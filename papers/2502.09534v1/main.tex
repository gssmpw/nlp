\documentclass{article}

\usepackage[utf8]{inputenc} % allow utf-8 input
\usepackage[T1]{fontenc}    % use 8-bit T1 fonts
\usepackage{hyperref}       % hyperlinks
\usepackage{url}            % simple URL typesetting
\usepackage{booktabs}       % professional-quality tables
\usepackage{amsfonts}       % blackboard math symbols
\usepackage{nicefrac}       % compact symbols for 1/2, etc.
\usepackage{microtype}      % microtypography
\usepackage{xcolor}         % colors

\usepackage{graphicx}
\usepackage{subcaption}

% Attempt to make hyperref and algorithmic work together better:
\newcommand{\theHalgorithm}{\arabic{algorithm}}

% For theorems and such
\usepackage{amsmath}
\usepackage{amssymb}
\usepackage{mathtools}
\usepackage{amsthm}

% if you use cleveref..
\usepackage[capitalize,nameinlink,noabbrev]{cleveref}

%%%%%%%%%%%%%%%%%%%%%%%%%%%%%%%%
% THEOREMS
%%%%%%%%%%%%%%%%%%%%%%%%%%%%%%%%
\usepackage{thm-restate}
\theoremstyle{plain}
\newtheorem{theorem}{Theorem}[section]
\newtheorem{proposition}[theorem]{Proposition}
\newtheorem{lemma}[theorem]{Lemma}
\newtheorem{corollary}[theorem]{Corollary}
\theoremstyle{definition}
\newtheorem{definition}[theorem]{Definition}
\newtheorem{assumption}[theorem]{Assumption}
\theoremstyle{remark}
\newtheorem{remark}[theorem]{Remark}


%
\setlength\unitlength{1mm}
\newcommand{\twodots}{\mathinner {\ldotp \ldotp}}
% bb font symbols
\newcommand{\Rho}{\mathrm{P}}
\newcommand{\Tau}{\mathrm{T}}

\newfont{\bbb}{msbm10 scaled 700}
\newcommand{\CCC}{\mbox{\bbb C}}

\newfont{\bb}{msbm10 scaled 1100}
\newcommand{\CC}{\mbox{\bb C}}
\newcommand{\PP}{\mbox{\bb P}}
\newcommand{\RR}{\mbox{\bb R}}
\newcommand{\QQ}{\mbox{\bb Q}}
\newcommand{\ZZ}{\mbox{\bb Z}}
\newcommand{\FF}{\mbox{\bb F}}
\newcommand{\GG}{\mbox{\bb G}}
\newcommand{\EE}{\mbox{\bb E}}
\newcommand{\NN}{\mbox{\bb N}}
\newcommand{\KK}{\mbox{\bb K}}
\newcommand{\HH}{\mbox{\bb H}}
\newcommand{\SSS}{\mbox{\bb S}}
\newcommand{\UU}{\mbox{\bb U}}
\newcommand{\VV}{\mbox{\bb V}}


\newcommand{\yy}{\mathbbm{y}}
\newcommand{\xx}{\mathbbm{x}}
\newcommand{\zz}{\mathbbm{z}}
\newcommand{\sss}{\mathbbm{s}}
\newcommand{\rr}{\mathbbm{r}}
\newcommand{\pp}{\mathbbm{p}}
\newcommand{\qq}{\mathbbm{q}}
\newcommand{\ww}{\mathbbm{w}}
\newcommand{\hh}{\mathbbm{h}}
\newcommand{\vvv}{\mathbbm{v}}

% Vectors

\newcommand{\av}{{\bf a}}
\newcommand{\bv}{{\bf b}}
\newcommand{\cv}{{\bf c}}
\newcommand{\dv}{{\bf d}}
\newcommand{\ev}{{\bf e}}
\newcommand{\fv}{{\bf f}}
\newcommand{\gv}{{\bf g}}
\newcommand{\hv}{{\bf h}}
\newcommand{\iv}{{\bf i}}
\newcommand{\jv}{{\bf j}}
\newcommand{\kv}{{\bf k}}
\newcommand{\lv}{{\bf l}}
\newcommand{\mv}{{\bf m}}
\newcommand{\nv}{{\bf n}}
\newcommand{\ov}{{\bf o}}
\newcommand{\pv}{{\bf p}}
\newcommand{\qv}{{\bf q}}
\newcommand{\rv}{{\bf r}}
\newcommand{\sv}{{\bf s}}
\newcommand{\tv}{{\bf t}}
\newcommand{\uv}{{\bf u}}
\newcommand{\wv}{{\bf w}}
\newcommand{\vv}{{\bf v}}
\newcommand{\xv}{{\bf x}}
\newcommand{\yv}{{\bf y}}
\newcommand{\zv}{{\bf z}}
\newcommand{\zerov}{{\bf 0}}
\newcommand{\onev}{{\bf 1}}

% Matrices

\newcommand{\Am}{{\bf A}}
\newcommand{\Bm}{{\bf B}}
\newcommand{\Cm}{{\bf C}}
\newcommand{\Dm}{{\bf D}}
\newcommand{\Em}{{\bf E}}
\newcommand{\Fm}{{\bf F}}
\newcommand{\Gm}{{\bf G}}
\newcommand{\Hm}{{\bf H}}
\newcommand{\Id}{{\bf I}}
\newcommand{\Jm}{{\bf J}}
\newcommand{\Km}{{\bf K}}
\newcommand{\Lm}{{\bf L}}
\newcommand{\Mm}{{\bf M}}
\newcommand{\Nm}{{\bf N}}
\newcommand{\Om}{{\bf O}}
\newcommand{\Pm}{{\bf P}}
\newcommand{\Qm}{{\bf Q}}
\newcommand{\Rm}{{\bf R}}
\newcommand{\Sm}{{\bf S}}
\newcommand{\Tm}{{\bf T}}
\newcommand{\Um}{{\bf U}}
\newcommand{\Wm}{{\bf W}}
\newcommand{\Vm}{{\bf V}}
\newcommand{\Xm}{{\bf X}}
\newcommand{\Ym}{{\bf Y}}
\newcommand{\Zm}{{\bf Z}}

% Calligraphic

\newcommand{\Ac}{{\cal A}}
\newcommand{\Bc}{{\cal B}}
\newcommand{\Cc}{{\cal C}}
\newcommand{\Dc}{{\cal D}}
\newcommand{\Ec}{{\cal E}}
\newcommand{\Fc}{{\cal F}}
\newcommand{\Gc}{{\cal G}}
\newcommand{\Hc}{{\cal H}}
\newcommand{\Ic}{{\cal I}}
\newcommand{\Jc}{{\cal J}}
\newcommand{\Kc}{{\cal K}}
\newcommand{\Lc}{{\cal L}}
\newcommand{\Mc}{{\cal M}}
\newcommand{\Nc}{{\cal N}}
\newcommand{\nc}{{\cal n}}
\newcommand{\Oc}{{\cal O}}
\newcommand{\Pc}{{\cal P}}
\newcommand{\Qc}{{\cal Q}}
\newcommand{\Rc}{{\cal R}}
\newcommand{\Sc}{{\cal S}}
\newcommand{\Tc}{{\cal T}}
\newcommand{\Uc}{{\cal U}}
\newcommand{\Wc}{{\cal W}}
\newcommand{\Vc}{{\cal V}}
\newcommand{\Xc}{{\cal X}}
\newcommand{\Yc}{{\cal Y}}
\newcommand{\Zc}{{\cal Z}}

% Bold greek letters

\newcommand{\alphav}{\hbox{\boldmath$\alpha$}}
\newcommand{\betav}{\hbox{\boldmath$\beta$}}
\newcommand{\gammav}{\hbox{\boldmath$\gamma$}}
\newcommand{\deltav}{\hbox{\boldmath$\delta$}}
\newcommand{\etav}{\hbox{\boldmath$\eta$}}
\newcommand{\lambdav}{\hbox{\boldmath$\lambda$}}
\newcommand{\epsilonv}{\hbox{\boldmath$\epsilon$}}
\newcommand{\nuv}{\hbox{\boldmath$\nu$}}
\newcommand{\muv}{\hbox{\boldmath$\mu$}}
\newcommand{\zetav}{\hbox{\boldmath$\zeta$}}
\newcommand{\phiv}{\hbox{\boldmath$\phi$}}
\newcommand{\psiv}{\hbox{\boldmath$\psi$}}
\newcommand{\thetav}{\hbox{\boldmath$\theta$}}
\newcommand{\tauv}{\hbox{\boldmath$\tau$}}
\newcommand{\omegav}{\hbox{\boldmath$\omega$}}
\newcommand{\xiv}{\hbox{\boldmath$\xi$}}
\newcommand{\sigmav}{\hbox{\boldmath$\sigma$}}
\newcommand{\piv}{\hbox{\boldmath$\pi$}}
\newcommand{\rhov}{\hbox{\boldmath$\rho$}}
\newcommand{\upsilonv}{\hbox{\boldmath$\upsilon$}}

\newcommand{\Gammam}{\hbox{\boldmath$\Gamma$}}
\newcommand{\Lambdam}{\hbox{\boldmath$\Lambda$}}
\newcommand{\Deltam}{\hbox{\boldmath$\Delta$}}
\newcommand{\Sigmam}{\hbox{\boldmath$\Sigma$}}
\newcommand{\Phim}{\hbox{\boldmath$\Phi$}}
\newcommand{\Pim}{\hbox{\boldmath$\Pi$}}
\newcommand{\Psim}{\hbox{\boldmath$\Psi$}}
\newcommand{\Thetam}{\hbox{\boldmath$\Theta$}}
\newcommand{\Omegam}{\hbox{\boldmath$\Omega$}}
\newcommand{\Xim}{\hbox{\boldmath$\Xi$}}


% Sans Serif small case

\newcommand{\Gsf}{{\sf G}}

\newcommand{\asf}{{\sf a}}
\newcommand{\bsf}{{\sf b}}
\newcommand{\csf}{{\sf c}}
\newcommand{\dsf}{{\sf d}}
\newcommand{\esf}{{\sf e}}
\newcommand{\fsf}{{\sf f}}
\newcommand{\gsf}{{\sf g}}
\newcommand{\hsf}{{\sf h}}
\newcommand{\isf}{{\sf i}}
\newcommand{\jsf}{{\sf j}}
\newcommand{\ksf}{{\sf k}}
\newcommand{\lsf}{{\sf l}}
\newcommand{\msf}{{\sf m}}
\newcommand{\nsf}{{\sf n}}
\newcommand{\osf}{{\sf o}}
\newcommand{\psf}{{\sf p}}
\newcommand{\qsf}{{\sf q}}
\newcommand{\rsf}{{\sf r}}
\newcommand{\ssf}{{\sf s}}
\newcommand{\tsf}{{\sf t}}
\newcommand{\usf}{{\sf u}}
\newcommand{\wsf}{{\sf w}}
\newcommand{\vsf}{{\sf v}}
\newcommand{\xsf}{{\sf x}}
\newcommand{\ysf}{{\sf y}}
\newcommand{\zsf}{{\sf z}}


% mixed symbols

\newcommand{\sinc}{{\hbox{sinc}}}
\newcommand{\diag}{{\hbox{diag}}}
\renewcommand{\det}{{\hbox{det}}}
\newcommand{\trace}{{\hbox{tr}}}
\newcommand{\sign}{{\hbox{sign}}}
\renewcommand{\arg}{{\hbox{arg}}}
\newcommand{\var}{{\hbox{var}}}
\newcommand{\cov}{{\hbox{cov}}}
\newcommand{\Ei}{{\rm E}_{\rm i}}
\renewcommand{\Re}{{\rm Re}}
\renewcommand{\Im}{{\rm Im}}
\newcommand{\eqdef}{\stackrel{\Delta}{=}}
\newcommand{\defines}{{\,\,\stackrel{\scriptscriptstyle \bigtriangleup}{=}\,\,}}
\newcommand{\<}{\left\langle}
\renewcommand{\>}{\right\rangle}
\newcommand{\herm}{{\sf H}}
\newcommand{\trasp}{{\sf T}}
\newcommand{\transp}{{\sf T}}
\renewcommand{\vec}{{\rm vec}}
\newcommand{\Psf}{{\sf P}}
\newcommand{\SINR}{{\sf SINR}}
\newcommand{\SNR}{{\sf SNR}}
\newcommand{\MMSE}{{\sf MMSE}}
\newcommand{\REF}{{\RED [REF]}}

% Markov chain
\usepackage{stmaryrd} % for \mkv 
\newcommand{\mkv}{-\!\!\!\!\minuso\!\!\!\!-}

% Colors

\newcommand{\RED}{\color[rgb]{1.00,0.10,0.10}}
\newcommand{\BLUE}{\color[rgb]{0,0,0.90}}
\newcommand{\GREEN}{\color[rgb]{0,0.80,0.20}}

%%%%%%%%%%%%%%%%%%%%%%%%%%%%%%%%%%%%%%%%%%
\usepackage{hyperref}
\hypersetup{
    bookmarks=true,         % show bookmarks bar?
    unicode=false,          % non-Latin characters in AcrobatÕs bookmarks
    pdftoolbar=true,        % show AcrobatÕs toolbar?
    pdfmenubar=true,        % show AcrobatÕs menu?
    pdffitwindow=false,     % window fit to page when opened
    pdfstartview={FitH},    % fits the width of the page to the window
%    pdftitle={My title},    % title
%    pdfauthor={Author},     % author
%    pdfsubject={Subject},   % subject of the document
%    pdfcreator={Creator},   % creator of the document
%    pdfproducer={Producer}, % producer of the document
%    pdfkeywords={keyword1} {key2} {key3}, % list of keywords
    pdfnewwindow=true,      % links in new window
    colorlinks=true,       % false: boxed links; true: colored links
    linkcolor=red,          % color of internal links (change box color with linkbordercolor)
    citecolor=green,        % color of links to bibliography
    filecolor=blue,      % color of file links
    urlcolor=blue           % color of external links
}
%%%%%%%%%%%%%%%%%%%%%%%%%%%%%%%%%%%%%%%%%%%



\title{Fast Tensor Completion via Approximate Richardson Iteration}

\author[1]{Mehrdad Ghadiri}
%Emails: \texttt{mehrdadg@mit.edu, fahrbach@google.com, yb.kook@gatech.edu, jadbabai@mit.edu}}}
\author[2]{Matthew Fahrbach}
\author[3]{Yunbum Kook}
\author[1]{Ali Jadbabaie}

\affil[1]{Massachusetts Institute of Technology, \texttt{\{mehrdadg,jadbabai\}@mit.edu}}
\affil[2]{Google Research, \texttt{fahrbach@google.com}}
\affil[3]{Georgia Institute of Technology, \texttt{yb.kook@gatech.edu}}

\date{}

\begin{document}

\maketitle

\begin{abstract}  
Test time scaling is currently one of the most active research areas that shows promise after training time scaling has reached its limits.
Deep-thinking (DT) models are a class of recurrent models that can perform easy-to-hard generalization by assigning more compute to harder test samples.
However, due to their inability to determine the complexity of a test sample, DT models have to use a large amount of computation for both easy and hard test samples.
Excessive test time computation is wasteful and can cause the ``overthinking'' problem where more test time computation leads to worse results.
In this paper, we introduce a test time training method for determining the optimal amount of computation needed for each sample during test time.
We also propose Conv-LiGRU, a novel recurrent architecture for efficient and robust visual reasoning. 
Extensive experiments demonstrate that Conv-LiGRU is more stable than DT, effectively mitigates the ``overthinking'' phenomenon, and achieves superior accuracy.
\end{abstract}  
\section{Introduction}
\label{sec:introduction}
The business processes of organizations are experiencing ever-increasing complexity due to the large amount of data, high number of users, and high-tech devices involved \cite{martin2021pmopportunitieschallenges, beerepoot2023biggestbpmproblems}. This complexity may cause business processes to deviate from normal control flow due to unforeseen and disruptive anomalies \cite{adams2023proceddsriftdetection}. These control-flow anomalies manifest as unknown, skipped, and wrongly-ordered activities in the traces of event logs monitored from the execution of business processes \cite{ko2023adsystematicreview}. For the sake of clarity, let us consider an illustrative example of such anomalies. Figure \ref{FP_ANOMALIES} shows a so-called event log footprint, which captures the control flow relations of four activities of a hypothetical event log. In particular, this footprint captures the control-flow relations between activities \texttt{a}, \texttt{b}, \texttt{c} and \texttt{d}. These are the causal ($\rightarrow$) relation, concurrent ($\parallel$) relation, and other ($\#$) relations such as exclusivity or non-local dependency \cite{aalst2022pmhandbook}. In addition, on the right are six traces, of which five exhibit skipped, wrongly-ordered and unknown control-flow anomalies. For example, $\langle$\texttt{a b d}$\rangle$ has a skipped activity, which is \texttt{c}. Because of this skipped activity, the control-flow relation \texttt{b}$\,\#\,$\texttt{d} is violated, since \texttt{d} directly follows \texttt{b} in the anomalous trace.
\begin{figure}[!t]
\centering
\includegraphics[width=0.9\columnwidth]{images/FP_ANOMALIES.png}
\caption{An example event log footprint with six traces, of which five exhibit control-flow anomalies.}
\label{FP_ANOMALIES}
\end{figure}

\subsection{Control-flow anomaly detection}
Control-flow anomaly detection techniques aim to characterize the normal control flow from event logs and verify whether these deviations occur in new event logs \cite{ko2023adsystematicreview}. To develop control-flow anomaly detection techniques, \revision{process mining} has seen widespread adoption owing to process discovery and \revision{conformance checking}. On the one hand, process discovery is a set of algorithms that encode control-flow relations as a set of model elements and constraints according to a given modeling formalism \cite{aalst2022pmhandbook}; hereafter, we refer to the Petri net, a widespread modeling formalism. On the other hand, \revision{conformance checking} is an explainable set of algorithms that allows linking any deviations with the reference Petri net and providing the fitness measure, namely a measure of how much the Petri net fits the new event log \cite{aalst2022pmhandbook}. Many control-flow anomaly detection techniques based on \revision{conformance checking} (hereafter, \revision{conformance checking}-based techniques) use the fitness measure to determine whether an event log is anomalous \cite{bezerra2009pmad, bezerra2013adlogspais, myers2018icsadpm, pecchia2020applicationfailuresanalysispm}. 

The scientific literature also includes many \revision{conformance checking}-independent techniques for control-flow anomaly detection that combine specific types of trace encodings with machine/deep learning \cite{ko2023adsystematicreview, tavares2023pmtraceencoding}. Whereas these techniques are very effective, their explainability is challenging due to both the type of trace encoding employed and the machine/deep learning model used \cite{rawal2022trustworthyaiadvances,li2023explainablead}. Hence, in the following, we focus on the shortcomings of \revision{conformance checking}-based techniques to investigate whether it is possible to support the development of competitive control-flow anomaly detection techniques while maintaining the explainable nature of \revision{conformance checking}.
\begin{figure}[!t]
\centering
\includegraphics[width=\columnwidth]{images/HIGH_LEVEL_VIEW.png}
\caption{A high-level view of the proposed framework for combining \revision{process mining}-based feature extraction with dimensionality reduction for control-flow anomaly detection.}
\label{HIGH_LEVEL_VIEW}
\end{figure}

\subsection{Shortcomings of \revision{conformance checking}-based techniques}
Unfortunately, the detection effectiveness of \revision{conformance checking}-based techniques is affected by noisy data and low-quality Petri nets, which may be due to human errors in the modeling process or representational bias of process discovery algorithms \cite{bezerra2013adlogspais, pecchia2020applicationfailuresanalysispm, aalst2016pm}. Specifically, on the one hand, noisy data may introduce infrequent and deceptive control-flow relations that may result in inconsistent fitness measures, whereas, on the other hand, checking event logs against a low-quality Petri net could lead to an unreliable distribution of fitness measures. Nonetheless, such Petri nets can still be used as references to obtain insightful information for \revision{process mining}-based feature extraction, supporting the development of competitive and explainable \revision{conformance checking}-based techniques for control-flow anomaly detection despite the problems above. For example, a few works outline that token-based \revision{conformance checking} can be used for \revision{process mining}-based feature extraction to build tabular data and develop effective \revision{conformance checking}-based techniques for control-flow anomaly detection \cite{singh2022lapmsh, debenedictis2023dtadiiot}. However, to the best of our knowledge, the scientific literature lacks a structured proposal for \revision{process mining}-based feature extraction using the state-of-the-art \revision{conformance checking} variant, namely alignment-based \revision{conformance checking}.

\subsection{Contributions}
We propose a novel \revision{process mining}-based feature extraction approach with alignment-based \revision{conformance checking}. This variant aligns the deviating control flow with a reference Petri net; the resulting alignment can be inspected to extract additional statistics such as the number of times a given activity caused mismatches \cite{aalst2022pmhandbook}. We integrate this approach into a flexible and explainable framework for developing techniques for control-flow anomaly detection. The framework combines \revision{process mining}-based feature extraction and dimensionality reduction to handle high-dimensional feature sets, achieve detection effectiveness, and support explainability. Notably, in addition to our proposed \revision{process mining}-based feature extraction approach, the framework allows employing other approaches, enabling a fair comparison of multiple \revision{conformance checking}-based and \revision{conformance checking}-independent techniques for control-flow anomaly detection. Figure \ref{HIGH_LEVEL_VIEW} shows a high-level view of the framework. Business processes are monitored, and event logs obtained from the database of information systems. Subsequently, \revision{process mining}-based feature extraction is applied to these event logs and tabular data input to dimensionality reduction to identify control-flow anomalies. We apply several \revision{conformance checking}-based and \revision{conformance checking}-independent framework techniques to publicly available datasets, simulated data of a case study from railways, and real-world data of a case study from healthcare. We show that the framework techniques implementing our approach outperform the baseline \revision{conformance checking}-based techniques while maintaining the explainable nature of \revision{conformance checking}.

In summary, the contributions of this paper are as follows.
\begin{itemize}
    \item{
        A novel \revision{process mining}-based feature extraction approach to support the development of competitive and explainable \revision{conformance checking}-based techniques for control-flow anomaly detection.
    }
    \item{
        A flexible and explainable framework for developing techniques for control-flow anomaly detection using \revision{process mining}-based feature extraction and dimensionality reduction.
    }
    \item{
        Application to synthetic and real-world datasets of several \revision{conformance checking}-based and \revision{conformance checking}-independent framework techniques, evaluating their detection effectiveness and explainability.
    }
\end{itemize}

The rest of the paper is organized as follows.
\begin{itemize}
    \item Section \ref{sec:related_work} reviews the existing techniques for control-flow anomaly detection, categorizing them into \revision{conformance checking}-based and \revision{conformance checking}-independent techniques.
    \item Section \ref{sec:abccfe} provides the preliminaries of \revision{process mining} to establish the notation used throughout the paper, and delves into the details of the proposed \revision{process mining}-based feature extraction approach with alignment-based \revision{conformance checking}.
    \item Section \ref{sec:framework} describes the framework for developing \revision{conformance checking}-based and \revision{conformance checking}-independent techniques for control-flow anomaly detection that combine \revision{process mining}-based feature extraction and dimensionality reduction.
    \item Section \ref{sec:evaluation} presents the experiments conducted with multiple framework and baseline techniques using data from publicly available datasets and case studies.
    \item Section \ref{sec:conclusions} draws the conclusions and presents future work.
\end{itemize}
% !TEX root =  ../main.tex
\section{Background on causality and abstraction}\label{sec:preliminaries}

This section provides the notation and key concepts related to causal modeling and abstraction theory.

\spara{Notation.} The set of integers from $1$ to $n$ is $[n]$.
The vectors of zeros and ones of size $n$ are $\zeros_n$ and $\ones_n$.
The identity matrix of size $n \times n$ is $\identity_n$. The Frobenius norm is $\frob{\mathbf{A}}$.
The set of positive definite matrices over $\reall^{n\times n}$ is $\pd^n$. The Hadamard product is $\odot$.
Function composition is $\circ$.
The domain of a function is $\dom{\cdot}$ and its kernel $\ker$.
Let $\mathcal{M}(\mathcal{X}^n)$ be the set of Borel measures over $\mathcal{X}^n \subseteq \reall^n$. Given a measure $\mu^n \in \mathcal{M}(\mathcal{X}^n)$ and a measurable map $\varphi^{\V}$, $\mathcal{X}^n \ni \mathbf{x} \overset{\varphi^{\V}}{\longmapsto} \V^\top \mathbf{x} \in \mathcal{X}^m$, we denote by $\varphi^{\V}_{\#}(\mu^n) \coloneqq \mu^n(\varphi^{\V^{-1}}(\mathbf{x}))$ the pushforward measure $\mu^m \in \mathcal{M}(\mathcal{X}^m)$. 


We now present the standard definition of SCM.

\begin{definition}[SCM, \citealp{pearl2009causality}]\label{def:SCM}
A (Markovian) structural causal model (SCM) $\scm^n$ is a tuple $\langle \myendogenous, \myexogenous, \myfunctional, \zeta^\myexogenous \rangle$, where \emph{(i)} $\myendogenous = \{X_1, \ldots, X_n\}$ is a set of $n$ endogenous random variables; \emph{(ii)} $\myexogenous =\{Z_1,\ldots,Z_n\}$ is a set of $n$ exogenous variables; \emph{(iii)} $\myfunctional$ is a set of $n$ functional assignments such that $X_i=f_i(\parents_i, Z_i)$, $\forall \; i \in [n]$, with $ \parents_i \subseteq \myendogenous \setminus \{ X_i\}$; \emph{(iv)} $\zeta^\myexogenous$ is a product probability measure over independent exogenous variables $\zeta^\myexogenous=\prod_{i \in [n]} \zeta^i$, where $\zeta^i=P(Z_i)$. 
\end{definition}
A Markovian SCM induces a directed acyclic graph (DAG) $\mathcal{G}_{\scm^n}$ where the nodes represent the variables $\myendogenous$ and the edges are determined by the structural functions $\myfunctional$; $ \parents_i$ constitutes then the parent set for $X_i$. Furthermore, we can recursively rewrite the set of structural function $\myfunctional$ as a set of mixing functions $\mymixing$ dependent only on the exogenous variables (cf. \cref{app:CA}). A key feature for studying causality is the possibility of defining interventions on the model:
\begin{definition}[Hard intervention, \citealp{pearl2009causality}]\label{def:intervention}
Given SCM $\scm^n = \langle \myendogenous, \myexogenous, \myfunctional, \zeta^\myexogenous \rangle$, a (hard) intervention $\iota = \operatorname{do}(\myendogenous^{\iota} = \mathbf{x}^{\iota})$, $\myendogenous^{\iota}\subseteq \myendogenous$,
is an operator that generates a new post-intervention SCM $\scm^n_\iota = \langle \myendogenous, \myexogenous, \myfunctional_\iota, \zeta^\myexogenous \rangle$ by replacing each function $f_i$ for $X_i\in\myendogenous^{\iota}$ with the constant $x_i^\iota\in \mathbf{x}^\iota$. 
Graphically, an intervention mutilates $\mathcal{G}_{\mathsf{M}^n}$ by removing all the incoming edges of the variables in $\myendogenous^{\iota}$.
\end{definition}

Given multiple SCMs describing the same system at different levels of granularity, CA provides the definition of an $\alpha$-abstraction map to relate these SCMs:
\begin{definition}[$\abst$-abstraction, \citealp{rischel2020category}]\label{def:abstraction}
Given low-level $\mathsf{M}^\ell$ and high-level $\mathsf{M}^h$ SCMs, an $\abst$-abstraction is a triple $\abst = \langle \Rset, \amap, \alphamap{} \rangle$, where \emph{(i)} $\Rset \subseteq \datalow$ is a subset of relevant variables in $\mathsf{M}^\ell$; \emph{(ii)} $\amap: \Rset \rightarrow \datahigh$ is a surjective function between the relevant variables of $\mathsf{M}^\ell$ and the endogenous variables of $\mathsf{M}^h$; \emph{(iii)} $\alphamap{}: \dom{\Rset} \rightarrow \dom{\datahigh}$ is a modular function $\alphamap{} = \bigotimes_{i\in[n]} \alphamap{X^h_i}$ made up by surjective functions $\alphamap{X^h_i}: \dom{\amap^{-1}(X^h_i)} \rightarrow \dom{X^h_i}$ from the outcome of low-level variables $\amap^{-1}(X^h_i) \in \datalow$ onto outcomes of the high-level variables $X^h_i \in \datahigh$.
\end{definition}
Notice that an $\abst$-abstraction simultaneously maps variables via the function $\amap$ and values through the function $\alphamap{}$. The definition itself does not place any constraint on these functions, although a common requirement in the literature is for the abstraction to satisfy \emph{interventional consistency} \cite{rubenstein2017causal,rischel2020category,beckers2019abstracting}. An important class of such well-behaved abstractions is \emph{constructive linear abstraction}, for which the following properties hold. By constructivity, \emph{(i)} $\abst$ is interventionally consistent; \emph{(ii)} all low-level variables are relevant $\Rset=\datalow$; \emph{(iii)} in addition to the map $\alphamap{}$ between endogenous variables, there exists a map ${\alphamap{}}_U$ between exogenous variables satisfying interventional consistency \cite{beckers2019abstracting,schooltink2024aligning}. By linearity, $\alphamap{} = \V^\top \in \reall^{h \times \ell}$ \cite{massidda2024learningcausalabstractionslinear}. \cref{app:CA} provides formal definitions for interventional consistency, linear and constructive abstraction.
\section{Approximate Richardson Iteration}
\label{sec:lifted_regression}

We now present our main techniques for reducing tensor completion to tensor decomposition.
When we use ALS to solve a TC problem,
we must efficiently solve least-squares problems $\min_{\mat x} \, \norm{\mat{Px}-\mat q}_2$.
The rows of the design matrix $\mat{P}$ correspond to the \emph{subset of observations} in the TC problem,
so $\mat{P}$ does not necessarily have the structure of the design matrix in the full TD problem.

A direct approach is to compute the closed-form solution $\mat x^* = (\mat P^\top \mat P)^{-1}\mat P^\top \mat q$, but computing $(\mat P^\top \mat P)^{-1}$ is often impractical.
Two techniques are commonly used to overcome this:
(1) iterative methods and (2) row sampling.
Iterative methods repeat the same relatively cheap per-step computation
\emph{many times} to approximate the original expensive computation.
Row sampling methods (e.g., leverage score sampling) randomly pick rows of $\mat P$
and solve a least-squares problem on the sampled rows to obtain an approximate solution to the original problem with high probability.
Directly computing leverage scores for a general $\mat{P}$, however,
is \emph{also prohibitively expensive} since it requires computing
the same matrix $(\mat P^\top \mat P)^{-1}$ (see \Cref{app:leverage-score} for details).

We show that our novel \emph{approximate-mini-ALS} method
is a principled approach for tensor completion.
In \Cref{ssec:lifting}, we show how lifting \emph{restores the structure} of the full TD ALS update step,
enabling fast least-squares methods for a larger (but equivalent) problem.
In \Cref{subsec:iterative_methods}, we show that iteratively solving this lifted problem (i.e., mini-ALS) is connected to an iterative method called the \emph{Richardson iteration}~\citep{richardson1911approximate},
which we can also view as a matrix-splitting method.
In other words, mini-ALS and the Richardson iteration with a certain preconditioner give the same sequence of iterates $\{\mat{x}^{(k)}\}_{k \ge 0}$.
Lastly in~\Cref{sec:approx-solve-lifted},
we prove novel convergence guarantees for \emph{approximately} solving the lifted problem
(i.e., for approximate-mini-ALS).
This allows us to directly use fast leverage-score sampling algorithms for
CP decomposition~\citep{cheng2016spals,larsen2022practical,bharadwaj2023fast},
Tucker decomposition~\citep{diao2019optimal,fahrbach2022subquadratic},
and TT decomposition~\citep{bharadwaj2024efficient}
as blackbox subroutines.

\subsection{Lifting to a Structured Problem}
\label{ssec:lifting}
Consider the linear regression problem with $\mat{P}\in\R^{|\Omega| \times R}$ and $\mat{q} \in \R^{|\Omega|}$ given by
\begin{equation}
\label{eqn:input_regression}
    \mat{x}^* = \argmin_{\mat{x} \in \R^R} \,\norm{\mat{P} \mat{x} - \mat{q}}_{2}^2\,.
\end{equation}
If there exists a tall structured matrix $\mat{A} \in \R^{I \times R}$
with a subset of rows $\Omega \subseteq [I]$ such that $\mat{A}_{\Omega} = \mat{P}$
(permutations of the rows allowed),
then we can lift \eqref{eqn:input_regression} to a higher-dimensional problem while preserving the optimal solution.

\begin{restatable}[]{lemma}{LiftedRegression}
\label{lem:lifted_regression}
Let $\mat{b} \in \R^I$ be the lifted response such that $\mat{b}_{\Omega} = \mat{q}$
and $\mat{b}_{\overline{\Omega}}$ is a free variable. If
\begin{equation}
\label{eqn:lifted_regression}
    (\mat{x}^*, \mat{b}^*_{\overline{\Omega}})
    =
    \argmin_{\mat{x} \in \R^R, \mat{b}_{\overline{\Omega}} \in \R^{I - |\Omega|}}\, \norm*{\mat{A} \mat{x} - \mat{b}}_{2}\,,
\end{equation}
then $\mat{x}^*$ also minimizes \eqref{eqn:input_regression},
i.e., the original linear regression problem $\min_{\mat{x} \in \R^R}\, \norm{\mat{P} \mat{x} - \mat{q}}_{2}^2$.
\end{restatable}

\begin{proof}
For any $\mat{x}$, we have
\[
    \norm{\mat{A}\mat{x}-\mat{b}}_2^2
    =
    \norm{\mat{A}_{\Omega}\mat{x} - \mat{b}_{\Omega}}_{2}^2
    +
    \norm{\mat{A}_{\overline\Omega}\mat{x} - \mat{b}_{\overline\Omega}}_{2}^2\,,
\]
so
\[
\min_{\mat{x}}\norm{\mat{P} \mat{x} - \mat{q}}_2^2 \leq \min_{\mat{x},\mat{b}_{\overline{\Omega}}} \norm{\mat{A}\mat{x}-\mat{b}}_2^2\,.
\]
Moreover, for any $\mat{x}$, taking $\mat{b}_{\overline \Omega} = \mat{A}_{\overline \Omega}\mat{x}$ gives us $\norm{\mat{A}_{\overline\Omega}\mat{x} - \mat{b}_{\overline\Omega}}_{2}^2=0$, which implies that
\[
\min_{\mat{x}} \norm{\mat{P} \mat{x} - \mat{q}}_2^2 \geq \min_{\mat{x},\mat{b}_{\overline{\Omega}}} \norm{\mat{A}\mat{x}-\mat{b}}_2^2\,.
\]
Therefore,
\[
\min_{\mat{x}}\norm{\mat{P} \mat{x} - \mat{q}}_2^2 = \min_{\mat{x},\mat{b}_{\overline{\Omega}}} \norm{\mat{A}\mat{x}-\mat{b}}_2^2\,,
\]
and $\mat{x}^*$ also minimizes \eqref{eqn:input_regression}.
\end{proof}

For the rest of this section,
let $\widetilde{\mat{P}} \in \R^{I\times R}$, $\widetilde{\mat{q}}\in \R^{I}$
be the zero-masked lifted matrix and vector such that
\[
    (\widetilde{\mat{P}}_{\Omega}, \widetilde{\mat{P}}_{\overline{\Omega}}) = (\mat{A}_{\Omega}, \boldsymbol{0})
    \quad
    \text{and}
    \quad
    (\widetilde{\mat{q}}_{\Omega}, \widetilde{\mat{q}}_{\overline{\Omega}})
    =
    (\mat{b}_{\Omega}, \boldsymbol{0}).
\]

\begin{restatable}{lemma}{ConvexQuadratic}
\label{lem:lifted_problem_is_convex_quadratic}
Problem~\ref{eqn:lifted_regression} is a convex quadratic program.
\end{restatable}

\begin{proof}
Since $\widetilde{\mat{q}}\in \R^{I}$ is defined as $\widetilde{\mat{q}}_{\Omega} = \mat{b}_{\Omega}$ and $\widetilde{\mat{q}}_{\overline{\Omega}} = \boldsymbol{0}$,
we can write \eqref{eqn:lifted_regression} in the following equivalent manner:
\[
    (\mat{x}^*, \mat{b}^*_{\overline{\Omega}})
    =
    \argmin_{\mat{x} \in \R^R, \mat{b}_{\overline{\Omega}}\in\R^{I-|\Omega|}}
    \norm*{
    \begin{bmatrix}
        \mat{A} & -\mat{I}_{:,\overline{\Omega}}
    \end{bmatrix}
    \begin{bmatrix}
        \mat{x} \\ \mat{b}_{\overline{\Omega}}
    \end{bmatrix}
    -
    \widetilde{\mat{q}}
    }_{2}^2\,,
\]
where $\mat{I}$ is the $I\times I$ identity matrix.
\end{proof}

\begin{remark}
Problem~\ref{eqn:lifted_regression} is  not a linear regression problem
with (structured) design matrix~$\mat{A}$ since there are $\mat{b}_{\overline{\Omega}}$ variables in the response.
There is, however, enough structure to employ block minimization to alternate between minimizing $\mat{x}$ and $\mat{b}_{\overline{\Omega}}$.
\end{remark}

\subsection{Iterative Methods for the Lifted Problem}
\label{subsec:iterative_methods}

Iterative methods for solving linear systems and regression problems have a long history and have been used to expedite several algorithms in theory and practice.
The algorithms we consider use the exact arithmetic model, but all of these methods can be carried out with numbers with $\log \nicefrac{\kappa}\epsilon$ bits, where $\kappa$ is the condition number of the matrix (see, e.g., \citet{ghadiri2023bit,ghadiri2024improving}).
There is a literature on \emph{inexact} Richardson iteration for solving linear systems, 
but they require the error $\widehat{\epsilon}$ to be smaller than than $1/\kappa$,
which is not achievable with leverage-score sampling \cite{golub1988convergence,golub1997closer}.

\begin{lemma}[{Preconditioned Richardson iteration, \citep[Lemma 6.1]{lee2024techniques}}]
\label{lem:richardson_iteration}
Consider the least-squares problem $\mat{x}^* = \argmin_{\mat{x} \in \R^R}\, \norm{\mat{P}\mat{x} - \mat{q}}$.
Let $\mat{M}$ be a matrix such that $\mat{P}^\top \mat{P} \preccurlyeq \mat{M} \preccurlyeq \beta \cdot \mat{P}^\top \mat{P}$
for some $\beta \ge 1$,
and consider the Richardson iteration:
\[
    \mat{x}^{(k+1)} = \mat{x}^{(k)} - \mat{M}^{-1}(\mat{P}^\top \mat{P} \mat{x}^{(k)} - \mat{P}^\top \mat{q})\,.
\]
Then, we have that
\[
    \norm{\mat{x}^{(k+1)} - \mat{x}^*}_{\mat{M}}
    \le
    \Bigl(1 - \frac{1}{\beta}\Bigr)
    \norm{\mat{x}^{(k)} - \mat{x}^*}_{\mat{M}}\,.
\]
\end{lemma}

We now present a key lemma showing that alternating minimization between $\mat{x}$ and $\mat{b}_{\overline{\Omega}}$ corresponds to preconditioned Richardson iterations
on the original least-squares problem.
Below, one can easily check that $\mat{A}$, $\widetilde{\mat{P}}$, and $\widetilde{\mat{q}}$
in our lifted approach satisfy this condition.

\begin{restatable}{lemma}{RichardsonSimulation}
\label{lemma:richardson-simulation}
Let $\mat{A}, \widetilde{\mat{P}} \in \R^{I \times R}$, $\widetilde{\mat{q}}\in\R^{I}$ such that $\widetilde{\mat{P}}-\mat{A}$ and
$\bigl[\begin{matrix}
\widetilde{\mat{P}} & \widetilde{\mat{q}}
\end{matrix}\bigr]$
are orthogonal, i.e., $(\widetilde{\mat{P}}-\mat{A})^\top \bigl[\begin{matrix} \widetilde{\mat{P}} & \widetilde{\mat{q}} \end{matrix}\bigr] = \mat{0}$.
Then, the iterative method
\begin{align*}
    \widetilde{\mat{q}}^{(k)} & = \widetilde{\mat{q}} + (\mat{A} - \widetilde{\mat{P}})\, \mat{x}^{(k)}\,, \\
    \mat{x}^{(k+1)} & = \argmin_{\mat{x} \in \R^{R}}\, \norm{\mat{A} \mat{x} - \widetilde{\mat{q}}^{(k)}}_2^2\,,
\end{align*}
simulates Richardson iterations with preconditioner $\mat{A}^\top \mat{A}$
for the regression problem $\min_{\mat{x}}\, \norm{\widetilde{\mat{P}} \mat{x} - \widetilde{\mat{q}}}_2^2$,
i.e.,
\begin{equation}
    \label{eq:update-rule}
    \mat{x}^{(k+1)}
    =
    \mat{x}^{(k)} - (\mat{A}^\top \mat{A})^{-1} (\widetilde{\mat{P}}^\top \widetilde{\mat{P}} \mat{x}^{(k)} - \widetilde{\mat{P}}^\top \widetilde{\mat{q}})\,.
\end{equation}
\end{restatable}

\begin{proof}
Assume that $\mat{A}^\top \mat{A}$ is full rank.
Solving the normal equation for $\mat{x}^{(k+1)}$,
\begin{align*}
\mat{x}^{(k+1)}
&= (\mat{A}^\top \mat{A})^{-1} \mat{A}^\top \widetilde{\mat{q}}^{(k)}\\
&= (\mat{A}^\top \mat{A})^{-1} \mat{A}^\top (\widetilde{\mat{q}} + (\mat{A} - \widetilde{\mat{P}})\, \mat{x}^{(k)})
\\ & =
\mat{x}^{(k)} - (\mat{A}^\top \mat{A})^{-1} \mat{A}^\top (\widetilde{\mat{P}} \mat{x}^{(k)} - \widetilde{\mat{q}})\,.
\end{align*}
Since $\widetilde{\mat{P}}-\mat{A}$ and $\bigl[\begin{matrix}
\widetilde{\mat{P}} & \widetilde{\mat{q}}
\end{matrix}\bigr]$ are orthogonal, 
\begin{align*}
    \mat{A}^\top (\widetilde{\mat{P}} \mat{x}^{(k)} - \widetilde{\mat{q}})
    &=
    \bigl(\widetilde{\mat{P}} - (\widetilde{\mat{P}} - \mat{A})\bigr)^\top \bigl[\begin{matrix}
        \widetilde{\mat{P}} & \widetilde{\mat{q}}
    \end{matrix}\bigr] \begin{bmatrix}
    \mat{x}^{(k)} \\ -1
    \end{bmatrix} \\
    &= 
    \widetilde{\mat{P}}^\top (\widetilde{\mat{P}} \mat{x}^{(k)} - \widetilde{\mat{q}})\,.
\end{align*}
Therefore,
\[
    \mat{x}^{(k+1)}
    =
    \mat{x}^{(k)} - (\mat{A}^\top \mat{A})^{-1} (\widetilde{\mat{P}}^\top \widetilde{\mat{P}} \mat{x}^{(k)} - \widetilde{\mat{P}}^\top \widetilde{\mat{q}})\,,
\]
which completes the proof.
\end{proof}

\begin{remark}
In the tensor completion setting, $\mat{A}-\widetilde{\mat{P}}$ vanishes over $\Omega$,
so $\widetilde{\mat{q}}^{(k)}$ only updates entries in $\overline{\Omega}$ while maintaining $\mat{q}$ on $\Omega$.
Thus, computing $\mat{x}^{(k+1)}$ corresponds to
\begin{align*}
    \mat{x}^{(k+1)}
    &=
    \argmin_{\mat{x} \in \R^R}\, \norm{\mat{Ax}-\widetilde{\mat{q}}^{(k)}}^2_2
    = 
    \argmin_{\mat{x} \in \R^R}\, \bigl\{\norm{\mat{Px}-\mat{q}}^2_2 + \norm{\mat{A}_{\overline \Omega}\,(\mat x - \mat x^{(k)})}_2^2\bigr\}\,.
\end{align*}
\end{remark}

\subsection{Approximately Solving the Lifted Problem}
\label{sec:approx-solve-lifted}

We have shown that alternating minimization for the lifted problem~\eqref{eqn:lifted_regression}
has strong connections to preconditioned Richardson iteration
and inherits its convergence guarantees.
However, for this observation to be useful,
we need to use fast regression algorithms for the $\mat{x}^{(k+1)}$ updates that \emph{exploit the structure} of $\mat{A}$,
i.e.,
when solving $\min_{\mat x}\, \norm{\mat{Ax} - \widetilde{\mat{q}}^{(k)}}_2$.

This is where leverage score sampling comes in to play.
We exploit the structure of $\mat{A}$ to efficiently compute its leverage scores,
and then we solve the regression problem efficiently but \emph{approximately}.
By using a sketching method, our work deviates from the standard (exact) Richardson iteration.

Our next result shows how using approximate least-squares solutions
\emph{in each step of block minimization}
affects the convergence guarantee of our lifted iterative method.

\begin{algorithm2e}[t]
    \caption{\LiftedApproximateSolver}
    \label{alg:approximate-lifting}
	\BlankLine
	\KwData{$\mat{A},\widetilde{\mat{P}} \in \R^{I \times R}$, $\widetilde{\mat{q}}\in\R^{I}$, $\beta \geq 1$, $\epsilon \in (0,1)$, $\widehat{\epsilon} \in [0, 1/\beta^2)$ with $\widetilde{\mat{P}}^\top \widetilde{\mat{P}} \preceq \mat{A}^\top \mat{A} \preceq \beta \cdot \widetilde{\mat{P}}^\top \widetilde{\mat{P}}$}
	\KwResult{$\widetilde{\mat{x}} \in \R^{R}$}
	\BlankLine
	Initialize $\mat{x}^{(0)}=\boldsymbol{0}$ \\
        
        \For{$k = 0, 1, \dots, \ceil*{\frac{\log(\sfrac{2\beta}{\epsilon})}{2\,(\sfrac{1}{\beta} - \sqrt{\widehat{\epsilon}})} }$
        }{  
            Set $\widetilde{\mat{q}}^{(k)} \gets \widetilde{\mat{q}} + (\mat{A} - \widetilde{\mat{P}})\, \mat{x}^{(k)}$
            \hfill {\color{Navy} \tcp{\texttt{Implicit}}}
            Set $\mat{x}^{(k+1)}$ to a vector such that $ \norm{\mat{A} \mat{x}^{(k+1)} - \widetilde{\mat{q}}^{(k)}}_2^2 \leq(1+\widehat{\epsilon})\,\min_{\mat{x}}\, \norm{\mat{A} \mat{x} - \widetilde{\mat{q}}^{(k)}}_2^2$
        }
    \Return $\mat{x}^{(k)}$
\end{algorithm2e}

\begin{restatable}{theorem}{ApproximateRichardson}
\label{thm:approximate-richardson}
Let $\mat{A},\widetilde{\mat{P}} \in \R^{I \times R}$, $\widetilde{\mat{q}}\in\R^{I}$, and $\beta \ge 1$
such that
$\widetilde{\mat{P}}-\mat{A}$ and $\bigl[\begin{matrix} \widetilde{\mat{P}} & \widetilde{\mat{q}} \end{matrix}\bigr]$ are orthogonal, and
\[
    \widetilde{\mat{P}}^\top \widetilde{\mat{P}}
    \preceq
    \mat{A}^\top \mat{A}
    \preceq
    \beta \cdot \widetilde{\mat{P}}^\top \widetilde{\mat{P}}\,.
\]
Let $\epsilon \in (0,1), \widehat{\epsilon} \in [0,1/\beta^2)$ and
\ApproximateSolve be an algorithm that for any
$\widehat{\mat{x}}\in\R^{R}$ and $\mat{f}=\widetilde{\mat{q}}+(\mat{A}-\widetilde{\mat{P}})\, \widehat{\mat{x}}$,
computes $\overline{\mat{x}} \in \R^{R}$ in time $O(T)$ such that
\[
\norm{\mat{A}\overline{\mat{x}}-\mat{f}}_2^2\leq (1+\widehat{\epsilon})\,\min_{\mat{x}}\, \norm{\mat{A}\mat{x}-\mat{f}}_2^2\,.
\]
Then, Algorithm~\ref{alg:approximate-lifting} returns an approximate solution
$\widetilde{\mat{x}} \in \R^{R}$, using  \ApproximateSolve as a subroutine, such that
\begin{align*}
    \norm{\widetilde{\mat{P}} \widetilde{\mat{x}}-\widetilde{\mat{q}}}_2^2
    &\leq
    \parens*{1 + \frac{2 \widehat{\epsilon}}{(\sfrac{1}{\beta} - \sqrt{\widehat{\epsilon}})^2}}\, \min_{\mat{x}}\, \norm{\widetilde{\mat{P}} \mat{x}-\widetilde{\mat{q}}}_2^2 
    + \epsilon\, \norm{\widetilde{\mat{P}}\,(\widetilde{\mat{P}}^\top \widetilde{\mat{P}})^{-1} \widetilde{\mat{P}}^\top \widetilde{\mat{q}}}_2^2\,,
\end{align*}
in $O\parens*{\frac{\beta}{1-\sqrt{\widehat{\epsilon}} \beta} \cdot  T \log \sfrac{\beta}\epsilon}$ time.
\end{restatable}

\begin{proof}
We show that Algorithm~\ref{alg:approximate-lifting} gives the desired output.
Suppose \ApproximateSolve yields $\mat{x}^{(k+1)}$ for given inputs $\mat{A},\widetilde{\mat{P}},\widetilde{\mat{q}}$, and $\widetilde{\mat{q}}^{(k)}$ (i.e., $\widehat{\mat{x}} \gets \mat{x}^{(k)}$, $\mat{f}\gets \widetilde{\mat{q}}^{(k)}$, and $\overline{\mat{x}} \gets \mat{x}^{(k+1)}$), which satisfies
\[
    \norm{\mat{A}\mat{x}^{(k+1)}-\widetilde{\mat{q}}^{(k)}}_2^2\leq (1+\widehat{\epsilon})\,\min_{\mat{x}} \norm{\mat{A}\mat{x}-\widetilde{\mat{q}}^{(k)}}_2^2
    = (1+\widehat{\varepsilon})\,\norm{\pi_{\mat A ^\perp}\widetilde{\mat{q}}^{(k)}}_2^2\,.
\]
We can also decompose the LHS using $\widetilde{\mat{q}}^{(k)} = \pi_{\mat A} \widetilde{\mat{q}}^{(k)} + \pi_{\mat A ^\perp} \widetilde{\mat{q}}^{(k)}$ as follows:
\[
    \norm{\mat{A}\mat{x}^{(k+1)}-\widetilde{\mat{q}}^{(k)}}_2^2
    = 
    \norm{\mat{A}\mat{x}^{(k+1)} - \pi_{\mat{A}}\widetilde{\mat{q}}^{(k)}}_2^2 + \norm{\pi_{\mat{A}^{\perp}} \widetilde{\mat{q}}^{(k)}}_2^2\,.
\]
Combining the above, we get
\[
    \norm{\mat{A}\mat{x}^{(k+1)} - \pi_{\mat{A}}\widetilde{\mat{q}}^{(k)}}_2^2
    \leq
    \widehat{\epsilon}\, \norm{\pi_{\mat{A}^{\perp}}\widetilde{\mat{q}}^{(k)}}_2^2\,.
\]
Denoting $\mat{x}^* = (\widetilde{\mat{P}}^\top \widetilde{\mat{P}})^{-1} \widetilde{\mat{P}}^\top \widetilde{\mat{q}} = \argmin_{\mat x}\,\norm{\mat{Bx}-\widetilde{\mat{q}}}_2$ and using the triangle inequality,
\begin{align}
\nonumber
    \norm{\mat{A}\mat{x}^{(k+1)} - \mat{A}\mat{x}^*}_2
    &
    \leq \norm{\mat{A}\mat{x}^{(k+1)} - \pi_{\mat A}\widetilde{\mat{q}}^{(k)}}_2 + \norm{\pi_{\mat A}\widetilde{\mat{q}}^{(k)} - \mat{A} \mat{x}^*}_2
    \\ & \leq \label{eq:total-bound-on-a-norm}
    \sqrt{\widehat{\epsilon}}\, \norm{\pi_{\mat{A}^{\perp}}\widetilde{\mat{q}}^{(k)}}_2 + \norm{\pi_{\mat A}\widetilde{\mat{q}}^{(k)} - \mat{A} \mat{x}^*}_2\,.
\end{align}

We now bound each term in the RHS. As for the second term, since $\widetilde{\mat{q}}^{(k)} = \widetilde{\mat{q}} + (\mat{A} - \widetilde{\mat{P}})\, \mat{x}^{(k)}$ and $(\widetilde{\mat{P}}-\mat{A})^\top \begin{bmatrix}
\widetilde{\mat{P}} & \widetilde{\mat{q}} \end{bmatrix} = 0$, by \Cref{lemma:richardson-simulation},
\begin{align*}
    (\mat{A}^\top \mat{A})^{-1} \mat{A}^\top\widetilde{\mat{q}}^{(k)}
    &=
    \argmin_{\mat x}\,\norm{\mat{Ax}-\widetilde{\mat{q}}^{(k)}}_2 \\
    &=
    \mat{x}^{(k)} - (\mat{A}^\top \mat{A})^{-1} (\widetilde{\mat{P}}^\top\widetilde{\mat{P}}\mat{x}^{(k)} - \widetilde{\mat{P}}^\top\widetilde{\mat{q}})\,,
\end{align*}
which is exactly a Richardson iteration with preconditioner $\mat M\gets \mat{A}^\top \mat{A}$ in \Cref{lem:richardson_iteration} (satisfying $\widetilde{\mat{P}}^\top \widetilde{\mat{P}} \preceq \mat{A}^\top \mat{A}\preceq \beta\,\widetilde{\mat{P}}^\top \widetilde{\mat{P}}$). 
Thus, $\norm{(\mat{A}^\top \mat{A})^{-1} \mat{A}^\top\widetilde{\mat{q}}^{(k)} - \mat{x}^*}_{\mat{A}^\top \mat{A}} \leq (1-\beta^{-1})\,\norm{\mat{x}^{(k)} - \mat{x}^*}_{\mat{A}^\top \mat{A}}$, and
\begin{equation}
    \label{eq:second-term-bound-on-a-norm}
    \norm{\pi_{\mat A}\widetilde{\mat{q}}^{(k)} - \mat{A} \mat{x}^*}_2
    \leq
    \parens*{1-\frac 1 \beta}\, \norm{\mat{A} \mat{x}^{(k)} - \mat{A}\mat{x}^*}_2\,.
\end{equation}

Regarding the first term in \eqref{eq:total-bound-on-a-norm}, since $\mat{Ax}^{(k)}$ is in the column space of $\mat{A}$,
\[
\pi_{\mat A^\perp}\widetilde{\mat{q}}^{(k)} 
= 
\pi_{\mat A^\perp}\bigl(\widetilde{\mat{q}} +(\mat{A-B})\,\mat x^{(k)}\bigr)
=
\pi_{\mat A^\perp}(\widetilde{\mat{q}} - \widetilde{\mat{P}} \mat{x}^{(k)})\,.
\]
Therefore,
\begin{align*}
\norm{\pi_{\mat A^\perp}\widetilde{\mat{q}}^{(k)}}_2^2
& \leq 
\norm{\widetilde{\mat{q}} - \widetilde{\mat{P}} \mat{x}^{(k)}}_2^2
\\ & =
\norm{\widetilde{\mat{P}} \mat{x}^{*} - \widetilde{\mat{P}} \mat{x}^{(k)}}_2^2 + \min_{\mat{x}}\, \norm{\widetilde{\mat{P}} \mat{x}-\widetilde{\mat{q}}}_2^2
\\ & \leq 
\norm{\mat{A} \mat{x}^{*} - \mat{A} \mat{x}^{(k)}}_2^2 + \min_{\mat{x}}\, \norm{\widetilde{\mat{P}} \mat{x}-\widetilde{\mat{q}}}_2^2\,,
\end{align*}
where the last inequality follows from $\widetilde{\mat{P}}^\top \widetilde{\mat{P}} \preceq \mat{A}^\top \mat{A}$.
Thus,
\begin{equation}
\label{eq:first-term-bound-on-a-norm}
\norm{\pi_{\mat A^\perp}\widetilde{\mat{q}}^{(k)}}_2
\leq 
\norm{\mat{A} \mat{x}^{*} - \mat{A} \mat{x}^{(k)}}_2 + \min_{\mat{x}}\, \norm{\widetilde{\mat{P}} \mat{x}-\widetilde{\mat{q}}}_2\,.
\end{equation}

Combining \eqref{eq:total-bound-on-a-norm}, \eqref{eq:second-term-bound-on-a-norm}, and \eqref{eq:first-term-bound-on-a-norm}, we have
\[
    \norm{\mat{A}\mat{x}^{(k+1)} - \mat{A}\mat{x}^*}_2
    \leq
    \parens*{1-\frac{1}{\beta} + \sqrt{\widehat{\epsilon}}}\, \norm{\mat{A} \mat{x}^{*} - \mat{A} \mat{x}^{(k)}}_2 + \sqrt{\widehat{\epsilon}}\, \min_{\mat{x}}\, \norm{\widetilde{\mat{P}} \mat{x}-\widetilde{\mat{q}}}_2\,.
\]
Denoting $\alpha=1-\frac{1}{\beta} + \sqrt{\widehat{\epsilon}}\,$, by induction, we have
\begin{align}
\nonumber
\norm{\mat{A}\mat{x}^{(k)} - \mat{A}\mat{x}^*}_2 
& \leq 
\alpha^k\, \norm{\mat{A} \mat{x}^{*} - \mat{A} \mat{x}^{(0)}}_2 + (1+\alpha+\alpha^2 + \cdots + \alpha^{k-1}) \times \sqrt{\widehat{\epsilon}}\,\min_{\mat{x}}\, \norm{\widetilde{\mat{P}} \mat{x}-\widetilde{\mat{q}}}_2
\\ & =
\label{eq:col-space-bound-for-a}
\alpha^k\, \norm{\mat{A} \mat{x}^{*} - \mat{A} \mat{x}^{(0)}}_2 + \frac{1-\alpha^k}{1-\alpha}\times \sqrt{\widehat{\epsilon}}\, \min_{\mat{x}}\, \norm{\widetilde{\mat{P}} \mat{x}-\widetilde{\mat{q}}}_2\,.
\end{align}

We also have
\begin{align}
\nonumber
\norm{\widetilde{\mat{P}} \mat{x}^{(k)}-\widetilde{\mat{q}}}_2^2 
&= \norm{\widetilde{\mat{P}} \mat{x}^{(k)} - \pi_{\widetilde{\mat{P}}}\widetilde{\mat{q}}}_2^2 + \norm{\pi_{\widetilde{\mat{P}}^\perp}\widetilde{\mat{q}}}_2^2
\\ & = 
\label{eq:total-error}
\norm{\widetilde{\mat{P}} \mat{x}^{(k)}-\widetilde{\mat{P}} \mat{x}^*}_2^2 + \min_{\mat{x}}\, \norm{\widetilde{\mat{P}} \mat{x} - \widetilde{\mat{q}}}_2^2\,.
\end{align}
We then bound the first term by using $\widetilde{\mat{P}}^\top \widetilde{\mat{P}} \preceq \mat{A}^\top \mat{A} \preceq \beta\, \widetilde{\mat{P}}^\top \widetilde{\mat{P}}$ and \eqref{eq:col-space-bound-for-a} as follows:
\begin{align}
    \nonumber
    \norm{\widetilde{\mat{P}} \mat{x}^{(k)}-\widetilde{\mat{P}} \mat{x}^*}_2^2 
    &\leq
    \norm{\mat{A} \mat{x}^{(k)}-\mat{A} \mat{x}^*}_2^2 \\
    &\leq
    \nonumber
    2\alpha^{2k}\, \norm{\mat{A} \mat{x}^{*} - \mat{A} \mat{x}^{(0)}}_2^2 + 2\, \parens*{\frac{1-\alpha^k}{1-\alpha}}^2 \times \widehat{\epsilon}\, \min_{\mat{x}}\, \norm{\widetilde{\mat{P}} \mat{x}-\widetilde{\mat{q}}}_2^2 \\
    &\leq 
    \nonumber
    2 \beta\alpha^{2k}\, \norm{\widetilde{\mat{P}} \mat{x}^{*} - \widetilde{\mat{P}} \mat{x}^{(0)}}_2^2 + 2\, \parens*{\frac{1-\alpha^k}{1-\alpha}}^2 \times \widehat{\epsilon}\, \min_{\mat{x}}\, \norm{\widetilde{\mat{P}} \mat{x}-\widetilde{\mat{q}}}_2^2\,.
\end{align}
Putting this bound back into \eqref{eq:total-error},
\[
\norm{\widetilde{\mat{P}} \mat{x}^{(k)}-\widetilde{\mat{q}}}_2^2 \leq 2 \beta\alpha^{2k}\, \norm{\widetilde{\mat{P}} \mat{x}^{*} - \widetilde{\mat{P}} \mat{x}^{(0)}}_2^2 + \parens*{1 + 2\widehat{\epsilon}\, \parens*{\frac{1-\alpha^k}{1-\alpha}}^2}\, \min_{\mat{x}}\, \norm{\widetilde{\mat{P}} \mat{x}-\widetilde{\mat{q}}}_2^2\,.
\]

Setting
\[
k = \ceil*{\frac{\log(\sfrac{2\beta}{\epsilon})}{2\,(\nicefrac 1 \beta -\sqrt{\widehat{\epsilon}})}}\,,
\]
we have
\[
    \norm{\widetilde{\mat{P}} \mat{x}^{(k)}-\widetilde{\mat{q}}}_2^2
    \leq
    \epsilon\, \norm{\widetilde{\mat{P}} \mat{x}^{*} - \widetilde{\mat{P}} \mat{x}^{(0)}}_2^2 + \parens*{1+ \frac{2 \widehat{\epsilon}}{(\nicefrac 1  \beta - \sqrt{\widehat{\epsilon}})^2} }\, \min_{\mat{x}}\, \norm{\widetilde{\mat{P}} \mat{x}-\widetilde{\mat{q}}}_2^2\,,
\]
which completes the proof with $\mat{x}^{(0)}=\boldsymbol{0}$.
\end{proof}

\begin{remark}
To better understand the theorem, observe that
$\widetilde{\mat{P}}^\top \widetilde{\mat{P}} = \mat{P}^\top \mat{P}$ is a $\beta$-spectral approximation of $\mat{A}^\top \mat{A}$,
$\varepsilon$ controls the reducible error $\varepsilon\, \norm{\widetilde{\mat{P}} \mat{x}^*}_{2}^2$,
and $(1 + \widehat{\varepsilon})$ is the error in the approximate least-square update for each $\mat{x}^{(k)}$.
\end{remark}

\paragraph{Bounding $\beta$.}
First, observe that in the case of TD,
we have $\mat{P} = \mat{A}$, so $\beta = 1$.
More generally, if $\textnormal{rank}(\mat{A}) = s \le \min\{I, R\}$
and $\mat{A} = \mat{U}\mat{\Sigma}\mat{V}^\top$ is a compressed SVD, then $\mat{A}$ is said to satisfy  the \emph{standard incoherence condition} with parameter $\mu$ \citep{chen2015incoherence} if
\[
    \max_{i\in[I]} \norm*{\mat{e}_{i}^\top \mat{U}}_2 \leq \sqrt{\frac{\mu s}{I}}\,,
    \quad
    \max_{r\in[R]} \norm*{\mat{V} ^\top\mat{e}_r}_2 \leq \sqrt{\frac{\mu s}{R}}\,.
\]
The $\norm{\mat{e}_{i}^\top \mat{U}}_2^2$ and $\norm{\mat{V} ^\top\mat{e}_r}_2^2$ values are the leverage scores of the rows and columns of $\mat{A}$.
Applying \citet[Lemma 4]{cohen2015uniform},
if each row of $\mat{A}$ is observed with probability $p$ such that $p \geq \frac{c\mu s \log s}{I}$ for some absolute constant $c$, then 
\[
    \frac{1}{2} \,\mat{A}^\top \mat{A}
    \preceq
    \frac{1}{p} \, \mat{A}_{\Omega}^\top \mat{A}_{\Omega}
    \preceq
    \frac{3}{2}\, \mat{A}^\top \mat{A}\,,
\]
which gives $\beta=2/p$.
Let $\zeta = \max_{i\in[I]} \norm{\mat{a}_i}_2$, where $\mat{a}_i$ is row $i$ of $\mat{A}$.
Then, the $\alpha \zeta^2$-ridge leverage scores of $\mat{A}$ (i.e., $\mat{a}_i^\top (\mat{A}^\top \mat{A} + \alpha \zeta^2 \cdot\mat{I}_{R})^{-1} \mat{a}_i$), for $\alpha\geq 1$, are at most $1/\alpha$. If $p$ is the observation rate, taking $\alpha = \frac{c\log s}{p}$ gives the required incoherence condition. This can be done by introducing an $\ell_2$-regularization term to the TC optimization problem (i.e., solving a ridge regression problem in each ALS step).
Note that usually $\alpha$ can be chosen to be much smaller in practice.

\section{Sampling Methods for Tensor Completion}
\label{sec:sampling-for-completion}
We are now ready to efficiently solve the \emph{unstructured} least-squares problem \eqref{eqn:input_regression} induced by ALS for tensor completion, i.e., for $\mat{P}\in\R^{|\Omega|\times R}$ and observations $\mat{q} \in \R^{|\Omega|}$, find
\[
    \mat{x}^* = \argmin_{\mat{x} \in \R^R} \,\norm{\mat{P} \mat{x} - \mat{q}}_{2}^2\,.
\]
As in Algorithm~\ref{alg:approximate-lifting},
we lift this problem to higher dimension to get a structured design matrix $\mat{A}$,
and use a known fast algorithm for approximately solving the structured least-squares problem in each step of approximate-mini-ALS.
For a given $\widehat{\epsilon} \in (0,1/\beta^2)$, the approximate solver computes a solution $\overline{\mat{x}}\in \R^R$ in time $O(T_{\widehat{\epsilon}})$ such that 
\[
\norm{\mat A \overline{\mat x} - \mat b}_2^2 \leq (1+\widehat \epsilon)\,\min_{\mat x} \norm{\mat A \mat x -\mat b}_2^2\,.
\]
Therefore, for a desired $\varepsilon_1 \in (0,1)$, we set $\widehat \epsilon = \Theta(\varepsilon_1 / \beta^2)$ and use a sufficiently small $\epsilon \gets \varepsilon_2$ in \cref{thm:approximate-richardson}.
Putting everything together, Algorithm~\ref{alg:approximate-lifting} finds an approximate solution $\widetilde{\mat x}\in \R^R$ in time $O(\beta T_{\varepsilon_1\beta^{-2}}\log\frac{\beta}{\epsilon_2})$ that satisfies
\begin{align}
    \norm{\mat P \widetilde{\mat x} - \mat q}_2^2
    &\leq
    (1+\varepsilon_1)\,\norm{\pi_{\mat P^\perp }\mat q}_2^2 + \varepsilon_2\, \norm{\mat \pi_{\mat P}\mat q}_2^2\,,\label{eq:tc-approx-sol}
\end{align}
where $\mat{\pi}_{\mat{P}}$ and $\mat{\pi}_{\mat{P}^\perp}$ are the orthogonal projection matrices into the column space and null space of $\mat{P}$, respectively~(see \Cref{app:least-square-regression}).
With this in hand, we now present the running times of our lifted iterative method for TC problems
by combining \cref{thm:approximate-richardson} with state-of-the-art tensor decomposition results based on leverage score sampling.

\subsection{CP Completion}
\label{subsec:CP-completion}
Each ALS update step for CP completion solves a regression problem where the design matrix is the Khatri--Rao product:
for $\mat{A}^{(k)} \in \R^{I_k \times R}$,
$\mat A^{\neq k} := \bigodot_{n=1,n\neq k}^N \mat A^{(n)} \in \R^{I_{\neq k}\times R}$,
and $\mat Q = (\mat X^\top_{(k)})_\Omega \in \R^{|\Omega| \times I_k}$,
\[
    \mat A^{(k)}
    \gets
    \argmin_{\mat{A} \in \R^{I_k \times R}} \, \bigl\lVert  (\mat A^{\neq k})_\Omega \,\mat{A}^\top - \mat Q \bigr\rVert_{\frobenius}\,.
\]
The design matrix $(\mat A^{\neq k})_\Omega$ does not necessarily have any structure,
so a direct method relies on solving the normal equation, which takes $O(R^\omega +R |\Omega|(R+I_k))$ time.
Thus, the running time of \emph{one round} of ALS, i.e., updating all $N$ factors,
is
\[
    O\parens*{N\parens*{R^\omega +R^2 |\Omega|}+ R |\Omega|\sum_{n=1}^N I_n}.
\]

Previous work on CP tensor decomposition \citep{cheng2016spals, larsen2022practical, bharadwaj2023fast}
developed fast methods for efficiently computing the leverage scores of a Khatri--Rao product matrix.
In particular, \citet{bharadwaj2023fast} designed a data structure for computing and maintaining the
leverage scores of $\mat A^{\neq k}$ during ALS updates. 
This approach requires sampling $\tilde{O}(R/\varepsilon)$ rows of $\mat A^{\neq k}$.
Due to the Khatri--Rao product structure, each row of $\mat A^{\neq k}$ can be mapped to a sequence of one choice from the rows of each $\mat{A}^{(n)}$ for $n\in [N]\backslash \{k\}$.
Hence, sampling a row from $\mat{A}^{\neq k}$ is equivalent to the following:
for each $n \in [N]\backslash\{k\}$, sample a row from $\mat A^{(n)}$ according to some conditional distribution given sampled rows from $\mat{A}^{(1)}, \dots, \mat{A}^{(n-1)}$, and then compute the Hadamard product of $N-1$ sampled rows.
Maintaining the full $I_n$-dimensional vector for a conditional probability for each $n$ is costly,
so \citet{bharadwaj2023fast} developed a binary tree-based data structure to speed up leverage-score sampling for $\mat A^{\neq k}$.

Applying \citet[Corollary 3.3]{bharadwaj2023fast}, one round of ALS runs in time
$
    \tilde{O} \parens{ \varepsilon^{-1} \sum_{n=1}^N \parens*{ I_n R^2 + NR^3} }.
$
Using their CP TD algorithm as the approximate solver in Algorithm~\ref{alg:approximate-lifting}, and combining its guarantee with \cref{thm:approximate-richardson},
we can extend their approach to CP completion.

\begin{corollary}
There is an ALS CP completion algorithm such that
(i) after a factor matrix update, each row of $\mat{A}^{(n)}$ satisfies \eqref{eq:tc-approx-sol},
and (ii) the total running time of one round is
\[
    \tilde{O} \parens*{\frac{\beta^2}{\varepsilon_1} \sum_{n=1}^N \parens*{I_n R^2 + NR^3} \log \frac{1}{\epsilon_2}}\,.
\]
\end{corollary}

\noindent
There is \emph{no dependence} on $|\Omega|$ in the running time due to leverage score sampling, i.e., it runs in sublinear time.

\subsection{Tucker Completion}

\citet{fahrbach2022subquadratic} designed block-sketching techniques and fast Kronecker product-matrix multiplication algorithms
to exploit the ALS structure for Tucker decomposition.

\subsubsection{Core Tensor Update}
Recall that for a Tucker decomposition we use the notation $I=\prod_{n\in[N]} I_n$ and $R=\prod_{n\in[N]} R_n$.
The ALS core tensor update in the Tucker completion problem is
\[
    \tensor{G}
    \leftarrow
    \argmin_{\tensor{G}' \in\R^{R_1 \times \cdots \times R_N}} \norm*{\parens*{\bigotimes_{n=1}^N\mat{A}^{(n)}}_{\Omega} \hspace{-0.2cm}\vvec(\tensor{G}') - \vvec(\tensor{X})_{\Omega}}_2.
\]
The design matrix above restricted to $\Omega$ is exactly $\mat P$ in our general setup.

We compare the running times of the direct method and our lifting approach.
In the former, we can compute an exact solution to a least-squares problem $\mat x^* = (\mat P^\top \mat P)^{-1} \mat P^\top \mat q$ in time $O(|\Omega| R^2 + R^\omega)$.

To achieve a fast lifted method, we solve the second step of Algorithm~\ref{alg:approximate-lifting}
using the leverage score sampling-based
core tensor update algorithm in \citep[Theorem 1.2]{fahrbach2022subquadratic}
with running time
\[
    \tilde{O} \parens*{
        \sum_{n=1}^N \parens*{ I_n R_n  + \frac{R_n^\omega N^2}{\varepsilon^2}} + \frac{R^{2-\theta^*}}{\varepsilon}
    }\,,
\]
where $\theta^*>0$ is an optimizable constant depending on $\{R_n\}_{n\in[N]}$.
Using this as the \ApproximateSolve subroutine in \Cref{thm:approximate-richardson}, we achieve the following.

\begin{corollary}
\label{cor:fast_tucker_completion_core_tensor_update}
There is an algorithm that computes an ALS Tucker completion core tensor update satisfying \eqref{eq:tc-approx-sol}
in time
\begin{equation}
\label{eqn:fast_tucker_completion_core_tensor_update}
    \tilde{O}\parens*{\parens*{
         \sum_{n=1}^N \parens*{I_nR_n  + \beta^4 R_n^\omega N^2 \varepsilon_1^{-2}}
         +
         \frac{\beta^2 R^{2-\theta^*}}{\varepsilon_1}} \beta\log\frac{1}{\varepsilon_2}
    }\,.
\end{equation}
\end{corollary}

\subsubsection{Factor Matrix Update}
The ALS factor matrix update for $\mat{A}^{(k)}$ in the Tucker completion problem is
\[
    \mat{A}^{(k)}
    \leftarrow
    \hspace{-0.1cm}
    \argmin_{\mat{A}\in \R^{I_k \times R_k}} \norm*{ \parens*{
        \parens*{\bigotimes_{n=1,n\neq k}^N \hspace{-0.10cm} \mat{A}^{(n)} } \mat{G}_{(k)}^\top }_\Omega  \hspace{-0.10cm} \mat{A}^\top \hspace{-0.10cm} - \mat Q}_\frobenius,
\]
where $\mat Q= (\mat X^\top_{(k)})_\Omega \in \R^{|\Omega| \times I_k}$ is a sparse matrix of observations.
The running time of a direct method that solves the normal equation is
$O(R_k^\omega + R_k |\Omega| (R_{\neq k} + R_k + I_k))$, where $R_{\neq k} = R/R_k$.

The running time of the sampling-based factor-matrix update for $\mat{A}^{(k)}$ in \citet[Theorem 1.2]{fahrbach2022subquadratic}
for the full decomposition problem is
\[
    \tilde{O}\parens*{
        \sum_{n=1}^N \parens*{I_nR_n + \frac{R_n^\omega N^2}{\varepsilon^2} + I_k R R_n} + \frac{I_k R_{\neq k}^{2-\theta^*}}{\varepsilon}
    }\,.
\]

\noindent
Combining this result with \cref{thm:approximate-richardson},
Algorithm~\ref{alg:approximate-lifting} has the following running time for a factor matrix update.

\begin{corollary}
There is an algorithm that computes an ALS Tucker completion factor matrix update for $\mat{A}^{(k)}$,
with each row of $\mat{A}^{(k)}$ satisfying \eqref{eq:tc-approx-sol},
in time
\[
\textstyle
    \tilde{O} \parens*{
        \parens*{
            \sum_{n=1}^N \parens*{I_nR_n  + \beta^4 R_n^\omega N^2 \epsilon_1^{-2}}
            +
            \frac{\beta^2 I_k R_{\neq k}^{2-\theta^*}}{\epsilon_1} + I_k R\sum_{n=1}^N R_n } \beta \log \frac{1}{\epsilon_2}
    }\,.
\]
\end{corollary}

\subsection{TT Completion}

Each ALS step for TT decomposition solves the following least-squares problem
with a Kronecker product-type design matrix:
for $\mat{A}^{\neq k} := \mat{A}_{< k}\otimes \mat{A}^\top_{> k} \in \R^{I_{\neq k} \times (R_{k-1}R_k)}$ and $\mat Q = (\mat{X}_{(k)}^\top)_\Omega \in \R^{|\Omega| \times I_k}$,
\[
    \tensor{A}^{(k)}
    \gets
    \argmin_{\tensor{B} \in \R^{R_{k-1}\times I_k \times R_k}}
    \norm*{\parens*{\mat{A}^{\neq k}}_\Omega\, (\mat{B}_{(2)})^\top - \mat Q }_{\frobenius}\,.
\]
Solving this directly with the normal equation takes
$O(\bar R_k^\omega +\bar R_k |\Omega| (\bar R_k + I_k))$ time for $\bar R_k := R_{k-1} R_k$.
Thus, the time for one round of ALS is
\[
    O\parens*{
        \sum_{n=1}^N \parens*{\bar R_n^\omega + \bar R_n |\Omega| \parens*{\bar R_n +I_n}}
    }.
\]

\citet{bharadwaj2024efficient} proposed a sampling-based ALS algorithm that crucially relies a \emph{canonical form} of the TT decomposition with respect to the index $k$. Any TT decomposition can be converted to this form through a QR decomposition, and this form ensures that $(\mat A^{\neq k})^\top \mat A^{\neq k} = I_{R_{k-1} R_k}$.
It follows that the leverage scores of $\mat A^{\neq k}$ are simply the diagonal entries of $\mat A^{\neq k} (\mat A^{\neq k})^\top = (\mat A_{< k}\mat A_{<k}^\top) \otimes (\mat A_{> k}^\top\mat A_{>k})$.

It follows from properties of the Kronecker product~\citep{diao2019optimal} that
$
    \ell_{i^{\neq k}}(\mat A^{\neq k})
    =
    \ell_{i_{< k}}(\mat A_{< k}) \cdot \ell_{i_{> k}}(\mat A^\top_{> k})\,,
$
where $i^{\neq k} = \underline{i_1\cdots i_{k-1} i_{k+1}\cdots i_N}$, $i_{< k} = \underline{i_1\cdots i_{k-1}}$, and $i_{> k} = \underline{i_{k+1}\cdots i_N}$ (see \Cref{app:tt-decomposition-details} for notation).
Therefore, efficient leverage score sampling for $\mat A^{\neq k}$ reduces to that for $\mat A_{< k}$ and $\mat A_{> k}$.
To this end, \citet{bharadwaj2024efficient} adopt an approach similar to \citet{bharadwaj2023fast} for leverage score-based CP decomposition.
Each row of $\mat A_{< k}$ corresponds to a series of one slice for each third-order tensor $\mat A^{(n)}$ for $n<k$, which results in a series of conditional sampling steps using a data structure adapted from the one used for CP decomposition.
In contrast, \citet[Corollary 4.4]{bharadwaj2024efficient} show that one round of approximate TT-core updates,
if $R_n = R$ for all $n \in [N-1]$, can run in time
$\tilde{O} \parens{ R^4 \varepsilon^{-1} \sum_{n = 1}^N \parens*{N + I_n}}$,
which leads to the following result.

\begin{corollary}
There is an ALS TT completion algorithm such that
(i) after a TT-core update, each row fiber of $\tensor{A}^{(n)}$
satisfies \eqref{eq:tc-approx-sol},
and (ii) the total running time of one round is
\[
    \tilde{O} \parens*{
        \frac{\beta^2 R^4}{\epsilon_1} \sum_{n=1}^N \parens*{N + I_n} \log\frac{1}{\epsilon_2}
    }\,.
\]
\end{corollary}

\section{Experiments}
\label{sec:experiments}
The experiments are designed to address two key research questions.
First, \textbf{RQ1} evaluates whether the average $L_2$-norm of the counterfactual perturbation vectors ($\overline{||\perturb||}$) decreases as the model overfits the data, thereby providing further empirical validation for our hypothesis.
Second, \textbf{RQ2} evaluates the ability of the proposed counterfactual regularized loss, as defined in (\ref{eq:regularized_loss2}), to mitigate overfitting when compared to existing regularization techniques.

% The experiments are designed to address three key research questions. First, \textbf{RQ1} investigates whether the mean perturbation vector norm decreases as the model overfits the data, aiming to further validate our intuition. Second, \textbf{RQ2} explores whether the mean perturbation vector norm can be effectively leveraged as a regularization term during training, offering insights into its potential role in mitigating overfitting. Finally, \textbf{RQ3} examines whether our counterfactual regularizer enables the model to achieve superior performance compared to existing regularization methods, thus highlighting its practical advantage.

\subsection{Experimental Setup}
\textbf{\textit{Datasets, Models, and Tasks.}}
The experiments are conducted on three datasets: \textit{Water Potability}~\cite{kadiwal2020waterpotability}, \textit{Phomene}~\cite{phomene}, and \textit{CIFAR-10}~\cite{krizhevsky2009learning}. For \textit{Water Potability} and \textit{Phomene}, we randomly select $80\%$ of the samples for the training set, and the remaining $20\%$ for the test set, \textit{CIFAR-10} comes already split. Furthermore, we consider the following models: Logistic Regression, Multi-Layer Perceptron (MLP) with 100 and 30 neurons on each hidden layer, and PreactResNet-18~\cite{he2016cvecvv} as a Convolutional Neural Network (CNN) architecture.
We focus on binary classification tasks and leave the extension to multiclass scenarios for future work. However, for datasets that are inherently multiclass, we transform the problem into a binary classification task by selecting two classes, aligning with our assumption.

\smallskip
\noindent\textbf{\textit{Evaluation Measures.}} To characterize the degree of overfitting, we use the test loss, as it serves as a reliable indicator of the model's generalization capability to unseen data. Additionally, we evaluate the predictive performance of each model using the test accuracy.

\smallskip
\noindent\textbf{\textit{Baselines.}} We compare CF-Reg with the following regularization techniques: L1 (``Lasso''), L2 (``Ridge''), and Dropout.

\smallskip
\noindent\textbf{\textit{Configurations.}}
For each model, we adopt specific configurations as follows.
\begin{itemize}
\item \textit{Logistic Regression:} To induce overfitting in the model, we artificially increase the dimensionality of the data beyond the number of training samples by applying a polynomial feature expansion. This approach ensures that the model has enough capacity to overfit the training data, allowing us to analyze the impact of our counterfactual regularizer. The degree of the polynomial is chosen as the smallest degree that makes the number of features greater than the number of data.
\item \textit{Neural Networks (MLP and CNN):} To take advantage of the closed-form solution for computing the optimal perturbation vector as defined in (\ref{eq:opt-delta}), we use a local linear approximation of the neural network models. Hence, given an instance $\inst_i$, we consider the (optimal) counterfactual not with respect to $\model$ but with respect to:
\begin{equation}
\label{eq:taylor}
    \model^{lin}(\inst) = \model(\inst_i) + \nabla_{\inst}\model(\inst_i)(\inst - \inst_i),
\end{equation}
where $\model^{lin}$ represents the first-order Taylor approximation of $\model$ at $\inst_i$.
Note that this step is unnecessary for Logistic Regression, as it is inherently a linear model.
\end{itemize}

\smallskip
\noindent \textbf{\textit{Implementation Details.}} We run all experiments on a machine equipped with an AMD Ryzen 9 7900 12-Core Processor and an NVIDIA GeForce RTX 4090 GPU. Our implementation is based on the PyTorch Lightning framework. We use stochastic gradient descent as the optimizer with a learning rate of $\eta = 0.001$ and no weight decay. We use a batch size of $128$. The training and test steps are conducted for $6000$ epochs on the \textit{Water Potability} and \textit{Phoneme} datasets, while for the \textit{CIFAR-10} dataset, they are performed for $200$ epochs.
Finally, the contribution $w_i^{\varepsilon}$ of each training point $\inst_i$ is uniformly set as $w_i^{\varepsilon} = 1~\forall i\in \{1,\ldots,m\}$.

The source code implementation for our experiments is available at the following GitHub repository: \url{https://anonymous.4open.science/r/COCE-80B4/README.md} 

\subsection{RQ1: Counterfactual Perturbation vs. Overfitting}
To address \textbf{RQ1}, we analyze the relationship between the test loss and the average $L_2$-norm of the counterfactual perturbation vectors ($\overline{||\perturb||}$) over training epochs.

In particular, Figure~\ref{fig:delta_loss_epochs} depicts the evolution of $\overline{||\perturb||}$ alongside the test loss for an MLP trained \textit{without} regularization on the \textit{Water Potability} dataset. 
\begin{figure}[ht]
    \centering
    \includegraphics[width=0.85\linewidth]{img/delta_loss_epochs.png}
    \caption{The average counterfactual perturbation vector $\overline{||\perturb||}$ (left $y$-axis) and the cross-entropy test loss (right $y$-axis) over training epochs ($x$-axis) for an MLP trained on the \textit{Water Potability} dataset \textit{without} regularization.}
    \label{fig:delta_loss_epochs}
\end{figure}

The plot shows a clear trend as the model starts to overfit the data (evidenced by an increase in test loss). 
Notably, $\overline{||\perturb||}$ begins to decrease, which aligns with the hypothesis that the average distance to the optimal counterfactual example gets smaller as the model's decision boundary becomes increasingly adherent to the training data.

It is worth noting that this trend is heavily influenced by the choice of the counterfactual generator model. In particular, the relationship between $\overline{||\perturb||}$ and the degree of overfitting may become even more pronounced when leveraging more accurate counterfactual generators. However, these models often come at the cost of higher computational complexity, and their exploration is left to future work.

Nonetheless, we expect that $\overline{||\perturb||}$ will eventually stabilize at a plateau, as the average $L_2$-norm of the optimal counterfactual perturbations cannot vanish to zero.

% Additionally, the choice of employing the score-based counterfactual explanation framework to generate counterfactuals was driven to promote computational efficiency.

% Future enhancements to the framework may involve adopting models capable of generating more precise counterfactuals. While such approaches may yield to performance improvements, they are likely to come at the cost of increased computational complexity.


\subsection{RQ2: Counterfactual Regularization Performance}
To answer \textbf{RQ2}, we evaluate the effectiveness of the proposed counterfactual regularization (CF-Reg) by comparing its performance against existing baselines: unregularized training loss (No-Reg), L1 regularization (L1-Reg), L2 regularization (L2-Reg), and Dropout.
Specifically, for each model and dataset combination, Table~\ref{tab:regularization_comparison} presents the mean value and standard deviation of test accuracy achieved by each method across 5 random initialization. 

The table illustrates that our regularization technique consistently delivers better results than existing methods across all evaluated scenarios, except for one case -- i.e., Logistic Regression on the \textit{Phomene} dataset. 
However, this setting exhibits an unusual pattern, as the highest model accuracy is achieved without any regularization. Even in this case, CF-Reg still surpasses other regularization baselines.

From the results above, we derive the following key insights. First, CF-Reg proves to be effective across various model types, ranging from simple linear models (Logistic Regression) to deep architectures like MLPs and CNNs, and across diverse datasets, including both tabular and image data. 
Second, CF-Reg's strong performance on the \textit{Water} dataset with Logistic Regression suggests that its benefits may be more pronounced when applied to simpler models. However, the unexpected outcome on the \textit{Phoneme} dataset calls for further investigation into this phenomenon.


\begin{table*}[h!]
    \centering
    \caption{Mean value and standard deviation of test accuracy across 5 random initializations for different model, dataset, and regularization method. The best results are highlighted in \textbf{bold}.}
    \label{tab:regularization_comparison}
    \begin{tabular}{|c|c|c|c|c|c|c|}
        \hline
        \textbf{Model} & \textbf{Dataset} & \textbf{No-Reg} & \textbf{L1-Reg} & \textbf{L2-Reg} & \textbf{Dropout} & \textbf{CF-Reg (ours)} \\ \hline
        Logistic Regression   & \textit{Water}   & $0.6595 \pm 0.0038$   & $0.6729 \pm 0.0056$   & $0.6756 \pm 0.0046$  & N/A    & $\mathbf{0.6918 \pm 0.0036}$                     \\ \hline
        MLP   & \textit{Water}   & $0.6756 \pm 0.0042$   & $0.6790 \pm 0.0058$   & $0.6790 \pm 0.0023$  & $0.6750 \pm 0.0036$    & $\mathbf{0.6802 \pm 0.0046}$                    \\ \hline
%        MLP   & \textit{Adult}   & $0.8404 \pm 0.0010$   & $\mathbf{0.8495 \pm 0.0007}$   & $0.8489 \pm 0.0014$  & $\mathbf{0.8495 \pm 0.0016}$     & $0.8449 \pm 0.0019$                    \\ \hline
        Logistic Regression   & \textit{Phomene}   & $\mathbf{0.8148 \pm 0.0020}$   & $0.8041 \pm 0.0028$   & $0.7835 \pm 0.0176$  & N/A    & $0.8098 \pm 0.0055$                     \\ \hline
        MLP   & \textit{Phomene}   & $0.8677 \pm 0.0033$   & $0.8374 \pm 0.0080$   & $0.8673 \pm 0.0045$  & $0.8672 \pm 0.0042$     & $\mathbf{0.8718 \pm 0.0040}$                    \\ \hline
        CNN   & \textit{CIFAR-10} & $0.6670 \pm 0.0233$   & $0.6229 \pm 0.0850$   & $0.7348 \pm 0.0365$   & N/A    & $\mathbf{0.7427 \pm 0.0571}$                     \\ \hline
    \end{tabular}
\end{table*}

\begin{table*}[htb!]
    \centering
    \caption{Hyperparameter configurations utilized for the generation of Table \ref{tab:regularization_comparison}. For our regularization the hyperparameters are reported as $\mathbf{\alpha/\beta}$.}
    \label{tab:performance_parameters}
    \begin{tabular}{|c|c|c|c|c|c|c|}
        \hline
        \textbf{Model} & \textbf{Dataset} & \textbf{No-Reg} & \textbf{L1-Reg} & \textbf{L2-Reg} & \textbf{Dropout} & \textbf{CF-Reg (ours)} \\ \hline
        Logistic Regression   & \textit{Water}   & N/A   & $0.0093$   & $0.6927$  & N/A    & $0.3791/1.0355$                     \\ \hline
        MLP   & \textit{Water}   & N/A   & $0.0007$   & $0.0022$  & $0.0002$    & $0.2567/1.9775$                    \\ \hline
        Logistic Regression   &
        \textit{Phomene}   & N/A   & $0.0097$   & $0.7979$  & N/A    & $0.0571/1.8516$                     \\ \hline
        MLP   & \textit{Phomene}   & N/A   & $0.0007$   & $4.24\cdot10^{-5}$  & $0.0015$    & $0.0516/2.2700$                    \\ \hline
       % MLP   & \textit{Adult}   & N/A   & $0.0018$   & $0.0018$  & $0.0601$     & $0.0764/2.2068$                    \\ \hline
        CNN   & \textit{CIFAR-10} & N/A   & $0.0050$   & $0.0864$ & N/A    & $0.3018/
        2.1502$                     \\ \hline
    \end{tabular}
\end{table*}

\begin{table*}[htb!]
    \centering
    \caption{Mean value and standard deviation of training time across 5 different runs. The reported time (in seconds) corresponds to the generation of each entry in Table \ref{tab:regularization_comparison}. Times are }
    \label{tab:times}
    \begin{tabular}{|c|c|c|c|c|c|c|}
        \hline
        \textbf{Model} & \textbf{Dataset} & \textbf{No-Reg} & \textbf{L1-Reg} & \textbf{L2-Reg} & \textbf{Dropout} & \textbf{CF-Reg (ours)} \\ \hline
        Logistic Regression   & \textit{Water}   & $222.98 \pm 1.07$   & $239.94 \pm 2.59$   & $241.60 \pm 1.88$  & N/A    & $251.50 \pm 1.93$                     \\ \hline
        MLP   & \textit{Water}   & $225.71 \pm 3.85$   & $250.13 \pm 4.44$   & $255.78 \pm 2.38$  & $237.83 \pm 3.45$    & $266.48 \pm 3.46$                    \\ \hline
        Logistic Regression   & \textit{Phomene}   & $266.39 \pm 0.82$ & $367.52 \pm 6.85$   & $361.69 \pm 4.04$  & N/A   & $310.48 \pm 0.76$                    \\ \hline
        MLP   &
        \textit{Phomene} & $335.62 \pm 1.77$   & $390.86 \pm 2.11$   & $393.96 \pm 1.95$ & $363.51 \pm 5.07$    & $403.14 \pm 1.92$                     \\ \hline
       % MLP   & \textit{Adult}   & N/A   & $0.0018$   & $0.0018$  & $0.0601$     & $0.0764/2.2068$                    \\ \hline
        CNN   & \textit{CIFAR-10} & $370.09 \pm 0.18$   & $395.71 \pm 0.55$   & $401.38 \pm 0.16$ & N/A    & $1287.8 \pm 0.26$                     \\ \hline
    \end{tabular}
\end{table*}

\subsection{Feasibility of our Method}
A crucial requirement for any regularization technique is that it should impose minimal impact on the overall training process.
In this respect, CF-Reg introduces an overhead that depends on the time required to find the optimal counterfactual example for each training instance. 
As such, the more sophisticated the counterfactual generator model probed during training the higher would be the time required. However, a more advanced counterfactual generator might provide a more effective regularization. We discuss this trade-off in more details in Section~\ref{sec:discussion}.

Table~\ref{tab:times} presents the average training time ($\pm$ standard deviation) for each model and dataset combination listed in Table~\ref{tab:regularization_comparison}.
We can observe that the higher accuracy achieved by CF-Reg using the score-based counterfactual generator comes with only minimal overhead. However, when applied to deep neural networks with many hidden layers, such as \textit{PreactResNet-18}, the forward derivative computation required for the linearization of the network introduces a more noticeable computational cost, explaining the longer training times in the table.

\subsection{Hyperparameter Sensitivity Analysis}
The proposed counterfactual regularization technique relies on two key hyperparameters: $\alpha$ and $\beta$. The former is intrinsic to the loss formulation defined in (\ref{eq:cf-train}), while the latter is closely tied to the choice of the score-based counterfactual explanation method used.

Figure~\ref{fig:test_alpha_beta} illustrates how the test accuracy of an MLP trained on the \textit{Water Potability} dataset changes for different combinations of $\alpha$ and $\beta$.

\begin{figure}[ht]
    \centering
    \includegraphics[width=0.85\linewidth]{img/test_acc_alpha_beta.png}
    \caption{The test accuracy of an MLP trained on the \textit{Water Potability} dataset, evaluated while varying the weight of our counterfactual regularizer ($\alpha$) for different values of $\beta$.}
    \label{fig:test_alpha_beta}
\end{figure}

We observe that, for a fixed $\beta$, increasing the weight of our counterfactual regularizer ($\alpha$) can slightly improve test accuracy until a sudden drop is noticed for $\alpha > 0.1$.
This behavior was expected, as the impact of our penalty, like any regularization term, can be disruptive if not properly controlled.

Moreover, this finding further demonstrates that our regularization method, CF-Reg, is inherently data-driven. Therefore, it requires specific fine-tuning based on the combination of the model and dataset at hand.
\section{Conclusion}
In this work, we propose a simple yet effective approach, called SMILE, for graph few-shot learning with fewer tasks. Specifically, we introduce a novel dual-level mixup strategy, including within-task and across-task mixup, for enriching the diversity of nodes within each task and the diversity of tasks. Also, we incorporate the degree-based prior information to learn expressive node embeddings. Theoretically, we prove that SMILE effectively enhances the model's generalization performance. Empirically, we conduct extensive experiments on multiple benchmarks and the results suggest that SMILE significantly outperforms other baselines, including both in-domain and cross-domain few-shot settings.

\bibliographystyle{abbrvnat}
\bibliography{references}

\newpage
\appendix

\section{Missing Details for \Cref{sec:preliminaries}}

\subsection{Least-Squares Linear Regression}
\label{app:least-square-regression}

For a design matrix $\mat{A}\in \R^{n\times d}$ and response $\mat{b}\in \R^n$, consider the least-squares problem
\[
    \mat{x}^* = \argmin_{\mat x\in \R^d}\,\norm{\mat{Ax}-\mat{b}}_2\,.
\]
The solution $\mat{x}^*$ can be obtained by solving the \emph{normal equation} $\mat{A}^\top\mat{A}\mat{x}^*=\mat{A}^\top \mat{b}$.
Therefore, $\mat{x}^* = (\mat{A}^\top\mat{A})^{-1}\mat{A}^\top \mat{b}$.
If $\mat{A}^\top\mat{A}$ is singular, then we can use the \emph{pseudoinverse} $\mat{A}^+$.

The \emph{orthogonal projection matrix} $\pi_{\mat{A}}:\R^n \to \R^n$ onto the image space of $\mat{A}$ is defined by
$\pi_{\mat{A}} = \mat{A}\,(\mat{A}^\top\mat{A})^{-1}\mat{A}^\top$, and satisfies $\pi_{\mat{A}}^2 = \pi_{\mat{A}}$ and $\norm{\pi_{\mat A}\mat v}_2 \leq \norm{\mat v}_2$ for any $\mat{v} \in \R^n$. 
Recall that any $\mat{v}\in \R^n$ can be uniquely decomposed as $\mat v = \pi_{\mat{A}}\mat v + \pi_{\mat{A}^\perp}\mat v$, where $\pi_{\mat{A}^\perp}=\mat{I}_{n} - \pi_{\mat{A}}$ is the orthogonal projection to the orthogonal subspace of $\textnormal{colsp}(\mat{A})$.

Given $\mat{b} = \pi_{\mat{A}}\mat b + \pi_{\mat A ^\perp}\mat b$, the first term is the \emph{reducible error} by regressing $\mat b$ on $\mat x$,
i.e., taking the optimum $\mat{x}^*$ so that $\mat{Ax}^* = \pi_{\mat{A}}\mat{b}$.
The second term  $\pi_{\mat{A}^\perp}\mat{b}$ is the \emph{irreducible error}, i.e., $\min\,\norm{\mat{Ax}-\mat b}_2 =\norm{\pi_{\mat A^\perp}\mat b}_2$.


\subsection{Leverage Score Sampling for Tensor Decomposition}
\label{app:leverage-score}

ALS formulations show how each tensor decomposition step reduces to solving a least-squares problem of the form
$\min_{\mat{x}}\,\norm{\mat{A}\mat x - \mat{b}}_2$ with a highly structured $\mat{A}$.
While we can find the optimum in closed form via $(\mat{A}^\top \mat{A})^{+} \mat{A}^\top \mat{b}$, matrix $\mat{A}$ has $I = I_1 \cdots I_N$ rows corresponding to each entry of the tensor (i.e., it is a tall skinny matrix), which can make using the normal equation challenging in practice.

Randomized sketching methods are a popular approach to approximately solving this problem with faster running times with high probability.
In general, these approach sample rows of $\mat{A}$ according to the probability distribution defined by the \emph{leverage scores} of rows, resulting in a random sketching matrix $\mat{S}$ whose height is much smaller than that of $\mat{A}$. For a matrix $\mat{A} \in \R^{I\times R}$ with ($I\gg R$), the leverage scores of $\mat{A}$ is the vector $\ell\in [0,1]^I$ defined by
\[
\ell_i \defeq \bigl(\mat{A}\,(\mat{A^\top A})^+\mat{A}^\top\bigr)_{ii}\,.
\]
Then, for a given $\varepsilon, \delta \in (0,1)$, the sketching algorithm samples $\tilde{O}(\nicefrac{R}{\varepsilon\delta})$ many rows, where the $i$-th row is drawn with probability $\ell_i  / \sum_i \ell_i = \ell_i / \text{rank}\,(\mat{A})$.
With probability at least $1-\delta$, we can guarantee that
\[
\min_{\mat{x}} \,\norm{\mat{S}\mat{A}\mat{x}-\mat{S}\mat{b}}_2 \leq (1+\varepsilon)\, \min_{\mat{x}}\,\norm{\mat{A}\mat x - \mat{b}}_2\,.
\]
The reduced number of rows in $\mat{SA}$ leads to better running times for the least-squares solves.
However, na\"ively computing leverage scores takes as long as computing the closed-form optimum since we need to compute $(\mat{A}^\top \mat{A})^{+}$.
This is where the \emph{structure} of the design matrix $\mat{A}$ comes in to play, i.e., to speed up the leverage score computations.


\subsection{Tensor-Train Decomposition}
\label{app:tt-decomposition-details}

Given a tensor-train (TT) decomposition $\{\tensor{A}^{(n)}\}_{n=1}^{N}$ and index $n\in[N]$, define the
\emph{left chain} $\mat{A}_{<n} \in \R^{(I_1 \cdots I_{n-1}) \times R_{n-1}}$
and
the \emph{right chain} $\mat{A}_{> n}\in \R^{R_n \times (I_{n+1} \cdots I_N)}$ as:
\begin{align*}
    a_{< n}(\underline{i_1\dots i_{n-1}}, r_{n-1})
    &=
    \sum_{r_0,\dots,r_{n-1}} \prod_{k=1}^{n-1} a^{(k)}_{r_{k-1}i_kr_k}
    \\
    a_{> n}(r_n, \underline{i_{n+1}\dots i_N})
    &=
    \sum_{r_{n+1},\dots,r_N} \prod_{k=n+1}^N a^{(k)}_{r_{k-1}i_kr_k}\,,
\end{align*}
where for any $i_s \in [I_s]$ with $s (\neq n)\in [N]$, $\underline{i_1\dots i_{n-1}}:= 1+ \sum_{k=1}^{n-1} (i_k -1)\prod_{j=1}^{k-1} I_j$ and $\underline{i_{n+1}\dots i_N}:= 1+ \sum_{k=n+1}^{N} (i_k -1)\prod_{j=n+1}^{k-1} I_j$.
When ALS optimizes $\tensor{A}^{(n)}$ with all other TT-cores fixed, it solves the regression problem:
\[
    \tensor{A}^{(n)} \!\!
    \gets \!\!
    \argmin_{\tensor{B} \in \R^{R_{n-1}\times I_n \times R_n}}
    \bigl\lVert (\mat{A}_{<n}\kron \mat{A}^\top_{>n})\,\mat{B}_{(2)}^\top - \mat{X}_{(n)}^\top \bigr\rVert_{\frobenius}\,,
\]
which is equivalent to solving $I_n$ Kronecker regression problems in $\R^{\prod_{k\neq n} I_k}$.


\subsection{Tensor Networks}
\label{app:tensor-networks}

A \emph{tensor network} (TN) is a powerful framework that can represent any factorization of a tensor,
so it can recover the three decompositions above as special cases.
A TN decomposition $\text{TN}(\tensor A^{(1)},\dots,\tensor A^{(N)})$ represents a given tensor $\tensor X$ with $N$ tensors $\tensor{A}^{(1)},\dots, \tensor A^{(N)}$ and a \emph{tensor diagram}. As in the above decompositions, the goal is to compute
\[
\argmin_{\tensor A^{(1)},\dots,\tensor A^{(N)}}\,\norm{\tensor X - \text{TN}(\tensor A^{(1)},\dots,\tensor A^{(N)})}_\frobenius^2\,.
\]
A tensor diagram\footnote{We refer readers to \url{https://tensornetwork.org/diagrams/} for more details.} consists of nodes with dangling edges, where a node indicates a tensor, and its dangling edge represents a mode, so that the number of dangling edges is the order of the tensor. For example, a node without an edge indicates a scalar, one with one dangling edge is a vector, and one with two dangling edges is a matrix.

When two dangling edges of two nodes are connected, we say that the two tensors are \emph{contracted} along that mode (i.e., a mode product of those two tensors). For example, when a node with two dangling edges shares one edge with another node with one dangling edge, it indicates a matrix-vector multiplication. Hence, the number of unmatched dangling edges in a tensor diagram corresponds to the order of its representing tensor.

\paragraph{ALS for TN decomposition.}
Given a TN decomposition $\{\tensor{A}^{(n)}\}_{n\in[N]}$, when ALS optimizes a tensor $\tensor A^{(n)}$ with all others fixed, it solves a linear regression problem of the form
\[
\tensor{A}^{(n)} \gets \argmin_{\tensor{B}} \, \lVert \mat A_{\neq n}\mat{B} - \mat{X}\rVert_{\frobenius}\,,
\]
where $\mat A_{\neq n}$ is an appropriate matricization depending on $\tensor A^{(1)},\dots,\tensor A^{(n-1)}, \tensor A^{(n+1)},\dots,\tensor A^{(N)}$, and $\mat B$ and $\mat X$ are suitable matricizations of $\tensor B$ and $\tensor X$, respectively. Structure of $\mat A_{\neq n}$ can be specified through a new tensor diagram obtained by removing the node of $\tensor A^{(n)}$ from the original tensor network diagram.

Just as the ALS approaches for other decomposition algorithms, \citet{malik2022sampling} proposed a sampling-based approach via leverage scores. First of all, they pointed out that $\mat{A}_{\neq n}^\top \mat A_{\neq n}$ can be efficiently computed by exploiting inherent structure of $\mat A_{\neq n}$ (i.e., contract a series of matched edges in a tensor diagram in an appropriate order). They then presented a leverage-score sampling method that draws rows of $\mat A_{\neq n}$ without materializing a full probability vector, and in spirit this approach is similar to one used for the CP decomposition in \cref{subsec:CP-completion}.


% \section{Reproducibility Checklist}
% Please refer to the technical appendix titled 'Reproducibility Checklist', which is attached in the supplementary material of this submission.

\section{Experimental Setup \label{sec:hyperParams}} 
All training experiments were performed
on public datasets using a single A100 40GB GPU for a
maximum of two days. All experiments were conducted using PyTorch, and results are averaged over three seeds. All hyperparameters are detailed in Tab.~\ref{tab:NLPhyperpams} and Tab.~\ref{tab:Vsionhyperpams}.

\begin{table}[h]
\centering
\small
\begin{tabular}{l c}
\toprule
\textbf{Parameter} & \textbf{Value} \\
\midrule
Model-width & 192 \\
Number of layers & 24 \\
Number of patches & 196 \\
%Scan Mode {\color{red} ???} & one directional \\
Batch-size & 512 \\
Optimizer & AdamW \\
Momentum & \( \beta_1, \beta_2 = 0.9, 0.999 \) \\
Base learning rate & $5e-4$ \\
Weight decay & 0.1 \\
Dropout & 0 \\
Training epochs & 300 \\
Learning rate schedule & cosine decay \\
Warmup epochs & 5 \\
Warmup schedule & linear \\ 
Degree of Taylor approx. (Eq.~\ref{eq:simplifiedModel}) & 3 \\
\bottomrule
\end{tabular}
\caption{Hyperparameters for image-classification via Vision Mamba variants} 
\label{tab:Vsionhyperpams}
\end{table}

\begin{table}[h]
\centering
\small
\begin{tabular}{l c}
\toprule
\textbf{Parameter} & \textbf{Value} \\
\midrule
Model-width & 386 \\
Number of layers & 12 \\
Context-length (training) & 1024 \\
Batch-size & 32 \\
Optimizer & AdamW \\
Momentum & \( \beta_1, \beta_2 = 0.9, 0.999 \) \\
Base learning rate & $1.5e-3$ \\
Weight decay & 0.01 \\
Dropout & 0 \\
Training epochs & 20 \\
Learning rate schedule & cosine decay  \\
Warmup epochs & 1  \\
Warmup schedule & linear  \\ 
Degree of Taylor approx. (Eq.~\ref{eq:simplifiedModel}) &  3\\
\bottomrule
\end{tabular}
\caption{Hyperparameters for language modeling via Mamba-based LMs} 
\label{tab:NLPhyperpams}
\end{table}

% \section{Additional Background Material}

% {\noindent\textbf{Rademacher Complexities}}
% We explore the generalization capabilities of overparameterized NNs by analyzing their Rademacher complexity. This measure provides an upper bound on the worst-case generalization gap, which represents the difference between training and testing errors within a specific hypothesis class. It is defined as the expected performance of the class averaged over all possible data labelings, with labels independently and uniformly drawn from the set $\{\pm 1\}$. For further details, refer to \citep{mohri2018foundations, Shalev-Shwartz2014, bartlett2002rademacher}.


% \begin{definition}[Rademacher Complexity] Let $\mathcal{F}$ be a set of real-valued functions $f_w:\mathcal{X} \to \mathbb{R}^\mathcal{C}$ defined over a set $\mathcal{X}$. Given a fixed sample $X = \{ x_j\}_{j=1}^m \in \mathcal{X}^m$, the empirical Rademacher complexity of $\mathcal{F}$ is defined as follows: 
% \begin{equation*}
% \mathcal{R}_{X}(\mathcal{F}) ~:=~ \frac{1}{m} \mathbb{E}_{\xi: \xi_{ic} \sim U[\{\pm 1\}]} \left[ \sup_{f_w \in \mathcal{F}} \sum^{m}_{j=1}\sum^{\mathcal{C}}_{c=1} \xi_{ic} f_w(x_j)_c  \right].
% \end{equation*}
% \end{definition}

% {\color{red}
% It should be moved:
% We focus on training models for classification tasks. The problem is formally defined by a distribution $P$ over pairs $(x,y)\in \cX\times \cY$, where $\cX \subset \R^{\mathcal{D}}$ represents the input space, and $\cY \subset \R^\mathcal{C}$ is the label space containing one-hot encoded labels for the integers $1,\ldots,\mathcal{C}$. 

% We define a hypothesis class $\cF \subset \{f_w:\cX\to \R^\mathcal{C}\}$ (such as a neural network architecture), where each function \( f_w \in \cF \) is parameterized by a vector of trainable weights $w$. Given any input \( x \in \cX \), \( f_w \) provides a predicted label, and the performance is evaluated based on the \emph{expected error}, \(\err_P(f_w) := \mathbb{E}_{(x, y) \sim P}\left[\mathbb{I}\left[\max_{j \neq y}(f_w(x)_j) \geq f_w(x)_{y}\right]\right]\). Here, the indicator function \(\mathbb{I}\) returns 1 for True and 0 for False.

% Since the full distribution \( P \) is not directly accessible, our objective is to train a model \( f_w \) using a training dataset \( S = \{(x_i, y_i)\}_{i = 1}^m \) consisting of independent and identically distributed (i.i.d.) samples from \( P \). We aim to achieve accurate predictions while applying regularization to control the complexity of \( f_w \). 

% To denote the entire $n$th row and $n$th column of a matrix $M$, we use the notation:
% \begin{align*}
% M_{n*} & \text{ denotes the entire } n \text{th row of matrix } M. \\
% M_{*n} & \text{ denotes the entire } n \text{th column of matrix } M. \\
% M_{nk} & \text{ denotes the element in the } n \text{th row and } k \text{th column.}
% \end{align*}
% {\bf Selective State Space Models.\enspace} 
% Time-variant SSMs, namely, the matrices $A,B,C$ of each channel are modified over $L$ time steps. We are focusing on selective SSMs of the following form. 
% A neural network $f_w : \mathbb{R}^{D \times L} \rightarrow \mathbb{R}^{\mathcal{C}}$ takes a sequence $x=(x_{*1},...,x_{*L}) \in \mathbb{R}^{D \times L}$ as input where $L$ is the length of the sequence and $D$ is the dimension of the tokens $x_i$. We denote
% \begin{align*}
%     B_i &= B x_{*i}, B \in \mathbb{R}^{N \times D} \\
%     C_i &= C x_{*i}, C \in \mathbb{R}^{N \times D} \\
%     \Delta_{*i} &= S_{\Delta} x_{*i}, S_{\Delta} \in \mathbb{R}^{D \times D} \rightarrow \Delta \in \mathbb{R}^{D \times L} \\
%     \bar{A}_{j,i} &= (z^2 + \alpha z)(\Delta_{j,i} * \underbrace{A_{j*}}_{\textbf{ $j$th row of A } } ), A \in \mathbb{R}^{D \times N}, \text{$(z^2+\alpha z)$ is an activation function, } \alpha \geq 0 \\ 
%     W &\in \mathbb{R}^{\mathcal{C} \times D} 
% \end{align*}
% }

\end{document}
