\documentclass{article}

\usepackage[utf8]{inputenc} % allow utf-8 input
\usepackage[T1]{fontenc}    % use 8-bit T1 fonts
\usepackage{hyperref}       % hyperlinks
\usepackage{url}            % simple URL typesetting
\usepackage{booktabs}       % professional-quality tables
\usepackage{amsfonts}       % blackboard math symbols
\usepackage{nicefrac}       % compact symbols for 1/2, etc.
\usepackage{microtype}      % microtypography
\usepackage{xcolor}         % colors

\usepackage{graphicx}
\usepackage{subcaption}

% Attempt to make hyperref and algorithmic work together better:
\newcommand{\theHalgorithm}{\arabic{algorithm}}

% For theorems and such
\usepackage{amsmath}
\usepackage{amssymb}
\usepackage{mathtools}
\usepackage{amsthm}

% if you use cleveref..
\usepackage[capitalize,nameinlink,noabbrev]{cleveref}

%%%%%%%%%%%%%%%%%%%%%%%%%%%%%%%%
% THEOREMS
%%%%%%%%%%%%%%%%%%%%%%%%%%%%%%%%
\usepackage{thm-restate}
\theoremstyle{plain}
\newtheorem{theorem}{Theorem}[section]
\newtheorem{proposition}[theorem]{Proposition}
\newtheorem{lemma}[theorem]{Lemma}
\newtheorem{corollary}[theorem]{Corollary}
\theoremstyle{definition}
\newtheorem{definition}[theorem]{Definition}
\newtheorem{assumption}[theorem]{Assumption}
\theoremstyle{remark}
\newtheorem{remark}[theorem]{Remark}

\newcommand{\thought}[1]{{\color[rgb]{0.2,0.39,0.66}(#1)}}
\newcommand{\todo}[1]{{\color[rgb]{1.0,0.0,0.0}(#1)}}
\newcommand{\hsh}[1]{{\color{green!50!black} Henrik: #1}}
\newcommand{\st}[1]{{\color{red!50!black} Sebastian: #1}}

\newcommand{\ulm}[1]{_{\scaleto{\mathrm{#1}}{3pt}}}
\newcommand\at[2]{\left.#1\right|_{#2}}











\newtheorem{assumption}{Assumption}

\DeclareMathOperator*{\argmax}{arg\,max}
\DeclareMathOperator*{\argmin}{arg\,min}

\newcommand{\swname}[1]{\texttt{#1}}
\newcommand{\ie}{i\/.\/e\/.,\/~}
\newcommand{\eg}{e\/.\/g\/.,\/~}
\newcommand{\cf}{cf\/.\/~}

\newcommand{\fig}{Fig\/.\/~}
\newcommand{\defn}{Def\/.\/~}
\newcommand{\sect}{Sec\/.\/~}
\newcommand{\tabl}{Tab\/.\/~}
\newcommand{\algo}{Algorithm~}
\newcommand{\theo}{Theorem~}

\newcommand{\bnnl}{3 hidden layers}
\newcommand{\bnnn}{50 neurons}
\newcommand{\bnna}{tanh activations}

\newcommand{\capt}[1]{\mdseries{\emph{#1}}}

\newcommand{\videolink}{at \url{https://youtu.be/_d7AqTRjz6g}}
\newcommand{\codelink}{\url{https://github.com/wheelbot/mini-wheelbot}}

\newcommand{\fakepar}[1]{\vspace{0mm}\noindent\textbf{#1.}}

\newcommand{\needref}{\textcolor{red}{[REF]}}

\newcommand{\plotfontsize}{9pt}


\title{Fast Tensor Completion via Approximate Richardson Iteration}

\author[1]{Mehrdad Ghadiri}
%Emails: \texttt{mehrdadg@mit.edu, fahrbach@google.com, yb.kook@gatech.edu, jadbabai@mit.edu}}}
\author[2]{Matthew Fahrbach}
\author[3]{Yunbum Kook}
\author[1]{Ali Jadbabaie}

\affil[1]{Massachusetts Institute of Technology, \texttt{\{mehrdadg,jadbabai\}@mit.edu}}
\affil[2]{Google Research, \texttt{fahrbach@google.com}}
\affil[3]{Georgia Institute of Technology, \texttt{yb.kook@gatech.edu}}

\date{}

\begin{document}

\maketitle

\begin{abstract}
Retrieval-Augmented Generation (RAG) is often used with Large Language Models (LLMs) to infuse domain knowledge or user-specific information. In RAG, given a user query, a retriever extracts chunks of relevant text from a knowledge base. These chunks are sent to an LLM as part of the input prompt. Typically, any given chunk is repeatedly retrieved across user questions. However, currently, for every question, attention-layers in LLMs fully compute the key values (KVs) repeatedly for the input chunks, as state-of-the-art methods cannot reuse KV-caches when chunks appear at arbitrary locations with arbitrary contexts. Naive reuse leads to output quality degradation.  This leads to potentially redundant computations on expensive GPUs and increases latency. In this work, we propose \sys, a system for managing and reusing precomputed KVs corresponding to the text chunks (we call \textit{chunk-caches}) in RAG-based systems. We present how to identify \hl{\textit{chunk-caches} that are reusable}, how to efficiently perform a small fraction of recomputation to \textit{fix} the cache to maintain output quality, and how to efficiently store and evict \textit{chunk-caches} in the hardware for maximizing reuse while masking any overheads. With real production workloads as well as synthetic datasets, we show that \sys reduces redundant computation by \textbf{51\%} over SOTA prefix-caching and \textbf{75\%} over full recomputation.
\hl{Additionally, with continuous batching on a real production workload, we get a \textbf{1.6$\times$} speedup in throughput and a \textbf{2$\times$} reduction in end-to-end response latency over prefix-caching while maintaining quality, for both the \llama-3-8B and \llama-3-70B models. 
}
\end{abstract}





\documentclass[../main.tex]{subfiles}
\graphicspath{{../images/}}
\makeatletter
\def\input@path{{../images/}}
\makeatother
\begin{document}
\section{Introduction}
\begin{figure}
\centering
\begin{tikzpicture}
\node[inner sep=0pt] (ws) at (0, 0) {
\includegraphics[height=.4\textwidth, trim={10cm 0 10cm 0},clip]{world_space.png}};
\node[inner sep=0pt] (cs) at (6,0) {\includegraphics[height=.4\textwidth, trim={10cm 1cm 10cm 4cm},clip]{conf_space.png}};
\end{tikzpicture}
\vspace{-5pt}
\label{fig:pbrm_intro}
\caption{\textbf{Left}: Shows world space obstacles as grey spheres. Robots start and goal configuration is colored red and green, respectively. Configurations along the computed path are colored transparent blue. \textbf{Right:} Mapped world space scenario to configuration space. Obstacle region is the grey mesh. Red spheres are collision-free regions computed by the neural SCDF. The optimized shortest path in the convex corridor is the blue curve.}
\vspace{-25pt}
\end{figure}
Motion planning is the problem of finding a collision-free trajectory that connects a given start and goal configuration. The planning takes place in the configuration space of the robot. For single body robots, like mobile robots or drones, the configuration space and the world space are usually the same. This simplifies the planning, since explicit obstacle representations are available which enables geometrical tools like separating hyperplanes, smallest distance to obstacles etc., to be used when designing motion planning algorithms. For multi-body robots like manipulators, the situation is completely different. The world space obstacles are usually mapped to non-convex regions, and to make the problem even harder, the mapping is usually not known. Forming explicit representations of the obstacle region in the configuration space is usually too expensive or intractable. Despite all of this, sampling based planners are used with great success, which mainly is due to their use of implicit representations of the obstacle region. The basic idea is to construct a graph in the configuration space that covers and connects the collision-free region. From this graph, a path can be extracted that connects a given start and goal configuration. The approach is computationally expensive, since the graph is constructed with the smallest geometrical building block available, points, which represents a collision-check. Furthermore, the extracted paths from the graph are non-smooth and jagged due to the stochastic nature of the approach. This adds an additional post-processing step to the process, where the paths are shortcutted and smoothened, before the path can be used for tracking. Clearly a lot of time is invested to form this graph and produce smooth paths. Thus, if the obstacles start to move, then all of this work is done in no use, since all points that make up this graph need to be re-verified, which is simply too time consuming to be done in real time.
\\\\
In this work, we want to address the existing drawbacks of the sampling based planners. Our main contribution is an improved motion planner where each vertex in the graph covers a collision-free region in the form of a sphere instead of a point and where the edges are formed with neighboring intersecting spheres. This representation has the advantage of instead of returning piecewise linear paths, returning a sequence of overlapping spheres, i.e. a convex corridor, that connects a given start and goal configuration, illustrated in Figure \ref{fig:pbrm_intro}. This convex corridor allows us to use convex optimization to produce smooth trajectories, instead of computationally expensive post-processing methods. The representation further allows us to estimate the coverage of the collision-free space, which gives us awareness and feedback in the offline roadmap construction phase. Finally, our representation is simple to adapt to moving obstacles, simply requery for the new radii and recheck for intersections. 
\\\\
The spherical collision-free regions are formed using a signed distance function (SDF), which is a function that returns the smallest distance from an arbitrary point to the boundary of an obstacle. As the name implies, the distance is signed, thus if the point is inside the obstacle it is negative otherwise positive. If the distance is positive, a sphere with radius equal to the distance is guaranteed to cover a collision-free region. Using an SDF in motion planning is not new, but what is novel about our approach is that we express the distance in the configuration space instead of the world space and by doing so allows us to form these convex collision-free regions. We refer to the resulting SDF as a signed configuration distance function (SCDF). Computing an SCDF analytically is non-trivial, our approach is therefore to parameterize the SCDF with a deep neural network and learn the mapping by supervised learning. Our resulting neural SCDF can compute distances for different parameter values of obstacle shapes and we also show how multiple distances can be combined, thus making our approach flexible.
\section{Related work}
Motion planning algorithms can roughly be divided into three families, grid-based, sampling based and optimization based methods. Grid-based methods (GBM) discretize the planning space from which a graph is then compiled. A standard search method is A$^\star$ \citep{a_star}, which is classified as an \textit{informed} search method, since it employs a heuristic function to speed up the search. A$^\star$ guarantees to return an optimal path at the level of discretization used. GBMs usually discretize the planning space by a regular lattice and this limits the GBMs to problems with low dimensionality due to the curse of dimensionality. Thus, GBMs are usually limited to single-body robots where the degrees of freedom (DOF) are low. To overcome the inherent scaling problem with the GBMs, stochastic methods are usually used for multi-body robots. These methods are termed as sampling-based methods (SBM) and core members within this family are the rapidly-exploring random trees (RRT) \citep{rrt} and the probabilistic roadmap (PRM) \citep{prm}. RRT grows a tree from the start configuration and explores the collision-free region in a rapid way until it is able to connect to the goal region. RRT is usually improved by bi-directional planning \citep{rrt_connect}, i.e. an additional tree is grown from the goal configuration and the trees are tested for connection after any tree has been expanded. RRT is a single-query method, thus it searches for a path from scratch each time it is queried. Contrary to this, PRM is a multi-query method, which solves for multiple queries without starting from scratch. PRM does this by creating a roadmap (graph) that covers the collision-free space as an offline step. The graph is then used to solve for multiple queries. PRMs are used in cases where the environment does not change since the extra offline step is too computationally costly and needs to be re-done if the environment is changed. In our work, we address this inherent issue by using a different roadmap representation. Our vertices in the graph cover a collision-free region in the form of spheres and we form the edges by checking for intersecting spheres. If something in the environment changes, we recompute the spheres radii and recheck the intersections, without relying on collision detection. We use a trained neural network to compute the sphere radius, therefore querying for the radius can be done fast, hence our representation enables the PRM for dynamic environments.
\\\\
In the recent decades, optimization based methods (OBM) \citep{chomp, schulman, itomp, stomp} have been introduced as an alternative to SBM for multi-body robots. Like the SBM, the OBMs scale well to higher dimensional problems and produce smoother motion. It is common to use a SDF in the optimization since it is a smooth function, thus enabling gradient-based methods. However, the standard way of expressing the SDF is in world space. The distance therefore needs to be mapped to the configuration space by the forward kinematics. This mapping makes the optimization problem a non-linear program (NLP), which is computationally expensive to solve. Recently, a different approach has been proposed. In \cite{mp_gcs} motion planning is formulated as a convex optimization problem by using the graph of convex sets framework \citep{gcs}. The underlying idea is to decompose the collision-free space into intersecting convex sets from which a convex optimization problem is formulated. In cases where an explicit representation of the obstacles in the configuration space exists, like for single-body robots, creating collision-free convex regions can be done fast \citep{iris}. For multi-body robots, this is non-trivial. Existing work does this successfully \citep{iris_nlp, iris_c} by an optimization based approach, but the methods are still too time consuming to be used in the presence of moving obstacles. Our approach is instead to use deep learning to learn an SDF expressed in the configuration space. With this, we can query for shortest distances to the collision boundary, which allows us to expand spherical regions which are collision-free. Our approach is fast and therefore enables our suggested roadmap planner to be used in dynamic environments.
\\\\
Recent research has focused on learning collision detection \citep{fk_kernel_distance, diffco, graphdistnet} by predicting the signed distance between the robot links and the surrounding obstacles in the world space. The learned SDF is used in trajectory optimization but since the distance is expressed in the world space, the problem becomes an NLP and therefore takes a long time to solve. We take a novel approach and suggest to instead express the signed distance in the configuration space. This allows us to improve the PRM at the same time as it enables convex optimization for trajectory optimization, which runs faster and is more reliable than NLP solvers. In \cite{cspf} a learned signed distance function in the configuration space is proposed similar to our approach. However, their approach is restricted to point cloud representations, while we propose to represent the obstacles as parameterized geometric shapes, e.g. spheres. Furthermore, we also show how to use our learned SCDF to improve an existing roadmap planner.
\section{Problem formulation}
A robot is located in the world space, $\W \subset \R^3 $. The unique location of the robot is given by its configuration $\q \in \C$, where $\C$ is the configuration space. The set of points covered by the robots bodies at a certain configuration is expressed as $\B(\q) \subset \W$. The robot is surrounded by $\NrObst$ obstacles $\O = \bigcup_{i=1}^{\NrObst} \O_i$, where  $\O_i \subset \W$. The representation of the obstacle in the configuration space is the set $\C\O_i = \{\q \in \C \: |\: \B(\q) \cap \O_i \neq \emptyset \}$. The obstacle space is formed as $\Co = \bigcup_{i=1}^{\NrObst} \C \O_i$. The complement is referred to as the free space, $\Cf = \C \setminus \Co$. The path planning problem is a tuple, ($\Cf$, $\qStart$, $\qGoal$), where we want to connect a query pair, consisting of a start, $\qStart$, and goal configuration, $\qGoal$, with a geometric path, $\q(s): [0, 1] \mapsto \Cf$, such that $\q(0)=\qStart$ and $\q(1)=\qGoal$, or report correctly when such a path does not exist.
\end{document}


\section{Preliminaries}\label{sec:preliminaries}



%We denote by $(\Ac(x_\Ac),\Bc(x_\Bc))(z)$ a random execution of $\pi$ with private inputs $(x_\Ac,y_\Ac)$, and common input $z$.

%\Jnote{Move to DP}
% At the end of such an execution, the protocol outputs a public transcript denoted by the random variable $\trans_\pi(x_\Ac,x_\Ac,z)$ we denotes the common as $\out(\trans_\pi(x_\Ac,x_\Ac,z)$, and each party $\Pc \in \set{\Ac,\Bc}$ obtains his view denoted $\view^\Pc_\pi(x_\Ac,x_\Bc,z)$, which may also contain a ``local output'' \Jnote{Local} $\out^\Pc(x_\Ac,x_\Bc,z)$ (if the protocol specifies such an output). \Jnote{Common output, and parties output}


\subsection{Distributions and Random Variables}\label{sec:prelim:dist}
The support of a distribution $P$ over a finite set $\cS$ is defined by $\Supp(P) \eqdef \set{x\in \cS: P(x)>0}$. For a distribution or a random variable $D$, let $d\from D$ denote that $d$ was sampled according to $D$. Similarly,  for a set $\cS$, let $x \from \cS$ denote that $x$ is drawn uniformly from $\cS$, and denote by $\cU_{\cS}$ the uniform distribution over $\cS$. For a finite set $\cX$ and a distribution $C_X$ over $\cX$, we use the capital letter $X$ to denote the random variable that takes values in $\cX$ and is sampled according to $C_X$. The {\sf statistical distance} (\aka {\sf~variation distance}) of two distributions $P$ and $Q$ over a discrete domain $\cX$ is defined by $\sdist{P}{Q} \eqdef \max_{\cS\subseteq \cX} \size{P(\cS)-Q(\cS)} = \frac{1}{2} \sum_{x \in \cS}\size{P(x)-Q(x)}$. 
For a vector $x = (x_1,\ldots,x_n)$ and index $i\in [n]$, we let $x_{-i} = (x_1,\ldots,x_{i-1},x_{i+1},\ldots,x_n)$ and $x^{(i)} = (x_1,\ldots,x_{i-1}, -x_i, x_{i+1},\ldots,x_n)$, for a set $\cS \subseteq [n]$ we let $x_{\cS} = (x_i)_{i \in \cS}$ and $x_{-\cS} = (x_i)_{i \in [n]\setminus \cS}$, and for a vector $r \in \zo^n$ we let $x_r = (x_i)_{\set{i \colon r_i = 1}}$ and $x_{-r} = (x_i)_{\set{i \colon r_i = 0}}$.

%For $n \in \N$ we let $U_n$ be the uniform distribution over $\oo^n$, and let $S_n$ be the distribution induces by the sum of $n$ i.i.d.\ random variables, each is distributed according to $U_1$. Let $\cN(0,1)$ be the standard normal distribution.
%For a distribution $\cD$ and a function $f$, we define by $f(\cD)$ the distribution that is induced by the output of $f(x)$ for $x \from \cD$. 





% \begin{theorem}[\cite{McGregorMPRTV10}]\label{thm:sv-extracotr}
% 	\Enote{Remove if not needed}
% 	There is a constant $c$ to make the following holds. Let $X$ be an $\alpha$-SV source on $\{0,1\}^n$, let $Y$ be a source on $\{0,1\}^n$ with min-entropy at least $\beta n$ (independent from $X$), and let $Z=\ip{X,Y}\mbox{mod m}$ for some $m\in\mathbb{N}$. Then for every $\delta\in[0,1]$, the random variable $(Y,Z)$ is $\delta$-close to $(Y,U)$ where $U$ is uniform on $\mathbb{Z}_m$ and independent of $Y$, provided that
% 	$$
% 	n\geq c\cdot\frac{m^2}{\alpha\beta}\cdot\log(\frac{m}{\beta})\cdot\log(\frac{m}{\delta}).
% 	$$
% \end{theorem}



\Enote{I removed the definition of DP since it already appears in the intro}
\remove{
\subsection{Differential Privacy}\label{sec:prelim:DP}
We use the following standard definition of (information theoretic) differential privacy, due to \citet{DMNS06}. For notational convenience, we focus on databases over $\oo$.
\begin{definition}[Differentially private mechanisms]\label{def:mech}
	A randomized function $f\colon\oo^n\mapsto \zs$ is an {\sf $n$-size, $(\eps,\delta)$-differentially private mechanism} (denoted $(\eps,\delta)$-\DP) if for every neighboring $w,w'\in \oo^n$ and every function $g\colon \zs\mapsto \zo$, it holds that 
	$$
	\pr{g(f(w))=1}\leq e^{\eps}\cdot \pr{g(f(w'))=1} +\delta.
	$$ 	
	If $\delta=0$, we omit it from the notation.
\end{definition}
}


\subsubsection{Computational Differential Privacy}
There are several ways for defining computational differential privacy (see \cref{sec:related-works}). We use the most relaxed version due to \cite{BNO08}. For notational convenience, we focus on databases over $\oo$.
\begin{definition}[Computational differentially private mechanisms]\label{def:ComMech}
	A randomized function ensemble $f=\set{f_\pk\colon\oo^{n(\pk)}\mapsto \zs}$ is an {\sf $n$-size, $(\eps,\delta)$-computationally differentially private} (denoted $(\eps,\delta)$-$\CDP$) if for every poly-size circuit family $\set{\Ac_\pk}_{\pk\in \N}$, the following holds for every large enough $\pk$ and every neighboring $w,w'\in\oo^{n(\pk)}$:
	$$
	\pr{\Ac_\pk(f_\pk(w))=1}\leq e^{\eps(\pk)}\cdot \pr{\Ac_\pk(f_\pk(w'))=1} +\delta(\pk).
	$$ 
	If $\delta(\pk) = \negl(\pk)$, we omit it from the notation. 
\end{definition}



\subsubsection{Two-Party Differential Privacy}\label{sec:DP}
In this section we formally define distributed differential privacy mechanism (\ie protocols). %For the ease of notation, we consider protocol with no common input.

\begin{definition}\label{def:DP}%\Nnote{fix security parameter}
	A two-party protocol $\Pi=(\Ac,\Bc)$ is {\sf $(\eps,\delta)$-differentially private}, denoted $(\eps,\delta)$-$\DP$, if the following holds for every algorithm $\Dc$: let $\V^\Pc(x,y)(\pk)$ be the view of party $\Pc$ in a random execution of $\Pi(x,y)(1^\pk)$. Then for every $\pk,n \in \N$, $x\in \oo^n$ and neighboring $y,y'\in\oo^n$:
	\begin{align*}
	\pr{\Dc(V^\Ac(x,y)(\pk))=1}\le e^{\eps(\pk)}\cdot \pr{\Dc(V^\Ac (x,y')(\pk))=1}+\delta(\pk),
	\end{align*} 
	and for every $y\in \oo^n$ and neighboring $x,x'\in\oo^{n}$:
	\begin{align*}
	\pr{\Dc(V^\Bc(x,y)(\pk))=1}\le e^{\eps(\pk)}\cdot \pr{\Dc(V^\Bc (x',y)(\pk))=1}+\delta(\pk).
	\end{align*} 	
	Protocol $\Pi$ is {\sf $(\eps,\delta)$-computational differentially private}, denoted $(\eps,\delta)$-$\CDP$, if the above inequalities only hold for a non-uniform \ppt $\Dc$ and large enough $\pk$. We omit $\delta = \negl(\pk)$ from the notation. \footnote{Note that define we give for two-party differentially private protocols is a semi-honest definition, in which we ask for the security to hold when the parties interact in an honest execution of the protocol. Since we are proving a lower bound, starting from this weaker guarantee (as opposed to security against malicious players), yields a stronger result.}
\end{definition}
%We omit $\delta$ from the notation if $\delta$ is a negligible function of $n$.

%\Enote{simulation-based}
\begin{remark}[The definition for computational differential privacy we use]\label{rem:comDPChannel} 
	An alternative, stronger definition of computational differential privacy, known as simulation-based computational differential privacy, requires that the distribution of each party’s view be computationally indistinguishable from a distribution that ensures privacy in an information-theoretic sense. \cref{def:DP} is a weaker notion in comparison. Consequently, establishing a lower bound for a protocol that satisfies this weaker guarantee (as we do in this work) yields a stronger result.%Actually, our lower bound only requires the privacy to hold against \emph{uniform} external observer.
	%\Nnote{Maybe add: When only interesting in \Dp against external observer, the two definitions can be achieve using key-agreement and (single-party) \Dp mechanism. }
\end{remark}




\subsection{Useful Claims}
\remove{
In this section, we state generic lemmas and propositions that we will use later in our proofs.

The following lemma which we prove in \cref{sec:missing-proofs:distance-I}, measures the distance between two uniform stings conditioned one a random index $i$ either being fixed to $0$ or to $1$.

\def\distanceILemma{
    Let $R \la \zo^n$. For any (randomized) function $f:\{0,1\}^n\rightarrow \{0,1\}$ and $\alpha > 0$, it holds that
    \begin{align}\label{eq:f-alpha}
        \ppr{i \la [n]}{\size{\:\ex{f(R) \mid R_i = 0}-\ex{f(R) \mid R_i = 1}\:}\geq \alpha} \leq \frac{2}{n \alpha^2},
    \end{align}
    where the expectations are taken over $R$ and the randomness of $f$.
}

\begin{lemma}\label{lem:distance-I}
    \distanceILemma
\end{lemma}
}

The following two propositions state that given the output of a differentially private function, it is not possible to predict well even a random index (even if all other indexes are leaked). The first proposition handles the information-theoretic case and the second handles the computation case. Both propositions are proven in \cref{sec:missing-proofs:hard-to-guess}. 

\def\propHardToGuessInf{
    Let $f\colon \oo^n \rightarrow \cY$ be an $(\eps,\delta)$-\DP function, let $g \colon [n] \times \oo^{n-1} \times \cY \rightarrow \set{-1,1,\bot}$ be a (randomized) function, and let $X = (X_1,\ldots,X_n) \la \oo^n$. Then the following holds for every $i \in [n]$ where $X_i^* = g(i,X_{-i},f(X_1,\ldots,X_n))$:
    \begin{align*}
        \pr{X_i^* = X_i} \leq e^{\eps}\cdot \pr{X_i^* = -X_i} + \delta.
    \end{align*}
}

\begin{proposition}\label{prop:hard-to-guess-inf}
    \propHardToGuessInf
\end{proposition}


\def\propHardToGuessComp{
    Let $f = \set{f_{\pk} \colon \oo^{n(\pk)} \rightarrow \zo^{m(\pk)}}_{\pk \in \bbN}$ be an $(\eps,\delta)$-\CDP function ensemble, and let $\set{g_{\pk}}_{\pk \in \bbN}$ be a poly-size circuit family. Then, for large enough $\pk$ and $X = (X_1,\ldots,X_{n(\pk)}) \la \oo^{n(\pk)}$, the following holds for every $i \in [n(\pk)]$ where $X_i^* = g_{\pk}(i,X_{-i},f_{\pk}(X_1,\ldots,X_n))$:
    \begin{align*}
        \pr{X_i^* = X_i} \leq e^{\eps(\pk)}\cdot \pr{X_i^* = -X_i} + \delta(\pk).
    \end{align*}
}

\begin{proposition}\label{prop:hard-to-guess-comp}
    \propHardToGuessComp
\end{proposition}





\remove{
\Enote{Chao's old statement:}
\begin{lemma}\label{lem:distance-I-old}
        Let $R \la \zo^n$. 
	For any function $f:\{0,1\}^n\rightarrow \{0,1\}$ and $\alpha<0.01$, it holds that
	$$
	\Pr_{i\la[n]}\left[\: \size{\:\mathbb{E}[f(R) \mid R_i = 0]-\mathbb{E}[f(R) \mid R_i = 1]\:}\geq \alpha\right]\leq \frac{2+2\log(\frac{1}{\alpha})}{n\alpha^2}.
	$$
\end{lemma}
\begin{proof}
	Define $S_1=\{r \in \zo^n \colon f(r)=1\}$. Then for any $i\in[n]$, we have
	$$
	\begin{array}{rl}
		\size{\mathbb{E}[f(R) \mid R_i = 0]-\mathbb{E}[f(R) \mid R_i = 1]}
		&=\size{\Pr[R\in S_1|R_i=0]-\Pr[R\in S_1|R_i=1]}\\
		&=\size{\frac{\Pr[R_i=0|R\in S_1]\cdot\Pr[R\in S_1]}{\Pr[R_i=0]}-\frac{\Pr[R_i=1|R\in S_1]\cdot\Pr[R\in S_1]}{\Pr[R_i=1]}}\\
		&=\frac{2\size{S_1}}{2^n}\size{\Pr[R_i=0|R\in S_1]-\Pr[R_i=1|R\in S_1]}
	\end{array}
	$$
	When $|S_1|\leq \alpha\cdot 2^{n-1}$, we have $\size{\mathbb{E}[f(R) \mid R_i = 0]-\mathbb{E}[f(R) \mid R_i = 1]}\leq\frac{2\size{S_1}}{2^n}\leq \alpha$ for any $i\in[n]$. Hence, in the following, we assume $|S_1|> \alpha\cdot 2^{n-1}$.

	%Define $I_{bad}=\{i|\size{\Pr[R_i=0|R\in S_1]-\Pr[R_i=1|R\in S_1]}>2\alpha\}$ and $k=\size{I_{bad}}$, then for any $i\notin I_{bad}$, we have 
    %$$
    %\begin{array}{rl}
    %    2\alpha&\geq \size{\Pr[R_i=0|R\in S_1]-\Pr[R_i=1|R\in S_1]}\\
    %    &=\size{\frac{\Pr[R\in S_1|R_i=0]\cdot\Pr[R_i=0]}{\Pr[R\in S_1]}-\frac{\Pr[R\in S_1|R_i=1]\cdot\Pr[R_i=1]}{\Pr[R\in S_1]}}\\
    %    &=\size{\Pr[R\in S_1|R_i=0]-\Pr[R\in S_1|R_i=1]}\cdot\frac{1}{2\Pr[R\in S_1]}\\
    %    &\geq \size{\mathbb{E}[f(R) \mid R_i = 0]-\mathbb{E}[f(R) \mid R_i = 1]}\cdot \frac{1}{2},
    %\end{array}
    %$$ 
    %where the last inequality is because $\Pr[R\in S_1]\leq 1$. So that $\size{\mathbb{E}}[f(R) \mid R_i = 0]-\mathbb{E}[f(R) \mid R_i = 1]\leq %4\alpha$.
    Define $I_{bad}=\{i \colon \size{\Pr[R_i=0|R\in S_1]-\Pr[R_i=1|R\in S_1]} \geq 2\alpha\}$ and $k=\size{I_{bad}}$, and denote $I_{bad}=\{i_1,\dots,i_k\}$. Define $(X_{i_1}, \ldots X_{i_k}) = (R_{i_1},\dots,R_{i_k})\mid_{R \in S_1}$. 
    Consider the min-entropy
	$$
	\begin{array}{rl}
		H_{min}(X_{i_1},\dots,X_{i_k})&\leq H(X_{i_1},\dots,X_{i_k})\\
		&\leq \sum_{j=1}^k H(X_{i_j})\\
		&\leq k\cdot \left(-(\frac{1}{2}+2\alpha)\cdot\log(\frac{1}{2}+2\alpha)-(\frac{1}{2}-2\alpha)\cdot\log(\frac{1}{2}-2\alpha)\right)\\
            &=k\cdot \left(-(\frac{1}{2}+2\alpha)\cdot(\log(1+4\alpha)-1)-(\frac{1}{2}-2\alpha)\cdot(\log(1-4\alpha)-1)\right)\\
            &=k\cdot \left(1-(\frac{1}{2}+2\alpha)\cdot\log(1+4\alpha)-(\frac{1}{2}-2\alpha)\cdot\log(1-4\alpha)\right),
		
	\end{array}
	$$
	where $H_{min}(Y)$ is the minimum entropy of $Y$ and $H(Y)$ is the Shannon entropy of $Y$.\Enote{add to preliminaries.}
        The third inequality holds since by the definition of $I_{bad}$, for every $j \in [k]$ it holds that $\size{\pr{X_{i_j} = 1}-\pr{X_{i_j} = 0}} > 2\alpha$, and therefore $H(X_{i_j}) \leq H(1/2 + 2\alpha)$\Enote{define}.
	
	Therefore, there exists $b_1,\dots,b_k\in\{0,1\}$, such that 
	
	\begin{align}\label{eq:min-entropy-result}
		\Pr\left[(R_{i_1},\ldots,R_{i_k}) = (b_1,\ldots,b_k) \mid R\in S_1\right]
		&= \pr{(X_{i_1},\ldots,X_{i_k}) = (b_1,\ldots,b_k)}\\
		&= 2^{-H_{min}(X_{i_1},\dots,X_{i_k})}\nonumber\\
		&\geq 2^{k\cdot \left(-1+(\frac{1}{2}+2\alpha)\cdot\log(1+4\alpha)+(\frac{1}{2}-2\alpha)\cdot\log(1-4\alpha)\right)}.\nonumber
	\end{align}
	
	Let $S_{bad}=\{r \in \zo^n  \colon \set{(r_{i_1},\ldots,r_{i_k}) = (b_1,\ldots,b_k)} \land \set{r\in S_1}\}$.
	It holds that
	\begin{align*}
		|S_{bad}|
		&= \size{S_1} \cdot \Pr\left[(R_{i_1},\ldots,R_{i_k}) = (b_1,\ldots,b_k) \mid R\in S_1\right]\\
		&\geq \alpha\cdot 2^{n-1}\cdot2^{k\cdot \left(-1+(\frac{1}{2}+2\alpha)\cdot\log(1+4\alpha)+(\frac{1}{2}-2\alpha)\cdot\log(1-4\alpha)\right)},
	\end{align*} 
	where the inequality holds by \cref{eq:min-entropy-result} and since $\size{S_1} \geq \alpha\cdot 2^{n-1}$.
	Notice that any string in $S_{bad}$ depends on at most $n-k$ bits. It implies that $|S_{bad}|\leq 2^{n-k}$. Therefore, we have
	$$
	\begin{array}{rl}
		&2^{n-k}\geq \alpha\cdot 2^{n-1}\cdot2^{k\cdot \left(-1+(\frac{1}{2}+2\alpha)\cdot\log(1+4\alpha)+(\frac{1}{2}-2\alpha)\cdot\log(1-4\alpha)\right)} \\
		\Rightarrow& n-k \geq \log \alpha+n-1+k\cdot \left(-1+(\frac{1}{2}+2\alpha)\cdot\log(1+4\alpha)+(\frac{1}{2}-2\alpha)\cdot\log(1-4\alpha)\right)\\
		\Rightarrow& 1-\log \alpha \geq k\cdot((\frac{1}{2}+2\alpha)\cdot\log(1+4\alpha)+(\frac{1}{2}-2\alpha)\cdot\log(1-4\alpha))\\
		\Rightarrow& 1-\log \alpha \geq k\cdot(4\alpha\cdot\log(1+4\alpha)+(\frac{1}{2}-2\alpha)\cdot\log(1-16\alpha^2))\\
        \Rightarrow& 1-\log\alpha \geq k\cdot(15.9\alpha^2-8\alpha^2+32\alpha^3)=k\cdot(7.9\alpha^2+32\alpha^3)>0.5k\alpha^2\\
		\Rightarrow& k\leq \frac{2-2\log \alpha}{\alpha^2} = \frac{2+2\log (1/\alpha)}{\alpha^2},
	\end{array}
	$$
	Where the third transition holds since 
	\begin{align*}
		\lefteqn{(\frac{1}{2}+2\alpha)\cdot\log(1+4\alpha)+(\frac{1}{2}-2\alpha)\cdot\log(1-4\alpha)}\\
		&= 4\alpha\cdot\log(1+4\alpha) + (\frac{1}{2}-2\alpha)\paren{\log(1+4\alpha)+\log(1-4\alpha)}\\
		&= 4\alpha\cdot\log(1+4\alpha)+(\frac{1}{2}-2\alpha)\cdot\log(1-16\alpha^2),
	\end{align*}
	and the forth transition holds since $4\alpha\cdot\log(1+4\alpha)+(\frac{1}{2}-2\alpha)\cdot\log(1-16\alpha^2) > 15.9\alpha^2-8\alpha^2+32\alpha^3$ for $\alpha < 0.01$.
	Thus, we conclude that 
	$$
	\Pr_{i\la[n]}\left[\size{\mathbb{E}[f(R) \mid R_i=0]-\mathbb{E}[f(R) \mid R_i = 1]}\geq \alpha\right]\leq \frac{k}{n}\leq \frac{2+2\log (1/\alpha)}{n\alpha^2}.
	$$
\end{proof}
}


\subsection{Channels and Two-Party Protocols}\label{sec:protocol}

\paragraph{Channels.}A channel is simply a distribution of a pair of tuples defined as follows. 
\begin{definition}[Channels]\label{def:channel} A {\sf channel} $C_{(X,U)(Y,V)}$ of size $\isize$ over alphabet $\Sigma$ is a probability distribution over $(\Sigma^\isize \times\zo^\ast) \times(\Sigma^\isize \times\zo^\ast)$. The ensemble $C_{(X,U)(Y,V)}= \set{C_{(X_\pk,U_\pk)(Y_\pk,V_\pk)}}_{\pk\in \N}$ is an $\isize$-size channel ensemble, if for every $\pk\in \N$, $C_{(X_\pk,U_\pk)(Y_\pk,V_\pk)}$ is an $\isize(\pk)$-size channel. %We denote a channel of size one by a \emph{single-bit} channel. 
We refer to $X$ and $Y$ as the {\sf local outputs}, and to $U$ and $V$ as the {\sf views}.	
\end{definition}

We view a  channel as the experiment in which there are two parties $\Ac$ and $\Bc$.  Party $\Ac$ receives ``output'' $X$ and ``view'' $U$, and party $\Bc$ receives ``output'' $Y$ and ``view'' $V$. Unless stated otherwise, the channels we consider are over the alphabet $\Sigma = \oo$. We naturally identify channels with the distribution that characterizes their output.








\subsubsection{Two-Party Protocols}

A two-party protocol $\Pi=(\Ac,\Bc)$ is \ppt if the running time of both parties is polynomial in their input length. We let $\Pi(x,y)(z)$ or $(\Ac(x),\Bc(y))(z)$ denote a random execution of $\Pi$ on a common input $z$, and private inputs $x,y$.%We assume \wlg that a protocol has a common output (part of its transcript).\Jnote{This is not really the case we consider in this paper..}

\begin{definition}[Oracle-aided protocols]\label{def:ChannelAidedProtocol}
	In a two-party protocol $\Pi$ with oracle access to a {\sf protocol} $\Psi$, denoted $\Pi^\Psi$, the parties make use of the \textit{next-message function} of $\Psi$.\footnote{The function that on a partial view of one of the parties, returns its next message.} In a two-party protocol $\Pi$ with oracle access to a {\sf channel} $C_{Z W}$, denoted $\Pi^C$, the parties can jointly invoke $C$ for several times. In each call, an independent pair $(z,w)$ is sampled according to $C_{Z W}$, one party gets $z$, the other gets $w$.
\end{definition}


\begin{definition}[The channel of a protocol]\label{def:ChannlOfProtocol}
	For a no-input two-party protocol $\Pi= (\Ac,\Bc)$, we associate the channel $C_\Pi$, defined by $\C_\Pi= C_{(X, U),(Y, V)}$, where $X$ and $Y$ are the local outputs of $\Ac$ and $\Bc$ (respectively) and
	$U$ and $V$ are the local views of $\Ac$ and $\Bc$ (respectively).
    
	For a two-party protocol $\Pi$ that gets a security parameter $1^\pk$ as its (only, common) input, we associate the channel ensemble $ \set{C_{\Pi(1^\pk)}}_{\pk\in \N}$. 
\end{definition}

\begin{definition}[$(\alpha,\gamma)$-Accurate channel]\label{def:accurate-func}
	A channel $C = C_{(X, U),(Y, V)}$ is {\sf $(\alpha,\gamma)$-accurate for the function $f$}, if $\ppr{C}{\size{\out(V)-f(X,Y)}\leq \alpha}\ge \gamma$, where $\out(V)$ is the designated output.
    A channel ensemble $C_{(X, U),(Y, V)}= \set{C_{(X_\pk, U_\pk),(Y_\pk, V_\pk)}}_{\pk\in \N}$ is  $(\alpha,\gamma)$-accurate for  $f$ if $C_{(X_\pk, U_\pk),(Y_\pk, V_\pk)}$ is $(\alpha(\pk),\gamma(\pk))$-accurate for $f$, for every $\pk \in \N$.
\end{definition}

\subsubsection{Differentially Private Channels}\label{sec:DPChannel}
Differentially private channels are naturally defined as follows:
\begin{definition}[Differentially private channels]\label{def:DPChannel}
	An $n$-size channel $C = C_{(X, U),(Y, V)}$ with $X, Y$ over $\oo^n$ 
	is {\sf$(\eps,\delta)$-differentially private} (denoted $(\eps,\delta)$-$\DP$) if for every $x \in \Supp(X)$ there exists an $n$-size $(\eps,\delta)$-$\DP$ mechanisms $\Mc_x$ such that $(X,Y,U) \equiv (X,Y,\Mc_X(Y))$, and for every $y \in \Supp(Y)$ there exists an $n$-size $(\eps,\delta)$-$\DP$ mechanisms $\Mc_y'$ such that $(X,Y,V) \equiv (X,Y,\Mc_Y'(X))$. In addition, we say that the channel is \emph{uniform} if $X$ and $Y$ are independent random variables uniformly distributed in $\oo^n$. 
\end{definition}

\begin{definition}[Computational differentially private channels]\label{def:CDPChannel}
	An $n$-size channel ensemble $C = \set{C_{(X_\pk, U_\pk),(Y_\pk, V_\pk)}}_{\pk\in\N}$ with $X_\pk, Y_\pk$ over $\oo^n$ 
	is {\sf$(\eps,\delta)$-computationally differentially private} (denoted $(\eps,\delta)$-$\CDP$) if for every ensemble $\set{x_\pk \in \Supp(X_\pk)}_{\pk\in\N}$ there exists an $n$-size $(\eps,\delta)$-\CDP mechanisms ensemble $\set{\Mc_{x_\pk}}_{\pk\in\N}$ such that $(X_\pk,Y_\pk,U_\pk) \equiv (X_\pk,Y_\pk,\Mc_{X_\pk}(Y_\pk))$, for every $\pk\in\N$, and for every ensemble $\set{y_\pk \in \Supp(Y_\pk)}_{\pk\in\N}$ there exists an $n$-size $(\eps,\delta)$-$\CDP$ mechanisms ensemble $\set{\Mc'_{y_\pk}}_{\pk\in\N}$ such that $(X_\pk,Y_\pk,V_\pk) \equiv (X_\pk,Y_\pk,\Mc_{Y_\pk}'(X_\pk))$ for every $\pk\in \N$. In addition, we say that the channel is \emph{uniform} if $X_\pk$ and $Y_\pk$ are independent random variables uniformly distributed in $\{\pm 1\}^n$ for all $\pk\in\N$.
\end{definition}




% \begin{lemma}~\label{lem:dp-sv-source}
% 	Let $P$ be an $\varepsilon$-DP randomized protocol. Let $X$ and $Y$ be independent random variables uniformly distributed in $\{\pm 1\}^n$ and let random variable $\Pi(X,Y)$ denote the transcript of running $P(X,y)$. Then for every $\pi\in Supp(\Pi)$, the random variables corresponding to the inputs conditioned on transcript $\pi$, $X_\pi$ and $Y_\pi$, are independent $e^{-\varepsilon}$-strong SV source.
% \end{lemma}





\subsubsection{Weak Erasure Channel (\WEC)}

\begin{definition}[\WEC]\label{def:WEC}
	A channel $((O_A,V_A), (O_B,V_B))$ with $O_A \in \set{0,1}$ and $O_B \in \set{0,1,\bot}$ is a {\sf weak erasure channel}, denoted $(\alpha,p,q)$-$\WEC$, if:
	\begin{itemize}
		%\item $O_A\in \set{-1,1}$ and $O_B\in \set{-1,1,\bot}$.
		\item Random erasure: $\pr{O_B = \perp} = 1/2$.
		
		\item Agreement: $\pr{O_A\ne O_B\mid O_B\ne \bot}\le \alpha$.
		
		\item Secrecy:
		
		\begin{enumerate}
			\item For every algorithm $\Dc$ it holds that\label{WEC:item:A}
			\begin{align*}
				%\size{\pr{\Ac(O_A,V_A) = 1 \mid O_B \neq \perp} - \pr{\Ac(O_A,V_A) = 1 \mid O_B = \perp}} \le p
				\size{\pr{\Dc(V_A) = 1 \mid O_B \neq \perp} - \pr{\Dc(V_A) = 1 \mid O_B = \perp}} \le p
			\end{align*}
			(Alice doesn't know if $O_B = \perp$.)
			
			\item For every algorithm $\Dc$ it holds that\label{WEC:item:B}
			\begin{align*}
				\pr{\Dc(V_B) = O_A \mid O_B=\bot} \leq \frac{1+q}{2}.
			\end{align*}
			(i.e., if $O_B=\bot$, Bob don't know what is the value of $O_A$).
			
			%\item $SD((O_A U|O_B=\bot),(O_A U|O_B\ne \bot))\le p$ (The sender don't know if $O_B=\bot$).
			
			%\item $SD(V O_A|O_B=\bot,V(-O_A)|O_B=\bot)\le q$ (If $O_B=\bot$, Bob don't know what the value of $O_A$).
		\end{enumerate}
	\end{itemize}
   We say that a channel ensemble $C=\set{C_\pk}_{\pk\in N}$ is a {\sf computational weak erasure channel}, denoted $(\alpha,p,q)$-\CompWEC, if for every \ppt algorithm $\Dc$ and every sufficiently large $\pk\in\N$, $C_\pk$ satisfies the properties stated in the items above, where the secrecy property holds with respect to a \ppt algorithm $\Dc$. A protocol $\Lambda$ is said to be $(\alpha,p,q)$-$\CompWEC$, if the ensemble induces by the protocol (that is, $C=\set{C_{\Lambda(\pk)}}_{\pk\in\N}$) is $(\alpha,p,q)$-$\CompWEC$.  
\end{definition}



\subsubsection{Approximate Weak Erasure Channel (\AWEC)}\label{sec:AWEC}

\begin{definition}[\AWEC]\label{def:AWEC}
	A channel $C = ((O_A,V_A), (O_B,V_B))$ over $([-n,n] \times \zo^*) \times (([-n,n] \cup \bot)  \times \zo^*)$ is an {\sf approximate weak erasure channel}, denoted $(\ell,\alpha,p,q)$-\AWEC if:
	\begin{itemize}
		
		\item Random erasure: $\pr{O_B = \perp} = 1/2$.
		
		\item Accuracy: $\pr{\size{O_A - O_B} > \ell \mid O_B \ne \bot}\le \alpha$.
		
		\item Secrecy:
		
		\begin{enumerate}
			\item For every algorithm $\Dc$ it holds that\label{AWEC:item:A}
			\begin{align*}
				%\size{\pr{\Ac(O_A,V_A) = 1 \mid O_B \neq \perp} - \pr{\Ac(O_A,V_A) = 1 \mid O_B = \perp}} \le p
				\size{\pr{\Dc(V_A) = 1 \mid O_B \neq \perp} - \pr{\Dc(V_A) = 1 \mid O_B = \perp}} \le p
			\end{align*}
			(Alice doesn't know if $O_B=\bot$).
			
			\item For every algorithm $\Dc$ it holds that\label{AWEC:item:B}
			\begin{align*}
				\pr{\size{\Dc(V_B) - O_A} \leq 1000 \ell \mid O_B=\bot} \leq q.
			\end{align*}
			(i.e., if $O_B=\bot$, Bob can't estimate the value of $O_A$ with error $\leq 1000 \ell$).
		\end{enumerate}
	\end{itemize}
     We say that a channel ensemble $C=\set{C_\pk}_{\pk\in N}$ is a {\sf computational approximate weak erasure channel}, denoted $(\ell,\alpha,p,q)$-\CompAWEC, if for every \ppt algorithm $\Dc$ and every sufficiently large $\pk\in\N$, $C_\pk$ satisfies the properties stated in the items above. A protocol $\Gamma$ is said to be $(\ell,\alpha,p,q)$-$\CompAWEC$, if the ensemble induced by the protocol (that is, $C=\set{C_{\Gamma(\pk)}}_{\pk\in\N}$) is $(\ell,\alpha,p,q)$-$\CompAWEC$.  
\end{definition}

We will make use of the following lemma, which shows that for some choices of the parameters, \AWEC implies \WEC. The lemma is proven in \cref{sec:AWEC-to-WEC}.

\begin{lemma}\label{lemma:AWEC-to-WEC}
	For every $\ell> 0$, there exists a \ppt protocol $\Lambda = (\Pc_1,\Pc_2)$ such that given an oracle access to an $(\ell,\alpha,p,q)$-\AWEC $C$, the channel $\tilde{C}$ induced by $\Lambda^C$ is $(\alpha'=\alpha+0.001,\: p' = p ,\:  q' = 1/2 + 2(q+0.01))$-\WEC.
	Furthermore, the proof is constructive in a black-box manner:
	\begin{enumerate}
		\item There exists an oracle-aided \ppt algorithm $\Ec_1$ such that for every channel $C = ((\OA,\VA), (\OB,\VB))$ and algorithm $\Dc$ violating the \WEC secrecy property~\ref{WEC:item:A} of $\tilde{C}$, algorithm $\Ec_1^{\Dc}$ violates the \AWEC secrecy property~\ref{AWEC:item:A} of $C$.
		
		\item There exists an oracle-aided \ppt algorithm $\Ec_2$ such that for every channel $C = ((\OA,\VA), (\OB,\VB))$ and algorithm $\Dc$ violating the \WEC secrecy property~\ref{WEC:item:B} of $\tilde{C}$, algorithm $\Ec_2^{\Dc}$ violates the \AWEC secrecy property~\ref{AWEC:item:B} of $C$.
	\end{enumerate}
\end{lemma}

Since \cref{lemma:AWEC-to-WEC} is constructive, the following is an immediate corollary.
\begin{corollary}\label{cor:CompAWEC to CompWEC}
There exists an oracle aided \ppt protocol $\Lambda$, such that given a protocol $\Gamma$ that induces $(\ell,\alpha,p,q)$-\CompAWEC, it holds that $\Lambda^\Gamma$ is $(\alpha'=\alpha+0.001,\: p' = p ,\:  q' = 1/2 + 2(q+0.01))$-\CompWEC.  
\end{corollary}
\begin{proof}[Proof of \ref{cor:CompAWEC to CompWEC}]
Let $\Lambda$ be the \ppt algorithm guaranteed  by Lemma \ref{lemma:AWEC-to-WEC}. Given an $(\ell,\alpha,p,q)$-\CompAWEC protocol $\Gamma$, we define $\Lambda(\pk)={\Lambda^{\Gamma(\pk)}(\pk)}$. Assume towards a contradiction that $\Lambda$ is not a $(\alpha',p',q')$-\CompWEC. It follows that there exists a \ppt $\Dc$ that for infinity many $\pk\in\N$ contradicts one of the \WEC secrecy properties of channel ensemble $\set{C_{\Lambda(\pk)}}_{\pk\in\N}$. Fix $\pk\in\N$ for which this holds. By Lemma \ref{lemma:AWEC-to-WEC}, there exists a \ppt $\Ec^\Dc$ that for every such $\pk$  contradicts one of the secrecy properties of the channel $C_{\Gamma(\pk)}$. This implies that for infinity many $\pk\in\N$, $\Ec^\Dc$  contradict the secrecy of the channel ensemble $\set{C_{\Gamma(\pk)}}_{\pk\in\N}$, which is a contradiction since this would means that $\Gamma$ is not a $(\ell,\alpha,p,q)$-\CompAWEC.       
\end{proof}



\subsection{Oblivious Transfer (\OT)}

\paragraph{Secure Computation.}
We use the standard notion of securely computing a functionality, \cf  \cite{Goldreich04}.
\begin{definition}[Secure computation]\label{def:SFE}
	A two-party protocol {\sf securely computes a functionality $f$}, if it does so according to the real/ideal paradigm.   We add the term perfectly/statistically/computationally/non-uniform computationally, if the simulator's output is  perfect/statistical/computationally indistinguishable/  non-uniformly indistinguishable from  the real distribution.  The protocol have the above notions of security {\sf against semi-honest  adversaries}, if its security only  guaranteed to holds against an adversary that follows the prescribed protocol.   Finally, for the case of perfectly secure computation, we naturally apply the above notion also to the non-asymptotic case: the protocol with no security parameter perfectly  compute a functionality $f$.
	
	A two-party protocol {\sf securely computes a functionality ensemble $f$ with oracle to a channel $C$}, if it does so according to the above definition when the parties have access to a trusted party computing $C$. All the above adjectives naturally extend to this setting.
\end{definition}

\paragraph{Oblivious Transfer.}
The (one-out-of-two) oblivious transfer functionality is defined as follows.
\begin{definition}[oblivious transfer functionality $f_{\OT}$]\label{def:OTfunc}
	The oblivious transfer functionality over $\zo \times (\zs)^2$ is defined by  $f_{\OT} (i,(\sigma_0,\sigma_1)) = (\perp,\sigma_i)$.
\end{definition}
A protocol is $\ast$ secure OT,   for \\$\ast\in \set{\text{semi-honest statistically/computationally/computationally non-uniform}}$, if it  compute the $f_{\OT}$  functionality with $\ast$ security.





% \begin{definition}[Computational oblivious transfer, semi-honest model]
% A protocol $\Pi=(\Ac,\Bc)$ is a semi-honest 1-out-of-2 computational oblivious transfer (comp-OT) protocol if the following holds. Given a common input $1^{\pk}$, the parties $\Ac$ and $\Bc$ run the protocol $\Pi(1^\pk)$ (in an honest manner) and    
% $\Ac$ outputs $X=(m_1,m_2)\in \zo\times\zo$ and has a view $U$ and $\Bc$ outputs $Y=(i,\hat{m})\in\zo\times\zo$ and has a view $V$, and the following properties are satisfied:
% \begin{enumerate}
%     \item \textbf{Correctness:} 
%     $\pr{\hat{m}\neq m_i}<\negl(\pk).$ 
    
%     \item \textbf{A's Privacy:} For every \ppt $\Dc$ and every sufficiently large $\pk$:
%     $\pr{\Dc(V)=m_{i-1}}<(1+\negl(\pk))/2$
    
%     \item \textbf{B's Privacy:} For every \ppt $\Dc$ and every sufficiently large $\pk$:
%     $\pr{\Dc(U)=i}<(1+\negl(\pk))/2$  
% \end{enumerate}
% \end{definition}

We make use of the following useful results by Wullschleger on oblivious transfer amplification from weak channels.
\begin{theorem}[\cite{Wullschleger09}, from \WEC to statistically secure \OT]\label{thm:WEC TO OT IT}
    There exists an oracle aided protocol $\Pi$ such that the following holds: Given a $(\alpha,p,q)$-\WEC $C$, if $44(\alpha+p)\le 1-q$ then $\Pi^{C}(1^\pk)$ is a semi-honest statistically secure \OT.
\end{theorem}

The following computational version of \cref{thm:WEC TO OT IT} is implicit in \cite{Wullschleger09} and is based on the computational proof explicitly stated in \cite{Wul07} (see Section 6 in \cite{Wullschleger09} for discussion).   

\begin{theorem}[\cite{Wullschleger09,   Wul07}, from \CompWEC to computinally secure \OT]\label{thm:WEC TO OT Comp}
    There exists an oracle aided protocol $\Pi$ such that the following holds: Given a $(\alpha,p,q)$-\CompWEC protocol $\Lambda$, if $44(\alpha+p)\le 1-q$ then $\Pi^{\Lambda}$ is a semi-honest computational secure \OT.
\end{theorem}



% \begin{definition}[Computational 1-out-of-2 Oblivious Transfer, semi-honest model]
% A protocol $\Pi=(\Ac,\Bc)$ is a semi-honest 1-out-of-2 $(\eps,\alpha,\beta)$-oblivious transfer (OT) protocol if the following holds. 

% The parties $\Ac$ and $\Bc$ run the protocol (in an honest manner) and    
% $\Ac$ outputs $X=(m_1,m_2)\in \zo\times\zo$ and has a view $U$ and $\Bc$ outputs $Y=(i,\hat{m})\in\zo\times\zo$ and has a view $V$, and following properties are satisfied:
% \begin{enumerate}
%     \item \textbf{Correctness:} 
%     $\pr{\hat{m}\neq m_i}<\eps.$ 
    
%     \item \textbf{A's Privacy:} For every adversary $\Dc$:
%     $\pr{\Dc(V)=m_{i-1}}<(1+\alpha)/2$
    
%     \item \textbf{B's Privacy:} For every adversary $\Dc$: $\pr{\Dc(U)=i}<(1+\beta)/2$  
% \end{enumerate}
% \end{definition}
\section{Approximate Richardson Iteration}
\label{sec:lifted_regression}

We now present our main techniques for reducing tensor completion to tensor decomposition.
When we use ALS to solve a TC problem,
we must efficiently solve least-squares problems $\min_{\mat x} \, \norm{\mat{Px}-\mat q}_2$.
The rows of the design matrix $\mat{P}$ correspond to the \emph{subset of observations} in the TC problem,
so $\mat{P}$ does not necessarily have the structure of the design matrix in the full TD problem.

A direct approach is to compute the closed-form solution $\mat x^* = (\mat P^\top \mat P)^{-1}\mat P^\top \mat q$, but computing $(\mat P^\top \mat P)^{-1}$ is often impractical.
Two techniques are commonly used to overcome this:
(1) iterative methods and (2) row sampling.
Iterative methods repeat the same relatively cheap per-step computation
\emph{many times} to approximate the original expensive computation.
Row sampling methods (e.g., leverage score sampling) randomly pick rows of $\mat P$
and solve a least-squares problem on the sampled rows to obtain an approximate solution to the original problem with high probability.
Directly computing leverage scores for a general $\mat{P}$, however,
is \emph{also prohibitively expensive} since it requires computing
the same matrix $(\mat P^\top \mat P)^{-1}$ (see \Cref{app:leverage-score} for details).

We show that our novel \emph{approximate-mini-ALS} method
is a principled approach for tensor completion.
In \Cref{ssec:lifting}, we show how lifting \emph{restores the structure} of the full TD ALS update step,
enabling fast least-squares methods for a larger (but equivalent) problem.
In \Cref{subsec:iterative_methods}, we show that iteratively solving this lifted problem (i.e., mini-ALS) is connected to an iterative method called the \emph{Richardson iteration}~\citep{richardson1911approximate},
which we can also view as a matrix-splitting method.
In other words, mini-ALS and the Richardson iteration with a certain preconditioner give the same sequence of iterates $\{\mat{x}^{(k)}\}_{k \ge 0}$.
Lastly in~\Cref{sec:approx-solve-lifted},
we prove novel convergence guarantees for \emph{approximately} solving the lifted problem
(i.e., for approximate-mini-ALS).
This allows us to directly use fast leverage-score sampling algorithms for
CP decomposition~\citep{cheng2016spals,larsen2022practical,bharadwaj2023fast},
Tucker decomposition~\citep{diao2019optimal,fahrbach2022subquadratic},
and TT decomposition~\citep{bharadwaj2024efficient}
as blackbox subroutines.

\subsection{Lifting to a Structured Problem}
\label{ssec:lifting}
Consider the linear regression problem with $\mat{P}\in\R^{|\Omega| \times R}$ and $\mat{q} \in \R^{|\Omega|}$ given by
\begin{equation}
\label{eqn:input_regression}
    \mat{x}^* = \argmin_{\mat{x} \in \R^R} \,\norm{\mat{P} \mat{x} - \mat{q}}_{2}^2\,.
\end{equation}
If there exists a tall structured matrix $\mat{A} \in \R^{I \times R}$
with a subset of rows $\Omega \subseteq [I]$ such that $\mat{A}_{\Omega} = \mat{P}$
(permutations of the rows allowed),
then we can lift \eqref{eqn:input_regression} to a higher-dimensional problem while preserving the optimal solution.

\begin{restatable}[]{lemma}{LiftedRegression}
\label{lem:lifted_regression}
Let $\mat{b} \in \R^I$ be the lifted response such that $\mat{b}_{\Omega} = \mat{q}$
and $\mat{b}_{\overline{\Omega}}$ is a free variable. If
\begin{equation}
\label{eqn:lifted_regression}
    (\mat{x}^*, \mat{b}^*_{\overline{\Omega}})
    =
    \argmin_{\mat{x} \in \R^R, \mat{b}_{\overline{\Omega}} \in \R^{I - |\Omega|}}\, \norm*{\mat{A} \mat{x} - \mat{b}}_{2}\,,
\end{equation}
then $\mat{x}^*$ also minimizes \eqref{eqn:input_regression},
i.e., the original linear regression problem $\min_{\mat{x} \in \R^R}\, \norm{\mat{P} \mat{x} - \mat{q}}_{2}^2$.
\end{restatable}

\begin{proof}
For any $\mat{x}$, we have
\[
    \norm{\mat{A}\mat{x}-\mat{b}}_2^2
    =
    \norm{\mat{A}_{\Omega}\mat{x} - \mat{b}_{\Omega}}_{2}^2
    +
    \norm{\mat{A}_{\overline\Omega}\mat{x} - \mat{b}_{\overline\Omega}}_{2}^2\,,
\]
so
\[
\min_{\mat{x}}\norm{\mat{P} \mat{x} - \mat{q}}_2^2 \leq \min_{\mat{x},\mat{b}_{\overline{\Omega}}} \norm{\mat{A}\mat{x}-\mat{b}}_2^2\,.
\]
Moreover, for any $\mat{x}$, taking $\mat{b}_{\overline \Omega} = \mat{A}_{\overline \Omega}\mat{x}$ gives us $\norm{\mat{A}_{\overline\Omega}\mat{x} - \mat{b}_{\overline\Omega}}_{2}^2=0$, which implies that
\[
\min_{\mat{x}} \norm{\mat{P} \mat{x} - \mat{q}}_2^2 \geq \min_{\mat{x},\mat{b}_{\overline{\Omega}}} \norm{\mat{A}\mat{x}-\mat{b}}_2^2\,.
\]
Therefore,
\[
\min_{\mat{x}}\norm{\mat{P} \mat{x} - \mat{q}}_2^2 = \min_{\mat{x},\mat{b}_{\overline{\Omega}}} \norm{\mat{A}\mat{x}-\mat{b}}_2^2\,,
\]
and $\mat{x}^*$ also minimizes \eqref{eqn:input_regression}.
\end{proof}

For the rest of this section,
let $\widetilde{\mat{P}} \in \R^{I\times R}$, $\widetilde{\mat{q}}\in \R^{I}$
be the zero-masked lifted matrix and vector such that
\[
    (\widetilde{\mat{P}}_{\Omega}, \widetilde{\mat{P}}_{\overline{\Omega}}) = (\mat{A}_{\Omega}, \boldsymbol{0})
    \quad
    \text{and}
    \quad
    (\widetilde{\mat{q}}_{\Omega}, \widetilde{\mat{q}}_{\overline{\Omega}})
    =
    (\mat{b}_{\Omega}, \boldsymbol{0}).
\]

\begin{restatable}{lemma}{ConvexQuadratic}
\label{lem:lifted_problem_is_convex_quadratic}
Problem~\ref{eqn:lifted_regression} is a convex quadratic program.
\end{restatable}

\begin{proof}
Since $\widetilde{\mat{q}}\in \R^{I}$ is defined as $\widetilde{\mat{q}}_{\Omega} = \mat{b}_{\Omega}$ and $\widetilde{\mat{q}}_{\overline{\Omega}} = \boldsymbol{0}$,
we can write \eqref{eqn:lifted_regression} in the following equivalent manner:
\[
    (\mat{x}^*, \mat{b}^*_{\overline{\Omega}})
    =
    \argmin_{\mat{x} \in \R^R, \mat{b}_{\overline{\Omega}}\in\R^{I-|\Omega|}}
    \norm*{
    \begin{bmatrix}
        \mat{A} & -\mat{I}_{:,\overline{\Omega}}
    \end{bmatrix}
    \begin{bmatrix}
        \mat{x} \\ \mat{b}_{\overline{\Omega}}
    \end{bmatrix}
    -
    \widetilde{\mat{q}}
    }_{2}^2\,,
\]
where $\mat{I}$ is the $I\times I$ identity matrix.
\end{proof}

\begin{remark}
Problem~\ref{eqn:lifted_regression} is  not a linear regression problem
with (structured) design matrix~$\mat{A}$ since there are $\mat{b}_{\overline{\Omega}}$ variables in the response.
There is, however, enough structure to employ block minimization to alternate between minimizing $\mat{x}$ and $\mat{b}_{\overline{\Omega}}$.
\end{remark}

\subsection{Iterative Methods for the Lifted Problem}
\label{subsec:iterative_methods}

Iterative methods for solving linear systems and regression problems have a long history and have been used to expedite several algorithms in theory and practice.
The algorithms we consider use the exact arithmetic model, but all of these methods can be carried out with numbers with $\log \nicefrac{\kappa}\epsilon$ bits, where $\kappa$ is the condition number of the matrix (see, e.g., \citet{ghadiri2023bit,ghadiri2024improving}).
There is a literature on \emph{inexact} Richardson iteration for solving linear systems, 
but they require the error $\widehat{\epsilon}$ to be smaller than than $1/\kappa$,
which is not achievable with leverage-score sampling \cite{golub1988convergence,golub1997closer}.

\begin{lemma}[{Preconditioned Richardson iteration, \citep[Lemma 6.1]{lee2024techniques}}]
\label{lem:richardson_iteration}
Consider the least-squares problem $\mat{x}^* = \argmin_{\mat{x} \in \R^R}\, \norm{\mat{P}\mat{x} - \mat{q}}$.
Let $\mat{M}$ be a matrix such that $\mat{P}^\top \mat{P} \preccurlyeq \mat{M} \preccurlyeq \beta \cdot \mat{P}^\top \mat{P}$
for some $\beta \ge 1$,
and consider the Richardson iteration:
\[
    \mat{x}^{(k+1)} = \mat{x}^{(k)} - \mat{M}^{-1}(\mat{P}^\top \mat{P} \mat{x}^{(k)} - \mat{P}^\top \mat{q})\,.
\]
Then, we have that
\[
    \norm{\mat{x}^{(k+1)} - \mat{x}^*}_{\mat{M}}
    \le
    \Bigl(1 - \frac{1}{\beta}\Bigr)
    \norm{\mat{x}^{(k)} - \mat{x}^*}_{\mat{M}}\,.
\]
\end{lemma}

We now present a key lemma showing that alternating minimization between $\mat{x}$ and $\mat{b}_{\overline{\Omega}}$ corresponds to preconditioned Richardson iterations
on the original least-squares problem.
Below, one can easily check that $\mat{A}$, $\widetilde{\mat{P}}$, and $\widetilde{\mat{q}}$
in our lifted approach satisfy this condition.

\begin{restatable}{lemma}{RichardsonSimulation}
\label{lemma:richardson-simulation}
Let $\mat{A}, \widetilde{\mat{P}} \in \R^{I \times R}$, $\widetilde{\mat{q}}\in\R^{I}$ such that $\widetilde{\mat{P}}-\mat{A}$ and
$\bigl[\begin{matrix}
\widetilde{\mat{P}} & \widetilde{\mat{q}}
\end{matrix}\bigr]$
are orthogonal, i.e., $(\widetilde{\mat{P}}-\mat{A})^\top \bigl[\begin{matrix} \widetilde{\mat{P}} & \widetilde{\mat{q}} \end{matrix}\bigr] = \mat{0}$.
Then, the iterative method
\begin{align*}
    \widetilde{\mat{q}}^{(k)} & = \widetilde{\mat{q}} + (\mat{A} - \widetilde{\mat{P}})\, \mat{x}^{(k)}\,, \\
    \mat{x}^{(k+1)} & = \argmin_{\mat{x} \in \R^{R}}\, \norm{\mat{A} \mat{x} - \widetilde{\mat{q}}^{(k)}}_2^2\,,
\end{align*}
simulates Richardson iterations with preconditioner $\mat{A}^\top \mat{A}$
for the regression problem $\min_{\mat{x}}\, \norm{\widetilde{\mat{P}} \mat{x} - \widetilde{\mat{q}}}_2^2$,
i.e.,
\begin{equation}
    \label{eq:update-rule}
    \mat{x}^{(k+1)}
    =
    \mat{x}^{(k)} - (\mat{A}^\top \mat{A})^{-1} (\widetilde{\mat{P}}^\top \widetilde{\mat{P}} \mat{x}^{(k)} - \widetilde{\mat{P}}^\top \widetilde{\mat{q}})\,.
\end{equation}
\end{restatable}

\begin{proof}
Assume that $\mat{A}^\top \mat{A}$ is full rank.
Solving the normal equation for $\mat{x}^{(k+1)}$,
\begin{align*}
\mat{x}^{(k+1)}
&= (\mat{A}^\top \mat{A})^{-1} \mat{A}^\top \widetilde{\mat{q}}^{(k)}\\
&= (\mat{A}^\top \mat{A})^{-1} \mat{A}^\top (\widetilde{\mat{q}} + (\mat{A} - \widetilde{\mat{P}})\, \mat{x}^{(k)})
\\ & =
\mat{x}^{(k)} - (\mat{A}^\top \mat{A})^{-1} \mat{A}^\top (\widetilde{\mat{P}} \mat{x}^{(k)} - \widetilde{\mat{q}})\,.
\end{align*}
Since $\widetilde{\mat{P}}-\mat{A}$ and $\bigl[\begin{matrix}
\widetilde{\mat{P}} & \widetilde{\mat{q}}
\end{matrix}\bigr]$ are orthogonal, 
\begin{align*}
    \mat{A}^\top (\widetilde{\mat{P}} \mat{x}^{(k)} - \widetilde{\mat{q}})
    &=
    \bigl(\widetilde{\mat{P}} - (\widetilde{\mat{P}} - \mat{A})\bigr)^\top \bigl[\begin{matrix}
        \widetilde{\mat{P}} & \widetilde{\mat{q}}
    \end{matrix}\bigr] \begin{bmatrix}
    \mat{x}^{(k)} \\ -1
    \end{bmatrix} \\
    &= 
    \widetilde{\mat{P}}^\top (\widetilde{\mat{P}} \mat{x}^{(k)} - \widetilde{\mat{q}})\,.
\end{align*}
Therefore,
\[
    \mat{x}^{(k+1)}
    =
    \mat{x}^{(k)} - (\mat{A}^\top \mat{A})^{-1} (\widetilde{\mat{P}}^\top \widetilde{\mat{P}} \mat{x}^{(k)} - \widetilde{\mat{P}}^\top \widetilde{\mat{q}})\,,
\]
which completes the proof.
\end{proof}

\begin{remark}
In the tensor completion setting, $\mat{A}-\widetilde{\mat{P}}$ vanishes over $\Omega$,
so $\widetilde{\mat{q}}^{(k)}$ only updates entries in $\overline{\Omega}$ while maintaining $\mat{q}$ on $\Omega$.
Thus, computing $\mat{x}^{(k+1)}$ corresponds to
\begin{align*}
    \mat{x}^{(k+1)}
    &=
    \argmin_{\mat{x} \in \R^R}\, \norm{\mat{Ax}-\widetilde{\mat{q}}^{(k)}}^2_2
    = 
    \argmin_{\mat{x} \in \R^R}\, \bigl\{\norm{\mat{Px}-\mat{q}}^2_2 + \norm{\mat{A}_{\overline \Omega}\,(\mat x - \mat x^{(k)})}_2^2\bigr\}\,.
\end{align*}
\end{remark}

\subsection{Approximately Solving the Lifted Problem}
\label{sec:approx-solve-lifted}

We have shown that alternating minimization for the lifted problem~\eqref{eqn:lifted_regression}
has strong connections to preconditioned Richardson iteration
and inherits its convergence guarantees.
However, for this observation to be useful,
we need to use fast regression algorithms for the $\mat{x}^{(k+1)}$ updates that \emph{exploit the structure} of $\mat{A}$,
i.e.,
when solving $\min_{\mat x}\, \norm{\mat{Ax} - \widetilde{\mat{q}}^{(k)}}_2$.

This is where leverage score sampling comes in to play.
We exploit the structure of $\mat{A}$ to efficiently compute its leverage scores,
and then we solve the regression problem efficiently but \emph{approximately}.
By using a sketching method, our work deviates from the standard (exact) Richardson iteration.

Our next result shows how using approximate least-squares solutions
\emph{in each step of block minimization}
affects the convergence guarantee of our lifted iterative method.

\begin{algorithm2e}[t]
    \caption{\LiftedApproximateSolver}
    \label{alg:approximate-lifting}
	\BlankLine
	\KwData{$\mat{A},\widetilde{\mat{P}} \in \R^{I \times R}$, $\widetilde{\mat{q}}\in\R^{I}$, $\beta \geq 1$, $\epsilon \in (0,1)$, $\widehat{\epsilon} \in [0, 1/\beta^2)$ with $\widetilde{\mat{P}}^\top \widetilde{\mat{P}} \preceq \mat{A}^\top \mat{A} \preceq \beta \cdot \widetilde{\mat{P}}^\top \widetilde{\mat{P}}$}
	\KwResult{$\widetilde{\mat{x}} \in \R^{R}$}
	\BlankLine
	Initialize $\mat{x}^{(0)}=\boldsymbol{0}$ \\
        
        \For{$k = 0, 1, \dots, \ceil*{\frac{\log(\sfrac{2\beta}{\epsilon})}{2\,(\sfrac{1}{\beta} - \sqrt{\widehat{\epsilon}})} }$
        }{  
            Set $\widetilde{\mat{q}}^{(k)} \gets \widetilde{\mat{q}} + (\mat{A} - \widetilde{\mat{P}})\, \mat{x}^{(k)}$
            \hfill {\color{Navy} \tcp{\texttt{Implicit}}}
            Set $\mat{x}^{(k+1)}$ to a vector such that $ \norm{\mat{A} \mat{x}^{(k+1)} - \widetilde{\mat{q}}^{(k)}}_2^2 \leq(1+\widehat{\epsilon})\,\min_{\mat{x}}\, \norm{\mat{A} \mat{x} - \widetilde{\mat{q}}^{(k)}}_2^2$
        }
    \Return $\mat{x}^{(k)}$
\end{algorithm2e}

\begin{restatable}{theorem}{ApproximateRichardson}
\label{thm:approximate-richardson}
Let $\mat{A},\widetilde{\mat{P}} \in \R^{I \times R}$, $\widetilde{\mat{q}}\in\R^{I}$, and $\beta \ge 1$
such that
$\widetilde{\mat{P}}-\mat{A}$ and $\bigl[\begin{matrix} \widetilde{\mat{P}} & \widetilde{\mat{q}} \end{matrix}\bigr]$ are orthogonal, and
\[
    \widetilde{\mat{P}}^\top \widetilde{\mat{P}}
    \preceq
    \mat{A}^\top \mat{A}
    \preceq
    \beta \cdot \widetilde{\mat{P}}^\top \widetilde{\mat{P}}\,.
\]
Let $\epsilon \in (0,1), \widehat{\epsilon} \in [0,1/\beta^2)$ and
\ApproximateSolve be an algorithm that for any
$\widehat{\mat{x}}\in\R^{R}$ and $\mat{f}=\widetilde{\mat{q}}+(\mat{A}-\widetilde{\mat{P}})\, \widehat{\mat{x}}$,
computes $\overline{\mat{x}} \in \R^{R}$ in time $O(T)$ such that
\[
\norm{\mat{A}\overline{\mat{x}}-\mat{f}}_2^2\leq (1+\widehat{\epsilon})\,\min_{\mat{x}}\, \norm{\mat{A}\mat{x}-\mat{f}}_2^2\,.
\]
Then, Algorithm~\ref{alg:approximate-lifting} returns an approximate solution
$\widetilde{\mat{x}} \in \R^{R}$, using  \ApproximateSolve as a subroutine, such that
\begin{align*}
    \norm{\widetilde{\mat{P}} \widetilde{\mat{x}}-\widetilde{\mat{q}}}_2^2
    &\leq
    \parens*{1 + \frac{2 \widehat{\epsilon}}{(\sfrac{1}{\beta} - \sqrt{\widehat{\epsilon}})^2}}\, \min_{\mat{x}}\, \norm{\widetilde{\mat{P}} \mat{x}-\widetilde{\mat{q}}}_2^2 
    + \epsilon\, \norm{\widetilde{\mat{P}}\,(\widetilde{\mat{P}}^\top \widetilde{\mat{P}})^{-1} \widetilde{\mat{P}}^\top \widetilde{\mat{q}}}_2^2\,,
\end{align*}
in $O\parens*{\frac{\beta}{1-\sqrt{\widehat{\epsilon}} \beta} \cdot  T \log \sfrac{\beta}\epsilon}$ time.
\end{restatable}

\begin{proof}
We show that Algorithm~\ref{alg:approximate-lifting} gives the desired output.
Suppose \ApproximateSolve yields $\mat{x}^{(k+1)}$ for given inputs $\mat{A},\widetilde{\mat{P}},\widetilde{\mat{q}}$, and $\widetilde{\mat{q}}^{(k)}$ (i.e., $\widehat{\mat{x}} \gets \mat{x}^{(k)}$, $\mat{f}\gets \widetilde{\mat{q}}^{(k)}$, and $\overline{\mat{x}} \gets \mat{x}^{(k+1)}$), which satisfies
\[
    \norm{\mat{A}\mat{x}^{(k+1)}-\widetilde{\mat{q}}^{(k)}}_2^2\leq (1+\widehat{\epsilon})\,\min_{\mat{x}} \norm{\mat{A}\mat{x}-\widetilde{\mat{q}}^{(k)}}_2^2
    = (1+\widehat{\varepsilon})\,\norm{\pi_{\mat A ^\perp}\widetilde{\mat{q}}^{(k)}}_2^2\,.
\]
We can also decompose the LHS using $\widetilde{\mat{q}}^{(k)} = \pi_{\mat A} \widetilde{\mat{q}}^{(k)} + \pi_{\mat A ^\perp} \widetilde{\mat{q}}^{(k)}$ as follows:
\[
    \norm{\mat{A}\mat{x}^{(k+1)}-\widetilde{\mat{q}}^{(k)}}_2^2
    = 
    \norm{\mat{A}\mat{x}^{(k+1)} - \pi_{\mat{A}}\widetilde{\mat{q}}^{(k)}}_2^2 + \norm{\pi_{\mat{A}^{\perp}} \widetilde{\mat{q}}^{(k)}}_2^2\,.
\]
Combining the above, we get
\[
    \norm{\mat{A}\mat{x}^{(k+1)} - \pi_{\mat{A}}\widetilde{\mat{q}}^{(k)}}_2^2
    \leq
    \widehat{\epsilon}\, \norm{\pi_{\mat{A}^{\perp}}\widetilde{\mat{q}}^{(k)}}_2^2\,.
\]
Denoting $\mat{x}^* = (\widetilde{\mat{P}}^\top \widetilde{\mat{P}})^{-1} \widetilde{\mat{P}}^\top \widetilde{\mat{q}} = \argmin_{\mat x}\,\norm{\mat{Bx}-\widetilde{\mat{q}}}_2$ and using the triangle inequality,
\begin{align}
\nonumber
    \norm{\mat{A}\mat{x}^{(k+1)} - \mat{A}\mat{x}^*}_2
    &
    \leq \norm{\mat{A}\mat{x}^{(k+1)} - \pi_{\mat A}\widetilde{\mat{q}}^{(k)}}_2 + \norm{\pi_{\mat A}\widetilde{\mat{q}}^{(k)} - \mat{A} \mat{x}^*}_2
    \\ & \leq \label{eq:total-bound-on-a-norm}
    \sqrt{\widehat{\epsilon}}\, \norm{\pi_{\mat{A}^{\perp}}\widetilde{\mat{q}}^{(k)}}_2 + \norm{\pi_{\mat A}\widetilde{\mat{q}}^{(k)} - \mat{A} \mat{x}^*}_2\,.
\end{align}

We now bound each term in the RHS. As for the second term, since $\widetilde{\mat{q}}^{(k)} = \widetilde{\mat{q}} + (\mat{A} - \widetilde{\mat{P}})\, \mat{x}^{(k)}$ and $(\widetilde{\mat{P}}-\mat{A})^\top \begin{bmatrix}
\widetilde{\mat{P}} & \widetilde{\mat{q}} \end{bmatrix} = 0$, by \Cref{lemma:richardson-simulation},
\begin{align*}
    (\mat{A}^\top \mat{A})^{-1} \mat{A}^\top\widetilde{\mat{q}}^{(k)}
    &=
    \argmin_{\mat x}\,\norm{\mat{Ax}-\widetilde{\mat{q}}^{(k)}}_2 \\
    &=
    \mat{x}^{(k)} - (\mat{A}^\top \mat{A})^{-1} (\widetilde{\mat{P}}^\top\widetilde{\mat{P}}\mat{x}^{(k)} - \widetilde{\mat{P}}^\top\widetilde{\mat{q}})\,,
\end{align*}
which is exactly a Richardson iteration with preconditioner $\mat M\gets \mat{A}^\top \mat{A}$ in \Cref{lem:richardson_iteration} (satisfying $\widetilde{\mat{P}}^\top \widetilde{\mat{P}} \preceq \mat{A}^\top \mat{A}\preceq \beta\,\widetilde{\mat{P}}^\top \widetilde{\mat{P}}$). 
Thus, $\norm{(\mat{A}^\top \mat{A})^{-1} \mat{A}^\top\widetilde{\mat{q}}^{(k)} - \mat{x}^*}_{\mat{A}^\top \mat{A}} \leq (1-\beta^{-1})\,\norm{\mat{x}^{(k)} - \mat{x}^*}_{\mat{A}^\top \mat{A}}$, and
\begin{equation}
    \label{eq:second-term-bound-on-a-norm}
    \norm{\pi_{\mat A}\widetilde{\mat{q}}^{(k)} - \mat{A} \mat{x}^*}_2
    \leq
    \parens*{1-\frac 1 \beta}\, \norm{\mat{A} \mat{x}^{(k)} - \mat{A}\mat{x}^*}_2\,.
\end{equation}

Regarding the first term in \eqref{eq:total-bound-on-a-norm}, since $\mat{Ax}^{(k)}$ is in the column space of $\mat{A}$,
\[
\pi_{\mat A^\perp}\widetilde{\mat{q}}^{(k)} 
= 
\pi_{\mat A^\perp}\bigl(\widetilde{\mat{q}} +(\mat{A-B})\,\mat x^{(k)}\bigr)
=
\pi_{\mat A^\perp}(\widetilde{\mat{q}} - \widetilde{\mat{P}} \mat{x}^{(k)})\,.
\]
Therefore,
\begin{align*}
\norm{\pi_{\mat A^\perp}\widetilde{\mat{q}}^{(k)}}_2^2
& \leq 
\norm{\widetilde{\mat{q}} - \widetilde{\mat{P}} \mat{x}^{(k)}}_2^2
\\ & =
\norm{\widetilde{\mat{P}} \mat{x}^{*} - \widetilde{\mat{P}} \mat{x}^{(k)}}_2^2 + \min_{\mat{x}}\, \norm{\widetilde{\mat{P}} \mat{x}-\widetilde{\mat{q}}}_2^2
\\ & \leq 
\norm{\mat{A} \mat{x}^{*} - \mat{A} \mat{x}^{(k)}}_2^2 + \min_{\mat{x}}\, \norm{\widetilde{\mat{P}} \mat{x}-\widetilde{\mat{q}}}_2^2\,,
\end{align*}
where the last inequality follows from $\widetilde{\mat{P}}^\top \widetilde{\mat{P}} \preceq \mat{A}^\top \mat{A}$.
Thus,
\begin{equation}
\label{eq:first-term-bound-on-a-norm}
\norm{\pi_{\mat A^\perp}\widetilde{\mat{q}}^{(k)}}_2
\leq 
\norm{\mat{A} \mat{x}^{*} - \mat{A} \mat{x}^{(k)}}_2 + \min_{\mat{x}}\, \norm{\widetilde{\mat{P}} \mat{x}-\widetilde{\mat{q}}}_2\,.
\end{equation}

Combining \eqref{eq:total-bound-on-a-norm}, \eqref{eq:second-term-bound-on-a-norm}, and \eqref{eq:first-term-bound-on-a-norm}, we have
\[
    \norm{\mat{A}\mat{x}^{(k+1)} - \mat{A}\mat{x}^*}_2
    \leq
    \parens*{1-\frac{1}{\beta} + \sqrt{\widehat{\epsilon}}}\, \norm{\mat{A} \mat{x}^{*} - \mat{A} \mat{x}^{(k)}}_2 + \sqrt{\widehat{\epsilon}}\, \min_{\mat{x}}\, \norm{\widetilde{\mat{P}} \mat{x}-\widetilde{\mat{q}}}_2\,.
\]
Denoting $\alpha=1-\frac{1}{\beta} + \sqrt{\widehat{\epsilon}}\,$, by induction, we have
\begin{align}
\nonumber
\norm{\mat{A}\mat{x}^{(k)} - \mat{A}\mat{x}^*}_2 
& \leq 
\alpha^k\, \norm{\mat{A} \mat{x}^{*} - \mat{A} \mat{x}^{(0)}}_2 + (1+\alpha+\alpha^2 + \cdots + \alpha^{k-1}) \times \sqrt{\widehat{\epsilon}}\,\min_{\mat{x}}\, \norm{\widetilde{\mat{P}} \mat{x}-\widetilde{\mat{q}}}_2
\\ & =
\label{eq:col-space-bound-for-a}
\alpha^k\, \norm{\mat{A} \mat{x}^{*} - \mat{A} \mat{x}^{(0)}}_2 + \frac{1-\alpha^k}{1-\alpha}\times \sqrt{\widehat{\epsilon}}\, \min_{\mat{x}}\, \norm{\widetilde{\mat{P}} \mat{x}-\widetilde{\mat{q}}}_2\,.
\end{align}

We also have
\begin{align}
\nonumber
\norm{\widetilde{\mat{P}} \mat{x}^{(k)}-\widetilde{\mat{q}}}_2^2 
&= \norm{\widetilde{\mat{P}} \mat{x}^{(k)} - \pi_{\widetilde{\mat{P}}}\widetilde{\mat{q}}}_2^2 + \norm{\pi_{\widetilde{\mat{P}}^\perp}\widetilde{\mat{q}}}_2^2
\\ & = 
\label{eq:total-error}
\norm{\widetilde{\mat{P}} \mat{x}^{(k)}-\widetilde{\mat{P}} \mat{x}^*}_2^2 + \min_{\mat{x}}\, \norm{\widetilde{\mat{P}} \mat{x} - \widetilde{\mat{q}}}_2^2\,.
\end{align}
We then bound the first term by using $\widetilde{\mat{P}}^\top \widetilde{\mat{P}} \preceq \mat{A}^\top \mat{A} \preceq \beta\, \widetilde{\mat{P}}^\top \widetilde{\mat{P}}$ and \eqref{eq:col-space-bound-for-a} as follows:
\begin{align}
    \nonumber
    \norm{\widetilde{\mat{P}} \mat{x}^{(k)}-\widetilde{\mat{P}} \mat{x}^*}_2^2 
    &\leq
    \norm{\mat{A} \mat{x}^{(k)}-\mat{A} \mat{x}^*}_2^2 \\
    &\leq
    \nonumber
    2\alpha^{2k}\, \norm{\mat{A} \mat{x}^{*} - \mat{A} \mat{x}^{(0)}}_2^2 + 2\, \parens*{\frac{1-\alpha^k}{1-\alpha}}^2 \times \widehat{\epsilon}\, \min_{\mat{x}}\, \norm{\widetilde{\mat{P}} \mat{x}-\widetilde{\mat{q}}}_2^2 \\
    &\leq 
    \nonumber
    2 \beta\alpha^{2k}\, \norm{\widetilde{\mat{P}} \mat{x}^{*} - \widetilde{\mat{P}} \mat{x}^{(0)}}_2^2 + 2\, \parens*{\frac{1-\alpha^k}{1-\alpha}}^2 \times \widehat{\epsilon}\, \min_{\mat{x}}\, \norm{\widetilde{\mat{P}} \mat{x}-\widetilde{\mat{q}}}_2^2\,.
\end{align}
Putting this bound back into \eqref{eq:total-error},
\[
\norm{\widetilde{\mat{P}} \mat{x}^{(k)}-\widetilde{\mat{q}}}_2^2 \leq 2 \beta\alpha^{2k}\, \norm{\widetilde{\mat{P}} \mat{x}^{*} - \widetilde{\mat{P}} \mat{x}^{(0)}}_2^2 + \parens*{1 + 2\widehat{\epsilon}\, \parens*{\frac{1-\alpha^k}{1-\alpha}}^2}\, \min_{\mat{x}}\, \norm{\widetilde{\mat{P}} \mat{x}-\widetilde{\mat{q}}}_2^2\,.
\]

Setting
\[
k = \ceil*{\frac{\log(\sfrac{2\beta}{\epsilon})}{2\,(\nicefrac 1 \beta -\sqrt{\widehat{\epsilon}})}}\,,
\]
we have
\[
    \norm{\widetilde{\mat{P}} \mat{x}^{(k)}-\widetilde{\mat{q}}}_2^2
    \leq
    \epsilon\, \norm{\widetilde{\mat{P}} \mat{x}^{*} - \widetilde{\mat{P}} \mat{x}^{(0)}}_2^2 + \parens*{1+ \frac{2 \widehat{\epsilon}}{(\nicefrac 1  \beta - \sqrt{\widehat{\epsilon}})^2} }\, \min_{\mat{x}}\, \norm{\widetilde{\mat{P}} \mat{x}-\widetilde{\mat{q}}}_2^2\,,
\]
which completes the proof with $\mat{x}^{(0)}=\boldsymbol{0}$.
\end{proof}

\begin{remark}
To better understand the theorem, observe that
$\widetilde{\mat{P}}^\top \widetilde{\mat{P}} = \mat{P}^\top \mat{P}$ is a $\beta$-spectral approximation of $\mat{A}^\top \mat{A}$,
$\varepsilon$ controls the reducible error $\varepsilon\, \norm{\widetilde{\mat{P}} \mat{x}^*}_{2}^2$,
and $(1 + \widehat{\varepsilon})$ is the error in the approximate least-square update for each $\mat{x}^{(k)}$.
\end{remark}

\paragraph{Bounding $\beta$.}
First, observe that in the case of TD,
we have $\mat{P} = \mat{A}$, so $\beta = 1$.
More generally, if $\textnormal{rank}(\mat{A}) = s \le \min\{I, R\}$
and $\mat{A} = \mat{U}\mat{\Sigma}\mat{V}^\top$ is a compressed SVD, then $\mat{A}$ is said to satisfy  the \emph{standard incoherence condition} with parameter $\mu$ \citep{chen2015incoherence} if
\[
    \max_{i\in[I]} \norm*{\mat{e}_{i}^\top \mat{U}}_2 \leq \sqrt{\frac{\mu s}{I}}\,,
    \quad
    \max_{r\in[R]} \norm*{\mat{V} ^\top\mat{e}_r}_2 \leq \sqrt{\frac{\mu s}{R}}\,.
\]
The $\norm{\mat{e}_{i}^\top \mat{U}}_2^2$ and $\norm{\mat{V} ^\top\mat{e}_r}_2^2$ values are the leverage scores of the rows and columns of $\mat{A}$.
Applying \citet[Lemma 4]{cohen2015uniform},
if each row of $\mat{A}$ is observed with probability $p$ such that $p \geq \frac{c\mu s \log s}{I}$ for some absolute constant $c$, then 
\[
    \frac{1}{2} \,\mat{A}^\top \mat{A}
    \preceq
    \frac{1}{p} \, \mat{A}_{\Omega}^\top \mat{A}_{\Omega}
    \preceq
    \frac{3}{2}\, \mat{A}^\top \mat{A}\,,
\]
which gives $\beta=2/p$.
Let $\zeta = \max_{i\in[I]} \norm{\mat{a}_i}_2$, where $\mat{a}_i$ is row $i$ of $\mat{A}$.
Then, the $\alpha \zeta^2$-ridge leverage scores of $\mat{A}$ (i.e., $\mat{a}_i^\top (\mat{A}^\top \mat{A} + \alpha \zeta^2 \cdot\mat{I}_{R})^{-1} \mat{a}_i$), for $\alpha\geq 1$, are at most $1/\alpha$. If $p$ is the observation rate, taking $\alpha = \frac{c\log s}{p}$ gives the required incoherence condition. This can be done by introducing an $\ell_2$-regularization term to the TC optimization problem (i.e., solving a ridge regression problem in each ALS step).
Note that usually $\alpha$ can be chosen to be much smaller in practice.

\section{Sampling Methods for Tensor Completion}
\label{sec:sampling-for-completion}
We are now ready to efficiently solve the \emph{unstructured} least-squares problem \eqref{eqn:input_regression} induced by ALS for tensor completion, i.e., for $\mat{P}\in\R^{|\Omega|\times R}$ and observations $\mat{q} \in \R^{|\Omega|}$, find
\[
    \mat{x}^* = \argmin_{\mat{x} \in \R^R} \,\norm{\mat{P} \mat{x} - \mat{q}}_{2}^2\,.
\]
As in Algorithm~\ref{alg:approximate-lifting},
we lift this problem to higher dimension to get a structured design matrix $\mat{A}$,
and use a known fast algorithm for approximately solving the structured least-squares problem in each step of approximate-mini-ALS.
For a given $\widehat{\epsilon} \in (0,1/\beta^2)$, the approximate solver computes a solution $\overline{\mat{x}}\in \R^R$ in time $O(T_{\widehat{\epsilon}})$ such that 
\[
\norm{\mat A \overline{\mat x} - \mat b}_2^2 \leq (1+\widehat \epsilon)\,\min_{\mat x} \norm{\mat A \mat x -\mat b}_2^2\,.
\]
Therefore, for a desired $\varepsilon_1 \in (0,1)$, we set $\widehat \epsilon = \Theta(\varepsilon_1 / \beta^2)$ and use a sufficiently small $\epsilon \gets \varepsilon_2$ in \cref{thm:approximate-richardson}.
Putting everything together, Algorithm~\ref{alg:approximate-lifting} finds an approximate solution $\widetilde{\mat x}\in \R^R$ in time $O(\beta T_{\varepsilon_1\beta^{-2}}\log\frac{\beta}{\epsilon_2})$ that satisfies
\begin{align}
    \norm{\mat P \widetilde{\mat x} - \mat q}_2^2
    &\leq
    (1+\varepsilon_1)\,\norm{\pi_{\mat P^\perp }\mat q}_2^2 + \varepsilon_2\, \norm{\mat \pi_{\mat P}\mat q}_2^2\,,\label{eq:tc-approx-sol}
\end{align}
where $\mat{\pi}_{\mat{P}}$ and $\mat{\pi}_{\mat{P}^\perp}$ are the orthogonal projection matrices into the column space and null space of $\mat{P}$, respectively~(see \Cref{app:least-square-regression}).
With this in hand, we now present the running times of our lifted iterative method for TC problems
by combining \cref{thm:approximate-richardson} with state-of-the-art tensor decomposition results based on leverage score sampling.

\subsection{CP Completion}
\label{subsec:CP-completion}
Each ALS update step for CP completion solves a regression problem where the design matrix is the Khatri--Rao product:
for $\mat{A}^{(k)} \in \R^{I_k \times R}$,
$\mat A^{\neq k} := \bigodot_{n=1,n\neq k}^N \mat A^{(n)} \in \R^{I_{\neq k}\times R}$,
and $\mat Q = (\mat X^\top_{(k)})_\Omega \in \R^{|\Omega| \times I_k}$,
\[
    \mat A^{(k)}
    \gets
    \argmin_{\mat{A} \in \R^{I_k \times R}} \, \bigl\lVert  (\mat A^{\neq k})_\Omega \,\mat{A}^\top - \mat Q \bigr\rVert_{\frobenius}\,.
\]
The design matrix $(\mat A^{\neq k})_\Omega$ does not necessarily have any structure,
so a direct method relies on solving the normal equation, which takes $O(R^\omega +R |\Omega|(R+I_k))$ time.
Thus, the running time of \emph{one round} of ALS, i.e., updating all $N$ factors,
is
\[
    O\parens*{N\parens*{R^\omega +R^2 |\Omega|}+ R |\Omega|\sum_{n=1}^N I_n}.
\]

Previous work on CP tensor decomposition \citep{cheng2016spals, larsen2022practical, bharadwaj2023fast}
developed fast methods for efficiently computing the leverage scores of a Khatri--Rao product matrix.
In particular, \citet{bharadwaj2023fast} designed a data structure for computing and maintaining the
leverage scores of $\mat A^{\neq k}$ during ALS updates. 
This approach requires sampling $\tilde{O}(R/\varepsilon)$ rows of $\mat A^{\neq k}$.
Due to the Khatri--Rao product structure, each row of $\mat A^{\neq k}$ can be mapped to a sequence of one choice from the rows of each $\mat{A}^{(n)}$ for $n\in [N]\backslash \{k\}$.
Hence, sampling a row from $\mat{A}^{\neq k}$ is equivalent to the following:
for each $n \in [N]\backslash\{k\}$, sample a row from $\mat A^{(n)}$ according to some conditional distribution given sampled rows from $\mat{A}^{(1)}, \dots, \mat{A}^{(n-1)}$, and then compute the Hadamard product of $N-1$ sampled rows.
Maintaining the full $I_n$-dimensional vector for a conditional probability for each $n$ is costly,
so \citet{bharadwaj2023fast} developed a binary tree-based data structure to speed up leverage-score sampling for $\mat A^{\neq k}$.

Applying \citet[Corollary 3.3]{bharadwaj2023fast}, one round of ALS runs in time
$
    \tilde{O} \parens{ \varepsilon^{-1} \sum_{n=1}^N \parens*{ I_n R^2 + NR^3} }.
$
Using their CP TD algorithm as the approximate solver in Algorithm~\ref{alg:approximate-lifting}, and combining its guarantee with \cref{thm:approximate-richardson},
we can extend their approach to CP completion.

\begin{corollary}
There is an ALS CP completion algorithm such that
(i) after a factor matrix update, each row of $\mat{A}^{(n)}$ satisfies \eqref{eq:tc-approx-sol},
and (ii) the total running time of one round is
\[
    \tilde{O} \parens*{\frac{\beta^2}{\varepsilon_1} \sum_{n=1}^N \parens*{I_n R^2 + NR^3} \log \frac{1}{\epsilon_2}}\,.
\]
\end{corollary}

\noindent
There is \emph{no dependence} on $|\Omega|$ in the running time due to leverage score sampling, i.e., it runs in sublinear time.

\subsection{Tucker Completion}

\citet{fahrbach2022subquadratic} designed block-sketching techniques and fast Kronecker product-matrix multiplication algorithms
to exploit the ALS structure for Tucker decomposition.

\subsubsection{Core Tensor Update}
Recall that for a Tucker decomposition we use the notation $I=\prod_{n\in[N]} I_n$ and $R=\prod_{n\in[N]} R_n$.
The ALS core tensor update in the Tucker completion problem is
\[
    \tensor{G}
    \leftarrow
    \argmin_{\tensor{G}' \in\R^{R_1 \times \cdots \times R_N}} \norm*{\parens*{\bigotimes_{n=1}^N\mat{A}^{(n)}}_{\Omega} \hspace{-0.2cm}\vvec(\tensor{G}') - \vvec(\tensor{X})_{\Omega}}_2.
\]
The design matrix above restricted to $\Omega$ is exactly $\mat P$ in our general setup.

We compare the running times of the direct method and our lifting approach.
In the former, we can compute an exact solution to a least-squares problem $\mat x^* = (\mat P^\top \mat P)^{-1} \mat P^\top \mat q$ in time $O(|\Omega| R^2 + R^\omega)$.

To achieve a fast lifted method, we solve the second step of Algorithm~\ref{alg:approximate-lifting}
using the leverage score sampling-based
core tensor update algorithm in \citep[Theorem 1.2]{fahrbach2022subquadratic}
with running time
\[
    \tilde{O} \parens*{
        \sum_{n=1}^N \parens*{ I_n R_n  + \frac{R_n^\omega N^2}{\varepsilon^2}} + \frac{R^{2-\theta^*}}{\varepsilon}
    }\,,
\]
where $\theta^*>0$ is an optimizable constant depending on $\{R_n\}_{n\in[N]}$.
Using this as the \ApproximateSolve subroutine in \Cref{thm:approximate-richardson}, we achieve the following.

\begin{corollary}
\label{cor:fast_tucker_completion_core_tensor_update}
There is an algorithm that computes an ALS Tucker completion core tensor update satisfying \eqref{eq:tc-approx-sol}
in time
\begin{equation}
\label{eqn:fast_tucker_completion_core_tensor_update}
    \tilde{O}\parens*{\parens*{
         \sum_{n=1}^N \parens*{I_nR_n  + \beta^4 R_n^\omega N^2 \varepsilon_1^{-2}}
         +
         \frac{\beta^2 R^{2-\theta^*}}{\varepsilon_1}} \beta\log\frac{1}{\varepsilon_2}
    }\,.
\end{equation}
\end{corollary}

\subsubsection{Factor Matrix Update}
The ALS factor matrix update for $\mat{A}^{(k)}$ in the Tucker completion problem is
\[
    \mat{A}^{(k)}
    \leftarrow
    \hspace{-0.1cm}
    \argmin_{\mat{A}\in \R^{I_k \times R_k}} \norm*{ \parens*{
        \parens*{\bigotimes_{n=1,n\neq k}^N \hspace{-0.10cm} \mat{A}^{(n)} } \mat{G}_{(k)}^\top }_\Omega  \hspace{-0.10cm} \mat{A}^\top \hspace{-0.10cm} - \mat Q}_\frobenius,
\]
where $\mat Q= (\mat X^\top_{(k)})_\Omega \in \R^{|\Omega| \times I_k}$ is a sparse matrix of observations.
The running time of a direct method that solves the normal equation is
$O(R_k^\omega + R_k |\Omega| (R_{\neq k} + R_k + I_k))$, where $R_{\neq k} = R/R_k$.

The running time of the sampling-based factor-matrix update for $\mat{A}^{(k)}$ in \citet[Theorem 1.2]{fahrbach2022subquadratic}
for the full decomposition problem is
\[
    \tilde{O}\parens*{
        \sum_{n=1}^N \parens*{I_nR_n + \frac{R_n^\omega N^2}{\varepsilon^2} + I_k R R_n} + \frac{I_k R_{\neq k}^{2-\theta^*}}{\varepsilon}
    }\,.
\]

\noindent
Combining this result with \cref{thm:approximate-richardson},
Algorithm~\ref{alg:approximate-lifting} has the following running time for a factor matrix update.

\begin{corollary}
There is an algorithm that computes an ALS Tucker completion factor matrix update for $\mat{A}^{(k)}$,
with each row of $\mat{A}^{(k)}$ satisfying \eqref{eq:tc-approx-sol},
in time
\[
\textstyle
    \tilde{O} \parens*{
        \parens*{
            \sum_{n=1}^N \parens*{I_nR_n  + \beta^4 R_n^\omega N^2 \epsilon_1^{-2}}
            +
            \frac{\beta^2 I_k R_{\neq k}^{2-\theta^*}}{\epsilon_1} + I_k R\sum_{n=1}^N R_n } \beta \log \frac{1}{\epsilon_2}
    }\,.
\]
\end{corollary}

\subsection{TT Completion}

Each ALS step for TT decomposition solves the following least-squares problem
with a Kronecker product-type design matrix:
for $\mat{A}^{\neq k} := \mat{A}_{< k}\otimes \mat{A}^\top_{> k} \in \R^{I_{\neq k} \times (R_{k-1}R_k)}$ and $\mat Q = (\mat{X}_{(k)}^\top)_\Omega \in \R^{|\Omega| \times I_k}$,
\[
    \tensor{A}^{(k)}
    \gets
    \argmin_{\tensor{B} \in \R^{R_{k-1}\times I_k \times R_k}}
    \norm*{\parens*{\mat{A}^{\neq k}}_\Omega\, (\mat{B}_{(2)})^\top - \mat Q }_{\frobenius}\,.
\]
Solving this directly with the normal equation takes
$O(\bar R_k^\omega +\bar R_k |\Omega| (\bar R_k + I_k))$ time for $\bar R_k := R_{k-1} R_k$.
Thus, the time for one round of ALS is
\[
    O\parens*{
        \sum_{n=1}^N \parens*{\bar R_n^\omega + \bar R_n |\Omega| \parens*{\bar R_n +I_n}}
    }.
\]

\citet{bharadwaj2024efficient} proposed a sampling-based ALS algorithm that crucially relies a \emph{canonical form} of the TT decomposition with respect to the index $k$. Any TT decomposition can be converted to this form through a QR decomposition, and this form ensures that $(\mat A^{\neq k})^\top \mat A^{\neq k} = I_{R_{k-1} R_k}$.
It follows that the leverage scores of $\mat A^{\neq k}$ are simply the diagonal entries of $\mat A^{\neq k} (\mat A^{\neq k})^\top = (\mat A_{< k}\mat A_{<k}^\top) \otimes (\mat A_{> k}^\top\mat A_{>k})$.

It follows from properties of the Kronecker product~\citep{diao2019optimal} that
$
    \ell_{i^{\neq k}}(\mat A^{\neq k})
    =
    \ell_{i_{< k}}(\mat A_{< k}) \cdot \ell_{i_{> k}}(\mat A^\top_{> k})\,,
$
where $i^{\neq k} = \underline{i_1\cdots i_{k-1} i_{k+1}\cdots i_N}$, $i_{< k} = \underline{i_1\cdots i_{k-1}}$, and $i_{> k} = \underline{i_{k+1}\cdots i_N}$ (see \Cref{app:tt-decomposition-details} for notation).
Therefore, efficient leverage score sampling for $\mat A^{\neq k}$ reduces to that for $\mat A_{< k}$ and $\mat A_{> k}$.
To this end, \citet{bharadwaj2024efficient} adopt an approach similar to \citet{bharadwaj2023fast} for leverage score-based CP decomposition.
Each row of $\mat A_{< k}$ corresponds to a series of one slice for each third-order tensor $\mat A^{(n)}$ for $n<k$, which results in a series of conditional sampling steps using a data structure adapted from the one used for CP decomposition.
In contrast, \citet[Corollary 4.4]{bharadwaj2024efficient} show that one round of approximate TT-core updates,
if $R_n = R$ for all $n \in [N-1]$, can run in time
$\tilde{O} \parens{ R^4 \varepsilon^{-1} \sum_{n = 1}^N \parens*{N + I_n}}$,
which leads to the following result.

\begin{corollary}
There is an ALS TT completion algorithm such that
(i) after a TT-core update, each row fiber of $\tensor{A}^{(n)}$
satisfies \eqref{eq:tc-approx-sol},
and (ii) the total running time of one round is
\[
    \tilde{O} \parens*{
        \frac{\beta^2 R^4}{\epsilon_1} \sum_{n=1}^N \parens*{N + I_n} \log\frac{1}{\epsilon_2}
    }\,.
\]
\end{corollary}

\section{Experiments: Planning outperforms Heuristics}
\label{sec:experiment}

We begin our empirical demonstrations by showcasing the effectiveness of our planning framework on both synthetic and real datasets. We focus on the simplest planning algorithm, 1-step lookaheads (Algorithm~\ref{alg:complete}), and show that even basic planning can hold great promise. 
We illustrate our framework using two uncertainty quantification modules---GPs and 
\ensembles/ \ensembleplus. 

Throughout this section, we focus on evaluating the mean squared error of 
a regression model $\model$,  and develop adaptive policies that minimize uncertainty on $g(f)$ defined in~\eqref{eqn:l2-g-f}.
When GPs provide a valid model of uncertainty, 
our experiments show that our planning framework significantly outperforms other baselines. 
We further demonstrate that our conceptual framework extends to deep learning-based uncertainty quantification methods such as  \ensembleplus while highlighting computational challenges that need to be resolved in order to scale our ideas. 
For simplicity, we assume a naive predictor, i.e., $\psi(\cdot) \equiv 0$. However, we emphasize that this problem is just as complex as if we were using a sophisticated model $\psi(.)$. The performance gap between the algorithms 
primarily depends
on the level  of uncertainty in our prior beliefs.

To evaluate the performance of our algorithm, we benchmark it against several baselines. 
%Active learning baselines use an acquisition function $\ac$ to select points that have the highest   function value: $X\opt_t \in \argmax_{X \in \xpoolj{t}} \ac({X})$ at every step $t$. These methods may also need an UQ module, which we simply use the same UQ module as in our algorithm, and it  outputs $V(X)$ that measures the the uncertainty of each point $X \in \xpoolj{t}$.
Our first set of baselines are from active learning~\citep{AggarwalKoGuHaPh14}:
\\ % \noindent\textbf{Active Learning Heuristics:} 
\textbf{(1)} 
\textsf{Uncertainty Sampling (Static):}  In this approach, we query the samples for which the model is least certain about. Specifically, we estimate the variance of the latent output $f(X)$ for each $X \in \xpool$ using the UQ module and select the top-$K$ points with the highest uncertainty. \\
\textbf{(2)} \textsf{Uncertainty Sampling (Sequential):} This is a greedy heuristic that sequentially selects the points with the highest uncertainty within a batch, while updating the posterior beliefs using pseudo labels from the current posterior state. Unlike \textsf{Uncertainty Sampling (Static)}, this method takes into account the information gained from each point within batch, and hence tries to diversify the selected points within a batch. 

 
We also compare our approach to the  \textbf{(3)} \textsf{Random Sampling}, which selects each batch uniformly at random from the pool. Additionally, we compare solving the planning problem using  \textsf{REINFORCE}-based policy gradients with   $\mathsf{Smoothed\text{-}Autodiff}$ policy gradients.\footnote{Our code repository is available at
  \url{https://github.com/namkoong-lab/adaptive-labeling}.}
%Detailed experimental setups are provided in Section \ref{sec:details-experiments}.

%We repeat all experiments with 10 random seeds.




\begin{figure}[t]
\centering
\begin{minipage}[b]{0.49\textwidth}
\centering
\includegraphics[width=\textwidth, height=5cm]{figures/original_scale/Var_of_l_2_loss.pdf}
\caption{(Synthetic data) Variance of mean squared loss evaluated through the posterior belief $\mu_t$ at each horizon $t$. This is the objective that policy gradient methods like \textsf{REINFORCE} and $\ouralgo$ optimizes. 1-step lookaheads are surprisingly effective even in long horizons.}
\label{fig:var-l2-sim}
\end{minipage}
\hfill
\begin{minipage}[b]{0.49\textwidth}
\centering \includegraphics[width=\textwidth, height=5cm]{figures/original_scale/Error_of_estimated_model_l_2_loss.pdf}
\caption{(Synthetic data) Error between MSE calculated based on collected data $\mc{D}^{0:T}$ vs. population oracle MSE over $\mc{D}_{\rm eval} \sim P_X$. Reducing uncertainty over posteriors directly leads to better OOD evaluations. 1-step lookaheads significantly outperform active learning heuristics in small horizons.}
\label{fig:mean-l2-sim}
\end{minipage}
%\caption{Simulated data for GPs}
%\label{fig:both_plots}
\end{figure}

\subsection{Planning with Gaussian processes}
\label{sec:experiment-plan-GP}
We now briefly describe the data generation process for the GP experiments,  deferring a more detailed discussion of the dataset generation to Section~\ref{sec:details-experiments}. 
We use both the synthetic data and the real data to test our methodology.
For the \emph{simulated data},  we construct a setting where the general population is distributed across \emph{51 non-overlapping clusters} while the initial labeled data $\dtrain$ just comes from one cluster. In contrast, both $\dpool \defeq (\xpool,\ypool),\deval \defeq (\xeval,\yeval)$ are generated   from all the clusters. 
We begin with a low-dimensional scenario, generating a one-dimensional regression setting using a GP. %Gaussian Process (GP).
Although the data-generating process is not known to the algorithms,  we assume that the GP hyperparameters are known to all the algorithms
to ensure fair comparisons. This can be viewed as a setting where our prior is well-specified, allowing us to isolate the effects
of different policy optimization approaches
 without any concerns about the misspecified priors. We select $10$ batches, each of size $K=5$ across $T = 10$ time horizons.

To examine the robustness of our method against the distributional assumptions made  in the simulated case, we then move to a real dataset where the correct prior is not known. We simulate selection bias from the eICU dataset~\citep{PollardJoRaCeMaBa18}, which contains real-world patient data with in-hospital mortality outcomes. 
We conduct a $k$-means clustering to generate 51 clusters and then select data from those clusters. We view this to be a credible replication of practice, as severe distribution shifts are common due to selection bias in clinical labels.  To convert the binary mortality labels into a regression setting, we train a  random forest classifier and fit a GP on predicted scores, which serves as the UQ module for all the algorithms. As before, the task is to select 10 batches, each consisting of 5 samples, across 10 time horizons.

 In Figures~\ref{fig:var-l2-sim} and~\ref{fig:mean-l2-sim}, we present results for the simulated data. 
Figure~\ref{fig:var-l2-sim} shows the variance of $\ell_2$ loss, and Figure~\ref{fig:mean-l2-sim} presents the error in the estimated $\ell_2$ loss using $\mu_t$ (relative to true $\ell_2$ loss, that is unknown to the algorithm). 
As we can see from these plots, our method one-step lookahead  gives substantial improvements  over active learning baselines and random sampling. In addition,
compared to the one-step lookahead planning approach using \textsf{REINFORCE}-based policy gradients, 
we observe that $\mathsf{Smoothed\text{-}Autodiff}$-based policy gradients provide significantly more robust performance over all horizons.

In Figures~\ref{fig:var-l2-real}~and~\ref{fig:mean-l2-real}, we observe similar findings on the eICU data. We see that planning policies (\textsf{REINFORCE} and $\mathsf{Smoothed\text{-}Autodiff}$) consistently outperform other heuristics by a large margin.  Active learning baselines perform poorly in these small-horizon batched problems and can sometimes be even worse than the random search baselines.  Overall, our results show the importance of careful planning in adaptive labeling for reliable model evaluation. 

We offer some intuition as to why one-step lookahead planning may outperform other heuristic algorithms. 
 First,  \textsf{Uncertainty sampling (Static)} while myopically selects the
 top-$K$ inputs with the highest uncertainty, it fails to consider 
the overlap in information content among the ``best” instances; see \citep{AggarwalKoGuHaPh14} for more details. 
In other words,  it might acquire points from the same region with high uncertainty while failing to induce diversity among the batch.
Although \textsf{Uncertainty Sampling (Sequential)} somewhat addresses the issue of information overlap, a significant drawback of 
this algorithm
is the disconnect between the objective we aim to optimize and the algorithm. For example, it might sample from a region with high uncertainty but very low density. 

\begin{figure}[t]
\centering
\begin{minipage}[b]{0.48\textwidth}
\centering
\includegraphics[width=\textwidth, height=5cm]{figures/original_scale/Var_of_l_2_loss_real.pdf}
\caption{(Real-world eICU data) Variance of mean squared loss evaluated through the posterior belief $\mu_t$ at each horizon $t$. Even 1-step lookaheads are extremely effective planners, and auto-differentiation-based pathwise policy gradients provide a reliable optimization algorithm based on low-variance gradient estimates.}
\label{fig:var-l2-real}
\end{minipage}
\hfill
\begin{minipage}[b]{0.48\textwidth}
\centering \includegraphics[width=\textwidth, height=5cm]{figures/original_scale/Error_of_estimated_model_l_2_loss_real.pdf}
\caption{(Real-world eICU data) Error between MSE calculated based on collected data $\mc{D}^{0:T}$ vs. population oracle MSE over $\mc{D}_{\rm eval} \sim P_X$. Reducing uncertainty over posteriors directly leads to better OOD evaluations. Our method significantly outperforms active learning-based heuristics, and random sampling.}
\label{fig:mean-l2-real}
\end{minipage}
%\caption{Real data for GPs}
\end{figure}
 
%\vspace{-1.5cm}
% \begin{wrapfigure}{r}{.32\columnwidth}
%   \vspace{-.5cm} 
%   \centering
% \includegraphics[scale=.29]{figures/Var of l2l_2 loss.pdf}
%   \vspace{-0.2cm}
%   \caption{Results of GP}
% \label{fig:var-l2-gp}
%   \vspace{-0.1cm}
% \end{wrapfigure}


% Attempts have been made  in the past to address these  drawbacks heuristically  (see \citep{AggarwalKoGuHaPh14}). We give a unified computational framework while approaching the problem in a more principled manner and solving it more optimally.




\subsection{Planning with  neural network-based uncertainty quantification methods ($\ensembleplus$)}


We now provide a proof-of-concept that shows the generalizability of our conceptual framework  to the deep learning-based UQ modules, specifically focusing on $\ensembleplus$ due to their previously observed superior performance~\citep{OsbandWenAsDwIbLuRo23}. Recall that implementing our framework with deep learning-based UQ modules  requires us to retrain the model across multiple possible random actions $\bm{a}(\theta)$ sampled from the current policy $\pi_\theta$.
This requires significant computational resources, in sharp contrast to the GPs where the posteriors are in closed form and can be readily updated and differentiated. 

Due to the computational constraints, we test $\ensembleplus$ on a toy setting to demonstrate the generalizability of our framework. We consider a setting where the general population consists of four clusters, while the initial labeled data only comes from one cluster. Again we generate data using GPs.  The task is to select a batch of 2 points in one horizon. We detail the $\ensembleplus$ architecture in Section \ref{sec:details-experiments}, and we assume prior uncertainty to be large (depends on the scaling of the prior generating functions). 
The results are summarized in the Table~\ref{tab:UQ_ensemble}.

% \begin{table}[H]
% \vspace{-10pt}
% \caption{Performance under \ensembleplus as UQ module}
%     \centering
%     \begin{tabular}{|m{3cm}|m{2.5cm}|m{2cm}|} 
%     \hline
%       Algorithm   & Variance of $\loss_2$ loss estimate & Error of $\loss_2$ loss estimate  \\ \hline Random Sampling 
%          & $1710.9 \pm 1352.1$ & $8.67\pm6.62$ 
%       \\ \hline \ouralgo & $1.30 \pm 0.68$ & $0.91\pm0.25$ \\ \hline
%     \end{tabular}
%     \label{tab:UQ_ensemble}
%     %\vspace{-10pt}
% \end{table}




\begin{table}[h]
\vspace{-10pt}
\caption{Performance under \ensembleplus as the UQ module}
\centering
\begin{tabular}{|l|l|l|}
\hline
Algorithm   & Variance of $\loss_2$ loss estimate & Error of $\loss_2$ loss estimate  \\
\hline
\textsf{Random sampling} & 7129.8 $\pm$ 1027.0 & 136.2 $\pm$ 8.28 \\ \hline
\textsf{Uncertainty sampling (Static)} & 10852 $\pm$ 0.0 & 162.156 $\pm$ 0.0 \\ \hline
\textsf{Uncertainty sampling (Sequential)} & 8585.5 $\pm$ 898.9 & 144 $\pm$ 6.93 \\ \hline
\textsf{REINFORCE} & 1697.1 $\pm$ 0.0 & 45.27 $\pm$ 0.0 \\ \hline
\ouralgo & 1697.1 $\pm$ 0.0 & 45.27 $\pm$ 0.0 \\ \hline
\end{tabular}
%\caption{Comparison of different algorithms based on variance   and   error in $\ell_2$ loss estimation with Ensemble $+$ as the UQ module. Our results demonstrate that {\ouralgo} and REINFORCE outperformthe other active learning based heuristics, confirming the benefits of our MDP formulation for the adaptive labeling problem, as also demonstrated in Section 4.\\
%\footnotesize{Experimental details: We use Gaussian Processes as our data generating process, GP parameters are the same as in Section D.3.  The task is to select a batch of 2 points along one horizon.The marginal distribution $p_X$ has 4 \textit{non-overlapping} clusters. Initial data comes from one cluster, while pool and evaluation points comes from all the clusters. We have $20$ initial labeled data points, $10$ pool points, and $252$ evaluation points.  Training procedures are similar to the one in Section D.3.} }
\label{tab:UQ_ensemble}
\end{table}



% We faced  issues in scaling up these experiments which will be our focus in the future. 





% \begin{itemize}
%     \item Posteriors should be consistent. Two dimensions: even with less training,  
%     \item the inference should be  fast enough
% \end{itemize}


% Potential research directions for uncertainty quantification

% In this section we consider a simple setting We consider a simpler setting and 


% For synthetic dataset generation, we use ...... For real datasets, we use ...... We compare our methodolgy to several baselines ()    This Section is structured as follows:
% \begin{itemize}
%     \item \textbf{GPs, square loss objective} (Section \ref{}): 
%     %the broad aim of the experiments  in this section is to isolate the performance of our methodology without any concerns for the inefficiencies induced due to a mis-specified prior or imperfect posterior inference. To accomplish this we generate synthetic datasets using GPs (detailed later). We use the well specified prior (GPs - with same hyperparameter setting) as our UQ module.   
%      As GPs provide differentaible posterior inference - any errors induced due to imperfect posterior updates are also isolated. We note that under this setting
%      \item In Section\ref{} we demonstrate why our methodology performs better than other baselines - by devising various synthetic experiments ()
%     \item  \textbf{UQ Benchmarking }(Section \ref{}): Before diving into the experiments using $\ensembleplus$ and ENNs,  we showcase our benchmarking experiments in Section \ref{}. We use real datasets We observe that ENNs perform better
%      \item \textbf{Ensemble $+$}, objective: recall, accuracy
%     \item \textbf{ENN}, objective: recall, accuracy
% \end{itemize}




% In Section {}, we test 
% \subsection{Experimental details}

% \begin{itemize}
%     \item UQ methodologies - GPs, ENNs
%     \item Objectives - Recall,  ATE
%     \item Datasets - ATE-synthetic datasets, Recall-synthetic, real datasets
%     \item Baselines - 
%     \begin{itemize}
%         \item Random sampling
%         \item Active learning - Uncertainty based sampling - In regression setting almost all of the 
%         \item Myopic greedy - Greedy Batch based sampling
%         \item Policy Gradient
%     \end{itemize}
    
% \end{itemize}

% \subsection{Experiments}
%     \begin{itemize}
%     \item GPs with square loss
%     \item Benchmarking ENN
%         \item ENNs with ATE
%         \item ENNs with Recall
%     \end{itemize}

% \subsection{Benefits over other algorithms - intuition and experiments}

%Active learning - Myopic greedy / Don't rely on the objective rather some entropy version.


%%% Local Variables:
%%% mode: latex
%%% TeX-master: "main"
%%% End:

\section*{Conclusion}
This paper aims to enhance our understanding of the computational complexity of computing various Shapley value variants. We found that for various ML models --- including decision trees, regression tree ensembles, weighted automata, and linear regression --- both local and global interventional and baseline SHAP can be computed in polynomial time under HMM modeled distributions. This extends popular algorithms, such as TreeSHAP, beyond their empirical distributional scope. We also establish strict complexity gaps between the various SHAP variants (baseline, interventional, and conditional) and prove the intractability of computing SHAP for tree ensembles and neural networks in simplified scenarios. Overall, we present SHAP as a versatile framework whose complexity depends on four key factors: \begin{inparaenum}[(i)] \item model type, \item SHAP variant, \item distribution modeling approach, \item and local vs. global explanations\end{inparaenum}. We believe this perspective provides deeper insight into the computational complexity of SHAP, paving the way for future work.




%We believe that our framework provides a more intricate understanding of SHAP computation complexity across different models, distributions, and variants, paving the way for further research.

Our work opens promising directions for future research. First, expanding our computational analysis to other SHAP-related metrics, such as asymmetric SHAP~\citep{frye20} and SAGE~\citep{covert2020understanding}, would be valuable. Additionally, we aim to explore more expressive distribution classes and relaxed assumptions beyond those in Section \ref{sec:tractable} while maintaining tractable SHAP computation. Finally, when exact computation is intractable (Section \ref{sec:intractable}), investigating the approximability of SHAP metrics through approximation and parameterized complexity theory~\citep{downey2012parameterized} is an important direction.

%Our work opens several promising avenues for future research on the computational properties of explainable AI methods, with a particular focus on SHAP. First, it would be interesting to broaden the computational analysis conducted in this work to include other popular SHAP-related metrics in the literature, such as asymmetric SHAP \cite{frye20} and SAGE \cite{covert2020understanding}. Also, in the future, we aim to explore more expressive distribution classes and relaxed distributional assumptions—extending beyond those examined in Section \ref{sec:tractable} —that still yield tractable SHAP computation. Finally, when exact computation proves intractable (Section \ref{sec:intractable}), it is worthwhile to theoretically investigate the question of the approximability of computing the SHAP metrics across various configurations, through the lens of approximation and parametrized complexity theory \cite{arora2009computational}.

%This paper aims to deepen our understanding of the computational complexity involved in obtaining different Shapley value variants. We found that for a variety of ML models, including decision trees, tree ensembles for regression, weighted automata, and linear regression models — computing both local and global interventional and baseline SHAP can be done in polynomial time when distributions are modeled by HMMs. This extends the distributional scope of popular algorithms like TreeSHAP, which is limited to empirical distributions. Additionally, we demonstrate a strict complexity gap between SHAP variants, showing that interventional and baseline SHAP can be strictly easier to compute than conditional SHAP. Despite these positive results, we uncovered intractability for various SHAP variants in neural networks and tree ensembles. Finally, we provided generalized complexity relations across SHAP variants. We believe that our framework offers a deeper understanding of the complexity involved in computing SHAP across various variants, models, distributions, as well as in both local and global computations, laying the groundwork for future research.

\bibliographystyle{abbrvnat}
\bibliography{references}

\newpage
\appendix

\section{Missing Details for \Cref{sec:preliminaries}}

\subsection{Least-Squares Linear Regression}
\label{app:least-square-regression}

For a design matrix $\mat{A}\in \R^{n\times d}$ and response $\mat{b}\in \R^n$, consider the least-squares problem
\[
    \mat{x}^* = \argmin_{\mat x\in \R^d}\,\norm{\mat{Ax}-\mat{b}}_2\,.
\]
The solution $\mat{x}^*$ can be obtained by solving the \emph{normal equation} $\mat{A}^\top\mat{A}\mat{x}^*=\mat{A}^\top \mat{b}$.
Therefore, $\mat{x}^* = (\mat{A}^\top\mat{A})^{-1}\mat{A}^\top \mat{b}$.
If $\mat{A}^\top\mat{A}$ is singular, then we can use the \emph{pseudoinverse} $\mat{A}^+$.

The \emph{orthogonal projection matrix} $\pi_{\mat{A}}:\R^n \to \R^n$ onto the image space of $\mat{A}$ is defined by
$\pi_{\mat{A}} = \mat{A}\,(\mat{A}^\top\mat{A})^{-1}\mat{A}^\top$, and satisfies $\pi_{\mat{A}}^2 = \pi_{\mat{A}}$ and $\norm{\pi_{\mat A}\mat v}_2 \leq \norm{\mat v}_2$ for any $\mat{v} \in \R^n$. 
Recall that any $\mat{v}\in \R^n$ can be uniquely decomposed as $\mat v = \pi_{\mat{A}}\mat v + \pi_{\mat{A}^\perp}\mat v$, where $\pi_{\mat{A}^\perp}=\mat{I}_{n} - \pi_{\mat{A}}$ is the orthogonal projection to the orthogonal subspace of $\textnormal{colsp}(\mat{A})$.

Given $\mat{b} = \pi_{\mat{A}}\mat b + \pi_{\mat A ^\perp}\mat b$, the first term is the \emph{reducible error} by regressing $\mat b$ on $\mat x$,
i.e., taking the optimum $\mat{x}^*$ so that $\mat{Ax}^* = \pi_{\mat{A}}\mat{b}$.
The second term  $\pi_{\mat{A}^\perp}\mat{b}$ is the \emph{irreducible error}, i.e., $\min\,\norm{\mat{Ax}-\mat b}_2 =\norm{\pi_{\mat A^\perp}\mat b}_2$.


\subsection{Leverage Score Sampling for Tensor Decomposition}
\label{app:leverage-score}

ALS formulations show how each tensor decomposition step reduces to solving a least-squares problem of the form
$\min_{\mat{x}}\,\norm{\mat{A}\mat x - \mat{b}}_2$ with a highly structured $\mat{A}$.
While we can find the optimum in closed form via $(\mat{A}^\top \mat{A})^{+} \mat{A}^\top \mat{b}$, matrix $\mat{A}$ has $I = I_1 \cdots I_N$ rows corresponding to each entry of the tensor (i.e., it is a tall skinny matrix), which can make using the normal equation challenging in practice.

Randomized sketching methods are a popular approach to approximately solving this problem with faster running times with high probability.
In general, these approach sample rows of $\mat{A}$ according to the probability distribution defined by the \emph{leverage scores} of rows, resulting in a random sketching matrix $\mat{S}$ whose height is much smaller than that of $\mat{A}$. For a matrix $\mat{A} \in \R^{I\times R}$ with ($I\gg R$), the leverage scores of $\mat{A}$ is the vector $\ell\in [0,1]^I$ defined by
\[
\ell_i \defeq \bigl(\mat{A}\,(\mat{A^\top A})^+\mat{A}^\top\bigr)_{ii}\,.
\]
Then, for a given $\varepsilon, \delta \in (0,1)$, the sketching algorithm samples $\tilde{O}(\nicefrac{R}{\varepsilon\delta})$ many rows, where the $i$-th row is drawn with probability $\ell_i  / \sum_i \ell_i = \ell_i / \text{rank}\,(\mat{A})$.
With probability at least $1-\delta$, we can guarantee that
\[
\min_{\mat{x}} \,\norm{\mat{S}\mat{A}\mat{x}-\mat{S}\mat{b}}_2 \leq (1+\varepsilon)\, \min_{\mat{x}}\,\norm{\mat{A}\mat x - \mat{b}}_2\,.
\]
The reduced number of rows in $\mat{SA}$ leads to better running times for the least-squares solves.
However, na\"ively computing leverage scores takes as long as computing the closed-form optimum since we need to compute $(\mat{A}^\top \mat{A})^{+}$.
This is where the \emph{structure} of the design matrix $\mat{A}$ comes in to play, i.e., to speed up the leverage score computations.


\subsection{Tensor-Train Decomposition}
\label{app:tt-decomposition-details}

Given a tensor-train (TT) decomposition $\{\tensor{A}^{(n)}\}_{n=1}^{N}$ and index $n\in[N]$, define the
\emph{left chain} $\mat{A}_{<n} \in \R^{(I_1 \cdots I_{n-1}) \times R_{n-1}}$
and
the \emph{right chain} $\mat{A}_{> n}\in \R^{R_n \times (I_{n+1} \cdots I_N)}$ as:
\begin{align*}
    a_{< n}(\underline{i_1\dots i_{n-1}}, r_{n-1})
    &=
    \sum_{r_0,\dots,r_{n-1}} \prod_{k=1}^{n-1} a^{(k)}_{r_{k-1}i_kr_k}
    \\
    a_{> n}(r_n, \underline{i_{n+1}\dots i_N})
    &=
    \sum_{r_{n+1},\dots,r_N} \prod_{k=n+1}^N a^{(k)}_{r_{k-1}i_kr_k}\,,
\end{align*}
where for any $i_s \in [I_s]$ with $s (\neq n)\in [N]$, $\underline{i_1\dots i_{n-1}}:= 1+ \sum_{k=1}^{n-1} (i_k -1)\prod_{j=1}^{k-1} I_j$ and $\underline{i_{n+1}\dots i_N}:= 1+ \sum_{k=n+1}^{N} (i_k -1)\prod_{j=n+1}^{k-1} I_j$.
When ALS optimizes $\tensor{A}^{(n)}$ with all other TT-cores fixed, it solves the regression problem:
\[
    \tensor{A}^{(n)} \!\!
    \gets \!\!
    \argmin_{\tensor{B} \in \R^{R_{n-1}\times I_n \times R_n}}
    \bigl\lVert (\mat{A}_{<n}\kron \mat{A}^\top_{>n})\,\mat{B}_{(2)}^\top - \mat{X}_{(n)}^\top \bigr\rVert_{\frobenius}\,,
\]
which is equivalent to solving $I_n$ Kronecker regression problems in $\R^{\prod_{k\neq n} I_k}$.


\subsection{Tensor Networks}
\label{app:tensor-networks}

A \emph{tensor network} (TN) is a powerful framework that can represent any factorization of a tensor,
so it can recover the three decompositions above as special cases.
A TN decomposition $\text{TN}(\tensor A^{(1)},\dots,\tensor A^{(N)})$ represents a given tensor $\tensor X$ with $N$ tensors $\tensor{A}^{(1)},\dots, \tensor A^{(N)}$ and a \emph{tensor diagram}. As in the above decompositions, the goal is to compute
\[
\argmin_{\tensor A^{(1)},\dots,\tensor A^{(N)}}\,\norm{\tensor X - \text{TN}(\tensor A^{(1)},\dots,\tensor A^{(N)})}_\frobenius^2\,.
\]
A tensor diagram\footnote{We refer readers to \url{https://tensornetwork.org/diagrams/} for more details.} consists of nodes with dangling edges, where a node indicates a tensor, and its dangling edge represents a mode, so that the number of dangling edges is the order of the tensor. For example, a node without an edge indicates a scalar, one with one dangling edge is a vector, and one with two dangling edges is a matrix.

When two dangling edges of two nodes are connected, we say that the two tensors are \emph{contracted} along that mode (i.e., a mode product of those two tensors). For example, when a node with two dangling edges shares one edge with another node with one dangling edge, it indicates a matrix-vector multiplication. Hence, the number of unmatched dangling edges in a tensor diagram corresponds to the order of its representing tensor.

\paragraph{ALS for TN decomposition.}
Given a TN decomposition $\{\tensor{A}^{(n)}\}_{n\in[N]}$, when ALS optimizes a tensor $\tensor A^{(n)}$ with all others fixed, it solves a linear regression problem of the form
\[
\tensor{A}^{(n)} \gets \argmin_{\tensor{B}} \, \lVert \mat A_{\neq n}\mat{B} - \mat{X}\rVert_{\frobenius}\,,
\]
where $\mat A_{\neq n}$ is an appropriate matricization depending on $\tensor A^{(1)},\dots,\tensor A^{(n-1)}, \tensor A^{(n+1)},\dots,\tensor A^{(N)}$, and $\mat B$ and $\mat X$ are suitable matricizations of $\tensor B$ and $\tensor X$, respectively. Structure of $\mat A_{\neq n}$ can be specified through a new tensor diagram obtained by removing the node of $\tensor A^{(n)}$ from the original tensor network diagram.

Just as the ALS approaches for other decomposition algorithms, \citet{malik2022sampling} proposed a sampling-based approach via leverage scores. First of all, they pointed out that $\mat{A}_{\neq n}^\top \mat A_{\neq n}$ can be efficiently computed by exploiting inherent structure of $\mat A_{\neq n}$ (i.e., contract a series of matched edges in a tensor diagram in an appropriate order). They then presented a leverage-score sampling method that draws rows of $\mat A_{\neq n}$ without materializing a full probability vector, and in spirit this approach is similar to one used for the CP decomposition in \cref{subsec:CP-completion}.


\section{Experiments}\label{sec:experiments_extra}
The code used in this work will be made publicly available later.
%{The code to replicate our results can be found in this public Github repositroy:}
%\AL{make a new public github repo and include link later}
%\AN{We should adhere to dual blind policy during review process. Please refer to https://icml.cc/Conferences/2025/AuthorInstructions}
%\AN{This mean it is safe to submit the code as a supplement file (zipped). After the acceptance, we can make the code open on github.}
%\AL{Yes, I will consult you on this tomorrow, thank you.}

\subsection{Pseudocode and training settings for mean-field experiments} \label{subsec:pseudocode}

For experiments concerning MFNNs, the output of a neuron in a two-layer MFNN is modelled by: $h(x_i, z_i) = R\tanh(x_i^3) \tanh(x_i^{1\top} z_i + x_i^2)$, where $x_i = (x_i^1, x_i^2, x_i^3) \in \bR^{d + 1 + 1}$ is its parameter, $z_i$ is the given input and $R$ is a scaling constant. The $\tanh$ activation function is placed on the second layer as boundedness of the model is crucial for our analysis. Noisy gradient descent is then used to train neural networks for $T$ epochs each. We omit the pseudocode for training MFNNs with MFLD since it is identical to the backpropagation with noisy gradient descent algorithm.

\paragraph{Algorithm \ref{alg:circle_data}} Generate the double circle data: $\mathcal{D} = \left( z_i, y_i\right)^n_{i=1}$, $z_i \in \bR^2, y_i \in \bR$ before splitting it into $\mathcal{D}_\text{train}$ and $\mathcal{D}_{\text{test}}$. We set $n=200$, $r_\text{inner}=1$, $r_\text{outer}=2$ and use an 80-20 train-test split for the data.

\paragraph{Algorithm \ref{alg:multi_index_data}} Generate the $k$ multi-index data: $\mathcal{D} = \left( z_i, y_i\right)^n_{i=1}$, $z_i \in \bR^d, y_i \in \bR$. A key step is normalizing and projecting $z_i$ to the inside of a $d$-dimensional hypersphere. We set $n = 500$, $d = 100$, $r=5$, $k=100$ and $\bar{R} = 100$. 

\paragraph{Algorithm \ref{alg:classification}} Describes how we obtain and test the performance of merged MFNNs against (an approximation to) the mean-field limit by computing the sup-norm between both outputs. The relevant results are stored into a dictionary for plotting the heatmaps. The training procedure is identical for both the classification and regression problem. Let $M_\text{max} = 20$ and $N_\text{list} = \{50, 100, \dots ,500 \}$ denote the maximum number of networks to merge and list of neuron settings to train in parallel respectively. We set the hyperparameters for training as follows:
\begin{itemize}
    \item Classification: $R=10$, $N_\infty=10000$, $\eta = 0.1$, $\lambda' = 0.1$, $\lambda = 0.01$, $T=200$ and loss function: logistic loss
    \item Regression: $R=10$, $N_\infty=10000$, $\eta = 0.01$, $\lambda' = 0.1$, $\lambda = 0.01$, $T=100$ and loss function: mean squared error
\end{itemize}


\begin{algorithm} 
\caption{Generate data points along cocentric 2D circles}\label{alg:circle_data}
\begin{algorithmic}[1]
\REQUIRE $n$, $r_{\text{inner}}$, $r_{\text{outer}}$
\ENSURE Dataset $\mathcal{D} = \{(z_i, y_i)\}_{i=1}^n$
\STATE Initialize $\mathcal{D} \gets \emptyset$
\FOR{$i = 1$ to $n$}
    \STATE Sample $\theta \sim \text{Uniform}(0, 2\pi)$
    \STATE Sample $\xi_1, \xi_2 \sim \text{Normal}(0, 0.1)$
    \IF{$i <  n/2$}
        \STATE $r \gets r_{\text{inner}}$
        \STATE $y_i \gets -1$
    \ELSE
        \STATE $r \gets r_{\text{outer}}$
        \STATE $y_i \gets +1$
    \ENDIF
    \STATE Compute Cartesian coordinates: $z_i = (r \cos(\theta) + \xi_1, r \sin(\theta) + \xi_2)$
    \STATE Add $(z_i, y_i)$ to $\mathcal{D}$
\ENDFOR
\STATE Randomly shuffle $\mathcal{D}$
\STATE Split $\mathcal{D}$ into $\mathcal{D}_{\text{train}}$ (80\%) and $\mathcal{D}_{\text{test}}$ (20\%)
\STATE \textbf{return} $\mathcal{D}_{\text{train}}, \mathcal{D}_{\text{test}}$
\end{algorithmic}
\end{algorithm}

\clearpage
\begin{figure}[H]
\vspace{-1.5em}
\begin{algorithm}[H] 
\caption{Generate $k$ multi-index data}
\begin{algorithmic}[1] \label{alg:multi_index_data}
\REQUIRE $n$, $d$, $r$, $k$, $\bar{R}$
\ENSURE Dataset $\mathcal{D} = \{(z_i, y_i)\}_{i=1}^n$
\STATE Initialize $\mathcal{D} \gets \emptyset$
\FOR{$i = 1$ to $n$}
    \STATE Sample $\zeta \sim \text{Normal}(0,1)$
    \STATE $\zeta \gets \zeta^{(1/d)} \times r$ \hfill \COMMENT{Get scaling constant}
    \STATE Sample $z \sim \text{Normal}\left(0, \text{I}_d \right)$
    
    \STATE $z_i \gets z/ |z|$ \hfill \COMMENT{Normalize} 
    \STATE $z_i \gets z_i \times \zeta$ \hfill \COMMENT{Project}
    \STATE $y_i \gets 0$
    \FOR{$j = 1$ to $k$}
        \STATE $y_i \gets y_i + \tanh \left(z_i^j \right)$
    \ENDFOR
\STATE $y_i \gets y_i \times (\bar{R}/k)$
\STATE Add $(z_i, y_i)$ to $\mathcal{D}$
\ENDFOR
\STATE Split $\mathcal{D}$ into $\mathcal{D}_{\text{train}}$ (80\%) and $\mathcal{D}_{\text{test}}$ (20\%)
\STATE \textbf{return} $\mathcal{D}_{\text{train}}, \mathcal{D}_{\text{test}}$
\end{algorithmic}
\end{algorithm}
\end{figure}

\begin{algorithm} 
\caption{Training and merging MFNNs}\label{alg:classification}
\begin{algorithmic}[H]
\REQUIRE $\mathcal{D}_{\text{train}}, \mathcal{D}_{\text{test}} = (z_\text{test}, y_\text{test})$, $N_\infty$, $N_\text{list}$, $M_\text{max}$
\ENSURE Dictionary \textit{sup\_norm\_dic} maps $N$ to the average sup\_norm
\STATE $h_\infty \gets$ Train a MFNN with $N_\infty$ neurons on $\mathcal{D}_{\text{train}}$
\STATE $\hat{y}_\infty \gets$Use $h_\infty$ to predict on $\mathcal{D}_{\text{test}}$
\STATE Initialize \textit{sup\_norm\_dic} $\gets \{\}$
\FOR{$N \in N_\text{list}$}
    \STATE  $\{h_N^{1}, h_N^{2}, \dots h^{M_\text{max}}_N \} \gets$Train $M_\text{max}$ MFNNs with $N$ neurons on $\mathcal{D}_{\text{train}}$
    \STATE Initialize \textit{sup\_norm\_lst} $\gets []$

    \FOR{$M \in \{1, 2, \dots M_\text{max} \}$}
        \STATE \textit{sup\_norm\_total} $\gets 0$
        \FOR{50 iterations}
            \STATE Randomly sample $M$ networks from $\left \{ h_N^1, h_N^2, \dots, h^{M_\text{max}}_N \right \}$
            \STATE $h_{MN} \gets$Merge the $M$ networks to form a new neural network 
            \STATE $\hat{y} \gets$Use $h_{MN}$ to predict on $\mathcal{D}_{\text{test}}$
            \STATE \textit{sup\_norm} $\gets \text{max}\left(|\hat{y} - \hat{y}_\infty |\right)$
            \STATE $\text{\textit{sup\_norm\_total}} \gets \text{\textit{sup\_norm\_total}} + \text{\textit{sup\_norm}}$
        \ENDFOR
        \STATE Append \textit{sup\_norm\_total}/ 50 to \textit{sup\_norm\_lst}
    \ENDFOR
    \STATE \textit{sup\_norm\_dic}[\textit{N}] $\gets$ \textit{sup\_norm\_lst}
\ENDFOR
\STATE \textbf{return} \textit{sup\_norm\_dic}
\end{algorithmic}
\end{algorithm}


\clearpage

\subsection{Additional MFNN experiments} \label{subsec:additional_experiments}
Beyond examining the effect of both $M$ and $N$ on sup norm, we also compare the convergence rate of MFNNs using different $\lambda \in \{10^{-1}, 10^{-2}, 10^{-3}, 10^{-4} \}$ on the multi-index regression problem. Since the training dataset is small and we intend to investigate high $\lambda$, we have to consider the low epoch setting to prevent deterioration of generalization capabilities. We train 20 networks in parallel and average the MSE (in log-scale) at each epoch, repeating this for $N\in \{300, 400, \dots, 800\}$. Figure \ref{fig:experiments_extra} shows that higher $\lambda$ improves the convergence speed of particles and makes training more stable. Finally, we merge networks with the same hyperparameters for comparison across different $\lambda$. A similar trend is observed in Table \ref{table:experiments_extra}, highlighting the efficacy of PoC-based ensembling when training for fewer epochs with a high $\lambda$.

\begin{figure}[H]
\vskip 0.2in
\begin{center}
\centerline{\includegraphics[width=\columnwidth]{extra_experiment.png}}
\caption{Averaged test ln(MSE) of singular MFNNs, across different $N$ and $\lambda$ for 5 epochs}  
\label{fig:experiments_extra}
\end{center}
\end{figure}

\begin{table*}[th]
    \centering
    \caption{MSE comparison between merging $M=20$ networks across different $N$ and $\lambda$ after 5 epochs.}
    \label{table:experiments_extra}
    \begin{footnotesize}
    \begin{tabular}{ccccccc} 
    \toprule
    & \multicolumn{6}{c}{$\boldsymbol{N}$} \\
    $\boldsymbol{\lambda}$ & 300 & 400 & 500 & 600 & 700 & 800\\
    \midrule
    $10^{-1}$ & \underline{0.9132253} & \underline{0.9040508}& \underline{0.9075238}& \underline{0.9044338}& \underline{0.9030165}&  \underline{0.9022377}\\
    $10^{-2}$ & 1.2325489& 1.2229528& 1.2166352& 1.1978958 & 1.1921849& 1.1654898\\
    $10^{-3}$ & 1.5718020& 1.5668763& 1.5607907& 1.5581368& 1.5282313& 1.5234329\\
    $10^{-4}$ & 1.6987042& 1.6887244& 1.6631799& 1.6135653& 1.5860944& 1.5821924\\
    \bottomrule
    \end{tabular}
    \end{footnotesize}
\end{table*}


\clearpage


\subsection{LoRA for finetuning language models}\label{subsec:experiments_extra_lora}
To examine the effect of $\lambda$, we perform LoRA and PoC-based merging by varying $\lambda \in \{0,10^{-5},10^{-4}\}$ with one-epoch training. We optimize eight LoRA parameters of rank $N=32$ in parallel using noisy AdamW with the speficied $\lambda$. Table \ref{table:LoRA_comparison_1epoch} summarizes the results. For LoRA, the table lists the best result among the eight LoRA parameters based on the average accuracy across all datasets and also provides the average accuracies of the eight parameters for each dataset. We observed that for Llama2-7B with $\lambda=0$ and  $\lambda=10^{-5}$, the chances of the optimization converging are very low. Consequently, both the average accuracy of eight LoRAs and the accuracy of PoC-based merging are also low. This is because the regularization strength $\lambda$ controls the optimization speed as seen in Theorem \ref{theorem:mfld_convergence}. On the other hand, by using a high constant $\lambda=10^{-4}$ the average performance was improved, and PoC-based merging achieved quite high accuracy even with only one-epoch of training. This result suggests using high $\lambda$ to reduce the training costs, provided it does not negatively affect generalization error. For Llama3-8B, one-epoch training is sufficient to converge, and while LoRA performed well and PoC-based merging further improved the accuracies.

\begin{table*}[th]
    \centering
    \caption{Accuracy comparison of LoRA and PoC-based merging for finetuning Llama models (1 epoch).}
    \label{table:LoRA_comparison_1epoch}
    \begin{footnotesize}
    \begin{tabular}{cccccccccccc}
        \toprule
        \textbf{Model} & \textbf{Method} & \textbf{$\lambda$} & \textbf{SIQA} & \textbf{PIQA} & \textbf{WinoGrande} & \textbf{OBQA} & \textbf{ARC-c} & \textbf{ARC-e} & \textbf{BoolQ} & \textbf{HellaSwag} & \textbf{Ave.} \\
        \midrule
        \multirow{11}{*}{\begin{tabular}{c}Llama2\\7B\end{tabular}}
            & LoRA (best) & $0$  & $80.55$ & $82.86$ & $83.19$ & $81.60$ & $71.08$ & $84.51$ & $71.90$ & $90.21$ & $80.74$ \\
            & LoRA (ave.) & $0$ & $64.73$  & $76.31$ & $77.76$ & $68.70$ & $57.02$ & $69.02$ & $69.04$ & $70.63$ & $69.15$ \\
            & \textbf{PoC merge} & $0$ & $32.29$ & $62.57$ & $83.58$ & $22.20$ & $28.41$ & $29.42$ & $61.53$ & $28.50$ & $43.56$ \\
            \cmidrule(lr){2-12}
            & LoRA (best) & $10^{-5}$ & $80.14$ & $82.37$ & $83.43$ & $80.40$ & $68.86$ & $83.42$ & $71.68$ & $89.94$ & $80.03$ \\
            & LoRA (ave.) & $10^{-5}$  & $74.37$ & $74.12$ & $80.55$ & $67.50$ & $58.34$ & $71.98$ & $69.43$ & $66.25$ & $70.32$ \\
            & \textbf{PoC merge} & $10^{-5}$ & $74.56$ & $83.84$ & $85.16$ & $60.00$ & $63.14$ & $78.37$ & $68.72$ & $92.77$ & $75.82$ \\
            \cmidrule(lr){2-12}
            & LoRA (best) & $10^{-4}$ & $78.20$ & $80.90$ & $81.22$ & $78.40$ & $65.19$ & $79.00$ & $69.97$ & $86.50$ & $77.42$ \\
            & LoRA (ave.) & $10^{-4}$  & $74.42$ & $77.70$ & $76.08$ & $75.93$ & $60.93$ & $76.25$ & $65.68$ & $66.71$ & $71.71$ \\
            & \textbf{PoC merge} & $10^{-4}$ & $80.76$ & $82.15$ & $84.85$ & $84.80$ & $71.25$ & $85.35$ & $72.26$ & $91.65$ & $81.63$ \\
        \midrule
        \multirow{11}{*}{\begin{tabular}{c}Llama3\\8B\end{tabular}}
            & LoRA (best) & $0$ & $80.45$ & $88.47$ & $86.82$ & $87.60$ & $82.25$ & $90.87$ & $73.85$ & $95.78$ & $85.76$ \\
            & LoRA (ave.) & $0$  & $80.51$ & $88.87$ & $86.85$ & $87.00$ & $80.78$ & $90.98$ & $73.71$ & $95.84$ & $85.57$ \\
            & \textbf{PoC merge} & $0$ & $81.73$ & $88.96$ & $87.77$ & $88.00$ & $81.40$ & $91.71$ & $74.46$ & $96.45$ & $86.31$ \\
            \cmidrule(lr){2-12}
            & LoRA (best) & $10^{-5}$ & $80.50$ & $88.68$ & $86.98$ & $86.80$ & $81.48$ & $91.12$ & $75.14$ & $95.97$ & $85.83$ \\
            & LoRA (ave.) & $10^{-5}$  & $80.83$ & $88.64$ & $86.85$ & $87.05$ & $80.39$ & $90.76$ & $71.54$ & $95.87$ & $85.24$ \\
            & \textbf{PoC merge} & $10^{-5}$ & $81.53$ & $89.45$ & $87.92$ & $87.80$ & $82.25$ & $91.79$ & $75.54$ & $96.44$ & $86.59$ \\
            \cmidrule(lr){2-12}
            & LoRA (best) & $10^{-4}$ & $80.30$ & $88.57$ & $86.42$ & $87.20$ & $78.07$ & $89.81$ & $73.61$ & $95.14$ & $84.89$ \\
            & LoRA (ave.) & $10^{-4}$ & $80.00$ & $88.20$ & $85.69$ & $86.23$ & $78.86$ & $89.48$ & $73.08$ & $95.05$ & $84.57$ \\
            & \textbf{PoC merge} & $10^{-4}$ & $80.71$ & $89.72$ & $88.08$ & $89.00$ & $82.17$ & $91.79$ & $74.56$ & $96.36$ & $86.55$ \\
        \bottomrule
    \end{tabular}
    \end{footnotesize}
\end{table*}


\end{document}
