\section{Introduction}\label{sec:introduction}
\section{Introduction}


\begin{figure}[t]
\centering
\includegraphics[width=0.6\columnwidth]{figures/evaluation_desiderata_V5.pdf}
\vspace{-0.5cm}
\caption{\systemName is a platform for conducting realistic evaluations of code LLMs, collecting human preferences of coding models with real users, real tasks, and in realistic environments, aimed at addressing the limitations of existing evaluations.
}
\label{fig:motivation}
\end{figure}

\begin{figure*}[t]
\centering
\includegraphics[width=\textwidth]{figures/system_design_v2.png}
\caption{We introduce \systemName, a VSCode extension to collect human preferences of code directly in a developer's IDE. \systemName enables developers to use code completions from various models. The system comprises a) the interface in the user's IDE which presents paired completions to users (left), b) a sampling strategy that picks model pairs to reduce latency (right, top), and c) a prompting scheme that allows diverse LLMs to perform code completions with high fidelity.
Users can select between the top completion (green box) using \texttt{tab} or the bottom completion (blue box) using \texttt{shift+tab}.}
\label{fig:overview}
\end{figure*}

As model capabilities improve, large language models (LLMs) are increasingly integrated into user environments and workflows.
For example, software developers code with AI in integrated developer environments (IDEs)~\citep{peng2023impact}, doctors rely on notes generated through ambient listening~\citep{oberst2024science}, and lawyers consider case evidence identified by electronic discovery systems~\citep{yang2024beyond}.
Increasing deployment of models in productivity tools demands evaluation that more closely reflects real-world circumstances~\citep{hutchinson2022evaluation, saxon2024benchmarks, kapoor2024ai}.
While newer benchmarks and live platforms incorporate human feedback to capture real-world usage, they almost exclusively focus on evaluating LLMs in chat conversations~\citep{zheng2023judging,dubois2023alpacafarm,chiang2024chatbot, kirk2024the}.
Model evaluation must move beyond chat-based interactions and into specialized user environments.



 

In this work, we focus on evaluating LLM-based coding assistants. 
Despite the popularity of these tools---millions of developers use Github Copilot~\citep{Copilot}---existing
evaluations of the coding capabilities of new models exhibit multiple limitations (Figure~\ref{fig:motivation}, bottom).
Traditional ML benchmarks evaluate LLM capabilities by measuring how well a model can complete static, interview-style coding tasks~\citep{chen2021evaluating,austin2021program,jain2024livecodebench, white2024livebench} and lack \emph{real users}. 
User studies recruit real users to evaluate the effectiveness of LLMs as coding assistants, but are often limited to simple programming tasks as opposed to \emph{real tasks}~\citep{vaithilingam2022expectation,ross2023programmer, mozannar2024realhumaneval}.
Recent efforts to collect human feedback such as Chatbot Arena~\citep{chiang2024chatbot} are still removed from a \emph{realistic environment}, resulting in users and data that deviate from typical software development processes.
We introduce \systemName to address these limitations (Figure~\ref{fig:motivation}, top), and we describe our three main contributions below.


\textbf{We deploy \systemName in-the-wild to collect human preferences on code.} 
\systemName is a Visual Studio Code extension, collecting preferences directly in a developer's IDE within their actual workflow (Figure~\ref{fig:overview}).
\systemName provides developers with code completions, akin to the type of support provided by Github Copilot~\citep{Copilot}. 
Over the past 3 months, \systemName has served over~\completions suggestions from 10 state-of-the-art LLMs, 
gathering \sampleCount~votes from \userCount~users.
To collect user preferences,
\systemName presents a novel interface that shows users paired code completions from two different LLMs, which are determined based on a sampling strategy that aims to 
mitigate latency while preserving coverage across model comparisons.
Additionally, we devise a prompting scheme that allows a diverse set of models to perform code completions with high fidelity.
See Section~\ref{sec:system} and Section~\ref{sec:deployment} for details about system design and deployment respectively.



\textbf{We construct a leaderboard of user preferences and find notable differences from existing static benchmarks and human preference leaderboards.}
In general, we observe that smaller models seem to overperform in static benchmarks compared to our leaderboard, while performance among larger models is mixed (Section~\ref{sec:leaderboard_calculation}).
We attribute these differences to the fact that \systemName is exposed to users and tasks that differ drastically from code evaluations in the past. 
Our data spans 103 programming languages and 24 natural languages as well as a variety of real-world applications and code structures, while static benchmarks tend to focus on a specific programming and natural language and task (e.g. coding competition problems).
Additionally, while all of \systemName interactions contain code contexts and the majority involve infilling tasks, a much smaller fraction of Chatbot Arena's coding tasks contain code context, with infilling tasks appearing even more rarely. 
We analyze our data in depth in Section~\ref{subsec:comparison}.



\textbf{We derive new insights into user preferences of code by analyzing \systemName's diverse and distinct data distribution.}
We compare user preferences across different stratifications of input data (e.g., common versus rare languages) and observe which affect observed preferences most (Section~\ref{sec:analysis}).
For example, while user preferences stay relatively consistent across various programming languages, they differ drastically between different task categories (e.g. frontend/backend versus algorithm design).
We also observe variations in user preference due to different features related to code structure 
(e.g., context length and completion patterns).
We open-source \systemName and release a curated subset of code contexts.
Altogether, our results highlight the necessity of model evaluation in realistic and domain-specific settings.






Recent advancements in large language models (LLMs) \cite{gpt4o,llama3,qwen2} have greatly propelled the evolution of natural language processing, positioning them as foundational models for artificial general intelligence (AGI). Despite the remarkable reasoning ability \cite{openai_o1}, LLMs are still limited in accessing real-time information and lack of domain-specific knowledge, which is outside the pre-training corpus. To address these limitations, retrieval-augmented generation (RAG) \cite{gao2023retrieval} has become a popular paradigm in adding new knowledge to the static LLMs by retrieving relevant documents into the context of LLM generation. 

Existing RAG methods typically retrieve documents independently, making it difficult to capture complex relationships between pieces of knowledge \cite{karpukhin2020dense,bge_m3,moreira2024nv}. This limitation hampers the performance of LLMs in integrating knowledge across document boundaries, particularly in multi-hop reasoning tasks \cite{yang2018hotpotqa,trivedi2022musique} and real-world applications like legal judgment \cite{kang2024bridging} and medical diagnoses \cite{jin-etal-2019-pubmedqa}, which require reasoning over multiple sources. Although recent methods have expanded the retrieval process into multiple steps and incorporate LLM reasoning, they still encounter high computational costs due to iterative retrieval and reasoning with LLMs \cite{trivedi2023interleaving,sunthink,joshi2024reaper}.

Recently, graph-enhanced retrieval augmented generation (GraphRAG) \cite{peng2024graph,han2024retrieval} has emerged as a novel solution that builds a graph structure to explicitly model the intricate relationships between knowledge. This enables the development of a graph-enhanced retriever to identify relevant information using graphs. The structural nature of graphs allows GraphRAG to capture global context and dependencies among documents, significantly improving reasoning across multiple sources \cite{edge2024local}. 
Methods like HippoRAG \cite{gutiérrez2024hipporag} enhance retrieval by employing a personalized PageRank algorithm to locate relevant knowledge with graphs. However, these algorithms rely solely on the graph structure, which is often noisy or incomplete, limiting their overall performance.
%However, the graph structure in GraphRAG may contain noise or be incomplete, which hampers the retriever's ability to locate relevant knowledge solely based on graph structure \cite{gutiérrez2024hipporag}. 
Alternative methods \cite{mavromatis2024gnn,he2024g} incorporate graph neural networks (GNNs) into the retrieval process. These methods have shown impressive performance due to GNNs' powerful reasoning capabilities in handling incomplete graphs \cite{galkintowards}. Nevertheless, they still face limitations in generalizability since they require training from scratch on new datasets.

% GFM
Nowadays, the search for a foundation GNN model that can transfer and generalize across different datasets has been an active research topic. Ideally, a foundation GNN or graph foundation model (GFM) can benefit from large-scale training and generalize across diverse graphs \cite{maoposition,liu2023towards}. Efforts have been made to identify transferable graph tokens (e.g., motifs, sub-trees, and relation graphs) \cite{galkintowards,wang2024gft,xia2024opengraph} that can be shared among different graphs for GFM design. However, these methods primarily focus on graph-related tasks (e.g., node classification and link prediction), leaving the design of a GFM to enhance LLMs' reasoning ability unexplored.
%How to design a GFM to enhance the reasoning of LLM remains an open question.

% Foundation models (FMs) \cite{bommasani2021opportunities} renovate the landscape of AI by providing a large-scale model that can be directly applied to diverse datasets and applications without expensive training. FMs, trained on massive datasets, are expected to fit the neural scaling law \cite{hestness2017deep} as the scale of training data and model size grows, which has been validated in recent foundation models in computer vision \cite{dehghani2023scaling} and natural language processing \cite{kaplan2020scaling}. However, how to design a general and scalable graph foundation model (GFM) that can be directly applied to various unseen datasets to power GraphRAG remains an open challenge \cite{maoposition,liu2023towards}.

To bridge the gap, in this paper, we propose an effective, efficient, and general graph foundation model for retrieval augmented generation (\ourmethod), thereby enhancing LLMs' reasoning ability.
%
As shown in \Cref{fig:intro}, we create a \emph{knowledge graph index} (KG-index) from documents in each dataset. The KG-index consists of interconnected factual triples pointing to the original documents, which serves as a structural knowledge index across multiple sources, enhancing the integration of diverse knowledge for complex reasoning tasks \cite{gutiérrez2024hipporag}. 
%This aligns with the hippocampal memory indexing theory \cite{teyler1986hippocampal}, where the KG-index functions like an artificial hippocampus to store associations between knowledge memories, enhancing the integration of diverse knowledge for complex reasoning tasks\cite{gutiérrez2024hipporag}.
% As shown in \Cref{fig:intro}, we begin by creating a \emph{knowledge graph index} (KG-index) from documents in each dataset. The KG-index holds a set of interconnected triples pointing to the original documents, serving as a structural index of knowledge across multiple documents. The KG-index 
% aligns with the human hippocampal memory indexing theory \cite{teyler1986hippocampal} where the KG-index acts like an artificial hippocampal index to store associations between knowledge memory, enhancing the integration of diverse knowledge for complex reasoning tasks \cite{gutiérrez2024hipporag}. 
%
Then, we present the \emph{graph foundation model retriever} (GFM retriever), driven by a query-dependent GNN that captures complex query-knowledge relationships in a unified,  transferable space of semantics and graph structure. Through multi-layer message passing, the GFM retriever enables efficient multi-hop retrieval in a single step, surpassing previous multi-step methods.
%
The GFM retriever, with 8M parameters, undergoes a two-stage training: \emph{unsupervised KG completion pre-training} and \emph{supervised document retrieval fine-tuning} on large-scale datasets, including 60 knowledge graphs with over 14M triples and 700k documents. This large-scale training ensures the generalizability of GFM retriever to be applied directly to unseen datasets without further training.

In experiments, \ourmethod achieves state-of-the-art performance across three multi-hop QA datasets, demonstrating its effectiveness and efficiency in multi-hop reasoning. It also generalizes well across seven RAG datasets from diverse domains, such as biomedical, customer service, and general knowledge, without requiring additional training. Furthermore, \ourmethod follows the neural scaling law \cite{hestness2017deep}, whose performance benefits from training data and model size scaling, emphasizing its potential as a foundational model for future improvements.

The main contributions of this paper are as follows:
\begin{itemize}
\item We introduce a graph foundation model for retrieval augmented generation (\ourmethod), powered by a novel query-dependent GNN to enable efficient multi-hop retrieval within a single step.
\item We train a large-scale model with 8M parameters, marking the first graph foundation model (GFM) that can be applied directly to various unseen datasets for retrieval augmented generation.
\item We evaluate \ourmethod on three multi-hop QA datasets and seven domain-specific RAG datasets, achieving state-of-the-art performance across all, demonstrating its effectiveness, efficiency, generalizability, and potential as a foundational model for further enhancement.
\end{itemize}
