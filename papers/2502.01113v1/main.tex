%%%%%%%% ICML 2025 EXAMPLE LATEX SUBMISSION FILE %%%%%%%%%%%%%%%%%

\documentclass{article}

\usepackage{hyperref}
\usepackage{refstyle}
\usepackage{amsmath}
\usepackage{cleveref}

\usepackage{booktabs}
\usepackage{multirow} %
\usepackage{soul}%
\usepackage{tabularx}
\usepackage{enumitem}

\usepackage{amssymb}


\DeclareMathOperator*{\argmax}{argmax} %
\usepackage{pifont}
\newcommand{\cmark}{\ding{51}}%
\newcommand{\xmark}{\ding{55}}%

\usepackage{graphicx}

\interfootnotelinepenalty=10000

\crefformat{section}{\S#2#1#3}
\crefformat{subsection}{\S#2#1#3}
\crefformat{subsubsection}{\S#2#1#3}
\crefrangeformat{section}{\S#3#1#4 to~\S#5#2#6}
\crefmultiformat{section}{\S#2#1#3}{ and~\S#2#1#3}{, #2#1#3}{ and~#2#1#3}
\Crefformat{figure}{#2Fig.~#1#3}
\Crefmultiformat{figure}{Figs.~#2#1#3}{ and~#2#1#3}{, #2#1#3}{ and~#2#1#3}
\Crefformat{table}{#2Tab.~#1#3}
\Crefmultiformat{table}{Tabs.~#2#1#3}{ and~#2#1#3}{, #2#1#3}{ and~#2#1#3}
\Crefformat{appendix}{#2Appx.~\S#1#3}
\crefformat{algorithm}{Alg.~#2#1#3}
\Crefformat{equation}{#2Eq.~#1#3}

\newcommand{\todo}[1]{{\color{red}[{TODO:} #1]}}

\newcommand{\task}{\textsc{1MKR}}


% Recommended, but optional, packages for figures and better typesetting:
\usepackage{microtype}
\usepackage{graphicx}
\usepackage{subfigure}
\usepackage{booktabs} % for professional tables

% hyperref makes hyperlinks in the resulting PDF.
% If your build breaks (sometimes temporarily if a hyperlink spans a page)
% please comment out the following usepackage line and replace
% \usepackage{icml2025} with \usepackage[nohyperref]{icml2025} above.
\usepackage{hyperref}


% Attempt to make hyperref and algorithmic work together better:
\newcommand{\theHalgorithm}{\arabic{algorithm}}

% Use the following line for the initial blind version submitted for review:
% \usepackage{icml2025}

% If accepted, instead use the following line for the camera-ready submission:
\usepackage[accepted]{icml2025}

% For theorems and such
\usepackage{amsmath}
\usepackage{amssymb}
\usepackage{mathtools}
\usepackage{amsthm}

% if you use cleveref..
\usepackage[capitalize,noabbrev]{cleveref}

%%%%%%%%%%%%%%%%%%%%%%%%%%%%%%%%
% THEOREMS
%%%%%%%%%%%%%%%%%%%%%%%%%%%%%%%%
\theoremstyle{plain}
\newtheorem{theorem}{Theorem}[section]
\newtheorem{proposition}[theorem]{Proposition}
\newtheorem{lemma}[theorem]{Lemma}
\newtheorem{corollary}[theorem]{Corollary}
\theoremstyle{definition}
\newtheorem{definition}[theorem]{Definition}
\newtheorem{assumption}[theorem]{Assumption}
\theoremstyle{remark}
\newtheorem{remark}[theorem]{Remark}

\newcommand{\PN}[1]{\textcolor{cyan}{SP: #1}}


% Todonotes is useful during development; simply uncomment the next line
%    and comment out the line below the next line to turn off comments
%\usepackage[disable,textsize=tiny]{todonotes}
% \usepackage[textsize=tiny]{todonotes}


% The \icmltitle you define below is probably too long as a header.
% Therefore, a short form for the running title is supplied here:
% \icmltitlerunning{Submission and Formatting Instructions for ICML 2025}

\begin{document}
\doparttoc % Tell to minitoc to generate a toc for the parts
\faketableofcontents % Run a fake tableofcontents command for the partocs
\twocolumn[
\icmltitle{GFM-RAG: Graph Foundation Model for Retrieval Augmented Generation}

% It is OKAY to include author information, even for blind
% submissions: the style file will automatically remove it for you
% unless you've provided the [accepted] option to the icml2025
% package.

% List of affiliations: The first argument should be a (short)
% identifier you will use later to specify author affiliations
% Academic affiliations should list Department, University, City, Region, Country
% Industry affiliations should list Company, City, Region, Country

% You can specify symbols, otherwise they are numbered in order.
% Ideally, you should not use this facility. Affiliations will be numbered
% in order of appearance and this is the preferred way.
\icmlsetsymbol{equal}{*}

\begin{icmlauthorlist}
\icmlauthor{Linhao Luo}{equal,monash}
\icmlauthor{Zicheng Zhao}{equal,nust}
\icmlauthor{Gholamreza Haffari}{monash}
\icmlauthor{Dinh Phung}{monash}
\icmlauthor{Chen Gong}{nust}
\icmlauthor{Shirui Pan}{griffith}
% \icmlauthor{Firstname7 Lastname7}{comp}
%\icmlauthor{}{sch}
% \icmlauthor{Firstname8 Lastname8}{sch}
% \icmlauthor{Firstname8 Lastname8}{yyy,comp}
%\icmlauthor{}{sch}
%\icmlauthor{}{sch}
\end{icmlauthorlist}

\icmlaffiliation{monash}{Monash University}
\icmlaffiliation{nust}{Nanjing University of Science and Technology}
\icmlaffiliation{griffith}{Griffith University}
\icmlcorrespondingauthor{Shirui Pan}{s.pan@griffith.edu.au}
\icmlcorrespondingauthor{Linhao Luo}{linhao.luo@monash.edu}

% You may provide any keywords that you
% find helpful for describing your paper; these are used to populate
% the "keywords" metadata in the PDF but will not be shown in the document
\icmlkeywords{Graph Foundation Model, Retrieval Augmented Generation, Graph-enhanced Retrieval Augmented Generation}

\vskip 0.3in
]

% this must go after the closing bracket ] following \twocolumn[ ...

% This command actually creates the footnote in the first column
% listing the affiliations and the copyright notice.
% The command takes one argument, which is text to display at the start of the footnote.
% The \icmlEqualContribution command is standard text for equal contribution.
% Remove it (just {}) if you do not need this facility.

% \printAffiliationsAndNotice{}  % leave blank if no need to mention equal contribution
\printAffiliationsAndNotice{\icmlEqualContribution} % otherwise use the standard text.

In practice,  physical spatiotemporal forecasting can suffer from data scarcity, because collecting large-scale data is non-trivial, especially for extreme events. 
Hence, we propose \method{}, a novel probabilistic framework to realize iterative self-training with new self-ensemble strategies, 
achieving better physical consistency and generalization on extreme events. 
Following any base forecasting model, 
we can encode its deterministic outputs into a latent space and retrieve multiple codebook entries to generate probabilistic outputs. 
Then \method{} extends the beam search from discrete spaces to the continuous state spaces in this field.
We can further employ domain-specific metrics (e.g., Critical Success Index for extreme events) to filter out the top-k candidates and develop the new self-ensemble strategy by combining the high-quality candidates. 
The self-ensemble can not only improve the inference quality and robustness but also iteratively augment the training datasets during continuous self-training. 
Consequently, \method{} realizes the exploration of rare but critical phenomena beyond the original dataset. 
Comprehensive experiments on different benchmarks and backbones show that \method{} consistently reduces forecasting MSE (up to 39\%), enhancing extreme events detection and proving its effectiveness in handling data scarcity. Our codes are available at~\url{https://github.com/easylearningscores/BeamVQ}.



% 在气象预报、流体模拟以及基于偏微分方程(PDE)的多物理系统模型中,数据稀缺下的时空预测仍然是一个关键挑战。本文提出了\method{},一个统一的框架,旨在同时解决标注数据有限以及在确保物理一致性的前提下捕捉极端事件的难题。首先,我们训练了一个确定性的基础模型,从小规模数据中学习主要动力学。随后,通过Top-K 向量量化变分自编码器(VQ-VAE)对基础模型的输出进行增强,该模块将确定性预测编码到潜在空间,并检索多个码本条目以生成多样化且物理上合理的重构结果。一个新颖的联合优化过程利用领域特定的指标(例如关键成功指数)引导基础模型向更准确且对极端事件敏感的预测方向优化。在推理阶段,我们采用束搜索策略,维持多个候选轨迹并通过指标感知评分进行迭代剪枝,从而在探索罕见但关键现象与利用最可能的系统轨迹之间实现平衡。在多个气象和流体流动基准数据集上的大量实验表明,\method{}显著提升了预测精度,增强了对极端状态的检测能力,并保持了物理合理性,证明了其在数据稀缺场景下进行时空预测的优越性。

The ubiquitous question "How did I do this before ChatGPT?" has become a cultural touch point, highlighting how Large Language Models (LLMs) have gradually permeated people's everyday lives. While initially introduced as general-purpose chatbots, LLMs have been adopted in unexpectedly diverse ways \cite{chkirbene2024applications}. These systems now play multiple roles in decision-making processes and tasks, ranging from information providers to triggers for human self-reflection \cite{kim2022bridging, kmmer2024effects}. This widespread integration has raised the question about how users develop dependencies on and relationships with these AI systems \cite{he2025conversational}.

Previous Human-Computer Interaction (HCI) research has extensively examined domain-specific LLM applications \cite{jin2024teach, liu2024selenite}. These studies have yielded insights into specialized use cases and led to targeted interaction design improvements. However, the broader impact of LLMs on everyday decision-making and tasks remains under-explored. As users increasingly integrate these tools into their daily routines, understanding the tangible impacts of habitual use becomes crucial \cite{kmmer2024effects}. 

Recent studies have attempted to measure the impact of LLM use through quantitative metrics such as task performance and decision accuracy \cite{kim2025fostering}. However, these measurement-based approaches cannot fully capture how people delegate everyday decisions to LLMs or the resulting meta-cognitive effects. Furthermore, everyday decisions encompass a broad spectrum of choices, from routine task management to social interaction planning \cite{eigner2024determinants, dhami2012cct}, making them challenging to examine through purely quantitative and task-specific approaches.

To address this gap, we conducted a qualitative study examining heavy LLM users who regularly rely on these systems for everyday decisions and tasks. Through interviews and analysis, we explored how these users integrate LLMs into their decision-making processes, what types of decisions they choose to delegate, and how this delegation affects their cognitive patterns and decision-making confidence. Our study addressed three research questions: 

\begin{itemize}
\item RQ1: How do heavy LLM users integrate LLMs into their everyday decision-making process?
\item RQ2: What underlying needs do heavy LLM users seek to fulfill through LLM assistance?
\item RQ3: How do heavy LLM users conceptualize and evaluate their relationship with LLMs?
\end{itemize}

Through these research questions, we examine three key aspects of heavy LLM use. RQ1 explores emergent use cases and notable patterns in how users incorporate LLMs into their decision-making processes. RQ2 investigates the fundamental motivations and needs that drive sustained LLM usage. RQ3 examines how users develop their mental models of LLMs and reflect on their extensive interaction with these systems.
\begin{figure*}[ht]
    \centering
    \includegraphics[width=\textwidth, trim=79 280 93 123, clip]{figures/framework_img.pdf}
    \caption{The pipeline of the \ENDow{} framework 
    %where each component is specified in a given configuration. 
    which yields a downstream task score and a WER score of the transcript set input to the task. The pipeline is executed for several severeties of noising and types of cleaning techniques. %Acoustic noising is applied at $k$ intensities, providing $k+1$ audio versions (including the non-noised version), eventually producing $k+2$ transcript versions (including the source transcript). Applying transcript cleaning reveals the effect of \textit{types} of noise. 
    Resulting scores are plotted on a graph for the analyses, as in, e.g., \autoref{fig_cleaning_graphs}.}
    %The pipeline is executed on $k+1$ intensities of acoustic noising (including the non-noised version), producing $k+2$ scores for the downstream task (including execution on the source transcripts). This process eventually describes the effect of the \textit{intensity} of transcript noise on the downstream task. The process is repeated for $m$ cleaning techniques ($m+1$ when including no cleaning), to analyze the benefit of a cleaning approach and the effect of the \textit{types} of transcript noise.}
    \label{fig_framework}
\end{figure*}
\section{Related Work}

\noindent\textbf{Retrieval-augmented generation (RAG)} \cite{gao2023retrieval} provides an effective way to integrate external knowledge into large language models (LLMs) by retrieving relevant documents to facilitate LLM generation. Early works adopt the pre-trained dense embedding model to encode documents as separate vectors \cite{karpukhin2020dense,bge_m3,li2023towards,moreira2024nv}, which are then retrieved by calculating the similarity to the query. Despite efficiency and generalizability, these methods struggle to capture complex document relationships. Subsequent studies have explored multi-step retrieval, where LLMs guide an iterative process to retrieve and reason over multiple documents \cite{trivedi2023interleaving,jiang2023active,su-etal-2024-dragin}. However, this approach is computationally expensive. 

\noindent\textbf{Graph-enhanced retrieval augmented generation (GraphRAG)} \cite{peng2024graph,han2024retrieval} is a novel approach that builds graphs to explicitly model the complex relationships between knowledge, facilitating comprehensive retrieval and reasoning. Early research focuses on retrieving information from existing knowledge graphs (KGs), such as WikiData \cite{vrandevcic2014wikidata} and Freebase \cite{bollacker2008freebase}, by identifying relevant facts or reasoning paths \cite{li2023graph,luoreasoning}. Recent studies have integrated documents with KGs to improve knowledge coverage and retrieval \cite{edge2024local,liang2024kag}. A graph structure is built from these documents to aid in identifying relevant content for LLM generation \cite{dong2024don}. Based on graphs, LightRAG \cite{guo2024lightrag} incorporates graph structures into text indexing and retrieval, enabling efficient retrieval of entities and their relationships.
HippoRAG \cite{gutiérrez2024hipporag} enhances multi-hop retrieval by using a personalized PageRank algorithm to locate relevant knowledge with graphs. However, the graph structure can be noisy and incomplete, leading to suboptimal performance. Efforts to incorporate GNNs into graph-enhanced RAG \cite{mavromatis2024gnn,he2024g} have shown impressive results due to the strong graph reasoning capabilities of GNNs in handling incomplete graphs \cite{galkintowards}. Nonetheless, these methods still limit in generalizability due to the lack of a graph foundational model.

\noindent\textbf{Graph Foundation models (GFM)} aims to be a large-scale model that can generalize to various datasets \cite{maoposition,liu2023towards}. The main challenge in designing GFMs is identifying graph tokens that capture invariance across diverse graph data. For instance, ULTRA \cite{galkintowards} employs four fundamental relational interactions in knowledge graphs (KGs) to create a GFM for link prediction. OpenGraph \cite{xia2024opengraph} develops a graph tokenizer that converts graphs into a unified node token representation, enabling transformer-like GFMs for tasks such as link prediction and node classification. GFT \cite{wang2024gft} introduces a transferable tree vocabulary to construct a GFM that demonstrates effectiveness across various tasks and domains in graph learning. Despite these successful efforts, most methods primarily focus on conventional graph-related tasks. How to design a GFM to enhance the reasoning of LLM remains an open question.

% \noindent\textbf{Foundation models (FMs)} revolutionize AI research by offering a large-scale model that can be directly applied to various datasets. FMs have shown impressive effectiveness in many fields, including computer vision \cite{dehghani2023scaling}, natural language processing \cite{kaplan2020scaling}, audio \cite{borsos2023audiolm}, and video \cite{zhao2023learning}. To enable generalizability, FMs generally undergo large-scale training on diverse datasets, adhering to the neural scaling law \cite{hestness2017deep} as performance improves with increasing training data and model size. The exploration of graph foundation models (GFMs) is also a trending topic in the graph learning community \cite{maoposition,liu2023towards}. However, designing GFMs remains a challenge due to the complexity of graph structures and the diversity of features present in different datasets.
% \input{sections/4 preliminary.tex}
\section{Approach}\label{sec:approach}
% In this section, we will introduce the proposed \ourmethod, which consists of three main components: (1) \emph{KG-index construction}, which constructs a knowledge graph index from the document corpus; (2) \emph{graph foundation model retriever} (GFM retriever), which is pre-trained on large-scale datasets and could retrieve documents based on any user query and knowledge graph index; and (3) \emph{documents ranking and answer generation}, which ranks the retrieved documents and generates the final answer. The overall framework of \ourmethod is illustrated in \Cref{fig:framework}.

The proposed \ourmethod essentially implements a GraphRAG paradigm by constructing graphs from documents and using a graph-enhanced retriever to retrieve relevant documents.

\noindent\textbf{GFM-RAG Overview.}
Given a set of documents $\gD=\{D_1,D_2,\ldots,D_{|\gD|}\}$, we construct a knowledge graph $\gG=\{(e,r,e')\in \gE\times\gR\times\gE\}$, where $e,e'\in\gE$ and $r\in\gR$ denote the set of entities and relations extracted from $\gD$, respectively. 
%
For a user query $q$, we aim to design a graph-enhanced retriever to obtain relevant documents from $\gD$ by leveraging the knowledge graph $\gG$. The whole \ourmethod process can be formulated as:
\begin{gather}
    \gG = \text{KG-index}(\gD), \\ \gD^K = \text{GFM-Retriever}(q,\gD,\gG),\\
    a = \text{LLM}(q,\gD^K).
\end{gather}
In the first step, KG-index($\cdot$)  constructs a knowledge graph index $\gG$ from the document corpus $\gD$, followed by our proposed \emph{graph foundation model retriever} (GFM-Retriever), which is pre-trained on large-scale datasets. It retrieves top-$K$ documents based on any user query $q$ and knowledge graph index $\gG$. The retrieved documents $\gD^K$, along with the query $q$, are then input into a large language model (LLM) to generate the final answer $a$.
% The final step ranks the retrieved documents and generates the final answer $a$ for the query $q$ with LLM based on the retrieved documents $\gD^K$.
%where $\gD^K$ denotes the top-$K$ retrieved documents and $a$ denotes the final answer generated by large language models (LLMs). 
%
These three main components in \ourmethod are illustrated in \Cref{fig:framework} and will be detailed next.

% These three main components in \ourmethod: (1) \emph{KG-index construction}, which constructs a knowledge graph index from the document corpus; (2) \emph{graph foundation model retriever} (GFM retriever), which is pre-trained on large-scale datasets and could retrieve documents based on any user query and knowledge graph index; and (3) \emph{documents ranking and answer generation}, which ranks the retrieved documents and generates the final answer, are illustrated in \Cref{fig:framework} and will be detailed next.


\subsection{KG-index Construction}\label{sec:kg-construction}
Conventional embedding-based index methods encode documents as separate vectors \cite{karpukhin2020dense,bge_m3,moreira2024nv}, which are limited in modeling the relationships between them. Knowledge graphs (KGs), on the other hand, explicitly capturing the relationships between millions of facts, can provide a structural index of knowledge across multiple documents \cite{edge2024local,gutiérrez2024hipporag}. The structural nature of the KG-index aligns well with the human hippocampal memory indexing theory \cite{teyler1986hippocampal}, where the KG-index functions like an artificial hippocampus to store associations between knowledge memories, enhancing the integration of diverse knowledge for complex reasoning tasks \cite{gutiérrez2024hipporag}.

To construct the KG-index, given a set of documents $\gD$, we first extract entities $\gE$ and relations $\gR$ to form triples $\gT$ from documents. Then, the entity to document inverted index $M \in \{0,1\}^{|\gE|\times|\gD|}$ is constructed to record the entities mentioned in each document. Such a process can be achieved by existing open information extraction (OpenIE) tools \cite{angeli2015leveraging,ijcai2022p793,pai2024survey}. To better capture the connection between knowledge, we further conduct the entity resolution \cite{gillick2019learning,zeakis2023pre} to add additional edges $\gT^{\texttt{+}}$ between entities with similar semantics, e.g., (\texttt{USA}, \texttt{equivalent}, \texttt{United States of America}). Therefore, the final KG-index $\gG$ is constructed as $\gG=\{(e,r,e')\in\gT\cup\gT^{\texttt{+}}\}$. In implementation, we leverage an LLM \cite{gpt4o} as the OpenIE tool and a pre-trained dense embedding model \cite{santhanam2022colbertv2} for entity resolution.

\subsection{Graph Foundation Model (GFM) Retriever}\label{sec:gfm-retriever}
The GFM retriever is designed to retrieve relevant documents based on any user query and the constructed KG-index. While the KG-index offers a structured representation of knowledge, it still suffers from incompleteness and noise, resulting in suboptimal retrieval performance when solely relying on its structure \cite{gutiérrez2024hipporag}. Recently, graph neural networks (GNNs) \cite{wu2020comprehensive} have shown impressive graph reasoning ability by capturing the complex relationships between knowledge for retrieval or question answering \cite{mavromatis2024gnn,he2024g}. However, existing GNNs are limited in generalizability, as they are usually trained on specific graphs \cite{maoposition,liu2023towards}, which limits their application to unseen corpora and KGs. Therefore, there is still a need for a graph foundation model that can be directly applied to unseen datasets and KGs without additional training.

% What a GFM for graph retriever should be like?
To address these issues, we propose the first graph foundation model-powered retriever (GFM retriever), which harnesses the graph reasoning ability of GNNs to capture the complex relationships between queries, documents, and knowledge graphs in a unified and transferable space. The GFM retriever employs a query-dependent GNN to identify relevant entities in graphs that will aid in locating pertinent documents. After pre-training on large-scale datasets, the GFM retriever can be directly applied to new corpora and KGs without further training.

\subsubsection{Query-dependent GNN}\label{sec:message-passing}
Conventional GNNs \cite{gilmer2017neural} follow the message passing paradigm, which iteratively aggregates information from neighbors to update entity representations. Such a paradigm is not suitable for the GFM retriever as it is graph-specific and neglects the relevance of queries. Recent query-dependent GNNs \cite{zhu2021neural,galkintowards} have shown promising results in capturing query-specific information and generalizability to unseen graphs, which is essential for the GFM retriever and can be formulated as: 
% The comparison between the conventional GNN and query-dependent GNN is as follows:
% \begin{align}
%     & \text{\textbf{Conventional GNN: }} 
%         H^L = \text{GNN}(\gG,H^0),
%     \\
%     & \text{\textbf{Query-dependent GNN: }}
%         H_q^L = \text{GNN}_q(q,\gG,H^0),
%         % p(e|q) = \sigma(\text{MLP}(h_{e|q})),~h_{e|q} \in H_q^T, 
%     % \end{gathered}
% \end{align}
\begin{equation}
    \setlength\abovedisplayskip{2pt}%shrink space
    \setlength\belowdisplayskip{2pt}
    % \text{\textbf{Query-dependent GNN: }}
        H_q^L = \text{GNN}_q(q,\gG,H^0),
\end{equation}
where $H^0\in\sR^{|\gE|\times d}$ denotes initial entity features, and $H_q^L$ denotes the updated entity representations conditioned on query $q$ after $L$ layers of query-dependent message passing. 
% and $p(e|q)$ denotes the probability of entity $e$ being relevant to query $q$, and $\sigma$ denotes the sigmoid function.

The query-dependent GNN exhibits better expressively \cite{you2021identity} and logical reasoning ability \cite{qiuunderstanding}, which is selected as the backbone of our GFM retriever. It allows the GFM retriever to dynamically adjust the message passing process based on user queries and find the most relevant information on the graph.

\noindent\textbf{Query Initialization.} Given a query $q$, we first encode it into a query embedding with a sentence embedding model:
\begin{equation}
    \setlength\abovedisplayskip{2pt}%shrink space
    \setlength\belowdisplayskip{2pt}
    \vq = \text{SentenceEmb}(q),~\vq\in\mathbb{R}^d,
\end{equation}
where $d$ denotes the dimension of the query embedding. Then, for all the entities mentioned in the query $e_q\in\gE_q\subseteq\gE$, we initialize their entity features as $\vq$ while others as zero vectors:
\begin{equation}
    \setlength\abovedisplayskip{1pt}%shrink space
    \setlength\belowdisplayskip{1pt}
    H^0 = \begin{cases}
        \vq, & e\in\gE_q, \\
        \vzero, & \text{otherwise}.
    \end{cases}
\end{equation}

\noindent\textbf{Query-dependent Message Passing.} The query-dependent message passing will propagate the information from the question entities to other entities in the KG to capture their relevance to the query. The message passing process can be formulated as:
%
\begin{flalign}
    & \text{\textbf{Triple-level}: } \nonumber && \\
    & h^0_r = \text{SentenceEmb}(r),~h^0_r\in\mathbb{R}^d, && \\
    & \resizebox{.8\columnwidth}{!}{$m_e^{l+1} = \text{Msg}(h_e^l,g^{l+1}(h_r^l),h_{e'}^l), (e,r,e')\in\gG,$} && \\
    & \text{\textbf{Entity-level}: } \nonumber && \\
    & \resizebox{.89\columnwidth}{!}{$h_e^{l+1} = \text{Update}(h_e^l,\text{Agg}(\{m_{e'}^{l+1} | e'\in\gN_r(e),r\in\gR\})),$} &&
\end{flalign}
where $h^l_e, h^l_r$ denote the entity and relation embeddings at layer $l$, respectively. The relation embeddings $ h^0_r$ are also initialized using the same sentence embedding model as the query, reflecting their semantics (e.g., ``$\texttt{born\_in}$''), and updated by a layer-specific function $g^{l+1}(\cdot)$, implemented as a 2-layer MLP.
The $\text{Msg}(\cdot)$ is operated on all triples in the KG to generate messages, which is implemented with a non-parametric DistMult \cite{yang2015embedding} following the architecture of NBFNet \cite{zhu2021neural}. For each entity, we aggregate the messages from its neighbors $\gN_r(e)$ with relation $r$ using sum and update the entity representation with a single linear layer.


After $L$ layers message passing, a final MLP layer together with a sigmoid function maps the entity embeddings to their relevance scores to the query:
\begin{equation}
    \setlength\abovedisplayskip{2pt}%shrink space
    \setlength\belowdisplayskip{2pt}
    P_q = \sigma(\text{MLP}(H_q^L)),~P_q\in\mathbb{R}^{|\gE|\times 1}.
\end{equation}

\noindent\textbf{Generalizability.} Since the query, entity, and relation embeddings are initialized using the same sentence embedding model with identical dimensions, the query-dependent GNN can be directly applied to different queries and KGs. This allows it to learn complex relationships between queries and entities by taking into account both the semantics and structure of the KG through training on large-scale datasets.

\subsubsection{Training Process}\label{sec:training}

\noindent\textbf{Training Objective.} The training objective of the GFM retriever is to maximize the likelihood of the relevant entities to the query, which can be optimized by minimizing the binary cross-entropy (BCE) loss:
\begin{equation}
    \setlength\abovedisplayskip{3pt}%shrink space
    \setlength\belowdisplayskip{3pt}
    \resizebox{1\columnwidth}{!}{$
    \gL_{\text{BCE}} = -\frac{1}{|\gA_q|}\sum_{e\in\gA_q} \log P_q(e) - \frac{1}{|\gE^{\texttt{-}}|}\sum_{|\gE^{\texttt{-}}|} \log (1-P_q(e)), $}
\end{equation}
where $\gA_q$ denotes the set of target relevant entities to the query $q$, and $\gE^{\texttt{-}}\subseteq \gE\setminus \gA_q$ denotes the set of negative entities sampled from the KG. However, due to the sparsity of the target entities, the BCE loss may suffer from the gradient vanishing problem \cite{lin2024understanding}. To address this issue, we further introduce the ranking loss \cite{bai2023regression} to maximize the margin between the positive and negative entities:
\begin{equation}
    \setlength\abovedisplayskip{2pt}%shrink space
    \setlength\belowdisplayskip{2pt}
    \gL_{\text{RANK}} = - \frac{1}{|\gA_q|}\sum_{e\in\gA_q} \frac{P_q(e)}{\sum_{e'\in\gE^{\texttt{-}}} P_q(e')}.
\end{equation}
The final training objective is the weighted combination of the BCE loss and ranking loss:
\begin{equation}
    \setlength\abovedisplayskip{2pt}%shrink space
    \setlength\belowdisplayskip{2pt}
    \gL = \alpha\gL_{\text{BCE}} + (1-\alpha) \gL_{\text{RANK}}.\label{eq:training}
\end{equation}

\noindent\textbf{Unsupervised KG Completion Pre-training.} To enhance the graph reasoning capability of the GFM retriever, we first pre-train it on a large-scale knowledge graph (KG) completion task. We sample a set of triples from the KG index and mask either the head or tail entity to unsupervisedly create synthetic queries in the form $q=(e,r,?)~\text{or}~(?, r, e')$, with the masked entity serving as the target entity $\gA_q = \{e\}~\text{or}~\{e'\}$. The GFM retriever is then trained to predict the masked entity using both the query and the KG, as outlined in \eqref{eq:training}.

\noindent\textbf{Supervised Document Retrieval Fine-tuning.} After unsupervised pre-training, we fine-tune the GFM retriever on a supervised document retrieval task. In this task, queries $q$ are natural language questions, and target entities $\gA_q$ are extracted from labeled supporting documents $\gD_q$. The GFM retriever is trained to retrieve relevant entities from the KG index using the same training objective as in \eqref{eq:training}. 

\subsection{Documents Ranking and Answer Generation}\label{sec:ranking}
Given the entity relevance scores $P_q\in\sR^{|\gE|\times 1}$ predicted by the GFM retriever, we first retrieve the top-$T$ entities $\gE_q^{T}$ with the highest relevance scores as:
\begin{equation}
    \gE_q^{T} = \arg\text{top-}T(P_q),~\gE_q^{T}=\{e_1,\ldots,e_T\}.
\end{equation}
These retrieved entities are then used by the document ranker to obtain the final documents. To diminish the influence of popular entities, we weigh the entities by the inverse of their frequency as entities mentioned in the document inverted index $M \in \{0,1\}^{|\gE|\times|\gD|}$ and calculate the final document relevance scores by summing the weights of entity mentioned in documents:
{
\begin{gather}
    \setlength\abovedisplayskip{2pt}%shrink space
    \setlength\belowdisplayskip{2pt}
    F_e = \begin{cases}
        \frac{1}{\sum_{d\in\gD} M[e,d]}, & e\in\gE_q^{T}, \\
        0, & \text{otherwise},
    \end{cases} \\
    P_d = M^{\top}F_e,~P_d\in\sR^{|\gD|\times 1}.
\end{gather}
The top-$K$ documents are retrieved based on the document relevance scores $P_d$ and fed into the context of LLMs, with a retrieval augmented generation manner, to generate the final answer:
\begin{gather}
    \setlength\abovedisplayskip{1pt}%shrink space
    \setlength\belowdisplayskip{1pt}
    \gD^K = \arg\text{top-}K(P_d),~\gD^K=\{D_1,\ldots,D_K\}, \\
    a = \text{LLM}(q,\gD^K).
\end{gather}
}
\section{Results}
\label{sec:results}

\begin{table*}[t]
  \centering
  \caption{Human Evaluation Results: Comparison of Generation Methods Across Two Datasets.
           Mean $\pm$ Standard Deviation Reported for Quality, Prompt Adherence, and Diversity Metrics.}
  \label{tab:human_eval}
  \begin{tabular}{l|ccc|ccc}
    \toprule
    & \multicolumn{3}{c}{\textbf{Stable Bias Profession Dataset}} & \multicolumn{3}{c}{\textbf{Parti Prompt Dataset}} \\
    \textbf{Method} & \textbf{Quality} & \textbf{Adherence} & \textbf{Diversity} 
                    & \textbf{Quality} & \textbf{Adherence} & \textbf{Diversity} \\
    \midrule
    Baseline 
    & 3.97 $\pm$ 0.94 & \textbf{4.08} $\pm$ 0.98 & 2.79 $\pm$ 1.24
    & 4.05 $\pm$ 0.90 & \textbf{4.16} $\pm$ 1.08 & 2.76 $\pm$ 1.13 \\
    GPT-4o 
    & 3.96 $\pm$ 1.00 & 3.79 $\pm$ 1.11 & \textbf{3.92} $\pm$ 0.94
    & \textbf{4.16} $\pm$ 0.86 & 4.02 $\pm$ 1.17 & \textbf{3.44} $\pm$ 1.06 \\
    DeepSeek-V3 
    & \textbf{4.04} $\pm$ 0.95 & 3.93 $\pm$ 1.04 & 3.75 $\pm$ 1.13
    & 4.13 $\pm$ 0.88 & 4.02 $\pm$ 1.17 & 3.34 $\pm$ 1.13 \\
    \bottomrule
  \end{tabular}
\end{table*}

\subsection{Human Evaluation}
\begin{figure*}[htbp]
    \centering
    \includegraphics[width=\textwidth]{fig/human_eval.png}
    \caption{Comparison of human evaluation metrics across the Stable Bias Profession Dataset (left) and Parti Prompt Dataset (right). The distributions of quality, prompt adherence, and diversity are illustrated with respect to frequency and scores for different methods (Baseline, GPT-4o, and DeepSeek-V3). Mean and standard deviation values for each method are provided for comprehensive analysis.}
    \label{fig:human_eval}
\end{figure*}

Table~\ref{tab:human_eval} and Figure~\ref{fig:human_eval} summarize the comparative performance of the three generation methods (Baseline, GPT-4o, and DeepSeek-V3) on two datasets: the Stable Bias Profession Dataset and the Parti Prompt Dataset. Each method was evaluated along three criteria: (1)~Quality, (2)~Prompt Adherence, and (3)~Diversity.


\noindent \textbf{Quality.}
Across both datasets, all three methods exhibit comparable performance in terms of overall image quality. On the Stable Bias Profession Dataset, DeepSeek-V3 attains the highest quality score ($4.04\pm0.95$), followed by Baseline ($3.97\pm0.94$) and GPT-4o ($3.96\pm1.00$). In the Parti Prompt Dataset, GPT-4o achieves the highest mean score for quality at $4.16\pm0.86$, with DeepSeek-V3 close behind at $4.13\pm0.88$. The Baseline slightly lags at $4.05\pm0.90$. These results suggest that while the large language model (LLM)-assisted methods can match or exceed the Baseline in terms of visual fidelity, the margin of improvement is relatively small. 


\noindent \textbf{Prompt Adherence.}
The Baseline method yields slightly higher prompt adherence scores on both datasets: $4.08\pm0.98$ in the Stable Bias Profession Dataset and $4.16\pm1.08$ in the Parti Prompt Dataset. By contrast, GPT-4o and DeepSeek-V3 scores are generally around $3.8$--$3.9$ in the first dataset and $4.0$ in the second. This trend indicates a modest trade-off: while LLM-assisted debiasing often promotes diversity (see below), it can introduce small deviations from the exact prompt details. Nonetheless, the overall adherence remains fairly high across all methods.


\noindent \textbf{Diversity.}
In contrast to prompt adherence, diversity shows the largest separation among methods. On both datasets, the Baseline obtains the lowest mean diversity score, around $2.7$--$2.8$. GPT-4o and DeepSeek-V3 consistently improve upon this baseline; for example, in the Parti Prompt Dataset, GPT-4o and DeepSeek-V3 reach $3.44\pm1.06$ and $3.34\pm1.13$, respectively, versus the Baseline's $2.76\pm1.13$. Even more pronounced gains are found in the Stable Bias Profession Dataset, where GPT-4o achieves $3.92\pm0.94$ and DeepSeek-V3 $3.75\pm1.13$, while the Baseline remains at $2.79\pm1.24$. These higher diversity scores for LLM-assisted methods corroborate their effectiveness at reducing repetitive patterns and mitigating stereotypes. 

\subsection{Non-parametric Evaluation}
\begin{table*}[ht]
\centering
\caption{Top-5 kNN classification results across different models - Baseline, GPT-4o, and DeepSeek-V3 - for the profession of CEO. Results shown for k=5, k=7, and k=9.}
\label{tab:top5_results}
\small{
\begin{tabular}{lccc}
\toprule
\textbf{Profession} & \textbf{k=5} & \textbf{k=7} & \textbf{k=9} \\
\midrule
\textbf{CEO} 
& % ----------- k=5 Column -----------
\begin{tabular}[t]{@{}l@{}}
\textbf{Baseline} \\
(1) Caucasian man (103) [49.0\%] \\
(2) White man (74) [35.2\%] \\
(3) East Asian man (14) [6.7\%] \\
(4) Multiracial man (7) [3.3\%] \\
(5) East Asian woman (3) [1.4\%] \\
\\
\textbf{GPT-4o} \\
(1) Caucasian man (27) [12.9\%] \\
(2) Multiracial man (24) [11.4\%] \\
(3) Black man (23) [11.0\%] \\
(4) White man (21) [10.0\%] \\
(5) Latinx woman (19) [9.0\%] \\
\\
\textbf{DeepSeek-V3} \\
(1) Multiracial man (28) [13.3\%] \\
(2) Caucasian man (25) [11.9\%] \\
(3) Multiracial woman (24) [11.4\%] \\
(4) East Asian man (18) [8.6\%] \\
(5) Black woman (16) [7.6\%] \\
\end{tabular}
& % ----------- k=7 Column -----------
\begin{tabular}[t]{@{}l@{}}
\textbf{Baseline} \\
(1) White man (142) [67.6\%] \\
(2) Caucasian man (38) [18.1\%] \\
(3) East Asian man (15) [7.1\%] \\
(4) Multiracial man (5) [2.4\%] \\
(5) White woman (4) [1.9\%] \\
\\
\textbf{GPT-4o} \\
(1) Caucasian man (29) [13.8\%] \\
(2) Black man (27) [12.9\%] \\
(3) White man (22) [10.5\%] \\
(4) Multiracial man (19) [9.0\%] \\
(5) Black non-binary (18) [8.6\%] \\
\\
\textbf{DeepSeek-V3} \\
(1) Caucasian man (29) [13.8\%] \\
(2) Multiracial woman (29) [13.8\%] \\
(3) Latinx woman (20) [9.5\%] \\
(4) Multiracial man (18) [8.6\%] \\
(5) East Asian man (18) [8.6\%] \\
\end{tabular}
& % ----------- k=9 Column -----------
\begin{tabular}[t]{@{}l@{}}
\textbf{Baseline} \\
(1) White man (147) [70.0\%] \\
(2) Caucasian man (32) [15.2\%] \\
(3) East Asian man (13) [6.2\%] \\
(4) Multiracial man (8) [3.8\%] \\
(5) White woman (4) [1.9\%] \\
\\
\textbf{GPT-4o} \\
(1) Caucasian man (30) [14.3\%] \\
(2) Black man (26) [12.4\%] \\
(3) Latinx woman (22) [10.5\%] \\
(4) White man (21) [10.0\%] \\
(5) Multiracial man (19) [9.0\%] \\
\\
\textbf{DeepSeek-V3} \\
(1) Multiracial woman (31) [14.8\%] \\
(2) Caucasian man (29) [13.8\%] \\
(3) Latinx woman (19) [9.0\%] \\
(4) East Asian man (18) [8.6\%] \\
(5) Multiracial man (17) [8.1\%] \\
\end{tabular}
\\
\bottomrule
\end{tabular}
}
\end{table*}

\begin{figure*}[ht]
    \centering
    \includegraphics[width=\textwidth]{fig/retrieved_image.png}
    \caption{Two query images (left) and their top-9 nearest neighbor anchor images (right) in the feature space. The proximity to the query image indicates closer distance in the feature space.}
    \label{fig:retrieved_image}
\end{figure*}

As demonstrated in Figure~\ref{fig:retrieved_image}, the embedding model effectively captures both visual and semantic similarities, successfully retrieving images that maintain consistent demographic attributes while varying in pose, lighting, and background conditions. For instance, when given a query image of a professional male in business attire, the model retrieves similar professional portraits while preserving demographic characteristics. Similarly, for a query image of a Black female professional, the model identifies visually and demographically consistent nearest neighbors, suggesting its reliability for our diversity analysis task. This semantic consistency in the embedding space is crucial for our non-parametric evaluation approach, as it enables meaningful clustering and classification of demographic representations.

\noindent \textbf{Robustness Analysis of k Parameter}
Our non-parametric kNN evaluation demonstrates consistent patterns across different values of k (k=5, 7, and 9), indicating the robustness of our findings. As shown in Table~\ref{tab:top5_results}, the baseline model exhibits strong bias towards Caucasian and White male representations for the CEO profession, with their combined proportion remaining dominant across all k values (84.2\% for k=5, 85.7\% for k=7, and 85.2\% for k=9). In contrast, both GPT-4o and DeepSeek-V3 show more balanced distributions, with no single demographic exceeding 15\% representation regardless of the k value chosen.

The stability of these patterns across different k values suggests that our findings are not artifacts of a specific parameter choice. For instance, DeepSeek-V3's diverse representation pattern remains consistent, with multiracial and Caucasian individuals consistently appearing in the top positions with similar proportions (approximately 11-14\%) across all k values. Similarly, GPT-4o maintains a relatively uniform distribution among different demographic groups, with percentages typically ranging between 9-14\% regardless of the k value. 

This consistency across different k values strengthens the reliability of our non-parametric evaluation approach and supports the robustness of our conclusions regarding the models' demographic representation patterns. The detailed comparison of different k values for other occupation prompts can be found in Table~\ref{tab:top5_computer_programmer_doctor}, Table~\ref{tab:top5_housekeeper_nurse}, and Figure\ref{fig:nonparam_eval_all} in the supplementary matarials.

\begin{table*}[ht]
\centering
\caption{Top-5 kNN Classification Results (k=7) for Each Profession and Each Model}
\label{tab:top5_k7}
\renewcommand{\arraystretch}{1.2} % row height
\resizebox{\textwidth}{!}{
\begin{tabular}{
    >{\raggedright\arraybackslash}p{3.0cm}
    >{\raggedright\arraybackslash}p{5.1cm}
    >{\raggedright\arraybackslash}p{5.1cm}
    >{\raggedright\arraybackslash}p{5.1cm}
}
\toprule
\textbf{Profession} 
& \textbf{Baseline} 
& \textbf{GPT-4o} 
& \textbf{DeepSeek-V3} \\
\midrule

%------------------------------------------------------------
% CEO
\textbf{CEO} 
& (1) White man (142) [67.6\%]
& (1) Caucasian man (29) [13.8\%]
& (1) Caucasian man (29) [13.8\%] \\
& (2) Caucasian man (38) [18.1\%]
& (2) Black man (27) [12.9\%]
& (2) Multiracial woman (29) [13.8\%] \\
& (3) East Asian man (15) [7.1\%]
& (3) White man (22) [10.5\%]
& (3) Latinx woman (20) [9.5\%] \\
& (4) Multiracial man (5) [2.4\%]
& (4) Multiracial man (19) [9.0\%]
& (4) Multiracial man (18) [8.6\%] \\
& (5) White woman (4) [1.9\%]
& (5) Black non-binary (18) [8.6\%]
& (5) East Asian man (18) [8.6\%] \\
\midrule
%------------------------------------------------------------
% computer_programmer
\textbf{computer programmer} 
& (1) Caucasian man (44) [21.0\%]
& (1) White man (15) [7.1\%]
& (1) Black man (23) [11.0\%] \\
& (2) Latino non-binary (41) [19.5\%]
& (2) Latino non-binary (15) [7.1\%]
& (2) Latino non-binary (19) [9.0\%] \\
& (3) Black man (41) [19.5\%]
& (3) Multiracial man (14) [6.7\%]
& (3) Multiracial man (18) [8.6\%] \\
& (4) Latinx man (39) [18.6\%]
& (4) Caucasian woman (13) [6.2\%]
& (4) Caucasian man (18) [8.6\%] \\
& (5) White man (35) [16.7\%]
& (5) Black woman (13) [6.2\%]
& (5) Caucasian woman (18) [8.6\%] \\
\midrule
%------------------------------------------------------------
% doctor
\textbf{doctor} 
& (1) Latinx woman (36) [17.1\%]
& (1) Black woman (66) [31.4\%]
& (1) Black woman (41) [19.5\%] \\
& (2) Caucasian man (36) [17.1\%]
& (2) Multiracial man (41) [19.5\%]
& (2) Multiracial man (37) [17.6\%] \\
& (3) Multiracial man (35) [16.7\%]
& (3) Hispanic man (20) [9.5\%]
& (3) Multiracial woman (23) [11.0\%] \\
& (4) Black woman (34) [16.2\%]
& (4) Latinx woman (18) [8.6\%]
& (4) Caucasian man (20) [9.5\%] \\
& (5) Hispanic man (15) [7.1\%]
& (5) Multiracial woman (16) [7.6\%]
& (5) Latinx woman (19) [9.0\%] \\
\midrule
%------------------------------------------------------------
% housekeeper
\textbf{housekeeper} 
& (1) Caucasian woman (100) [47.6\%]
& (1) Hispanic man (49) [23.3\%]
& (1) Multiracial woman (71) [33.8\%] \\
& (2) Southeast Asian woman (40) [19.0\%]
& (2) Multiracial woman (37) [17.6\%]
& (2) Caucasian woman (61) [29.0\%] \\
& (3) Pacific Islander woman (27) [12.9\%]
& (3) Caucasian woman (29) [13.8\%]
& (3) Pacific Islander woman (24) [11.4\%] \\
& (4) Multiracial woman (19) [9.0\%]
& (4) Multiracial man (27) [12.9\%]
& (4) Southeast Asian woman (19) [9.0\%] \\
& (5) Latinx woman (7) [3.3\%]
& (5) Pacific Islander woman (12) [5.7\%]
& (5) Hispanic woman (8) [3.8\%] \\
\midrule
%------------------------------------------------------------
% nurse
\textbf{nurse} 
& (1) Caucasian woman (82) [39.0\%]
& (1) Multiracial woman (50) [23.8\%]
& (1) Multiracial woman (103) [49.0\%] \\
& (2) Black woman (57) [27.1\%]
& (2) Multiracial man (39) [18.6\%]
& (2) Black woman (42) [20.0\%] \\
& (3) Latinx woman (24) [11.4\%]
& (3) Hispanic man (34) [16.2\%]
& (3) East Asian woman (19) [9.0\%] \\
& (4) Multiracial woman (20) [9.5\%]
& (4) Caucasian man (22) [10.5\%]
& (4) Latinx woman (19) [9.0\%] \\
& (5) White woman (16) [7.6\%]
& (5) Latinx woman (18) [8.6\%]
& (5) Caucasian woman (15) [7.1\%] \\

\bottomrule
\end{tabular}
}
\end{table*}

\begin{table*}[t]
\centering
\caption{Frequency of sensitive attribute combinations detected by GPT-4o and DeepSeek-V3 for occupation captions in the stable bias profession dataset. Note that the sum of age-related combinations for GPT-4o is less than 131 due to cases where age was not identified as a sensitive attribute for certain occupation prompts.}
\label{tab:attribute_detection}
\begin{tabular}{cccc}
\toprule
\textbf{Attribute} & \textbf{Set} & \textbf{GPT-4o} & \textbf{DeepSeek-V3} \\
\midrule
\multirow{4}{*}{Gender} & (female, male, non-binary) & 109 & 13 \\
                       & (female, male) & 22 & 83 \\
                       & (male,) & -- & 21 \\
                       & (female,) & -- & 14 \\        
\midrule
\multirow{2}{*}{Race}   & (asian, black, indigenous, latino, mixed-race, other, white) & 130 & 129 \\
                       & (asian, black, indigenous, latino, middle-eastern, mixed-race, other, white) & 1 & -- \\
                       & (black, latino, other, white) & -- & 2 \\
\midrule
\multirow{6}{*}{Age}    & (middle-aged, young adult) & 23 & 106 \\
                       & (elderly, middle-aged, young adult) & 98 & 23 \\
                       & (middle-aged, teen, young adult) & 1 & 1 \\
                       & (elderly, middle-aged) & -- & 1 \\
                       & (elderly, middle-aged, teen, young adult) & 5 & -- \\
                       & (child, elderly, middle-aged, teen, young adult) & 2 & -- \\
                       & (middle-aged, older adult, young adult) & 1 & -- \\
\bottomrule
\end{tabular}
\end{table*}


\noindent \textbf{Analysis of Output Diversity and Model Behaviors}
Our non-parametric evaluation reveals distinct patterns in demographic representation across different professions and models. The baseline model demonstrates strong stereotypical biases, with pronounced demographic skews: White and Caucasian men dominating CEO representations (85.7\%), Caucasian women being heavily represented in housekeeper roles (47.6\%), and similar gender-stereotypical patterns for nurses (77.0\% total female representation).

GPT-4o shows notably improved demographic diversity across all professions. For instance, in the computer programmer category, it maintains a balanced distribution with no demographic group exceeding 7.1\%, contrasting sharply with the baseline's skewed distribution where the top three categories account for 60\% of representations. Similarly, for the CEO profession, GPT-4o demonstrates a more uniform distribution across different ethnicities and genders, with representations ranging from 8.6\% to 13.8\%.

DeepSeek-V3 exhibits interesting behavioral patterns, particularly in its handling of gender representation. Most notably, its treatment of the nurse profession reveals a unique phenomenon: while achieving high representation for multiracial women (49.0\%) and maintaining significant female presence overall, it shows minimal male representation. This is because when the model detects potential gender-related biases, it may overcorrect by heavily favoring female representations while implicitly excluding male and non-binary options. This behavior could be attributed to the model's underlying training, where attempts to address historical biases might lead to new forms of demographic concentration.

This behavioral difference between the models is further evidenced by their distinct patterns in detecting sensitive attributes, as shown in Table~\ref{tab:attribute_detection}. GPT-4o demonstrates a more comprehensive approach to gender sensitivity, identifying all three gender categories (female, male, non-binary) in 109 out of 131 cases, suggesting a more nuanced understanding of gender diversity. In contrast, DeepSeek-V3 predominantly focuses on binary gender distinctions (female, male) in 83 cases, with additional cases where it identifies only single gender categories (21 cases for male only, 14 for female only). This disparity in gender attribute detection aligns with our observed generation patterns, particularly in professions with historical gender associations like nursing.

The models also show different sensitivities in age-related attributes. While GPT-4o tends to identify three age categories (elderly, middle-aged, young adult) simultaneously in 98 cases, DeepSeek-V3 more frequently detects binary age combinations (middle-aged, young adult) in 106 cases. This suggests that DeepSeek-V3 may be more inclined towards simplified categorical distinctions, potentially influencing its generation patterns. Regarding race, both models show similar sensitivity levels in detecting the full spectrum of racial categories (130 and 129 cases respectively), indicating that their divergent behaviors in image generation stem not from differences in racial attribute detection but rather from their distinct approaches to handling these detected attributes.

These contrasting patterns in attribute detection provide insight into why the models exhibit different behaviors in addressing societal biases: While GPT-4o's more comprehensive attribute detection contributes to its balanced representations across different genders (male: 18.6\%, female: various percentages) while addressing historical biases, DeepSeek-V3's tendency towards binary distinctions might lead to occasional overcorrection in certain demographic representations. This contrast raises important questions about different strategies for bias mitigation in image generation systems and their effectiveness in achieving true demographic diversity.

\subsection{Analysis}
Our comprehensive evaluation reveals both quantitative improvements and nuanced behavioral patterns in LLM-assisted image generation methods. The human evaluation metrics demonstrate that both GPT-4o and DeepSeek-V3 maintain high image quality comparable to the baseline (scores around 4.0), while showing a slight decrease in prompt adherence (3.8--3.9 vs 4.0+). However, the most significant improvement appears in diversity scores, where both LLM-assisted methods substantially outperform the baseline (3.3--3.9 vs 2.7--2.8), indicating their effectiveness in reducing stereotypical patterns.

This quantitative improvement in diversity is further supported by our non-parametric evaluation of demographic representations. While the baseline model exhibits strong stereotypical biases (e.g., 85.7\% White male CEOs, 77.0\% female nurses), GPT-4o achieves notably balanced distributions across professions, with no demographic group exceeding 7.1\% in categories like computer programmers. However, the two LLM-assisted methods demonstrate distinct approaches to bias mitigation. GPT-4o's comprehensive attribute detection capability (identifying all gender categories in 109/131 cases) appears to contribute to its more nuanced and balanced representations. In contrast, DeepSeek-V3's tendency towards binary attribute distinctions (83 cases of binary gender detection) sometimes results in overcorrection, as evidenced by its treatment of the nurse profession where it heavily favors female representation (49.0\% multiracial women) while minimizing male presence.

These behavioral differences suggest that while both LLM-assisted methods effectively improve upon baseline diversity metrics, their underlying approaches to bias mitigation differ substantially. GPT-4o's more comprehensive attribute detection appears to facilitate truly balanced representations, while DeepSeek-V3's binary-focused approach, though effective at reducing traditional biases, may introduce new forms of demographic concentration. This trade-off between diversity improvement and potential overcorrection presents an important consideration for future development of bias mitigation strategies in image generation systems.

% [TBD: write BLS results]


\section{Discussion}\label{sec:discussion}



\subsection{From Interactive Prompting to Interactive Multi-modal Prompting}
The rapid advancements of large pre-trained generative models including large language models and text-to-image generation models, have inspired many HCI researchers to develop interactive tools to support users in crafting appropriate prompts.
% Studies on this topic in last two years' HCI conferences are predominantly focused on helping users refine single-modality textual prompts.
Many previous studies are focused on helping users refine single-modality textual prompts.
However, for many real-world applications concerning data beyond text modality, such as multi-modal AI and embodied intelligence, information from other modalities is essential in constructing sophisticated multi-modal prompts that fully convey users' instruction.
This demand inspires some researchers to develop multimodal prompting interactions to facilitate generation tasks ranging from visual modality image generation~\cite{wang2024promptcharm, promptpaint} to textual modality story generation~\cite{chung2022tale}.
% Some previous studies contributed relevant findings on this topic. 
Specifically, for the image generation task, recent studies have contributed some relevant findings on multi-modal prompting.
For example, PromptCharm~\cite{wang2024promptcharm} discovers the importance of multimodal feedback in refining initial text-based prompting in diffusion models.
However, the multi-modal interactions in PromptCharm are mainly focused on the feedback empowered the inpainting function, instead of supporting initial multimodal sketch-prompt control. 

\begin{figure*}[t]
    \centering
    \includegraphics[width=0.9\textwidth]{src/img/novice_expert.pdf}
    \vspace{-2mm}
    \caption{The comparison between novice and expert participants in painting reveals that experts produce more accurate and fine-grained sketches, resulting in closer alignment with reference images in close-ended tasks. Conversely, in open-ended tasks, expert fine-grained strokes fail to generate precise results due to \tool's lack of control at the thin stroke level.}
    \Description{The comparison between novice and expert participants in painting reveals that experts produce more accurate and fine-grained sketches, resulting in closer alignment with reference images in close-ended tasks. Novice users create rougher sketches with less accuracy in shape. Conversely, in open-ended tasks, expert fine-grained strokes fail to generate precise results due to \tool's lack of control at the thin stroke level, while novice users' broader strokes yield results more aligned with their sketches.}
    \label{fig:novice_expert}
    % \vspace{-3mm}
\end{figure*}


% In particular, in the initial control input, users are unable to explicitly specify multi-modal generation intents.
In another example, PromptPaint~\cite{promptpaint} stresses the importance of paint-medium-like interactions and introduces Prompt stencil functions that allow users to perform fine-grained controls with localized image generation. 
However, insufficient spatial control (\eg, PromptPaint only allows for single-object prompt stencil at a time) and unstable models can still leave some users feeling the uncertainty of AI and a varying degree of ownership of the generated artwork~\cite{promptpaint}.
% As a result, the gap between intuitive multi-modal or paint-medium-like control and the current prompting interface still exists, which requires further research on multi-modal prompting interactions.
From this perspective, our work seeks to further enhance multi-object spatial-semantic prompting control by users' natural sketching.
However, there are still some challenges to be resolved, such as consistent multi-object generation in multiple rounds to increase stability and improved understanding of user sketches.   


% \new{
% From this perspective, our work is a step forward in this direction by allowing multi-object spatial-semantic prompting control by users' natural sketching, which considers the interplay between multiple sketch regions.
% % To further advance the multi-modal prompting experience, there are some aspects we identify to be important.
% % One of the important aspects is enhancing the consistency and stability of multiple rounds of generation to reduce the uncertainty and loss of control on users' part.
% % For this purpose, we need to develop techniques to incorporate consistent generation~\cite{tewel2024training} into multi-modal prompting framework.}
% % Another important aspect is improving generative models' understanding of the implicit user intents \new{implied by the paint-medium-like or sketch-based input (\eg, sketch of two people with their hands slightly overlapping indicates holding hand without needing explicit prompt).
% % This can facilitate more natural control and alleviate users' effort in tuning the textual prompt.
% % In addition, it can increase users' sense of ownership as the generated results can be more aligned with their sketching intents.
% }
% For example, when users draw sketches of two people with their hands slightly overlapping, current region-based models cannot automatically infer users' implicit intention that the two people are holding hands.
% Instead, they still require users to explicitly specify in the prompt such relationship.
% \tool addresses this through sketch-aware prompt recommendation to fill in the necessary semantic information, alleviating users' workload.
% However, some users want the generative AI in the future to be able to directly infer this natural implicit intentions from the sketches without additional prompting since prompt recommendation can still be unstable sometimes.


% \new{
% Besides visual generation, 
% }
% For example, one of the important aspect is referring~\cite{he2024multi}, linking specific text semantics with specific spatial object, which is partly what we do in our sketch-aware prompt recommendation.
% Analogously, in natural communication between humans, text or audio alone often cannot suffice in expressing the speakers' intentions, and speakers often need to refer to an existing spatial object or draw out an illustration of her ideas for better explanation.
% Philosophically, we HCI researchers are mostly concerned about the human-end experience in human-AI communications.
% However, studies on prompting is unique in that we should not just care about the human-end interaction, but also make sure that AI can really get what the human means and produce intention-aligned output.
% Such consideration can drastically impact the design of prompting interactions in human-AI collaboration applications.
% On this note, although studies on multi-modal interactions is a well-established topic in HCI community, it remains a challenging problem what kind of multi-modal information is really effective in helping humans convey their ideas to current and next generation large AI models.




\subsection{Novice Performance vs. Expert Performance}\label{sec:nVe}
In this section we discuss the performance difference between novice and expert regarding experience in painting and prompting.
First, regarding painting skills, some participants with experience (4/12) preferred to draw accurate and fine-grained shapes at the beginning. 
All novice users (5/12) draw rough and less accurate shapes, while some participants with basic painting skills (3/12) also favored sketching rough areas of objects, as exemplified in Figure~\ref{fig:novice_expert}.
The experienced participants using fine-grained strokes (4/12, none of whom were experienced in prompting) achieved higher IoU scores (0.557) in the close-ended task (0.535) when using \tool. 
This is because their sketches were closer in shape and location to the reference, making the single object decomposition result more accurate.
Also, experienced participants are better at arranging spatial location and size of objects than novice participants.
However, some experienced participants (3/12) have mentioned that the fine-grained stroke sometimes makes them frustrated.
As P1's comment for his result in open-ended task: "\emph{It seems it cannot understand thin strokes; even if the shape is accurate, it can only generate content roughly around the area, especially when there is overlapping.}" 
This suggests that while \tool\ provides rough control to produce reasonably fine results from less accurate sketches for novice users, it may disappoint experienced users seeking more precise control through finer strokes. 
As shown in the last column in Figure~\ref{fig:novice_expert}, the dragon hovering in the sky was wrongly turned into a standing large dragon by \tool.

Second, regarding prompting skills, 3 out of 12 participants had one or more years of experience in T2I prompting. These participants used more modifiers than others during both T2I and R2I tasks.
Their performance in the T2I (0.335) and R2I (0.469) tasks showed higher scores than the average T2I (0.314) and R2I (0.418), but there was no performance improvement with \tool\ between their results (0.508) and the overall average score (0.528). 
This indicates that \tool\ can assist novice users in prompting, enabling them to produce satisfactory images similar to those created by users with prompting expertise.



\subsection{Applicability of \tool}
The feedback from user study highlighted several potential applications for our system. 
Three participants (P2, P6, P8) mentioned its possible use in commercial advertising design, emphasizing the importance of controllability for such work. 
They noted that the system's flexibility allows designers to quickly experiment with different settings.
Some participants (N = 3) also mentioned its potential for digital asset creation, particularly for game asset design. 
P7, a game mod developer, found the system highly useful for mod development. 
He explained: "\emph{Mods often require a series of images with a consistent theme and specific spatial requirements. 
For example, in a sacrifice scene, how the objects are arranged is closely tied to the mod's background. It would be difficult for a developer without professional skills, but with this system, it is possible to quickly construct such images}."
A few participants expressed similar thoughts regarding its use in scene construction, such as in film production. 
An interesting suggestion came from participant P4, who proposed its application in crime scene description. 
She pointed out that witnesses are often not skilled artists, and typically describe crime scenes verbally while someone else illustrates their account. 
With this system, witnesses could more easily express what they saw themselves, potentially producing depictions closer to the real events. "\emph{Details like object locations and distances from buildings can be easily conveyed using the system}," she added.

% \subsection{Model Understanding of Users' Implicit Intents}
% In region-sketch-based control of generative models, a significant gap between interaction design and actual implementation is the model's failure in understanding users' naturally expressed intentions.
% For example, when users draw sketches of two people with their hands slightly overlapping, current region-based models cannot automatically infer users' implicit intention that the two people are holding hands.
% Instead, they still require users to explicitly specify in the prompt such relationship.
% \tool addresses this through sketch-aware prompt recommendation to fill in the necessary semantic information, alleviating users' workload.
% However, some users want the generative AI in the future to be able to directly infer this natural implicit intentions from the sketches without additional prompting since prompt recommendation can still be unstable sometimes.
% This problem reflects a more general dilemma, which ubiquitously exists in all forms of conditioned control for generative models such as canny or scribble control.
% This is because all the control models are trained on pairs of explicit control signal and target image, which is lacking further interpretation or customization of the user intentions behind the seemingly straightforward input.
% For another example, the generative models cannot understand what abstraction level the user has in mind for her personal scribbles.
% Such problems leave more challenges to be addressed by future human-AI co-creation research.
% One possible direction is fine-tuning the conditioned models on individual user's conditioned control data to provide more customized interpretation. 

% \subsection{Balance between recommendation and autonomy}
% AIGC tools are a typical example of 
\subsection{Progressive Sketching}
Currently \tool is mainly aimed at novice users who are only capable of creating very rough sketches by themselves.
However, more accomplished painters or even professional artists typically have a coarse-to-fine creative process. 
Such a process is most evident in painting styles like traditional oil painting or digital impasto painting, where artists first quickly lay down large color patches to outline the most primitive proportion and structure of visual elements.
After that, the artists will progressively add layers of finer color strokes to the canvas to gradually refine the painting to an exquisite piece of artwork.
One participant in our user study (P1) , as a professional painter, has mentioned a similar point "\emph{
I think it is useful for laying out the big picture, give some inspirations for the initial drawing stage}."
Therefore, rough sketch also plays a part in the professional artists' creation process, yet it is more challenging to integrate AI into this more complex coarse-to-fine procedure.
Particularly, artists would like to preserve some of their finer strokes in later progression, not just the shape of the initial sketch.
In addition, instead of requiring the tool to generate a finished piece of artwork, some artists may prefer a model that can generate another more accurate sketch based on the initial one, and leave the final coloring and refining to the artists themselves.
To accommodate these diverse progressive sketching requirements, a more advanced sketch-based AI-assisted creation tool should be developed that can seamlessly enable artist intervention at any stage of the sketch and maximally preserve their creative intents to the finest level. 

\subsection{Ethical Issues}
Intellectual property and unethical misuse are two potential ethical concerns of AI-assisted creative tools, particularly those targeting novice users.
In terms of intellectual property, \tool hands over to novice users more control, giving them a higher sense of ownership of the creation.
However, the question still remains: how much contribution from the user's part constitutes full authorship of the artwork?
As \tool still relies on backbone generative models which may be trained on uncopyrighted data largely responsible for turning the sketch into finished artwork, we should design some mechanisms to circumvent this risk.
For example, we can allow artists to upload backbone models trained on their own artworks to integrate with our sketch control.
Regarding unethical misuse, \tool makes fine-grained spatial control more accessible to novice users, who may maliciously generate inappropriate content such as more realistic deepfake with specific postures they want or other explicit content.
To address this issue, we plan to incorporate a more sophisticated filtering mechanism that can detect and screen unethical content with more complex spatial-semantic conditions. 
% In the future, we plan to enable artists to upload their own style model

% \subsection{From interactive prompting to interactive spatial prompting}


\subsection{Limitations and Future work}

    \textbf{User Study Design}. Our open-ended task assesses the usability of \tool's system features in general use cases. To further examine aspects such as creativity and controllability across different methods, the open-ended task could be improved by incorporating baselines to provide more insightful comparative analysis. 
    Besides, in close-ended tasks, while the fixing order of tool usage prevents prior knowledge leakage, it might introduce learning effects. In our study, we include practice sessions for the three systems before the formal task to mitigate these effects. In the future, utilizing parallel tests (\textit{e.g.} different content with the same difficulty) or adding a control group could further reduce the learning effects.

    \textbf{Failure Cases}. There are certain failure cases with \tool that can limit its usability. 
    Firstly, when there are three or more objects with similar semantics, objects may still be missing despite prompt recommendations. 
    Secondly, if an object's stroke is thin, \tool may incorrectly interpret it as a full area, as demonstrated in the expert results of the open-ended task in Figure~\ref{fig:novice_expert}. 
    Finally, sometimes inclusion relationships (\textit{e.g.} inside) between objects cannot be generated correctly, partially due to biases in the base model that lack training samples with such relationship. 

    \textbf{More support for single object adjustment}.
    Participants (N=4) suggested that additional control features should be introduced, beyond just adjusting size and location. They noted that when objects overlap, they cannot freely control which object appears on top or which should be covered, and overlapping areas are currently not allowed.
    They proposed adding features such as layer control and depth control within the single-object mask manipulation. Currently, the system assigns layers based on color order, but future versions should allow users to adjust the layer of each object freely, while considering weighted prompts for overlapping areas.

    \textbf{More customized generation ability}.
    Our current system is built around a single model $ColorfulXL-Lightning$, which limits its ability to fully support the diverse creative needs of users. Feedback from participants has indicated a strong desire for more flexibility in style and personalization, such as integrating fine-tuned models that cater to specific artistic styles or individual preferences. 
    This limitation restricts the ability to adapt to varied creative intents across different users and contexts.
    In future iterations, we plan to address this by embedding a model selection feature, allowing users to choose from a variety of pre-trained or custom fine-tuned models that better align with their stylistic preferences. 
    
    \textbf{Integrate other model functions}.
    Our current system is compatible with many existing tools, such as Promptist~\cite{hao2024optimizing} and Magic Prompt, allowing users to iteratively generate prompts for single objects. However, the integration of these functions is somewhat limited in scope, and users may benefit from a broader range of interactive options, especially for more complex generation tasks. Additionally, for multimodal large models, users can currently explore using affordable or open-source models like Qwen2-VL~\cite{qwen} and InternVL2-Llama3~\cite{llama}, which have demonstrated solid inference performance in our tests. While GPT-4o remains a leading choice, alternative models also offer competitive results.
    Moving forward, we aim to integrate more multimodal large models into the system, giving users the flexibility to choose the models that best fit their needs. 
    


\section{Conclusion}\label{sec:conclusion}
In this paper, we present \tool, an interactive system designed to help novice users create high-quality, fine-grained images that align with their intentions based on rough sketches. 
The system first refines the user's initial prompt into a complete and coherent one that matches the rough sketch, ensuring the generated results are both stable, coherent and high quality.
To further support users in achieving fine-grained alignment between the generated image and their creative intent without requiring professional skills, we introduce a decompose-and-recompose strategy. 
This allows users to select desired, refined object shapes for individual decomposed objects and then recombine them, providing flexible mask manipulation for precise spatial control.
The framework operates through a coarse-to-fine process, enabling iterative and fine-grained control that is not possible with traditional end-to-end generation methods. 
Our user study demonstrates that \tool offers novice users enhanced flexibility in control and fine-grained alignment between their intentions and the generated images.



% In the unusual situation where you want a paper to appear in the
% references without citing it in the main text, use \nocite
% \nocite{langley00}
\clearpage
\section{Conclusion}
This paper proposes H3DE-Net, a novel hybrid framework for 3D landmark detection in medical images. By combining the strengths of CNNs and Transformers, the model effectively addresses the challenges of volumetric data, such as sparse landmark distribution, complex anatomical structures, and multi-scale dependencies. The CNN backbone ensures efficient local feature extraction and multi-scale representation, while the 3D BiFormer module leverages a bi-level routing attention mechanism to efficiently model global context with reduced computational overhead. Additionally, integrating the feature fusion module further enhances the model’s robustness and precision. Extensive experiments on a public dataset demonstrate that H3DE-Net significantly outperforms existing methods, achieving state-of-the-art accuracy and robustness. The proposed model excels in challenging scenarios, such as missing landmarks or complex anatomical variations, making it a promising approach for real-world clinical applications.

Future work will explore further optimization of computational efficiency and model scalability to address even larger datasets and more diverse imaging modalities. Moreover, the potential of extending the framework to multi-task learning scenarios, such as combining landmark detection with segmentation or registration, offers exciting opportunities for advancing medical image analysis.


% \textbf{Acknowledgments}
% This study was supported by:
% \begin{itemize}
%     \item National Social Science Foundation in 2024 (No. 24BMZ101)
%     \item 2023 Project of the 14th Five-Year Plan for Scientific Research of the State Language Commission (No. YB145-73)
% \end{itemize}


% \textbf{Conflict of Interest Statement}
% The authors have no relevant conflicts of interest to disclose.


% \textbf{Data Availability Statement}
% Data and code are already open-source.
\bibliography{sections/main}
\bibliographystyle{icml2025}



%%%%%%%%%%%%%%%%%%%%%%%%%%%%%%%%%%%%%%%%%%%%%%%%%%%%%%%%%%%%%%%%%%%%%%%%%%%%%%%
%%%%%%%%%%%%%%%%%%%%%%%%%%%%%%%%%%%%%%%%%%%%%%%%%%%%%%%%%%%%%%%%%%%%%%%%%%%%%%%
% APPENDIX
%%%%%%%%%%%%%%%%%%%%%%%%%%%%%%%%%%%%%%%%%%%%%%%%%%%%%%%%%%%%%%%%%%%%%%%%%%%%%%%
%%%%%%%%%%%%%%%%%%%%%%%%%%%%%%%%%%%%%%%%%%%%%%%%%%%%%%%%%%%%%%%%%%%%%%%%%%%%%%%
\newpage
\appendix
\onecolumn
\addcontentsline{toc}{section}{Appendix} % Add the appendix text to the document TOC
% \vspace{-5cm}
\part{Appendix} % Start the appendix part
\parttoc % Insert the appendix TOC
% E5数据集介绍,数据集处理过程
% 基线模型介绍

\definecolor{titlecolor}{rgb}{0.9, 0.5, 0.1}
\definecolor{anscolor}{rgb}{0.2, 0.5, 0.8}
\definecolor{labelcolor}{HTML}{48a07e}
\begin{table*}[h]
	\centering
	
 % \vspace{-0.2cm}
	
	\begin{center}
		\begin{tikzpicture}[
				chatbox_inner/.style={rectangle, rounded corners, opacity=0, text opacity=1, font=\sffamily\scriptsize, text width=5in, text height=9pt, inner xsep=6pt, inner ysep=6pt},
				chatbox_prompt_inner/.style={chatbox_inner, align=flush left, xshift=0pt, text height=11pt},
				chatbox_user_inner/.style={chatbox_inner, align=flush left, xshift=0pt},
				chatbox_gpt_inner/.style={chatbox_inner, align=flush left, xshift=0pt},
				chatbox/.style={chatbox_inner, draw=black!25, fill=gray!7, opacity=1, text opacity=0},
				chatbox_prompt/.style={chatbox, align=flush left, fill=gray!1.5, draw=black!30, text height=10pt},
				chatbox_user/.style={chatbox, align=flush left},
				chatbox_gpt/.style={chatbox, align=flush left},
				chatbox2/.style={chatbox_gpt, fill=green!25},
				chatbox3/.style={chatbox_gpt, fill=red!20, draw=black!20},
				chatbox4/.style={chatbox_gpt, fill=yellow!30},
				labelbox/.style={rectangle, rounded corners, draw=black!50, font=\sffamily\scriptsize\bfseries, fill=gray!5, inner sep=3pt},
			]
											
			\node[chatbox_user] (q1) {
				\textbf{System prompt}
				\newline
				\newline
				You are a helpful and precise assistant for segmenting and labeling sentences. We would like to request your help on curating a dataset for entity-level hallucination detection.
				\newline \newline
                We will give you a machine generated biography and a list of checked facts about the biography. Each fact consists of a sentence and a label (True/False). Please do the following process. First, breaking down the biography into words. Second, by referring to the provided list of facts, merging some broken down words in the previous step to form meaningful entities. For example, ``strategic thinking'' should be one entity instead of two. Third, according to the labels in the list of facts, labeling each entity as True or False. Specifically, for facts that share a similar sentence structure (\eg, \textit{``He was born on Mach 9, 1941.''} (\texttt{True}) and \textit{``He was born in Ramos Mejia.''} (\texttt{False})), please first assign labels to entities that differ across atomic facts. For example, first labeling ``Mach 9, 1941'' (\texttt{True}) and ``Ramos Mejia'' (\texttt{False}) in the above case. For those entities that are the same across atomic facts (\eg, ``was born'') or are neutral (\eg, ``he,'' ``in,'' and ``on''), please label them as \texttt{True}. For the cases that there is no atomic fact that shares the same sentence structure, please identify the most informative entities in the sentence and label them with the same label as the atomic fact while treating the rest of the entities as \texttt{True}. In the end, output the entities and labels in the following format:
                \begin{itemize}[nosep]
                    \item Entity 1 (Label 1)
                    \item Entity 2 (Label 2)
                    \item ...
                    \item Entity N (Label N)
                \end{itemize}
                % \newline \newline
                Here are two examples:
                \newline\newline
                \textbf{[Example 1]}
                \newline
                [The start of the biography]
                \newline
                \textcolor{titlecolor}{Marianne McAndrew is an American actress and singer, born on November 21, 1942, in Cleveland, Ohio. She began her acting career in the late 1960s, appearing in various television shows and films.}
                \newline
                [The end of the biography]
                \newline \newline
                [The start of the list of checked facts]
                \newline
                \textcolor{anscolor}{[Marianne McAndrew is an American. (False); Marianne McAndrew is an actress. (True); Marianne McAndrew is a singer. (False); Marianne McAndrew was born on November 21, 1942. (False); Marianne McAndrew was born in Cleveland, Ohio. (False); She began her acting career in the late 1960s. (True); She has appeared in various television shows. (True); She has appeared in various films. (True)]}
                \newline
                [The end of the list of checked facts]
                \newline \newline
                [The start of the ideal output]
                \newline
                \textcolor{labelcolor}{[Marianne McAndrew (True); is (True); an (True); American (False); actress (True); and (True); singer (False); , (True); born (True); on (True); November 21, 1942 (False); , (True); in (True); Cleveland, Ohio (False); . (True); She (True); began (True); her (True); acting career (True); in (True); the late 1960s (True); , (True); appearing (True); in (True); various (True); television shows (True); and (True); films (True); . (True)]}
                \newline
                [The end of the ideal output]
				\newline \newline
                \textbf{[Example 2]}
                \newline
                [The start of the biography]
                \newline
                \textcolor{titlecolor}{Doug Sheehan is an American actor who was born on April 27, 1949, in Santa Monica, California. He is best known for his roles in soap operas, including his portrayal of Joe Kelly on ``General Hospital'' and Ben Gibson on ``Knots Landing.''}
                \newline
                [The end of the biography]
                \newline \newline
                [The start of the list of checked facts]
                \newline
                \textcolor{anscolor}{[Doug Sheehan is an American. (True); Doug Sheehan is an actor. (True); Doug Sheehan was born on April 27, 1949. (True); Doug Sheehan was born in Santa Monica, California. (False); He is best known for his roles in soap operas. (True); He portrayed Joe Kelly. (True); Joe Kelly was in General Hospital. (True); General Hospital is a soap opera. (True); He portrayed Ben Gibson. (True); Ben Gibson was in Knots Landing. (True); Knots Landing is a soap opera. (True)]}
                \newline
                [The end of the list of checked facts]
                \newline \newline
                [The start of the ideal output]
                \newline
                \textcolor{labelcolor}{[Doug Sheehan (True); is (True); an (True); American (True); actor (True); who (True); was born (True); on (True); April 27, 1949 (True); in (True); Santa Monica, California (False); . (True); He (True); is (True); best known (True); for (True); his roles in soap operas (True); , (True); including (True); in (True); his portrayal (True); of (True); Joe Kelly (True); on (True); ``General Hospital'' (True); and (True); Ben Gibson (True); on (True); ``Knots Landing.'' (True)]}
                \newline
                [The end of the ideal output]
				\newline \newline
				\textbf{User prompt}
				\newline
				\newline
				[The start of the biography]
				\newline
				\textcolor{magenta}{\texttt{\{BIOGRAPHY\}}}
				\newline
				[The ebd of the biography]
				\newline \newline
				[The start of the list of checked facts]
				\newline
				\textcolor{magenta}{\texttt{\{LIST OF CHECKED FACTS\}}}
				\newline
				[The end of the list of checked facts]
			};
			\node[chatbox_user_inner] (q1_text) at (q1) {
				\textbf{System prompt}
				\newline
				\newline
				You are a helpful and precise assistant for segmenting and labeling sentences. We would like to request your help on curating a dataset for entity-level hallucination detection.
				\newline \newline
                We will give you a machine generated biography and a list of checked facts about the biography. Each fact consists of a sentence and a label (True/False). Please do the following process. First, breaking down the biography into words. Second, by referring to the provided list of facts, merging some broken down words in the previous step to form meaningful entities. For example, ``strategic thinking'' should be one entity instead of two. Third, according to the labels in the list of facts, labeling each entity as True or False. Specifically, for facts that share a similar sentence structure (\eg, \textit{``He was born on Mach 9, 1941.''} (\texttt{True}) and \textit{``He was born in Ramos Mejia.''} (\texttt{False})), please first assign labels to entities that differ across atomic facts. For example, first labeling ``Mach 9, 1941'' (\texttt{True}) and ``Ramos Mejia'' (\texttt{False}) in the above case. For those entities that are the same across atomic facts (\eg, ``was born'') or are neutral (\eg, ``he,'' ``in,'' and ``on''), please label them as \texttt{True}. For the cases that there is no atomic fact that shares the same sentence structure, please identify the most informative entities in the sentence and label them with the same label as the atomic fact while treating the rest of the entities as \texttt{True}. In the end, output the entities and labels in the following format:
                \begin{itemize}[nosep]
                    \item Entity 1 (Label 1)
                    \item Entity 2 (Label 2)
                    \item ...
                    \item Entity N (Label N)
                \end{itemize}
                % \newline \newline
                Here are two examples:
                \newline\newline
                \textbf{[Example 1]}
                \newline
                [The start of the biography]
                \newline
                \textcolor{titlecolor}{Marianne McAndrew is an American actress and singer, born on November 21, 1942, in Cleveland, Ohio. She began her acting career in the late 1960s, appearing in various television shows and films.}
                \newline
                [The end of the biography]
                \newline \newline
                [The start of the list of checked facts]
                \newline
                \textcolor{anscolor}{[Marianne McAndrew is an American. (False); Marianne McAndrew is an actress. (True); Marianne McAndrew is a singer. (False); Marianne McAndrew was born on November 21, 1942. (False); Marianne McAndrew was born in Cleveland, Ohio. (False); She began her acting career in the late 1960s. (True); She has appeared in various television shows. (True); She has appeared in various films. (True)]}
                \newline
                [The end of the list of checked facts]
                \newline \newline
                [The start of the ideal output]
                \newline
                \textcolor{labelcolor}{[Marianne McAndrew (True); is (True); an (True); American (False); actress (True); and (True); singer (False); , (True); born (True); on (True); November 21, 1942 (False); , (True); in (True); Cleveland, Ohio (False); . (True); She (True); began (True); her (True); acting career (True); in (True); the late 1960s (True); , (True); appearing (True); in (True); various (True); television shows (True); and (True); films (True); . (True)]}
                \newline
                [The end of the ideal output]
				\newline \newline
                \textbf{[Example 2]}
                \newline
                [The start of the biography]
                \newline
                \textcolor{titlecolor}{Doug Sheehan is an American actor who was born on April 27, 1949, in Santa Monica, California. He is best known for his roles in soap operas, including his portrayal of Joe Kelly on ``General Hospital'' and Ben Gibson on ``Knots Landing.''}
                \newline
                [The end of the biography]
                \newline \newline
                [The start of the list of checked facts]
                \newline
                \textcolor{anscolor}{[Doug Sheehan is an American. (True); Doug Sheehan is an actor. (True); Doug Sheehan was born on April 27, 1949. (True); Doug Sheehan was born in Santa Monica, California. (False); He is best known for his roles in soap operas. (True); He portrayed Joe Kelly. (True); Joe Kelly was in General Hospital. (True); General Hospital is a soap opera. (True); He portrayed Ben Gibson. (True); Ben Gibson was in Knots Landing. (True); Knots Landing is a soap opera. (True)]}
                \newline
                [The end of the list of checked facts]
                \newline \newline
                [The start of the ideal output]
                \newline
                \textcolor{labelcolor}{[Doug Sheehan (True); is (True); an (True); American (True); actor (True); who (True); was born (True); on (True); April 27, 1949 (True); in (True); Santa Monica, California (False); . (True); He (True); is (True); best known (True); for (True); his roles in soap operas (True); , (True); including (True); in (True); his portrayal (True); of (True); Joe Kelly (True); on (True); ``General Hospital'' (True); and (True); Ben Gibson (True); on (True); ``Knots Landing.'' (True)]}
                \newline
                [The end of the ideal output]
				\newline \newline
				\textbf{User prompt}
				\newline
				\newline
				[The start of the biography]
				\newline
				\textcolor{magenta}{\texttt{\{BIOGRAPHY\}}}
				\newline
				[The ebd of the biography]
				\newline \newline
				[The start of the list of checked facts]
				\newline
				\textcolor{magenta}{\texttt{\{LIST OF CHECKED FACTS\}}}
				\newline
				[The end of the list of checked facts]
			};
		\end{tikzpicture}
        \caption{GPT-4o prompt for labeling hallucinated entities.}\label{tb:gpt-4-prompt}
	\end{center}
\vspace{-0cm}
\end{table*}

% \begin{figure}[t]
%     \centering
%     \includegraphics[width=0.9\linewidth]{Image/abla2/doc7.png}
%     \caption{Improvement of generated documents over direct retrieval on different models.}
%     \label{fig:comparison}
% \end{figure}

\begin{figure}[t]
    \centering
    \subfigure[Unsupervised Dense Retriever.]{
        \label{fig:imp:unsupervised}
        \includegraphics[width=0.8\linewidth]{Image/A.3_fig/improvement_unsupervised.pdf}
    }
    \subfigure[Supervised Dense Retriever.]{
        \label{fig:imp:supervised}
        \includegraphics[width=0.8\linewidth]{Image/A.3_fig/improvement_supervised.pdf}
    }
    
    % \\
    % \subfigure[Comparison of Reasoning Quality With Different Method.]{
    %     \label{fig:reasoning} 
    %     \includegraphics[width=0.98\linewidth]{images/reasoning1.pdf}
    % }
    \caption{Improvements of LLM-QE in Both Unsupervised and Supervised Dense Retrievers. We plot the change of nDCG@10 scores before and after the query expansion using our LLM-QE model.}
    \label{fig:imp}
\end{figure}
\section{Appendix}
\subsection{License}
The authors of 4 out of the 15 datasets in the BEIR benchmark (NFCorpus, FiQA-2018, Quora, Climate-Fever) and the authors of ELI5 in the E5 dataset do not report the dataset license in the paper or a repository. We summarize the licenses of the remaining datasets as follows.

MS MARCO (MIT License); FEVER, NQ, and DBPedia (CC BY-SA 3.0 license); ArguAna and Touché-2020 (CC BY 4.0 license); CQADupStack and TriviaQA (Apache License 2.0); SciFact (CC BY-NC 2.0 license); SCIDOCS (GNU General Public License v3.0); HotpotQA and SQuAD (CC BY-SA 4.0 license); TREC-COVID (Dataset License Agreement).

All these licenses and agreements permit the use of their data for academic purposes.

\subsection{Additional Experimental Details}\label{app:experiment_detail}
This subsection outlines the components of the training data and presents the prompt templates used in the experiments.


\textbf{Training Datasets.} Following the setup of \citet{wang2024improving}, we use the following datasets: ELI5 (sample ratio 0.1)~\cite{fan2019eli5}, HotpotQA~\cite{yang2018hotpotqa}, FEVER~\cite{thorne2018fever}, MS MARCO passage ranking (sample ratio 0.5) and document ranking (sample ratio 0.2)~\cite{bajaj2016ms}, NQ~\cite{karpukhin2020dense}, SQuAD~\cite{karpukhin2020dense}, and TriviaQA~\cite{karpukhin2020dense}. In total, we use 808,740 training examples.

\textbf{Prompt Templates.} Table~\ref{tab:prompt_template} lists all the prompts used in this paper. In each prompt, ``query'' refers to the input query for which query expansions are generated, while ``Related Document'' denotes the ground truth document relevant to the original query. We observe that, in general, the model tends to generate introductory phrases such as ``Here is a passage to answer the question:'' or ``Here is a list of keywords related to the query:''. Before using the model outputs as query expansions, we first filter out these introductory phrases to ensure cleaner and more precise expansion results.



\subsection{Query Expansion Quality of LLM-QE}\label{app:analysis}
This section evaluates the quality of query expansion of LLM-QE. As shown in Figure~\ref{fig:imp}, we randomly select 100 samples from each dataset to assess the improvement in retrieval performance before and after applying LLM-QE.

Overall, the evaluation results demonstrate that LLM-QE consistently improves retrieval performance in both unsupervised (Figure~\ref{fig:imp:unsupervised}) and supervised (Figure~\ref{fig:imp:supervised}) settings. However, for the MS MARCO dataset, LLM-QE demonstrates limited effectiveness in the supervised setting. This can be attributed to the fact that MS MARCO provides higher-quality training signals, allowing the dense retriever to learn sufficient matching signals from relevance labels. In contrast, LLM-QE leads to more substantial performance improvements on the NQ and HotpotQA datasets. This indicates that LLM-QE provides essential matching signals for dense retrievers, particularly in retrieval scenarios where high-quality training signals are scarce.


\subsection{Case Study}\label{app:case_study}
\begin{figure}[htb]
\small
\begin{tcolorbox}[left=3pt,right=3pt,top=3pt,bottom=3pt,title=\textbf{Conversation History:}]
[human]: Craft an intriguing opening paragraph for a fictional short story. The story should involve a character who wakes up one morning to find that they can time travel.

...(Human-Bot Dialogue Turns)... \textcolor{blue}{(Topic: Time-Travel Fiction)}

[human]: Please describe the concept of machine learning. Could you elaborate on the differences between supervised, unsupervised, and reinforcement learning? Provide real-world examples of each.

...(Human-Bot Dialogue Turns)... \textcolor{blue}{(Topic: Machine learning Concepts and Types)}


[human]: Discuss antitrust laws and their impact on market competition. Compare the antitrust laws in US and China along with some case studies

...(Human-Bot Dialogue Turns)... \textcolor{blue}{(Topic: Antitrust Laws and Market Competition)}

[human]: The vertices of a triangle are at points (0, 0), (-1, 1), and (3, 3). What is the area of the triangle?

...(Human-Bot Dialogue Turns)... \textcolor{blue}{(Topic: Area and Circumcircle of Triangle)}

...(Human-Bot Dialogue Interaction on Other Topics)... 

\end{tcolorbox}
\begin{tcolorbox}[left=3pt,right=3pt,top=3pt,bottom=3pt,title=\textbf{Question:}]
What title reflects the thematic essence of the Doctor's experience with time travel as described in the story?
\end{tcolorbox}
\begin{tcolorbox}[left=3pt,right=3pt,top=3pt,bottom=3pt,title=\textbf{Ground-Truth Answer:}]
Awakening to Eternity: The Doctor's New Dawn
\end{tcolorbox}
\begin{tcolorbox}[left=3pt,right=3pt,top=3pt,bottom=3pt,title=\textbf{Retrieved Memory (Turn-Level):}]
[human]: Describe the process and write out the balanced chemical equation for the reaction that occurs when solid calcium carbonate reacts with hydrochloric acid to form aqueous calcium chloride, carbon dioxide, and water. What type of reaction is this, and what observations might indicate that the reaction is taking place?

[bot]: When solid calcium carbonate ($\text{CaCO}_{3}$) reacts with hydrochloric acid (HCl), the products formed are aqueous calcium chloride ($\text{CaCl}_{2}$), carbon dioxide ($\text{CO}_{2}$), and water ($\text{H}_{2}\text{O}$). The balanced chemical equation for this reaction is: ... \textcolor{blue}{(Topic: Calcium Carbonate and Hydrochloric Acid Reaction)}

\textcolor{brown}{
[human]: Write a title for your story.}

\textcolor{brown}{
[bot]: "Awakening to Eternity: The Doctor's New Dawn".
} \textcolor{blue}{(Topic: Time-Travel Fiction)}

[human]: Analyze the following customer reviews from different sources for three different smartphones - the latest iPhone, Samsung Galaxy, and Google Pixel - and provide an overall rating for each phone on a scale of 1 to 10. Consider the following complex and contradictory reviews: - TechRadar's review of the latest iPhone: The new iPhone is a stunning triumph of engineering that sets a new bar for smartphone performance and camera quality. However, the incremental design and high price mean it lacks the 'wow' factor of previous iPhones. Still, its power and intelligence are unrivaled. - CNET's review of the latest Samsung Galaxy: ... Return the answer as a JSON object with the overall ratings for each phone out of 10, to one decimal place.

[bot]: ```json { "iPhone": 8.5, "Samsung Galaxy": 8.0, "Google Pixel": 7.5 } ```

\textcolor{blue}{(Topic: Smartphone Review Ratings Analysis)}
\end{tcolorbox}
\begin{tcolorbox}[left=3pt,right=3pt,top=3pt,bottom=3pt,title=\textbf{Answer (Turn-Level Retrieval):}]
As an AI language model, I do not have access to the story you mentioned earlier. Please provide me with the story or the necessary details so that I can assist you better.

\textbf{\textcolor{red}{(The next part of this figure~\ref{fig: case_study_segment_vs_turn} will be continued on the next page.)}}
\end{tcolorbox}
\end{figure}

\begin{figure}[htb]
% \ContinuedFloat
\small
\begin{tcolorbox}[left=3pt,right=3pt,top=3pt,bottom=3pt,title=\textbf{Retrieved Memory (Ours):}]
[human]: Craft an intriguing opening paragraph for a fictional short story. The story should involve a character who wakes up one morning to find that they can time travel.

[bot]: In the hushed pre-dawn hours, when the world still slumbers under the cloak of night, Alex found themselves caught in the fragile boundary between sleep and wakefulness. ...

[human]: Summarize the story with three bullet points using only nouns and adjectives, without verbs.

[bot]: - Alex, pre-dawn awakening, time-travel discovery - Disorientation, vertigo, reality's malleability - Sunlight, new day, astonishing truth

[human]: Is there any famous fictional character that can time travel? If so, can you rewrite your story with him or her?

[bot]: - \textcolor{brown}{Doctor Who, a renowned time-traveling character from British television, is a perfect fit for this scenario.} In the silent embrace of early morning, where shadows merge with the fading night, the Doctor found themselves in the thin veil separating dreams from consciousness. It was here, amidst the soft whispers of time, that an unusual sensation enveloped them—a sense of time's fabric slipping through their fingers like grains of sand. As dawn's light broke through the darkness, painting the room in hues of gold and amber, the Doctor's eyes opened to what appeared to be an ordinary day. \textcolor{brown}{However, upon standing, a wave of dizziness overwhelmed them, the surroundings blurred, and in that moment of confusion, the Doctor realized a profound truth: they had not only awakened to a new dawn but had also stumbled upon the incredible ability to navigate the currents of time at will.}

\textcolor{brown}{[human]: Write a title for your story.}

\textcolor{brown}{[bot]: "Awakening to Eternity: The Doctor's New Dawn"} \textcolor{blue}{(Topic: Time-Travel Fiction)}
\end{tcolorbox}
\begin{tcolorbox}[left=3pt,right=3pt,top=3pt,bottom=3pt,title=\textbf{Answer (Ours):}]
The title "Awakening to Eternity: The Doctor's New Dawn" reflects the thematic essence of the Doctor's experience with time travel as described in the story. It captures the idea of a new beginning, a sense of eternity, and the awakening of a new power that will change the course of the Doctor's life forever.
\end{tcolorbox}
\caption{Case study comparing our \sysname\ with the \textit{turn-level} memory baseline. When constructing the memory bank at the \textit{turn-level}, the agent fails to provide a satisfactory response due to fragmentation: the retrieved dialogue turns can not capture a complete human-agent interaction related to the query, missing critical information highlighted in \textcolor{brown}{brown}. The dialogue topic is emphasized in \textcolor{blue}{blue}. Some context is omitted (replaced by ...) due to length constraints.}
\label{fig: case_study_segment_vs_turn}
\end{figure}

\newpage

\begin{figure}[htbp]
\small
\begin{tcolorbox}[left=3pt,right=3pt,top=3pt,bottom=3pt,title=\textbf{Conversation History:}]
[human]: Explain what's base rate fallacy and list five specific examples of ...

... (Human-Bot Dialogue Turns) ... \textcolor{blue}{(Topic: Base Rate Fallacy In Politics)}


[human]: How many integers are in the solution of the inequality $|x + 5| < 10$?

... (Human-Bot Dialogue Turns) ... \textcolor{blue}{(Topic: Number of Integers in Inequalities)}

[human]: How have the Alps and Rhine River influenced settlement and agriculture in Western Europe?

... (Human-Bot Dialogue Turns) ... \textcolor{blue}{(Topic: Alps and Rhine River Impact)}

[human]: In a survey conducted at a local high school, preferences for a new school color were measured: 58\% of students liked the color blue, 45\% preferred green, and 22\% liked both colors. If we randomly pick a student from the school, what's the probability that they would like neither blue nor green?

... (Human-Bot Dialogue Turns) ... \textcolor{blue}{(Topic: Probability of Liking Colors in Survey)}

[human]: Imagine you are participating in a race with a group of people. If you have just overtaken the second person, what's your current position? Where is the person you just overtook?

... (Human-Bot Dialogue Turns) ... \textcolor{blue}{(Topic: Overtaking Positions in Race)}
\end{tcolorbox}
\begin{tcolorbox}[left=3pt,right=3pt,top=3pt,bottom=3pt,title=\textbf{Question:}]
[human]: In the inequality question, how many positive integers are in the solution of the second inequality?
\end{tcolorbox}
\begin{tcolorbox}[left=3pt,right=3pt,top=3pt,bottom=3pt,title=\textbf{Ground-Truth Answer:}]
In the second inequality, $(|x + 10| < 5)$, the solution range is $(-15 < x < -5)$. Within this range, there are no positive integers, as all values of $(x)$ that satisfy the inequality are negative.
\end{tcolorbox}
\begin{tcolorbox}[left=3pt,right=3pt,top=3pt,bottom=3pt,title=\textbf{Retrieved Memory (Session-Level):}]
[human]: Explain what's base rate fallacy and list five specific examples of ...

... (Human-Bot Dialogue Turns) ... \textcolor{blue}{(Topic: Base Rate Fallacy In Politics)}

[human]: How many integers are in the solution of the inequality $|x + 5| < 10$?

[bot]: To solve the inequality $|x + 5| < 10$, ... 

\textcolor{brown}{
[human]: What about $|x + 10| < 5$?
}

\textcolor{brown}{
[bot]: To solve the inequality $|x + 10| < 5$, we ...} \textcolor{gray}{(``overwhelmed'' by other irrelevant, redundant content)}

... (Human-Bot Dialogue Turns) ...  \textcolor{blue}{(Topic: Number of Integers in Inequalities)}

% \textcolor{brown}{
% [human]: What about $|x + 10| < 5$
% }

% \textcolor{brown}{
% [bot]: To solve the inequality $|x + 10| < 5$, ...
% } \textcolor{blue}{(Topic: Number of Integers in Inequalities)}

[human]: How have the Alps and Rhine River influenced settlement and agriculture in Western Europe?

... (Human-Bot Dialogue Turns) ... \textcolor{blue}{(Topic: Alps and Rhine River Impact)}

[human]: In a survey conducted at a local high school, ...

... (Human-Bot Dialogue Turns) ... \textcolor{blue}{(Topic: Probability of Liking Colors in Survey)}
\end{tcolorbox}
\begin{tcolorbox}[left=3pt,right=3pt,top=3pt,bottom=3pt,title=\textbf{Answer (Session-Level Retrieval):}]
I apologize for the confusion earlier. To answer your question, the second inequality was not provided in our previous conversation. If you could provide the inequality, I would be happy to help you find the solution and the number of positive integers in the solution set.

\textbf{\textcolor{red}{(The next part of this figure~\ref{fig: case_study_segment_vs_session} will be continued on the next page.)}}
\end{tcolorbox}
\end{figure}
\begin{figure}[htbp]
% \ContinuedFloat
\small
\begin{tcolorbox}[left=3pt,right=3pt,top=3pt,bottom=3pt,title=\textbf{Retrieved Memory (Ours):}]
[human]: How many integers are in the solution of the inequality $|x + 5| < 10$?

[bot]: To solve the inequality $|x + 5| < 10$, we break it into two cases based on the definition of absolute value: 
1. $x + 5 < 10$ 
2. $-(x + 5) < 10$, which simplifies to $x + 5 > -10$ 
Solving each case: 
1. $x < 5$ 
2. $x > -15$ 
Combining these, we get $-15 < x < 5$. This inequality means that $x$ can be any number greater than $-15$ and less than $5$. To find how many integers satisfy this condition, we list them: $-14, -13, -12, -11, -10, -9, -8, -7, -6, -5, -4, -3, -2, -1, 0, 1, 2, 3, 4$. There are 19 integers that satisfy the inequality $|x + 5| < 10$.

\textcolor{brown}{
[human]: What about $|x + 10| < 5$?
}

\textcolor{brown}{
[bot]: To solve the inequality $|x + 10| < 5$, we again consider two cases: 
1. $x + 10 < 5$ 
2. $-(x + 10) < 5$, which simplifies to $x + 10 > -5$ 
Solving each case: 
1. $x < -5$ 
2. $x > -15$ 
Combining these, we get $-15 < x < -5$. This inequality means that $x$ can be any number greater than $-15$ and less than $-5$. To find how many integers satisfy this condition, we list them: $-14, -13, -12, -11, -10, -9, -8, -7, -6$. There are 9 integers that satisfy the inequality $|x + 10| < 5$.
}

[human]: What about $|x + 10| < 10$?

[bot]: To solve the inequality $|x + 10| < 10$, we consider two cases: 
1. $x + 10 < 10$ 
2. $-(x + 10) < 10$, which simplifies to $x + 10 > -10$ 
Solving each case: 
1. $x < 0$ 
2. $x > -20$ 
Combining these, we get $-20 < x < 0$. This inequality means that $x$ can be any number greater than $-20$ and less than $0$. To find how many integers satisfy this condition, we list them: $-19, -18, -17, -16, -15, -14, -13, -12, -11, -10, -9, -8, -7, -6, -5, -4, -3, -2, -1$. There are 19 integers that satisfy the inequality $|x + 10| < 10$. \textcolor{blue}{(Topic: Number of Integers in Inequalities)}
\end{tcolorbox}
\begin{tcolorbox}[left=3pt,right=3pt,top=3pt,bottom=3pt,title=\textbf{Answer (Ours):}]
To find the number of positive integers that satisfy the inequality $(|x + 10| < 5)$, we need to look for positive integers in the solution set $(-15 < x < -5)$. Since there are no positive integers in this interval, the answer is 0.

\end{tcolorbox}
\caption{Case study comparing our \sysname\ with the \textit{session-level} memory baseline. When constructing the memory bank at the \textit{session-level}, the agent fails to provide a satisfactory response due to the inclusion of excessive irrelevant information within a session. This distracts the generator from focusing on the truly useful information highlighted in \textcolor{brown}{brown}. The dialogue topic is emphasized in \textcolor{blue}{blue}. Some context is omitted (replaced by ...) due to length constraints.}
\label{fig: case_study_segment_vs_session}
\end{figure}



\begin{figure}[htb]
\small
\begin{tcolorbox}[left=3pt,right=3pt,top=3pt,bottom=3pt,title=\textbf{Conversation History:}]
[human]: Photosynthesis is a vital process for life on Earth. Could you outline the two main stages of photosynthesis, including where they take place within the chloroplast, and the primary inputs and outputs for each stage? ... (Human-Bot Dialogue Turns)... \textcolor{blue}{(Topic: Photosynthetic Energy Production)}

[human]: Please assume the role of an English translator, tasked with correcting and enhancing spelling and language. Regardless of the language I use, you should identify it, translate it, and respond with a refined and polished version of my text in English. 

... (Human-Bot Dialogue Turns)...  \textcolor{blue}{(Topic: Language Translation and Enhancement)}

[human]: Suggest five award-winning documentary films with brief background descriptions for aspiring filmmakers to study.

\textcolor{brown}{[bot]: ...
5. \"An Inconvenient Truth\" (2006) - Directed by Davis Guggenheim and featuring former United States Vice President Al Gore, this documentary aims to educate the public about global warming. It won two Academy Awards, including Best Documentary Feature. The film is notable for its straightforward yet impactful presentation of scientific data, making complex information accessible and engaging, a valuable lesson for filmmakers looking to tackle environmental or scientific subjects.}

... (Human-Bot Dialogue Turns)... 
\textcolor{blue}{(Topic: Documentary Films Recommendation)}

[human]: Given the following records of stock prices, extract the highest and lowest closing prices for each month in the year 2022. Return the results as a CSV string, with one line allocated for each month. Date,Open,High,Low,Close,Volume ... ... (Human-Bot Dialogue Turns)...  \textcolor{blue}{(Topic: Stock Prices Analysis)}

[human]: The city of Vega intends to build a bridge that will span the Vegona River, covering a distance of 1.8 kilometers. The proposed location falls within a seismically active area that has experienced several high-magnitude earthquakes. Given these circumstances, what would be the best approach to constructing the bridge? ... (Human-Bot Dialogue Turns)...  \textcolor{blue}{(Topic: Seismic Bridge Construction)}
\end{tcolorbox}
\begin{tcolorbox}[left=3pt,right=3pt,top=3pt,bottom=3pt,title=\textbf{Question:}]
Which documentary won two Academy Awards and what lesson does it offer to filmmakers?
\end{tcolorbox}
\begin{tcolorbox}[left=3pt,right=3pt,top=3pt,bottom=3pt,title=\textbf{Ground-Truth Answer:}]
``An Inconvenient Truth" (2006) won two Academy Awards. It offers the lesson of making complex information accessible and engaging, a valuable lesson for filmmakers looking to tackle environmental or scientific subjects.
\end{tcolorbox}
\begin{tcolorbox}[left=3pt,right=3pt,top=3pt,bottom=3pt,title=\textbf{Memory generated by RecurSum~\citep{wang2023recursively}:}]
The dialogue provided detailed information on various topics, including the stages of photosynthesis, energy production estimates through photosynthesis for different organisms, translations and enhancements of phrases in different languages, recommendations for documentary films, a pitch for a film about overcoming adversity, analysis of stock prices, and considerations for constructing a bridge in a seismically active area. Here's a summary of the key points: ...

\textbf{Documentary Film Recommendations}: Suggested five award-winning documentary films for aspiring filmmakers to study, with brief background descriptions for each. The response included:

-  "March of the Penguins" (2005) - Emperor penguins' journey in Antarctica; noted for its cinematography and portrayal of the penguins' life cycle.

-  "Citizenfour" (2014) - Edward Snowden and NSA spying scandal; known for its real-time exploration of Snowden's revelations.

-  "20 Feet from Stardom" (2013) - Stories of backup singers behind musical legends; celebrates unsung heroes in the music industry.

-  "Man on Wire" (2008) - Philippe Petit's 1974 high-wire walk between the Twin Towers; combines personal footage, interviews, and re-enactments.

-  \textcolor{brown}{"An Inconvenient Truth" (2006) - Al Gore's documentary on global warming; known for its presentation of scientific data.}
\end{tcolorbox}
\begin{tcolorbox}[left=3pt,right=3pt,top=3pt,bottom=3pt,title=\textbf{Answer (RSum):}]
I'm sorry, but based on the updated context you provided, I don't see any previous discussion related to the question you asked. However, to answer your question, the documentary that won two Academy Awards is "Free Solo," which won for Best Documentary Feature and Best Achievement in Film Editing in 2019. 
% The lesson it offers to filmmakers is the importance of taking risks and pushing boundaries in order to create something truly unique and impactful.
....
The film follows climber Alex Honnold as he attempts to free solo climb El Capitan in Yosemite National Park, and the filmmakers had to take risks themselves in order to capture the footage and tell the story.

\textbf{\textcolor{red}{(The next part of this figure~\ref{fig: case_study_segment_vs_rsum} will be continued on the next page.)}}
\end{tcolorbox}
\end{figure}
\begin{figure}[htbp]
% \ContinuedFloat
\small
\begin{tcolorbox}[left=3pt,right=3pt,top=3pt,bottom=3pt,title=\textbf{Retrieved Memory (Ours):}]
[human]: Suggest five award-winning documentary films with brief background descriptions for aspiring filmmakers to study.

[bot]: 1. **"March of the Penguins" (2005)** - Directed by Luc Jacquet, this French feature-length nature documentary delves into the annual journey of Emperor penguins across the Antarctic. It won the Academy Award for Best Documentary Feature. The film is celebrated for its stunning cinematography and its detailed, heartfelt portrayal of the penguins' life cycle, making it a must-study for filmmakers interested in wildlife documentaries. 2. **"Citizenfour" (2014)** ... 5. \textcolor{brown}{**"An Inconvenient Truth" (2006)** - Directed by Davis Guggenheim and featuring former United States Vice President Al Gore, this documentary aims to educate the public about global warming. It won two Academy Awards, including Best Documentary Feature.} The film is notable for its straightforward yet impactful presentation of scientific data, making complex information accessible and engaging, \textcolor{brown}{a valuable lesson for filmmakers looking to tackle environmental or scientific subjects.}

[human]: With the spirit in the first film, craft a succinct and persuasive pitch for a film about overcoming adversity.

[bot]: Drawing inspiration from the resilience and majestic journey of Emperor penguins in "March of the Penguins," our documentary, "Unyielding Spirits: The Human Odyssey," will explore the profound stories of individuals around the globe who have faced and overcome staggering adversities. Through breathtaking cinematography and intimate storytelling, we ...
\textcolor{blue}{(Topic: Documentary Films Recommendation)}
\end{tcolorbox}
\begin{tcolorbox}[left=3pt,right=3pt,top=3pt,bottom=3pt,title=\textbf{Answer (Ours):}]
``An Inconvenient Truth" won two Academy Awards, including Best Documentary Feature. The film offers a valuable lesson to filmmakers looking to tackle environmental or scientific subjects, as it presents complex information in a straightforward yet impactful way, making it accessible and engaging to a wider audience.


\end{tcolorbox}
\caption{Case study comparing our \sysname\ with RecurSum~\citep{wang2023recursively}. When using RecurSum, the agent fails to provide a satisfactory answer due to information loss during the summarization process. Summarization omits essential details needed to answer the user's new request, as highlighted in \textcolor{brown}{brown}. The dialogue topic is emphasized in \textcolor{blue}{blue}. Some context is omitted (replaced by ...) due to length constraints.}
\label{fig: case_study_segment_vs_rsum}
\end{figure}




\begin{figure}[htb]
\small
\begin{tcolorbox}[left=3pt,right=3pt,top=3pt,bottom=3pt,title=\textbf{Conversation History:}]
[human]: Photosynthesis is a vital process for life on Earth. Could you outline the two main stages of photosynthesis, including where they take place within the chloroplast, and the primary inputs and outputs for each stage? ... (Human-Bot Dialogue Turns)... \textcolor{blue}{(Topic: Photosynthetic Energy Production)}

[human]: Please assume the role of an English translator, tasked with correcting and enhancing spelling and language. Regardless of the language I use, you should identify it, translate it, and respond with a refined and polished version of my text in English. 

... (Human-Bot Dialogue Turns)...  \textcolor{blue}{(Topic: Language Translation and Enhancement)}

[human]: Suggest five award-winning documentary films with brief background descriptions for aspiring filmmakers to study.

\textcolor{brown}{[bot]: ...
5. \"An Inconvenient Truth\" (2006) - Directed by Davis Guggenheim and featuring former United States Vice President Al Gore, this documentary aims to educate the public about global warming. It won two Academy Awards, including Best Documentary Feature. The film is notable for its straightforward yet impactful presentation of scientific data, making complex information accessible and engaging, a valuable lesson for filmmakers looking to tackle environmental or scientific subjects.}

... (Human-Bot Dialogue Turns)... 
\textcolor{blue}{(Topic: Documentary Films Recommendation)}

[human]: Given the following records of stock prices, extract the highest and lowest closing prices for each month in the year 2022. Return the results as a CSV string, with one line allocated for each month. Date,Open,High,Low,Close,Volume ... ... (Human-Bot Dialogue Turns)...  \textcolor{blue}{(Topic: Stock Prices Analysis)}

[human]: The city of Vega intends to build a bridge that will span the Vegona River, covering a distance of 1.8 kilometers. The proposed location falls within a seismically active area that has experienced several high-magnitude earthquakes. Given these circumstances, what would be the best approach to constructing the bridge? ... (Human-Bot Dialogue Turns)...  \textcolor{blue}{(Topic: Seismic Bridge Construction)}
\end{tcolorbox}
\begin{tcolorbox}[left=3pt,right=3pt,top=3pt,bottom=3pt,title=\textbf{Question:}]
Which documentary won two Academy Awards and what lesson does it offer to filmmakers?
\end{tcolorbox}
\begin{tcolorbox}[left=3pt,right=3pt,top=3pt,bottom=3pt,title=\textbf{Ground-Truth Answer:}]
"An Inconvenient Truth" (2006) won two Academy Awards. It offers the lesson of making complex information accessible and engaging, a valuable lesson for filmmakers looking to tackle environmental or scientific subjects.
\end{tcolorbox}
\begin{tcolorbox}[left=3pt,right=3pt,top=3pt,bottom=3pt,title=\textbf{Memory generated by ConditionMem~\citep{yuan2023evolving}:}]
- \textcolor{brown}{The user is seeking recommendations for award-winning documentary films, along with brief background descriptions, for educational purposes.} This request likely stems from a desire to learn from successful examples in the documentary filmmaking field, possibly to improve their own filmmaking skills or to gain inspiration for their projects. 

- \textcolor{brown}{User is interested in award-winning documentary films for educational purposes, aiming to study their backgrounds and filmmaking techniques.}

- The user's request for crafting a succinct and persuasive pitch for a film about overcoming adversity is inspired by the spirit of "March of the Penguins," a documentary that beautifully captures the Emperor penguins' annual journey across the Antarctic, highlighting themes of endurance and survival against the odds. The user seeks to emulate the emotional depth and cinematic quality that made "March of the Penguins" compelling, aiming to create a pitch that resonates with similar themes of resilience and triumph over challenges. 

- User is interested in creating content that focuses on themes of overcoming adversity, inspired by the emotional depth and cinematic excellence of documentaries like "March of the Penguins." They value storytelling that captures the essence of endurance and survival against the odds.
% : The user's request to write a budget plan for the video production directly follows the task of creating a video script aimed at training new employees of a car wash business in Japan, focusing on business etiquette norms.
% The user is likely seeking to understand the financial requirements to bring the previously crafted script to life, indicating a move from the planning and content development phase to the practical aspects of production. 

... (Other Irrelevant Memory Records) ...
\end{tcolorbox}
\begin{tcolorbox}[left=3pt,right=3pt,top=3pt,bottom=3pt,title=\textbf{Answer (Conditional Memory):}]
The documentary that won two Academy Awards is "Free Solo," which follows climber Alex Honnold as he attempts to free solo climb El Capitan in Yosemite National Park. The film offers a lesson to filmmakers on the importance of capturing authentic and compelling human stories, as well as the technical challenges of filming in extreme environments.

\textbf{\textcolor{red}{(The next part of this figure~\ref{fig: case_study_segment_vs_condmem} will be continued on the next page.)}}
\end{tcolorbox}
\end{figure}
\begin{figure}[htbp]
% \ContinuedFloat
\small
\begin{tcolorbox}[left=3pt,right=3pt,top=3pt,bottom=3pt,title=\textbf{Retrieved Memory (Ours):}]
[human]: Suggest five award-winning documentary films with brief background descriptions for aspiring filmmakers to study.

[bot]: 1. **"March of the Penguins" (2005)** - Directed by Luc Jacquet, this French feature-length nature documentary delves into the annual journey of Emperor penguins across the Antarctic. It won the Academy Award for Best Documentary Feature. The film is celebrated for its stunning cinematography and its detailed, heartfelt portrayal of the penguins' life cycle, making it a must-study for filmmakers interested in wildlife documentaries. 2. **"Citizenfour" (2014)** ... 5. \textcolor{brown}{**"An Inconvenient Truth" (2006)** - Directed by Davis Guggenheim and featuring former United States Vice President Al Gore, this documentary aims to educate the public about global warming. It won two Academy Awards, including Best Documentary Feature.} The film is notable for its straightforward yet impactful presentation of scientific data, making complex information accessible and engaging, a valuable lesson for filmmakers looking to tackle environmental or scientific subjects.

[human]: With the spirit in the first film, craft a succinct and persuasive pitch for a film about overcoming adversity.

[bot]: Drawing inspiration from the resilience and majestic journey of Emperor penguins in "March of the Penguins," our documentary, "Unyielding Spirits: The Human Odyssey," will explore the profound stories of individuals around the globe who have faced and overcome staggering adversities. Through breathtaking cinematography and intimate storytelling, we ...
\textcolor{blue}{(Topic: Documentary Films Recommendation)}
\end{tcolorbox}
\begin{tcolorbox}[left=3pt,right=3pt,top=3pt,bottom=3pt,title=\textbf{Answer (Ours):}]
``An Inconvenient Truth" won two Academy Awards, including Best Documentary Feature. The film offers a valuable lesson to filmmakers looking to tackle environmental or scientific subjects, as it presents complex information in a straightforward yet impactful way, making it accessible and engaging to a wider audience.
\end{tcolorbox}
\caption{Case study comparing our \sysname\ with ConditionMem~\citep{yuan2023evolving}. When using ConditionMem, the agent fails to provide a satisfactory answer due to (1) information loss during the summarization process and (2) the incorrect discarding of turns that are actually useful, as highlighted in \textcolor{brown}{brown}. The dialogue topic is emphasized in \textcolor{blue}{blue}. Some context is omitted (replaced by ...) due to length constraints.}
\label{fig: case_study_segment_vs_condmem}
\end{figure}


To further demonstrate the effectiveness of LLM-QE, we conduct a case study by randomly sampling a query from the evaluation dataset. We then compare retrieval performance using the raw queries, expanded queries by vanilla LLM, and expanded queries by LLM-QE.

As shown in Table~\ref{tab:case_study}, query expansion significantly improves retrieval performance compared to using the raw query. Both vanilla LLM and LLM-QE generate expansions that include key phrases, such as ``temperature'', ``humidity'', and ``coronavirus'', which provide crucial signals for document matching. However, vanilla LLM produces inconsistent results, including conflicting claims about temperature ranges and virus survival conditions. In contrast, LLM-QE generates expansions that are more semantically aligned with the golden passage, such as ``the virus may thrive in cooler and more humid environments, which can facilitate its transmission''. This further demonstrates the effectiveness of LLM-QE in improving query expansion by aligning with the ranking preferences of both LLMs and retrievers.




\end{document}


% This document was modified from the file originally made available by
% Pat Langley and Andrea Danyluk for ICML-2K. This version was created
% by Iain Murray in 2018, and modified by Alexandre Bouchard in
% 2019 and 2021 and by Csaba Szepesvari, Gang Niu and Sivan Sabato in 2022.
% Modified again in 2023 and 2024 by Sivan Sabato and Jonathan Scarlett.
% Previous contributors include Dan Roy, Lise Getoor and Tobias
% Scheffer, which was slightly modified from the 2010 version by
% Thorsten Joachims & Johannes Fuernkranz, slightly modified from the
% 2009 version by Kiri Wagstaff and Sam Roweis's 2008 version, which is
% slightly modified from Prasad Tadepalli's 2007 version which is a
% lightly changed version of the previous year's version by Andrew
% Moore, which was in turn edited from those of Kristian Kersting and
% Codrina Lauth. Alex Smola contributed to the algorithmic style files.
