\section{Related Works}
\vspace{-0.5mm}
\subsection{LiDAR (Inertial) SLAM}
\vspace{-1mm}

LiDAR-based SLAM has long been significantly influenced by LOAM \cite{LOAM}, which splits the SLAM problem into two parallel tasks: odometry and mapping. By extracting edge and planar features from LiDAR scans and optimizing feature correspondences over long-term optimization, LOAM achieves high accuracy with low computational cost in structured environments. This approach has inspired subsequent works such as LeGO-LOAM \cite{LeGO-LOAM} and LIO-SAM \cite{LIO-SAM}. LeGO-LOAM introduced lightweight ground-optimized segmentation and loop closure mechanisms. LIO-SAM innovated through Inertial Measurement Unit (IMU) tight-coupling in factor graphs \cite{FactorGraph} with the Bayes tree \cite{ISAM2}, enabling motion-aware submap matching and drift-aware loop closure integration. However, traditional methods like \cite{LOAM} and its variants struggle in featureless environments or with LiDARs that have a small field of view. To address these challenges, FAST-LIO \cite{FAST-LIO} introduces an iEKF update mechanism, enabling real-time alignment of each scan with an incrementally built map. In its later iteration, FAST-LIO2, based on the ikd-Tree, replaces feature matching with point-to-plane ICP on raw points, offering improved robustness and state-of-the-art performance across various environments. DLIO \cite{DLIO} further utilizes analytical equations for fast and parallelizable motion correction. By directly registering scans to the map and employing a nonlinear geometric observer, DLIO improves both accuracy and computational efficiency. LTA-OM \cite{LTAOM} integrates FAST-LIO2 for front-end odometry and STD \cite{STD} for loop closure, further incorporating loop optimization. Its multisession operation enables dynamic map updates and robust localization, yielding performance gains over state-of-the-art SLAM systems. iG-LIO \cite{ig-lio} introduces a tightly-coupled LIO framework using GICP, a voxel-based surface covariance estimator that replaces kd-tree-based method \cite{GICP}, \cite{VGICP} to accelerate processing in dense scans, and an incremental voxel map, enhancing efficiency while maintaining state-of-the-art accuracy. Recently, correspondence-based methods employ geometric primitives and robust estimation to reduce problem size and prune correspondences \cite{TEASER, Quatro}, yet point feature degradation and rigid assumptions yield unreliable matches and unresolved non‑planar ambiguities \cite{Segregator, Outram}. Although these methods achieve competitive performance in efficiency and accuracy, they exhibit inherent limitations in diverse environments due to their over-reliance on geometric features.
% Although these methods achieve a good balance between efficiency and precision, they exhibit inherent limitations in geometrically degenerate environments due to their over-reliance on geometric features and high dependence on point cloud density.
% While Quatro \cite{Quatro} employs quasi-SE(3) constraints to mitigate degeneracy, studies \cite{Segregator,Outram} show that its rigid, planar assumption fails to resolve non-planar ambiguities.
% Recently, correspondence-based methods use geometric primitives and robust estimation to reduce problem size and prune correspondences \cite{TEASER, Quatro}
% However, \cite{TEASER, Quatro} show that point feature degradation yields unreliable correspondences, while \cite{Segregator, Outram} reveal Quatro’s rigid assumptions fail to resolve non‑planar ambiguities.

\vspace{-1mm}

\subsection{Other Information Assisted LiDAR (Inertial) SLAM}
\vspace{-1mm}
In addition to LiDAR SLAM, several methods have been proposed to enhance robustness by incorporating additional information. I-LOAM \cite{I-LOAM} and Intensity-SLAM \cite{I-SLAM} integrate intensity as a similarity metric, incorporating it into a weighted ICP approach. Similarly, approaches like \cite{I-LOAM,I-SLAM} use high-intensity points as an additional feature class for more accurate registration. \cite{MD-SLAM} focuses on optimizing photometric errors using intensity, range, and normal images, but doesn't incorporate IMU data or perform motion undistortion. RI-LIO \cite{RI-LIO} integrates reflectivity images within a tightly-coupled LIO framework by leveraging photometric error minimization into the iEKF of \cite{FAST-LIO}, aiming to efficiently reduce the drift in geometric-only methods. COIN-LIO \cite{COIN-LIO} improves LIO by integrating LiDAR intensity with geometric registration, utilizing a novel image processing pipeline and feature selection strategy for enhanced robustness in geometrically degenerate environments, such as tunnels. By selecting complementary patches and continuously assessing feature quality, it performs well in these scenarios. However, its focus on geometrically degenerate environments (e.g., long corridors) limits its adaptability, resulting in compromised performance in more diverse environments. Critically, these methods either rely on dense-channel imaging LiDARs for reliable operation or suffer from scale distortions caused by spherical projection, which limits their localization accuracy and restricts their applicability to a broader range of LiDAR configurations, particularly for sparse-channel LiDAR systems. Therefore, we need another better source of information to assist LiDAR SLAM. 

\vspace{-1.5mm}
\subsection{Related BEV Approaches}
\vspace{-1mm}
Recent advances in LiDAR localization, place recognition and loop closure have explored BEV representations to improve accuracy.
MapClosures \cite{MapClosures} proposes a loop closure detection method for SLAM utilizing BEV density images with ORB features derived from local maps, enabling effective place recognition and detection of map-level closures. 
\cite{BVMatch} pioneers BEV-based place recognition by projecting 3D LiDAR scans to BEV images, generating rotation-invariant maximum index maps using Log-Gabor filters, and employing the novel Bird’s-Eye View Feature Transform (BVFT) for robust feature extraction and pose estimation. 
BEVPlace \cite{BEVPlace} and its extension \cite{BEVPlace++} further advance this concept using lightweight CNNs with rotation-equivariant modules and NetVLAD \cite{NetVLAD} global descriptors, achieving state-of-the-art performance in subtasks of global localization including place recognition and loop closure detection.
\par In contrast to direct point cloud matching or spherical projection based methods, BEV representations inherently avoid scale distortions by projecting 3D points into a unified 2D plane, thereby enabling stable improvements in SLAM performance across different LiDAR configurations as demonstrated in our experiments. Thereby, we propose BEV-LIO(LC), a novel LIO framework that leverages BEV features to tightly integrate geometric registration, reprojection error minimization, and loop closure.
\vspace{-0.5mm}


\begin{figure*}[!t]
   \vspace{-0.5mm}
\centering
    \vspace{-4.5mm}
\includegraphics[width=\linewidth]{image/second.pdf}
% \caption{System overview of BEV-LIO and BEV-LIO-LC. The LiDAR input is firstly preprocessed with imu input, then divided into geometric(green at right bottom) and reprojection(blue at left bottom, see Sec.\ref{projection},\ref{Feature},\ref{Reprojection Residual}) two parts. Both residuals are combined in an iterated update (red). The loop closure part(purple at top, see Sec.\ref{Loop Closure}) correct accumulated drift with factor graph.
% }
\caption{
System overview of BEV-LIO and BEV-LIO-LC. The system first preprocesses LiDAR scans with IMU motion compensation before constructing geometric and reprojection residuals. Geometric residuals (green, right bottom) are computed via ikd-Tree with kNN search and normal vector computation. Reprojection residuals (blue, left bottom) are derived from BEV image feature matching and constructed by pixel-point correspondence. Both residuals are fused within the iEKF for state estimation (red). To mitigate drift, BEV-LIO-LC incorporates a loop closure module (purple), which detects and verifies loops via global descriptor and BEV image matching and refines constraints in a factor graph.}
\label{system overview}
    \vspace{-5.5mm}
\end{figure*}

    \vspace{-0.5mm}