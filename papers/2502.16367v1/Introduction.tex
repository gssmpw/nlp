\section{Introduction}

Future wireless communications systems are expected to operate at millimeter-wave and sub-terahertz frequency bands and support a massive number of devices \cite{Rappaport2019} as in Internet of Things (IoT) scenarios \cite{Gupta_2015} and reach higher data rates \cite{Viswanathan_2020}. The transmission of higher data rates can represent a challenge in terms of the design of energy-efficient analog
to-digital converters (ADCs) since the power consumption in the ADC increases exponentially with amplitude resolution \cite{Murmann_2009} and quadratically with the sampling rate for bandwidths above 300MHz \cite{Murmann_ADC,cqabd}.
An established approach to decreasing the power consumption of each ADC is to consider 1-bit quantization \cite{bbprec,1bitidd,dqalms,dqarls,dynovs,comp}. In addition to reducing energy consumption, it reduces the complexity of the devices since automatic gain control can potentially be omitted. The loss of amplitude information can be compensated by increasing the sampling rate \cite{Gilbert_1993}. In a noise-free case, it has been shown that rates of $\log_2(M_\mathrm{Rx}+1)$ bits per Nyquist interval are achievable by $M_\mathrm{Rx}$-fold oversampling with respect to (w.r.t.) the Nyquist rate \cite{Shamai2_1994}.

The information must be encoded into the temporal samples with one-bit resolution. {Methods based on 1-bit quantization and oversampling have been introduced in \cite{Landau_2014, Son_2019,1bitcpm,dynovs}}. In \cite{Shamai2_1994} the constructed bandlimited transmit signal conveys the information into zero-crossing patterns. In this sense modulation schemes based on zero-crossing have been proposed in \cite{Peter_2021, Peter_2020, Bender_2019} with runlength limited (RLL) transmit sequences and \cite{Viveros_2023} with the time instance zero-crossing (TI ZX) modulation. The TI ZX modulation from \cite{Viveros_2023} encodes the information into the time-instance of zero-crossings in order to reduce the number of zero-crossings of the signal. Important precoding approaches for MIMO channels have been studied for systems with 1-bit quantization modulation based on the maximization of the minimum distance to the decision threshold (MMDDT) \cite{Viveros_2023}, minimum mean squared error (MMSE) \cite{Viveros_2023, Amine_2016, Jacobson_2016}, state-machine based waveform design optimization \cite{Viveros_2024} and quality of service (QOS) constraint \cite{Viveros_ssp_2021}. The proposed method in \cite{Viveros_ssp_2021}  minimizes the transmit power while taking into account quality of service constraints in terms of the minimum distance to the decision threshold. Some of these methods improve performance when combined with faster-than-Nyquist (FTN) signaling \cite{Mazo_1975}. Moreover, in \cite{Viveros_2023} a spectral efficiency lower bound is presented for the TI ZX modulation. Besides, in \cite{Viveros_2023} an analytical method is introduced for the TI ZX modulation with MMSE precoding. {Moreover, the study in \cite{Erico_2023} proposes a branch-and-bound method with  QOS for a solution that attains a target symbol-error probability}. \textcolor{r}{Other modulation schemes have been presented in \cite{mohamed_2017} for multiple-input multiple-output (MIMO) systems.}

 In this study, we consider a bandlimited multiuser MIMO downlink system with 1-bit quantization and oversampling, considering the TI ZX modulation \cite{Viveros_2023}. For this study, a semi-analytical symbol error rate upper bound is presented for the QOS constraint method with an established minimum distance to the decision threshold \cite{Viveros_ssp_2021}. {Different from \cite{Viveros_ssp_2021}, the precoding method is formulated such that the constraint is given in terms of a target  SER. Moreover considering the Gray coding for TI ZX modulation from \cite{Viveros_2023}, an approximate BER can also be defined as constraint. Numerical results illustrate the performance of the proposed and other competing techniques.} 
 
 The rest of this paper is organized as follows, Section~\ref{sec:system_model} details the considered system model. 
Afterwards, in Section~\ref{sec:TIZX} we explain the TI ZX modulation and the detection process. Section~\ref{sec:QOS} presents the optimization problem for QOS temporal precoding method. Then, the performance bound is introduced in Section~\ref{sec:bound}. Numerical results are presented in Section~\ref{sec:num_results} and
finally, we conclude in Section~\ref{sec:conclusiones}.

Notation: In the paper, all scalar values, vectors, and matrices are represented by $a$, ${\boldsymbol{x}}$ and ${\boldsymbol{X}}$, respectively. \textcolor{c}{The subscript $I/Q$ means that the process is done separately for the signal's phase and quadrature component.}
