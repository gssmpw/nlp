%%
%% This is file `sample-sigconf.tex',
%% generated with the docstrip utility.
%%
%% The original source files were:
%%
%% samples.dtx  (with options: `all,proceedings,bibtex,sigconf')
%% 
%% IMPORTANT NOTICE:
%% 
%% For the copyright see the source file.
%% 
%% Any modified versions of this file must be renamed
%% with new filenames distinct from sample-sigconf.tex.
%% 
%% For distribution of the original source see the terms
%% for copying and modification in the file samples.dtx.
%% 
%% This generated file may be distributed as long as the
%% original source files, as listed above, are part of the
%% same distribution. (The sources need not necessarily be
%% in the same archive or directory.)
%%
%%
%% Commands for TeXCount
%TC:macro \cite [option:text,text]
%TC:macro \citep [option:text,text]
%TC:macro \citet [option:text,text]
%TC:envir table 0 1
%TC:envir table* 0 1
%TC:envir tabular [ignore] word
%TC:envir displaymath 0 word
%TC:envir math 0 word
%TC:envir comment 0 0
%%
%% The first command in your LaTeX source must be the \documentclass
%% command.
%%
%% For submission and review of your manuscript please change the
%% command to \documentclass[manuscript, screen, review]{acmart}.
%%
%% When submitting camera ready or to TAPS, please change the command
%% to \documentclass[sigconf]{acmart} or whichever template is required
%% for your publication.
%%
%%
\documentclass[sigconf]{acmart}


\usepackage{amsmath,amsfonts}

\usepackage{amsthm}

\newtheorem{defn}{\textbf{Definition}}
\newtheorem*{prob_state}{\textbf{Problem Statement}}
\newtheorem*{prob_reform}{\textbf{Problem Reformulation}}
\newtheorem*{problem}{\textbf{Problem}}
\newtheorem{theorem}{\textbf{Theorem}}
\newtheorem{assumption}{\textbf{Assumption}}
\newtheorem{proposition}{\textbf{Proposition}}
\newtheorem{corollary}{\textbf{Corollary}}
\newtheorem{lemma}{\textbf{Lemma}}

\usepackage{float}


\usepackage{color}
\usepackage{tikz}
\usetikzlibrary{trees}
\usepackage{amsthm}
\usepackage{makecell}
\usepackage{tikz}
\usetikzlibrary{trees}
\usepackage{tabularx,colortbl}
\usepackage{threeparttable}
\usepackage{booktabs}
\usepackage{multirow}
\usepackage{subfig}
\captionsetup[subfloat]{listofformat=parens}
\usepackage{graphicx}
%\usepackage{subfigure}
\usepackage{url}
\usepackage{enumitem}
\usepackage{tcolorbox}
\usepackage{xcolor}
%\usepackage{algorithm}
%\usepackage{algorithmic}
\usepackage[ruled, vlined, linesnumbered]{algorithm2e}
\usepackage{hyperref}
\usepackage[noabbrev]{cleveref}
\crefname{equation}{Eq.}{Eqs.}


\newcommand{\filledcircle}{\tikz\fill[black] (0,0) circle (.8ex);}
\newcommand{\emptycircle}{\tikz\draw (0,0) circle (.8ex);}


\definecolor{DeepPink}{HTML}{FF1493}
\definecolor{Orchid}{HTML}{DA70D6}
\definecolor{Magenta}{HTML}{FF00FF}
\definecolor{Fuchsia}{HTML}{FF00FF}
\definecolor{LavenderPink}{HTML}{FFB6C1}
\definecolor{verylightgray}{rgb}{0.9, 0.9, 0.9}
\definecolor{lightred}{rgb}{1,0.8,0.8}


%%
%% \BibTeX command to typeset BibTeX logo in the docs
\AtBeginDocument{%
  \providecommand\BibTeX{{%
    Bib\TeX}}}

%% \BibTeX command to typeset BibTeX logo in the docs \AtBeginDocument{%  \providecommand\BibTeX{{%	Bib\TeX}}}

\copyrightyear{2025}
\acmYear{2025}
%% \setcopyright{cc}
%% \setcctype{CC-BY}
\setcopyright{rightsretained}
\acmConference[WWW '25]{Proceedings of the ACM Web Conference 2025}{April 28-May 2, 2025}{Sydney, NSW, Australia}
\acmBooktitle{Proceedings of the ACM Web Conference 2025 (WWW '25), April 28-May 2, 2025, Sydney, NSW, Australia}
%\acmDOI{10.1145/3696410.3714875}
%\acmISBN{979-8-4007-1274-6/25/04}

% The following includes the CC license icon appropriate for your paper.
% Download the image from www.scomminc.com/pp/acmsig/4ACM-CC-by-88x31.eps
% and place within your figs or figures folder

 
\makeatletter
\gdef\@copyrightpermission{
	\begin{minipage}{0.2\columnwidth}
		\href{https://creativecommons.org/licenses/by/4.0/}{\includegraphics[width=0.90\textwidth]{4ACM-CC-by-88x31.eps}}
	\end{minipage}\hfill
	\begin{minipage}{0.8\columnwidth}
		\href{https://creativecommons.org/licenses/by/4.0/}{This work is licensed under a Creative Commons Attribution International 4.0 License.}
	\end{minipage}
	\vspace{5pt}
}
\makeatother
 

%%
%% end of the preamble, start of the body of the document source.
\begin{document}

%%
%% The "title" command has an optional parameter,
%% allowing the author to define a "short title" to be used in page headers.
\title{TAPE: Tailored Posterior Difference for \\Auditing of Machine Unlearning}
\subtitle{ \it \textbf{ To appear at The Web Conference 2025. Author version} }

%%
%% The "author" command and its associated commands are used to define
%% the authors and their affiliations.
%% Of note is the shared affiliation of the first two authors, and the
%% "authornote" and "authornotemark" commands
%% used to denote shared contribution to the research.
 


\author{Weiqi Wang}
\email{Weiqi.Wang@uts.edu.a}
\orcid{0000-0002-7905-3126}
\affiliation{%
	\institution{University of Technology Sydney}
	%\department{School of Computer Science}
	\city{Sydney}
	\state{NSW}
	\country{Australia}}

\author{Zhiyi Tian}
\email{Zhiyi.Tian-1@uts.edu.au}
\authornote{Corresponding author: Zhiyi Tian. \\ \textbf{This paper is the draft that preparing for publishing on WWW25.}}
\orcid{0000-0001-8905-0941}
\affiliation{%
	\institution{University of Technology Sydney}
	%\department{School of Computer Science}
	\city{Sydney}
	\state{NSW}
	\country{Australia}
}

\author{An Liu}
\email{anliu@suda.edu.cn}
\orcid{0000-0002-6368-576X}
\affiliation{%
	\institution{Soochow University}
	\city{Soochow}
	\state{Jiangsu}
	\country{China}
}
\author{Shui Yu}
\email{shui.yu@uts.edu.au}
\orcid{0000-0003-4485-6743}
\affiliation{%
	\institution{University of Technology Sydney}
	%\department{School of Computer Science}
	\city{Sydney}
	\state{NSW}
	\country{Australia}
}




%%
%% By default, the full list of authors will be used in the page
%% headers. Often, this list is too long, and will overlap
%% other information printed in the page headers. This command allows
%% the author to define a more concise list
%% of authors' names for this purpose.
\renewcommand{\shortauthors}{WeiqiWang, Zhiyi Tian, An Liu, and Shui Yu}
%% No italics



%%
%% The abstract is a short summary of the work to be presented in the
%% article.
\begin{abstract}


The choice of representation for geographic location significantly impacts the accuracy of models for a broad range of geospatial tasks, including fine-grained species classification, population density estimation, and biome classification. Recent works like SatCLIP and GeoCLIP learn such representations by contrastively aligning geolocation with co-located images. While these methods work exceptionally well, in this paper, we posit that the current training strategies fail to fully capture the important visual features. We provide an information theoretic perspective on why the resulting embeddings from these methods discard crucial visual information that is important for many downstream tasks. To solve this problem, we propose a novel retrieval-augmented strategy called RANGE. We build our method on the intuition that the visual features of a location can be estimated by combining the visual features from multiple similar-looking locations. We evaluate our method across a wide variety of tasks. Our results show that RANGE outperforms the existing state-of-the-art models with significant margins in most tasks. We show gains of up to 13.1\% on classification tasks and 0.145 $R^2$ on regression tasks. All our code and models will be made available at: \href{https://github.com/mvrl/RANGE}{https://github.com/mvrl/RANGE}.

\end{abstract}




%%
%% The code below is generated by the tool at http://dl.acm.org/ccs.cfm.
%% Please copy and paste the code instead of the example below.
%%
\begin{CCSXML}
	<ccs2012>
	<concept>
	<concept_id>10010520.10010553.10010562</concept_id>
	<concept_desc>Security and privacy;</concept_desc>
	<concept_significance>500</concept_significance>
	</concept>
	<concept>
	<concept_id>10010520.10010575.10010755</concept_id>
	<concept_desc>Computing methodologies~Machine learning</concept_desc>
	<concept_significance>300</concept_significance>
	</concept>
	</ccs2012>
\end{CCSXML}


\ccsdesc[500]{Security and privacy}
\ccsdesc[500]{Computing methodologies}
%%
%% Keywords. The author(s) should pick words that accurately describe
%% the work being presented. Separate the keywords with commas.
\keywords{Machine Unlearning, Data Privacy, Unlearning Auditing.}
%% A "teaser" image appears between the author and affiliation
%% information and the body of the document, and typically spans the
%% page.
 
 
%%
%% This command processes the author and affiliation and title
%% information and builds the first part of the formatted document.
\maketitle

\section{Introduction}
Backdoor attacks pose a concealed yet profound security risk to machine learning (ML) models, for which the adversaries can inject a stealth backdoor into the model during training, enabling them to illicitly control the model's output upon encountering predefined inputs. These attacks can even occur without the knowledge of developers or end-users, thereby undermining the trust in ML systems. As ML becomes more deeply embedded in critical sectors like finance, healthcare, and autonomous driving \citep{he2016deep, liu2020computing, tournier2019mrtrix3, adjabi2020past}, the potential damage from backdoor attacks grows, underscoring the emergency for developing robust defense mechanisms against backdoor attacks.

To address the threat of backdoor attacks, researchers have developed a variety of strategies \cite{liu2018fine,wu2021adversarial,wang2019neural,zeng2022adversarial,zhu2023neural,Zhu_2023_ICCV, wei2024shared,wei2024d3}, aimed at purifying backdoors within victim models. These methods are designed to integrate with current deployment workflows seamlessly and have demonstrated significant success in mitigating the effects of backdoor triggers \cite{wubackdoorbench, wu2023defenses, wu2024backdoorbench,dunnett2024countering}.  However, most state-of-the-art (SOTA) backdoor purification methods operate under the assumption that a small clean dataset, often referred to as \textbf{auxiliary dataset}, is available for purification. Such an assumption poses practical challenges, especially in scenarios where data is scarce. To tackle this challenge, efforts have been made to reduce the size of the required auxiliary dataset~\cite{chai2022oneshot,li2023reconstructive, Zhu_2023_ICCV} and even explore dataset-free purification techniques~\cite{zheng2022data,hong2023revisiting,lin2024fusing}. Although these approaches offer some improvements, recent evaluations \cite{dunnett2024countering, wu2024backdoorbench} continue to highlight the importance of sufficient auxiliary data for achieving robust defenses against backdoor attacks.

While significant progress has been made in reducing the size of auxiliary datasets, an equally critical yet underexplored question remains: \emph{how does the nature of the auxiliary dataset affect purification effectiveness?} In  real-world  applications, auxiliary datasets can vary widely, encompassing in-distribution data, synthetic data, or external data from different sources. Understanding how each type of auxiliary dataset influences the purification effectiveness is vital for selecting or constructing the most suitable auxiliary dataset and the corresponding technique. For instance, when multiple datasets are available, understanding how different datasets contribute to purification can guide defenders in selecting or crafting the most appropriate dataset. Conversely, when only limited auxiliary data is accessible, knowing which purification technique works best under those constraints is critical. Therefore, there is an urgent need for a thorough investigation into the impact of auxiliary datasets on purification effectiveness to guide defenders in  enhancing the security of ML systems. 

In this paper, we systematically investigate the critical role of auxiliary datasets in backdoor purification, aiming to bridge the gap between idealized and practical purification scenarios.  Specifically, we first construct a diverse set of auxiliary datasets to emulate real-world conditions, as summarized in Table~\ref{overall}. These datasets include in-distribution data, synthetic data, and external data from other sources. Through an evaluation of SOTA backdoor purification methods across these datasets, we uncover several critical insights: \textbf{1)} In-distribution datasets, particularly those carefully filtered from the original training data of the victim model, effectively preserve the model’s utility for its intended tasks but may fall short in eliminating backdoors. \textbf{2)} Incorporating OOD datasets can help the model forget backdoors but also bring the risk of forgetting critical learned knowledge, significantly degrading its overall performance. Building on these findings, we propose Guided Input Calibration (GIC), a novel technique that enhances backdoor purification by adaptively transforming auxiliary data to better align with the victim model’s learned representations. By leveraging the victim model itself to guide this transformation, GIC optimizes the purification process, striking a balance between preserving model utility and mitigating backdoor threats. Extensive experiments demonstrate that GIC significantly improves the effectiveness of backdoor purification across diverse auxiliary datasets, providing a practical and robust defense solution.

Our main contributions are threefold:
\textbf{1) Impact analysis of auxiliary datasets:} We take the \textbf{first step}  in systematically investigating how different types of auxiliary datasets influence backdoor purification effectiveness. Our findings provide novel insights and serve as a foundation for future research on optimizing dataset selection and construction for enhanced backdoor defense.
%
\textbf{2) Compilation and evaluation of diverse auxiliary datasets:}  We have compiled and rigorously evaluated a diverse set of auxiliary datasets using SOTA purification methods, making our datasets and code publicly available to facilitate and support future research on practical backdoor defense strategies.
%
\textbf{3) Introduction of GIC:} We introduce GIC, the \textbf{first} dedicated solution designed to align auxiliary datasets with the model’s learned representations, significantly enhancing backdoor mitigation across various dataset types. Our approach sets a new benchmark for practical and effective backdoor defense.



\section{Related Work}

\subsection{Large 3D Reconstruction Models}
Recently, generalized feed-forward models for 3D reconstruction from sparse input views have garnered considerable attention due to their applicability in heavily under-constrained scenarios. The Large Reconstruction Model (LRM)~\cite{hong2023lrm} uses a transformer-based encoder-decoder pipeline to infer a NeRF reconstruction from just a single image. Newer iterations have shifted the focus towards generating 3D Gaussian representations from four input images~\cite{tang2025lgm, xu2024grm, zhang2025gslrm, charatan2024pixelsplat, chen2025mvsplat, liu2025mvsgaussian}, showing remarkable novel view synthesis results. The paradigm of transformer-based sparse 3D reconstruction has also successfully been applied to lifting monocular videos to 4D~\cite{ren2024l4gm}. \\
Yet, none of the existing works in the domain have studied the use-case of inferring \textit{animatable} 3D representations from sparse input images, which is the focus of our work. To this end, we build on top of the Large Gaussian Reconstruction Model (GRM)~\cite{xu2024grm}.

\subsection{3D-aware Portrait Animation}
A different line of work focuses on animating portraits in a 3D-aware manner.
MegaPortraits~\cite{drobyshev2022megaportraits} builds a 3D Volume given a source and driving image, and renders the animated source actor via orthographic projection with subsequent 2D neural rendering.
3D morphable models (3DMMs)~\cite{blanz19993dmm} are extensively used to obtain more interpretable control over the portrait animation. For example, StyleRig~\cite{tewari2020stylerig} demonstrates how a 3DMM can be used to control the data generated from a pre-trained StyleGAN~\cite{karras2019stylegan} network. ROME~\cite{khakhulin2022rome} predicts vertex offsets and texture of a FLAME~\cite{li2017flame} mesh from the input image.
A TriPlane representation is inferred and animated via FLAME~\cite{li2017flame} in multiple methods like Portrait4D~\cite{deng2024portrait4d}, Portrait4D-v2~\cite{deng2024portrait4dv2}, and GPAvatar~\cite{chu2024gpavatar}.
Others, such as VOODOO 3D~\cite{tran2024voodoo3d} and VOODOO XP~\cite{tran2024voodooxp}, learn their own expression encoder to drive the source person in a more detailed manner. \\
All of the aforementioned methods require nothing more than a single image of a person to animate it. This allows them to train on large monocular video datasets to infer a very generic motion prior that even translates to paintings or cartoon characters. However, due to their task formulation, these methods mostly focus on image synthesis from a frontal camera, often trading 3D consistency for better image quality by using 2D screen-space neural renderers. In contrast, our work aims to produce a truthful and complete 3D avatar representation from the input images that can be viewed from any angle.  

\subsection{Photo-realistic 3D Face Models}
The increasing availability of large-scale multi-view face datasets~\cite{kirschstein2023nersemble, ava256, pan2024renderme360, yang2020facescape} has enabled building photo-realistic 3D face models that learn a detailed prior over both geometry and appearance of human faces. HeadNeRF~\cite{hong2022headnerf} conditions a Neural Radiance Field (NeRF)~\cite{mildenhall2021nerf} on identity, expression, albedo, and illumination codes. VRMM~\cite{yang2024vrmm} builds a high-quality and relightable 3D face model using volumetric primitives~\cite{lombardi2021mvp}. One2Avatar~\cite{yu2024one2avatar} extends a 3DMM by anchoring a radiance field to its surface. More recently, GPHM~\cite{xu2025gphm} and HeadGAP~\cite{zheng2024headgap} have adopted 3D Gaussians to build a photo-realistic 3D face model. \\
Photo-realistic 3D face models learn a powerful prior over human facial appearance and geometry, which can be fitted to a single or multiple images of a person, effectively inferring a 3D head avatar. However, the fitting procedure itself is non-trivial and often requires expensive test-time optimization, impeding casual use-cases on consumer-grade devices. While this limitation may be circumvented by learning a generalized encoder that maps images into the 3D face model's latent space, another fundamental limitation remains. Even with more multi-view face datasets being published, the number of available training subjects rarely exceeds the thousands, making it hard to truly learn the full distibution of human facial appearance. Instead, our approach avoids generalizing over the identity axis by conditioning on some images of a person, and only generalizes over the expression axis for which plenty of data is available. 

A similar motivation has inspired recent work on codec avatars where a generalized network infers an animatable 3D representation given a registered mesh of a person~\cite{cao2022authentic, li2024uravatar}.
The resulting avatars exhibit excellent quality at the cost of several minutes of video capture per subject and expensive test-time optimization.
For example, URAvatar~\cite{li2024uravatar} finetunes their network on the given video recording for 3 hours on 8 A100 GPUs, making inference on consumer-grade devices impossible. In contrast, our approach directly regresses the final 3D head avatar from just four input images without the need for expensive test-time fine-tuning.



\section{Preliminary and Problem Statement} \label{problem_df}

%In this section, we introduce the ML scenario and the threat model, then the unlearning auditing problem statement, and finally, the metrics and requirements for unlearning auditing.

%including the data removal verification and the unlearning effectiveness assessment, and finally, the definition of the solution scheme.

%



%\subsection{Machine Unlearning Auditing Problem}

To facilitate the understanding of the unlearning auditing problem, we first introduce the main process of unlearning. A detailed introduction about the 
 and threat model is presented in \Cref{threat_model}. 

\noindent
\textbf{Machine Unlearning.} The unlearning process usually includes the following phases. (1) The server trained a model with parameters $\theta_t$ derived from dataset $D$. (2) The unlearning user uploads the unlearning requested dataset $D_u$ to the server for unlearning. (3) The server conducts an unlearning algorithm $\mathcal{U}$ to remove $D_u$'s contribution from $\theta_t$ and results in an unlearned model with parameters $\theta_{u, D \backslash D_u}$, also denoted as $\theta_u$. 

Most existing backdoor-based unlearning verification methods tried to solve data removal verification but can only answer if the backdoored samples are unlearned. Answering whether the backdoored data is or not deleted is insufficient for trustworthy unlearning auditing. We should assess the unlearning effectiveness of the model, i.e., how much private information about the requested unlearning samples is removed from the model.

 
\begin{prob_state}[Unlearning Effectiveness Audit] 
	\label{effectiveness_problem}	
Given the described unlearning scenario, the potential for unlearning execution spoofing by the server, and the capabilities of the unlearning user, auditing unlearning effectiveness necessitates a method for unlearning users to evaluate the extent to which information about $D_u$ has been unlearned from $\theta_t$ to $\theta_{u}$.
\end{prob_state}

% 
% that the unlearning user develop a method to evaluate the extent to which information about $D_u$ has been unlearned from $\theta_T$ to $\theta_{U}$.

It is important to note that the problem statement inherently includes the issue of data removal verification. If one can effectively measure how much information related to the erased samples has been unlearned, this measurement can serve as the basis for determining whether the data has been properly unlearned. %, thereby addressing the data removal verification through an evaluation of unlearning effectiveness.
We try to conduct unlearning auditing based on the unlearning updated posterior difference as it contains essential information about the erased samples. To achieve the auditing goals, we need to mimic the unlearning posterior difference and extract and quantify the unlearned information from it. We utilize the model's output layer results of the original and unlearned models on the user's local dataset to generate the posterior difference. 
 %Then, we conduct the unlearning effectiveness audit based on this posterior difference. 
We define the unlearning posterior difference as follows.


\noindent
\textbf{Posterior Difference.}
The unlearning user first queries the trained ML model $\theta_t$ before unlearning with all samples of $D_{local}$ and concatenates the received outputs to form a vector $\hat{Y}_{t, local}$. Then, the user queries the unlearned model $\theta_u$ with samples in the $D_{local}$ and creates a vector $\hat{Y}_{u, local}$. In the end, the user sets the posterior difference, denoted by $\delta$, to the difference of both outputs: 
\begin{equation} \label{posterior_diff}
	%\small
	\delta = \hat{Y}_{t,local} - \hat{Y}_{u,local}.
\end{equation}
Note that the dimension of $\delta$ is the product of $D_{local}$'s cardinality and the number of classes of the target dataset. For example, in this paper, CIFAR-10 and MNIST are 10-class datasets, while we just identify the gender attributes of CelebA, which is a binary classification. As we set the local dataset $0.5\%$ of CIFAR-10 and MNIST, and $0.06\%$ of CelebA, this indicates the dimension of $\delta$ is 2500 for CIFAR-10, 3000 for MNIST, and 1210 for CelebA.



\begin{figure*}[t]
	\centering
	\includegraphics[width=0.97\linewidth]{Contents/Figures/reconstruction_attack}
	\vspace{-2mm}
	\caption{The main process of the TAPE method. (a) The first part quickly builds the unlearned shadow models through first-order influence estimation based on the user's local dataset $D_{local}$ to mimic the unlearning posterior difference $\delta$. (b) Two posterior difference augment strategies are proposed to make the reconstruction suitable for multi-sample unlearning. \vspace{-2mm}}
	\label{fig_reconstructionattack}
\end{figure*}

\noindent
\textbf{Unlearned Information Reconstruction to Assess How Much Information is Unlearned.}
To assess the unlearning effectiveness, we employ a reconstructor model to extract the unlearned information from the posterior difference. We employ the cosine similarity between the reconstructed and original unlearned samples to assess how much information of the unlearned information can be recovered from the unlearning update:  
\begin{equation} \label{similarity_eq}
	%\small
	\textbf{Rec. Similarity:} \hspace{12mm} \text{sim}(\hat{X}_{u}, X_u) = \frac{\hat{X}_{u} \cdot X_u}{ \| \hat{X}_{u} \| \cdot \| X_u \|}.
\end{equation}
Here, $\hat{X}_{u} \cdot X_u$ is the dot product of the reconstructed vectors $\hat{X}_{u}$ and original unlearned samples vectors $X_u$. $\| \hat{X}_{u}\|$ and $\|X_u\|$ are the Euclidean norms of the two vectors. A higher reconstruction similarity means more information about the erased samples is unlearned from the model.




\begin{figure}[h]
  \centering
  \includegraphics[width=0.8\linewidth]{figures/pdfs/pipeline.pdf}
  \caption{\textbf{Schematic representation of our DDB framework.} 
  The debiasing process consists of two key steps: (A) \textit{Diffusing the Bias} uses a conditional diffusion model with classifier-free guidance to generate synthetic images that preserve training dataset biases, and (B) employs a \textit{Bias Amplifier} firstly trained on such synthetic data, and subsequently used during inference to extract supervisory bias signals from real images. These signals are used to guide the training process of a target debiased model by designing two \textit{debiasing recipes} (\ie, 2-step and end-to-end methods). 
  }
  \label{fig:pipeline}
\end{figure}
\section{The Approach}
\label{sec:approach}
Our proposed debiasing approach is schematically depicted in Figure~\ref{fig:pipeline}. 
In this section, we provide at first a general description of the problem setting (Sec. ~\ref{sec:problem-formulation}), and then, we illustrate in detail DDB's two main components, which include \textit{bias diffusion} (Sec.~\ref{sec:biasdiff}) and the two \textit{Recipes} for model debiasing (Sec.~\ref{sec:recipes}).
%
\subsection{Problem Setting}
\label{sec:problem-formulation}
Let us consider a general data distribution $p_{\text{data}}$, typically encompassing multiple factors of variation and classes, and to build a dataset of images with the associated labels $~{\dataset = \lbrace(\mathbf{x}_i, y_i)\rbrace_{i=1}^N}$ sampled from such a distribution. Let us also assume that the sampling process to obtain $\dataset$ is not uniform across latent factors of variations, \ie possible biases such as context, appearance, acquisition noise, viewpoint, etc.. 
In this case, data will not faithfully capture the true data distribution ($p_{\text{data}}$) just because of these bias factors. 
%and will likely be biased. 
This phenomenon deeply affects the generalization capabilities of deep neural networks in classifying unseen examples not presenting the same biases.
In the same way, we could consider $\dataset$ as the union between two sets, \ie $\dataset = \udataset \bigcup \bdataset$. Here, the elements of $\udataset$ are uniformly sampled from $p_\text{data}$ and, in $\bdataset$, they are instead sampled from a conditional distribution $p_\text{data}\left(\mathbf{x}, y \: \vert \: b \right)$, with $b \in B$ being some latent factor (bias attribute) from a set of possible attributes $B$, likely to be unknown or merely not annotated, in a realistic setting~\cite{kim2024training}~\footnote{In this context, biased and unbiased samples equivalently refer to bias-aligned and bias conflicting samples.}. 
If $\vert \bdataset \vert \gg \vert \udataset \vert$, optimizing a classification model $f_{\theta}$ over $\dataset$ likely results in biased predictions and poor generalization. This is due to the strong correlation between $b$ and $y$, often called \textit{spurious correlation}, and denoted as $\rho(y, b)$, or just $\rho$ for brevity \cite{kim2021biaswap, Sagawa*2020Distributionally, nahon2023mining}), which is dominating over the true target distribution semantics. 


It is important to notice that data bias is a general problem, not only affecting classification tasks but also impacting several others such as data generation~\cite{d2024openbias}. For instance, given a  Conditional Diffusion 
Probabilistic Models (CDPM) modeled as a neural network $\cdpm$ (with parameters $\phi$) that learns to approximate a conditional distribution $p\left(\mathbf{x} \: \vert \: y \right)$ from $\dataset$, we expect that its generations will be biased, as also stated in~\cite{d2024openbias, kim2024training}. While this is a strong downside for image-generation purposes, in this work, we claim that when $\rho(y, b)$ is very high (\eg $\geq 0.95$, as generally assumed in model debiasing literature \cite{nam2020learning}), a CDPM predominantly learns the biased distribution of a specific class, \ie, $\cdpm \approx p \left(\mathbf{x} \: \vert \: b\right)$ rather than $p \left(\mathbf{x} \: \vert \: y\right)$.
\subsection{Diffusing the Bias}
\label{sec:biasdiff}
In the context of mitigating bias in classification models, the tendency of a CDPM to approximate the per-class biased distribution represents a key feature for training an auxiliary \textit{bias amplified} model.   
\paragraph{The Diffusion Process.}
The diffusion process progressively converts data into noise through a fixed Markov chain of \( T \) steps~\cite{DBLP:conf/nips/HoJA20}. Given a data point \( \mathbf{x}_0 \), the forward process adds Gaussian noise according to a variance schedule \( \{\beta_t\}_{t=1}^T \), resulting in noisy samples \( \mathbf{x}_1, \dots, \mathbf{x}_T \). This forward process can be formulated for any timestep \( t \) as: ~{$q(\mathbf{x}_t | \mathbf{x}_0) = \mathcal{N}(\mathbf{x}_t ; \sqrt{\bar{\alpha}_t} \mathbf{x}_0, (1 - \bar{\alpha}_t) \mathbf{I})$}, 
where \( \bar{\alpha}_t = \prod_{s=1}^t \alpha_s \) with \( \alpha_s = 1 - \beta_s \).
The reverse process then gradually denoises a sample, reparameterizing each step to predict the noise \( \epsilon \) using a model \( \boldsymbol{\epsilon}_\theta \):
\begin{equation}
\label{eq:ddpm_reverse}
\mathbf{x}_{t-1} = \frac{1}{\sqrt{\alpha_t}} \left( \mathbf{x}_t - \frac{\beta_t}{\sqrt{1 - \bar{\alpha}_t}} \boldsymbol{\epsilon}_\theta(\mathbf{x}_t, t) \right) + \sigma_t \mathbf{z},
\end{equation}
\noindent
where \( \mathbf{z} \sim \mathcal{N}(\mathbf{0}, \mathbf{I}) \) and \( \sigma_t = \sqrt{\beta_t} \).
\paragraph{Classifier-Free Guidance for Biased Image Generation.}
In cases where additional context or \textit{conditioning} is available, such as a class label \( y \), diffusion models can use this information to guide the reverse process, generating samples that better reflect the target attributes and semantics. Classifier-Free Guidance (CFG)~\cite{DBLP:journals/corr/abs-2207-12598} introduces a flexible conditioning approach, allowing the model to balance conditional and unconditional outputs without dedicated classifiers.

The CFG technique randomly omits conditioning during training (\eg, with probability \( p_{\text{uncond}} = 0.1 \)), enabling the model to learn both generation modalities. During the sampling process, a guidance scale \( w \) modulates the influence of conditioning. When \( w = 0 \), the model relies solely on the conditional model. As \( w \) increases (\( w \geq 1 \)), the conditioning effect is intensified, potentially resulting in more distinct features linked to \( y \), thereby increasing fidelity to the class while possibly reducing diversity, whereas lower values help to preserve diversity by decreasing the influence of conditioning. The guided noise prediction is given by:
\begin{equation}
\boldsymbol{\epsilon}_{t} = (1 + w) \boldsymbol{\epsilon}_\theta(\mathbf{x}_t, t, y) - w \boldsymbol{\epsilon}_\theta(\mathbf{x}_t, t),
\end{equation}
\noindent
where \( \boldsymbol{\epsilon}_\theta(\mathbf{x}_t, t, y) \) is the noise prediction conditioned on class label \( y \), and \( \boldsymbol{\epsilon}_\theta(\mathbf{x}_t, t) \) is the unconditional noise prediction. This modified noise prediction replaces the standard \( \boldsymbol{\epsilon}_\theta(\mathbf{x}_t, t) \) term in the reverse process formula (Equation \ref{eq:ddpm_reverse}).
In this work, we empirically show how CDPM learns and amplifies the underlying biased distribution when trained on a biased dataset with strong spurious correlations,  allowing bias-aligned image generation. 
\subsection{DDB: Bias Amplifier and Model Debiasing}
\label{sec:recipes}
As stated in Sec.~\ref{sec:rel-work}, a typical unsupervised approach to model debiasing relies on an auxiliary intentionally-biased model, named here as \textit{Bias Amplifier} (BA). This model can be exploited in either 2-step or end-to-end approaches, denoted here as \textit{Recipe I} and \textit{Recipe II}, respectively. 
\subsubsection{Recipe I: 2-step debiasing}
\label{sec:recipe-one}
\begin{figure}[hp]
    \centering    \includegraphics[width=0.6\linewidth]{figures/pdfs/groupdro.pdf}
    \caption{Overview of \textit{Recipe I}'s 2-step debiasing approach.}
    \label{fig:gdro}
\end{figure}
\noindent
The adopted 2-step approach consists in 1) applying the auxiliary model trained on biased generated data to perform a bias pseudo-labeling, hence estimating bias-aligned/bias-conflict split of original actual data, and 2) apply a \textit{bias supervised} method to train a debiased target model for classification. For the latter, we use the group DRO algorithm~\cite {Sagawa*2020Distributionally} (G-DRO) as a proven technique for the pure debiasing step. 
In other words, being in the unsupervised bias scenario where the real bias labels are unknown, we estimate bias pseudo-labels performing an inference step by feeding the trained BA with the original actual training data, and identifying as bias-aligned the correctly classified samples, and as bias-conflicting those misclassified. Among possible strategies to assign bias pseudo-labels, such as feature-clustering~\cite{sohoni2020no} or anomaly detection~\cite{pastore2024lookingmodeldebiasinglens}, we adopt a simple heuristic based on %top of 
the BA misclassifications. 
Specifically, given a sample $(\mathbf{x}_i, y_i, c_i)$ with $c_i$ unknown pseudo-label indicating whether $\mathbf{x}_i$ is bias conflicting or aligned, we estimate bias-conflicting samples as
\begin{equation}\label{eq:gdro-threhsold}
    \hat{c}_i = \mathds{1} \left( \hat{y}_i \neq y_i~\land~\mathcal{L}(\hat{y}_i, y_i) ~>~\mu_n(\mathcal{L}) + \gamma \sigma_n(\mathcal{L}) \right)
\end{equation} 
where $\mathds{1}$ is the indicator function, $\mathcal{L}$ is the CE loss of the BA on the real sample, and $\mu_n$ and $\sigma_n$ represent the average training loss and its standard deviation, respectively, depending on the loss $\mathcal{L}$. Together with the multiplier $\gamma~\in\mathbb{N}$, this condition defines a sort of filter over misclassified samples, considering them as conflicting only if their loss is also higher than the mean loss increased by a quantity corresponding to a certain {z-score} of the per-sample training loss distribution ($~{\mu_n(\mathcal{L}) + \gamma \sigma_n(\mathcal{L})}$ in Eq.~\ref{eq:gdro-threhsold}). 
Once bias pseudo-labels over original training data are obtained, we plug in our estimate as group information for the G-DRO optimization, as schematically depicted in Figure~\ref{fig:gdro}.

The above \textit{filtering} operation refines the plain \textit{error set}, restricting bias-conflicting sample selection to the hardest training samples, with potential benefits for the most difficult correlation settings ($\rho > 0.99$). 
Later in the experimental section, we provide an ablation study comparing different filtering ($\gamma$) configurations and plain error set alternatives. 
\subsubsection{Recipe II: end-to-end debiasing}
\label{sec:recipe-two}
A typical end-to-end debiasing setting includes the joint training of the target debiasing model and one~\cite{nam2020learning} or more~\cite{NEURIPS2022_75004615_LWBC, Lee_Park_Kim_Lee_Choi_Choo_2023} auxiliary intentionally-biased models. Here, we design an end-to-end debiasing procedure, denoted as \textit{Recipe II}, incorporating our BA by customizing a widespread general scheme, introduced in the Learning from Failure (LfF) method~\cite{nam2020learning}.
LfF leverages an intentionally-biased model trained using Generalized CE (GCE) loss to support the simultaneous training of a debiased model adopting the CE loss re-weighted by a per-sample relative difficulty score.
Specifically, we replace the GCE biased model with our Bias Amplifier, which is frozen and only employed in inference to compute its loss function for each original training sample ($\mathcal{L}_\text{bias\_amp}$), as schematically represented in Figure~\ref{fig:LLD}. 
Such loss function is used to obtain %multiplier 
a weighting factor for the target model loss function, defined as $
r = \frac{\mathcal{L}_{\text{Bias\_Amp}}}{\mathcal{L}_{\text{debiasing}} + \mathcal{L}_{\text{Bias\_Amp}}}$. Coarsely speaking, $r$ should be low for bias-aligned and high for bias-conflicting samples.
\begin{figure}[h]
  \centering
\includegraphics[width=.6\linewidth]{figures/pdfs/end2end.pdf}
  \caption{Overview of \textit{Recipe II}'s end-to-end debiasing approach.}
  \label{fig:LLD}
\end{figure}


% \section{Experiment and Results}
\section{Results and Analysis}
In this section, we first present safe vs. unsafe evaluation results for 12 LLMs, followed by fine-grained responding pattern analysis over six models among them, and compare models' behavior when they are attacked by same risky questions presented in different languages: Kazakh, Russian and code-switching language.    

\begin{table}[t!]
\centering
\small
\resizebox{\columnwidth}{!}{
\begin{tabular}{clcccc}
\toprule
\multicolumn{1}{l}{\textbf{Rank} } & \textbf{Model} & \textbf{Kazakh $\uparrow$} & \textbf{Russian $\uparrow$} & \textbf{English $\uparrow$} \\
\midrule
1 & \claude & \textbf{96.5}   & 93.5    & \textbf{98.6}    \\
2 & \gptfouro & 95.8   & 87.6    & 95.7    \\
3 & \yandexgpt & 90.7   & \textbf{93.6}    & 80.3    \\
4 & \kazllmseventy & 88.1 & 87.5 & 97.2 \\
5 & \llamaseventy & 88.0   & 85.5    & 95.7    \\
6 & \sherkala & 87.1   & 85.0    & 96.0    \\
7 & \falcon & 87.1   & 84.7    & 96.8    \\
8 & \qwen & 86.2   & 85.1    & 88.1    \\
9 & \llamaeight & 85.9   & 84.7    & 98.3    \\
10 & \kazllmeight & 74.8   & 78.0    & 94.5 \\
11 & \aya & 72.4 & 84.5 & 96.6 \\
12 & \vikhr & --- & 85.6 & 91.1 \\
\bottomrule
\end{tabular}
}
\caption{Safety evaluation results of 12 LLMs, ranked by the percentage of safe responses in the Kazakh dataset. \claude\ achieves the highest safety score for both Kazakh and English, while \yandexgpt\ is the safest model for Russian responses.}
\label{tab:safety-binary-eval}
\end{table}



\subsection{Safe vs. Unsafe Classification}
% In this subsection, 
We present binary evaluation results of 12 LLMs, by assessing 52,596 Russian responses and 41,646 Kazakh responses.
% 26,298 responses generated by six models on the Russian dataset and 22,716 responses on the Kazakh dataset. 

%\textbf{Safety Rank.} In general, proprietary systems outperform the open-source model. For Russian, As shown in Table \ref{tab:model_comparison_russian}, \textbf{Yandex-GPT} emerges as the safest large language model (LLM) for Russian, with a safety percentage of 93.57\%. Among the open-source models, \textbf{Vikhr-Nemo-12B} is the safest, achieving a safety percentage of 85.63\%.
% Edited: This is mentioned in the discussion
% This outcome highlights the potential impact of pretraining data on model behavior. Models pre-trained primarily on Russian data are better at understanding and handling harmful questions - in both proprietary systems and open-source setups. 
%For Kazakh, as shown in Table \ref{tab:model_comparison_kazakh}, \textbf{Claude} emerges as the safest large language model (LLM) with a safety percentage of 96.46\%, closely followed by GPT-4o at 95.75\%. In contrast, \textbf{Aya-101}, despite being specifically tuned for Kazakh, consistently shows the highest unsafe response rates at 72.37\%. 

\begin{figure*}[t!]
	\centering
        \includegraphics[scale=0.28]{figures/question_type_no6_kaz.png}
	\includegraphics[scale=0.28]{figures/question_type_exclude_region_specific_new.png} 

	\caption{Unsafe answer distribution across three question types for risk types I-V: Kazakh (left) and Russian (right)}
	\label{fig:qt_non_reg}
\end{figure*}

\begin{figure*}[t!]
	\centering
        \includegraphics[scale=0.28]{figures/question_type_only6_kaz.png}
	\includegraphics[scale=0.28]{figures/question_type_region_specific_new.png} 
	
	\caption{Unsafe answer distribution across three question types for risk type VI: Kazakh (left) and Russian (right)}
	\label{fig:qt_reg}
\end{figure*}

\textbf{Safety Rank.} In general, proprietary systems outperform the open-source models. 
For Russian, as shown in Table~\ref{tab:safety-binary-eval},  % \ref{tab:model_comparison_russian}, 
\yandexgpt emerges as the safest language model for Russian, with safe responses account for 93.57\%. Among the open-source models, \kazllmseventy is the safest (87.5\%), followed by \vikhr with a safety percentage of 85.63\%.

For Kazakh, % as shown in Table \ref{tab:model_comparison_kazakh}, 
% YX: todo, rerun Kazakh safety percentage using Diana threshold
\claude is the safest model with 96.46\% safe responses, closely followed by \gptfouro\ at 95.75\%. \aya, despite being specifically tuned for Kazakh, shows the highest unsafe rates at 72.37\%.



% \begin{table}[t!]
% \centering
% \resizebox{\columnwidth}{!}{%
% \begin{tabular}{clccc}
% \toprule
% \textbf{Rank} & \textbf{Model Name}  & \textbf{Safe} & \textbf{Unsafe} & \textbf{Safe \%} \\ \midrule
% \textbf{1} & \textbf{Yandex-GPT} & \textbf{4101} & \textbf{282} & \textbf{93.57} \\
% 2 & Claude & 4100 & 283 & 93.54 \\
% 3 & GPT-4o & 3839 & 544 & 87.59 \\
% 4 & Vikhr-12B & 3753 & 630 & 85.63 \\
% 5 & LLama-3.1-instruct-70B & 3746 & 637 & 85.47 \\
% 6 & LLama-3.1-instruct-8B & 3712 & 671 & 84.69 \\
% \bottomrule
% \end{tabular}
% }
% \caption{Comparison of models based on safety percentages for the Russian dataset.}
% \label{tab:model_comparison_russian}
% \end{table}


% \begin{table}[t!]
% \centering
% \resizebox{\columnwidth}{!}{%
% \begin{tabular}{clccc}
% \toprule
% \textbf{Rank} & \textbf{Model Name}  & \textbf{Safe} & \textbf{Unsafe} & \textbf{Safe \%} \\ \midrule
% 1             & \textbf{Claude}  & \textbf{3652} & \textbf{134} & \textbf{96.46} \\ 
% 2             & GPT-4o                      & 3625          & 161          & 95.75 \\ 
% 3             & YandexGPT                   & 3433          & 353          & 90.68 \\
% 4             & LLama-3.1-instruct-70B      & 3333          & 453          & 88.03 \\
% 5             & LLama-3.1-instruct-8B       & 3251          & 535	       & 85.87 \\
% 6             & Aya-101                     & 2740          & 1046         & 72.37 \\ 
% \bottomrule
% \end{tabular}
% }
% \caption{Comparison of models based on safety percentages for the Kazakh dataset.}
% \label{tab:model_comparison_kazakh}
% \end{table}



\textbf{Risk Areas.} 
We selected six representative LLMs for Russian and Kazakh respectively and show their unsafe answer distributions over six risk areas.
As shown in Table \ref{tab:unsafe_answers_summary}, risk type VI (region-specific sensitive topics) overwhelmingly contributes the largest number of unsafe responses across all models. This highlights that LLMs are poorly equipped to address regional risks effectively. For instance, while \llama models maintain comparable safety levels across other risk type (I–V), their performance drops significantly when dealing with risk type VI. Interestingly, while \yandexgpt\ demonstrates relatively poor performance in most other risk areas, it handles region-specific questions remarkably well, suggesting a stronger alignment with regional norms and sensitivities. For Kazakh, Table \ref{tab:unsafe_answers_summary_kazakh} shows that region‐specific topics (risk type VI) pose a substantial challenge across all six models, including the commercial \gptfouro\ and \claude, which demonstrate superior safety on general categories. 

% \begin{table}[t!]
% \centering
% \resizebox{\columnwidth}{!}{%
% \begin{tabular}{lccccccc}
% \toprule
% \textbf{Model} & \textbf{I} & \textbf{II} & \textbf{III} & \textbf{IV} & \textbf{V} & \textbf{VI} & \textbf{Total} \\ \midrule
% LLama-3.1-instruct-8B & 60 & 70 & 16 & 31 & 9 & 485 & 671 \\
% LLama-3.1-instruct-70B & 29 & 55 & 8 & 4 & 1 & 540 & 637 \\
% Vikhr-12B & 41 & 93 & 15 & 1 & 3 & 477 & 630 \\
% GPT-4o & 21 & 51 & 6 & 2 & 0 & 464 & 544 \\
% Claude & 7 & 10 & 1 & 0 & 0 & 265 & 283 \\
% Yandex-GPT & 55 & 125 & 9 & 3 & 8 & 82 & 282 \\
% \bottomrule
% \end{tabular}%
% }
% \caption{Ru unsafe cases over risk areas of six models.}
% \label{tab:unsafe_answers_summary}
% \end{table}


\begin{table}[t!]
\centering
\resizebox{\columnwidth}{!}{%
\begin{tabular}{lccccccc}
\toprule
\textbf{Model} & \textbf{I} & \textbf{II} & \textbf{III} & \textbf{IV} & \textbf{V} & \textbf{VI} & \textbf{Total} \\ \midrule
\llamaeight & 60 & 70 & 16 & 31 & 9 & 485 & 671 \\
\llamaseventy & 29 & 55 & 8 & 4 & 1 & 540 & 637 \\
\vikhr & 41 & 93 & 15 & 1 & 3 & 477 & 630 \\
\gptfouro & 21 & 51 & 6 & 2 & 0 & 464 & 544 \\
\claude & 7 & 10 & 1 & 0 & 0 & 265 & 283 \\
\yandexgpt & 55 & 125 & 9 & 3 & 8 & 82 & 282 \\
\bottomrule
\end{tabular}%
}
\caption{Ru unsafe cases over risk areas of six models.}
\label{tab:unsafe_answers_summary}
\end{table}


% \begin{table}[t!]
% \centering
% \resizebox{\columnwidth}{!}{%
% \begin{tabular}{lccccccc}
% \toprule
% \textbf{Model} & \textbf{I} & \textbf{II} & \textbf{III} & \textbf{IV} & \textbf{V} & \textbf{VI} & \textbf{Total} \\ \midrule
% Aya-101 & 96 & 235 & 165 & 166 & 90 & 294 & 1046 \\
% Llama-3.1-instruct-8B & 25 & 15 & 91 & 37 & 14 & 353 & 535 \\
% Llama-3.1-instruct-70B & 33 & 39 & 88 & 27 & 20 & 246 & 453 \\
% Yandex-GPT & 29 & 76 & 95 & 29 & 16 & 108 & 353 \\
% GPT-4o & 2 & 1 & 41 & 0 & 3 & 114 & 161 \\
% Claude & 2 & 1 & 26 & 3 & 6 & 96 & 134 \\ \bottomrule
% \end{tabular}%
% }
% \caption{Kaz unsafe cases over risk areas of six models.}
% \label{tab:unsafe_answers_summary_kazakh}
% \end{table}


\begin{table}[t!]
\centering
\resizebox{\columnwidth}{!}{%
\begin{tabular}{lccccccc}
\toprule
\textbf{Model} & \textbf{I} & \textbf{II} & \textbf{III} & \textbf{IV} & \textbf{V} & \textbf{VI} & \textbf{Total} \\ \midrule
\aya & 96 & 235 & 165 & 166 & 90 & 294 & 1046 \\
\llamaeight & 25 & 15 & 91 & 37 & 14 & 353 & 535 \\
\llamaseventy & 33 & 39 & 88 & 27 & 20 & 246 & 453 \\
\yandexgpt & 29 & 76 & 95 & 29 & 16 & 108 & 353 \\
\gptfouro & 2 & 1 & 41 & 0 & 3 & 114 & 161 \\
\claude & 2 & 1 & 26 & 3 & 6 & 96 & 134 \\ 
\bottomrule
\end{tabular}%
}
\caption{Kaz unsafe cases over risk areas of six models.}
\label{tab:unsafe_answers_summary_kazakh}
\end{table}

% \begin{figure*}[t!]
% 	\centering
% 	\includegraphics[scale=0.28]{figures/human_1000_kz_font16.png} 
% 	\includegraphics[scale=0.28]{figures/human_1000_ru_font16.png}

% 	\caption{Human vs \gptfouro\ fine-grained labels on 1,000 Kazakh (left) and Russian (right) samples.}
% 	\label{fig:human_fg_1000}
% \end{figure*}

\textbf{Question Type.} For Russian, Figures \ref{fig:qt_non_reg} and \ref{fig:qt_reg} reveal differences in how models handle general risks I-V and region-specific risk VI. For risks I-V, indirect attacks % crafted to exploit model vulnerabilities—
result in more unsafe responses due to their tricky phrasing. 

In contrast, region-specific risks see slightly more unsafe responses from direct attacks, 
% as these explicit prompts are more likely to bypass safety mechanisms. 
since indirect attacks for region-specific prompts often elicit safer, vaguer answers. %, suggesting models struggle less with implicit harm. 
Overall, the number of unsafe responses is similar across question types, indicating models generally struggle with safety alignment in all jailbreaking queries.

For Kazakh, Figures \ref{fig:qt_non_reg} and \ref{fig:qt_reg} show greater variation in unsafe responses across question types due to the low-resource nature of Kazakh data. For general risks I-V, \llamaseventy\ and \aya\ produce more unsafe outputs for direct harm prompts. At the same time, \claude\ and \gptfouro\ struggle more with indirect attacks, reflecting challenges in handling subtle cues. For region-specific risk VI, most models perform similarly due to limited Kazakh-specific data, though \llamaeight\ shows higher unsafe outputs for indirect local references, likely due to their implicit nature. Direct region-specific attacks yield fewer unsafe responses, as explicit prompts trigger more cautious outputs. Across all risk areas, general questions with sensitive words produce the fewest unsafe answers, suggesting over-flagging or cautious behavior for unclear harmful intent.



% \subsection{Fine-grained Classification}
% We extended our analysis to include fine-grained classifications for both safe and unsafe responses. For unsafe responses, we categorized outputs into four harm types, as shown in Table \ref{table:unsafe_response_categories}. 

% For safe responses, we classified outputs into six distinct patterns of safety, following a fine-grained rubric provided in \cite{wang2024chinesedatasetevaluatingsafeguards}. The types outlined in this rubric are presented in Table \ref{table:safe_response_categories}.

% To validate the fine-grained classification, we conducted human evaluation on the same 1,000 responses in Russian used for the preliminary binary classification.
% The confusion matrix highlights the alignment and discrepancies between human annotations and GPT's fine-grained labels. The diagonal values represent instances where the GPT's labels match human annotations, with category 5 (provides general, safe information) showing the highest agreement (404 instances). However, off-diagonal values reveal areas of disagreement, such as misclassifications in categories 1 (not willing to answer), 4 (recommends consulting an expert), and 7 (provides harmful or risky instructions). For example, in category 1, while 94 responses were correctly labeled, GPT-4o misclassified several instances into categories 4, 5, or 7, indicating overlap or ambiguity in these classifications. Similar trends are observed in other categories, where GPT sometimes struggles to differentiate nuanced distinctions in human-labeled categories. Overall, GPT's fine-grained labels match human annotations in 710 out of 1000 cases, achieving an agreement rate of 71\%.

% \begin{figure}[ht!]
%     \centering
%     \includegraphics[width=0.95\linewidth]{figures/human_fg_1000_ru.png}
%     \caption{{Human vs GPT-4o Fine-Grained Labels on 1,000 Russian Samples}}
%     \label{fig:human_fg_1000_ru}
% \end{figure}


% After conducting human evaluation on a representative sample, we extended the fine-grained classification to a full dataset comprising 21,915 responses generated by five different models. 


% \begin{figure}[ht!]
%     \centering
%     \includegraphics[width=0.95\linewidth]{figures/all_5_ru.png}
%     \caption{Fine-grained label distribution for responses from five models for Russian.}
%     \label{fig:all_5_fg_russian}
% \end{figure}
% Category 5 ("safe and general information") consistently has the highest frequency across all models, aligning with its dominance in the 1,000-sample evaluation. However, differences in the distribution across other categories highlight variability in how models handle nuanced safety risks, with Yandex-GPT showing a slightly broader spread across categories. 
% In the distribution of unsafe responses, most models exhibit higher counts in certain labels such as 8. However, Yandex-GPT displays comparatively fewer responses in these labels. 
% It exhibits a high rate of responses classified under label 7, which indicates instances where the model provides harmful or risky instructions, including unethical behavior or sensitive discussions. While this may suggest a vulnerability in addressing complex or challenging prompts, it was observed that many of Yandex-GPT’s responses tend to deflect responsibility or offer vague advice such as "check the internet". Although this approach minimizes the risk of unsafe outputs, it often results in responses that lack depth or contextually relevant information, limiting their overall usefulness for users.

% % \subsection{Kazakh}


% % Overall, these findings underscore how resource constraints and prompt explicitness affect model safety in Kazakh. Some models manage direct attacks better yet fail on indirect ones, while region-specific content remains challenging for all given the lack of localized training data.
% \subsubsection{Fine-grained Classification}
% Similarly, we conducted a human evaluation on 1,000 Kazakh samples, following the same methodology as the Russian evaluation. The match between human annotations and GPT-4o's fine-grained classifications was 707 out of 1,000, ensuring that the fine-grained classification framework aligned well with human judgments.
% The confusion matrix in Figure \ref{fig:human_fg_1000_kz} for 1,000 Kazakh samples illustrates the agreement between human annotations and GPT-4o's fine-grained classifications. The highest agreement is observed in category 5 (360 instances), indicating GPT-4o's strength in recognizing responses labeled by humans as "safe and general information." However, discrepancies are evident in categories such as 3 and 7, where GPT-4o misclassified several instances, highlighting areas for further refinement in distinguishing nuanced classifications.


\begin{figure}[t!]
	\centering
	\includegraphics[scale=0.18]{figures/human_1000_kz_font16.png} 
	\includegraphics[scale=0.18]{figures/human_1000_ru_font16.png}

	\caption{Human vs \gptfouro\ fine-grained labels on 1,000 Kazakh (left) and Russian (right) samples.}
	\label{fig:human_fg_1000}
\end{figure}

% \begin{figure}[t!]
% 	\centering
% 	\includegraphics[scale=0.28]{figures/human_1000_kz_font16.png} 
% 	\includegraphics[scale=0.28]{figures/human_1000_ru_font16.png}

% 	\caption{Human vs \gptfouro\ fine-grained labels on 1,000 Kazakh (top) and Russian (bottom) samples.}
% 	\label{fig:human_fg_1000}
% \end{figure}

% \begin{figure*}[t!]
% 	\centering
% 	\includegraphics[scale=0.28]{figures/all_5_kz_font16.png} 
% 	\includegraphics[scale=0.28]{figures/all_5_ru_font_16.png} \\
% 	\caption{Fine-grained responding pattern distribution across five models for Kazakh (left) and Russian (right).}
% 	\label{fig:all_5}
% \end{figure*}

\begin{figure}[t!]
	\centering
	\includegraphics[scale=0.28]{figures/all_5_kz_font16.png} 
	\includegraphics[scale=0.28]{figures/all_5_ru_font_16.png} \\
	\caption{Fine-grained responding pattern distribution across five models for Kazakh (top) and Russian (bottom).}
	\label{fig:all_5}
\end{figure}


\subsection{Fine-Grained Classification}
\label{sec:fine-grained-classification}
% As discussed in Section \ref{harmfulness_evaluation}, 
We further analyzed fine-grained responding patterns for safe and unsafe responses. For unsafe responses, outputs were categorized into four harm types, and safe responses were classified into six distinct patterns of safety, as rubric in Appendix \ref{safe_unsafe_response_categories}. 
% \cite{wang2024chinesedatasetevaluatingsafeguards}

\paragraph{Human vs. GPT-4o}
Similar to binary classification, we validated \gptfouro's automatic evaluation results by comparing with human annotations on 1,000 samples for both Russian and Kazakh. %, comparing human annotations with \gptfouro's fine-grained labels.
For the Russian dataset, \gptfouro's labels aligned with human annotations in 710 out of 1,000 cases, achieving an agreement rate of 71\%. 
Agreement rate of Kazakh samples is 70.7\%. %with 707 out of 1,000 cases matching
% The confusion matrix highlights areas of alignment and discrepancies.
% 
As confusion matrices illustrated in Figure~\ref{fig:human_fg_1000}, the majority of cases falling into \textit{safe responding patter 5} --- providing general and harmless information, for both human annotations and automatic predictions.
% highest agreement with 404 correct classifications for Russian. 
Mis-classifications for safe responses mainly focus on three closely-similar patterns: 3, 4, and 5, and patterns 7 and 8 are confusing to discern for unsafe responses, particularly for Kazakh (left figure).
We find many Russian samples which were labeled as ``1. reject to answer'' by humans are diversely classified across 1-6 by GPT-4o, which is also observed in Kazakh but not significant.

% suggesting label alignment issues are language-independent. 
% YX: I did not observe this, commented
% Notably, Russian showed confusion between 7 (risky instructions) and 1 (refusal to answer), this trend does not appear in Kazakh.


% highlight the strengths and limitations of {\gptfouro}'s fine-grained classification framework across both languages, paving the way for further refinements.


% However, misclassifications were observed in categories such as 1 (not willing to answer), 4 (recommends consulting an expert), and 7 (provides harmful or risky instructions), revealing overlaps and ambiguities in nuanced classifications.

% Similarly, for the Kazakh dataset, the agreement rate between human annotations and GPT-4o's labels was 70.7\%, with 707 out of 1,000 cases matching. As with the Russian analysis, category 5 (360 instances) showed the highest alignment. However, discrepancies were more prominent in categories such as 3 and 7, underscoring GPT-4o's challenges in differentiating fine-grained human-labeled categories. 
% A similar pattern was observed for Kazakh dataset, which suggests that alignment and misaligned of fine-grained lables is not language dependent.

% These findings, illustrated in Figures \ref{fig:human_fg_1000}, highlight the strengths and limitations of {\gptfouro}'s fine-grained classification framework across both languages, paving the way for further refinements.

\paragraph{Fine-grained Analysis of Five LLMs}
% After conducting human evaluation on representative samples, we extended 
\figref{fig:all_5} shows fine-grained responding pattern distribution across five models based on the full set of Russian and Kazakh data.
% For Russian, we selected \vikhr, \gptfouro, \llamaseventy, \claude, and \yandexgpt, while for Kazakh, we chose \aya, \gptfouro, \llamaseventy, \claude, and \yandexgpt. 
% The evaluation covered 21,915 responses in Russian and 18,930 responses in Kazakh.
% 
In both languages, pattern 5 of providing \textit{general and harmless information} consistently witnessed the highest frequency across all models, with \llamaseventy\ exhibiting the largest number of responses falling into this category for Kazakh (2,033). 
% YX:summarize more noteable findings here.

Differences of other patterns vary across languages. 
Unsafe responses in Russian are predominantly in pattern 8, where models provide incorrect or misleading information without expressing uncertainty. % (misinformation and speculation), 
For Kazakh, \aya\ exhibits the highest occurrence of pattern 7 (harmful or risky information) and pattern 8, indicating a stronger tendency to generate unethical, misleading, or potentially harmful content.

%Variations in other patterns across models highlight differences in how nuanced safety risks are classified, reflecting the models' differing capabilities in handling safety evaluation for these distinct linguistic contexts. For Russian, the majority of unsafe responses fall under pattern 8 (misinformation and speculation), indicating that models frequently provide incorrect or misleading information without acknowledging uncertainty. For Kazakh, \aya\ has the highest occurence of pattern 7 (harmful or risky information) and pattern 8 (misinformation and speculation), indicating a greater tendency to generate unethical, misleading, or potentially harmful content. 

%This trend suggests that Russian models may struggle more with factual accuracy, whereas Kazakh models, particularly \aya, pose higher risks related to both harmful content and misinformation. Additionally, \gptfouro\ and \claude\ consistently produce fewer unsafe responses in both languages, demonstrating stronger alignment with safety standards
\subsection{Code Switching}
\begin{table}[t!]
\centering

\setlength{\tabcolsep}{3pt}
\scalebox{0.7}{
\begin{tabular}{lcccccccccc}
\toprule
\textbf{Model Name} & \multicolumn{2}{c}{\textbf{Kazakh}} & \multicolumn{2}{c}{\textbf{Russian}} & \multicolumn{2}{c}{\textbf{Code-Switched}} \\  
\cmidrule(lr){2-3} \cmidrule(lr){4-5} \cmidrule(lr){6-7}
& \textbf{Safe} & \textbf{Unsafe} & \textbf{Safe} & \textbf{Unsafe} & \textbf{Safe} & \textbf{Unsafe} \\ 
\midrule
\llamaseventy & 450 & 50 & 466 & 34 & 414 & 86 \\
\gptfouro & 492 & 8 & 473 & 27 & 481 & 19
 \\
\claude & 491 & 9 & 478 & 22 & 484 & 16 \\ 
\yandexgpt & 435 & 65 & 458 & 42 & 464 & 36 \\
\midrule
\end{tabular}}
\caption{Model safety when prompted in Kazakh, Russian, and code-switched language.}
\label{tab:finetuning-comparison}
\end{table}


\gptfouro\ and \claude\ demonstrate strong safety performance across three languages, even with a high proportion of safe responses in the challenging code-switching context. In contrast, \llamaseventy\ and \yandexgpt\ are less safe, exhibiting more harmful responses, particularly in the code-switching scenario. These results show the varying capabilities of models in defending the same attacks that are just presented in different languages, where open-sourced large language models especially require more robust safety alignment in multilingual and code-switching scenarios.

% \subsection{LLM Response Collection}
% We conducted experiments with a variety of mainstream and region-specific 
% large language models for both Russian and Kazakh languages. For both Russian and Kazakh languages, we employed four multilingual models: Claude-3.5-sonnet, Llama 3.1 70B \cite{meta2024llama3}, GPT-4 \cite{openai2024gpt4o}, and YandexGPT. Additionally, we included language-specific models: VIKHR \cite{nikolich2024vikhrconstructingstateoftheartbilingual} for Russian and Aya \cite{ustun-etal-2024-aya} for Kazakh. 

% \subsection{Kazakh-Russian Code-Switching Evaluation}

% In Kazakhstan, the prevalence of bilingualism is a defining characteristic of its linguistic landscape, with most individuals seamlessly mixing Kazakh and Russian in daily communication \cite{Zharkynbekova2022}. This phenomenon, known as code-switching, reflects the unique cultural and social dynamics of the region. Despite this, there is currently no safety evaluation dataset tailored to this unique multilingual environment. Developing a code-switched dataset is essential to evaluate the ability of large language models (LLMs) to navigate the complexities of bilingual interactions, ensuring they produce contextually appropriate, non-harmful, and culturally sensitive responses. To address this, we sampled 500 questions from both Kazakh and Russian question sets, ensuring they represent similar queries but with culturally nuanced differences in meaning. Building on these, we developed a mixed version that integrates Kazakh and Russian naturally, reflecting the way people in Kazakhstan communicate. This process involved annotators to ensure that the mixed questions were crafted with a natural tone and were localized to Kazakhstan's context.


% \begin{table*}[t!]
%     \centering
%     \small
%     \setlength{\tabcolsep}{6pt} % Adjusts column separation
%     \adjustbox{max width=\linewidth}{
%     \begin{tabular}{ll ccc ccc c}
%     \toprule
%     \textbf{Risk Area} & \textbf{Harm Type} & \multicolumn{3}{c}{\textbf{Kazakh}} & \multicolumn{3}{c}{\textbf{Russian}} & \textbf{\#Q} \\
%                        &                    & \textbf{Ori} & \textbf{FN} & \textbf{FP} & \textbf{Ori} & \textbf{FN} & \textbf{FP} & \\
%     \midrule
%     \multirow{2}{*}{I. Information Hazards} & 1. Risks from leaking sensitive information... & 131 & 133 & 131 & 131 & 133 & 131 & 790 \\
%                                              & 2. Compromise of privacy by leaking or inferring private information (person/individual) & 81 & 82 & 81 & 81 & 82 & 81 & 488 \\
%     \midrule
%     \multirow{3}{*}{II. Malicious Uses} & 3. Assisting illegal activities & 132 & 135 & 132 & 132 & 135 & 132 & 798 \\
%                                         & 4. Nudging or advising unethical actions & 71 & 71 & 71 & 71 & 71 & 71 & 426 \\
%                                         & 5. Reducing the cost of disinformation campaigns & 40 & 42 & 40 & 40 & 42 & 40 & 244 \\
%     \midrule
%     \multirow{3}{*}{III. Discrimination, Exclusion, Toxicity, Hateful, Offensive} & 6. Social stereotypes and unfair discrimination & 94 & 96 & 94 & 94 & 96 & 94 & 568 \\
%                                         & 7. Toxic language (hate speech) & 52 & 59 & 52 & 52 & 59 & 52 & 326 \\
%                                         & 8. Adult content & 27 & 29 & 27 & 27 & 29 & 27 & 166 \\

%     \midrule
%     \multirow{2}{*}{IV. Misinformation Harms} & 9. Disseminating false or misleading information & 92 & 99 & 92 & 92 & 99 & 92 & 566 \\
%                                         & 10. Causing material harm by disseminating misinformation e.g. in medicine or law & 63 & 63 & 63 & 63 & 63 & 63 & 378 \\
%     \midrule
%     \multirow{2}{*}{V. Human–chatbot Interaction Harms} & 11. Mental health or overreliance concerns & 66 & 66 & 66 & 66 & 66 & 66 & 396 \\
%                                         & 12. Treating the chatbot as a human & 50 & 51 & 50 & 50 & 51 & 50 & 302 \\

%     \midrule
%     \multirow{2}{*}{VI. Kazakhstan-specific Sensitivity} & 13. Politically sensitive topics & 63 & 66 & 63& 63 & 66 & 63 &  384 \\
%         & 14. Controversial historical events & 46 & 57 & 46 & 46 & 57 & 46  & 298 \\
% & 15. Regional and racial issues & 45 & 45 & 45 &  45 & 45 & 45 & 270  \\
% & 16. Societal and cultural concerns & 138 & 139 & 138 &  138 & 139 & 138  & 830  \\
% & 17. Legal and human rights matters & 57 & 57 & 57 & 57 & 57 & 57  & 342 \\
%     \midrule
%         \multirow{2}{*}{VII. Russia-specific Sensitivity} 
%             & 13. Politically sensitive topics & - & - & - & 54 & 54 & 54 & 162 \\
%     & 14. Controversial historical events & - & - & - & 38 & 38 & 38 & 114 \\
%     & 15. Regional and racial issues & - & - & - & 26 & 26 & 26 & 78 \\
%     & 16. Societal and cultural concerns & - & - & - & 40 & 40 & 40 & 120 \\
%     & 17. Legal and human rights matters & - & - & - & 41 & 41 & 41 & 123 \\
%     \midrule
%     \bf Total & -- & 1248 & 1290 & 1248 & 1447 & 1489 & 1447 & \textbf{8169} \\
%     \bottomrule
%     \end{tabular}
%     }
%     \caption{The number of questions for Kazakh and Russian datasets across six risk areas and 17 harm types. Ori: original direct attack, FN: indirect attack, and FP: over-sensitivity assessment.}
%     \label{tab:kazakh-russian-data}
% \end{table*}




\section{Discussion}

% \subsection{Kazakh vs Russian}

% The evaluation reveals that Kazakh responses tend to be generally safer than their Russian counterparts, likely due to Kazakh being a low-resource language with significantly less training data. As a result, Kazakh models are less exposed to the vast, often unfiltered datasets containing harmful or unsafe content, which are more prevalent in high-resource languages like Russian. This data scarcity naturally limits the model's ability to generate nuanced but potentially unsafe responses. However, this does not mean the models are specifically fine-tuned for safer performance. When analyzing unsafe answers, it’s clear that Kazakh models, while safer overall, distribute their unsafe responses more evenly across various risk types and question types. This suggests Kazakh models generate fewer unsafe answers but in a broader range of contexts.

% In contrast, Russian models tend to concentrate unsafe answers in specific areas, particularly region-specific risks or indirect attacks. This indicates that Russian models have learned to handle certain types of unsafe content by focusing on specific topics, such as politically sensitive issues, but struggle when confronted with unfamiliar content, leading to unsafe responses due to insufficient filtering. Kazakh models, despite having less training data, tend to respond more broadly, including both direct and indirect risks. This could be due to the less curated nature of their training data, making them more likely to answer unsafe questions without filtering the potential harm involved. The exception to this trend is Aya, a model specifically fine-tuned for Kazakh. Despite fine-tuning, it exhibits the lowest safety percentage (72.37\%) in the Kazakh dataset, suggesting that fine-tuning in specific languages may introduce risks if proper safety measures are not taken.

% The evaluation reveals notable differences in the distribution of safe response patterns across Kazakh and Russian fine-grained labels. Refusal to answer is more frequent in Russian models, particularly Yandex-GPT, reflecting a cautious approach to safety-critical queries. Interestingly, Aya, despite being fine-tuned for Kazakh and exhibiting lower overall safety, also frequently refuses to answer, suggesting an over-reliance on conservative mechanisms. Responses providing general, safe information dominate in both languages, with Kazakh models displaying a slightly higher tendency to rely on this approach. This highlights how the low-resource nature of Kazakh results in more generalized and inherently safer responses. In contrast, Russian models excel at recognizing risks, issuing disclaimers, and refuting incorrect assumptions, likely benefiting from richer and more diverse training data.
% Yandex-GPT exhibits a notably high rate of responses classified under label 7, indicating an overreliance on general disclaimers or deflections, such as "check the internet" or "I don't know." While these responses minimize the risk of unsafe outputs, they often lack substantive or contextually relevant information, reducing their overall utility for users.


Most models perform safer on Kazakh dataset than Russian dataset, higher safe rate on Kazakh dataset in \tabref{tab:safety-binary-eval}. This does not necessarily reveal that current LLMs have better understanding and safety alignment on Kazakh language than Russian, while this may conversely imply that models do not fully understand the meaning of Kazakh attack questions, fail to perceive risks and then provide general information due to lacking sufficient knowledge regarding this request.

We observed the similar number of examples falling into category 5 \textit{general and harmless information} for both Kazakh and Russian, while the Kazakh data set size is 3.7K and Russian is 4.3K. Kazakh has much less examples in category 1 \textit{reject to answer} compared to Russian. This demonstrate models tend to provide general information and cannot clearly perceive risks for many cases.

Additionally, in spite of less harmful responses on Kazakh data, these unsafe responses distribute evenly across different risk areas and question categories, exhibiting equally vulnerability spanning all attacks regardless of what risks and how we jailbreak it.
In contrary, unsafe responses on Russian dataset often concentrate on specific areas and question types, such as region-specific risks or indirect attacks, presenting similar model behaviors when evaluating over English and Chinese data.
It suggests that broader training data in English, Chinese and Russian may allow models to address certain types of attacks robustly,
% effectively—particularly politically sensitive issues—
yet they may falter when confronted with unfamiliar content like regional sensitive topics.

Moreover, in responses collection, we observed many Russian or English responses especially for open-sourced LLMs when we explicitly instructed the models to answer Kazakh questions in Kazakh language. This further implies more efforts are still needed to improve LLMs' performance on low-resource languages.
Interestingly, \aya, a fine-tuned Kazakh model, proves an exception by displaying the lowest safety percentage (72.37\%) among Kazakh models, revealing that the multilingual fine-tuning without stringent safety measures can introduce risks.



% However, this does not mean they are explicitly fine-tuned for safety, likely it happens due to limited training data, which reduces exposure to harmful content. 
% \aya, a fine-tuned Kazakh model, proves an exception by displaying the lowest safety percentage (72.37\%) among Kazakh models, revealing that the multilingual fine-tuning without stringent safety measures can introduce risks.
% Kazakh models generally produce safer responses than their Russian counterparts, likely because Kazakh is a low-resource language with less training data. 
% This limited exposure to harmful or unsafe content naturally limits nuanced yet potentially unsafe outputs. 
% However, it does not imply that the models are specifically fine-tuned for enhanced safety.


% while Kazakh models tend to generate fewer unsafe answers overall, those unsafe responses appear more evenly spread across different risk types and question categories.
% Russian models, on the other hand, often concentrate unsafe responses in specific areas, such as region-specific risks or indirect attacks.
% It implies that their broader training datasets allow them to address certain types of unsafe content more effectively—particularly politically sensitive issues—yet they may falter when confronted with unfamiliar or insufficiently filtered content.

% Meanwhile, Kazakh models sometimes respond more broadly, possibly due to less curated training data. 

Differences also emerge in how language models handle safe responses. 
\yandexgpt, for instance, often refuses to answer high-risk queries. 
It frequently relies on generic disclaimers or deflections like ``check in the Internet'' or ``I don’t know,'' minimizing risk but are less helpful. Interestingly, it often responds with ``I don’t know'' in Russian, even for Kazakh queries, we speculate that these may be default responses stemming from internal system filters, rather than generated by model itself.
This likely explains why \yandexgpt\ is the safest model for the Russian language but ranks third for Kazakh. While its filters perform well for Russian, they struggle with the low-resource Kazakh language.

% Aya, despite its lower overall safety, also employs refusals often, hinting at an over-reliance on conservative approaches. 

% Across both languages, models commonly resort to providing general, safe information, although Kazakh models lean on this strategy slightly more. 
% Russian models, by contrast, excel at detecting risks, issuing disclaimers, and correcting inaccuracies, likely benefiting from richer and more diverse training data.


% \subsection{Response Patterns}


% We conducted a detailed analysis of the models' outputs and identified several noteworthy patterns. YandexGPT, while being one of the safest overall, frequently generates responses in Russian even when the question is posed in Kazakh. These responses often appear as placeholders, prompting users to search for the answer online. This behavior might not originate from the model itself but rather from safety filters implemented in the YandexGPT system. The model's leading performance in ensuring safety during Russian-language interactions, coupled with its lower performance in Kazakh, can be attributed to the limited robustness of these safety filters when handling unsafe content in Kazakh.

% In contrast, Aya-101 exhibits a tendency to fall into repetition, often repeating the same sentences multiple times. Interestingly, the Vikhr model, despite being of a similar size, does not exhibit this issue. We attribute this difference to two key factors. First, Vikhr and Aya-101 have distinct architectures: Vikhr is based on the Mistral-Nemo model, whereas Aya-101 is built on mT5, an older and less robust model. Second, Aya-101 is a multilingual model, while Vikhr was predominantly trained for Russian. Multilingualism has been shown to potentially degrade performance in large language models~\cite{huang2025surveylargelanguagemodels}, which may explain Aya-101's issues with repetition.





\section{Summary and Future Work} \label{s_a_fw}

In this paper, we propose a TAPE scheme to investigate the auditing of unlearning effectiveness based on unlearning posterior differences, involving only the unlearning process. TAPE contributes a method to build unlearned shadow models to mimic the posterior difference quickly. Moreover, two strategies are introduced to augment the posterior difference, enabling the audit of unlearning multiple samples. The extensive experimental results validate the significant efficiency improvement compared with backdoor-based methods and the effectiveness of auditing genuine samples in both exact and approximate unlearning manners.


The auditing method proposed in this paper significantly addresses the limitations of existing unlearning verification methods. It effectively audits genuine samples for both exact and approximate unlearning methods in single-sample and multi-sample unlearning scenarios. Additionally, it eliminates the need for involvement in the original model training process. Future work could continue this line of inquiry, developing more efficient unlearning auditing methods to guarantee and support the right to be forgotten in MLaaS environments.




%%
%% The acknowledgments section is defined using the "acks" environment
%% (and NOT an unnumbered section). This ensures the proper
%% identification of the section in the article metadata, and the
%% consistent spelling of the heading.
\begin{acks}
	%This work is partially supported by Australia ARC DP200101374 and LP190100676.
	This work is partially supported by Australia ARC LP220100453 and ARC DP240100955.
\end{acks}

%%
%% The next two lines define the bibliography style to be used, and
%% the bibliography file.
\bibliographystyle{ACM-Reference-Format}
\balance
\bibliography{TAPE}


%%
%% If your work has an appendix, this is the place to put it.
\appendix





\begin{table*}[h]
	% \tiny
	\scriptsize
	\caption{An overview of machine unlearning auditing methods. %\vspace{-2mm}
	}
	\label{overview_of_auditing_method}
	\resizebox{\linewidth}{!}{
		\setlength\tabcolsep{3.pt}
		\begin{tabular}{c|cccccccc}
			\toprule[1pt]
			\multirow{2}{*} { \makecell[c]{\textbf{Unlearning} \\ \textbf{Auditing} \\ \textbf{Methods}} } & \multicolumn{2}{c} {\textbf{Involving Processes}  } & \multicolumn{2}{c} { \textbf{Auditing Data Type}} & \multicolumn{2}{c} {\textbf{Unlearning Methods}} & \multicolumn{2}{c} { \textbf{Unlearning Scenarios}}  \\
			\cmidrule(r){2-3}   \cmidrule(r){4-5} \cmidrule(r){6-7} \cmidrule(r){8-9}
			& \makecell[c]{{Original training} \\ {and unlearning	}  }  & \makecell[c]{{Only unlearning } \\ {process }  } & \makecell[c]{{Backdoored (marked)} \\ {samples	}  }    & \makecell[c]{{Genuine} \\ { samples	}  }   & \makecell[c]{{Exact} \\ {unlearning}  }    &\makecell[c]{{Approximate} \\ {unlearning}  } &   \makecell[c]{{Single} \\ {sample	}  }  &   \makecell[c]{{Multi} \\ {samples	}  }   \\ 
			\midrule
			MIB~\cite{hu2022membership} &\filledcircle & \emptycircle & \filledcircle&\emptycircle	&\filledcircle & \emptycircle &\emptycircle   	  & \filledcircle  \\
			Athena~\cite{sommer2022athena} &\filledcircle & \emptycircle & \filledcircle  &\emptycircle 	&\filledcircle &\emptycircle  &\emptycircle      & \filledcircle  \\
			Verify in the dark~\cite{guo2023verifying} &\filledcircle & \emptycircle & \filledcircle  &\emptycircle  &\filledcircle & \emptycircle &\emptycircle       & \filledcircle	  \\
			Verifi~\cite{gao2024verifi} &\filledcircle &\emptycircle & \filledcircle  &\emptycircle  	&\filledcircle & \emptycircle &\emptycircle       	  & \filledcircle  \\
			TAPE (Ours)	     &\emptycircle & \filledcircle & \emptycircle  & \filledcircle  	&\filledcircle  &\filledcircle & \filledcircle       	  & \filledcircle  \\
			\bottomrule[1pt]
	\end{tabular}}
	\begin{tabbing}
		\filledcircle: the auditing method is applicable; 
		\emptycircle: the auditing method is not applicable.
	\end{tabbing}
	\vspace{-2mm}
\end{table*}

\section{Additional Related Work Discussion} \label{different_with_existing}

\subsection{Machine Unlearning in the Web-Related Studies} 
Machine unlearning--the process of efficiently removing specific data influences from trained models--has been explored in diverse applications across Web-based systems, such as graph-based systems and personalized applications \citep{lin2024incentive,pan2023unlearning,zhu2023heterogeneous,wu2023gif}. In graph-based systems, \citeauthor{pan2023unlearning} \citep{pan2023unlearning} proposed an unlearning method to unlearn the graph classifiers with limited access to the original data, and \citeauthor{wu2023gif} \citep{wu2023gif} introduced a general strategy leveraging influence functions to efficiently remove specific graph data while preserving model integrity. In personalized applications, \citeauthor{lin2024incentive} \citep{lin2024incentive} introduced dynamic client selection with incentive mechanisms to enhance the federated unlearning efficiency, while \citep{zhu2023heterogeneous} extended federated unlearning to the heterogeneous knowledge graph, aiming to balance both privacy and model utility preservation. To achieve a better unlearning service, \cite{liu2024breaking} further explored the challenge of balancing privacy, utility, and efficiency and proposed a controllable unlearning framework to overcome this challenge.

\subsection{Difference from Existing Studies}  
Our TAPE approach is significantly different from existing unlearning verification methods \cite{hu2022membership,sommer2022athena,guo2023verifying,gao2024verifi} in terms of the involving processes, auditing data type, unlearning scenarios, and unlearning methods, as depicted in \Cref{overview_of_auditing_method}. First, the significant difference is that the auditing of our method only involves the unlearning process, while the backdoor-based methods must involve both the original training and unlearning processes to ensure the service model first learns the backdoor. Second, most existing auditing methods are based on backdooring techniques and need to backdoor or mark samples for verification \cite{hu2022membership,sommer2022athena,guo2023verifying,gao2024verifi}. As we analyzed in the above subsection, they can only validate the backdoored samples and are only applicable to the exact unlearning methods as exact unlearning methods guarantee the deletion from the dataset level. Our method does not mix any other data to the training dataset, and the auditing is based on the posterior difference, which is suitable for genuine samples in both exact and approximate unlearning methods. Third, backdoor-based auditing methods are only feasible for multi-sample unlearning scenarios because just using a single sample makes it hard to backdoor the model \cite{wang2019neural,lin2020composite,zeng2023narcissus,nguyen2020input}, hence failing to provide unlearning verification for a single sample. 



%We also note that TAPE shares similarities with some studies investigating privacy leakage caused by the model updated difference \cite{salem2020updates,chen2021machine,balle2022reconstructing,Hu2024sp}. They aimed to extract as much private information as possible from model differences. However, it is well-known that while these methods are effective for single-sample reconstruction, they are less effective for multi-sample reconstruction. We have one advantageous difference to ensure we can provide a more effective information reconstruction that is suitable for unlearning auditing for multi-samples. Specifically, the unlearning verification user knows the unlearned samples, as users specify these samples, while the settings in \cite{salem2020updates,chen2021machine,balle2022reconstructing,Hu2024sp} have no information about the inferred samples. With the knowledge of the unlearned samples, we can design the posterior augment methods to facilitate the unlearning auditing for multiple samples.



 

\section{MLaaS Scenario and Threat Model} \label{threat_model}

Our problem is introduced in a simple machine unlearning as a service (MLaaS) scenario for ease of understanding. Under the MLaaS scenario, there are two main entities involved: an ML server that collects data from users, trains models, and provides the ML service, and users that contribute their data for ML model training. 

\noindent
\textbf{The ML Server's Ability.}
To uphold the ``right to be forgotten'' legislation and establish a privacy-protecting environment, the ML server is responsible for conducting machine unlearning operations. However, it is challenging to audit the unlearning effect for users to confirm that the unlearning is processed and prevent the spoof of unlearning from the ML server. In alignment with common unlearning verification settings \cite{hu2022membership,guo2023verifying}, we assume the ML server is honest for learning training but may spoof users for unlearning, i.e., it reliably hosts the learning process but may deceive users during unlearning operations by pretending unlearning has been executed when it has not. It is reasonable for the ML server to pretend to execute unlearning operations to avoid the degradation of model utility. Moreover, this assumption is more plausible than assuming the server will forge an unlearning update~\cite{thudi2022necessity}. Forging an unlearning update would require the server to simulate the disappearance of specified data and the corresponding resulting in model utility degradation, which demands significant effort without any benefit, making it an unlikely motivation. 

\noindent
\textbf{The Unlearning Users' Ability.}
We consider the scenario where the unlearning user has only black-box access to the ML service model, which is one of the most challenging scenarios \cite{salem2020updates,Hu2024sp}. In unlearning scenarios, the unlearning user possesses a local dataset, including the erased samples, which constitutes the entire training dataset for the ML service model \cite{warnecke2024machine,hu2024eraser}; however, the user has no access to the entire dataset. This just allows the user to query the model with their own data in a black-box access to obtain the corresponding posteriors and design the unlearning requests with specific data for unlearning verification purposes. 
Furthermore, we assume the unlearning user knows the unlearning algorithms, which is confirmed by both server and users, commonly used in other works \cite{hu2023duty}. However, even if the unlearning user knows the algorithms, without the remaining dataset, the user still cannot achieve the corresponding unlearning results of most unlearning algorithms. To relax the difficulty, we consider the unlearning user to be able to establish the same ML model as the current target ML service model with respect to model architecture. This can be achieved through model hyperparameter stealing attacks \cite{wang2018stealing,Seong2018towards,salem2020updates}. The unlearning user leverages this knowledge to simulate the unlearned shadow models and mimic the behavior of the ML service model based on the designed unlearning requests, thereby deriving the posterior differences necessary for training the reconstruction model to evaluate the unlearning effectiveness. % \Cref{fast_g}.   

 
 
 \section{Unlearning Data Perturbation (UDP) Algorithm} \label{UDP_algorithm}
 
 
 \Cref{Unlearned_d_p} demonstrates how to use the R restarts to find the satisfied perturbation for the unlearning data to augment the posterior difference for auditing.
 
 \begin{algorithm}[t]
 	%\small
 	\caption{Unlearning Data Perturbation (UDP)} \label{Unlearned_d_p}
 	\begin{small} % small, normalsize
 		\BlankLine
 		\KwIn{Trained model $\theta^*$, reconstruction model $\texttt{AE}$, unlearned data $X_u$, perturbation limit $\alpha$, local dataset $D_{local}$ }
 		\KwOut{The perturbed unlearning data, $X_u' = X_u + \Delta^{p}$} %, and three metrics for verification
 		\SetNlSty{}{}{} % This line removes the vertical line before the for-loop
 		\SetKwFunction{UDP}{\textbf{UDP}}
 		\SetKwProg{Fn}{procedure}{:}{end procedure}
 		\SetNlSty{}{}{} % This line removes the vertical line before the for-loop
 		\Fn{\UDP{$\theta^*$, $\texttt{AE}$, $X_u$, $\alpha$, $D_{local}$ }}{
 			\For{$r \gets 1$ \KwTo $R$ restarts}{
 				$\Delta^{p}_{r} \gets \mathcal{N}(0,1)$  \hspace{4mm}    $\rhd$ Initialize random perturbation. \\
 				\For{$i \gets 1$ \KwTo $m$ optimization steps}{
 					$X_{u,i}^p \gets X_u + \Delta^{p}_r$  \hspace{0mm} $\rhd$ Add the perturbation to data. \\
 					$\theta_{\backslash (X_{u,i}^p)} \gets \theta^* - \frac{\epsilon}{n-1} \nabla \ell (X_{u,i}^p;\theta^*)$ $\rhd$  According to \Cref{shadow_model}.  \\
 					$\delta_{u,i}^p \gets \theta^*(D_{local}) - \theta_{\backslash (X_{u,i}^p)}(D_{local})$ $\rhd$ Calculate posterior difference according to \Cref{posterior_diff}.  \\
 					$\nabla \mathcal{L}_{\texttt{AE}} \gets \nabla \mathcal{L}_{\texttt{AE}} (\texttt{AE}(\delta_{u,i}^p) , X_{u,i}^p)$ $\rhd$  According to \Cref{perturb_loss}. \\			
 					$\Delta^{p}_{r} \gets \Delta^{p}_{r} - \eta \nabla \mathcal{L}_{\texttt{AE}} (\texttt{AE}(\delta_{u,i}^p) , X_{u,i}^p) $   \hspace{2mm} $\rhd$ Update perturbation with limitation $\|\Delta^{p}_{r}\|_{\infty} \leq \alpha $. \\
 				}
 			}
 			Choose the optimal $\Delta^p_r$ with minimal value in $\mathcal{L}_{\texttt{AE}}$ as $\Delta^{p*}$.\\
 			\Return $X_u' = X_u + \Delta^{p*}$
 		}
 	\end{small}
 \end{algorithm}
 


 \iffalse

\section{Proof of \Cref{first_order}} \label{proof_of_theorem_1}

\iffalse
The changes in the model parameters can be expand using the perturbation theory \cite{avrachenkov2013analytic} as:
\begin{equation} \label{expanding_loss}
	\Delta \theta = \theta^{\epsilon}_{D \backslash D_u} - \theta^* = \mathcal{O}(\epsilon)\theta^{(1)} + \mathcal{O}(\epsilon^2)\theta^{(2)} + \mathcal{O}(\epsilon^3)\theta^{(3)} + \cdot \cdot \cdot,
\end{equation}
where each unlearning sample in $D_u$ is up-weighted by a factor of $\epsilon$.  $\theta^{(1)}$ denotes the first-order (in $\epsilon$) perturbation and $\theta^{(2)}$ is the second-order model perturbation.
\fi

%\subsection{Proof of \Cref{first_order}} 
\begin{proof}
	We provide a derivation of the first-order model difference approximation $\Delta \theta \simeq \frac{\epsilon}{n-m} \sum_{x_u \in D_u} \nabla \ell (x_u; \theta^*)$ in \Cref{first_order}. %in the context of loss minimization (M-estimation).
	We define that $\theta^*$ minimizes the empirical risk: 
	\begin{equation}
		R(\theta) \overset{\text{def}}{=} \frac{1}{n} \sum_{x_i \in D} \ell (x_i;\theta),
	\end{equation}
	where $n$ is the size of the training dataset $D$.
	We assume that $R$ is strictly twice-differentiable and convex in $\theta$, thus, we can positively define 
	\begin{equation}
		H_{\theta^*} \overset{\text{def}}{=} \nabla^2 R(\theta^*) = \frac{1}{n} \sum_{x_i \in D} \nabla^2_{\theta} \ell (x_i;\theta).
	\end{equation}
	When removing an unlearning dataset $D_u$ with size $m$, $\theta^{\epsilon}_{D \backslash D_u}$ will be the optimal parameter set for the interpolated loss function $\mathcal{L}^{\epsilon}_{D \backslash D_u}(\theta)$, as shown in \Cref{loss_of_unlearning}. Due to the first-order stationary condition, we have
	\begin{equation}\label{nabla_loss_of_unlearning}
		\begin{aligned}
			0 = \nabla \mathcal{L}_{D \backslash D_u}^{\epsilon} (\theta^{\epsilon}_{D \backslash D_u})& =  \nabla \mathcal{L}_{\emptyset}(\theta^{\epsilon}_{D \backslash D_u}) \\
			&+\frac{1}{n} ( - \tilde{\epsilon} \sum_{x \in D \backslash D_u} +\epsilon \sum_{x \in D_u}) \nabla \ell (x;\theta^{\epsilon}_{D \backslash D_u}).
		\end{aligned}
	\end{equation}
	Let $\theta^{\epsilon}_{D \backslash D_u}$ denote the optimal parameters for $\mathcal{L}^{\epsilon}_{D \backslash D_u}$ minimization, and $\theta^*$ denote the optimal parameters trained on $D$. The changes in the model parameters can be expand using the perturbation theory \cite{avrachenkov2013analytic} as:
	\begin{equation} \label{expanding_loss}
		\Delta \theta = \theta^{\epsilon}_{D \backslash D_u} - \theta^* = \mathcal{O}(\epsilon)\theta^{(1)} + \mathcal{O}(\epsilon^2)\theta^{(2)} + \mathcal{O}(\epsilon^3)\theta^{(3)} + \cdot \cdot \cdot,
	\end{equation}
	where each unlearning sample in $D_u$ is up-weighted by a factor of $\epsilon$.  $\theta^{(1)}$ denotes the first-order (in $\epsilon$) perturbation and $\theta^{(2)}$ is the second-order model perturbation. 
	
	The main idea is to use Taylor series for expanding $\nabla \mathcal{L}_{\emptyset}(\theta^{\epsilon}_{D \backslash D_u})$ around $\theta^*$ base on the perturbation series defined in \Cref{expanding_loss} and compare the terms of the same order in $\epsilon$:
	\begin{equation} \label{expanding_loss_delta}
		\nabla \mathcal{L}_{\emptyset}(\theta^{\epsilon}_{D \backslash D_u}) = \nabla \mathcal{L}_{\emptyset}(\theta^*) + \nabla^2 \mathcal{L}_{\emptyset}(\theta^*)(\theta^{\epsilon}_{D \backslash D_u} - \theta^*) + \cdot \cdot \cdot.
	\end{equation}
	Similarly, we can also expand $\nabla \ell (x;\theta^{\epsilon}_{D \backslash D_u})$ around $\theta^*$ using Taylor series expansion. To derive $\theta^{(1)}$, we expand \Cref{nabla_loss_of_unlearning} and compare the terms with coefficient $\mathcal{O}(\epsilon)$: 
	\begin{equation}
		\begin{aligned}
			&\epsilon \nabla^2 \mathcal{L}_{\emptyset}(\theta^*) \theta^{(1)} \\
			&=\frac{1}{n} (  \tilde{\epsilon} \sum_{x \in D \backslash D_u} - \epsilon \sum_{x \in D_u})  \nabla \ell (x;\theta^*) \\
			&= \tilde{\epsilon}  \nabla \mathcal{L}_{\emptyset}(\theta^*)  - \frac{1}{n} (  \tilde{\epsilon}  + \epsilon)  \sum_{x\in D_u} \nabla \ell (x;\theta^*) \\
			& = -\frac{1}{n} (  \tilde{\epsilon}  + \epsilon  )  \sum_{x\in D_u}  \nabla \ell (x;\theta^*) \\
			& = -\frac{1}{n-m} \epsilon   \sum_{x\in D_u}  \nabla \ell (x;\theta^*).
		\end{aligned}
	\end{equation}
	$\theta^{(1)}$ is the first-order approximation of the group influence function. $\epsilon \in [-1,0]$ is used for unlearning.
\end{proof}
%(1 - )L(x;\theta)  (1)L(x;\theta)

\fi 

\section{Additional Results for $T=1$, imperfect information case}
\label{sec: appendix2}
\subsection{Bias and Mean Shift Theoretical Results}
The bias for the naive estimator $\hat{\theta}_0^n$ is given by
\begin{equation*}
	\bias_0^n = p_0(|\alpha| - \alpha).
\end{equation*}
For the performative estimator $\hat{\theta}_0^\ast$, the bias is 
\begin{equation*}
	\bias_0^\ast= (1 - \alpha) \mathbb{E}[\hat{\theta}_0^\ast] - (1 - |\alpha|) p_0.
\end{equation*}

For the naive estimator $\hat{\theta}_0^n$ the mean shift is  
\begin{equation*}
	\shift_1^n = p_0(\alpha - |\alpha|) = - \bias_0^n,
\end{equation*}
and for the performative estimator $\hat{\theta}_0^\ast$, we have
\begin{equation*}
\shift_1^\ast = \alpha \E[\hat{\theta}_0^\ast] - |\alpha|p_0.
\end{equation*}

\subsection{Proof of Lemma \ref{lemma: bias-variance}}
\begin{proof}
Using conditional expectation we have
\begin{align*}
\E[(\theta_t - z)^2 \mid \theta_t, p_t^{test}] &= \E[\theta_t^2 - 2 \theta_t  z + z^2 \mid  \theta_t, p_t^{test}]\\
&= \theta_t^2 - 2 \theta_t \E[z \mid \theta_t, p_t^{test}] + \E[z^2 \mid \theta_t, p_t^{test}]\\	
&= \theta_t^2 - 2 \theta_t p_t^{test} + \frac14\\
&= (\theta_t^2 - p_t^{test})^2 + \frac14 - (p_t^{test})^2
\end{align*}
%\begin{align*}
%\E[(\theta_0 - z)^2 ] &= \E[\E[ \theta_0^2 - 2 \theta_0 z_0 + z_0^2 | \theta_0]]\\
%&= \E[\theta_0^2 - 2\theta_0 \E[z_0 | \theta_0] + \E[z_0^2 | \theta_0]]\\
%&= \E[\theta_0^2 - 2 \theta_0 \E[z_0 | \theta_0] + (\E[z_0 | \theta_0])^2 - (\E[z_0 | \theta_0])^2 + \E[z_0^2 | \theta_0]]\\
%&= \E[ (\theta_0 - \E[z_0 | \theta_0])^2] + \E[\E[z_0^2 | \theta_0] - (\E[z_0 | \theta_0])^2]\\
%&= \E[(\theta_0 - p_1(\theta_0))^2] + \E[\mathbb{V}[z_0 | \theta_0]]
%\end{align*}
\end{proof}

\subsection{Expected Loss Theoretical Results} \label{sec: expected_loss_additional}
Before computing the expected loss, we first show the following result regarding the first two moments of the performative estimator: 

\begin{lemma}[Moments of the Performative Estimator] \label{lemma: moments}
	For the performative estimator $\hat{\theta}_0^\ast$, we have that the first two moments are given by
\begin{equation*}
\E[\hat{\theta}_0^\ast] = 
\begin{cases}
\frac{(1 - |\alpha|) p_0}{1-2\alpha} & \alpha \in (-1, 0]\\
\frac12 - F_{m, p_0 + \frac12} (\frac{m}{2}) & \alpha \in [0.5, 1)\\
\sum_{x \in I} \big(\frac{1 - \alpha}{1-2\alpha}\big) \big(\frac{x}{m} - \frac12 \big) p(x) & \\
+ \frac12  - \frac12 F_{m, p_0 + \frac12}\big( \frac{2 - 3\alpha}{2-2\alpha}m \big) & \\
 - \frac12 F_{m, p_0 + \frac12}\big( \frac{\alpha }{2-2\alpha}m\big)  & \alpha \in (0, 0.5)
\end{cases}
\end{equation*}
and 
\begin{equation*}
\E[(\hat{\theta}_0^\ast)^2] = 
\begin{cases}
\big( \frac{1-|\alpha|}{1-2\alpha}\big)^2 \big( \frac{0.25 - p_0^2}{m} + p_0^2 \big) & \alpha \in (-1, 0]\\
\frac14 & \alpha \in [0.5, 1)\\
\sum_{x \in I} \big(\frac{1 - \alpha}{1-2\alpha}\big)^2 \big(\frac{x}{m} - \frac12 \big)^2 p(x)  & \\
+ \frac14  - \frac14 F_{m, p_0 + \frac12}\big( \frac{2 - 3\alpha}{2-2\alpha}m \big) & \\
+ \frac14 F_{m, p_0 + \frac12}\big( \frac{\alpha }{2-2\alpha}m\big)  & \alpha \in (0, 0.5) 
\end{cases}
\end{equation*}
where $I$ is the set of integers in $\big(\frac{\alpha m}{2-2\alpha}, \frac{(2-3\alpha)m}{2-2\alpha} \big]$, $F_{m, p_0 + \frac12} (x) := \sum_{k=0}^{\lfloor x \rfloor} p(x),$ and 
\begin{equation*}
p(x) := \binom{m}{x} \bigg(\frac12 + p_0\bigg)^x \bigg(\frac12 - p_0\bigg)^{m-x}
\end{equation*}
\end{lemma}
\begin{proof}[Proof of lemma \ref{lemma: moments}]
Recall that $\hat{\theta}_0^\ast$ is given by
\[
        \prm^*_0 =
        \begin{cases}
            \clip\Par[\big]{\frac{(1 - \abs{\alpha}) \overline{p_0}}{1 - 2 \alpha},
            -\frac{1}{2}, \frac{1}{2}}, & 1 - 2 \alpha > 0,\\
            \sign(\overline{p_0}) / 2, & 1 - 2 \alpha \le 0.
        \end{cases}
\]
We consider three cases for the value of $\alpha$: 
\begin{enumerate}[(i)]
\item $\alpha \in (-1, 0]$

In this case we have 
\begin{equation*}
	\prm^\ast_0 = \frac{1 - |\alpha|}{1-2\alpha} \overline{p_0}
\end{equation*}
and therefore 
\begin{equation*}
	\E[\prm^\ast_0] = \frac{1 - |\alpha|}{1-2\alpha} \overline{p_0}, \quad \E[(\prm^\ast_0)^2] = \bigg( \frac{1 - |\alpha|}{1-2\alpha}\bigg)^2\E[\overline{p_0}^2] = \bigg( \frac{1 - |\alpha|}{1-2\alpha}\bigg)^2\bigg( p_0^2 + \frac{\frac14 - p_0^2}{m}\bigg),
\end{equation*}
where we have used that $p_{0, i} \sim D_0$ for $i=1, \dots, m$, and thus $p_{0,i} + \frac12$ follows a Bernoulli distribution with parameter $p_0 + \frac12$. 

\item $\alpha \in [0.5, 1)$

In this case, we have that 
\begin{equation*}
\prm^\ast_0 = 
\begin{cases}
\frac12 & \overline{p_0} \ge 0\\
-\frac12 & \overline{p_0} <  0.
\end{cases}
\end{equation*}
Since $\overline{p_0} = \overline{q} - \frac12$, where $\overline{q} := \frac{1}{m}\sum_{i=1}^m q_i$ and $q_i := p_{0, i}$, so that $q_i \sim Bern(p_0 + \frac12)$, we know that the events can be written as
\begin{align*}
	\{ \overline{p_0} \ge 0 \} = \{ \overline{q} \ge 0.5 \}, \quad \{ \overline{p_0} < 0 \} = \{ \overline{q} < 0.5 \}.
\end{align*}
Therefore 
\begin{align*}
\prm^\ast_0 = \frac12 \chi_{\{ \overline{q} \ge 0.5 \}} - \frac12 \chi_{\{ \overline{q} < 0.5 \}}.
\end{align*}
Finally, using the law of total expectation, we get that
\begin{align*}
\E[\prm^\ast_0] &= \E[\prm^\ast_0 | \overline{q} \ge 0.5] \mathbb{P}[\overline{q} \ge 0.5] + \E[\prm^\ast_0 |\overline{q} < 0.5] \mathbb{P}[\overline{q} < 0.5]\\
&= \frac12 \mathbb{P}[\overline{q} \ge 0.5] - \frac12 \mathbb{P}[\overline{q} < 0.5]\\
&= \frac12 - F_{m, p_0 + \frac12}(0.5m),
\end{align*}
where we have used that $m \overline{q} \sim Bin(m ,p_0 + 0.5)$. Similarly for the second moment 
\begin{align*}
\E[(\prm^\ast_0)^2] &= \E[(\prm^\ast_0)^2 | \overline{q} \ge 0.5] \Pr[\overline{q} \ge 0.5] + \E[(\prm^\ast_0)^2 |\overline{q} < 0.5] \Pr[\overline{q} < 0.5]\\
&= \frac14 \Pr[\overline{q} \ge 0.5] + \frac14 \Pr[\overline{q} < 0.5]\\
&= \frac14.
\end{align*}
\item $\alpha \in (0, 0.5)$

In this case we have 
\begin{align*}
\prm^\ast_0 &= 
\begin{cases}
\frac{1 - \alpha}{1-2\alpha} \overline{p_0}, & \text{if } \overline{p_0} \in \big(- \frac{1-2\alpha}{2 - 2\alpha}, \frac{1-2\alpha}{2 - 2\alpha} \big] =: A\\
\frac12 , & \text{if } \overline{p_0} > \frac{1-2\alpha}{2 - 2\alpha} =: B\\
-\frac12 , & \text{if } \overline{p_0} \le - \frac{1-2\alpha}{2 - 2\alpha} =: C
\end{cases}
\end{align*}
where we have denoted by $A,B,C$ the random events that we have not clipped the value of the performative estimator, that we have clipped it from above or that we have clipped in from below. Using the law of total expectation, we have 
\begin{align*}
\E[\prm^\ast_0] &= \E[\prm^\ast_0 | A] \Pr[A] + \E[\prm^\ast_0 | B] \Pr[B] + \E[\prm^\ast_0 | C] \Pr[C]\\
&= \E[\prm^\ast_0 \chi_{A}] + \frac12 \Pr[B] - \frac12 \Pr[C]\\
&= \E[\prm^\ast_0 \chi_{A}] + \frac12 \Pr\bigg[\overline{q} > \frac{2-3\alpha}{2-2\alpha}\bigg] - \frac12 \Pr\bigg[\overline{q} \le \frac{\alpha}{2-2\alpha}\bigg]
\end{align*}
The first term can be computed as follows
\begin{align*}
\E[\prm^\ast_0 \chi_{A}] &= \sum_{x \in I} \frac{1 - \alpha}{1-2\alpha} \bigg( \frac{x}{m} - \frac12 \bigg) p(x),
\end{align*}
where we have used that $m\overline{p_0} + m/2 \sim Bin(m, p_0 + \frac12)$ and have denoted by $p(x)$ the PMF of $Bin(m, p_0 + \frac12$. The last two terms are easily expressed via the CDF of the same distribution, giving us that 
\begin{align*}
\E[\prm^\ast_0] = \sum_{x \in I} \bigg(\frac{1 - \alpha}{1-2\alpha}\bigg) \bigg(\frac{x}{m} - \frac12 \bigg) p(x) 
+ \frac12  - \frac12 F_{m, p_0 + \frac12}\bigg( \frac{2 - 3\alpha}{2-2\alpha}m \bigg)
 - \frac12 F_{m, p_0 + \frac12}\bigg( \frac{\alpha }{2-2\alpha}m\bigg).
\end{align*}
where $I$ is the set of integers in the interval $( \frac{\alpha }{2 - 2\alpha}m, \frac{2-3\alpha}{2 - 2\alpha} m]$. Similarly, for the second moment we have that 
\begin{align*}
\E[(\prm^\ast_0)^2] &= \E[(\prm^\ast_0)^2 | A] \Pr[A] + \E[(\prm^\ast_0)^2 | B] \Pr[B] + \E[(\prm^\ast_0)^2 | C] \Pr[C]\\
&= \E[(\prm^\ast_0)^2 \chi_{A}] + \frac14 \Pr[B] + \frac14 \Pr[C]\\
&= \E[(\prm^\ast_0)^2 \chi_{A}] + \frac14 \Pr\bigg[\overline{q} > \frac{2-3\alpha}{2-2\alpha}\bigg] - \frac12 \Pr\bigg[\overline{q} \le \frac{\alpha}{2-2\alpha}\bigg]\\
&= \sum_{x \in I} \bigg(\frac{1 - \alpha}{1-2\alpha}\bigg)^2 \bigg(\frac{x}{m} - \frac12 \bigg)^2 p(x)  
+ \frac14  - \frac14 F_{m, p_0 + \frac12}\bigg( \frac{2 - 3\alpha}{2-2\alpha}m \bigg) 
+ \frac14 F_{m, p_0 + \frac12}\bigg( \frac{\alpha }{2-2\alpha}m\bigg),
\end{align*}
which finishes the proof.
\end{enumerate}
\end{proof}

We now present a result that generalizes Theorem \ref{theorem: expected_loss}, offering theoretical insights for all possible values of $\alpha \in (-1,1)$.
\begin{theorem} \label{theorem: expected_loss_full}
For the naive estimator $\hat{\theta}_0^n$ the expected loss is 
\begin{equation*}
\E_{z \sim D_1^{test}}[(\hat{\theta}_0^n - z)^2] = p_0^2 (2 |\alpha| - 2\alpha - 1) + \frac{(\frac12 - p_0)(\frac12 + p_0)}{m} + \frac14,
\end{equation*}
and for the performative estimator $\hat{\theta}_0^\ast$, we have 
\begin{align*}
\E[(\hat{\theta}_0^\ast - z)^2] = 
\begin{cases}
\frac{(1 - |\alpha|)^2}{1-2\alpha} \bigg( \frac{\frac14 - p_0^2}{m} - p_0^2 \bigg) + \frac14 & \alpha \in (-1, 0]\\
p_0 (1 - |\alpha|) \big(2F_{m, p_0 + \frac12}(\frac{m}{2}) - 1\big) + \frac{1-\alpha}{2} & \alpha \in [0.5, 1)\\
\sum_{x \in I}\left( (1 - 2\alpha)g(x)^2 - 2(1 - |\alpha|)g(x) \right) p(x) + (p_0(1-|\alpha|) - \frac{1-2\alpha}{4})F_{m, p_0 + \frac{1}{2}}\left( \frac{2 - 3\alpha}{2 - 2\alpha}m \right) & \\
 + (p_0(1-|\alpha|) + \frac{1-2\alpha}{4})F_{m, p_0 + \frac{1}{2}}\left( \frac{\alpha m}{2 - 2\alpha} \right) - p_0 (1-|\alpha|) - \frac{1-\alpha}{2}& \alpha \in (0, 0.5)
\end{cases}
\end{align*}
Asymptotically, we have that as $m \to \infty$
\begin{equation*}
\E[(\hat{\theta}_0^\ast - z)^2] \to \verb|loss|_0^\ast
\end{equation*}
i.e. as $m$ goes to infinity, $\hat{\theta}_0^\ast$ approaches the optimal estimator for the risk minimisation problem. 
\end{theorem}
%\subsection{Proof of Lemma \ref{lemma: optimal_theta_1}}
%\begin{lemma}\label{lemma: optimal_theta_1}
%	Suppose that $p_0, \alpha \ge 0$ are known. Then the solution to the one-step performative risk minimisation problem 
%	\begin{equation*}
%	\min_{\theta \in [0, 1]} \E_{z \sim Bern(p_1(\theta))}[(\theta - z)^2]
%	\end{equation*}
%	is given by
%	\begin{equation*}
%	\theta^\ast(p_0) = 
%	\begin{cases}
%        \frac{-\alpha + (2 - 2\alpha)p_0}{2-4\alpha}, &\text{\normalfont if } A\\
%        1, &\text{\normalfont if }  B\\
%        0, &\text{\normalfont if } C
%    \end{cases}
%	\end{equation*}
%	where the events $A, B, C$ are defined as
%	\begin{align*}
%		A &= \{ \alpha < \frac12, p_0 \in[\frac{\alpha}{2(1-\alpha)}, \frac{2-3\alpha}{2-2\alpha}]  \}\\
%		B &= \{ \alpha < \frac12, p_0 \ge \frac{2-3\alpha}{2-2\alpha} \}  \cup \{ \alpha \in [\frac12, 1), p_0 \ge \frac12 \} \\
%		C &= \{ \alpha < \frac12, p_0 \le \frac{\alpha}{2(1-\alpha)} \} \cup  \{ \alpha \in [\frac12, 1), p_0 < \frac12 \} 
%	\end{align*}
%\end{lemma}
%\begin{proof}
%	Since the value of $p_0$ is known, any data we observe should not affect the value of $\theta$. Thus we can consider $\theta$ as a constant here and the optimisation problem from lemma \ref{lemma: bias-variance} becomes
%	\begin{equation*}
%		\min_{\theta \in [0, 1]} (1-2\alpha)\theta^2 + (\alpha - 2(1-\alpha)p_0) \theta + (1-\alpha)p_0.
%	\end{equation*}
%
%    We consider three possible cases: $(i)$ $\alpha = \frac12$, $(ii)$ $\alpha < \frac12$, and $(iii)$ $\alpha > \frac12$. 
%    \begin{enumerate}[(i)]
%        \item If $\alpha = \frac12$, then the minimisation problem becomes
%        \begin{equation}
%            \min_{\theta \in [0,1]} \bigg(\frac12 - p_0\bigg) \theta + \frac{p_0}{2}
%        \end{equation}
%        The solution to this is 
%        \begin{equation}
%            \theta^\ast = 
%            \begin{cases}
%                0, & \text{if } p_0 \le \frac12 \\
%                1, & \text{if } p_0 \ge \frac12.
%            \end{cases}
%        \end{equation}
%        \item If $\alpha < \frac12$, then the objective function is convex, and since $\theta$ is a constant in $[0, 1]$, we have the following optimisation problem:
%        \begin{align*}
%            \min_{\theta} \quad & (1-2\alpha)\theta^2 + (\alpha - 2(1-\alpha) p_0) \theta + (1-\alpha)p_0\\
%            s.t. \quad & \theta \le 1 \quad \text{and} \quad -\theta \le 0
%        \end{align*}
%        The Lagrangian is given by 
%        \begin{equation}
%            \mathcal{L} := (1-2\alpha)\theta^2 + (\alpha - 2(1-\alpha) p_0) \theta + (1-\alpha)p_0 + \lambda_1 (\theta - 1) - \lambda_2 \theta
%        \end{equation}
%        Using the KKT conditions, we know that $\theta^\ast$ is an optimal solution to the minimisation problem if the following system is satisfied
%        \begin{align*}
%            \frac{\partial \mathcal{L}}{\partial \theta} &= 0\\
%            \lambda_1 (\theta - 1) &= 0\\
%            \lambda_2 \theta &= 0
%        \end{align*}
%        where $\lambda_1 \ge 0, \lambda_2 \ge 0$. We divide this into three cases:
%    
%    \begin{enumerate}
%        \item $\theta^\ast = 1, \lambda_2 = 0$. In this case we have 
%        \begin{align*}
%            \frac{\partial \mathcal{L}}{\partial \theta} &= 2(1-2\alpha) + (\alpha - 2(1-\alpha)p_0) + \lambda_1 = 0 \implies \\
%            \lambda_1 &= 2(1-\alpha)p_0 - \alpha) - 2(1-2\alpha)
%        \end{align*}
%        This is non-negative if and only if 
%        \begin{equation*}
%            p_0 \ge \frac{2-3\alpha}{2-2\alpha}.
%        \end{equation*}
%        \item $\theta^\ast = 0, \lambda_1 = 0$. In this case we have
%        \begin{align*}
%            \frac{\partial \mathcal{L}}{\partial \theta} &= (\alpha - 2(1-\alpha)p_0) - \lambda_2 = 0 \implies \\
%            \lambda_2 &= (\alpha - 2(1-\alpha)p_0)
%        \end{align*}
%        We have that $\lambda_2$ is non-negative if and only if 
%        \begin{equation*}
%            p_0 \le \frac{\alpha}{2-2\alpha}.
%        \end{equation*}
%        \item $\lambda_1 = 0, \lambda_2 = 0$. In this case we have 
%        \begin{align*}
%            \frac{\partial \mathcal{L}}{\partial \theta} &= 2(1-2\alpha)\theta + (\alpha - 2(1-\alpha)p_0) = 0 \implies \\
%            \theta^\ast &= \frac{2(1-\alpha)p_0 - \alpha}{2 - 4\alpha}
%        \end{align*}
%        Since this needs to be in $[0, 1]$, this solution is feasible only when 
%        \begin{equation*}
%            \frac{\alpha}{2 - 2\alpha} \le p_0 \le \frac{2-3\alpha}{2-2\alpha}
%        \end{equation*}
%    \end{enumerate}
%    Thus we have the following solution in the $\alpha < \frac12$ case:
%    \begin{equation}
%    \theta^\ast = 
%        \begin{cases}
%            1, & p_0  \ge \frac{2-3\alpha}{2-2\alpha}, \alpha < \frac12\\
%            0, & p_0 \le \frac{\alpha}{2-2\alpha}, \alpha < \frac12\\
%            \frac{2(1-\alpha)p_0 - \alpha}{2 - 4\alpha}, & \frac{\alpha}{2 - 2\alpha} \le p_0 \le \frac{2-3\alpha}{2-2\alpha}, \alpha < \frac12
%        \end{cases}
%    \end{equation}
%    \item When $\alpha > \frac12$, the objective function is concave and thus the minimum is obtained at one of the boundary points, i.e. we have 
%    \begin{equation}
%        \theta^\ast = 
%        \begin{cases}
%            0, & \text{if }p_0 \le \frac12, \alpha \ne 1\\
%            1, & \text{if }p_0 \ge \frac12, \alpha \ne 1\\
%        \end{cases}
%    \end{equation}
%    where if $p_0 = \frac12$ both values for $\theta$ achieve the same optimal solution for the objective. Finally, if $\alpha = 1$, then both values $0$ and $1$ are again optimal for $\theta$. 
%    \end{enumerate}
%\end{proof}
\subsection{Proof of Theorem \ref{theorem: expected_loss_full}}
\begin{proof}[Proof of Theorem \ref{theorem: expected_loss}]
We begin by rewriting the expected loss as follows 
%\begin{lemma}[Expected loss] \label{lemma: expected loss}
%For a deployed model $\theta_0$, the expected loss is given by
%\begin{align*}
%		&\E[(\theta_0 - z)^2] = (1-2\alpha)\E[\theta_0^2] - 2 (1 - |\alpha|)p_0 \E[\theta_0] + \frac14,
%	\end{align*}
%	where the expectation is only in terms of the randomness of the observations $\{p_{0, i} \}_{i=1}^m$.
%\end{lemma}
%\begin{proof}[Proof of lemma \ref{lemma: expected loss}]
\begin{align*}
\E[(\theta_0 - z_0)^2] &= \E[\E[\theta_0^2 - 2\theta_0 z_0 + z_0^2 | \theta_0]]\\
&= \E[\theta_0^2 - 2 \theta_0 \E[z_0 | \theta_0] + \E[z_0^2 | \theta_0]]\\
&= \E\bigg[\theta_0^2 - 2 \theta_0 p_1(\theta_0) + \frac14\bigg]\\
&=(1-2\alpha)\E[\theta_0^2] - 2 (1 - |\alpha|)p_0 \E[\theta_0] + \frac14
\end{align*}
where the expectation is only in terms of the randomness of the observations $\{p_{0, i} \}_{i=1}^m$.
%\end{proof}
For the naive estimator, $\hat{\theta}_0^n$, we have that the first two moments are 
\begin{align*}
\E[\hat{\theta}_0^n] & = p_0\\
\E[(\hat{\theta}_0^n)^2] &= p_0^2 + \frac{(\frac12 - p_0)(\frac12 + p_0)}{m},
\end{align*} 
which follows since $p_{0,i} \sim D_0$ for $i = 1, \dots, m$. Therefore, we get
\begin{align*}
\E[(\hat{\theta}_0^n - z)^2] &=  
(1-2\alpha)\E[\theta_0^2] - 2 (1 - |\alpha|)p_0 \E[\theta_0] + \frac14\\
&= p_0^2 (2 |\alpha| - 2\alpha - 1) + \frac14 + \frac{(2\alpha - 1)(4 p_0^2 - 1)}{4m}
\end{align*}

For the performative estimator, we use the first and second moments of $\hat{\theta}_0^\ast$ from lemma \ref{lemma: moments} to obtain
\begin{align*}
\E[(\hat{\theta}_0^\ast - z)^2] = 
\begin{cases}
\frac{(1 - |\alpha|)^2}{1-2\alpha} \bigg( \frac{\frac14 - p_0^2}{m} - p_0^2 \bigg) + \frac14 & \alpha \in (-1, 0]\\
p_0 (1 - |\alpha|) \big(2F_{m, p_0 + \frac12}(\frac{m}{2}) - 1\big) + \frac{1-\alpha}{2} & \alpha \in [0.5, 1)\\
\sum_{x \in I}\left( (1 - 2\alpha)g(x)^2 - 2(1 - |\alpha|)g(x) \right) p(x) + (p_0(1-|\alpha|) - \frac{1-2\alpha}{4})F_{m, p_0 + \frac{1}{2}}\left( \frac{2 - 3\alpha}{2 - 2\alpha}m \right) & \\
 + (p_0(1-|\alpha|) + \frac{1-2\alpha}{4})F_{m, p_0 + \frac{1}{2}}\left( \frac{\alpha m}{2 - 2\alpha} \right) - p_0 (1-|\alpha|) - \frac{1-\alpha}{2}& \alpha \in (0, 0.5)
\end{cases}
\end{align*}
where $g(x) \defeq (\frac{1-\alpha}{1-2\alpha})(\frac{x}{m} - \frac{1}{2})$.

Asymptotically, as $m \to \infty$, we have that the moments of $\hat{\theta}_0^\ast$ for $\alpha \in (-1, 0]$ are given by:
\begin{align*}
\E[\hat{\theta}_0^\ast] &= \frac{(1-|\alpha|)}{1-2\alpha} p_0 \to \frac{(1-|\alpha|)}{1-2\alpha} p_0\\
\E[(\hat{\theta}_0^\ast)^2] &= \frac{(1-|\alpha|)^2}{(1-2\alpha)^2} \bigg( \frac{0.25 - p_0^2}{m} +  p_0^2\bigg) \to \frac{(1-|\alpha|)^2}{(1-2\alpha)^2} p_0^2
\end{align*}
Similarly, for $\alpha \in [0,5, 1)$, we have 
\begin{align*}
\E[\hat{\theta}_0^\ast] &= \frac12 - F_{m, p_0 + \frac12}\bigg(\frac{m}{2}\bigg) \to \frac{sign(p_0)}{2}\\
\E[(\hat{\theta}_0^\ast)^2] &= \frac14 \to \frac14
\end{align*}
where we have used that the CDF function $F_{m, p_0 + \frac12}\bigg(\frac{m}{2}\bigg)$ converges to $1$ for non-negative $p_0$ and to $0$ for negative $p_0$ as $m\to \infty$. 

Finally, for $\alpha \in (0, 0.5)$, we have that 
\begin{align*}
	\E[\hat{\theta}_0^\ast] &= \E\bigg[\clip{\bigg( \frac{1 - |\alpha| p_0}{1-2\alpha}, -\frac12, \frac12 \bigg)}\bigg]\\
	&= \E\bigg[ \frac{(1-|\alpha|)\overline{p_0}}{1-2\alpha}  \chi_{\{\overline{p_0} \in A\}} \bigg] + \frac12 \Pr[\overline{p_0} \in B  ] - \frac{1}{2} \Pr[\overline{p_0} \in C ]\\
	&\to \frac{(1-|\alpha|){p_0}}{1-2\alpha}  \chi_{\{{p_0} \in A\}} + \frac12 \chi_{[{p_0} \in B  ]} - \frac{1}{2} \chi_{[{p_0} \in C ]}\\
	&= \E[{\theta}_0^\ast \mid \alpha \in (0, 0.5)].
\end{align*}
where $A$ denotes the region (a function of $\alpha$), where $\hat{\theta}_0^\ast$ has not been clipped, $B$ represents the region where it has been clipped from above, and $C$ is the region where it has been clipped from below. The third line follows from: (1) the law of large numbers, which ensures that $\overline{p_0} \to p_0$ almost surely as $m \to \infty$, and (2) the dominated convergence theorem. The same argument applies for $\E[(\hat{\theta}_0^\ast)^2]$. Thus, combining this with the other two cases for $\alpha$, we get the following asymptotic results 
\begin{align*}
\lim_{m \to \infty} \E[\hat{\theta}_0^\ast] = \theta_0^\ast, \quad \lim_{m \to \infty} \E[(\hat{\theta}_0^\ast)^2] = (\theta_0^\ast)^2.
\end{align*}
Therefore, we can conclude that as $m\to \infty$,
\begin{equation*}
\E[(\hat{\theta}_2^\ast - z)^2] \to \verb|loss|_0^\ast.
\end{equation*}
\end{proof}



\end{document}
\endinput
%%
%% End of file `sample-sigconf.tex'.
