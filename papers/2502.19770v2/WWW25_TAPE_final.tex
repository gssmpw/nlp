%%
%% This is file `sample-sigconf.tex',
%% generated with the docstrip utility.
%%
%% The original source files were:
%%
%% samples.dtx  (with options: `all,proceedings,bibtex,sigconf')
%% 
%% IMPORTANT NOTICE:
%% 
%% For the copyright see the source file.
%% 
%% Any modified versions of this file must be renamed
%% with new filenames distinct from sample-sigconf.tex.
%% 
%% For distribution of the original source see the terms
%% for copying and modification in the file samples.dtx.
%% 
%% This generated file may be distributed as long as the
%% original source files, as listed above, are part of the
%% same distribution. (The sources need not necessarily be
%% in the same archive or directory.)
%%
%%
%% Commands for TeXCount
%TC:macro \cite [option:text,text]
%TC:macro \citep [option:text,text]
%TC:macro \citet [option:text,text]
%TC:envir table 0 1
%TC:envir table* 0 1
%TC:envir tabular [ignore] word
%TC:envir displaymath 0 word
%TC:envir math 0 word
%TC:envir comment 0 0
%%
%% The first command in your LaTeX source must be the \documentclass
%% command.
%%
%% For submission and review of your manuscript please change the
%% command to \documentclass[manuscript, screen, review]{acmart}.
%%
%% When submitting camera ready or to TAPS, please change the command
%% to \documentclass[sigconf]{acmart} or whichever template is required
%% for your publication.
%%
%%
\documentclass[sigconf]{acmart}


\usepackage{amsmath,amsfonts}

\usepackage{amsthm}

\newtheorem{defn}{\textbf{Definition}}
\newtheorem*{prob_state}{\textbf{Problem Statement}}
\newtheorem*{prob_reform}{\textbf{Problem Reformulation}}
\newtheorem*{problem}{\textbf{Problem}}
\newtheorem{theorem}{\textbf{Theorem}}
\newtheorem{assumption}{\textbf{Assumption}}
\newtheorem{proposition}{\textbf{Proposition}}
\newtheorem{corollary}{\textbf{Corollary}}
\newtheorem{lemma}{\textbf{Lemma}}

\usepackage{float}


\usepackage{color}
\usepackage{tikz}
\usetikzlibrary{trees}
\usepackage{amsthm}
\usepackage{makecell}
\usepackage{tikz}
\usetikzlibrary{trees}
\usepackage{tabularx,colortbl}
\usepackage{threeparttable}
\usepackage{booktabs}
\usepackage{multirow}
\usepackage{subfig}
\captionsetup[subfloat]{listofformat=parens}
\usepackage{graphicx}
%\usepackage{subfigure}
\usepackage{url}
\usepackage{enumitem}
\usepackage{tcolorbox}
\usepackage{xcolor}
%\usepackage{algorithm}
%\usepackage{algorithmic}
\usepackage[ruled, vlined, linesnumbered]{algorithm2e}
\usepackage{hyperref}
\usepackage[noabbrev]{cleveref}
\crefname{equation}{Eq.}{Eqs.}


\newcommand{\filledcircle}{\tikz\fill[black] (0,0) circle (.8ex);}
\newcommand{\emptycircle}{\tikz\draw (0,0) circle (.8ex);}


\definecolor{DeepPink}{HTML}{FF1493}
\definecolor{Orchid}{HTML}{DA70D6}
\definecolor{Magenta}{HTML}{FF00FF}
\definecolor{Fuchsia}{HTML}{FF00FF}
\definecolor{LavenderPink}{HTML}{FFB6C1}
\definecolor{verylightgray}{rgb}{0.9, 0.9, 0.9}
\definecolor{lightred}{rgb}{1,0.8,0.8}


%%
%% \BibTeX command to typeset BibTeX logo in the docs
\AtBeginDocument{%
  \providecommand\BibTeX{{%
    Bib\TeX}}}

%% \BibTeX command to typeset BibTeX logo in the docs \AtBeginDocument{%  \providecommand\BibTeX{{%	Bib\TeX}}}

\copyrightyear{2025}
\acmYear{2025}
%% \setcopyright{cc}
%% \setcctype{CC-BY}
\setcopyright{rightsretained}
\acmConference[WWW '25]{Proceedings of the ACM Web Conference 2025}{April 28-May 2, 2025}{Sydney, NSW, Australia}
\acmBooktitle{Proceedings of the ACM Web Conference 2025 (WWW '25), April 28-May 2, 2025, Sydney, NSW, Australia}
%\acmDOI{10.1145/3696410.3714875}
%\acmISBN{979-8-4007-1274-6/25/04}

% The following includes the CC license icon appropriate for your paper.
% Download the image from www.scomminc.com/pp/acmsig/4ACM-CC-by-88x31.eps
% and place within your figs or figures folder

 
\makeatletter
\gdef\@copyrightpermission{
	\begin{minipage}{0.2\columnwidth}
		\href{https://creativecommons.org/licenses/by/4.0/}{\includegraphics[width=0.90\textwidth]{4ACM-CC-by-88x31.eps}}
	\end{minipage}\hfill
	\begin{minipage}{0.8\columnwidth}
		\href{https://creativecommons.org/licenses/by/4.0/}{This work is licensed under a Creative Commons Attribution International 4.0 License.}
	\end{minipage}
	\vspace{5pt}
}
\makeatother
 

%%
%% end of the preamble, start of the body of the document source.
\begin{document}

%%
%% The "title" command has an optional parameter,
%% allowing the author to define a "short title" to be used in page headers.
\title{TAPE: Tailored Posterior Difference for \\Auditing of Machine Unlearning}
\subtitle{ \it \textbf{ To appear at The Web Conference 2025. Author version} }

%%
%% The "author" command and its associated commands are used to define
%% the authors and their affiliations.
%% Of note is the shared affiliation of the first two authors, and the
%% "authornote" and "authornotemark" commands
%% used to denote shared contribution to the research.
 


\author{Weiqi Wang}
\email{Weiqi.Wang@uts.edu.a}
\orcid{0000-0002-7905-3126}
\affiliation{%
	\institution{University of Technology Sydney}
	%\department{School of Computer Science}
	\city{Sydney}
	\state{NSW}
	\country{Australia}}

\author{Zhiyi Tian}
\email{Zhiyi.Tian-1@uts.edu.au}
\authornote{Corresponding author: Zhiyi Tian. \\ \textbf{This paper is the draft that preparing for publishing on WWW25.}}
\orcid{0000-0001-8905-0941}
\affiliation{%
	\institution{University of Technology Sydney}
	%\department{School of Computer Science}
	\city{Sydney}
	\state{NSW}
	\country{Australia}
}

\author{An Liu}
\email{anliu@suda.edu.cn}
\orcid{0000-0002-6368-576X}
\affiliation{%
	\institution{Soochow University}
	\city{Soochow}
	\state{Jiangsu}
	\country{China}
}
\author{Shui Yu}
\email{shui.yu@uts.edu.au}
\orcid{0000-0003-4485-6743}
\affiliation{%
	\institution{University of Technology Sydney}
	%\department{School of Computer Science}
	\city{Sydney}
	\state{NSW}
	\country{Australia}
}




%%
%% By default, the full list of authors will be used in the page
%% headers. Often, this list is too long, and will overlap
%% other information printed in the page headers. This command allows
%% the author to define a more concise list
%% of authors' names for this purpose.
\renewcommand{\shortauthors}{WeiqiWang, Zhiyi Tian, An Liu, and Shui Yu}
%% No italics



%%
%% The abstract is a short summary of the work to be presented in the
%% article.
\begin{abstract}

% Recent works to jointly reconstruct 3D human and object from a single RGB image, are mostly model-based, that fail to capture the fine details of the clothed human body and object surface. In this paper, we introduce ReCHOR, a novel, model-free, first-method to produce realistic clothed human-object reconstructions from a monocular view. This is extremely challenging due to human-object occlusions, diverse interactions and depth ambiguity, as it needs to infer both 3D spatial awareness and high resolution details. Our core idea is based on estimating neural implicit representations for human and object respectively by an attention-based neural implicit model that attends to pixel-aligned features from both the global human-object image for spatial awareness and  the local separate view of human and object images for high quality details. Additionally, the network is conditioned on semantic features from an initial estimated human-object pose prior and a generative diffusion model that inpaints occluded regions, thus enabling the retrieval of details from them.
% We also propose a synthetic dataset with rendered scenes of diverse, inter-occluded 3D human and object scans, to train our network. We evaluate our method on the synthetic and real world BEHAVE dataset. Our experiments show that our method outperforms the SOTA in achieving realistic clothed human-object reconstructions.
Recent approaches to jointly reconstruct 3D humans and objects from a single RGB image represent 3D shapes with template-based or coarse models, which fail to capture details of loose clothing on human bodies. In this paper, we introduce a novel implicit approach for jointly reconstructing realistic 3D clothed humans and objects from a monocular view. For the first time, we model both the human and the object with an implicit representation, allowing to capture more realistic details such as clothing. This task is extremely challenging due to human-object occlusions and the lack of 3D information in 2D images, often leading to poor detail reconstruction and depth ambiguity. To address these problems, we propose a novel attention-based neural implicit model that leverages image pixel alignment from both the input human-object image for a global understanding of the human-object scene and from local separate views of the human and object images to improve realism with, for example, clothing details. Additionally, the network is conditioned on semantic features derived from an estimated human-object pose prior, which provides 3D spatial information about the shared space of humans and objects. To handle human occlusion caused by objects, we use a generative diffusion model that inpaints the occluded regions, recovering otherwise lost details. For training and evaluation, we introduce a synthetic dataset featuring rendered scenes of inter-occluded 3D human scans and diverse objects. Extensive evaluation on both synthetic and real-world datasets demonstrates the superior quality of the proposed human-object reconstructions over competitive methods.
\end{abstract}


%%
%% The code below is generated by the tool at http://dl.acm.org/ccs.cfm.
%% Please copy and paste the code instead of the example below.
%%
\begin{CCSXML}
	<ccs2012>
	<concept>
	<concept_id>10010520.10010553.10010562</concept_id>
	<concept_desc>Security and privacy;</concept_desc>
	<concept_significance>500</concept_significance>
	</concept>
	<concept>
	<concept_id>10010520.10010575.10010755</concept_id>
	<concept_desc>Computing methodologies~Machine learning</concept_desc>
	<concept_significance>300</concept_significance>
	</concept>
	</ccs2012>
\end{CCSXML}


\ccsdesc[500]{Security and privacy}
\ccsdesc[500]{Computing methodologies}
%%
%% Keywords. The author(s) should pick words that accurately describe
%% the work being presented. Separate the keywords with commas.
\keywords{Machine Unlearning, Data Privacy, Unlearning Auditing.}
%% A "teaser" image appears between the author and affiliation
%% information and the body of the document, and typically spans the
%% page.
 
 
%%
%% This command processes the author and affiliation and title
%% information and builds the first part of the formatted document.
\maketitle

\section{Introduction}
\label{sec:intro}
% Image editing methods in diffusion models depend on user-defined control directions - users can unlock their creativity using these methods by specifying the desired manipulation through prompts~\cite{gandikota2023concept}, reference images~\cite{ruiz2022dreambooth, kumari2022customdiffusion, gal2022image, chen2024trainingfreeregionalpromptingdiffusion}, or attribute vectors~\cite{parmar2023zero,hertz2022prompt}. In this work, we ask a fundamentally different question: \emph{Can we automatically discover the underlying visual structure of a concept within diffusion model's knowledge?} %Rather than requiring user-specified controls, we aim to decompose the model's internal knowledge into meaningful directions.

% This question touches on a fundamental limitation in how we interact with diffusion models. Current control methods ~\cite{zhang2023addingconditionalcontroltexttoimage, gandikota2023concept, ye2023ipadaptertextcompatibleimage,ye2023ipadaptertextcompatibleimage, hertz2024stylealignedimagegeneration, li2023photomaker, shi2024instantbooth, chen2024trainingfreeregionalpromptingdiffusion} require users to specify their desired manipulations in advance, limiting interactive creativity. This contrasts with natural human artistic workflows, where creators dynamically explore creative ideas while jointly refining them toward meaningful artistic outcomes~\cite{hoffmann2016modeling}. This synergy between specification and exploration is not new to generative models. Early GAN architectures naturally developed disentangled latent spaces that enabled continuous\cite{harkonen2020ganspace,radford2015unsupervised, wu2021stylespace, shen2020interfacegan}, compositional control over generated images. Users could explore these spaces to discover interesting variations that would be difficult to describe in words~\cite{wu2021stylespace}, then combine them to achieve their creative goals~\cite{grabe2022towards}. 


% While diffusion models have largely superseded GANs in conditional image synthesis~\cite{dhariwal2021diffusion},  their underlying structure remains less understood. Diffusion models achieve remarkable diversity through high-dimensional latents, unlike GANs' compact latent spaces.  With a single prompt, diffusion models can generate radically different variations through different random initializations of input noise. We ask - Is it possible to discover interpretable structure within this vast space of variations?

Text-to-image diffusion models are capable of generating remarkable visual variations from a single prompt through different random initializations. However, this vast creative potential remains largely opaque to users---while we can generate diverse images, we lack understanding of the underlying structure of these variations. This presents a fundamental challenge: how can we discover and expose the latent visual capabilities encoded within these models?

\let\thefootnote\relax \footnote{$^{*}$Correspondence to \texttt{gandikota.ro@northeastern.edu}}

The challenge touches on a key limitation in how we interact with diffusion models today. Current control methods require users to explicitly specify their desired edits in advance through prompts~\cite{gandikota2023concept}, reference images~\cite{zhang2023addingconditionalcontroltexttoimage, chen2024trainingfreeregionalpromptingdiffusion, ruiz2022dreambooth,kumari2022customdiffusion, Ryu_lora, hu2021lora}, or attribute vectors~\cite{ye2023ipadaptertextcompatibleimage, hertz2024stylealignedimagegeneration, li2023photomaker, shi2024instantbooth,parmar2023zero,hertz2022prompt}. That contrasts sharply with natural human creative workflows, where artists dynamically explore creative ideas and jointly refine them toward meaningful artistic outcomes~\cite{hoffmann2016modeling}. The need for pre-specified controls creates a barrier between users and the full creative potential of these models.

Interestingly, earlier generative models like GANs~\cite{gans,karras2019style,brock2018large} naturally developed more interpretable internal structures. Their compact latent spaces often exhibited emergent disentanglement~\cite{harkonen2020ganspace,radford2015unsupervised, wu2021stylespace, shen2020interfacegan}, enabling continuous and compositional control over generated images. Users could explore these spaces to discover interesting variations that would be difficult to describe in words~\cite{wu2021stylespace}, then combine them to achieve their creative goals~\cite{grabe2022towards}.

Diffusion models have largely superseded GANs in conditional image synthesis~\cite{dhariwal2021diffusion}, achieving greater diversity through much higher-dimensional latents. And yet an understanding of the underlying structure of these larger latent spaces has remained elusive. In this work, we ask a fundamental question: \emph{Can we automatically discover the visual structure within a diffusion model's knowledge of a concept?} Rather than requiring user-specified controls, we aim to decompose the model's internal representations into expressive directions that users can explore and combine.

To address these needs, we present \textbf{SliderSpace}, a framework that brings systematic explorability to diffusion models. Given just a text prompt, SliderSpace discovers a canonical set of meaningful, diverse, and controllable directions within the model's knowledge of that concept. Each direction is implemented as a low-rank adapter~\cite{hu2021lora} that can be scaled and composed with others, allowing users to explore and smoothly combine different aspects of variation, as shown in Figure~\ref{fig:intro}.

We ground SliderSpace discovery in three key requirements for meaningful decomposition of a diffusion model's visual manifold: 
\begin{enumerate}
    \item \textbf{Unsupervised Discovery:} The decomposition process should emerge from the intrinsic structure of the model's learned representation, rather than being guided by predefined attributes. This ensures we capture the true topology of the model's knowledge space rather than projecting our assumptions onto it.
    
    \item \textbf{Semantic Orthogonality:} Each discovered control must represent a distinct semantic direction. This is enforced in a semantic feature space, like CLIP, where every slider has an orthogonal effect in embeddings. This prevents discovering multiple controls that create similar semantic effects, making the system more efficient and easier.
    
    \item \textbf{Distribution Consistency:} Directions must induce consistent transformations across both random seeds and prompt variations. 
\end{enumerate}

These requirements naturally lead to our proposed framework, which we formalize in Section~\ref{sec:method}. As we show in our experiments, SliderSpace is architecture-agnostic, working with both conventional U-Net based models like Stable Diffusion~\cite{rombach2022high, rombach2022sd20, podell2023sdxl, turbo, dmd} and recent transformer-based architectures like Flux~\cite{flux}.

We demonstrate the expressiveness of SliderSpace through three applications: First, we show how SliderSpace can decompose high-level concepts into diverse and expressive components, revealing the natural axes of variation in the model's understanding. Second, we explore artistic style variation, where SliderSpace discovers directions that match or exceed the diversity of manually curated artist lists while being judged more useful by human evaluators. Finally, we show how SliderSpace can help reverse the mode collapse commonly observed in distilled diffusion models, restoring diversity while maintaining generation speed.

Beyond providing practical creative control, SliderSpace opens new avenues for understanding and utilizing the latent capabilities of diffusion models. By mapping these models' visual potential into intuitive, composable directions, we take a step toward making their creative possibilities more accessible and interpretable to users.

% Image editing methods in diffusion models unlock the creativity of users. In this work we ask an alternate question: \emph{Can we organize and expose what of the diffusion model is already capable of?}.
% Existing methods for controlling image generation typically require users to manually specify edit directions for desired changes. This process is time-consuming, requires technical expertise, and limits the spontaneity of the creative process. For instance, if a user wants to adjust the smile of a generated person, they must explicitly request this edit, often through imprecise prompt engineering or model fine-tuning. This approach of predefined controls or manual specifications restricts users from fully exploring the latent capabilities of the model. There may be interesting stylistic variations or attributes that the model can generate, but users have no easy way to discover or utilize these.

% Natural visual disentanglement was an emergent property in the latent space of Generative Adversarial Models (GANs) \cite{harkonen2020ganspace,radford2015unsupervised, wu2021stylespace, shen2020interfacegan}. In particular, it has been observed that StyleGAN~\cite{karras2019style} stylespace neurons offer detailed control over many meaningful aspects of images that would be difficult to describe in words~\cite{wu2021stylespace}. However, diffusion models do not share such a compact latent space~\cite{park2023unsupervised}; and efforts to uncover such a space in the semantic embeddings of the text conditioning have met with limited success \nik{Nick - is there a specific citation you were thinking about?}.

% In this work we introduce \textbf{SliderSpace}, which takes a step towards uncovering an analogous low dimensional representation of diffusion models' visual breadth; in essence treating the diffusion model as many generators sharing parameters, where a particular generator is defined by a specific prompt. For a given prompt we sample many random seeds (and optionally prompt expansions using an LLM), generate the corresponding images, and apply an off the shelf feature extractor (in this work CLIP, but our method can be applied to any differentiable feature extractor). We use PCA to analyze these features, and for each of the leading $k$ principal components we train a LoRA \cite{} which causes the diffusion model to produces images which increase the feature magnitude along that component when passed back through the same feature extractor. This leads to a 'Slider' for each principal component, because each LoRA can be scaled and applied to the original diffusion model, continuously varying those visual features in the generated results (as measured, in our case, by CLIP).

% There are many other works that enhance the controllability of diffusion models. One common approach is enabling users to add spatial constraints to a generation either manually, or via a reference image \cite{zhang2023addingconditionalcontroltexttoimage, chen2024trainingfreeregionalpromptingdiffusion}, a second is leveraging more abstract embeddings (e.g. identity, style) extracted from a reference image \cite{ye2023ipadaptertextcompatibleimage, hertz2024stylealignedimagegeneration, li2023photomaker, shi2024instantbooth}, a third is finetuning a foundation model to better generate a concept important to the user \cite{ruiz2022dreambooth, kumari2022customdiffusion, Ryu_lora, hu2021lora}, and a fourth (most relevant to this work) is finding low-rank adaptors of the model based on a prompt or small training set which can be scaled to provide continous control over one aspect of generated image (e.g. night vs day, basic vs luxury, etc.) \cite{gandikota2023concept}. SliderSpace is complementary to all of these methods and offers something distinct. All of the other methods we are aware require the user (and / or model designer) to know in advance what type of control they want. In contrast SliderSpace assists users in discovering and controlling hidden capabilities present in the diffusion model's distribution of possible generations.

%We propose that truly intuitive creative control in a text-to-image model should meet three key criteria: \emph{discoverability}, \emph{intuitiveness}, and \emph{specificity}. The model should reveal controllable attributes that may not be immediately obvious, offer controls that are easy to understand and manipulate, and ensure each control affects a distinct attribute of the generated image.

% We demonstrate the utility and power of SliderSpace using three applications built on top of SDXL-DMD \cite{dmd}, because its fast generation speed lends itself well to the continuous control offered by SliderSpace.

% First, we study concept decomposition (Section \ref{sec:concept_exp}), where we learn sliders for a specific concept (e.g. 'monster', 'waterfall', 'car'). Through quantitative metrics of diversity and text alignment we demonstrate that the learned sliders dramatically boost the diversity of generations when randomly applied without harming text alignment; we also ask humans to qualitatively judge these results in a user study where they find the SliderSpace results to be more 'Diverse', 'Useful', and 'Creative' than our baselines.

% Second, we attempt to compare the automatic discoveries of SliderSpace to a large scale manual study of artistic styles (Section \ref{sec:art_exp}), open-sourced by ParrotZone \cite{parrotzone}. In this study SDXL was prompted with over 4300 artist names,  and based on visual inspection the cases of successful stylistic mimicry recorded. Quantitatively SliderSpace more closely matches the distribution of artistic variation discovered by ParrotZone than other baselines, and in our user studies was judged to be significantly more 'Diverse' and 'Useful' than the baselines. To our surprise humans even judged SliderSpace results to be slightly more 'Diverse' than the results generated by the manually discovered artist names of \cite{parrotzone}.

% Third, we attempt to use SliderSpace to reverse the mode collapse commonly observed in distilled few-step diffusion models relative to the original teacher model (Section \ref{sec:diverse_exp}). We quantitatively demonstrate that applying SliderSpace to SDXL-DMD leads to more closely matching the distribution of images by the original teacher, SDXL.

%Through extensive experiments on various state-of-the-art text-to-image models, we demonstrate that SliderSpace significantly enhances user control and creative expression in AI-assisted image generation tasks. Our method enables a range of applications, including concept decomposition and control, diversity improvement in generated images, customization dissection and edits, and the exploration of artistic styles inherent in the model.

% SliderSpace goes beyond providing a practical tool for enhanced creative control. By mapping the visual potential of diffusion models it can open new avenues for generative creativity and deepens our understanding of each model's hidden potential.
\section{Related Work}
\label{sec:related_work}

The original investigation \cite{gibson1979ecological} on the relationship between visual perception and human action defines \emph{affordance} as the opportunities for interaction with the surrounding environment. Behavioral studies on regular and cognitively impaired persons have shown evidence that perception results in both visual and motor signals in the human brain. An extended study \cite{anderson2002attentional} shows that visual attention to the spatial characteristics of the perceived objects initiates automatic motor signals for different actions. In computer vision, human affordance learning involves novel pose prediction such that the estimated pose represents a valid human action within the scene context. The task is fundamental to many problems requiring robust semantic reasoning about the environment, such as human motion synthesis \cite{wang2021scene} and scene-aware human pose generation \cite{wang2017binge, roy2016multi, zhang2022inpaint, yao2023scene}.

Earlier methods of affordance learning have explored knowledge mining \cite{zhu2014reasoning} and multimodal feature cues \cite{roy2016multi} to address the problem. In \cite{zhu2014reasoning}, the authors use a Markov Logic Network for constructing a knowledge base by extracting several object attributes from different image and metadata sources, which can perform various downstream visual inference tasks without any additional classifier, including zero-shot affordance prediction. In \cite{roy2016multi}, the authors use depth map, surface normals, and segmentation map as multimodal cues to train a multi-scale convolutional neural network (CNN) for scene-level semantic label assignment associated with specific human actions. In \cite{do2018affordancenet}, the authors design a multi-branch end-to-end CNN with two separate pathways for object detection and affordance label assignment to achieve high real-time inference throughput. Researchers \cite{chuang2018learning} have also explored socially imposed constraints for affordance learning. In \cite{chuang2018learning}, the authors propose a graph neural network (GNN) to propagate contextual scene information from egocentric views for action-object affordance reasoning.

Probabilistic modeling of scene-aware human motion generation also involves semantic reasoning of human interaction with the environment. Initial works on human motion synthesis have taken different architectural approaches, such as sequence-to-sequence models \cite{barsoum2018hp}, generative adversarial networks (GAN) \cite{barsoum2018hp, cai2018deep, yang2018pose}, graph convolutional networks (GCN) \cite{yan2019convolutional}, and variational autoencoders (VAE) \cite{guo2020action2motion}. However, these methods have mostly ignored the role of environmental semantics. Due to potential uncertainty in human motion, in a recent approach \cite{wang2021scene}, the authors address such motion synthesis with a GAN conditioned on scene attributes and motion trajectory to predict probable body pose dynamics.

One key challenge of human affordance generation in 2D scenes is the lack of large-scale datasets with rich pose annotations. In \cite{wang2017binge}, the authors compile the only public dataset of annotated human body poses in complex 2D indoor scenes by extracting frames from sitcom videos. Aiming to generate a contextually valid human affordance at a user-defined location, the authors propose sampling the scale and deformation parameters for an existing human pose template using a VAE conditioned on the localized image patches as scene context. In \cite{zhang2022inpaint}, the authors introduce a two-stage GAN architecture for achieving a similar goal by estimating the affine bounding box parameters to localize a probable human in the scene and then generating a potential body pose at that location. The method uses the input scene, corresponding depth, and segmentation maps as semantic guidance. In \cite{yao2023scene}, the authors propose a transformer-based approach with knowledge distillation for generating human affordances in 2D indoor scenes.



\section{Preliminary and Problem Statement} \label{problem_df}

%In this section, we introduce the ML scenario and the threat model, then the unlearning auditing problem statement, and finally, the metrics and requirements for unlearning auditing.

%including the data removal verification and the unlearning effectiveness assessment, and finally, the definition of the solution scheme.

%



%\subsection{Machine Unlearning Auditing Problem}

To facilitate the understanding of the unlearning auditing problem, we first introduce the main process of unlearning. A detailed introduction about the 
 and threat model is presented in \Cref{threat_model}. 

\noindent
\textbf{Machine Unlearning.} The unlearning process usually includes the following phases. (1) The server trained a model with parameters $\theta_t$ derived from dataset $D$. (2) The unlearning user uploads the unlearning requested dataset $D_u$ to the server for unlearning. (3) The server conducts an unlearning algorithm $\mathcal{U}$ to remove $D_u$'s contribution from $\theta_t$ and results in an unlearned model with parameters $\theta_{u, D \backslash D_u}$, also denoted as $\theta_u$. 

Most existing backdoor-based unlearning verification methods tried to solve data removal verification but can only answer if the backdoored samples are unlearned. Answering whether the backdoored data is or not deleted is insufficient for trustworthy unlearning auditing. We should assess the unlearning effectiveness of the model, i.e., how much private information about the requested unlearning samples is removed from the model.

 
\begin{prob_state}[Unlearning Effectiveness Audit] 
	\label{effectiveness_problem}	
Given the described unlearning scenario, the potential for unlearning execution spoofing by the server, and the capabilities of the unlearning user, auditing unlearning effectiveness necessitates a method for unlearning users to evaluate the extent to which information about $D_u$ has been unlearned from $\theta_t$ to $\theta_{u}$.
\end{prob_state}

% 
% that the unlearning user develop a method to evaluate the extent to which information about $D_u$ has been unlearned from $\theta_T$ to $\theta_{U}$.

It is important to note that the problem statement inherently includes the issue of data removal verification. If one can effectively measure how much information related to the erased samples has been unlearned, this measurement can serve as the basis for determining whether the data has been properly unlearned. %, thereby addressing the data removal verification through an evaluation of unlearning effectiveness.
We try to conduct unlearning auditing based on the unlearning updated posterior difference as it contains essential information about the erased samples. To achieve the auditing goals, we need to mimic the unlearning posterior difference and extract and quantify the unlearned information from it. We utilize the model's output layer results of the original and unlearned models on the user's local dataset to generate the posterior difference. 
 %Then, we conduct the unlearning effectiveness audit based on this posterior difference. 
We define the unlearning posterior difference as follows.


\noindent
\textbf{Posterior Difference.}
The unlearning user first queries the trained ML model $\theta_t$ before unlearning with all samples of $D_{local}$ and concatenates the received outputs to form a vector $\hat{Y}_{t, local}$. Then, the user queries the unlearned model $\theta_u$ with samples in the $D_{local}$ and creates a vector $\hat{Y}_{u, local}$. In the end, the user sets the posterior difference, denoted by $\delta$, to the difference of both outputs: 
\begin{equation} \label{posterior_diff}
	%\small
	\delta = \hat{Y}_{t,local} - \hat{Y}_{u,local}.
\end{equation}
Note that the dimension of $\delta$ is the product of $D_{local}$'s cardinality and the number of classes of the target dataset. For example, in this paper, CIFAR-10 and MNIST are 10-class datasets, while we just identify the gender attributes of CelebA, which is a binary classification. As we set the local dataset $0.5\%$ of CIFAR-10 and MNIST, and $0.06\%$ of CelebA, this indicates the dimension of $\delta$ is 2500 for CIFAR-10, 3000 for MNIST, and 1210 for CelebA.



\begin{figure*}[t]
	\centering
	\includegraphics[width=0.97\linewidth]{Contents/Figures/reconstruction_attack}
	\vspace{-2mm}
	\caption{The main process of the TAPE method. (a) The first part quickly builds the unlearned shadow models through first-order influence estimation based on the user's local dataset $D_{local}$ to mimic the unlearning posterior difference $\delta$. (b) Two posterior difference augment strategies are proposed to make the reconstruction suitable for multi-sample unlearning. \vspace{-2mm}}
	\label{fig_reconstructionattack}
\end{figure*}

\noindent
\textbf{Unlearned Information Reconstruction to Assess How Much Information is Unlearned.}
To assess the unlearning effectiveness, we employ a reconstructor model to extract the unlearned information from the posterior difference. We employ the cosine similarity between the reconstructed and original unlearned samples to assess how much information of the unlearned information can be recovered from the unlearning update:  
\begin{equation} \label{similarity_eq}
	%\small
	\textbf{Rec. Similarity:} \hspace{12mm} \text{sim}(\hat{X}_{u}, X_u) = \frac{\hat{X}_{u} \cdot X_u}{ \| \hat{X}_{u} \| \cdot \| X_u \|}.
\end{equation}
Here, $\hat{X}_{u} \cdot X_u$ is the dot product of the reconstructed vectors $\hat{X}_{u}$ and original unlearned samples vectors $X_u$. $\| \hat{X}_{u}\|$ and $\|X_u\|$ are the Euclidean norms of the two vectors. A higher reconstruction similarity means more information about the erased samples is unlearned from the model.






\section{TAPE Methodology} \label{muv_method}

%Tailored Posterior Difference for Auditing of Machine Unlearning

\subsection{Overview of the TAPE}

%Based on the previously introduction for machine unlearning auditing, w

%We propose a solution that trans a Reconstructor using posterior differences of unlearning to extract the unlearned information. 
We illustrate the overview methodology process in \Cref{fig_reconstructionattack}, which includes two main steps. 

%we employ a reconstructor model to extract the unlearned information from the posterior difference.

\noindent
\textbf{Unlearned Shadow Model Building.} In this step, we propose a method to quickly build the unlearned shadow models with only the user's samples. Our method utilizes the first-order influence estimation function to effectively estimate the unlearning influence and remove it from the original model, thus quickly mimicking the unlearned model to generate the posterior differences. 


\noindent
\textbf{Reconstructor Training.} We then train a reconstructor model to evaluate how much information about the erased samples is unlearned. An unlearned data perturbation strategy and an unlearned influence-based division strategy are proposed to augment the posterior differences for reconstruction for multiple samples. Both strategies are implemented utilizing the advantage that the unlearning user knows and prepares the unlearned samples.


%Existing privacy reconstructing methods~\cite{Hu2024sp,salem2020updates,balle2022reconstructing} are only suitable for recovering a single sample but fail to recover multiple samples for a model or posterior difference. 

%We address this limitation by proposing 

%, because the user specifies these unlearned samples as unlearning requests.


 
\subsection{Constructing Unlearned Shadow Model to Mimic Posterior Difference} \label{fast_g}


The unlearning user possesses a local dataset $D_{local}$, including the unlearned data $D_u$. %, which was once used to train the ML service model. Now, the user wants to unlearn $D_u$ from the ML service model and verify the unlearning effectiveness. 
As the unlearning verification is executed on the unlearning user side, the user can utilize the local dataset $D_{local}$ to construct unlearned shadow models to mimic posterior differences.  
%Only VBU \cite{nguyen2020variational} can implement unlearning with solely the unlearned samples; however, it is only suitable for Bayesian models.

\noindent
\textbf{Constructing Unlearned Shadow Model.} Many existing machine unlearning algorithms rely on the assistance of the remaining dataset $D \backslash D_u$. We propose a method based on the influence function theory \cite{koh2017understanding,basu2020second,bae2022if} in ML to quickly approximate an unlearned shadow model with only the unlearned data $D_u$. Specifically, when we remove $D_u$ from a trained model $\theta_t$ for unlearning, the empirical risk minimization (ERM) can be written as:
\begin{equation}
	\small
	\mathcal{L}_{D \backslash D_u}(\theta) = \frac{1}{n - m} \sum_{x \in D \backslash D_u} \ell (x; \theta),
\end{equation}
where $n$ is the size of $D$, $m$ is the size of $D_u$, and $\ell(x;\theta)$ is the loss. 

Similar to \cite{basu2020second}, we evaluate the effect of up-weighting a group of training samples on model parameters. Note that in this case, the updated weights must still form a valid distribution. Specifically, if a group of training samples is up-weighted, the weights of the remaining samples should be down-weighted to preserve the sum to one constraint of weights in the ERM formulation. 
We assume that the weights of samples in $D_u$ have been up-weighted all by $\epsilon$ and use $\frac{m}{n}$ to denote the fraction of up-weighted training samples. This results in a down-weighting of the rest of the training data by $\tilde{\epsilon} = \frac{m}{n-m} \epsilon$, to preserve the empirical weight distribution of the training dataset. Then, the ERM can be translated as:
\begin{equation} \label{loss_of_unlearning}
	\mathcal{L}^{\epsilon}_{D \backslash D_u} (\theta) = \frac{1}{n} (\sum_{x \in D \backslash D_u} (1 - \tilde{\epsilon})\ell(x;\theta) + \sum_{x \in D_u}(1+\epsilon)\ell(x;\theta)).
\end{equation}
In the above equation, if $\epsilon=0$, we get the original loss function $\mathcal{L}_{\emptyset}(\theta)$ (none of the training data points are unlearned) and if $\epsilon=-1$, we get the loss function $\mathcal{L}_{D \backslash D_u}(\theta)$ (specified samples are removed).

%\begin{equation}
%\theta^{\epsilon}_{D \backslash D_u} = \arg \min_{\theta} \mathcal{L}^{\epsilon}_{D \backslash D_u} (\theta),
%\end{equation}
%where

Let $\theta^{\epsilon}_{D \backslash D_u}$ denote the optimal parameters for $\mathcal{L}^{\epsilon}_{D \backslash D_u}$ minimization, and $\theta^*$ denote the optimal parameters trained on $D$. The unlearned shadow models can be approximately achieved by removing the estimated data influence from the trained model as follows,
\begin{equation} \label{shadow_model}
	\theta^{\epsilon}_{D \backslash D_u}  =  \theta_t  -  \frac{\epsilon}{n-m} \sum_{x_u \in D_u} \nabla \ell (x_u; \theta_t),
\end{equation}
where $\epsilon \in [-1,0]$ is used for unlearning, $m$ is the size of the erased dataset and $n$ is the size of the training dataset. $\Delta \theta \simeq  - \frac{\epsilon}{n-m} \sum_{x_u \in D_u} \nabla \ell (x_u; \theta_t)$ is the estimaed data influence at current trained model $\theta_t$. We omit the proof of the shadow model estimation in \Cref{shadow_model} as it is similar to the proofs in \cite{koh2017understanding,basu2020second}. Constructing the unlearned shadow model based on \Cref{shadow_model} only relies on the unlearned samples and is convenient for the user to implement. 

%%We present the proof of the data influence estimation in \Cref{first_order} at Appendix~\ref{proof_of_theorem_1}. 
%Calculating the model difference using the above approximation method has been demonstrated to be much more efficient than the naive retraining method.  During the practical training process, we employ the stochastic estimation method to further speed up the Hessian matrix calculation following \cite{koh2017understanding}. 

\iffalse
\begin{theorem}[Unlearned Shadow Model based on First-order Influence Estimation] 
	\label{first_order}
	When computing the influence of a few samples, such as $m$ samples, the scaling of the first-order $\theta^{(1)}$ is $\frac{m}{n}$, while the scaling of the second-order coefficient is $\frac{m^2}{n^2}$, which is very small when $n$ is large. Thus, the second-order term can be ignored, and the data influence at current model $\theta^*$ can be estimated as $\Delta \theta \simeq  - \frac{\epsilon}{n-m} \sum_{x_u \in D_u} \nabla \ell (x_u; \theta^*)$. Then unlearned shadow model can be achieved based on the estimation as 
	
	
\end{theorem}
\fi 
 


\noindent
\textbf{Mimicking Posterior Difference.} 
With the above method to construct unlearned shadow models, then, we can easily achieve the mimicked posterior differences. For instance, assuming the local dataset $D_{local}$ contains $m$ samples $X_1, X_2, ..., X_m$, we can construct $m$ unlearned shadow models $\theta_{D \backslash X_1}, \theta_{D \backslash X_2}, ..., \theta_{D \backslash X_m}$ for each sample, where $\backslash X$ means unlearning the sample $X$. Based on these unlearned shadow models, we can mimic the corresponding unlearned posterior, $\hat{Y}_{\backslash X_1, local}, \hat{Y}_{\backslash X_2, local}, ..., \hat{Y}_{\backslash X_m, local}$, using the local dataset. The posterior difference can be calculated through \Cref{posterior_diff}, denoted as $\delta_1, \delta_2, ..., \delta_m$, as shown in \Cref{fig_reconstructionattack}. Together with the corresponding shadow unlearning set's ground truth information, the training data for the reconstructor model to evaluate the unlearned information is derived.



%We can achieve the best auditing effect if we use the flattened parameters of the service model difference. However, using all the parameters as input for the Verifier and Reconstructor would be computationally expensive for large models. We design the multi-task information bottleneck (MT-IB) structure and only choose the difference of the informative representation layer to replace the whole model difference to improve the scalability. 


 
 

\subsection{Reconstructor Model Training with Two Strategies for Multiple Samples Auditing} \label{two_s}
 
%\subsubsection{} \label{recons}

\noindent
\textbf{Reconstructor Training for Unlearning Effectiveness Assessment.} Like \citep{salem2020updates}, we employ the autoencoder ($\texttt{AE}$) architecture to construct the Reconstructor, which includes an encoder and a decoder, as shown in \Cref{fig_reconstructionattack}(b). Its goal is to learn an efficient encoding for the posterior differences $\delta$. The encoder encodes the posterior difference into a latent vector $\mu$, and the decoder decodes the latent vector to reconstruct the unlearned samples. We employ mean squared error (MSE) as the loss function, $\mathcal{L}_{\texttt{AE}} = ||\hat{X_u} - X_u||^2_2,$ where $\hat{X_u} = \texttt{AE} (\delta_u)$ is the reconstructed sample for $X_u$. 

Existing studies \cite{salem2020updates, Hu2024sp, balle2022reconstructing} showed effective reconstruction for a single sample for the updated model difference. However, they are infeasible for reconstructing multiple samples. For unlearning effectiveness auditing, the unlearning user has the knowledge of the unlearned samples. With this advantage, we design two strategies: one augments posterior differences by perturbing unlearned data before unlearning and one augments posterior differences by individually dividing the posterior difference after unlearning, enabling evaluate how much information is unlearned for multiple samples. 


 

%\subsubsection{Unlearned Data Perturbation to Augment Posterior Difference} \label{data_Perturbation}

%Many unlearning methods directly compare the posterior difference between $\hat{Y}_{t, D_u}$ and $\hat{Y}_{u, D_u}$ to evaluate the unlearning effectiveness. Usually, they treat a degradation of the accurate prediction probability as a sign of successful unlearning for the unlearned model \cite{bourtoule2021machine,warnecke2024machine}. However, not all unlearning operations will cause a significant degradation of posterior probability on the unlearned samples. The model performance will remain, especially when there are some samples in the remaining dataset that are similar to the erased samples. It will also hinder the audit of unlearning effectiveness.  

\noindent
\textbf{Unlearned Data Perturbation before Unlearning.} 
We propose an unlearned data perturbation method to augment posterior difference, assisting the unlearning effectiveness verification. Specifically, we hope to introduce a perturbation $\Delta^p$ to the unlearned sample $X_u$ to augment the unlearned posterior for the reconstructor, so that it can effectively evaluate how much information is unlearned. At the same time, unlearning the perturbed specified data should maintain the unlearned model's utility on the remaining dataset. Since our final purpose is to improve the reconstructed information, we can formalize the unlearned data perturbation as follows to find the suitable perturbation. 
\begin{equation} \label{perturb_loss}
	\small
	\begin{aligned}
			&\min_{\Delta^p} \mathcal{L}_{\texttt{AE}} (\hat{X_u}', X_u + \Delta^p) \\
			&\text{s.t.} \hspace{3mm}  \Delta^p \in \arg \min_{\theta_{ \backslash (X_u + \Delta^p)}} \sum_{x \in D_r} \ell (x;\theta_{\backslash (X_u + \Delta^p)})
	\end{aligned}
\vspace{-2mm}
\end{equation}
where $\hat{X_u}' = \texttt{AE}(\delta_{\backslash (X_u + \Delta^p)})$, meaning that the samples are reconstructed based on the posterior difference that unlearns the perturbed data $X_u + \Delta^p$. We define the constraint that $ \Delta^p : \| \Delta^p \|_{\infty} \leq \alpha$ to ensure that the perturbed data will not be too different from the original data. We can combine these two losses together and treat them as two objectives, thus can be optimized with two-objective optimization methods \cite{sener2018multi,poirion2017descent,guo2020learning}. During the perturbation optimization process, we fix the trained model $\theta_t$ and the reconstruction model $\texttt{AE}$. We only update the perturbation $\Delta^p$ of $X_u$ to induce an augmented unlearned posterior difference $\delta_{\backslash (X_u + \Delta^p)}$, which improves the reconstruction effect. To find an effective perturbation, we can employ the restars technique, which is inspired from \cite{GeipingFHCT0G21,QinMGKDFDSK19}, and we provide the corresponding algorithm in \Cref{UDP_algorithm}. 

\iffalse
, and the corresponding algorithm is presented in \Cref{Unlearned_d_p}.


\begin{algorithm}[t]
	%\small
	\caption{Unlearning Data Perturbation (UDP)} \label{Unlearned_d_p}
	\begin{small} % small, normalsize
		\BlankLine
		\KwIn{Trained model $\theta^*$, reconstruction model $\texttt{AE}$, unlearned data $X_u$, perturbation limit $\alpha$, local dataset $D_{local}$ }
		\KwOut{The perturbed unlearning data, $X_u' = X_u + \Delta^{p}$} %, and three metrics for verification
		\SetNlSty{}{}{} % This line removes the vertical line before the for-loop
		\SetKwFunction{UDP}{\textbf{UDP}}
		\SetKwProg{Fn}{procedure}{:}{end procedure}
		\SetNlSty{}{}{} % This line removes the vertical line before the for-loop
		\Fn{\UDP{$\theta^*$, $\texttt{AE}$, $X_u$, $\alpha$, $D_{local}$ }}{
			\For{$r \gets 1$ \KwTo $R$ restarts}{
				$\Delta^{p}_{r} \gets \mathcal{N}(0,1)$  \hspace{4mm}    $\rhd$ Initialize random perturbation. \\
				\For{$i \gets 1$ \KwTo $m$ optimization steps}{
					$X_{u,i}^p \gets X_u + \Delta^{p}_r$  \hspace{0mm} $\rhd$ Add the perturbation to data. \\
					$\theta_{\backslash (X_{u,i}^p)} \gets \theta^* - \frac{\epsilon}{n-1} \nabla \ell (X_{u,i}^p;\theta^*)$ $\rhd$  According to \Cref{shadow_model}.  \\
					$\delta_{u,i}^p \gets \theta^*(D_{local}) - \theta_{\backslash (X_{u,i}^p)}(D_{local})$ $\rhd$ Calculate posterior difference according to \Cref{posterior_diff}.  \\
					$\nabla \mathcal{L}_{\texttt{AE}} \gets \nabla \mathcal{L}_{\texttt{AE}} (\texttt{AE}(\delta_{u,i}^p) , X_{u,i}^p)$ $\rhd$  According to \Cref{perturb_loss}. \\			
					$\Delta^{p}_{r} \gets \Delta^{p}_{r} - \eta \nabla \mathcal{L}_{\texttt{AE}} (\texttt{AE}(\delta_{u,i}^p) , X_{u,i}^p) $   \hspace{2mm} $\rhd$ Update perturbation with limitation $\|\Delta^{p}_{r}\|_{\infty} \leq \alpha $. \\
				}
			}
			Choose the optimal $\Delta^p_r$ with minimal value in $\mathcal{L}_{\texttt{AE}}$ as $\Delta^{p*}$.\\
			\Return $X_u' = X_u + \Delta^{p*}$
		}
	\end{small}
\end{algorithm}
\fi 


%\subsubsection{Unlearned Influence-based Division}


\noindent
\textbf{Unlearning Influence-based Division after Unlearning.} 
The unlearning influence-based division strategy utilizes the convenient properties of the first-order data influence estimation. After achieving the overall posterior differences for multiple samples $\delta_{\backslash D_u}$, the user can quickly estimate the basic data influence for each integrated sample $x_u \in D_u$, and we divide the overall posterior difference according to the weight of each sample's influence. We assume the divided posterior difference of the integrated sample obeys a Gaussian distribution:
%\vspace{-2mm}
\begin{equation} 
	\vspace{-2mm}
	\small
	\begin{aligned}
		\delta_{\backslash x_u}  \sim \mathcal{N}(\frac{\delta_{\backslash D_u} }{ \sum_{x_u \in D_u} \nabla  \ell (x_u;\theta_t)} \cdot \nabla  \ell (x_u;\theta_t), \sigma^2), \\
		\text{s.t.} \   \delta_{\backslash D_u} = \sum_{x_u \in D_u}  \delta_{\backslash x_u} ,
	\end{aligned}
%\vspace{-1mm}
\end{equation}
where we keep the divided posterior difference values as the mean and add a random deviation to it; meanwhile, we keep the sum of all the split slice posterior differences $\sum_{x_u \in D_u}  \delta_{\backslash x_u}$ equal to the overall posterior difference $\delta_{\backslash D_u}$. Thus, we ensure every reconstruction has a unique divided posterior difference without additional change or noise in the original $\delta_{\backslash D_u}$. This operation changes the reconstruction task for multiple samples based on the same $\delta_{\backslash D_u}$ as reconstructing every single sample of $D_u$ based on multiple divided posterior differences.  




\iffalse 
\vspace{2mm}
\noindent
\textbf{Masking.} Masking is mainly used to improve the reconstruction quality for multiple samples. This is because training a qualified Reconstructor for unlearning effectiveness assessment is much more difficult than training a Verifier for data removal verification based on the model difference. Although a regular autoencoder performs well on some simple datasets, it is hard to reconstruct a clear image on complex datasets. Compared with \cite{salem2020updates}, our advantage is that the server can access the full training data samples, enabling us to implement the masking strategy. 
Inspired by MAE \cite{he2022masked}, we randomly mask 70\% pixels of the specified-unlearned samples and assemble the masked samples with the model difference as the input for Reconstructor training. 
\begin{equation}
	\hat{X}_u = \text{Reconstructor}(\texttt{mask}(X_u) + \Delta_z \theta_{x_u})
\end{equation}
This strategy combines the core idea of unlearning difference reconstruction training and MAE, improving the reconstruction quality significantly.




\subsubsection{Verifier Training for Data Removal Verification} \label{verifier_describe}



Different from existing works that infer if samples are unlearned in a black-box setting \cite{chen2021machine,kurmanji2024towards,lu2022label}, we utilize the white-box access of the server to construct a more powerful Verifier to validate the unlearning execution. With access to the full training dataset, we directly construct the verification training data, $(\Delta \theta_{a, unl.}, Y_{unl.})$ as the positive samples and $(\Delta \theta_{a, rem.}, Y_{rem.})$ as the negative samples, to train the Verifier model. $\Delta \theta_{a, unl.}$ are the simulated model differences where the unlearned model is trained using the remaining dataset without the unlearned samples. 
And $\Delta \theta_{a, rem.}$ are the simulated model difference where the unlearned model is trained using the remaining dataset still containing the erased samples. We treat them as comparative samples and train a classifying model, Verifier, so that if the model difference of new unlearning request $\Delta \theta_{x_u}$ comes, we can infer if it has been unlearned.  
\fi 



\section{Experimental Results}
\begin{table*}[t]
\centering
\caption{Total Variation Distance on CIFAR-10-LT ($N_l = 500$, $M_l = 4000$) with different class imbalance ratios $\gamma_l$ and $\gamma_u$ under five different unlabeled class distributions.}
\label{tab:cifar10-tv}
\resizebox{\textwidth}{!}{
\begin{tabular}{lccccccccccc}
\toprule
& & \multicolumn{2}{c}{consistent} & \multicolumn{2}{c}{uniform} & \multicolumn{2}{c}{reversed} & \multicolumn{2}{c}{middle} & \multicolumn{2}{c}{head-tail} \\
\cmidrule(lr){3-4} \cmidrule(lr){5-6} \cmidrule(lr){7-8} \cmidrule(lr){9-10} \cmidrule(lr){11-12}
& & $\gamma_l = 150$ & $\gamma_l = 100$ & $\gamma_l = 150$ & $\gamma_l = 100$ & $\gamma_l = 150$ & $\gamma_l = 100$ & $\gamma_l = 150$ & $\gamma_l = 100$ & $\gamma_l = 150$ & $\gamma_l = 100$ \\
Model & Estimator & $\gamma_u = 150$ & $\gamma_u = 100$ & $\gamma_u = 1$ & $\gamma_u = 1$ & $\gamma_u = 1/150$ & $\gamma_u = 1/100$ & $\gamma_u = 150$ & $\gamma_u = 100$ & $\gamma_u = 150$ & $\gamma_u = 100$ \\
\midrule
Supervised & MLLS & 0.269 ± 0.252 & 0.038 ± 0.006 & 0.251 ± 0.046 & 0.255 ± 0.060 & 0.429 ± 0.028 & 0.493 ± 0.050 & 0.333 ± 0.042 & 0.320 ± 0.009 & 0.457 ± 0.034 & 0.444 ± 0.043 \\
Supervised & RLLS & 0.043 ± 0.001 & 0.044 ± 0.010 & 0.348 ± 0.034 & 0.305 ± 0.068 & 0.769 ± 0.016 & 0.678 ± 0.028 & 0.430 ± 0.008 & 0.368 ± 0.013 & 0.539 ± 0.018 & 0.503 ± 0.020 \\
\midrule
MLE & IPW & 0.027 ± 0.001 & 0.027 ± 0.000 & 0.319 ± 0.072 & 0.243 ± 0.010 & 0.674 ± 0.020 & 0.646 ± 0.041 & 0.438 ± 0.020 & 0.454 ± 0.026 & 0.547 ± 0.049 & 0.491 ± 0.059 \\
MLE & OR & 0.045 ± 0.004 & 0.042 ± 0.000 & 0.215 ± 0.026 & 0.203 ± 0.032 & 0.433 ± 0.017 & 0.395 ± 0.033 & 0.193 ± 0.006 & 0.209 ± 0.037 & 0.307 ± 0.147 & 0.249 ± 0.130 \\
MLE & DR & 0.090 ± 0.002 & 0.079 ± 0.000 & 0.407 ± 0.027 & 0.360 ± 0.007 & 0.425 ± 0.007 & 0.421 ± 0.029 & 0.256 ± 0.001 & 0.286 ± 0.031 & 0.435 ± 0.136 & 0.362 ± 0.122 \\
\midrule
EM & IPW & 0.035 ± 0.002 & 0.040 ± 0.001 & 0.021 ± 0.001 & 0.029 ± 0.015 & 0.303 ± 0.187 & 0.091 ± 0.010 & 0.119 ± 0.011 & 0.105 ± 0.022 & 0.104 ± 0.026 & 0.104 ± 0.051 \\
EM & OR & 0.037 ± 0.003 & 0.042 ± 0.002 & 0.016 ± 0.001 & 0.024 ± 0.012 & 0.269 ± 0.183 & 0.090 ± 0.008 & 0.122 ± 0.012 & 0.103 ± 0.022 & 0.072 ± 0.012 & 0.073 ± 0.024 \\
EM & DR & 0.034 ± 0.004 & 0.037 ± 0.001 & 0.014 ± 0.001 & 0.027 ± 0.020 & 0.264 ± 0.191 & 0.092 ± 0.005 & 0.111 ± 0.019 & 0.097 ± 0.026 & 0.077 ± 0.016 & 0.073 ± 0.028 \\
\midrule
SimPro & IPW & 0.070 ± 0.011 & 0.058 ± 0.000 & 0.046 ± 0.001 & 0.049 ± 0.005 & 0.254 ± 0.074 & 0.223 ± 0.098 & 0.097 ± 0.025 & 0.067 ± 0.002 & 0.105 ± 0.066 & 0.110 ± 0.079 \\
SimPro & OR & 0.071 ± 0.012 & 0.058 ± 0.000 & 0.045 ± 0.001 & 0.049 ± 0.006 & 0.040 ± 0.003 & 0.059 ± 0.017 & 0.074 ± 0.006 & 0.075 ± 0.002 & 0.033 ± 0.003 & 0.033 ± 0.003 \\
SimPro & DR & 0.017 ± 0.004 & 0.026 ± 0.001 & 0.019 ± 0.002 & 0.018 ± 0.003 & 0.039 ± 0.003 & 0.058 ± 0.025 & 0.091 ± 0.007 & 0.031 ± 0.001 & 0.015 ± 0.003 & 0.019 ± 0.007 \\
\bottomrule
\end{tabular}
}
\end{table*}


\begin{table*}[t]
\centering
\caption{Total Variation Distance on CIFAR-100-LT ($N_l = 50$, $M_l = 400$) with different class imbalance ratios $\gamma_l$ and $\gamma_u$ under five different unlabeled class distributions.}
\label{tab:cifar100-tv}
\resizebox{\textwidth}{!}{
\begin{tabular}{lccccccccccc}
\toprule
& & \multicolumn{2}{c}{consistent} & \multicolumn{2}{c}{uniform} & \multicolumn{2}{c}{reversed} & \multicolumn{2}{c}{middle} & \multicolumn{2}{c}{head-tail} \\
\cmidrule(lr){3-4} \cmidrule(lr){5-6} \cmidrule(lr){7-8} \cmidrule(lr){9-10} \cmidrule(lr){11-12}
& & $\gamma_l = 20$ & $\gamma_l = 10$ & $\gamma_l = 20$ & $\gamma_l = 10$ & $\gamma_l = 20$ & $\gamma_l = 10$ & $\gamma_l = 20$ & $\gamma_l = 10$ & $\gamma_l = 20$ & $\gamma_l = 10$ \\
Model & Estimator & $\gamma_u = 20$ & $\gamma_u = 10$ & $\gamma_u = 1$ & $\gamma_u = 1$ & $\gamma_u = 1/20$ & $\gamma_u = 1/10$ & $\gamma_u = 20$ & $\gamma_u = 10$ & $\gamma_u = 20$ & $\gamma_u = 10$ \\
\midrule
Supervised & MLLS & 0.707 ± 0.016 & 0.313 ± 0.100 & 0.445 ± 0.172 & 0.309 ± 0.119 & 0.383 ± 0.075 & 0.397 ± 0.006 & 0.570 ± 0.001 & 0.373 ± 0.107 & 0.543 ± 0.009 & 0.231 ± 0.057 \\
Supervised & RLLS & 0.520 ± 0.007 & 0.133 ± 0.003 & 0.337 ± 0.125 & 0.253 ± 0.082 & 0.424 ± 0.060 & 0.463 ± 0.003 & 0.454 ± 0.021 & 0.306 ± 0.074 & 0.460 ± 0.028 & 0.241 ± 0.040 \\
\midrule
MLE & IPW & 0.075 ± 0.000 & 0.071 ± 0.001 & 0.229 ± 0.001 & 0.167 ± 0.002 & 0.565 ± 0.005 & 0.443 ± 0.007 & 0.415 ± 0.000 & 0.311 ± 0.005 & 0.343 ± 0.000 & 0.280 ± 0.001 \\
MLE & OR & 0.065 ± 0.002 & 0.061 ± 0.001 & 0.200 ± 0.007 & 0.143 ± 0.001 & 0.526 ± 0.011 & 0.399 ± 0.023 & 0.360 ± 0.003 & 0.256 ± 0.012 & 0.328 ± 0.003 & 0.266 ± 0.005 \\
MLE & DR & 0.149 ± 0.019 & 0.145 ± 0.010 & 0.243 ± 0.004 & 0.214 ± 0.019 & 0.568 ± 0.005 & 0.464 ± 0.014 & 0.403 ± 0.014 & 0.309 ± 0.012 & 0.365 ± 0.007 & 0.320 ± 0.004 \\
\midrule
EM & IPW & 0.097 ± 0.008 & 0.092 ± 0.004 & 0.239 ± 0.007 & 0.179 ± 0.003 & 0.478 ± 0.012 & 0.329 ± 0.020 & 0.262 ± 0.016 & 0.202 ± 0.003 & 0.312 ± 0.002 & 0.227 ± 0.001 \\
EM & OR & 0.121 ± 0.007 & 0.108 ± 0.005 & 0.261 ± 0.007 & 0.189 ± 0.004 & 0.489 ± 0.013 & 0.335 ± 0.020 & 0.274 ± 0.016 & 0.211 ± 0.004 & 0.336 ± 0.003 & 0.235 ± 0.001 \\
EM & DR & 0.125 ± 0.005 & 0.111 ± 0.004 & 0.269 ± 0.007 & 0.194 ± 0.005 & 0.497 ± 0.010 & 0.336 ± 0.024 & 0.281 ± 0.019 & 0.219 ± 0.008 & 0.336 ± 0.007 & 0.233 ± 0.004 \\
\midrule
SimPro & IPW & 0.125 ± 0.001 & 0.100 ± 0.005 & 0.166 ± 0.007 & 0.141 ± 0.009 & 0.353 ± 0.023 & 0.261 ± 0.008 & 0.202 ± 0.003 & 0.158 ± 0.005 & 0.277 ± 0.009 & 0.197 ± 0.003 \\
SimPro & OR & 0.133 ± 0.005 & 0.100 ± 0.004 & 0.160 ± 0.007 & 0.138 ± 0.010 & 0.322 ± 0.014 & 0.253 ± 0.008 & 0.202 ± 0.003 & 0.156 ± 0.005 & 0.269 ± 0.006 & 0.191 ± 0.004 \\
SimPro & DR & 0.122 ± 0.003 & 0.106 ± 0.006 & 0.188 ± 0.009 & 0.149 ± 0.006 & 0.343 ± 0.023 & 0.257 ± 0.007 & 0.219 ± 0.010 & 0.172 ± 0.002 & 0.279 ± 0.007 & 0.198 ± 0.004 \\
\bottomrule
\end{tabular}
}
\end{table*}
\begin{table*}[t]
\centering
\caption{Top-1 accuracy (\%) on CIFAR-10-LT ($N_l = 500$, $M_l = 4000$) with different class imbalance ratios $\gamma_l$ and $\gamma_u$ under five different unlabeled class distributions. In most settings, our two stage algorithm improves SimPro (9 / 10) and BOAT (8 / 10). We use {\green green} to indicate when our plug-in improves and {\red red} when it degrades the base model.}
\label{tab:cifar10-acc}
\resizebox{\textwidth}{!}{
\begin{tabular}{lcccccccccc}
\toprule

& \multicolumn{2}{c}{consistent} & \multicolumn{2}{c}{uniform} & \multicolumn{2}{c}{reversed} & \multicolumn{2}{c}{middle} & \multicolumn{2}{c}{head-tail} \\
\cmidrule(lr){2-3} \cmidrule(lr){4-5} \cmidrule(lr){6-7} \cmidrule(lr){8-9} \cmidrule(lr){10-11}

& $\gamma_l = 150$ & $\gamma_l = 100$ & $\gamma_l = 150$ & $\gamma_l = 100$ & $\gamma_l = 150$ & $\gamma_l = 100$ & $\gamma_l = 150$ & $\gamma_l = 100$ & $\gamma_l = 150$ & $\gamma_l = 100$ \\
& $\gamma_u = 150$ & $\gamma_u = 100$ & $\gamma_u = 1$ & $\gamma_u = 1$ & $\gamma_u = 1/150$ & $\gamma_u = 1/100$ & $\gamma_u = 150$ & $\gamma_u = 100$ & $\gamma_u = 150$ & $\gamma_u = 100$ \\

\midrule

FixMatch & 62.9 $\pm$ 0.36 & 67.8 $\pm$ 1.13 & 67.6 $\pm$ 2.56 & 73.0 $\pm$ 3.81 & 59.9 $\pm$ 0.82 & 62.5 $\pm$ 0.94 & 64.3 $\pm$ 0.63 & 71.7 $\pm$ 0.46 & 58.3 $\pm$ 1.46 & 66.6 $\pm$ 0.87 \\
CReST+ & 67.5 $\pm$ 0.45 & 76.3 $\pm$ 0.86 & 74.9 $\pm$ 0.90 & 82.2 $\pm$ 1.53 & 62.0 $\pm$ 1.18 & 62.9 $\pm$ 1.39 & 58.5 $\pm$ 0.68 & 71.4 $\pm$ 0.60 & 59.3 $\pm$ 0.72 & 67.2 $\pm$ 0.48 \\
DASO & 70.1 $\pm$ 1.81 & 76.0 $\pm$ 0.37 & 83.1 $\pm$ 0.47 & 86.6 $\pm$ 0.84 & 64.0 $\pm$ 0.11 & 71.0 $\pm$ 0.95 & 69.0 $\pm$ 0.31 & 73.1 $\pm$ 0.68 & 70.5 $\pm$ 0.59 & 71.1 $\pm$ 0.32 \\
% w/ ACR$\dagger$ (Wei \& Gan, 2023) & 70.9 $\pm$ 0.37 & 76.1 $\pm$ 0.42 & 91.9 $\pm$ 0.02 & 92.5 $\pm$ 0.19 & 83.2 $\pm$ 0.39 & 85.2 $\pm$ 0.12 & 77.6 $\pm$ 0.20 & 79.3 $\pm$ 0.30 & 73.8 $\pm$ 0.83 & 79.3 $\pm$ 0.48 \\
% w/ SimPro & 74.2 $\pm$ 0.90 & 80.7 $\pm$ 0.30 & 93.6 $\pm$ 0.08 & 93.8 $\pm$ 0.10 & 83.5 $\pm$ 0.95 & 85.8 $\pm$ 0.48 & 82.6 $\pm$ 0.38 & 84.8 $\pm$ 0.54 & 81.0 $\pm$ 0.27 & 83.0 $\pm$ 0.36 \\
Supervised & 63.2 $\pm$ 0.14 & 66.0 $\pm$ 0.27 & 63.3 $\pm$ 0.28 & 65.8 $\pm$ 0.19 & 63.1 $\pm$ 0.19 & 65.9 $\pm$ 0.51 & 63.5 $\pm$ 0.22 & 65.8 $\pm$ 0.03 & 63.0 $\pm$ 0.18 & 66.4 $\pm$ 0.07 \\
\midrule
EM & 69.1 $\pm$ 1.29 & 73.8 $\pm$ 0.71 & 94.0 $\pm$ 0.08 & 93.2 $\pm$ 0.94 & 76.6 $\pm$ 2.72 & 82.2 $\pm$ 0.24 & 79.5 $\pm$ 0.35 & 81.6 $\pm$ 0.58 & 79.2 $\pm$ 0.50 & 79.8 $\pm$ 0.17 \\
\midrule
SimPro & 74.4 $\pm$ 0.71 & 79.7 $\pm$ 0.45 & 93.3 $\pm$ 0.10 & 93.3 $\pm$ 0.47 & 83.8 $\pm$ 0.80 & 84.1 $\pm$ 0.24 & 78.7 $\pm$ 0.30 & 84.2 $\pm$ 0.26 & 81.2 $\pm$ 0.20 & 82.0 $\pm$ 1.07 \\
% \midrule
SimPro+ & \green 77.8 $\pm$ 1.50 & \green 81.2 $\pm$ 0.39 & \green 93.7 $\pm$ 0.07 & \green 93.7 $\pm$ 0.24 & \red 83.3 $\pm$ 0.38 & \green 84.7 $\pm$ 0.78 & \green 79.2 $\pm$ 0.70 & \green 85.4 $\pm$ 0.66 & \green 81.3 $\pm$ 0.27 & \green 82.5 $\pm$ 0.56 \\
\midrule
BOAT & 80.5 $\pm$ 0.39 & 83.3 $\pm$ 0.27 & 93.9 $\pm$ 0.03 & 94.1 $\pm$ 0.10 & 79.7 $\pm$ 0.25 & 81.1 $\pm$ 0.15 & 79.7 $\pm$ 1.15 & 81.6 $\pm$ 0.09 & 79.4 $\pm$ 0.44 & 80.9 $\pm$ 0.16 \\
% \midrule
BOAT+ & \green 81.6 $\pm$ 0.15 & \green 83.8 $\pm$ 0.04 & \red 93.7 $\pm$ 0.23 & 94.1 $\pm$ 0.17 & \green 80.4 $\pm$ 0.71 & \green 81.7 $\pm$ 0.38 & \green 80.3 $\pm$ 0.28 & \green 83.1 $\pm$ 0.45 & \green 79.7 $\pm$ 0.29 & \green 81.0 $\pm$ 0.36 \\
\bottomrule
\end{tabular}
}
\end{table*}

\begin{table*}[t]
\centering
\caption{Top-1 accuracy (\%) on CIFAR-100-LT ($N_l = 50$, $M_l = 400$) with different class imbalance ratios $\gamma_l$ and $\gamma_u$ under five different unlabeled class distributions. Despite poor estimation in stage 1, our approach does not degrade the accuracy for most of the settings. We use {\green green} to indicate when our plug-in improves and {\red red} when it degrades the base method.}
\label{tab:cifar100-acc}
\resizebox{\textwidth}{!}{
\begin{tabular}{lccccccccccc}
\toprule

& \multicolumn{2}{c}{consistent} & \multicolumn{2}{c}{uniform} & \multicolumn{2}{c}{reversed} & \multicolumn{2}{c}{middle} & \multicolumn{2}{c}{head-tail} \\
\cmidrule(lr){2-3} \cmidrule(lr){4-5} \cmidrule(lr){6-7} \cmidrule(lr){8-9} \cmidrule(lr){10-11}

& $\gamma_l = 20$ & $\gamma_l = 10$ & $\gamma_l = 20$ & $\gamma_l = 10$ & $\gamma_l = 20$ & $\gamma_l = 10$ & $\gamma_l = 20$ & $\gamma_l = 10$ & $\gamma_l = 20$ & $\gamma_l = 10$ \\
& $\gamma_u = 20$ & $\gamma_u = 10$ & $\gamma_u = 1$ & $\gamma_u = 1$ & $\gamma_u = 1/20$ & $\gamma_u = 1/10$ & $\gamma_u = 20$ & $\gamma_u = 10$ & $\gamma_u = 20$ & $\gamma_u = 10$ \\

\midrule
% FixMatch & 40.0 $\pm$ 0.96 & 45.2 $\pm$ 0.55 & 39.6 $\pm$ 1.16 & \\
% CReST+ & 40.1 $\pm$ 1.28 & 44.5 $\pm$ 0.94 & 37.6 $\pm$ 0.88 & \\
% DASO & 43.0 $\pm$ 0.15 & 49.8 $\pm$ 0.24 & 49.4 $\pm$ 0.93 & \\
Supervised & 32.4 $\pm$ 0.40 & 38.4 $\pm$ 0.18 & 32.7 $\pm$ 0.25 & 38.0 $\pm$ 0.22 & 32.5 $\pm$ 0.51 & 38.4 $\pm$ 0.43 & 32.3 $\pm$ 0.08 & 37.9 $\pm$ 0.43 & 32.1 $\pm$ 0.33 & 38.2 $\pm$ 0.38 \\
% \midrule
EM & 42.4 $\pm$ 0.43 & 49.6 $\pm$ 0.30 & 50.9 $\pm$ 0.27 & 58.0 $\pm$ 0.35 & 42.1 $\pm$ 0.16 & 49.8 $\pm$ 0.47 & 42.8 $\pm$ 0.41 & 49.6 $\pm$ 0.36 & 41.5 $\pm$ 1.26 & 49.5 $\pm$ 0.18 \\
\midrule
SimPro & 42.5 $\pm$ 0.58 & 49.6 $\pm$ 0.22 & 51.7 $\pm$ 0.22 & 58.1 $\pm$ 0.53 & 44.9 $\pm$ 0.21 & 51.8 $\pm$ 0.42 & 42.7 $\pm$ 0.06 & 49.8 $\pm$ 0.45 & 43.3 $\pm$ 0.76 & 50.9 $\pm$ 0.19 \\
% \midrule
SimPro+ & \green 42.8 $\pm$ 0.49 & \green 50.1 $\pm$ 0.33 & \red 51.6 $\pm$ 0.63 & \red 57.8 $\pm$ 0.38 & \red 44.7 $\pm$ 0.51 & \red 51.4 $\pm$ 0.88 & \green 43.4 $\pm$ 0.58 & \green 50.4 $\pm$ 0.28 & \green 43.8 $\pm$ 0.50 & \red 50.7 $\pm$ 0.76 \\
\midrule
BOAT & 43.7 $\pm$ 0.16 & 51.4 $\pm$ 0.32 & 55.1 $\pm$ 0.95 & 60.5 $\pm$ 0.15 & 43.1 $\pm$ 0.49 & 52.7 $\pm$ 0.23 & 43.6 $\pm$ 0.19 & 51.4 $\pm$ 0.39 & 43.9 $\pm$ 0.42 & 51.4 $\pm$ 0.14 \\
% \midrule
BOAT+ & \green 44.8 $\pm$ 0.13 & 51.4 $\pm$ 0.51 & \red 53.8 $\pm$ 0.32 & 60.5 $\pm$ 0.69 & \green 43.4 $\pm$ 0.56 & \red 52.4 $\pm$ 0.36 & \green 43.9 $\pm$ 0.59 & \red 50.8 $\pm$ 0.09 & \red 43.6 $\pm$ 0.50 & \green 51.9 $\pm$ 0.49 \\
\bottomrule
\end{tabular}
}
\end{table*}

We perform experiments for each stage of our algorithm. In the first stage, we compare among various methods to estimate the unlabeled class distribution $P(Y|A=0)$, showing that SimPro + DR performs well. In the second stage, we freeze the unlabeled class distribution, using our best estimator  SimPro + DR, and plug it into 2 SOTA semi-supervised learning algorithms, SimPro and BOAT~\cite{boat}. We show that this simple procedure improves the existing methods, and is even capable of improving the original SimPro when used for both stages.


% \textbf{Datasets} We adopt 4 standard benchmarks used frequently in other semi-supervised learning work: CIFAR-10, CIFAR-100~\cite{cifar}, STL-10~\cite{stl10} and Imagenet-127~\cite{cossl}. To match our RTSSL setting, we create long-tailed labeled and unlabeled sets from CIFAR-10 and CIFAR-100. Specifically, we use $\gamma_l$ and $n_1$ to denote the imbalance ratio and the head class's number of samples of the labeled data, the remaining class's size is computed as $n_c = n_1 \times \gamma_l^{-\frac{c-1}{C-1}}$ and likewise, $\gamma_u$ and $m_1$ of the unlabeled data. For CIFAR-10, we fix $n_1=500$ and $m_1=4000$. We test 2 different configurations $\gamma_l=\gamma_c=150$ and $\gamma_l=\gamma_c=100$. We further permute classes the unlabeled sets in 5 ways: consistent, uniform, reversed, middle and headtail, similar to \cite{simpro} and visualized in figure~\ref{fig:distribution}, which results in 10 different datasets in total. Similarly for CIFAR-100, we fix $n_1=500$ and $m_1=4000$, use 2 configurations $\gamma_l=\gamma_c=20$ and $\gamma_l=\gamma_c=10$, and permute the classes in 5 ways, resulting in 10 datasets as well. For STL-10, the unlabeled set has no ground truth labels, therefore we use all samples in the head class and set the imbalance ratio $\gamma_l$ to $10$ or $20$. Imagenet-127 is a naturally long-tailed dataset with 127 classes. We train on 32x32 and 64x64 image resolutions following ~\cite{cossl}.


\textbf{Datasets} We evaluate our method on four standard semi-supervised learning benchmarks: CIFAR-10, CIFAR-100~\cite{cifar}, STL-10~\cite{stl10}, and Imagenet-127~\cite{cossl}. To simulate RTSSL, we construct long-tailed labeled and unlabeled sets for CIFAR-10 and CIFAR-100. The labeled data follows an imbalance ratio $\gamma_l$ with head class size $n_1$, while the remaining class sizes are computed as $n_c = n_1 \times \gamma_l^{-\frac{c-1}{C-1}}$. The unlabeled data follows a similar setup with $\gamma_u$ and $m_1$.  

For CIFAR-10, we set $n_1 = 500$, $m_1 = 4000$, and test two configurations: $\gamma_l = \gamma_u = 150$ and $\gamma_l = \gamma_u = 100$. We generate 10 datasets by permuting the unlabeled class distributions in five ways: \textit{consistent, uniform, reversed, middle}, and \textit{head-tail}, as in~\cite{simpro}. CIFAR-100 follows the same setup with $n_1 = 50$, $m_1 = 400$, and $\gamma_l, \gamma_u$ values of 20 and 10.  

For STL-10, where unlabeled data lacks ground-truth labels, we use all head-class samples and set $\gamma_l$ to 10 or 20. Imagenet-127 is naturally long-tailed with 127 classes, and we train on 32$\times$32 and 64$\times$64 resolutions as in~\cite{cossl}.


\paragraph{Training.} We follow the implementation and hyperparameter settings of \cite{simpro}. We defer these details in \cref{subsec:training-setting}. One important exception is that for Imagenet-127, we use the smaller Wide ResNet-28-2 in stage 1 and the larger ResNet-50 for stage 2, to demonstrate that a smaller model is sufficient for distribution estimation.


\begin{table}[t]
\small
\centering
\caption{Top-1 Accuracy (\%) on STL-10. Our two-stage algorithms improves both SimPro and BOAT for both settings.}
\label{tab:stl10-acc}
% \resizebox{\linewidth}{!}{
\begin{tabular}{lcc}
\toprule
Method & $\gamma_l=10$ & $\gamma_l=20$ \\ \hline
Supervised & 73.9 $\pm$ 0.57 & 70.4 $\pm$ 0.95 \\
\midrule
MLE & 67.6 $\pm$ 0.57 & 58.9 $\pm$ 4.05 \\
\midrule
EM & 84.9 $\pm$ 0.14 & 83.6 $\pm$ 0.25 \\
\midrule
SimPro & 82.4 $\pm$ 1.57 & 80.5 $\pm$ 0.96 \\
SimPro+ & \green 83.9 $\pm$ 0.76 & \green 82.7 $\pm$ 0.86 \\
\midrule
BOAT & 83.8 $\pm$ 0.20 & 82.0 $\pm$ 0.34 \\
BOAT+ & \green 84.1 $\pm$ 0.38 & \green 82.4 $\pm$ 0.10 \\
\bottomrule
\end{tabular}
\end{table}















\begin{table}[t]
% \setlength{\tabcolsep}{3.5pt}
\small
\centering
\caption{Top-1 Accuracy (\%) on Imagenet-127. Our two-stage approach improves both SimPro and BOAT for both resolutions.}
\label{tab:imagenet-127-acc}
% \resizebox{\linewidth}{!}{
\begin{tabular}{lcc}
\toprule
Method & $32 \times 32$ & $64 \times 64$ \\ \hline
SimPro & 54.8 & 63.7 \\
SimPro+ & \green 55.1 & \green 64.2 \\
\midrule
BOAT & 51.6 & 58.7 \\
BOAT+ & \green 52.0 & \green 59.2 \\

\bottomrule
\end{tabular}
% }
\end{table}


\begin{table}[t]
% \setlength{\tabcolsep}{3.5pt}
\small\centering
\caption{Total Variation Distance on Imagenet-127}
\label{tab:imagenet-127-tv}
% \resizebox{\linewidth}{!}{
\begin{tabular}{cccc}
\toprule
Method & Estimator & $32 \times 32$ & $64 \times 64$ \\ \hline
MLE & IPW  & 0.103 $\pm$ 0.034 & 0.051 $\pm$ 0.000 \\
MLE & OR  & 0.153 $\pm$ 0.052 & 0.041 $\pm$ 0.000 \\
MLE & DR  & \green 0.100 $\pm$ 0.029 & \green 0.075 $\pm$ 0.003 \\
\midrule
EM & IPW  & 0.141 $\pm$ 0.006 & 0.163 $\pm$ 0.010 \\
EM & OR  & 0.205 $\pm$ 0.006 & 0.236 $\pm$ 0.011 \\
EM & DR  & \green 0.024 $\pm$ 0.001 & \green 0.042 $\pm$ 0.004 \\
\midrule
SimPro & IPW  & 0.041 $\pm$ 0.012 & 0.224 $\pm$ 0.040 \\
SimPro & OR  & 0.036 $\pm$ 0.014 & 0.291 $\pm$ 0.079 \\
SimPro & DR  & \green 0.017 $\pm$ 0.000 & \green 0.037 $\pm$ 0.004 \\
\bottomrule
\end{tabular}
% }
\end{table}

\subsection{Better results on label distribution} 
\label{subsec:label}
We have mentioned various ways throughout the papers to estimate the unlabeled class distribution. In what follows, method consists of a model, which is how the learning is done, and an estimator, which is how the final distribution is estimated using parameters learned from the model.

%\begin{enumerate}
%\item 
\noindent
\textbf{Supervised}. The model is trained on the labeled set only and used to estimate the unlabeled class distribution \cite{unifiedlabelshift}. 2 successful estimators for this setting are \textbf{RLLS} \cite{rlls} and \textbf{MLLS} \cite{mlls}. 

%\item 
\noindent\textbf{MLE}. The model is trained by directly maximizing the likelihood \cref{eq:likelihood}. We also use the decomposition $P(Y|X)$ and $P(A|Y)$, and write the unlabeled term as $P(A=0, X) = \sum_{c} P(Y=c|X) P(A=0|Y=c)$, which enables gradient descent training on these parameters. This is also the MLE method to estimate $P(A|Y)$ in \cite{arelabelsinformative}.

%\item 
\noindent\textbf{EM}. We further test the EM algorithm in \cref{subsec:em}. In particular we also use strong and weak augmentations similar to FixMatch, but not the pseudo labeling operator. Confidence thresholding removes the soft predictions of the non-dominant classes, which may be better to keep since our target of the first stage is the global class statistics. We also try 3 estimators on this model.

%\item 
\noindent\textbf{SimPro} \cite{simpro} can be seen as our previous EM but also with FixMatch's confidence thresholding and logit adjustment loss in \cref{subsec:simpro}. Confidence thresholding is a powerful regularization technique that encodes the assumption that classes are well separated \cite{entropyminimization}, but can introduce bias to the estimation, which justifies the use of DR.
%\end{enumerate}

% For semi-supervised methods MLE, EM and SimPro, as we now have additional information on the missingness mechanism, we can use 3 estimators OR, IPW and DR presented in \cref{subsec:2-stage}


Results on \cref{tab:cifar10-tv} presents the performance of various models and estimators on CIFAR-10. We can see that SimPro + DR performs best. In contrast, SimPro + OR, SimPro's original way of estimating $P(Y|A=0)$, and SimPro + IPW tend to underperform EM on the consistent and uniform datasets. The consistent setting is worth noting, since it arises when data is sampled uniformly at random for labeling,  representative of a large number of real world situations. EM is competitive to SimPro as well even without pseudo labeling, but overall we found this regularization to offer significant gains in the reversed, middle and head-tail settings. Finally, Supervised with either MLLS or RLLS estimators performs much worse than the semi-supervise methods.

\cref{tab:imagenet-127-tv} aligns with the observations  made in \cref{tab:cifar10-tv}. In particular, SimPro + DR is the best method for class distribution estimation of the much larger Imagenet-127. We also found that a small neural network and a small image resolution is sufficient for the distribution estimation of the much larger dataset Imagenet-127. This matches our intuition that the finite-dimensional parameter is easier to learn.

\cref{tab:cifar100-tv} shows that most methods understandably struggle to estimate the class distributions in CIFAR-100. This is because there are few samples in each class, the head class has 10 times less samples while the number of classes multiplies 10 times compared to CIFAR-10. We see here that SimPro + DR is not the best method, but the relative gap between estimators are small.

% Among the models, the supervised baseline do not perform well even in the consistent setting, showing that when unlabeled data is available during training, learning from them can be valuable for class distribution estimation, especially in the cases with little labeled data like ours. Both the MLE and supervised models perform badly on the reversed, middle and head-tail settings

% Among the estimators, we see that DR boosts the performance of SimPro and EM in CIFAR-10, and of all semi-supervised models in Imagenet-127. It does not improve MLE on CIFAR-10, and it does not improve on CIFAR-100. However, for most of the time, the decrease is not much. In constrast, IPW estimators can be significantly worse, for example in the reversed setting of CIFAR-10, where the distance is $0.254$ for $\gamma_l=150$ and $0.233$ for $\gamma_l=100$, compared to OR's 0.040 and 0.059. 

% Both the MLE and supervised models perform badly on the reversed, middle and head-tail settings. EM does a decent job, though not as well as SimPro, on all 5 distribution settings of CIFAR-10. However, on Imagenet-127, EM without DR performs worse than MLE.

% We note that the performance on DR is similar to OR in these cases, showing that DR has a double robustness property. While IPW only relies on the finite-dimensional $P(A|Y)$, which intuitively is easy to estimate, we found that the inverse probability weight can nevertheless be unstable when some probabilities are small, and this is where DR shows its strength by combining both IPW and OR.



\subsection{Two-stage algorithm improves accuracy}

In the second stage of our algorithm, we freeze our estimation and plug it in SimPro and BOAT. We denote SimPro+ and BOAT+ for algorithms that use our first stage estimate.



\cref{tab:cifar10-acc} shows that for CIFAR-10 SimPro+ and BOAT+ improve over their original versions across most settings, leading to large improvements in both the consistent and middle class distribution settings. In particular, our two-stage approach improves SimPro in 9 / 10 settings and BOAT in 8 / 10 settings.
We also observe consistent improvements ove both base algorithms, SimPro and BOAT, for several other datasets. \cref{tab:stl10-acc} demonstrates improvements for 2 / 2 class imbalance ratios in STL-10 and \cref{tab:imagenet-127-acc} for 2 / 2  different resolutions of ImageNet-127. 


We also evaluate on CIFAR-100 for multiple unlabeled  class distribution settings and with mediocre class label distribution estimates in stage 1, demonstrate no degradation in accuracy in stage 2. As shown in \cref{tab:cifar100-acc}, the two stage algorithm with a mediocre stage 1 estimation leads to parity with the baseline. Stage 2 provides small improvements in 5 / 10 settings for SimPro and in 4 / 10 (with 2 ties) for BOAT.


\subsection{Ablation Study: Alternative implementations.}
\label{subsec:ablation-1}
In this section, we ablate on our 2-stage choice. Specifically, we consider 2 alternative implementations:
\paragraph{\textbf{Doubly-robust risk}}  
This approach is \cite{arelabelsinformative, onnonrandommissinglabels}, as discussed in \cref{sec:background}. we consider the doubly-robust risk as our training loss. We use the missingness mechanism estimation from stage-1 of SimPro+ for fair comparison. \cref{eq:dr-risk} is used for training in which the pseudo-labeling operators can be applied straightforwardly. More detail in \cref{subsec:dr-risk}
\paragraph{\textbf{Batch-update doubly-robust $P(Y|A)$}} Different from SimPro+, here we remove the first stage and instead update our doubly robust estimation of the unlabeled class distribution using a moving average of the batch statistics.

\cref{tab:cifar10-ablation-1} shows that the batch-update version of SimPro+ is significantly worse on the consistent and uniform settings, while the doubly-robust risk is worst overall, especially in the reversed setting where $P(A|Y)$ is very small for the labeled tail classes, causing instability issues during training. In conclusion, our 2-stage approach is the best.


\begin{table}[t]
\small
\centering
\caption{Top-1 Accuracy (\%) on CIFAR-10. We compare our 2-stage SimPro+ with 1) an 1-stage alternative that updates and uses the doubly-robust estimation on-the-fly and 2) SimPro with doubly-robust risk. We use $\gamma_l=150$. {\green green} color indicates that our method performs best.}
\label{tab:cifar10-ablation-1}
\resizebox{\linewidth}{!}{
\begin{tabular}{lccccc}
\toprule
Method & consistent & uniform & reversed & middle & headtail\\ \hline
SimPro+ & \green 77.8 & \green 93.7 & \green 83.3 & \green 79.2 & \green 81.3 \\
batch-update & 71.9 & 91.4 & 82.6 & 78.6 & 81.2 \\
DR-risk & 72.1 & 89.8 & 67.1 & 75.6 & 79.5 \\
\bottomrule
\end{tabular}
}
\end{table}




\section{Summary and Future Work} \label{s_a_fw}

In this paper, we propose a TAPE scheme to investigate the auditing of unlearning effectiveness based on unlearning posterior differences, involving only the unlearning process. TAPE contributes a method to build unlearned shadow models to mimic the posterior difference quickly. Moreover, two strategies are introduced to augment the posterior difference, enabling the audit of unlearning multiple samples. The extensive experimental results validate the significant efficiency improvement compared with backdoor-based methods and the effectiveness of auditing genuine samples in both exact and approximate unlearning manners.


The auditing method proposed in this paper significantly addresses the limitations of existing unlearning verification methods. It effectively audits genuine samples for both exact and approximate unlearning methods in single-sample and multi-sample unlearning scenarios. Additionally, it eliminates the need for involvement in the original model training process. Future work could continue this line of inquiry, developing more efficient unlearning auditing methods to guarantee and support the right to be forgotten in MLaaS environments.




%%
%% The acknowledgments section is defined using the "acks" environment
%% (and NOT an unnumbered section). This ensures the proper
%% identification of the section in the article metadata, and the
%% consistent spelling of the heading.
\begin{acks}
	%This work is partially supported by Australia ARC DP200101374 and LP190100676.
	This work is partially supported by Australia ARC LP220100453 and ARC DP240100955.
\end{acks}

%%
%% The next two lines define the bibliography style to be used, and
%% the bibliography file.
\bibliographystyle{ACM-Reference-Format}
\balance
\bibliography{TAPE}


%%
%% If your work has an appendix, this is the place to put it.
\appendix





\begin{table*}[h]
	% \tiny
	\scriptsize
	\caption{An overview of machine unlearning auditing methods. %\vspace{-2mm}
	}
	\label{overview_of_auditing_method}
	\resizebox{\linewidth}{!}{
		\setlength\tabcolsep{3.pt}
		\begin{tabular}{c|cccccccc}
			\toprule[1pt]
			\multirow{2}{*} { \makecell[c]{\textbf{Unlearning} \\ \textbf{Auditing} \\ \textbf{Methods}} } & \multicolumn{2}{c} {\textbf{Involving Processes}  } & \multicolumn{2}{c} { \textbf{Auditing Data Type}} & \multicolumn{2}{c} {\textbf{Unlearning Methods}} & \multicolumn{2}{c} { \textbf{Unlearning Scenarios}}  \\
			\cmidrule(r){2-3}   \cmidrule(r){4-5} \cmidrule(r){6-7} \cmidrule(r){8-9}
			& \makecell[c]{{Original training} \\ {and unlearning	}  }  & \makecell[c]{{Only unlearning } \\ {process }  } & \makecell[c]{{Backdoored (marked)} \\ {samples	}  }    & \makecell[c]{{Genuine} \\ { samples	}  }   & \makecell[c]{{Exact} \\ {unlearning}  }    &\makecell[c]{{Approximate} \\ {unlearning}  } &   \makecell[c]{{Single} \\ {sample	}  }  &   \makecell[c]{{Multi} \\ {samples	}  }   \\ 
			\midrule
			MIB~\cite{hu2022membership} &\filledcircle & \emptycircle & \filledcircle&\emptycircle	&\filledcircle & \emptycircle &\emptycircle   	  & \filledcircle  \\
			Athena~\cite{sommer2022athena} &\filledcircle & \emptycircle & \filledcircle  &\emptycircle 	&\filledcircle &\emptycircle  &\emptycircle      & \filledcircle  \\
			Verify in the dark~\cite{guo2023verifying} &\filledcircle & \emptycircle & \filledcircle  &\emptycircle  &\filledcircle & \emptycircle &\emptycircle       & \filledcircle	  \\
			Verifi~\cite{gao2024verifi} &\filledcircle &\emptycircle & \filledcircle  &\emptycircle  	&\filledcircle & \emptycircle &\emptycircle       	  & \filledcircle  \\
			TAPE (Ours)	     &\emptycircle & \filledcircle & \emptycircle  & \filledcircle  	&\filledcircle  &\filledcircle & \filledcircle       	  & \filledcircle  \\
			\bottomrule[1pt]
	\end{tabular}}
	\begin{tabbing}
		\filledcircle: the auditing method is applicable; 
		\emptycircle: the auditing method is not applicable.
	\end{tabbing}
	\vspace{-2mm}
\end{table*}

\section{Additional Related Work Discussion} \label{different_with_existing}

\subsection{Machine Unlearning in the Web-Related Studies} 
Machine unlearning--the process of efficiently removing specific data influences from trained models--has been explored in diverse applications across Web-based systems, such as graph-based systems and personalized applications \citep{lin2024incentive,pan2023unlearning,zhu2023heterogeneous,wu2023gif}. In graph-based systems, \citeauthor{pan2023unlearning} \citep{pan2023unlearning} proposed an unlearning method to unlearn the graph classifiers with limited access to the original data, and \citeauthor{wu2023gif} \citep{wu2023gif} introduced a general strategy leveraging influence functions to efficiently remove specific graph data while preserving model integrity. In personalized applications, \citeauthor{lin2024incentive} \citep{lin2024incentive} introduced dynamic client selection with incentive mechanisms to enhance the federated unlearning efficiency, while \citep{zhu2023heterogeneous} extended federated unlearning to the heterogeneous knowledge graph, aiming to balance both privacy and model utility preservation. To achieve a better unlearning service, \cite{liu2024breaking} further explored the challenge of balancing privacy, utility, and efficiency and proposed a controllable unlearning framework to overcome this challenge.

\subsection{Difference from Existing Studies}  
Our TAPE approach is significantly different from existing unlearning verification methods \cite{hu2022membership,sommer2022athena,guo2023verifying,gao2024verifi} in terms of the involving processes, auditing data type, unlearning scenarios, and unlearning methods, as depicted in \Cref{overview_of_auditing_method}. First, the significant difference is that the auditing of our method only involves the unlearning process, while the backdoor-based methods must involve both the original training and unlearning processes to ensure the service model first learns the backdoor. Second, most existing auditing methods are based on backdooring techniques and need to backdoor or mark samples for verification \cite{hu2022membership,sommer2022athena,guo2023verifying,gao2024verifi}. As we analyzed in the above subsection, they can only validate the backdoored samples and are only applicable to the exact unlearning methods as exact unlearning methods guarantee the deletion from the dataset level. Our method does not mix any other data to the training dataset, and the auditing is based on the posterior difference, which is suitable for genuine samples in both exact and approximate unlearning methods. Third, backdoor-based auditing methods are only feasible for multi-sample unlearning scenarios because just using a single sample makes it hard to backdoor the model \cite{wang2019neural,lin2020composite,zeng2023narcissus,nguyen2020input}, hence failing to provide unlearning verification for a single sample. 



%We also note that TAPE shares similarities with some studies investigating privacy leakage caused by the model updated difference \cite{salem2020updates,chen2021machine,balle2022reconstructing,Hu2024sp}. They aimed to extract as much private information as possible from model differences. However, it is well-known that while these methods are effective for single-sample reconstruction, they are less effective for multi-sample reconstruction. We have one advantageous difference to ensure we can provide a more effective information reconstruction that is suitable for unlearning auditing for multi-samples. Specifically, the unlearning verification user knows the unlearned samples, as users specify these samples, while the settings in \cite{salem2020updates,chen2021machine,balle2022reconstructing,Hu2024sp} have no information about the inferred samples. With the knowledge of the unlearned samples, we can design the posterior augment methods to facilitate the unlearning auditing for multiple samples.



 

\section{MLaaS Scenario and Threat Model} \label{threat_model}

Our problem is introduced in a simple machine unlearning as a service (MLaaS) scenario for ease of understanding. Under the MLaaS scenario, there are two main entities involved: an ML server that collects data from users, trains models, and provides the ML service, and users that contribute their data for ML model training. 

\noindent
\textbf{The ML Server's Ability.}
To uphold the ``right to be forgotten'' legislation and establish a privacy-protecting environment, the ML server is responsible for conducting machine unlearning operations. However, it is challenging to audit the unlearning effect for users to confirm that the unlearning is processed and prevent the spoof of unlearning from the ML server. In alignment with common unlearning verification settings \cite{hu2022membership,guo2023verifying}, we assume the ML server is honest for learning training but may spoof users for unlearning, i.e., it reliably hosts the learning process but may deceive users during unlearning operations by pretending unlearning has been executed when it has not. It is reasonable for the ML server to pretend to execute unlearning operations to avoid the degradation of model utility. Moreover, this assumption is more plausible than assuming the server will forge an unlearning update~\cite{thudi2022necessity}. Forging an unlearning update would require the server to simulate the disappearance of specified data and the corresponding resulting in model utility degradation, which demands significant effort without any benefit, making it an unlikely motivation. 

\noindent
\textbf{The Unlearning Users' Ability.}
We consider the scenario where the unlearning user has only black-box access to the ML service model, which is one of the most challenging scenarios \cite{salem2020updates,Hu2024sp}. In unlearning scenarios, the unlearning user possesses a local dataset, including the erased samples, which constitutes the entire training dataset for the ML service model \cite{warnecke2024machine,hu2024eraser}; however, the user has no access to the entire dataset. This just allows the user to query the model with their own data in a black-box access to obtain the corresponding posteriors and design the unlearning requests with specific data for unlearning verification purposes. 
Furthermore, we assume the unlearning user knows the unlearning algorithms, which is confirmed by both server and users, commonly used in other works \cite{hu2023duty}. However, even if the unlearning user knows the algorithms, without the remaining dataset, the user still cannot achieve the corresponding unlearning results of most unlearning algorithms. To relax the difficulty, we consider the unlearning user to be able to establish the same ML model as the current target ML service model with respect to model architecture. This can be achieved through model hyperparameter stealing attacks \cite{wang2018stealing,Seong2018towards,salem2020updates}. The unlearning user leverages this knowledge to simulate the unlearned shadow models and mimic the behavior of the ML service model based on the designed unlearning requests, thereby deriving the posterior differences necessary for training the reconstruction model to evaluate the unlearning effectiveness. % \Cref{fast_g}.   

 
 
 \section{Unlearning Data Perturbation (UDP) Algorithm} \label{UDP_algorithm}
 
 
 \Cref{Unlearned_d_p} demonstrates how to use the R restarts to find the satisfied perturbation for the unlearning data to augment the posterior difference for auditing.
 
 \begin{algorithm}[t]
 	%\small
 	\caption{Unlearning Data Perturbation (UDP)} \label{Unlearned_d_p}
 	\begin{small} % small, normalsize
 		\BlankLine
 		\KwIn{Trained model $\theta^*$, reconstruction model $\texttt{AE}$, unlearned data $X_u$, perturbation limit $\alpha$, local dataset $D_{local}$ }
 		\KwOut{The perturbed unlearning data, $X_u' = X_u + \Delta^{p}$} %, and three metrics for verification
 		\SetNlSty{}{}{} % This line removes the vertical line before the for-loop
 		\SetKwFunction{UDP}{\textbf{UDP}}
 		\SetKwProg{Fn}{procedure}{:}{end procedure}
 		\SetNlSty{}{}{} % This line removes the vertical line before the for-loop
 		\Fn{\UDP{$\theta^*$, $\texttt{AE}$, $X_u$, $\alpha$, $D_{local}$ }}{
 			\For{$r \gets 1$ \KwTo $R$ restarts}{
 				$\Delta^{p}_{r} \gets \mathcal{N}(0,1)$  \hspace{4mm}    $\rhd$ Initialize random perturbation. \\
 				\For{$i \gets 1$ \KwTo $m$ optimization steps}{
 					$X_{u,i}^p \gets X_u + \Delta^{p}_r$  \hspace{0mm} $\rhd$ Add the perturbation to data. \\
 					$\theta_{\backslash (X_{u,i}^p)} \gets \theta^* - \frac{\epsilon}{n-1} \nabla \ell (X_{u,i}^p;\theta^*)$ $\rhd$  According to \Cref{shadow_model}.  \\
 					$\delta_{u,i}^p \gets \theta^*(D_{local}) - \theta_{\backslash (X_{u,i}^p)}(D_{local})$ $\rhd$ Calculate posterior difference according to \Cref{posterior_diff}.  \\
 					$\nabla \mathcal{L}_{\texttt{AE}} \gets \nabla \mathcal{L}_{\texttt{AE}} (\texttt{AE}(\delta_{u,i}^p) , X_{u,i}^p)$ $\rhd$  According to \Cref{perturb_loss}. \\			
 					$\Delta^{p}_{r} \gets \Delta^{p}_{r} - \eta \nabla \mathcal{L}_{\texttt{AE}} (\texttt{AE}(\delta_{u,i}^p) , X_{u,i}^p) $   \hspace{2mm} $\rhd$ Update perturbation with limitation $\|\Delta^{p}_{r}\|_{\infty} \leq \alpha $. \\
 				}
 			}
 			Choose the optimal $\Delta^p_r$ with minimal value in $\mathcal{L}_{\texttt{AE}}$ as $\Delta^{p*}$.\\
 			\Return $X_u' = X_u + \Delta^{p*}$
 		}
 	\end{small}
 \end{algorithm}
 


 \iffalse

\section{Proof of \Cref{first_order}} \label{proof_of_theorem_1}

\iffalse
The changes in the model parameters can be expand using the perturbation theory \cite{avrachenkov2013analytic} as:
\begin{equation} \label{expanding_loss}
	\Delta \theta = \theta^{\epsilon}_{D \backslash D_u} - \theta^* = \mathcal{O}(\epsilon)\theta^{(1)} + \mathcal{O}(\epsilon^2)\theta^{(2)} + \mathcal{O}(\epsilon^3)\theta^{(3)} + \cdot \cdot \cdot,
\end{equation}
where each unlearning sample in $D_u$ is up-weighted by a factor of $\epsilon$.  $\theta^{(1)}$ denotes the first-order (in $\epsilon$) perturbation and $\theta^{(2)}$ is the second-order model perturbation.
\fi

%\subsection{Proof of \Cref{first_order}} 
\begin{proof}
	We provide a derivation of the first-order model difference approximation $\Delta \theta \simeq \frac{\epsilon}{n-m} \sum_{x_u \in D_u} \nabla \ell (x_u; \theta^*)$ in \Cref{first_order}. %in the context of loss minimization (M-estimation).
	We define that $\theta^*$ minimizes the empirical risk: 
	\begin{equation}
		R(\theta) \overset{\text{def}}{=} \frac{1}{n} \sum_{x_i \in D} \ell (x_i;\theta),
	\end{equation}
	where $n$ is the size of the training dataset $D$.
	We assume that $R$ is strictly twice-differentiable and convex in $\theta$, thus, we can positively define 
	\begin{equation}
		H_{\theta^*} \overset{\text{def}}{=} \nabla^2 R(\theta^*) = \frac{1}{n} \sum_{x_i \in D} \nabla^2_{\theta} \ell (x_i;\theta).
	\end{equation}
	When removing an unlearning dataset $D_u$ with size $m$, $\theta^{\epsilon}_{D \backslash D_u}$ will be the optimal parameter set for the interpolated loss function $\mathcal{L}^{\epsilon}_{D \backslash D_u}(\theta)$, as shown in \Cref{loss_of_unlearning}. Due to the first-order stationary condition, we have
	\begin{equation}\label{nabla_loss_of_unlearning}
		\begin{aligned}
			0 = \nabla \mathcal{L}_{D \backslash D_u}^{\epsilon} (\theta^{\epsilon}_{D \backslash D_u})& =  \nabla \mathcal{L}_{\emptyset}(\theta^{\epsilon}_{D \backslash D_u}) \\
			&+\frac{1}{n} ( - \tilde{\epsilon} \sum_{x \in D \backslash D_u} +\epsilon \sum_{x \in D_u}) \nabla \ell (x;\theta^{\epsilon}_{D \backslash D_u}).
		\end{aligned}
	\end{equation}
	Let $\theta^{\epsilon}_{D \backslash D_u}$ denote the optimal parameters for $\mathcal{L}^{\epsilon}_{D \backslash D_u}$ minimization, and $\theta^*$ denote the optimal parameters trained on $D$. The changes in the model parameters can be expand using the perturbation theory \cite{avrachenkov2013analytic} as:
	\begin{equation} \label{expanding_loss}
		\Delta \theta = \theta^{\epsilon}_{D \backslash D_u} - \theta^* = \mathcal{O}(\epsilon)\theta^{(1)} + \mathcal{O}(\epsilon^2)\theta^{(2)} + \mathcal{O}(\epsilon^3)\theta^{(3)} + \cdot \cdot \cdot,
	\end{equation}
	where each unlearning sample in $D_u$ is up-weighted by a factor of $\epsilon$.  $\theta^{(1)}$ denotes the first-order (in $\epsilon$) perturbation and $\theta^{(2)}$ is the second-order model perturbation. 
	
	The main idea is to use Taylor series for expanding $\nabla \mathcal{L}_{\emptyset}(\theta^{\epsilon}_{D \backslash D_u})$ around $\theta^*$ base on the perturbation series defined in \Cref{expanding_loss} and compare the terms of the same order in $\epsilon$:
	\begin{equation} \label{expanding_loss_delta}
		\nabla \mathcal{L}_{\emptyset}(\theta^{\epsilon}_{D \backslash D_u}) = \nabla \mathcal{L}_{\emptyset}(\theta^*) + \nabla^2 \mathcal{L}_{\emptyset}(\theta^*)(\theta^{\epsilon}_{D \backslash D_u} - \theta^*) + \cdot \cdot \cdot.
	\end{equation}
	Similarly, we can also expand $\nabla \ell (x;\theta^{\epsilon}_{D \backslash D_u})$ around $\theta^*$ using Taylor series expansion. To derive $\theta^{(1)}$, we expand \Cref{nabla_loss_of_unlearning} and compare the terms with coefficient $\mathcal{O}(\epsilon)$: 
	\begin{equation}
		\begin{aligned}
			&\epsilon \nabla^2 \mathcal{L}_{\emptyset}(\theta^*) \theta^{(1)} \\
			&=\frac{1}{n} (  \tilde{\epsilon} \sum_{x \in D \backslash D_u} - \epsilon \sum_{x \in D_u})  \nabla \ell (x;\theta^*) \\
			&= \tilde{\epsilon}  \nabla \mathcal{L}_{\emptyset}(\theta^*)  - \frac{1}{n} (  \tilde{\epsilon}  + \epsilon)  \sum_{x\in D_u} \nabla \ell (x;\theta^*) \\
			& = -\frac{1}{n} (  \tilde{\epsilon}  + \epsilon  )  \sum_{x\in D_u}  \nabla \ell (x;\theta^*) \\
			& = -\frac{1}{n-m} \epsilon   \sum_{x\in D_u}  \nabla \ell (x;\theta^*).
		\end{aligned}
	\end{equation}
	$\theta^{(1)}$ is the first-order approximation of the group influence function. $\epsilon \in [-1,0]$ is used for unlearning.
\end{proof}
%(1 - )L(x;\theta)  (1)L(x;\theta)

\fi 

 



% in \Cref{UEV_algorithm}.


\iffalse
\section{Datasets} \label{datasets_appendix}

\begin{table}[h]
	\scriptsize
	\caption{Dataset statistics.}
	\label{dataset_table}
	\vspace{-3mm}
	\resizebox{\linewidth}{!}{
		\setlength\tabcolsep{7.5pt}
		\begin{tabular}{cccc}
			\toprule[0.8pt]
			Dataset & Feature Dimension  & \#. Classes & \#. Samples \\
			\midrule
			MNIST & 28×28×1 & 10 & 70,000  \\  
			\rowcolor{verylightgray}
			CIFAR10 & 32×32×3 & 10 & 60,000  \\  
			STL-10 & 96x96x3 &10 & 5000 \\
			\rowcolor{verylightgray}
			CelebA & 178×218×3 & 2 (Gender) & 202,599 \\
			\bottomrule[0.8pt]
	\end{tabular}}
\end{table}

The statistics of all datasets used in our experiments are listed and introduced in \Cref{dataset_table}. MNIST, CIFAR10, and STL-10 are benchmark datasets utilized for 10-class image classification tasks, offering a range of objective categories with varying levels of learning complexity. Our experiment on CelebA is to identify the gender attributes of the face images. The task is a binary classification problem, different from the ones on MNIST, CIFAR10 and STL-10. We also introduce them below

\begin{itemize}
	\item \textbf{MNIST.} MNIST contains 60,000 handwritten digit images for the training and 10,000 handwritten digit images for the testing. All these black and white digits are size normalized, and centered in a fixed-size image with 28 × 28 pixels.
	\item \textbf{CIFAR10.} CIFAR10 dataset consists of 60,000 32x32 colour images in 10 classes, with 6,000 images per class. There are 50,000 training images and 10,000 test images.
	\item \textbf{STL-10.} STL-10 dataset consists of 13,000 color images with 5,000 training images and 8,000 test images. STL-10 has 10 classes of airplanes, birds, cars, cats, dear, dogs, horses, monkeys, ships, and trucks with each image having a higher resolution of 96x96 pixels. Compared to the above two datasets, STL-10 can be considered as a more challenging dataset with higher learning complexity.
	\item  \textbf{CelebA.} CelebA is a large-scale face attributes dataset with more than 200,000 celebrity images, each with 40 attribute annotations, and the size of each image is 178×218.
\end{itemize}

\fi 




\section{The Verifier Training Process}  \label{verifi_train}



\begin{algorithm}[h]
	%\small
	\caption{Verifier Model Training (VMT)} \label{verifier_trianing}
	\begin{small} % small, normalsize
		\BlankLine
		\KwIn{Reconstruction model $\texttt{AE}$, posterior differences $\delta$, local dataset $D_{local}$, unlearned dataset $D_u$ }
		\KwOut{The Verifier Model $\mathcal{V}$}  
		\SetNlSty{}{}{} % This line removes the vertical line before the for-loop
		\SetKwFunction{VMT}{\textbf{VMT}}
		\SetKwProg{Fn}{procedure}{:}{end procedure}
		\SetNlSty{}{}{} % This line removes the vertical line before the for-loop
		\Fn{\VMT{$\texttt{AE}$, $\delta$, $D_{local}$, $D_u$}}{
			Initialize a verification dataset $D_{veri.}$ \\
			\For{$x_u$ in $D_u$, $x_i$ in $D_{local} \backslash D_u$ }{
				$D_{veri.}$ adds the positive sample ($\texttt{AE}(\delta_{\backslash x_u}), x_u; 1$) \\
				$D_{veri.}$ adds the negative sample ($\texttt{AE}(\delta_{\backslash x_u}), x_i; 0$) \\
			}
			Initialize a Verifier model $\mathcal{V}$ \\
			Train $\mathcal{V}$ on the constructed $D_{veri.}$ using a cross entropy loss  \\
			\Return the trained $\mathcal{V}$
		}
	\end{small}
\end{algorithm}


%The Verifier Model Training (VMT) algorithm is designed to construct a model that can distinguish between the unlearned and still remaining data instances. We first construct a verification dataset. For each instance in the unlearned dataset, the reconstructor model reconstructs information from the posterior difference that unlearns the instance, and we set the corresponding label equal to 1. For each instance in the local dataset that is not part of the unlearned dataset, we set a negative label for the instance and the reconstructed sample pair. These samples are added to the verification dataset. A verifier model is then initialized and trained on this constructed dataset using a cross-entropy loss. We return the final trained Verifier model.



This Verifier aims to identify if the recovered samples are unlearned samples. Specifically, we first construct a verification dataset $D_{veri.}$. For each instance in the unlearned dataset, and the reconstructor model reconstructs based on the posterior difference of the instance, and we set the corresponding label equal to 1. We add it as the postive sample ($\texttt{AE}(\delta_{\backslash x_u}), x_u; 1$) into $D_{veri.}$ For each instance in the local dataset that is not part of the unlearned dataset, we set a negative label for the instance and the reconstructed sample pair, i.e. ($\texttt{AE}(\delta_{\backslash x_u}), x_i; 0$). These samples are added to the verification dataset too. A verifier model is then initialized and trained on this constructed dataset using a cross-entropy loss. The Verifier model training algorithm is presented in \Cref{verifier_trianing}. %We return the final trained Verifier model.



 
\section{Metrics and Requirements for Auditing} \label{detailed_metrics}
%We now define the requirements of an effective solution to the unlearning effectiveness auditing problem. In terms of verification capacity and model utility, the scheme must be able to assess unlearning effectiveness, verify the data removal, and preserve functionality.




\begin{table*}[t]
	% \tiny
	\scriptsize
	\caption{Overall Evaluation Results on MNIST, CIFAR10, STL-10, and CelebA. 	\vspace{-2mm}}
	\label{tab_total}
	\resizebox{\linewidth}{!}{
		\setlength\tabcolsep{4.pt}
		\begin{tabular}{c|cccccccccccc}
			\toprule[1pt]
			\multirow{2}{*} { \makecell[c]{\textbf{Single-Sample} \\ \textbf{Unlearning Auditing}} } & \multicolumn{3}{c} {MNIST, $\text{\it ESS}=1$}& \multicolumn{3}{c} {CIFAR10, $\text{\it ESS}=1$} & \multicolumn{3}{c} {STL-10, $\text{\it ESS}=1$}  & \multicolumn{3}{c} {CelebA, $\text{\it ESS}=1$} \\
			\cmidrule(r){2-4}   \cmidrule(r){5-7} \cmidrule(r){8-10} \cmidrule(r){11-13}
			& Original & MIB \cite{hu2022membership}  	 & TAPE	  & Original	& MIB    & TAPE	 & Original	 &   MIB    & TAPE  & Original	 &   MIB    & TAPE \\
			\midrule %\thinmidrule
			Running time (s)  	  & 620 &  	637								& \textbf{143}  		         &651  &  672	& \textbf{135}	 	&781 		& 815 & \textbf{79.81}  &1546 		& 1622 & \textbf{32.76} \\
			Model Utility (Acc.)	 &\textbf{99.14\%}       & 98.31\%   &\textbf{99.14\%}         & \textbf{81.62\%} 	  & 79.45\%   & \textbf{81.62\%}   &\textbf{68.99\%} & 67.54\% & \textbf{68.99\%}  & \textbf{96.93\% } & 96.05\%  & \textbf{96.93\% }  \\
			Rec. Sim.  		 	& -  		 & -  					&\textbf{0.965}   						 			& - & - & \textbf{0.974} & -  	   &-& \textbf{0.175}  & -  	   &-& \textbf{0.977} \\
			Unl. Verifiability     &  $0.00\%$   & $0.00\%$  &\textbf{99.43\%}    &   $0.00\%$ &  $0.00\%$ & \textbf{98.76\%}   &  $0.00\%$  &  $0.00\%$   & \textbf{84.00\%}  &  $0.00\%$ &  $0.00\%$ & \textbf{97.64\%} \\
			\midrule[0.12em]
			\multirow{2}{*} {\makecell[c]{\textbf{Multi-Sample} \\ \textbf{Unlearning Auditing}}} & \multicolumn{3}{c} {MNIST, $\text{\it ESS}=20$}& \multicolumn{3}{c} {CIFAR10, $\text{\it ESS}=20$} & \multicolumn{3}{c} {STL-10, $\text{\it ESS}=2$} & \multicolumn{3}{c} {CelebA, $\text{\it ESS}=20$} \\
			\cmidrule(r){2-4}   \cmidrule(r){5-7} \cmidrule(r){8-10} \cmidrule(r){11-13}
			& Original & MIB \cite{hu2022membership}   	 & TAPE & Original & MIB   & TAPE	& Original	 &   MIB   & TAPE & Original	 &   MIB   & TAPE  \\
			\midrule 
			Running time (s)   & 613  & 638    &  \textbf{113}    & 644   & 673 			& \textbf{113} 	 & 781	& 809  & \textbf{74.90}   & 1570	& 1663  & \textbf{21.43}  \\
			Model Acc.	     & \textbf{99.05\%}   & 98.73\%    & \textbf{99.05\%}    & \textbf{81.62\%}    & 79.13\%    & \textbf{81.62\%} & \textbf{68.99\%} &  67.26\% &  \textbf{68.99\%}  &  \textbf{97.01\%} &  96.88\%  &  \textbf{97.01\%} \\
			Rec. Sim.  			    &-   			& -  			& \textbf{0.933}  			&-& -									& \textbf{0.973}					 	&- & - & \textbf{0.174}  &- & - & \textbf{0.970} \\
			Unl. Verifiability   	  & $0.00\%$   & $0.00\%$ & \textbf{98.67\%}     & $0.00\%$ & $0.00\%$ & \textbf{97.44\%}  	& $0.00\%$ & $0.00\%$ & \textbf{84.40\%} &  $0.00\%$ &  $0.00\%$  &  \textbf{94.57\%} \\
			\bottomrule[1pt]
	\end{tabular}}
	%	\vspace{-4mm}
\end{table*}




%\subsubsection{The Ability to Assess Unlearning Effectiveness and Verify Data Removal} \label{similarity_ability}

 
\noindent
\textbf{Data Removal Verifiability.} 
Existing backdoor-based unlearning verification methods can only provide the data removal verifiability based on the backdoor attack success rate \cite{guo2023verifying,hu2022membership}. We also train a Verifier (a classifying model) to identify the reconstructed data of the unlearned samples and the reconstructed data of the samples that still remain. We propose Verifiability to evaluate the accuracy of the Verifier, which calculates the correct classifying rate as 
\begin{equation} \label{verifiability_def}
	\textbf{Verifiability: } \hspace{4mm}  V =  \frac{ 1}{m}\sum_{x_u \in D_u} \mathbb{I}( \text{Verifier}(\delta_u, x_{u})=1),  
\end{equation}
where $m$ is the size of the unlearned dataset $D_{u}$ and $\mathbb{I}$ is the indicator function that equals 1 when its argument is true ($\text{Verifier}(\delta_u, x_{u})=1$) and 0 otherwise. 






\iffalse
%\subsubsection{Functionality Preservation}
\noindent
\textbf{Functionality Preservation.} 
The unlearning auditing scheme should not come at the cost of significant model utility degradation. In other words, the performance of the model after processing auditing should be only slightly worse than, if not equivalent to, the originally trained model. The formal definition is 
\begin{equation}
	\Pr_{(x,y) \in D} [\theta(x) =y]   \approx \Pr_{(x,y) \in D}[\theta_{\texttt{A}}(x)=y].
\end{equation}
It checks that the performance of the model with an auditing module $\theta_{\texttt{A}}$ does not deviate too much from those of original trained $\theta$.  

\fi




\section{Additional Experiments} \label{additional_exp}


\subsection{Overview Evaluation of TAPE} \label{overall_eval_app}

We demonstrate the overview evaluation results of different unlearning auditing methods on MNIST, CIFAR10, STL-10 and CelebA, presented in \Cref{tab_total}. The upper half of \Cref{tab_total} demonstrates the evaluations of the single-sample unlearning auditing, and the lower half of \Cref{tab_total} presents the evaluations of the multi-sample unlearning auditing. The bolded values indicate the best performance among the compared methods. We fill a dash when the method does not contain the evaluation metrics. 


 
\noindent
\textbf{Setup.}
We measure auditing methods based on the four above-introduced evaluation metrics in single-sample and multi-sample unlearning scenarios. In single-sample verification, the Erased Sample Size ({\it ESS}) is equal to 1 and $\text{\it ESS}=20$ for the multi-sample scenario. On STL-10, we set $\text{\it ESS}=2$ for the multi-sample scenario, as STL-10 only contains 5000 training samples, which is much smaller than other datasets. The evaluation here is tested based on the retraining-based unlearning method SISA \cite{bourtoule2021machine}. To better illustrate the functionality preservation and efficiency, we record the performance of solely training the original model, shown as ``Original'' in \Cref{tab_total}.

 

\noindent
\textbf{Evaluation of Efficiency.}
Since TAPE does not involve the original model training process, it consumes much less running time than MIB and ``Original''. The ``Original'' is training the original model before unlearning, and the MIB method needs to backdoor the model during the initial model training process before unlearning. Specifically, TAPE achieves more than $4.5\times$ speedup in efficiency on MNIST, $5\times$ speedup on CIFAR10, $10\times$ speedup on STL-10, and $50\times$ speedup on CelebA. On CelebA, the best speedup is up to $75\times$.




\noindent
\textbf{Evaluation of Functionality Preservation.}
The effect of functionality preservation is measured by model accuracy. In both single-sample and multi-sample unlearning auditing, our TAPE always achieves better functionality preservation than MIB. The highest accuracy preservation is around $2\%$, achieved on CIFAR10. The reason is that the MIB method needs to mix backdoored samples into the training dataset, and the backdoored samples with modified labels will negatively influence model utility. On the contrary, the TAPE scheme is independent of the original model training process; hence, our method will not influence the model utility of the original ML service model, keeping the same model accuracy as ``Original'', demonstrating better functionality preservation.


\noindent
\textbf{Evaluation of Unlearning Auditing Effect.}
We use reconstruction similarity to measure how much information about the specified samples is unlearned. The MIB method is unable to provide such an assessment of unlearned information for evaluation of unlearning effectiveness. Hence, we fill a dash of MIB in this metric. Reconstruction for a single sample always achieves better results than for multiple samples, which confirms our previous analysis and existing works \cite{salem2020updates,balle2022reconstructing}. The unlearned posterior difference of a single sample contains more information about such a sample than a posterior difference of multiple samples, as information from multiple samples is interwoven together in one posterior difference.  


To align with existing unlearning verification methods, we propose the verifiability metric to evaluate the data removal status, which is defined in \Cref{verifiability_def}. Since the erased sample size is small and only genuine unlearned samples are evaluated in this experiment, the MIB cannot successfully verify the unlearning of any genuine samples in \Cref{tab_total}. In both single-sample and multi-sample unlearning scenarios, TAPE provides effective data removal verification (accuracy larger than $95\%$ on MNIST, CIFAR10, and CelebA). Moreover, data removal status verification for a single sample always achieves better results than for multiple samples.





\iffalse
\begin{tcolorbox}[colback=white, boxrule=0.3mm]
	\noindent \textbf{Takeaway 1.} TAPE achieves the best efficiency and model functionality preservation as our scheme is independent of the original model training. Moreover, TAPE is effective in providing unlearning auditing for genuine unlearned samples, while MIB does not achieve this.
\end{tcolorbox}
\fi 

%(both how much information is unlearned and data removal status verification)


\subsection{Impact on Efficiency of Erased Samples Size ($\text{\it ESS}$)}  \label{imp_of_efficiency_ESS}


\noindent
\textbf{Impact on Efficiency.} 
The main components of the running time of TAPE are building unlearned shadow models and training the reconstructor. The running time of these two processes is highly related to the size of the user's local dataset. In our experiments, we randomly select $0.5\%$ samples on MNIST and CIFAR10 and choose $0.06\%$ samples on CelebA as the local dataset.

\Cref{evaluation_of_running_time} shows the running time of TAPE and MIB on the three datasets. The running time of TAPE has no significant relationship with the $\text{\it ESS}$ because the running time of TAPE (shadow model building and reconstructor training) highly depends on the size of the user's local dataset. For MIB, the running time has no obvious variations when $\text{\it ESS}$ increases. This is because the MIB verification preparation is accompanied by the original model training, which is heavily related to the size of training datasets. 
TAPE has a much more efficient running time compared with MIB, as TAPE is independent of the original model training. 



\begin{figure}[t]
	\centering
	%\hspace{-3mm}
	%\vspace{-2mm}
	\subfloat{    
		\includegraphics[scale=0.4]{Contents/Figures/Experiments_r/On_MNIST/Running_time/mnist_rt_sample_size_bar}
	}
	\vspace{-1mm}
	\caption{Running time about different $ESS$.}
	\vspace{-2mm}
	\label{evaluation_of_running_time} 
\end{figure}









\iffalse


\begin{figure*}[t]
	\centering
	\includegraphics[width=0.97\linewidth]{../../../../PycharmProjects/MUV_by_reconstruction/Experiments/On_MNIST/Ablation_exp/second_order_compare}
	\vspace{-2mm}
	\caption{Comparison between the first-order and second-order model differences. The legends stand for the first-order and second-order model difference when the erased data is ``In'' or ``Not In'' the remaining dataset.}
	\vspace{-2mm}
	\label{fig:secondordercompare}
\end{figure*}




\section{Additional Ablation Study Experiments} \label{ad_exp}








\subsection{Impact of first-order and second-order influence estimation for unlearned shadow model building}

Usually, the first-order model difference approximation is effective when the size $m$ of $D_u$ is small. However, when computing the group of many erased samples model difference, the second-order coefficient is in the order of $\frac{m^2}{n^2}$, which can be large when the size of $D_u$ is large. In this situation, we need to take both first-order perturbation $\theta^{(1)}$  and second-order perturbation $\theta^{(2)}$ into account. %Below is the corresponding calculation for second order model difference approximation. 
We present the experiments of the first-order and second-order model differences in \Cref{fig:secondordercompare}.
Theoretically, the MUA-MD of the second-order model difference will perform better than the MUA-MD of the first-order model difference because the second-order model difference approximation includes additional second-order information. In \cite{basu2020second}, the authors conducted experiments to remove 50\% samples to demonstrate the improvement of the second-order influence approximation. 
However, since we only unlearn from 1 to 100 samples, the second-order coefficient in this situation is from $\frac{1}{60000^2}$ to $\frac{100^2}{60000^2}$, which is still very small.  Hence, in this scenario, the impact of the second order is minimal, resulting in similar model performances. % when evaluated from the perspectives of Average UE, reconstruction similarity, and verifiability. From the running time perspective, calculating the second-order model difference takes more time than only calculating the first-order model difference, which is apparent when $\it{ESS}=1$.


\begin{theorem}[Second order model difference approximation] 
	\label{second_order}
	If the third-derivative of the loss function at $\theta^*$ is sufficiently small, the second-order influence function for a group of erased samples $D_u$ is:
	\begin{equation}
		\Delta \theta \simeq \mathcal{I}^{(2)}(D_u) = \mathcal{I}^{(1)}(D_u) + \mathcal{I}'(D_u)
	\end{equation}
	where 
	\begin{equation}
		\small
		\mathcal{I}'(D_u) = \frac{1}{n-m} (I - (\nabla^2 L_{\emptyset}(\theta^*))^{-1} \cdot \frac{1}{m} \sum_{x_u \in D_u} \nabla^2 L(x_u;\theta^*) ) \mathcal{I}^{(1)} (D_u).
	\end{equation}
\end{theorem}
\fi



%in \cite{}, they conducted experiments to remove 50\% samples to demonstrate the improvement of the second-order influence approximation.

% that's all folks


\iffalse
\section{Preliminaries}\label{pr_section}
\subsection{Machine Unlearning}
%Recent legislation such as GDPR and CCPA enact the ``right to be forgotten'', which allows individuals to request the removal of their specified data from trained models to preserve their privacy. 

%In machine unlearning, the model server should remove the influence of the specified samples $D_u$ from models that were trained based on the training set $D$ \cite{cao2015towards,neel2021descent}. The server needs to guarantee that the unlearned model should perform like the model that is retrained without seeing the erased samples $D_u$. 

%There are already many machine unlearning methods. Here, we briefly introduce the \textit{gold-standard} retraining method and two representative \textit{exact} and \textit{approximate} unlearning methods.



\noindent
\textbf{Retraining from Scratch.} The gold-standard machine unlearning method is to retrain the whole model from scratch. Formally, there is already a trained model with optimal parameters denoting as $\theta^*$, and its corresponding training dataset is $D$. This approach retrains a new model $\theta_u$ on the remaining dataset $D_r = D \backslash D_u$, where $D_u$ is the requested unlearning dataset. We call this $\theta_u$ the parameters of the unlearned model. Although retraining one hundred percent satisfies the requirements of unlearning, the computational and storage overhead of retraining is too large when the original dataset $D$ is large, and the unlearning requests are frequent. 

%To reduce the computational cost, several efficient exact and approximate approaches have been proposed.

\vspace{2mm}
\noindent
\textbf{SISA.} 
SISA is a successful unlearning extension based on naive retraining to improve the unlearning efficiency \cite{bourtoule2021machine,koch2023no}. Most other exact unlearning methods adopt a similar assemble strategy like SISA \cite{chen2022graph,yan2022arcane,chen2022recommendation}. The main process of SISA divides the full data $D$ into several shards $D^1, D^2, ..., D^k$ and trains sub-models with parameters $\theta^1, \theta^2, ..., \theta^k$ for each shard. When the server receives a request for unlearning sample $x_u$, it just needs to retrain the sub-model $\theta^i$ of shard $D^i$ that contains $x_u$. Since the size of the shard, $D^i$, is much smaller than $D$, the computational overhead of SISA is much smaller than the naive retraining method.



\vspace{2mm}
\noindent
\textbf{HBU.} Hessian matrix based unlearning (HBU) is a representative approximate unlearning method that aims to unlearn a posterior directly based on the original model quickly \cite{sekhari2021remember,guo2019certified,mehta2022deep,warnecke2021machine}. \citeauthor{guo2019certified} \cite{guo2019certified} proposed a certified-removal mechanism, which is based on the Hessian matrix and achieves unlearning by Newton Update. An example of unlearning one sample $x_u$ can be defined as 
\[
\theta_u = \theta^* + H^{-1}_{\theta^*} \nabla L(x_u, \theta^*),
\]
where $H^{-1}$ is the inverse Hessian matrix of loss function $L(D \backslash x_u,\cdot)$ at $\theta^*$, denoted as $H_{\theta^*} = \nabla^2 L(D \backslash x_u, \theta^*)$. Other approximate unlearning methods that directly unlearn based on the original trained models also employ the update equation like this but with different unlearning loss functions \cite{nguyen2020variational,fu2022knowledge,wang2023machine}.



\subsection{Verification of Machine Unlearning}
Most existing unlearning verification methods utilize backdoors to assist in verifying whether the data is unlearned \cite{hu2022membership,sommer2022athena,guo2023verifying}. These verification methods mixed backdoored samples $D_b$ with users' normal data $D_u$ before uploading for original model training, which can be described as $D_{u, b} = D_u \cup D_b$. By mixing a certain number of backdoored samples, the model will be backdoored during the model training process.  
Then, the user can verify if his/her data is learned or unlearned in the model by testing whether backdoors can attack the model. However, we find that the $D_u$ and $D_b$ mixed in the user's dataset $D_{u,b}$ are actually two distinct sub-sets. The appearance or disappearance of the backdoor can only verify if the backdoored data $D_b$ is used or not used in the model, while it cannot represent the users' normal data $D_u$ is used or not.



To better illustrate the different model performance of $D_u$ and $D_b$ during training, we conduct an approximate unlearning (VBU~\cite{nguyen2020variational}) on MNIST, presented in \Cref{fig:mnistepochaccdrop}. The model accuracy diminished rapidly on $D_b$ while slowly on unlearned data $D_u$ and test data during the unlearning training process, which was also studied in \cite{gao2023backdoor}. The evidence clearly shows that backdoor accuracy only indicates whether backdoored samples are unlearned in the model, while it does not serve as proof of unlearning for normal data.

%an experiment to show their performance when executing, The model accuracy performs variously on backdoored data $D_b$ and normal data $D_u$ during the unlearning training process. 

\fi



\end{document}
\endinput
%%
%% End of file `sample-sigconf.tex'.
