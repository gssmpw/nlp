 % This must be in the first 5 lines to tell arXiv to use pdfLaTeX, which is strongly recommended.
\pdfoutput=1
% In particular, the hyperref package requires pdfLaTeX in order to break URLs across lines.

\documentclass[11pt]{article}

% Remove the "review" option to generate the final version.
\usepackage[]{acl}

% Standard package includes
\usepackage{times}
\usepackage{latexsym}

% Additional packages
\usepackage{graphicx}
\usepackage{color}
\usepackage{xspace}
\usepackage{amsmath,amsfonts}
\usepackage{algorithm}
\usepackage{algorithmic}
\usepackage{textcomp}
\usepackage{xcolor}
\usepackage{booktabs}
\usepackage{booktabs,xltabular}
\usepackage{subcaption}
\usepackage{listings}
\usepackage{adjustbox}
\usepackage[most]{tcolorbox}

% %%% added by Chen Huang %%%
% \usepackage{easyReview}

\usepackage{url}
%%% end add %%%

%%% Spacing around floats %%% 
\setlength{\floatsep}{11pt plus 2pt minus 4pt}
\setlength{\textfloatsep}{11pt plus 2pt minus 4pt}
\setlength{\dblfloatsep}{\floatsep}
\setlength{\dbltextfloatsep}{11pt plus 2pt minus 4pt}
\setlength{\intextsep}{\floatsep}
\setlength{\abovecaptionskip}{5pt plus 3pt minus 2pt}

% \newtcolorbox{promptBox}{
%     colback=gray!20, % Background color of the box
%     colframe=gray!40, % Color of the frame
%     fonttitle=\bfseries,
%     coltitle=black,
%     colbacktitle=gray!40
% }

\newtcolorbox{promptBox}{
    colback=gray!10, % Background color of the box
    colframe=black!40, % Color of the frame
    fonttitle=\bfseries,
    coltitle=black,
    colbacktitle=gray!40
}

% Set global settings for listings here
\lstset{
    basicstyle=\ttfamily, % Change the font style
    xleftmargin=\parindent, % Adjust left margin
    xrightmargin=\parindent % Adjust right margin
}

\lstdefinestyle{codestyle}{
    backgroundcolor=\color{white},   
    commentstyle=\color{green},
    keywordstyle=\color{blue},
    numberstyle=\tiny\color{black},
    stringstyle=\color{purple},
    basicstyle=\ttfamily\footnotesize,
    breakatwhitespace=false,         
    breaklines=true,                 
    captionpos=b,                    
    keepspaces=true,                 
    numbers=left,                    
    numbersep=5pt,                  
    showspaces=false,                
    showstringspaces=false,
    showtabs=false,                  
    tabsize=2
}
\lstset{style=codestyle}

% For proper rendering and hyphenation of words containing Latin characters (including in bib files)
\usepackage[T1]{fontenc}
% For Vietnamese characters
% \usepackage[T5]{fontenc}
% See https://www.latex-project.org/help/documentation/encguide.pdf for other character sets

% This assumes your files are encoded as UTF8
\usepackage[utf8]{inputenc}

\usepackage{microtype}

\newcommand{\TheName}{\textsc{Constructa}}

\title{\TheName{}: Automating Commercial Construction Schedules in
Fabrication Facilities with Large Language Models}

% \author{First Author \\
%   Affiliation / Address line 1 \\
%   Affiliation / Address line 2 \\
%   Affiliation / Address line 3 \\
%   \texttt{email@domain} \\\And
%   Second Author \\
%   Affiliation / Address line 1 \\
%   Affiliation / Address line 2 \\
%   Affiliation / Address line 3 \\
%   \texttt{email@domain} \\}

% \author{
%     \textbf{Yifan Zhang}\textsuperscript{1,2}\textsuperscript{\thanks{\hspace{5pt}Work done during a GenAI research internship at Intel Incubation and Disruptive Innovation~(IDI) Group.}}\hspace{5pt}
%     \textbf{Xue Yang}\textsuperscript{2}\vspace{5pt}\\
%     \includegraphics[height=10pt]{vanderbilt_logo.png} Vanderbilt University\textsuperscript{1}\hspace{5pt}
%     \includegraphics[height=10pt]{intel_logo_new.png} Intel Corporation\textsuperscript{2}\\
%     \normalsize
%     \texttt{\{yifan.zhang.2\}@vanderbilt.edu}\hspace{5pt} 
%     \texttt{\{xue.yang\}@intel.com}
% }

\author{
    \textbf{Yifan Zhang}\textsuperscript{1,2}\textsuperscript{\dag}\hspace{5pt}
    \textbf{Xue Yang}\textsuperscript{2}\vspace{5pt}\\
    \includegraphics[height=10pt]{vanderbilt_logo.png} Vanderbilt University\textsuperscript{1}\hspace{5pt}
    \includegraphics[height=10pt]{intel_logo_new.png} Intel Corporation\textsuperscript{2}\\
    \normalsize
    \texttt{\{yifan.zhang.2\}@vanderbilt.edu}\hspace{5pt} 
    \texttt{\{xue.yang\}@intel.com}
}

% \small
% \textsuperscript{1}Department of Computer Science, Vanderbilt University, Nashville, TN, USA \\ 
% \small
% \textsuperscript{2}Incubation and Disruptive Innovation~(IDI), Intel Corporation, Santa Clara, CA, USA \\

\begin{document}

\maketitle

\begingroup
\renewcommand\thefootnote{\dag}
\footnotetext{Work done during a GenAI research internship at Intel Incubation and Disruptive Innovation~(IDI) Group.}
\endgroup

\begin{abstract}

% Leveraging LLM to do automated planning, especially in traditional industry, has been an underestimated but emerging research direction. Specifically in commericial construction optimization, the inherent compexity of automated scheduling to some degree slows down the process of automation and still largely requires architect's expertise to manually create plan for the schedule. In this paper, we present a novel framework, named \Thename{}, for automatically optimizing construction schedules in complex commercial projects such as semiconductor fabrication with the power of Large Language Models (LLMs). Specifically, (1) observing the incapability of LLM in understanding commercial construction, we designed static Retrieval-Augmented Generation (RAG) method to infuse LLMs with construction knowledge; (2) we incorporate multiple typs of context sampling methods based on the archtect's expertise to provide LLMs the most relevant context in addition to the knowledge, and (3) we deployed a preference alignmend-based method called CPA-DPO to further combine the knowledge and context into dense information in schedule automation. Our experiment shows that \TheName{} have notable accuracy missing value prediction, dependency analysis and automated planning in our proprietary data. We hope this work can  inspire the community to further deploy full-fledged LLM-based automation modeling pipeline for industry vertical applications.

% Automating planning with Large Language Models (LLMs) presents significant opportunities for advancing traditional industries, yet it remains underexplored. In commercial construction optimization, the inherent complexity of automated scheduling often necessitates manual intervention from architects, ensuring precision but adding to the process's intricacy. We propose \TheName{}, a novel framework designed to optimize construction schedules in complex projects such as semiconductor fabrication by harnessing the power of LLMs. Specifically, \ThenName{} addresses key challenges by: (1) introducing a static Retrieval-Augmented Generation (RAG) method to infuse construction knowledge into LLMs, overcoming their inherent limitations in understanding this domain; (2) leveraging diverse, context-sampling techniques inspired by architectural expertise to enrich input relevance; and (3) deploying a preference alignment approach, CPA-DPO, to integrate knowledge and context into actionable outputs for schedule automation. Experiments on proprietary data demonstrate that \Thename{} significantly outperforms baseline methods across missing value prediction, dependency analysis, and automated planning tasks. Our work highlights the potential of LLMs to revolutionize industry-specific workflows, inspiring further research into scalable, domain-adapted automation frameworks.

Automating planning with LLMs presents transformative opportunities for traditional industries, yet remains underexplored. In commercial construction, the complexity of automated scheduling often requires manual intervention to ensure precision. We propose \TheName{}, a novel framework leveraging LLMs to optimize construction schedules in complex projects like semiconductor fabrication. \TheName{} addresses key challenges by: (1) integrating construction-specific knowledge through static RAG; (2) employing context-sampling techniques inspired by architectural expertise to provide relevant input; and (3) deploying Construction DPO to align schedules with expert preferences using RLHF. Experiments on proprietary data demonstrate performance improvements of +42.3\% in missing value prediction, +79.1\% in dependency analysis, and +28.9\% in automated planning compared to baseline methods, showcasing its potential to revolutionize construction workflows and inspire domain-specific LLM advancements.

\end{abstract}

\begin{figure*}[ht]
    \centering
    \includegraphics[width=\linewidth]{figs/constructa_overview.pdf}
    \caption{Overview of the \TheName{} system. (a) The initial construction schedule is created by experts and refined with contextual activity and site samples. (b) Contextualized activity aggregates hierarchical, first-order, and sequential relations. (c) Knowledge vectorization embeds and retrieves construction knowledge for optimization. (d) Construction preference alignment uses RLHF to align schedules with expert rules and preferences.}
    \label{fig:constructa_overview}
\end{figure*}

\section{Introduction}

Automating construction schedules in large-scale commercial projects, such as semiconductor fabrication, is an inherently complex task due to the dynamic nature of project contexts, intricate dependency structures, and the critical need for expert-driven decision-making~\cite{neelamkavil2009automation,azimi2011framework}. The difficulty lies in managing the vast number of interdependent activities, each with unique resource requirements and constraints, while simultaneously adapting to real-time changes and unforeseen disruptions~\cite{zavadskas2004multicriteria}. These factors necessitate seamless integration of domain knowledge and human expertise to ensure project feasibility and efficiency. Traditional methods, relying on rigid rules and static assumptions, often fail to adapt to the variability and uncertainty inherent in large-scale construction projects, leaving a critical need for more flexible and context-aware approaches~\cite{alegre2016engineering,al2020uncertainty}.

% Automating construction schedules in large-scale projects, such as semiconductor fabrication, is inherently complex due to dynamic contexts, intricate dependencies, and the need for expert decision-making~\cite{neelamkavil2009automation,azimi2011framework}. Managing numerous interdependent tasks with unique constraints, while adapting to real-time changes, requires seamless integration of domain knowledge and human expertise~\cite{zavadskas2004multicriteria}. Traditional rule-based methods, reliant on static assumptions, often fail to handle the variability and uncertainty of large-scale construction, highlighting the need for more flexible, context-aware approaches~\cite{alegre2016engineering,al2020uncertainty}.

Despite recent advancements in machine learning, the potential of large language models (LLMs) for construction scheduling remains underexplored due to several limitations. LLMs, pretrained on broad datasets, lack the domain-specific knowledge needed for intricate project dependencies and constraints~\cite{xu2024llm,banerjee2024cps}. Moreover, the size and complexity of construction plans make it impractical to load entire projects into LLMs for automation~\cite{gidado1996project}. Instead, construction scheduling demands dynamic handling of real-time updates and evolving conditions. LLMs face three key challenges: (1) capturingthe intricate dependencies between construction activities, (2) adapting to context-sensitive changes in task priorities or resource availability, and (3) aligning outputs with expert-driven preferences. These challenges highlight the need for tailored frameworks to bridge the gap between LLM capabilities and the demands of large-scale construction projects.

% Despite advancements in machine learning, LLMs remain underutilized for construction scheduling due to key limitations. Pretrained on general datasets, they lack domain-specific knowledge for handling intricate dependencies and constraints~\cite{xu2024llm,banerjee2024cps}. The complexity of construction plans also makes loading entire projects impractical~\cite{gidado1996project}. Effective scheduling requires real-time updates and evolving priorities, presenting challenges in (1) capturing task dependencies, (2) adapting to dynamic changes, and (3) aligning with expert preferences. These gaps underscore the need for tailored frameworks to meet the demands of large-scale construction.

To address these limitations, we present \TheName{}\footnote{\TheName{} and Construction RLHF are used interchangeably in this paper.}, a novel framework designed to optimize construction schedules dynamically by leveraging LLMs with three key components: (1) Static Retrieval-Augmented Generation (SRAG or Static RAG), which introduces domain-specific construction knowledge, enabling LLMs to understand definitions, rules, and constraints critical to commercial construction; (2) Contextualized Knowledge RAG~(Knowledge RAG or KRAG), which incorporates the expertise of architects by dynamically sampling context-sensitive information, ensuring the relevance of inputs to evolving project conditions; and (3) Construction RLHF, which aligns the outputs of LLMs with expert feedback to enhance their in-depth understanding and produce human-aligned scheduling decisions.

We evaluate \TheName{} on a proprietary dataset comprising 4,340 semiconductor fabrication activities characterized by intricate dependencies and constraints. \TheName{} delivers substantial performance improvements, including a 42.3\% boost in missing value prediction, 79.1\% in dependency analysis, and 28.9\% in automated planning compared to baseline methods. Further analysis across levels and areas shows adaptability, while Construction RLHF distills raw data into actionable insights, demonstrating scalability and robustness for complex construction tasks.


\begin{figure*}[htbp]
    \centering
    \includegraphics[width=\textwidth]{figs/constructa_cpa_rlhf_backup.pdf}
    \caption{Illustration of the CPA-RLHF process. \textcircled{1} Raw contexts and rules are input for comprehension. \textcircled{2} The Plan Agent refines these into filtered contexts and rules. \textcircled{3} Completions are evaluated and stored in the Preference Database. \textcircled{4} The Expert Agent aligns outputs with project preferences. Part (a) collects data for preference model training, and part (b) aligns preferences for accurate planning.}
    \label{fig:constructa_cpa}
\end{figure*}

\section{Methodology}

Our methodology starts with an expert-provided schedule (Figure~\ref{fig:constructa_overview}, part (a)) and refines it using Static RAG for retrieval, Knowledge RAG for dependencies, and Construction RLHF for rule alignment (parts (c), (b), and (d)). The outputs, including retrieved knowledge and preference-aligned task relationships, are integrated into prompts for dynamic, context-aware scheduling.

% Our methodology begins with an expert-provided schedule (Figure~\ref{fig:constructa_overview}, part (a)) and refines it using Static RAG for knowledge retrieval, Knowledge RAG for task dependency capture, and Construction RLHF for aligning schedules with expert rules (parts (c), (b), and (d)). The resulting outputs, including retrieved knowledge, contextual task relationships, and preference-aligned rules, are combined into prompts to enable dynamic and context-aware construction scheduling.

% Our methodology starts with an initial construction schedule provided by experts, as shown in part (a) of Figure~\ref{fig:constructa_overview}, and refines it through Static RAG for knowledge retrieval, Knowledge RAG for capturing task dependencies, and Construction RLHF for aligning schedules with expert rules, illustrated in parts (c), (b), and (d), respectively. The outputs from these components, including retrieved knowledge, contextual task relationships, and preference-aligned rules, are integrated into a comprehensive prompt that guides the automation of commercial construction tasks, ensuring dynamic and context-aware scheduling.

\subsection{Static Retrieval-Augmented Generation}

Static RAG equips LLMs with construction-specific knowledge, as shown in part (c) of Figure~\ref{fig:constructa_overview}. It bridges the gap between general-purpose models and scheduling needs by generating embeddings for retrieval, with Local Static RAG providing precise definitions and Global Static RAG offering broader domain knowledge.

\textbf{Local Static RAG} provides precise definitions for construction-specific terms like Work Breakdown Structure (WBS) using curated online resources. For each term \( t \) in the terminology set \( \mathcal{T} \), its definition \( d_t \) is retrieved and embedded as \( e_t = f_\text{embed}(d_t) \) using an embedding model \( f_\text{embed} \). These embeddings are stored for contextualizing activities in schedule optimization.

\textbf{Global Static RAG} retrieves domain-specific knowledge from resources like textbooks or manuals. Raw text \( \mathcal{D} \) is cleaned and segmented into chunks \( \mathcal{C} = \{c_1, c_2, \dots, c_n\} \), each embedded as \( e_{c_i} = f_\text{embed}(c_i) \) and stored in a database. For a query \( q \), the system retrieves the most relevant chunk \( c^* \) by maximizing similarity \( \text{sim}(e_q, e_{c_i}) \), where \( e_q = f_\text{embed}(q) \) and \( c^* = \arg \max_{c_i \in \mathcal{C}} \text{sim}(e_q, e_{c_i}) \). Combining Local and Global Static RAG ensures precise definitions and broad domain knowledge for construction scheduling.

\subsection{Contextual Knowledge RAG}

Contextual Knowledge RAG samples task-specific contexts from a dependency graph \( G = (V, E) \), where \( V \) represents activities and \( E \) their dependencies. As shown in part (b) of Figure~\ref{fig:constructa_overview}, it aggregates hierarchical, first-order, and sequential relationships, using the combined context to retrieve relevant embeddings from the knowledge database for construction scheduling.

\textbf{Sequential Context} captures predecessor and successor activities up to three hops by traversing the graph in both directions. Random paths are sampled to reflect relevant sequential relationships while avoiding revisits and cycles, ensuring the selection of meaningful task flows.

\textbf{Hierarchical Context} retrieves nodes within the same Work Breakdown Structure (WBS) up to two levels. Tasks sharing WBS attributes are identified, and bidirectional traversal ensures that hierarchically consistent nodes are included in the context.

\textbf{First-Order Context} includes direct predecessors and successors of the target node, focusing on immediate task dependencies critical for accurate schedule representation.

Each task \( i \) is assigned a combined context \( C_i = \{\text{FirstOrder}(i), \text{Hierarchical}(i), \text{Sequential}(i)\} \), reflecting one-hop, two-hop, and three-hop constraints. Using the same embedding model as Static RAG, embeddings for \( C_i \) retrieve local knowledge and the top three global knowledge chunks from books and references, balancing dependencies to optimize rule generation and scheduling.

\subsection{Construction RLHF}

The Construction RLHF pipeline (Figure~\ref{fig:constructa_cpa}) refines schedules by integrating expert feedback and dynamic adjustments. Starting with raw contexts and rules (\textcircled{1}), the Plan Agent combines task-specific details with context retrieved from SRAG and KRAG (\textcircled{2}). Refined outputs, evaluated as positive or negative completions, are stored in the Preference Database (\textcircled{3}). The smaller Expert Agent\footnote{The dual-agent structure enables the smaller LLM to memorize preferences while the larger LLM automates schedules.}, compared to the Plan Agent, utilizes this feedback and memorized domain knowledge to ensure schedules align with dynamic project requirements (\textcircled{4}), supporting robust and adaptive scheduling.

% The Construction RLHF pipeline (Figure~\ref{fig:constructa_cpa}) integrates dynamic feedback and expert preferences to refine construction schedules. It begins with raw construction contexts and rules (\textcircled{1}), which the Plan Agent refines by incorporating task-specific details and context retrieved through SRAG and KRAG (\textcircled{2}). The outputs, consisting of refined contexts and planning rules, are evaluated as positive or negative completions and stored in the Preference Database for preference alignment (\textcircled{3}). The smaller Expert Agent leverages this feedback to further enhance the outputs, relying on memorized domain-specific knowledge to ensure schedules align with dynamic project requirements (\textcircled{4}). By iteratively integrating these elements, the pipeline dynamically adjusts to project-specific constraints, supporting robust and adaptive scheduling.

\textbf{CPA-RLHF} acts as the overarching framework, transforming the initial construction schedule into a dynamic environment for offline reinforcement learning. This is achieved by masking certain ground-truth values to simulate real-world uncertainties, effectively leveraging the expertise of architects in providing feedback on schedule optimization. The masked environment serves as a feedback loop where evaluated completions inform the refinement of the preference model. This process enables CPA-RLHF to address complex scheduling requirements by integrating domain knowledge, contextual adjustments, and expert preferences.

Within this framework, \textbf{CPA-DPO} refines the preference alignment process through supervised fine-tuning (SFT) and direct preference optimization. SFT establishes an initial alignment by minimizing the cross-entropy loss \( L_{\text{SFT}} = -\frac{1}{N} \sum_{i=1}^{N} y_i \log p_i \), grounding the model in expert-labeled schedules to produce coherent and contextually relevant outputs. Building on this, the preference alignment phase optimizes the total loss \( L_{\text{total}} = L_{\text{SFT}} + \alpha L_{\text{CR}} + \beta L_{\text{PA}} \), where \( \alpha \) and \( \beta \) balance contributions from Context-Rule Interaction Loss (\( L_{\text{CR}} \)) and Preference Alignment Loss (\( L_{\text{PA}} \)). The latter, defined as \( L_{\text{PA}} = -\frac{1}{N} \sum_{i=1}^{N} \left( y_i \log(p_i) + (1 - y_i) \log(1 - p_i) \right) \), ensures model outputs align with expert-defined preferences while respecting project constraints. This integrated approach enables the model to dynamically adapt to construction complexities, improving task prioritization and resource allocation.


% The final schedule \( S \) minimizes project completion time \( \sum_{i \in V} T_i \), subject to task constraints \( r_i(S) = \text{true}, \forall i \in V \). This hybrid approach generates schedules that respect expert preferences and project constraints, enabling adaptive and efficient construction scheduling.

% \subsection{Construction RLHF}

% The scheduling process is further refined using Reinforcement Learning from Human Feedback (RLHF), specifically Direct Preference Optimization (DPO). This involves aligning generated schedules with expert-defined preferences through a two-step training process.

% \subsubsection{Supervised Fine-Tuning (SFT)}

% During SFT, the model is fine-tuned using expert-labeled schedules to minimize the cross-entropy loss, defined as \( L_{\text{SFT}} = -\frac{1}{N} \sum_{i=1}^{N} y_i \log p_i \), where \( y_i \) is the true label and \( p_i \) is the predicted probability for task \( i \).

% \subsubsection{Direct Preference Optimization (DPO)}

% After SFT, the model undergoes DPO training to align outputs with expert preferences while adhering to contextual constraints. The total loss \( L_{\text{total}} \), combining SFT loss, Context-Rule Interaction Loss \( L_{\text{CR}} \), and Preference Alignment Loss \( L_{\text{PA}} \), is defined as \( L_{\text{total}} = L_{\text{SFT}} + \alpha L_{\text{CR}} + \beta L_{\text{PA}} \), where \( \alpha \) and \( \beta \) control the balance of loss components. The schedule \( S \) is then optimized to minimize the total project completion time \( \sum_{i \in V} T_i \), subject to all task constraints \( r_i(S) = \text{true} \, \forall i \in V \), ensuring adherence to project requirements.

% This hybrid approach enables CPA to incorporate domain knowledge, contextual relationships, and expert preferences into a robust, adaptive framework for large-scale construction scheduling.


% \section{Methodology}
% Our methodology combines data-driven context extraction, rule-based optimization, and preference alignment to create an adaptive scheduling framework. This approach captures project dependencies and expert preferences through structured data collection, context embedding, and reinforcement learning, providing a robust foundation for responsive construction scheduling.

% \subsection{Data Collection and Context Embedding}

% We first collect construction project data, including work breakdown structure (WBS), activity dependencies, and site conditions. This data is used to construct a dependency graph, where each node represents a construction activity. Each edge in the graph represents the dependency between activities, forming a directed acyclic graph (DAG). We apply pre-trained transformer models to generate context embeddings for each node. The embedding \( e_i \) for a node \( i \) is computed as:

% \begin{equation}
%     e_i = f_\text{embed}(X_i)
% \end{equation}

% where \( X_i \) is the raw feature vector for activity \( i \), and \( f_\text{embed} \) represents the embedding model (e.g., Sentence-Transformer or BERT).

% The similarity between two activities in terms of their construction context is calculated using cosine similarity:

% \begin{equation}
%     \text{sim}(e_i, e_j) = \frac{e_i \cdot e_j}{\|e_i\| \|e_j\|}
% \end{equation}

% This embedding-based approach provides the CPA system with a contextualized understanding of each task’s relationship to the project, ensuring that generated schedules accurately reflect dependencies and project constraints.

% \subsection{Unified Sampling of Contexts}
% To ensure that all relevant dependencies are considered in the scheduling process, we use a unified sampling method to extract three types of contexts:
% - Sequential Context: Extracts predecessors and successors up to 3 hops.
% - Hierarchical Context: Extracts nodes within the same WBS, up to 2 hops.
% - First-Order Context: Direct predecessors and successors.

% Let \( G = (V, E) \) represent the dependency graph, where \( V \) is the set of activities, and \( E \) is the set of dependencies. The set of sampled contexts for a node \( i \) can be written as:

% \begin{equation}
%     C_i = \{ \text{Sequential}(i), \text{Hierarchical}(i), \text{FirstOrder}(i) \}
% \end{equation}

% These contexts are used as input for rule generation and optimization.

% The combined sampling approach enables CPA to account for multiple dependency types simultaneously, enhancing its ability to generate context-sensitive schedules tailored to project specifics.

% \subsection{Rule Generation and Schedule Prediction}
% Using Azure OpenAI, we generate construction rules based on context embeddings and predefined prompts. These rules define constraints on task scheduling, dependencies, and resource allocation. The Plan Agent uses these rules to dynamically adjust the construction schedule. The process of rule generation is modeled as a sequence-to-sequence task:

% \begin{equation}
%     r = g_\text{rule}(C_i)
% \end{equation}

% where \( C_i \) is the combined context for node \( i \), and \( g_\text{rule} \) is the rule generation model based on LLMs.

% The predicted schedule \( S \) is generated by minimizing the total project completion time while satisfying all constraints:

% \begin{equation}
%     S = \arg \min \sum_{i \in V} T_i \quad \text{s.t.} \quad r_i(S) = \text{true} \ \forall i \in V
% \end{equation}

% where \( T_i \) is the completion time of activity \( i \), and \( r_i \) represents the rules generated for activity \( i \).

% This rule-based optimization approach ensures that the schedule adheres to project constraints while being adaptable to changes, a crucial advantage for complex construction projects.

% \section{Plan Agent and DPO Model}
% The Plan Agent integrates raw construction context, filtered rules, and expert feedback to generate optimized schedules. The core of this system is a language model fine-tuned using a two-stage training process: initial Supervised Fine-Tuning (SFT) followed by Direct Preference Optimization (DPO). The objective is to align the generated schedules with expert preferences while ensuring adherence to contextual and rule-based constraints.

% This hybrid training approach is designed to produce schedules that align with the nuances of expert preferences, offering a more human-centered, flexible, and adaptable scheduling solution.

% \subsection{Training Process Overview}
% The training process consists of two main stages:

% \begin{enumerate}
%     \item \textbf{Supervised Fine-Tuning (SFT) Initialization}: The model is first fine-tuned on a subset of data where the generated schedules exactly match the ground truth provided by experts. This step initializes the model parameters toward producing accurate and contextually appropriate outputs.

%     \item \textbf{Direct Preference Optimization (DPO~\cite{xu2024dpo}) Training}: Following SFT, the model undergoes DPO training using a novel loss function that promotes preference alignment and adherence to contextual and rule-based information.
% \end{enumerate}

% This two-phase approach allows CPA to prioritize human-defined preferences while maintaining a strong foundation in accurate and contextually relevant scheduling practices.

% \subsection{Supervised Fine-Tuning (SFT) Initialization}
% During the SFT stage, the model is trained to minimize the \textbf{Language Modeling Loss} ($L_{\text{LM}}$) on data samples with exact matches to expert schedules. The loss is calculated using cross-entropy between the predicted tokens and the ground truth:

% \begin{equation}
%     L_{\text{SFT}} = L_{\text{LM}}
% \end{equation}

% This initialization phase ensures that the model is capable of generating coherent, expert-aligned schedules from the start, providing a foundation for more refined preference alignment through DPO.

% \subsection{Direct Preference Optimization (DPO) Training}
% After SFT initialization, the model is further trained using the DPO approach. To effectively model the interaction between \textbf{context} and \textbf{rules} during preference alignment, we introduce a novel loss function, $L_{\text{total}}$. This loss promotes preference-aligned outputs by enhancing the model’s adherence to both contextual and rule-based information. It comprises three components: \textbf{Language Modeling Loss} ($L_{\text{LM}}$), \textbf{Context-Rules Interaction Loss} ($L_{\text{CR}}$), and \textbf{Preference Alignment Loss} ($L_{\text{PA}}$).

% \begin{equation}
%     L_{\text{total}} = L_{\text{LM}} + \alpha \cdot L_{\text{CR}} + \beta \cdot L_{\text{PA}}
% \end{equation}

% where $\alpha$ and $\beta$ balance the importance of each component.

% By integrating these losses, the DPO model learns to prioritize expert preferences in ways that enhance schedule quality, aligning generated outputs with both human and project requirements.

% \subsubsection{Language Modeling Loss}
% The \textbf{Language Modeling Loss} ($L_{\text{LM}}$) ensures coherence in the generated language, calculated using cross-entropy loss.

% \subsubsection{Context-Rules Interaction Loss}
% The \textbf{Context-Rules Interaction Loss} ($L_{\text{CR}}$) encourages alignment between context and rule-based constraints through a contrastive mechanism:

% \begin{equation}
%     \begin{aligned}
%     L_{\text{CR}} = \frac{1}{N} \sum_{i=1}^{N} \max \Big(0, & \, d(f_{\text{context}}(x_i), f_{\text{rule}}(x_i)) \\
%     & - f_{\text{output}}(x_i) + \delta \Big)
%     \end{aligned}
% \end{equation}

% where:
% \begin{itemize}
%     \item $f_{\text{context}}(x_i)$ and $f_{\text{rule}}(x_i)$ are embeddings for the context and rule constraints,
%     \item $f_{\text{output}}(x_i)$ represents the model's output embedding,
%     \item $d(\cdot, \cdot)$ measures alignment (e.g., cosine similarity),
%     \item $\delta$ is a margin controlling sensitivity to discrepancies.
% \end{itemize}

% This loss term is critical for ensuring the model respects context-rule alignment in the generated outputs, supporting accurate scheduling decisions.

% \subsubsection{Preference Alignment Loss}
% The \textbf{Preference Alignment Loss} ($L_{\text{PA}}$) is a binary cross-entropy loss comparing generated responses to preference labels:

% \begin{equation}
%     L_{\text{PA}} = -\frac{1}{N} \sum_{i=1}^{N} \left( y_i \log(p_i) + (1 - y_i) \log(1 - p_i) \right)
% \end{equation}

% where $y_i$ denotes the preference label for output $x_i$, and $p_i$ is the model's predicted probability of preference alignment.

% This final loss component ensures that generated schedules align closely with human-defined preferences, optimizing project timelines and task prioritization in line with expert guidance.

% \subsection{Parameter-Efficient Fine-Tuning with LoRA}
% To efficiently fine-tune the language model without updating all parameters, we employ Low-Rank Adaptation (LoRA). LoRA adds trainable rank-decomposed weight matrices to existing weights, signiAficantly reducing the number of trainable parameters. This approach enables effective fine-tuning while keeping computational costs manageable.

% LoRA’s lightweight fine-tuning strategy allows for flexible and efficient updates to the model, essential for handling the diverse requirements of large-scale construction projects.

% \subsection{Implementation Details}
% The training process leverages several techniques to optimize performance:

% \begin{itemize}
%     \item \textbf{4-bit Quantization}: The base model is quantized to 4-bit precision using the \texttt{BitsAndBytes} library, reducing memory footprint and computational requirements.
%     \item \textbf{Gradient Checkpointing}: Enabled to save memory during training by trading computation for memory.
%     \item \textbf{Mixed Precision Training}: Utilized to accelerate training and reduce memory usage through lower-precision computations where appropriate.
%     \item \textbf{Supervised Fine-Tuning before DPO}: The initial SFT provides a strong starting point, enhancing the effectiveness of subsequent DPO training.
%     \item \textbf{Optimization}: The AdamW optimizer is used with a learning rate tuned for effective convergence on the reduced parameter set.
% \end{itemize}

% These techniques together ensure a smooth, computationally efficient training process that allows CPA to maintain optimal performance under resource constraints.

% \section{Experiments}

% To evaluate the effectiveness of our proposed framework, we conducted experiments focusing on its performance in optimizing construction schedules. These tests measure the system's ability to improve task efficiency and align scheduling with expert preferences across various construction activities.

% \subsection{Dataset and Setup}
% We conducted experiments using data from multiple semiconductor fabrication projects. The data includes tasks such as HVAC installation, chemical piping, and cleanroom preparation. For each activity, we collected WBS information, task dependencies, and resource requirements. The system was tested on its ability to generate optimized schedules that minimize delays and align with expert feedback.

% This setup provides a comprehensive testing ground for evaluating the CPA framework, allowing us to assess its performance across varied, realistic project conditions.

% \subsection{Construction RLHF}

% The scheduling process is refined using Reinforcement Learning from Human Feedback (RLHF), with Direct Preference Optimization (DPO) as its core mechanism. To initialize the model, we employ Supervised Fine-Tuning (SFT) as a practical trick, briefly fine-tuning the model on expert-labeled schedules. This establishes a strong baseline, allowing the DPO stage to focus on aligning outputs with expert preferences.

% \subsubsection{Preference Alignment for Construction Scheduling}

% In the DPO stage, we align generated schedules with expert-defined preferences while respecting contextual and rule-based constraints. The total loss is defined as:

% \[
% L_{\text{total}} = \alpha L_{\text{CR}} + \beta L_{\text{PA}},
% \]

% where \( L_{\text{CR}} \), the Context-Rule Interaction Loss, ensures alignment between contextual embeddings and rule constraints, and \( L_{\text{PA}} \), the Preference Alignment Loss, aligns outputs with human-defined preferences. 

% The Preference Alignment Loss is formulated as a binary cross-entropy loss:

% \[
% L_{\text{PA}} = -\frac{1}{N} \sum_{i=1}^{N} \left( y_i \log(p_i) + (1 - y_i) \log(1 - p_i) \right),
% \]

% where \( y_i \) denotes the preference label for output \( x_i \), and \( p_i \) is the model's predicted probability of alignment with the expert's preference. 

% To optimize the final schedule \( S \), we minimize the total project completion time while ensuring all task constraints are met:

% \[
% S = \arg \min \sum_{i \in V} T_i \quad \text{s.t.} \quad r_i(S) = \text{true}, \, \forall i \in V,
% \]

% where \( T_i \) is the completion time of activity \( i \), and \( r_i(S) \) represents the rule-based constraints satisfied by the schedule \( S \). This formulation ensures that generated schedules are both efficient and aligned with expert-driven project requirements, enabling robust and adaptive construction scheduling.

\section{Experimental Design}

This section outlines the experimental configurations for Static RAG, Knowledge RAG and Construction RLHF, emphasizing embedding methods, model configurations, and optimization strategies.

\textbf{Static and Knowledge RAG} The SRAG setup used 500-token chunks for efficient processing, with embeddings generated via \texttt{all-MiniLM-L6-v2}\footnote{\url{https://huggingface.co/sentence-transformers/all-MiniLM-L6-v2}}. Static RAG focused on terminologies and definitions, while Knowledge RAG retrieved context from manuals and domain-specific references.

\textbf{Construction RLHF} The Plan Agent used GPT-4o~\cite{islam2024gpt}, and the Expert Agent employed Llama3.2-3B model~\cite{touvron2023llama} for expert preference alignment. Training involved 10 epochs of SFT for initialization, followed by 10 epochs of CPA-DPO for preference refinement. The trained Expert Agent supported contextual refinements. 

\textbf{LLM Training Configuration} Efficient training was achieved using 4-bit quantization, gradient checkpointing, mixed precision training, and the AdamW optimizer~\cite{zhuang2022understanding}. Data collection employed a random seed of 42, while inference utilized a seed of 12345, ensuring the generation of diverse datasets to enhance generalizability.

\textbf{Prompt Design} Comprehensive prompt categories tailored for each task are provided in the appendix to address construction-specific challenges effectively. Each result reflects the top-2 predictions (\( k = 2 \)) for enhanced accuracy, with Construction RLHF ensembled with KRAG to combine expert alignment and domain-specific knowledge retrieval.

\section{Result and Analysis}

We evaluate \TheName{} across key scheduling tasks, highlighting its ability to address complex dependencies, handle missing data, and align schedules with expert-defined constraints.

\subsection{Evaluation Metrics}

\TheName{} is evaluated using three key metrics to assess its ability to predict missing elements in construction schedules while ensuring logical consistency and expert alignment.

\textbf{Missing Value Prediction (MVP)} measures the model’s ability to reconstruct values from three randomly removed columns. This tests its capability to handle incomplete data while preserving schedule coherence and minimizing disruptions caused by missing information.

\textbf{Dependency Analysis (DA)} evaluates prediction accuracy for relational columns, including Activity Status, Level, Area, and Discipline. Since these dependencies define task sequencing and workflow constraints, this metric ensures that predicted schedules maintain logical task relationships and prevent inconsistencies.

\textbf{Automated Planning (AP)} assesses the model’s ability to predict Current Start and Current Finish dates while considering real-world constraints. It measures how well the generated schedules align with expert workflows, resource availability, and project feasibility to ensure practical execution.

% \TheName{} is evaluated using three key metrics, each designed to assess its ability to predict missing elements in construction schedules while ensuring coherence, logical consistency, and alignment with expert-driven planning principles.

% \textbf{Missing Value Prediction (MVP)} evaluates the model's ability to reconstruct values from three randomly removed columns. This tests the system's robustness in handling incomplete data by ensuring schedule completeness while maintaining consistency with available project information.

% \textbf{Dependency Analysis (DA)} assesses the prediction accuracy for relational columns, including Activity Status, Level, Area, and Discipline. Since construction schedules rely on well-defined dependencies, this metric measures the model's capability to infer and preserve logical task relationships, preventing inconsistencies in execution sequences.

% \textbf{Automated Planning (AP)} focuses on predicting the Current Start and Current Finish columns, which are critical for scheduling accuracy. This metric evaluates the extent to which the model-generated schedules adhere to real-world project constraints, expert-defined workflows, and overall project feasibility, ensuring alignment with practical construction planning requirements.


% \subsection{Evaluation Metrics}

% \TheName{} is evaluated using three metrics, each testing its ability to predict missing elements in construction schedules:

% \textbf{Missing Value Prediction (MVP)} evaluates the model's ability to reconstruct values from three randomly removed columns. This ensures the completeness of the construction schedule and tests the system's capacity to handle missing data.

% \textbf{Dependency Analysis (DA)} assesses the prediction accuracy for relational columns, such as Activity Status, Level, Area, and Discipline. This metric measures the model's capability to understand and maintain logical sequencing of tasks.

% \textbf{Automated Planning (AP)} focuses on predicting the Current Start and Current Finish columns. It evaluates the alignment of generated schedules with project constraints and expert preferences, ensuring practicality and accuracy in planning.

% The CPA system demonstrated a 15\% reduction in delays for HVAC installations and a 10\% improvement in chemical piping tasks. These improvements highlight the effectiveness of CPA in achieving faster and more compliant schedules compared to traditional methods, showcasing the benefits of preference-based optimization.

\begin{table}[t]
\centering
\resizebox{\columnwidth}{!}{
\begin{tabular}{l|ccc|c}
\toprule
Model Config            & MVP (\%)         & DA (\%)         & AP (\%)         & Avg (\%) \\ \midrule
GPT-4o~(Basic Context)  & 14.6             & 3.1             & 8.4             & 8.7      \\ \midrule
+ Static RAG            & 11.6             & 1.6             & 12.5            & 8.6      \\
+ Knowledge RAG         & 51.4             & 77.9            & 25.9            & 51.7     \\
+ Construction RLHF     & 56.9             & 82.2            & 37.3            & 58.8     \\ \midrule
Gain (\TheName{} vs. BC) & \textbf{+42.3}  & \textbf{+79.1}  & \textbf{+28.9}  & \textbf{+50.1}  \\ 
\bottomrule
\end{tabular}
}
% \caption{Performance comparison of various pretraining configurations for construction schedule optimization.}
\caption{Performance comparison of pretraining configurations for construction schedule optimization. 
\textbf{Basic Context (BC)} refers to GPT-4o without retrieval augmentation or RLHF, relying only on general pretraining knowledge by sampling random rows as context.}
\label{tab:ablation_study_construction}
\end{table}

\begin{table*}[h]
\centering
\resizebox{\textwidth}{!}{
\begin{tabular}{l|l|cccc|cccc|cccc}
\toprule
Group & Discipline & \multicolumn{4}{c|}{MVP (\%)} & \multicolumn{4}{c|}{DA (\%)} & \multicolumn{4}{c}{AP (\%)} \\ 
 &  & BC & SRAG & KRAG & RLHF & BC & SRAG & KRAG & RLHF & BC & SRAG & KRAG & RLHF \\ \midrule
CSA & CSA.Arch.Arch-D & 5.6 & 4.4 & 23.3 & 25.6 & 1.7 & 2.5 & 39.2 & 43.3 & 0.8 & 5.0 & 15.0 & 19.2 \\
 & CSA.Arch.CRCs-D & 6.7 & 6.7 & 40.0 & 40.0 & 0.0 & 0.0 & 65.0 & 65.0 & 0.0 & 0.0 & 17.5 & 20.0 \\
 & CSA.Arch.Metal & 6.5 & 4.5 & 32.7 & 37.8 & 2.1 & 0.7 & 61.0 & 64.2 & 3.1 & 7.5 & 10.8 & 16.8 \\
 & CSA.Arch.RF & 9.5 & 6.3 & 38.1 & 42.9 & 0.9 & 0.6 & 49.7 & 53.3 & 2.1 & 6.0 & 8.0 & 17.3 \\ 
 & CSA.Arch.WPRF & 0.0 & 0.0 & 33.3 & 33.3 & 0.0 & 0.0 & 25.0 & 25.0 & 0.0 & 25.0 & 0.0 & 0.0 \\ 
 & CSA.Civil.Earthwork & 11.3 & 6.9 & 32.8 & 38.2 & 1.6 & 0.4 & 47.2 & 52.0 & 3.6 & 6.9 & 10.1 & 16.5 \\ 
 & CSA.Struc.Concrete & 8.8 & 7.4 & 29.2 & 33.0 & 1.1 & 1.3 & 35.5 & 39.1 & 4.6 & 6.7 & 12.9 & 20.0 \\ 
 & CSA.Struc.Modules & 6.2 & 5.6 & 37.1 & 41.7 & 2.7 & 0.5 & 61.8 & 66.5 & 3.8 & 6.9 & 11.6 & 19.7 \\
 & CSA.Struc.Piers & 7.5 & 6.7 & 30.0 & 36.7 & 0.0 & 0.0 & 45.0 & 47.5 & 7.5 & 5.0 & 22.5 & 32.5 \\
 & CSA.Struc.Steel & 8.0 & 7.8 & 30.8 & 34.0 & 1.7 & 1.0 & 50.3 & 53.6 & 3.5 & 7.4 & 15.4 & 21.6 \\ 
 & CSA.Struc.Strut & 9.0 & 5.6 & 33.5 & 37.9 & 1.8 & 0.8 & 60.1 & 64.9 & 6.5 & 5.8 & 18.2 & 27.1 \\
 \midrule
MEP & MEP.Mech.Dry & 4.6 & 6.4 & 37.6 & 38.8 & 2.5 & 0.4 & 65.7 & 68.0 & 4.6 & 6.4 & 18.2 & 25.4 \\
 & MEP.Mech.Wet & 0.0 & 0.0 & 66.7 & 66.7 & 0.0 & 0.0 & 75.0 & 75.0 & 0.0 & 0.0 & 50.0 & 50.0 \\ 
 & MEP.Proc.HP & 3.2 & 5.4 & 33.3 & 40.8 & 1.4 & 0.2 & 61.5 & 66.5 & 3.2 & 5.4 & 14.0 & 23.7 \\ 
 & MEP.Proc.LP & 4.3 & 4.8 & 35.2 & 39.3 & 1.6 & 0.5 & 61.1 & 68.2 & 4.3 & 6.9 & 14.6 & 22.9 \\ 
 & MEP.Proc.Vac & 7.5 & 5.2 & 32.4 & 36.7 & 1.1 & 0.7 & 57.5 & 63.9 & 7.5 & 6.8 & 19.6 & 31.1 \\ 
 & MEP.Proc.Waste & 7.9 & 6.4 & 33.1 & 38.8 & 1.4 & 0.4 & 63.0 & 67.9 & 5.2 & 6.8 & 18.2 & 25.2 \\ 
 & MEP.Proc.Water & 5.7 & 6.4 & 37.6 & 38.8 & 3.0 & 0.5 & 65.7 & 70.5 & 3.8 & 7.6 & 14.8 & 23.2 \\
 \midrule
 Avg & & 6.2 & 5.2 & 36.4 & 40.2 & 1.5 & 0.6 & 55.8 & 59.2 & 3.7 & 6.6 & 17.4 & 24.3 \\
\bottomrule
\end{tabular}
}
% \caption{Grouped performance comparison across construction schedule optimization tasks.}
\caption{Grouped performance comparison across construction schedule optimization tasks. 
SRAG retrieves domain-specific definitions, KRAG structures context using activity relationships, and RLHF aligns predictions with expert feedback. Results show notable gains in MVP, DA, and AP, especially in CSA and MEP disciplines.}
\label{tab:cross_column_ablation_study_with_rlhf_full}
\end{table*}

% \subsection{Overall Performance Gains}

% Table~\ref{tab:ablation_study_construction} demonstrates the overall performance improvements of \TheName{} across MVP, DA, and AP tasks. Static RAG shows limited impact, with marginal or decreased performance, indicating that basic retrieval alone is insufficient for complex scheduling tasks. Adding Knowledge RAG significantly boosts MVP and DA by incorporating contextualized domain knowledge, improving the model’s understanding of dependencies. The integration of Construction RLHF achieves the highest gains, with MVP improving by +42.3\%, DA by +79.1\%, and AP by +28.9\%. These results highlight the effectiveness of \TheName{} in capturing intricate scheduling constraints, refining predictions, and adapting to dynamic construction workflows.

\subsection{Overall Performance Gains}

Table~\ref{tab:ablation_study_construction} demonstrates the overall performance improvements of \TheName{} across MVP, DA, and AP tasks. Static RAG shows limited impact, with marginal or decreased performance, as it provides domain knowledge without contextual adaptation. Knowledge RAG boosts MVP and DA by incorporating task-specific dependencies, improving inference of missing values and logical sequencing. Construction RLHF achieves the highest gains, improving MVP by +42.3\%, DA by +79.1\%, and AP by +28.9\%. These results highlight the effectiveness of \TheName{} in addressing complex construction scheduling tasks.

% Table~\ref{tab:ablation_study_construction} demonstrates the overall performance improvements of \TheName{} across MVP, DA, and AP tasks. Static RAG shows limited impact, with marginal or decreased performance. Adding Knowledge RAG significantly boosts MVP and DA, while integrating Construction RLHF achieves the highest gains, improving MVP by +42.3\%, DA by +79.1\%, and AP by +28.9\%. These results highlight the effectiveness of \TheName{} in addressing complex construction scheduling tasks.

\begin{figure*}[h]
    \centering
    \includegraphics[width=\textwidth]{figs/performance_comparison_beautified.pdf}
    \caption{Performance Comparison Across Levels and Areas. This plot shows the performance of various metrics, including Basic Context, Static RAG, Knowledge RAG, and Construction RLHF, for three tasks (Automated Planning, Dependency Analysis, and Missing Value Prediction) across different levels and areas.}
    \label{fig:performance_comparison}
\end{figure*}

\begin{figure*}[h]
    \centering
    \includegraphics[width=\textwidth]{figs/distribution_by_source_with_dpo_output.pdf}
    \caption{Context length distributions for AP, DA, and MVP sources, highlighting reductions achieved through CPA-DPO. The shorter contexts effectively maintain performance while improving efficiency in schedule optimization.}
    \label{fig:distribution_by_source}
\end{figure*}

\subsection{Construction Disciplines, Levels, and Areas in Evaluation}

Effective construction scheduling depends on disciplines, structural levels, and spatial areas, each with unique dependencies. We evaluate \TheName{} across these dimensions to ensure adaptability to real-world constraints.

\textbf{Disciplines} Construction projects encompass Civil, Structural, and Architectural (CSA) and Mechanical, Electrical, and Plumbing (MEP) disciplines. CSA tasks, such as structural assemblies and load-bearing elements, require precise sequencing for stability. MEP tasks, including waste processing and high-pressure systems, demand coordinated integration for efficient infrastructure.  

\textbf{Levels} Evaluation covers Equipment (EQ), Utility Level (UL), Standard Floor (SF), and Roof Floor (RF). SF and RF are the most complex, with RF requiring detailed sequencing for reinforcements and installations.  

\textbf{Areas} Performance is analyzed in construction zones such as 6E, 9E, and SU. High-complexity areas like SU E and 10E have dense interdependencies, making effective scheduling essential for coordination and resource optimization.  

% Effective construction scheduling depends on multiple factors, including distinct disciplines, structural levels, and spatial areas, each with unique dependencies. To evaluate \TheName{}, we analyze its performance across these key dimensions to ensure adaptability to real-world construction constraints.

% \textbf{Disciplines} Construction projects involve Civil, Structural, and Architectural (CSA) and Mechanical, Electrical, and Plumbing (MEP) disciplines. CSA tasks, such as structural assemblies and load-bearing elements, require precise sequencing for stability. MEP tasks, including waste processing and high-pressure systems, demand intricate coordination for efficient infrastructure integration.  

% \textbf{Levels} Evaluation includes Equipment (EQ), Utility Level (UL), Standard Floor (SF), and Roof Floor (RF). SF and RF present the most complexity, with RF requiring detailed sequencing for structural reinforcements and installations.  

% \textbf{Areas} Performance is analyzed in construction zones such as 6E, 9E, and SU. High-complexity areas like SU E and 10E have dense interdependencies, making effective scheduling critical for coordination and resource optimization.  

\subsection{Performance by Discipline}

The grouped results in Table~\ref{tab:cross_column_ablation_study_with_rlhf_full} provide insights into \TheName{}’s performance across construction disciplines. For CSA disciplines, including CSA.Struc.Modules and CSA.Struc.Piers, \TheName{} excels in accurately modeling dependencies and generating optimized schedules, effectively addressing challenges such as sequencing structural assemblies, ensuring load-bearing integrity, and maintaining alignment with construction constraints. 

Similarly, for MEP disciplines, including MEP.Proc.Waste and MEP.Proc.HP, significant improvements are observed in DA and AP, demonstrating \TheName{}’s ability to capture intricate interdependencies between mechanical, electrical, and plumbing systems. This highlights the model’s robustness in specialized workflows where precise coordination of installations and operational constraints is critical to overall project efficiency.

% \subsection{Performance by Discipline}

% The grouped results in Table~\ref{tab:cross_column_ablation_study_with_rlhf_full} provide insights into \TheName{}’s performance across construction disciplines. For CSA disciplines, such as CSA.Struc.Modules and CSA.Struc.Piers, \TheName{} excels in accurately modeling dependencies and generating optimized schedules, addressing challenges like sequencing structural assemblies. Similarly, for MEP disciplines, such as MEP.Proc.Waste and MEP.Proc.HP, significant improvements are observed in DA and AP, showcasing the robustness of \TheName{} in specialized workflows.

\subsection{Performance by Level and Area}

Figure~\ref{fig:performance_comparison} compares performance across construction levels (EQ, UL, SF, RF) and areas (6E, 9E, SU). \TheName{} consistently outperforms other methods across all categories, demonstrating its ability to adapt to varying spatial and structural complexities. 

For levels, the largest improvements are observed in SF and RF, highlighting the model's capability to handle complex roof-level dependencies, structural reinforcements, and standard floor operations with greater accuracy. The gains in RF indicate that \TheName{} effectively accounts for elevated sequencing constraints and installation workflows that are more intricate at higher levels.

For areas, \TheName{} achieves the highest gains in zones with high complexity, such as SU E and 10E, where interdependencies between tasks are more intricate. This suggests that \TheName{} effectively learns and adapts to localized construction constraints, optimizing sequencing and resource allocation in highly constrained or densely coordinated zones.

% \subsection{Performance by Level and Area}

% Figure~\ref{fig:performance_comparison} compares performance across construction levels (e.g., EQ, UL, SF, RF) and areas (e.g., 6E, 9E, SU). \TheName{} consistently outperforms other methods across all categories. For levels, the largest improvements are observed in SF and RF, reflecting the model's capability to handle complex roof-level dependencies and standard floor operations. For areas, \TheName{} achieves the highest gains in zones with high complexity, such as SU E and 10E, where interdependencies are more intricate. 

\subsection{Knowledge Distillation and Observations}

Figure~\ref{fig:distribution_by_source} shows the reduced context length after CPA-DPO alignment, demonstrating effective knowledge distillation from the Plan Agent to the Expert Agent. By filtering out redundant details and retaining only essential scheduling constraints, \TheName{} enhances efficiency while preserving decision-making accuracy. By prioritizing critical dependencies, it enables more precise scheduling adjustments and minimizes the risk of misaligned task sequencing.

% Figure~\ref{fig:distribution_by_source} shows the reduced context length after CPA-DPO alignment, demonstrating effective knowledge distillation from the Plan Agent to the Expert Agent. By filtering out redundant details and retaining only essential scheduling constraints, \TheName{} enhances efficiency while preserving decision-making accuracy. 

Another key observation is that \TheName{} refines scheduling inputs by reducing context length while preserving essential constraints. CPA-DPO alignment streamlines DA and MVP, filtering excess details that obscure dependencies. This distillation enhances adaptability by emphasizing key relational structures, improving interpretability and alignment with industry requirements.

% A key observation is that \TheName{} learns to extract domain-relevant constraints, generating concise yet informative scheduling recommendations. This targeted distillation improves adaptability to complex project structures, enabling the model to generalize across varying construction scenarios while maintaining consistency in decision-making. Additionally, the process reinforces model interpretability, making it easier to analyze and validate predictions, which is crucial for industry adoption.

\section{Future Applications and Industry Adoption}

\TheName{} presents strong potential for LLM adoption in construction scheduling, improving automation, adaptability, and decision support. Traditional methods struggle with real-time changes, while \TheName{} continuously refines schedules based on evolving constraints~\cite{pan2021automated,neelamkavil2009automation}. By learning from historical schedules and domain-specific constraints, it optimizes resource allocation, mitigates conflicts, and enhances project execution.

For broader adoption, \TheName{} can integrate with existing construction management software as an intelligent planning tool. Its ability to handle dynamic scheduling and dependency modeling makes it valuable for large-scale projects. Future work will address deployment challenges, including computational efficiency, latency, and seamless integration with industry platforms~\cite{zhang2023rule,amer2023construction}, ensuring scalability for commercial applications such as semiconductor fabrication.

% \subsection{Knowledge Distillation and Observations}

% Figure~\ref{fig:distribution_by_source} shows the reduced context length after CPA-DPO alignment, demonstrating effective knowledge distillation from the Plan Agent to the Expert Agent. This process retains only essential scheduling constraints, improving model efficiency without compromising performance.  

% One key observation from this process is that \TheName{} learns to extract domain-relevant scheduling constraints while filtering out redundant details. This targeted distillation allows the model to generate concise yet informative scheduling recommendations, enhancing adaptability to complex project structures while maintaining decision consistency. The distilled knowledge not only streamlines inference but also reinforces model interpretability, making \TheName{} well-suited for dynamic construction environments.

% \section{Future Applications and Industry Adoption}

% \TheName{} presents a strong case for LLM adoption in the construction industry, demonstrating substantial improvements in automated scheduling tasks. Beyond academic validation, its practical impact lies in its ability to integrate with existing construction management software, where it can serve as an intelligent augmentation tool for project planning and execution. 

% One key area for industry adoption is the automation of real-time scheduling adjustments~\cite{pan2021automated,neelamkavil2009automation}. Traditional methods struggle with dynamically evolving project conditions, whereas \TheName{} can continuously refine schedules based on new constraints, improving adaptability. Additionally, its ability to learn from historical schedules and domain-specific constraints positions it as a valuable asset for optimizing resource allocation and mitigating scheduling conflicts. 

% Future work will focus on addressing practical deployment challenges, including computational efficiency, latency, and integration with industry-standard platforms~\cite{zhang2023rule,amer2023construction}. Ensuring that \TheName{} operates within real-world infrastructure constraints will be critical to its successful adoption in large-scale commercial projects such as semiconductor fabrication.

% \subsection{Knowledge Distillation and Future Applications}

% Figure~\ref{fig:distribution_by_source} shows the reduced context length after CPA-DPO alignment, achieving concise outputs while improving task performance. This reflects effective knowledge distillation from the Plan Agent to the Expert Agent, enabling \TheName{} to optimize schedules efficiently and laying the foundation for scalable, real-time industry-level recommender systems.

% Our system significantly improved construction schedule optimization compared to traditional methods. By integrating human preferences with context-aware rule generation, we achieved faster task completion times and higher alignment with expert feedback. The planned implementation of the DPO model is expected to further enhance this process, enabling more granular adjustments based on real-time data.

% \subsection{Analysis of CPA-RLHF for Schedule Automation}

% The results in Table~\ref{tab:cross_column_ablation_study_with_rlhf_full} illustrate the significant impact of \textbf{CPA-RLHF} on construction schedule automation across various tasks and disciplines. Compared to other methods like Basic Context (BC), Static RAG (SRAG), and Knowledge RAG (KRAG), CPA-RLHF achieves consistent and substantial improvements, showcasing its effectiveness in modeling contextual dependencies and optimizing schedules in commercial construction projects.

% Specifically, CPA-RLHF demonstrates its strength in the following areas:
% \begin{itemize}
%     \item \textbf{CSA.Struc.Modules}: This discipline focuses on assembling structural modules such as prefabricated steel components. In commercial construction, module interdependencies often require precise sequencing to ensure structural stability and efficient resource allocation. CPA-RLHF significantly improves Dependency Analysis (DA) accuracy, leveraging its ability to model these complex dependencies effectively.
%     \item \textbf{MEP.Proc.Waste}: This discipline deals with planning and managing systems for waste processing and disposal, a critical part of industrial and large-scale commercial projects. CPA-RLHF enhances Automated Planning (AP) by incorporating contextual reasoning to optimize task order and resource usage, which are essential for maintaining regulatory compliance and environmental standards.
% \end{itemize}

% These results highlight how CPA-RLHF addresses the unique challenges of commercial construction scheduling, particularly in disciplines that involve intricate dependencies or specialized workflows. Its capacity to adapt to varied construction contexts makes it a robust tool for advancing schedule automation in this domain.

% The results in Table~\ref{tab:cross_column_ablation_study_with_rlhf_full} highlight the effectiveness of CPA-RLHF in automating construction schedules across various tasks and disciplines. Compared to methods like Basic Context (BC), SRAG, and KRAG, \TheName{} consistently achieves superior performance, demonstrating its ability to model contextual dependencies and optimize schedules for complex commercial construction projects.

% \TheName{} excels in disciplines like CSA.Struc.Modules, where precise sequencing of structural module assembly is critical, and MEP.Proc.Waste, which involves optimizing waste management systems. By effectively modeling interdependencies and incorporating contextual reasoning, CPA-RLHF significantly enhances Dependency Analysis (DA) accuracy and Automated Planning (AP) efficiency, addressing the unique challenges of specialized workflows in commercial construction.


% % \paragraph{Principle for Selecting and Ordering Levels and Areas} 
% \textbf{Frequency of Representation:} Levels and areas that appear most frequently in the dataset are prioritized to ensure the analysis has sufficient data points for meaningful insights. 
% \textbf{Relevance to Construction Scenarios:} Critical levels such as RF (Roof Level) and SF (Standard Floor), which are significant in structural and construction tasks, are included. Similarly, areas like SU (subsections) and zones with high complexity (e.g., 10E, 9E) are selected to reflect realistic site-specific challenges. 
% \textbf{Impact on Pipeline:} Levels and areas where the Construction RLHF method demonstrates notable improvements are emphasized, ensuring the analysis captures the benefits of this approach. The levels (EQ, UL, SF, RF) are ordered hierarchically, moving upward from equipment to the roof, while areas are ordered geographically (e.g., from 6E to 10E) and clustered by site-specific sections like SU C, SU E, and SU W.

% % \paragraph{Insights from the Figure} 
% \textbf{Construction RLHF:} The performance comparison across levels and areas reveals that CPA-RLHF provides consistent improvements in schedule automation. While some tasks, levels, and areas show minimal differences between methods, CPA-RLHF demonstrates significant benefits in automated planning (AP) across all levels and areas, emphasizing its robustness in construction schedule optimization.

% The selection and ordering of levels and areas prioritize those with frequent representation in the dataset, high relevance to construction tasks, and where the Construction RLHF method shows notable improvements. Critical levels like RF (Roof Level) and SF (Standard Floor) and complex areas such as SU (subsections) and zones like 10E and 9E are included to reflect realistic construction challenges. The levels are ordered hierarchically from equipment to the roof, while areas are clustered geographically. The analysis demonstrates that CPA-RLHF consistently enhances automated planning, showing significant improvements across all levels and areas in construction scheduling.

% \section{Discussion}

% Modern recommender systems leverage reinforcement learning to adapt dynamically to user preferences in real-time, optimizing personalization through continuous updates. In contrast, the CPA-DPO framework applies preference-based reinforcement learning to construction scheduling, where dependencies are rigid and retraining occurs at fixed intervals. This fixed-interval approach aligns with construction project phases, enabling systematic updates while balancing computational efficiency and expert feedback integration.

% Unlike the continuous updates of recommender systems, CPA-DPO refines scheduling decisions periodically, ensuring adaptability to evolving project needs without overburdening production systems. By leveraging advancements in preference alignment, CPA-DPO demonstrates the potential to bring the dynamic capabilities of recommender systems into structured domains like construction, enhancing scalability and real-world applicability.

% \section{Discussion}

% \TheName{}'s dynamic adaptation to user preferences and real-time contexts makes it ideal for industry-scale recommender systems, especially in rapidly changing environments like scheduling. Its integration of static knowledge, context-aware embeddings, and preference alignment ensures flexibility and precision. This framework represents a visionary step toward intelligent, adaptive systems capable of transforming complex industrial workflows.

\section{Related Works}

Research on LLM-powered construction scheduling is limited, with prior work focusing on deterministic methods and RL in other domains~\cite{srivastava2022imperative,dashti2021integrated,bademosi2021factors,pan2021automated,li2021optimal}. This work pioneers construction automation using RAG and RLHF.

% \section{Related Works}

% Research on LLM-powered construction scheduling is limited, with prior work focusing on deterministic optimization and RL in related domains~\cite{srivastava2022imperative,dashti2021integrated,bademosi2021factors,pan2021automated,li2021optimal}, which struggle with dynamic project constraints. This work pioneers construction automation by integrating RAG and RLHF, enabling adaptive decision-making and improved dependency modeling.  

\subsection{Construction Automation}

Traditional construction automation has predominantly utilized deterministic scheduling algorithms~\cite{peiris2023production,khodabakhshian2023deterministic,peiris2023production} and rule-based systems~\cite{zhang2023rule,amer2023construction,augar2024rule}. While these methods are effective in static environments, they often fail to adapt to the dynamic and complex nature of real-world construction projects, which involve evolving dependencies and resource constraints~\cite{xie2023case,al2024generation,parekh2024automating,he2024real,huang2024cross}. Our approach addresses these limitations by integrating domain-specific knowledge and context, enabling more flexible and responsive scheduling.

\subsection{Retrieval-Augmented Generation~(RAG)}

Retrieval-Augmented Generation (RAG) techniques enhance language models by incorporating external knowledge sources, improving their ability to generate contextually relevant information~\cite{gao2023retrieval,chen2024benchmarking,jiang2024longrag,li2024malmixer,acharya2025optimizing}. However, existing RAG methods may not effectively retrieve and integrate the highly specialized and structured information required for construction scheduling~\cite{zhao2024retrieval,fan2024survey,barnett2024seven}. Our method overcomes this challenge by employing a static RAG framework tailored to the construction domain, ensuring the retrieval of precise and pertinent information that informs scheduling decisions.

% \subsection{Reinforcement Learning from Human Feedback~(RLHF)}

% Reinforcement Learning from Human Feedback (RLHF), including methods such as Direct Preference Optimization (DPO), aligns model outputs with human preferences through comparative feedback~\cite{wang2023rlhf,yang2024multi,dong2024rlhf,xu2024dpo,saeidi2024insights}. However, applying RLHF in traditional industries like construction remains challenging due to the need for domain-specific knowledge, complex dependencies, and expert-driven priorities~\cite{wang2024comprehensive,xiao2024comprehensive,feng2024towards}. While RLHF has been applied in various domains, its use in construction scheduling remains underexplored. Our approach extends DPO by incorporating construction-specific knowledge and structured context, resulting in schedules that better reflect expert preferences and project-specific requirements.

\subsection{Reinforcement Learning from Human Feedback~(RLHF)}

Reinforcement Learning from Human Feedback (RLHF), including Direct Preference Optimization (DPO), aligns model outputs with human preferences through comparative feedback~\cite{wang2023rlhf,yang2024multi,dong2024rlhf,xu2024dpo,saeidi2024insights}. In software engineering, RLHF has been used to enhance model alignment with human reasoning, leveraging human attention and feedback to improve code summarization, model focus, and explainability~\cite{bansal2023modeling,karas2024tale,li2024machines,zhang2024eyetrans}. Additionally, studies show that LLMs can learn structured decision patterns from human-provided code comments and summarization patterns~\cite{zhang2022pre,zhang2022leveraging}, demonstrating RLHF’s potential for domains requiring contextual understanding, such as construction.

% Reinforcement Learning from Human Feedback (RLHF), including methods such as Direct Preference Optimization (DPO), aligns model outputs with human preferences through comparative feedback~\cite{wang2023rlhf,yang2024multi,dong2024rlhf,xu2024dpo,saeidi2024insights}. In software engineering, RLHF has been explored to enhance model alignment with human reasoning and improve representation learning. Studies have leveraged human attention to refine code summarization, align model focus with programmer cognition, and improve explainability~\cite{bansal2023modeling,karas2024tale,li2024machines,zhang2024eyetrans}. Additionally, research has shown that human feedback, such as code comments and summarization patterns, can be systematically learned by LLMs~\cite{zhang2022pre,zhang2022leveraging}. This enhances their ability to learn structured decision patterns from human feedback, demonstrating the potential for adapting similar approaches to domains requiring specialized contextual understanding, such as construction.

However, applying RLHF in traditional industries like construction remains challenging due to the need for domain-specific knowledge, complex dependencies, and expert-driven priorities~\cite{wang2024comprehensive,xiao2024comprehensive,feng2024towards}. While RLHF has been applied in various domains, its use in construction scheduling remains underexplored. Our approach extends DPO by incorporating construction-specific knowledge and structured context, resulting in schedules that better reflect expert preferences and project-specific requirements.

\section{Conclusion}

In conclusion, we presented \TheName{}, an approach for automating construction schedules by integrating LLMs, contextualized knowledge RAG, and RLHF to optimize workflows with expert input. This framework advances traditional methods, offering flexibility, scalability, and adaptability for large-scale projects with complex dependencies. Future work includes implementing the Construction DPO model, incorporating multimodal inputs, and evolving \TheName{} into a dynamic recommender system for continuous project adaptation.

\section*{Acknowledgment}

This work was supported by Intel Corporation\footnote{\url{https://www.intel.com/content/www/us/en/homepage.html}}, specifically through the Incubation and Disruptive Innovation group. We also appreciate the collaboration and insights from Intel Foundry\footnote{\url{https://www.intel.com/content/www/us/en/foundry/overview.html}} employees, whose expertise in semiconductor fabrication has guided our exploration of leveraging LLMs to automate construction processes in chip manufacturing.


% \begin{abstract}

% In this paper, we present a novel framework for optimizing construction schedules in complex commercial projects such as semiconductor fabrication. Our framework leverages reinforcement learning, large language models (LLMs), and expert preferences to dynamically adjust construction schedules. The proposed system integrates context-aware activity sampling, rule-based scheduling, and preference alignment using a Direct Preference Optimization (DPO) model. We show that this approach significantly improves task completion times and alignment with human preferences compared to traditional methods.
% \end{abstract}

% \section{Introduction}
% The optimization of construction schedules in large-scale commercial projects, particularly semiconductor fabrication, is an ongoing challenge. Current methods often rely on rigid rule-based systems, which fail to adapt to changing project contexts or human preferences. In this paper, we introduce \textit{Construction Preference Alignment} (CPA), a framework that integrates large language models (LLMs), context sampling, and reinforcement learning to dynamically adjust schedules in response to real-time data and expert feedback.

% \begin{figure*}[ht]
%     \centering
%     \includegraphics[width=\linewidth]{figs/constructa_overview.pdf}
%     \caption{Overview of the \TheName{} system. (a) The initial construction schedule is provided by a construction expert and adjusted based on contextual activity and site samples. (b) Contextualized activity aggregates hierarchical, first-order, and sequential relations. (c) Knowledge vectorization saves and retrieves construction knowledge for optimization. (d) Construction preference alignment optimizes the schedule using reinforcement learning based on combined knowledge and context.}
%     \label{fig:constructa_overview}
% \end{figure*}

% \section{Related Work}
% Prior work in construction optimization has primarily focused on deterministic scheduling algorithms or rule-based systems, which struggle to adapt to dynamic, real-world project conditions. Reinforcement learning has been applied in other domains, but its use in construction scheduling is still limited. Our work builds on recent advancements in preference-based reinforcement learning, particularly Direct Preference Optimization (DPO), to create a flexible system for construction scheduling that learns from both project data and expert preferences.

% \section{Methodology}
% \subsection{Data Collection and Context Embedding}
% We first collect construction project data, including work breakdown structure (WBS), activity dependencies, and site conditions. This data is used to construct a dependency graph, where each node represents a construction activity. Each edge in the graph represents the dependency between activities, forming a directed acyclic graph (DAG). We apply pre-trained transformer models to generate context embeddings for each node. The embedding \( e_i \) for a node \( i \) is computed as:

% \begin{equation}
%     e_i = f_\text{embed}(X_i)
% \end{equation}

% where \( X_i \) is the raw feature vector for activity \( i \), and \( f_\text{embed} \) represents the embedding model (e.g., Sentence-Transformer or BERT).

% The similarity between two activities in terms of their construction context is calculated using cosine similarity:

% \begin{equation}
%     \text{sim}(e_i, e_j) = \frac{e_i \cdot e_j}{\|e_i\| \|e_j\|}
% \end{equation}

% \subsection{Unified Sampling of Contexts}
% To ensure that all relevant dependencies are considered in the scheduling process, we use a unified sampling method to extract three types of contexts:
% - Sequential Context: Extracts predecessors and successors up to 3 hops.
% - Hierarchical Context: Extracts nodes within the same WBS, up to 2 hops.
% - First-Order Context: Direct predecessors and successors.

% Let \( G = (V, E) \) represent the dependency graph, where \( V \) is the set of activities, and \( E \) is the set of dependencies. The set of sampled contexts for a node \( i \) can be written as:

% \begin{equation}
%     C_i = \{ \text{Sequential}(i), \text{Hierarchical}(i), \text{FirstOrder}(i) \}
% \end{equation}

% These contexts are used as input for rule generation and optimization.

% \subsection{Rule Generation and Schedule Prediction}
% Using Azure OpenAI, we generate construction rules based on context embeddings and predefined prompts. These rules define constraints on task scheduling, dependencies, and resource allocation. The Plan Agent uses these rules to dynamically adjust the construction schedule. The process of rule generation is modeled as a sequence-to-sequence task:

% \begin{equation}
%     r = g_\text{rule}(C_i)
% \end{equation}

% where \( C_i \) is the combined context for node \( i \), and \( g_\text{rule} \) is the rule generation model based on LLMs.

% The predicted schedule \( S \) is generated by minimizing the total project completion time while satisfying all constraints:

% \begin{equation}
%     S = \arg \min \sum_{i \in V} T_i \quad \text{s.t.} \quad r_i(S) = \text{true} \ \forall i \in V
% \end{equation}

% where \( T_i \) is the completion time of activity \( i \), and \( r_i \) represents the rules generated for activity \( i \).

% \begin{figure*}[htbp]
%     \centering
%     \includegraphics[width=\textwidth]{figs/constructa_cpa_rlhf.pdf}
%     \caption{Diagram illustrating the Construction Preference Alignment DPO~(CPA-DPO) process. The Plan Agent combines raw context and rules to generate filtered context and rules, which are aligned using preferences stored in the database.}
%     \label{fig:constructa_cpa}
% \end{figure*}

% \section{Plan Agent and DPO Model}

% The Plan Agent integrates raw construction context, filtered rules, and expert feedback to generate optimized schedules. The core of this system is a language model fine-tuned using a two-stage training process: initial Supervised Fine-Tuning (SFT) followed by Direct Preference Optimization (DPO). The objective is to align the generated schedules with expert preferences while ensuring adherence to contextual and rule-based constraints.

% \subsection{Training Process Overview}

% The training process consists of two main stages:

% \begin{enumerate}
%     \item \textbf{Supervised Fine-Tuning (SFT) Initialization}: The model is first fine-tuned on a subset of data where the generated schedules exactly match the ground truth provided by experts. This step initializes the model parameters toward producing accurate and contextually appropriate outputs.

%     \item \textbf{Direct Preference Optimization (DPO~\cite{xu2024dpo}) Training}: Following SFT, the model undergoes DPO training using a novel loss function that promotes preference alignment and adherence to contextual and rule-based information.
% \end{enumerate}

% \subsection{Supervised Fine-Tuning (SFT) Initialization}

% During the SFT stage, the model is trained to minimize the \textbf{Language Modeling Loss} ($L_{\text{LM}}$) on data samples with exact matches to expert schedules. The loss is calculated using cross-entropy between the predicted tokens and the ground truth:

% \begin{equation}
%     L_{\text{SFT}} = L_{\text{LM}}
% \end{equation}

% This step ensures the model learns to generate coherent schedules that closely align with expert expectations.

% \subsection{Direct Preference Optimization (DPO) Training}

% After SFT initialization, the model is further trained using the DPO approach. To effectively model the interaction between \textbf{context} and \textbf{rules} during preference alignment, we introduce a novel loss function, $L_{\text{total}}$. This loss promotes preference-aligned outputs by enhancing the model’s adherence to both contextual and rule-based information. It comprises three components: \textbf{Language Modeling Loss} ($L_{\text{LM}}$), \textbf{Context-Rules Interaction Loss} ($L_{\text{CR}}$), and \textbf{Preference Alignment Loss} ($L_{\text{PA}}$).

% \begin{equation}
%     L_{\text{total}} = L_{\text{LM}} + \alpha \cdot L_{\text{CR}} + \beta \cdot L_{\text{PA}}
% \end{equation}

% where $\alpha$ and $\beta$ balance the importance of each component.

% \subsubsection{Language Modeling Loss}
% The \textbf{Language Modeling Loss} ($L_{\text{LM}}$) ensures coherence in the generated language, calculated using cross-entropy loss.

% \subsubsection{Context-Rules Interaction Loss}
% The \textbf{Context-Rules Interaction Loss} ($L_{\text{CR}}$) encourages alignment between context and rule-based constraints through a contrastive mechanism:

% \begin{equation}
%     \begin{aligned}
%     L_{\text{CR}} = \frac{1}{N} \sum_{i=1}^{N} \max \Big(0, & \, d(f_{\text{context}}(x_i), f_{\text{rule}}(x_i)) \\
%     & - f_{\text{output}}(x_i) + \delta \Big)
%     \end{aligned}
% \end{equation}

% where:
% \begin{itemize}
%     \item $f_{\text{context}}(x_i)$ and $f_{\text{rule}}(x_i)$ are embeddings for the context and rule constraints,
    
%     \item $f_{\text{output}}(x_i)$ represents the model's output embedding,
    
%     \item $d(\cdot, \cdot)$ measures alignment (e.g., cosine similarity),
    
%     \item $\delta$ is a margin controlling sensitivity to discrepancies.
% \end{itemize}

% \subsubsection{Preference Alignment Loss}
% The \textbf{Preference Alignment Loss} ($L_{\text{PA}}$) is a binary cross-entropy loss comparing generated responses to preference labels:

% \begin{equation}
%     L_{\text{PA}} = -\frac{1}{N} \sum_{i=1}^{N} \left( y_i \log(p_i) + (1 - y_i) \log(1 - p_i) \right)
% \end{equation}

% where $y_i$ denotes the preference label for output $x_i$, and $p_i$ is the model's predicted probability of preference alignment.

% \subsection{Parameter-Efficient Fine-Tuning with LoRA}

% To efficiently fine-tune the language model without updating all parameters, we employ Low-Rank Adaptation (LoRA). LoRA adds trainable rank-decomposed weight matrices to existing weights, significantly reducing the number of trainable parameters. This approach enables effective fine-tuning while keeping computational costs manageable.

% \subsection{Implementation Details}

% The training process leverages several techniques to optimize performance:

% \begin{itemize}
%     \item \textbf{4-bit Quantization}: The base model is quantized to 4-bit precision using the \texttt{BitsAndBytes} library, reducing memory footprint and computational requirements.
    
%     \item \textbf{Gradient Checkpointing}: Enabled to save memory during training by trading computation for memory.
    
%     \item \textbf{Mixed Precision Training}: Utilized to accelerate training and reduce memory usage through lower-precision computations where appropriate.
    
%     \item \textbf{Supervised Fine-Tuning before DPO}: The initial SFT provides a strong starting point, enhancing the effectiveness of subsequent DPO training.
    
%     \item \textbf{Optimization}: The AdamW optimizer is used with a learning rate tuned for effective convergence on the reduced parameter set.
% \end{itemize}

% \subsection{Conclusion}

% By integrating SFT before DPO and utilizing techniques like LoRA and quantization, the Plan Agent's DPO model effectively aligns generated schedules with expert preferences while maintaining computational efficiency. This multi-stage training process ensures that the model generates contextually coherent, rule-compliant, and preference-aligned outputs, enhancing the optimization of construction schedules based on both automated systems and human expertise.

% \section{Experiments}
% \subsection{Dataset and Setup}
% We conducted experiments using data from multiple semiconductor fabrication projects. The data includes tasks such as HVAC installation, chemical piping, and cleanroom preparation. For each activity, we collected WBS information, task dependencies, and resource requirements. The system was tested on its ability to generate optimized schedules that minimize delays and align with expert feedback.

% \subsection{Baseline Comparison}
% Our CPA system was compared against traditional rule-based scheduling methods. We evaluated performance based on the following metrics:
% - Time-to-completion
% - Rule adherence
% - Expert preference alignment

% The CPA system demonstrated a 15\% reduction in delays for HVAC installations, and a 10\% improvement in chemical piping tasks.

% \begin{table}[h]
% \centering
% \resizebox{\columnwidth}{!}{
% \begin{tabular}{l|ccc|c}
% \toprule
% Model Config       & MVP & DA & AP & Avg \\ \midrule
% GPT-4o                          & --                                & --                           & --                          & --           \\
% + Contextualized Activity          & --                                & --                           & --                          & --           \\
% + Knowledge RAG                    & --                                & --                           & --                          & --           \\
% + Construction RLHF                & --                                & --                           & --                          & --           \\ \bottomrule
% \end{tabular}
% }
% \caption{Performance comparison of various pretraining configurations for construction schedule optimization tasks. Each row represents a sequential modification applied to the model in the previous row. Metrics include exact match for Missing Value Prediction, Dependency Analysis, and Automated Planning, consistent with prior studies.}
% \label{tab:ablation_study_construction}
% \end{table}


% \section{Results and Discussion}

% \begin{figure*}[h]
%     \centering
%     \includegraphics[width=\textwidth]{figs/distribution_by_source_with_dpo_output.pdf}
%     \caption{Distribution of Predict Context and Polished Context Lengths by Source. This plot illustrates the frequency distributions of context lengths based on source type, providing insight into how context lengths vary across different sources.}
%     \label{fig:distribution_by_source}
% \end{figure*}


% Our system significantly improved construction schedule optimization compared to traditional methods. By integrating human preferences with context-aware rule generation, we achieved faster task completion times and higher alignment with expert feedback. The planned implementation of the DPO model is expected to further enhance this process, enabling more granular adjustments based on real-time data.

% \begin{table*}[h]
% \centering
% \resizebox{\textwidth}{!}{
% \begin{tabular}{l|ccc|ccc|ccc|c}
% \toprule
% Discipline           & \multicolumn{3}{c|}{MVP}  & \multicolumn{3}{c|}{DA} & \multicolumn{3}{c|}{AP} & Avg \\ \midrule
%                      & EM  & Prec & Rec          & DC  & Prec & Rec       & SA  & Tim & RO      &  \\ \hline
% HVAC                 & --  & --   & --           & --  & --   & --        & --  & --  & --      & -- \\
% Electrical           & --  & --   & --           & --  & --   & --        & --  & --  & --      & -- \\
% Piping               & --  & --   & --           & --  & --   & --        & --  & --  & --      & -- \\
% General Construction & --  & --   & --           & --  & --   & --        & --  & --  & --      & -- \\
% \bottomrule
% \end{tabular}
% }
% \caption{Discipline-wise evaluation of \TheName{} across core tasks: Missing Value Prediction (MVP), Dependency Analysis (DA), and Automated Planning (AP). Metrics include Exact Match (EM), Precision (Prec), Recall (Rec), Dependency Completeness (DC), Schedule Accuracy (SA), Timeliness (Tim), and Resource Optimization (RO).}
% \label{tab:discipline_wise_evaluation}
% \end{table*}

% \section{Conclusion and Future Work}
% We introduced a novel system for optimizing construction schedules using LLMs, context sampling, and reinforcement learning. Our CPA framework integrates human preferences into the scheduling process, providing a flexible and adaptable solution for large-scale commercial projects. Future work will focus on implementing the DPO model and extending the system to handle multi-modal inputs such as video data for task monitoring.

\bibliography{anthology}

% BACKUP 1: Description for Sequential Context Subgraph 1:
% In a Sequential Context subgraph used for project scheduling, the structure typically consists of nodes representing individual tasks or activities required to complete a project. Each node is connected by directed edges that denote the sequence and dependencies between tasks, illustrating the flow of work. A task without predecessors represents an entry point that can begin immediately, serving as the project's initiation node. Conversely, an exit node with no successors indicates the final task or culmination of project activities. The edges connecting the nodes signify dependency relationships, such as Finish-to-Start (FS) where one task must be completed before the subsequent task can begin, Finish-to-Finish (FF) where two tasks must complete simultaneously, Start-to-Start (SS) requiring concurrent initiations, or Start-to-Finish (SF) where a succeeding task needs to begin before the predecessor concludes, although SF is less commonly used. This graph structure fundamentally aids in visualizing task scheduling, identifying critical paths, and highlighting potential bottlenecks that could delay project delivery. By analyzing these dependencies, project managers can prioritize resource allocation, mitigate risks, and enhance workflow efficiency, fundamentally leveraging the framework to ensure timely project completion.

% Description for Sequential Context Subgraph 2:
% In a sequential context subgraph for project scheduling, the structure is composed of nodes and directed edges which represent the tasks and their dependencies, respectively. Each node denotes a distinct task or activity that is part of the overarching project. The nodes are usually annotated with relevant details such as the task duration, resource requirements, and earliest start or finish times. Directed edges between nodes illustrate dependencies or precedence relationships, indicating the sequential order in which tasks must be performed. An edge from node A to node B implies that task A must be completed before task B can commence. This sequential arrangement is crucial for identifying critical paths, optimizing project timelines, and ensuring that resource allocation is managed effectively. The graph starts with nodes that have no predecessors, often referred to as entry nodes, and concludes with nodes that have no successors, known as exit nodes. By analyzing this graph, project managers can pinpoint bottlenecks, forecast potential delays, and adjust schedules dynamically to accommodate unforeseen changes.


% BACKUP 2:

% Description for Hierarchical Context Subgraph 1:
% In a Hierarchical Context subgraph for project scheduling, nodes represent distinct components or phases of a project, such as tasks, milestones, and deliverables, while edges illustrate the dependencies or relationships between these components. The hierarchy of the graph typically starts with a root node that signifies the overall project, cascading down to sub-nodes that denote major project divisions or phases. These phases further decompose into individual tasks or milestones. Edges between nodes represent dependencies, indicating the sequence or precedence relationships that must be upheld; for example, a directed edge from Task A to Task B signifies that Task B cannot commence until Task A is completed. Additionally, edges can represent temporal constraints such as start-to-start, start-to-finish, finish-to-start, and finish-to-finish relationships. The hierarchy and dependencies captured by these edges and nodes are crucial for identifying the critical path, allocating resources efficiently, and ensuring that progress aligns with time and budget constraints. This structured approach allows project managers to visualize and adjust the scheduling as needed to optimize workflow and resource allocation while maintaining flexibility to adapt to changes.

% Description for Hierarchical Context Subgraph 2:
% In project scheduling, a Hierarchical Context subgraph provides a visual representation of the various tasks, their dependencies, and the project structure. Nodes in this graph typically represent tasks, milestones, or deliverables that need to be accomplished as part of the project. Each node is connected by edges that represent dependencies between these tasks, indicating the order in which tasks should be completed. The hierarchy is evident in the layered structure of the graph, where top-level nodes might represent major project phases or deliverables, and lower-level nodes detail the specific tasks necessary to complete each phase. This structured approach allows project managers to easily identify critical paths, which are sequences of dependent tasks determining the shortest possible project duration. Arrows on these edges specify the direction of task flow, emphasizing predecessor-successor relationships essential for sequential task completion. In addition to these dependencies, other annotations like estimated task duration or resource allocation might be included, offering a comprehensive overview necessary for effective project planning and adjustment. This hierarchical representation aids in locating bottlenecks, optimizing resource allocation, and ensuring that all prerequisites for tasks and milestones are addressed to mitigate risk and facilitate the smooth execution of the project plan.

% BACKUP 3: 

% Description for First-Order Context Subgraph 1:
% In a First-Order Context subgraph for project scheduling, the structure is typically composed of nodes and edges that represent tasks and their dependencies, respectively. Each node in the subgraph signifies a discrete project task or milestone that must be completed, while directed edges between nodes denote the prerequisite relationships or dependencies among these tasks. This dependency indicates a precedence constraint, meaning that the task at the start of the edge must be completed before the task at the end can commence. Within this subgraph, the nodes can vary in complexity, representing single activities or composite tasks that might encompass several interrelated activities. The edges carry either temporal constraints, like start-to-start, start-to-finish, finish-to-start, or finish-to-finish, or resource-related constraints, where certain tasks require shared resources, limiting their simultaneity. The topological structure of this subgraph impacts the project's critical path, which is the longest sequence of dependent tasks that determine the minimal project duration. Analyzing these dependencies allows project managers to identify potential bottlenecks and optimize scheduling to improve efficiency and resource allocation. Furthermore, this subgraph can highlight critical nodes that have multiple dependencies, serving as indicators for key project risks or points where schedule slippage could occur.

% Description for First-Order Context Subgraph 2:
% A First-Order Context subgraph for project scheduling comprises nodes that represent specific tasks, milestones, or deliverables within a project and edges that symbolize the dependencies or relationships between these tasks. Each node functions as a discrete unit of work with attributes such as duration, resource requirements, and start and end dates. Nodes can either be independent or dependent on other nodes, depending on the project's workflow and sequencing needs. Edges, depicted as directed arrows, define the logical and temporal dependencies between nodes, indicating the order in which tasks must be completed. These dependencies can be classified as Finish-to-Start (FS), Start-to-Start (SS), Finish-to-Finish (FF), or Start-to-Finish (SF), each dictating a specific kind of relationship. For instance, a Finish-to-Start (FS) relationship implies that one task must be completed before the subsequent task can commence. This structure allows project managers to assess critical paths, identify potential bottlenecks, and optimize resource allocation by evaluating how delays in one task may impact the completion of dependent tasks. Overall, understanding the interplay between nodes and edges in this subgraph is crucial for effective project scheduling and management, offering a visual and analytical approach to balancing multiple project constraints.

% \subsection{Unified Context Sampling Visualization}

% To support effective construction scheduling, we employ a unified sampling method that extracts three distinct types of contextual information from project activities: Sequential Context, Hierarchical Context, and First-Order Context. Each method offers a unique approach to capturing dependencies and relationships among construction activities, facilitating comprehensive schedule optimization. Figures \ref{fig:sequential_context}, \ref{fig:hierarchical_context}, and \ref{fig:first_order_context} illustrate the structure and details of each context sampling method.

% \begin{figure*}[ht]
%     \centering
%     \includegraphics[width=\textwidth]{figs/sequential_context.pdf}
%     \caption{Sequential Context Sampling: This sampling method extracts nodes up to three hops away from each selected activity, representing predecessors and successors. Sequential sampling highlights dependencies that span across multiple stages in the construction workflow, enabling the model to understand task sequences and critical paths that influence the overall project schedule.}
%     \label{fig:sequential_context}
% \end{figure*}

% In Figure \ref{fig:sequential_context}, Sequential Context Subgraph 1 (left) shows a network of activities where nodes represent individual tasks required for project completion, connected by directed edges that denote task dependencies. Each node connects to predecessors and successors up to three hops away, capturing dependencies such as Finish-to-Start (FS), Finish-to-Finish (FF), Start-to-Start (SS), and, though less common, Start-to-Finish (SF) relationships. This structure is critical for visualizing the overall task flow, identifying critical paths, and highlighting potential bottlenecks that could delay project delivery. Sequential Context Subgraph 2 (right) extends this by including a larger set of interconnected nodes, where tasks are annotated with additional details such as task duration, resource requirements, and start or finish times. This dense layout offers a comprehensive view of task sequences, helping project managers forecast delays, pinpoint bottlenecks, and dynamically adjust schedules to accommodate unforeseen changes.

% \begin{figure*}[ht]
%     \centering
%     \includegraphics[width=\textwidth]{figs/hierarchical_context.pdf}
%     \caption{Hierarchical Context Sampling: This sampling focuses on capturing nodes within the same Work Breakdown Structure (WBS) up to two hops. Hierarchical context provides insights into tasks grouped by project phases, illustrating how dependencies within each WBS segment affect the schedule's progression.}
%     \label{fig:hierarchical_context}
% \end{figure*}

% Figure \ref{fig:hierarchical_context} shows the Hierarchical Context Sampling. Hierarchical Context Subgraph 1 (left) presents nodes representing major project phases or milestones and their sub-tasks, organized within a structured hierarchy. Starting from a root node that signifies the overall project, dependencies cascade down through the graph, capturing relationships such as Start-to-Start and Finish-to-Start within a single WBS segment. This layout allows for visualizing dependencies specific to each phase, which is crucial for managing resources and time within discrete project stages. Hierarchical Context Subgraph 2 (right) shows a more streamlined arrangement, where tasks follow a linear progression, emphasizing phase-aligned scheduling adjustments. This structure helps project managers identify the critical path within each phase and adjust scheduling as needed to optimize workflow and resource allocation, while ensuring flexibility to adapt to phase-specific constraints and objectives.

% \begin{figure*}[ht]
%     \centering
%     \includegraphics[width=\textwidth]{figs/first_order_context.pdf}
%     \caption{First-Order Context Sampling: This method captures only direct predecessors and successors for each selected activity. First-order context highlights immediate task dependencies, providing a concise view of direct task relationships essential for high-priority scheduling adjustments.}
%     \label{fig:first_order_context}
% \end{figure*}

% Figure \ref{fig:first_order_context} illustrates the First-Order Context Sampling. First-Order Context Subgraph 1 (left) shows a minimal structure with only one dependency, representing a direct, Finish-to-Start relationship between two tasks. This sparse setup allows for focused adjustments on critical dependencies without the complexity of additional nodes, making it ideal for high-priority scheduling where immediate, direct task relationships are paramount. First-Order Context Subgraph 2 (right) presents a more intricate structure with multiple tasks directly connected to a central node. This setup captures immediate predecessors and successors, including Start-to-Start (SS) and Finish-to-Finish (FF) dependencies, providing a concise overview of key relationships around the central task. Such a layout enables project managers to address dependencies that directly impact the timing and prioritization of essential tasks, helping maintain schedule adherence while focusing on high-impact areas of the project.

% Each sampling method uniquely extracts relevant information from the project table, allowing the model to adaptively balance broad, phase-level dependencies with immediate task relationships. This unified approach to context sampling is instrumental in generating a well-rounded understanding of the construction schedule, enabling dynamic and context-aware adjustments.

% \subsection{General Predefined Prompt Categories and Context Mapping}

% The prompt system utilizes predefined categories and context mappings to structure the data collection for various tasks in construction scheduling. These predefined categories align with specific aspects of construction project analysis, helping the language model understand the objectives and apply suitable strategies across different prompts.

% \begin{figure*}[h]
% \begin{tcolorbox}[colback=white, colframe=gray!70, colbacktitle=gray!20, coltitle=black, title=Sequential Context (Context 1), width=\textwidth, boxsep=5pt, left=3pt, right=3pt, top=5pt, bottom=5pt]
%     \textbf{Activity Sequence and Timing} \newline
%     List the sequence of construction activities based on the 'Current Start' and 'Current Finish' dates, ensuring they follow the correct order as indicated by 'Predecessor Details' and 'Successor Details'.
%     \tcblower
%     \textbf{Calculate Activity Duration} \newline
%     Based on the 'Current Start' and 'Current Finish' dates, calculate the duration for each activity and establish the step-by-step timeline for the project.
% \end{tcolorbox}
% \caption*{\textit{The Sequential Context focuses on tasks that require understanding the order and duration of activities, aiding the model in providing structured timelines.}}
% \end{figure*}

% \begin{figure*}[h]
% \begin{tcolorbox}[colback=white, colframe=gray!70, colbacktitle=gray!20, coltitle=black, title=First-Order Context (Context 2), width=\textwidth, boxsep=5pt, left=3pt, right=3pt, top=5pt, bottom=5pt]
%     \textbf{Analyze Time Relationships} \newline
%     Analyze the 'Predecessor Details' and 'Successor Details' to determine the time domain relationship between activities. Identify which activities are in parallel and the number of branches in the dependency graph.
%     \tcblower
%     \textbf{Overlapping Disciplines} \newline
%     Identify overlapping disciplines from the 'Discipline' column and infer which dependency graphs can be merged based on this overlap.
%     \tcblower
%     \textbf{Inter-Disciplinary Dependencies} \newline
%     Examine the inter-dependency between different disciplines and create a map of these relationships.
%     \tcblower
%     \textbf{Area-Based Dependencies} \newline
%     Using the 'Area' column, analyze area-based dependencies and how they affect the sequence of construction activities.
% \end{tcolorbox}
% \caption*{\textit{The First-Order Context is used for tasks requiring analysis of dependencies, both temporal and spatial, across disciplines and areas.}}
% \end{figure*}

% \begin{figure*}[h]
% \begin{tcolorbox}[colback=white, colframe=gray!70, colbacktitle=gray!20, coltitle=black, title=Hierarchical Context (Context 3), width=\textwidth, boxsep=5pt, left=3pt, right=3pt, top=5pt, bottom=5pt]
%     \textbf{Hierarchical Tree Structure} \newline
%     Organize the activities into a hierarchical tree structure based on their WBS and identify any activities that should be sequential but are not currently listed as such.
%     \tcblower
%     \textbf{Assess Sequence Reconstruction} \newline
%     For each activity, determine if the sequence can be recovered from the given data. If not, specify what critical information is missing and suggest how to bridge the identified gaps.
% \end{tcolorbox}
% \caption*{\textit{The Hierarchical Context helps the model understand hierarchical structures, organizing tasks based on a tree-like structure and identifying gaps in the sequence.}}
% \end{figure*}

% \subsection{Task-Specific Prompts for Data Collection}

% For each specific task (AP, MVP, and DA), we have designed prompts to guide the language model in generating outputs relevant to construction scheduling.

% \begin{figure*}[h]
% \begin{tcolorbox}[colback=white, colframe=gray!70, colbacktitle=gray!20, coltitle=black, title=Automated Planning (AP) Prompts, width=\textwidth, boxsep=5pt, left=3pt, right=3pt, top=5pt, bottom=5pt]
%     \textbf{AP - Part 1} \newline
%     You are a virtual construction expert collaborating with a larger LLM to automate the construction schedule. Use the 'Current Start' and 'Current Finish' dates in the context to ensure tasks are scheduled based on their dependencies. Explain how the selected rules help guide the automation of task sequencing and timing.
%     \tcblower
%     \textbf{AP - Part 2} \newline
%     Justify why these specific rules and context elements are crucial for automating the schedule. Describe the connection between the context and rules, and provide logical reasoning for why these choices will result in a successful automation process.
% \end{tcolorbox}
% \caption*{\textit{The AP prompts focus on scheduling construction activities based on start and finish dates, with additional context to justify the automation logic.}}
% \end{figure*}

% \begin{figure*}[h]
% \begin{tcolorbox}[colback=white, colframe=gray!70, colbacktitle=gray!20, coltitle=black, title=Missing Value Prediction (MVP) Prompts, width=\textwidth, boxsep=5pt, left=3pt, right=3pt, top=5pt, bottom=5pt]
%     \textbf{MVP - Part 1} \newline
%     Based on the following information, choose the correct values for the missing columns. Return the values as a list, separated by commas, with each value enclosed within [Value] and [/Value] tags. The list should contain exactly three values, corresponding to the columns listed in the same order.
%     \tcblower
%     \textbf{MVP - Part 2} \newline
%     This part provides the row input, static knowledge, and context information that the model will use to identify missing values and fill them accurately.
% \end{tcolorbox}
% \caption*{\textit{The MVP prompts focus on predicting missing values based on static knowledge and contextual clues.}}
% \end{figure*}

% \begin{figure*}[h]
% \begin{tcolorbox}[colback=white, colframe=gray!70, colbacktitle=gray!20, coltitle=black, title=Dependency Analysis (DA) Prompts, width=\textwidth, boxsep=5pt, left=3pt, right=3pt, top=5pt, bottom=5pt]
%     \textbf{DA - Part 1} \newline
%     You are a virtual construction expert collaborating with a larger LLM to analyze dependencies between construction activities. Focus on identifying key dependencies using the 'Predecessor Details' and 'Successor Details' in the context. Explain how and why the selected rules are relevant for understanding the dependencies between activities.
%     \tcblower
%     \textbf{DA - Part 2} \newline
%     Connect these rules to specific parts of the context. Ensure that the relationship between the context and rules is clearly articulated, showing logical reasoning behind the choices made for this analysis.
% \end{tcolorbox}
% \caption*{\textit{The DA prompts guide the model in identifying and explaining dependencies between construction activities.}}
% \end{figure*}

% \begin{figure*}[h]
% \begin{tcolorbox}[colback=white, colframe=gray!70, colbacktitle=gray!20, coltitle=black, title=DPO Polishing Prompt, width=\textwidth, boxsep=5pt, left=3pt, right=3pt, top=5pt, bottom=5pt]
%     \textbf{Preference Alignment} \newline
%     As a virtual construction scheduling expert, refine the following output to ensure it aligns with expert expectations. The output should provide coherent and contextually relevant responses to scheduling needs, integrating expert rules and project-specific knowledge seamlessly. Emphasize adherence to preferences and explain any dependencies or task prioritizations that support an optimized construction schedule.
% \end{tcolorbox}
% \caption*{\textit{The DPO prompt refines responses to align with expert preferences, ensuring coherence and relevance in the generated outputs.}}
% \end{figure*}

% \appendix
% \section*{Appendix: Additional Experiments}

% In this appendix, we provide further details on the experiments conducted, including sensitivity analysis on the context embedding models and variations of the preference alignment strategy. We experimented with different transformer-based models for embedding generation and found that models such as BERT and Sentence-Transformers provided optimal performance for our context-based scheduling tasks.

% \subsection{Correlation and Similarity Analysis of Project Attributes}

% Understanding the relationships among project attributes is essential for optimizing scheduling and dependency management in project planning. We perform two types of analyses to capture both linear correlations and deeper semantic relationships:

% \begin{itemize}
%     \item \textbf{Correlation Analysis (Encoded Data)}: This approach examines linear dependencies between attributes using encoded data. To achieve this, we encoded categorical attributes as numeric codes, then computed the Pearson correlation coefficients across project attributes. This allows us to identify direct dependencies that impact the project timeline and resource allocation.
    
%     \item \textbf{Cosine Similarity Analysis (Embeddings)}: Using embeddings generated from the \texttt{distilbert-base-uncased}\footnote{\url{https://huggingface.co/docs/transformers/en/model_doc/distilbert}} pre-trained language model, this analysis captures semantic similarities among attributes. This method uncovers implicit, context-based relationships that linear correlations may miss, providing insights into roles, locations, and dependencies. The embeddings for each attribute were generated by averaging the hidden states from the model’s final layer, offering a contextual understanding of each attribute’s role in the project.
% \end{itemize}

% \begin{figure*}[h]
%     \centering
%     \includegraphics[width=\textwidth]{figs/combined_similarity_matrices.pdf}
%     \caption{Combined Correlation and Cosine Similarity Heatmaps for Project Attributes. The left plot illustrates the correlation matrix based on encoded project data, highlighting linear relationships among attributes. The right plot presents the cosine similarity matrix based on embeddings, revealing deeper semantic associations among attributes.}
%     \label{fig:combined_similarity_matrices}
% \end{figure*}

% \noindent Figure \ref{fig:combined_similarity_matrices} displays the results from both analyses, each providing unique insights:

% \textbf{Correlation Matrix (Encoded Data)}: The left heatmap highlights linear relationships among attributes, with several notable correlations:

% \begin{itemize}
%     \item \textbf{Current Start and Current Finish}: The high correlation here reflects the dependency between start and finish dates, a foundational aspect of project scheduling.
    
%     \item \textbf{Activity Status and Project Phase}: Correlations between activity status and project phase suggest that certain statuses align with specific phases, informing phase-based scheduling prompts.
    
%     \item \textbf{Predecessor and Successor}: Strong correlation indicates that tasks have sequential dependencies, essential for creating an accurate task sequence.
% \end{itemize}

% In summary, these correlations reveal structural dependencies in project attributes, assisting in identifying key points in the scheduling and sequencing workflow. These insights enable more effective scheduling strategies by understanding which attributes inherently impact each other.

% \textbf{Cosine Similarity Matrix (Embeddings)}: The right heatmap reveals semantic relationships between attributes, which help identify context-based dependencies:

% \begin{itemize}
%     \item \textbf{Subcontractor and Superintendent}: High similarity implies overlapping responsibilities between these roles, which can guide role-based dependencies in scheduling.
    
%     \item \textbf{Discipline and Zone}: This similarity reflects the association between certain disciplines and zones, useful for location-based dependency prompts.
    
%     \item \textbf{Project Phase and Activity Status}: Semantic alignment between phases and statuses provides a structured basis for task progression, useful for designing prompts that ensure coherent task sequences.
% \end{itemize}

% Overall, these embedding-based relationships uncover context-driven dependencies beyond simple correlations, offering a richer view of the project structure. Such insights are critical for tasks involving nuanced scheduling needs, as they reveal role interactions and locational dependencies that direct scheduling and resource assignment decisions.

% \subsection{Unified Context Sampling Visualization}

% To support effective construction scheduling, we employ a unified sampling method that extracts three distinct types of contextual information from project activities: Sequential Context, Hierarchical Context, and First-Order Context. Each method offers a unique approach to capturing dependencies and relationships among construction activities, facilitating comprehensive schedule optimization. Figures \ref{fig:sequential_context}, \ref{fig:hierarchical_context}, and \ref{fig:first_order_context} illustrate the structure and details of each context sampling method.

% \begin{figure*}[ht]
%     \centering
%     \includegraphics[width=\textwidth]{figs/sequential_context.pdf}
%     \caption{Sequential Context Sampling: This sampling method extracts nodes up to three hops away from each selected activity, representing predecessors and successors. Sequential sampling highlights dependencies that span across multiple stages in the construction workflow, enabling the model to understand task sequences and critical paths that influence the overall project schedule.}
%     \label{fig:sequential_context}
% \end{figure*}

% In Figure \ref{fig:sequential_context}, Sequential Context Subgraph 1 (left) captures a network of activities where each node represents an individual task, connected by directed edges indicating dependencies such as Finish-to-Start (FS), Finish-to-Finish (FF), Start-to-Start (SS), and Start-to-Finish (SF). This structure is critical for understanding the flow of tasks and for identifying critical paths that could affect project timelines. Sequential Context Subgraph 2 (right) extends the model’s perspective by including a more interconnected set of tasks annotated with details like duration and resource requirements, offering a holistic view of task sequences. This broader context helps in pinpointing bottlenecks and assessing how delays in one activity might impact others across different stages of the project.

% \begin{figure*}[ht]
%     \centering
%     \includegraphics[width=\textwidth]{figs/hierarchical_context.pdf}
%     \caption{Hierarchical Context Sampling: This sampling method captures nodes within the same Work Breakdown Structure (WBS) segment up to two hops. Hierarchical sampling provides insights into dependencies within structured phases, illustrating how tasks grouped by project phases affect the schedule’s progression and resource distribution.}
%     \label{fig:hierarchical_context}
% \end{figure*}

% Figure \ref{fig:hierarchical_context} showcases Hierarchical Context Sampling. In Hierarchical Context Subgraph 1 (left), nodes represent key project phases or milestones, organized hierarchically with dependencies cascading down from a root node representing the overarching project. This sampling captures dependencies like Start-to-Start and Finish-to-Start within a single WBS segment, essential for managing resources efficiently within defined project phases. Hierarchical Context Subgraph 2 (right) focuses on a more linear progression of tasks within each phase, emphasizing dependencies that influence phase-specific timelines. This method is beneficial for ensuring resource allocation and for identifying potential delays that might affect phase completion without impacting unrelated phases.

% \begin{figure*}[ht]
%     \centering
%     \includegraphics[width=\textwidth]{figs/first_order_context.pdf}
%     \caption{First-Order Context Sampling: This sampling captures only direct predecessors and successors for each activity, highlighting immediate dependencies. First-order sampling provides a concise view of direct relationships essential for quick, high-priority scheduling adjustments and dependency management.}
%     \label{fig:first_order_context}
% \end{figure*}

% Figure \ref{fig:first_order_context} illustrates First-Order Context Sampling, which emphasizes immediate task relationships. First-Order Context Subgraph 1 (left) presents a simple structure with a single Finish-to-Start dependency, enabling the model to prioritize tasks that have direct scheduling impacts. First-Order Context Subgraph 2 (right) is more complex, capturing multiple direct dependencies around a central node with relationships like Start-to-Start (SS) and Finish-to-Finish (FF). This sampling is especially useful for addressing immediate scheduling needs, where adjusting critical dependencies can help project managers focus on tasks that have the most significant impact on project timelines.

% Each sampling method extracts specific patterns and relationships from the construction schedule, allowing the model to balance a broad view of phase-level dependencies with immediate task-specific adjustments. This unified approach to context sampling is instrumental in generating a comprehensive understanding of project scheduling, enabling dynamic, context-aware adjustments that support optimized construction workflows.

% \subsection{General Predefined Prompt Categories and Context Mapping}

% The prompt system utilizes predefined categories and context mappings to structure data collection for various tasks in construction scheduling. These categories are aligned with specific aspects of project analysis, helping the language model understand the objectives and apply appropriate strategies for generating structured and accurate responses across different tasks.

% \begin{figure*}[h]
% \begin{tcolorbox}[colback=white, colframe=gray!70, colbacktitle=gray!20, coltitle=black, title=Sequential Context (Context 1), width=\textwidth, boxsep=5pt, left=3pt, right=3pt, top=5pt, bottom=5pt]
%     \textbf{Activity Sequence and Timing} \newline
%     List the sequence of construction activities based on the 'Current Start' and 'Current Finish' dates, ensuring they follow the correct order as indicated by 'Predecessor Details' and 'Successor Details'.
%     \tcblower
%     \textbf{Calculate Activity Duration} \newline
%     Based on the 'Current Start' and 'Current Finish' dates, calculate the duration for each activity and establish the step-by-step timeline for the project.
% \end{tcolorbox}
% The Sequential Context prompt is designed to capture the linear progression of activities in construction. By focusing on the order and duration of activities, this context prompt aids in generating structured timelines, enabling the model to outline a clear sequence and allocate resources efficiently.
% \end{figure*}

% \begin{figure*}[h]
% \begin{tcolorbox}[colback=white, colframe=gray!70, colbacktitle=gray!20, coltitle=black, title=First-Order Context (Context 2), width=\textwidth, boxsep=5pt, left=3pt, right=3pt, top=5pt, bottom=5pt]
%     \textbf{Analyze Time Relationships} \newline
%     Analyze the 'Predecessor Details' and 'Successor Details' to determine the time domain relationship between activities. Identify which activities are in parallel and the number of branches in the dependency graph.
%     \tcblower
%     \textbf{Overlapping Disciplines} \newline
%     Identify overlapping disciplines from the 'Discipline' column and infer which dependency graphs can be merged based on this overlap.
%     \tcblower
%     \textbf{Inter-Disciplinary Dependencies} \newline
%     Examine the inter-dependency between different disciplines and create a map of these relationships.
%     \tcblower
%     \textbf{Area-Based Dependencies} \newline
%     Using the 'Area' column, analyze area-based dependencies and how they affect the sequence of construction activities.
% \end{tcolorbox}
% The First-Order Context prompt focuses on immediate dependencies and relationships between tasks. By analyzing time, disciplinary overlaps, and area-based dependencies, this prompt enables the model to capture critical dependencies that could impact the flow of work and resource allocation across parallel activities.
% \end{figure*}

% \begin{figure*}[h]
% \begin{tcolorbox}[colback=white, colframe=gray!70, colbacktitle=gray!20, coltitle=black, title=Hierarchical Context (Context 3), width=\textwidth, boxsep=5pt, left=3pt, right=3pt, top=5pt, bottom=5pt]
%     \textbf{Hierarchical Tree Structure} \newline
%     Organize the activities into a hierarchical tree structure based on their WBS and identify any activities that should be sequential but are not currently listed as such.
%     \tcblower
%     \textbf{Assess Sequence Reconstruction} \newline
%     For each activity, determine if the sequence can be recovered from the given data. If not, specify what critical information is missing and suggest how to bridge the identified gaps.
% \end{tcolorbox}
% The Hierarchical Context prompt helps the model understand hierarchical structures in project planning. By focusing on organizing tasks based on work breakdown structure (WBS), this context prompt aids in identifying gaps in sequencing and structuring project phases logically.
% \end{figure*}

% \subsection{Task-Specific Prompts for Data Collection}

% For each specific task (Automated Planning (AP), Missing Value Prediction (MVP), Dependency Analysis (DA), and Direct Preference Optimization (DPO)), we have designed prompts to guide the language model in generating outputs relevant to construction scheduling.

% \begin{figure*}[h]
% \begin{tcolorbox}[colback=white, colframe=gray!70, colbacktitle=gray!20, coltitle=black, title=Automated Planning (AP) Prompts, width=\textwidth, boxsep=5pt, left=3pt, right=3pt, top=5pt, bottom=5pt]
%     \textbf{AP - Part 1} \newline
%     You are a virtual construction expert collaborating with a larger LLM to automate the construction schedule. Use the 'Current Start' and 'Current Finish' dates in the context to ensure tasks are scheduled based on their dependencies. Explain how the selected rules help guide the automation of task sequencing and timing.
%     \tcblower
%     \textbf{AP - Part 2} \newline
%     Justify why these specific rules and context elements are crucial for automating the schedule. Describe the connection between the context and rules, and provide logical reasoning for why these choices will result in a successful automation process.
% \end{tcolorbox}
% The AP prompt focuses on scheduling construction activities based on start and finish dates, with an emphasis on the rules that support task sequencing and timing. This prompt aims to ensure coherent automation logic while aligning with project constraints and expert expectations.
% \end{figure*}

% \begin{figure*}[h]
% \begin{tcolorbox}[colback=white, colframe=gray!70, colbacktitle=gray!20, coltitle=black, title=Missing Value Prediction (MVP) Prompts, width=\textwidth, boxsep=5pt, left=3pt, right=3pt, top=5pt, bottom=5pt]
%     \textbf{MVP - Part 1} \newline
%     Based on the following information, choose the correct values for the missing columns. Return the values as a list, separated by commas, with each value enclosed within [Value] and [/Value] tags. The list should contain exactly three values, corresponding to the columns listed in the same order.
%     \tcblower
%     \textbf{MVP - Part 2} \newline
%     This part provides the row input, static knowledge, and context information that the model will use to identify missing values and fill them accurately.
% \end{tcolorbox}
% The MVP prompt is essential for accurately predicting missing data in construction tables, using both static knowledge and contextual details. This prompt is designed to help the model make accurate value predictions, enhancing data completeness and reliability.
% \end{figure*}

% \begin{figure*}[h]
% \begin{tcolorbox}[colback=white, colframe=gray!70, colbacktitle=gray!20, coltitle=black, title=Dependency Analysis (DA) Prompts, width=\textwidth, boxsep=5pt, left=3pt, right=3pt, top=5pt, bottom=5pt]
%     \textbf{DA - Part 1} \newline
%     You are a virtual construction expert collaborating with a larger LLM to analyze dependencies between construction activities. Focus on identifying key dependencies using the 'Predecessor Details' and 'Successor Details' in the context. Explain how and why the selected rules are relevant for understanding the dependencies between activities.
%     \tcblower
%     \textbf{DA - Part 2} \newline
%     Connect these rules to specific parts of the context. Ensure that the relationship between the context and rules is clearly articulated, showing logical reasoning behind the choices made for this analysis.
% \end{tcolorbox}
% The DA prompt guides the model in identifying and explaining dependencies between construction activities, with emphasis on critical tasks and their interactions. This prompt supports dependency mapping, which is crucial for project planning and risk management.
% \end{figure*}

% \begin{figure*}[h]
% \begin{tcolorbox}[colback=white, colframe=gray!70, colbacktitle=gray!20, coltitle=black, title=Context Polishing for DPO Prompts, width=\textwidth, boxsep=5pt, left=3pt, right=3pt, top=5pt, bottom=5pt]
%     \textbf{Context Polishing for DPO} \newline
%     As a virtual construction scheduling expert, refine the following output to ensure it aligns with expert expectations. The output should provide coherent and contextually relevant responses to scheduling needs, integrating expert rules and project-specific knowledge seamlessly. Emphasize adherence to preferences and explain any dependencies or task prioritizations that support an optimized construction schedule.
% \end{tcolorbox}
% The Context Polishing prompt refines responses to align with expert preferences, ensuring coherent, relevant outputs for scheduling needs. It supports the Direct Preference Optimization (DPO) process, enhancing the alignment of generated content with real-world project standards and expectations.
% \end{figure*}

\appendix
\section*{Appendix: Additional Details}
% \renewcommand{\thesubsection}{\Alph{section}.\arabic{subsection}}
\renewcommand{\thesubsection}{A.\arabic{subsection}} % Subsections numbered as A.1, A.2, etc.

In this appendix, we provide comprehensive details on the experiments conducted, including sensitivity analysis on context embedding models, variations of preference alignment strategies, the complexity analysis of the construction dependency graph, and the detailed design of context sampling methods, prompt categories, and task-specific prompts.

\subsection{Complexity of the Construction Dependency Graph}

Understanding the structural complexity of the dependency graph is critical for automating construction schedules effectively. We analyzed two key metrics to highlight the challenges posed by real-world construction scenarios (Figure~\ref{fig:degree_hop_distribution}):

\begin{itemize}
    \item \textbf{Degree Distribution}: This metric captures the number of connections each activity node has within the dependency graph. As shown in Figure~\ref{fig:degree_hop_distribution}, the degree distribution exhibits a mean value of 3.86, with some nodes having as many as 20 connections. These values indicate the extensive interdependencies among activities, which require careful management to maintain project feasibility and avoid resource bottlenecks.
    
    \item \textbf{Maximal Hop Distribution}: This measures the farthest distance, in terms of hops, to dependent nodes. The average maximal hop distance is 13.93, with the highest value reaching 73. These long-range dependencies demonstrate the need for multi-level propagation strategies to capture hierarchical and sequential task relationships effectively.
\end{itemize}

\begin{figure}[t]
    \centering
    \includegraphics[width=\columnwidth]{figs/degree_hop_distribution_teel.pdf}
    \caption{Distribution of degree and maximal hop for dependency graph nodes. The left plot shows the degree distribution, reflecting task interconnectivity, while the right plot presents the maximal hop distribution, highlighting long-range task dependencies.}
    \label{fig:degree_hop_distribution}
\end{figure}

\noindent These metrics emphasize the intricate nature of construction scheduling, with both high interconnectivity and significant multi-level dependencies. The insights derived from these analyses underline the importance of advanced frameworks like \TheName{} to manage such complexity in commercial construction projects.

\subsection{Correlation and Similarity Analysis of Project Attributes}

Understanding relationships among project attributes is essential for optimizing construction scheduling and dependency management. We conducted two types of analyses to capture both linear correlations and deeper semantic relationships:

\begin{itemize}
    \item \textbf{Correlation Analysis (Encoded Data)}: We examined linear dependencies between attributes by encoding categorical data as numeric codes and calculating Pearson correlation coefficients across project attributes. This method identifies direct dependencies that impact the project timeline and resource allocation, revealing structural insights into task sequences.
    
    \item \textbf{Cosine Similarity Analysis (Embeddings)}: Using embeddings generated from the \texttt{distilbert-base-uncased}\footnote{\url{https://huggingface.co/distilbert-base-uncased}}
 pre-trained language model, we captured semantic relationships among attributes that linear correlations might miss. This analysis highlights implicit, context-driven dependencies such as role interactions and spatial relationships, providing a nuanced view of project structure.
\end{itemize}

\begin{figure*}[h]
    \centering
    \includegraphics[width=\textwidth]{figs/combined_similarity_matrices.pdf}
    \caption{Combined Correlation and Cosine Similarity Heatmaps for Project Attributes. The left plot illustrates the correlation matrix based on encoded project data, highlighting linear relationships among attributes. The right plot presents the cosine similarity matrix based on embeddings, revealing deeper semantic associations among attributes.}
    \label{fig:combined_similarity_matrices}
\end{figure*}

\noindent Figure \ref{fig:combined_similarity_matrices} displays the results from both analyses, each providing unique insights:

\textbf{Correlation Matrix (Encoded Data)}: The left heatmap highlights linear relationships among attributes, with several notable correlations:

\begin{itemize}
    \item \textbf{Current Start and Current Finish}: The high correlation here reflects the dependency between start and finish dates, a foundational aspect of project scheduling.
    
    \item \textbf{Activity Status and Project Phase}: Correlations between activity status and project phase suggest that certain statuses align with specific phases, informing phase-based scheduling prompts.
    
    \item \textbf{Predecessor and Successor}: Strong correlation indicates that tasks have sequential dependencies, essential for creating an accurate task sequence.
\end{itemize}

In summary, these correlations reveal structural dependencies in project attributes, assisting in identifying key points in the scheduling and sequencing workflow. These insights enable more effective scheduling strategies by understanding which attributes inherently impact each other.

\textbf{Cosine Similarity Matrix (Embeddings)}: The right heatmap reveals semantic relationships between attributes, which help identify context-based dependencies:

\begin{itemize}
    \item \textbf{Subcontractor and Superintendent}: High similarity implies overlapping responsibilities between these roles, which can guide role-based dependencies in scheduling.
    
    \item \textbf{Discipline and Zone}: This similarity reflects the association between certain disciplines and zones, useful for location-based dependency prompts.
    
    \item \textbf{Project Phase and Activity Status}: Semantic alignment between phases and statuses provides a structured basis for task progression, useful for designing prompts that ensure coherent task sequences.
\end{itemize}

Overall, these embedding-based relationships uncover context-driven dependencies beyond simple correlations, offering a richer view of the project structure. Such insights are critical for tasks involving nuanced scheduling needs, as they reveal role interactions and locational dependencies that direct scheduling and resource assignment decisions.

\subsection{Unified Context Sampling Visualization}

To support effective construction scheduling, we employ a unified sampling method that extracts three distinct types of contextual information from project activities: Sequential Context, Hierarchical Context, and First-Order Context. Each method offers a unique approach to capturing dependencies and relationships among construction activities, facilitating comprehensive schedule optimization. Figures \ref{fig:sequential_context}, \ref{fig:hierarchical_context}, and \ref{fig:first_order_context} illustrate the structure and details of each context sampling method.

\begin{figure*}[ht]
    \centering
    \includegraphics[width=\textwidth]{figs/sequential_context.pdf}
    \caption{Sequential Context Sampling: This sampling method extracts nodes up to three hops away from each selected activity, representing predecessors and successors. Sequential sampling highlights dependencies that span across multiple stages in the construction workflow, enabling the model to understand task sequences and critical paths that influence the overall project schedule.}
    \label{fig:sequential_context}
\end{figure*}

In Figure \ref{fig:sequential_context}, Sequential Context Subgraph 1 (left) shows a network of activities where nodes represent individual tasks required for project completion, connected by directed edges that denote task dependencies. Each node connects to predecessors and successors up to three hops away, capturing dependencies such as Finish-to-Start (FS), Finish-to-Finish (FF), Start-to-Start (SS), and, though less common, Start-to-Finish (SF) relationships. This structure is critical for visualizing the overall task flow, identifying critical paths, and highlighting potential bottlenecks that could delay project delivery. Sequential Context Subgraph 2 (right) extends this by including a larger set of interconnected nodes, where tasks are annotated with additional details such as task duration, resource requirements, and start or finish times. This dense layout offers a comprehensive view of task sequences, helping project managers forecast delays, pinpoint bottlenecks, and dynamically adjust schedules to accommodate unforeseen changes.

\begin{figure*}[ht]
    \centering
    \includegraphics[width=\textwidth]{figs/hierarchical_context.pdf}
    \caption{Hierarchical Context Sampling: This sampling focuses on capturing nodes within the same Work Breakdown Structure (WBS) up to two hops. Hierarchical context provides insights into tasks grouped by project phases, illustrating how dependencies within each WBS segment affect the schedule’s progression.}
    \label{fig:hierarchical_context}
\end{figure*}

Figure \ref{fig:hierarchical_context} shows the Hierarchical Context Sampling. Hierarchical Context Subgraph 1 (left) presents nodes representing major project phases or milestones and their sub-tasks, organized within a structured hierarchy. Starting from a root node that signifies the overall project, dependencies cascade down through the graph, capturing relationships such as Start-to-Start and Finish-to-Start within a single WBS segment. This layout allows for visualizing dependencies specific to each phase, which is crucial for managing resources and time within discrete project stages. Hierarchical Context Subgraph 2 (right) shows a more streamlined arrangement, where tasks follow a linear progression, emphasizing phase-aligned scheduling adjustments. This structure helps project managers identify the critical path within each phase and adjust scheduling as needed to optimize workflow and resource allocation, while ensuring flexibility to adapt to phase-specific constraints and objectives.

\begin{figure*}[ht]
    \centering
    \includegraphics[width=\textwidth]{figs/first_order_context.pdf}
    \caption{First-Order Context Sampling: This method captures only direct predecessors and successors for each selected activity. First-order context highlights immediate task dependencies, providing a concise view of direct task relationships essential for high-priority scheduling adjustments.}
    \label{fig:first_order_context}
\end{figure*}

Figure \ref{fig:first_order_context} illustrates the First-Order Context Sampling. First-Order Context Subgraph 1 (left) shows a minimal structure with only one dependency, representing a direct, Finish-to-Start relationship between two tasks. This sparse setup allows for focused adjustments on critical dependencies without the complexity of additional nodes, making it ideal for high-priority scheduling where immediate, direct task relationships are paramount. First-Order Context Subgraph 2 (right) presents a more intricate structure with multiple tasks directly connected to a central node. This setup captures immediate predecessors and successors, including Start-to-Start (SS) and Finish-to-Finish (FF) dependencies, providing a concise overview of key relationships around the central task. Such a layout enables project managers to address dependencies that directly impact the timing and prioritization of essential tasks, helping maintain schedule adherence while focusing on high-impact areas of the project.

Each sampling method uniquely extracts relevant information from the project table, allowing the model to adaptively balance broad, phase-level dependencies with immediate task relationships. This unified approach to context sampling is instrumental in generating a well-rounded understanding of the construction schedule, enabling dynamic and context-aware adjustments.

\subsection{General Predefined Prompt Categories and Context Mapping}

The prompt system utilizes predefined categories and context mappings to structure data collection for various tasks in construction scheduling. Each category aligns with specific aspects of project analysis, guiding the language model to interpret context effectively. This design ensures the capture of dependencies, durations, and resource-based relationships essential for scheduling. 

\begin{itemize}
    \item \textbf{Activity Sequence and Timing}: This prompt helps the model list construction activities based on 'Current Start' and 'Current Finish' dates, following dependencies defined by 'Predecessor Details' and 'Successor Details'. This captures the linear progression of tasks, aiding structured timeline generation.

    \item \textbf{Calculate Activity Duration}: Focusing on each activity's duration based on start and finish dates, this prompt aids in establishing a timeline for the project. The model uses these durations to enhance scheduling precision and identify critical periods in the workflow.

    \item \textbf{Hierarchical Tree Structure}: By organizing tasks according to the Work Breakdown Structure (WBS), this prompt helps arrange tasks hierarchically and identify sequential requirements, essential for maintaining the logical flow within each project phase.

    \item \textbf{Assess Sequence Reconstruction}: This prompt directs the model to assess if task sequences can be reconstructed from available data, highlighting missing elements. Such reconstruction ensures dependencies are respected, crucial for seamless project continuity.

    \item \textbf{Analyze Time Relationships}: By analyzing time-based dependencies (e.g., FS, SS), this prompt helps identify parallel tasks and branches in dependency graphs, enabling effective time management across activities.

    \item \textbf{Overlapping Disciplines and Inter-Disciplinary Dependencies}: These prompts capture dependencies across overlapping and interconnected disciplines, facilitating resource alignment and identifying areas where interdisciplinary coordination is needed.

    \item \textbf{Area-Based Dependencies}: This prompt encourages the model to examine how dependencies align with specific areas, ensuring location-based planning aligns with the project’s spatial organization.
\end{itemize}

\subsection{Task-Specific Prompts for Data Collection}

For each specific task (Automated Planning (AP), Missing Value Prediction (MVP), Dependency Analysis (DA), and Construction Preference Alignment Direct Preference Optimization (CPA-DPO)), dedicated prompts have been designed to guide the language model in generating relevant outputs. Here’s an outline of each task-specific prompt:

\begin{itemize}
    \item \textbf{Prompt for AP}: This prompt instructs the model to focus on scheduling tasks based on 'Current Start' and 'Current Finish' dates, ensuring that task sequences respect dependencies. By using rules for sequencing and timing, the AP prompt facilitates logical task progression, essential for maintaining project coherence.

    \item \textbf{Prompt for MVP}: This prompt guides the model to predict missing values using both context and generated rules. It emphasizes the identification of critical data points for completion, enhancing data quality and completeness in project tables.

    \item \textbf{Prompt for DA}: Instructing the model to examine dependencies based on 'Predecessor Details' and 'Successor Details,' the DA prompt helps the model identify crucial task interactions. This supports dependency mapping, crucial for understanding the ripple effects of scheduling changes.

    \item \textbf{Context Polishing for CPA-DPO}: This prompt refines the generated output, ensuring it aligns with expert standards. The model adjusts for adherence to preferences, dependencies, and task prioritization, essential for optimized scheduling.
\end{itemize}

% Each of these prompts is designed to target specific needs within construction scheduling, ensuring the model outputs data that aligns with project management best practices. The combination of predefined prompt categories and task-specific prompts ensures that the model is well-equipped to handle the complexities of construction scheduling tasks dynamically and contextually.

Each prompt targets specific construction scheduling needs, aligning outputs with project management best practices and dynamically addressing task complexities.

\subsection{Industry Relevance and Considerations}

The automation of construction scheduling has long been an industry challenge due to the dynamic nature of project constraints, interdependent tasks, and expert-driven decision-making. While traditional methods rely on predefined heuristics and rule-based scheduling, they struggle to adapt to unexpected changes in workforce availability, material delays, or regulatory shifts. Large-scale projects, such as semiconductor fabrication, further complicate scheduling due to high coordination demands across multiple disciplines. Addressing these challenges requires an intelligent, adaptive system capable of learning from past schedules and dynamically updating plans based on new constraints.

A major consideration in adopting LLM-driven solutions for construction is their real-world integration and deployment feasibility. Existing project management software, such as Primavera P6 and BIM-based scheduling tools, is widely used by industry professionals. For AI-driven scheduling to be effective, it must complement these tools rather than replace them. The ability of retrieval-augmented models to incorporate structured industry knowledge and expert-aligned reinforcement learning provides a pathway for seamless integration, allowing construction professionals to leverage AI insights while maintaining human oversight in critical decision-making.

Additionally, concerns about data dependency and scalability must be addressed for broader industry adoption. While proprietary datasets are necessary for high-fidelity scheduling predictions, future research could explore the use of open-source construction datasets or synthetic data generation techniques to improve model robustness across diverse projects. Furthermore, factors such as computational overhead, latency, and cost must be considered in deployment, ensuring that AI-powered scheduling remains practical for real-world applications. By tackling these challenges, LLM-driven scheduling can move from a research prototype to a reliable industry tool that enhances efficiency, reduces project risks, and scales across complex construction environments.


\begin{figure*}[h]
\begin{tcolorbox}[colback=white, colframe=gray!70, colbacktitle=gray!20, coltitle=black, title=Sequential Context (Context 1), width=\textwidth, boxsep=5pt, left=3pt, right=3pt, top=5pt, bottom=5pt]
    \textbf{Activity Sequence and Timing} \newline
    List the sequence of construction activities based on the 'Current Start' and 'Current Finish' dates, ensuring they follow the correct order as indicated by 'Predecessor Details' and 'Successor Details'.
    \tcblower
    \textbf{Calculate Activity Duration} \newline
    Based on the 'Current Start' and 'Current Finish' dates, calculate the duration for each activity and establish the step-by-step timeline for the project.
\end{tcolorbox}
The Sequential Context prompt is designed to capture the linear progression of activities in construction. By focusing on the order and duration of activities, this context prompt aids in generating structured timelines, enabling the model to outline a clear sequence and allocate resources efficiently.
\end{figure*}

\begin{figure*}[h]
\begin{tcolorbox}[colback=white, colframe=gray!70, colbacktitle=gray!20, coltitle=black, title=First-Order Context (Context 2), width=\textwidth, boxsep=5pt, left=3pt, right=3pt, top=5pt, bottom=5pt]
    \textbf{Analyze Time Relationships} \newline
    Analyze the 'Predecessor Details' and 'Successor Details' to determine the time domain relationship between activities. Identify which activities are in parallel and the number of branches in the dependency graph.
    \tcblower
    \textbf{Overlapping Disciplines} \newline
    Identify overlapping disciplines from the 'Discipline' column and infer which dependency graphs can be merged based on this overlap.
    \tcblower
    \textbf{Inter-Disciplinary Dependencies} \newline
    Examine the inter-dependency between different disciplines and create a map of these relationships.
    \tcblower
    \textbf{Area-Based Dependencies} \newline
    Using the 'Area' column, analyze area-based dependencies and how they affect the sequence of construction activities.
\end{tcolorbox}
The First-Order Context prompt focuses on immediate dependencies and relationships between tasks. By analyzing time, disciplinary overlaps, and area-based dependencies, this prompt enables the model to capture critical dependencies that could impact the flow of work and resource allocation across parallel activities.
\end{figure*}

\begin{figure*}[h]
\begin{tcolorbox}[colback=white, colframe=gray!70, colbacktitle=gray!20, coltitle=black, title=Hierarchical Context (Context 3), width=\textwidth, boxsep=5pt, left=3pt, right=3pt, top=5pt, bottom=5pt]
    \textbf{Hierarchical Tree Structure} \newline
    Organize the activities into a hierarchical tree structure based on their WBS and identify any activities that should be sequential but are not currently listed as such.
    \tcblower
    \textbf{Assess Sequence Reconstruction} \newline
    For each activity, determine if the sequence can be recovered from the given data. If not, specify what critical information is missing and suggest how to bridge the identified gaps.
\end{tcolorbox}
The Hierarchical Context prompt helps the model understand hierarchical structures in project planning. By focusing on organizing tasks based on work breakdown structure (WBS), this context prompt aids in identifying gaps in sequencing and structuring project phases logically.
\end{figure*}

\begin{figure*}[h]
\begin{tcolorbox}[colback=white, colframe=gray!70, colbacktitle=gray!20, coltitle=black, title=Automated Planning (AP) Prompts, width=\textwidth, boxsep=5pt, left=3pt, right=3pt, top=5pt, bottom=5pt]
    \textbf{AP - Part 1} \newline
    You are a virtual construction expert collaborating with a larger LLM to automate the construction schedule. Use the 'Current Start' and 'Current Finish' dates in the context to ensure tasks are scheduled based on their dependencies. Explain how the selected rules help guide the automation of task sequencing and timing.
    \tcblower
    \textbf{AP - Part 2} \newline
    Justify why these specific rules and context elements are crucial for automating the schedule. Describe the connection between the context and rules, and provide logical reasoning for why these choices will result in a successful automation process.
\end{tcolorbox}
The AP prompt focuses on scheduling construction activities based on start and finish dates, with an emphasis on the rules that support task sequencing and timing. This prompt aims to ensure coherent automation logic while aligning with project constraints and expert expectations.
\end{figure*}

\begin{figure*}[h]
\begin{tcolorbox}[colback=white, colframe=gray!70, colbacktitle=gray!20, coltitle=black, title=Missing Value Prediction (MVP) Prompts, width=\textwidth, boxsep=5pt, left=3pt, right=3pt, top=5pt, bottom=5pt]
    \textbf{MVP - Part 1} \newline
    Based on the following information, choose the correct values for the missing columns. Return the values as a list, separated by commas, with each value enclosed within [Value] and [/Value] tags. The list should contain exactly three values, corresponding to the columns listed in the same order.
    \tcblower
    \textbf{MVP - Part 2} \newline
    This part provides the row input, static knowledge, and context information that the model will use to identify missing values and fill them accurately.
\end{tcolorbox}
The MVP prompt is essential for accurately predicting missing data in construction tables, using both static knowledge and contextual details. This prompt is designed to help the model make accurate value predictions, enhancing data completeness and reliability.
\end{figure*}

\begin{figure*}[h]
\begin{tcolorbox}[colback=white, colframe=gray!70, colbacktitle=gray!20, coltitle=black, title=Dependency Analysis (DA) Prompts, width=\textwidth, boxsep=5pt, left=3pt, right=3pt, top=5pt, bottom=5pt]
    \textbf{DA - Part 1} \newline
    You are a virtual construction expert collaborating with a larger LLM to analyze dependencies between construction activities. Focus on identifying key dependencies using the 'Predecessor Details' and 'Successor Details' in the context. Explain how and why the selected rules are relevant for understanding the dependencies between activities.
    \tcblower
    \textbf{DA - Part 2} \newline
    Connect these rules to specific parts of the context. Ensure that the relationship between the context and rules is clearly articulated, showing logical reasoning behind the choices made for this analysis.
\end{tcolorbox}
The DA prompt guides the model in identifying and explaining dependencies between construction activities, with emphasis on critical tasks and their interactions. This prompt supports dependency mapping, which is crucial for project planning and risk management.
\end{figure*}

% \begin{figure*}[h]
% \begin{tcolorbox}[colback=white, colframe=gray!70, colbacktitle=gray!20, coltitle=black, title=Context Polishing for CPA-DPO Prompts, width=\textwidth, boxsep=5pt, left=3pt, right=3pt, top=5pt, bottom=5pt]
%     \textbf{Context Polishing for CPA-DPO} \newline
%     As a virtual construction scheduling expert, refine the following output to ensure it aligns with expert expectations. The output should provide coherent and contextually relevant responses to scheduling needs, integrating expert rules and project-specific knowledge seamlessly. Emphasize adherence to preferences and explain any dependencies or task prioritizations that support an optimized construction schedule.
% \end{tcolorbox}
% The Context Polishing prompt refines responses to align with expert preferences, ensuring coherent, relevant outputs for scheduling needs. It supports the Direct Preference Optimization (DPO) process, enhancing the alignment of generated content with real-world project standards and expectations.
% \end{figure*}

\begin{figure*}[h]
\begin{tcolorbox}[colback=white, colframe=gray!70, colbacktitle=gray!20, coltitle=black, title=Context Polishing for CPA-DPO Prompts, width=\textwidth, boxsep=5pt, left=3pt, right=3pt, top=5pt, bottom=5pt]
    \textbf{Context Polishing for CPA-DPO - Part 1} \newline
    As a virtual construction scheduling expert, refine the following output to ensure it aligns with expert expectations. Your role involves guiding a larger LLM by providing clear context, expert rules, and structured instructions for three primary tasks:
    \begin{itemize}
        \item \textbf{Missing Value Prediction:} Select and explain relevant context elements crucial for filling in missing values. Use expert rules to guide predictions and clarify their connection to the context.
        \item \textbf{Dependency Analysis:} Analyze and explain activity dependencies using 'Predecessor Details' and 'Successor Details.' Highlight how the rules inform these relationships.
        \item \textbf{Schedule Automation:} Automate task scheduling using 'Current Start' and 'Current Finish' dates, prioritizing based on criticality and dependencies. Apply rules to ensure task order and dependencies are respected.
    \end{itemize}
    \tcblower
    \textbf{Context Polishing for CPA-DPO - Part 2} \newline
    The output should provide coherent and contextually relevant responses to scheduling needs, integrating expert rules and project-specific knowledge seamlessly. Emphasize adherence to preferences and explain any dependencies or task prioritizations that support an optimized construction schedule.
\end{tcolorbox}
The Context Polishing prompt ensures that responses align with expert preferences, providing clear, structured guidance for missing value prediction, dependency analysis, and schedule automation. It supports the Direct Preference Optimization (DPO) process by enhancing the alignment of generated content with real-world project standards and expectations.
\end{figure*}

% This appendix presents an exhaustive description of the structured prompts used in construction schedule automation. Each prompt category has been tailored to capture specific scheduling nuances, from dependency analysis to automation, ensuring the model provides outputs aligned with expert project management standards.

% This appendix details the structured prompts used in construction schedule automation, designed to address specific scheduling aspects like dependency analysis and automation, ensuring outputs meet expert project management standards.

% \bibliographystyle{acl_natbib}

\end{document}
