\begin{table}[t]
\small
  \centering
  \begin{tabular}{l|ccc}
    \hline
    \textbf{Dataset} & \textbf{\method{}} & \textbf{Only Image} & \textbf{Only Text} \\
    \hline
    TABMWP  & \textbf{87.50} & 76.86 & 85.71 \\
    WTQ     & \textbf{55.77} & 51.47 & 48.84 \\
    HiTab   & \textbf{63.00} & 62.18 & 57.36 \\
    TAT-QA  & \textbf{60.75} & 57.38 & 55.95 \\
    FeTaQA  & \textbf{33.18} & 29.50 & 32.31 \\
    \hline
    TabFact  & \textbf{82.27} & 79.88 & 80.14 \\
    InfoTabs & 75.74 & \textbf{76.16 }& 72.75 \\
    \hline
  \end{tabular}
  \caption{\label{tab:ablation_hippo}Performance of \method{} Using the Table Representation of Different Modalities. All models are implemented with MiniCPM-V-2.6.}
\end{table}
\section{Appendix}

\subsection{License}
This study uses MMTab (Apache-2.0 License\footnote{\href{https://www.apache.org/licenses/LICENSE-2.0}{https://www.apache.org/licenses/LICENSE-2.0}}), 
TABMWP (MIT License\footnote{\href{https://opensource.org/licenses/MIT}{https://opensource.org/licenses/MIT}}), 
WTQ (MIT License), 
HiTab (C-UDA License\footnote{\href{https://cdla.dev/computational-use-of-data-agreement-v1-0/}{https://cdla.dev/computational-use-of-data-agreement-v1-0/}}), 
TAT-QA (MIT License), 
FeTaQA (CC-BY-SA 4.0 License\footnote{\href{https://creativecommons.org/licenses/by-sa/4.0/}{https://creativecommons.org/licenses/by-sa/4.0/}}), 
TabFact (CC-BY-SA 4.0 License), 
and InfoTabs (Apache-2.0 License) in experiments. 
All of these licenses and agreements allow the use of their data for academic purposes.

\subsection{Performance of \method{} Using Table Representations of Different Modalities}
This section shows the performance of \method{} using various table representations: text-based, image-based, and multi-modal table representations.

As shown in Table~\ref{tab:ablation_hippo}, \method{} with a multi-modal table representation outperforms both text-based and image-based representations across most TQA and TFV datasets. Specifically, on average, the multi-modal representation yields a significant improvement of over 3.5\% across all datasets, compared to both image-based and text-based representations. The superior performance underscores \method{}'s ability to effectively learn and integrate semantic information from both text-based and image-based representations, leading to more comprehensive table understanding.

\subsection{Effectiveness of Different Training Strategies}
This experiment evaluates the effectiveness of table understanding by examining the prediction consistency across table understanding models trained using different strategies.

As shown in Figure~\ref{fig:consistency}, we evaluate the consistency of Zero-Shot, SFT, DPO, and \method{} methods in predicting the golden labels. Specifically, we randomly sample 500 cases and ask each model to generate 10 outputs for consistency evaluation. A higher consistency score indicates that the model produces more confident predictions aligned with the ground truth. Overall, DPO-based optimization methods improve the consistency of MLLM predictions with respect to the ground truth, suggesting that DPO assigns a higher probability than SFT for generating the correct answer by learning from preference pairs. Notably, \method{} further enhances its prediction consistency on both datasets, demonstrating that the training strategy of \method{} helps MLLMs make more confident and accurate predictions. \method{} enhances the DPO training process by sampling more diverse responses from the table representations of different modalities.
\begin{figure}[t]
    \centering
    \subfigure[TAT-QA.]{
    \includegraphics[width=0.45\linewidth]{image/consistency_ratio_sampled_tat-qa.pdf}
        \label{fig:subfig1}
    }
    \subfigure[TabFact.]{
        \includegraphics[width=0.45\linewidth]{image/consistency_ratio_sampled_tabfact.pdf}
        \label{fig:subfig2}
    }
    \caption{The Consistency of Different Models for Ground Truth Label Prediction.}
    \label{fig:consistency}
\end{figure}


\begin{table}[t]
\small
\centering
\begin{tabular}{lccc} 
\hline
\multirow{2}{*}{\textbf{Method}} & \multicolumn{3}{c}{\textbf{Table Size}} \\ \cline{2-4}
& Small & Medium  & Large  \\ \hline
Zero-Shot & 56.20 & 45.20 & 35.65 \\ \hline
\method{} & \textbf{59.94} & \textbf{46.98} & \textbf{43.94} \\ \hline
\end{tabular}
\caption{Performance of Zero-Shot and \method{} Models on the WTQ dataset.}
\label{tab:performance_different_table_size}
\end{table}

\definecolor{titlecolor}{rgb}{0.9, 0.5, 0.1}
\definecolor{anscolor}{rgb}{0.2, 0.5, 0.8}
\definecolor{labelcolor}{HTML}{48a07e}
\begin{table*}[h]
	\centering
	
 % \vspace{-0.2cm}
	
	\begin{center}
		\begin{tikzpicture}[
				chatbox_inner/.style={rectangle, rounded corners, opacity=0, text opacity=1, font=\sffamily\scriptsize, text width=5in, text height=9pt, inner xsep=6pt, inner ysep=6pt},
				chatbox_prompt_inner/.style={chatbox_inner, align=flush left, xshift=0pt, text height=11pt},
				chatbox_user_inner/.style={chatbox_inner, align=flush left, xshift=0pt},
				chatbox_gpt_inner/.style={chatbox_inner, align=flush left, xshift=0pt},
				chatbox/.style={chatbox_inner, draw=black!25, fill=gray!7, opacity=1, text opacity=0},
				chatbox_prompt/.style={chatbox, align=flush left, fill=gray!1.5, draw=black!30, text height=10pt},
				chatbox_user/.style={chatbox, align=flush left},
				chatbox_gpt/.style={chatbox, align=flush left},
				chatbox2/.style={chatbox_gpt, fill=green!25},
				chatbox3/.style={chatbox_gpt, fill=red!20, draw=black!20},
				chatbox4/.style={chatbox_gpt, fill=yellow!30},
				labelbox/.style={rectangle, rounded corners, draw=black!50, font=\sffamily\scriptsize\bfseries, fill=gray!5, inner sep=3pt},
			]
											
			\node[chatbox_user] (q1) {
				\textbf{System prompt}
				\newline
				\newline
				You are a helpful and precise assistant for segmenting and labeling sentences. We would like to request your help on curating a dataset for entity-level hallucination detection.
				\newline \newline
                We will give you a machine generated biography and a list of checked facts about the biography. Each fact consists of a sentence and a label (True/False). Please do the following process. First, breaking down the biography into words. Second, by referring to the provided list of facts, merging some broken down words in the previous step to form meaningful entities. For example, ``strategic thinking'' should be one entity instead of two. Third, according to the labels in the list of facts, labeling each entity as True or False. Specifically, for facts that share a similar sentence structure (\eg, \textit{``He was born on Mach 9, 1941.''} (\texttt{True}) and \textit{``He was born in Ramos Mejia.''} (\texttt{False})), please first assign labels to entities that differ across atomic facts. For example, first labeling ``Mach 9, 1941'' (\texttt{True}) and ``Ramos Mejia'' (\texttt{False}) in the above case. For those entities that are the same across atomic facts (\eg, ``was born'') or are neutral (\eg, ``he,'' ``in,'' and ``on''), please label them as \texttt{True}. For the cases that there is no atomic fact that shares the same sentence structure, please identify the most informative entities in the sentence and label them with the same label as the atomic fact while treating the rest of the entities as \texttt{True}. In the end, output the entities and labels in the following format:
                \begin{itemize}[nosep]
                    \item Entity 1 (Label 1)
                    \item Entity 2 (Label 2)
                    \item ...
                    \item Entity N (Label N)
                \end{itemize}
                % \newline \newline
                Here are two examples:
                \newline\newline
                \textbf{[Example 1]}
                \newline
                [The start of the biography]
                \newline
                \textcolor{titlecolor}{Marianne McAndrew is an American actress and singer, born on November 21, 1942, in Cleveland, Ohio. She began her acting career in the late 1960s, appearing in various television shows and films.}
                \newline
                [The end of the biography]
                \newline \newline
                [The start of the list of checked facts]
                \newline
                \textcolor{anscolor}{[Marianne McAndrew is an American. (False); Marianne McAndrew is an actress. (True); Marianne McAndrew is a singer. (False); Marianne McAndrew was born on November 21, 1942. (False); Marianne McAndrew was born in Cleveland, Ohio. (False); She began her acting career in the late 1960s. (True); She has appeared in various television shows. (True); She has appeared in various films. (True)]}
                \newline
                [The end of the list of checked facts]
                \newline \newline
                [The start of the ideal output]
                \newline
                \textcolor{labelcolor}{[Marianne McAndrew (True); is (True); an (True); American (False); actress (True); and (True); singer (False); , (True); born (True); on (True); November 21, 1942 (False); , (True); in (True); Cleveland, Ohio (False); . (True); She (True); began (True); her (True); acting career (True); in (True); the late 1960s (True); , (True); appearing (True); in (True); various (True); television shows (True); and (True); films (True); . (True)]}
                \newline
                [The end of the ideal output]
				\newline \newline
                \textbf{[Example 2]}
                \newline
                [The start of the biography]
                \newline
                \textcolor{titlecolor}{Doug Sheehan is an American actor who was born on April 27, 1949, in Santa Monica, California. He is best known for his roles in soap operas, including his portrayal of Joe Kelly on ``General Hospital'' and Ben Gibson on ``Knots Landing.''}
                \newline
                [The end of the biography]
                \newline \newline
                [The start of the list of checked facts]
                \newline
                \textcolor{anscolor}{[Doug Sheehan is an American. (True); Doug Sheehan is an actor. (True); Doug Sheehan was born on April 27, 1949. (True); Doug Sheehan was born in Santa Monica, California. (False); He is best known for his roles in soap operas. (True); He portrayed Joe Kelly. (True); Joe Kelly was in General Hospital. (True); General Hospital is a soap opera. (True); He portrayed Ben Gibson. (True); Ben Gibson was in Knots Landing. (True); Knots Landing is a soap opera. (True)]}
                \newline
                [The end of the list of checked facts]
                \newline \newline
                [The start of the ideal output]
                \newline
                \textcolor{labelcolor}{[Doug Sheehan (True); is (True); an (True); American (True); actor (True); who (True); was born (True); on (True); April 27, 1949 (True); in (True); Santa Monica, California (False); . (True); He (True); is (True); best known (True); for (True); his roles in soap operas (True); , (True); including (True); in (True); his portrayal (True); of (True); Joe Kelly (True); on (True); ``General Hospital'' (True); and (True); Ben Gibson (True); on (True); ``Knots Landing.'' (True)]}
                \newline
                [The end of the ideal output]
				\newline \newline
				\textbf{User prompt}
				\newline
				\newline
				[The start of the biography]
				\newline
				\textcolor{magenta}{\texttt{\{BIOGRAPHY\}}}
				\newline
				[The ebd of the biography]
				\newline \newline
				[The start of the list of checked facts]
				\newline
				\textcolor{magenta}{\texttt{\{LIST OF CHECKED FACTS\}}}
				\newline
				[The end of the list of checked facts]
			};
			\node[chatbox_user_inner] (q1_text) at (q1) {
				\textbf{System prompt}
				\newline
				\newline
				You are a helpful and precise assistant for segmenting and labeling sentences. We would like to request your help on curating a dataset for entity-level hallucination detection.
				\newline \newline
                We will give you a machine generated biography and a list of checked facts about the biography. Each fact consists of a sentence and a label (True/False). Please do the following process. First, breaking down the biography into words. Second, by referring to the provided list of facts, merging some broken down words in the previous step to form meaningful entities. For example, ``strategic thinking'' should be one entity instead of two. Third, according to the labels in the list of facts, labeling each entity as True or False. Specifically, for facts that share a similar sentence structure (\eg, \textit{``He was born on Mach 9, 1941.''} (\texttt{True}) and \textit{``He was born in Ramos Mejia.''} (\texttt{False})), please first assign labels to entities that differ across atomic facts. For example, first labeling ``Mach 9, 1941'' (\texttt{True}) and ``Ramos Mejia'' (\texttt{False}) in the above case. For those entities that are the same across atomic facts (\eg, ``was born'') or are neutral (\eg, ``he,'' ``in,'' and ``on''), please label them as \texttt{True}. For the cases that there is no atomic fact that shares the same sentence structure, please identify the most informative entities in the sentence and label them with the same label as the atomic fact while treating the rest of the entities as \texttt{True}. In the end, output the entities and labels in the following format:
                \begin{itemize}[nosep]
                    \item Entity 1 (Label 1)
                    \item Entity 2 (Label 2)
                    \item ...
                    \item Entity N (Label N)
                \end{itemize}
                % \newline \newline
                Here are two examples:
                \newline\newline
                \textbf{[Example 1]}
                \newline
                [The start of the biography]
                \newline
                \textcolor{titlecolor}{Marianne McAndrew is an American actress and singer, born on November 21, 1942, in Cleveland, Ohio. She began her acting career in the late 1960s, appearing in various television shows and films.}
                \newline
                [The end of the biography]
                \newline \newline
                [The start of the list of checked facts]
                \newline
                \textcolor{anscolor}{[Marianne McAndrew is an American. (False); Marianne McAndrew is an actress. (True); Marianne McAndrew is a singer. (False); Marianne McAndrew was born on November 21, 1942. (False); Marianne McAndrew was born in Cleveland, Ohio. (False); She began her acting career in the late 1960s. (True); She has appeared in various television shows. (True); She has appeared in various films. (True)]}
                \newline
                [The end of the list of checked facts]
                \newline \newline
                [The start of the ideal output]
                \newline
                \textcolor{labelcolor}{[Marianne McAndrew (True); is (True); an (True); American (False); actress (True); and (True); singer (False); , (True); born (True); on (True); November 21, 1942 (False); , (True); in (True); Cleveland, Ohio (False); . (True); She (True); began (True); her (True); acting career (True); in (True); the late 1960s (True); , (True); appearing (True); in (True); various (True); television shows (True); and (True); films (True); . (True)]}
                \newline
                [The end of the ideal output]
				\newline \newline
                \textbf{[Example 2]}
                \newline
                [The start of the biography]
                \newline
                \textcolor{titlecolor}{Doug Sheehan is an American actor who was born on April 27, 1949, in Santa Monica, California. He is best known for his roles in soap operas, including his portrayal of Joe Kelly on ``General Hospital'' and Ben Gibson on ``Knots Landing.''}
                \newline
                [The end of the biography]
                \newline \newline
                [The start of the list of checked facts]
                \newline
                \textcolor{anscolor}{[Doug Sheehan is an American. (True); Doug Sheehan is an actor. (True); Doug Sheehan was born on April 27, 1949. (True); Doug Sheehan was born in Santa Monica, California. (False); He is best known for his roles in soap operas. (True); He portrayed Joe Kelly. (True); Joe Kelly was in General Hospital. (True); General Hospital is a soap opera. (True); He portrayed Ben Gibson. (True); Ben Gibson was in Knots Landing. (True); Knots Landing is a soap opera. (True)]}
                \newline
                [The end of the list of checked facts]
                \newline \newline
                [The start of the ideal output]
                \newline
                \textcolor{labelcolor}{[Doug Sheehan (True); is (True); an (True); American (True); actor (True); who (True); was born (True); on (True); April 27, 1949 (True); in (True); Santa Monica, California (False); . (True); He (True); is (True); best known (True); for (True); his roles in soap operas (True); , (True); including (True); in (True); his portrayal (True); of (True); Joe Kelly (True); on (True); ``General Hospital'' (True); and (True); Ben Gibson (True); on (True); ``Knots Landing.'' (True)]}
                \newline
                [The end of the ideal output]
				\newline \newline
				\textbf{User prompt}
				\newline
				\newline
				[The start of the biography]
				\newline
				\textcolor{magenta}{\texttt{\{BIOGRAPHY\}}}
				\newline
				[The ebd of the biography]
				\newline \newline
				[The start of the list of checked facts]
				\newline
				\textcolor{magenta}{\texttt{\{LIST OF CHECKED FACTS\}}}
				\newline
				[The end of the list of checked facts]
			};
		\end{tikzpicture}
        \caption{GPT-4o prompt for labeling hallucinated entities.}\label{tb:gpt-4-prompt}
	\end{center}
\vspace{-0cm}
\end{table*}
\subsection{Performance of \method{} on Tables of Different Scales}
In this section, we analyze the performance of \method{} and Zero-Shot on tables of varying scales.

In our experiments, we categorize tables into three groups: Small, Medium, and Large. Specifically, Small refers to tables with fewer than 1,000 tokens, Medium includes tables with 1,000 to 2,000 tokens, and Large encompasses tables with more than 2,000 tokens. The distribution is as follows: Small tables (70.28\%), Medium tables (19.45\%), and Large tables (10.27\%).

As shown in Table~\ref{tab:performance_different_table_size}, both models exhibit a decrease in accuracy as table size increases, highlighting the challenge of capturing and reasoning with complex information from larger tables. Notably, \method{} consistently outperforms Zero-Shot across all table scales, particularly for large tables, demonstrating its superior robustness in handling larger tables. This performance advantage suggests that \method{} remains effective even as the complexity and scale of the tabular data increase.

\subsection{Additional Experimental Details}\label{app:experiments}
In this section, we provide a detailed description of the steps to construct the DPO training data.

The training data is sourced from~\citet{ZhengFSS0J024}. The inputs are categorized into three types: text-based table representations ($L(T)$), image-based table representations ($V(T)$), and multi-modal table representations, which are formed by concatenating the text-based and image-based representations ($L(T)$, $V(T)$). The construction of the DPO training dataset utilized \texttt{TABMWP}, \texttt{WTQ}, \texttt{TAT-QA}, \texttt{TabFact}, and \texttt{InfoTabs}. We exclude \texttt{FeTaQA} due to its evaluation metric being BLEU, which does not focus on the accuracy of question answering. Additionally, the \texttt{HiTab} dataset is excluded because it involves multi-level tables, which present formatting challenges when converted to Markdown. This conversion can lead to formatting inconsistencies, making it less suitable for training. From each of the chosen datasets, we extract 2,000 instances, resulting in a combined dataset of 10,000 training instances.

For data sampling, we use the MiniCPM-V-2.6 model with a temperature setting of 1 to generate 10 candidate responses for each modality. These responses are rigorously evaluated against ground truth answers to assess their accuracy. For DPO training, the ground truth is labeled as the positive response $y^+$, while the most frequent incorrect response is designated as the negative one $y^-$ for DPO training (Eq.~\ref{eq:dpo}).


\subsection{Prompt Templates Used in \method{}} \label{app:hippo_prompt}
We follow the approach of previous work~\cite{ZhengFSS0J024} modifying the prompt templates to better align with our objectives for multi-modal table representations. The prompt templates used in our experiments are shown in Figure~\ref{fig:prompt}.

\subsection{Prompt Templates Used to Generate Chain-of-Thought}\label{app:cot_prompt}
In this section, we represent the CoT (Chain of Thought) prompt we used in Figure~\ref{fig:cot_prompt}. For each table representation: image, text, and multi-modal, we provide the model with both the table representation and its corresponding answer. The model is then instructed to generate a modality-specific thinking step.

\begin{figure}[t]
\centering
\includegraphics[width=\linewidth]{image/cot_prompt}
\caption{CoT Prompt Templates Used in \method{}.} 
\label{fig:cot_prompt}
\end{figure}