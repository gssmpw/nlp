\section{Introduction}
Tabular data is pervasive in our daily lives, appearing in formats such as databases, scientific articles, web pages, and spreadsheets~\cite{chen2000mining,hurst2000interpretation,hu2023chatdb}. The structured nature of tabular data enables the systematic organization of information into rows and columns, facilitating efficient sorting, querying, and manipulation~\citep{pujara2021tables, chen2020open}. Consequently, table understanding and reasoning have emerged as a significant area of interest in NLP, garnering much attention from researchers~\citep{bao2018table, zhang2024survey}.
\section{Example}
\label{app:example}

Table~\ref{tab:example} shows a counseling session between a client agent and CAMI, with the client's profile shown in Table~\ref{tab:example profile}. During the session, CAMI explores Health and Relationships topics in the first 6 turns.  It subsequently performs Step Into operation from Relationships to Parenting topics in Turn 7, and from Parenting to Paternal Bond topics in Turn 9, before a Switch operation to Habituation topic in Turn 11.  The full exploration path is visualized in Figure~\ref{fig:exploration path}.

\begin{table*}[ht]
\centering
\begin{tabularx}{\textwidth}{c X}
\toprule
        {\bf Turn ID} & {\bf Utterances}                     \\ \midrule
1       & Counselor: Hello. How are you?                                                                                                                                                                                                                                                                                                                     \\ \hline
2       & Client: I am good. What about you?                                                                                                                                                                                                                                                                                                                 \\ \hline
3       & Counselor: I'm doing well, thanks for asking! Can you share how you've been feeling about your health lately, especially in relation to smoking?                                                                                                                                                                                                   \\ \hline
4       & Client: I've been pretty busy with work and juggling family stuff. Sometimes I just think about how nice it is to take a break with my coworkers outside, chatting and during those moments, life feels a bit more relaxed. You know how it is—it’s nice to unwind.                                                                                \\ \hline
5       & Counselor: It sounds like smoking is a comforting routine for you amidst your busy life with work and your daughter. I wonder how it might impact the time you spend with your family.                                                                                                                                                             \\ \hline
6       & Client: I get what you mean, but I really don't see my smoking as an issue. It helps me manage stress, and I don’t think it affects my time with my daughter. We still have good moments together, and I wouldn't want to change that.                                                                                                             \\ \hline
7       & Counselor: It sounds like you find comfort in smoking to manage stress, and that feels crucial to your parenting now. I wonder, though, how you might feel if your daughter started to see smoking as a way to unwind, too. What impact would that have on your relationship with her?                                                             \\ \hline
8       & Client: That’s a tough thought. I remember starting young and how hard it was to quit later. I wouldn’t want that for her. It makes me reflect on what I’m modeling.                                                                                                   \\ \hline
9       & Counselor: It's great that you're reflecting on your modeling for your daughter. Perhaps seeing smoking as something that may weaken your bond could help reshape how you approach stress—considering healthier ways to connect can actually strengthen your relationship and set a positive example for her.                                      \\ \hline
10      & Client: I appreciate that perspective. I do want to set a strong example for her. My experience with my own family and their struggles makes me cautious. I see the patterns and don’t want her to feel that smoking is a go-to solution for stress, like I did.                                                                                   \\ \hline
11      & Counselor: It sounds like your family’s struggles have really shaped your perspective on smoking. You want to protect your daughter from those same patterns, knowing that smoking can not only impact her stress management but also her overall learning and adaptation to challenges. That’s a powerful motivation.                             \\ \hline
12      & Client: Exactly. I want her to learn healthier coping strategies instead of following in my footsteps. I realize that I need to make a change for both our sakes. It’s not just about me; it’s about creating an environment where she can thrive without those habits. It's time to find better ways to manage stress that we can share together.   
\\ \bottomrule
\end{tabularx}
\caption{Counseling Session between CAMI and a Simulated Client.}
\label{tab:example}
\end{table*}



\begin{table*}[ht]
\centering
\begin{tabularx}{\textwidth}{l X}
\toprule
{\bf Profile Item}        & {\bf Description}                     \\ \midrule
Behavioral Problem  & Smoking \\ \hline
State of Mind       & Precontemplation \\ \hline
Personas            & You smoked when you were young.", "You have a 12-year-old daughter. \newline The friends of the your daughter have tried smoking. \newline When you are at work, you all go out and smoke and chit chat and talk about things. \newline You like to have a cigarette after eating. \\ \hline
Motivation          & You are motivated because of the risk of smoking in habituation for your daughter, as you don't want your daughter to think smoking is good thing.    \\ \hline
Beliefs             & You feel smoking enjoyable and relaxing. \newline You would be left out if you would hate to not smoke. \newline You use smoking as a way to manage weight and control eating habits. \newline You have been smoking for a long time and feels it's almost impossible to quit. \\ \hline
Motivation Topics          & Interpersonal Relationships, Parenting, Habituation
\\ \bottomrule
\end{tabularx}
\caption{Client's Profile in the Example.}
\label{tab:example profile}
\end{table*}


\begin{figure*}
    \centering
    \includegraphics[width=\linewidth]{figs/example_explore_path.pdf}
    \caption{Topic Exploration Path by the Counselor in the Example.}
    \label{fig:exploration path}
\end{figure*}



Thrived on the logical reasoning capabilities of Large Language Models (LLMs), using LLMs for dealing with table-related tasks has become a mainstream research direction~\cite{chen2023large,zhang2024survey,dong2024large}. Existing table understanding methods convert tables into linear text sequences and focus on designing prompts or instructions to stimulate LLMs to conduct effective reasoning over tables~\cite{chen2023large,wang2024chainoftable}. However, they typically provide a fixed text representation of the tabular format for reasoning. Recent studies have also shown that LLMs are sensitive to the text representation of tables~\cite{liu-etal-2024-rethinking}, motivating researchers to explore the most suitable text-based tabular formats for different table understanding scenarios~\cite{zhang2024flextaf,sui2024table,singha2023tabular}.


Besides text-based table representations, many works use the screenshot of table as its image-based representation during reasoning to explore the effectiveness of Multi-modal Large Language Models (MLLMs) in understanding table images~\cite{deng2024tables,ZhengFSS0J024}. As shown in Figure~\ref{fig:example}, both text-based and image-based table representations potentially play distinct roles in enhancing the table reasoning abilities of MLLMs. Specifically, in the first case, the question asks, ``What is the total number of wins listed for the United States?'', which requires the model to identify the wins of the United States, namely ``18'', ``2'' and ``2'', and then sum them to obtain the correct answer, ``22''. The text-based table representation enables LLMs to produce the correct answer because the question relies more on the arithmetic ability of language models. In contrast, the image-based table representation allows MLLMs to correctly answer the question in the second case. This is enabled by the visual annotation of teams with different colors to represent the win-loss situation. Both the color and cell position in the image provide crucial semantics to help MLLMs accurately answer the question. Despite these advantages, existing works~\cite{deng2024tables,ZhengFSS0J024} mainly focus on investigating the table understanding capabilities of MLLMs using unimodal representations, leaving room for further exploration of multi-modal representations to enable more effective table reasoning.

This paper introduces the \textbf{H}ybr\textbf{I}d-modal \textbf{P}reference o\textbf{P}timizati\textbf{O}n (\method{}) model, which integrates both text-based and image-based table representations for enhancing the table understanding capability of MLLMs. Specifically, \method{} proposes a Hybrid-Modal Preference Optimization method to guide MLLMs in answering questions by leveraging more comprehensive information from different modalities of table representations. \method{} prompts the MLLM to generate responses based on both unimodal and multi-modal representations of the table. Then, it selects the most representative negative responses using the self-consistency~\cite{liu-etal-2024-rethinking} of MLLMs when answering questions based on different modalities, thereby mitigating unnecessary modality bias during training. These negative responses are subsequently collected to optimize the MLLMs using the DPO method~\cite{rafailov2023direct}, helping the model to assign higher probabilities to ground truth answers over negative responses.

Our experiments demonstrate the effectiveness of our \method{} model by achieving more than a 4\% improvement over different table understanding models, which underscores the importance of incorporating both text-based and image-based representations in table understanding tasks. Additionally, \method{} significantly enhances the performance of MLLMs even with unimodal table representations, illustrating the generalization ability of our training method. Our further analyses show that \method{} optimizes MLLMs to better extract semantic information, generate more consistent answers, and engage in diverse reasoning processes based on table representations of different modalities, thereby enabling more accurate predictions based on multi-modal table representations.