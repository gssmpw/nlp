\documentclass{iise}

\frenchspacing
\linespread{1.2}

\usepackage{bbold}
\usepackage{amsfonts,amsmath,amssymb,amsthm}
\usepackage{pifont}
\usepackage{accents}
\usepackage{mathrsfs}
\usepackage[dvipsnames]{xcolor}
\usepackage{pdflscape}
\usepackage{mathtools}
\usepackage{cancel}
\usepackage{hyperref}
\usepackage{nicefrac}

\usepackage{booktabs}
\usepackage{todonotes}

\usepackage{multirow}

\usepackage{float}
\floatstyle{ruled}
\newfloat{model}{thp}{lop}
\floatname{model}{Model}
\newcounter{models}

\theoremstyle{definition}
\newtheorem{definition}{Definition}[section]
\newtheorem{theorem}{Theorem}[section]
\newtheorem{lemma}[theorem]{Lemma}
\newtheorem{proposition}[theorem]{Proposition}
\newtheorem{conjecture}[theorem]{Conjecture}
\newtheorem{corollary}{Corollary}[theorem]

\usepackage{algorithmic}
\usepackage[ruled,linesnumbered,noresetcount,vlined]{algorithm2e}

\newcommand{\bo}[1]{\boldsymbol{\operatorname{{#1}}}}
\newcommand{\proj}{\operatorname{{proj}}}
\newcommand{\cmark}{\ding{51}}
\newcommand{\xmark}{\ding{55}}

\title{\titlesize Dual Conic Proxy for Semidefinite Relaxation of\\ AC Optimal Power Flow}

\author{Guancheng Qiu, Mathieu Tanneau, Pascal Van Hentenryck \\ 
Georgia Institute of Technology, Atlanta, USA \\
}
\authorlist{Qiu, Tanneau and Van Hentenryck}

\begin{document}

\pagestyle{headings}

\maketitle

\begin{abstract}
{\small
The nonlinear, non-convex AC Optimal Power Flow (AC-OPF) problem is fundamental for power systems operations.
The intrinsic complexity of AC-OPF has fueled a growing interest in the development of optimization proxies for the problem, i.e., machine learning models that predict high-quality, close-to-optimal solutions.
More recently, dual conic proxy architectures have been proposed, which combine machine learning and convex relaxations of AC-OPF, to provide valid certificates of optimality using learning-based methods.
Building on this methodology, this paper proposes, for the first time, a dual conic proxy architecture for the semidefinite (SDP) relaxation of AC-OPF problems.
Although the SDP relaxation is stronger than the second-order cone relaxation considered in previous work, its practical use has been hindered by its computational cost.
The proposed method combines a neural network with a differentiable dual completion strategy that leverages the structure of the dual SDP problem.
This approach guarantees dual feasibility, and therefore valid dual bounds, while providing orders of magnitude of speedups compared to interior-point algorithms.
The paper also leverages self-supervised learning, which alleviates the need for time-consuming data generation and allows to train the proposed models efficiently.
Numerical experiments are presented on several power grid benchmarks with up to 500 buses.
The results demonstrate that the proposed SDP-based proxies can outperform weaker conic relaxations, while providing several orders of magnitude speedups compared to a state-of-the-art interior-point SDP solver.
}
\end{abstract}

\section*{Keywords}
AC optimal power flow, convex relaxation, semidefinite programming, neural network, self-supervised learning

\section{Problem Studied}\label{sec:def}
We first present Fixed-Radius Near Neighbor (FRNN) queries and then formalize Aggregation Queries over Nearest Neighbors (AQNNs) that build on them. We then state our problem.

\subsection{Nearest Neighbor Queries}\label{subsec:FRNN}
We build on generalized Fixed-Radius Near Neighbor (FRNN) queries \cite{FRNNSurvey}. Given a dataset \( D \), a query object \( q \), a radius \( r \), and a distance function \( dist \), a generalized FRNN query retrieves all nearest neighbors of \( q \) within radius \( r \). More formally:
\[
NN_D(q, r) = \{x \in D \mid dist(x, q) \leq r\},
\]
where \(x\) is any data point in \(D\) and \(dist(x, q)\) denotes the distance between them. We use \(|NN_D(q,r)|\) to denote the neighborhood size of \(q\). As shown in Fig. \ref{fig:framework}, given a radius \(r\) and a target patient \(q\), patients in the dotted circle are nearest neighbors, and the neighborhood size is 6.

\subsection{Aggregation Queries over Nearest Neighbors}\label{subsec:AQNN} 
Given an FRNN query object \(q\) in dataset \(D\), a radius \(r\), and an attribute \(\texttt{attr}\), an Aggregation Query over Nearest Neighbors (AQNN) is defined as:
\[ \text{agg}(NN_D(q,r)[\texttt{attr}]) \]
where agg is an aggregation function, such as $\mathtt{AVG}$, $\mathtt{SUM}$, and $\mathtt{PCT}$, and \(NN_D(q,r)[\texttt{attr}]\) denotes the bag of values of attribute \texttt{attr} of all FRNN results of \(q\) within radius \(r\). 
% \end{definition}

An AQNN expresses aggregation operations to capture key insights about the neighborhood of a query object. For example, \(\mathtt{AVG}\) can be used to reflect the average heart rate or systolic blood pressure of patients in the neighborhood, providing a measure of typical health conditions. \(\mathtt{SUM}\) is useful for assessing cumulative effects, such as the total cost of treatments in the neighborhood that instructs public policy in terms of health. Similarly, $\mathtt{PCT}$ can be used to find the proportion of patients in the neighborhood of a patient of interest, relative to the population in the dataset.
%\laks{Why is finding the total \#meds to NNs or the total treatment cost of everyone in the NN interesting?}

% \texttt{MIN} and \texttt{MAX} are not included in the aggregation functions because they only capture extreme values, which may not represent the typical characteristics of the nearest neighbors and are more sensitive to outliers. 
% \laks{AVG is also sensitive to outliers, but we still allow it. isn't the real reason we don't consider MIN/MAX because they are amenable to estimation via sampling?} We choose \texttt{PCT} instead of \texttt{COUNT} in order to provide a normalized measure that remains comparable across different neighborhood sizes. It allows for more consistent interpretation of relative popularity \cite{moore1989introduction}.


Fig. \ref{fig:framework} illustrates an example of an AQNN: ``\textit{Find the average systolic blood pressure of patients similar to an insomnia patient \(q\)}''. The aggregation function is \(\mathtt{AVG}\) and the target attribute of interest is systolic blood pressure. Exact query evaluation requires consulting physicians (or predicting embeddings by an expensive machine learning model) for all 500 patients in \(D\) and calculate \(q\)'s nearest neighbors wrt \(r\) \cite{DBLP:journals/isci/RodriguesGSBA21}. We refer to such highly accurate but computationally expensive models as \textit{oracle models}, denoted as \(O\), including deep learning models trained on domain-specific data or human expert annotations \cite{DBLP:conf/sigmod/LuCKC18}. Using oracle models is very expensive \cite{sze2017efficient, DujianPQA, DBLP:journals/pvldb/KangGBHZ20}. To address that, we seek an approximate solution by \textit{proxy models}, denoted as \(P\), that are at least one order of magnitude cheaper than oracle models. In the example, if consulting physicians for one patient incurs one cost unit, calling a cheap machine learning model instead incurs at most \(0.1\) cost unit. Once the similar patients are identified, their systolic blood pressure values are averaged and returned as  output. The use of a proxy model may reduce the accuracy of the neighborhood prediction and hence, we should judiciously call oracle and proxy models to minimize the error of aggregate results.

Note that the values of the target attribute \texttt{attr} are \textit{not} predicted but are instead known quantities.

\subsection{Problem Statement}
Given an AQNN, our goal is to return an approximate aggregate result by leveraging both oracle and proxy models while reducing error and cost.



\documentclass[../main.tex]{subfiles}
\graphicspath{{../images/}}
\makeatletter
\def\input@path{{../images/}}
\makeatother
\begin{document}
\section{Introduction}
\begin{figure}
\centering
\begin{tikzpicture}
\node[inner sep=0pt] (ws) at (0, 0) {
\includegraphics[height=.4\textwidth, trim={10cm 0 10cm 0},clip]{world_space.png}};
\node[inner sep=0pt] (cs) at (6,0) {\includegraphics[height=.4\textwidth, trim={10cm 1cm 10cm 4cm},clip]{conf_space.png}};
\end{tikzpicture}
\vspace{-5pt}
\label{fig:pbrm_intro}
\caption{\textbf{Left}: Shows world space obstacles as grey spheres. Robots start and goal configuration is colored red and green, respectively. Configurations along the computed path are colored transparent blue. \textbf{Right:} Mapped world space scenario to configuration space. Obstacle region is the grey mesh. Red spheres are collision-free regions computed by the neural SCDF. The optimized shortest path in the convex corridor is the blue curve.}
\vspace{-25pt}
\end{figure}
Motion planning is the problem of finding a collision-free trajectory that connects a given start and goal configuration. The planning takes place in the configuration space of the robot. For single body robots, like mobile robots or drones, the configuration space and the world space are usually the same. This simplifies the planning, since explicit obstacle representations are available which enables geometrical tools like separating hyperplanes, smallest distance to obstacles etc., to be used when designing motion planning algorithms. For multi-body robots like manipulators, the situation is completely different. The world space obstacles are usually mapped to non-convex regions, and to make the problem even harder, the mapping is usually not known. Forming explicit representations of the obstacle region in the configuration space is usually too expensive or intractable. Despite all of this, sampling based planners are used with great success, which mainly is due to their use of implicit representations of the obstacle region. The basic idea is to construct a graph in the configuration space that covers and connects the collision-free region. From this graph, a path can be extracted that connects a given start and goal configuration. The approach is computationally expensive, since the graph is constructed with the smallest geometrical building block available, points, which represents a collision-check. Furthermore, the extracted paths from the graph are non-smooth and jagged due to the stochastic nature of the approach. This adds an additional post-processing step to the process, where the paths are shortcutted and smoothened, before the path can be used for tracking. Clearly a lot of time is invested to form this graph and produce smooth paths. Thus, if the obstacles start to move, then all of this work is done in no use, since all points that make up this graph need to be re-verified, which is simply too time consuming to be done in real time.
\\\\
In this work, we want to address the existing drawbacks of the sampling based planners. Our main contribution is an improved motion planner where each vertex in the graph covers a collision-free region in the form of a sphere instead of a point and where the edges are formed with neighboring intersecting spheres. This representation has the advantage of instead of returning piecewise linear paths, returning a sequence of overlapping spheres, i.e. a convex corridor, that connects a given start and goal configuration, illustrated in Figure \ref{fig:pbrm_intro}. This convex corridor allows us to use convex optimization to produce smooth trajectories, instead of computationally expensive post-processing methods. The representation further allows us to estimate the coverage of the collision-free space, which gives us awareness and feedback in the offline roadmap construction phase. Finally, our representation is simple to adapt to moving obstacles, simply requery for the new radii and recheck for intersections. 
\\\\
The spherical collision-free regions are formed using a signed distance function (SDF), which is a function that returns the smallest distance from an arbitrary point to the boundary of an obstacle. As the name implies, the distance is signed, thus if the point is inside the obstacle it is negative otherwise positive. If the distance is positive, a sphere with radius equal to the distance is guaranteed to cover a collision-free region. Using an SDF in motion planning is not new, but what is novel about our approach is that we express the distance in the configuration space instead of the world space and by doing so allows us to form these convex collision-free regions. We refer to the resulting SDF as a signed configuration distance function (SCDF). Computing an SCDF analytically is non-trivial, our approach is therefore to parameterize the SCDF with a deep neural network and learn the mapping by supervised learning. Our resulting neural SCDF can compute distances for different parameter values of obstacle shapes and we also show how multiple distances can be combined, thus making our approach flexible.
\section{Related work}
Motion planning algorithms can roughly be divided into three families, grid-based, sampling based and optimization based methods. Grid-based methods (GBM) discretize the planning space from which a graph is then compiled. A standard search method is A$^\star$ \citep{a_star}, which is classified as an \textit{informed} search method, since it employs a heuristic function to speed up the search. A$^\star$ guarantees to return an optimal path at the level of discretization used. GBMs usually discretize the planning space by a regular lattice and this limits the GBMs to problems with low dimensionality due to the curse of dimensionality. Thus, GBMs are usually limited to single-body robots where the degrees of freedom (DOF) are low. To overcome the inherent scaling problem with the GBMs, stochastic methods are usually used for multi-body robots. These methods are termed as sampling-based methods (SBM) and core members within this family are the rapidly-exploring random trees (RRT) \citep{rrt} and the probabilistic roadmap (PRM) \citep{prm}. RRT grows a tree from the start configuration and explores the collision-free region in a rapid way until it is able to connect to the goal region. RRT is usually improved by bi-directional planning \citep{rrt_connect}, i.e. an additional tree is grown from the goal configuration and the trees are tested for connection after any tree has been expanded. RRT is a single-query method, thus it searches for a path from scratch each time it is queried. Contrary to this, PRM is a multi-query method, which solves for multiple queries without starting from scratch. PRM does this by creating a roadmap (graph) that covers the collision-free space as an offline step. The graph is then used to solve for multiple queries. PRMs are used in cases where the environment does not change since the extra offline step is too computationally costly and needs to be re-done if the environment is changed. In our work, we address this inherent issue by using a different roadmap representation. Our vertices in the graph cover a collision-free region in the form of spheres and we form the edges by checking for intersecting spheres. If something in the environment changes, we recompute the spheres radii and recheck the intersections, without relying on collision detection. We use a trained neural network to compute the sphere radius, therefore querying for the radius can be done fast, hence our representation enables the PRM for dynamic environments.
\\\\
In the recent decades, optimization based methods (OBM) \citep{chomp, schulman, itomp, stomp} have been introduced as an alternative to SBM for multi-body robots. Like the SBM, the OBMs scale well to higher dimensional problems and produce smoother motion. It is common to use a SDF in the optimization since it is a smooth function, thus enabling gradient-based methods. However, the standard way of expressing the SDF is in world space. The distance therefore needs to be mapped to the configuration space by the forward kinematics. This mapping makes the optimization problem a non-linear program (NLP), which is computationally expensive to solve. Recently, a different approach has been proposed. In \cite{mp_gcs} motion planning is formulated as a convex optimization problem by using the graph of convex sets framework \citep{gcs}. The underlying idea is to decompose the collision-free space into intersecting convex sets from which a convex optimization problem is formulated. In cases where an explicit representation of the obstacles in the configuration space exists, like for single-body robots, creating collision-free convex regions can be done fast \citep{iris}. For multi-body robots, this is non-trivial. Existing work does this successfully \citep{iris_nlp, iris_c} by an optimization based approach, but the methods are still too time consuming to be used in the presence of moving obstacles. Our approach is instead to use deep learning to learn an SDF expressed in the configuration space. With this, we can query for shortest distances to the collision boundary, which allows us to expand spherical regions which are collision-free. Our approach is fast and therefore enables our suggested roadmap planner to be used in dynamic environments.
\\\\
Recent research has focused on learning collision detection \citep{fk_kernel_distance, diffco, graphdistnet} by predicting the signed distance between the robot links and the surrounding obstacles in the world space. The learned SDF is used in trajectory optimization but since the distance is expressed in the world space, the problem becomes an NLP and therefore takes a long time to solve. We take a novel approach and suggest to instead express the signed distance in the configuration space. This allows us to improve the PRM at the same time as it enables convex optimization for trajectory optimization, which runs faster and is more reliable than NLP solvers. In \cite{cspf} a learned signed distance function in the configuration space is proposed similar to our approach. However, their approach is restricted to point cloud representations, while we propose to represent the obstacles as parameterized geometric shapes, e.g. spheres. Furthermore, we also show how to use our learned SCDF to improve an existing roadmap planner.
\section{Problem formulation}
A robot is located in the world space, $\W \subset \R^3 $. The unique location of the robot is given by its configuration $\q \in \C$, where $\C$ is the configuration space. The set of points covered by the robots bodies at a certain configuration is expressed as $\B(\q) \subset \W$. The robot is surrounded by $\NrObst$ obstacles $\O = \bigcup_{i=1}^{\NrObst} \O_i$, where  $\O_i \subset \W$. The representation of the obstacle in the configuration space is the set $\C\O_i = \{\q \in \C \: |\: \B(\q) \cap \O_i \neq \emptyset \}$. The obstacle space is formed as $\Co = \bigcup_{i=1}^{\NrObst} \C \O_i$. The complement is referred to as the free space, $\Cf = \C \setminus \Co$. The path planning problem is a tuple, ($\Cf$, $\qStart$, $\qGoal$), where we want to connect a query pair, consisting of a start, $\qStart$, and goal configuration, $\qGoal$, with a geometric path, $\q(s): [0, 1] \mapsto \Cf$, such that $\q(0)=\qStart$ and $\q(1)=\qGoal$, or report correctly when such a path does not exist.
\end{document}

\section{AC-OPF and a Semidefinite Relaxation}
\label{sec:formulation}

The imaginary unit is denoted by $\im$, i.e., $\im^{2} \, {=} \, -1$.
The complex conjugate of $z \, {\in} \, \mathbb{C}$ is $z^{\star}$.
$\Re(\cdot)$ and $\Im(\cdot)$ denotes the real and imaginary parts of a complex number.
In all that follows, let $\E_{ij}$ denote a square matrix of appropriate dimension, whose $(i, j)$ entry is equal to $1$, and all other zeros.
The identity matrix is denoted by $I$.
The symmetric and skew-symmetric parts of square matrix $\mathbf{A}$ are denoted by $\mathbf{A}^{+} = (\mathbf{A} + \mathbf{A}^{\top})/2$ and $\mathbf{A}^{-} = (\mathbf{A} - \mathbf{A}^{\top})/2$.
Note that $\mathbf{A} = \mathbf{A}^{+} + \mathbf{A}^{-}$.
The smallest eigenvalue of a Hermitian matrix $\mathbf{X}$ is denoted by $\lambda_{min}(\mathbf{X})$.
The cones of positive semidefinite Hermitian (resp. symmetric) matrices of order $n$ is denoted by $\mathbb{H}_{+}^{n}$ (resp. $\mathbb{S}^{n}_{+}$),
and the second-order cone of order $n$ is denoted by $\mathcal{Q}^{n} = \{x \in \mathbb{R}^{n} \, | \, x_{1} \geq \sqrt{x_{2}^{2} + ... + x_{n}^{2}} \}$.

The paper considers a power grid, represented as a simple directed graph, whose sets of buses and branches are denoted by $\mathcal{N}$ and $\mathcal{E}$, respectively.
Each branch is represented as a directed edge $(i, j)$ from bus $i$ to $j$, with admittance matrix
\begin{align}
    Y_{ij} =
    \begin{pmatrix}
        \yff_{ij} & \yft_{ij}\\
        \ytf_{ij} & \ytt_{ij}
    \end{pmatrix}
    = 
    \begin{pmatrix}
        \gff_{ij} + \im \bff_{ij} & \gft_{ij} + \im \bft_{ij}\\
        \gtf_{ij} + \im \btf_{ij} & \gtt_{ij} + \im \btt_{ij}
    \end{pmatrix}
    \in 
    \mathbb{C}^{2 \times 2}.
\end{align}
The shunt admittance and power demand at node $i$ are denoted by $Y^{s}_{i} = g^{s}_{i} + \im b^{s}_{i} \in \mathbb{C}$ and $\Sd_{i} = \pd_{i} + \im \qd_{i} \in \mathbb{C}$, respectively.
For ease of reading and without loss of generality, we formulate the problems assuming that exactly one generator is connected to each bus, and that generation costs are linear.



\subsection{The AC-Optimal Power Flow Formulation}

The formulation of AC-OPF considered in this paper is detailed in Model~\ref{model:AC-OPF}.
The decision variables comprise complex generation $\Sg = \pg + \im \qg$,
complex nodal voltage $\V = \vm \angle \va$,
and forward and reverse power flows $\Sf = \pf + \im \qf$ and $\St = \pt + \im \qt$.
The objective \eqref{eq:ACOPF:objective} minimizes total generation costs.
Constraint \eqref{eq:ACOPF:kirchhoff} enforces Kirchhoff current law at each bus.
Constraints \eqref{eq:ACOPF:ohm_fr}--\eqref{eq:ACOPF:ohm_to} express power flows on each branch using Ohm's law,
and constraints \eqref{eq:ACOPF:thermal_limits} enforce thermal limits on forward and reverse power flows.
Finally, constraints \eqref{eq:ACOPF:voltage_bounds} and \eqref{eq:ACOPF:reactive_dispatch_bounds} enforce minimum and maximum limits on nodal voltage magnitude and power generation.

\begin{figure}[!t]
    \centering
    \begin{minipage}[t]{0.47\textwidth}
        \begin{model}[H]
            \caption{The AC-OPF model}
            \label{model:AC-OPF}
            \begin{subequations}
            \footnotesize
            \label{eq:ACOPF}
            \begin{align}
                \min \quad 
                & \sum_{i \in \mathcal{N}} c_{i} \pg_{i}
                    \label{eq:ACOPF:objective}
                    \\
                \textrm{s.t.} \quad
                & \Sg_{i} - \Sd_{i} - Y_{i}^{s^{*}} |\V_{i}|^{2} = \sum_{(i,j) \in \mathcal{E}} \Sf_{ij} + \sum_{(j,i) \in \mathcal{E}} \St_{ji}
                    && \forall i \in \mathcal{N}
                    \label{eq:ACOPF:kirchhoff}
                    \\
                & \Sf_{ij} = {\yff_{ij}}^{*} |\V_{i}|^{2} - {\yft_{ij}}^{*} \V_{i} \V_{j}^{*} 
                    && \forall ij \in \mathcal{E}
                    \label{eq:ACOPF:ohm_fr}
                    \\
                & \St_{ij} = {\ytt_{ij}}^{*} |\V_{j}|^{2} - {\ytf_{ij}}^{*} \V_{i}^{*} \V_{j}
                    && \forall ij \in \mathcal{E}
                    \label{eq:ACOPF:ohm_to}
                    \\
                & |\Sf_{ij}|, |\St_{ij}| \leq \bar{s}_{ij}
                    && \forall ij \in \mathcal{E}
                    \label{eq:ACOPF:thermal_limits}
                    \\
                & \vmmin_{i} \leq |\V_{i}| \leq \vmmax_{i} 
                    && \forall i \in \mathcal{N}
                    \label{eq:ACOPF:voltage_bounds}
                    \\
                & \Sgmin_{i} \leq \Sg_{i} \leq \Sgmax_{i}
                    && \forall i \in \mathcal{N}
                    \label{eq:ACOPF:reactive_dispatch_bounds}
            \end{align}
            \end{subequations}
            \vspace{0.5em}
        \end{model}
    \end{minipage}
    \hfill
    \begin{minipage}[t]{0.45\textwidth}
        \begin{model}[H]
            \caption{The SDP-OPF model}
            \label{model:SDP-OPF}
            \begin{subequations}
            \footnotesize
            \begin{align}
                \min \quad 
                & \sum_{i \in \mathcal{G}} c_{i} \pg_{i}
                    \label{eq:SDPOPFC:obj}
                    \\
                \textrm{s.t.} \quad
                & \Sg_{i} - \Sd_{i} - Y_{i}^{s^{*}} \Wsdp_{ii} = \sum_{(i,j) \in \mathcal{E}} \Sf_{ij} + \sum_{(j,i) \in \mathcal{E}} \St_{ji}
                    && \forall i \in \mathcal{N}
                    \label{eq:SDPOPFC:kirchhoff}
                    \\
                & \Sf_{ij} = \inner{\yff_{ij} \E_{ii} + \yft_{ij} \E_{ij}}{\Wsdp}
                    && \forall ij \in \mathcal{E}
                    \\
                & \St_{ij} = \inner{\ytt_{ij} \E_{jj} + \ytf_{ij} \E_{ij}}{\Wsdp}
                    && \forall ij \in \mathcal{E}
                    \\
                & 
                    (\bar{s}_{ij}, \pf_{ij}, \qf_{ij}), (\bar{s}_{ij}, \pt_{ij}, \qt_{ij}) \in \mathcal{Q}^{3}
                    && \forall ij \in \mathcal{E}
                    \\
                & \vmmin_{i}^{2} \leq \Wsdp_{ii} \leq \vmmax_{i}^{2} 
                    && \forall i \in \mathcal{N} \\
                & \Sgmin_{i} \leq \Sg_{i} \leq \Sgmax_{i}
                    && \forall i \in \mathcal{N}
                    \label{eq:SDPOPFC:qg:bounds}
                    \\
                & \Wsdp \in \mathbb{H}^{|\mathcal{N}|}_{+}
                    \label{eq:SDPOPFC:psd}
            \end{align}
            \end{subequations}
        \end{model}
    \end{minipage}
\end{figure}



\subsection{A Semidefinite Relaxation of AC-OPF}

Introducing the change of variable $\Wsdp = \V \V^{\star}$ yields an equivalent formulation of AC-OPF if $\Wsdp \succeq 0$ and $\text{rank}(\Wsdp) = 1$.
Relaxing the rank constraint on $\Wsdp$ yields a semidefinite relaxation SDP-OPF, originally proposed in \cite{Bai2008_SDPRelaxationOPF}, which is stated in Model \ref{model:SDP-OPF}.
Because SDP-OPF is a relaxation of AC-OPF, its optimal value is a valid lower bound on the optimal value of AC-OPF.

Model \ref{model:DSDP-OPF} presents the conic dual of SDP-OPF, where $\mathcal{A}_{R}(\lambda, \mu, \nu)$ and $\mathcal{A}_{I}(\lambda, \mu, \nu)$ are defined as
\begin{align}
    \label{eq:AR_def}
    \mathcal{A}_{R}(\lambda, \mu, \nu) = 
        & \sum_{i \in \mathcal{N}} 
            (- g_{i}^{s} \lambdaP_{i} + b_{i}^{s} \lambdaQ_{i}) \E_{ii} 
            + (\muWmMin_{i} - \muWmMax_{i}) \E_{ii}\\
        &  + \sum_{(i, j) \in \mathcal{E}} \left(
                  \lambdaPf_{ij} \big( \gff_{ij} \E_{ii} + \gft_{ij} \E^{+}_{ij} \big)
                + \lambdaPt_{ij} \big( \gtt_{ij} \E_{jj} + \gtf_{ij} \E^{+}_{ij} \big)
                - \lambdaQf_{ij} \big( \bff_{ij} \E_{ii} + \bft_{ij} \E^{+}_{ij} \big)
                - \lambdaQt_{ij} \big( \btt_{ij} \E_{jj} + \btf_{ij} \E^{+}_{ij} \big)
                \right) \nonumber\\
    % & &+ & (-\muAngleDiffMin_{ij} \tan(\dvamin) + \muAngleDiffMax_{ij} \tan(\dvamax)) \sympart(\E_{ij}) \\
    \mathcal{A}_{I}(\lambda, \mu, \nu) = 
        & \sum_{(i, j) \in \mathcal{E}} \left(
            \lambdaPf_{ij} \bft_{ij} \E^{-}_{ij}
            - \lambdaPt_{ij} \btf_{ij} \E^{-}_{ij}
            + \lambdaQf_{ij} \gft_{ij} \E^{-}_{ij}
            - \lambdaQt_{ij} \gtf_{ij} \E^{-}_{ij}
            % + (\muAngleDiffMin_{ij} - \muAngleDiffMax_{ij}) \E^{-}_{ij}
        \right).
\end{align}
By weak duality, any dual solution that satisfies \eqref{eq:DSDPOPF:pg}-\eqref{eq:DSDPOPF:psd} yields a valid lower bound on the optimal value of SDP-OPF and, in turn, the optimal value of AC-OPF.

\begin{model}[!t]
\caption{The DSDP-OPF model}
\label{model:DSDP-OPF}
\begin{subequations}
    \small
    \begin{align}
        \max_{\lambda, \mu, \nu} \quad 
        & \sum_{i \in \mathcal{N}} \left(
            \pd_{i} \lambdaP_{i}
            + \qd_{i} \lambdaQ_{i}
            + \vmmin_{i}^{2} \muWmMin_{i} - \vmmax_{i}^{2} \muWmMax_{i}
            + \pgmin_{i} \muPgMin_{i} - \pgmax_{i} \muPgMax_{i}
            + \qgmin_{i} \muQgMin_{i} - \qgmax_{i} \muQgMax_{i}
        \right)
        - \sum_{e \in \mathcal{E}} \bar{s}_{e} \left( \nuThermalSfr_{e} + \nuThermalSto_{e} \right)
        \label{eq:DSDPOPF:obj}
        \\
        \text{s.t.} \quad
        & 
            \lambdaP_{i} + \muPgMin_{i} - \muPgMax_{i} = c_{i}
            && \forall i \in \mathcal{N}
            \label{eq:DSDPOPF:pg}\\
        & 
            \lambdaQ_{i} + \muQgMin_{i} - \muQgMax_{i} = 0
            && \forall i \in \mathcal{N}
            \label{eq:DSDPOPF:qg}\\
        & 
            -\lambdaP_{i} - \lambdaPf_{ij} + \nuThermalPfr_{ij} = 0
            && \forall ij \in \mathcal{E}
            \label{eq:DSDPOPF:pf}\\
        & 
            -\lambdaQ_{i} - \lambdaQf_{ij} + \nuThermalQfr_{ij} = 0
            && \forall ij \in \mathcal{E}
            \label{eq:DSDPOPF:qf}\\
        & 
            -\lambdaP_{j} - \lambdaPt_{ij }+ \nuThermalPto_{ij} = 0
            && \forall ij \in \mathcal{E}
            \label{eq:DSDPOPF:pt}\\
        & 
            -\lambdaQ_{j} - \lambdaQt_{ij} + \nuThermalQto_{ij} = 0
            && \forall ij \in \mathcal{E}
            \label{eq:DSDPOPF:qt}\\
        &
            \mathcal{A}_{R}(\lambda, \mu, \nu) + \im \mathcal{A}_{I}(\lambda, \mu, \nu) + \Ssdp = 0
            &&
            \label{eq:DSDPOPF:W}
            \\
        % Domain of dual variables
        & 
            \muPgMin, \muPgMax, \muQgMin, \muQgMax, \muWmMin, \muWmMax \geq 0
            && 
            \label{eq:DSDPOPF:non_negative} \\
        & 
            \nuThermalfr_{ij} = (\nuThermalSfr_{ij}, \nuThermalPfr_{ij}, \nuThermalQfr_{ij}) \in \mathcal{Q}^{3},
            \nuThermalto_{ij}  = (\nuThermalSto_{ij}, \nuThermalPto_{ij}, \nuThermalQto_{ij})\in \mathcal{Q}^{3}
            && \forall ij \in \mathcal{E}
            \label{eq:DSDPOPF:cone:nu} \\
        &
            \Ssdp \in \hermat^{|\mathcal{N}|}_{+}
            \label{eq:DSDPOPF:psd}
            && 
    \end{align}
\end{subequations}
\end{model}

%\vspace{-15pt}

\section{Separation Logic Predicate Synthesis via \tool}
\label{sec:SLsynthesis}

Having described the enhanced \emph{general-purpose} predicate
synthesis algorithm from positive-only examples,
%
we now show how to instantiate it for synthesis of inductive SL
predicates and improve the efficiency of the search algorithm by
exploiting domain-specific SL insights. We further discuss the
SL-validity of the synthesised predicates and the completeness of the
search algorithm.

\subsection{SL Predicates: Basics and Intricacies}
\label{sec:default}
 
\begin{figure}[!t]
  \centering
  \[
\begin{aligned}
  \sym{predicate} & ::= \sym{main\_pred} \;|\; \sym{main\_pred}  \sym{invented\_pred}\ast \\
  \sym{main\_pred} & ::= \sym{base\_clause}(\pre{main\_head}) \;|\; \sym{rec\_clause}(\pre{main\_head})\ast \\
  \sym{invented\_pred} & ::= \sym{base\_clause}(\pre{inv\_head}) \;|\; \sym{rec\_clause}(\pre{inv\_head})\ast \\
  \sym{base\_clause}(H) & ::= H(\codeinmath{This}, \sym{args}) \leftarrow \sym{base\_lit}\ast, \sym{pure\_lit}\ast \\
  \sym{rec\_clause}(H) & ::= H(\codeinmath{This}, \sym{args}) \leftarrow \sym{pointer\_lit}\ast, \sym{rec\_lit}\ast, \sym{pure\_lit}\ast \\
  \sym{literal}(R) & ::= R(\sym{args}) \\
  % Define the specific types of literals
  \sym{base\_lit} & ::= \sym{literal}(\pre{base\_pred}) \qquad\qquad \texttt{\% Pre-defined  for spatial relations} \\
  \sym{pure\_lit} & ::= \sym{literal}(\pre{pure\_pred}) \qquad \qquad\enspace \quad \texttt{\% Pre-defined  for pure relations} \\
  \sym{pointer\_lit} & ::= \pre{domain}(\codeinmath{This}, \sym{var}) \qquad\qquad\quad\enspace \texttt{\% Extract from the memory graphs} \\
  \sym{rec\_lit} & ::= \sym{literal}(\sym{head}) \\
  % General concepts
  \sym{args} & ::= \sym{var} \;|\; \sym{var}, \sym{args} \\
  \sym{var} & ::= \codeinmath{X1} \;|\; \codeinmath{X2} \;|\; \dots \;|\; \codeinmath{This} \\
  \sym{head} & ::= \pre{main\_head} \;|\; \pre{inv\_head} \quad \texttt{\% From the task or randomly generated}
\end{aligned}
\]
\caption{The grammar of the SL predicates, in basic Backus–Naur form
  (BNF), extended with (1) meta-variables $(\cdot)$ for specialising
  the symbols, and (2) pre-defined atoms denoted by $\pre{X}$ (with
  comments of their origins).}
  \label{fig:grammar}
\end{figure}

We define the space of SL predicates in a way standard for
Syntax-Guided Synthesis (SyGuS)~\cite{Alur-al:FMCAD13}.
%
The grammar of the SL predicates is shown in \autoref{fig:grammar}. An
SL predicate is either having a shape with a single main predicate, or
shaped by a main predicate together with a set of invented
\emph{auxiliary} predicates, which are required in the case of nested
linked structures.
%

Specific to the predicates,
both main predicate and invented predicates consist of the base and recursive clauses, where the base clause is the one that does not have any recursive calls, and the recursive clause is the one that has recursive calls. The head literal (\ie, before $\leftarrow$) in each clause has a fixed argument \pcode{This} that denotes the base address of the data structure (similar to the \textit{this} reference in object-oriented programming).
% 
The body literals (\ie, after $\leftarrow$) in the clauses are defined in terms of different predicates: the base (and pure) predicates are pre-defined, but extensible, to capture the spatial relation among the pointer for the base clause (the pure constraints among variables in clauses, respectively); the domain predicates describe the points-to relations can be obtained from the memory graphs; the recursive predicates are the recursive calls to the main or invented predicates.

% To define a tractable search procedure


Three aspects in the grammar in \autoref{fig:grammar} contribute to the
infinite synthesis search space: (1) the length of clauses, (2) the
number of sub-clauses for each predicate, (3) the arity of the
invented predicates. 
%
% As customary in SyGuS, we bound them with constants.
%
For our task, we noticed that predicates for real-world structures
rarely require more than 10 literals in their bodies; two sub-clauses
for each predicate are sufficient to capture the common structures;
and the arity of the invented predicates is set to be not more than
the arity of the main predicates. Such bounds for hypothesis space are
common in almost all synthesis-by-example tools~(\eg,
\cite{cropper2021learning,lee2021combining,Si-al:FSE18}), not only to
make the synthesis tractable, but also to avoid
overfitting~\cite{PadhiMN019} (\eg, a predicate disjointing facts of
all examples).
%

Below, we discuss two challenges in make SL predicate synthesis
effective and efficient, together with how we address them in \tool.

% %
% The restriction of the search space is also a
% common solution to \emph{overfitting}, which is common in
% synthesis-by-example methods: there is always a
% predicate disjointing facts of all examples, but it is
% likely to be overfitted for specific examples. By providing a finite
% search space, such problem is eliminated.

% The outline  approaches here are
% presented in the context of our SL-specific setting, but are also
% applicable to other ILP tasks.


\subsubsection{Semantic-Based Pruning.}
\label{sec:semantics}

In most existing syntax-guided synthesisers \cite{cropper2021learning,Alur-al:FMCAD13,Si-al:FSE18}, the search is accelerated by pruning of the hypothesis search
space by employing the general \emph{syntax}
restrictions.
%
Other than limiting the syntax, we apply the following \emph{semantic}
properties' restriction of Separation Logic predicates to the search.
%
% Specifically, we encode the properties of SL predicates (\eg, \emph{minimum
%   reachability, pointer functionality}) with ASP so that many invalid
% outputs from \popper are eliminated. 
%
\begin{enumerate}
  \item \emph{Basic reachability}: no points-to relation appears in the
    body other than the ones from the \pcode{this} pointer. Thus, the clause \pcode{p(X, Y) :- next(X, Y), next(Y, Z), ...} is not 
    allowed as a candidate, because we expect all locations in the body to be
    accessible from \pcode{this} via fields.
  %
  \item \emph{Basic assumptions}: the base (non-recursive) clause
    restricts \pcode{this} pointer to either be \code{null} or to equal to
    another pointer parameter variable. \Eg, \pcode{p(X, Y) :-
      nullptr(X), ...} is allowed, but \pcode{p(X, Y) :-
      next(X, Y), ...} cannot be a base clause.
  %
  \item \emph{Restricted use of} \code{null}: if a variable \pcode{X} is
    a null-pointer (denoted by \pcode{nullptr(X)}), no
    more \pcode{X} occurs in the clause. \Eg, the clause \pcode{p(X, Y) :- nullptr(X), next(X, Y)}
    is not allowed.
  %
  \item \emph{Quasi-well-founded recursion of payload}: the pure argument passed to a
    recursive call should (non-strictly) decrease. \Eg, for a clause
    \pcode{p(X, S) :- next(X, Z), p(Z, S1), ...}, the set \pcode{S} should contains \pcode{S1}. This
    is a common assumption in recursive program synthesis \cite{albarghouthi2013recursive,lee2021combining}.
  %
  \item \emph{Heap functionality}: points-to relations of the same field
    should not target different locations. \Eg, a candidate clause cannot be \pcode{p(X, Y) :- next(X, Z), next(X, Z1), ...}.
  %
  \end{enumerate}

\noindent
%
This list of search constraints represents a combination of the
properties implied by SL semantics (in a Java-style field-based memory
model) as well as by common properties of data structures, which are
essential for the efficient search of SL predicates.
%
The exact encodings of these constraints in ASP are provided and explained in \autoref{app:slsemantics}.

\subsubsection{Free Variables and Auxiliary Placeholders.}
\label{sec:auxiliary}

Free variables are common in SL predicates, \eg, the (implicitly
existentially-quantified) location \pcode{Y} in the base clause of the
 doubly linked list below:
%
\begin{minted}[fontsize=\small]{prolog}
  dll(X, Y) :- nullptr(X).
  dll(X, Y) :- next(X, Z), prev(X, Y), dll(Z, X).
\end{minted}
%
Unfortunately, completeness guarantees of pruning discussed in \autoref{sec:popper2}  do not hold for
predicates with free variables in the sense that
 a complete (\ie, valid) hypothesis with free
variables might  be wrongfully pruned during the search~\cite[\S{4.5}]{cropper2021learning}.
%
To address this problem, we introduce \emph{auxiliary placeholders}
into the search as a way to express predicate clauses with free
variables.
%
For example, the following doubly linked list predicate can be
regarded the same as the one above with \pcode{anypointer()}
placeholder, and \emph{can} be synthesised.
%
\begin{minted}[fontsize=\small]{prolog}
  dll(X, Y) :- nullptr(X), anypointer(Y).
  dll(X, Y) :- next(X, Z), prev(X, Y), dll(Z, X).
\end{minted}
%
On a technical level, this requires adding an ASP constraint (shown in \autoref{app:auxiliary})
that forces the parameter of the placeholder predicate (\pcode{Y}
here) to appear \emph{twice} in the whole clause, so it could be later
translated into a single occurrence of a free variable.

% \subsubsection{Hypothesis Specificity.}
% \label{sec:specificity}

% Having applied the clause minimisation for redundancy elimination, the
% synthesiser is often left with the problem to choose the best
% hypothesis from the set of ``canonical'' candidates, none of which
% entail each other.
% %
% Our novel notion of specificity is aimed to provide an ordering that
% helps to make such a choice.
% %
% As an example, consider
% %
% the predicates \pcode{p()} and \pcode{q()} of the same size, defined
% as \pcode{p(A, B) :- succ(A, B)} and \pcode{q(A, B) :- less_than(A,
%   B)}.
% %
% But based on the meaning of the predicates, we should know
% that \pcode{succ(A, B)} is a stronger statement than
% \pcode{less_than(A, B)}, so \pcode{p(A, B)} is more specific than
% \pcode{q(A, B)}. With this to be considered, the specificity of a
% hypothesis is defined by the following (strict) partial order.


% \begin{definition}[Hypothesis Specificity]
% \label{def:spec}
% Given two hypotheses $A, B$ with the same arity and the same number of
% clauses, $A$ is \emph{more specific than} $B$ (denoted by $A \prec B$)
% \Iff either ($i$) $\mathit{size}(B) < \mathit{size}(A)$ (\ie, $A$ has
% strictly more literals), or ($ii$)
% $\mathit{size}(A)=\mathit{size}(B)$, and $\exists\mathit{l}_1$,
% $\mathit{l}_2$, s.t. $B(\mathit{l}_1/\mathit{l}_2) = A$, and
% $\mathit{l}_1 \models \mathit{l}_2$, and
% $\mathit{l}_2 \not\models \mathit{l}_1$, where
% $\mathit{l}_1/\mathit{l}_2$ denotes the replacement of the literal
% $\mathit{l}_2$ in a sub-clause of $B$ by $\mathit{l}_1$.
% \end{definition}

% \todo{}
% We conclude this section with the following formal proposition stating
% that our synthesis algorithm returns a \emph{locally-optimal} hypothesis 
%  in the search space with specificity as the
% metric.

% \begin{theorem}
%   \label{thm:specific}
%   The hypothesis returned by the positive-only learning in
%   \emph{\autoref{alg:popper}} is the most specific (i.e., the local
%   minimum of the specificity) predicate that is complete in the search
%   space defined by the algorithm's initial constraints
%   (\pcode{in_cons}) and the size limit (\pcode{max_size}) parameters.
% \end{theorem}
% \begin{proof}[Proof]
%   By induction on the size limit \pcode{max_size} of the predicate: when \pcode{max_size} is 0, there is no predicate hypotheses, so \pcode{None} is the most specific one. Then assume that the theorem holds for \pcode{max_size} $n$, \ie, \pcode{sol_i} is the most specific; we prove it for \pcode{max_size} $n+1$.

%   When \pcode{max_size} is $n+1$, based on the while loop in
%   \autoref{alg:popper}, the search space for $n+1$ is the search space
%   for $n$ plus when \pcode{size} is $n+1$. By the induction
%   hypothesis, \pcode{sol_i} is the most specific in the search space
%   for $n$, and the output \pcode{sol} is either \pcode{sol_i} or the
%   more specific one in $n+1$. Therefore, \pcode{sol} is the most
%   specific in the search space with $n+1$ as \pcode{max_size}.

% \end{proof}

\subsection{Ensuring SL Validity in \prolog}
\label{sec:sldomain}

An astute reader can notice that the validity of the synthesised
predicates is not immediate due to our treatment of \prolog clauses as
SL assertions: the conjunction in \prolog does not guarantee the
\emph{separating conjunction} (\pcode{*}) in the SL sense. As an
example, consider the following simplified \prolog predicate for
binary trees:
%
\begin{minted}[fontsize=\small]{prolog}
  bi_tree(X) :- nullptr(X). 
  bi_tree(X) :- t1(X, L), t2(X, R), bi_tree(L), bi_tree(R).
\end{minted}
%
In this case, an instance of \pcode{bi_tree(X)} being evaluated to be
\emph{true} in \prolog can imply \emph{false} under SL semantics that
enforces heap disjointness: considering a memory graph with two nodes
%
\begin{minted}[fontsize=\small]{prolog}
  t1(n1, n2). t2(n1, n2). t1(n2, null). t2(n2, null).
\end{minted}
%
so that the graph fact \pcode{bi_tree(n1)} is provable in \prolog, but
the clauses \pcode{bi_tree(L)} and \pcode{bi_tree(R)} are
\emph{non-disjoint}.
%
Notice that, in our inductive synthesis setting, this situation would
correspond to having \emph{multiple} occurrences of the same points-to
fact in a memory graph representing a positive example for the
predicate, but should not be allowed by the definition of separating
conjunction.

To avoid this source of unsoundness, we use a straightforward solution
that enforces such separating conjunction semantics in \prolog: a
valid SL predicate is a complete \prolog predicate where the positive
examples succeed using each points-to fact \emph{exactly} one time (a
semantic property of SL assertions known as \emph{linearity}).
%
For the complete \prolog but invalid SL predicates, we also use the
\textit{specialisation} rule in \autoref{sec:popper2} to prune them:
if a predicate violates the linearity, then a more constrained one
will also violate it; this contributes to the new pruning in
line~15 of \autoref{alg:popper}.

We establish the following property of our SL-specific predicate
synthesis stating that, for the predicates in \tool's search space in
\autoref{sec:default}, if a memory graph is provable in \prolog with
linearity, then the corresponding heap is valid under SL semantics.

\begin{theorem}[SL Validity]
\label{thm:validity}
Let \pcode{F(h)} denote the memory graph of a heap \pcode{h}. For any
output predicate \pcode{p(X)} of \tool and any heap \pcode{h}, the
following fact holds: 
%
  \pcode{F(h)} $\models_{\prolog+\text{Lin}}$
\pcode{p(X)} $\Rightarrow$ \pcode{h} $\models_{\text{SL}}$ \pcode{p(X)}.
% \begin{center}
% \end{center}
\end{theorem}




\subsection{The \tool Algorithm}
\label{sec:tool}

The only remaining step before putting all the pieces together is to
select the desired predicate from the set of non-comparable solutions
of positive-only learning. 
%
Even though predicates from POL can be conjuncted in general, the
semantics of SL predicates following the definition in
\autoref{sec:default} is more restrictive and the conjunction of valid
SL predicates may result in an ill-formed or a constantly false one. 
%
We found in practice that after the semantics-based normalisation from
\autoref{sec:normalise}, the number of literals can serve as a
\emph{good enough} specificity metric among incomparable predicates,
since containing more literals is likely to contain more information
or constraints about the heap structure. 
%
Following this intuition, we define the synthesis algorithm with
MAX\_POL function, which obtains the largest predicate from POL as per
\autoref{alg:popper}.

\begin{algorithm}[!t]
  \caption{The \tool loop for inductive predicate synthesis}
  \label{alg:sippy}
  \begin{algorithmic}[1]
  \small
  \Require memory graphs consist of \pcode{graph_bk, exs}
  \Procedure{Sippy}{\pcode{graph_bk, exs}}
      \State \pcode{graph_cons, shapes} = \pcode{graph_info(graph_bk, exs)}
      \State \pcode{max_var} = \pcode{max_body} = 1
      \State \pcode{sol} = \pcode{True} \Comment{The most general solution as initialisation}
      \For{\pcode{shape} in \pcode{shapes}}
        \State \pcode{max_size} = \pcode{maxsize(max_body, shape)}
        \State \pcode{h} = \Call{MAX\_POL}{\pcode{graph_bk, exs, graph_cons, max_size}}
        \If{\pcode{h} $\prec$ \pcode{sol}} \Comment{A more specific predicate is obtained}
            \State \pcode{max_var, max_body} = \pcode{(var_of(h), size_of(h))} + $\delta$
            \State \pcode{sol} = \pcode{h}
        \ElsIf{\pcode{sol} == True} \Comment{No predicate is yet learned}
            \If{\pcode{max_var} == \pcode{upper_bound}}
                \State \textbf{continue} \Comment{Try the next shape}
            \EndIf
            \State \pcode{max_var, max_body} += (1, 1)
        \Else
            \State \textbf{break} \Comment{No more specific predicate is found}
        \EndIf
      \EndFor
      \State \Return \pcode{sol}
  \EndProcedure
  \end{algorithmic}
\end{algorithm}



\autoref{alg:sippy} summarises the internal workings of \tool.
%
Our synthesiser takes as inputs memory graphs encoded as sets of logic
facts (\eg, \pcode{graph_bk}, such as \pcode{next(..)} and
\pcode{value(..)} from \autoref{fig:sorted-list}), positive examples of
heaps on which a predicate holds (\eg, \pcode{exs} as \pcode{srtl(..)}
from \autoref{fig:sorted-list}), so that the shape (matched with
pre-defined shapes in \autoref{sec:default}) a set of ASP constraints
(\pcode{graph_cons}) describing the information in the graphs (such as
the arity and types of the predicates to be learned) are obtained
(line~2).
%
Two parameters (line~3) for positive-only learning (MAX\_POL), (1)~the maximum number of
variables and (2)~the maximum size of the body of a predicate clause
for restricting the search space, are gradually increased during the
search using the following empirical strategy:
%
if no solution is valid (line~11), we either increase both parameters
by one to enlarge the space until finding one (line~14), or the
attempt on the current predicate shape fails (\ie, the upper bound of
the search space is reached), then
\tool will try synthesising using the next shape (line~13, \ie, more auxiliary predicates);
%
when obtaining one new better predicate than the existing, the search
parameters are both increased by a parameter~$\delta$ to find a
possibly more specific predicate (line~9), and the solution is
updated (line~10); if the learned predicate in the larger search space
is not better than the previous, we stop the search and output
(line~15-16).
%
The role of the parameter $\delta$ is, thus, to provide a ``margin''
for the completeness of the search: it is not guaranteed that \tool
will find the most specific solution \emph{across all possible search
  spaces}, but only in the search-space that is bound by the returned
output's parameters \emph{plus}~$\delta$.\footnote{We choose it to be (1,2) in our experiment from the natural observation: for our domain, we expect to have one body literal where the predicate is generating a new variable, and one more body literal that uses the new variable.}
%
Note that line~6 of \autoref{alg:sippy} features a function
\pcode{maxsize()} that calculates the maximum size of the search space based on the maximum number of variables and the predicate shape setting.

Finally, we provide a correctness argument for \tool. The soundness of
synthesising \emph{consistent} (\ie, inhabited) and \emph{well-formed}
(\ie, finitely provable) SL predicates is guaranteed by the soundness
of classic ILP and \autoref{thm:validity}. The following ``local''
completeness states that, given the output of \tool, no more specific
output can be discovered, \emph{even in} the larger search space
obtained by increasing the search parameters \emph{once} by $\delta$
at the line~9 of \autoref{alg:sippy}.

\begin{theorem}[Local Completeness of \tool]
\label{thm:completeness}
If the output of \tool is a predicate with the maximum number of
variables $m$ and the maximum length of the body $n$, then there is no
predicate with the maximum length of the body $m'$ and the maximum
number of variables $n'$, where $(m',~n')-(m,~n) = \delta$, that is
more specific than the output predicate.
\end{theorem}
\begin{proof}[Proof]
  Directly by contradiction and Theorem 3.1. Assume that the output solution \pcode{sol} is with size $(m,~n)$, and it is not the most specific one in size $(m',~n') = (m,~n) + \delta$.

 Because \pcode{sol} is the output, the search space is set to be $(m',~n')$ after the loop it is obtained. With Theorem 3.1 and the assumption, there is a solution \pcode{sol}$'$ in $(m',~n')$ that is more specific than \pcode{sol}, which is a contradiction with the output \pcode{sol}. Thus, \pcode{sol} is the most specific one in $(m',~n')$.
\end{proof}







% \subsection{Domain-Specific Pruning}
% \label{sec:dsc}

% So far we have shown how to encode the syntax of SL predicates
% (\autoref{sec:default}) as well as their basic semantic properties
% that guarantee validity of the solutions (\autoref{sec:sldomain}).
% %
% To enable even more aggressive yet sound search space pruning, we have
% encoded more SL properties as ASP search constraints:
% %
% \begin{enumerate}
% \item \emph{Basic Reachability}: no points-to relation appears in the
%   body other than the ones from the \pcode{this} pointer. For instance, the
%   predicate like \pcode{p(X, Y) :- next(X, Y), next(Y, Z), ...}, is not
%   allowed, because we expect all locations in the body to be
%   accessible from \pcode{this} via fields.
% %
% \item \emph{Basic Assumptions}: the base (non-recursive) clause
%   restricts \pcode{this} pointer to either be \code{null} or to equal to
%   another pointer parameter variable. For instance, \pcode{p(X, Y) :-
%     nullptr(X), ...} can be the base clause, but \pcode{p(X, Y) :-
%     next(X, Y), ...} cannot.
% %
% \item \emph{Restricted use of} \code{null}: if a variable \pcode{X} is
%   a null-pointer (denoted by \pcode{nullptr(X)}), no
%   more \pcode{X} should occur in the clause body. For example, \pcode{p(X, Y) :- nullptr(X), next(X, Y)}
%   is not allowed.
% %
% \item \emph{Quasi-well-founded recursion}: the pure argument passed to a
%   recursive call should (non-strictly) decrease. For instance,
%   \pcode{p(X, Y) :- next(X, Z), Y1 == Y+1, p(Z, Y1)} is not valid. This
%   is a common assumption in recursive program synthesis, which is also
%   suitable for our task.
% %
% \item \emph{Heap functionality}: points-to relations of the same field
%   should not target different locations. For instance, \pcode{p(X, Y)
%     :- next(X, Z), next(X, Z1), ...} is not allowed.
% %
% \end{enumerate}
% %
% This list of search constraints represents a combination of the
% properties implied by SL semantics (in a Java-style field-based memory
% model) as well as by common properties of shapes of data structures
% considered.
% %
% The rules above are merely optimisations: they are not necessary to
% ensure correctness of \tool and serve to restrict the search space to
% be a refined (but expressive) domain of SL predicate. 



\begin{table}[ht!]
\centering
\caption{\textbf{Super Resolution Performance Results.} Our proposed WGAN EEG Spatial Upsampling method significantly outperforms a baseline of Bicubic Interpolation commonly used in EEG upsampling pipelines.}
\label{tab:results}
\resizebox{0.8\linewidth}{!}{%
\begin{tabular}{@{}cccccc@{}}
\toprule
\multirow{2}{*}{\textbf{Dataset}} & \multirow{2}{*}{\textbf{Scale}} & \multicolumn{2}{c}{\textbf{Bicubic}} & \multicolumn{2}{c}{\textbf{WGAN}} \\ \cmidrule(l){3-6} 
                      &   & \textbf{MSE} & \textbf{MAE} & \textbf{MSE}    & \textbf{MAE}   \\
\toprule
\multirow{2}{*}{Val}  & 2 & 3.71E7       & 3.89E3       & \textbf{2.01E3} & \textbf{24.38} \\
                      & 4 & 7.23E7       & 6.42E3       & \textbf{8.53E3} & \textbf{63.83} \\
\midrule
\multirow{2}{*}{Test} & 2 & 3.75E7       & 3.91E3       & \textbf{2.06E3} & \textbf{24.66} \\
                      & 4 & 7.30E7       & 6.45E3       & \textbf{8.68E3} & \textbf{64.39} \\
\bottomrule
\end{tabular}%
}
\end{table}
\section*{Conclusion}
This paper aims to enhance our understanding of the computational complexity of computing various Shapley value variants. We found that for various ML models --- including decision trees, regression tree ensembles, weighted automata, and linear regression --- both local and global interventional and baseline SHAP can be computed in polynomial time under HMM modeled distributions. This extends popular algorithms, such as TreeSHAP, beyond their empirical distributional scope. We also establish strict complexity gaps between the various SHAP variants (baseline, interventional, and conditional) and prove the intractability of computing SHAP for tree ensembles and neural networks in simplified scenarios. Overall, we present SHAP as a versatile framework whose complexity depends on four key factors: \begin{inparaenum}[(i)] \item model type, \item SHAP variant, \item distribution modeling approach, \item and local vs. global explanations\end{inparaenum}. We believe this perspective provides deeper insight into the computational complexity of SHAP, paving the way for future work.




%We believe that our framework provides a more intricate understanding of SHAP computation complexity across different models, distributions, and variants, paving the way for further research.

Our work opens promising directions for future research. First, expanding our computational analysis to other SHAP-related metrics, such as asymmetric SHAP~\citep{frye20} and SAGE~\citep{covert2020understanding}, would be valuable. Additionally, we aim to explore more expressive distribution classes and relaxed assumptions beyond those in Section \ref{sec:tractable} while maintaining tractable SHAP computation. Finally, when exact computation is intractable (Section \ref{sec:intractable}), investigating the approximability of SHAP metrics through approximation and parameterized complexity theory~\citep{downey2012parameterized} is an important direction.

%Our work opens several promising avenues for future research on the computational properties of explainable AI methods, with a particular focus on SHAP. First, it would be interesting to broaden the computational analysis conducted in this work to include other popular SHAP-related metrics in the literature, such as asymmetric SHAP \cite{frye20} and SAGE \cite{covert2020understanding}. Also, in the future, we aim to explore more expressive distribution classes and relaxed distributional assumptions—extending beyond those examined in Section \ref{sec:tractable} —that still yield tractable SHAP computation. Finally, when exact computation proves intractable (Section \ref{sec:intractable}), it is worthwhile to theoretically investigate the question of the approximability of computing the SHAP metrics across various configurations, through the lens of approximation and parametrized complexity theory \cite{arora2009computational}.

%This paper aims to deepen our understanding of the computational complexity involved in obtaining different Shapley value variants. We found that for a variety of ML models, including decision trees, tree ensembles for regression, weighted automata, and linear regression models — computing both local and global interventional and baseline SHAP can be done in polynomial time when distributions are modeled by HMMs. This extends the distributional scope of popular algorithms like TreeSHAP, which is limited to empirical distributions. Additionally, we demonstrate a strict complexity gap between SHAP variants, showing that interventional and baseline SHAP can be strictly easier to compute than conditional SHAP. Despite these positive results, we uncovered intractability for various SHAP variants in neural networks and tree ensembles. Finally, we provided generalized complexity relations across SHAP variants. We believe that our framework offers a deeper understanding of the complexity involved in computing SHAP across various variants, models, distributions, as well as in both local and global computations, laying the groundwork for future research.

\section*{Acknowledgements}
This research is partly funded by NSF award 2112533 and ARPA-E PERFORM award AR0001136.

\bibliography{ref}

\end{document}
