\section{Introduction}
\label{sec:introduction}
The AC Optimal Power Flow (AC-OPF) problem aims to find the most cost-effective to dispatch power generation while satisfying demand and engineering constraints.
Despite its importance to power system operation, planning and electricity markets, its complex, nonlinear, and non-convex nature makes it challenging to solve and limits its practical use.

The need to repeatedly solve similar instances has led to the development of optimization proxies, i.e., machine learning (ML) models which can approximate the input-output mapping of AC-OPF solvers in milliseconds.
Most existing proxies for DC- and AC-OPF focus on predicting primal solutions without providing optimality guarantees \cite{fioretto2020predicting,donti2021dc3,huang2021deepopf,Chen2023_E2ELR}.
Hence, these approaches may provide sub-optimal solutions, which represents significant economic losses, and are therefore not sufficient for electricity market-clearing applications, which require near-optimal solutions \cite{MISO_BPM_002}.

To address this limitation, Dual Conic Proxies (DCP) were introduced in \cite{qiu2024dual}, which proposes to learn to generate valid lower bounds by predicting feasible dual solutions for a second-order cone (SOC) relaxation of AC-OPF proposed by Jabr \cite{Jabr2006_SOCRelaxationOPF}.
A more general Dual Lagrangian Learning framework was then proposed in   \cite{tanneau2024dual} for conic optimization problems, with results reported on linear and SOC problems.
While the approach of \cite{qiu2024dual} scales to large systems, it is intrinsically limited by the strength of the Jabr relaxation.
The reader is referred to \cite{Molzahn2019_OPF_survey} and to \cite{qiu2024dual,tanneau2024dual} for a detailed review of convex relaxations of AC-OPF, and of dual optimization proxies, respectively.

It is well-known that semidefinite (SDP) formulations provide strong relaxations for OPF problems \cite{kocuk2016strong,Molzahn2019_OPF_survey}, at the price of significant computational and numerical challenges.
As a first step towards achieving practical DCP methodology for SDP problems, this work develops, for the first time, a 
DCP methodology for an SDP relaxation of AC-OPF.
The paper presents a new dual-feasible architecture for this SDP relaxation, and conducts numerical results on small-to-medium power grids that demonstrate the benefits of the proposed approach.