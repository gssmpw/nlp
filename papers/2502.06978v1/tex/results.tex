\section{Numerical Experiments}
\label{sec:results}

\newcommand{\ieeeXXXS}{\texttt{ieee14}}
\newcommand{\ieeeXXS}{\texttt{ieee30}}
\newcommand{\pegaseXS}{\texttt{pegase89}}
\newcommand{\ieeeXS}{\texttt{ieee118}}
\newcommand{\ieeeS}{\texttt{ieee300}}



\subsection{Experiment Details}
\label{sec:results:data_generation}

The paper conducts experiments on several systems from the PGLib library v21.07 \cite{babaeinejadsarookolaee2019power} with up to 500 buses.
AC-OPF, SOC-OPF and SDP-OPF instances are formulated using PowerModels \cite{power_models}, and solved with Ipopt \cite{wachter2006implementation} and Mosek 10 \cite{mosek}, respectively.
Both solvers use tolerances of $10^{-6}$, with other parameters set to default.
For each system, 20,000 OPF instances are generated by perturbing reference $\Sd$ values using the method in \cite{anonymous2024pglearn}.
Each dataset is then split into training (90\%), validation (5\%) and testing (5\%).
The solutions of the solvers are only used for evaluating optimality gaps during testing, and are not used in training nor validation.

The proposed SDP proxy is evaluated against the SOC proxy proposed in \cite{qiu2024dual}, which is based on the SOC relaxation of AC-OPF.
Both proxies use the same training, validation and test data.
The DCP proxies are implemented in Python 3.10 using PyTorch 2.0 \cite{paszke2019pytorch} and trained with the Adam optimizer \cite{kingma2014adam}.
All experiments are conducted on Intel Xeon 2.7GHz CPU machines running Linux and equipped with Tesla RTX 6000 GPUs.



\subsection{Performance Evaluation}
The quality of the DCP dual solutions is evaluated in terms of optimality gap with respect to the objective values of Mosek solutions (to the dual problems) and those of Ipopt solutions (to the original AC-OPF).

The quality of a dual bound is evaluated through its dual optimality gap, defined as 
\begin{equation}
    \text{dual optimality gap} = \frac{z^{*}_{AC} - z}{z^{*}_{AC}} \geq 0,
\end{equation}
where $z$ is a valid dual bound, obtained from a dual-feasible solution of the SOC or SDP relaxation, and $z^{*}_{AC}$ is the objective value of the best-known AC-OPF solution, obtained by Ipopt.
The paper also reports the gap closed by the proposed SDP proxy compared to the SOC proxy of \cite{qiu2024dual}, defined as
\begin{align}
    \label{eq:gap_closed}
    \text{gap closed} = \frac{\hat{z}_{SDP} - \hat{z}_{SOC}}{z^{*}_{SDP} - \hat{z}_{SOC}},  
\end{align}
where $z^{*}_{SDP}$, $\hat{z}_{SDP}$ and $\hat{z}_{SOC}$ denote the dual bound obtained from an optimal dual SDP solution, the predicted dual SDP solution, and the predicted dual SOC solution, respectively.
A positive (resp. negative) gap closed indicates that the SDP proxy yields better (resp. worse) dual bounds that the SOC proxy; a 100\% gap closed indicates the SDP proxy produces dual-optimal solutions.
The paper uses the geometric mean $\mu(x_{1}, ..., x_{n}) \, {=} \sqrt[n]{x_{1} x_{2} ... x_{n}}$ to report mean dual gaps, and the arithmetic mean to report mean gap closed, since the latter may take negative values.



\subsection{Dual Conic Proxy Performance}
\label{sec:results:DCP}

\begin{table}[!t]
    \centering
    \caption{DCP Performance Results}
    \label{tab:results:gaps}
    \begin{tabular}{lrrrrr}
        \toprule
        & \multicolumn{4}{c}{Dual Optimality Gap w.r.t. $z^{*}_{AC}$}\\
        \cmidrule(lr){2-5}
        System
            & \multicolumn{1}{c}{$z^{*}_{SOC}$}
            & \multicolumn{1}{c}{$z^{*}_{SDP}$}
            & \multicolumn{1}{c}{$\hat{z}_{SOC}$}
            & \multicolumn{1}{c}{$\hat{z}_{SDP}$}
            & $^{\dagger}$Gap closed
            \\
\midrule
\multirow{1}{*}{\ieeeXXXS}
        & 0.101 (0.009)
        & 0.000 (0.001)
        & \textbf{0.153} (0.050)
        & 0.404 (0.062)
        & -170.232 (54.157)
        \\
\multirow{1}{*}{\texttt{ieee30}}
        & 21.062 (2.490)
        & 0.029 (0.030)
        & 24.338 (3.074)
        & \textbf{23.758} (2.656)
        & 1.456 (12.851)
        \\
\multirow{1}{*}{\ieeeXS}
        & 0.837 (0.683)
        & 0.037 (0.110)
        & 2.110 (1.008)
        & \textbf{0.713} (0.700)
        & 61.261 (21.094)
        \\
\multirow{1}{*}{\ieeeS}
        & 2.009 (0.663)
        & 0.373 (0.915)
        & \textbf{5.580} (1.516)
        & 5.693 (1.687)
        & -2.601 (\phantom{0}4.761)
        \\
\multirow{1}{*}{\texttt{goc500}}
        & 0.097 (0.024)
        & 0.000 (0.000)
        & 0.384 (0.564)
        & \textbf{0.229} (0.582)
        & 34.326 (25.455)
        \\
        \bottomrule
    \end{tabular} \\
    \footnotesize{All values are in \%. Mean and standard deviation (in parentheses) are presented. $^{\dagger}$Gap closed by dual SDP proxy ($\hat{z}_{SDP}$) compared to SOC proxy ($\hat{z}_{SOC}$) with respect to ground truth SDP bound ($z^{*}_{SDP}$), see Eq \eqref{eq:gap_closed}.}
\end{table}

Table \ref{tab:results:gaps} presents the performance of the SOC and SDP proxies.
The proposed SDP proxy yields smaller dual gaps than the SOC proxy on the 30-, 118- and 500-bus systems.
It is important to note that the performance of the SOC proxy is intrinsically limited by the quality of the SOC relaxation, which can be significantly worse than the SDP relaxation, e.g., on the {\ieeeXXS} system ($>20\%$ vs $0.029\%$ gap).
In particular, on the 118-bus system,
\emph{the SDP proxy outperforms the SOC relaxation} which, in turn, implies that no SOC proxy can match the SDP proxy on this test case.

Furthermore, the results illustrate the challenges of training high-quality SDP proxy, especially on the {\ieeeXXS} system.
This is attributed to the higher output dimension of the SDP proxy, and the additional difficulty of the PSD constraint \eqref{eq:DSDPOPF:psd}.
The numerical results of Table \ref{tab:results:gaps} thus confirm the benefits of using the SDP relaxation as the basis for building high-quality dual proxies for AC-OPF.

Finally, timing results, not reported for lack of space, further demonstrate that the SDP proxy achieves several orders of magnitude speedups compared to Mosek.