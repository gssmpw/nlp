\documentclass{article}
\usepackage{graphicx} % Required for inserting images

\title{Mathematical Models}
\author{Andrew Zheng}
\date{September 2023}

\usepackage{amssymb}
\usepackage{amsmath}
\usepackage{hyperref}
\usepackage{cleveref}

\numberwithin{equation}{section}
\usepackage{xcolor}

\usepackage{biblatex} %Imports biblatex package
\addbibresource{references.bib} %Import the bibliography file

\begin{document}



\subsection{Analytic formulas for singular integral}\label{appendix: formual for singular integral}

The integral we want to integrate is

\begin{align}
    I_s &= \sum_{i=1}^3 q(V_i) \int_\Delta \frac{\frac{|\y|}{R} (-y_x n_x - y_y n_y)}{4\pi(y_x^2+y_y^2)^{3/2}} l_i(\y) \, dA_{\y} \\
    &= \sum_{i=1}^3 q(V_i) \int_\Delta \frac{(-y_x n_x - y_y n_y)}{4\pi R|\y|^2} l_i(\y) \, dA_{\y}
\end{align}

Expanding our affine functions $l_i(\y)$, the integrals we need to compute are
\begin{align*}
I_7:=\int_\Delta \frac{y_x}{|\y|^2}~dA_{\y},~ I_8:=\int_\Delta \frac{y_y}{|\y|^2}~dA_{\y}, ~ I_9:=\int_\Delta \frac{y_x^2}{|\y|^2}~dA_{\y}, 
\\ I_{10}:=\int_\Delta \frac{y_y^2}{|\y|^2}~dA_{\y}, ~ I_{11}:=\int_\Delta \frac{y_xy_y}{|\y|^2}~dA_{\y}.
\end{align*}

The desired integral is now

\begin{align*}
I_s = -\frac{1}{4\pi R}\sum_{i=1}^3 \gamma(V_i) \big[ (n_x l_{i,0} ) I_7 + (n_y l_{i,0}) I_8 + (n_x l_{i,x}) I_9 + (n_y l_{i,y} ) I_{10} + (n_x l_{i,y} + n_y l_{i, x}) I_{11} \big].
\end{align*}

In polar coordinates, the integrals are of the form
\begin{equation*}
    \int_{r_\mathrm{start}}^{r_\mathrm{end}} \int_{\theta_\mathrm{start}(r)}^{\theta_\mathrm{end}(r)} r^{a+b-1}(\cos(\theta))^a(\sin(\theta))^b ~d\theta dr
\end{equation*}
for $(a, b) \in \{ (1, 0), (0, 1), (2, 0), (0, 2), (1, 1)\}$. The integrals in $\theta$ are the same as before. The integrals in $r$ are now 

\begin{enumerate}
\item $$\int \frac{1}{r} = \log(r), \quad r\ne 0$$

\item $$\int \frac{\sqrt{r^2 - d^2}}{r} = \begin{cases}
    \sqrt{r^2 - d^2} - d \arctan\left( \frac{\sqrt{r^2 - d^2}}{d}\right) & d \ne 0 \\
    r & d = 0
\end{cases}$$

\item $$\int r \, dr = \frac{1}{2} r^2$$

\item $$\int r \arccos\left(\frac{d}{r}\right) = \begin{cases}
    -\frac{d}{2}\sqrt{r^2 - d^2}  + \frac{r^2}{2} \arccos\left(\frac{d}{r}\right) & d \ne 0 \\
    \frac{\pi r^2}{4}& d = 0
\end{cases}$$
\end{enumerate}


For $I_7$, we have 
\begin{equation}
\begin{split}
    I_7  & = \int_{r_\mathrm{start}}^{r_\mathrm{end}} \int_{\phi_\mathrm{start} + \mathrm{sign}_\mathrm{start}\arccos\left(d_\mathrm{start}/r\right)}^{\phi_\mathrm{end} + \mathrm{sign}_\mathrm{end}\arccos\left(d_\mathrm{end}/r\right)} \cos(\theta)\, d\theta \, dr \\
    & = (d_\mathrm{end}\sin(\phi_\mathrm{end}) - d_\mathrm{start}\sin(\phi_\mathrm{start})) \int_{r_\mathrm{start}}^{r_\mathrm{end}} \frac{1}{r}\, dr \\
    & + \mathrm{sign}_\mathrm{end}\cos(\phi_\mathrm{end}) \int_{r_\mathrm{start}}^{r_\mathrm{end}} \frac{\sqrt{r^2 - d_\mathrm{end}^2}}{r}\, dr \\
    & - \mathrm{sign}_\mathrm{start}\cos(\phi_\mathrm{start}) \int_{r_\mathrm{start}}^{r_\mathrm{end}} \frac{\sqrt{r^2 - d_\mathrm{start}^2}}{r}\, dr \\
\end{split}
\end{equation}

For $I_8$, we have 
\begin{equation}
\begin{split}
    I_8  & = \int_{r_\mathrm{start}}^{r_\mathrm{end}} \int_{\phi_\mathrm{start} + \mathrm{sign}_\mathrm{start}\arccos\left(d_\mathrm{start}/r\right)}^{\phi_\mathrm{end} + \mathrm{sign}_\mathrm{end}\arccos\left(d_\mathrm{end}/r\right)} \sin(\theta)\, d\theta \, dr \\
    & = - (d_\mathrm{end}\cos(\phi_\mathrm{end}) - d_\mathrm{start}\cos(\phi_\mathrm{start})) \int_{r_\mathrm{start}}^{r_\mathrm{end}} \frac{1}{r}\, dr \\
    & + \mathrm{sign}_\mathrm{end}\sin(\phi_\mathrm{end}) \int_{r_\mathrm{start}}^{r_\mathrm{end}} \frac{\sqrt{r^2 - d_\mathrm{end}^2}}{r}\, dr \\
    & - \mathrm{sign}_\mathrm{start}\sin(\phi_\mathrm{start}) \int_{r_\mathrm{start}}^{r_\mathrm{end}} \frac{\sqrt{r^2 - d_\mathrm{start}^2}}{r}\, dr \\
\end{split}
\end{equation}

For $I_9$, we have 
\begin{equation}
\begin{split}
    I_9  & = \int_{r_\mathrm{start}}^{r_\mathrm{end}} \int_{\phi_\mathrm{start} + \mathrm{sign}_\mathrm{start}\arccos\left(d_\mathrm{start}/r\right)}^{\phi_\mathrm{end} + \mathrm{sign}_\mathrm{end}\arccos\left(d_\mathrm{end}/r\right)} r(\cos(\theta))^2 \, d\theta \, dr \\
    & = \frac{1}{2}(\phi_\mathrm{end} - \phi_\mathrm{start} - \frac{\sin(2\phi_\mathrm{end})}{2} + \frac{\sin(2\phi_\mathrm{start})}{2}) \int_{r_\mathrm{start}}^{r_\mathrm{end}} r \, dr \\
    & + \frac{1}{2}\int_{r_\mathrm{start}}^{r_\mathrm{end}} r\arccos\left(\frac{d_\mathrm{end}}{r}\right)\, dr \\ 
    & - \frac{1}{2}\int_{r_\mathrm{start}}^{r_\mathrm{end}} r\arccos\left(\frac{d_\mathrm{start}}{r}\right)\, dr \\ 
    & + \frac{1}{2}d_\mathrm{end}\cos(2\phi_\mathrm{end}) \int_{r_\mathrm{start}}^{r_\mathrm{end}} \frac{\sqrt{r^2-d_\mathrm{end}^2}}{r}\, dr \\
    & - \frac{1}{2}d_\mathrm{start}\cos(2\phi_\mathrm{start}) \int_{r_\mathrm{start}}^{r_\mathrm{end}} \frac{\sqrt{r^2-d_\mathrm{start}^2}}{r}\, dr \\
    & + \frac{1}{2}(d_\mathrm{end}^2\sin(2\phi_\mathrm{end}) - d_\mathrm{start}^2\sin(2\phi_\mathrm{start})) \int_{r_\mathrm{start}}^{r_\mathrm{end}} \frac{1}{r} \, dr \\ 
\end{split}
\end{equation}

For $I_{10}$, we have 
\begin{equation}
\begin{split}
    I_{10} & = \int_{r_\mathrm{start}}^{r_\mathrm{end}} \int_{\phi_\mathrm{start} + \mathrm{sign}_\mathrm{start}\arccos\left(d_\mathrm{start}/r\right)}^{\phi_\mathrm{end} + \mathrm{sign}_\mathrm{end}\arccos\left(d_\mathrm{end}/r\right)} r(\sin(\theta))^2 \, d\theta \, dr \\
    & = \frac{1}{2}(\phi_\mathrm{end} - \phi_\mathrm{start} + \frac{\sin(2\phi_\mathrm{end})}{2} - \frac{\sin(2\phi_\mathrm{start})}{2}) \int_{r_\mathrm{start}}^{r_\mathrm{end}} r \, dr \\
    & + \frac{1}{2}\int_{r_\mathrm{start}}^{r_\mathrm{end}} r\arccos\left(\frac{d_\mathrm{end}}{r}\right)\, dr \\ 
    & - \frac{1}{2}\int_{r_\mathrm{start}}^{r_\mathrm{end}} r\arccos\left(\frac{d_\mathrm{start}}{r}\right)\, dr \\ 
    & - \frac{1}{2}d_\mathrm{end}\cos(2\phi_\mathrm{end}) \int_{r_\mathrm{start}}^{r_\mathrm{end}} \frac{\sqrt{r^2-d_\mathrm{end}^2}}{r}\, dr \\
    & + \frac{1}{2}d_\mathrm{start}\cos(2\phi_\mathrm{start}) \int_{r_\mathrm{start}}^{r_\mathrm{end}} \frac{\sqrt{r^2-d_\mathrm{start}^2}}{r}\, dr \\
    & - \frac{1}{2}(d_\mathrm{end}^2\sin(2\phi_\mathrm{end}) - d_\mathrm{start}^2\sin(2\phi_\mathrm{start})) \int_{r_\mathrm{start}}^{r_\mathrm{end}} \frac{1}{r} \, dr \\ 
\end{split}
\end{equation}

For $I_{11}$, we have 
\begin{equation}
\begin{split}
    I_{11}  & = \int_{r_\mathrm{start}}^{r_\mathrm{end}} \int_{\phi_\mathrm{start} + \mathrm{sign}_\mathrm{start}\arccos\left(d_\mathrm{start}/r\right)}^{\phi_\mathrm{end} + \mathrm{sign}_\mathrm{end}\arccos\left(d_\mathrm{end}/r\right)} r\cos(\theta)\sin(\theta) \, d\theta \, dr \\
    & = -\frac{1}{2} (d_\mathrm{end}^2 \cos(2\phi_\mathrm{end}) - d_\mathrm{start}^2 \cos(2\phi_\mathrm{start})) \int_{r_\mathrm{start}}^{r_\mathrm{end}} \frac{1}{r}\, dr \\
    & - \frac{1}{2}\left((\sin(\phi_\mathrm{end}))^2 - \sin(\phi_\mathrm{start}))^2\right)\int_{r_\mathrm{start}}^{r_\mathrm{end}} r\, dr \\
    & + \frac{1}{2}\mathrm{sign}_\mathrm{end} d_\mathrm{end}\sin(2\phi_\mathrm{end}) \int_{r_\mathrm{start}}^{r_\mathrm{end}} \frac{\sqrt{r^2 - d_\mathrm{end}^2}}{r}\, dr \\
    & - \frac{1}{2}\mathrm{sign}_\mathrm{start} d_\mathrm{start}\sin(2\phi_\mathrm{start}) \int_{r_\mathrm{start}}^{r_\mathrm{end}} \frac{\sqrt{r^2 - d_\mathrm{start}^2}}{r}\, dr \\
\end{split}
\end{equation}

Though many of these integrals in $r$ may not exist if $r_\mathrm{start} = 0$, note that $d_\mathrm{end}=d_\mathrm{start}=0$ in those cases since the origin is a vertex of the triangle.



To see why this is useful, consider the integral
\begin{equation}
    \int_{\Delta_S} \frac{f(s, t)}{(s^2+t^2)^{\frac{\alpha}{2}}}\, dsdt.
\end{equation}
Given that $f$ is continuous, this has a singularity of type $\alpha$ on $\Delta_S$. 
If we do a change of variable with the Duffy transform, which we denote as $(s, t) = D(s_1, s_2)$, we get 
\begin{equation}
        \int_0^1 \int_0^1 \frac{f(D(s_1, s_2))J(D)}{s_1^{\alpha\beta}((1-s_2)^2+s_2^2)^{\frac{\alpha}{2}}}\, ds\,dt.
\end{equation}
Calculating the Jacobian of $D$, we get $J(D) = \beta s_1^{2\beta - 1}$. Hence, to remove the singularity, we require that 
\begin{equation}\label{Duffy condition}
    2\beta - 1 - \alpha\beta = \beta (2-\alpha) - 1 \geq 0.
\end{equation}
Thus, the Duffy transform can remove singularities for $\alpha \in [0, 2)$. 


\begin{figure}
    \centering
    \includegraphics[width=1.0\linewidth]{Figures/near_singular_analytic_time_result.png}
    \caption{Wall-clock time comparison between the analytic method and adaptive method (integral2 from Matlab) on 2000 tests. The analytic method was run on Rust while the adaptive method was run in Matlab. Both languages are relatively optimized for mathematical computations, but the analytic method is more than $10^3$ times faster than the adaptive method.}
    \label{fig:near singular analytic time}
\end{figure}
As seen in \autoref{fig:near singular analytic results}, the relative difference between the two methods is always below $10^{-10}$, even when the distance between $\x$ and $\Delta$ becomes very small. If we plotted absolute difference, most values are around $10^{-16}$, which is double-precision. Furthermore, \autoref{fig:near singular analytic time} shows that with around the same accuracy, the adaptive method is on average more than $10^3$ times faster than the adaptive method. 

\end{document}

