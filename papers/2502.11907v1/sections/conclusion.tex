\documentclass[../paper.tex]{subfiles}
\usepackage{graphicx} % Required for inserting images
\graphicspath{{\subfix{../Figures/}}}

\usepackage{amssymb}
\usepackage{amsmath}
\usepackage{amsthm}
\usepackage{hyperref}

\usepackage{biblatex} %Imports biblatex package
\addbibresource{references.bib} %Import the bibliography file

\usepackage{xcolor}

\begin{document}

In the fields of numerical PDEs, many problems are reformulated into integral equations. For Fredholm integral equations, the kernel functions are often weakly singular, and becomes strongly singular when the domain of integration is approximated by a finite number of elements. This is often the case in boundary element methods (BEM), so it is vital to be able to calculate these integrals accurately. Previous practices include setting the strongly singular integrals to zero and stating the error is bounded by the size of the simplex, evaluating only the non-singular part of the integral, or using adaptive refinement with a combination of the previous methods. Most previous analytic methods were only applicable to the 2D case, or when the singularity was of type $r^n, n\geq -1$.

However, with our new methods, we can provide fast computational algorithms that lowers the error of these strongly singular integrals even the case where the kernel function $K(\x, \y)$ has singularity $r^{-2}$.
The Interpolation-Duffy method uses a quadrature rule on the unit square and interpolates the normal vector $\nx$ to heuristically stop the new domain $\Delta$ from introducing a new singularity. 
The Geometric method incorporates the true geometry of $\partial\Omega$ in a push-forward map so that the singularity caused by approximating $\partial\Omega$ by $\Delta$ is never present in the first place.
Both methods are better than simply setting the integral to zero and much faster than adaptive refinement methods, but the Geometric method is much more accurate than the Interpolation-Duffy method as it integrates on the true geometry. 
However, the Geometric method requires an extra step of calculating the second fundamental form, either from the function that defines the surface, or by using a finite difference scheme when only a collection of points on the surface is given.
Even when the integral is not strongly singular but "near-singular" due to $\x$ being near $\Delta$, we provide analytic formulas that are just as accurate as adaptive methods when integrating on the simplex $\Delta$ while being much faster. 

All formulas for the Geometric method for singular integrals and the analytic method for near singular methods are presented for when $p(\y)$ is a degree two polynomial of its components, but can easily be extended to higher polynomials.
Analytic formulas for integrating $G(\x, \y)p(\y)$ are also presented for linear $p$. As there is no singularity in these integrals, they are accurate no matter where $\x$ resides.

We believe that with these new formulas, many numerical algorithms of solving PDEs such as BEM \cite{ramvsak20073d, ren2015analytical, bohm2024efficient, atkinson1997numerical} can greatly increase their accuracy without the need to heavily refine their discretization of the domain. Especially in the case of BEM, which requires solving a dense linear system, not needing to refine triangular elements allows for a much smaller matrix. This can lead to much faster algorithms and also smaller memory storage due to less elements needed to accurately approximate the domain of integration. 

\end{document}
