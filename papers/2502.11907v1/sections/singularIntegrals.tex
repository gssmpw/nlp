\documentclass[../paper.tex]{subfiles}
\usepackage{graphicx} % Required for inserting images
\graphicspath{{\subfix{../Figures/}}}

\usepackage{amssymb}
\usepackage{amsmath}
\usepackage{hyperref}
\usepackage{cleveref}

\numberwithin{equation}{section}
\usepackage{xcolor}

\usepackage{biblatex} %Imports biblatex package
\addbibresource{references.bib} %Import the bibliography file

\begin{document}
In the previous section, we were able to directly integrate \autoref{eq: general integral problem gradient green} when $\x\notin\Delta$ because $K$ was not singular on $\Delta$. Once $x\in\Delta$, the integral becomes strongly singular and all the previous formulas fail. Specifically, some of the integrals in $r$ become divergent. As the pre-discretization integral are clearly not infinite, we stress that this is due to the process of approximating a $C^2$ boundary with flat triangles, and not a feature of the integral we are trying to approximate. 

Let $V_1, V_2, V_3$ be three points on the surface $\partial\Omega$ and let $\partial\Omega_V\subset\partial\Omega$ be the curved triangular patch on which we want to integrate our function. The specifics of how to define this triangular patch is discussed later. The true integral we want to evaluate is 
\begin{equation}\label{eq: True grad of Green integral on C2 boundary}
    I_{\text{true}} = \int_{\partial\Omega_V} \frac{\partial G}{\partial n(\x)}(\x, \y) p(\y)\, dS(\y).
\end{equation}
In this section, we give two different methods to approximate $I_{\text{true}}$ when $\x$ is one of the vertices of the triangular patch. Without loss of generality, let us assume that $\x = V_1$.
This commonly occurs in numerical algorithms \cite{hsiao2008boundary} as the points $\x$ are usually collocation points chosen to be vertices of the triangles in $\partial\Omega_{\Delta}$.

As stated before, if we approximate $\partial\Omega_V$ with a flat triangle $\Delta$, this integral becomes undefined. Thus, we need to either keep the true domain of integration, or we integrate a new function on $\Delta$ whose integral value is finite and hopefully approximates $I_{\text{true}}$.
Previous methods that keep the true domain $\partial\Omega_V$ do not have explicit generalizable formulas \cite{mantivc1994computing}, or are applicable only in the $2$D setting with line elements \cite{sladek2000optimal}. Others use the second approach and simply integrate the zero function on $\Delta$, or only evaluate the finite part of the divergent integral \cite{jarvenpaa2003singularity}.
The second approach can be viewed as trying to approximate infinity with a finite value that hopefully is close to $I_{\text{true}}$.

We first present the Geometric method, which keeps the domain to be $\partial\Omega_V$ and approximates it with analytic closed form expressions. We then present the I-D method, which approximates the domain with $\Delta$, but integrates a heuristically chosen similar integral that is a weakly singular integral. 

\subsection{Geometric Method}
Instead of approximating an ``infinity" that was intended to approximate $I_{\text{true}}$, the Geometric method directly approximates $I_{\text{true}}$ by using geometric properties of the true domain $\partial\Omega$. 

Let us first rotate and shift the coordinate system so that $V_1$ is the origin and its normal vector $n(V_1) = (0, 0, 1)^T$. Using $(s_1, s_2)$ as the coordinate system for the XY-plane that is now tangent to $\partial\Omega$, we have a push-forward map $\pi_*$ from the $(s_1, s_2)$-plane onto a local region of $\partial\Omega$ that contains the three vertices.
Let $pV_2$ and $pV_3$ be the projections of $V_2$ and $V_3$ onto the $(s_1, s_2)$-plane and $\Delta (pV_1pV_2pV_3)$ be the triangle that joins $V_1, pV_2$, and $pV_3$. Let $\partial\Omega_V$ denote the triangular patch on $\partial\Omega$ defined by $\pi^{-1}(\Delta (pV_1pV_2pV_3))$.
With our rotations, the $(s_1, s_2)$-plane is tangent to $\partial\Omega_V$ and $\partial\Omega_V$ can be represented as a graph $s_3 = f(x^1, x^2)$.

Instead of approximating the integral on $\partial\Omega_V$ by integrating the same integrand on $\Delta(pV_1pV_2pV_3)$, we instead integrate directly on $\partial\Omega_V$ using a push-forward map. Integration on these curved triangular patches has been presented before \cite{cai2002singularity}, but only with singularity $r^{-1}$. 

Let us denote the desired integral as 
\begin{equation}
    I = \int_{\partial\Omega_V} \frac{(V_1 - \y)\cdot n(V_1)}{|V_1 - \y|^3} p(\y)\, dS(\y).
\end{equation}
Using the push-forward map, the integral can also be written as 
\begin{equation}
    I = \int_{p(V_1V_2V_3)} \pi_* \left[\frac{(V_1 - \y)\cdot n(V_1)}{|V_1 - \y|^3} p(\y)\right]\, \pi_*\left[ dS(\y) \right].
\end{equation}
As $V_1 = 0$, and $n(V_1) = (0, 0, 1)^T$, the integral simplifies to  
\begin{equation}
    I = \int_{p(V_1V_2V_3)} \pi_* \left[(-\y)\cdot \begin{bmatrix}
        0\\
        0\\
        1\\
    \end{bmatrix}|\y|^{-3} p(\y) \right]\, \pi_*\left[ dS(\y) \right].
\end{equation}
As we positioned our tangent plane to intersect the surface at the origin, $dS(\y)$ and $\pi_*[dS(\y)]$ differ by $O(|\y|^2)$ \cite{toponogov2006differential}.
Since $\y\in \partial\Omega_V$ can be represented as $(s_1, s_2, f(s_1, s_2))$, we wish to approximate $f(s_1, s_2)$.
The second fundamental form $K_{ij}$ (expressed in the basis $\partial_1$, $\partial_2$ corresponding to $(s_1, s_2)$) of $\partial\Omega_j$ at $V_1$ is the Hessian of $f$ at $V_1$. 
\begin{equation}
    K_{ij}(V_1) = [\partial_{ij}f](V_1).
\end{equation}
In the special case of $\Omega_j$ being a sphere of radius $R$, the second fundamental form is $K_{ij} = -R^{-1} \delta_{ij}$ where we use the outward pointing normal.  
Using the Taylor expansion of $f$ at $V_1=0$, we have 
\begin{equation}
    \pi_*\left[\y\cdot \begin{bmatrix}
        0\\
        0\\
        1\\
    \end{bmatrix}\right] = \sum_{i, j = 1}^2 \frac{1}{2} \partial_{ij}f(V_1)\cdot s_is_j + O(|(s_1, s_2)|^3) = \sum_{i, j = 1}^2 \frac{1}{2} K_{ij}s_is_j + O(|(s_1, s_2)|^3).
\end{equation}
This is because $f(0) = 0$ and all first order derivatives of $f$ are zero as $\partial\Omega_V$ is tangent to the $(s_1, s_2)$-plane.
We can also approximate $f(s_1, s_2)$ with only the first degree Taylor expansion with error $O(|(s_1, s_2|^2)$, so $|\y|^{-3} = (s_1^2 + x_2^2)^{-3/2} + O(|(s_1, s_2)|^2)$. Hence, our integral becomes
\begin{equation}
    \begin{split}
            I &= -\frac{1}{2} \int_{\Delta(pV_1pV_2pV_3)}\pi_*[p](s_1, s_2) \sum_{i, j = 1}^2 K_{ij}s_is_j \left[ (s_1)^2 + (s_2)^2 \right]^{-\frac{3}{2}}\, d(s_1, s_2)\\
    & + O(|(s_1, s_2)|^2).
    \end{split}
\end{equation}

Now we wish to approximate $\pi_*[p]$ via a polynomial function. 
Using our previous interpolation method, we have that 
\begin{equation}
    \pi_*[p](x^1, x^2) \approx \sum_{k=1}^M \pi_*[p](pQ_k)l_k(x^1, x^2),
\end{equation}
for some quadrature points $pQ_k\in \Delta(pV_1pV_2pV_3)$. Each $pQ_k$ is the projection of a point $Q_k\in \partial\Omega_V$ onto the $(x^1, x^2)-$plane.
By definition of the push-forward, we get that this is equivalent to
\begin{equation}
    \pi_*[p](x^1, x^2) \approx \sum_{k=1}^M p(Q_k)l_k(x^1, x^2).
\end{equation}
If we choose a quadrature with enough nodes, this is exact for our polynomial $p$, so there is no error in this part of the formula. Hence, our integral is approximately
\begin{equation}
    I \approx \sum_{k=1}^M -\frac{p(Q_k)}{2} \int_{\Delta(pV_1pV_2pV_3)} \sum_{i, j = 1}^2 K_{ij}x^ix^j l_k(x^1, x^2) \left[ (x^1)^2 + (x^2)^2 \right]^{-3/2}\, d(x^1)\, d(x^2)
\end{equation}
and the error is approximately quadratic in $|\y|$. 
Due to the lack of constant terms in the numerator, the integral always exists. 
Converting to polar coordinates, we have integrals of the form
\begin{equation}
    \int_{\Delta(pV_1pV_2pV_3)} r^{a+b+1-3}(\cos(\theta))^a(\sin(\theta))^b \, d\theta\, dr,
\end{equation} 
where $a$ and $b$ non-negative integers that represent the powers of the two coordinates in the polynomials $l_k$. 

Instead of integrating in $\theta$ first like before, we integrate in $r$ first and then integrate in $\theta$. 
The reason for this is due to the fact that $pV_1$ is the origin. 
Having a fixed vertex at the origin means that we can perform a rotation so that $pV_2$ lies on the positive $s_1$-line, which then means that the integral in $\theta$ is a single segment unlike in previous cases. 
Let $\Delta(pV_1pV_2pV_3)$ be positively oriented and use a rotation such that $pV_2$ lies on the $s_1$-axis. 
This way, $\theta\in[0, \theta_{pV_3}]$ where $\theta_{pV_3}$ is the angle of $pV_3$ and the $s_1$-axis. 
As a function of $\theta$, we have that 
\begin{equation}
    r(\theta) =  \frac{|pV_2| \sin(\theta_{2})}{\sin(\theta + \theta_{2})},
\end{equation}
where $\theta_{2}$ is the angle between $pV_1pV_2$ and $pV_2pV_3$. This equation is obtained using the sine law, which states that 
\begin{equation}
    \frac{\sin(\theta_{2})}{r} = \frac{\sin(\pi - \theta - \theta_{2})}{|pV_2|}.
\end{equation}
For a visualization, see \autoref{fig: radius as a function of theta}. 
\begin{figure}
\begin{center}
\begin{tikzpicture}

% draw the axes
\coordinate (xaxis) at (4.5,0);
\coordinate (yaxis) at (0,3.5);
\draw[->] (0,0) -- (xaxis) node[right] {$s_1$};
\draw[->] (0,0) -- (yaxis) node[above] {$s_2$};

% draw the triangle
\coordinate (V1) at (0.0,0.0);
\coordinate (V2) at (4.0,0.0);
\coordinate (V3) at (2.0,3.0);
\coordinate (midP) at (2.2, 2.7);
\draw[fill=blue!20,opacity=0.5] (V1) -- (V2) -- (V3) -- cycle;

% label the vertices of the triangle
\node at (V1) [above left] {$\boldsymbol{pV_1}$};
\node at (V2) [above right] {$\boldsymbol{pV_2}$};
\node at (V3) [above] {$\boldsymbol{pV_3}$};
\fill (V1) circle (2pt);
\fill (V2) circle (2pt);
\fill (V3) circle (2pt);

% draw the radius
\node at (midP) [above right] {$\boldsymbol{W}$};
\fill (midP) circle (2pt);
\draw[dashed] (V1) -- (midP);

% label the angles
\draw pic["$\theta_2$", draw=black, <-, angle eccentricity=1.2, angle radius=1cm] {angle = V3--V2--V1};
\draw pic["$\psi$", draw=black, ->, angle eccentricity=1.2, angle radius=0.7cm] {angle = V1--midP--V2};
\draw pic["$\theta$", draw=black, ->, angle eccentricity=1.2, angle radius=0.7cm] {angle = V2--V1--midP};

\end{tikzpicture}
\end{center}
\caption{A diagram showing how to calculate $r(\theta)$. 
As $\theta$ grows, the sine law gives that the ratio between $r$ and $\sin(\theta_2)$ is the same as the ratio between $|pV_2|$ and $\sin(\psi)$.} \label{fig: radius as a function of theta}
\end{figure}
Hence the integrals we need to compute are of the form
\begin{equation}
    \int_0^{\theta_\mathrm{end}} (\cos(\theta))^a (\sin(\theta))^b \int_0^{r(\theta)} r^{a+b+1-3}\, dr \, d\theta.
\end{equation}
The integrals in $r$ are trivial as $a+b+1-3\geq 0$, so we get 
\begin{equation}
    \int_0^{\theta_\mathrm{end}} (\cos(\theta))^a (\sin(\theta))^b \frac{1}{a+b-1}\left(\frac{|pV_2|\sin(\theta_2)}{\sin(\theta+\theta_2)}\right)^{a+b-1}\, d\theta.
\end{equation}

Letting 
\begin{align*}
    \Tilde{I}_{x^2} &= \int_0^{\theta_\mathrm{end}} \frac{\cos^2(\theta)}{\sin(\theta+\theta_2)}\, d\theta, \quad &\Tilde{I}_{y^2} = \int_0^{\theta_\mathrm{end}} \frac{\sin^2(\theta)}{\sin(\theta+\theta_2)}\, d\theta, \\
    \Tilde{I}_{x^2y} &= \int_0^{\theta_\mathrm{end}} \frac{\cos^2(\theta)\sin(\theta)}{\sin^2(\theta+\theta_2)}\, d\theta, \quad &\Tilde{I}_{xy^2} = \int_0^{\theta_\mathrm{end}} \frac{\cos(\theta)\sin^2(\theta)}{\sin^2(\theta+\theta_2)}\, d\theta, \\
    \Tilde{I}_{x^3} &= \int_0^{\theta_\mathrm{end}} \frac{\cos^3(\theta)}{\sin^2(\theta+\theta_2)}\, d\theta, \quad &\Tilde{I}_{y^3} = \int_0^{\theta_\mathrm{end}} \frac{\sin^3(\theta)}{\sin^2(\theta+\theta_2)}\, d\theta, \\
\end{align*}
the desired integral can be written as some linear combination of these integrals.
The formulas for $\Tilde{I}_{x^2}, \ldots, \Tilde{I}_{y^3}$ can be found in Appendix \autoref{appendix: formula for geometric method}. 

An important remark is that the Geometric method can still be used when $\x$ is not a vertex. The only difference is that now we can no longer position the triangle to have one edge on the $s_1$-axis. We would instead need to determine bounds of integration using the algorithm from the previous section \ref{sec: near singular integrals}.

\subsection{Calculating the Second Fundamental Form}
Since the Geometric method requires knowledge of the second fundamental form of a surface, we explain what it is and how to calculate it numerically.
Intuitively, the second fundamental form is an extrinsic curvature that describes how curved a surface is. For a more rigorous and geometric understanding, see \cite{toponogov2006differential, do2016differential} or any textbook on Differential Geometry. The theory contains many definitions and notions of fundamental forms, but we present a simple method of computing the second fundamental form for manifolds embedded in the standard Euclidean metric of $\mathbb{R}^3$. Given a point $\y\in\partial\Omega$, we first perform rotations and translations so that the surface is tangent to the $XY$-plane, intersects it at the origin, and the outward normal is pointing up. If the surface $\partial\Omega$ is the graph of a function $z = f(s_1, s_2)$, then the second fundamental form at the point $(x, y)$ is $K_{ij} = \frac{\partial^2 f}{\partial s_i \partial s_j}(x, y)$. It is just the coefficients of the second order terms in the Taylor expansion of $f$ at $(x, y)$. 

For the sphere of radius $R$ centered at the origin, we first apply a rotation so that $(x, y) = (0, 0)$ and that we are on the upper half of the sphere. Locally, we have that $z = \sqrt{R^2 - x^2 - y^2}$. Thus, its Taylor expansion is 
\begin{equation}
    z = R - \frac{1}{R}x^2 - \frac{1}{R}y^2 + \cdots.
\end{equation}
Hence, $K_{ij} = -\delta_{ij}R^{-1}$ as we stated before. 

This shows that if the user can locally represent the surface as the graph of a function, the second fundamental form can be explicitly calculated analytically. If the surface is defined only by a collection of points in a mesh and no such parametrization function is available, one can still approximate the second derivatives. At any point $\y$ in the mesh, you can use a finite difference scheme and the nearby mesh points to approximate the second derivative \cite{liszka1980finite}.

\subsection{Interpolation-Duffy Method}
We now present the I-D method, which is a quadrature method inspired by the generalized Duffy transform \cite{mousavi2010generalized}. Unlike the Geometric method, the I-D method does not use any geometric information of the boundary, thus it is much less accurate than the Geometric method. We introduce it only as a comparison and as another method that still does better than simply setting the integral to zero. 

To simplify the integration, we first find the affine map $T$ which maps the unit simplex $\Delta_S$ to $\Delta$. Here, the unit simplex is defined as 
\begin{equation}
    \Delta_S = \{(s, t, 0)\in \mathbb{R}^3: s+t=1, 0\leq s \leq 1, 0 \leq t \leq 1\}.
\end{equation}
This exists as $\Delta$ is a flat triangle in $\mathbb{R}^3$, so $T$ is just a composition of rotations, sheers, and translations. Furthermore, it is easy to see that the Jacobian of $T$ is a constant which is just $J(T) = \frac{1}{2}\text{Area}(\Delta)$. Since $\x$ is a vertex of $\Delta$, we choose the map $T$ that maps the origin to $\x$. 

Now, we can write our integral as 
\begin{equation}
    J(T)\int_{\Delta_S} \frac{(\x - T(s, t))\cdot \nx}{|\x - T(s, t)|^3}p(T(s, t))\, dA.
\end{equation}
As the singularity is at the origin, we can use the Duffy transform, which maps the unit square to the unit simplex via 
\begin{equation}
    s = (1-s_2)s_1^\beta, \quad t = s_1^\beta s_2.
\end{equation}
Choosing $\beta = 1$, this makes singularities of type $r^{-1}$ removable, which one would think matches our kernel function. One of the powers in the denominator is canceled out by the $(\x-\y)$ and another is canceled from the dot product between $(\x-\y)$ and $\nx$, however the second statement is not true on $\Delta$. 
Even when $\y$ approaches $\x$, the dot product does not approach zero as $\y$ is a point in $\Delta$ and not in $\partial\Omega$. 
Given a triangle $\Delta$ in our triangulation and a map $T:\Delta_S \rightarrow \Delta$, we have that the normal vector on $\Delta$ can be represented as 
\begin{equation}
    n(\Delta) = \pm \frac{\frac{\partial T}{\partial s} \times \frac{\partial T}{\partial t}}{|\frac{\partial T}{\partial s} \times \frac{\partial T}{\partial t}|}.
\end{equation}
It is the cross product of two sides of the triangle, and the sign is chosen so that it is the outward normal. 

Some authors that encounter this strongly singular integral simply approximate it by setting it to zero \cite{ramvsak20073d, ren2015analytical, bohm2024efficient}, with the reasoning that they choose the normal $\nx$ to just be $n(\Delta)$. 
As $(\x-\y)$ lives on the same plane as $\Delta$, $\nx \cdot (\x-\y) = 0$ so the strongly singular integral is set to be zero. 
Their reasoning is that with a fine enough triangulation mesh, the principle value integral is zero. 

Instead of following the approach of these authors, we propose another idea that may solve this problem that does not require very fine meshes. 
To make sure that the dot product with the normal vector $\nx$ still cancels out one of the singularities, we require that as $\y$ approaches $\x$, $(\x-\y)$ becomes orthogonal to $\nx$. 
However, away from $\x$, we do not want $(\x-\y)\cdot \nx$ to simply be zero, as that would defeat the purpose of even trying to evaluate the singular integral. 

We define an augmented normal vector $N(\x, \y)$ on a triangle $\Delta$ as a function similar to a convex combination of its true normal vector and the normal vector of the triangle $\Delta$.
\begin{equation}
    N(\x, \y) = \alpha \nx + (1-\alpha) n(\Delta)
\end{equation}
where $\alpha$ as defined as a rapidly changing smooth bump function:
\begin{equation}
    \alpha(y) = \begin{cases}
        0, & \text{if } \beta \leq a_1 \\
        1 - \exp{\left(1 + \frac{1}{(a_3(\beta-a_1))^2 - 1}\right)}, & \text{if } a_1 < \beta \leq a_2 \\
        1, & \text{if } \beta > a_2 \\
    \end{cases}
\end{equation}
and
\begin{equation}
    a_3 = \frac{1}{a_2-a_1},\quad \beta = \frac{|\x-\y|}{\max_{\y\in\Delta}{|\x-\y|}}.
\end{equation}
We choose $a_2 = a_1 + 0.1$ and 
\begin{equation*}
    a_1 = \begin{cases}
        0.40 - 0.37\max_{\y\in\Delta}{|\x-\y|}, &\max_{\y\in\Delta}{|\x-\y|} > 0.1 \\ 
        0.418 - 0.55\max_{\y\in\Delta}{|\x-\y|}, & \max_{\y\in\Delta}{|\x-\y|} \in (0.05, 0.1] \\
        0.4187 - 0.62\max_{\y\in\Delta}{|\x-\y|}, & \max_{\y\in\Delta}{|\x-\y|} \in (0.01, 0.05] \\
        1.0, &\max_{\y\in\Delta}{|\x-\y|} \leq 0.01 \\ 
    \end{cases}.
\end{equation*}
These were chosen after multiple tests. The true optimal values we hypothesize should be some polynomial of degree at least two or a function that behaves like $f(x) = 1/x$ while depending on the curvature of the geometry and the edge lengths of the triangle. 
When $\y$ becomes sufficiently close to $\x$, the augmented normal vector $N(\x, \y)$ becomes perpendicular to $(\x-\y)$, which makes the singularity removable. 
As $\y$ gets further away from $\x$, $N(\x, \y)$ rapidly becomes $\nx$. 
Hence, this augmented kernel function has a singularity of type $\alpha=1$ at the origin, so the Duffy transform makes it twice differentiable on the unit square. Hence, the convergence rate with increasing number of nodes on the unit square follows the theory for twice differentiable functions functions. A simple quadrature scheme for the unit square is the tensor product of Gauss-Legendre quadrature, though others can also be used.

As our choice of how to define $N(\x, \y)$ can be viewed as a heuristic approximation, we believe that this I-D method is not the best method of approximating the true integral. It is simply a method that allows quadrature methods to approximate the true integral with a similar, but different, integral that does not diverge. 

\subsection{Comparison between the two methods}
The Geometric method is much faster than the I-D method due to the I-D method potentially using a large number of quadrature nodes on the unit square. 
The Geometric method is also approximating the true integral on $\partial\Omega$, while the I-D method is evaluating an augmented finite integral on $\Delta$ that hopefully approximates the true finite integral on $\partial\Omega$. 
However, the Geometric method requires knowledge of the second fundamental form of $\partial\Omega$, which requires an extra step to compute. 

\end{document}

