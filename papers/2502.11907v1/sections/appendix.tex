\documentclass[../paper.tex]{subfiles}
\usepackage{graphicx} % Required for inserting images
\graphicspath{{\subfix{../Figures/}}}

\usepackage{amssymb}
\usepackage{amsmath}
\usepackage{amsthm}
\usepackage{hyperref}

\usepackage{biblatex} %Imports biblatex package
\addbibresource{references.bib} %Import the bibliography file

\begin{document}

We list all the formulas needed to evaluate the integrals shown in this paper, and some remarks on their stability. 

\subsection{Formulas for near singular integrals}\label{appendix: formula for integral of normal derivative of single layer potential}
Given that 
\begin{align*}
I_0:=\int_\Delta \frac{1}{(y_x^2+y_y^2+c^2)^{3/2}} ~dA_{\y},~ I_x:=\int_\Delta \frac{y_x}{(y_x^2+y_y^2+c^2)^{3/2}}~dA_{\y},\\ 
I_y:=\int_\Delta \frac{y_y}{(y_x^2+y_y^2+c^2)^{3/2}}~dA_{\y},~ I_{x^2}:=\int_\Delta \frac{y_x^2}{(y_x^2+y_y^2+c^2)^{3/2}}~dA_{\y}, \\
I_{y^2}:=\int_\Delta \frac{y_y^2}{(y_x^2+y_y^2+c^2)^{3/2}}~dA_{\y},~ I_{xy}:=\int_\Delta \frac{y_xy_y}{(y_x^2+y_y^2+c^2)^{3/2}}~dA_{\y}.
\end{align*}

The integrals that we want to analytically write out in polar coordinates are 
\begin{equation}\label{single-layer potential on triangle}
    \int_{r_\mathrm{start}}^{r_\mathrm{end}} \int_{\theta_\mathrm{start}(r)}^{\theta_\mathrm{end}(r)} \frac{r^{a+b+1}\cos^a(\theta)\sin^b(\theta)}{(r^2+c^2)^{3/2}} ~d\theta dr,
\end{equation}
in which $(a,b)\in \left\{(0,0),(1,0),(0,1),(2,0),(0,2),(1,1)\right\}$. The integrals in $\theta$ are easy to integrate. After integrating, we substitute $\theta = \phi \pm \arccos(d/r)$ to see what we need to integrate in $r$.

\begin{align*}
(a, b) = (0, 0): &\quad \int 1 ~d\theta = \theta = \phi \pm \arccos(d/r), \\
(a, b) = (1, 0): &\quad \int \cos\theta ~d\theta = \sin \theta = \sin(\phi \pm \arccos(d/r)) \\
&\quad= \frac{d}{r} \sin\phi \pm \frac{\sqrt{r^2-d^2}}{r}\cos\phi, \\
(a, b) = (0, 1): &\quad \int \sin\theta ~d\theta = -\cos\theta = -\cos(\phi \pm \arccos(d/r)) \\ 
&\quad = -\frac{d}{r} \cos\phi \pm \frac{\sqrt{r^2-d^2}}{r}\sin\phi, \\
(a, b) = (2, 0): &\quad \int (\cos\theta)^2 ~d\theta = \frac12\theta + \frac12\cos\theta\sin\theta \\
&\quad = \frac12 (\phi \pm \arccos(d/r)) \pm \cos(2\phi)\frac{d\sqrt{r^2-d^2}}{2r^2} + \left(\frac{d^2}{2r^2}-\frac14\right)\sin(2\phi), \\
(a, b) = (0, 2): &\quad \int (\sin\theta)^2 ~d\theta = 
\frac12\theta - \frac12\cos\theta\sin\theta  \\
&\quad = \frac12 (\phi \pm \arccos(d/r)) \mp \cos(2\phi)\frac{d\sqrt{r^2-d^2}}{2r^2} - \left(\frac{d^2}{2r^2}-\frac14\right)\sin(2\phi), \\
(a, b) = (1, 1): &\quad \int \cos\theta\sin\theta ~d\theta = -\frac12 (\cos \theta)^2 = -\frac12 (\cos(\phi \pm \arccos(d/r)))^2 \\
&\quad = -\frac{d^2}{2r^2}\cos(2\phi) - \frac{1}{2}(\sin(\phi))^2 \pm \frac{d\sqrt{r^2-d^2}}{2r^2}\sin(2\phi).\\
\end{align*}

When you plug these $\theta$ integrals into Equation \ref{single-layer potential on triangle}, there are five integrals in $r$: 
\begin{enumerate}
\item $$\int \frac{r}{(r^2+c^2)^{3/2}}~ dr = -\frac{1}{\sqrt{r^2+c^2}},$$

\item $$\int \frac{r}{(r^2+c^2)^{3/2}} \arccos{\frac{d}{r}}~ dr = \begin{cases}
-\frac{\arccos{\frac{d}{r}}}{\sqrt{r^2+c^2}}+\frac{1}{c}\arctan\left(\frac{c}{d}\sqrt{\frac{r^2-d^2}{r^2+c^2}}\right), & d \ne 0, l\neq 0 \\
-\frac{\pi/2}{\sqrt{r^2+c^2}}, & d = 0 
\end{cases},
$$

\item $$
    \int \frac{r}{(r^2+c^2)^{3/2}}\sqrt{r^2-d^2}~ dr = \frac12 \log \left(2 \sqrt{r^2+c^2}
    \sqrt{r^2-d^2}+c^2-d^2+2 r^2\right)-\sqrt{\frac{r^2-d^2}{r^2+c^2}},
$$

\item $$\int \frac{r^3}{(r^2+c^2)^{3/2}}~ dr = \frac{r^2 + 2c^2}{\sqrt{r^2+c^2}},$$

\item \begin{align*} & \int \frac{r^3}{(r^2+c^2)^{3/2}} \arccos{\frac{d}{r}}~ dr \\
& \quad =
\begin{cases}
 -2 c~\text{arctan} \left(\frac{c}{d} \sqrt{\frac{r^2-d^2}{r^2+c^2}} \right)+ \frac{r^2 + 2c^2}{\sqrt{r^2+c^2}}\arccos\frac{d}{r} - d~ \text{arctanh}\left( \sqrt{\frac{r^2-d^2}{r^2+c^2}} \right), & d \ne 0, r\ne 0 \\
 \frac{\pi}{2}\frac{r^2 + 2c^2}{\sqrt{r^2+c^2}}, & d = 0
 \end{cases}.
\end{align*}
\end{enumerate}
Due to the first integral, we see that $c$ cannot equal to 0 when $r=0$ (see Equation \ref{equation: I1}).
In the case when $c=0$, the five integrals become 
\begin{enumerate}
\item $$\int \frac{1}{r^2}~ dr = -\frac{1}{r},$$
\item $$\int \frac{1}{r^2} \arccos{\frac{d}{r}}~ dr = \begin{cases}
-\frac{\arccos\frac{d}{r}}{r} + \frac{\sqrt{r^2-d^2}}{rd}, & d \ne 0, l\neq 0 \\
-\frac{\pi/2}{r}, & d = 0 
\end{cases}
,$$
\item $$
     \int \frac{1}{r^2}\sqrt{r^2-d^2}~ dr = 
     \begin{cases}
         -\frac{\sqrt{r^2 - d^2}}{r} + \log \left( \frac{\sqrt{r^2-d^2} + r}{d}\right), & d\ne 0 \\
         \log(r), & d = 0
     \end{cases},
$$
\item $$\int 1~ dr = r,$$
\item \begin{align*} & \int \arccos{\frac{d}{r}}~ dr \\
& \quad =
\begin{cases}
 r \arccos(\frac{d}{r}) - d \log\left( \frac{\sqrt{r^2 - d^2} + r}{d}\right), & d \ne 0, r\ne 0 \\
 \frac{\pi}{2}r, & d = 0
 \end{cases}.
\end{align*}
\end{enumerate}
Due to numerical precision, some terms can give NaN values (such as $\text{arctanh}\left( \sqrt{\frac{r^2-d^2}{r^2+c^2}} \right)$ when $d^2, c^2 \approx 1e-16$), so these terms are calculated using the first four terms of their Taylor expansion. 

Now we can write out the full formulas for the integrals. For $I_0$, we have 
\begin{equation}\label{equation: I1}
\begin{split}
    I_0  & = \int_{r_\mathrm{start}}^{r_\mathrm{end}} \int_{\phi_\mathrm{start} + \mathrm{sign}_\mathrm{start}\arccos\left(d_\mathrm{start}/r\right)}^{\phi_\mathrm{end} + \mathrm{sign}_\mathrm{end}\arccos\left(d_\mathrm{end}/r\right)} \frac{r}{(r^2 + c^2)^{3/2}}\, d\theta \, dr \\
    & = (\phi_\mathrm{end} - \phi_\mathrm{start}) \int_{r_\mathrm{start}}^{r_\mathrm{end}} \frac{r}{(r^2+c^2)^{3/2}}\, dr \\
    & + \mathrm{sign}_\mathrm{end}\int_{r_\mathrm{start}}^{r_\mathrm{end}} \frac{r}{(r^2 + c^2)^{3/2}}\arccos\left(\frac{r_\mathrm{end}}{r}\right)\, dr \\
    & - \mathrm{sign}_\mathrm{start}\int_{r_\mathrm{start}}^{r_\mathrm{end}} \frac{r}{(r^2 + c^2)^{3/2}}\arccos\left(\frac{d_\mathrm{start}}{r}\right)\, dr.
\end{split}
\end{equation}
It is important to know that when we have $\phi_\mathrm{end} - \phi_\mathrm{start}$, we need to check for branch cuts in $\arctan$. 
These is because $\phi$ was calculated via $\arctan$, which is multi-valued. 
If we did not check for branch cuts, we could be potentially integrating $[\theta_\mathrm{start}, \theta_\mathrm{end} + 2\pi]$ instead of $[\theta_\mathrm{start}, \theta_\mathrm{end}]$. 
When checking the branch cut, we just need to check if $\theta_\mathrm{end} - \theta_\mathrm{start}\in [0, 2\pi]$, \textbf{NOT} if $\phi_\mathrm{end} - \phi_\mathrm{start}\in [0, 2\pi]$. 
This branch cut problem also shows up in $I_{x^2}$ and $I_{y^2}$. 

For $I_x$, we have 
\begin{equation}
    \begin{split}
        I_x & = \int_{r_\mathrm{start}}^{r_\mathrm{end}} \int_{\phi_\mathrm{start} + \mathrm{sign}_\mathrm{start}\arccos(d_\mathrm{start}/r)}^{\phi_\mathrm{end} + \mathrm{sign}_\mathrm{end}\arccos(d_\mathrm{end}/r)} \frac{r^2 \cos(\theta)}{(r^2+c^2)^{3/2}}\, d\theta\, dr \\ 
        & = \left(d_\mathrm{end}\sin(\phi_\mathrm{end}) - d_\mathrm{start}\sin(\phi_\mathrm{start})\right) \int_{r_\mathrm{start}}^{r_\mathrm{end}}\frac{r}{(r^2+c^2)^{3/2}}\, dr \\
        & + \mathrm{sign}_\mathrm{end}\cos(\phi_\mathrm{end}) \int_{r_\mathrm{start}}^{r_\mathrm{end}}\frac{r\sqrt{r^2-d_\mathrm{end}^2}}{(r^2+c^2)^{3/2}}\, dr \\
        & - \mathrm{sign}_\mathrm{start}\cos(\phi_\mathrm{start}) \int_{r_\mathrm{start}}^{r_\mathrm{end}}\frac{r\sqrt{r^2-d_\mathrm{start}^2}}{(r^2+c^2)^{3/2}}\, dr.
    \end{split}
\end{equation}

For $I_y$, we have 
\begin{equation}
\begin{split}
    I_y & = \int_{r_\mathrm{start}}^{r_\mathrm{end}} \int_{\phi_\mathrm{start} + \mathrm{sign}_\mathrm{start}\arccos(d_\mathrm{start}/r)}^{\phi_\mathrm{end} + \mathrm{sign}_\mathrm{end}\arccos(d_\mathrm{end}/r)} \frac{r^2 \sin(\theta)}{(r^2+c^2)^{3/2}}\, d\theta\, dr \\ 
    & = \left(d_\mathrm{start}\cos(\phi_\mathrm{start}) - d_\mathrm{end}\cos(\phi_\mathrm{end})\right) \int_{r_\mathrm{start}}^{r_\mathrm{end}}\frac{r}{(r^2+c^2)^{3/2}}\, dr \\
    & + \mathrm{sign}_\mathrm{end}\sin(\phi_\mathrm{end}) \int_{r_\mathrm{start}}^{r_\mathrm{end}}\frac{r\sqrt{r^2-d_\mathrm{end}^2}}{(r^2+c^2)^{3/2}}\, dr \\
    & - \mathrm{sign}_\mathrm{start}\sin(\phi_\mathrm{start}) \int_{r_\mathrm{start}}^{r_\mathrm{end}}\frac{r\sqrt{r^2-d_\mathrm{start}^2}}{(r^2+c^2)^{3/2}}\, dr.
\end{split}
\end{equation}

For $I_{x^2}$, we have 
\begin{equation}
    \begin{split}
        I_{x^2} & = \int_{r_\mathrm{start}}^{r_\mathrm{end}} \int_{\phi_\mathrm{start} + \mathrm{sign}_\mathrm{start}\arccos(d_\mathrm{start}/r)}^{\phi_\mathrm{end} + \mathrm{sign}_\mathrm{end}\arccos(d_\mathrm{end}/r)} \frac{r^3 \cos^2(\theta)}{(r^2+c^2)^{3/2}}\, d\theta\, dr \\ 
        & = \frac{1}{2}(\phi_\mathrm{end} - \frac{\sin(2\phi_\mathrm{end})}{2} - \phi_\mathrm{start} + \frac{\sin(2\phi_\mathrm{start})}{2})\int_{r_\mathrm{start}}^{r_\mathrm{end}} \frac{r^3}{(r^2+c^2)^{3/2}}\, dr \\
        & + \frac{\mathrm{sign}_\mathrm{end}}{2}\int_{r_\mathrm{start}}^{r_\mathrm{end}} \frac{r^3\arccos(\frac{d_\mathrm{end}}{r})}{(r^2+c^2)^{3/2}}\, dr\\
        & - \frac{\mathrm{sign}_\mathrm{start}}{2}\int_{r_\mathrm{start}}^{r_\mathrm{end}} \frac{r^3\arccos(\frac{d_\mathrm{start}}{r})}{(r^2+c^2)^{3/2}}\, dr\\
        & + \frac{1}{2}(d_\mathrm{end}^2\sin(2\phi_\mathrm{end}) - d_\mathrm{start}^2\sin(2\phi_\mathrm{start})) \int_{r_\mathrm{start}}^{r_\mathrm{end}} \frac{r}{(r^2+c^2)^{3/2}}\, dr\\
        & + \frac{1}{2}\mathrm{sign}_\mathrm{end}d_\mathrm{end}\cos(2\phi_\mathrm{end}) \int_{r_\mathrm{start}}^{r_\mathrm{end}} \frac{r\sqrt{r^2-d_\mathrm{end}^2}}{(r^2+c^2)^{3/2}}\, dr\\
        & - \frac{1}{2}\mathrm{sign}_\mathrm{start}d_\mathrm{start}\cos(2\phi_\mathrm{start}) \int_{r_\mathrm{start}}^{r_\mathrm{end}} \frac{r\sqrt{r^2-d_\mathrm{start}^2}}{(r^2+c^2)^{3/2}}\, dr.
    \end{split}
\end{equation}

For $I_{y^2}$. we have 
\begin{equation}
    \begin{split}
        I_{y^2} & = \int_{r_\mathrm{start}}^{r_\mathrm{end}} \int_{\phi_\mathrm{start} + \mathrm{sign}_\mathrm{start}\arccos(d_\mathrm{start}/r)}^{\phi_\mathrm{end} + \mathrm{sign}_\mathrm{end}\arccos(d_\mathrm{end}/r)} \frac{r^3 \sin^2(\theta)}{(r^2+c^2)^{3/2}}\, d\theta\, dr \\ 
        & = \frac{1}{2}\left(\phi_\mathrm{end} + \frac{\sin(2\phi_\mathrm{end})}{2} - \phi_\mathrm{start} - \frac{\sin(2\phi_\mathrm{start})}{2}\right)\int_{r_\mathrm{start}}^{r_\mathrm{end}} \frac{r^3}{(r^2+c^2)^{3/2}}\, dr \\
        & + \frac{\mathrm{sign}_\mathrm{end}}{2}\int_{r_\mathrm{start}}^{r_\mathrm{end}} \frac{r^3\arccos(\frac{d_\mathrm{end}}{r})}{(r^2+c^2)^{3/2}}\, dr\\
        & - \frac{\mathrm{sign}_\mathrm{start}}{2}\int_{r_\mathrm{start}}^{r_\mathrm{end}} \frac{r^3\arccos(\frac{d_\mathrm{start}}{r})}{(r^2+c^2)^{3/2}}\, dr\\
        & + \frac{1}{2}(- d_\mathrm{end}^2\sin(2\phi_\mathrm{end}) + d_\mathrm{start}^2\sin(2\phi_\mathrm{start})) \int_{r_\mathrm{start}}^{r_\mathrm{end}} \frac{r}{(r^2+c^2)^{3/2}}\, dr\\
        & - \frac{1}{2}\mathrm{sign}_\mathrm{end}d_\mathrm{end}\cos(2\phi_\mathrm{end}) \int_{r_\mathrm{start}}^{r_\mathrm{end}} \frac{r\sqrt{r^2-d_\mathrm{end}^2}}{(r^2+c^2)^{3/2}}\, dr\\
        & + \frac{1}{2}\mathrm{sign}_\mathrm{start}d_\mathrm{start}\cos(2\phi_\mathrm{start}) \int_{r_\mathrm{start}}^{r_\mathrm{end}} \frac{r\sqrt{r^2-d_\mathrm{start}^2}}{(r^2+c^2)^{3/2}}\, dr.
    \end{split}
\end{equation}

For $I_{xy}$, we have 
\begin{equation}
    \begin{split}
        I_{xy} & = \int_{r_\mathrm{start}}^{r_\mathrm{end}} \int_{\phi_\mathrm{start} + \mathrm{sign}_\mathrm{start}\arccos(d_\mathrm{start}/r)}^{\phi_\mathrm{end} + \mathrm{sign}_\mathrm{end}\arccos(d_\mathrm{end}/r)} \frac{r^3 \sin(\theta)\cos(\theta)}{(r^2+c^2)^{3/2}}\, d\theta\, dr \\
        & = \frac{1}{2}(\sin^2(\phi_\mathrm{start}) - \sin^2(\phi_\mathrm{end})) \int_{r_\mathrm{start}}^{r_\mathrm{end}} \frac{r^3}{(r^2+c^2)^{3/2}}\, dr\\
        & + \frac{1}{2}(d_\mathrm{start}^2 \cos(2\phi_\mathrm{start}) - d_\mathrm{end}^2\cos(2\phi_\mathrm{end})) \int_{r_\mathrm{start}}^{r_\mathrm{end}} \frac{r}{(r^2+c^2)^{3/2}}\, dr\\
        & + \frac{1}{2}\mathrm{sign}_\mathrm{end}\sin(2\phi_\mathrm{end})d_\mathrm{end} \int_{r_\mathrm{start}}^{r_\mathrm{end}} \frac{r\sqrt{r^2-d_\mathrm{end}^2}}{(r^2+c^2)^{3/2}}\, dr\\
        & - \frac{1}{2}\mathrm{sign}_\mathrm{start}\sin(2\phi_\mathrm{start})d_\mathrm{start} \int_{r_\mathrm{start}}^{r_\mathrm{end}} \frac{r\sqrt{r^2-d_\mathrm{start}^2}}{(r^2+c^2)^{3/2}}\, dr.
    \end{split}
\end{equation}

For higher powers, integrating in theta first gives
\begin{align*}
(a, b) = (3, 0): &\quad \int \cos^3(\theta)\, d\theta = \sin(\theta) - \frac{1}{3}\sin^3(\theta),\\
(a, b) = (2, 1): &\quad \int \cos^2(\theta)\sin(\theta)\, d\theta = -\frac{1}{3} \cos^3(\theta),\\
(a, b) = (1, 2): &\quad \int \cos(\theta)\sin^2(\theta)\, d\theta = \frac{1}{3} \sin^3(\theta),\\
(a, b) = (0, 3): &\quad \int \sin^3(\theta)\, d\theta = -\cos(\theta) + \frac{1}{3}\cos^3(\theta).
\end{align*}
Plugging in $\theta = \phi\pm\arccos(d/r)$, we get for $(a, b) = (3, 0)$
\begin{equation*}
\begin{split}
    (a, b) = (3, 0) &:\left(\sin(\theta)\frac{d}{r} \pm \cos(\phi) \frac{\sqrt{r^2 - d^2}}{r}\right) \\
&-\frac{1}{3}\left[ \sin^3(\phi)\frac{d^3}{r^3} \pm 3 \sin^2(\phi)\cos(\phi) \frac{d^2\sqrt{r^2-d^2}}{r^3}\right. \\ 
&+ \left. 3\sin(\phi)\cos^2(\phi)\frac{d(r^2-d^2)}{r^3} \pm \cos^3(\phi) \frac{(r^2-d^2)^{3/2}}{r^3}\right].
\end{split}    
\end{equation*}
For $(a, b) = (0, 3)$, we have 
\begin{equation*}
\begin{split}
    (a, b) = (0, 3) &:-\left(\cos(\theta)\frac{d}{r} \mp \sin(\phi) \frac{\sqrt{r^2 - d^2}}{r}\right) \\
&+\frac{1}{3}\left[ \cos^3(\phi)\frac{d^3}{r^3} \mp 3 \cos^2(\phi)\sin(\phi) \frac{d^2\sqrt{r^2-d^2}}{r^3}\right. \\ 
&+ \left. 3\cos(\phi)\sin^2(\phi)\frac{d(r^2-d^2)}{r^3} \mp \sin^3(\phi) \frac{(r^2-d^2)^{3/2}}{r^3}\right].
\end{split}    
\end{equation*}
Integrating in $r$, we have the extra integral is
\begin{equation*}
    \begin{split}
        \int \frac{r(r^2-d^2)^{3/2}}{(r^2+c^2)^{3/2}}\, dr &= 
        \sqrt{\frac{r^2-d^2}{r^2+c^2}} \frac{r^2+2d^2+3c^2}{2} \\
        &- \frac{3(c^2+d^2)\log(\sqrt{r^2+c^2} + \sqrt{r^2-d^2})}{2}.
    \end{split}
\end{equation*}
When $c=0$, we have that the integral becomes
\begin{equation*}
    \int \frac{(r^2-d^2)^{3/2}}{r^2}\, dr = \left(\frac{d^2}{r} + \frac{r}{2}\right)\sqrt{r^2-d^2} - \frac{3d^2}{2}\log(r + \sqrt{r^2-d^2}).
\end{equation*}
These equations can then be used to get the integrals for $(a, b) = (2, 1)$ or $(a, b) = (1, 2)$.
% ---------------------------------------------------------------------------------------------------
\subsection{Formulas for the Geometric method}\label{appendix: formula for geometric method}
Recall that in the Geometric method, the integrals are of the form
\begin{equation}
    \int_0^{\theta_\mathrm{end}} (\cos(\theta))^a (\sin(\theta))^b \int_0^{r(\theta)} r^{a+b+1-3}\, dr \, d\theta.
\end{equation}
This simple formula is only because we can fix one edge to be the $X$-axis, so that the integral in $\theta$ starts at zero. The integrals in $r$ are trivial as $a+b+1-3\geq 0$, so we get 
\begin{equation}
    \int_0^{\theta_\mathrm{end}} (\cos(\theta))^a (\sin(\theta))^b \frac{1}{a+b-1}\left(\frac{|pV_2|\sin(\theta_2)}{\sin(\theta+\theta_2)}\right)^{a+b-1}\, d\theta.
\end{equation}
We shall only consider the cases in which $(a, b)\in \{(1, 1), (2, 0), (0, 2), (2, 1), (1, 2), (3, 0), (0, 3)\}$. These correspond to when we approximate $p$ using linear functions, but it is not difficult to write similar formulas for higher powers. 
% In the case where $K_{ij} = R^{-1}\delta_{ij}$, we only need to consider $(a, b) \in \{(2, 0), (2, 1), (3, 0), (0, 2), (1, 2), (0, 3)\}$. 

For notation, we write
\begin{align*}
    \Tilde{I}_{x^2} &= \int_0^{\theta_\mathrm{end}} \frac{\cos^2(\theta)}{\sin(\theta+\theta_2)}\, d\theta, \quad &\Tilde{I}_{y^2} = \int_0^{\theta_\mathrm{end}} \frac{\sin^2(\theta)}{\sin(\theta+\theta_2)}\, d\theta, \\
    \Tilde{I}_{x^2y} &= \int_0^{\theta_\mathrm{end}} \frac{\cos^2(\theta)\sin(\theta)}{\sin^2(\theta+\theta_2)}\, d\theta, \quad &\Tilde{I}_{xy^2} = \int_0^{\theta_\mathrm{end}} \frac{\cos(\theta)\sin^2(\theta)}{\sin^2(\theta+\theta_2)}\, d\theta, \\
    \Tilde{I}_{x^3} &= \int_0^{\theta_\mathrm{end}} \frac{\cos^3(\theta)}{\sin^2(\theta+\theta_2)}\, d\theta, \quad &\Tilde{I}_{y^3} = \int_0^{\theta_\mathrm{end}} \frac{\sin^3(\theta)}{\sin^2(\theta+\theta_2)}\, d\theta.
\end{align*}

The integrals are as follows.
\begin{align*}
    (a, b) = (1, 1) &: \int \frac{\cos(\theta)\sin(\theta)}{\sin(\theta+\theta_2)}\, d\theta \\
    &\qquad = \sin(\theta-\theta_2) - \frac{\sin(2\theta_2)}{2} \log\left(\tan(\frac{\theta+\theta_2}{2})\right),\\
    (a, b) = (2, 0) &: \int \frac{\cos^2(\theta)}{\sin(\theta + \theta_2)}\, d\theta \\
    &\qquad = \cos(\theta - \theta_2) + \cos^2(\theta_2)\log\left(\tan\left(\frac{\theta + \theta_2}{2}\right)\right),\\
    (a, b) = (0, 2) &: \int \frac{\sin^2(\theta)}{\sin(\theta + \theta_2)}\, d\theta \\
    &\qquad =  -\cos(\theta - \theta_2) + \sin^2(\theta_2)\log\left(\tan\left(\frac{\theta + \theta_2}{2}\right)\right),\\
    (a, b) = (2, 1) &: \int \frac{\cos^2(\theta)\sin(\theta)}{\sin^2(\theta + \theta_2)}\, d\theta \\
    &\qquad = \frac{1}{4}\big(-2(\cos(\theta_2) + 3\cos(3\theta_2))\mathrm{arctanh}\left(\cos(\theta_2) - \sin(\theta_2)\tan\left(\frac{\theta}{2}\right)\right) \\
    &\qquad + \csc(\theta + \theta_2)(2\sin(2\theta-\theta_2) + \sin(\theta_2) + 3\sin(3\theta_2))\big),\\
    (a, b) = (1, 2) &: \int \frac{\cos(\theta)\sin^2(\theta)}{\sin^2(\theta + \theta_2)}\, d\theta \\
    &\qquad = \frac{1}{4}\big(-2(\sin(\theta_2) - 3\sin(3\theta_2))\mathrm{arctanh}\left(\cos(\theta_2) - \sin(\theta_2)\tan\left(\frac{\theta}{2}\right)\right) \\
    &\qquad - \csc(\theta + \theta_2)(2\cos(2\theta-\theta_2) + \cos(\theta_2) - 3\cos(3\theta_2))\big),\\
    (a, b) = (3, 0) &: \int \frac{\cos^3(\theta)}{\sin^2(\theta + \theta_2)}\, d\theta \\
    &\qquad = -\cos(\theta-2\theta_2) - 6\cos(\theta_2)\sin^2(\theta_2)\mathrm{arctanh}\left( \cos(\theta_2) - \sin(\theta_2)\tan\left(\frac{\theta}{2}\right)\right)\\
    &\qquad + \sin^3(\theta_2)\csc(\theta+\theta_2),\\
    (a, b) = (0, 3) &: \int \frac{\sin^3(\theta)}{\sin^2(\theta + \theta_2)}\, d\theta \\
    &\qquad = -\sin(\theta-2\theta_2) - 6\cos^2(\theta_2)\sin(\theta_2)\mathrm{arctanh}\left(\cos(\theta_2) - \sin(\theta_2)\tan\left(\frac{\theta}{2}\right)\right)\\
    &\qquad -\cos^3(\theta_2)\csc(\theta+\theta_2).
\end{align*}

\subsection{Formulas for Green's function}\label{appendix: formula for single layer potential}
Though most of the paper discusses integrating $K(\x, \y) p(\y)$, we also provide analytic equations for integrating $G(\x, \y)p(\y)$ where $G$ is defined from \autoref{eq: Green in 3D} and $p(\y)$ is a linear function.
Given 
\begin{align*}
J_0&:=\int_\Delta \frac{1}{(y_x^2+y_y^2+c^2)^{1/2}} ~dA_{\y}, \\
J_x&:=\int_\Delta \frac{y_x}{(y_x^2+y_y^2+c^2)^{1/2}}~dA_{\y}, \\
J_y&:=\int_\Delta \frac{y_y}{(y_x^2+y_y^2+c^2)^{1/2}}~dA_{\y}.
\end{align*}

The desired integral is 
\begin{align*}
J = -\frac{1}{4\pi}\sum_{i=1}^3 \gamma(V_i) \big[ l_{i,0}J_0 + l_{i,x}J_x + l_{i,y}J_y\big].
\end{align*}

The integrals are of the form

\begin{equation}
    \int_{r_\mathrm{start}}^{r_l} \int_{\theta_\mathrm{start}}^{\theta_\mathrm{end}} \frac{r^{a+b+1}(\cos(\theta))^a(\sin(\theta))^b}{(r^2 + c^2)^{1/2}}
\end{equation}
for $(a, b) = \{ (0, 0), (0, 1), (1, 0)\}$. The integrals in $\theta$ are 

\begin{align*}
    (a, b) = (0, 0): &\quad \int 1 \, d\theta = \theta = \phi \pm \arccos\left(\frac{d}{r}\right), \\
    (a, b) = (1, 0): &\quad \int \cos(\theta) \, d\theta = \sin(\theta) = \sin\left(\phi \pm \arccos\left(\frac{d}{r}\right)\right) \\
    &\qquad =  \frac{d}{r} \sin(\phi) \pm \frac{\sqrt{r^2-d^2}}{r}\cos(\phi),\\
    (a, b) = (0, 1): &\quad \int \sin(\theta)\, d\theta = -\cos(\theta) = -\cos\left(\phi \pm \arccos\left(\frac{d}{r}\right)\right) \\
    &\qquad =  -\frac{d}{r} \cos(\phi) \pm \frac{\sqrt{r^2-d^2}}{r}\sin(\phi).
\end{align*}

The integrals in $r$ are calculated to be 
\begin{enumerate}
    \item \begin{align*}
        \int \frac{r}{(r^2 + c^2)^{1/2}}\, dr &= \sqrt{r^2 + c^2},
    \end{align*}
    \item \begin{align*}
        \int \frac{r \arccos(\frac{d}{r})}{(r^2 + c^2)^{1/2}}\, dr &= \begin{cases}
            \sqrt{r^2 + c^2}\arccos(\frac{d}{r}) - c\arcsin(\frac{c\sqrt{\frac{r^2-d^2}{r^2}}}{\sqrt{c^2 + d^2}}) - d~\text{arctanh}(\sqrt{\frac{r^2-d^2}{r^2+c^2}}) & d, r \ne 0\\
            \frac{\pi}{2}\sqrt{r^2 + c^2} & d = 0\\
        \end{cases},
    \end{align*}
    \item \begin{align*}
        \int \frac{r\sqrt{r^2-d^2}}{(r^2 + c^2)^{1/2}}\, dr &= \frac{1}{2} \left(\sqrt{r^2+c^2}\sqrt{r^2-d^2} + (c^2+d^2)\log(\sqrt{r^2+c^2} - \sqrt{r^2-d^2})\right).
    \end{align*}
\end{enumerate}
Note here that the sign of $c$ does not matter on both sides of the equations. In the right hand side of the second integral, $\arcsin$ is a odd function, so the sign of the $c$ inside and in front of $\arcsin$ cancel each other out. Unlike the Neumann problem, $c$ can equal to 0 on the left hand side even if $r = 0$. When $c=0$, the integrals are
\begin{enumerate}
    \item \begin{align*}
        \int \frac{r}{(r^2)^{1/2}}\, dr &= r,\\
    \end{align*}
    \item \begin{align*}
        \int \frac{r \arccos(\frac{d}{r})}{(r^2)^{1/2}}\, dr &= \begin{cases}
            r\arccos(\frac{d}{r}) + \frac{d}{2}\log\left(-\frac{\sqrt{r^2-d^2}-r}{\sqrt{r^2-d^2}+r}\right) & d, r \ne 0\\
            \frac{\pi}{2}r & d = 0\\
        \end{cases},
    \end{align*}
    \item \begin{align*}
        \int \frac{r\sqrt{r^2-d^2}}{(r^2)^{1/2}}\, dr &= \begin{cases}
            \frac{1}{2} \left(r\sqrt{r^2-d^2} + d^2\log(r - \sqrt{r^2-d^2})\right) & d, r \ne 0\\
            \frac{1}{2}r^2 & d = 0
        \end{cases}.
    \end{align*}
\end{enumerate}


Hence, 
\begin{align*}
    J_0  & = \int_{r_\mathrm{start}}^{r_\mathrm{end}} \int_{\phi_\mathrm{start} + \mathrm{sign}_\mathrm{start}\arccos\left(d_\mathrm{start}/r\right)}^{\phi_\mathrm{end} + \mathrm{sign}_\mathrm{end}\arccos\left(d_\mathrm{end}/r\right)} \frac{r}{(r^2 + c^2)^{1/2}}\, d\theta \, dr \\
    & = (\phi_\mathrm{end} - \phi_\mathrm{start}) \int_{r_\mathrm{start}}^{r_\mathrm{end}} \frac{r}{(r^2+c^2)^{1/2}}\, dr\\ 
    & + \mathrm{sign}_\mathrm{end}\int_{r_\mathrm{start}}^{r_\mathrm{end}} \frac{r}{(r^2 + c^2)^{1/2}}\arccos\left(\frac{d_\mathrm{end}}{r}\right)\, dr \\
    & - \mathrm{sign}_\mathrm{start}\int_{r_\mathrm{start}}^{r_\mathrm{end}} \frac{r}{(r^2 + c^2)^{1/2}}\arccos\left(\frac{d_\mathrm{start}}{r}\right)\, dr.
\end{align*}
\begin{align*}
    J_x  & = \int_{r_\mathrm{start}}^{r_\mathrm{end}} \int_{\phi_\mathrm{start} + \mathrm{sign}_\mathrm{start}\arccos\left(d_\mathrm{start}/r\right)}^{\phi_\mathrm{end} + \mathrm{sign}_\mathrm{end}\arccos\left(d_\mathrm{end}/r\right)} \frac{r^2 \cos(\theta)}{(r^2 + c^2)^{1/2}}\, d\theta \, dr \\
    & = (d_\mathrm{end}\sin(\phi_\mathrm{end}) - d_\mathrm{start}\sin(\phi_\mathrm{start})) \int_{r_\mathrm{start}}^{r_\mathrm{end}} \frac{r}{(r^2 + c^2)^{1/2}}\, dr\\
    & + \mathrm{sign}_\mathrm{end}\cos(\phi_\mathrm{end})\int_{r_\mathrm{start}}^{r_\mathrm{end}} \frac{r\sqrt{r^2-d_\mathrm{end}^2}}{(r^2+c^2)^{1/2}}\, dr\\
    & - \mathrm{sign}_\mathrm{start}\cos(\phi_\mathrm{start})\int_{r_\mathrm{start}}^{r_\mathrm{end}} \frac{r\sqrt{r^2-d_\mathrm{start}^2}}{(r^2+c^2)^{1/2}}\, dr.
\end{align*}
\begin{align*}
    J_y  & = \int_{r_\mathrm{start}}^{r_\mathrm{end}} \int_{\phi_\mathrm{start} + \mathrm{sign}_\mathrm{start}\arccos\left(d_\mathrm{start}/r\right)}^{\phi_\mathrm{end} + \mathrm{sign}_\mathrm{end}\arccos\left(d_\mathrm{end}/r\right)} \frac{r^2 \sin(\theta)}{(r^2 + c^2)^{1/2}}\, d\theta \, dr \\
    & = -(d_\mathrm{end}\cos(\phi_\mathrm{end}) - d_\mathrm{start}\cos(\phi_\mathrm{start})) \int_{r_\mathrm{start}}^{r_\mathrm{end}} \frac{r}{(r^2 + c^2)^{1/2}}\, dr\\
    & + \mathrm{sign}_\mathrm{end}\sin(\phi_\mathrm{end})\int_{r_\mathrm{start}}^{r_\mathrm{end}} \frac{r\sqrt{r^2-d_\mathrm{end}^2}}{(r^2+c^2)^{1/2}}\, dr\\
    & - \mathrm{sign}_\mathrm{start}\sin(\phi_\mathrm{start})\int_{r_\mathrm{start}}^{r_\mathrm{end}} \frac{r\sqrt{r^2-d_\mathrm{start}^2}}{(r^2+c^2)^{1/2}}\, dr.
\end{align*}

Just like before, we need to be careful of the branch cut problem in $J_x$. As all integrals are finite even when $c=0$, $\x$ can be any arbitrary point in $\mathbb{R}^3$.

\end{document}

