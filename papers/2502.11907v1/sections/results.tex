\documentclass[../paper.tex]{subfiles}
\usepackage{graphicx} % Required for inserting images
\graphicspath{{\subfix{../Figures/}}}

\usepackage{amssymb}
\usepackage{amsmath}
\usepackage{amsthm}
\usepackage{hyperref}

\usepackage{biblatex} %Imports biblatex package
\addbibresource{references.bib} %Import the bibliography file

\usepackage{xcolor}

\begin{document}

To test our new methods for calculating integrals, we test them based on their accuracy and time. For accuracy, we calculate the relative absolute difference of the values computed via the different methods and Matlab's \texttt{integral2} function. \texttt{integral2} uses adaptive quadrature, which is known to be accurate for small enough tolerance. For time, we record the wall-clock time of calculating each integral. For the I-D method, we use a $7\times 7$ Gauss-Legendre quadrature rule on the unit square. 

\subsection{Comparisons on near-singular integrals}
Let us consider evaluating the integral 
\begin{equation}\label{eq: K for simulation}
    I = \int_{\Delta} \frac{(\x-\y)\cdot \nx}{4\pi |\x - \y|^3} \, dS(\y)
\end{equation}
where $\Delta$ is a triangle in the triangulation approximation of the boundary $\partial\Omega = S^1$.
We first present our results on the analytic equations given to evaluate the case when $\x$ does not lie in $\Delta$.
As stated in previous sections, the integral is not singular at all, but its behaviour is near-singular. 
The integral is evaluated in 2000 different tests. 
In each test, the first two components of $\Delta$'s vertices are uniformly sampled from $[-0.05, 0.05]$ while the third component is calculated so that the vertices lie on $S^1$. $\x= [0, 0, c]$ where $c$ is uniformly sampled from $[-0.05, 0.05]$ but its normal vector $\nx$ is fixed to be $[\sqrt{3}/3, \sqrt{3}/3, \sqrt{3}/3]$. This is because after rotations, the normal vectors $\nx$ are different in each test case. 
\begin{figure}
    \centering
    \includegraphics[width=1.0\linewidth]{Figures/near_singular_analytic_error_result.png}
    \caption{Comparison between the values of the integral \autoref{eq: K for simulation} using the Analytic Method and an adaptive method using \texttt{integral2} from Matlab on 2000 tests. $\Delta$ is assumed to have vertices on $S^1$ with their first two components uniformly distributed in $[-0.05, 0.05]^2$. $\x = [0, 0, c]$ where $c$ is uniformly distributed in [-0.05, 0.05]. Absolute and relative tolerance for \texttt{integral2} was set to $10^{-14}$. The relative difference between the two methods is always below $10^{-10}$.}
    \label{fig:near singular analytic results}
\end{figure}
The results of these tests are shown in \autoref{fig:near singular analytic results}. 
In terms of runtime, the analytic method is on average a thousand times faster than \texttt{integral2}, which uses an adaptive refinement algorithm. This difference in time can be reduced if the absolute and relative tolerances for \texttt{integral2} are increased. However, due to the different programing languages, the wall-times cannot be directly compared. Even when calculating
\begin{equation}\label{eq: K*y for simulation}
    I = \int_{\Delta} \frac{(\x-\y)\cdot \nx}{4\pi |\x - \y|^3} y_x\, dS(\y),
\end{equation}
where $y_x$ is the first component of $y$, we obtain very similar results shown in \autoref{fig:near singular y analytic results}. 
\begin{figure}
    \centering
    \includegraphics[width=1.0\linewidth]{Figures/near_singular_analytic_y_error_result.png}
    \caption{Comparison between the values of the integral \autoref{eq: K*y for simulation} using the Analytic Method and \texttt{integral2} on 2000 tests. $\Delta$ is assumed to be on the $XY$-plane with vertices uniformly distributed in $[-0.05, 0.05]^2$. $\x = [0, 0, c]$ where $c$ is uniformly distributed in [-0.05, 0.05]. Absolute and relative tolerance for \texttt{integral2} was set to $10^{-14}$. The relative difference between the two methods is always below $10^{-7}$.}
    \label{fig:near singular y analytic results}
\end{figure}

For integrals that do not have strong singularities at all, such as \autoref{eq: general integral problem green}, its similar analytic equations give the same accuracy. Though these equations give values that are very accurate, this analytic method is slower than a simple 3-point quadrature method due to the algorithm to find the bounds of integration.  
It is up to the users to decide when to use this analytic method and when to use a $n$-point quadrature method with small $n$. 

\subsection{Comparisons on singular integrals}
In the next test, we still consider \autoref{eq: K for simulation} but now $\x = (0, 0, 0)$ is one of its vertices. 
In each test case, $\Delta$ is simulated the same way as the previous tests, but now $\x$ and $\nx$ are fixed so that $\x$ is the first vertex of $\Delta$ and its normal is the normal vector of $S^1$ at $\x$. 
When running the adaptive method, we actually integrate $K(\x, \y)$ on the true triangular patch on $S^1$. This is because the integral on $\Delta$ is divergent.

\begin{figure}
    \centering
    \includegraphics[width=0.9\linewidth]{Figures/singular_error_result.png}
    \caption{Relative difference of computing the integral \autoref{eq: K for simulation} when $\x$ is a vertex of $\Delta$ on 4000 tests using different methods compared to Mathlab's adaptive method \texttt{integral2}. The Zero method means simply setting the integral to equal to zero. In the left subplot, the X-axis is the average norm of the vertices of $\Delta$. The blue dashed line represents a line of slope one, while the black dashed line has slope two. On right subplot, the X-axis is the diameter of $\Delta$. Both subplots show that the Geometric method has a much smaller relative difference compared to the other two methods, though its error depends on the size of $\Delta$. The slope of the line that givers a lower bound for the difference for the Geometric method seems to have slope one. 
    The I-D method on average performs better than the Zero method, but its relative error is high for very small or skewed triangles.  
    Absolute and relative tolerance for \texttt{integral2} was set to $10^{-8}$.} 
    \label{fig:singular error result}
\end{figure}
Since \texttt{integral2} is evaluating the true integral on a subset on $S^1$, the relative difference of the I-D method compared to \texttt{integral2} is very high. This is because a lot of accuracy is lost the moment we approximate a smooth curved domain with piecewise linear planar triangles. 
In comparison, the Geometric method is much more accurate as it incorporates geometric properties of $\partial\Omega$ into its calculations to actually evaluate on $\partial\Omega$. 

For the Geometric method, the closer the region of integration is to the origin, the better it approximates the adaptive method. Hence, larger triangles result in larger approximation errors. 
In the log-log plot, we see that the line that upper bounds the error has a slope of approximately two, which is what our analysis in \autoref{sec: singular integrals} stated.

For the I-D method, the relative difference in the I-D method compared to \texttt{integral2} seems to be more related to the shape of the triangles $\Delta$ and not its size. This is shown as the relative error seems to decrease in the right subplot of \autoref{fig:singular error result}. However, we believe that this error can be reduced if better methods of interpolating the normal vector (such as using some geometric information) are found. 
More accurate quadrature rules on the unit square can also be used to give better results. However, without information of $\partial\Omega$, we do not expect the I-D method to get as accurate as the Geometric method. 

It is also important to note that if the tolerance is decreased, \texttt{integral2} will start to give warnings. Hence, it is not known if \texttt{integral2} is the best approximation of the true integral values.
Unfortunately, analytic equations for the true solution are not available for this example on $S^1$, so we can only compare to \texttt{integral2}. 
\begin{figure}
    \centering
    \includegraphics[width=0.9\linewidth]{Figures/singular_time_result.png}
    \caption{Wall-clock time for the different methods when integrating \autoref{eq: K for simulation} on 4000 tests. The Geometric method is consistently the fastest, with the I-D method around 50 times faster than \texttt{integral2}. The Geometric and I-D methods were run on Rust while \texttt{integral2} was run in Matlab.}
    \label{fig:singular time result}
\end{figure}
In terms of runtime, \autoref{fig:singular time result} shows that the Geometric method is around ten times faster than the I-D methods. Though both methods look to be significantly faster than \texttt{integral2}, the times are not directly comparable as Matlab is not a compiled language. However, both methods are still likely to be much faster than adaptive methods as both methods require much less calculations. 
\end{document}
