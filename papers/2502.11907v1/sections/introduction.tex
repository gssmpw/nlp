\documentclass[../paper.tex]{subfiles}
\usepackage{graphicx} % Required for inserting images
\graphicspath{{\subfix{../Figures/}}}

\usepackage{amssymb}
\usepackage{amsmath}
\usepackage{amsthm}
\usepackage{hyperref}

\usepackage{biblatex} %Imports biblatex package
\addbibresource{references.bib} %Import the bibliography file

\begin{document}

Let $\Omega\subset \mathbb{R}^3$ with $C^2$ boundary. Consider the problem of integrating a function $f$ over $\partial\Omega$. Assuming that $f\in L^1(\partial\Omega)$, this is a difficult problem numerically. 
Unless there are very easy analytic formulas that represent the integral, one will have to create a quadrature scheme for this integral.
As the geometry of $\partial\Omega$ becomes more complicated, an accurate quadrature scheme will be harder and harder to find.
One common algorithm to integrate functions on arbitrary smooth boundaries is to first discretize $\partial\Omega$ into elements on which very accurate quadrature schemes exist \cite{hsiao2008boundary}. In 2D, common elements are line segments, while in 3D they are simplexes or triangles which we denote by $\Delta$. The method of approximating $\partial\Omega$ with a finite number of triangles is often referred to as triangulation, and we denote by $\partial\Omega_{\Delta}$ as the collection of those triangles. 
There are many algorithms available that can perform this triangulation \cite{persson2004simple, delaunay1934spheres} and simple implementations can be found in most commonly used programming languages. 

Now that we have approximated $\partial\Omega$ with $\partial\Omega_{\Delta}$, we can write
\begin{equation}
    I(f) = \int_{\partial\Omega}f(\x)\,dS(\x) \approx \sum_{\Delta \in \partial\Omega_{\Delta}} \int_{\Delta} f(\x)\, dS(\x).
\end{equation}
However, the function $f$ is usually weakly singular on $\partial\Omega$ and strongly singular on the triangles $\Delta$. Thus, by triangulation, we approximate a finite integral with an undefined integral.

In the field of integral operators, partial differential equations (PDEs) are reformulated into integral equations. Some of these integral equations are Fredholm integrals of the form
\begin{equation}
    u(\x) + \int_{\partial\Omega} K(\x, \y) u(\y)\, d\y.
\end{equation}
The kernel function $K(\x, \cdot)$ is usually weakly singular on $\partial\Omega$. Through triangulation, the approximations of $\partial\Omega$ using elements creates new singularities in $K$ that make it strongly singular. 

A common example of a function that is weakly integrable on $\partial\Omega$ but not on its triangulation is the normal derivative of Green's function for the Laplace equation in three dimensions, which shows up in many electrostatics problems \cite{griffith2013electrophysiology}. Letting 
\begin{equation}\label{eq: Green in 3D}
    G_{\mathbb{R}^3}(\x, \y) = -\frac{1}{2\pi |\x - \y|},
\end{equation}
we have that its normal derivative in $\x$ is 
\begin{equation}
    K_{\mathbb{R}^3}(\x, \y) = \frac{\partial G}{\partial \nx}(\x, \y) = \frac{(\x-\y)\cdot \nx}{4\pi |\x-\y|^3}.
\end{equation}

In more generality, the integrals we want to evaluate are
\begin{align}
    \int_{\Delta} G(\x, \y)p(\y)\, dS(\y) &=  \int_{\Delta} \frac{g(\x, \y)}{|\x-\y|}p(\y)\, dS(\y), \label{eq: general integral problem green} \\
    \int_{\Delta} \frac{\partial G}{\partial \nx}(\x, \y)p(\y)\, dS(\y) &= \int_{\Delta} \frac{f(\x, \y) \cdot \nx}{|\x-\y|^3}(\x, \y)p(\y)\, dS(\y)\label{eq: general integral problem gradient green}
\end{align}
where $G$ has a type $r^{-1} = |\x-\y|^{-1}$ singularity and $f$ is a smooth function. Common examples that fit this frame work is the Green's function for 3D Laplace, Helmholtz, and Screened-Poisson equations. We also assume that $p$ is some polynomial, and $\Delta$ is some triangle in $\mathbb{R}^3$. This is because in most integral equations, one represents $u$ in some basis of polynomials.

When evaluating \autoref{eq: general integral problem green} when $\x$ is very close to $\Delta$, previous papers have presented methods involving the Duffy transform \cite{duffy1982quadrature}, polar coordinate systems \cite{cai2002singularity}, and singularity extraction methods \cite{jarvenpaa2003singularity}. Though they work well for integrals of the form \autoref{eq: general integral problem green}, they fail when trying to evaluate \autoref{eq: general integral problem gradient green}. This is especially true when $\x$ is in $\Delta$. If one were to use the recursive formulas of \cite{jarvenpaa2003singularity} for \autoref{eq: general integral problem gradient green} with $\x$ being a vertex of $\Delta$, their answer would simply be zero if $p$ is a constant. This is especially problematic as most polynomial basis functions include a constant term. 

% Edit the following paragraph (See Spyros comments)
As $\x$ approaches $\y$, the denominator behaves like $r^3$. One power is canceled out from the $(\x-\y)$ in the numerator, another is canceled out from $dS(\y)$. When $\y$ approaches $\x$, the dot product $(\x - \y)$ becomes orthogonal to $\nx$. Thus, the original integrand on the $C^2$ boundary only has a removable singularity. On triangles, $\x$ being very close to $\y$ means that they are on the same triangle $\Delta$. This results in $(\x-\y)$ being on the same plane as the triangle for all $\y\in\Delta$. If $\Delta$ is not orthogonal to $\nx$, a new singularity is introduced. If $\Delta$ is orthogonal to $\nx$, then the entire integral is zero as the numerator is zero for all $\y\in\Delta$. Both cases result in very bad approximations of the true integral. 
Even if $\x$ does not lie on $\Delta$, just by being close enough, the integral can numerically appear singular, making the error large.

The fundamental problem of the integral \autoref{eq: general integral problem gradient green} when $\x$ is in $\Delta$ is that we are trying to approximate the true weakly singular integral $I_{\text{true}}$ with a strongly singular integral $I = \infty$. Previous analytic methods \cite{jarvenpaa2003singularity} then approximate this divergent integral by computing only its finite part. The Interpolation-Duffy (I-D) method approximates this divergent integral by computing an similar but augmented integral that is only weakly singular integral on a triangle. What both cases are trying to do is to approximate infinity with a finite value that hopefully is close to $I_{\text{true}}$. 
Instead of approximating an ``infinity" that was intended to approximate $I_{\text{true}}$, we introduce a new evaluation method, which we call the geometric method, that directly approximates $I_{\text{true}}$ by using geometric information of the true domain $\partial\Omega$. 

The paper is divided into 3 parts. In \autoref{sec: near singular integrals}, we introduce analytical formulas that can integrate \autoref{eq: general integral problem gradient green} when $\x$ is near $\Delta$ but not on it. In \autoref{sec: singular integrals}, we present two different methods to handle the strongly singular integrals of \autoref{eq: general integral problem gradient green} when $\x$ is a vertex of $\Delta$. In \autoref{sec: results} we give numerical simulations of our new methods and see how they compare to the standard adaptive quadrature method of MATLAB. Finally, we end the paper with an appendix that gives explicit formulas of some integrals that arise in our methods for low degrees of $p$.

% Add link here later
An implementation of our new methods in Rust is available on \href{https://github.com/AfZheng126/AnalyticIntegrals}{Github}, and instructions on how to call it using C or Matlab are provided in the repository. 

Through this paper, we use the notation that underlined variables $\x$ represents vectors, while $x$ represents a real value. 
We will use $\Delta$ to represent triangles instead of the Laplacian operator.

\end{document}


