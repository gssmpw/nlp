\section{Introduction}
The rapid advancement of Large Language Models (LLMs) agents has enabled their widespread deployment in various applications, including high-stakes domains such as finance~\cite{DBLP:conf/aaaiss/YuLCJLZLSK24}, healthcare~\cite{DBLP:journals/corr/abs-2401-05654}, autonomous driving~\cite{DBLP:journals/itsm/CuiMCYW24}, and chemical laboratories handling hazardous materials~\cite{DBLP:journals/natmi/BranCSBWS24}. 
These agents use LLMs for processing and external tools for executing actions.

While the external tools expand the capabilities of LLMs, they also introduce risks, such as the threat of indirect prompt injection (IPI) attacks~\cite{DBLP:conf/acl/ZhanLYK24,DBLP:conf/ccs/AbdelnabiGMEHF23}. 
In an IPI attack, adversaries embed malicious instructions into external data sources accessed by the agent, aiming to manipulate its behavior. 
Such attacks are especially dangerous due to their ease of execution and potential to cause significant harm, including unauthorized transactions, data leaks, or even physical damage.




Given the severity of these risks,  it is crucial to develop effective defense mechanisms against IPI attacks.
A robust defense must not only withstand current threats but also anticipate future adaptive attacks—those specifically designed after the defense is fully disclosed~\cite{DBLP:conf/icml/AthalyeC018, DBLP:conf/icml/MazeikaPYZ0MSLB24}.
In standard computer security and ML security, adaptive attacks serve as a standard approach to test the reliability of defenses~\cite{katz2007introduction,DBLP:conf/nips/TramerCBM20}.

Prior studies have shown that non-adaptive attacks can greatly underestimate a system's vulnerabilities, as defenses that seem robust under these attacks may be entirely compromised by adaptive ones, drastically reducing accuracy and exposing a false sense of security~\cite{DBLP:conf/icml/AthalyeC018}.
While defenses against IPI attacks have been proposed~\cite{DBLP:journals/corr/abs-2312-14197}, no studies have yet explored their effectiveness against adaptive attacks, leaving their robustness in question.

\section{System}\label{sec:system}
We consider systems in the form
%
\begin{subequations}\label{eq:system}
	\begin{align}
		\label{eq:system:x0}
		x(t) &= x_0(t), & t &\in (-\infty, t_0], \\
		%
		\label{eq:system:x}
		\dot x(t) &= f(x(t), z(t)), & t &\in [t_0, t_f],
	\end{align}
\end{subequations}
%
where $t \in \R$ is time, $t_0, t_f \in \R$ are the initial and final time, $x: \R \rightarrow \R^{n_x}$ is the state, and $x_0: \R \rightarrow \R^{n_x}$ is the initial state function. Furthermore, $f: \R^{n_x} \times \R^{n_z} \rightarrow \R^{n_x}$ is the right-hand side function, and the memory state, $z: \R \rightarrow \R^{n_z}$, is given by the convolution
%
\begin{subequations}\label{eq:system:delay}
	\begin{align}
		\label{eq:system:z}
		z(t) &= \int\limits_{-\infty}^t \alpha(t - s) \odot r(s) \incr s, \\
		%
		\label{eq:system:r}
		r(t) &= h(x(t)),
	\end{align}
\end{subequations}
%
where $r: \R \rightarrow \R^{n_z}$ is the delayed variable, and each element of $\alpha: \Rnn \rightarrow \Rnn^{n_z}$ is a \emph{regular} kernel (see Definition~\ref{def:regular:kernel}). Furthermore, $h: \R^{n_x} \rightarrow \R^{n_z}$ is the memory function. We assume that $f$ and $h$ are differentiable in their arguments, and we refer to the paper by Ponosov et al.~\cite[Thm.~1]{Ponosov:etal:2004} for more details on the existence and uniqueness of solutions to the initial value problem~\eqref{eq:system}--\eqref{eq:system:delay}. See also the book by Hale and Lunel~\cite{Hale:Lunel:1993}.
%
\begin{definition}\label{def:regular:kernel}
	A scalar-valued kernel, $\alpha: \Rnn \rightarrow \Rnn$, is \emph{regular} if it satisfies the following properties.
	%
	\begin{enumerate}
		\item It is non-negative and bounded, i.e., $0 \leq \alpha(t) \leq K$ for all $t \in \Rnn$ and for some finite $K \in \Rp$.
		%
		\item It is continuous, i.e., for all $\epsilon \in \Rp$ and $t \in \Rnn$, there exists a $\delta \in \Rp$ such that $|\alpha(s) - \alpha(t)| < \epsilon$ for all $s \in \Rnn$ satisfying $|s - t| < \delta$.
		%
		\item It is normalized such that
	\end{enumerate}
	%
	\begin{align}\label{eq:kernel:normalization}
		\int\limits_0^\infty \alpha(t) \incr t &= 1.
	\end{align}
\end{definition}
%
For a given system of DDEs with distributed time delays, each element of $\alpha$ may not satisfy~\eqref{eq:kernel:normalization}. However, as they are assumed to be nonzero and non-negative, it is straightforward to normalize them. Next, we present a few well-known corollaries about the steady states of~\eqref{eq:system:x}--\eqref{eq:system:delay} and their stability.
%
\begin{corollary}\label{thm:steady:state}
	A state $\bar x \in \R^{n_x}$ is a steady state of the system~\eqref{eq:system:x}--\eqref{eq:system:delay} if
	%
	\begin{align}\label{eq:steady:state}
		0 &= f(\bar x, \bar z), &
		\bar z &= \bar r = h(\bar x).
	\end{align}
\end{corollary}

\begin{proof}
	In steady state, $x(t) = \bar x$ for all $t$. Consequently, $r(t) = \bar r = h(\bar x)$ and
	%
	\begin{align}
		z(t)
		&= \int_{-\infty}^t \alpha(t - s) \odot \bar r \incr s
		 = \int_{-\infty}^t \alpha(t - s) \incr s \odot \bar r
		 = \bar r,
	\end{align}
	%
	for all $t$, where we have used the property~\eqref{eq:kernel:normalization} of each element of $\alpha$.
\end{proof}
%
\begin{corollary}\label{thm:stability}
	The system~\eqref{eq:system:x}--\eqref{eq:system:delay} is locally asymptotically stable around a steady state, $\bar x$, satisfying~\eqref{eq:steady:state} if $\real \lambda < 0$ for all $\lambda \in \C$ that satisfy the characteristic equation
	%
	\begin{align}\label{eq:characteristic:equation}
		\det\left(F - \lambda I + G \int_0^\infty e^{-\lambda s} \diag \alpha(s) \incr s H\right) = 0,
	\end{align}
	%
	where $I \in \R^{n_x \times n_x}$ is an identity matrix.
	%
	The matrices $F \in \R^{n_x \times n_x}$, $G \in \R^{n_x \times n_z}$, and $H \in \R^{n_z \times n_x}$ are the Jacobians of the right-hand side function and the delay function evaluated in the steady state:
	%
	\begin{align}\label{eq:jacobians}
		F &= \pdiff{f}{x}(\bar x, \bar z), &
		G &= \pdiff{f}{z}(\bar x, \bar z), &
		H &= \pdiff{h}{x}(\bar x).
	\end{align}
	%
\end{corollary}

\begin{proof}
	The linearized system corresponding to~\eqref{eq:system:x}--\eqref{eq:system:delay} describes the evolution of the deviation variable $X: \R \rightarrow \R^{n_x}$:
	%
	\begin{align}\label{eq:linearized:system}
		\dot X(t) &= F X(t) + G \int_{-\infty}^t \alpha(t - s) \odot H X(s) \incr s, &
		X(t) &= x(t) - \bar x.
	\end{align}
	%
	See, e.g., \cite{Cushing:1975, Cushing:1977, Miller:1972} for proofs of the condition~\eqref{eq:characteristic:equation} for asymptotic stability of the linearized system in~\eqref{eq:linearized:system}.
\end{proof}

\section{Defenses Against Timing Attacks}
This section examines defense mechanisms aimed at mitigating the vulnerabilities faced by the timing stack's three layers. Table~\ref{tab:defense-examples} provides representative examples of this work and the extent to which it addresses timing stack issues. In this analysis, we focus exclusively on system-based solutions for safeguarding timing architectures, deferring the discussion of theory-based approaches for the next section.

\begin{table*}[t]
\footnotesize
\centering
\begin{tabular}{ p{4.75cm}  p{1cm}  p{1.25cm}  p{1.25cm}  p{1.25cm}  p{1.5cm}  p{1.5cm}  p{1.5cm}  }
 \multicolumn{1}{c}{} & \multicolumn{2}{c}{Hardware Issues} & \multicolumn{2}{c}{Software Issues} & \multicolumn{3}{c}{Network Issues}\\
 \cmidrule(lr){2-3} \cmidrule(lr){4-5} \cmidrule(lr){6-8}
    & \textit{Side Ch.} & \textit{Flaw. Des.} & \textit{Soft. Bugs} & \textit{TEE Issues} & \textit{Auth. Issues} & \textit{Avail. Issues} & \textit{Impl. Issues} \\
 \hline
 Wei et. al.~\cite{lfi-ro-based} & \halfcirc & \emptycirc & \emptycirc & \emptycirc & \emptycirc & \emptycirc & \emptycirc \\
 Arm Generic Timer~\cite{arm-generic-timer} & \emptycirc & \halfcirc & \emptycirc & \emptycirc & \emptycirc & \emptycirc & \emptycirc \\
 TPM Counters~\cite{ftpm} & \emptycirc & \fullcirc & \emptycirc & \emptycirc & \emptycirc & \emptycirc & \emptycirc \\
 \hline
 Timeseal~\cite{time-stack-timeseal} & \emptycirc & \emptycirc & \halfcirc & \halfcirc & \emptycirc & \emptycirc & \emptycirc \\
 T3E~\cite{trusted-time-t3e} & \emptycirc & \emptycirc & \halfcirc & \halfcirc & \emptycirc & \emptycirc & \emptycirc \\
Scone~\cite{sandbox-scone} & \emptycirc & \emptycirc & \emptycirc & \halfcirc & \emptycirc & \emptycirc & \emptycirc \\
SeCloak~\cite{sandbox-secloak} & \emptycirc & \emptycirc & \emptycirc & \halfcirc & \emptycirc & \emptycirc & \emptycirc \\
\hline
Cryptographic Communications~\cite{net-sync-ptp-sec, ntpv4-rfc} & \emptycirc & \emptycirc & \emptycirc & \emptycirc & \halfcirc & \emptycirc & \emptycirc \\
Chronos~\cite{net-sync-chronos} & \emptycirc & \emptycirc & \emptycirc & \emptycirc & \halfcirc & \fullcirc & \emptycirc \\
Semperfi et. al.~\cite{gps-anti-spoofing-semperfi} & \emptycirc & \emptycirc & \emptycirc & \emptycirc & \halfcirc & \fullcirc & \emptycirc \\
\hline
\end{tabular}
\caption{Examples of representative papers that propose mitigation for timing stack issues (\textit{Ixx}). 
A full circle indicates the research paper mitigates all issues in the category, a half circle indicates that some of the issues in a category are addressed and an empty circle signifies the lack of proposed mitigation for the given category. Note that implementation issues ($I17$) are not addressed by system-based approach as they arise from errors in execution of this approach itself.}
\label{tab:defense-examples}
\end{table*}

\subsection{Securing the Hardware}
We begin by highlighting design solutions that address issues arising from the physical side channel and the design limitations of the timing circuitry.

\noindent\textbf{\texttt{D01.} Laser Fault Injection Countermeasures.} In response to the rising threat of laser fault injection (LFI), a significant body of research has focused on detection techniques~\cite{lfi-multi-spot, lfi-tdc, lfi-ro-based}. These efforts concentrate on identifying lasers incident on the SoC and provide countermeasures against laser-based computational faults ($I02$). However, these methods may not detect laser-based attacks on crystal oscillators which are external to the SoC. Nevertheless, they offer a promising starting point for developing countermeasures against laser attacks on crystal oscillators ($I01$). For instance, He et al. introduced a ring oscillator-based watchdog that analyzes clock signal irregularities to detect LFI attacks~\cite{lfi-ro-based}. Such solutions can be further developed to detect attacks on the external oscillators by analyzing the analog clock signal generated by them.

\noindent\textbf{\texttt{D02.} Monotonic and Fixed Frequency Counters.} Many contemporary System-on-Chips (SoCs) incorporate specialized counters that are monotonic i.e. consistently counting in a single direction and immune to resets. Further, the frequency of these counters remains independent of the DVFS mechanism, thwarting any attempt by a privileged adversary to exploit energy management interfaces for attacks on the timing stack ($I03$). ARM's generic timer~\cite{arm-generic-timer} and Intel's TSC are notable examples~\footnote{when virtualization extensions are disabled.}~\cite{intel-tsc}. Monotonic counters, uninfluenced by DVFS, are also prevalent in Trusted Platform Modules (TPMs) embedded in modern SoCs\cite{ftpm}. However, TPMs, situated outside processor chips, suffer from substantial access latency, constraining their utility~\cite{time-stack-timeseal}. These fixed rate monotonic counters present a substantial advance towards fixing hardware design issues that could enable timing attacks. However, despite these advancements, the challenge of counter manipulation still persists on legacy hardware and modern systems that incorporate virtualization extensions (see $I04$). 

\subsection{Software Defenses}
This subsection delves into solutions designed to protect the integrity of \textit{local clock}'s data. 

\noindent\textbf{\texttt{D03.} Trusted Timing Services.} Development efforts have been concentrated on integrating trusted timing services within TEEs to counter vulnerabilities in system software and device drivers ($I05, I06$). For instance, ARM's TrustZone provides a privileged TEE with direct access to a secure counter and timer~\cite{arm-generic-timer}, enabling TEE software to maintain a \textit{trusted local clock}. Intel SGX, a user-space TEE, benefits from solutions like Timeseal~\cite{time-stack-timeseal}, which secures the enclave's trusted timing API against delay attacks, and T3E~\cite{trusted-time-t3e}, utilizing secure TPM counters to provide trusted timing within the SGX enclave. These approaches for constructing a trusted timing stack are not confined to Intel and ARM-based TEEs but are extensible to other TEE architectures. Despite their significance, these trusted timing solutions exhibit several notable limitations: (i) the application code requiring trusted time must execute within the TEE's isolated environments, which enlarges the Trusted Computing Base (TCB), thereby increasing the security risk to the TEE software\footnote{TCB refers to the code and data that reside inside a TEE.}; and (ii) crucially, these trusted timing solutions do not incorporate secure time-synchronization services.

\noindent\textbf{\texttt{D04.} Trusted I/O.} Researchers have proposed several solutions to enable user-space TEEs' (Figure~\ref{fig:userspace-tee}) direct access to I/O devices ($I08$). One such effort, Aurora~\cite{sandbox-aurora}, allows Intel SGX to access high-resolution counters in the hardware, among other peripherals. SGX enclave's access to these counters can improve trusted time services such as Timeseal~\cite{time-stack-timeseal} when integrated with it. Other initiatives like Scone~\cite{sandbox-scone} and SGXIO~\cite{sgxio} provide secure network I/O to the SGX enclave, essential for synchronizing trusted time stacks inside TEEs with network time. However, they only safeguard the integrity and confidentiality of network packets, falling short of preventing delay attacks by untrusted system software.

\subsection{Secure Time Synchronization}
Time synchronization security has garnered significant focus within timing security research. Efforts in this domain have concentrated on enhancing various aspects of time-sync protocols to fortify them against adversarial actions.

\noindent\textbf{\texttt{D05.} Cryptographic Communications.} Time-sync protocols can enhance their security against packet manipulation attacks ($I10$, $I11$) by utilizing authentication and encryption, provided they adhere to the following conditions: (i) cryptographic functions should fully protect network packets~\cite{net-sync-ptp-covert-channel}, and (ii) their operation should not require time synchronization as a prerequisite~\cite{ntp-replay-drop-attack}. For example, the authentication and encryption mechanisms introduced by the NTS standard (RFC8915~\cite{nts-rfc}) for NTP's client-server mode fulfill both criteria. Conversely, PTP's authentication mechanisms fail to meet these conditions because: (i) the correction field in the PTP packet header remains unprotected by authentication, and (ii) the recommended authentication protocol, TESLA, requires devices to be loosely synchronized. Despite these shortcomings, such security measures constitute the primary defense against network adversaries and should be implemented by time-sync protocols to either partially or completely mitigate packet manipulation attacks. It is important to note that packet manipulation by malicious time servers~\cite{shark-ntp-pool} remains feasible and necessitates further defense mechanisms.

\noindent\textbf{\texttt{D06.} Multipath Time Transfer.} Delay attacks $I14$ cannot be mitigated using cryptographic mechanisms and must be prevented using other techniques. Using multiple paths for synchronization between two systems is one strategy to mitigate these attacks. This approach forces the attacker to identify and introduce delay along all the time-sync paths which is a significantly more challenging task. Mizrahi et al. propose a game-theoretic model for such a multipath time synchronization scheme~\cite{multi-path-game-theory}. Similarly, Chronos~\cite{net-sync-chronos} employs multi-path synchronization by querying reference time from various NTP servers. This approach is not only effective in mitigating delay attacks $I14$ but it also mitigates packet drop $I15$ attacks. However, the multi-path approach is ineffective against a malicious device that functions as bottleneck on the path between the client and the time server(s).

\noindent\textbf{\texttt{D07.} Algorithmic Updates.} Recent research has focused on algorithmic approaches for mitigating delay ($I14$) and replay ($I12$) attacks. For instance, Fatima et al.\cite{net-sync-feedforward} present a feedforward clock model, for PTP, along with an algorithm proficient in detecting delay-free packets. They remove rest of the packets that were potentially delayed or replayed by an adversary before estimating time-sync parameters (offset and skew). Likewise, Chronos\cite{net-sync-chronos} reinforces the multi-path time-sync approach using a byzantine fault tolerance-based algorithm for the random selection of time servers in each synchronization round. Adopting these algorithmic updates will enhance time-sync protocol's resiliency against delay attacks, but it does not completely prevent performance degradation~\cite{net-sync-feedforward}.

\noindent\textbf{\texttt{D08.} GPS Anti-jamming \& spoofing.} The prevalence of GPS jamming ($I16$) and spoofing ($I13$) attacks against critical navigation systems have motivated a large number of studies. These works have demonstrated the ability to acquire weak GPS signals amidst jamming, allowing for an average positioning error of 16m~\cite{gps-anti-jamming-post-correlation, gps-anti-jamming-post-wavelet}. Addressing GPS spoofing, Khalajmehrabadi et al. present detection techniques that estimate the extent of the spoofing signal, empowering devices to take necessary mitigation measures~\cite{gps-anti-spoofing-tsarm}. Building on this work, Lee et al. introduced techniques to prevent GPS spoofing for static receivers~\cite{gps-anti-spoofing-static}. And Semperfi et. al.~\cite{gps-anti-spoofing-semperfi} developed anti-spoofing technique for mobile GPS receivers, such as UAVs, enhancing robustness for diverse navigation systems. Location and time information in GPS signals is tightly coupled. It means that these techniques mitigate both location and time-sync errors under adversarial conditions.
To address this gap, we conduct a comprehensive evaluation of existing IPI defenses by testing their resilience to adaptive attacks.
Our goal is to find adversarial attack methods to compromise these defenses. 
In the context of IPI attacks, the attacker can only manipulate the content of external sources, such as reviews or emails~\cite{DBLP:conf/acl/ZhanLYK24,DBLP:conf/ccs/AbdelnabiGMEHF23}. 
Therefore, adaptive attacks in this setting involve crafting adversarial examples in the external content to manipulate the LLM agent. 
This is similar to adversarial attacks in jailbreak attacks of LLMs, where the goal is to bypass models' safety alignment and trigger harmful outputs
~\cite{DBLP:journals/corr/abs-2307-15043, DBLP:conf/iclr/LiuXCX24, zhu2023autodan}.
Building on this, we leverage strong attack strategies from jailbreak settings to design adaptive attacks in the IPI context. 


We implement eight different defenses against IPI attacks on two types of LLM agents and design adaptive attacks to expose their vulnerabilities.  
Our results show that adaptive attacks consistently achieve success rates above 50\% across the targeted defenses and LLM agents—exceeding the ASR before deploying any defense and significantly outperforming non-adaptive attacks.  
These findings reveal weaknesses in current defense strategies and emphasize the need to test defenses against adaptive attacks to ensure robustness and reliability.






