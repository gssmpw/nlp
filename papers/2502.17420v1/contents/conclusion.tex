\section{Conclusion}
This work advances the understanding of refusal mechanisms in LLMs by introducing gradient-based representation engineering as a powerful tool for identifying and analyzing refusal directions. Our method yields more effective refusal directions with fewer side effects, demonstrating its viability for extracting meaningful structures while allowing for greater modeling flexibility.
We establish that refusal behaviors can be better understood via high-dimensional polyhedral cones in activation space rather than a single linear direction, highlighting their complex spatial structures. Additionally, we introduce representational independence and show that within this space of independent directions multiple refusal directions exist and correspond to distinct mechanisms.
Our gradient-based representation engineering approach can be extended to identify various concepts beyond refusal by simply changing the optimization targets.
The generated findings provide new insights into the geometry of aligned LLMs, highlighting the importance of structured, gradient-based approaches in LLM interpretability and safety. %
