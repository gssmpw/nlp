\documentclass{article}
\usepackage[T1]{fontenc} % add special characters (e.g., umlaute)
\usepackage[utf8]{inputenc} % set utf-8 as default input encoding
\usepackage{ismir,amsmath,cite,url}
\usepackage{graphicx}
\usepackage{color}
\usepackage{booktabs}
\usepackage{lineno}
\usepackage[bookmarks=false]{hyperref}
\hypersetup{colorlinks=true,urlcolor=blue}

%\linenumbers

% ---------------------------------------------------------------------------
\title{Exploring Internet Radio Across the Globe with the MIRAGE Online Dashboard}

% \multauthor
%     {Ngan V.T. Nguyen\textsuperscript{1,2} \hspace{1cm} Elizabeth A.M. Acosta\textsuperscript{2} \hspace{1cm} Tommy Dang\textsuperscript{2} \hspace{1cm} David R.W. Sears\textsuperscript{2}} {\textsuperscript{1}Vietnam National University, Ho Chi Minh City, Vietnam \\ \textsuperscript{2}Texas Tech University, Lubbock TX, USA}

\fourauthors
  {Ngan V.T. Nguyen} {University of Science, VNU-HCMUS \\ {\tt \small nvtngan@hcmus.edu.vn}}
  %{Ngan V.T. Nguyen} {University of Science (VNU-HCMUS)\\Vietnam National University \\ {\tt nvtngan@hcmus.edu.vn}}
  {Elizabeth A.M. Acosta} {Texas Tech University\\{\tt \small liz.acosta@ttu.edu}} 
  {Tommy Dang} {Texas Tech University \\ {\tt \small tommy.dang@ttu.edu}}
  {David R.W. Sears} {Texas Tech University\\{\tt \small david.sears@ttu.edu}}
  
\def\authorname{Nguyen et al.}

\sloppy % please retain sloppy command for improved formatting

\begin{document}

%
\maketitle
%
\begin{abstract}
%Radio remains a central site of culture and daily life for communities around the world, but its recent convergence with the internet has yet to receive significant attention in the music research community. To address this issue, 
This study presents the \textit{Music Informatics for Radio Across the GlobE} (MIRAGE) online dashboard, which allows users to access, interact with, and export metadata (e.g., artist name, track title) and musicological features (e.g., instrument list, voice type, key/mode) for 1 million events streaming on 10,000 internet radio stations across the globe. Users can search for stations or events according to several criteria, display, analyze, and listen to the selected station/event lists using interactive visualizations that include embedded links to streaming services, and finally export relevant metadata and visualizations for further study. %In doing so, the dashboard allows researchers with potentially little training in computational methods to select and analyze a subset of stations or songs (i.e., to develop their own sub-corpora) for a medium that places diversity center stage.
\end{abstract}

\section{Introduction}\label{sec:introduction}

Despite its scholarly neglect relative to television, film, and print \cite{Lacey2018}, radio’s convergence with the internet has extended its reach via web browsers and smartphone apps, enabling the medium to persist as a central site of culture and daily life for communities around the world \cite{Bottomley2020, Glantz2016}. The recent resurgence of pirate and community radio stations on the internet alongside national and multinational networks also reflects internet radio’s lower production costs relative to short-wave terrestrial (e.g., FM or AM) radio \cite{Wall2004}, resulting in a diverse range of both standardized and specialized programming \cite{Chambers2003, Hendy2000, Uimonen2017}. 

And yet, the volume and scope of much of the research in fields like radio studies has been freighted heavily towards the Global North \cite{Lacey2018}. %, owing to entrenched postcolonial legacies and their effects on the differential access to archives and the overwhelming dominance of Western frames of reference and theory \cite{Lacey2018}. %These problems are perhaps even more acute for music theory, a discipline that often restricts its purview to the traditions found either in European music from about 1600 to around 1950 (the period of “common practice”; \cite{Hyer2002}), or (more recently) in Anglophone popular music from the latter half of the twentieth century \cite{Ewell2020}. 
In doing so, the research program just described attempts to situate listeners within a particular musical tradition (e.g., western classical or popular music), rather than within a particular geographic environment (e.g., El Paso, Texas) where myriad musical traditions might co-exist. As a result, music's vast global marketplace has yet to receive sustained scholarly attention in the MIR community. 

To address this issue, this study presents the development release (v0.2) of the \textit{Music Informatics for Radio Across the GlobE} (MIRAGE) online dashboard, which allows users with potentially little training in computational methods to access, interact with, and export metadata (e.g., artist name, track title) and musicological features (e.g., instrument list, voice type, key/mode) for 1 million events streaming on 10,000 internet radio stations across the globe. To that end, Section~\ref{sec:background} summarizes previous research on the development of digitized music corpora and cultural databases. Next, Section~\ref{sec:metacorpus} presents the MIRAGE-MetaCorpus, Section~\ref{sec:dashboard} introduces the MIRAGE online dashboard, and Section~\ref{sec: use case} offers a potential use case. Finally, Section~\ref{sec:conclusion} discusses limitations and future directions for the MIRAGE project.  

\section{Previous Research}\label{sec:background}

In recent years, researchers in music theory, music information retrieval (MIR), and radio/media studies have developed digitized music corpora and cultural databases that represent data in machine-readable symbolic and audio formats. 

\begin{figure*}[t!]
    \centering
    \begin{minipage}[t!]{.45\linewidth} 
        \centering
        \begin{tabular}{rrrrr} % Table generated by Excel2LaTeX from sheet 'MIRAGE_descriptivestats'
        \toprule
              & \multicolumn{2}{c}{\textbf{Radio}} & \multicolumn{2}{c}{\textbf{Sovereign}} \\
        \multicolumn{1}{l}{\textbf{Continent}} & \multicolumn{2}{c}{\textbf{Stations}} & \multicolumn{2}{c}{\textbf{States}} \\
        \midrule
        \multicolumn{1}{l}{Africa} &        392  & \multicolumn{1}{l}{(4\%)} & 39    & \multicolumn{1}{l}{(58\%)} \\
        \multicolumn{1}{l}{Asia} &        653  & \multicolumn{1}{l}{(7\%)} & 38    & \multicolumn{1}{l}{(59\%)} \\
        \multicolumn{1}{l}{Europe} &     5,161  & \multicolumn{1}{l}{(52\%)} & 49    & \multicolumn{1}{l}{(60\%)} \\
        \multicolumn{1}{l}{North America} &     2,243  & \multicolumn{1}{l}{(22\%)} & 33    & \multicolumn{1}{l}{(67\%)} \\
        \multicolumn{1}{l}{Oceania} &        222  & \multicolumn{1}{l}{(2\%)} & 5     & \multicolumn{1}{l}{(17\%)} \\
        \multicolumn{1}{l}{South America} &     1,329  & \multicolumn{1}{l}{(13\%)} & 13    & \multicolumn{1}{l}{(87\%)} \\
        \textbf{TOTAL} &   10,000  &       & 177   &  \\
        \bottomrule
        \end{tabular}%
    \end{minipage}
    \hspace{2em}
    \begin{minipage}[t!]{.45\linewidth}
        \centering
        \includegraphics[scale=.058]{figs/MIRAGE_map.png}
    \end{minipage}
    \caption{Descriptive statistics (left) and geographic map (right) of the radio stations in MIRAGE-MetaCorpus. The size of each bubble represents the number of stations at that location.}
    \label{fig:tableANDmap}
\end{figure*}

In computational music theory, heavily curated corpora (100s of songs) like the McGill Billboard and Rolling Stone-200 data sets include expert annotations for musical parameters like harmony, meter, and melody, for example, but remain restricted to Anglophone popular music traditions \cite{Burgoyne2011,Declercq2011}. What is more, the limited size of symbolic corpora makes comparative research especially difficult \cite{Sears2021}. 

In MIR, corpora like the Million Song data set address issues of scale while avoiding copyright infringement by providing researchers with publicly available metadata and musicological features protected under fair use for a large collection of songs hosted on commercial music-streaming services like last.fm \cite{Bertin-Mahieux2011}. Nevertheless, the size, scope, and format of these projects require extensive training in distant-reading (i.e., computational) methods \cite{Aizenberg2012, Dang2012, Silva2017}. As a result, MIR corpora sometimes eschew the kinds of musical engagements favored by scholars in humanities disciplines using close-reading methodologies. Finally, the projects referenced above do not include information about the geographic location of the music encountered by listeners in everyday life.

%In comparative musicology, the Garland Encyclopedia of World Music \cite{Savage2015}, the Natural History of Song Discography \cite{Mehr2019}, and the Global Jukebox \cite{Wood2022} offer cross-cultural databases of music that include annotations for numerous aspects of musical style. The Global Jukebox is the largest of these, consisting of annotations for nearly 6,000 traditional songs from over 1,000 societies \cite{Lomax1968, Savage2018}. However, the principal motivation for these projects was to allow researchers to examine the relationships between performance style and social interaction through analysis of annotated datasets of hand-coded variables. In doing so, these projects explore the (folk) musical traditions indigenous to specific human societies, rather than those encountered on internet radio, which may include (but not be limited to) those indigenous traditions.

Finally, in radio/media studies, researchers routinely employ interview and survey methodologies to explore radio stations across the globe \cite{AlaFossi2008, Lind1999}, in some cases by selecting samples from radio-station directories hosted online \cite{Kuhn2011}. The now defunct ComFM, for example, included a catalogue of web-radio stations classified according to geographic region and type of programming. Other current internet radio directories like radio.co and internet-radio.com offer searchable databases consisting of several thousand stations, but they do not permit users to access or export the entire database for further analysis. 

The MIRAGE online dashboard addresses these issues by offering a global archive of the musical traditions encountered on internet radio. For this reason, the dashboard's database could serve MIR tasks like music recommendation and genre classification, but the dashboard itself also allows researchers with potentially little training in computational methods to select and analyze a subset of events or stations (i.e., to develop their own sub-corpora). Finally, like previous MIR projects \cite{Bertin-Mahieux2011}, the MIRAGE online dashboard avoids copyright infringement by including publicly available metadata and musicological features protected under fair use while enabling users to stream recordings using embedded links to commercial services like Spotify and YouTube. 

%By comparison, the MIRAGE dashboard will include station-level, human annotations (e.g., city, country, station format) for 10,000 radio stations.   

%The purpose of the MIRAGE project is to embrace the diversity of musical traditions on internet radio – including (but not limited to) those represented in the aforementioned databases – as well as the (geographic) contexts in which contemporary listeners might hear them. 

\section{MIRAGE MetaCorpus}\label{sec:metacorpus}

The core database for the MIRAGE online dashboard is MIRAGE-MetaCorpus, which currently consists of metadata and musicological features for 1 million events that streamed on 10,000 internet radio stations across the globe. In this context, an `event’ could represent a musical work of some kind, or a radio program like a podcast or a call-in show. %To avoid confusion, however, we will refer to events as `songs' throughout this paper.

\subsection{Collecting MetaData}

Following \cite{Aizenberg2012}, data collection consisted of three stages: station-list and event-list collection (\textit{Stage 1}), station-list review (\textit{Stage 2}), and event-list parsing (\textit{Stage 3}).

\subsubsection{Stage 1: Collecting Station/Event Lists}
Toward Stage 1, the research team collected metadata for an initial list of internet radio stations and then monitored the station streams to obtain additional metadata from the stream encoder. To that end, we monitored radio stations in real time on Radio Garden,\footnote{\href{https://radio.garden}{https://radio.garden}.} a streaming service with an open-access application programming interface (API) that allows users to select and play publicly available radio streams using an interactive representation of the globe. %Station-list metadata include each station's name, format, form, frequency (if applicable), and streaming url, among others.


Between the months October to January 2022-2023, a random sample of 10,000 stations from the initial station list was monitored throughout the 24-hour day -- but avoiding each ten-minute period at the top and bottom of the hour when advertising is most frequent -- in order to obtain additional metadata from the stream encoder for 100 events from each station, resulting in an initial list of 1 million events. The monitoring algorithm also excluded an event if the stream description did not include metadata, or if the metadata featured advertising terms or reflected a station blackout period (e.g., `advert', `commercial', `unknown', `blackout', etc.).%Metadata obtained from the stream encoder included the station's name and description, along with information about the content of the stream itself, such as the stream's content description (e.g., artist name, track title), codec, bitrate, and framerate. The monitoring algorithm also excluded a song if the stream title did not include metadata, or if the metadata featured advertising terms or a reflected a station blackout period (e.g., `advert', `commercial', 'unknown', 'blackout', etc.). 

During event-list collection, additional metadata for each location in the initial station list was also included from the Natural Earth map data set,\footnote{\href{https://www.naturalearthdata.com/}{https://www.naturalearthdata.com/}.} which provides public-domain vector and map raster data along with accompanying metadata.  

Shown in Figure \ref{fig:tableANDmap}, the selected station list represents 177 of the globe's 305 sovereign states. As a random sample of Radio Garden's station list (i.e., the \textit{Radio Garden sample}), this release of the MIRAGE-MetaCorpus (v0.2) therefore reflects the prevalence of internet radio stations across the globe on the Radio Garden streaming service.

% Table generated by Excel2LaTeX from sheet 'encoding'
\begin{table}[t!]
  \centering
    \small
    \begin{tabular}{l|l}
        \toprule
        \multicolumn{1}{p{11.8em}|}{\hspace{-.5em} \textbf{\texttt{<location>}}} &  \\
        \multicolumn{1}{p{11.8em}|}{\hspace{-.5em} \enspace \texttt{<city>}\textsuperscript{ab}} & \multicolumn{1}{p{11em}}{ \hspace{-.5em} Johor Bahru} \\
        \multicolumn{1}{p{11.8em}|}{\hspace{-.5em} \enspace \texttt{<country>}\textsuperscript{ab}} & \multicolumn{1}{p{11em}}{\hspace{-.5em} Malaysia} \\
        \multicolumn{1}{p{11.8em}|}{\hspace{-.5em} \enspace \texttt{<country\_GDP>}\textsuperscript{b}} & \multicolumn{1}{p{11em}}{\hspace{-.5em} 863 Billion} \\
        \multicolumn{1}{p{11.8em}|}{\hspace{-.5em} \enspace \texttt{<coordinates>}\textsuperscript{ab}} & \multicolumn{1}{p{11em}}{\hspace{-.5em} 103.6545°, 1.4783°} \\
        \multicolumn{1}{p{11.8em}|}{\hspace{-.5em} \enspace \textbf{\texttt{<station>} }} &  \\
        \multicolumn{1}{p{11.8em}|}{\hspace{-.5em} \enspace \enspace \texttt{<name>}\textsuperscript{cd} } & \multicolumn{1}{p{11em}}{\hspace{-.5em} Best FM } \\
        \multicolumn{1}{p{11.8em}|}{\hspace{-.5em} \enspace \enspace \texttt{<form>}\textsuperscript{cd}} & \multicolumn{1}{p{11em}}{\hspace{-.5em} Simulcast (FM 104.1)} \\
        \multicolumn{1}{p{11.8em}|}{\hspace{-.5em} \enspace \enspace \texttt{<format>}\textsuperscript{d}} & \multicolumn{1}{p{11em}}{\hspace{-.5em} Adult Contemporary} \\
        \multicolumn{1}{p{11.8em}|}{\hspace{-.5em} \enspace \enspace \texttt{<genre>}\textsuperscript{cd}} & \multicolumn{1}{p{11em}}{\hspace{-.5em} pop, Indonesian pop} \\
        \multicolumn{1}{p{11.8em}|}{\hspace{-.5em} \enspace \enspace \texttt{<website>}\textsuperscript{cd}} & \multicolumn{1}{p{11em}}{\hspace{-.5em} \href{http://www.bestfm.com.my}{http://www.bestfm.com.my}} \\
        \multicolumn{1}{p{11.8em}|}{\hspace{-.5em} \enspace \enspace \textbf{\texttt{<event>}}} &  \\
        \multicolumn{1}{p{11.8em}|}{\hspace{-.5em} \enspace \enspace \enspace \texttt{<time@station>}\textsuperscript{c}} & \multicolumn{1}{l}{\hspace{-.5em} 12/28/2022 9:37} \\
       %\multicolumn{1}{p{11.8em}|}{\hspace{-.5em} \enspace \enspace \enspace \texttt{<url>}\textsuperscript{a}} & \multicolumn{1}{p{11em}}{\hspace{-.5em} /listen/best-fm/7MwM5caF} \\
        \multicolumn{1}{p{11.8em}|}{\hspace{-.5em} \enspace \enspace \enspace \texttt{<description>}\textsuperscript{c}} & \multicolumn{1}{p{11em}}{\hspace{-.5em} Aisha Retno – Sutera} \\
        \multicolumn{1}{p{11.8em}|}{\hspace{-.5em} \enspace \enspace \enspace \texttt{<reliability>}\textsuperscript{e}} & \multicolumn{1}{p{11em}}{\hspace{-.5em} 1} \\
        \multicolumn{1}{p{11.8em}|}{\hspace{-.5em} \enspace \enspace \enspace \textbf{\texttt{<artist>}}} &  \\
        \multicolumn{1}{p{11.8em}|}{\hspace{-.5em} \enspace \enspace \enspace \enspace \enspace \texttt{<name>}\textsuperscript{f}} & \multicolumn{1}{p{11em}}{\hspace{-.5em} Aisha Retno} \\
        \multicolumn{1}{p{11.8em}|}{\hspace{-.5em} \enspace \enspace \enspace \enspace \enspace \texttt{<type>}\textsuperscript{f}} & \multicolumn{1}{p{11em}}{\hspace{-.5em} musical artist} \\
        \multicolumn{1}{p{11.8em}|}{\hspace{-.5em} \enspace \enspace \enspace \enspace \enspace \texttt{<gender>}\textsuperscript{f}} & \multicolumn{1}{p{11em}}{\hspace{-.5em} female} \\
        \multicolumn{1}{p{11.8em}|}{\hspace{-.5em} \enspace \enspace \enspace \enspace \enspace \texttt{<country>}\textsuperscript{f}} & \multicolumn{1}{p{11em}}{\hspace{-.5em} Malaysia} \\
        \multicolumn{1}{p{11.8em}|}{\hspace{-.5em} \enspace \enspace \enspace \enspace \enspace \texttt{<genre>}\textsuperscript{f}} & \multicolumn{1}{p{11em}}{\hspace{-.5em} pop} \\
        %\multicolumn{1}{p{11.8em}|}{\hspace{-.5em} \enspace \enspace \enspace \enspace \enspace \texttt{<popularity>}\textsuperscript{f}} & \multicolumn{1}{l}{\hspace{-.5em} 44} \\
        \multicolumn{1}{p{11.8em}|}{\hspace{-.5em} \enspace \enspace \enspace \enspace \enspace \texttt{<instruments>}\textsuperscript{f}} & \multicolumn{1}{p{11em}}{\hspace{-.5em} piano, voice } \\
        \multicolumn{1}{p{11.8em}|}{\hspace{-.5em} \enspace \enspace \enspace \textbf{\texttt{<track>}}} &  \\
        \multicolumn{1}{p{11.8em}|}{\hspace{-.5em} \enspace \enspace \enspace \enspace \enspace \texttt{<title>}\textsuperscript{f}} & \multicolumn{1}{p{11em}}{\hspace{-.5em} Sutera} \\
        \multicolumn{1}{p{11.8em}|}{\hspace{-.5em} \enspace \enspace \enspace \enspace \enspace \texttt{<duration>}\textsuperscript{f}} & \multicolumn{1}{l}{\hspace{-.5em} 03:18} \\
        \multicolumn{1}{p{11.8em}|}{\hspace{-.5em} \enspace \enspace \enspace \enspace \enspace \texttt{<year\_released>}\textsuperscript{f}} & \multicolumn{1}{l}{\hspace{-.5em} 2022} \\
        \multicolumn{1}{p{11.8em}|}{\hspace{-.5em} \enspace \enspace \enspace \enspace \enspace \texttt{<key>}\textsuperscript{f}} & \multicolumn{1}{p{11em}}{\hspace{-.5em} C minor} \\
        \multicolumn{1}{p{11.8em}|}{\hspace{-.5em} \enspace \enspace \enspace \enspace \enspace \texttt{<language>}\textsuperscript{f}} & \multicolumn{1}{p{11em}}{\hspace{-.5em} Malay} \\
        \bottomrule
    \end{tabular}%
    \caption{Left: Selected variables from the encoding scheme for MIRAGE-MetaCorpus, expressed in pseudocode. Metadata were obtained from the following sources: \textsuperscript{a} Radio Garden API; \textsuperscript{b} Natural Earth map data set; \textsuperscript{c} Internet Radio Station Stream Encoder; \textsuperscript{d} Annotator Review; \textsuperscript{e} Monitoring/Matching Algorithm; \textsuperscript{f} Online Music Libraries. Right: An example of the metadata for an event in MIRAGE-MetaCorpus.}
  \label{tab:encoding}
\end{table}

\subsubsection{Stage 2: Reviewing Stations}
Toward Stage 2, a team of six human annotators began reviewing station-level metadata from the Radio Garden API and radio-station stream encoder in 2023-2024. For each station, an annotator reviewed the station's website url, station name, city, and country for incorrect/missing spelling, capitalization, punctuation, and diacritics. Next, the list of genres, formats, and terrestrial (FM/AM) station frequencies (if applicable) were reviewed and/or included using information on the station website. Finally, the annotator reviewed the corresponding event list for each station to determine the percentage of events that featured reliable stream-description metadata (i.e., artist name, track title). 

Currently, the research team has reviewed over 6,000 stations and plans to complete station-list review by 2025. %Lists featuring $>90\%$ of events with reliable stream-description metadata were then included in the final parsing stage.    

\subsubsection{Stage 3: Parsing Events}

Toward Stage 3, additional metadata were collected for each event using the Spotify and WikiData online music libraries.\footnote{\href{https://open.spotify.com}{https://open.spotify.com}; \href{https://www.wikidata.org}{https://www.wikidata.org}.} Specifically, the team queried each API using each event's stream description. The obtained list of matching queries was then filtered using a normalized edit distance measure. Query lists featuring more than one matching entry based on normalized edit distance were then ranked by release date, and the track with the oldest release date was selected.

%Together, online music libraries provided additional metadata about each artist (e.g., name, genres, number of followers, country of birth/citizenship, etc.) and track (e.g., title, duration, year released, etc.), as well as musicological features associated with the selected track (e.g., key, mode, form, danceability, etc.). 

%\begin{figure}[b!]
%  \centering
%  \begin{minipage}[t]{1\linewidth} 
%    \includegraphics[width=\textwidth,scale=.39]{figs/Multiflatform1.png}
%  \end{minipage}
%  \begin{minipage}[t]{1\linewidth} 
%    \includegraphics[width=\textwidth,scale=.39]{figs/Multiflatform2.png}
%  \end{minipage}
%  \caption{The MIRAGE online dashboard was designed to support multiple devices and platforms using user-customized layouts.}
%  \label{fig:multiplatform}
%\end{figure}

\begin{figure*}[t!]
  \centering
  \includegraphics[width=.8\textwidth,scale=.39]{figs/Overview2.png}
  \caption{An overview of the MIRAGE online dashboard (v0.2).}
  \label{fig:mdash}
\end{figure*}

\subsection{MetaData Variables}

%The current release of the MIRAGE-MetaCorpus (v0.2) contains a station list consisting of 10,000 stations and an event list consisting of 1 million events, with 100 events monitored from each station. These lists contain all collected metadata from the Radio Garden API, the Natural Earth data set, and the station's stream encoder. Currently, the research team has also reviewed ~6,000 stations and plans to complete station-list review by 2025. Finally, the entire event list has been parsed to provide additional metadata from online music libraries. 

Each event in MIRAGE-MetaCorpus includes metadata for 100 variables obtained from the Radio Garden API (RG), the Natural Earth map data set (NE), the internet radio station stream encoder (SE), annotator review (AR), or using the online music libraries WikiData (WD), MusicBrainz (MB), Spotify (SP), Musixmatch (MX), YouTube (YT), Genius (GE), and AZlyrics (AZ). Shown in Table \ref{tab:encoding}, these metadata reflect information about each event's location, station, event, artist, and track. For example, location metadata includes variables like the city, country, and geographic coordinates of the monitored event, as well as demographic data like the country's population and GDP. Station metadata includes its name, form (a webcast stream, or a stream simulcast on the internet and terrestrial radio frequencies), formats (e.g., Top 40), and the station’s website url. Event metadata includes variables like the local time when the station was monitored and the event's identifying metadata, such as the name of the artist and title of the recording. Finally, artist and track metadata include variables like the name and type of the artist, and if the artist is a group, a list of the group’s members and their demographic information (their listed genders, sexual orientations, and ethnicities), the group’s country of origin by birth and/or citizenship, the title and duration of the track, and its year of release. %Finally, all metadata were stored in UTF-8 format, which pragmatically supports any encoded natural language that can be represented by unicode characters.  

\subsection{MetaData Access \& Export}

Users may access the complete MIRAGE-Metacorpus with the online dashboard.\footnote{The MIRAGE online dashboard is available at \href{https://pearl-laboratory.github.io/mirage-mc/}{https://pearl-laboratory.github.io/mirage-mc/}.} In addition, public-domain metadata from MIRAGE-MetaCorpus are available for download in an open-access repository on Zenodo \cite{Sears2024}, which includes both the complete data set and a subset of the data set for which the metadata obtained from the station's stream encoder and the corresponding metadata provided by online music libraries was deemed a reliable match (i.e., where the normalized edit distance measure between the two metadata character strings was $\geq$.90 on a 0–1 scale).  

\section{MIRAGE online dashboard}\label{sec:dashboard}

The MIRAGE online dashboard is an open-access web application that enables users to effortlessly navigate and engage with radio-station metadata and musicological features at various levels of detail. The dashboard's layout consists of fully interactive panes displaying relevant information from MIRAGE-MetaCorpus. The dashboard is also compatible with multiple platforms and operating systems, so users may access and interact with the dashboard from any internet-connected device. %Moreover, the dashboard's intuitive user interface makes data analysis and interpretation simple for users of different technical proficiency levels. 

%The toolbar at the top allows users to control the dashboard's theme (day/night), layout (e.g., adding/deleting panes, reorganizing the position of eahch pane using an intuitive drag/drop feature), font size, and language, along with undo/redo buttons and a search bar. 


The complete technology stack of the dashboard includes Node.js for the server, MongoDB for the database, and React for the front end. This integration of technology guarantees a smooth user experience and effective data processing. Moreover, the dashboard may be tailored to accommodate individual users' distinct requirements and inclinations, rendering it a versatile instrument for analyzing radio stations. What is more, incorporating these technologies enables instantaneous data updates and interactive functionalities, thereby boosting the overall user experience over subsequent versions of the dashboard. In addition, the MIRAGE dashboard offers sophisticated search and filtering tools and the ability to export metadata and visualizations in URL, CSV, PNG, and SVG formats, allowing users to study and share data easily.  
 
\subsection{Structure \& Processing}\label{subsec:processing}

Shown in Figure~\ref{fig:mdash}, the MIRAGE dashboard's layout is divided into two groups: a toolbox on the top (A) and data-visualization panels below (B-D), making it easy for users to navigate and analyze information. The toolbox at the top includes options for language preference, panel-display customization, and searching. The data visualization panels show the data in various formats, such as charts, graphs, and tables, for straightforward interpretation and analysis. The panels can also dock to allow the user to create a customized layout, or open to another window (or undock) to permit a more detailed view suitable for multiple-screen presentations.

Shown in Figure~\ref{fig:database}, the database is partitioned into five tables: location, station, event, artist, and track. The data are structured in this manner to facilitate convenient retrieval and examination of each category while minimizing duplication. In this way, the database allows for easy filtering and sorting based on specific criteria, enhancing the overall efficiency of data analysis. Additionally, partitioning of data into separate tables helps to prevent errors and inconsistencies in data entry and manipulation.

\begin{figure}[b!]
  \centering
  \begin{minipage}{1\linewidth} 
    \includegraphics[width=\textwidth,scale=.39]{figs/database_v2.png}
  \end{minipage}
  \caption{Database tables and connections for the MIRAGE online dashboard.}
  \label{fig:database}
\end{figure}

\subsection{Layout}\label{subsec:system}
\subsubsection{Earth-View Panel}
The 3D interactive Earth-view (or `globe') panel visualizes the number of stations across the globe. Shown in Figure~\ref{fig:mdash}, each hexagonal-shaped vertical bar identifies the locations where radio stations reside. The height of each bar represents the number of stations at that geographic location, and the bars are also color-coded by country. The Earth-view panel is also linked to the event-list panel such that when a user selects a specific location on the Earth view, the event-list panel automatically filters (i.e., restricts) the station- and event-level metadata to the selected location. In this way, users may compare the number of stations in various regions and discern any recurring patterns or trends.  

%Additionally, users have the ability to export the metadata for the purpose of conducting further analysis or generating reports, facilitating the seamless sharing of valuable insights with data scientists and musicologists without the learning curve. %The tool's interactive nature improves the user experience and enables a more profound comprehension of the analyzed data. Overall, this tool provides a user-friendly platform for exploring and interpreting data related to radio stations worldwide. Its intuitive design and customizable features make it an invaluable resource for professionals in the field of media research or music industry analysis.

\subsubsection{Event-List Panels}
Once users have selected a specific location on the interactive Earth-view panel or using the search function on the toolbox, they can retrieve metadata for the top 1,000 most recent entries in the event-list panel. Users can also select and add events to the selected event-list panel for further analysis and/or export, enabling users to revise their search parameters without losing selected metadata. The event-list panels also allow users to download the contents of either table in CSV format, or obtain a URL to share the results of their most recent search with another user.

\begin{figure}[t!]
  \centering
  \begin{minipage}{1\linewidth} 
    \includegraphics[width=\textwidth,scale=.38]{figs/song.png}
    \vspace{-.5em}
  \end{minipage}
  % \caption{An example of the song-detail panel in the MIRAGE online dashboard.}
  % \label{fig:songdetail}
    \begin{minipage}{1\linewidth} 
    \includegraphics[width=\textwidth,scale=.38]{figs/listen.png}
  \end{minipage}
  \caption{Top: Examples of the event-detail (top) and listen (bottom) panels in the MIRAGE online dashboard.}
  \label{fig:songdetail}
\end{figure}

% \begin{figure}[t!]
%   \centering
%   \begin{minipage}{1\linewidth} 
%     \includegraphics[width=\textwidth,scale=.38]{figs/listen.png}
%   \end{minipage}
%   \caption{An example of the listen panel in the MIRAGE online dashboard.}
%   \label{fig:listen}
% \end{figure}

\begin{figure*}[t!]
  \centering
  \begin{minipage}{1\textwidth} 
    \includegraphics[width=\textwidth,scale=1]{figs/analysis_fig.png}
  \end{minipage}
  \caption{Event-List Visualization Pane for events by Indonesian artists (left) or Indonesian stations (right) in the MIRAGE online dashboard. The red star indicates the position of Anggun's ``Snow on the Sahara.''}
  \label{fig:analysis_fig}
\end{figure*}

% \begin{figure*}[t!]
%   \centering
%   \begin{minipage}{1\textwidth} 
%     \includegraphics[width=\textwidth,scale=1]{figs/surfacemap.png}
%   \end{minipage}
%   \caption{Statistical surface maps representing the proportion of songs on US radio by artists from ``Top 3'' countries (left) and Latin American countries (right).}
%   \label{fig:surfacemap}
% \end{figure*}

\subsubsection{Map Panel}
The map panel enables users to readily visualize the events' geographic distribution in the event-list panel. Each dot on the map reflects the precise position of a particular event, and the size of the dot represents the number of events at that position. If the user selects an event from the event-list panel, that event will be represented by a red dot in the map panel. In this way, the map panel offers users a distinctive method for visualizing the geographical variety of the events in the event-list panel. %Users can readily discern patterns and trends in the geographical distribution of songs. Moreover, the interactive map improves the entire user experience by providing a visual depiction of the data. 

%Furthermore, the song-list panel allows the user to review a maximum of 1000 song/station entries. 

\subsubsection{Event-Detail \& Listen Panels}
Shown in Figure~\ref{fig:songdetail}, the event-detail panel displays the currently selected event from the event-list panel. The content is categorized into four sections: radio-station metadata (e.g., name, location, formats, url, etc.), event metadata (i.e., stream description), artist details (e.g., name(s), gender(s), group affiliations, instrument(s), etc.), and track metadata (e.g., track title, duration, key/mode,  etc.). Figure~\ref{fig:songdetail}, for example, presents all available metadata for Indonesian singer Anggun's ``Snow on the Sahara,'' which streamed on Radio ITB86 in Jakarta, Indonesia on December 16, 2022. Note that users can obtain a list of demographic (e.g., gender, nationality, etc.) and musicological (e.g., list of instruments, vocal type, associated genres, etc.) information about Anggun, review additional metadata about the song itself (year released, language, the song's lyricist(s), etc.), and finally navigate to other websites and online music libraries using the provided hyperlinks.% to the station's website url, location, Spotify, YouTube, MusicBrainz, and Wikipedia, among others.  %It is important to acknowledge that the data has undergone processing from several sources using comprehensive matching algorithms to guarantee the database's accuracy and consistency.

Finally, the listen panel allows users to stream available recordings using embedded links to the integrated Spotify and YouTube platforms. Although not all events are available on both platforms, the dashboard is regularly updated to ensure that the provided information is current.

\subsubsection{Event-List Visualization Panel}

Shown in Figure~\ref{fig:analysis_fig}, the event-list visualization panel allows users to explore the searched or selected event list using interactive bar, scatter, and histogram plots. For each plot, users may select the appropriate metadata variable(s) from a dropdown list, edit the plot using Plotly Chart Studio,\footnote{\href{chart-studio.plotly.com}{chart-studio.plotly.com}.}, and finally export the plot in SVG format.  


\section{Example Use Case}\label{sec: use case}

The metadata and visualizations produced by the MIRAGE online dashboard have numerous applications for users. Figure~\ref{fig:analysis_fig}, for example, examines Anggun's ``Snow on the Sahara'' within the context of events produced by Indonesian artists across the globe (left), or streaming on Indonesian radio stations (right). The MIRAGE-MetaCorpus features 37 events (and 12 tracks) produced by Anggun, of which 21 were ``Snow on the Sahara'' (or its French language version, ``La neige au Sahara''). 

Among Indonesian artists, music genres familiar to western listeners like pop, pop-rock, and alternative rock rank in the top 10, along with characteristic southeast Asian genres like dangdut and koplo. Among events streaming on Indonesian stations, music by Indonesian artists also ranks first, though several Anglophone countries also rank in the top ten (USA, UK, etc.). Scatter plots of a two-dimensional arousal-valence emotion space and a two-component solution from a principal components analysis of the track's danceability, speechiness, acousticness, liveness, and instrumentalness further reveal the track's unconventional expressive and musical characteristics relative to the other tracks produced by Indonesian artists or streaming on Indonesian stations.  Finally, histograms of the track's popularity and year of release reflect the song's enduring popularity more than two decades after its initial release.

%The metadata and visualizations produced by the MIRAGE online dashboard have numerous applications for users, but perhaps the most important is the user's ability to  to export the obtained metadata for further analysis. Using geo-tagged metadata from the MIRAGE online dashboard, for example, users could produce statistical surface maps of a selected country or continent, or of the entire globe, based on metadata variables like station format or artist country. 

%By way of example, Figure~\ref{fig:surfacemap} %explores the demographic makeup of artists on American internet radio in MIRAGE-MetaCorpus. Artists from the sample reflected in Figure~\ref{fig:surfacemap} hailed from 158 countries across the globe, with the top three countries of the USA, the United Kingdom (UK), and Canada accounting for 82\% of the songs in the song list. Using inverse distance weighting \cite{OSullivan2010}, these surface maps reflect the proportion of artists from either the top three countries of the USA, UK, and Canada (left), or Latin American countries in Central and South America and the Caribbean that together accounted for 3\% of all songs in the song list (right).   

%Although stations predominantly feature artists from the top three countries throughout much of the continental USA, Latin American artists also feature prominently at stations located in major metropolitan areas (e.g., Los Angeles, Chicago, New York), southern-border towns (El Paso TX, Edinburgh TX), and coastal cities (Doral FL, Egg Harbor Township NJ). These differences in the diversity of artists across stations may partly reflect each station’s network (e.g., community, national, transnational) and format (e.g., Top 40, Caribbean, Easy Listening), as well as population demographics of the region(s) in question. Expanding this analysis to countries across the globe would allow users to estimate a diversity index and reveal potential asymmetries in the exchange of songs across countries.  

% In Figure~\ref{fig:uc1}, there is a clearly different ratio of the number of streams and the number of stations between London (989/139) and Crawley (300/3). This demonstrates a significant disparity in stream density between the two locations. The visual depiction helps to highlight the disparity and allows for a more comprehensive analysis of the data.
% \begin{figure}[h]
%   \centering
%   \begin{minipage}{1\linewidth} 
%     \includegraphics[width=\textwidth,scale=.38]{figs/uc1.png}
%   \end{minipage}
%   \caption{London and Crawley case: Number of stations vs Number of streams.}
%   \label{fig:uc1}
% \end{figure}

%\section{Limitations \& Ethical Considerations}\label{sec:limitations}

\section{Conclusion \& Ethical Considerations}\label{sec:conclusion}
This development release (v0.2) of the MIRAGE online dashboard provides a snapshot of the contemporary global listening landscape for scholars across the (digital) humanities. Our purpose in doing so is to facilitate cross-cultural, comparative research, which has become a pressing concern in several music disciplines \cite{Ewell2020, Jacoby2020, Lacey2018, Savage2012}. To that end, the MIRAGE-MetaCorpus features metadata for 1 million events that streamed on 10,000 radio stations across the globe, and the dashboard is interoperable with several platforms and operating systems \cite{Moss2021, Wilkinson2016}. %To maintain our commitment to the principles of findability, accessibility, interoperability, and reusability (FAIR) that guide the development of digital resources \cite{Moss2021, Wilkinson2016}, the dashboard also provides advanced search and filtering capabilities, allowing users to export metadata and visualizations in URL, CSV, PNG, and SVG formats. %Additionally, it offers convenient features for analyzing and sharing data. Its intuitive design and customizable features make it an invaluable resource for professionals in the field of media research or music industry analysis.

As a metadata repository, the MIRAGE-MetaCorpus contains links to online resources that we do not control. To mitigate the potential for dataset degradation over time, the research team plans to update (and collect additional) metadata annually. Nevertheless, the attribution metadata provided by the radio station's stream encoder does not always reliably match metadata provided by online music libraries. In our view, nonmatching (or `unreliable') metadata allow the research community to evaluate the coverage (i.e., bias) of online music libraries for the music found on internet radio. Nevertheless, MIRAGE users should be aware of the potential for matching errors. For tasks where higher match quality is important, users may search for reliable metadata in the online dashboard, or export reliable subsets of the MIRAGE-MetaCorpus. 

Similarly, this project provides access to metadata and musicological features produced by proprietary (or otherwise undisclosed) algorithms, often trained on western musical traditions and their associated organizational principles. As a result, we encourage the research community to treat the attribution metadata in MIRAGE as a starting point for developing corpora and methodologies involving other musical traditions \cite{Born2020, Huang2023}. 

In developing the MIRAGE online dashboard, the research team has attempted to protect the interests of copyright holders by only including publicly available metadata protected under fair use while enabling users to stream recordings using embedded links to commercial services like Spotify or YouTube. The dashboard also adheres to the user agreements from the libraries and streaming services mentioned above (e.g., Radio Garden, Spotify, WikiData), according to which users may access and interact with all data on the online dashboard, but they may only export public-domain data for further analysis and study (i.e., from the Radio Garden API, the Natural Earth data set, station stream encoder, and WikiData). Perhaps most importantly, this project did not directly record/store audio from station streams at any point in the data-collection pipeline. 

Nevertheless, we acknowledge the concerns of copyright holders (artists, radio stations, online music libraries, and streaming services) who do not wish to share attribution metadata about their work (e.g., artist demographics, track details, etc.). We only provide links to publicly available sources and do not own the copyright for any music referenced in the MIRAGE-MetaCorpus. For that reason, copyright holders may request the removal of metadata from the MIRAGE project.\footnote{Please contact \href{mailto:miragedashboard@gmail.com}{miragedashboard@gmail.com}.}  

In addition to completing station-list review for the remaining stations in MIRAGE-MetaCorpus, future versions of the dashboard will transition from React+Nodejs to Remix in order to enhance the speed of queries and allow users to access and review more than 1,000 events simultaneously in the event-list panel. The team also plans to conduct a usability study to examine the dashboard's practical utility, as well as incorporate additional customizable sampling and visualization tools like statistical surface maps to enhance the user's exploration of metadata variables in MIRAGE \cite{OSullivan2010}. In doing so, we hope future versions of this dashboard will facilitate cross-cultural, comparative research for a medium that places diversity center stage. %The team also plans to incorporate additional customizable visualization tools like statistical surface maps to enhance the user's exploration of metadata variables in MIRAGE. In doing so, we hope future versions of this dashboard will facilitate cross-cultural, comparative research for a medium that places diversity center stage. %we hope that the music research community will explore the musical traditions on a medium that places diversity center stage.



% For bibtex users:
\bibliography{ISMIRtemplate}

\end{document}

