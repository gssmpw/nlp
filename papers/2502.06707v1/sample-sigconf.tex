%%
%% This is file `sample-sigconf.tex',
%% generated with the docstrip utility.
%%
%% The original source files were:
%%
%% samples.dtx  (with options: `all,proceedings,bibtex,sigconf')
%% 
%% IMPORTANT NOTICE:
%% 
%% For the copyright see the source file.
%% 
%% Any modified versions of this file must be renamed
%% with new filenames distinct from sample-sigconf.tex.
%% 
%% For distribution of the original source see the terms
%% for copying and modification in the file samples.dtx.
%% 
%% This generated file may be distributed as long as the
%% original source files, as listed above, are part of the
%% same distribution. (The sources need not necessarily be
%% in the same archive or directory.)
%%
%%
%% Commands for TeXCount
%TC:macro \cite [option:text,text]
%TC:macro \citep [option:text,text]
%TC:macro \citet [option:text,text]
%TC:envir table 0 1
%TC:envir table* 0 1
%TC:envir tabular [ignore] word
%TC:envir displaymath 0 word
%TC:envir math 0 word
%TC:envir comment 0 0
%%
%%
%% The first command in your LaTeX source must be the \documentclass
%% command.
%%
%% For submission and review of your manuscript please change the
%% command to \documentclass[manuscript, screen, review]{acmart}.
%%
%% When submitting camera ready or to TAPS, please change the command
%% to \documentclass[sigconf]{acmart} or whichever template is required
%% for your publication.
%%
%%
\documentclass[sigconf]{acmart}
% \documentclass[sigconf, anonymous, review]{acmart}
% \settopmatter{printacmref=false} % Removes citation information below abstract
% \renewcommand\footnotetextcopyrightpermission[1]{} % removes footnote with conference information in first column
% \pagestyle{plain} % removes running headers

%%
%% \BibTeX command to typeset BibTeX logo in the docs
\AtBeginDocument{%
  \providecommand\BibTeX{{%
    Bib\TeX}}}

%% Rights management information.  This information is sent to you
%% when you complete the rights form.  These commands have SAMPLE
%% values in them; it is your responsibility as an author to replace
%% the commands and values with those provided to you when you
%% complete the rights form.
\setcopyright{acmlicensed}
\copyrightyear{2018}
\acmYear{2018}
\acmDOI{XXXXXXX.XXXXXXX}



%%%%%%%%%%%%%%%%%%%%%%%%% Self-Define %%%%%%%%%%%%%%%%%%%%%%%%%%%%%%%
\newcommand{\TODO}[1]{\textbf{\color{red}[TODO: #1]}}
\usepackage{xspace}
\usepackage{multicol, multirow, threeparttable}
\usepackage{graphicx}
\usepackage{caption}
\usepackage{subcaption}
\newcommand{\NAME}{\TODO{NAME}\xspace}

%\usepackage{marvosym}
\usepackage{fontawesome}
\usepackage{enumitem}

\newcommand{\name}{FinMamba\xspace}
\usepackage[capitalize]{cleveref}
\crefname{section}{Sec.}{Secs.}
\Crefname{section}{Section}{Sections}
\Crefname{table}{Table}{Tables}
\crefname{table}{Tab.}{Tabs.}
\crefformat{equation}{Eq.~#2#1#3}

\usepackage{pifont} % \ding{xx}
\usepackage{bbding}
\usepackage{fontawesome}

\definecolor{mypurple}{rgb}{0.7, 0, 0.9}
\definecolor{myorange}{rgb}{0.9, 0.7, 0}
\definecolor{bl}{rgb}{0.25, 0.5, 0.9}
\newcommand{\best}[1]{{\textbf{\textcolor{red}{#1}}}}
\newcommand{\second}[1]{{\textcolor{bl}{\underline{#1}}}}
%%%%%%%%%%%%%%%%%%%%%%%%% Self-Define %%%%%%%%%%%%%%%%%%%%%%%%%%%%%%%


%% These commands are for a PROCEEDINGS abstract or paper.
% \acmConference[Conference acronym 'XX]{Make sure to enter the correct
%   conference title from your rights confirmation emai}{June 03--05,
%   2018}{Woodstock, NY}
%%
%%  Uncomment \acmBooktitle if the title of the proceedings is different
%%  from ``Proceedings of ...''!
%%
%%\acmBooktitle{Woodstock '18: ACM Symposium on Neural Gaze Detection,
%%  June 03--05, 2018, Woodstock, NY}
% \acmISBN{978-1-4503-XXXX-X/18/06}


%%
%% Submission ID.
%% Use this when submitting an article to a sponsored event. You'll
%% receive a unique submission ID from the organizers
%% of the event, and this ID should be used as the parameter to this command.

%\acmSubmissionID{123-A56-BU3}

%%
%% For managing citations, it is recommended to use bibliography
%% files in BibTeX format.
%%
%% You can then either use BibTeX with the ACM-Reference-Format style,
%% or BibLaTeX with the acmnumeric or acmauthoryear sytles, that include
%% support for advanced citation of software artefact from the
%% biblatex-software package, also separately available on CTAN.
%%
%% Look at the sample-*-biblatex.tex files for templates showcasing
%% the biblatex styles.
%%

\usepackage{fancyhdr}
\pagestyle{empty}

%%
%% The majority of ACM publications use numbered citations and
%% references.  The command \citestyle{authoryear} switches to the
%% "author year" style.
%%
%% If you are preparing content for an event
%% sponsored by ACM SIGGRAPH, you must use the "author year" style of
%% citations and references.
%% Uncommenting
%% the next command will enable that style.
%%\citestyle{acmauthoryear}



%%
%% end of the preamble, start of the body of the document source.
\begin{document}

%%
%% The "title" command has an optional parameter,
%% allowing the author to define a "short title" to be used in page headers.
% \title{Stock Movement Prediction Based on Market-Aware Heterogeneous Graph and Multi-Scale Mamba}
\title{\name: Market-Aware Graph Enhanced Multi-Level Mamba for Stock Movement Prediction}
\renewcommand{\shorttitle}{\name: Market-Aware Graph Enhanced Multi-Level Mamba for Stock Movement Prediction}

%%
%% The "author" command and its associated commands are used to define
%% the authors and their affiliations.
%% Of note is the shared affiliation of the first two authors, and the
%% "authornote" and "authornotemark" commands
%% used to denote shared contribution to the research.
\author{
    %Authors
    % All authors must be in the same font size and format.
    Yifan Hu\textsuperscript{\rm 1,2,}*, 
    Peiyuan Liu\textsuperscript{\rm 1,}*, 
    Yuante Li\textsuperscript{\rm 2}, 
    Dawei Cheng\textsuperscript{\rm 2,}\footnotemark[2],\\
    Naiqi Li\textsuperscript{\rm 1}, 
    Tao Dai\textsuperscript{\rm 3},
    Jigang Bao\textsuperscript{\rm 1}, 
    Shu-tao Xia\textsuperscript{\rm 1}, 
}
\affiliation{
    %Afiliations
    \textsuperscript{\rm 1}Tsinghua Shenzhen International Graduate School \ \ \ \ 
    \textsuperscript{\rm 2}Tongji University \ \ \ \ 
    \textsuperscript{\rm 3}Shenzhen University\\
    \textsuperscript{*}Equal Contribution \ \ \ \ 
    \textsuperscript{\footnotemark[2]}Corresponding Author\\
    \faEnvelopeO\ Primary contact: huyf0122@gmail.com\\
    \country{}
}


% \author{Yifan Hu}
% \authornote{Both authors contributed equally to this research.}
% \affiliation{%
%   \institution{Tongji University}
%   \city{Shanghai}
%   \country{China}}
% \email{huyf0122@gmail.com}

% \author{Peiyuan Liu}
% \authornotemark[1]
% \affiliation{%
%  \institution{Tsinghua Shenzhen International Graduate School}
%  \city{Shenzhen}
%  \country{China}}
% \email{peiyuanliu.edu@gmail.com}

% \author{Yuante Li}
% \affiliation{%
%   \institution{Tongji University}
%   \city{Shanghai}
%   \country{China}}
% \email{liyuante@tongji.edu.cn}

% \author{Dawei Cheng}
% \authornote{Corresponding author: Dawei Cheng}
% \affiliation{%
%   \institution{Tongji University}
%   \city{Shanghai}
%   \country{China}}
% \email{dcheng@tongji.edu.cn}

% \author{Naiqi Li}
% \affiliation{%
%  \institution{Tsinghua Shenzhen International Graduate School}
%  \city{Shenzhen}
%  \country{China}}
% \email{linaiqi@sz.tsinghua.edu.cn}

% \author{Tao Dai}
% \affiliation{%
%   \institution{Shenzhen University}
%   \city{Shenzhen}
%   \country{China}}
% \email{daitao.edu@gmail.com}


% \author{Jigang Bao}
% \affiliation{%
%  \institution{Tsinghua Shenzhen International Graduate School}
%  \city{Shenzhen}
%  \country{China}}
% \email{baojg19@mails.tsinghua.edu.cn}

% \author{Shu-tao Xia}
% \affiliation{%
%  \institution{Tsinghua Shenzhen International Graduate School}
%  \city{Shenzhen}
%  \country{China}}
% \email{xiast@sz.tsinghua.edu.cn}

%%
%% By default, the full list of authors will be used in the page
%% headers. Often, this list is too long, and will overlap
%% other information printed in the page headers. This command allows
%% the author to define a more concise list
%% of authors' names for this purpose.
% \renewcommand{\shortauthors}{Trovato et al.}
\renewcommand{\shortauthors}{Yifan Hu, Peiyuan Liu, Yuante Li, Dawei Cheng, Naiqi Li, Tao Dai, Jigang Bao, Shu-tao Xia}

%%
%% The abstract is a short summary of the work to be presented in the
%% article.
\begin{abstract}
  % 近年来, 将时间序列特征和与时间序列之间的相关性相结合已成为Stock movement prediction的一种普遍方法
  % 然而,由于金融数据的non-stationarity以及低信噪比 和 市场的动态性以及复杂性, 它们仍面临个两个limitations
  % 其一, Existing financial data mining literature 难以 高效 且 低内存占用 地从更长的历史数据中挖掘有效时序依赖; 其二, 标的中股票间关系随着宏观市场动态变化,它们也无法根据Market来自适应建模multifaceted股票间关系
  Recently, combining stock features with inter-stock correlations has become a common and effective approach for stock movement prediction. However, financial data presents significant challenges due to its low signal-to-noise ratio and the dynamic complexity of the market, which give rise to two key limitations in existing methods. First, the relationships between stocks are highly influenced by multifaceted factors including macroeconomic market dynamics, and current models fail to adaptively capture these evolving interactions under specific market conditions. Second, for the accuracy and timeliness required by real-world trading, existing financial data mining methods struggle to extract beneficial pattern-oriented dependencies from long historical data while maintaining high efficiency and low memory consumption. 
  To address the limitations, we propose \name, a Mamba-GNN-based framework for market-aware and multi-level hybrid stock movement prediction.
  Specifically, we devise a dynamic graph to learn the changing representations of inter-stock relationships by integrating a pruning module that adapts to market trends. Afterward, with a selective mechanism, the multi-level Mamba discards irrelevant information and resets states to skillfully recall historical patterns across multiple time scales with linear time costs, which are then jointly optimized for reliable prediction. 
  Extensive experiments on U.S. and Chinese stock markets demonstrate the effectiveness of our proposed \name, achieving state-of-the-art prediction accuracy and trading profitability, while maintaining low computational complexity. The code is available at \url{https://github.com/TROUBADOUR000/FinMamba}.
\end{abstract}

%%
%% The code below is generated by the tool at http://dl.acm.org/ccs.cfm.
%% Please copy and paste the code instead of the example below.
%%
% \begin{CCSXML}
% <ccs2012>
%    <concept>
%        <concept_id>10010147.10010178</concept_id>
%        <concept_desc>Computing methodologies~Artificial intelligence</concept_desc>
%        <concept_significance>500</concept_significance>
%        </concept>
%    <concept>
%        <concept_id>10010405.10003550</concept_id>
%        <concept_desc>Applied computing~Electronic commerce</concept_desc>
%        <concept_significance>500</concept_significance>
%        </concept>
%  </ccs2012>
% \end{CCSXML}
% \ccsdesc[500]{Information systems~Data mining}
% \ccsdesc[500]{Applied computing~Economics}


%%
%% Keywords. The author(s) should pick words that accurately describe
%% the work being presented. Separate the keywords with commas.
\keywords{Quantitative investment, Stock Movement Prediction, Mamba}
%% A "teaser" image appears between the author and affiliation
%% information and the body of the document, and typically spans the
%% page.

\received{20 February 2007}
\received[revised]{12 March 2009}
\received[accepted]{5 June 2009}

%%
%% This command processes the author and affiliation and title
%% information and builds the first part of the formatted document.
\maketitle

\section{Introduction}
% stock movement prediction 的重要性和挑战,传统方法问题
% 现有的研究方法
  % 为了捕获时序依赖,RNN CNN transformer... figure 1 中 引入mamba的优点
  % 为了建模股票间关系,静态图 异构图 ... 然而单一方面的考虑难以建模股票市场,从先验的序列相关性和后验的industry 引入graph构建
  % 另外,现有研究方法另一个问题 没有考虑与市场交互,引入具体模块
% 因此,提出了...


% lpy
Stock movement prediction plays a pivotal role in data science-driven quantitative trading applications due to its potential to guide profitable investment strategies~\cite{survey,cisthpan,MDGNN}. Unlike general time series forecasting, stock prices record human-brain-armed game behaviors, which are influenced by various factors, including investor behavior, economic indicators, political events, and global news. 
This high level of volatility and the multifaceted nature of these influences make accurate prediction a particularly daunting challenge. 
Traditional machine learning approaches~\cite{alpha,randomforest} have been used to address these challenges, but often require expert-designed features and struggle to capture the intricate dependencies between stocks. Recently, deep learning-based methods~\cite{CTTS,ALSP,lsrigru} have shown great promise in overcoming these limitations by effectively combining stock features with inter-stock correlations~\cite{tcgpn}. However, such a paradigm could still be unsatisfactory due to the following two limitations. 
% which can be broadly categorized into RNN-based~\cite{rnn}, CNN-based~\cite{cnn,CTTS}, and Transformer-based~\cite{ALSP,master} methods.

% origin ==========
% Stock movement prediction is a critical area of research in quantitative trading because of its potential to guide profitable investment strategies~\cite{survey}. 
% Unlike general time series forecasting, stock prices are influenced by a complex interplay of factors from various aspects, including investor behavior, economic indicators, political events, global news, and so on. This highly volatile and multifaceted nature makes accurate prediction of stock movement an exceptionally challenging task. 
% Traditional machine learning approaches have been used to address these challenges, but often require expert-designed features and struggle to capture the intricate dependencies between stocks.
% Recently, deep learning techniques have shown promise in improving prediction accuracy by exploiting non-linear temporal dependencies and capturing relationships between stocks.

% =================

% example
% StockFormer~\cite{StockFormer} utilizes a series of attention structures in the decision-making module to learn a mixture of predictive states and relational states.
% HATR~\cite{hatr} combines a hierarchical temporal module to capture multi-grained dynamic patterns of stocks and a multi-graph interaction module to learn correlations among stocks.
% However, current approaches still struggle to fully grasp the complexities of financial time series dynamics and the intricate connections among different stocks, limiting their effectiveness.

\begin{figure}[t]
    \centering
    \includegraphics[width=0.45\textwidth]{images/correlation_index.pdf}
    \caption{
         Market fluctuations have affected the correlation among stocks. In the heatmap above, the horizontal axis represents trading days, while the vertical axis displays the average correlation of each stock with other stocks on a given day. The line chart below reflects the macroeconomic market volatility. The analysis reveals that, during past market downturns, stock correlations tend to intensify. A discernible pattern emerges: stock correlations increase when the market index falls and diminish when the market index rises.
    }
    \Description{..}
    \label{fig:corr&index}
\end{figure}


\ding{182} \textbf{\textit{Multifaceted inter-stock relationships under various market conditions.}}
As is well known, synergy within the stock market is a key aspect, with related stocks often moving in sync~\cite{ALSP}, providing valuable insights for more accurate predictions. Meanwhile, inter-stock relationships are influenced by various factors, such as sector dynamics, regulatory politics, and the macroeconomic market environment. Therefore, comprehensive modeling of relationships between stocks is crucial.
Some of the existing methods~\cite{StockFormer,hist,ALSP} rely on incomplete prior knowledge, which may introduce biases. For instance, solely grouping stocks by industry sectors can overlook cross-sector influences from broader macroeconomic factors. 
Additionally, stock relationships are dynamic, evolving with market fluctuations, making it difficult for static models to accurately reflect these changes. In contrast, posterior-based methods construct adjacency matrices to model dynamic correlations between stocks based on dynamic time warping~\cite{cisthpan}, Pearson correlation coefficients~\cite{thgnn}, Euclidean distance~\cite{tcgpn} or learning-based methods~\cite{ADBTR,master}. However, posterior methods can lead to spurious correlations, where stocks with no actual relationship exhibit high sequence similarity due to coincidental trends. Thus, we propose a graph-based structure to integrate both static prior correlations, such as industry classifications, and dynamic posterior correlations derived from stock price sequences, providing a more comprehensive framework for capturing inter-stock dependencies.

Furthermore, the role of macroeconomic markets in shaping stock relationships is often overlooked, despite their significant influence on inter-stock correlations~\cite{market}. As the stock market evolves, the inter-stock relationships also shift dynamically. As \cref{fig:corr&index} depicts, during financial crises, stocks within the same industry tend to exhibit stronger correlations due to shared risk factors and similar investor behavior~\cite{DIMPFL201410}. This phenomenon is amplified by the impact of market indices, which heighten the influence of macroeconomic conditions on stock performance during turbulent periods~\cite{MA2022102940}. Therefore, market indices are not only critical for gauging overall market performance but also play a key role in driving the dynamic relationships between stocks. To better capture these shifts, we propose a graph optimization strategy that refines stock relationship modeling by leveraging market index feedback, retaining dominant edges based on index movements while pruning those that fail to reflect dynamic changes. This approach enables the model to adaptively adjust inter-stock correlations in response to market fluctuations, resulting in more flexible predictions.


\begin{figure}[t]
    \centering
    \includegraphics[width=0.475\textwidth]{images/intro.pdf}
    \caption{
    Comparison of (a) Transformer's self-attention mechanism and (b) Mamba's selective mechanism in stock movement prediction. Mamba is more effective in recalling similar historical patterns and avoiding overemphasis on outliers compared to Transformer.
    }
    \Description{..}
    \label{fig:intro}
\end{figure}

% Additionally, real-time market trading imposes efficiency requirements for timely decision-making.

\ding{183} \textbf{\textit{Pattern-oriented dependencies under the timeliness constraints.}}
% 介绍股票数据来引出真实时效性要求下有效依赖的作用
The time series studied in the deep learning literature~\cite{pdf,timebridge} tend to exhibit some regularity (e.g., solar energy, traffic flow), facilitating the extraction of highly relevant features from similar repetitive patterns~\cite{generative}.
In contrast, stock prices often lack regular repetitive patterns due to low signal-to-noise ratios~\cite{FactorVAE}, with similar patterns appearing at different periods and amplitudes.
Recently, the Transformer~\cite{attention}, which calculates interactions between all sequence elements to capture their dependencies, has become a mainstream approach.
However, its self-attention mechanism often assigns disproportionate importance to outliers, leading to an overemphasis on anomalous patterns and hindering the effective capture of smoother, more common trends across different time scales~\cite{amd} (see \cref{fig:intro}(a)). 
% global effective receptive field
After that, Mamba~\cite{mamba} has shown great potential in dynamic sequence modeling, with notable success in the other fields~\cite{mambair, mambavision}. The selective mechanism of Mamba enables it to recall temporal patterns from historical inputs, making the model particularly well-suited for capturing temporal dependencies in stock prices (see \cref{fig:intro}(b)). 
Moreover, real-time market trading imposes strict efficiency requirements for timely decision-making~\cite{ccso}. In light of this, Mamba offers a key advantage with its lower linear complexity compared to the Transformer, significantly enhancing prediction efficiency.

Additionally, stock prices exhibit distinct characteristics across different time levels, gradually transitioning from micro to macro levels~\cite{LUO2021101512}. On short-term, minute-level levels, price fluctuations are primarily driven by market microstructures, such as trader sentiment, high-frequency trading strategies, and market liquidity, often influenced by news, rumors, or technical indicators. As the time horizon extends to daily levels, mid-term factors like corporate performance, industry trends, and policy changes come into play~\cite{LU201777}. At weekly or monthly levels, macroeconomic forces such as economic growth, inflation, and monetary policy become the dominant drivers, reflecting structural market shifts and long-term trends. Therefore, analyzing stock price movements on a single time scale fails to comprehensively capture the full spectrum of temporal dependencies.\cite{ijcai2020p640}.

% origin ==========
% For exploiting non-linear temporal dependencies, the Transformer architecture~\cite{attention} is widely used, which excels in modeling long-range dependencies due to the self-attention mechanisms that solve the problem of CNNs and RNNs. 
% Lately, Mamba~\cite{mamba} has shown great potential in modeling time series dynamics and has made progress in the field of computer vision~\cite{mambair, mambavision}. 
% As shown in \cref{fig:intro}, compared to Transformers, Mamba is more suitable for modeling financial time series such as stocks. 
% Specifically, high stock volatility is reflected in the interaction between price fluctuation patterns over different time spans, including both long-term (e.g.,A,B) and short-term(e.g.,C,D) spans. This requires a time-varying model capable of selectively remembering or ignoring inputs depending on their context. 
% The attention mechanism encodes each time step by considering its weighted relevance to the entire sequence. However, global attention mechanisms tend to assign high scores to outlier points~\cite{dlinear, ALSP}, making them susceptible to the non-stationary noise typical of stock data. As a result, the model tends to focus excessively on outlier-heavy patterns while struggling to effectively capture more common, smoother patterns across different time horizons.
% In contrast, Mamba's selection mechanism enables the parameters governing interactions along the sequence to be input-dependent, effectively balancing short-term and long-term dependencies~\cite{cmamba}. This accounts for the time spans between stock patterns and dynamically adjusts the behavior of stock trading series accordingly.
% Moreover, compared to Transformers, Mamba has linear time complexity, which saves computing resources. 
% In addition, stock data exhibit temporal variations across multiple scales. Intuitively, for example, minute-by-minute price fluctuations may follow daily market trends, while weekly or monthly movements may be driven by broader economic factors. This inspires us to model state spaces across scales, capturing the dynamics at both granular and macro levels.
% =================

% lpy
% To capture relationships between stocks, existing methods~\cite{hist,ALSP} often rely on prior knowledge, such as industry-based classifications. However, these priors are often incomplete and may introduce biases. For instance, solely grouping stocks by industry can overlook cross-sector influences from broader macroeconomic factors. Additionally, stock relationships are dynamic, evolving with market fluctuations, making it difficult for static models to accurately reflect these changes. In contrast, posterior-based methods construct adjacency matrices to model dynamic correlations between stocks based on dynamic time warping~\cite{cisthpan}, Pearson correlation coefficients~\cite{thgnn}, Euclidean distance~\cite{tcgpn} or learning-based methods~\cite{ADBTR,master}. However, posterior methods can lead to spurious correlations, where stocks with no actual relationship exhibit high sequence similarity due to coincidental trends. To address these challenges, we propose a graph-based structure to integrate both static prior correlations, such as industry classifications, and dynamic posterior correlations derived from stock price sequences, providing a more comprehensive and accurate framework for capturing inter-stock dependencies.

% origin =====================
% For capturing relationships between stocks, some previous work \cite{hist,ALSP} has relied on predefined concepts, relationships, or rules to construct static correlation graphs. 
% For example, static sector-based correlations are defined by grouping stocks from the same industry together. 
% However, predefined relationships are often incomplete and can introduce bias. For instance, classifying stocks solely by industry could overlook cross-sector influences, such as macroeconomic factors that affect multiple sectors simultaneously. Furthermore, the relationships between stocks are dynamic and evolve with market variation. 
% To address these issues, other approaches have introduced heterogeneous graphs that capture the dynamic correlations between sequences by constructing adjacency matrices using methods such as dynamic time warping~\cite{cisthpan}, Pearson correlation coefficients~\cite{thgnn}, Euclidean distance~\cite{tcgpn} or learning-based methods~\cite{ADBTR,master}. 
% However, this posterior correlations can lead to the phenomenon of spurious relationships, where two unrelated stocks show a strong sequence correlation simply because of similar movements. In addition, accurate prediction of stock movements requires taking into account the complex and multifaceted relationships between stocks. 
% This inspires us to alleviate such phenomenon in the posterior correlations through a priori correlations, and to comprehensively model the correlations between stocks from multiple correlation perspectives.
% =====================

% lpy
% Furthermore, the role of market indices in shaping stock relationships is often overlooked, despite their significant influence on inter-stock correlations~\cite{market}. As the stock market evolves, the relationships between stocks also shift dynamically. For instance, during financial crises, stocks within the same industry tend to exhibit stronger correlations due to shared risk factors and similar investor behavior~\cite{DIMPFL201410}. This phenomenon is amplified by the impact of market indices, which heighten the influence of macroeconomic conditions on stock performance during turbulent periods~\cite{MA2022102940}. Therefore, market indices are not only critical for gauging overall market performance but also play a key role in driving the dynamic relationships between stocks. To better capture these shifts, we propose a graph optimization strategy that refines stock relationship modeling by leveraging market index feedback, retaining dominant edges based on index movements while pruning those that fail to reflect dynamic changes. This approach enables the model to adaptively adjust inter-stock correlations in response to market fluctuations, resulting in more flexible and accurate predictions.

% origin =====================
% Another limitation of existing works is that they ignore the relationship between the market index and stock correlations. As the market evolves, the relationships between stocks will dynamically become effective or ineffective. For instance, during a financial crisis, stocks within the same sector may exhibit stronger correlations due to shared risks and investor behavior. We should retain the edges that dominate based on market feedback and prune those that fail to convey market dynamics, thereby adaptively enhancing the model’s response to market changes to improve predictive accuracy. This is discussed in detail in \cref{chapter:p}.
% =====================

% lpy
% In addition to inter-stock relationships, stock prices exhibit distinct characteristics across different time scales, gradually transitioning from micro to macro levels~\cite{LUO2021101512}. On short-term, minute-level scales, price fluctuations are primarily driven by market microstructures, such as trader sentiment, high-frequency trading strategies, and market liquidity, often influenced by news, rumors, or technical indicators. As the time horizon extends to daily scales, mid-term factors like corporate performance, industry trends, and policy changes come into play~\cite{LU201777}. At weekly or monthly scales, macroeconomic forces such as economic growth, inflation, and monetary policy become the dominant drivers, reflecting structural market shifts and long-term trends. Therefore, analyzing stock price movements on a single time scale fails to comprehensively capture the full spectrum of temporal dependencies.\cite{ijcai2020p640}.

Based on the above motivations, we propose \name with Market-Aware Graph (MAG) and Multi-Level Mamba (MLM) to address existing limitations on the 1) multifaceted inter-stock relationships and 2) pattern-oriented characteristics of stock prices. 
Specifically, the MAG module adapts to effectively capture inter-stock relationships by integrating both the posterior and prior information, while refining according to macroeconomic market conditions. Then, the MLM module jointly models the micro- and macro-time level dependencies across the financial time series. Extensive experiments on U.S. and Chinese stock markets demonstrate the effectiveness of the proposed \name, along with its low computational complexity. 
In a nutshell, our contributions are summarized as follows:

\begin{itemize}[itemsep=-0.11em]
    \item To the best of our knowledge, this is the first work that enhances the inter-stock relationships under specific macroeconomic market conditions, leading to a more impressive understanding of market dynamics.
    \item We propose \name that constructs a Market-Aware Graph integrating both static priors and dynamic posteriors, refines these relationships using market index feedback, and models stock pattern-oriented dependencies across multiple levels through a Multi-Level Mamba framework.
    \item Extensive experiments on both U.S. and Chinese real-time stock markets demonstrate the state-of-the-art performance of the proposed \name with high computational efficiency. The visualization results provided valuable insights into the dynamics of stock patterns.
\end{itemize}


% Origin ===================
% Beyond the above motivations, we propose \name with temporal stock correlation graph and multi-scale mamba. Overall, we use the correlation between sequences as a posterior relationship and the primary and secondary industry classifications as a prior relationship. This allows us to comprehensively construct the relationships between stocks for each trading day from multiple perspectives. Subsequently, we utilize market-aware graph parsification to further retain the edges that play a dominant role on the current trading day, while pruning away edges that introduce unnecessary noise. Then, we analyze the data using graph embeddings to obtain a comprehensive representation of stock data. Finally, we employ multi-scale Mamba to model temporal dependencies at both micro and macro levels, enabling dynamic and effective stock movement prediction. 
% Our contributions are summarized as follows:
% ===================

% \begin{itemize}
%     \item We propose a novel \name for stock movement prediction to effectively capture the stock correlation and stock temporal dependencies. We also introduce a graph sparsification method based on stock market to adapt to varying market scenarios.
%     \item We delve into the issues of Mamba in stock temporal dynamics modeling and  present a reformed multi-scale method to exploit similar patterns across scales.
%     \item We conduct extensive experiments in both the U.S. and Chinese datasets, demonstrating the superior performance and generalization ability of \name.
% \end{itemize}


\section{Related Work}

\subsection{Stock Movement Prediction}
Due to the growing interest in stock market investing and the evolution of deep learning, most contemporary research endeavors now center around implementing and refining deep learning models in the realm of stock movement prediction. 
Consequently, combining stock features with inter-stock correlations has become a common and effective approach for stock movement prediction.
For example, THGNN~\cite{thgnn} employs a temporal and heterogeneous graph framework to extract insights from price history and relationships. MASTER~\cite{master} mines the momentary and cross-time stock correlation with learning-based methods. CI-STHPAN~\cite{cisthpan} constructs channel independent hypergraphs among stocks with similar stock price trends based on dynamic time warping.
Despite their success, current methodologies may still fall short by overlooking the fact that the relationship between individual stocks and the market index is such that it strengthens when the index falls and weakens when the index rises, especially during periods of market volatility.


\subsection{Mamba and State Space Models}
State Space Models (SSMs)~\cite{s4,mamba,mamba2} have emerged as a promising architecture for sequence modeling. Mamba, leveraging selective SSMs and a hardware-optimized algorithm, has achieved strong performance in several areas, including Natural Language Processing~\cite{mamba}, Computer Vision~\cite{mambair, mambavision}. 
For time series forecasting, Timemachine~\cite{timemachine} selects contents
for prediction against global and local contextual information with four SSM modules. Chimera~\cite{chimera} incorporates 2-dimensional SSMs with different discretization processes and input-dependent parameters to dynamically model the dependencies.
However, the application of Mamba in quantitative trading is still in its infancy, with approaches such as MambaStock~\cite{mambastock} applying the original Mamba for individual stock modeling and SAMBA~\cite{samba} using bidirectional Mamba blocks to capture long-term dependencies.
In our work, we enhance Mamba by modeling similar stock patterns across different time spans through multi-level projection, while also incorporating inter-stock relationships and market influences to improve stock movement prediction.


\section{Problem Definition}\label{chapter:p}
In this section, we will introduce some concepts in our proposed \name framework and formally define the problem of stock movement prediction. 
For certain concepts and phenomena, we provide examples to facilitate understanding.

\textbf{Definition 1. Stock Context.}
The set of all stocks is defined as $\mathcal{S}=\{s_1, s_2, ..., s_N\}\in\mathbb{R}^{N\times L\times F}$, where $s_i$ represents a specific stock, $N$ denotes the total number of stocks, $L$ denotes the length of the lookback window and $F$ is the number of features.
For any given stock $s_i$, its data on trading day $t$ is defined as $s_i^t\in\mathbb{R}^F$. Closing price $p_i^t$ is one of the features of $s_i^t$ and a one-day return ratio $r_i^t = \frac{p_i^t-p_i^{t-1}}{p_i^{t-1}}$.
On any given trading day $t$, there exists an optimal ranking of the stock scores $Y^t=\{y_1^t\geq y_2^t\geq...\geq y_N^t\}$. 
For any two stocks $s_i, s_j\in\mathcal{S}$, if $r_i^t\geq r_j^t$, there exists an overall order between the ranks $y_i^t\geq y_j^t$.
% such that there exists an overall order between the ranks $y_i^t\geq y_j^t$ for any two stocks $s_i, s_j\in\mathcal{S}$, if $r_i^t\geq r_j^t$.
Such an ordering of stocks $\mathcal{S}$ on a trading day $t$ represents a ranking list, where stocks achieving higher ranking scores $Y$ are expected to achieve a higher
investment revenue (profit) on day $t$. 

\textbf{Definition 2. Industry Decay Matrix.}
Investors believe that firms with similar industry characteristics should earn similar returns on average~\cite{cisthpan}. The Industry Decay Matrix $D$ is designed to enhance the similarity characteristics within industries. Specifically, it assigns different decay coefficients to firms that belong to the same secondary industry $Se(\cdot)$, the same primary industry $Pr(\cdot)$, or different industries.

\begin{figure}[t]
    \centering
    \includegraphics[width=0.45\textwidth]{images/industry.png}
    \caption{
         Stock movements among different primary and secondary industries.
    }
    \Description{..}
    \label{fig:industry}
\end{figure}

\textbf{Example 1.}
\cref{fig:industry} shows an example of fairly consistent movements among stocks within the same secondary industries. Moreover, within the same primary industries, such as Software \& Services and Semiconductors, there are also similar movements over certain periods, reflecting higher-level relationships.% between the stocks.

\textbf{Definition 3. Dynamic Stock Correlation Graph.}
Given that stock relationships are subject to daily changes and shaped by market dynamics, we propose a Dynamic Stock Correlation Graph (DSCG) to capture and represent them. Let's define the DSCG as $\mathcal{G}=\{g^t\}_{t=1}^L$, where $g^t=\{\mathcal{V}, \mathcal{E}^t\}$ represents a specific stock correlation for one day. In graphs, the node set $\mathcal{V}$ denotes each stock and the edge set $\mathcal{E}$ represents stock correlations. 
Each node $v_i$ corresponds to a stock $s_i$.
Each edge $e_{ij}^t$ is assigned a weight $A^t[i,j]$, representing the relationship between $s_i$ and $s_j$ on trading day $t$, where $A^t$ is the adjacency matrix. The edges with the top $K$ similarities indicate the dominant interactive relationships on trading day $t$.

\textbf{Definition 4. Market Index.}
A market index $M$ is defined as a statistical measure that tracks the performance of a specific segment of the macroeconomic financial market.



\begin{figure*}[!t]
    \centering
    \includegraphics[width=0.9525\textwidth]{images/pipline.pdf}
    \caption{
    Overall structure of proposed \name. 
    1. Market-Aware Graph models both the prior long-term and the posterior short-term correlations between stocks and adaptively selects dominant relationships based on macroeconomic market index trends. 
    2. Multi-Level Mamba captures similar stock movement patterns across both coarse-grained and fine-grained levels.
    }
    \Description{..}
    \label{fig:pipline}
\end{figure*}


\textbf{Example 2.}
\cref{fig:corr&index} visualizes the correlation between the market index and individual stocks in the NASDAQ 100 from 2018 to 2023. A clear pattern emerges: \textbf{\textit{the stock correlation strengthens as the market index falls and weakens as the market index rises}}.
% 2020年3月,新冠疫情在美国爆发;OPEC和俄罗斯未达成减产协议、国际原油暴跌。美国企业债债务风险暴露,美元流动性危机爆发,无差别抛售导致国债收益率大幅反弹、美股四次熔断
For instance, in March 2020, the COVID-19 outbreak in the U.S. revealed corporate debt risks and triggered a liquidity crisis, causing four circuit breakers in the stock market. %causing a sharp rebound in Treasury yields and four circuit breakers in the stock market.

% 有几种猜想 1.投资人角度 2.巨大现金流进入 3. ...
We propose the following \textbf{\textit{hypothesis}} to explain this phenomenon:
(1) From the perspective of investors and risk concentration, when the market index declines, systemic risk rises and investor pessimism increases, leading to greater synchronization among stocks and higher correlation. Conversely, when the market performs well, optimism drives capital into diverse sectors, causing stock performance to diverge and reducing correlation.
%From the perspective of investors and risk concentration, a decline in the market index is often accompanied by an increase in systemic risk across the market. Investors generally take a more pessimistic view of the future during such periods. As a result, most stock prices are driven by the same macroeconomic factors and market sentiment that are systemic in nature, leading to greater synchronization among stocks. This causes the correlation between stocks to increase. Conversely, when the market is doing well or the index is higher, investor optimism typically drives capital into different sectors and individual stocks, particularly those with different growth potential. During such periods, the performance of different industries or individual stocks diverges, leading to a decrease in stock correlations.
(2) When the index rises, significant capital inflows into individual stocks can lower serial correlations, as diversified investment strategies spread buying interest, making price movements less synchronized.
% When the index rises, large inflows of capital into individual stocks can reduce serial correlations. This is because diversified investment strategies can spread buying interest across different stocks, causing their price movements to become less synchronized.


\textbf{Problem 1. Stock Movement Prediction.}
Formally, given the stock-specific information (e.g., historical price, stock correlation) of $\mathcal{S}$, we aim to learn a ranking function that outputs a score $Y^{L+1}$ to rank each stock $s_i$ on the next day regarding expected profit.



\section{Method}

In the following sections, we will describe the architecture of the \name as illustrated in \cref{fig:pipline}, including the Market-Aware Graph and the Multi-Level Mamba.
Market-Aware Graph extracts the dependencies between stocks over a period of time under specific market conditions, including both the posterior short-term relationship between stock sequences and the prior long-term industry relationship, while Multi-Level Mamba effectively captures similar stock movement patterns at multiple levels.


\subsection{Market-Aware Graph}

Market-Aware Graph captures stock node representations from graphs generated on each trading day and consists of three key components: Dynamic Stock Correlation Graph Generation, Market-Aware Graph Sparsification, and Graph Attention Aggregation.

\paragraph{\textbf{Dynamic Stock Correlation Graph Generation}}
To model the comprehensive inter-stock correlations, represented by the adjacency matrix $A$, we approach from the perspective of the posterior sequence correlations $Q$ and the prior relationships $D$ of the primary and secondary industries.

% Graph Snapshot
% 预先定义的静态图的问题; 直接基于历史信息构造关系的可行性和合理之处 (公司间有多种关系,例如 shareholders and invested companies, Franchisor and Franchisee, Parent Company and Subsidiary 难以用预先定义的图全面表示; 而直接利用相似度计算关系是容易实现且有效的,不会被有歧义或错误的披露和新闻报道影响)
% 逻辑有问题,应该改为: 股票间关系是复杂多方面的,从先验静态关系图和后验历史相似度单独建模都是不足够的,因此我们结合了两者
% In light of the relationships between stocks are multifaceted and changing on a daily basis~\cite{MDGNN}, it is evident that the methodologies employed in previous studies~\cite{sgnn,sgnn1}, which establish static correlation graphs relied on predefined concepts or rules, or directly generating relationships based on market history trends~\cite{tcgpn, ECHOGL} without extra domain-specific knowledge or news sources, are inadequate for accurately capturing the dynamic nature of stock correlations in real-time. 
% In contrast, directly generating relationships based on market history trends has shown practical effectiveness~\cite{tcgpn, ECHOGL}, as it does not require extra domain-specific knowledge or news sources and is relatively straightforward to implement.

In light of the fact that relationships between stocks are multifaceted and evolve on a daily basis~\cite{MDGNN}, it is evident that the methodologies employed in previous studies, which rely on prior static correlation graphs based on predefined concepts or rules~\cite{sgnn,sgnn1}, or those that generate posterior relationships directly from market history trends without incorporating additional domain-specific knowledge or news sources~\cite{thgnn, tcgpn, ECHOGL}, are inadequate to accurately capture the dynamic and real-time inter-stock correlations.

Given these considerations, we first calculate the posterior Spearman coefficient between the actual stocks $s_i,i\in\{1,2,...,N\}$ within a $L$-length look-back window to measure the interactions between the stocks, yielding the similarity matrix $Q\in\mathbb{R}^{N\times N}$. On trading day $t$, the similarity between $s_i$ and $s_j$ can be formalized with:

\begin{equation}
    Q_{ij}^{t}=1-\frac{6\sum_{c=t-L+1}^{t}(R(s_i^c)-R(s_j^c))^2}{L(L^2-1)}
\end{equation}
where $R(\cdot)$ is the rank function.

% 单一的关系(比如历史的相似度)不能很好的反正整个市场,需要额外引入行业信息
% Furthermore, the intricate and multifaceted nature of stock markets makes it inadvisable to rely on a single type of relationship (e.g., historical similarity) to capture the complex interconnections between stocks~\cite{MDGNN}. 
% It is essential to examine correlations from a more comprehensive perspective.% in order to gain a more accurate understanding of the overall performance of the markets.
Subsequently, we propose the prior Industry Decay Matrix $D\in\mathbb{R}^{N\times N}$, which models the relationships from an industry perspective, thereby extending the analysis beyond the inherent correlations within financial sequences themselves. To model the variety of different types of relationships existing between companies (e.g., Franchisor and Franchisee, shareholders and invested companies), we introduce primary $Pr(\cdot)$ and secondary $Se(\cdot)$ industries, and define $D$ as follows:
\begin{equation}
D_{ij}=
\begin{cases}
    \ 1, \ \ \ \text{if}\ \ Se(i)=Se(j)\\
    \delta_1, \ \ \text{elif}\ \ Pr(i)=Pr(j)\\
    \delta_2, \ \ \text{otherwise}
\end{cases}
\end{equation}

Thus, $Q$ represents a posterior short-term relationship that changes daily, while $D$ is a constant prior long-term relationship. Ultimately, we obtain an adjacency matrix $A^t\in\mathbb{R}^{N\times N}$ by performing a dot product $A^t=Q^t\cdot D$, where $A^t[i,j]$ indicates an edge $e_{ij}^t\in\mathcal{E}^t$. 
Then a fully connected graph $g^t=\{\mathcal{V}, \mathcal{E}^t\}$, representing a comprehensive and refined correlations between stocks $S$, is generated.


\paragraph{\textbf{Market-Aware Graph Sparsification}}
% 在伴随市场变化的长期实践中,投资者的一个基本观察结果是,股票间相互作用动态地开始生效并到期。传统投资者反复进行统计检查,筛选所有潜在的股票对,以识别选择共整合的关系进行投资组合。这些特征的选择过程繁琐被证明是非常低效的;此外,该策略只能投资于有限数量的共整合股票,导致投资风险更高,策略稳定性较低;且与基于深度学习的方法相结合时面临挑战。为了节省人力,我们设计了Market-Aware Graph Sparsification,它结合了市场信息来进行自动股票间关系选择

In the long-term process of adapting to market changes, a fundamental observation of investors is that interactions between stocks evolve over time, dynamically activating and expiring. This means that certain relationships may effectively reflect similar stock movements in certain market conditions but not in others.
Traditional investors perform repeated statistical checks to identify relationships that are effective for portfolio selection, which is labour-intensive and challenging when combined with deep learning-based methods. To reduce manual effort, we propose the Market-Aware Graph Sparsification module, which incorporates market information to automatically select important inter-stock relationships.

Specifically, to ensure fairness without introducing additional information, for trading day $t$, the market index $M^t\in\mathbb{R}^{1\times L\times F}$ is represented by the mean value of the stocks included in the target set of length $L$ preceding that day, rather than utilising an actual market index. 
The process of calculating market-aware sparsity level is formalized as follows:
\begin{equation}
    \kappa^t=\tau\cdot \text{Sigmod}(\text{Inception}(M^t))
\end{equation}
where $\tau$ is a constant used to adjust the sparsity level. 
First, the 2D tensor $M$ is processed by a parameter-efficient inception block~\cite{inception}, designated as $Inception(\cdot)$, which incorporates multi-scale 2D kernels and is widely recognized as a prominent vision backbone. 
Subsequently, the output of $Inception(\cdot)$ passes through the $sigmoid$ activation function to obtain the pruning sparsity level $\kappa^t$. 

Finally, adaptive pruning is performed on the already generated fully connected graph $g^t$ to obtain the dominant $K$ correlations between stocks as perceived by the market over a certain period. The process of sparsification is formalized as follows:
\begin{equation}
    g_{final}^{t}=\{V,\mathcal{E}^t\backslash\mathcal{E}^{t'}\},\
    \mathcal{E}^{t'} = \text{topK}(-A^t[i,j],\lceil \kappa^t \times |\mathcal{E}^{t}| \rceil)
    %\kappa^t=\frac{|\mathcal{E}^{t'}|}{|\mathcal{E}^t|}
\end{equation}
where $g_{final}^t$ only modifies the edge set $\mathcal{E}^t$ without altering the node set $\mathcal{V}$, and $\mathcal{E}^{t'}$ denotes the removed edges. The retained edges $e_{ij}^t\in\mathcal{E}^t\backslash\mathcal{E}^{t'}$ correspond to the edges with the top $K$ largest weights $A^t[i,j]$, where $K=|\kappa^t\cdot N^2|$.


\paragraph{\textbf{Graph Attention Aggregation}}
% As stated above, a set of DSCG $\mathcal{G}$ is generated based on historical similarity and industry relationships between stocks. %The approach allows for the capture of intricate representations of stock relationships. 
After generating a set of DSCG $\mathcal{G}$, we leverage a multi-head attention mechanism to aggregate the messages from neighboring nodes within graph structures. 
Concretely, for each stock $s_i$ (node $v_i$) on the trading day $t$, we compute the attention coefficient $\alpha_{ij}^t$ of its neighboring stock $s_j$ (node $v_j$), denoting the importance of the edge $e_{ij}^t$ in the pruned graph $g_{final}^t$. It can formalized with:
\begin{equation}
    \alpha_{v_i, v_j}^{k} = \frac{\text{exp}(\text{LeakyReLU}(a_k^T[Wh_i^t||Wh_j^t]))}{\sum_{s_u^t\in\mathcal{N}(v_i)}\text{exp}(\text{LeakyReLU}(a_k^T[Wh_i^t||Wh_u^t]))}
\end{equation}
where $h_i^t, h_j^t\in\mathbb{R}^{L\times F}$ is the feature representation of nodes $v_i$ and $v_j$, $W\in\mathbb{R}^F$ is the learnable weight matrix, $a_k^T\in\mathbb{R}^{2F}$ is the weight vector of relation $e_{ij}^t$ in $k$-th head and function $\mathcal{N}(v_i)$ denotes the set of neighbors of node $v_i$.

Then we aggregate the features of neighboring nodes of node $v_i$ using attention coefficients to generate a neighboring representation $z_i^t\in\mathbb{R}^{1\times L\times F}$:

% \vskip -0.1in

\begin{equation}
z_i^t=\sigma\bigg(\bigg|\bigg|_{k=1}^{K}\sum_{v_j\in\mathcal{N}(v_i)}\alpha_{v_i, v_j}^{k}h_j^t\bigg)
\end{equation}


where $\sigma$ denotes a non-linear activation function GELU and $\big|\big|_{k=1}^{K}$ denotes the concatenation of the outputs from all $K$ heads.

Once we have the original representations $s_i^t$ and the neighboring representation $z_i^t$, we concatenate them and get the stock embedding $P=\text{Stack}(s_i^t||z_i^t)|_{i=1}^{N}\ \in\mathbb{R}^{N\times L\times 2F}$.

\subsection{Multi-Level Mamba}

Compared to other financial time series forecasting methods (e.g., RNN, CNN, Transformer), Mamba introduces a input-dependent selection mechanism that balances short-term and long-term dependencies, which is suitable for modeling stock data where similar stock patterns occur over different time spans.
% 之前模块服务mamba
In earlier modules, we have already generated sequences that capture both short-term and long-term relationships, providing a strong foundation for this adaptive mechanism.
However, stock data exhibit different temporal variations across various scales. Intuitively, in a stock price series, daily fluctuations are influenced by monthly economic trends, which in turn are influenced by annual market cycles. To better model the diverse patterns within the non-stationary dynamics of stocks, we designed a Multi-Level Mamba.

% 多个线性映射为什么是多尺度
% 多个线性映射共同作用下,数据被投影到多个不同的特征空间中。这些特征空间各自捕捉到了输入数据的不同尺度,提供了更全面的特征表示
Technically, given the input $P\in\mathbb{R}^{N\times L\times 2F}$, we project it onto $k$ different levels through $k$ distinct linear mappings, each designed to capture features at varying levels of granularity.
These mappings, parameterized by unique transformation matrices, allow the model to extract multi-level pattern-oriented information by emphasizing either fine-grained local details or broader global patterns, enabling a richer representation.
For each level $i\in\{1,2,...,k\}$, it is fed into Mamba block, where the continuous state space mechanism produces a response $o^t_i\in\mathbb{R}^{N\times D}$ based on the observation of hidden state $h^t_i\in\mathbb{R}^{N\times D}$ and the input $x_i^t=\text{Linear}_i(P)^t\in\mathbb{R}^{N\times 2F}$, where $D$ is the output dimension. It can be formulated as:
\begin{equation}
\begin{aligned}
    h^t_i &= \bar{A}^t_i h^{t-1}_{i}+\bar{B}^t_i x_i^t\\
    o^t_i &= C^t_i h^t_i
\end{aligned}
\end{equation}
where $\bar{A}^i_t$, $\bar{B}^t_i$ and $C^t_i$ are the parameters of Mamba block~\cite{mamba}.

Afterward, we apply another linear mapping to restore the original level and concatenate them. Finally, a linear layer is used to obtain the predicted score $y^t\in\mathbb{R}^{N\times 1}$ of next trading day and $Y$ is formed by concatenating $y^t$ along the time dimension.
\begin{equation}
    y^t = \text{Linear}(||_{i=1}^{k}[\text{Re-Level-Proj.}(o^t_i+x_i^t)])
\end{equation}

Empowered by the multi-level state space mechanism, \name learns similar stock patterns over random intervals at different levels, enabling it to integrate complementary forecasting capabilities from mixed multi-level series.


\renewcommand{\arraystretch}{0.95}
\begin{table*}[!ht]
\setlength{\tabcolsep}{2pt}
\caption{Performance evaluation of compared models for financial time series forecasting in CSI 300, CSI 500, S\&P 500 and NASDAQ 100 datasets. The best and second-best results are in \best{bold} and \second{underlined}, respectively.}
\resizebox{\textwidth}{!}
{
\begin{tabular}{c|ccccc|ccccc|ccccc|ccccc}
\toprule
 \multicolumn{1}{c}{\multirow{2}{*}{Models}} & \multicolumn{5}{c}{CSI 300} & \multicolumn{5}{c}{CSI 500} & \multicolumn{5}{c}{S\&P 500}& \multicolumn{5}{c}{NASDAQ 100} \\
      
\cmidrule(lr){2-6} \cmidrule(lr){7-11} \cmidrule(lr){12-16} \cmidrule(lr){17-21}

\multicolumn{1}{c}{} & ARR$\uparrow$ & AVol$\downarrow$ & MDD$\downarrow$ & ASR$\uparrow$ & IR$\uparrow$ & ARR$\uparrow$ & AVol$\downarrow$ & MDD$\downarrow$ & ASR$\uparrow$ & IR$\uparrow$ & ARR$\uparrow$ & AVol$\downarrow$ & MDD$\downarrow$ & ASR$\uparrow$ & IR$\uparrow$ & ARR$\uparrow$ & AVol$\downarrow$ & MDD$\downarrow$ & ASR$\uparrow$ & IR$\uparrow$ \\
\midrule
BLSW~\citeyearpar{blsw} & -0.076 & \best{0.113} & -0.231 & -0.670 & \second{0.311} & 0.110 & 0.227 & -0.155 & 0.485 & 0.446 & 0.199 & 0.318 & -0.223 & 0.626 & 0.774 & 0.368 & 0.339 & -0.222 & 1.086 & 1.194 \\
CSM~\citeyearpar{csm} & -0.185 & 0.204 & -0.293 & -0.907 & -0.935 & 0.015 & 0.229 & -0.179 & 0.066 & 0.001 & 0.099 & 0.250 & -0.139 & 0.396 & 0.584 & 0.116 & 0.242 & -0.145 & 0.479 & 0.603 \\
\midrule
% PPO~\citeyearpar{ppo} & -0.096 & \best{0.045} & \second{-0.120} & -2.138 & -2.234 & -0.032 & \best{0.015} & \best{-0.040} & -2.041 & -2.075 & 0.020 & \best{0.089} & \best{-0.067} & 0.220 & 0.263 & 0.148 & \best{0.118} & \second{-0.104} & 1.259 & 1.237 \\
AlphaStock~\citeyearpar{alphastock} & -0.164 & 0.153 & -0.245 & -1.072 & -1.098 & -0.017 & 0.148 & -0.166 & -0.115 & -0.043 & 0.122 & 0.140 & -0.126 & 0.871 & 0.892 & 0.372 & 0.178 & -0.134 & 1.781 & 1.869 \\
DeepPocket~\citeyearpar{deeppocket} & -0.036 & 0.135 & -0.175 & -0.270 & -0.258 & 0.006 & \best{0.127} & -0.148 & 0.050 & 0.115 & 0.165 & \second{0.142} & -0.126 & 1.165 & 1.045 & 0.346 & \best{0.157} & -0.116 & 2.197 & 1.882 \\
\midrule
Transformer~\citeyearpar{attention} & -0.240 & 0.156 & -0.281 & -1.543 & -1.695 & 0.154 & 0.156 & -0.135 & 0.986 & 0.867 & 0.135 & 0.159 & -0.140 & 0.852 & 0.908 & 0.268 & 0.175 & -0.131 & 1.531 & 1.441 \\
Mamba~\citeyearpar{mamba} & -0.141 & 0.155 & -0.290 & -0.910 & -1.240 & 0.076 & 0.157 & -0.180 & 0.484 & 0.337 & 0.141 & 0.170 & -0.156 & 0.829 & 0.892 & 0.281 & 0.170 & -0.143 & 1.651 & 1.504 \\
FactorVAE~\citeyearpar{FactorVAE} & -0.048 & \second{0.134} & -0.175 & -0.335 & -0.348 & 0.006 & \best{0.127} & -0.147 & 0.047 & 0.112 & 0.160 & \second{0.142} & -0.132 & 1.128 & 1.013 & 0.356 & \second{0.159} & -0.119 & 2.234 & 1.907 \\
THGNN~\citeyearpar{thgnn} & -0.015 & 0.172 & -0.152 & \second{-0.088} & -0.003 & 0.048 & \second{0.128} & -0.141 & 0.375 & 0.432 & 0.271 & \best{0.141} & -0.094 & 1.921 & 1.778 & 0.644 & 0.204 & -0.146 & 3.147 & 2.543 \\ 
MambaStock~\citeyearpar{mambastock} & -0.132 & 0.158 & -0.227 & -0.836 & -0.857 & 0.106 & 0.154 & -0.158 & 0.690 & 0.339 & 0.145 & 0.158 & -0.146 & 0.916 & 0.932 & 0.280 & 0.166 & -0.140 & 1.684 & 1.489 \\
CL~\citeyearpar{CL} & -0.035 & 0.148 & -0.183 & -0.241 & -0.193 & 0.051 & 0.146 & -0.128 & 0.351 & 0.390 & 0.308 & 0.189 & -0.171 & 1.629 & 1.451 & 0.351 & 0.172 & -0.122 & 2.041 & 1.821 \\
MASTER~\citeyearpar{master} & \second{0.102} & 0.151 & \second{-0.126} & 0.681 & 0.726 & 0.128 & 0.130 & \best{-0.098} & 0.989 & 0.997 & \second{0.335} & 0.171 & -0.134 & \second{1.958} & \second{1.895} & \second{0.654} & 0.188 & -0.102 & \second{3.479} & 2.683 \\ 
CI-STHPAN~\citeyearpar{cisthpan} & -0.078 & 0.167 & -0.144 & -0.466 & -0.355 & 0.021 & 0.151 & -0.129 & 0.136 & 0.211 & 0.123 & 0.233 & -0.254 & 0.527 & 0.632 & 0.454 & 0.208 & -0.124 & 2.178 & 1.855 \\ 
VGNN~\citeyearpar{VGNN} & -0.037 & 0.163 & -0.197 & -0.227 & -0.201 & 0.111 & 0.166 & -0.175 & 0.668 & 0.564 & 0.299 & 0.202 & -0.169 & 1.473 & 1.406 & 0.616 & 0.181 & \second{-0.099} & 3.405 & \second{2.798} \\ 
\midrule
PatchTST~\citeyearpar{patchtst} & -0.224 & 0.158 & -0.279 & -1.415 & -1.563 & 0.118 & 0.152 & -0.127 & 0.776 & 0.735 & 0.146 & 0.167 & -0.140 & 0.877 & 0.949 & 0.239 & 0.185 & -0.138 & 1.296 & 1.233 \\
iTransformer~\citeyearpar{itransformer} & -0.115 & 0.145 & -0.190 & -0.793 & -0.775 & \second{0.214} & 0.168 & -0.164 & \second{1.276} & \second{1.173} & 0.159 & 0.170 & -0.139 & 0.941 & 0.955 & 0.188 & 0.196 & -0.202 & 0.963 & 0.937 \\
TimeMixer~\citeyearpar{timemixer} & -0.156 & 0.159 & -0.232 & -0.983 & -1.028 & 0.078 & 0.153 & -0.114 & 0.511 & 0.385 & 0.254 & 0.162 & -0.131 & 1.568 & 1.448 & 0.264 & 0.188 & -0.131 & 1.401 & 1.346 \\
Crossformer~\citeyearpar{crossformer} & -0.071 & 0.162 & -0.437 & -0.383 & -0.039 & 0.163 & 0.217 & -0.238 & 0.686 & 0.650 & 0.284 & 0.159 & \second{-0.114} & 1.786 & 1.646 & 0.363 & 0.181 & -0.167 & 2.010 & 1.860 \\
\midrule
\name (Ours) & \best{0.106} & 0.163 & \best{-0.102} & \best{0.653} & \best{0.705} & \best{0.227} & 0.163 & \second{-0.106} & \best{1.389} & \best{1.339} & \best{0.341} & 0.164 & \best{-0.086} & \best{2.063} & \best{1.937} & \best{0.685} & 0.190 & \best{-0.097} & \best{3.601} & \best{2.851} \\
\bottomrule
\end{tabular}
}
\label{tab:mainexp}
\end{table*}
\renewcommand{\arraystretch}{1.0}



\subsection{Optimization Objectives}
In stock movement prediction, the learning goal of \name is to estimate the predicted $y^t$ denoting the positive return of all $N$ stocks on trading day t. For overall optimization, we combine point-wise regression loss and pair-wise ranking loss as:
\begin{equation}
    \mathcal{L}_{RP}=\frac{1}{L}\sum_{t=1}^{L}(\sum_{i=1}^{N}||y_i^t-r_i^t||^2+\eta\sum_{i=1}^{N}\sum_{j=1}^{N}max(0,-(y_i^t-y_j^t)(r_i^t-r_j^t)))
\end{equation}

Moreover, to guarantee that the learned stock embeddings efficiently capture the correlation within the dynamic graph structure and address the graph information bottleneck~\cite{gib} issue by minimizing uncertainty in the graph, we incorporate GIB loss $\mathcal{L}_{GIB}$ to minimize mutual information between the aggregation embedding and the original input, ensuring key information is retained while reducing irrelevant details.


\begin{equation}
    \mathcal{L}_{GIB}=I(Z;X)=\frac{1}{L}\sum_{t=1}^{L} \bigg(\sum_{i=1}^{N}\frac{||\text{mean}(z_i^t)-\text{mean}(s_i^t)||^2}{\text{var}(z_i^t)+\text{var}(s_i^t)}\bigg)
\end{equation}

Combining these two loss, we reach the complete end-to-end loss function with a weighting coefficient $\lambda$:
\begin{equation}
    \mathcal{L}=\mathcal{L}_{RP} + \lambda\mathcal{L}_{GIB}
\end{equation}



\section{Expeiments}


\subsection{Experiment Setup}

\paragraph{\textbf{Datasets.}}
We conduct thorough experiments in both the U.S. and Chinese stock markets, selecting entities from the 
S\&P 500\footnote{\url{https://hk.finance.yahoo.com/quote/\%5EGSPC/history/}}, 
NASDAQ 100\footnote{\url{https://hk.finance.yahoo.com/quote/\%5EIXIC/history}}, 
CSI 300\footnote{\url{https://cn.investing.com/indices/csi300}}, and 
CSI 500\footnote{\url{https://cn.investing.com/indices/china-securities-500}}. Our datasets comprise historical day-level market information, such as close, open, high, low, turnover and volume, from 2018 to 2023. See \cref{app:data} for a more detailed description of the datasets.

% ~\footnote{\url{https://hk.finance.yahoo.com/quote/\%5EGSPC/history/}}
% ~\footnote{\url{https://hk.finance.yahoo.com/quote/\%5EIXIC/history}}
% ~\footnote{\url{https://cn.investing.com/indices/csi300}}
% ~\footnote{\url{https://cn.investing.com/indices/china-securities-500}}

\paragraph{\textbf{Baseline Models.}}
We compare \name with other competitive models from different categories as follows: 
(1) Quantitative Investment Methods:
a. Classic strategies: Buying-Loser-Selling-Winner (BLSW)~\cite{blsw} and Cross-Sectional Mean reversion (CSM)~\cite{csm}; 
b. Deep Reinforcement Learning methods: AlphaStock~\cite{alphastock}, DeepPocket~\cite{deeppocket};
c. Deep Learning methods: Transformer~\cite{attention}, Mamba \cite{mamba}, FactorVAE~\cite{FactorVAE}, THGNN~\cite{thgnn}, MambaStock~\cite{mambastock}, CL~\cite{CL}, MASTER~\cite{master}, CI-STHPAN~\cite{cisthpan} and VGNN~\cite{VGNN};
(2) General Time Series Forecasting methods: PatchTST \cite{patchtst}, iTransformer~\cite{itransformer}, TimeMixer \cite{timemixer}, and Crossformer~\cite{crossformer}.

\paragraph{\textbf{Metric.}}
We employ five widely used metrics to evaluate the overall performance of each method: Annual Return Ratio (ARR), Annual Volatility (AVol), Maximum Draw Down (MDD), Annual Sharpe Ratio (ASR), and Information Ratio (IR). Lower absolute values of AVol and MDD, along with higher ARR, ASR, and IR, indicate better performance. See \cref{app:metric} for detailed descriptions.


\paragraph{\textbf{Implementation Details.}}
Our experiment is trained on the NVIDIA V100 GPU, and all models are built using PyTorch \cite{pytorch}.
The training and validation sets are kept consistent for all models. The number of GNN layers is 2, the number of levels is 2, and the window size is 20. See \cref{app:imp} for detailed settings.


\renewcommand{\arraystretch}{0.85}
\begin{table*}[t]
\caption{Component ablation of \name in CSI 500 and NASDAQ 100 datasets.}
\resizebox{0.95\textwidth}{!}
{
\begin{tabular}{c|cccccc|cccccc}
\toprule
 \multicolumn{1}{c}{\multirow{2}{*}{Models}} & \multicolumn{6}{c}{CSI 500} & \multicolumn{6}{c}{NASDAQ 100} \\
      
\cmidrule(lr){2-7} \cmidrule(lr){8-13}

\multicolumn{1}{c}{} & ARR$\uparrow$ & AVol$\downarrow$ & MDD$\downarrow$ & ASR$\uparrow$ & CR$\uparrow$ & \multicolumn{1}{c}{IR$\uparrow$} & ARR$\uparrow$ & AVol$\downarrow$ & MDD$\downarrow$ & ASR$\uparrow$ & CR$\uparrow$ & IR$\uparrow$  \\
\midrule
w/o industry decay matrix & 0.146 & 0.165 & -0.111 & 0.887 & 1.312 & 0.914 & 0.677 & 0.190 & -0.101 & 3.559 & 6.717 & 2.825 \\
w/o Market-Aware Sparsification & 0.064 & 0.162 & -0.186 & 0.391 & 0.341 & 0.462 & 0.399 & 0.184 & -0.112 & 2.166 & 3.543 & 1.923 \\
w/o Dynamic Stock Correlation & 0.128 & 0.168 & -0.113 & 0.757 & 1.125 & 0.799 & 0.384 & 0.188 & -0.115 & 2.043 & 3.351 & 1.831 \\
single level Mamba & 0.129 & 0.163 & -0.110 & 0.789 & 1.172 & 0.827 & 0.432 & 0.184 & -0.115 & 2.347 & 3.754 & 2.052 \\
w/o Mamba & 0.046 & \best{0.148} & -0.130 & 0.310 & 0.353 & 0.378 & 0.312 & \best{0.179} & -0.149 & 1.743 & 2.098 & 1.631 \\
FinTransformer & 0.194 & 0.168 & -0.144 & 1.152 & 1.352 & 1.142 & 0.671 & 0.183 & -0.117 & \best{3.676} & 5.726 & 2.717 \\
\midrule
\name & \best{0.227} & 0.163 & \best{-0.106} & \best{1.389} & \best{2.135} & \best{1.339} & \best{0.685} & 0.190 & \best{-0.097} & 3.601 & \best{7.085} & \best{2.851} \\
\bottomrule
\end{tabular}
}
\label{tab:ablation}
\end{table*}
\renewcommand{\arraystretch}{1.0}

\begin{figure*}[!htb]
    \centering
    \begin{subfigure}[!b]{0.73\textwidth}
        \centering
        \includegraphics[width=\textwidth]{images/mamba_embed.pdf} 
        \caption{}
        \label{fig:subfig1}
    \end{subfigure}
    \hfill
    \begin{subfigure}[!b]{0.26\textwidth}
        \centering
        \includegraphics[width=\textwidth]{images/usage.pdf} 
        \caption{}
        \label{fig:subfig2}
    \end{subfigure}
    \caption{(a) \textcolor{mypurple}{a} to \textcolor{mypurple}{d} and \textcolor{myorange}{e} to \textcolor{myorange}{h} represent two groups of similar stock movement patterns. \name effectively captures strong intra-group correlations and exhibits low inter-group correlations. (b) Our proposed \name demonstrates superior efficiency and effectiveness, achieving lower inference time (ms/iter) and reduced memory usage (MB) with longer lookback horizons.}
    \label{fig:mamba_trans}
    \Description{..}
\end{figure*}

\paragraph{\textbf{Trading Protocols.}}
Following~\cite{thgnn}, we use the daily buy-hold-sell trading strategy to evaluate the performance of stock movement prediction methods in terms of returns. 
During the test period, the trading process for each day is simulated as follows: First, at the end of the trading on day $t$, the traders use \name to generate prediction scores and rank expected returns for all stocks. Then, at the opening of $t+1$, the traders sell the stocks purchased on day $t$ and buy those with higher predicted returns, focusing on the top $k$ ranked stocks. If a stock continues to be ranked with the highest predicted returns, it remains in the trader's portfolio. Notably, the experiment does not take transaction costs into account. 


\subsection{Experiment Results}

% 为什么 PPO 等强化学习算法的AVol和MDD这么低? 
% DRL它通过与市场环境的交互,不断调整策略,以应对不同市场条件的变化。这种自适应能力使得模型在降低风险(如MDD)和控制波动(AVol)方面表现更优; 目标函数通常是最大化长期累积回报,能够在长期持仓的基础上优化决策,从而降低长期的最大回撤和波动率
% 而深度学习时序预测模型通常只是根据历史数据来预测未来趋势,缺少实时决策和动态反馈的机制

% 为什么 stockmamba 的 MDD 相比深度学习时序预测模型也较低? 可能是通过剪枝引入与市场环境交互,有效评估复杂的非线性市场行为

The overall performance is reported in \cref{tab:mainexp}. Our proposed \name outperforms all other methods on most metrics. Based on these experimental findings, we draw the following conclusions:

\paragraph{\textbf{TSF methods.}} 
Deep learning-based general time series forecasting (TSF) methods, such as Crossformer~\cite{crossformer} and iTransformer~\cite{itransformer}, do not exploit the graph relationships between stocks, making it difficult for them to achieve state-of-the-art performance.
Moreover, such TSF methods rely primarily on historical data to predict future trends, lacking real-time decision-making and dynamic feedback mechanisms. Given the complexity of stock markets, relying solely on predictions is often insufficient to mitigate risk.

% \paragraph{\textbf{DRL methods.}} 
% Deep reinforcement learning (DRL) methods perform well on AVol and MDD metric, with DeepPocket~\cite{deeppocket} performing good across four datasets. However, \textbf{\textit{``You can't make an omelet without breaking eggs.''}}, which explains why these methods underperform on other critical return-related metrics such as ARR and ASR.
% Specifically, DRL methods continuously adjust its strategy by interacting with the market environment. Its objective function is typically focused on maximizing long-term cumulative returns, allowing for optimal long-term holding decisions. This adaptability allows DRL models to perform better at controlling volatility (AVol).

\paragraph{\textbf{Graph-based quantitative investment methods.}} 
Graph-based stock prediction models, such as THGNN~\cite{thgnn}, VGNN~\cite{VGNN} and CI-STHPAN~\cite{cisthpan} perform exceptionally well, demonstrating that incorporating relationships between stocks can significantly improve predictive performance. By leveraging graph structures to simulate synergies, these models capture the complex interactions and dependencies that exist between different stocks.

\paragraph{\textbf{Marker-Aware quantitative investment methods.}} 
The proposed \name refines the inter-stock dependencies in the global set of stocks based on market feedback, while MASTER~\cite{master} employs a gating mechanism to select effective factors for individual stocks, achieving \best{optimal} and \second{suboptimal} results respectively. This demonstrates the powerful ability of market information to optimize portfolio strategies. It is important to note that \name performs well in ARR, ASR, CR, and IR across all four datasets, highlighting its superiority in stock movement prediction. \name also shows strong performance in MDD, confirming that market-aware graph effectively captures complex, non-linear market behaviors by incorporating interactions with the market environment.


\subsection{Ablation Study}

To evaluate the effectiveness of each module in \name, we perform extensive ablation experiments on the CSI 500 and NASDAQ 100 datasets, with results shown in \cref{tab:ablation}. 
Excluding the industry decay matrix $D$ (w/o industry decay matrix) reduces performance, confirming the importance of our primary and secondary industry-related graph construction. 
Omitting Market-Aware Sparsification (w/o Market-Aware Sparsification) leads to a significant drop in performance, highlighting the module's role in leveraging the relationship between the macroeconomic market index and whole stocks set and eliminating unnecessary noise introduced by irrelevant edges. 
Furthermore, ignoring stock relationships and treating Mamba purely as a time-series prediction task (w/o Dynamic Stock Correlation) or ignoring time-series information and using GNN purely for aggregating stock relationships (w/o Mamba) both result in performance degradation.


For a detailed comparison between Transformer~\cite{attention} and Mamba \cite{mamba},
% 指标
we first replace Mamba in \name with Transformer (\textit{FinTransformer}) for experiments, resulting in a slight performance decline.
% 可视化
Subsequently, we visualize the model embeddings of the PEP stock from 23/06/21 to 23/09/20 in \cref{fig:subfig1} (i).
%, where the horizontal axis represents 64 trading days and the vertical axis represents the 64-dimensional embedding. 
In the daily candlestick chart, we identify two groups of similar stock movement patterns: \textcolor{mypurple}{a} to \textcolor{mypurple}{d} and \textcolor{myorange}{e} to \textcolor{myorange}{h}. 
In the \name embedding, we calculate the cosine similarity between pattern \textcolor{mypurple}{a} and all other patterns, and also for pattern \textcolor{myorange}{e}, as shown in \cref{fig:subfig1} (ii). The correlation matrix of the eight patterns is presented in \cref{fig:subfig1} (iii).
%\textcolor{mypurple}{a} and other time steps (top right bar graph) and similarly for pattern \textcolor{myorange}{e} (bottom right bar graph).
% 感觉逻辑不太对 要修改
% We observe that patterns \textcolor{mypurple}{a-d} exhibit similar movements, with Mamba identifying high similarity during these days, while a shows low similarity with \textcolor{myorange}{f}, \textcolor{myorange}{g}, and \textcolor{myorange}{h}. In contrast, 
We observe that \name effectively captures strong intra-group correlations, modeling well the temporal dependencies within similar patterns, and exhibits low inter-group correlations, reducing the impact of noise.
This suggests that \name can selectively retain similar inputs across different time intervals, making it highly effective for modeling stock data where recurring patterns may appear over varying time spans.
% In contrast, Transformer struggles to capture such information.
% Moreover, both embeddings display distinct vertical stripe patterns, possibly due to long-term features like price trends and trading volume, which remain consistent across different days. Furthermore, in the Transformer embeddings, the cosine similarity between different trading days is higher, indicating greater similarity and redundancy across days. This may explain why Mamba is better for modeling financial time series data.
% 内存效率
Lastly, we evaluate the model complexity in terms of memory usage (MB) and inference time (ms/iter) for longer lookback horizons, as depicted in \cref{fig:subfig2}. The lightweight and efficient design of \name, with its linear complexity, meets the timeliness requirement of algorithmic trading.

\begin{figure}[!t]
    \centering
    \includegraphics[width=0.48\textwidth]{images/topk_ratio.pdf}
    \caption{
         The retaining topK ratio with market index.
    }
    \Description{..}
    \label{fig:topk_ratio}
\end{figure}

\begin{figure}[!t]
    \centering
    \includegraphics[width=0.48\textwidth]{images/para_sensitivity.pdf}
    \caption{
         Portfolio performance of \name in terms of the length of lookback horizon $L$, the number of GNN layers, the weight of hinge loss $\eta$, and the number of levels $k$ in MLM.
    }
    \Description{..}
    \label{fig:para}
\end{figure}

\subsection{Sparsity Level $\kappa$ in Market-Aware Graph}

As shown in the \cref{fig:topk_ratio}, our Market-Aware Graph Sparsification is highly effective. When the market index is high, the relationships between stocks become weaker and it tends to retain fewer edges. Conversely, when the market index is low, the interactions between stocks become stronger and it tends to retain more edges.


\subsection{Parameter Sensitivity}
\cref{fig:para} shows the effect of four hyperparameters on the NASDAQ 100 dataset. It appears that increasing lookback horizons enhances performance, but only up to a certain length, beyond which the additional input introduces more noisy signals, diminishing the effectiveness of information capture. A similar pattern occurs when increasing the number of GNN layers and the number of levels in MLM. In addition, since hinge loss guides the model more effectively in learning ranking information, it improves ranking performance but is constrained by MSE loss, which limits prediction accuracy.


\subsection{Case Study}

On 15 September 2023, the United Auto Workers (UAW) initiated its first-ever simultaneous strike against the ``Detroit Big Three'' automakers: General Motors, Ford, and Stellantis. Lasting a month and a half, the strike is expected to hinder these companies' ability to compete with non-union automakers like Tesla and foreign brands. This case study examines the stock market's response, focusing on the relationship between GM (General Motors), F (Ford) and TSLA (Tesla).
As in \cref{fig:case}, the correlation between GM and F remained strong throughout all three periods, reflecting their close ties as traditional auto giants. In contrast, TSLA's correlation with GM and F was weaker in the first period, perhaps reflecting its distinct market position and growth prospects. However, in the subsequent periods, as the strike continued and competition intensified, the relationship between TSLA and the other two stocks notably strengthened. This suggests that \name more accurately captures the evolving dynamics between the stocks, reflecting their interdependencies.


\begin{figure}[!t]
    \centering
    \includegraphics[width=0.45\textwidth]{images/case_study.pdf}
    \caption{
         Case analysis of multifaceted inter-stock correlations. The upper half represents the correlation between the three stocks over different periods, while the lower half represents the stock price movements of the three stocks.
    }
    \Description{..}
    \label{fig:case}
\end{figure}

\vspace{-0.28em}


\section{Conclusion}

In this paper, we introduce a novel framework, \name, which integrates a Market-Aware Graph and Multi-Level Mamba architecture. Through comprehensive quantitative and qualitative analysis of real-world stock market data, including CSI, NASDAQ, and S\&P, we explore the ability of \name to effectively extract features from financial time series and dynamic relational graphs enriched by market feedback. Our analysis demonstrates its potential for accurate stock movement prediction across diverse market environments. Looking ahead, we will further explore the broader applications of the Mamba architecture in quantitative trading.


%%
%% The acknowledgments section is defined using the "acks" environment
%% (and NOT an unnumbered section). This ensures the proper
%% identification of the section in the article metadata, and the
%% consistent spelling of the heading.
% \begin{acks}
% To Robert, for the bagels and explaining CMYK and color spaces.
% \end{acks}

%%
%% The next two lines define the bibliography style to be used, and
%% the bibliography file.
\clearpage
\balance
\bibliographystyle{ACM-Reference-Format}
\bibliography{sample-base}


%%
%% If your work has an appendix, this is the place to put it.
\clearpage
\appendix

\section{Experiment Details}

\subsection{Datasets Details}\label{app:data}
For the four datasets (S\&P 500, NASDAQ 100, CSI300, CSI500), the train set is from 2018-01-01 to 2021-12-31, the validation set is from 2022-01-01 to 2022-12-31, and the test set is from 2023-01-01 to 2023-12-31. The statistics of each dataset is shown in \cref{tab:detaildataset}.

\renewcommand{\arraystretch}{0.5}
\begin{table}[htb]
\setlength{\tabcolsep}{6.5pt}
\caption{Detailed dataset descriptions.}
\resizebox{0.475\textwidth}{!}
{
\begin{tabular}{c|c|c|c|c}
\toprule

Dataset & Stocks (Nodes) & train set & validation set & test set \\

\midrule
CSI 300 & 285 & 943 & 242 & 242  \\
\midrule
CSI 500 & 450 & 943 & 242 & 242  \\
\midrule
S\&P 500 & 498 & 1008 & 251 & 249  \\
\midrule
NASDAQ 100 & 99 & 1008 & 251 & 249  \\

\bottomrule

\end{tabular}
}
\label{tab:detaildataset}
\end{table}


\subsection{Baseline Models}

We briefly describe the selected baseline models:

(1) Quantitative Investment Methods:

a. Classical Strategy:
\begin{itemize}
\item Buying-Loser-Selling-Winner (BLSW)~\cite{blsw}: which utilizes reversal strategies by identifying trends that are likely to reverse based on technical indicators and taking positions opposite to prevailing market direction.
\item Cross-Sectional Mean reversion (CSM)~\cite{csm}: whcih employs momentum strategies by identifying assets or securities with strong recent price trends and taking positions in the direction of those trends.
\end{itemize}

b. Deep Reinforcement Learning methods:
\begin{itemize}
\item AlphaStock~\cite{alphastock}: which is the first to offer an interpretable investment strategy using deep reinforcement attention networks. 
\item DeepPocket~\cite{deeppocket}: which consists of an autoencoder for feature extraction, a convolutional network to collect underlying information shared among financial instruments and an actor–critic RL agent. The source code is available at \url{https://github.com/MCCCSunny/DeepPocket}.
\end{itemize}

c. Deep Learning Methods:
\begin{itemize}
\item Transformer~\cite{attention} is a universal DL method featuring an Encoder-Decoder framework based on self-attention.
\item Mamba~\cite{mamba} is a state space model introducing a data-dependent selection mechanism that balances short-term and long-term dependencies.
\item FactorVAE~\cite{FactorVAE}: which integrates the dynamic factor model (DFM) with the variational autoencoder (VAE), and proposes a prior-posterior learning method which can approximate an optimal posterior factor model with future information. The source code is available at \url{https://github.com/ytliu74/FactorVAE}.
\item THGNN~\cite{thgnn}: which designs a temporal and heterogeneous graph neural network model to learn the dynamic relations among price movements in financial time series. The source code is available at \url{https://github.com/TongjiFinLab/THGNN}.
\item MambaStock~\cite{mambastock}: which mines historical stock market data with Mamba framework, eliminating the need for meticulous feature engineering or extensive preprocessing. The source code is available at \url{https://github.com/zshicode/MambaStock}.
\item CL~\cite{CL}: which predicts stock price movements by comparing textual and quantitative features of the current time interval against those of a prior time span for contrastive learning.
\item MASTER~\cite{master}: which models the momentary and cross-time stock correlation and leverages market information for automatic feature selection. The source code is available at \url{https://github.com/SJTU-DMTai/MASTER}.
\item CI-STHPAN~\cite{cisthpan}: which is a two-stage framework for stock selection, involving Transformer and HGAT based stock time series self-supervised pre-training and stock-ranking based downstream task fne-tuning. The source code is available at \url{https://github.com/Harryx2019/CI-STHPAN}.
\item VGNN~\cite{VGNN}: which is a decoupled graph learning framework for stock prediction with a tensor-based fusion module, a hybrid attention mechanism and a message-passing mechanism. The source code is available at \url{https://github.com/JiwenHuangFIC/VGNN}.
\end{itemize}

(2) Time Series Forecasting Methods:
\begin{itemize}
\item PatchTST~\cite{patchtst} is a Transformer-based model utilizing patching and channel independence technique. It also enable effective pre-training and transfer learning across datasets. The source code is available at \url{https://github.com/yuqinie98/PatchTST}.
\item iTransformer~\cite{itransformer} embeds each time series as variate tokens and is a fundamental backbone for time series forecasting. The source code is available at \url{https://github.com/thuml/iTransformer}.
\item TimeMixer~\cite{timemixer} is a Linear-based model enabling the combination of the multiscale information in both history extraction and future prediction phases. The source code is available at \url{hhttps://github.com/kwuking/TimeMixer}.
\item Crossformer~\cite{crossformer} is a Transformer-based model introducing the Dimension-Segment-Wise (DSW) embedding and Two-Stage Attention (TSA) layer to effectively capture cross-time and cross-dimension dependencies. The source code is available at \url{https://github.com/Thinklab-SJTU/Crossformer}.
\end{itemize}



\subsection{Metric Details}\label{app:metric}

We employ five widely used metrics to evaluate the overall performance of each method: Annual Return Ratio (ARR), Annual Volatility (AVol), Maximum Draw Down (MDD), Annual Sharpe Ratio (ASR) and Information Ratio (IR). The lower the absolute values of AVol and MDD, the higher the value of ARR, ASR, and IR, and the better the performance.
\begin{itemize}
    \item ARR measures the percentage increase or decrease in the value of an investment over the course of a year.
    \begin{equation}
        \text{ARR}=(1+\text{Total Return})^{\frac{1}{n}}-1
    \end{equation}
    \item AVol quantifies the volatility of an investment's return over the course of a year. $R_p$ denotes the daily return of the portfolio.
    \begin{equation}
        \text{AVol}=\sqrt{\text{Var}(R_p)}
    \end{equation}
    \item MDD represents the maximum drop from a peak to a trough in an investment's value.
    \begin{equation}
        \text{MDD}=-\text{max}\bigg(\frac{p_{peak}-p_{trough}}{p_{peak}}\bigg)
    \end{equation}
    \item ASR measures the risk-adjusted return of an investment over one year.
    \begin{equation}
        \text{ASR}=\frac{\text{ARR}}{\text{AVol}}
    \end{equation}
    % \item CR compares the average annual return of an investment to its maximum drawdown.
    % \begin{equation}
    %     \text{CR}=\frac{\text{ARR}}{|\text{MDD}|}
    % \end{equation}
    \item IR measures the excess return of an investment relative to a benchmark adjusted for its volatility. $R_b$ is the daily return of the market index. 
    \begin{equation}
        \text{IR} = \frac{\text{mean}(R_p-R_b)}{\text{std}(R_p-R_b)}
    \end{equation} 
\end{itemize}
     

\subsection{Implementation Details}\label{app:imp}
Our experiment is trained on one NVIDIA V100 GPU, and all models are built using PyTorch~\cite{pytorch}. The training and validation sets are kept consistent for all baseline models. The number of GNN layers is $2$, the level in MLM is $2$, and the lookback horizon is $20$. The learning rate is $0.01$ for S\&P 500 and NASDAQ 100, and $0.03$ for CSI 300 and CSI 500. The weight of the hinge loss is set to $3.0$, while the weight of the GIB loss is set to $1.0$. We set $k$ to $9$ for the selected top-$k$ ranked stocks in the experiment. 


\section{Extra Experimental Results}

Here we provide the extra results and analysis of experiments for \name.

\subsection{Additional Cases of Dynamic Interplay Between Stock Market Indices and Inter-Stock Correlations}

As shown in \cref{fig:corr&index}, the correlation between stocks becomes stronger when the market index falls and weaker when the market index rises. 
In particular, the relationship between stocks becomes very tight during several market downturns. 
For example, US stocks experienced a sharp pullback in the fourth quarter of 2018, driven by a combination of inflation concerns, interest rate hikes, a global economic slowdown and technical market factors. 
Similarly, from January to October 2022, inflationary pressures, post-pandemic supply chain disruptions and Federal Reserve policy shifts led to a prolonged decline in US equities.



\subsection{System Deployment Framework}
To validate our proposed \name in a realistic environment, we conduct an online deployment in the Chinese stock market.% from January 1, 2024 to June 30, 2024.
The deployed architecture of \name is shown in \cref{fig:deploy_framework}. To efficiently adapt to changing market conditions, the \name is updated offline once a week. The newly updated model is then used to make online trading decisions throughout the following week.
In the live trading process, the server connects trading signals directly to the exchanges, with the event processing engine managing the flow of real-time trade orders. The market data system provides investors with up-to-date market information. The server database aggregates three types of data: live exchange data, historical data stored in memory, and real-time streaming data obtained from the broker's scraper service. All of this data is refined and transformed into actionable market information.
Trader applications subscribe to these market data, which are then stored in an in-memory engine before being passed to the \name trading decision agent.
Trading signals generated by \name are first validated by an independent risk management system before being passed to the event processing engine. Final trade orders are routed to the Order Manager, where they are encrypted using exchange-provided APIs before being sent back to the exchange for execution.

\begin{figure}[!ht]
    \centering
    \includegraphics[width=0.48\textwidth]{images/deploy_framework.pdf}
    \caption{
         The system framework of our proposed \name
in quantitative trading scenarios.
    }
    \label{fig:deploy_framework}
\end{figure}

\subsection{Online Deployment}

To validate our model in a realistic environment, we conduct an online deployment based on the above framework in the Chinese stock market from 1 January 2024 to 30 June 2024.
Based on the predictions, we employ different strategies to trade within the first half hour of the next trading day’s opening.
The accumulative wealth of \name and the market return are depicted in \cref{fig:deployment}. Over half a year, all strategies in our model significantly outperform the market.


\begin{figure}[!ht]
    \centering
    \includegraphics[width=0.48\textwidth]{images/deployment.png}
    \caption{
         The performance of the strategy backtest.
    }
    \Description{..}
    \label{fig:deployment}
\end{figure}

\end{document}
\endinput
%%
%% End of file `sample-sigconf.tex'.
