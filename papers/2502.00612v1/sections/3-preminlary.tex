\section{Preliminaries}
In this section, we begin by presenting the time series patterns observed in web service traffic, illustrated with a specific web service traffic example. Subsequently, we provide an overview of the ecological causality theory, Convergent Cross Mapping (CCM)~\cite{sugihara2012detecting}, which serves as the theoretical foundation for our proposed CCMPlus module for web service traffic prediction.
\subsection{Web Service Traffic}
% As illustrated in Figure~\ref{fig:a_azure_web_service}, the web service traffic time series from Microsoft Azure exhibits significant fluctuations and frequent changes. These complex patterns pose a well-recognized challenge for predicting web service traffic, as highlighted by the research community~\cite{pan2023magicscaler, zou2024optscaler}.
Figure~\ref{fig:a_azure_web_service} presents an illustrative example of web service traffic time series from Microsoft Azure.
It exhibits significant fluctuations and frequent changes, primarily driven by human behavior and activity patterns.
The web traffic prediction task can be formulated as follows:
\begin{equation}
y(t) = P(y(t-\alpha), y(t-2\alpha), \dots, y(t-k\alpha)), \nonumber
\end{equation}
where $y(t)$ is the number of request at time $t$, $\alpha$ denotes the prediction granularity, and $k$ is the historicial sequence length.
However, accurately predicting web service traffic remains a well-recognized challenge due to the inherent complexity of traffic patterns, as highlighted by the research community~\cite{pan2023magicscaler, zou2024optscaler}.
\subsection{Convergent Cross Mapping}
 Convergent Cross Mapping (CCM) theory published in Science~\cite{sugihara2012detecting} was originally proposed in the field of ecology and is designed to detect causal relationships between species. The exposition of CCM is often illustrated using the context of the Lorenz system, as depicted in Figure~\ref{fig:ccm_lorenz_system}.

As depicted in Figure~\ref{fig:ccm_lorenz_system}, the trajectory of the Lorenz system forms a manifold \( M \) in the state space. This manifold \( M \) consists of a collection of points that represent all possible states of the Lorenz system over time, with these points connected to create a structured geometric space. The manifold, also referred to as the attractor, encompasses all trajectories and potential states \( \underline{m}(t) \) of the system. Each state \( \underline{m}(t) \) corresponds to a point in \( M \), represented by the coordinate vector \( \underline{m}(t) = [X(t), Y(t), Z(t)] \).

The shadow manifold, \( M_x \) or \( M_y \), represents the projection of the original manifold \( M \) onto the system variables \( X \) or \( Y \), respectively. Specifically, a lagged coordinate embedding utilizes \( E \) time-lagged values of \( X(t) \) as coordinate axes to reconstruct the shadow manifold \( M_x \). A point on \( M_x \), denoted as \( \underline{x}(t) \), is an \( E \)-dimensional vector expressed as:
\begin{equation}
\underline{x}(t) = [X(t), X(t-\tau), X(t-2\tau), \dots, X(t-(E-1)\tau)],
\nonumber
\end{equation}
where \( \tau \) is a positive time lag, and \( E \) denotes the embedding dimension. In Figure~\ref{fig:ccm_lorenz_system}, $E=3$. Similarly, the same approach applies to points \( \underline{y}(t) \) in the manifold \( M_y \), defined as:
\begin{equation}
\underline{y}(t) = [Y(t), Y(t-\tau), Y(t-2\tau), \dots, Y(t-(E-1)\tau)].
\nonumber
\end{equation}

\subsubsection{Cross Mapping}
\textbf{Cross mapping} refers to the process of identifying contemporaneous points in the manifold \( M_x \) of one variable \( X \) based on points in the manifold \( M_y \) of another variable \( Y \). Specifically, given a point \( \underline{y}(t) \) in the manifold \( M_y \), the corresponding point in time from the manifold \( M_x \) is \( \underline{x}(t) \). 

As illustrated in Figure~\ref{fig:ccm_cross_mapping}, if \( Y \) exerts a causal effect on \( X \), information from \( Y \) will be stored in \( X \). Consequently, the neighbors of \( \underline{x}(t) \) in \( M_x \) will correspond to points with the same time indices in \( M_y \), and these corresponding points will also be neighbors of \( \underline{y}(t) \). However, as noted in~\cite{sugihara2012detecting}, if \( Y \) has no causal effect on \( X \), the information about \( Y \) in \( X \) will be incomplete. As a result, the timely corresponding points in \( M_y \) will diverge and no longer be neighbors of \( \underline{y}(t) \).

\begin{figure}[t]
  \centering
  \includegraphics [width=\columnwidth]{figures/new_lorenz_system_compressed.pdf}
  \caption{The Convergent Cross Mapping (CCM) theory is often explained using the Lorenz system, utilizing its original manifold and corresponding shadow manifolds.}
  \label{fig:ccm_lorenz_system}
\end{figure}

\begin{figure}[t]
  \centering
  \includegraphics[width=0.85\linewidth,keepaspectratio]{figures/cross_mapping.pdf} 
  \caption{Cross mapping. The point \( \underline{y}(t) \) in the manifold \( M_y \) corresponds to the contemporaneous point in time \( \underline{x}(t) \) in the manifold \( M_x \).}
  \label{fig:ccm_cross_mapping}
\end{figure}

\begin{figure}[t]
  \centering
  \includegraphics[width=0.85\linewidth,keepaspectratio]{figures/correlation_coefficient_plot.png} 
  \caption{Convergent predictability as the time series length increases, assuming $Y$ has a causal effect on $X$.
}
  \label{fig:ccm_convergence}
\end{figure}

\subsubsection{Convergence}
\textbf{Convergence} in CCM implies that if the variable \( Y \) has a causal effect on \( X \), extending the observation period improves the ability to predict \( Y \) using points on the shadow manifold \( M_x \), as illustrated in Figure~\ref{fig:ccm_convergence}. A longer observation period provides more trajectories to fill the gaps in the manifold, resulting in a more defined structure, which enhances the prediction of \( \hat{Y}(t) \mid M_x \). Conversely, if two variables do not have a causal relationship, refining their manifolds will not lead to an improvement in predictive accuracy.

\subsubsection{CCM Procedure}
\label{sec:ccm_procedure}
The CCM procedure for detecting whether variable \( Y \) has causal effects on \( X \) consists of four key steps:

\begin{itemize}
    \item \textbf{Step 1: Construct the shadow manifold \( M_x \).}  
    Consider two time-evolving variables \( X(k) \) and \( Y(k) \) of length \( L \), where the time index \( k \) ranges from \( 1 \) to \( L \). The shadow manifold \( M_x \) is constructed by forming lagged coordinate vectors:
    \[
    \underline{x}(t) = [X(t), X(t-\tau), X(t-2\tau), \dots, X(t-(E-1)\tau)],
    \]
    for \( t = 1+(E-1)\tau \) to \( t=L \). Here, \( \tau \) represents the time lag, and \( E \) denotes the embedding dimension.

    \item \textbf{Step 2: Identify nearest neighbors in \( M_x \).}  
    To estimate \( Y(t) \) for a specific \( t \) in the range \( 1+(E-1)\tau \) to \( L \), use the shadow manifold \( M_x \). Denote the estimated value as \( \hat{Y}(t) \mid M_x \). Begin by locating the contemporaneous lagged coordinate vector \( \underline{x}(t) \) in \( M_x \), and find its \( E+1 \) nearest neighbors. Note that $E+1$ is the minimum number of points needed for a bounding simplex in an $E$-dimensional space. Let the time indices of these neighbors (ranked by proximity) be \( t_1, t_2, \dots, t_{E+1} \). The nearest neighbors of \( \underline{x}(t) \) in \( M_x \) are therefore denoted by \( \underline{x}(t_i) \), where \( i = 1, \dots, E+1 \).

    \item \textbf{Step 3: Estimate \( Y(t) \) using locally weighted means.}  
    The time indices \( t_1, t_2, \dots, t_{E+1} \) corresponding to the nearest neighbors of \( \underline{x}(t) \) are used to identify points in the variable \( Y(k) \). These points are then used to estimate \( Y(t) \) through a locally weighted mean of the \( E+1 \) values \( Y(t_i) \):
    \begin{equation}
    \hat{Y}(t) \mid M_X = \sum_{i=1}^{E+1} w_i Y(t_i),
    \nonumber
    \end{equation}
    where \( w_i \) represents the weight based on the distance between \( \underline{x}(t) \) and its \( i{\text{-th}} \) nearest neighbor in \( M_x \), and \( Y(t_i) \) are the contemporaneous values of variable \( Y(k) \). The weights \( w_i \) are determined by:
    \begin{equation}
    w_i = \frac{u_i}{\sum_{j=1}^{E+1} u_j},
    \nonumber
    \end{equation}
where
    \begin{equation}
    u_i = \exp \left\{ -\frac{d[\underline{x}(t), \underline{x}(t_i)]}{d[\underline{x}(t), \underline{x}(t_1)]} \right\}.
    \nonumber
    \end{equation}
    Here, \( d[\underline{x}(t), \underline{x}(t_i)] \) denotes the Euclidean distance between the two vectors.

    \item \textbf{Step 4: Calculate the correlation coefficient \( r \).}  
    Finally, calculate the correlation coefficient \( r \) between \( \hat{Y}(t) \) and \( Y(t) \), where \( t \) ranges from \( 1+(E-1)\tau \) to \( L \):
    \begingroup
\footnotesize
\begin{equation}
    r = \frac{\sum_{t=1+(E-1)\tau}^L \left( Y(t) - \overline{Y}(t) \right) \left( \hat{Y}(t) - \overline{\hat{Y}}(t) \right)}
    {\sqrt{\sum_{t=1+(E-1)\tau}^L \left( Y(t) - \overline{Y}(t) \right)^2 \sum_{t=1+(E-1)\tau}^L \left( \hat{Y}(t) - \overline{\hat{Y}}(t) \right)^2}}.
\nonumber
\end{equation}
\endgroup
    If variable \( Y \) has causal effects on \( X \), \( \hat{Y}(t) \) will converge to \( Y(t) \) as the observation period increases. In ideal cases, the correlation coefficient \( r \) will approach 1.
\end{itemize}























