\section{Introduction}
% Artificial intelligence (AI) techniques power a diverse range of applications, including computer vision~\cite{liu2024visual, li2021paint4poem, liu2024improved} and natural language processing~\cite{tian2022anti,touvron2023llama,tian-etal-2024-generic, kalla2023study}, both of which play a pivotal role in supporting various social and practical activities. As AI continues to advance, the use of web services has grown substantially~\cite{shen2024artificial, martin2020ai}. Predicting web service traffic carries significant social and practical value, with applications spanning dynamic resource scaling~\cite{pan2023magicscaler,zou2024optscaler}, load balancing~\cite{pavlenko2024vertically}, system anomaly detection~\cite{mitropoulou2024anomaly}, service-level agreement compliance~\cite{liao2024retrospecting}, fraud detection~\cite{Al-talak2021}, and so on. These capabilities not only enhance system performance but also improve the overall user experience with these technologies~\cite{kumar2024tpmcf}.

User-oriented web services continue to grow exponentially, especially with the advancements in artificial intelligence (AI) techniques~\cite{shen2024artificial, martin2020ai, tian-etal-2024-generic}, which have significantly accelerated the development of a diverse range of customized web applications. 
These services attract a substantial user base and play a pivotal role in enabling various social and practical activities. 
For instance, YouTube has amassed 2.7 billion users~\footnote{https://www.globalmediainsight.com/blog/youtube-users-statistics/}, powered by its cutting-edge recommendation algorithms.
Accurately predicting web service traffic carries significant social and practical value, with applications spanning dynamic resource scaling~\cite{pan2023magicscaler,zou2024optscaler}, load balancing~\cite{pavlenko2024vertically}, system anomaly detection~\cite{mitropoulou2024anomaly}, service-level agreement compliance~\cite{liao2024retrospecting}, fraud detection~\cite{Al-talak2021}, and so on. These capabilities not only enhance system performance but also improve the overall user experience with these technologies~\cite{kumar2024tpmcf}.

Co-located, long-running web services often experience diverse workload patterns~\cite{zou2024optscaler}. Web service traffic are characterized by frequent and significant fluctuations over time, driven by heterogeneous user behaviors at any given moment. These factors collectively make predicting web service traffic a highly challenging task~\cite{straesser2025trust, pan2023magicscaler}.
Previous works conduct extensive research to predict the web service traffic, often formulating it as a typical time series forecasting task. These approaches can broadly be categorized into statistical~\cite{kapgate2014weighted, hu2016autoscaling, kumar2016forecasting}, machine learning~\cite{issa2017using, daraghmeh2018linear, hu2019stream}, and deep learning~\cite{ruan2023workload, guo2018applying, pan2023magicscaler, zou2024optscaler} methods.
While statistical methods struggle to handle multi-dimensional and non-linear traffic data, machine learning methods address these limitations but fail to achieve the same level of accuracy as deep learning approaches. 
Among the deep learning based methods, the recent advancements in Transformer~\cite{vaswani2017attention} architecture have demonstrated superior performance in sequential prediction tasks. Consequently, Transformer-based methods have emerged, achieving promising results~\cite{qi2022performer} in web service traffic prediction.
However, they did not pay enough attention to the causal relationship across web services.

In addition to the three types of specialized methods previously discussed for predicting web service traffic, general time series forecasting methods are also widely employed in practice~\cite{zou2024optscaler, alharthi2024auto}. These approaches are typically divided into two categories~\cite{qiu2024tfb}: statistical methods and neural network-based methods. Since statistical methods often fail to capture complex temporal features, neural network-based methods generally achieve superior performance~\cite{wutimesnet}. Neural network-based approaches can be further classified into five paradigms~\cite{wangtimemixer, tan2024language}: Recurrent neural network (RNN)-based, convolutional neural network (CNN)-based, Transformer-based, multi-layer perceptron (MLP)-based, and large language model (LLM)-based methods. CNN-based methods~\cite{wang2023micn, hewage2020temporal} use convolutional kernels along the temporal dimension to identify sequential patterns, whereas RNN-based methods~\cite{franceschi2019unsupervised, dudukcu2023temporal, peng2024reservoir} rely on recurrent structures to model temporal state transitions. Transformer-based methods~\cite{zhou2022fedformer, liuitransformer, chenpathformer, wang2024timexer} are widely recognized for their ability to effectively extract features using attention mechanisms. 
LLM-based methods~\cite{tan2024language, touvron2023llama, jintime} leverage advanced reasoning capabilities by processing time series data through specially designed prompts, offering a promising avenue for temporal modeling. 
Finally, MLP-based methods~\cite{wang2024timexer, olivares2023neural, daslong, wutimesnet} strike a compelling balance between forecasting accuracy and computational efficiency. It is regrettable that when advanced general time series forecasting methods are applied to web traffic prediction, the causal relationships among services are not utilized.

Drawing inspiration from ecological causality, as illustrated in Figure~\ref{fig:rabbit}, where grass abundance and rabbit populations influence one another iteratively~\cite{ji2022superprocesses}, we identify analogous patterns in web service traffic as shown in Figure~\ref{fig:services_causality}. Empirical observations of Google Trends data reveal a causal relationship between leisure websites, such as Netflix, and work-related software, like Outlook. Specifically, increased web traffic to Netflix corresponds to decreased traffic to Outlook, and vice versa. 
These empirical observations suggest the existence of latent causal relationships underlying human-driven web behaviors. By uncovering and leveraging these causal relationships, we can achieve more accurate modeling of service traffic patterns.
Building on this motivation and the causality theory of Convergent Cross Mapping (CCM)~\cite{sugihara2012detecting}, which originates from ecology, we introduce the CCMPlus module. This module extracts features from web service traffic time series while including causal relationships among services, thereby enhancing the accuracy of web service traffic prediction. Furthermore, the CCMPlus module could integrate easily with existing time series forecasting models, enriching them with more informative, causally-aware features.

Concretely, we extend the CCM theory to effectively merge with neural networks, resulting in the development of the CCMPlus module. The CCMPlus module operates in three key steps: first, it extracts initial feature representations from web service traffic time series; second, it computes a causal correlation matrix from a multi-manifold space based on these feature representations; and finally, it applies the causal correlation matrix to the initial feature representations, generating a resulting feature representation that incorporates informative causal information. This enhanced feature representation can then be concatenated with the feature representations of web service traffic time series extracted by existing time series models, thereby improving prediction accuracy.

Our main contributions in this work can be summarized as follows:
\begin{itemize}
    \item \textbf{Method:} 
    The CCMPlus module enhances existing time series forecasting models by generating feature representations that incorporate causal relationships across web services, addressing a critical limitation of many previous methods. Additionally, the CCMPlus module is designed for seamless integration with existing time series forecasting models, further contributing to improved prediction accuracy.
    \item \textbf{Theory:} We justify that the causal correlation matrix generated by the CCMPlus module effectively captures causal relationships across web services, enabling the incorporation of these relationships into web traffic prediction methods.
    \item \textbf{Experiments:} Experiments conducted on three real-world web service traffic datasets (Alibaba Group, Microsoft Azure, and Ant Group) demonstrate that our method achieves superior performance in terms of Mean Squared Error (MSE) and Mean Absolute Error (MAE) compared to previous state-of-the-art (SOTA) methods, thereby validating the effectiveness of the CCMPlus module.
\end{itemize}

The remainder of the paper is organized as follows. Section 2 reviews related work. Section 3 provides preliminary information about our proposed method. Section 4 elaborates on the proposed framework, encompassing the CCMPlus and time series backbone modules. Section 5 compares experimental results from the proposed method and prevalent methods, and analyzes the method and results. Section 6 concludes this paper.

