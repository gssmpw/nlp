
% VLDB template version of 2020-08-03 enhances the ACM template, version 1.7.0:
% https://www.acm.org/publications/proceedings-template
% The ACM Latex guide provides further information about the ACM template

\documentclass[sigconf, nonacm]{acmart}
\pdfcompresslevel=2
% \pdfobjcompresslevel=2
\setcitestyle{numbers,open={[},close={]}} % Adjusts citation style
\usepackage{balance}
\usepackage{graphicx}
\usepackage{float}
\usepackage{booktabs}
\usepackage{morewrites}
\usepackage{algorithm}
% Some math packages, if needed:
\let\Bbbk\relax
\usepackage{amsmath,amssymb}
% The 'algpseudocode' environment provides \Require, \Ensure, \State, etc.
\usepackage{algpseudocode}
% 'Statex' is not defined by default; either remove it or define it. 
% \newcommand{\Statex}{\StatexIndent}


\newcommand{\mz}[1]{\textcolor{red}{mz:#1}}
%% The following content must be adapted for the final version
% paper-specific
\newcommand\vldbdoi{XX.XX/XXX.XX}
\newcommand\vldbpages{XXX-XXX}
% issue-specific
\newcommand\vldbvolume{14}
\newcommand\vldbissue{1}
\newcommand\vldbyear{2020}
% should be fine as it is
\newcommand\vldbauthors{\authors}
\newcommand\vldbtitle{\shorttitle} 
% leave empty if no availability url should be set
\newcommand\vldbavailabilityurl{URL_TO_YOUR_ARTIFACTS}
% whether page numbers should be shown or not, use 'plain' for review versions, 'empty' for camera ready
\newcommand\vldbpagestyle{plain} 

\begin{document}
\title{Using Causality for Enhanced Prediction of Web Traffic Time Series}

%%
%% The "author" command and its associated commands are used to define the authors and their affiliations.
\author{Chang Tian}
\affiliation{%
  \institution{KU Leuven}
  \streetaddress{Celestijnenlaan 200A}
  \city{Leuven}
  \state{Belgium}
  \postcode{3001 Leuven}
}
% \email{chang.tian@kuleuven.be}

\author{Mingzhe Xing}
\affiliation{%
  \institution{Zhongguancun Laboratory}
  \city{Beijing}
  \country{China}
}
% \email{xingmz@zgclab.edu.cn}

\author{Zenglin Shi}
\affiliation{%
  \institution{Hefei University of Technology}
  \city{Hefei}
  \country{China}
}
% \email{zenglin.shi@hfut.edu.cn}

\author{Matthew B. Blaschko}
\affiliation{%
  \institution{KU Leuven}
  \city{Leuven}
  \country{Belgium}
}
% \email{matthew.blaschko@esat.kuleuven.be}


% \author{Wenpeng Yin}
% \affiliation{%
%   \institution{Pennsylvania State University}
%   \city{Pennsylvania}
%   \country{United States}
% }
% \email{wenpeng@psu.edu}

\author{Yinliang Yue
}
\affiliation{%
  \institution{Zhongguancun Laboratory}
  \city{Beijing}
  \country{China}
}
% \email{yueyl@zgclab.edu.cn}

\author{Marie-Francine Moens}
\affiliation{%
  \institution{KU Leuven}
  \city{Leuven}
  \country{Belgium}
}
% \email{sien.moens@kuleuven.be}

%%
%% The abstract is a short summary of the work to be presented in the
%% article.
\begin{abstract}
Predicting web service traffic has significant social value, as it can be applied to various practical scenarios, including but not limited to dynamic resource scaling, load balancing, system anomaly detection, service-level agreement compliance, and fraud detection. Web service traffic is characterized by frequent and drastic fluctuations over time and are influenced by heterogeneous web user behaviors, making accurate prediction a challenging task. Previous research has extensively explored statistical approaches, and neural networks to mine features from preceding service traffic time series for prediction. However, these methods have largely overlooked the causal relationships between services. Drawing inspiration from causality in ecological systems, we empirically recognize the causal relationships between web services. To leverage these relationships for improved web service traffic prediction, we propose an effective neural network module, CCMPlus, designed to extract causal relationship features across services. This module can be seamlessly integrated with existing time series models to consistently enhance the performance of web service traffic predictions. We theoretically justify that the causal correlation matrix generated by the CCMPlus module captures causal relationships among services. Empirical results on real-world datasets from Microsoft Azure, Alibaba Group, and Ant Group confirm that our method surpasses state-of-the-art approaches in Mean Squared Error (MSE) and Mean Absolute Error (MAE) for predicting service traffic time series. These findings highlight the efficacy of leveraging causal relationships for improved predictions.
\end{abstract}

\maketitle
%%% do not modify the following VLDB block %%
%%% VLDB block start %%%
\pagestyle{\vldbpagestyle}
% \begingroup\small\noindent\raggedright\textbf{PVLDB Reference Format:}\\
% \vldbauthors. \vldbtitle. PVLDB, \vldbvolume(\vldbissue): \vldbpages, \vldbyear.\\
% \href{https://doi.org/\vldbdoi}{doi:\vldbdoi}
% \endgroup
% \begingroup
% \renewcommand\thefootnote{}\footnote{\noindent
% This work is licensed under the Creative Commons BY-NC-ND 4.0 International License. Visit \url{https://creativecommons.org/licenses/by-nc-nd/4.0/} to view a copy of this license. For any use beyond those covered by this license, obtain permission by emailing \href{mailto:info@vldb.org}{info@vldb.org}. Copyright is held by the owner/author(s). Publication rights licensed to the VLDB Endowment. \\
% \raggedright Proceedings of the VLDB Endowment, Vol. \vldbvolume, No. \vldbissue\ %
% ISSN 2150-8097. \\
% \href{https://doi.org/\vldbdoi}{doi:\vldbdoi} \\
% }\addtocounter{footnote}{-1}\endgroup
%%% VLDB block end %%%

%%% do not modify the following VLDB block %%
%%% VLDB block start %%%
% \ifdefempty{\vldbavailabilityurl}{}{
% \vspace{.3cm}
% \begingroup\small\noindent\raggedright\textbf{PVLDB Artifact Availability:}\\
% The source code, data, and/or other artifacts have been made available at \url{https://github.com/changtianluckyforever/CCMPlus}.
% \endgroup
% }
%%% VLDB block end %%%

\begin{figure*}[t]
  \centering
  \includegraphics[width=0.85\linewidth, keepaspectratio]{figures/services_causality.pdf}
  \caption{This figure illustrates 7-day search interest data for Netflix and Microsoft Outlook from Google Trends. The vertical axis represents normalized search interest as a percentage relative to the highest point on the chart for the specified time period. 
  % A value of 100 indicates the peak popularity of the term, while a value of 50 signifies that the term's popularity is half of the peak value.
  The horizontal axis corresponds to the time period from January 5th, 2025 to January 12th, 2025.}
  \label{fig:services_causality}
\end{figure*}

\begin{figure}[t]
  \centering
    \includegraphics[width=0.7\columnwidth, keepaspectratio]{figures/ccm_intro.pdf}
  \caption{Illustration of ecological and analogical web service causality. The rabbit population influences grass abundance~\cite{ji2022superprocesses}, where an increase in rabbits leads to a decrease in grass. This principle inspires our CCMPlus module, which captures causal effects among web services.}
  \label{fig:rabbit}
\end{figure}

%!TEX root = gcn.tex
\section{Introduction}
Graphs, representing structural data and topology, are widely used across various domains, such as social networks and merchandising transactions.
Graph convolutional networks (GCN)~\cite{iclr/KipfW17} have significantly enhanced model training on these interconnected nodes.
However, these graphs often contain sensitive information that should not be leaked to untrusted parties.
For example, companies may analyze sensitive demographic and behavioral data about users for applications ranging from targeted advertising to personalized medicine.
Given the data-centric nature and analytical power of GCN training, addressing these privacy concerns is imperative.

Secure multi-party computation (MPC)~\cite{crypto/ChaumDG87,crypto/ChenC06,eurocrypt/CiampiRSW22} is a critical tool for privacy-preserving machine learning, enabling mutually distrustful parties to collaboratively train models with privacy protection over inputs and (intermediate) computations.
While research advances (\eg,~\cite{ccs/RatheeRKCGRS20,uss/NgC21,sp21/TanKTW,uss/WatsonWP22,icml/Keller022,ccs/ABY318,folkerts2023redsec}) support secure training on convolutional neural networks (CNNs) efficiently, private GCN training with MPC over graphs remains challenging.

Graph convolutional layers in GCNs involve multiplications with a (normalized) adjacency matrix containing $\numedge$ non-zero values in a $\numnode \times \numnode$ matrix for a graph with $\numnode$ nodes and $\numedge$ edges.
The graphs are typically sparse but large.
One could use the standard Beaver-triple-based protocol to securely perform these sparse matrix multiplications by treating graph convolution as ordinary dense matrix multiplication.
However, this approach incurs $O(\numnode^2)$ communication and memory costs due to computations on irrelevant nodes.
%
Integrating existing cryptographic advances, the initial effort of SecGNN~\cite{tsc/WangZJ23,nips/RanXLWQW23} requires heavy communication or computational overhead.
Recently, CoGNN~\cite{ccs/ZouLSLXX24} optimizes the overhead in terms of  horizontal data partitioning, proposing a semi-honest secure framework.
Research for secure GCN over vertical data  remains nascent.

Current MPC studies, for GCN or not, have primarily targeted settings where participants own different data samples, \ie, horizontally partitioned data~\cite{ccs/ZouLSLXX24}.
MPC specialized for scenarios where parties hold different types of features~\cite{tkde/LiuKZPHYOZY24,icml/CastigliaZ0KBP23,nips/Wang0ZLWL23} is rare.
This paper studies $2$-party secure GCN training for these vertical partition cases, where one party holds private graph topology (\eg, edges) while the other owns private node features.
For instance, LinkedIn holds private social relationships between users, while banks own users' private bank statements.
Such real-world graph structures underpin the relevance of our focus.
To our knowledge, no prior work tackles secure GCN training in this context, which is crucial for cross-silo collaboration.


To realize secure GCN over vertically split data, we tailor MPC protocols for sparse graph convolution, which fundamentally involves sparse (adjacency) matrix multiplication.
Recent studies have begun exploring MPC protocols for sparse matrix multiplication (SMM).
ROOM~\cite{ccs/SchoppmannG0P19}, a seminal work on SMM, requires foreknowledge of sparsity types: whether the input matrices are row-sparse or column-sparse.
Unfortunately, GCN typically trains on graphs with arbitrary sparsity, where nodes have varying degrees and no specific sparsity constraints.
Moreover, the adjacency matrix in GCN often contains a self-loop operation represented by adding the identity matrix, which is neither row- nor column-sparse.
Araki~\etal~\cite{ccs/Araki0OPRT21} avoid this limitation in their scalable, secure graph analysis work, yet it does not cover vertical partition.

% and related primitives
To bridge this gap, we propose a secure sparse matrix multiplication protocol, \osmm, achieving \emph{accurate, efficient, and secure GCN training over vertical data} for the first time.

\subsection{New Techniques for Sparse Matrices}
The cost of evaluating a GCN layer is dominated by SMM in the form of $\adjmat\feamat$, where $\adjmat$ is a sparse adjacency matrix of a (directed) graph $\graph$ and $\feamat$ is a dense matrix of node features.
For unrelated nodes, which often constitute a substantial portion, the element-wise products $0\cdot x$ are always zero.
Our efficient MPC design 
avoids unnecessary secure computation over unrelated nodes by focusing on computing non-zero results while concealing the sparse topology.
We achieve this~by:
1) decomposing the sparse matrix $\adjmat$ into a product of matrices (\S\ref{sec::sgc}), including permutation and binary diagonal matrices, that can \emph{faithfully} represent the original graph topology;
2) devising specialized protocols (\S\ref{sec::smm_protocol}) for efficiently multiplying the structured matrices while hiding sparsity topology.


 
\subsubsection{Sparse Matrix Decomposition}
We decompose adjacency matrix $\adjmat$ of $\graph$ into two bipartite graphs: one represented by sparse matrix $\adjout$, linking the out-degree nodes to edges, the other 
by sparse matrix $\adjin$,
linking edges to in-degree nodes.

%\ie, we decompose $\adjmat$ into $\adjout \adjin$, where $\adjout$ and $\adjin$ are sparse matrices representing these connections.
%linking out-degree nodes to edges and edges to in-degree nodes of $\graph$, respectively.

We then permute the columns of $\adjout$ and the rows of $\adjin$ so that the permuted matrices $\adjout'$ and $\adjin'$ have non-zero positions with \emph{monotonically non-decreasing} row and column indices.
A permutation $\sigma$ is used to preserve the edge topology, leading to an initial decomposition of $\adjmat = \adjout'\sigma \adjin'$.
This is further refined into a sequence of \emph{linear transformations}, 
which can be efficiently computed by our MPC protocols for 
\emph{oblivious permutation}
%($\Pi_{\ssp}$) 
and \emph{oblivious selection-multiplication}.
% ($\Pi_\SM$)
\iffalse
Our approach leverages bipartite graph representation and the monotonicity of non-zero positions to decompose a general sparse matrix into linear transformations, enhancing the efficiency of our MPC protocols.
\fi
Our decomposition approach is not limited to GCNs but also general~SMM 
by 
%simply 
treating them 
as adjacency matrices.
%of a graph.
%Since any sparse matrix can be viewed 

%allowing the same technique to be applied.

 
\subsubsection{New Protocols for Linear Transformations}
\emph{Oblivious permutation} (OP) is a two-party protocol taking a private permutation $\sigma$ and a private vector $\xvec$ from the two parties, respectively, and generating a secret share $\l\sigma \xvec\r$ between them.
Our OP protocol employs correlated randomnesses generated in an input-independent offline phase to mask $\sigma$ and $\xvec$ for secure computations on intermediate results, requiring only $1$ round in the online phase (\cf, $\ge 2$ in previous works~\cite{ccs/AsharovHIKNPTT22, ccs/Araki0OPRT21}).

Another crucial two-party protocol in our work is \emph{oblivious selection-multiplication} (OSM).
It takes a private bit~$s$ from a party and secret share $\l x\r$ of an arithmetic number~$x$ owned by the two parties as input and generates secret share $\l sx\r$.
%between them.
%Like our OP protocol, o
Our $1$-round OSM protocol also uses pre-computed randomnesses to mask $s$ and $x$.
%for secure computations.
Compared to the Beaver-triple-based~\cite{crypto/Beaver91a} and oblivious-transfer (OT)-based approaches~\cite{pkc/Tzeng02}, our protocol saves ${\sim}50\%$ of online communication while having the same offline communication and round complexities.

By decomposing the sparse matrix into linear transformations and applying our specialized protocols, our \osmm protocol
%($\prosmm$) 
reduces the complexity of evaluating $\numnode \times \numnode$ sparse matrices with $\numedge$ non-zero values from $O(\numnode^2)$ to $O(\numedge)$.

%(\S\ref{sec::secgcn})
\subsection{\cgnn: Secure GCN made Efficient}
Supported by our new sparsity techniques, we build \cgnn, 
a two-party computation (2PC) framework for GCN inference and training over vertical
%ly split
data.
Our contributions include:

1) We are the first to explore sparsity over vertically split, secret-shared data in MPC, enabling decompositions of sparse matrices with arbitrary sparsity and isolating computations that can be performed in plaintext without sacrificing privacy.

2) We propose two efficient $2$PC primitives for OP and OSM, both optimally single-round.
Combined with our sparse matrix decomposition approach, our \osmm protocol ($\prosmm$) achieves constant-round communication costs of $O(\numedge)$, reducing memory requirements and avoiding out-of-memory errors for large matrices.
In practice, it saves $99\%+$ communication
%(Table~\ref{table:comm_smm}) 
and reduces ${\sim}72\%$ memory usage over large $(5000\times5000)$ matrices compared with using Beaver triples.
%(Table~\ref{table:mem_smm_sparse}) ${\sim}16\%$-

3) We build an end-to-end secure GCN framework for inference and training over vertically split data, maintaining accuracy on par with plaintext computations.
We will open-source our evaluation code for research and deployment.

To evaluate the performance of $\cgnn$, we conducted extensive experiments over three standard graph datasets (Cora~\cite{aim/SenNBGGE08}, Citeseer~\cite{dl/GilesBL98}, and Pubmed~\cite{ijcnlp/DernoncourtL17}),
reporting communication, memory usage, accuracy, and running time under varying network conditions, along with an ablation study with or without \osmm.
Below, we highlight our key achievements.

\textit{Communication (\S\ref{sec::comm_compare_gcn}).}
$\cgnn$ saves communication by $50$-$80\%$.
(\cf,~CoGNN~\cite{ccs/KotiKPG24}, OblivGNN~\cite{uss/XuL0AYY24}).

\textit{Memory usage (\S\ref{sec::smmmemory}).}
\cgnn alleviates out-of-memory problems of using %the standard 
Beaver-triples~\cite{crypto/Beaver91a} for large datasets.

\textit{Accuracy (\S\ref{sec::acc_compare_gcn}).}
$\cgnn$ achieves inference and training accuracy comparable to plaintext counterparts.
%training accuracy $\{76\%$, $65.1\%$, $75.2\%\}$ comparable to $\{75.7\%$, $65.4\%$, $74.5\%\}$ in plaintext.

{\textit{Computational efficiency (\S\ref{sec::time_net}).}} 
%If the network is worse in bandwidth and better in latency, $\cgnn$ shows more benefits.
$\cgnn$ is faster by $6$-$45\%$ in inference and $28$-$95\%$ in training across various networks and excels in narrow-bandwidth and low-latency~ones.

{\textit{Impact of \osmm (\S\ref{sec:ablation}).}}
Our \osmm protocol shows a $10$-$42\times$ speed-up for $5000\times 5000$ matrices and saves $10$-2$1\%$ memory for ``small'' datasets and up to $90\%$+ for larger ones.


\section{Related Work}

\subsection{Personalization and Role-Playing}
Recent works have introduced benchmark datasets for personalizing LLM outputs in tasks like email, abstract, and news writing, focusing on shorter outputs (e.g., 300 tokens for product reviews \citep{kumar2024longlamp} and 850 for news writing \citep{shashidhar-etal-2024-unsupervised}). These methods infer user traits from history for task-specific personalization \citep{sun-etal-2024-revealing, sun-etal-2025-persona, pal2024beyond, li2023teach, salemi2025reasoning}. In contrast, we tackle the more subjective problem of long-form story writing, with author stories averaging 1500 tokens. Unlike prior role-playing approaches that use predefined personas (e.g., Tony Stark, Confucius) \citep{wang-etal-2024-rolellm, sadeq-etal-2024-mitigating, tu2023characterchat, xu2023expertprompting}, we propose a novel method to infer story-writing personas from an author’s history to guide role-playing.


\subsection{Story Understanding and Generation}  
Prior work on persona-aware story generation \citep{yunusov-etal-2024-mirrorstories, bae-kim-2024-collective, zhang-etal-2022-persona, chandu-etal-2019-way} defines personas using discrete attributes like personality traits, demographics, or hobbies. Similarly, \citep{zhu-etal-2023-storytrans} explore story style transfer across pre-defined domains (e.g., fairy tales, martial arts, Shakespearean plays). In contrast, we mimic an individual author's writing style based on their history. Our approach differs by (1) inferring long-form author personas—descriptions of an author’s style from their past works, rather than relying on demographics, and (2) handling long-form story generation, averaging 1500 tokens per output, exceeding typical story lengths in prior research.

\begin{figure*}[t]
  \centering
    \includegraphics[width=0.9\linewidth,keepaspectratio]{figures/a_webservice_azure.pdf}
  \caption{Web service traffic time series from the Microsoft Azure cluster. The horizontal axis represents time progression, while the vertical axis indicates the number of service requests at each time point.}
  \label{fig:a_azure_web_service}
\end{figure*}

\section{Preliminaries}
In this section, we begin by presenting the time series patterns observed in web service traffic, illustrated with a specific web service traffic example. Subsequently, we provide an overview of the ecological causality theory, Convergent Cross Mapping (CCM)~\cite{sugihara2012detecting}, which serves as the theoretical foundation for our proposed CCMPlus module for web service traffic prediction.
\subsection{Web Service Traffic}
% As illustrated in Figure~\ref{fig:a_azure_web_service}, the web service traffic time series from Microsoft Azure exhibits significant fluctuations and frequent changes. These complex patterns pose a well-recognized challenge for predicting web service traffic, as highlighted by the research community~\cite{pan2023magicscaler, zou2024optscaler}.
Figure~\ref{fig:a_azure_web_service} presents an illustrative example of web service traffic time series from Microsoft Azure.
It exhibits significant fluctuations and frequent changes, primarily driven by human behavior and activity patterns.
The web traffic prediction task can be formulated as follows:
\begin{equation}
y(t) = P(y(t-\alpha), y(t-2\alpha), \dots, y(t-k\alpha)), \nonumber
\end{equation}
where $y(t)$ is the number of request at time $t$, $\alpha$ denotes the prediction granularity, and $k$ is the historicial sequence length.
However, accurately predicting web service traffic remains a well-recognized challenge due to the inherent complexity of traffic patterns, as highlighted by the research community~\cite{pan2023magicscaler, zou2024optscaler}.
\subsection{Convergent Cross Mapping}
 Convergent Cross Mapping (CCM) theory published in Science~\cite{sugihara2012detecting} was originally proposed in the field of ecology and is designed to detect causal relationships between species. The exposition of CCM is often illustrated using the context of the Lorenz system, as depicted in Figure~\ref{fig:ccm_lorenz_system}.

As depicted in Figure~\ref{fig:ccm_lorenz_system}, the trajectory of the Lorenz system forms a manifold \( M \) in the state space. This manifold \( M \) consists of a collection of points that represent all possible states of the Lorenz system over time, with these points connected to create a structured geometric space. The manifold, also referred to as the attractor, encompasses all trajectories and potential states \( \underline{m}(t) \) of the system. Each state \( \underline{m}(t) \) corresponds to a point in \( M \), represented by the coordinate vector \( \underline{m}(t) = [X(t), Y(t), Z(t)] \).

The shadow manifold, \( M_x \) or \( M_y \), represents the projection of the original manifold \( M \) onto the system variables \( X \) or \( Y \), respectively. Specifically, a lagged coordinate embedding utilizes \( E \) time-lagged values of \( X(t) \) as coordinate axes to reconstruct the shadow manifold \( M_x \). A point on \( M_x \), denoted as \( \underline{x}(t) \), is an \( E \)-dimensional vector expressed as:
\begin{equation}
\underline{x}(t) = [X(t), X(t-\tau), X(t-2\tau), \dots, X(t-(E-1)\tau)],
\nonumber
\end{equation}
where \( \tau \) is a positive time lag, and \( E \) denotes the embedding dimension. In Figure~\ref{fig:ccm_lorenz_system}, $E=3$. Similarly, the same approach applies to points \( \underline{y}(t) \) in the manifold \( M_y \), defined as:
\begin{equation}
\underline{y}(t) = [Y(t), Y(t-\tau), Y(t-2\tau), \dots, Y(t-(E-1)\tau)].
\nonumber
\end{equation}

\subsubsection{Cross Mapping}
\textbf{Cross mapping} refers to the process of identifying contemporaneous points in the manifold \( M_x \) of one variable \( X \) based on points in the manifold \( M_y \) of another variable \( Y \). Specifically, given a point \( \underline{y}(t) \) in the manifold \( M_y \), the corresponding point in time from the manifold \( M_x \) is \( \underline{x}(t) \). 

As illustrated in Figure~\ref{fig:ccm_cross_mapping}, if \( Y \) exerts a causal effect on \( X \), information from \( Y \) will be stored in \( X \). Consequently, the neighbors of \( \underline{x}(t) \) in \( M_x \) will correspond to points with the same time indices in \( M_y \), and these corresponding points will also be neighbors of \( \underline{y}(t) \). However, as noted in~\cite{sugihara2012detecting}, if \( Y \) has no causal effect on \( X \), the information about \( Y \) in \( X \) will be incomplete. As a result, the timely corresponding points in \( M_y \) will diverge and no longer be neighbors of \( \underline{y}(t) \).

\begin{figure}[t]
  \centering
  \includegraphics [width=\columnwidth]{figures/new_lorenz_system_compressed.pdf}
  \caption{The Convergent Cross Mapping (CCM) theory is often explained using the Lorenz system, utilizing its original manifold and corresponding shadow manifolds.}
  \label{fig:ccm_lorenz_system}
\end{figure}

\begin{figure}[t]
  \centering
  \includegraphics[width=0.85\linewidth,keepaspectratio]{figures/cross_mapping.pdf} 
  \caption{Cross mapping. The point \( \underline{y}(t) \) in the manifold \( M_y \) corresponds to the contemporaneous point in time \( \underline{x}(t) \) in the manifold \( M_x \).}
  \label{fig:ccm_cross_mapping}
\end{figure}

\begin{figure}[t]
  \centering
  \includegraphics[width=0.85\linewidth,keepaspectratio]{figures/correlation_coefficient_plot.png} 
  \caption{Convergent predictability as the time series length increases, assuming $Y$ has a causal effect on $X$.
}
  \label{fig:ccm_convergence}
\end{figure}

\subsubsection{Convergence}
\textbf{Convergence} in CCM implies that if the variable \( Y \) has a causal effect on \( X \), extending the observation period improves the ability to predict \( Y \) using points on the shadow manifold \( M_x \), as illustrated in Figure~\ref{fig:ccm_convergence}. A longer observation period provides more trajectories to fill the gaps in the manifold, resulting in a more defined structure, which enhances the prediction of \( \hat{Y}(t) \mid M_x \). Conversely, if two variables do not have a causal relationship, refining their manifolds will not lead to an improvement in predictive accuracy.

\subsubsection{CCM Procedure}
\label{sec:ccm_procedure}
The CCM procedure for detecting whether variable \( Y \) has causal effects on \( X \) consists of four key steps:

\begin{itemize}
    \item \textbf{Step 1: Construct the shadow manifold \( M_x \).}  
    Consider two time-evolving variables \( X(k) \) and \( Y(k) \) of length \( L \), where the time index \( k \) ranges from \( 1 \) to \( L \). The shadow manifold \( M_x \) is constructed by forming lagged coordinate vectors:
    \[
    \underline{x}(t) = [X(t), X(t-\tau), X(t-2\tau), \dots, X(t-(E-1)\tau)],
    \]
    for \( t = 1+(E-1)\tau \) to \( t=L \). Here, \( \tau \) represents the time lag, and \( E \) denotes the embedding dimension.

    \item \textbf{Step 2: Identify nearest neighbors in \( M_x \).}  
    To estimate \( Y(t) \) for a specific \( t \) in the range \( 1+(E-1)\tau \) to \( L \), use the shadow manifold \( M_x \). Denote the estimated value as \( \hat{Y}(t) \mid M_x \). Begin by locating the contemporaneous lagged coordinate vector \( \underline{x}(t) \) in \( M_x \), and find its \( E+1 \) nearest neighbors. Note that $E+1$ is the minimum number of points needed for a bounding simplex in an $E$-dimensional space. Let the time indices of these neighbors (ranked by proximity) be \( t_1, t_2, \dots, t_{E+1} \). The nearest neighbors of \( \underline{x}(t) \) in \( M_x \) are therefore denoted by \( \underline{x}(t_i) \), where \( i = 1, \dots, E+1 \).

    \item \textbf{Step 3: Estimate \( Y(t) \) using locally weighted means.}  
    The time indices \( t_1, t_2, \dots, t_{E+1} \) corresponding to the nearest neighbors of \( \underline{x}(t) \) are used to identify points in the variable \( Y(k) \). These points are then used to estimate \( Y(t) \) through a locally weighted mean of the \( E+1 \) values \( Y(t_i) \):
    \begin{equation}
    \hat{Y}(t) \mid M_X = \sum_{i=1}^{E+1} w_i Y(t_i),
    \nonumber
    \end{equation}
    where \( w_i \) represents the weight based on the distance between \( \underline{x}(t) \) and its \( i{\text{-th}} \) nearest neighbor in \( M_x \), and \( Y(t_i) \) are the contemporaneous values of variable \( Y(k) \). The weights \( w_i \) are determined by:
    \begin{equation}
    w_i = \frac{u_i}{\sum_{j=1}^{E+1} u_j},
    \nonumber
    \end{equation}
where
    \begin{equation}
    u_i = \exp \left\{ -\frac{d[\underline{x}(t), \underline{x}(t_i)]}{d[\underline{x}(t), \underline{x}(t_1)]} \right\}.
    \nonumber
    \end{equation}
    Here, \( d[\underline{x}(t), \underline{x}(t_i)] \) denotes the Euclidean distance between the two vectors.

    \item \textbf{Step 4: Calculate the correlation coefficient \( r \).}  
    Finally, calculate the correlation coefficient \( r \) between \( \hat{Y}(t) \) and \( Y(t) \), where \( t \) ranges from \( 1+(E-1)\tau \) to \( L \):
    \begingroup
\footnotesize
\begin{equation}
    r = \frac{\sum_{t=1+(E-1)\tau}^L \left( Y(t) - \overline{Y}(t) \right) \left( \hat{Y}(t) - \overline{\hat{Y}}(t) \right)}
    {\sqrt{\sum_{t=1+(E-1)\tau}^L \left( Y(t) - \overline{Y}(t) \right)^2 \sum_{t=1+(E-1)\tau}^L \left( \hat{Y}(t) - \overline{\hat{Y}}(t) \right)^2}}.
\nonumber
\end{equation}
\endgroup
    If variable \( Y \) has causal effects on \( X \), \( \hat{Y}(t) \) will converge to \( Y(t) \) as the observation period increases. In ideal cases, the correlation coefficient \( r \) will approach 1.
\end{itemize}

























\section{Method}

In Fig. \ref{fig:overview}, we illustrate two major stages of MedForge for collaborative model development, including feature branch development (Sec~\ref{branch}) and model merging (Sec~\ref{forging}). In the feature branch development, individual contributors (i.e., medical centers) could make individual knowledge contributions asynchronously. Our MedForge allows each contributor to develop their own plugin module and distilled data locally without the need to share any private data. In the model merging stage, MedForge enables multi-task knowledge integration by merging the well-prepared plugin module asynchronously. This key integration process is guided by the distilled dataset produced by individual branch contributors, resulting in a generalizable model that performs strongly among multiple tasks.


\subsection{Preliminary}
\label{pre}
In MedForge, the development of a multi-capability model relies on the multi-center and multi-task knowledge introduced by branch plugin modules and the distilled datasets.
The relationship between the main base model and branch plugin modules in our proposed MedForge is conceptually similar to the relationship between the main repository and its branches in collaborative software version control platforms (e.g., GitHub~\cite{github}). 
To facilitate plugin module training on branches and model merging, we use the parameter-efficient finetuning (PEFT) technique~\cite{hu2021lora} for integrating knowledge from individual contributors into the branch plugin modules. 

\subsubsection{Parameter-efficient Finetuning}
Compared to resource-intensive full-parameter finetuning, parameter-efficient finetuning (PEFT) only updates a small fraction of the pretrained model parameters to reduce computational costs and accelerate training on specific tasks. These benefits are particularly crucial in medical scenarios where computational resources are often limited.
As the representative PEFT technique, LoRA (Low-Rank Adaptation)~\cite{hu2021lora} is widely utilized in resource-constrained downstream finetuning scenarios. In our MedForge, each contributor trains a lightweight LoRA on a specific task as the branch plugin module. LoRA decomposes the weight matrices of the target layer into two low-rank matrices to represent the update made to the main model when adapting to downstream tasks. If the target weight matrix is $W_0 \in R^{d \times k}$, during the adaptation, the updated weight matrix can be represented as $W_0+\Delta W=W_0+B A$, where $B \in \mathbb{R}^{d \times r}, A \in \mathbb{R}^{r \times k}$ are the low-rank matrices with rank $r \ll  \min (d, k)$ and $AB$ constitute the LoRA module. 



\subsubsection{Dataset Distillation}
Dataset distillation~\cite{wang2018dataset, yu2023dataset, lei2023comprehensive} is particularly valuable for medicine scenarios that have limited storage capabilities, restricted transmitting bandwidth, and high concerns for data privacy~\cite{li2024dataset}. 
We leverage the power of dataset distillation to synthesize a small-scale distilled dataset from the original data.

The distilled datasets serve as the training set in the subsequent merging stage to allow multi-center knowledge integration. Models trained on this distilled dataset maintain comparable performance to those trained on the original dataset (\ref{tab:main_res}). Moreover, the distinctive visual characteristics among images of the raw dataset are blurred (see \ref{fig:overview}(a)), which alleviates the potential patient information leakage. 

To perform dataset distillation, we define the original dataset as $\mathcal{T}=\{x_i,y_i\}^N_{i=1}$ and the model parameters as $\theta$. The dataset distillation aims to synthesize a distilled dataset ${\mathcal{S}=\{{s_i},\tilde{y_i}\}^M_{i=1}}$ with a much smaller scale (${M \ll N}$), while models trained on $\mathcal{S}$ can show similar performance as models trained on $\mathcal{T}$. 
This process is achieved by narrowing the performance gap between the real dataset $\mathcal{T}$ and the synthesized dataset $\mathcal{S}$. In MedForge, we utilize the distribution matching (DM)~\cite{zhao2023dataset}, which increases data distribution similarity between the synthesized distilled data and the real dataset
The distribution similarity between the real and synthesized dataset is evaluated through the empirical estimate of the Maximum Mean Discrepancy (MMD)~\cite{gretton2012kernel}:
\begin{equation}
\mathbb{E}_{\boldsymbol{\vartheta} \sim P_{\vartheta}}\left\|\frac{1}{|\mathcal{T}|} \sum_{i=1}^{|\mathcal{T}|} \psi_{\boldsymbol{\vartheta}}\left(\boldsymbol{x}_i\right)-\frac{1}{|\mathcal{S}|} \sum_{j=1}^{|\mathcal{S}|} \psi_{\boldsymbol{\vartheta}}\left(\boldsymbol{s}_j\right)\right\|^2
\end{equation}

where $P_\vartheta$ is the distribution of network parameters, $\psi_{\boldsymbol{\vartheta}}$ is a feature extractor. Then the distillation loss $\mathcal{L}_{DM}$ is:
\begin{equation}\scalebox{0.9}{$
\mathcal{L}_{\mathrm{DM}}(\mathcal{T},\mathcal{S},\psi_{\boldsymbol{\vartheta}})=\sum_{c=0}^{C-1}\left\|\frac{1}{\left|\mathcal{T}_c\right|} \sum_{\mathbf{x} \in \mathcal{T}_c} \psi(\mathbf{x})-\frac{1}{\left|\mathcal{S}_c\right|} \sum_{\mathbf{s} \in \mathcal{S}_c} \psi(\mathbf{s})\right\|^2$}
\end{equation}

We also applied the Differentiable Siamese Augmentation (DSA) strategy~\cite{zhao2021dataset} in the training process of distilled data to enhance the quality of the distilled data. DSA could ensure the distilled dataset is representative of the original data by exploiting information in real data with various transformations. The distilled images extract invariant and critical features from these augmented real images to ensure the distilled dataset remains representative.
\begin{figure}[t]
    \centering
    \includegraphics[width=\linewidth]{assets/img/model_arch.png}
    \caption{\textbf{Main model architecture.} We adopt CLIP as the base module and attach LoRA modules to the visual encoder and visual projection as the plugin module. During all the procedures, only the plugin modules are tuned while the rest are frozen. We get the classification result by comparing the cosine similarity of the visual and text embeddings.}
    \label{fig:model_arch}
\end{figure}

\subsection{Feature Branch Development}
\label{branch}
In the feature branch development stage, the branch contributors are responsible for providing the locally trained branch plugin modules and the distilled data to the MedForge platform, as shown in Fig~\ref{fig:overview} (a).
In collaborative software development, contributors work on individual feature branches, push their changes to the main platform, and later merge the changes into the main branch to update the repository with new features. Inspired by such collaborative workflow, branch contributors in MedForge follow similar preparations before the merging stage, enabling the integration of diverse branch knowledge into the main branch while effectively utilizing local resources.

MedForge consistently keeps a base module and a forge item as the main branch. The base module preserves generative knowledge of the foundation model pretrained on natural image datasets (i.e., ImageNet~\cite{deng2009imagenet}), while the forge item contains model merging information that guides the integration of feature branch knowledge (i.e., a merged plugin module or the merging coefficients assigned to plugin modules). 
Similar to individual software developers working in their own branches, each branch contributor (e.g., individual medical centers) trains a task-specific plugin module using their private data to introduce feature branch knowledge into the main branch. These branch plugin modules are then committed and pushed to update the forge items of the main branch in the merging stage, thus enhancing the model's multi-task capabilities.


\begin{figure*}
    \centering
    \includegraphics[width=\textwidth]{assets/img/fusion.png}
    \caption{\textbf{The detailed methodology of the proposed Fusion.} Branch contributors can asynchronously commit and push their branch plugin modules and the distilled datasets. the plugin modules will then be weighted fused to the current main plugin module.}

    \label{fig:merge}
\end{figure*}


Regarding model architecture, MedForge contains a base module and a plugin module (Fig ~\ref{fig:model_arch}). The base module is pretrained on general datasets (e.g., ImageNet) and remains the model parameters frozen in all processes and branches (main and feature branches) to avoid catastrophic forgetting of foundational knowledge acquired from pretraining. Meanwhile, the plugin module is adaptable for knowledge integration and can be flexibly added or removed from the base module, allowing updates without affecting the base model. In our study, we use the pretrained CLIP~\cite{radford2021learning} model as the base module. For the language encoder and projection layer of the CLIP model, all the parameters are frozen, which enables us to directly leverage the language capability of the original CLIP model. For the visual encoder, we apply LoRA on weight matrices of query ($W_q$) and value ($W_v$), following the previous study~\cite{hu2021lora}. To better adapt the model to downstream visual tasks, we apply the LoRA technique to both the visual encoder and the visual projection, and these LoRA modules perform as the plugin module. During the training, only the plugin module (LoRA modules) participates in parameter updates, while the base module (the original CLIP model) remains unchanged. 

In addition to the plugin modules, the feature branch contributors also develop a distilled dataset based on their private local data, which encapsulates essential patterns and features, serving as the foundation for training the merging coefficients in the subsequent merging stage~\ref{forging}. Compared to previous model merging approaches that rely on whole datasets or few-shot sampling, distilled data is lightweight and representative, mitigating the privacy risks associated with sharing raw data. 
We illustrate our distillation procedure in Algorithm~\ref{algorithm:alg1}. In each distillation step, the synthesized data $\mathcal{S}$ will be updated by minimizing $\mathcal{L}_{DM}$.
\begin{algorithmic}[1]
    \STATE \textbf{Input:} A list of clauses $C$
    \STATE \textbf{Output:} List of primary outputs $PO$, primary inputs $PI$, intermediate variables $IV$, and Boolearn expressions $BE$
    \STATE $SC$ = [] \COMMENT{List of sub-clauses}
    \FOR{$i = 1$ to length($C$)}
        % \IF{$C[i] \cap SC = \emptyset$}
        %     \STATE Append \text{Simplify}(\text{FindBooleanExpression}([], $SC$)) to $BE$
        %     %\COMMENT{Simplify Boolean expression}
        %     \FOR{each item $w$ in $SC$}
        %         \IF{$w \notin IV$ and $w \neq v$}
        %             \STATE Append $w$ to $PI$
        %         \ENDIF
        %     \ENDFOR
        %     \STATE $SC$ = []
        % \ELSE
            \STATE Append $C[i]$ to $SC$
            \FOR{each item $v$ in $SC$}
                \IF{$v \notin PI$ and $v \notin IV$}
                    \STATE $f \gets \text{FindBooleanExpression}(v, SC)$ %\COMMENT{Find Boolean expression for $v$}
                    \STATE $g \gets \text{FindBooleanExpression}(\neg v, SC)$ %\COMMENT{Find Boolean expression for $\neg v$}
                    \IF{$f = \neg g$}
                        \STATE Append \text{Simplify}($f$) to $BE$ %\COMMENT{Simplify Boolean expression}
                        \IF{$f = True$ or $f = False$}
                            \STATE Append $v$ to $PO$
                        \ELSE
                            \STATE Append $v$ to $IV$
                        \ENDIF
                        \FOR{each item $w$ in $SC$}
                            \IF{$w \notin IV$ and $w \neq v$}
                                \STATE Append $w$ to $PI$
                            \ENDIF
                        \ENDFOR
                        \STATE SC = []
                        \STATE \textbf{break}
                    \ENDIF
                \ENDIF    
            \ENDFOR
        % \ENDIF
    \ENDFOR
    \STATE \textbf{Return} $PO, PI, IV, BE$
    \vspace{-0.65cm}
\end{algorithmic}



\subsection{MedForge Merging Stage}
\label{forging}
Following the feature branch development stage illustrated in Fig~\ref{fig:overview} (a), branch contributors push and merge their branch plugin modules along with the corresponding distilled dataset into the main branch, as shown in Fig~\ref{fig:overview} (b). Our MedForge allows an incremental capability accumulation from branches to construct a comprehensive medical model that can handle multiple tasks.

In the merging stage, the $i^{th}$ branch contributor is assigned a coefficient $w'_i$ for the contribution of merging, while the coefficient for the current main branch is $w_i$. By adaptively adjusting the value of coefficients, the main branch can balance and coordinate updates from different contributors, ultimately enhancing the overall performance of the model across multiple tasks.
The optimization of the coefficients is done by minimizing the cross-entropy loss for classification based on the distilled datasets. We also add $L1$ regularization to the loss to regulate the weights to avoid outlier coefficient values (e.g., extremely large or small coefficient values)~\cite{huang2023lorahub}. During optimization, following~\cite{huang2023lorahub}, we utilize Shiwa algorithm~\cite{liu2020versatile} to enable model merging under gradient-free conditions, with lower computational and time costs. The optimizer selector~\cite{liu2020versatile} automatically chooses the most suitable optimization method for coefficient optimization. 

In the following sections, we introduce the two merging methods used in our MedForge: Fusion and Mixture. In MedForge-Fusion, the parameters of the branch plugin modules are fused into the main branch after each round of the merging stage. For MedForge-Mixture, the outputs of the branch modules are weighted and summed based on their respective coefficients rather than directly applying the weighted sum to the model parameters. This largely preserves the internal parameter structure of each branch module.

\paragraph{MedForge-Fusion}
In MedForge, forge items are utilized to facilitate the integration of branch knowledge into the main branch.
For MedForge-Fusion, the forge item refers to adaptable main plugin modules. When the $i^{th}$ branch contributor pushes its branch plugin module $\theta'_i=A'_iB'_i$ to the main branch, the current main plugin module $\theta_{i-1}=A_{i-1}B_{i-1}$ will be updated to $\theta_{i}=A_{i}B_{i}$. The parameters of the branch and the current main plugin modules are weighted with coefficients and added to fuse a new version. The $A_i$, $B_i$ are the low-rank matrices composing the LoRA module $\theta_i$. The detailed fusion process can be represented as:
\begin{equation}
\theta_{i}=(w_i A_{i-1}+w'_i A'_i)(w_i B_{i-1}+ w'_i B'_i)
\end{equation}
Where $w_i$ is the coefficient assigned to the current main branch, while $w'_i$ is the coefficient assigned to the branch contributor. After this round of merging, the resulting plugin module $\theta_{i}$ is the updated version of main forging item, thus the main model is able to obtain new capacity introduced by the current branch contributor. When new contributors push their plugin modules and distilled datasets, the main branch can be incrementally updated through merging stages, and the optimization of the coefficients is guided by distilled data.
As shown in Fig.~\ref{fig:merge}, though multiple contributors commit their branch plugin modules and distilled datasets at different times, they can flexibly merge their plugin modules with the current main branch. After each merging round, the plugin module of the main branch will be updated, and thus the version iteration has been achieved.
\begin{figure*}[t]
    \centering
    \includegraphics[width=\textwidth]{assets/img/mixture.png}
    \caption{\textbf{The detailed methodology of the proposed Mixture.} Branch contributors can asynchronously commit and push their branch plugin modules and the distilled datasets. the outputs of different plugin modules will be weighted aggregated. The weights of each merging step will be saved.}

    \label{fig:mixmerge}
\end{figure*}


\paragraph{MedForge-Mixture}
To further improve the model merging performance, inspired by~\cite{zhao2024loraretriever}, we also propose medForge-mixture. For MedForge-Mixture, the forge items refer to the optimized coefficients.
As shown in Fig.~\ref{fig:mixmerge}, for MedForge-Mixture, the coefficient of each branch contributor is acquired and optimized based on distilled datasets. Then the outputs of plugin modules will be weighted combined with these coefficients to get the merged output. 

For each merging round, with branch contributor $i$, the branch coefficient is $w'_i$, the main coefficient is $w_i$, the branch plugin module is $\theta'_i=A'_iB'_i$, and the current main plugin module is $\theta_i=A_iB_i$. With the input $x$, the resulted MedForge-Mixture output can be represented as:
\begin{equation}
y_{i}=w_i A_{i-1} B_{i-1} x+w'_i A'_i B'_i x
\end{equation}

In this way, MedForge encourages additional contributors as the workflow supports continuous incremental knowledge updates.

Overall, both MedForge merging strategies greatly improve the communication efficiency among contributors. We use this design to build a multi-task medical foundation model that enhances the full utilization of resources in the medical community. For the MedForge-Fusion strategy, the main plugin module is updated after each merging round, thus avoiding storing the previous plugin modules and saving space. Meanwhile, the MedForge-Mixture strategy avoids directly updating the parameters of each plugin module, thus preserving their original structure and preventing the introduction of additional noise, which enhances the robustness and stability of the models.


\section{Experiments and Results}
\subsection{Experiment Settings}

\begin{table*}[ht]
    \centering
    % \small
    \caption{The main results of our experimentation. Each row group corresponds to the results for the given dataset, with each row showcasing the metric results for each model. The columns include all the main approaches, with \textbf{bold} highlighting the best result across all approaches.}
    \small
    \begin{tabular}{llccccc}
      \toprule
      Dataset & Model & Baseline & RAG & CoT & RaR & \rephrase \\
      \midrule
      \multirow[l]{3}{*}{TriviaQA}
          & Llama-3.2 3B  & 59.5 & 82.0 & 87.5  & 86.0 &  \textbf{88.5}    \\
          & Llama-3.1 8B  & 76.5 & 89.5 & 90.5  & 89.5 &  \textbf{92.5}    \\
          & GPT-4o    & 88.7 & 92.7 & 92.7  & 94.7 &  \textbf{96.7}    \\
      \midrule
      \multirow[l]{3}{*}{HotpotQA}
          & Llama-3.2 3B  &  17.5  & 26.0  & 26.5   & 25.0  &  \textbf{31.5}   \\
          & Llama-3.1 8B  &  23.0  & 26.5  & 31.0   & 28.5  &  \textbf{33.5}   \\
          & GPT-4o    &  44.0  & 45.3  & 46.7   & \textbf{47.3}  &  46.7   \\
      \midrule
      \multirow[l]{3}{*}{ASQA}
          & Llama-3.2 3B  &  14.2 & 21.5  & 21.9  & 23.5  &  \textbf{26.6}   \\ 
          & Llama-3.1 8B  &  14.6 & 23.1  & 24.8  & 25.5  &  \textbf{28.8}   \\ 
          & GPT-4o    &  26.8 & 30.4  & \textbf{31.9}  & 30.1 & 31.7 \\ 
      \bottomrule
    \end{tabular}
    \label{tab:main}
\end{table*}



\textbf{Datasets}. We conduct experiments on two datasets: CC-news\footnote{\href{https://huggingface.co/datasets/vblagoje/cc_news}{Huggingface: vblagoje/cc\_news}} and Wikipedia\footnote{\href{https://huggingface.co/datasets/legacy-datasets/wikipedia}{Huggingface: legacy-datasets/Wikipedia}}. CC-news is a large collection of news articles which includes diverse topics and reflects real-world temporal events. Meanwhile, Wikipedia covers general knowledge across a wide range of disciplines, such as history, science, arts, and popular culture.\\
\textbf{LLMs}: We experiment on three models including \gpt~(124M)~\cite{gpt2radford}, \pythia~(1.4B)~\cite{pythia}, and \llama~(7B)~\cite{llama2touvron2023}. This selection of models ensures a wide range of model sizes from small to large that allows us to analyze scaling effects and generalizability across different capacities. \\
\textbf{Evaluation Metrics}. For evaluating language modeling performance, we measure perplexity (PPL), as it reflects the overall effectiveness of the model and is often correlated with improvements in other downstream tasks~\cite{kaplan2020scalinglaws, lmsfewshot}. For defense effectiveness, we consider the attack area under the curve (AUC) value and True Positive Rate (TPR) at low False Positive Rate (FPR). In total, we perform 4 MIAs with different MIA signals. Given the sample $x$, the MIA signal function $f$ is formulated as follows: \\
$\bullet$ Loss~\cite{8429311} utilizes the negative cross entropy loss as the MIA signal. 
    \[f_\text{Loss}(x) = \mathcal{L}_\text{CE}(\theta; x) \]
$\bullet$ Ref-Loss~\cite{Carlini2020ExtractingTD} considers the loss differences between the target model and the attack reference model. To enhance the generality, our experiments ensure there is no data contamination between the training data of the target, reference, and attack models.
    \[f_\text{Ref}(x) = \mathcal{L}_\text{CE}(\theta; x) - \mathcal{L}_\text{CE}(\theta_\text{attack}; x) \]
$\bullet$ Min-K~\cite{shi2024detecting} leverages top K tokens that have the lowest loss values.
    \[f_\text{min-K}(x) = \frac{1}{|\text{min-K(x)}|} \sum_{t_i \in \text{min-K(x)}} -\log(P(t_i|t_{<i};\theta) \]
$\bullet$ Zlib~\cite{Carlini2020ExtractingTD} calibrates the loss signal with the zlib compression size.
    \[ f_\text{zlib}(x) = \mathcal{L}_\text{CE}(\theta; x) / \text{zlib}(x) \]

\noindent \textbf{Baselines}. We present the results of four baselines. \textit{Base} refers to the pretrained LLM without fine tuning. \textit{FT} represents the standard causal language modeling without protection. \textit{Goldfish}~\cite{hans2024be} implements a masking mechanism. \textit{DPSGD}~\cite{abadi2016deep, yu2022differentially} applies gradient clipping and injects noise to achieve  sample-level differential privacy.

\noindent \textbf{Implementation}. We conduct full fine-tuning for \gpt and \pythia. For computing efficiency, \llama fine-tuning is implemented using Low-Rank Adaptation (LoRA)~\cite{hu2022lora} which leads to \textasciitilde4.2M trainable parameters. Additionally, we use subsets of 3K samples to fine-tune the LLMs. We present other implementation details in Appendix~\ref{sec:app-implementation}.

\subsection{Overall Evaluation}
Table~\ref{tab:main_result} provides the overall evaluation compared to several baselines across large language model architectures and datasets. Among these two datasets, CCNews is more challenging, which  leads to higher perplexity  for all LLMs and fine-tuning methods. Additionally, the reference-model-based attack performs the best and demonstrates high privacy risks with attack AUC on the conventional fine-tuned models at 0.95 and 0.85 for Wikipedia and CCNews, respectively. Goldfish achieves similar PPL to the conventional FT method but fails to defend against MIAs. This aligns with the reported results by \citet{hans2024be} that Goldfish resists exact match attacks but only marginally affects MIAs. DPSGD provides a very strong protection in all settings (AUC < 0.55) but with a significant PPL tradeoff. Our proposed \methodname guarantees a robust protection, even slightly better than DPSGD, but with a notably smaller tradeoff on language modeling performance. For example, on the Wikipedia dataset, \methodname delivers perplexity reduction by 15\% to 27\%. Moreover, Table~\ref{tab:tpr} (Appendix~\ref{sec:app-add-res}) provides the TPR at 1\% FPR. Both DPSGD and \methodname successfully reduce the TPR to $\sim$0.02 for all LLMs and datasets. \textit{Overall, across multiple LLM architectures and datasets, \methodname consistently offers ideal privacy protection with  little trade-off in language modeling performance.}

\noindent \textbf{Privacy-Utility Trade-off.}
To investigate the privacy-utility trade-off of the methods, we vary the hyper-parameters of the fine-tuning methods. Particularly, for DPSGD, we adjust the privacy budget $\epsilon$ from (8, 1e-5)-DP to (100, 1e-5)-DP. We modify the masking percentage of Goldfish from 25\% to 50\%. Additionally, we vary the loss weight $\alpha$ from 0.2 to 0.8 for \methodname. Figure~\ref{fig:priv-ult-tradeoff} depicts the privacy-utility trade-off for GPT2 on the CCNews dataset. Goldfish, with very large masking rate (50\%), can slightly reduce the risk of the reference attack but can increase the risks of other attacks. By varying the weight $\alpha$, \methodname offers an adjustable trade-off between privacy protection and language modeling performance. \methodname largely dominates DPSGD and improves the language modeling performance by around 10\% with the ideal privacy protection against MIAs.

\begin{figure}[h]
    \centering
    \includegraphics[width=\linewidth]{figs/privacy-ultility-tradeoff.pdf}
    \caption{Privacy-utility trade-off of the methods while varying hyper-parameters. The Goldfish masking rate is set to 25\%, 33\%, and 50\%. The privacy budget $\epsilon$ of DPSGD is evaluated at 8, 16, 50, and 100. The weight $\alpha$ of \methodname is configured at 0.2, 0.5, and 0.8.}
    \label{fig:priv-ult-tradeoff}
\end{figure}


\subsection{Ablation Study}
\textbf{\methodname without reference models.} To study the impact of the reference model, we adapt \methodname to a non-reference version which directly uses the loss of the current training model (i.e., $s(t_i) = \mathcal{L}_{CE}(\theta; t_i)$) to select the learning and unlearning tokens. This means the unlearning tokens are the tokens that have smallest loss values. Figure~\ref{fig:ppl-auc-noref} presents the training loss and testing perplexity. There is an inconsistent trend of the training loss and testing perplexity. Although the training loss decreases overtime, the test perplexity increases. This result indicates that identifying appropriate unlearning tokens  without a reference model is challenging and conducting unlearning on an incorrect set hurts the language modeling performance.

\begin{figure}[htp]
    \centering
    \includegraphics[width=0.35\textwidth]{figs/train_loss_ppl_noref.pdf}
    \caption{Training Loss and Test Perplexity of \methodname without a reference model.
    % (\lrx{If time permits, it would be better to compare with our training curve here)}
    }
    \label{fig:ppl-auc-noref}
\end{figure}

\noindent \textbf{\methodname with out-of-domain reference models.} To examine the influence of the distribution gap in the reference model, we replace the in-domain trained reference model with the original pretrained base model. 
Figure~\ref{fig:ppl-auc-base-woasc} depicts the language modeling performance and privacy risks in this study. \methodname with an out-of-domain reference model can reduce the privacy risks but yield a significant gap in language modeling performance compared to \methodname using an in-domain reference model.

\noindent \textbf{\methodname without Unlearning.} To study the effects of unlearning tokens, we implement \methodname which use the first term of the loss only ({$\mathcal{L}_{\theta} = \mathcal{L}_{CE}(\theta; \mathcal{T}_h)$}). Figure~\ref{fig:ppl-auc-base-woasc} provides the perplexity and MIA AUC scores in this setting. Generally, without gradient ascent, \methodname can marginally reduce membership inference risks while slightly improving the language modeling performance. The token selection serves as a regularizer that helps to improve the language modeling performance. Additionally, tokens that are learned well in previous epochs may not be selected in the next epochs. This slightly helps to not amplify the memorization on these tokens over epochs.

\begin{figure}[htp]
    \centering
    \includegraphics[width=0.28\textwidth]{figs/auc_vs_ppl_base_woasc.pdf}
    \caption{Privacy-utility trade-off of \methodname with different settings: in-domain reference model, out-domain reference model, and without unlearning}
    \label{fig:ppl-auc-base-woasc}
\end{figure}


\subsection{Training Dynamics}
\textbf{Memorization and Generalization Dynamics}. Figure~\ref{fig:training-dynamics} (left) illustrates the training dynamics of conventional fine tuning and \methodname, while Figure~\ref{fig:training-dynamics} (middle) depicts the membership inference risks. Generally, the gap between training and testing loss of conventional fine-tuning steadily increases overtime, leading to model overfitting and high privacy risks. In contrast, \methodname maintains a stable equilibrium where the gap remains more than 10 times smaller. This equilibrium arises from the dual-purpose loss, which balances learning on hard tokens while actively unlearning memorized tokens. By preventing excessive memorization, \methodname mitigates membership inference risks and enhances generalization.

\begin{figure*}[htp]
    \centering
    \includegraphics[width=0.29\linewidth]{figs/loss_vs_steps_ft_duolearn.pdf}
    \includegraphics[width=0.29\linewidth]{figs/auc_vs_steps_ft_duolearn.pdf}
    \includegraphics[width=0.316\linewidth]{figs/cosine.pdf}
    \caption{Training dynamics of \methodname and the conventional fine-tuning approach. The left and middle figures provide the training-testing gap and membership inference risks, respectively. The testing~$\mathcal{L}_{CE}$ of FT and training~$\mathcal{L}_{CE}$ of \methodname are significantly overlapping, we provide the breakdown in Figure~\ref{fig:add-overlap-breakdown} in Appendix~\ref{sec:app-add-res}. The right figure depicts the cosine similarity of the learning and unlearning gradients of \methodname. Cosine similarity of 1 means entire alignment, 0 indicates orthogonality, and -1 presents full conflict.}
    \label{fig:training-dynamics}
\end{figure*}

\noindent \textbf{Gradient Conflicts}. To study the conflict between the learning and unlearning objectives in our dual-purpose loss function, we compute the gradient for each objective separately. We then calculate the cosine similarity of these two gradients. Figure~\ref{fig:training-dynamics} (right) provides the cosine similarity between two gradients over time. During training, the cosine similarity typically ranges from -0.15 to 0.15. This indicates a mix of mild conflicts and near-orthogonal updates. On average, it decreases from 0.05 to -0.1. This trend reflects increasing gradient misalignment. Early in training, the model may not have strongly learned or memorized specific tokens, so the conflicts are weaker. Overtime, as the model learns more and memorization grows, the divergence between hard and memorized tokens increases, making the gradients less aligned. This gradient conflict is the root of the small degradation of language modeling performance of \methodname compared to the conventional fine tuning approach.

\noindent \textbf{Token Selection Dynamics}. Figure~\ref{fig:token-selection} illustrates the token selection dynamics of \methodname during training. The figure shows that the token selection process is dynamic and changes over epochs. In particular, some tokens are selected as an unlearning from the beginning to the end of the training. This indicates that a token, even without being selected as a learning token initially, can be learned and memorized through the connections with other tokens. This also confirms that simple masking as in Goldfish is not sufficient to protect against MIAs. Additionally, there are a significant number of tokens that are selected for learning in the early epochs but unlearned in the later epochs. This indicates that the model learned tokens and then memorized them over epochs, and the during-training unlearning process is essential to mitigate the memorization risks.

\begin{figure}[htp]
    \centering
    \includegraphics[width=0.7\linewidth]{figs/token-selection-dynamics.pdf}
    \caption{Token Selection Dynamics of \methodname}
    \label{fig:token-selection}
    \vspace{-4mm}
\end{figure}

\subsection{Privacy Backdoor}
To study the worst case of privacy attacks and defense effectiveness under the state-of-the-art MIA, we perform a privacy backdoor -- Precurious~\cite{precurious}. In this setup, the target model undergoes continual fine-tuning from a warm-up model. The attacker then applies a reference-based MIA that leverages the warm-up model as the attack's reference. Table~\ref{tab:backdoor} shows the language modeling and MIA performance on CCNews with GPT-2. Precurious increases the MIA AUC score by 5\%. Goldfish achieves the lowest PPL, aligning with~\citet{hans2024be}, where the Goldfish masking mechanism acts as a regularizer that potentially enhances generalization. Both DPSGD and \methodname provide strong privacy protection, with \methodname offering slightly better defense while maintaining lower perplexity than DPSGD.

% \begin{table}[h]
%     \centering
%     \begin{tabular}{c|cc|cc}
%        \multirow{2}{*}{\textbf{Method}}  & \multicolumn{2}{c}{\textbf{CCNews}} & \multicolumn{2}{c}{\textbf{Wikipedia}} \\ 
%        & \textbf{PPL} & \textbf{AUC} & \textbf{PPL} & \textbf{AUC} \\ \hline
%        \textbf{FT}        & 21.593 & 0.911 \\
%        \textbf{Goldfish}  & \textbf{21.074} & 0.886 \\
%        \textbf{DPSGD}     & 23.279 & 0.533 \\
%        \textbf{DuoLearn}  & 22.296 & \textbf{0.499} \\
%     \end{tabular}
%     \caption{Caption}
%     \label{tab:my_label}
% \end{table}

\begin{table}[h]
    \centering
    \resizebox{\columnwidth}{!}{\begin{tabular}{c|cccccc}
        \textbf{Metric} & \textbf{WU} & \textbf{FT} & \textbf{GF} & \textbf{DP} & \textbf{DuoL} \\ \hline
        \textbf{PPL} & \textit{23.318} & 21.593 & \textbf{21.074} & 23.279 & 22.296  \\
        \textbf{AUC} & \textit{0.500} & 0.911 & 0.886 & 0.533 & \textbf{0.499} \\
    \end{tabular}}
    \caption{Experimental results of privacy backdoor for GPT2 on the CC-news dataset. WU stands for the warm-up model leveraged by Precurious. GF, DP, and DuoL are abbreviations of Goldfish, DPSGD, and \methodname}
    \label{tab:backdoor}
\end{table}

% \subsubsection{Hyperparameter Study}

% \subsubsection{Full fine-tuning versus Parameter efficent fine tuning}

% \subsubsection{Extending to Vision Language Models}




\section{Conclusion}
Inspired by causality in ecology, we propose the CCMPlus module, which captures causal relationships between web services and integrates with time series models to improve web service traffic prediction.

CCMPlus first constructs shadow manifolds for each web service time series. It then estimates each time point of target time series using points from the shadow manifolds of other web services. By comparing these estimations with the ground truth of the target time series, a causal correlation matrix is computed, quantifying inter-service causal dependencies. The causal matrix is then applied to the shadow manifold embeddings, generating the CCMPlus representation, which encodes causal relationships. This representation is concatenated with the time series model’s output, bridging the gap in causal inference and enhancing prediction performance.

We evaluate CCMPlus-integrated models on real-world web service traffic datasets from Alibaba Group, Microsoft Azure, and Ant Group. The results demonstrate that CCMPlus consistently improves web service traffic prediction, confirming its effectiveness.





% \section{Ethics Statement}
% Our research is solely dedicated to addressing the problem of forecasting web service traffic time series. Consequently, it does not involve any potential ethical risks.
% \begin{acks}
%  We would like to express our gratitude to our colleagues, families, and friends for their invaluable support throughout this work.
% \end{acks}

%\clearpage

\bibliographystyle{ACM-Reference-Format}
\bibliography{main}

\end{document}
\endinput
