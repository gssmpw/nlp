\begin{abstract}

Recent advances in generative models enable highly realistic image manipulations, creating an urgent need for robust forgery detection methods. Current datasets for training and evaluating these methods are limited in scale and diversity. To address this, we propose a methodology for creating high-quality inpainting datasets and apply it to create \textbf{\datasetname}, comprising over 95,000 inpainted images generated from 78,000 original images sourced from MS-COCO, RAISE, and OpenImages.
Our methodology consists of three components: (1) Semantically Aligned Object Replacement (SAOR) that identifies suitable objects through instance segmentation and generates contextually appropriate prompts, (2) Multiple Model Image Inpainting (MMII) that employs various state-of-the-art inpainting pipelines primarily based on diffusion models to create diverse manipulations, and (3) Uncertainty-Guided Deceptiveness Assessment (UGDA) that evaluates image realism through comparative analysis with originals. The resulting dataset surpasses existing ones in diversity, aesthetic quality, and technical quality. We provide comprehensive benchmarking results using state-of-the-art forgery detection methods, demonstrating the dataset's effectiveness in evaluating and improving detection algorithms. Through a human study with 42 participants on 1,000 images, we show that while humans struggle with images classified as deceiving by our methodology, models trained on our dataset maintain high performance on these challenging cases. Code and dataset are available at \url{https://github.com/mever-team/DiQuID}.
\end{abstract}
