\section{Conclusions and Future work}
\label{sec:conclusions}

We presented \emph{\datasetname}, a large-scale dataset for evaluating inpainting detection methods, built using a novel pipeline with three key components: SOAR for semantic-aware object selection, MMII for diverse inpainting manipulations, and UGDA for deceptiveness assessment. Built to be model-agnostic, our approach can integrate future improvements in segmentation, captioning, and language models.
Unlike previous datasets, our approach leverages multiple state-of-the-art inpainting pipeline and uses language models to generate contextually rich prompts, enhancing the aesthetic and technical quality of manipulated images.
Our benchmarking reveals both strengths and limitations of current detection methods, particularly for FR images and compressed images. Looking forward, our framework can benefit from advances in foundation models to improve prompt generation, realism assessment, and object selection. Future work should focus on developing detection architectures that are robust to compression while maintaining accurate localization of manipulated regions. We hope \emph{\datasetname} will drive progress in developing more robust forgery detection methods as AI-generated content proliferates.
\vspace{-14pt} 
\paragraph{Acknowledgments:} This work was supported by the Horizon Europe vera.ai project (grant no.101070093) and by the High Performance Computing infrastructure of the Aristotle University of Thessaloniki.
