\section{How discrete is language?}
\label{sec:how-discr-lang}

Linguists have found symbolic formalisms useful across all main domains of language, such as phonology~\cite{chomsky1968sound,PrinceSmolensky1993}, morphosyntax~\cite{Chomsky1957,bresnan1982mental,Langacker1987,pollard1994head,goldberg95}, semantics~\cite{montague74:EFL,partee+90,pustejovsky95}, and pragmatics~\cite{Grice1989,sperber-wilson95}.
In this article, I will focus on morphosyntax and semantics.

Work in morphosyntax posits for instance that words belong to different parts of speech (such as determiner, noun, or verb) and can stand in different syntactic relations (such as subject, object, or indirect object).
Languages mark morphosyntax formally, and restrictions in the co-occurrence of linguistic units (morpheme, words, clauses) are governed by morphosyntactic properties.
For instance, in English only verbs inflect for tense; and, in most, verbs past tense is signaled ty the suffix \textit{-ed} (``follow/followed'').
Similarly, only some verbs allow for indirect objects, and the indirect object in English is marked by the preposition \textit{to} (see example~(\ref{ex15})).
In many languages different units in the sentence display agreement~\cite{wechsler2003many}.
Example~(\ref{ex15}) showcases how, in Spanish, there is gender and number agreement within the noun phrase: the highlighted suffix {\textit{-a}} on the determiner and adjective mark feminine gender, in agreement with the noun's lexical gender.
Similarly, in English, subjects and verbs agree in number; in example~(\ref{ex2}), the singular subject (``A student'') cannot combine with a plural verb (``are'').

\renewcommand{\labelenumi}{(\theenumi)}
\begin{enumerate}
\item \label{ex15} John gave/*prepared a drink to Mary
\item \label{ex1} L\textbf{\underline{a}}s partes interesad\textbf{\underline{a}}s\\
  the.FEM.PL party.PL interested.FEM.PL\\
  `The interested parties'
\item \label{ex2} A student is/*are crossing the street
\end{enumerate}

In compositional semantics and the syntax-semantics interface, we find phenomena such as negation, where, in a sentential context, adding negation reverses polarity  \cite[][see example~(\ref{ex6})]{zeijlstra2007negation}, and anaphora, where syntactic constraints determine the shape of anaphoric pronouns: for instance, in~(\ref{ex3}), the pronoun ``him'' cannot refer to Mark~\cite{Chomsky1981}.

\renewcommand{\labelenumi}{(\theenumi)}
\begin{enumerate}[resume]
\item \label{ex6} I will/will not come to lunch
\item \label{ex3} Mark$_i$ combs himself$_i$/*him$_i$
\end{enumerate}

All of these phenomena are largely symbolic and discrete, in that there is no ``in between'' state: the choice between ``is'' and ``are'' is determined by the number of the subject; ``not'' is a like a binary switch for polarity in sentences; etc.
However, even in this realm one only needs to scratch the surface for discreteness to break down.
The border between parts of speech is notoriously fuzzy~\cite{Croft2001,Evans2009}; there is no universal agreed upon set of syntactic relations~\cite{dowty1991thematic}; negation is far from being a binary switch in many contexts (e.g., ``not unhappy'' does not mean ``happy''), and is hugely complex from a semantic point of view~\cite{zeijlstra2007negation};
and even agreement can break down~\cite{wechsler2003many}.

Consider agreement \textit{ad sensum}, exemplified in~(\ref{ex5}).
Here, the syntactic subject is the singular noun ``group'', but the plural form, forbidden in example~(\ref{ex2}), is allowed in this case.

\begin{enumerate}[resume]
\item \label{ex5} A group of students from New Zealand is/are crossing the street
\end{enumerate}

This example showcases the interaction between grammar and meaning, as \textit{ad sensum} agreement happens with singular head nouns that denote pluralities, such as ``group''.
Aspects of meaning that are conceptual in nature are, indeed, the source of much of language's fuzziness~\cite{wittgenstein53}:
Word meaning, for instance, is notoriously fuzzy, vague, and slippery.
As an example, in contrast to cases like (\ref{ex1}-\ref{ex3}) above, the similarities and differences between ``fast'' and ``swift'' are subtle, and there is no hard and fast rule to determine when to use one and when to use the other.
Moreover, while most words have many meanings, more often than not they are difficult to delineate~\cite{kilgarriff97}.
Hence, symbolic formalisms with discrete representations are highly problematic for word meaning~\cite{wittgenstein53,kilgarriff97,boleda2020distributional}.

Construction grammar, a family of theories within cognitive linguistics~\cite{Langacker1987,Lakoff1987,Fillmore1988,goldberg95,Croft2001}, has put the relationship between conceptual meaning and grammar center stage.
While these approaches still use discrete representations, they contest the existence of abstract syntactic rules of the sort exemplified in Figure~\ref{fig:main-fig} (top), which are advocated by generative linguists.
Scholars in construction grammar instead propose the existence of patterns (termed \textit{constructions}) at different levels of abstraction, consisting of pairings of form and meaning.
\footnote{``Any linguistic pattern is recognized as a construction as long as some aspect of its form
or function is not strictly predictable from its component parts or from other
constructions recognized to exist. In addition, patterns are stored as constructions even if
they are fully predictable as long as they occur with sufficient frequency.'' \cite[][p. 5]{goldberg05}}
Constructions are often semi-productive and heavily dependent on conceptual aspects of meaning, such that it is again difficult to establish hard and fast rules for their use that can be specified on formal grounds only.
For instance, the verb ``to sneeze'', which is not causative, can sometimes be used felicitously in a causative construction, as in example~(\ref{ex7}), attributed to Adele Goldberg by~\citet{hill2024transformersobviouslygoodmodels}.

\renewcommand{\labelenumi}{(\theenumi)}
\begin{enumerate}[resume]
\item \label{ex7} They sneezed the foam off the cappuccino
\end{enumerate}

It should however be noted that not all aspects of meaning are fuzzy; in particular, reference in language is largely discrete~\cite{frege1892}.
We use language to refer to entities and, from a linguistic point of view, there is nothing fuzzy in the distinction between, say, two people with the same name. Thus, whether ``Elizabeth Blackburn won the Nobel prize'' is true will depend on which Elizabeth Blackburn we're talking about in the given context.
\footnote{As of 2025, there are at least two Elizabeth Blackburns: a Nobel laureate and a judge in Florida.}
This is in contrast to conceptual aspects of meaning.

To sum up, this overview suggests that language is indeed both discrete and continuous; and that there is no neat discrete/continuous divide, nor any area of language that is completely discrete or completely continuous.
At the same time, there \textit{are} clearly areas that are more discrete (such as grammar) and areas that are more continuous (such as word meaning).
On the other hand, largely because of methodological limitations, most linguistic formalisms to date continue to be discrete.
\footnote{I should note that there have been several developments in integrating a probabilistic component, especially in semantics and pragmatics~\cite[see][for an overview]{erk2022probabilistic}.
Computational linguistics has also participated in the debate; for instance, researchers in the field have explored the combination of symbolic and distributed approaches to semantics, building on their complementary strengths and weaknesses~\cite[see][]{boleda-herbelot-2016-formal}.

However, I think it is fair to say that these efforts have not as yet succeeded in providing a unified linguistic framework that encompasses the phenomena reviewed in this section.}
Given the properties of language just discussed, and the fact that, as discussed in the introduction, neural networks afford the potential for both continuous and near-discrete behavior, we can expect LLMs to exploit this potential.

And this is indeed what recent literature on interpretability suggests.
In what follows, I will focus on providing evidence of near-discrete behavior, as continuous behaviors are already widely recognized in the field (e.g., in the literature on word embeddings, both static and contextualized).
Moreover, I will focus mainly on morphosyntax, an area that has received considerable attention in the interpretability literature.
