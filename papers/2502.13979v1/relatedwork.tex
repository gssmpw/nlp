\section{Related Work}
% \subsection{Financial Risk Management}
% Since the 1950s, the backbone of financial risk management has been statistical and regression analysis methods, which laid a robust groundwork for the quantitative analysis and systematic handling of risk~\cite{baesens2003benchmarking}. However, the global financial crisis in 2008 highlighted the inadequacies of these traditional models under extreme market conditions, especially their inability to adequately address the complex interconnections within the financial system~\cite{somin2020network}.

% In recent years, with the rapid development of machine learning and deep learning technologies, the field of financial risk management has encountered new opportunities for growth. 

Recent studies have utilized machine learning and deep learning technologies to capture and analyze complex data patterns, offering new perspectives and methods for risk assessment and management. 
% For example, the application of neural network technology to detect financial fraud, or the combination of deep reinforcement learning and graph network technologies to reduce the contagion risk in banking systems, demonstrates the potential and value of these new technologies in financial risk management.
% Many unified models based on the convolutional neural networks (CNNs) and long short-term memory (LSTM)
% are introduced to process financial data, providing insights into market fluctuations~\cite{lu2020cnn,chen2021constructing,cavalli2021cnn,wang2021stock}.
% 
% The graph neural network (GNN) family
% (e.g., 
% Risk-Rate~\cite{cheng2020contagious}, 
% Know-GNN~\cite{rao2021know}, MAGNN~\cite{cheng2022financial},  DGA-GNN~\cite{duan2024dga})
% is utilized to analyze trading data within financial networks, identifying unusual transactions and financial risks.
For example, TRACER~\cite{cheng2020contagious}, iConReg~\cite{cheng2022regulating}, SCRPF~\cite{cheng2023critical}, RisQNet~\cite{ijcai2024p817}
utilize graphs to depict the loan-guarantee relationships among small and medium-sized enterprises in the networked loan market, and construct effective deep graph neural networks to identify and curb the propagation of financial defaults.


In this paper, acknowledging the dynamic propagation of financial risks across spatial and temporal dimensions, and their evolving patterns, we introduce a novel and effective dynamic graph learning model designed for the recognition and analysis of financial risks.
% Besides, we develop a visualization analysis tool for financial risk propagation that performs causal analysis based on accurately identifying financial risks.
% \subsection{Graph Neural Network}


% Financial risks significantly affect financial institutions, markets, and the broader economy, thereby attracting widespread attention.
% 
% Early methods in financial risk recognition relied heavily on historical data and statistical models~\cite{santomero1997financial,prevost2000derivatives}. Techniques such as the value-at-risk ~\cite{duffie1997overview,linsmeier2000value} and expected shortfall~\cite{tasche2002expected,acerbi2002coherence,lazar2019model} were widely adopted for market risk assessment. Credit risk was evaluated through credit ratings~\cite{hilscher2017credit,carey2001parameterizing,pertaia2022new} and credit scoring models~\cite{blochlinger2006economic}. However, these traditional approaches often lacked the ability to capture the dynamic and interconnected nature of financial risks.

% Current studies usually leverage deep learning models for risk detection and analysis.
% For example,
% many unified models based on the convolutional neural networks (CNNs) and long short-term memory (LSTM)
% are introduced to process financial data, providing insights into market fluctuations~\cite{lu2020cnn,chen2021constructing,cavalli2021cnn,wang2021stock}.

% The graph neural network (GNN) family
% (e.g., 
% Risk-Rate~\cite{cheng2020contagious}, 
% Know-GNN~\cite{rao2021know}, MAGNN~\cite{cheng2022financial},  DGA-GNN~\cite{duan2024dga})
% is utilized to analyze trading data within financial networks, identifying unusual transactions and financial risks.

% \subsection{Causal Inference in Anomaly Detection}
% Causal inference has gained significant attention for revealing deep causal mechanisms within data, beyond mere correlations. It is extensively applied to enhance fairness, interpretability, and understanding of complex behaviors in anomaly detection tasks~\cite{lin2022causal,liu2023learning,carvalho2024invariant}.

% deep learning~\cite{zang2023discovering,yang2021causalvae,atzmon2020causal, madumal2020explainable,zhang2018fairness}.
% 
% For instance, the DECA method~\cite{atzmon2020causal} uses causal mechanisms to simulate and infer unprecedented combinations of attributes and objects, improving its prediction performance in zero-shot learning tasks. MCR multimodal causal reasoning framework~\cite{zang2023discovering} leverages causal relationships to enhance the understanding of associations between video content and text queries, significantly improving the quality of video question answering. 

% Meanwhile, recent studies have increasingly adopted causal inference techniques to tackle anomaly detection challenges~\cite{carvalho2024invariant,liu2023learning,lin2022causal}. For example, Liu et al.~\cite{liu2023learning} employed causal decomposition to extract pivotal features and identify complex anomalies, while Lin et al.~\cite{lin2022causal} leveraged causal graphs and interventions to analyze and address pseudo-label generation issues, markedly enhancing detection accuracy. Carvalho et al.~\cite{carvalho2024invariant} improved model adaptability to anomaly detection through invariant representation learning. However, research in this domain is still limited, primarily focusing on improving the models' generalization performance and detection capabilities, rather than exploring and quantifying the causal characteristics underlying the anomalies. Our study innovatively applies causal analysis to inductively summarize these anomalous behavioral patterns, offering robust support for developing effective preventative strategies.



% \vspace{-4mm}