\begin{abstract}
Image tokenization has enabled major advances in autoregressive image generation by providing compressed, discrete representations that are more efficient to process than raw pixels. While traditional approaches use 2D grid tokenization, recent methods like TiTok have shown that 1D tokenization can achieve high generation quality by eliminating grid redundancies. However, these methods typically use a fixed number of tokens and thus cannot adapt to an image's inherent complexity. 

We introduce \ours, a tokenizer that projects 2D images into variable-length, ordered 1D token sequences. For example, a $256\times256$ image can be resampled into anywhere from 1 to 256 discrete tokens, hierarchically and semantically compressing its information. By training a rectified flow model as the decoder and using nested dropout, \ours produces plausible reconstructions regardless of the chosen token sequence length.

We evaluate our approach in an autoregressive generation setting using a simple GPT-style Transformer. On ImageNet, this approach achieves an FID~$<$~2 across 8 to 128 tokens, outperforming TiTok and matching state-of-the-art methods with far fewer tokens. We further extend the model to support to text-conditioned image generation and examine how \ours relates to traditional 2D tokenization. A key finding is that \ours enables next-token prediction to describe images in a coarse-to-fine \textit{``visual vocabulary"}, and that the number of tokens to generate depends on the complexity of the generation task.

\end{abstract}
