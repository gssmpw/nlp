\section{Related Works}
\subsection{Psychological Inventories}
Personality inventories are widely used in psychology to understand individuals, which predict distinctive patterns of interpersonal interaction across contexts.
These assessments are often structured, theory-driven, and standardized.
Prominent instruments include the Myers-Briggs Type Indicator (MBTI) \citep{mbti}, NEO-PI-R \citep{costa2008revised}, and Comrey Personality Scales (CPS) \citep{cps}. Among these, the HEXACO model \citep{ashton2009hexaco} is particularly notable, offering a framework that encompasses six factors: \textit{Honesty-Humility, Emotionality, Extraversion, Agreeableness, Conscientiousness, and Openness to Experience}.
\citet{norton1978foundation} introduces the foundational construct of a communicator style. \citet{de2013communication} further explored communication style as a six-dimensional model.
Extensive research \citep{capraro2002myers, costa1992four, lee2004psychometric} has demonstrated the reliability and validity of these inventories.

\subsection{Emotional Support Dialogues}
Early efforts on emotional support dialogues focused on collecting emotional question-answer data from online platforms \citep{medeiros2018using, sharma2020computational, turcan-mckeown-2019-dreaddit, garg-etal-2022-cams}. These datasets laid the groundwork for understanding user emotions, but were limited to single-turn interactions.
Empathetic Dialogue dataset \citep{empatheticdialogue} addressed this by introducing multi-turn dialogues, crowd-sourced to simulate diverse empathetic interactions.
ESConv \citep{esconv} further advanced the field by introducing emotional support strategies collected from psychological theories, enabling chatbots to use these strategies for more empathetic and contextually appropriate responses.
Subsequent studies proposed using hierarchical graph networks to capture global emotions causes and user intentions \citep{peng2022control}, combining multiple emotional support strategies to enhance empathy \citep{tu-etal-2022-misc}, and implementing emotional support strategies and scenarios using LLMs to create the ExTES dataset \citep{zheng2024self}.


\subsection{Persona-Driven Emotional Support}
Recent advances have integrated personas into emotional support dialogues to enhance personalization and diversity. The ESC dataset \citep{zhang2024escot} introduced personas into the dialogue generation process. \citep{zhao2024esc} proposed a framework to extract personas from existing datasets for evaluation. Additionally, personas have been incorporated into chatbots to generate personalized responses \citep{tu2023characterchat, ait2023power, ma2024personality}.
These developments inspire our analysis of the relationship between personas, emotional support strategies, and dialogues, focusing on the extraction and use of personas in ESC.


\begin{figure}
\setlength{\abovecaptionskip}{5pt}   
\setlength{\belowcaptionskip}{0pt}
    \centering
    \includegraphics[width=1\linewidth]{img/personality-card-with-pic.pdf}
    \caption{An example of persona card.}
    \label{fig:card-img}
    \vspace{-0.3cm}
\end{figure}