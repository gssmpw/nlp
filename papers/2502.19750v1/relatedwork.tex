\section{Related Work}
The advances in numerical weather prediction have dominated weather and climate modeling over the last century. They model the complex Earth dynamics as coupled physical systems such as earth system models (ESM)~\citep{hurrell2013community}, integrating the simulations of the atmosphere, cryosphere, land, and ocean processes. With the development of machine learning models~\citep{DBLP:journals/corr/abs-1912-12180,guibas2021efficient} and the accessibility of high-quality weather data, various data-driven models have been proposed to mitigate the weaknesses of NWP such as high computational demands and sensitivity to initial conditions. Early studies target regional forecasting on specific variables, such as precipitation nowcasting in Hong Kong~\citep{shi2015convolutional}, wind prediction in Stuttgart~\citep{harbola2019one}, and air temperature prediction in Australia~\citep{salcedo2016monthly}. A notable progress is the publication of the ECMWF reanalysis v5 (ERA5) dataset~\citep{hersbach2020era5}, which combines historical observations with results from a high-fidelity integrated Forecasting System (IFS)~\citep{wedi2015modelling}. Based on such a dataset, pioneer works~\citep{scher2018toward,weyn2019can,weyn2020improving,rasp2020weatherbench,rasp2021data} study the global forecasting of specific variables such as 500 hPa Geopotential and 300 hPa zonal wind, but at relatively coarser resolutions~\citep{verma2024climode}.

Recently, the development of foundation models has significantly advanced data-driven weather and climate forecasting. They are trained on large-scale high-resolution global data and target various weather variables. FourCastNet~\citep{pathak2022fourcastnet} and a follow-up work SFNO~\citep{bonev2023spherical} are built on the framework of the Fourier Neural Operator~\citep{li2020fourier,guibas2021efficient}. Graph-based models~\citep{lam2022graphcast,keisler2022forecasting} such as GraphCast~\citep{keisler2022forecasting} first create a mesh grid on the spherical surface and perform message passing~\citep{DBLP:conf/iclr/LiuCZXZT0R24,DBLP:conf/kdd/ZhengLL0R24,DBLP:journals/tkde/LiuCHPZT22} on it. Besides these studies, a series of works are based on Transformer~\citep{vaswani2017attention,dosovitskiy2020image}. Pangu-Weather~\citep{bi2023accurate} leverages sliding window attention to model spatial relations. Based on a similar backbone, Fuxi~\citep{chen2023fuxi} and FengWu~\citep{chen2023fengwu} improve training strategies to reduce accumulation errors and incorporate multi-model/task perspectives respectively. Climax~\citep{nguyen2023climax} demonstrates its ability for various weather and climate tasks and CaFA~\citep{li2024cafa} considers the spherical geometry. 

Despite the progress of current foundation models, S2S forecasting receives less attention due to its difficulty. \citeauthor{hwang2019improving}~\citep{hwang2019improving} and \citeauthor{he2022learning}~\citep{he2022learning} study regional S2S forecasting via traditional machine learning models such as AutoKNN and XGBoost~\citep{chen2016xgboost}. \citeauthor{weyn2021sub}~\citep{weyn2021sub} designs an ensemble system based on convolution neural networks to predict six atmospheric variables. It's only been recently that Climax~\citep{nguyen2023climax} and Fuxi-S2S~\citep{chen2024machine} have been developed to try to tackle these issues based on pre-trained foundation models. Therefore, how to build an effective data-driven S2S forecasting model is still an open problem. In this work, we propose CirT and study the performance of the direct prediction model and show that it outperforms current iterative models. Moreover, existing S2S models generally treat global data as planar which introduces geometric inconsistency while we leverage spherical inductive bias in model designs to alleviate such problems. 

Although both GraphCast~\citep{lam2022graphcast} and CirT leverage geometric inductive biases, GraphCast focuses on local state aggregation and relies on message passing~\citep{DBLP:conf/aaai/LiuRGCXTL23} to aggregate local information without explicitly accounting for spatial periodicity. In contrast, CirT employs circular patching to normalize patch geometry and leverages its Fourier representation, consisting of coefficients of periodic basis functions, as inputs to the transformer encoders. Compared with FourcastNet~\citep{pathak2022fourcastnet} which aims to design an efficient token mixer for Vision Transformers that can effectively handle high-resolution inputs, CirT performs multi-head attention in the frequency domain to model the interactions among weather patches across various latitudes. Moreover, FourcastNet employs regular grid patching while CirT introduces circular patching to standardize patch geometry.
Pangu Weather also designs an earth-specific positional bias to integrate spherical information, encoding relative coordinates with learnable parameters into the attention weight computation. Nevertheless, they still employ cube patching and implicitly learn the geometric bias from data, while CirT explicitly leverages spherical bias in the patching strategy.