\documentclass[lettersize,journal]{IEEEtran}
\usepackage{amsmath,amsfonts}
\usepackage{algorithmic}
\usepackage{algorithm}
\usepackage{array}
% \usepackage[caption=false,font=normalsize,labelfont=sf,textfont=sf]{subfig}
\usepackage{textcomp}
\usepackage{stfloats}
\usepackage{url}
\usepackage{verbatim}
\usepackage{graphicx}
\usepackage{cite}

% mycite 
\usepackage{multirow}
\usepackage{bbding}
\usepackage{subfigure}
\usepackage{booktabs}
\usepackage{color} 
\usepackage{siunitx}
\usepackage{colortbl}
% \usepackage{caption}
\definecolor{mygray}{gray}{.9}
\definecolor{mypurple}{RGB}{128,0,128}
\hyphenation{op-tical net-works semi-conduc-tor IEEE-Xplore}
% updated with editorial comments 8/9/2021


\begin{document}

\title{\color{black}{Multi-Dimensional Quality Assessment for Text-to-3D Assets: Dataset and Model}}
% Subjective-aligned database and metric for Text-to-3D content quality assessment.
% Quality Assessment for AI-generated 3D Content Guided by Text
% AI-generated Text-Guided 3D Content Quality Assessment 
% \author{IEEE Publication Technology,~\IEEEmembership{Staff,~IEEE,}
%         % <-this % stops a space
% \thanks{This paper was produced by the IEEE Publication Technology Group. They are in Piscataway, NJ.}% <-this % stops a space
% \thanks{Manuscript received April 19, 2021; revised August 16, 2021.}}

% % The paper headers
% \markboth{Journal of \LaTeX\ Class Files,~Vol.~14, No.~8, August~2021}%
% {Shell \MakeLowercase{\textit{et al.}}: A Sample Article Using IEEEtran.cls for IEEE Journals}

% \IEEEpubid{0000--0000/00\$00.00~\copyright~2021 IEEE}
% % Remember, if you use this you must call \IEEEpubidadjcol in the second
% % column for its text to clear the IEEEpubid mark.

\author{
Kang Fu*, Huiyu Duan*, Zicheng Zhang, Xiaohong Liu, \emph{Member, IEEE,}\\  Xiongkuo Min$^{\dagger}$, \emph{Member, IEEE,} Jia Wang, and Guangtao Zhai$^{\dagger}$, \emph{Fellow, IEEE
} % <-this % stops a space
% \author{
% Kang Fu, Guangtao Zhai$^\dag$, \emph{Senior Member, IEEE
% } % <-this % stops a space

% \IEEEcompsocitemizethanks{\IEEEcompsocthanksitem This work was supported in part by NSFC (No.62225112, No.61831015, No. 62301316), the Fundamental Research Funds for the Central Universities, National Key R\&D Program of China 2021YFE0206700, Shanghai Municipal Science and Technology Major Project (2021SHZDZX0102), STCSM 22DZ2229005, and the China Postdoctoral Science Foundation under Grant 2023TQ0212 and 2023M742298. \textit{(Corresponding Author: Guangtao Zhai.)} \protect}

\IEEEcompsocitemizethanks{\IEEEcompsocthanksitem Kang Fu, Huiyu Duan, Zicheng Zhang, Xiaohong Liu, Xiongkuo Min, Jia Wang and Guangtao Zhai are with Shanghai Jiao Tong University, 200240 Shanghai, China. E-mail:\{fuk20\-20, huiyuduan, zzc1998, xiaohongliu, minxiongkuo, jiawang, zhaiguangtao\}@sjtu.edu.cn. 
This work was supported in part by the National Key R\&D Program of China under Grant 2021YFE0206700, in part by the National Natural Science Foundation of China under Grants 62401365, 62271312, 62225112, 62132006, and in part by the Shanghai Pujiang Program under Grant 22PJ1407400. * Equal Contributions. $\dagger$ Corresponding Authors.\protect}
        % <-this % stops a space
}% <-this % stops a space

% \author{
% Kang Fu, Guangtao Zhai$^\dag$, \emph{Senior Member, IEEE
% } % <-this % stops a space
% }% <-this % stops a space

% \thanks{Manuscript received April 19, 2021; revised August 16, 2021.}}

\maketitle

\begin{abstract}

Recent advancements in text-to-image (T2I) generation have spurred the development of text-to-3D asset (T23DA) generation, leveraging pretrained 2D text-to-image diffusion models for text-to-3D asset synthesis. Despite the growing popularity of text-to-3D asset generation, its evaluation has not been well considered and studied. However, given the significant quality discrepancies among various text-to-3D assets, there is a pressing need for quality assessment models aligned with human subjective judgments. To tackle this challenge, we conduct a comprehensive study to explore the T23DA quality assessment (T23DAQA) problem in this work from both subjective and objective perspectives. Given the absence of corresponding databases, we first establish the largest text-to-3D asset quality assessment database to date, termed the AIGC-T23DAQA database. This database encompasses 969 validated 3D assets generated from 170 prompts via 6 popular text-to-3D asset generation models, and corresponding subjective quality ratings for these assets from the perspectives of quality, authenticity, and text-asset correspondence, respectively. Subsequently, we establish a comprehensive benchmark based on the AIGC-T23DAQA database, and devise an effective T23DAQA model to evaluate the generated 3D assets from the aforementioned three perspectives, respectively. Specifically, the proposed method utilizes the projection videos of text-to-3D assets to extract 3D shape, texture and text-asset correspondence features, then fuses them to calculate the final three preference scores respectively. Extensive experimental results demonstrate the effectiveness of the proposed T23DAQA method in evaluating the quality of AI generated 3D asset, which is more consistent with human perception. To the best of our knowledge, this is the first work that studies the problem of text-guided 3D generation quality assessment, and  The database is released at \url{https://github.com/ZedFu/T23DAQA}.

% The dataset is available at https:
\end{abstract}

\begin{IEEEkeywords}
text-to-3D asset generation, subjective quality assessment, objective quality assessment, artificial intelligence generated content (AIGC)
\end{IEEEkeywords}

\section{Introduction}

Large language models (LLMs) have achieved remarkable success in automated math problem solving, particularly through code-generation capabilities integrated with proof assistants~\citep{lean,isabelle,POT,autoformalization,MATH}. Although LLMs excel at generating solution steps and correct answers in algebra and calculus~\citep{math_solving}, their unimodal nature limits performance in plane geometry, where solution depends on both diagram and text~\citep{math_solving}. 

Specialized vision-language models (VLMs) have accordingly been developed for plane geometry problem solving (PGPS)~\citep{geoqa,unigeo,intergps,pgps,GOLD,LANS,geox}. Yet, it remains unclear whether these models genuinely leverage diagrams or rely almost exclusively on textual features. This ambiguity arises because existing PGPS datasets typically embed sufficient geometric details within problem statements, potentially making the vision encoder unnecessary~\citep{GOLD}. \cref{fig:pgps_examples} illustrates example questions from GeoQA and PGPS9K, where solutions can be derived without referencing the diagrams.

\begin{figure}
    \centering
    \begin{subfigure}[t]{.49\linewidth}
        \centering
        \includegraphics[width=\linewidth]{latex/figures/images/geoqa_example.pdf}
        \caption{GeoQA}
        \label{fig:geoqa_example}
    \end{subfigure}
    \begin{subfigure}[t]{.48\linewidth}
        \centering
        \includegraphics[width=\linewidth]{latex/figures/images/pgps_example.pdf}
        \caption{PGPS9K}
        \label{fig:pgps9k_example}
    \end{subfigure}
    \caption{
    Examples of diagram-caption pairs and their solution steps written in formal languages from GeoQA and PGPS9k datasets. In the problem description, the visual geometric premises and numerical variables are highlighted in green and red, respectively. A significant difference in the style of the diagram and formal language can be observable. %, along with the differences in formal languages supported by the corresponding datasets.
    \label{fig:pgps_examples}
    }
\end{figure}



We propose a new benchmark created via a synthetic data engine, which systematically evaluates the ability of VLM vision encoders to recognize geometric premises. Our empirical findings reveal that previously suggested self-supervised learning (SSL) approaches, e.g., vector quantized variataional auto-encoder (VQ-VAE)~\citep{unimath} and masked auto-encoder (MAE)~\citep{scagps,geox}, and widely adopted encoders, e.g., OpenCLIP~\citep{clip} and DinoV2~\citep{dinov2}, struggle to detect geometric features such as perpendicularity and degrees. 

To this end, we propose \geoclip{}, a model pre-trained on a large corpus of synthetic diagram–caption pairs. By varying diagram styles (e.g., color, font size, resolution, line width), \geoclip{} learns robust geometric representations and outperforms prior SSL-based methods on our benchmark. Building on \geoclip{}, we introduce a few-shot domain adaptation technique that efficiently transfers the recognition ability to real-world diagrams. We further combine this domain-adapted GeoCLIP with an LLM, forming a domain-agnostic VLM for solving PGPS tasks in MathVerse~\citep{mathverse}. 
%To accommodate diverse diagram styles and solution formats, we unify the solution program languages across multiple PGPS datasets, ensuring comprehensive evaluation. 

In our experiments on MathVerse~\citep{mathverse}, which encompasses diverse plane geometry tasks and diagram styles, our VLM with a domain-adapted \geoclip{} consistently outperforms both task-specific PGPS models and generalist VLMs. 
% In particular, it achieves higher accuracy on tasks requiring geometric-feature recognition, even when critical numerical measurements are moved from text to diagrams. 
Ablation studies confirm the effectiveness of our domain adaptation strategy, showing improvements in optical character recognition (OCR)-based tasks and robust diagram embeddings across different styles. 
% By unifying the solution program languages of existing datasets and incorporating OCR capability, we enable a single VLM, named \geovlm{}, to handle a broad class of plane geometry problems.

% Contributions
We summarize the contributions as follows:
We propose a novel benchmark for systematically assessing how well vision encoders recognize geometric premises in plane geometry diagrams~(\cref{sec:visual_feature}); We introduce \geoclip{}, a vision encoder capable of accurately detecting visual geometric premises~(\cref{sec:geoclip}), and a few-shot domain adaptation technique that efficiently transfers this capability across different diagram styles (\cref{sec:domain_adaptation});
We show that our VLM, incorporating domain-adapted GeoCLIP, surpasses existing specialized PGPS VLMs and generalist VLMs on the MathVerse benchmark~(\cref{sec:experiments}) and effectively interprets diverse diagram styles~(\cref{sec:abl}).

\iffalse
\begin{itemize}
    \item We propose a novel benchmark for systematically assessing how well vision encoders recognize geometric premises, e.g., perpendicularity and angle measures, in plane geometry diagrams.
	\item We introduce \geoclip{}, a vision encoder capable of accurately detecting visual geometric premises, and a few-shot domain adaptation technique that efficiently transfers this capability across different diagram styles.
	\item We show that our final VLM, incorporating GeoCLIP-DA, effectively interprets diverse diagram styles and achieves state-of-the-art performance on the MathVerse benchmark, surpassing existing specialized PGPS models and generalist VLM models.
\end{itemize}
\fi

\iffalse

Large language models (LLMs) have made significant strides in automated math word problem solving. In particular, their code-generation capabilities combined with proof assistants~\citep{lean,isabelle} help minimize computational errors~\citep{POT}, improve solution precision~\citep{autoformalization}, and offer rigorous feedback and evaluation~\citep{MATH}. Although LLMs excel in generating solution steps and correct answers for algebra and calculus~\citep{math_solving}, their uni-modal nature limits performance in domains like plane geometry, where both diagrams and text are vital.

Plane geometry problem solving (PGPS) tasks typically include diagrams and textual descriptions, requiring solvers to interpret premises from both sources. To facilitate automated solutions for these problems, several studies have introduced formal languages tailored for plane geometry to represent solution steps as a program with training datasets composed of diagrams, textual descriptions, and solution programs~\citep{geoqa,unigeo,intergps,pgps}. Building on these datasets, a number of PGPS specialized vision-language models (VLMs) have been developed so far~\citep{GOLD, LANS, geox}.

Most existing VLMs, however, fail to use diagrams when solving geometry problems. Well-known PGPS datasets such as GeoQA~\citep{geoqa}, UniGeo~\citep{unigeo}, and PGPS9K~\citep{pgps}, can be solved without accessing diagrams, as their problem descriptions often contain all geometric information. \cref{fig:pgps_examples} shows an example from GeoQA and PGPS9K datasets, where one can deduce the solution steps without knowing the diagrams. 
As a result, models trained on these datasets rely almost exclusively on textual information, leaving the vision encoder under-utilized~\citep{GOLD}. 
Consequently, the VLMs trained on these datasets cannot solve the plane geometry problem when necessary geometric properties or relations are excluded from the problem statement.

Some studies seek to enhance the recognition of geometric premises from a diagram by directly predicting the premises from the diagram~\citep{GOLD, intergps} or as an auxiliary task for vision encoders~\citep{geoqa,geoqa-plus}. However, these approaches remain highly domain-specific because the labels for training are difficult to obtain, thus limiting generalization across different domains. While self-supervised learning (SSL) methods that depend exclusively on geometric diagrams, e.g., vector quantized variational auto-encoder (VQ-VAE)~\citep{unimath} and masked auto-encoder (MAE)~\citep{scagps,geox}, have also been explored, the effectiveness of the SSL approaches on recognizing geometric features has not been thoroughly investigated.

We introduce a benchmark constructed with a synthetic data engine to evaluate the effectiveness of SSL approaches in recognizing geometric premises from diagrams. Our empirical results with the proposed benchmark show that the vision encoders trained with SSL methods fail to capture visual \geofeat{}s such as perpendicularity between two lines and angle measure.
Furthermore, we find that the pre-trained vision encoders often used in general-purpose VLMs, e.g., OpenCLIP~\citep{clip} and DinoV2~\citep{dinov2}, fail to recognize geometric premises from diagrams.

To improve the vision encoder for PGPS, we propose \geoclip{}, a model trained with a massive amount of diagram-caption pairs.
Since the amount of diagram-caption pairs in existing benchmarks is often limited, we develop a plane diagram generator that can randomly sample plane geometry problems with the help of existing proof assistant~\citep{alphageometry}.
To make \geoclip{} robust against different styles, we vary the visual properties of diagrams, such as color, font size, resolution, and line width.
We show that \geoclip{} performs better than the other SSL approaches and commonly used vision encoders on the newly proposed benchmark.

Another major challenge in PGPS is developing a domain-agnostic VLM capable of handling multiple PGPS benchmarks. As shown in \cref{fig:pgps_examples}, the main difficulties arise from variations in diagram styles. 
To address the issue, we propose a few-shot domain adaptation technique for \geoclip{} which transfers its visual \geofeat{} perception from the synthetic diagrams to the real-world diagrams efficiently. 

We study the efficacy of the domain adapted \geoclip{} on PGPS when equipped with the language model. To be specific, we compare the VLM with the previous PGPS models on MathVerse~\citep{mathverse}, which is designed to evaluate both the PGPS and visual \geofeat{} perception performance on various domains.
While previous PGPS models are inapplicable to certain types of MathVerse problems, we modify the prediction target and unify the solution program languages of the existing PGPS training data to make our VLM applicable to all types of MathVerse problems.
Results on MathVerse demonstrate that our VLM more effectively integrates diagrammatic information and remains robust under conditions of various diagram styles.

\begin{itemize}
    \item We propose a benchmark to measure the visual \geofeat{} recognition performance of different vision encoders.
    % \item \sh{We introduce geometric CLIP (\geoclip{} and train the VLM equipped with \geoclip{} to predict both solution steps and the numerical measurements of the problem.}
    \item We introduce \geoclip{}, a vision encoder which can accurately recognize visual \geofeat{}s and a few-shot domain adaptation technique which can transfer such ability to different domains efficiently. 
    % \item \sh{We develop our final PGPS model, \geovlm{}, by adapting \geoclip{} to different domains and training with unified languages of solution program data.}
    % We develop a domain-agnostic VLM, namely \geovlm{}, by applying a simple yet effective domain adaptation method to \geoclip{} and training on the refined training data.
    \item We demonstrate our VLM equipped with GeoCLIP-DA effectively interprets diverse diagram styles, achieving superior performance on MathVerse compared to the existing PGPS models.
\end{itemize}

\fi 


Here, we discuss significant related works and potential countermeasures relevant to our Thor attack.

\paragrabf{Hertzbleed \cite{wang2022hertzbleed}.}
Hertzbleed leverages dynamic voltage and frequency scaling (DVFS) to transform power side-channel attacks into timing attacks. By exploiting the timing differences caused by frequency variations, even remote attacks become feasible. For instance, in an attack against Supersingular Isogeny Key Encapsulation (SIKE), they managed to recover 378 bits of the private key within 36 hours. Although this attack shares similarities with our work in exploiting frequency changes, DVFS can be managed by the CPU core and disabled in BIOS settings to mitigate the Hertzbleed attack. However, in our case, disabling DVFS is not a viable countermeasure since AMX, an on-chip accelerator, manages its power and frequency independently.

\paragrabf{Collide+Power \cite{kogler2023collide+}.}
The Collide+Power research focuses on the power leakage of the memory hierarchy via Running Average Power Limit (RAPL). If RAPL is unavailable, monitoring can be done through a throttling side-channel, albeit requiring more measurements. They demonstrated two types of attacks: Meltdown-style and Microarchitectural Data Sampling (MDS)-style. In Meltdown-style, targeting a shared cache among two processes on different cores, theoretically, one bit can be leaked in 99.95 days with power limit control or 2.86 years with stress-induced throttling. In MDS-style, where both victim and attacker run on different logical cores of the same physical core, data can be leaked from the L1/L2 cache at a rate of 4.82 bits per hour. Disabling simultaneous multithreading can mitigate MDS-style attacks. Generally, RAPL being a privileged interface is not accessible to unprivileged attackers, and throttling can be disabled by turning off DVFS.

\paragrabf{Platypus \cite{Lipp2021Platypus}.}
The Platypus attack reconstructed 509 RSA key bits using RAPL MSRs within Intel SGX enclaves. However, this attack vector has been mitigated by making RAPL a privileged interface.

\paragrabf{Neural Network Specific Attacks.}
Several studies have attacked neural network accelerators using power side channels. In the work by Wei et al. \cite{wei2018know}, an FPGA-based convolutional neural network accelerator was attacked, requiring physical access to recover the model's input image with up to 89\% accuracy. Effective mitigations include masking and random scheduling, although masking introduces significant overheads, as demonstrated in MaskedNet \cite{dubey2020maskednet}, increasing latency and area costs by 2.8x and 2.3x, respectively.

Open DNN Box \cite{Xiang2019OpenDB} inferred the weight sparsity of neural network models with 96.5\% accuracy on average. CSI NN \cite{236204} used power and electromagnetic traces to infer information about weights and architecture in fully connected neural networks. DeepEM \cite{Yu2020DeepEMDN} and DeepSniffer \cite{Hu2020DeepSnifferAD} collected electromagnetic traces to glean architectural information, with DeepEM specifically targeting binarized neural networks. Cache Telepathy \cite{244042}, GANRED \cite{Liu2020GANREDGR}, and DeepRecon \cite{Hong2018SecurityAO} employed well-known cache side channels like Flush+Reload and Prime+Probe to gather neural network insights. For these cache attacks, the attacker runs locally, and the presence of a shared cache is necessary. In contrast, our Thor attack doesn't need physical access or shared cache. It introduces a novel, data-dependent timing side-channel vulnerability specific to Intel AMX accelerators.

In the work by Gongye et al. \cite{9218707}, they attacked DNNs using a floating-point timing side channel to obtain weights and biases. They took advantage of the drastically different execution times for floating-point multiplication and addition in certain scenarios, such as when dealing with subnormal values, to launch their attack. However, with modern accelerators like Intel AMX, this floating-point timing vulnerability has been eliminated. Now, the execution time for tile multiplication remains constant, even for special cases like zero inputs. Specifically, the latency is fixed at 52 cycles and the throughput is 16. Despite this, to attack DNNs with more than one layer, cache monitoring or physical access is still required to measure the execution time of each layer.

% \paragrabf{Potential Countermeasures.}
% Eliminating the cooldown state could defend against Thor but at a high power cost since Intel AMX is an energy-intensive accelerator designed for AI tasks. Keeping AMX continuously active would be power-prohibitive.

% Masking is a proven countermeasure for protecting AI model parameters against power side-channel attacks and could be adapted for future AMX versions despite the performance overhead. Additionally, machine learning models should incorporate techniques to detect unusual usage patterns, which can help identify and thwart attacks attempting to infer parameter values using methods similar to our AMX-type attack. One well-known countermeasure to these timing attacks is to coarsen the timer. By reducing the timer's precision, it becomes much harder for attackers to measure the subtle differences in execution times that they rely on for their exploits.

% In summary, while various countermeasures exist for different attack vectors, protecting against Thor on Intel AMX accelerators requires novel approaches to power and frequency management alongside traditional techniques.




\section{Database Construction and Analysis}
In this section, we will describe the database construction and analysis in detail.
\begin{figure}[b]
    \centering
    \includegraphics[width = 0.48\textwidth]{Figs/pie.pdf}
    \caption{The Pie Chart of our used Prompt, which contains 11 challenge categories and 12 scene categories.}
    \label{fig:pie}
\end{figure}
\begin{figure*}[t]
    \centering
    \subfigure[3D assets generated by the prompt: ``a harp without any strings'']{\begin{minipage}[t]{\linewidth}
                \centering
                \includegraphics[width = 0.98\linewidth]{Figs/gallery-004.pdf}
                \end{minipage}}
                
    \subfigure[3D assets generated by the prompt: ``a pair of brown suede shoes'']{\begin{minipage}[t]{\linewidth}
                \centering
                \includegraphics[width = 0.98\linewidth]{Figs/gallery-012.pdf}
                \end{minipage}}
                
    \caption{Sample 3D assets from the AIGC-T23DAQA database, generated by Dreamfusion \cite{poole2022dreamfusion}, LatentNerf \cite{metzer2023latent}; Magic3D \cite{lin2023magic3d}, Prolificdreamer \cite{wang2024prolificdreamer}; SJC\cite{wang2023score}, TextMesh \cite{tsalicoglou2023textmesh} with the same input prompt respectively. (a) 3D assets generated by the prompt ``a harp without any strings''. (b) 3D assets generated by the prompt ``a pair of brown suede shoes''. This clearly shows that the visual quality of assets generated by different models varies greatly.}
    \label{fig:gallery}
    \vspace{-0.5cm}
\end{figure*}
\vspace{-15pt}
\subsection{Prompt Selection}
Compared to AIGC IQA and VQA databases, constructing text-to-3D asset quality assessment database mainly faces two difficulties: 1) The process of generating 3D asset from text is currently time-consuming, typically requiring 1 to 6 hours to generate one 3D asset. 2) The subjective experiment for evaluating generated 3D asset is also time-consuming, since subjects need to observe from whole directions and assess from multiple perspectives. Therefore, our constructed database is a enormous contribution to the field. First of all, meticulous prompts selection is important for text-to-3D asset quality assessment database construction. The selected prompts need to cover a wide range of real user inputs with a relatively small pool. PartiPrompts \cite{yu2022scaling} comprises 1600 varied English prompts designed to comprehensively assess and test the limits of text-to-image synthesis models. Following previous research \cite{wang2023aigciqa2023} we extracted 170 prompts from PartiPrompts, spanning 11 challenge categories and 12 scene categories. The distribution of selected scene and challenge categories is depicted in the pie chart of Fig. \ref{fig:pie}, which manifests that the prompts in our dataset exhibit a high level of scene diversity and encompass a broad spectrum of challenges.
\vspace{-15pt}
\subsection{3D Asset Generation}

To ensure asset diversity, AIGC-T23DAQA database contain six representative text-to-3D asset generation models. These current models typically comprise a 2D image generation module and a 3D asset representation module. When compared to other generation models, the diffusion model delivers exceptional results, establishing itself as the preferred foundational module for generating 2D images within these methodologies. For the 3D asset representation module, a variety of approaches are employed, including NeRF, Instatn-ngp, \textit{etc}. Dreamfusion \cite{poole2022dreamfusion} utilizes mip-NeRF 360 for 3D asset representation, while LatentNerf \cite{metzer2023latent} opts for vanilla NeRF. SJC \cite{wang2023score} employs voxel radiance fields to represent 3D asset, thereby enhancing the speed of the generation process. Conversely, TextMesh  \cite{tsalicoglou2023textmesh}, Magic3D \cite{lin2023magic3d}, and Prolificdreamer \cite{wang2024prolificdreamer} adopt a coarse-to-fine strategy. They commence with coarse 3D asset representations, using vanilla NeRF and Instatn-ngp, respectively, and subsequently refine the differentiable mesh into a fine representation. 
\begin{figure}[!t]
    \centering
    \subfigure[Prompt: ``an ostrich''.]{\begin{minipage}[t]{0.48\linewidth}
                \centering
                \includegraphics[width = 0.94\linewidth]{Figs/q0.pdf}
                \end{minipage}}
    \subfigure[Prompt: ``a comic about a boy and a tiger''.]{\begin{minipage}[t]{0.48\linewidth}
                \centering
                \includegraphics[width = 0.94\linewidth]{Figs/q1.pdf}
                \end{minipage}}
                
    \subfigure[Prompt: ``a fish without eyes''.]{\begin{minipage}[t]{0.48\linewidth}
                \centering
                \includegraphics[width = 0.94\linewidth]{Figs/a0.pdf}
                \end{minipage}}
    \subfigure[Prompt: ``a large present with a red ribbon to the left of a Christmas tree''.]{\begin{minipage}[t]{0.48\linewidth}
                \centering
                \includegraphics[width = 0.94\linewidth]{Figs/a1.pdf}
                \end{minipage}}
                
    \subfigure[Prompt: ``a robot cooking''.]{\begin{minipage}[t]{0.48\linewidth}
                \centering
                \includegraphics[width = 0.94\linewidth]{Figs/c0.pdf}
                \end{minipage}}
    \subfigure[Prompt: ``a bundle of blue and yellow flowers in a vase''.]{\begin{minipage}[t]{0.48\linewidth}
                \centering
                \includegraphics[width = 0.94\linewidth]{Figs/c1.pdf}
                \end{minipage}}
    \caption{Illustration of the differences between the three dimensions of quality ,authenticity, and text-3D correspondence. In each subfigure, the images in the top row are significantly better than the that in bottom row in terms of two perspectives, while similar or worse in terms of another perspective. (a) and (b) show examples that the authenticity and correspondence scores of the top images are higher, while the quality is similar. (c) and (d) show examples that the quality and correspondence scores of the top images are higher, while the authenticity is similar or lower. (e) and (f) show examples that the quality and authenticity scores of the top images are higher, while the correspondence is similar or lower. }
    \label{fig:xx}
    \vspace{-0.5cm}
\end{figure}
The generation process of text-to-3D asset was executed using open-source code \cite{threestudio2023} with default weights and configurations, resulting in a collection of 1020 instances (170 prompts × 6 models) of text-to-3D assets. Some examples of the 3D assets generated by the six text-to-3D asset generation models are illustrated in Fig.  \ref{fig:gallery}. Subsequently, we discarded 51 instances of failed asset generation, defined as cases where the entire spatial domain remained empty after-generation. Due to computational constraints, it is hard to render a generated 3D asset in real-time and evaluate it. Thus, we followed the method used in \cite{zhang2023eep} and projected the 3D asset into videos then conducted evaluation. This manipulation yielded 969 360-degree surround projection videos centered on the generated text-to-3D asset. Each video consists of comprised 120 frames with a resolution of 512 × 512 pixels and cumulative a total duration of 4 seconds. These projection videos were used for the subsequent subjective experiment.

\vspace{-10pt}
\subsection{Subjective Experiment}
\begin{figure}[!t]
    \centering
    \subfigure[Quality]{\begin{minipage}[t]{\linewidth}
                \centering
                \includegraphics[width = 0.98\linewidth]{Figs/assessment-perspectives-q.pdf}
                \end{minipage}}
                
    \subfigure[Authenticity]{\begin{minipage}[t]{\linewidth}
                \centering
                \includegraphics[width = 0.98\linewidth]{Figs/assessment-perspectives-a.pdf}
                \end{minipage}}

    \subfigure[Correspondence]{\begin{minipage}[t]{\linewidth}
                \centering
                \includegraphics[width = 0.98\linewidth]{Figs/assessment-perspectives-c.pdf}
                \end{minipage}}
    \caption{Illustration of the text-to-3d assets from the perspectives of quality, authenticity, and text-asset correspondence. The examples of good, fair, and poor quality are depicted in the first to third rows of (a). The examples illustrating good, fair, and poor authenticity are displayed in the first to third rows of (b). (c) showcases examples of good, fair, and poor correspondence generated by prompts ``a harp without any strings'', ``a knight holding a long sword'', and ``A cartoon tiger face''. }
    \label{fig:ap}
    \vspace{-0.4cm}
\end{figure}
To collect human visual preferences for text-to-3D assets, we further conducted a subjective evaluation experiment. As highlighted in prior AI generated asset quality assessment studies \cite{wang2023aigciqa2023, yang2024aigcoiqa2024}, the degradations of AI generated asset are significantly different from human captured or created asset, which need to be evaluated from multiple perception perspectives. \textcolor{black}{Based on traditional 3D quality assessment, which evaluates texture, color, and other visual quality attributes of the 3D asset, we selected the ``quality'' dimension for evaluation. Similar to AI-generated image and video quality assessment, in addition to assessing the visual quality of the 3D asset, we also need to evaluate its authenticity and correspondence to the text prompt. Therefore, we selected the dimensions of ``authenticity'' and ``correspondence''.} Hence, in this paper, we propose to evaluate human visual preferences for text-to-3D assets from three perspectives, including quality, authenticity, and text-asset correspondence. Fig. \ref{fig:xx} shows the differences between the selected three dimensions, which further manifests the importance, and significance of evaluating text-to-3D assets from multiple perspectives.
\textcolor{black}{Before each subject conducts the subjective experiment, we give a detailed instruction to subjects, which includes explaining to the subject the differences between ``quality'', ``authenticity'' and ``correspondence'' and showing examples of different degrees of each dimension. The ``quality" is the visual quality attribution of 3D asset including texture, color, integrity, etc, while the ``authenticity" refers to whether the 3D asset is consistent with the real world that the subject knows. The ``correspondence" is the alignment between the 3D asset and the input prompt text.} Then, participants were instructed to give their preference scores of text-to-3D assets based on the surrounding 360-degree projection videos. The first dimension for evaluating text-to-3D asset is ``quality'', which mainly evaluates the perception attributes including texture, color, integrity, details \textit{etc.}, analogous to traditional 3D models. Fig. \ref{fig:ap} (a) shows examples of the generated 3D asset with different ``quality'' levels. The second dimension for evaluating text-to-3D asset is ``authenticity'', which evaluates the perception attributes including unrealistic textures, shapes, \textit{etc.} It should be noted that compared to the authenticity attribute generally used in AIGC IQA, the degradation of the authenticity attribute for generated text-to-3D asset generally comes from the unrealistic or inconsistent multiple views. Fig. \ref{fig:ap} (b) shows examples of the generated 3D asset with different ``authenticity'' levels. Similar to AIGC IQA, and AIGC VQA methodologies, the correspondence between text, and 3D asset serves as another critical criterion in assessing text-to-3D asset quality, referred to as ``text-3D asset correspondence''. Fig. \ref{fig:ap} (c) shows examples of the generated 3D asset with different ``correspondence'' levels.
\begin{figure}[t]
    \centering
    \includegraphics[width = 0.48\textwidth]{Figs/gui.png}
    \caption{The illustration of the subjective assessment interface. The subject can evaluate their preferences of the text-to-3D assets, and record the quality, authenticity, correspondence scores with the scroll bars on the right.}
    \label{fig:gui}
    \vspace{-0.5cm}
\end{figure}

\begin{figure*}[t]
    \centering
    \subfigure[Quality MOS distribution.]{\begin{minipage}[t]{0.32\linewidth}
                \centering
                \includegraphics[width = 0.94\linewidth]{Figs/mosz_Quality.pdf}
                \end{minipage}}
    \subfigure[Authenticity MOS distribution.]{\begin{minipage}[t]{0.32\linewidth}
                \centering
                \includegraphics[width = 0.94\linewidth]{Figs/mosz_Authenticity.pdf}
                \end{minipage}}
    \subfigure[Correspondence MOS distribution.]{\begin{minipage}[t]{0.32\linewidth}
                \centering
                \includegraphics[width = 0.94\linewidth]{Figs/mosz_Correspondence.pdf}
                \end{minipage}}
    \caption{Distributions of the MOSs from the perspectives of quality, authenticity, and correspondence, respectively. These distributions exhibiting proposed T23DAQA database cover a wide range in terms of all perspectives. }
    \label{fig:mos}
    \vspace{-0.5cm}
\end{figure*}

We conducted the subjective experiment following the guidance in ITU-R BT.500-13 \cite{other:itu}. The experimental environment was arranged to simulate a typical indoor home setting with standard lighting conditions. The projection videos of text-to-3D asset, accompanied by the corresponding prompts, were presented randomly on a monitor with a resolution of $1920 \times 1080$. The interface, depicted in Fig. \ref{fig:gui}, facilitated viewer interaction, enabling navigation through previous, next, and replay options for the projection videos of the generated 3D asset. Additionally, three sliders ranging from 0 to 5, with a minimum interval of 0.1, were provided for participants to assign scores for quality, authenticity, and correspondence. 17 subjects (10 males and 7 females) participated in the subjective experiment, all possessing normal or corrected-to-normal vision. Each participant received detailed experimental instructions prior to engaging in the subjective evaluation. We divided the conversation of each participant in the subjective experiment into three subsets. For each participant, the database were randomly divided into three subsets, which are used in three subjective tests respectively. Each test lasted around one hour, followed by a 10-20 minutes break in between, and then the next test was performed.
\vspace{-15pt}
\subsection{Data Processing}

We followed the instructions of ITU \cite{other:itu} to conduct the outlier detection and subject rejection. \textcolor{black}{Specifically, for each evaluation dimension, we calculate the kurtosis of the raw subjective quality ratings for each generated 3D asset to determine whether the data follows a Gaussian or non-Gaussian distribution. For Gaussian distributions, a raw score is considered an outlier if it lies more than 2 standard deviations (std) from the mean. For non-Gaussian distributions, a score is deemed as an outlier if it is more than $\sqrt{20}$ standard deviations from the mean. Any subject whose evaluations exceed a 3\% outlier rate in any dimension is excluded from the analysis.} As a result, no subjects were rejected and the rejection ratio is 3\% for all ratings. Subsequently, we converted the raw ratings of the remaining valid subjective scores into Z-scores, which were then linearly scaled to the range of $\left [ 0,100 \right ]$. The final MOS is computed as follows:
\begin{equation}
z_{ij} = \frac{m_{ij}-\mu_{i}}{\sigma_{i}}, \quad z_{ij}^{'} = \frac{100\times(z_{ij} + 3)}{6}
\end{equation}

\begin{equation}
\text{MOS}_{j} = \frac{1}{N} z_{ij}^{'}
\end{equation}
where $m_{ij}$ is the subjective score given by the $i$-th subject to the $j$-th text-to-3D asset, $\mu_{i}$ and $\sigma_{i}$ is the mean score and the standard deviation given by the $i$-th subject respectively, $N$ is the total number of subjects. 
% \begin{figure}[t]
%     \centering
%     \includegraphics[width = 0.48\textwidth]{Figs/assessment-perspectives .pdf}
%     \caption{Illustration of the T23DCs from the perspectives of quality, authenticity, and text-content correspondence. The examples of good, fair, and poor quality are depicted in the first to third rows of (a). The examples illustrating good, fair, and poor authenticity are displayed in the first to third rows of (b). (c) showcases examples of good, fair, and poor correspondence generated by prompts such as "a harp without any strings", "a knight holding a long sword", and "A cartoon tiger face".}
%     \label{fig:ap}
% \end{figure}
\vspace{-15pt}
\subsection{Subjective Data Analysis}
Although a large number of text-to-3D asset generation models have been developed in recent years, the corresponding works that specifically analyze and compare their generation performance are lacking. Considering that the generation quality of the text-to-3D asset is influenced by multiple factors such as prompts, algorithms, \textit{etc}, which leads to diverse perceptual quality and affects the user experience, based on the established AIGC-T23DAQA database, we conduct an in-depth analysis for the collected MOSs from multiple perspectives as follows.

Fig. \ref{fig:mos} demonstrates the distribution of MOS values obtained from subjective experiments. It can be observed that the correspondence distribution surpasses both the quality and authenticity distributions, suggesting that the current generation models learn more towards correspondence while ignoring the quality and authenticity attributes. The reason for this phenomenon is that the current T23DA method utilizes text-to-image models to constrain the correspondence between images rendered from different perspectives and text. These text-to-image models are trained on a large number of text-image pairs and perform well in text-image correspondence, ensuring good correspondence between generated 3D asset and text; However, the text-generated image model cannot guarantee the geometric texture consistency of three-dimensional objects from different perspectives, resulting in the strange geometric shapes and floaters in generated 3D asset. As a result, the quality and authenticity of the generated 3D assets are poorer than those of correspondence. To enhance the overall user preferences in the future, it is more important to improve the quality and authenticity attributes for the generated 3D assets.

Fig. \ref{fig:moscompare} (a) compares the human preference MOSs for different models, including Dreamfusion \cite{poole2022dreamfusion}, LatentNerf \cite{metzer2023latent}; Magic3D \cite{lin2023magic3d}, Prolificdreamer \cite{wang2024prolificdreamer}; SJC\cite{wang2023score}, TextMesh \cite{tsalicoglou2023textmesh}. Fig. \ref{fig:moscompare} (b) compares the human preference MOSs for different prompt length. Prompt length is divided into six intervals on average, with 1-6 on the x-axis representing interval numbers from short to long. We can find from it that: 1) The 3D assets generated by different text-to-3D generation models have significantly different perceptual preferences, and even with the same input prompt, the quality, authenticity, and correspondence vary greatly across different text-to-3D asset methods. Models including Prolificdreamer \cite{wang2024prolificdreamer}, Magic3D \cite{lin2023magic3d}, and Prolificdreamer \cite{wang2024prolificdreamer} exhibit the best quality, authenticity, and correspondence respectively. The reasons for the subjective score differences among different models: From Figure 9 in the manuscript, it can be seen that the best quality, authenticity, and correspondence are Prolificdreamer, Magic3D, and Prolificdreamer respectively. Prolificdreamer uses variational score distillation to instead of score distillation sampling which used in other methods and solve the problems of over-saturation, over-smoothing, and low-diversity. So the Prolificdreamer has better quality and correspondence. Magic3D uses coarse-to-fine strategy to generate 3D asset and a sparse 3D hash grid structure to represent 3D asset, which can reduce the generation of floaters, making generated 3D asset more authenticity. 2) When the prompt is short (1 \& 2), the model is easy to generate high quality, authenticity, and correspondence 3D assets, However, as prompt length increases (3, 4 \& 5), text-to-3D generation models may struggle to meet the requirements of human preferences and the entire prompt, resulting in a decline in the quality, authenticity, and correspondence scores. Finally, when the prompt length is extreme long, the explicit descriptions make the quality, authenticity scores higher, while the correspondence scores are still lower than the prompt length of 1 \& 2. The reasons for subjective score differences in different prompt lengths: When the prompt length is short, the generated 3D asset is less constrained by the text, making it easy to achieve better text asset correspondence. However, as the length increases, the text-asset correspondence decreases; When the prompts are too long, a more detailed description can help the models generate better textures and geometry, resulting in better authenticity and quality.




\begin{figure}[ht]
    \centering

    \subfigure[]{\begin{minipage}[t]{0.9\linewidth}
                \centering
                \includegraphics[width = 0.93\linewidth]{Figs/catplot.pdf}
                \end{minipage}}

    \subfigure[]{\begin{minipage}[t]{0.9\linewidth}
                \centering
                \includegraphics[width = 0.93\linewidth]{Figs/promptplot.pdf}
                \end{minipage}}
                
    \caption{Illustration of the impact of different models and prompt lengths on the perceptual quality of T23DAs respectively. (a) shows the subjective quality, authenticity, and correspondence score of T23DAs with different methods including Dreamfusion, LatentNerf, Magic3D, Prolificdreamer, SJC, and TextMesh respectively. (b) shows the subjective quality, authenticity, and correspondence score of T23DAs with different prompt lengths. Prompt length is divided into six intervals on average, with 1-6 on the x-axis representing interval numbers from short to long.}
    \label{fig:moscompare}
    \vspace{-0.6cm}
\end{figure}


We applied Recurrency Sequence Processing to address the lack of consistency in the coarse dance representation of the~\cite{li2024lodge} model. We named this Recurrency Sequence Representation Learning as Dance Recalibration (DR). Dance recalibration uses \(n\) Dance Recalibration Blocks (DRB) corresponding to the length of the rough dance sequence to add sequential information to the rough dance representation to improve the consistency of the whole dance. The overall structure of our model is illustrated in Figure 1.

\begin{figure}[!t]
    \centering
    \includegraphics[width=\textwidth]{Figure1.eps}
    \caption{overall procedure of Pooling processing by our Pooling Block}
    \label{fig:enter-label4}
\end{figure}


\subsection{Dance Recalibration (DR)}
When the dance motion representation passes through the Dance Decoder Process using the~\cite{li2024lodge} model, it yields a coarse dance motion representation. During this process, the dance motion representations that pass through Global Diffusion follow a distribution but can output unstable values. This results in awkward dance motions when viewed from a sequential perspective. To address this issue, we added a Dance Recalibration Process.

DR fundamentally follows a structure similar to RNNs. Although RNNs are known to suffer from the gradient vanishing problem as they get deeper, the sequence length of the coarse dance representation in \cite{li2024lodge} is not long enough to cause this issue, making it suitable for use. Using LSTM or GRU, which solve the gradient vanishing problem, would make the model too complex and computationally expensive, making them unsuitable for use with the Denoising Diffusion Process \cite{ho2020denoising, song2020denoising}.

The coarse dance representation has 139 channels, consisting of 4-dim foot positions, 3-dim root translation, 6-dim rotaion information and 126-dim joint rotation channels. Of these, the 126-dim channels directly impact the dance motion, and all DR operations are performed using these 126 channels.

The values output from the Global Dance Decoder \(GD_{i}\), contain unstable dance motion information that follows a distribution. We construct Global Recalibrated Dance \(GRD_{i}\) by concatenating \(C\) the information from \(GRD_{i-1}\) with \(GD_{i}\) and applying pooling \(P\), thereby adding sequential information. However, using previous information as is may result in overly simple and smoothly connected dance motions. To prevent this, we add Gaussian noise \(G\) to the previous information \(GRD_{i-1}\) to produce more varied dance motions. This process is represented in Equations 1 below. The entire procedure is illustrated in Figure 2, 3.
\begin{equation}
    GRD_{i} = P(C(GD_{i} , GRD_{i-1} + G(Threshold))
\end{equation}



\begin{figure}[!t]
    \centering
    \includegraphics[width=\textwidth]{DanceRecalibration.eps}
    \caption{Overall of the Dance Recalibration Block Structure}
    \label{fig:enter-label1}
\end{figure}

\begin{figure}[!t]
    \centering
    \includegraphics[width=\textwidth]{DanceRecalibrationBlock.eps}
    \caption{The structure of the dance recalibration block}
    \label{fig:enter-label2}
\end{figure}

\subsection{Pooling Block}
Pooling \(P\) uses a simple pooling method. When \(GRD_{i}\) with added \(G\) and \(GD_{i+1}\) are input, they are concatenated into a \((Batch\times2\times126)\). First, we perform Layer Normalization to minimize differences between layers. Then, we pass through three simple 1D-Convolution Blocks, each followed by an activation function and batch normalization, to construct \(GRD_{i+1}\) that includes information from the previous dance motion. This procedure is illustrated in Figure 4.

\begin{figure}[!t]
    \centering
    \includegraphics[width=\textwidth]{Figure3.eps}
    \caption{overall procedure of Pooling processing by our Pooling Block}
    \label{fig:enter-label3}
\end{figure}

By following all these steps, each dance motion incorporates a bit of information from the previous dance motions, producing an overall coarse dance motion that follows the distribution of Global Diffusion while also retaining sequential information. This process is expressed in Equation 2:

\begin{equation}
    Total Coarse Dance Motion = C_{i=1}^{n}(P(C(GD_{i} , GRD_{i-1} + G(Threshold))), P(GD_{0}))
\end{equation}

We did not use bi-directional information because it complicates the calculations and can destabilize sequential information when using more than two \(GD_{i}\). Since there is a trade-off between generating complex dance motions and maintaining consistency, it is crucial to add appropriate noise. However, due to time constraints, we could not conduct various ablation studies.

\subsection{Experimental Setup}
\label{section:experimental_setup}
\textbf{Datasets:} Table~\ref{tab:datasets} provides a detailed breakdown of the SOTA intrusion datasets utilized in our study. 
%For each dataset we follow the data preparation steps outlined in section~\ref{section:data_preparation}. 
% \sean{is this section necessary with reduced page limit?}
% \begin{enumerate}
%     \item X-IIoTID \cite{al2021x}: The dataset consists of 59 features which are collected with the independence of devices and connectivity, generating a holistic intrusion data set to represent the heterogeneity of IIoT systems. It includes novel IIoT connectivity protocols, activities of various devices, and attack scenarios.  
%     \item WUSTL-IIoT \cite{zolanvari2021wustl}: WUSTL-IIoT aims to emulate real-world industrial systems. The dataset is deliberately unbalanced to imitate real-world industrial control systems, consisting of 41 features and 1,194,464 observations.
%     \item CICIDS2017 \cite{Sharafaldin2018TowardGA} The CICIDS2017 dataset includes a comprehensive collection of benign and malicious network traffic. It contains 80 features and represents a broad range of attacks, such as DoS, DDoS, Brute Force, XSS, and SQL Injection, across more than 2.8 million network flows. The dataset is widely used in evaluating intrusion detection systems.
%     \item UNSW-NB15 \cite{moustafa2015unsw, moustafa2016evaluation, moustafa2017novel, moustafa2017big, sarhan2020netflow} UNSW-NB15 is a comprehensive network intrusion dataset created by the University of New South Wales. It contains 49 features representing normal and malicious activities generated using IXIA's network traffic generator, covering a variety of contemporary attack types. 
% \end{enumerate}
For IIoT intrusion, we use IIoT datasets X-IIoTID \cite{al2021x} and WUSTL-IIoT \cite{zolanvari2021wustl}. We also include commonly used network intrusion datasets CICIDS2017 \cite{Sharafaldin2018TowardGA} and UNSW-NB15 \cite{moustafa2015unsw}. For X-IIoTID \cite{al2021x}, CICIDS2017 \cite{Sharafaldin2018TowardGA}, and UNSW-NB15 \cite{moustafa2015unsw}, we split the data across five experiences such that each experience contains two to four attacks. For WUSTL-IIoT \cite{zolanvari2021wustl}, we split the data across four experiences such that each experience contains one attack. We perform this data split to simulate an evolving data stream with emerging cyber attacks over time where each experience contains different attacks. 


%%%%%%%%%%%%%%%%%%%%%%%%%%%%%%%%%%%%%%%%%%%%%%%%%%%%%%%%%%%%%%%%%%%%%%%%%%%
\begin{table}[h]
    \caption{Selected Intrusion Datasets}
    \centering
    \label{tab:datasets}
    \resizebox{.99\columnwidth}{!}{
    \begin{tabular}{c|c|c|c|c}
    \hline
    Dataset    & Size      & Normal Data & Attack Data & Attack Types \\ 
    \hline
    X-IIoTID \cite{al2021x}   & 820,502   & 421,417     & 399,417     & 18           \\
    \hline
    WUSTL-IIoT \cite{zolanvari2021wustl} & 1,194,464 & 1,107,448   & 87,016      & 4       \\
    \hline
    CICIDS2017 \cite{Sharafaldin2018TowardGA} & 2,830,743 & 2,273,097 & 557,646 & 15 \\
    \hline
    UNSW-NB15 \cite{moustafa2015unsw}
 & 257,673 & 164,673 & 93,000 & 10 \\
    \hline
    \end{tabular}}
\end{table}
%%%%%%%%%%%%%%%%%%%%%%%%%%%%%%%%%%%%%%%%%%%%%%%%%%%%%%%%%%%%%%%%%%%%%%%%%%%

\textbf{Baselines:} %Due to the novelty of this problem formulation, there are no directly comparable methods. However, the most similar widely studied problem would be unsupervised continual learning (UCL). Therefore, 
We evaluate our algorithm against two SOTA unsupervised continual learning (UCL) algorithms: the Autonomous Deep Clustering Network (\textbf{ADCN}) \cite{ashfahani2023unsupervised}, and an autoencoder paired with K-Means clustering. The autoencoder K-Means model is combined with Learning without Forgetting \cite{lwf2019Li} continual learning loss; we refer to this model as \textbf{LwF}. Note that both \textbf{ADCN} and \textbf{LwF} require a small amount of labeled normal and attack data to perform classification. We also compare our approach against SOTA ND methods: local outlier factor (\textbf{LOF})\cite{Faber_2024}, one-class support vector machine (\textbf{OC-SVM})\cite{Faber_2024}, principal component analysis (\textbf{PCA})\cite{rios2022incdfm}, and Deep Isolation Forest (\textbf{DIF}) \cite{xu2023deep}. 
%We train the ND algorithms on the clean subset of normal data, $N_c$, and evaluate their performance on the remainder of the dataset. 
Since these ND models cannot be retrained on unlabeled contaminated data, continual learning is not feasible for these methods.

%an autoencoder with K-Means clustering paired with SOTA Learning without Forgetting (LwF) continual loss (LwF) \cite{lwf2019Li}.
%Notably, many SOTA UCL algorithms rely on image-specific contrastive pairs, which is not directly applicable to intrusion detection \cite{madaan2022representational, yu2023scale, fini2022self, liu2024unsupervised}.

%%%%%%%%%%%%%%%%%%%%%%%%%%%%%%%%%%%%%%%%%%%%%%%%%%%%%%%
\begin{figure*}
    \centering
    \includegraphics[width=.95\linewidth]{figures/cl_experiments.pdf}
    \caption{Continual learning metric results of ADCN\cite{ashfahani2023unsupervised}, LwF\cite{lwf2019Li}, and \Design{}}
    \label{fig:continual_methods_results}
\end{figure*}
%%%%%%%%%%%%%%%%%%%%%%%%%%%%%%%%%%%%%%%%%%%%%%%%%%%%%%%

\textbf{Evaluation Metrics:} To evaluate the model performance, we report $F_{1}$ score. Since there is a class imbalance within these datasets, to simulate real world IDS, $F_{1}$ score gives an accurate idea on attack detection. For the continual learning methods, we evaluate their performance at the end of each training experience on all experience test sets. This generates a matrix of $F_{1}$ score results $R_{ij}$ such that $i$ is the current training experience, and $j$ is the testing experience. To summarize this matrix of results, we report widely used CL metrics \cite{diaz2018don}: average $F_{1}$ score on current experience (AVG), forward transfer (FwdTrans), and backward transfer (BwdTrans). For a matrix $R_{ij}$ with $m$ total experiences, our metrics are formulated as follows: $\text{AVG}_{F_1} = \frac{\sum_{i = j} R_{ij}}{m}$; $\text{FwdTrans}_{F_1} = \frac{\sum_{j>i} R_{ij}}{\frac{m * (m-1)}{2}}$; $\text{BwdTrans}_{F_1} = \frac{\sum_{i}^m R_{mi} - R_{ii}}{\frac{m * (m-1)}{2}}$.
AVG is the average performance on the current test experience at every point of training. FwdTrans is the average performance on ``future'' experiences, which simulates performance on zero-day attacks. Finally, BwdTrans is the average change in performance of ``past'' test experiences at a ``future'' point of training. A negative BwdTrans indicates catastrophic forgetting, whereas a positive BwdTrans  indicates the model actually improved performance on past experiences after learning a future experience. Overall, AVG measures seen attacks, FwdTrans measures zero-day attacks, and BwdTrans measures forgetting. For all metrics, a higher positive result indicates a better performance. 

We also report the threshold-free metric Precision-Recall Area Under the Curve (PR-AUC) \cite{praucDavid06}. Since \Design{} requires selecting a threshold, PR-AUC allows us to assess model performance independently of the threshold. We choose PR-AUC over Receiver Operating Characteristic Area Under the Curve (ROC-AUC) because ROC-AUC can give misleadingly high results in the presence of class imbalance \cite{praucDavid06}.

\textbf{Hyperparameters:} %For $L_{CND}$ hyperparameters are the number of K-Means clusters $K$, the reconstruction loss strength $\lambda_R$,  the continual learning loss strength $\lambda_{CL}$, and the cluster separation loss margin $m$. 
We utilize \textit{elbow method} \cite{han2011data} for determining the number of clusters $K$. 
%It tests a range of $K$ values and then selects the value   where there is a significant change in slope, called the elbow point. 
%This resulted in $K$ values between 100-500. 
We set $\lambda_R$ and $\lambda_{CL}$ to 0.1, and for $m$ we use 2 after careful experimentation. For the AE modules of \Design{}, we use 4-layer MLP with 256 neurons in the hidden layers. We train it using Adam optimizer \cite{kingma2017adammethods} with a learning rate of 0.001. For PCA, we use the explained variance method and set it to 95\% \cite{rios2022incdfm}.

\textbf{Hardware:} We run our experiments on NVIDIA GeForce RTX 3090 GPU, with a AMD EPYC 7343 16-Core processor.

\subsection{Results}

\textbf{Continual Learning Comparison:} Fig.~\ref{fig:continual_methods_results} presents the results of our approach \Design{} compared with ADCN\cite{ashfahani2023unsupervised} and LwF\cite{lwf2019Li}. \Design{} shows the best performance on both seen (AVG) and unseen (FwdTrans) attacks across all datasets. \Design{} also has the highest BwdTrans on all except one dataset (UNSW-NB15). The average BwdTrans of \Design{} (0.87\%) is higher than the average BwdTrans of both ADCN (-0.06\%) and LwF (0.09\%). Notably, the BwdTrans of \Design{} is positive for three datasets. Indicating past experiences actually improve after training on future experiences for these datasets. Given the high FwdTrans as well, our approach finds features that generalize well to future experiences. 

Table~\ref{tab:improvement} shows the improvement of \Design{} over the UCL baselines on all datasets. Bold and underlined cases indicate the best and the second best improvements with respect to each metric, respectively. These improvements were calculated by comparing the performance of \Design{} to the baselines, where the improvement values represent the proportional increase over the baseline performance. We do not include BwdTrans because a proportional increase does not make sense for a metric that can be negative. \Design{} has up to $4.50\times$ and $6.1\times$ AVG improvement on ADCN and LwF, respectively. In addition, \Design{} has up to $6.47\times$ and $3.47\times$ FwdTrans improvement on ADCN and LwF. Averaged across all datasets, \Design{} shows a $1.88\times$ and $1.78\times$ improvement on AVG, and a $2.63\times$ and $1.60\times$ improvement on FwdTrans, compared to ADCN and LwF, respectively. %These results underscore the benefit of our continual novelty detection method \Design{}. The notably high FwdTrans score emphasizes how novelty detection can be used to identify unseen anomalous data, thereby significantly enhancing performance on zero-day attacks.

Overall, these results highlight the benefit of continual ND over UCL methods for IDS. \Design{}, with its PCA-based novelty detector, excels by effectively harnessing the normal data to identify attacks. A key strength of our approach lies in the assumption that normal data forms a distinct class, while everything else is treated as anomalous. This assumption is particularly well-suited to IDS. In contrast, methods like ADCN and LwF do not make this distinction where they handle both normal and attack data similarly, limiting their ability to fully exploit the inherent structure of the data. 



% %%%%%%%%%%%%%%%%%%%%%%%%%%%%%%%%%%%%%%%%%%%%%%%%%%%%%%%
% \begin{table}[]
% \centering
% \caption{\Design{} Percentage Improvement over UCL Baselines on AVG and FwdTrans}
% \label{tab:improvement}
% \begin{tabular}{|c|c|c|c|}
% \hline
% Baseline      & Dataset    & AVG  & FwdTrans  \\ \hline
% ADCN\cite{ashfahani2023unsupervised}          & X-IIoTID   & 101.88\%        & 400.35\%        \\ \cline{2-4} 
%               & WUSTL-IIoT & 349.86\%        & 546.68\%        \\ \cline{2-4} 
%               & CICIDS2017 & 37.19\%         & 73.46\%         \\ \cline{2-4} 
%               & UNSW-NB15  & 29.25\%         & 43.90\%         \\ \hline
% LwF\cite{lwf2019Li} & X-IIoTID   & 46.43\%         & 35.39\%         \\ \cline{2-4} 
%               & WUSTL-IIoT & 510.92\%        & 246.81\%        \\ \cline{2-4} 
%               & CICIDS2017 & 92.72\%         & 163.81\%        \\ \cline{2-4} 
%               & UNSW-NB15  & 11.07\%         & 2.20\%          \\ \hline
% \end{tabular}
% \end{table}
% %%%%%%%%%%%%%%%%%%%%%%%%%%%%%%%%%%%%%%%%%%%%%%%%%%%%%%%

%%%%%%%%%%%%%%%%%%%%%%%%%%%%%%%%%%%%%%%%%%%%%%%%%%%%%%%
\begin{table}[]
\centering
\caption{\Design{} Improvement over UCL Baselines}
\label{tab:improvement}
\scalebox{1}{
\begin{tabular}{|c|c|c|c|}
\hline
Baseline      & Dataset    & AVG  & FwdTrans  \\ \hline
ADCN\cite{ashfahani2023unsupervised}  & X-IIoTID   & $\underline{2.02\times}$  & $\underline{5.00\times}$   \\ \cline{2-4} 
                                      & WUSTL-IIoT & $\mathbf{4.50\times}$  & $\mathbf{6.47\times}$   \\ \cline{2-4} 
                                      & CICIDS2017 & $1.37\times$  & $1.73\times$   \\ \cline{2-4} 
                                      & UNSW-NB15  & $1.29\times$  & $1.44\times$   \\ \hline
LwF\cite{lwf2019Li}                   & X-IIoTID   & $1.46\times$  & $1.35\times$   \\ \cline{2-4} 
                                      & WUSTL-IIoT & $\mathbf{6.11\times}$  & $\mathbf{3.47\times}$   \\ \cline{2-4} 
                                      & CICIDS2017 & $\underline{1.93\times}$  & $\underline{2.64\times}$   \\ \cline{2-4} 
                                      & UNSW-NB15  & $1.11\times$  & $1.02\times$   \\ \hline
\end{tabular}}
\end{table}

%%%%%%%%%%%%%%%%%%%%%%%%%%%%%%%%%%%%%%%%%%%%%%%%%%%%%%%

%Figure~\ref{fig:XIIoT_graph} shows the $F_{1}$ score of ADCN and \Design{} for each experience on both datasets. Similarly, we use green and red colors for \Design{} and ADCN respectively. Notably for \Design{}, the $F_{1}$ score of each experience has little change over training time. This highlights the strength of novelty detection for IDSs, as even before seeing attacks \Design{} has good performance. On the other hand, ADCN test experiences do not improve until the associated training experience, meaning ADCN does not have an ability to generalize to future attacks. ADCN utilizes a subset of labeled data to assign labels to clusters. This subset of labeled might be causing ADCN to overfit to the attacks within the current experience, therefore leading ADCN to not generalize well. We can also clearly see that our approach is consistently better (higher $F_{1}$ score) than the state-of-the-art ADCN. 

% %%%%%%%%%%%%%%%%%%%%%%%%%%%%%%%%%%%%%%%%%%%%%%%%%%%%%%%
% \begin{figure*}[t]
%     \centering
%     \begin{subfigure}[t]{\linewidth}
%         \centering
%         \includegraphics[width=\linewidth]{figures/X-IIoTID-experiences.pdf}
%         \caption{X-IIoTID}
%         \label{fig:ADCN_XIIoT_results}
%     \end{subfigure}
%     \begin{subfigure}[t]{\linewidth}
%         \centering
%         \includegraphics[width=\linewidth]{figures/WUSTL-IIoT-experiences.pdf}
%         \caption{WUSTL-IIoT}
%         \label{fig:WUSTL-}
%     \end{subfigure}
%     \caption{$F_1$ Score of ADCN and \Design{} of each test experience over training experiences.}
%     \label{fig:XIIoT_graph}
% \end{figure*}
% %%%%%%%%%%%%%%%%%%%%%%%%%%%%%%%%%%%%%%%%%%%%%%%%%%%%%%%

\textbf{Novelty Detectors Comparison:} Fig.~\ref{fig:novelty_methods_results} compares LOF\cite{Faber_2024}, OC-SVM\cite{Faber_2024}, PCA\cite{rios2022incdfm}, and DIF \cite{xu2023deep} with \Design{} on all datasets. The average $F_{1}$ score of the novelty detection methods are compared to the AVG of \Design{}.  It can be seen \Design{} outperforms all other methods across all datasets. The two best performing methods are DIF and PCA. The average $F_{1}$ score improvement across all datasets of \Design{} is $1.16\times$ and $1.08\times$ over DIF and PCA, respectively. These results highlight the critical role of leveraging information from unsupervised data streams. Unlike these ND algorithms, \Design{} is capable of continuously learning from this unsupervised data, enabling it to enhance PCA reconstruction over time. By integrating evolving data patterns, \Design{} not only adapts to new anomalies but also improves its overall detection accuracy, demonstrating a clear advantage in dynamic environments.

%Given that \Design{} employs PCA detection, this indicates that the CFE effectively extracts useful features from the unlabeled training experiences. T

%%%%%%%%%%%%%%%%%%%%%%%%%%%%%%%%%%%%%%%%%%%%%%%%%%%%%%%   
\begin{figure}
    \centering
    \includegraphics[width=0.9\linewidth]{figures/novelty_detectors_experiments.pdf}
    \caption{Average $F_1$ score on all experiences of \Design{} and novelty detection methods: LOF, OC-SVM, PCA, DIF}
    \label{fig:novelty_methods_results}
\end{figure}
%%%%%%%%%%%%%%%%%%%%%%%%%%%%%%%%%%%%%%%%%%%%%%%%%%%%%%%
%%%%%%%%%%%%%%%%%%%%%%%%%%%%%%%%%%%%%%%%%%%%%%%%%%%%%%% 
\begin{figure}
    \centering
    \includegraphics[width=0.86\linewidth]{figures/novelty_detectors_pr_auc.pdf}
    \caption{Thresholding Free Evaluation of \Design{}}
    \label{fig:thresholding_free}
\end{figure}

%%%%%%%%%%%%%%%%%%%%%%%%%%%%%%%%%%%%%%%%%%%%%%%%%%%%%%%

\textbf{Pre-threshold Evaluation:} While thresholding plays a crucial role in attack decision-making, evaluating model prediction performance before applying threshold is also important. The UCL algorithms (ADCN\cite{ashfahani2023unsupervised} and LwF\cite{lwf2019Li}) do not output anomaly scores because they select classes based on the closest labeled cluster. Therefore we compare against the two best ND methods: DIF\cite{xu2023deep} and PCA\cite{rios2022incdfm}. Fig.~\ref{fig:thresholding_free} presents the PR-AUC values of DIF, PCA, and \Design{}. It can be seen that \Design{} provides the best threshold free results, which aligns with the threshold-based results presented earlier. The strong performance of \Design{} in both pre-threshold and threshold-based evaluations demonstrates that the model is robust regardless of the decision threshold. 

\subsection{Ablation Study}

To demonstrate the impact of our loss function components, we perform an ablation study. Table~\ref{tab:ablation_loss} shows the results of \Design{} with each loss function removed to demonstrate their individual effectiveness. Bold and underlined cases indicate the best and the second best performances with respect to each metric, respectively. \Design{} without reconstruction loss ($L_R$) and \Design{} without cluster separation loss ($L_{CS}$) performs worse in all categories. \Design{} without both $L_R$ and continual learning loss ($L_{CL}$) actually performs better AVG but has worse BwdTrans and FwdTrans. AVG does not account for past experiences, so the significantly negative BwdTrans indicates \Design{} w/o $L_R$ and $L_{CL}$ forgets, and therefore would perform worse on those experiences in the future. This would make sense as a regularization loss to improve continual learning would slightly decrease performance in non-continual scenario. Overall \Design{} has the best results when taking every metric category into account. Notably the low BwdTrans and FwdTrans of \Design{} (w/o $L_R$) showcases how the reconstruction loss helps \Design{} generalize better to unseen and past data. This highlights the power of $L_R$ to provide good features for continual learning. 

%%%%%%%%%%%%%%%%%%%%%%%%%%%%%%%%%%%%%%%%%%%%%%%%%%%%%%%%%%%%%%%%%%%%%
\begin{table}[]
\caption{Ablation Study of \Design{} Loss Functions}
\label{tab:ablation_loss}
\centering
\begin{tabular}{|c|c|c|c|}
\hline
Strategy                         & AVG              & BwdTrans        & FwdTrans         \\ \hline
CND-IDS                          &\underline{76.92\%}    & \textbf{0.87\%} & \textbf{73.70\%} \\ \hline
CND-IDS (w/o $L_{CS}$)           & 66.23\%          & \underline{0.09\%}    & 70.26\%          \\ \hline
CND-IDS (w/o $L_R$)              & 72.86\%          & -5.44\%         & 67.82\%          \\ \hline
CND-IDS (w/o $L_R$ and $L_{CL}$) & \textbf{79.92\%} & -11.26\%        & \underline{71.01\%}    \\ \hline
\end{tabular}
\end{table}
%%%%%%%%%%%%%%%%%%%%%%%%%%%%%%%%%%%%%%%%%%%%%%%%%%%%%%%%%%%%%%%%%%%%%%%

\subsection{Overhead Analysis}
%%%%%%%%%%%%%%%%%%%%%%%%%%%%%%%%%%%%%%%%%%%%%%%%%%%%%%%%%%%
% \begin{table}[]
% \centering
% \caption{Average training time and inference time per sample across all datasets in milliseconds}
% \label{tab:overhead}
% \begin{tabular}{|c|c|c|}
% \hline
% Strategy               & Inference Time(ms) \\ \hline
% \Design{}                   & 0.0019             \\ \hline
% ADCN\cite{ashfahani2023unsupervised}    & 0.4061             \\ \hline
% LwF\cite{lwf2019Li}           & 0.0677             \\ \hline
% DIF\cite{xu2023deep}         & 1.0535             \\ \hline
% PCA\cite{rios2022incdfm}       & 0.0018             \\ \hline
% \end{tabular}
% \end{table}
%%%%%%%%%%%%%%%%%%%%%%%%%%%%%%%%%%%%%%%%%%%%%%%%%%%%%%%%%%%%%
\begin{table}[]
\centering

\caption{Average inference time (in ms) per test sample}
\label{tab:overhead}
\scalebox{0.95}{
\begin{tabular}{|c|c|c|c|c|c|}
\hline
Strategy           & \Design{} & ADCN   & LwF    & DIF    & PCA    \\ \hline
Inference Time (ms) & \underline{0.0019}                     & 0.4061 & 0.0677 & 1.0535 & \textbf{0.0018} \\ \hline
\end{tabular}}
\end{table}
%%%%%%%%%%%%%%%%%%%%%%%%%%%%%%%%%%%%%%%%%%%%%%%%%%%%%%%%
Table~\ref{tab:overhead} evaluates the inference overhead of \Design{} compared to ADCN \cite{ashfahani2023unsupervised}, LwF \cite{lwf2019Li}, DIF \cite{xu2023deep}, and PCA \cite{rios2022incdfm}. %, excluding OC-SVM \cite{Faber_2024} and LOF \cite{Faber_2024} due to poor performance. 
\Design{} offers the fastest inference time among continual learning methods. Out of novelty detection methods, \Design{} is second only to PCA. We attribute the efficiency of \Design{} to avoiding the clustering classification used by LwF and ADCN. %\Design{} instead uses PCA reconstruction, which is much quicker than comparing data points to clusters. In addition, 
The difference between \Design{} and PCA is minimal, only 0.0001 milliseconds slower, due to the additional but lightweight step of encoding the data. Considering that the average median flow duration across datasets is 27.77 milliseconds, the overhead introduced by \Design{} is negligible in the context of real-time traffic flow.

%In this section we analyze the inference overhead of \Design{} compared to ADCN\cite{ashfahani2023unsupervised}, LwF\cite{lwf2019Li}, DIF\cite{xu2023deep}, and PCA\cite{rios2022incdfm}. We do not include OC-SVM\cite{Faber_2024} and LOF \cite{Faber_2024} due to weak performance. Table~\ref{tab:overhead} shows the average inference time in milliseconds per sample across all datasets. \Design{} has the best inference time besides PCA. We attribute this good inference time to \Design{} not using clustering classification like LwF and ADCN. Evidently, PCA reconstruction utilized by \Design{} is more time efficient than having to compare a data point to all saved clusters. Compared to pure PCA reconstruction, \Design{} is only 0.0001 ms slower. This small increase in inference time is due to the only added computation at inference is encoding the data with the encoder, which is simply a 4 layer MLP. Across all datasets, the average median travel flow duration is 27.77 ms, and the dataset with the quickest median travel flow is UNSW with 4.29 ms. Therefore the overhead introduced by \Design{} is irrelevant compared to the speed of the traffic flow. 

%\label{section:ablation_study}
%To assess the impact of our design choices, we perform an ablation study. Our goal is to analyze (i) threshold function evaluation, and (ii) novelty detection algorithm selection. 

 

%\textbf{Threshold Function Evaluation:} AE, PCA, and \Design{} all require a threshold to classify an anomaly based on the anomaly score. In all previously reported results, we select a widely used threshold that maximizes the $F_{1}$ score on the test set, i.e., Best-F. %This is not realistic but was used to compare the effectiveness of these methods. In this section 
%Here, we analyze three different threshold methods, which we denote: Best-F \cite{su2019robust}, Top-k \cite{zong2018deep}, and validation percentile (ValPer). Best-F uses the threshold that maximizes the $F_{1}$ score on test set. Top-k utilizes the contamination ratio $r$ of the test set, such that $r$ is the percentage of anomalies within the test set. Top-k selects a threshold so that the percentile of data within the test set classified as anomalies is equal to $r$. ValPer utilizes a validation set of normal data, and selects a threshold such that 99.7\% (3 standard deviations) of the normal data is within this threshold. 
%ValPer is the most realistic method as it does not rely on any information from the test set. 
%A breakdown of the $F_{1}$ score results for the different threshold methods is show in Table~\ref{tab:thresholding_results} where the best within each category is bolded. Overall Best-F performs significantly better than the other threshold methods, which is obvious as Best-F is an upper-bound for threshold selection. However the significant gap highlights the importance of threshold selection. Most importantly, \Design{} still performs better than PCA and AE through all threshold methods. 

%%%%%%%%%%%%%%%%%%%%%%%%%%%%%%%%%%%%%%%%%%%%%%%%%%%%%%%
%\begin{table}[]
%    \centering
%    \caption{Threshold Function Evaluation}
%    \resizebox{.97\columnwidth}{!}{
%    \begin{tabular}{c|c|c|c|c}
%        \hline
%         Dataset & Stategy & Best-F & Top-k & ValPer\\
%         \hline
%         & PCA  & 70.9 & 4.03 & 3.56 \\
%         \cline{2-5}
%         X-IIoTID & AE  & 75.6 & 4.03 & 29.4 \\
%         \cline{2-5}
%         & \Design{} & \textbf{78.8} & \textbf{5.63} &  %\textbf{52.9} \\	
%         \hline
%        & PCA  & 85.6 &19.9 & 52.8\\
%         \cline{2-5}
%         WUSTL-IIoT & AE  & 79.6 &19.7 & 37.8\\
%         \cline{2-5}
%         & \Design{} & \textbf{88.2} & \textbf{21.1} & \textbf{55.6}\\	
%         \hline
%    \end{tabular}}
%    \label{tab:thresholding_results}
%\end{table}
%%%%%%%%%%%%%%%%%%%%%%%%%%%%%%%%%%%%%%%%%%%%%%%%%%%%%%%

% %%%%%%%%%%%%%%%%%%%%%%%%%%%%%%%%%%%%%%%%%%%%%%%%%%%%%%%
% \begin{figure}
%     \centering
%     \includegraphics[width=0.95\linewidth]{figures/novelty_ablation.pdf}
%     \caption{Comparison of \Design{} with PCA and AE novelty detection models}
%     \label{fig:novelty_ablation_results}
% \end{figure}
% %%%%%%%%%%%%%%%%%%%%%%%%%%%%%%%%%%%%%%%%%%%%%%%%%%%%%%%

% \textbf{Novelty Detection Algorithm Selection:} For \Design{}, we select PCA as the novelty detection algorithm. As shown in Figure~\ref{fig:novelty_methods_results}, both PCA and AE perform well for detecting intrusions. Therefore, we test both AE and PCA as the novelty detection methods for \Design{}. Figure~\ref{fig:novelty_ablation_results} illustrates the AVG performance of \Design{} with AE and PCA as the novelty detection models. It is evident that PCA outperforms AE, justifying our selection of this algorithm for novelty detection. This could be because the CFE utilizes SAEs, which generate features based on the same reconstruction loss used by AE to classify anomalies. It may be beneficial to use PCA as it deconstructs the input in a different manner, thereby identifying different features and functioning better in conjunction with the SAE-based CFE.


We present RiskHarvester, a risk-based tool to compute a security risk score based on the value of the asset and ease of attack on a database. We calculated the value of asset by identifying the sensitive data categories present in a database from the database keywords. We utilized data flow analysis, SQL, and Object Relational Mapper (ORM) parsing to identify the database keywords. To calculate the ease of attack, we utilized passive network analysis to retrieve the database host information. To evaluate RiskHarvester, we curated RiskBench, a benchmark of 1,791 database secret-asset pairs with sensitive data categories and host information manually retrieved from 188 GitHub repositories. RiskHarvester demonstrates precision of (95\%) and recall (90\%) in detecting database keywords for the value of asset and precision of (96\%) and recall (94\%) in detecting valid hosts for ease of attack. Finally, we conducted an online survey to understand whether developers prioritize secret removal based on security risk score. We found that 86\% of the developers prioritized the secrets for removal with descending security risk scores.




%{\appendices
%\section*{Proof of the First Zonklar Equation}
%Appendix one text goes here.
% You can choose not to have a title for an appendix if you want by leaving the argument blank
%\section*{Proof of the Second Zonklar Equation}
%Appendix two text goes here.}



% \section{References Section}
% You can use a bibliography generated by BibTeX as a .bbl file.
%  BibTeX documentation can be easily obtained at:
%  http://mirror.ctan.org/biblio/bibtex/contrib/doc/
%  The IEEEtran BibTeX style support page is:
%  http://www.michaelshell.org/tex/ieeetran/bibtex/
 
%  % argument is your BibTeX string definitions and bibliography database(s)
% %\bibliography{IEEEabrv,../bib/paper}
% %
% \section{Simple References}
% You can manually copy in the resultant .bbl file and set second argument of $\backslash${\tt{begin}} to the number of references
%  (used to reserve space for the reference number labels box).

% \begin{thebibliography}{1}
% \bibliographystyle{IEEEtran}

% \bibitem{ref1}
% {\it{Mathematics Into Type}}. American Mathematical Society. [Online]. Available: https://www.ams.org/arc/styleguide/mit-2.pdf

% \bibitem{ref2}
% T. W. Chaundy, P. R. Barrett and C. Batey, {\it{The Printing of Mathematics}}. London, U.K., Oxford Univ. Press, 1954.

% \bibitem{ref3}
% F. Mittelbach and M. Goossens, {\it{The \LaTeX Companion}}, 2nd ed. Boston, MA, USA: Pearson, 2004.

% \bibitem{ref4}
% G. Gr\"atzer, {\it{More Math Into LaTeX}}, New York, NY, USA: Springer, 2007.

% \bibitem{ref5}M. Letourneau and J. W. Sharp, {\it{AMS-StyleGuide-online.pdf,}} American Mathematical Society, Providence, RI, USA, [Online]. Available: http://www.ams.org/arc/styleguide/index.html

% \bibitem{ref6}
% H. Sira-Ramirez, ``On the sliding mode control of nonlinear systems,'' \textit{Syst. Control Lett.}, vol. 19, pp. 303--312, 1992.

% \bibitem{ref7}
% A. Levant, ``Exact differentiation of signals with unbounded higher derivatives,''  in \textit{Proc. 45th IEEE Conf. Decis.
% Control}, San Diego, CA, USA, 2006, pp. 5585--5590. DOI: 10.1109/CDC.2006.377165.

% \bibitem{ref8}
% M. Fliess, C. Join, and H. Sira-Ramirez, ``Non-linear estimation is easy,'' \textit{Int. J. Model., Ident. Control}, vol. 4, no. 1, pp. 12--27, 2008.

% \bibitem{ref9}
% R. Ortega, A. Astolfi, G. Bastin, and H. Rodriguez, ``Stabilization of food-chain systems using a port-controlled Hamiltonian description,'' in \textit{Proc. Amer. Control Conf.}, Chicago, IL, USA,
% 2000, pp. 2245--2249.

% \end{thebibliography}


% \newpage

% \section{Biography Section}
% If you have an EPS/PDF photo (graphicx package needed), extra braces are
%  needed around the contents of the optional argument to biography to prevent
%  the LaTeX parser from getting confused when it sees the complicated
%  $\backslash${\tt{includegraphics}} command within an optional argument. (You can create
%  your own custom macro containing the $\backslash${\tt{includegraphics}} command to make things
%  simpler here.)
 
% \vspace{11pt}

% \bf{If you include a photo:}\vspace{-33pt}
% \begin{IEEEbiography}[{\includegraphics[width=1in,height=1.25in,clip,keepaspectratio]{fig1}}]{Michael Shell}
% Use $\backslash${\tt{begin\{IEEEbiography\}}} and then for the 1st argument use $\backslash${\tt{includegraphics}} to declare and link the author photo.
% Use the author name as the 3rd argument followed by the biography text.
% \end{IEEEbiography}

% \vspace{11pt}

% \bf{If you will not include a photo:}\vspace{-33pt}
% \begin{IEEEbiographynophoto}{John Doe}
% Use $\backslash${\tt{begin\{IEEEbiographynophoto\}}} and the author name as the argument followed by the biography text.
% \end{IEEEbiographynophoto}

% \vfill
\bibliographystyle{IEEEtran}
%argument is your BibTeX string definitions and bibliography database(s)
\bibliography{output}
% \newpage
% \section{Biography Section}
% If you have an EPS/PDF photo (graphicx package needed), extra braces are
%  needed around the contents of the optional argument to biography to prevent
%  the LaTeX parser from getting confused when it sees the complicated
%  $\backslash${\tt{includegraphics}} command within an optional argument. (You can create
%  your own custom macro containing the $\backslash${\tt{includegraphics}} command to make things
%  simpler here.)
 
% \vspace{11pt}
% \vspace{-0.5cm}
% \vspace{-80pt}
% \bf{If you include a photo:}\vspace{-33pt}


% \bf{If you will not include a photo:}\vspace{-33pt}
% \begin{IEEEbiographynophoto}{John Doe}
% Use $\backslash${\tt{begin\{IEEEbiographynophoto\}}} and the author name as the argument followed by the biography text.
% \end{IEEEbiographynophoto}

\vfill

\end{document}


