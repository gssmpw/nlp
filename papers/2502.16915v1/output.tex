\documentclass[lettersize,journal]{IEEEtran}
\usepackage{amsmath,amsfonts}
\usepackage{algorithmic}
\usepackage{algorithm}
\usepackage{array}
% \usepackage[caption=false,font=normalsize,labelfont=sf,textfont=sf]{subfig}
\usepackage{textcomp}
\usepackage{stfloats}
\usepackage{url}
\usepackage{verbatim}
\usepackage{graphicx}
\usepackage{cite}

% mycite 
\usepackage{multirow}
\usepackage{bbding}
\usepackage{subfigure}
\usepackage{booktabs}
\usepackage{color} 
\usepackage{siunitx}
\usepackage{colortbl}
% \usepackage{caption}
\definecolor{mygray}{gray}{.9}
\definecolor{mypurple}{RGB}{128,0,128}
\hyphenation{op-tical net-works semi-conduc-tor IEEE-Xplore}
% updated with editorial comments 8/9/2021


\begin{document}

\title{\color{black}{Multi-Dimensional Quality Assessment for Text-to-3D Assets: Dataset and Model}}
% Subjective-aligned database and metric for Text-to-3D content quality assessment.
% Quality Assessment for AI-generated 3D Content Guided by Text
% AI-generated Text-Guided 3D Content Quality Assessment 
% \author{IEEE Publication Technology,~\IEEEmembership{Staff,~IEEE,}
%         % <-this % stops a space
% \thanks{This paper was produced by the IEEE Publication Technology Group. They are in Piscataway, NJ.}% <-this % stops a space
% \thanks{Manuscript received April 19, 2021; revised August 16, 2021.}}

% % The paper headers
% \markboth{Journal of \LaTeX\ Class Files,~Vol.~14, No.~8, August~2021}%
% {Shell \MakeLowercase{\textit{et al.}}: A Sample Article Using IEEEtran.cls for IEEE Journals}

% \IEEEpubid{0000--0000/00\$00.00~\copyright~2021 IEEE}
% % Remember, if you use this you must call \IEEEpubidadjcol in the second
% % column for its text to clear the IEEEpubid mark.

\author{
Kang Fu*, Huiyu Duan*, Zicheng Zhang, Xiaohong Liu, \emph{Member, IEEE,}\\  Xiongkuo Min$^{\dagger}$, \emph{Member, IEEE,} Jia Wang, and Guangtao Zhai$^{\dagger}$, \emph{Fellow, IEEE
} % <-this % stops a space
% \author{
% Kang Fu, Guangtao Zhai$^\dag$, \emph{Senior Member, IEEE
% } % <-this % stops a space

% \IEEEcompsocitemizethanks{\IEEEcompsocthanksitem This work was supported in part by NSFC (No.62225112, No.61831015, No. 62301316), the Fundamental Research Funds for the Central Universities, National Key R\&D Program of China 2021YFE0206700, Shanghai Municipal Science and Technology Major Project (2021SHZDZX0102), STCSM 22DZ2229005, and the China Postdoctoral Science Foundation under Grant 2023TQ0212 and 2023M742298. \textit{(Corresponding Author: Guangtao Zhai.)} \protect}

\IEEEcompsocitemizethanks{\IEEEcompsocthanksitem Kang Fu, Huiyu Duan, Zicheng Zhang, Xiaohong Liu, Xiongkuo Min, Jia Wang and Guangtao Zhai are with Shanghai Jiao Tong University, 200240 Shanghai, China. E-mail:\{fuk20\-20, huiyuduan, zzc1998, xiaohongliu, minxiongkuo, jiawang, zhaiguangtao\}@sjtu.edu.cn. 
This work was supported in part by the National Key R\&D Program of China under Grant 2021YFE0206700, in part by the National Natural Science Foundation of China under Grants 62401365, 62271312, 62225112, 62132006, and in part by the Shanghai Pujiang Program under Grant 22PJ1407400. * Equal Contributions. $\dagger$ Corresponding Authors.\protect}
        % <-this % stops a space
}% <-this % stops a space

% \author{
% Kang Fu, Guangtao Zhai$^\dag$, \emph{Senior Member, IEEE
% } % <-this % stops a space
% }% <-this % stops a space

% \thanks{Manuscript received April 19, 2021; revised August 16, 2021.}}

\maketitle

\begin{abstract}

Recent advancements in text-to-image (T2I) generation have spurred the development of text-to-3D asset (T23DA) generation, leveraging pretrained 2D text-to-image diffusion models for text-to-3D asset synthesis. Despite the growing popularity of text-to-3D asset generation, its evaluation has not been well considered and studied. However, given the significant quality discrepancies among various text-to-3D assets, there is a pressing need for quality assessment models aligned with human subjective judgments. To tackle this challenge, we conduct a comprehensive study to explore the T23DA quality assessment (T23DAQA) problem in this work from both subjective and objective perspectives. Given the absence of corresponding databases, we first establish the largest text-to-3D asset quality assessment database to date, termed the AIGC-T23DAQA database. This database encompasses 969 validated 3D assets generated from 170 prompts via 6 popular text-to-3D asset generation models, and corresponding subjective quality ratings for these assets from the perspectives of quality, authenticity, and text-asset correspondence, respectively. Subsequently, we establish a comprehensive benchmark based on the AIGC-T23DAQA database, and devise an effective T23DAQA model to evaluate the generated 3D assets from the aforementioned three perspectives, respectively. Specifically, the proposed method utilizes the projection videos of text-to-3D assets to extract 3D shape, texture and text-asset correspondence features, then fuses them to calculate the final three preference scores respectively. Extensive experimental results demonstrate the effectiveness of the proposed T23DAQA method in evaluating the quality of AI generated 3D asset, which is more consistent with human perception. To the best of our knowledge, this is the first work that studies the problem of text-guided 3D generation quality assessment, and  The database is released at \url{https://github.com/ZedFu/T23DAQA}.

% The dataset is available at https:
\end{abstract}

\begin{IEEEkeywords}
text-to-3D asset generation, subjective quality assessment, objective quality assessment, artificial intelligence generated content (AIGC)
\end{IEEEkeywords}

% 
% 
The widespread integration of communication networks and smart devices in modern control systems has increased the vulnerability of industrial systems to online cyber-attacks, e.g., Industroyer, Blackenergy, etc \citep{osti_1505628}.
% Modern control systems have seen a large push to include communication networks and smart devices to increase performance, made possible by improvements in communication device cost and energy consumption. This trend has been coupled with the usage of open-standard communication protocols among industrial control systems, making them vulnerable to online cyber-attacks such as Industroyer, Blackenergy, etc \citep{osti_1505628}. 
To counter this, methods have been developed to improve security by achieving attack detection, mitigation, and monitoring, among others \citep{sandberg2022secure}. This paper focuses on active attack diagnosis to mitigate stealthy attacks. 
%
%\subsection{Literature review}

Active diagnosis techniques rely on the inclusion of additional moduli to control systems
% inclusion within the control system of additional moduli 
to alter the behavior of the system compared to information known by the attacker. 
For instance, the concept of additive watermarking was introduced in \cite{mo2015physical}, where noise signals of known mean and variance are added at the plant and compensated for it at the controller. 
This compensation, however, is not exact, causing some performance degradation. Thus, trade-offs between performance and detectability  are necessary \citep{zhu2023detection}.
% A later work \citep{zhu2023detection} designs the watermark signal by trading performance for detection. Thus, although additive watermarking serves as a good detection scheme, they endure performance losses even in the nominal case. 

In encrypted control \citep{darup2021encrypted}, the sensor data is encrypted, sent to the controller, and then operated on directly. Encrypted input signals are sent back to the plant for decryption. Although encryption is widespread in IT security, in control systems it presents some concerns, such as the introduction of time delays \citep{stabile2024verifiable}, while it may present inherent weaknesses \citep{alisic2023model}.
% they are not preferred as they introduce time delays \citep{stabile2024verifiable} which can cause instability, and some encryption schemes can be very weak  \citep{alisic2023model}. 

In moving target defense \citep{griffioen2020moving}, the plant is augmented with fictitious dynamics, known to the controller. The plant output is transmitted to the controller along with the fictitious states over a network under attack. 
The additional measurements then aide in the detection of attacks. 
This comes at the cost of higher communication bandwidth needs, which increases rapidly with the dimension of the augmented systems.
% Since the dynamics of the fictitious dynamics are exactly known to the controller, the attack is detected easily. However, when the scale of the system increases, the communication bandwidth used by moving the target defense approach increases rapidly. 

Other recently proposed works include two-way coding \citep{fang2019two}, a weak encryuption technique, and dynamic masking \citep{abdalmoaty2023privacy}, which enhances privacy as well as security, have been shown to be effective against zero-dynamics attacks.
% Two-way coding \citep{fang2019two} and dynamic masking \citep{abdalmoaty2023privacy} are other recently proposed approaches. Two-way coding is another form of weak encryption technique whilst dynamic masking proposes an architecture that enhances both privacy and security. These schemes are shown to be effective against zero dynamics attacks but remain to be studied for other classes of attacks. 
% Recent extensions include \citep{mukherjee2021secure,ramos2024privacy}.
% Some other works which are related are \citep{mukherjee2021secure}, an extension of \cite{fang2019two}. The work \citep{ramos2024privacy} is an extension of moving target defense for multi-agent systems. 
Furthermore, filtering techniques for attack detection are proposed by \cite{murguia2020security,hashemi2022codesign,escudero2023safety}, while not focusing on stealthy attacks.
% The works \citep{murguia2020security,hashemi2022codesign,escudero2023safety} develop filtering techniques to guarantee safety, without being focused on stealthy covert attacks.

Multiplicative watermarking (mWM) has been proposed by the authors as a diagnosis technique \citep{ferrari2020switching}. mWM consists of a pair of filters on each communication channel between the plant and its controller; the scheme is affine to weak encryption, whereby ``encoding'' and ``decoding'' are done by changing signals' dynamic characteristics through inverse pairs of filters. This enables original signals to be recovered exactly, and thus does not lead to performance degradation.
% A multiplicative watermark is an affine to a weak encryption technique, through which the signal is ``encoded'' by a filter, changing its dynamic behavior. The use of inverse pairs means that the original signal can be recovered, through ``decoding'' via an inverse filter. As such, differently to techniques based on additive watermarking, no performance is lost due to the injection of noise, and there are no bandwidth limitations.

%\subsection{Contributions}
One of the critical features of multiplicative watermarking is that to detect stealthy attacks, the mWM filter parameters must be switched over time. In this paper, an algorithm to optimally design the mWM parameters after a switching event is presented, enhancing detection performance, without changing the switching time.
% This is done without changing the switching time, which is taken as given.

\textcolor{black}{
To formalize the filter design problem, we suppose the defender is interested in optimal performance against adversaries injecting covert attacks with matched system parameters \citep{smith2015covert}, including the mWM parameters prior to the switch. This scenario represents a worst case where malicious agents can take full control of the system while remaining undetected.
Thus, the attack strategy is explicitly included within the formulation of the closed-loop system, and the mWM filters are chosen by solving an optimization problem minimizing the attack-energy-constrained output-to-output gain (AEC-OOG) \citep{anand2023risk}, a variation of the output-to-output gain proposed in  \cite{teixeira2015strategic}.
}
The main contributions of this paper are:
% We consider an adversary injecting a covert attack with matched system parameters \citep{smith2015covert}, i.e., an attacker with full knowledge of the control system parameters, including those of the mWM filters before the switch. This scenario is taken as a worst case, as it has been shown that this class of attacks can be made stealthy. To quantitatively define a cost, the output-to-output gain (OOG) \citep{teixeira2015strategic} is leveraged,
% a metric introduced to evaluate the impact of an additive attack in a control system. %Specifically, OOG evaluates the worst-case performance loss that an attacker injecting an undetectable attack can obtain. 
% Here, the maximum performance loss caused by a stealthy adversary with limited energy is taken, the attack-energy-constrained OOG (AEC-OOG) \citep{anand2023risk}. The main contributions of this paper are:
\begin{enumerate}
%[label=\alph*.]
\item The problem of optimally designing the switching mWM filters is formulated as an optimization problem, with the AEC-OOG is taken as the objective;%where the AEC-OOG is taken as the impact metric; 
\item The worst-case scenario of a covert attack with exact knowledge of plant and mWM filter parameters is embedded within the design problem;
% The optimization problem is defined to incorporate the worst-case scenario of a covert attack with exact knowledge of plant and mWM filter parameters;
\item The feasibility of the optimization problem is shown to be dependent only on stability conditions; 
\item A solution scheme is proposed to promote randomization of the mWM filter parameters such that an eavesdropping adversary cannot remain stealthy.
\end{enumerate} 

This builds on the results of \cite{ferrari2020switching}, where the focus was on the design of the switching protocols, rather than the parameters themselves.
Compared to previous work \citep{gallo2021design}, this paper introduces an optimization problem which is always feasible (thanks to the use of AEC-OOG in the objective), while also considering a more sophisticated class of covert attacks, where the presence of watermark is known to the adversary. 
Moreover, this paper poses a different objective than \citep{zhang2023hybrid}; indeed, while \citep{zhang2023hybrid} provided a design strategy to ensure certain privacy properties, in this paper we address the problem of optimal parameter design following a switching event.


%\subsection{Organization}
The rest of the paper is organized as follows. 
After formulating the problem in Section~\ref{sec:PF}, we propose our design algorithm in Section~\ref{sec:main}, and analyze its properties. It is then evaluated through a numerical example in Section~\ref{sec:NE}, and concluding remarks are given Section~\ref{sec:Con}.
% We provide the problem background in Section~\ref{sec:PF}. We formulate the design problem in Section~\ref{sec:main}, together with an analysis of its properties. The proposed algorithm is evaluated through a numerical example in Section \ref{sec:NE}. Concluding remarks are offered in Section \ref{sec:Con}.

A wealth of research exists looking at the effects of AI companions on humans, for example \citet{Brandtzaeg2022AIfriend, xie2022attachment}. Our paper instead focuses on evaluating the biases and stereotypes that chatbots perpetuate as it becomes increasingly important to mitigate their impacts.

Metrics play a crucial role in assessing {LLM}s, and a range of papers have produced quantitative evaluations of these models \citep{nangia-etal-2020-crows, dhamalabold2021, bellem2024are, wan2023biasasker}. Through the lens of gender, extensive work has been done on creating a metric for occupational bias \citep{kirk2024box, rudinger-etal-2018-wino}. \citet{bai2024measuring} is one of few papers that focus on more underlying gender biases in that it studies implicit (unintentional, automatic) rather than explicit (intentional, deliberate) bias. It does this by using the Implicit Association Test (IAT), commonly used for human biases, and modifies it to {LLM}s.

\subsection{Persona Bias in LLMs}

Research into {AI} personas find that, generally, the design and implementation of personas result in models reflecting existing human biases, as shown by \citet{cheng-etal-2023-marked}. They generated personas with different ethnicities and genders and then had the LLM describe itself in that personas voice. This output is compared to the unmarked default persona descriptions, i.e., White and Man, by finding words that statistically distinguish the two groups and comparing the generated descriptions to human-created ones. The results show that models positively stereotype and assume resilience in marked groups much more heavily than unmarked ones and much more often than humans do. \citet{wan-etal-2023-stochastic} aimed to categorise and measure ‘persona biases’ by creating a UniversalPersona dataset of generic and specific personas. These personas are measured against harmful expression (offensiveness, toxic continuation, and regard) and harmful agreement metrics (stereotype and toxic agreement). Findings show that models have fairness issues when taking on the role of a persona. This work is a continuation of that by \citet{deshpande-etal-2023-toxicity}, which shows that assigning a specific persona can increase toxicity up to six-fold. 

To uncover more implicit bias, \citet{gupta2024bias} evaluates the unintended effects of persona assignment by measuring the reasoning capability of different models on different tasks. The results are clear; although ChatGPT will unilaterally reply that there is no difference in the maths problem-solving skills between a physically-abled and disabled person, when adopting the identity of a physically-disabled person, it outputs that because of its disability, it is unable to perform calculations. The work by \citet{plaza2024angry} evaluates a more inferred bias that assumes women are more emotional than men, which {LLM}s seem to agree with; sadness is overwhelmingly linked with women, anger with men.

To date, no work has studied how assigning gendered personas to a model with an implied relationship with its user impacts model responses. Not acknowledging the user's role disregards the topic of sycophancy -- where {LLM}s may echo the opinions of the users they interact with. \citet{huang2024trustllm} and \citet{xu2024earthflatbecauseinvestigating} show that assigning the user a persona and then prompting the model with questions tends to have the model giving responses that would align with the user's persona. However, there is a research gap in how sycophancy may change when assigning a persona to the model system. The role of sycophancy is an essential question when focusing on {AI} companions, as the relationship between user and model is, at its core, intertwined \citep{sharma2023understandingsycophancylanguagemodels}.



\section{Database Construction and Analysis}
In this section, we will describe the database construction and analysis in detail.
\begin{figure}[b]
    \centering
    \includegraphics[width = 0.48\textwidth]{Figs/pie.pdf}
    \caption{The Pie Chart of our used Prompt, which contains 11 challenge categories and 12 scene categories.}
    \label{fig:pie}
\end{figure}
\begin{figure*}[t]
    \centering
    \subfigure[3D assets generated by the prompt: ``a harp without any strings'']{\begin{minipage}[t]{\linewidth}
                \centering
                \includegraphics[width = 0.98\linewidth]{Figs/gallery-004.pdf}
                \end{minipage}}
                
    \subfigure[3D assets generated by the prompt: ``a pair of brown suede shoes'']{\begin{minipage}[t]{\linewidth}
                \centering
                \includegraphics[width = 0.98\linewidth]{Figs/gallery-012.pdf}
                \end{minipage}}
                
    \caption{Sample 3D assets from the AIGC-T23DAQA database, generated by Dreamfusion \cite{poole2022dreamfusion}, LatentNerf \cite{metzer2023latent}; Magic3D \cite{lin2023magic3d}, Prolificdreamer \cite{wang2024prolificdreamer}; SJC\cite{wang2023score}, TextMesh \cite{tsalicoglou2023textmesh} with the same input prompt respectively. (a) 3D assets generated by the prompt ``a harp without any strings''. (b) 3D assets generated by the prompt ``a pair of brown suede shoes''. This clearly shows that the visual quality of assets generated by different models varies greatly.}
    \label{fig:gallery}
    \vspace{-0.5cm}
\end{figure*}
\vspace{-15pt}
\subsection{Prompt Selection}
Compared to AIGC IQA and VQA databases, constructing text-to-3D asset quality assessment database mainly faces two difficulties: 1) The process of generating 3D asset from text is currently time-consuming, typically requiring 1 to 6 hours to generate one 3D asset. 2) The subjective experiment for evaluating generated 3D asset is also time-consuming, since subjects need to observe from whole directions and assess from multiple perspectives. Therefore, our constructed database is a enormous contribution to the field. First of all, meticulous prompts selection is important for text-to-3D asset quality assessment database construction. The selected prompts need to cover a wide range of real user inputs with a relatively small pool. PartiPrompts \cite{yu2022scaling} comprises 1600 varied English prompts designed to comprehensively assess and test the limits of text-to-image synthesis models. Following previous research \cite{wang2023aigciqa2023} we extracted 170 prompts from PartiPrompts, spanning 11 challenge categories and 12 scene categories. The distribution of selected scene and challenge categories is depicted in the pie chart of Fig. \ref{fig:pie}, which manifests that the prompts in our dataset exhibit a high level of scene diversity and encompass a broad spectrum of challenges.
\vspace{-15pt}
\subsection{3D Asset Generation}

To ensure asset diversity, AIGC-T23DAQA database contain six representative text-to-3D asset generation models. These current models typically comprise a 2D image generation module and a 3D asset representation module. When compared to other generation models, the diffusion model delivers exceptional results, establishing itself as the preferred foundational module for generating 2D images within these methodologies. For the 3D asset representation module, a variety of approaches are employed, including NeRF, Instatn-ngp, \textit{etc}. Dreamfusion \cite{poole2022dreamfusion} utilizes mip-NeRF 360 for 3D asset representation, while LatentNerf \cite{metzer2023latent} opts for vanilla NeRF. SJC \cite{wang2023score} employs voxel radiance fields to represent 3D asset, thereby enhancing the speed of the generation process. Conversely, TextMesh  \cite{tsalicoglou2023textmesh}, Magic3D \cite{lin2023magic3d}, and Prolificdreamer \cite{wang2024prolificdreamer} adopt a coarse-to-fine strategy. They commence with coarse 3D asset representations, using vanilla NeRF and Instatn-ngp, respectively, and subsequently refine the differentiable mesh into a fine representation. 
\begin{figure}[!t]
    \centering
    \subfigure[Prompt: ``an ostrich''.]{\begin{minipage}[t]{0.48\linewidth}
                \centering
                \includegraphics[width = 0.94\linewidth]{Figs/q0.pdf}
                \end{minipage}}
    \subfigure[Prompt: ``a comic about a boy and a tiger''.]{\begin{minipage}[t]{0.48\linewidth}
                \centering
                \includegraphics[width = 0.94\linewidth]{Figs/q1.pdf}
                \end{minipage}}
                
    \subfigure[Prompt: ``a fish without eyes''.]{\begin{minipage}[t]{0.48\linewidth}
                \centering
                \includegraphics[width = 0.94\linewidth]{Figs/a0.pdf}
                \end{minipage}}
    \subfigure[Prompt: ``a large present with a red ribbon to the left of a Christmas tree''.]{\begin{minipage}[t]{0.48\linewidth}
                \centering
                \includegraphics[width = 0.94\linewidth]{Figs/a1.pdf}
                \end{minipage}}
                
    \subfigure[Prompt: ``a robot cooking''.]{\begin{minipage}[t]{0.48\linewidth}
                \centering
                \includegraphics[width = 0.94\linewidth]{Figs/c0.pdf}
                \end{minipage}}
    \subfigure[Prompt: ``a bundle of blue and yellow flowers in a vase''.]{\begin{minipage}[t]{0.48\linewidth}
                \centering
                \includegraphics[width = 0.94\linewidth]{Figs/c1.pdf}
                \end{minipage}}
    \caption{Illustration of the differences between the three dimensions of quality ,authenticity, and text-3D correspondence. In each subfigure, the images in the top row are significantly better than the that in bottom row in terms of two perspectives, while similar or worse in terms of another perspective. (a) and (b) show examples that the authenticity and correspondence scores of the top images are higher, while the quality is similar. (c) and (d) show examples that the quality and correspondence scores of the top images are higher, while the authenticity is similar or lower. (e) and (f) show examples that the quality and authenticity scores of the top images are higher, while the correspondence is similar or lower. }
    \label{fig:xx}
    \vspace{-0.5cm}
\end{figure}
The generation process of text-to-3D asset was executed using open-source code \cite{threestudio2023} with default weights and configurations, resulting in a collection of 1020 instances (170 prompts × 6 models) of text-to-3D assets. Some examples of the 3D assets generated by the six text-to-3D asset generation models are illustrated in Fig.  \ref{fig:gallery}. Subsequently, we discarded 51 instances of failed asset generation, defined as cases where the entire spatial domain remained empty after-generation. Due to computational constraints, it is hard to render a generated 3D asset in real-time and evaluate it. Thus, we followed the method used in \cite{zhang2023eep} and projected the 3D asset into videos then conducted evaluation. This manipulation yielded 969 360-degree surround projection videos centered on the generated text-to-3D asset. Each video consists of comprised 120 frames with a resolution of 512 × 512 pixels and cumulative a total duration of 4 seconds. These projection videos were used for the subsequent subjective experiment.

\vspace{-10pt}
\subsection{Subjective Experiment}
\begin{figure}[!t]
    \centering
    \subfigure[Quality]{\begin{minipage}[t]{\linewidth}
                \centering
                \includegraphics[width = 0.98\linewidth]{Figs/assessment-perspectives-q.pdf}
                \end{minipage}}
                
    \subfigure[Authenticity]{\begin{minipage}[t]{\linewidth}
                \centering
                \includegraphics[width = 0.98\linewidth]{Figs/assessment-perspectives-a.pdf}
                \end{minipage}}

    \subfigure[Correspondence]{\begin{minipage}[t]{\linewidth}
                \centering
                \includegraphics[width = 0.98\linewidth]{Figs/assessment-perspectives-c.pdf}
                \end{minipage}}
    \caption{Illustration of the text-to-3d assets from the perspectives of quality, authenticity, and text-asset correspondence. The examples of good, fair, and poor quality are depicted in the first to third rows of (a). The examples illustrating good, fair, and poor authenticity are displayed in the first to third rows of (b). (c) showcases examples of good, fair, and poor correspondence generated by prompts ``a harp without any strings'', ``a knight holding a long sword'', and ``A cartoon tiger face''. }
    \label{fig:ap}
    \vspace{-0.4cm}
\end{figure}
To collect human visual preferences for text-to-3D assets, we further conducted a subjective evaluation experiment. As highlighted in prior AI generated asset quality assessment studies \cite{wang2023aigciqa2023, yang2024aigcoiqa2024}, the degradations of AI generated asset are significantly different from human captured or created asset, which need to be evaluated from multiple perception perspectives. \textcolor{black}{Based on traditional 3D quality assessment, which evaluates texture, color, and other visual quality attributes of the 3D asset, we selected the ``quality'' dimension for evaluation. Similar to AI-generated image and video quality assessment, in addition to assessing the visual quality of the 3D asset, we also need to evaluate its authenticity and correspondence to the text prompt. Therefore, we selected the dimensions of ``authenticity'' and ``correspondence''.} Hence, in this paper, we propose to evaluate human visual preferences for text-to-3D assets from three perspectives, including quality, authenticity, and text-asset correspondence. Fig. \ref{fig:xx} shows the differences between the selected three dimensions, which further manifests the importance, and significance of evaluating text-to-3D assets from multiple perspectives.
\textcolor{black}{Before each subject conducts the subjective experiment, we give a detailed instruction to subjects, which includes explaining to the subject the differences between ``quality'', ``authenticity'' and ``correspondence'' and showing examples of different degrees of each dimension. The ``quality" is the visual quality attribution of 3D asset including texture, color, integrity, etc, while the ``authenticity" refers to whether the 3D asset is consistent with the real world that the subject knows. The ``correspondence" is the alignment between the 3D asset and the input prompt text.} Then, participants were instructed to give their preference scores of text-to-3D assets based on the surrounding 360-degree projection videos. The first dimension for evaluating text-to-3D asset is ``quality'', which mainly evaluates the perception attributes including texture, color, integrity, details \textit{etc.}, analogous to traditional 3D models. Fig. \ref{fig:ap} (a) shows examples of the generated 3D asset with different ``quality'' levels. The second dimension for evaluating text-to-3D asset is ``authenticity'', which evaluates the perception attributes including unrealistic textures, shapes, \textit{etc.} It should be noted that compared to the authenticity attribute generally used in AIGC IQA, the degradation of the authenticity attribute for generated text-to-3D asset generally comes from the unrealistic or inconsistent multiple views. Fig. \ref{fig:ap} (b) shows examples of the generated 3D asset with different ``authenticity'' levels. Similar to AIGC IQA, and AIGC VQA methodologies, the correspondence between text, and 3D asset serves as another critical criterion in assessing text-to-3D asset quality, referred to as ``text-3D asset correspondence''. Fig. \ref{fig:ap} (c) shows examples of the generated 3D asset with different ``correspondence'' levels.
\begin{figure}[t]
    \centering
    \includegraphics[width = 0.48\textwidth]{Figs/gui.png}
    \caption{The illustration of the subjective assessment interface. The subject can evaluate their preferences of the text-to-3D assets, and record the quality, authenticity, correspondence scores with the scroll bars on the right.}
    \label{fig:gui}
    \vspace{-0.5cm}
\end{figure}

\begin{figure*}[t]
    \centering
    \subfigure[Quality MOS distribution.]{\begin{minipage}[t]{0.32\linewidth}
                \centering
                \includegraphics[width = 0.94\linewidth]{Figs/mosz_Quality.pdf}
                \end{minipage}}
    \subfigure[Authenticity MOS distribution.]{\begin{minipage}[t]{0.32\linewidth}
                \centering
                \includegraphics[width = 0.94\linewidth]{Figs/mosz_Authenticity.pdf}
                \end{minipage}}
    \subfigure[Correspondence MOS distribution.]{\begin{minipage}[t]{0.32\linewidth}
                \centering
                \includegraphics[width = 0.94\linewidth]{Figs/mosz_Correspondence.pdf}
                \end{minipage}}
    \caption{Distributions of the MOSs from the perspectives of quality, authenticity, and correspondence, respectively. These distributions exhibiting proposed T23DAQA database cover a wide range in terms of all perspectives. }
    \label{fig:mos}
    \vspace{-0.5cm}
\end{figure*}

We conducted the subjective experiment following the guidance in ITU-R BT.500-13 \cite{other:itu}. The experimental environment was arranged to simulate a typical indoor home setting with standard lighting conditions. The projection videos of text-to-3D asset, accompanied by the corresponding prompts, were presented randomly on a monitor with a resolution of $1920 \times 1080$. The interface, depicted in Fig. \ref{fig:gui}, facilitated viewer interaction, enabling navigation through previous, next, and replay options for the projection videos of the generated 3D asset. Additionally, three sliders ranging from 0 to 5, with a minimum interval of 0.1, were provided for participants to assign scores for quality, authenticity, and correspondence. 17 subjects (10 males and 7 females) participated in the subjective experiment, all possessing normal or corrected-to-normal vision. Each participant received detailed experimental instructions prior to engaging in the subjective evaluation. We divided the conversation of each participant in the subjective experiment into three subsets. For each participant, the database were randomly divided into three subsets, which are used in three subjective tests respectively. Each test lasted around one hour, followed by a 10-20 minutes break in between, and then the next test was performed.
\vspace{-15pt}
\subsection{Data Processing}

We followed the instructions of ITU \cite{other:itu} to conduct the outlier detection and subject rejection. \textcolor{black}{Specifically, for each evaluation dimension, we calculate the kurtosis of the raw subjective quality ratings for each generated 3D asset to determine whether the data follows a Gaussian or non-Gaussian distribution. For Gaussian distributions, a raw score is considered an outlier if it lies more than 2 standard deviations (std) from the mean. For non-Gaussian distributions, a score is deemed as an outlier if it is more than $\sqrt{20}$ standard deviations from the mean. Any subject whose evaluations exceed a 3\% outlier rate in any dimension is excluded from the analysis.} As a result, no subjects were rejected and the rejection ratio is 3\% for all ratings. Subsequently, we converted the raw ratings of the remaining valid subjective scores into Z-scores, which were then linearly scaled to the range of $\left [ 0,100 \right ]$. The final MOS is computed as follows:
\begin{equation}
z_{ij} = \frac{m_{ij}-\mu_{i}}{\sigma_{i}}, \quad z_{ij}^{'} = \frac{100\times(z_{ij} + 3)}{6}
\end{equation}

\begin{equation}
\text{MOS}_{j} = \frac{1}{N} z_{ij}^{'}
\end{equation}
where $m_{ij}$ is the subjective score given by the $i$-th subject to the $j$-th text-to-3D asset, $\mu_{i}$ and $\sigma_{i}$ is the mean score and the standard deviation given by the $i$-th subject respectively, $N$ is the total number of subjects. 
% \begin{figure}[t]
%     \centering
%     \includegraphics[width = 0.48\textwidth]{Figs/assessment-perspectives .pdf}
%     \caption{Illustration of the T23DCs from the perspectives of quality, authenticity, and text-content correspondence. The examples of good, fair, and poor quality are depicted in the first to third rows of (a). The examples illustrating good, fair, and poor authenticity are displayed in the first to third rows of (b). (c) showcases examples of good, fair, and poor correspondence generated by prompts such as "a harp without any strings", "a knight holding a long sword", and "A cartoon tiger face".}
%     \label{fig:ap}
% \end{figure}
\vspace{-15pt}
\subsection{Subjective Data Analysis}
Although a large number of text-to-3D asset generation models have been developed in recent years, the corresponding works that specifically analyze and compare their generation performance are lacking. Considering that the generation quality of the text-to-3D asset is influenced by multiple factors such as prompts, algorithms, \textit{etc}, which leads to diverse perceptual quality and affects the user experience, based on the established AIGC-T23DAQA database, we conduct an in-depth analysis for the collected MOSs from multiple perspectives as follows.

Fig. \ref{fig:mos} demonstrates the distribution of MOS values obtained from subjective experiments. It can be observed that the correspondence distribution surpasses both the quality and authenticity distributions, suggesting that the current generation models learn more towards correspondence while ignoring the quality and authenticity attributes. The reason for this phenomenon is that the current T23DA method utilizes text-to-image models to constrain the correspondence between images rendered from different perspectives and text. These text-to-image models are trained on a large number of text-image pairs and perform well in text-image correspondence, ensuring good correspondence between generated 3D asset and text; However, the text-generated image model cannot guarantee the geometric texture consistency of three-dimensional objects from different perspectives, resulting in the strange geometric shapes and floaters in generated 3D asset. As a result, the quality and authenticity of the generated 3D assets are poorer than those of correspondence. To enhance the overall user preferences in the future, it is more important to improve the quality and authenticity attributes for the generated 3D assets.

Fig. \ref{fig:moscompare} (a) compares the human preference MOSs for different models, including Dreamfusion \cite{poole2022dreamfusion}, LatentNerf \cite{metzer2023latent}; Magic3D \cite{lin2023magic3d}, Prolificdreamer \cite{wang2024prolificdreamer}; SJC\cite{wang2023score}, TextMesh \cite{tsalicoglou2023textmesh}. Fig. \ref{fig:moscompare} (b) compares the human preference MOSs for different prompt length. Prompt length is divided into six intervals on average, with 1-6 on the x-axis representing interval numbers from short to long. We can find from it that: 1) The 3D assets generated by different text-to-3D generation models have significantly different perceptual preferences, and even with the same input prompt, the quality, authenticity, and correspondence vary greatly across different text-to-3D asset methods. Models including Prolificdreamer \cite{wang2024prolificdreamer}, Magic3D \cite{lin2023magic3d}, and Prolificdreamer \cite{wang2024prolificdreamer} exhibit the best quality, authenticity, and correspondence respectively. The reasons for the subjective score differences among different models: From Figure 9 in the manuscript, it can be seen that the best quality, authenticity, and correspondence are Prolificdreamer, Magic3D, and Prolificdreamer respectively. Prolificdreamer uses variational score distillation to instead of score distillation sampling which used in other methods and solve the problems of over-saturation, over-smoothing, and low-diversity. So the Prolificdreamer has better quality and correspondence. Magic3D uses coarse-to-fine strategy to generate 3D asset and a sparse 3D hash grid structure to represent 3D asset, which can reduce the generation of floaters, making generated 3D asset more authenticity. 2) When the prompt is short (1 \& 2), the model is easy to generate high quality, authenticity, and correspondence 3D assets, However, as prompt length increases (3, 4 \& 5), text-to-3D generation models may struggle to meet the requirements of human preferences and the entire prompt, resulting in a decline in the quality, authenticity, and correspondence scores. Finally, when the prompt length is extreme long, the explicit descriptions make the quality, authenticity scores higher, while the correspondence scores are still lower than the prompt length of 1 \& 2. The reasons for subjective score differences in different prompt lengths: When the prompt length is short, the generated 3D asset is less constrained by the text, making it easy to achieve better text asset correspondence. However, as the length increases, the text-asset correspondence decreases; When the prompts are too long, a more detailed description can help the models generate better textures and geometry, resulting in better authenticity and quality.




\begin{figure}[ht]
    \centering

    \subfigure[]{\begin{minipage}[t]{0.9\linewidth}
                \centering
                \includegraphics[width = 0.93\linewidth]{Figs/catplot.pdf}
                \end{minipage}}

    \subfigure[]{\begin{minipage}[t]{0.9\linewidth}
                \centering
                \includegraphics[width = 0.93\linewidth]{Figs/promptplot.pdf}
                \end{minipage}}
                
    \caption{Illustration of the impact of different models and prompt lengths on the perceptual quality of T23DAs respectively. (a) shows the subjective quality, authenticity, and correspondence score of T23DAs with different methods including Dreamfusion, LatentNerf, Magic3D, Prolificdreamer, SJC, and TextMesh respectively. (b) shows the subjective quality, authenticity, and correspondence score of T23DAs with different prompt lengths. Prompt length is divided into six intervals on average, with 1-6 on the x-axis representing interval numbers from short to long.}
    \label{fig:moscompare}
    \vspace{-0.6cm}
\end{figure}



\begin{figure*}
	\centering
	\includegraphics[width = \linewidth]{figure/AgentArena.pdf}
	\caption{\textbf{Stock Trading Workflow in \textit{Agent Trading Arena}.} 
	\textbf{Top:} Workflow of a trading day, including preparation, trading, and post-trading reflection. Agents discuss insights in the chat pool, analyze market trends, execute trades, and refine strategies based on performance.  
	\textbf{Bottom:} Example of agents' interactions in the chat pool and dynamic strategy updates.}
	\label{fig:AgentArena}
	\vspace{-3pt}
\end{figure*}

\section{Proposed Method}

% 核心部分visual representation,

To mitigate the influence of human prior knowledge and memory, we designed a closed-loop economic system~\citep{guo2024economics} called the \textit{Agent Trading Arena}, a zero-sum game simulating complex, quantitative real-world scenarios. The simulation workflow is illustrated in \autoref{fig:AgentArena} and further detailed in \autoref{appendix_arena}. In the \textit{Agent Trading Arena}, agents can invest in assets, earn dividends from holding assets, and pay daily expenses using virtual currency. The agent with the highest total return wins the game.

\subsection{Agent Trading Arena}

\paragraph{Structure of Agent Trading Arena.} 

To eliminate external knowledge biases, asset prices are determined by a bid-ask system, reflecting the prices at which buyers and sellers are willing to transact. The system evolves solely based on agents' actions and interactions, without external influences. This design ensures that the outcomes of agents' actions are not immediately apparent but unfold gradually, influenced by other agents' decisions.

To encourage active participation, a dividend mechanism is introduced. There are two primary sources of income in this system: capital gains from asset price differentials and dividends from holding assets. Dividends for each asset are distributed according to a predefined ratio, serving as an implicit anchor for asset prices. Agents holding more low-cost assets receive higher dividends. To prevent passive asset holding until the end of the game, agents must pay a daily capital cost proportional to their total wealth. These expenses are offset by asset dividends, and only agents with sufficient low-cost assets can cover costs. Under the pressure of significant daily expenses, agents must act swiftly and strategically, triggering frequent trades and price fluctuations to stimulate market activity. This dynamic mechanism ensures fairness in the zero-sum game while preventing agents from relying on fixed strategies to find optimal solutions.

\vspace{-3pt}

\paragraph{Agents Learn and Compete in Arena.}

The zero-sum game structure is crucial to eliminating the possibility of a universally optimal strategy. In fixed scenarios with a static optimal solution, agents could rely on predefined rules or memory-based approaches, bypassing adaptive decision-making. The zero-sum game ensures that there is no universally correct solution, with outcomes evolving dynamically based on agent interactions and competition. This design forces agents to continually adapt, learn from feedback, and develop context-dependent strategies, promoting deeper environmental exploration and preventing reliance on static or memory-driven solutions.

In the \textit{Agent Trading Arena}, agents are unaware of implicit rules, except for the objective to maximize their virtual wealth throughout the simulation. To win this zero-sum game, agents must effectively learn from experience, decipher hidden game rules, and develop strategies to counter competitors. This requires the ability to comprehend numerical feedback, formulate enduring strategies, and make informed decisions. Unlike other mathematical reasoning problems, the results of their actions unfold gradually and dynamically. Moreover, agents are easily misled by erroneous information from competitors, hindering their ability to discern strategic cues from competitors' textual data. Importantly, agents remain unaware of these implicit rules, so applying real-world knowledge does not benefit their performance. Therefore, agents must rely on experiential learning to decipher the hidden game rules and ultimately achieve victory.

\subsection{Types of Numerical Data Input}

\paragraph{Limitations of Textual Numerical Data.}

In the \textit{Agent Trading Arena}, the generated stock data is stored in numerical format. When used directly as input to an LLM, the models often struggle to interpret numerical data accurately or make sound decisions. To mitigate this, we convert the data into textual formats~\citep{numerical_text, long_text}, enhancing semantic features and clarifying output requirements to improve the models' understanding. During interactions, the LLMs process stock prices, trading volumes, and market indices presented as textual numerical data.

\begin{figure*}
	\centering
	\includegraphics[width = \linewidth]{figure/v_t.pdf}
	\caption{\textbf{Textual and Visual Representations of Corresponding Inputs and Outputs.} The left images display the agent’s Buy and Sell trading records, daily trade prices, and K-line charts for three stocks. The output from visual inputs (bottom right) captures overall stock trends and long-term behavior, while the output from textual inputs (top right) focuses on specific current prices.}
	\label{textual_visualized}
	\vspace{-3pt}
\end{figure*}

However, this textual approach reveals significant limitations. While the data is presented clearly, LLMs tend to focus excessively on specific values rather than identifying long-term trends or global patterns. They also struggle with understanding correlative relations and percentage changes, limiting their ability to assess differences and identify connections between data points. When analyzing time-series data with complex patterns, LLMs often fixate on individual data points, overlooking overarching relations. This issue is evident in the analysis output in the top-right corner of \autoref{textual_visualized}, where LLMs' focus on individual values impedes their ability to generalize, reducing their capacity to extract meaningful global insights.

Additionally, LLMs often overemphasize recent data while undervaluing historical information, even when prompted to consider its importance. This prevents them from effectively integrating past data and recognizing long-term patterns, complicating their understanding of numerical relations and trends. These challenges highlight the need for improved mechanisms to process numerical relations, identify global trends, and derive deeper insights from textual numerical data.

\vspace{-3pt}

\paragraph{Potential of Visual Numerical Data.}

Since textual numerical data often leads LLMs to focus on local details while neglecting broader relations, we investigated whether visual representations, such as scatter plots, line charts, and bar charts, could help LLMs better understand overall trends, similar to human reasoning. Thus, we transition from textual numerical data inputs to visualized formats ~\citep{storyllava}. As demonstrated in the bottom-right corner of \autoref{textual_visualized}, visual representations enable LLMs to more effectively grasp global trends, patterns, and relations that are often difficult to discern from textual numerical data alone.

These findings highlight the advantages of structured, visual numerical data, indicating that this format allows LLMs to more intuitively and comprehensively understand complex data, better capturing overall fluctuations, whereas text tends to focus on local details. By combining visualization and textual representations, LLMs not only overcome the challenges of relations in time-series data but also demonstrate better performance in identifying long-term trends and global patterns, while still attending to local details.

\subsection{Reflection Module}

We propose a strategy distillation method, illustrated in \autoref{fig:reflection}, that delivers real-time feedback to LLMs by analyzing both descriptive textual and visual numerical data. This enables the generation of new strategies and optimization of action plans. The approach allows agents to evaluate their results, refine strategies, and adapt continuously based on feedback. The process begins with assessing the day’s trajectory memory and associated strategies using an evaluation function. The strategic generation process leverages contrastive analysis of peak and nadir performers from the evaluation phase, creating bidirectional learning signals that inform subsequent iterations. This iterative cycle ensures continuous strategy evolution, fostering sustained improvement in decision-making.

\begin{figure}[t]
	\centering
	\includegraphics[width = \linewidth]{figure/reflection.pdf}
	\caption{\textbf{Design of the Reflection Module.} The process evaluates daily trajectory memory and strategies (top right), then generates new strategies (center) based on evaluation, environmental feedback (bottom right), and feedback from the 5 top- and bottom-performing strategies. Stock visualization (bottom left) enhances reflection, driving continuous improvement.}
	%The process evaluates daily trajectory memory and strategies, generating new strategies based on positive and negative feedback from the top- and bottom-performing strategies. Stock visualizations (bottom left) further enhance the reflection process, reinforcing continuous strategy refinement.}
	\label{fig:reflection}
	\vspace{-3pt}
\end{figure}

% We propose a strategy distillation method, illustrated in \autoref{fig:reflection}, that provides real-time feedback to LLMs by analyzing both descriptive textual and visualized numerical data. This enables the generation of new strategies and the optimization of action plans. The approach allows agents to assess their results, refine strategies, and continuously adapt based on feedback. The process begins by evaluating the day's trajectory memory and associated strategies using an evaluation function. From this assessment, new strategies are generated by selecting the top-performing and lowest-performing strategies, offering both positive and negative feedback. This iterative cycle ensures continuous strategy evolution, driving sustained improvement in decision-making.

The reflection module plays a crucial role in refining strategies by offering real-time feedback. It analyzes both descriptive textual and visual numerical data to generate new strategies and optimize action plans. Within the \textit{Agent Trading Arena}, the reflection module is triggered regularly to consolidate daily trading records and evaluate the effectiveness of strategies, refining both successful and unsuccessful experiences to guide future decisions. Ineffective strategies are stored in a strategy library for future reference, allowing agents to review and learn from past experiences. Further details can be found in \autoref{appendix_arena}.



\newpage
\section{Experiment}\label{sec-experiment}
\subsection{Experimental Setup}
We briefly introduce experimental settings to verify our proposed MoR, including Datasets \& Baselines, Implementation Details, and Evaluation Metrics. More details are in Appendix~\ref{app-expr-setting}.

\textbf{Datasets \& Baselines:} We use three TG-KBs from STaRK~\cite{wu2024stark} covering three knowledge domains, including E-commerce Products (Amazon), Academic Papers (MAG), and Biomedicine (Prime). We compare our MoR with baselines established by~\citet{wu2024stark} and categorize them into textual/structural/hybrid-based ones. More recent state-of-the-art hybird retrieval approaches fro TG-KBs such as KAR~\cite{xia2024knowledge} and MFAR$^{*}$~\cite{li2024multi} are also compared.


\textbf{Implementation Details:} 
To enhance the planning capability of our planning module, we fine-tune the Llama 3.2 (3B) on 1000 sampled queries with their corresponding ground-truth planning graphs, serving as the textual graph generator. In the absence of ground-truths, we synthesize them using LLMs. For the Prime dataset, we empirically find that directly prompting LLMs can hardly generate accurate planning graphs due to the lack of biomedical domain knowledge~\cite{Shen2024TagLLMRG}. Therefore, we adopt an alternative approach. First, we instruct LLMs to extract triplets from each query and then construct the planning graphs by merging triplets with shared entities. 
During mixed traversal, textual matching can be implemented using any lexical or semantic methods. For this study, we employ BM25 for Amazon and MAG and fine-tune a contriever to complement the biomedical knowledge for Prime.
To initialize the structural traversal, we employ textual matching to locate the top 5 nodes that are most relevant to the query as seeds. Additionally, at each layer, we incorporate the top 10 nodes retrieved via textual matching and append them to the current candidate set for the next round of traversal. Notably, due to the uncertainty of LLMs, the generated planning graphs can be invalid. In this case, we will directly conduct textual matching to retrieve candidates. For our ablations without reranker, we employ Ada-002~\cite{wu2024stark} with cosine similarity as the scorer to rank candidates for evaluating performance.

\textbf{Evaluation Metrics:}
We follow~\citet{wu2024stark} for evaluation by reporting Hit@1 (H@1), Hit@5 (H@5), Recall@20 (R@20), and mean reciprocal rank MRR to evaluate in the full spectrum. 


 

\newpage
\subsection{Overall Retrieval Performance}
We compare MoR with other baselines on three TG-KBs in Table~\ref{tab-merged}. Generally, hybrid methods, AvaTAR, KAR, MFAR$^{*}$, and our MoR, achieve better performance than purely textual or structural methods owing to their ability to integrate both structural and textual knowledge. 
Among all baselines, our proposed MoR achieves the overall best performance with a substantial margin on average, with the first ranking on MAG and the second ranking on Amazon/Prime datasets. This demonstrates the effectiveness of our proposed mixture of structural and textual knowledge retrieval. 
Textual retrieval performs better on Amazon than on MAG, suggesting that Amazon queries rely more on textual knowledge. In contrast, its weaker performance on MAG is due to MAG's lower textual richness and stronger structural signals. This disparity aligns with the distribution analysis presented by~\citet{wu2024stark} and supports our hypothesis that queries in different TG-KB datasets require varying desires for textual and structural knowledge. Meanwhile, structural retrieval methods such as conventional knowledge graph-based ones perform poorly because they are designed for graphs with minimal textual information compared to TG-KBs.
Different from Amazon and MAG, all existing methods without supervised tuning (e.g., Ada-002) exhibit significantly lower performance on Prime. This is due to the extreme domain expertise required in biology, where word-count-based, pre-trained textual similarity-based, and even more powerful LLMs are all poorly applicable here. Through fine-tuning, MFAR$^{*}$ and our proposed MoR generally achieve better performance, demonstrating the necessity of domain-specific knowledge for answering queries in knowledge-intensive domains. 




\newpage
\subsection{Ablation Study}
After verifying the superiority of MoR, we conduct ablation studies to assess its different components, including module and feature ablation.

\subsubsection{Module Ablation}


To assess the contribution of each module in MoR, namely, Text Matching-based Retrieval, Neighborhood-Fetching-based Structural Retrieval, and Reranker, we conduct a series of ablation experiments. First, we remove the Reranker, resulting in the variant MoR$_{\text{w/o R}}$. On top of that, we further separately eliminate Text Retrieval and Structural Retrieval, yielding MoR$_{\text{w/o RT}}$ and MoR$_{\text{w/o RS}}$, respectively.
As shown in Table~\ref{tab-merged}, the complete MoR framework consistently achieves the highest performance across all datasets, demonstrating the synergistic effect of the Textual Retriever, Structural Retriever, and Reranker.
After removing Reranker, MoR$_{\text{w/o R}}$ exhibits a consistent performance drop across all datasets and evaluation metrics. This underscores the importance of the Reranker in refining retrieval by filtering noisy candidates from the intermediate reasoning stage. 
Eliminating Text Retrieval, i.e., MoR$_{\text{w/o RT}}$, leads to a notable performance drop on Amazon but an unexpected improvement on MAG. This suggests that while textual knowledge benefits Amazon, it introduces misleading hard negatives that compromise the ranking method (e.g., Ada-002) for MAG. Conversely, removing Structural Retrieval, MoR$_{\text{w/o RS}}$, results in a slight performance decrease further on MAG, reinforcing the importance of structural knowledge in MAG-related queries.
%
These results underscore the Reranker's crucial role in adaptively harmonizing, balancing, and selecting knowledge from both structural and textual retrieval experts.






\begin{table}[t!]
\small
\setlength\tabcolsep{4.5pt}
\centering
\begin{tabular}{l|ccc|cccc}
\toprule
\textbf{Dataset} &\textbf{TF} & \textbf{SF} & \textbf{TI} & \textbf{H@1} & \textbf{H@5} & \textbf{R@20} & \textbf{MRR} \\ \midrule
\multirow{7}{*}{\textbf{MAG}} 
& \cmark & \xmark & \xmark & 48.96 & 73.02 & 72.44 & 59.79 \\
&      \xmark            & \cmark       &         \xmark         & 18.79 & 41.91 & 52.85 & 29.84 \\
&        \xmark          &         \xmark         & \cmark       & 18.16 & 41.53 & 52.78 & 29.31 \\
\cline{2-8}
& \cmark       & \cmark       &    \xmark              & 58.04 & 77.14 & 74.42 & 66.75 \\
& \cmark       &        \xmark          & \cmark       & \underline{58.16} & \underline{77.59} & \underline{74.96} & \underline{66.85} \\
&          \xmark        & \cmark       & \cmark       & 17.93 & 38.01 & 46.79 & 27.48 \\
\cline{2-8}
& \cmark       & \cmark       & \cmark       & \textbf{58.19} & \textbf{78.34} & \textbf{75.01} & \textbf{67.14} \\ \midrule
\multirow{7}{*}{\textbf{Amazon}}    
& \cmark       &      \xmark            &       \xmark           & \underline{51.21} & \underline{74.05} & \underline{59.79} & \underline{61.27} \\
&        \xmark          & \cmark       &      \xmark            & 8.09  & 24.48 & 25.62 & 16.94 \\
&         \xmark         &      \xmark            & \cmark       & 5.84  & 16.62 & 12.94 & 11.57 \\
\cline{2-8}
& \cmark       & \cmark       &      \xmark            & 50.91 & 73.38 & 59.58 & 61.15 \\
& \cmark       &         \xmark         & \cmark       & 51.09 & 73.56 & 59.61 & 61.14 \\
&            \xmark      & \cmark       & \cmark       & 8.09  & 24.48 & 25.62 & 16.94 \\
\cline{2-8}
& \cmark       & \cmark       & \cmark       & \textbf{52.19} & \textbf{74.65} & \textbf{59.92} & \textbf{62.24} \\ \bottomrule
\end{tabular}
\caption{Ablation study investigating the importance of three features, Textual Fingerprint (\textbf{TF}), Structural Fingerprint (\textbf{SF}), and Traversal Identifier (\textbf{TI}), of the traversal trajectories used in our Structure-aware Reranker.}
\label{tab-feature-ablation}
\vspace{-2ex}
\end{table}



\subsubsection{Feature Ablation}
The above ablation study highlights the crucial role of Structure-aware Reranker in adaptively integrating structural and textual knowledge. To further analyze the contributions of its three key features, \textbf{Textual Fingerprint (TF)}, \textbf{Structural Fingerprint (SF)}, and \textbf{Traversal Identifier (TI)} defined in Section~\ref{sec-organizing}, we conduct a feature ablation analysis and report retrieval performance across different feature configurations in Table~\ref{tab-feature-ablation}.
%Overall and individual performance
Overall, using three features together yields the best performance on both MAG and Amazon, highlighting their synergistic effect. Individually, TF contributes the most and outperforms SF and TI on both datasets. 
The reason is that based on the definition in Section~\ref{sec-organizing}, TF directly captures the relevance between the query and the retrieved nodes along the trajectory, whereas SF and TI primarily characterize the structural patterns and retrieval types, serving more as complementary factors. Therefore, equipping TF with these complementary factors (i.e., SF or TI) yields around 10\% additional gains on MAG. This is because SF and TI help the reranker selectively emphasize the relevance scores given by TF for certain nodes along the path. However, this boost is not observed on Amazon. We hypothesize that the textual knowledge needed there is predominantly derived from the final node on each path, making the structural cues provided by SF and TI less beneficial and even prone to overfitting. A deeper analysis to further justify this hypothesis is in Section~\ref{sec-further}. Overall, these findings underscore the varying importance of structural features in ranking across datasets.



\begin{table}[t!]
\small
\setlength\tabcolsep{4pt}
\centering
\begin{tabular}{l|ccc|ccc}
\toprule
\multirow{2}{*}{\textbf{Feature}} & \multicolumn{3}{c|}{\textbf{MAG}} & \multicolumn{3}{c}{\textbf{Amazon}} \\

 & H@1 & R@20 & MRR & H@1 & R@20 & MRR \\
\midrule
Last Node & 49.91 & 73.49 & 59.92 & 50.36 & 59.62 & 61.05   \\
Full Path & \textbf{58.19} & \textbf{75.01} & \textbf{67.14} & \textbf{52.19} & \textbf{59.92} & \textbf{62.24}   \\
\bottomrule
\end{tabular}
\caption{Comparing reranking performance using last node in the retrieved trajectory and the whole trajectory.}
\label{tab-Reranker-ablation}
\vspace{-2ex}
\end{table}

\begin{figure}[t!]
    \centering
    \includegraphics[width=0.49\textwidth, height = 0.22\textwidth]{figures/query-pattern-20250215.png}
    \vspace{-4.5ex}
    \caption{Imbalance number of queries and performance of different retrievers across different logical structures.}
    \label{fig-analysis}
    \vspace{-3ex}
\end{figure}





\subsection{Further Analysis}\label{sec-further}
This section understands MoR’s behavior by examining three questions, each of which enriches our insight into MoR’s functionality and offers novel perspectives inspiring future query retrieval research.

\textbf{Do structure signals affect reranking?}
To assess the impact of trajectory information on the Reranker's decision-making, we introduce a node-based Reranker that constructs trajectory features using only TF/SF/TI of the last node. In Table~\ref{tab-Reranker-ablation}, the path-based Reranker outperforms the node-based variant, especially on MAG. This highlights the critical role of trajectory features/structural knowledge in reranking. The minor performance boost on Amazon after switching to the full path trajectory indicates its textual knowledge preference over the last node rather than the whole trajectory.


\textbf{How does MoR perform on different logical structures?}
Figure~\ref{fig-analysis} shows the average performance of MoR on each query group categorized by their logical structures, where "Others" refer to queries with undefined logical structures in~\citet{wu2024stark} MoR consistently outperforms structural and textual retrievers across different logical structures. Among all queries, MoR performs the worst on "P → P" queries due to the ambiguity, although well-known, uniquely caused by repeated product entities from multi-step traversal.
The average-performing ``Others" group underscores the utility of diverse planning strategies for the same query.
Lastly, the skewed query distribution and retrieval performance across planning patterns reflect the varying nature of real-world planning needs. We hope these insights inspire research on data-centric reasoning designs and error control of planning.


\begin{figure}[t!]
    \centering
    \includegraphics[width=0.5\textwidth]{figures/heatmap-20250215.pdf}
    \vspace{-3ex}
    \caption{Saliency map visualization of query attention over three entities along the retrieved paths}
    \label{fig-map}
    \vspace{-2ex}
\end{figure}

\textbf{Does MoR indeed adaptively leverage the trajectory knowledge?} To understand how our proposed reranker prioritizes candidates in the Top-K results, we visualize the saliency map by computing the gradient of ranking scores with respect to the textual fingerprint (TF) of three nodes along the traversed path, which quantifies their importance for answering a given query. Figure~\ref{fig-map} illustrates this by analyzing trajectories for 100 ground-truth candidates across 100 queries on the Amazon and MAG datasets. Each dimension corresponds to a traversed node, with the final one representing the candidate itself. 
While the saliency score is concentrated in the last dimension for Amazon, 
MAG exhibits a more evenly distributed saliency pattern, where multiple nodes along the path contribute significantly to ranking score computation. This suggests that structural knowledge is more critical for answering queries in MAG, aligning with the previously observed lower performance of purely textual retrieval on MAG in Table~\ref{tab-merged}. Further case studies explain why the reranker attends different nodes for different queries. In Figure~\ref{fig-map}(a), the reranker favors the last two dimensions as the rich textual restriction (i.e., "Northwest Company..." and "NFL Seattle...") aids in identifying the correct node at the corresponding reasoning step, as discussed in Section~\ref{sec-reasoning}. The correct nodes, having higher similarity scores with the query, help guide the retrieval process toward the ground truth.
Conversely, in Figure~\ref{fig-map}(b),
since the first node ("University of Lausanne") helps narrow the search space and the last node ("frameless...") further filter candidates, both nodes have high saliency scores. Overall, our findings demonstrate that the reranker dynamically adapts its reliance on structural and textual knowledge depending on the dataset and query. 


Software development is increasingly conceived as a collaboration activity between developers and AIs. Indeed, IDEs already implement features to enable interactive development, with AI suggesting implementations that are reused by developers.

Although multiple studies show this interaction can be successful, there is still limited understanding of how the models must be configured and used in the context of code generation tasks. This study addresses this gap, systematically investigating the impact of several key parameters, including the repeated submission of a prompt to accommodate for the non-deterministic nature of the models.

Our study reveals several key findings about the usage of ChatGPT. In particular, we discovered how creativity, although up to a limited extent, is useful to increase the range of methods whose code can be generated correctly. A major role is played by parameter top-p, which is commonly underrated, and instead has a major impact on the correctness of the results, with lower values producing better results. Finally, prompts should be submitted multiple times, with $5$ repetitions combined with a temperature of $1.2$ resulting in an effective configuration in our experiments.  

Future work concerns two main research directions. One is about replicating this experiment with other AI assistants, to validate our findings in multiple contexts. The second research direction concerns finding strategies to deal with the need to submit the same prompt multiple times to obtain a useful result, and thus developing approaches able to select or merge multiple responses automatically. 




%{\appendices
%\section*{Proof of the First Zonklar Equation}
%Appendix one text goes here.
% You can choose not to have a title for an appendix if you want by leaving the argument blank
%\section*{Proof of the Second Zonklar Equation}
%Appendix two text goes here.}



% \section{References Section}
% You can use a bibliography generated by BibTeX as a .bbl file.
%  BibTeX documentation can be easily obtained at:
%  http://mirror.ctan.org/biblio/bibtex/contrib/doc/
%  The IEEEtran BibTeX style support page is:
%  http://www.michaelshell.org/tex/ieeetran/bibtex/
 
%  % argument is your BibTeX string definitions and bibliography database(s)
% %\bibliography{IEEEabrv,../bib/paper}
% %
% \section{Simple References}
% You can manually copy in the resultant .bbl file and set second argument of $\backslash${\tt{begin}} to the number of references
%  (used to reserve space for the reference number labels box).

% \begin{thebibliography}{1}
% \bibliographystyle{IEEEtran}

% \bibitem{ref1}
% {\it{Mathematics Into Type}}. American Mathematical Society. [Online]. Available: https://www.ams.org/arc/styleguide/mit-2.pdf

% \bibitem{ref2}
% T. W. Chaundy, P. R. Barrett and C. Batey, {\it{The Printing of Mathematics}}. London, U.K., Oxford Univ. Press, 1954.

% \bibitem{ref3}
% F. Mittelbach and M. Goossens, {\it{The \LaTeX Companion}}, 2nd ed. Boston, MA, USA: Pearson, 2004.

% \bibitem{ref4}
% G. Gr\"atzer, {\it{More Math Into LaTeX}}, New York, NY, USA: Springer, 2007.

% \bibitem{ref5}M. Letourneau and J. W. Sharp, {\it{AMS-StyleGuide-online.pdf,}} American Mathematical Society, Providence, RI, USA, [Online]. Available: http://www.ams.org/arc/styleguide/index.html

% \bibitem{ref6}
% H. Sira-Ramirez, ``On the sliding mode control of nonlinear systems,'' \textit{Syst. Control Lett.}, vol. 19, pp. 303--312, 1992.

% \bibitem{ref7}
% A. Levant, ``Exact differentiation of signals with unbounded higher derivatives,''  in \textit{Proc. 45th IEEE Conf. Decis.
% Control}, San Diego, CA, USA, 2006, pp. 5585--5590. DOI: 10.1109/CDC.2006.377165.

% \bibitem{ref8}
% M. Fliess, C. Join, and H. Sira-Ramirez, ``Non-linear estimation is easy,'' \textit{Int. J. Model., Ident. Control}, vol. 4, no. 1, pp. 12--27, 2008.

% \bibitem{ref9}
% R. Ortega, A. Astolfi, G. Bastin, and H. Rodriguez, ``Stabilization of food-chain systems using a port-controlled Hamiltonian description,'' in \textit{Proc. Amer. Control Conf.}, Chicago, IL, USA,
% 2000, pp. 2245--2249.

% \end{thebibliography}


% \newpage

% \section{Biography Section}
% If you have an EPS/PDF photo (graphicx package needed), extra braces are
%  needed around the contents of the optional argument to biography to prevent
%  the LaTeX parser from getting confused when it sees the complicated
%  $\backslash${\tt{includegraphics}} command within an optional argument. (You can create
%  your own custom macro containing the $\backslash${\tt{includegraphics}} command to make things
%  simpler here.)
 
% \vspace{11pt}

% \bf{If you include a photo:}\vspace{-33pt}
% \begin{IEEEbiography}[{\includegraphics[width=1in,height=1.25in,clip,keepaspectratio]{fig1}}]{Michael Shell}
% Use $\backslash${\tt{begin\{IEEEbiography\}}} and then for the 1st argument use $\backslash${\tt{includegraphics}} to declare and link the author photo.
% Use the author name as the 3rd argument followed by the biography text.
% \end{IEEEbiography}

% \vspace{11pt}

% \bf{If you will not include a photo:}\vspace{-33pt}
% \begin{IEEEbiographynophoto}{John Doe}
% Use $\backslash${\tt{begin\{IEEEbiographynophoto\}}} and the author name as the argument followed by the biography text.
% \end{IEEEbiographynophoto}

% \vfill
\bibliographystyle{IEEEtran}
%argument is your BibTeX string definitions and bibliography database(s)
\bibliography{output}
% \newpage
% \section{Biography Section}
% If you have an EPS/PDF photo (graphicx package needed), extra braces are
%  needed around the contents of the optional argument to biography to prevent
%  the LaTeX parser from getting confused when it sees the complicated
%  $\backslash${\tt{includegraphics}} command within an optional argument. (You can create
%  your own custom macro containing the $\backslash${\tt{includegraphics}} command to make things
%  simpler here.)
 
% \vspace{11pt}
% \vspace{-0.5cm}
% \vspace{-80pt}
% \bf{If you include a photo:}\vspace{-33pt}


% \bf{If you will not include a photo:}\vspace{-33pt}
% \begin{IEEEbiographynophoto}{John Doe}
% Use $\backslash${\tt{begin\{IEEEbiographynophoto\}}} and the author name as the argument followed by the biography text.
% \end{IEEEbiographynophoto}

\vfill

\end{document}


