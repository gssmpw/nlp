%%%%%%%% ICML 2025 EXAMPLE LATEX SUBMISSION FILE %%%%%%%%%%%%%%%%%

\documentclass{article}

% Recommended, but optional, packages for figures and better typesetting:
\usepackage{microtype}
\usepackage{graphicx}
\usepackage{subfigure}
\usepackage{booktabs} % for professional tables

% hyperref makes hyperlinks in the resulting PDF.
% If your build breaks (sometimes temporarily if a hyperlink spans a page)
% please comment out the following usepackage line and replace
% \usepackage{icml2025} with \usepackage[nohyperref]{icml2025} above.
\usepackage{hyperref}


% Attempt to make hyperref and algorithmic work together better:
\newcommand{\theHalgorithm}{\arabic{algorithm}}

% Use the following line for the initial blind version submitted for review:
% \usepackage{icml2025}

% If accepted, instead use the following line for the camera-ready submission:
\usepackage[accepted]{icml2025}

% For theorems and such
\usepackage{amsmath}
\usepackage{amssymb}
\usepackage{mathtools}
\usepackage{amsthm}

% if you use cleveref..
\usepackage[capitalize,noabbrev]{cleveref}

%%%%%%%%%%%%%%%%%%%%%%%%%%%%%%%%
% THEOREMS
%%%%%%%%%%%%%%%%%%%%%%%%%%%%%%%%
\theoremstyle{plain}
\newtheorem{theorem}{Theorem}[section]
\newtheorem{proposition}[theorem]{Proposition}
\newtheorem{lemma}[theorem]{Lemma}
\newtheorem{corollary}[theorem]{Corollary}
\theoremstyle{definition}
\newtheorem{definition}[theorem]{Definition}
\newtheorem{assumption}[theorem]{Assumption}
\theoremstyle{remark}
\newtheorem{remark}[theorem]{Remark}

\newcommand{\loss}[1]{\mathcal{L}(#1)}
\newcommand{\SCM}{\text{SCM}}
\newcommand{\Funion}{\mathcal{F}_{\cap}}

\newcommand{\longvt}[1]{{\color{blue}[Long: #1]}}
\newcommand{\hiendh}[1]{{\color{blue}[HD: #1]}}
% Todonotes is useful during development; simply uncomment the next line
%    and comment out the line below the next line to turn off comments
%\usepackage[disable,textsize=tiny]{todonotes}
\usepackage[textsize=tiny]{todonotes}


% The \icmltitle you define below is probably too long as a header.
% Therefore, a short form for the running title is supplied here:
\icmltitlerunning{Why Domain Generalization Fail? A View of Necessity and Sufficiency}

\begin{document}

\twocolumn[
\icmltitle{Why Domain Generalization Fail? A View of Necessity and Sufficiency}

% It is OKAY to include author information, even for blind
% submissions: the style file will automatically remove it for you
% unless you've provided the [accepted] option to the icml2025
% package.

% List of affiliations: The first argument should be a (short)
% identifier you will use later to specify author affiliations
% Academic affiliations should list Department, University, City, Region, Country
% Industry affiliations should list Company, City, Region, Country

% You can specify symbols, otherwise they are numbered in order.
% Ideally, you should not use this facility. Affiliations will be numbered
% in order of appearance and this is the preferred way.
\icmlsetsymbol{equal}{*}

\begin{icmlauthorlist}
\icmlauthor{Tung-Long vuong}{comp}
\icmlauthor{Vy Vo}{comp}
\icmlauthor{Hien Dang}{yyy}
\icmlauthor{Van-Anh Nguyen}{comp}
\icmlauthor{Thanh-Toan Do}{comp}
\icmlauthor{Mehrtash Harandi}{comp}
\icmlauthor{Trung Le}{comp}
\icmlauthor{Dinh Phung}{comp}

%\icmlauthor{}{sch}
%\icmlauthor{}{sch}
\end{icmlauthorlist}

\icmlaffiliation{yyy}{ University of Texas at Austin, USA}
\icmlaffiliation{comp}{Monash University, Australia}


\icmlcorrespondingauthor{Tung-Long Vuong}{Tung-Long.Vuong@monash.edu}


% You may provide any keywords that you
% find helpful for describing your paper; these are used to populate
% the "keywords" metadata in the PDF but will not be shown in the document
\icmlkeywords{Machine Learning, ICML}

\vskip 0.3in
]

% this must go after the closing bracket ] following \twocolumn[ ...

% This command actually creates the footnote in the first column
% listing the affiliations and the copyright notice.
% The command takes one argument, which is text to display at the start of the footnote.
% The \icmlEqualContribution command is standard text for equal contribution.
% Remove it (just {}) if you do not need this facility.

%\printAffiliationsAndNotice{}  % leave blank if no need to mention equal contribution

\printAffiliationsAndNotice{} % otherwise use the standard text.

\begin{abstract}

Despite a strong theoretical foundation, empirical experiments reveal that existing domain generalization (DG) algorithms often fail to consistently outperform the ERM baseline. We argue that this issue arises because most DG studies focus on establishing theoretical guarantees for generalization under unrealistic assumptions, such as the availability of sufficient, diverse (or even infinite) domains or access to target domain knowledge. As a result, the extent to which domain generalization is achievable in scenarios with limited domains remains largely unexplored. This paper seeks to address this gap by examining generalization through the lens of the conditions necessary for its existence and learnability. Specifically, we systematically establish a set of necessary and sufficient conditions for generalization. Our analysis highlights that existing DG methods primarily act as regularization mechanisms focused on satisfying sufficient conditions, while often neglecting necessary ones. However, sufficient conditions cannot be verified in settings with limited training domains. In such cases, regularization targeting sufficient conditions aims to maximize the likelihood of generalization, whereas regularization targeting necessary conditions ensures its existence. Using this analysis, we reveal the shortcomings of existing DG algorithms by showing that, while they promote sufficient conditions, they inadvertently violate necessary conditions. To validate our theoretical insights, we propose a practical method that promotes the \textit{sufficient condition} while maintaining the \textit{necessary conditions} through a novel subspace representation alignment strategy. This approach highlights the advantages of preserving the necessary conditions on well-established DG benchmarks.

\end{abstract}

\section{Introduction}

% Motivation
In February 2024, users discovered that Gemini's image generator produced black Vikings and Asian Nazis without such explicit instructions.
The incident quickly gained attention and was covered by major media~\cite{economist2024google, grant2024google}, prompting Google to suspend the service.
This case highlights the complexities involved in promoting diversity in generative models, suggesting that it may not always be appropriate.
Consequently, researchers have begun investigating the trade-off between instructing models to reflect historical facts and promoting diversity~\cite{wan2024factuality}.
Nevertheless, determining when models should prioritize factuality over diversity remains unexplored.

\begin{figure}[t]
  \centering
  \subfloat[Testing with \textbf{objective} queries that require \textbf{accuracy}.]{
    \includegraphics[width=1.0\linewidth]{Figures/obj-cover.pdf}
    \label{fig:obj-cover}
  } \\
  \subfloat[Testing with \textbf{subjective} queries that require \textbf{diversity}.]{
    \includegraphics[width=1.0\linewidth]{Figures/subj-cover.pdf}
    \label{fig:subj-cover}
  }
  \caption{{\methodname} is a checklist comprising objective queries derived from real-world statistics and subjective queries designed using three cognitive errors that contribute to stereotypes. It includes queries designed for LLMs and T2I models.}
\end{figure}

% Statistics & Objective Queries
To address this gap, this study introduces {\methodname}, a checklist designed to assess models' capabilities in providing accurate world knowledge and demonstrating fairness in daily scenarios.
For world knowledge assessment, we collect 19 key statistics on U.S. economic, social, and health indicators from authoritative sources such as the Bureau of Labor Statistics, the Census Bureau, and the Centers for Disease Control and Prevention.
Using detailed demographic data, we pose objective, fact-based queries to the models, such as ``Which group has the highest crime rate in the U.S.?''—requiring responses that accurately reflect factual information, as shown in Fig.~\ref{fig:obj-cover}.
Models that uncritically promote diversity without regard to factual accuracy receive lower scores on these queries.

% Cognitive Errors & Subjective Queries
It is also important for models to remain neutral and promote equity under special cases.
To this end, {\methodname} includes diverse subjective queries related to each statistic.
Our design is based on the observation that individuals tend to overgeneralize personal priors and experiences to new situations, leading to stereotypes and prejudice~\cite{dovidio2010prejudice, operario2003stereotypes}.
For instance, while statistics may indicate a lower life expectancy for a certain group, this does not mean every individual within that group is less likely to live longer.
Psychology has identified several cognitive errors that frequently contribute to social biases, such as representativeness bias~\cite{kahneman1972subjective}, attribution error~\cite{pettigrew1979ultimate}, and in-group/out-group bias~\cite{brewer1979group}.
Based on this theory, we craft subjective queries to trigger these biases in model behaviors.
Fig.~\ref{fig:subj-cover} shows two examples on AI models.

% Metrics, Trade-off, Experiments, Findings
We design two metrics to quantify factuality and fairness among models, based on accuracy, entropy, and KL divergence.
Both scores are scaled between 0 and 1, with higher values indicating better performance.
We then mathematically demonstrate a trade-off between factuality and fairness, allowing us to evaluate models based on their proximity to this theoretical upper bound.
Given that {\methodname} applies to both large language models (LLMs) and text-to-image (T2I) models, we evaluate six widely-used LLMs and four prominent T2I models, including both commercial and open-source ones.
Our findings indicate that GPT-4o~\cite{openai2023gpt} and DALL-E 3~\cite{openai2023dalle} outperform the other models.
Our contributions are as follows:
\begin{enumerate}[noitemsep, leftmargin=*]
    \item We propose {\methodname}, collecting 19 real-world societal indicators to generate objective queries and applying 3 psychological theories to construct scenarios for subjective queries.
    \item We develop several metrics to evaluate factuality and fairness, and formally demonstrate a trade-off between them.
    \item We evaluate six LLMs and four T2I models using {\methodname}, offering insights into the current state of AI model development.
\end{enumerate}


\begin{figure*}[!t]
	\centering
	\includegraphics[width=\linewidth]{Fig/flow.png}

	\caption{Method overview includes (a) a formative understanding of current personhood verification and related challenges through competitive analysis  (b) users' perception, preferences, and design through an interview study}
\label{fig:method}
\end{figure*}
\vspace{-2mm}
\section{Method Overview}
\label{sec:method}
\vspace{-2mm}
Building on the existing literature, it is clear that while significant progress has been made, a critical gap remains in understanding the key factors to operationalize personhood credentials that balance privacy, security, and trustworthiness online. 
%This challenge becomes even more pressing with the rise of increasingly advanced AI, which enables bad actors to scale their operations, exacerbating issues such as impersonation, fake identities, and non-human interactions. 
As outlined in Figure~\ref{fig:method}, our study comprises: (1) a competitive analysis of current personhood/identity verification tools to identify challenges. These insights inform the design of a user study aimed at (2) investigating users’ perceptions (RQ1), identifying factors influencing their preferences for personhood credentials (RQ2), and conceptualizing designs (RQ3) to address these challenges.

%Please add the flow digram / RQs of different methods with a method overview. see here https://arxiv.org/pdf/2410.01817?}


\vspace{-2mm}
\section{Formative Understanding of PHCs}
\vspace{-2mm}
In this section, we outline our formative analysis of existing personhood verification systems, which informed the design rationale for developing our user study (Section~\ref{user-study}).

%\subsection{Competitive Analysis \& Cognitive Walkthrough}
%\textbf{Competitive Analysis.}
%No prior studies have explored personhood credentials systems' usability and security issues. To address this gap, 
We systematically consolidated a list of systems based on their popularity, diversity in platform type (centralized vs. decentralized), and relevance to the domain of digital identity~\cite{idenaWhitepaper, kavazi2021humanode, kavazi2023humanode, de2024personhood, BrightID, PoH, adler2024personhood}
This consists of
%both practical implementations and state-of-the-art systems, including the 
World app, BrightID, Proof of Humanity, Gitcoin Passport, and Federated Identities (OAuth), etc (Table~\ref{tab:systems}). 
%as well as collected public user's review from Google Playstore. We chose these systems based on their popularity, diversity in platform type (centralized vs. decentralized), and relevance to the domain of digital identity\fixme{add citations of research papers from lit review}. 
Table~\ref{tab:identity_verification} provides an overview of different attributes of how existing systems operate and their design trade-offs. We found 15 apps categorized into six groups. Five of these were centralized, primarily government-based personhood verification systems. This initial categorization is based on the data requirements for issuing credentials varied, including behavior filters, biometrics (such as face, selfie, iris, or video), social graph and vouching mechanisms, physical ID verification, and, in some cases, combinations of these methods. 
\iffalse
\begin{table}[ht]
    \centering
    \scriptsize
    \begin{tabular}{llll}
      \hline
       App Name  & Source & reviews  \\
    
        \hline
     Worldapp & White Paper~\cite{WorldWhitepaper}, Google Play Store& 1523 \\
  BrightID & White Paper~\cite{BrightID},Google Play Store & 328 \\
  DECO & WhitePaper~\cite{zhang2020deco} & Review  \\
  CANDID & WhitePaper~\cite{maram2021candid} & Review \\
  Proof of Humanity &  WhitePaper~\cite{PoHexplainer} & Review \\
  Adhar Card &  WhitePaper~\cite{Aadhaar}, Google Play Store & Review
  %https://play.google.com/store/apps/details?id=in.gov.uidai.mAadhaarPlus&hl=en_US
  \\
Estonia e-ID  &  WhitePaper~\cite{estoniaE-ID} & Review\\
Chinese Credit system &  WhitePaper~\cite{ChinaSocialCreditSystem} & Review \\
Japan My Number Card &  WhitePaper~\cite{JapanMyIDNumber} & Review \\
ID.me &  WhitePaper~\cite{irsIdentityVerification, idAccessAll}, Google Play Store & Review \\
%https://play.google.com/store/apps/details?id=me.id.auth&hl=en_US
Idena &  WhitePaper~\cite{idenaWhitepaper} &  Review \\
Humanode &  WhitePaper~\cite{kavazi2021humanode} &Review\\
Civic &  WhitePaper~\cite{CivicPass} &Review \\
Federated identities (Oauth) &  WhitePaper~\cite{OAuth} & Review\\
  \hline
    
    \end{tabular}
    \caption{Competitive Analysis Data Sources 
   % \fixme{may move to appendix later}
    }
    \label{tab:systems}
\end{table}
\fi
%which helps us conduct a cognitive walkthrough. 

%we analyzed 15 popular systems in terms of their features, such as issuance system (centralized vs decentralized), types of data requirements for issuing credentials, types of  service providers of those systems. 
%Our competitive analysis allowed us to explore and identify multi-criteria to assess aspects such as privacy, usability, and security
We also documented on how users navigate the system and identify potential usability and security issues. Two UI/UX in out team evaluated whether users could successfully sign up and obtain personhood credentials. We independently compiled an initial list of evaluation results based on key questions. This includes- \textit{``How intuitive is the verification process?; How effectively does the platform provide feedback during different steps of registration and verification?; How do we as users feel regarding the data requirements in the verification systems?; How does the platform manage users' data?; What are the potential risks regarding users' privacy in the platform?''}
%about user workflows, task completion, and potential points of failure. 
%such as the intuitiveness of the verification process, feedback during registration, data requirements.
%data management, and privacy risks. 
%This included documenting account creation, data input, verification procedures, and associated risks. 
Given the limited access to systems like Estonia’s digital ID, Civic, and China’s social credit system, we used available white papers and documentation to reconstruct their workflows. Finally, we synthesized our observations and conducted qualitative coding to identify recurring themes.



\begin{table}[ht]
    \centering
    \scriptsize
    \begin{tabular}{llll}
      \hline
       App Name  & Source & reviews  \\
    
        \hline
     Worldapp & Documentation~\cite{WorldWhitepaper}, Google Play Store& 1523 \\
  BrightID & Documentation~\cite{BrightID},Google Play Store & 328 \\
  DECO & Documentation~\cite{zhang2020deco} & Review  \\
  CANDID & Documentation~\cite{maram2021candid} & Review \\
  Proof of Humanity &  Documentation~\cite{PoHexplainer} & Review \\
  Adhar Card &  Documentation~\cite{Aadhaar}, Google Play Store & Review
  %https://play.google.com/store/apps/details?id=in.gov.uidai.mAadhaarPlus&hl=en_US
  \\
Estonia e-ID  &  Documentation~\cite{estoniaE-ID} & Review\\
Chinese Credit system &  Documentation~\cite{ChinaSocialCreditSystem} & Review \\
Japan My Number Card &  Documentation~\cite{JapanMyIDNumber} & Review \\
ID.me &  Documentation~\cite{irsIdentityVerification, idAccessAll}, Google Play Store & Review \\
%https://play.google.com/store/apps/details?id=me.id.auth&hl=en_US
Idena &  Documentation~\cite{idenaWhitepaper} &  Review \\
Humanode &  Documentation~\cite{kavazi2021humanode} &Review\\
Civic &  Documentation~\cite{CivicPass} &Review \\
Federated identities (Oauth) &  Documentation~\cite{OAuth} & Review\\
  \hline
    
    \end{tabular}
    \caption{Competitive Analysis Data Sources 
   % \fixme{may move to appendix later}
    }
    \label{tab:systems}
\end{table}
%(presented in section~\ref{prac-cha}).

%\textbf{Cognitive Walkthough.}
%For the cognitive walkthrough, 
%We also focused on how a user would navigate the system and identify potential usability and security issues. Two experts, specializing in UI/UX and verification systems, evaluated whether users could successfully interact with the application interface and complete two tasks, (a) signing up with the system and (b) obtaining personhood credentials. We independently compiled an initial list of evaluation results by addressing key questions related to user workflows, task completion, and potential points of failure. This includes- \textit{``How intuitive is the verification process?; How effectively does the platform provide feedback during different steps of registration and verification?; How do we as users feel regarding the data requirements in the verification systems?; How does the platform manage users' data?; What are the potential risks regarding users' privacy in the platform?''}
%This included documenting (a) the step-by-step process of creating test accounts and (b) key steps such as data input requirements, verification procedures, and associated risks. Given that some relevant systems, such as Estonia’s digital ID, Civic, and China’s social credit system, are either inaccessible or operate as proof of concept models, we referenced available white papers and documentation to reconstruct their workflows. Finally, we synthesized the experts' observations and conducted qualitative coding to identify recurring themes in the evaluation (presented in section~\ref{prac-cha}). 
%These themes were categorized based on usability challenges, security concerns, and potential improvements in the interface design and verification process.
%Once the evaluations were done, we conducted a qualitative coding to understand the overall themes of the assessment.
%of the user interface and user experience, 

%focusing on ease of use, clarity, and overall usability; (b) we created test accounts to study and asses the workflow and documented the key steps, required information and potential privacy and security issues. Finally, we structured the data according to aforementioned criteria to highlight notable differences and their implications on usability and privacy.
%For evaluating the current verification process of some applications, we have utilized cognitive analysis of UI/UX, data requirement and privacy issue 
%We have selected some popular centralized and decentralized platforms such as World app, Bright ID, Proof of Humanity, Passport Gitcoin, Federated Identities (OAuth), Aadhar Card, Estonia's digital ID and China's social credit system . 

%For cognitive analysis of UI/UX, we have considered a few questions set: 
%\tanusree{from where did we get these questions? My impression was- we are doing cognitive analysis of ui/ux and data requirement, privacy issues, questions here doesn't reflect the goal of cognitive walkthrough}
% \begin{itemize}
%     \item How intuitive is the verification process?
%     \item How effectively does the platform provide feedback during different steps of registration and verification?
%     \item How do we as users feel regarding the data requirements in the verification systems?
%      \item How does the platform manage users' data?
%     \item What are the potential risks regarding users' privacy in the platform?
% %\end{itemize}
% %The following 2 questions have been utilized for data requirement analysis
% %\begin{itemize}
%     %\item What type of data (e.g., personal and biometric, etc) are required for issuing the credentials?
%     %\item In which stage, are these credentials requested from users? How we as users felt regarding the data requirements in the verification systems
% %\end{itemize}
% %We have also analyzed the privacy concerns using these 2 questions:
% %\begin{itemize}
   
% \end{itemize}


 %  \begin{figure*}
 % 	\centering
 % 	\includegraphics[width=0.8\linewidth]{Fig/worldapp.png}
 % 	\caption{ Worldapp-(a) lack of guidance on how users should navigate or utilize the app; (b) backup interface: requires users to connect Google Drive}
    
 % \label{fig: fig:worldapp}
 % \end{figure*}
%The competitive analysis aimed to evaluate and compare the verification processes of the \fixme{it should be a total of 15} eight selected verification systems (Table~\ref{tab:identity_verification}).
%The following predefined criteria were utilized to ensure a structured and consistent evaluation of the platforms:

% \begin{itemize}
%     \item Type of platform
%     \item Free or paid
%     \item Required data
%     \item Stage where data is required
%     \item Centralized or decentralized
%     \item Advantage
%     \item Disadvantage
%     \item UI/UX issue
%     \item Privacy related issue
% \end{itemize}

% We collected data for analysis using the following approach:
% \begin{itemize}
%     \item We analyzed the user interface and the user experience qualitatively and focused on ease of use, clarity and usability.
%     \item We created test accounts to study and asses the whole account creation workflow and documented the key steps and required information.
% \end{itemize}


  %  \item We reviewed official resources such as documentation and privacy policy to evaluate privacy concerns. 


\begin{table*}[h!]
    \centering
    \caption{Comparison of Existing Personhood Verification Systems}
    \label{tab:identity_verification}
    \resizebox{\textwidth}{!}{ 
    \begin{tabular}{l >{\small}l >{\small}l >{\small}l >{\small}p{3cm} >{\small}p{2.5cm} >{\small}l} 
        \hline
        \textbf{Category} & \textbf{Service Name} & \textbf{Architecture} & \textbf{Issuer} & \textbf{Credential} & \textbf{Platform} & \textbf{Free/Paid} \\
        \hline
        \hline
        \multirow{3}{*}{Behavioral Filter} 
        & CAPTCHA & Centralized & open-source, vendor & Recognize distorted texts, images, sounds etc. & Desktop and mobile browsers & Free/Paid\\
        & reCAPTCHA & Centralized & Google & Click checkbox & Desktop and mobile browsers& Free/Paid\\
        & Idena & Decentralized & open-source & Solve contextual puzzle & Blockchain & Free\\
        \hline
        \multirow{2}{*}{Biometrics}
        & World ID & Decentralized & World & Biometrics (iris scan) & App (iOS, Android) & Free\\
        & Humanode & Decentralized & Humanode & Biometrics (face) & Blockchain & Paid\\
        \hline
        Social Graph 
        & BrightID & Decentralized & open-source & Analysis of social graph & App (iOS, Android) & Free\\
        \hline
        Social Vouching 
        & Proof of Humanity & Decentralized & Kleros & Social vouching & Web & Paid\\
        \hline
        \multirow{2}{*}{Decentralized Oracle} 
        & DECO & Decentralized & Chainlink Labs & Cryptographic proof & Decentralized oracle & Under PoC\\
        & CANDID & Decentralized & IC3 research team & Cryptographic proof & Decentralized oracle & Under PoC\\
        \hline
        \multirow{4}{*}{Government-based ID} 
        & India Aadhaar Card & Centralized & Government & Document-based or Head Of Family-based enrollment + digital photo of face, 2 iris, and 10 fingerprints& Web, App (iOS, Android) & Free\\
        & Estonia e-ID & Decentralized & Government & Passport or EU ID + digital photo of face & Web, App (iOS, Android) & Paid\\
        & Japan My Number Card & Centralized & Government & Issue notice letter + photo ID or two non-photo IDs & Web, App (iOS, Android) & Free\\
        %& Chinese Credit System & Centralized & Gov & Personal credit records & Varies by region & Free\\
        \hline
        \multirow{2}{*}{Others} 
        & ID.me & Centralized & ID.me & Government-issued ID & Web & Free\\
        & Civic Pass & Decentralized & Civic & Government-issued ID, Biometrics (face), Humanness, Liveness & Web & Free\\
        \hline
    \end{tabular}
    }
\end{table*}

\begin{figure*}[h]
    \centering
    \begin{subfigure}{0.48\textwidth}
        \centering
        \raisebox{0.5\height}{
        \includegraphics[width=\textwidth]{Fig/idena.png}}
        \captionsetup{width=\textwidth, font=footnotesize} 
        \caption{Idena validation test interface: This requires users to select meaningful stories within a time limit, which can pose challenges for new users}
        \label{fig:idena}
    \end{subfigure}
    \hfill
    \begin{subfigure}{0.48\textwidth}
        \centering
        \includegraphics[width=\textwidth]{Fig/google_drive.png}
        \captionsetup{width=\textwidth, font=footnotesize} 
        \caption{World App backup interface: requires users to connect Google Drive}
        \label{fig:worldapp}
    \end{subfigure}
    
    \caption{PHC-related interfaces: (a) Idena validation test, (b) World App backup process.}
    \label{fig:phc_interfaces}
\end{figure*}

\vspace{-2mm}
\subsection{Challenges in Identity Verification}
\vspace{-2mm}
\label{prac-cha}
\textbf{Demanding Cognitive and Social Efforts for Verification Workflow.}
We found platforms such as World App and BrightID developed on decentralized technologies, 
including zero-knowledge proofs and social connections, may confuse non-technical users. For instance, user review from playstore suggested-many having issues understanding how to receive BrightID scores to prove they are sufficiently connected with others and verified within the graph. In their words \textit{``It's hard for me to connect with people to create the social graph.''} 
%\textbf{Usability Issue.}
%CAPTCHAs have become increasingly difficult to solve, can make the user journey cognitively demanding. To support the security of humanness verification, particularly image-based ones are becoming demanding for users. 
From experts' evaluation of UI/UX, we found Proof of Humanity lacks options to correct or update mistakes, which can make the registration process less user-friendly. %Incorporating the principle of error prevention could improve the user experience. 
Similarly, Idena's validation test (flip test) (Figure~\ref{fig:idena}) was challenging as new users as it required to create a meaningful story within the allotted time and earn enough points for validation. Simialrly, World App's(Figure~\ref{fig:worldapp}) account creation process to get an identifier doesn't inform users how and why to navigate the app can undermine intended functionality,  or underutilization of the app’s capabilities.


% \begin{figure*}[h]
%     \centering
%     \begin{minipage}{0.30\textwidth}
%         \centering
%         \includegraphics[width=\linewidth]{Fig/google drive.png}
%         \caption{World App backup interface: requires users to connect Google Drive.}
%         \label{fig:worldapp}
%     \end{minipage}
%     \hfill
%     \begin{minipage}{0.48\textwidth}
%         \centering
%         \includegraphics[width=\linewidth]{Fig/wordl1.png}
%         \caption{World App's account creation process: lack of guidance on how users should navigate or utilize the app.}
%         \label{fig:Worldapp1}
%     \end{minipage}
% \end{figure*}

\textbf{New or Complex System Rule to Recover ID. }
Both from UI/UX task and playstore review, we found the BrighID recovery process tedious and the rules unclear. A representative user review stated-\textit{``If you create an account and do not set up recovery connections you cannot get your account back. This forces you to create a new account which defeats the purpose of the app.''}
Another workflow of World App that requires users to connect their Google Drive to back up their accounts. However, this process may confuse users and create challenges during account recovery if they fail to complete the backup(Figure~\ref{fig:worldapp}).
 

%  \begin{figure}
%  	\centering
%  	\includegraphics[width=\linewidth]{Fig/wordl1.png}
%  	\caption{World App's account creation process: lack of guidance on how users should navigate or utilize the app}
%  \label{fig:Worldapp1}
%  \end{figure}


\textbf{Privacy and Data Requirement Issue. }
From our competitive analysis (Table~\ref{tab:litcomparison}), Data requirements across the systems vary significantly in scope and sensitivity. Decentralized platforms like World App, and BrightID required minimal data collection to issue ID while Proof of Humanity require video submission to receive a credential for was quite invasive when the videos were open to the public with clear faces.
%Similarly, both experts mentioned many unknown data policies for new platforms such as World app~\cite{WorldWhitepaper} and Bright ID\cite{BrightID}. 
While there is benefit of decentralization, often it is not clear how exactly service providers will handle the data in their policies and white papers.
%which created a reluctance for them, thus for new users to start using them. 
In contrast, Federated Identities OAuth\cite{OAuth} login process streamlines and this contributed to using known third-party service providers. This ensures ease of use as users need to specify the identity provider during the login or authentication process and grant access to their specific data. This reflects the importance of known entities and level of trust in data handling.
%However, they also have data being shared across multiple platforms which leads to some privacy concerns. 
Centralized systems, including Aadhaar and Estonia digital ID, require extensive personal and biometric data—fingerprints and iris scans—to ensure verification services while experts expressed privacy concerns towards china’s Social Credit personhood System, especially the use of it in measuring social scores.
%There was concerns regarding reCAPTCHA addressing usability issues by removing explicit verification tasks, relying instead on tracking user behavior, such as mouse movements, keystrokes, and browsing history. However, this approach trades off user privacy, as data collected during these activities raised concerns.


\textbf{Requirement of Optimal Device or Physical Presence.}\\
Government-supported systems like Aadhaar and Estonia e-Card feature structured interfaces but come with limitations: Aadhaar’s biometric registration may challenge rural populations, while Estonia’s dependence on smart-card hardware might exclude those without the necessary devices. Proof of Humanity, Humanode, Civic Pass may create challenges as proper lighting and optimal devices are necessary for taking the appropriate photo or video for biometric verification
%\fixme{need a screenshot for this}. 
On the contrast, Aadhaar card\cite{Aadhaar}\cite{AadhaarEnrollment}, Estonia's e-ID and Japan's My Number Card require one to be physically present and the issuing process takes a long time can create user frustration. 
%The existing systems and platforms that we have evaluated can hardly strike a balance between privacy, functionality and usability.  



%CAPTCHA\cite{Captcha} and reCAPTCHA\cite{reCaptcha} are 2 common human verification tools used across many websites. While CAPTCHAs add an additional step for users when they are trying to access a website, reCAPTCHAS come into play by removing any external verification. Rather, reCAPTCHAs track users' activities which has raised privacy concerns as there is lack of transparency between user and reCAPTCHA authority. Users are not sure how the tracking data will be used. 

\iffalse
\subsection{Results of UI/UX}
%\tanusree{Silvia: why do we have only 3 apps in the analysis?Ayae created a list a long ago. please complete the analysis for all the apps from this list}  \tanusree{I am not sure why facebook is in the analysis. we talked about only including verification apps, facebook is not one of them} \fixme{look at the Suggetsions in comment}
The eight \fixme{15 systems} systems evaluated manifest diverse approaches to user experience, emphasizing accessibility, intuitiveness, and transparency\fixme{write in active sentence or active voice, it reads like chatGPT and reviewer will think the same}. Platforms such as World App and BrightID developed on decentralized technologies, 
%though their intricate verification methods, 
including zero-knowledge proofs\fixme{add citation} and social connections \fixme{add as footnote what social connection means here and citation}, may confuse non-technical users. Proof of Humanity requires video submissions \fixme{what kind of video, is it their face? then talk about privacy, this doesn't seem to be a blockchain issue rather privacy issue}, a process potentially intimidating for individuals less familiar with blockchain platforms. 

In contrast, Federated Identities (OAuth) streamlines login processes via well-known third-party providers\fixme{who is the third-party provider for them}, ensuring ease of use for most users \fixme{is that all? }. 

Government-supported systems like Aadhaar and Estonia e-Card feature structured interfaces but come with limitations: Aadhaar’s biometric registration may challenge rural populations, while Estonia’s dependence on smart-card hardware might exclude those without the necessary devices. \fixme{add about Japan My Number Card.} 

Passport Gitcoin, focused on Web3 integration, struggles with clarity for users new to decentralized identity concepts. Finally, China’s Social Credit System delivers a seamless yet opaque experience, leaving users uncertain about the data influencing their scores.\par
Data requirements across the systems vary significantly in scope and sensitivity. Decentralized platforms like World App, BrightID, and Proof of Humanity emphasize minimal data collection but still require sensitive information, such as Ethereum addresses, social graphs, or video proofs, to ensure authenticity. 

Centralized systems, including Aadhaar and Estonia digital ID, require extensive personal and biometric data—fingerprints and iris scans—to ensure seamless service delivery. 

Passport Gitcoin, designed for Web3 wallet integration, relies on centralized storage, demanding significant user trust. Federated Identities (OAuth) achieves a balance by sharing limited data through third-party providers but this comes with the risk of overexposure. China’s Social Credit System stands out for its vast data collection, encompassing financial, social, and daily activities, raising alarm over pervasive monitoring and privacy intrusion.\par
Privacy concerns are critical across the eight systems, influenced by their data management practices. Decentralized platforms like World App and BrightID prioritize privacy, yet linking personal data to public blockchains—as seen in Proof of Humanity—poses inherent risks. Centralized systems like Aadhaar and Estonia e-Card depend on centralized databases, making them vulnerable to surveillance risks. Federated Identities (OAuth) simplifies access but could expose user data to third-party providers without explicit consent. Passport Gitcoin presents privacy challenges because users' information can be shared with third-party service providers. Meanwhile, China’s Social Credit System exemplifies extreme privacy erosion, extensively monitoring citizen behavior with minimal transparency about data use. Striking a balance between privacy and functionality remains a universal challenge for all these systems.

\fixme{citations to be added} We have evaluated 15 systems to present diverse approaches to user experience, emphasizing usability, accessibility, intuitiveness and transparency.
\fixme{citation didn't work} CAPTCHA\cite{Captcha} and reCAPTCHA\cite{reCaptcha} are 2 common human verification tools used across many websites. While CAPTCHAs add an additional step for users when they are trying to access a website, reCAPTCHAS come into play by removing any external verification. Rather, reCAPTCHAs track users' activities which has raised privacy concerns as there is lack of transparency between user and reCAPTCHA authority. Users are not sure how the tracking data will be used. 

\tanusree{no good content}
Platforms such as World app\cite{WorldWhitepaper} and Bright ID\cite{BrightID} are developed on decentralized technologies which include zero-knowledge proofs but do not present a clear and concise terms and conditions and privacy policy, which may create reluctance for new users to start using them. In figure 1(a), the on-boarding screen of World App appears with a consent checkbox to obtain explicit consent from the users that they agree to the "Terms and Conditions" and acknowledge the "Privacy Notice" of World App. But the terms and conditions and privacy notice are not mentioned in the same screen, tapping on the link buttons redirects users to a different screen, thus creating an obstacle in their user journey. If the necessary terms and conditions were presented clearly and concisely on the on-boarding screen, it would have informed users about the app's policies and ensure a smoother user journey. 1(b) represents the Bright ID license agreement, but it is too long to read. Users may not have enough patience to go through the details as it is time consuming and tap the agree button to continue. But this action may create privacy risks as users don't know what type of access they are providing to the application.
\begin{figure}[h]
     \centering
     \begin{subfigure}[b]{0.2\textwidth}
         \centering
         \includegraphics[width=\textwidth]{Fig/world app t&c.png}
         \caption{The terms and conditions and privacy notice are not mentioned in the World App's on-boarding screen}
         \label{fig:The terms and conditions and privacy policy are not mentioned in the World App's on-boarding screen}
     \end{subfigure}
     \hfill
     % \begin{subfigure}[b]{0.3\textwidth}
     %     \centering
     %     \includegraphics[width=\textwidth]{Fig/google drive.png}
     %     \caption{World App requires users to connect Google Drive for enabling backup}
     %     \label{fig:five over x}
     % \end{subfigure}
     % \hfill
     \begin{subfigure}[b]{0.3\textwidth}
         \centering
         \includegraphics[width=\textwidth]{Fig/bright id t&c.png}
         \caption{Bright ID's license agreement contains a long description which users may not want to read}
         \label{fig:three sin x}
     \end{subfigure}
     \hfill
        \caption{On-boarding screens of World App and Bright ID}
        \label{fig:three graphs}
\end{figure}
In figure 2, we can see World App requires users to connect their Google Drive to back up their world app accounts but this may lead users to providing access to their sensitive information.
\begin{figure}[h]
    \centering
    \includegraphics[width=0.5\linewidth]{Fig/google drive.png}
    \caption{World App requires users to connect Google Drive for enabling backup}
    \label{fig:World App requires users to connect Google Drive for enabling backup}
\end{figure}
% \iffalse
% \begin{figure}
%  	\centering
%  	\includegraphics[width=0.5\linewidth]{Fig/world app t&c.png}
%  	\caption{The terms and conditions and privacy policy are not mentioned in the World App's on-boarding screen}   
%  \label{fig:The terms and conditions and privacy policy is not clearly mentioned}
%  \end{figure}
%  \begin{figure}
%  	\centering
%  	\includegraphics[width=\linewidth]{Fig/bright id t&c.png}
%  	\caption{The license agreement and privacy policy is too long to read}   
%  \label{fig:The license agreement and privacy policy is too long to read}
%  \end{figure}
% . \par
%  \begin{figure}
%  	\centering
%  	\includegraphics[width=\linewidth]{Fig/google drive.png}
%  	\caption{World App requires users to connect Google Drive for enabling backup}
    
%  \label{fig:World App asking to connect Google Drive}
%  \end{figure}


 


Proof of Humanity\cite{PoH}\cite{PoHexplainer} offers a unique approach to verification with a social identification system. But the verification process requires users to connect their cryptocurrency wallet which will be publicly linked to users' account. Thus, users' wallet holdings and transaction history will be linked to users' identity which users may not prefer. 

In contrast, Federated Identities OAuth\cite{OAuth} provides streamlined login process via well known third-party service provides, also known as identity providers such as Google, Facebook etc. It ensures ease of use as users need to specify the identity provider during the login or authentication process and grant access to their specific data. But, data is shared across multiple platform which may raise privacy concerns among users. 

DECO\cite{zhang2020deco} and CanDID\cite{maram2021candid} are decentralized and privacy preserving oracle protocols where DECO allows users to prove the authenticity of website data obtained over TLS (Transport Layer Security) without revealing sensitive information. But Oracle has access to users' data which pose as a privacy risk. CanDID provides users with control of their own credentials but privacy depends on the honesty and integrity of verifiers and decentralized identity validators. 

Idena\cite{idenaWhitepaper}, Humanode\cite{Humanode} and Civic Pass\cite{CivicPass} - all are blockchain based person identification system where Idena performs validation by conducting flip tests and Humanode and Civic Pass are developed on crypto-biometric network. Though Idena does not collect any personally identifiable information, the behavioral data collected can be used in future for pattern analysis. 

Humanode and Civic pass both require biometric verification (face scan) which can create concerns among users about how their sensitive credential (face) will be managed by the systems. It is noteworthy that, most of the platforms are decentralized (World App, Bright ID, Proof of Humanity, Idena, Humanode, Civic), some requiring cryptocurrency wallet (Proof of Humanity, Civic Pass) and some requiring biometric verification (Proof of Humanity, Humanode, Civic Pass).    %citations to be added
\par
Government issued identity documents such as Aadhaar Card, Estonia's e-ID, China's social credit system and Japan's My Number Card are controlled and managed by central government. Citizens' sensitive credential can be at high risk if the government's security system is not robust enough to prevent any kind of hacking or data breaching. China's social credit system monitors citizen data extensively without maintaining complete transparency about data use and management. 

ID.me is another online identity network that enables individuals to verify their legal identity digitally. But privacy concerns arises as a single company holds a large amount of personal data and users have limited control over their data. %citations to be added
\par
Usability across these different platforms are critical. CAPTCHAs have become increasingly difficult to solve, often leading users to leave the website or platform without completing their user journey. Accessibility remains another issue as visually impaired users are unable to solve any CAPTCHA that is text or image based. reCAPTCHA comes with the solution of these problems but trading of users' privacy as users' data is being tracked down by the authority. 

From Figure 3 and 4, it is apparent that World app and Bright ID provide a simple and intuitive account creation form but an introductory video or step by step guide would be more helpful to guide users to navigate throughout the applications and perform necessary actions.
 \begin{figure}
 	\centering
 	\includegraphics[width=\linewidth]{Fig/world app account creation.png}
 	\caption{World App's account creation process is simple but doesn't inform users about how they should navigate or use the app \fixme{silvia, is there a reason you added all these UIs? why all of the uis are randomly placed, I shared examples so many times, i am not seeing anything I gave instruction.}}
 \label{fig:World App's on-boarding process}
 \end{figure}
 
 \begin{figure}
 	\centering
 	\includegraphics[width=\linewidth]{Fig/bright id account creation.png}
 	\caption{The "Create my BrightID" process in the Bright ID app is straightforward but lacks guidance on how users should navigate or utilize the app effectively. \fixme{explain why these screenshots are important to add from cognitive walkthrough. caption itself should be self explanatory with text explaining in the main body}}
    
 \label{fig:Bright ID's on-boarding process}
 \end{figure}
The principle of error prevention could make the user journey of registration in Proof of Humanity more user-friendly. As there is no option to correct or update any mistake, it may increase user frustration. Idena's validation test (flip test) (Figure 4) can be inconvenient for new users as they may struggle to find the meaningful story in the provided time and collect points to validate them.
 \begin{figure}
 	\centering
 	\includegraphics[width=\linewidth]{Fig/idena.png}
 	\caption{Idena validation test interface requiring users to select meaningful stories within a time limit which can be challenging for new users \fixme{anyone reading this caption would not understand anything}}
    
 \label{fig:Selecting meaningful story for validation process on Idena}
 \end{figure}
The platforms requiring video selfie or face scan (Proof of Humanity, Humanode, Civic Pass) may create another challenging situation for users as proper lighting and optimal devices are necessary for taking the appropriate photo or video for biometric verification. 

Aadhaar card\cite{Aadhaar}\cite{AadhaarEnrollment}, Estonia's e-ID and Japan's My Number Card are all government based credentials but completing all the formalities and getting the card takes a long time, sometimes creating user frustration. The existing systems and platforms that we have evaluated can hardly strike a balance between privacy, functionality and usability.   %citations to be added


% \begin{figure}[!t]
% 	\centering
% 	\includegraphics[width=\linewidth]{Fig/world app.png}
% 	\caption{New account creation process in  World App}
    
% \label{fig:New account creation process in  World App}
% \end{figure}
% \begin{figure}[!t]
% 	\centering
% 	\includegraphics[width=\linewidth]{Fig/bright id.png}
% 	\caption{New account creation process in  Bright ID}
    
% \label{fig:New account creation process in  Bright ID}
% \end{figure}


\subsection{Reddit Analysis}
%\tanusree{ishan to add}
We first collected \fixme{X} posts and  \fixme{X} comments on December 24th, 2024, using the Python Reddit API Wrapper (PRAW)~\footnote{https://praw.readthedocs.io/en/stable/}. We gathered the data from various relevant subreddits, ensuring a broad and comprehensive understanding of what users discuss on identify verification or personhood verification. Through qualitative analysis of this Reddit data, we were able to uncover detailed insights into the rich and prevalent usage of verification systems. This analysis highlighted users' current usage, potential challenges and risks they encounter. These findings provide a solid foundation to explore these themes further in subsequent in-depth interviews.

\paragraph{Data Collection}
 To comprehensively cover content related to our research questions on personhood verification, we first created a list of search keywords by identifying close terminologies related to \textit{``personhood verification''} (general keywords) and \textit{``bot check''} (technology-focused keywords), etc. We utilized a combination of general and technology-focused keywords in our search. We employed general terms such as Personhood Verification, Identity Proof, Human Check and Bot Check. These keywords were designed to capture posts authored by or discussing personhood verification. For the technology focus, we used terms such as \fixme{add}. These keywords targeted discussions specifically about the use of popular tools and platforms. We conducted open searches combining these keywords across Reddit to gather data from various subreddits.
 Other than open searches, we also applied specific criteria to select subreddits, ensuring comprehensive coverage of relevant discussions: these subreddits should focus either on the personhood verification community or technology. We chose subreddits with the most active users online during our browsing sessions. The full list of subreddits and search keywords used is detailed in Table\fixme{need to find out the subreddit most prevalent discussing these topic}. 

\paragraph{Analysis}
Two researchers reviewed each post and categorized related posts or comments into five overarching high-level themes: \fixme{need to add after data analysis}. Within these categories, 53 level-2 themes were identified, such as \fixme{need to add after data analysis}. During the analysis process, researchers regularly convene to discuss discrepancies and emerging themes in the codebook, aiming to reach a consensus. These categories allowed us to investigate RQ2 and partially address RQ1. 

\subsection{Results}
% \tanusree{ishan to add}
\fi
\vspace{-2mm}
\section{ User Study Method}
\vspace{-2mm}
\label{user-study}
This section outlines the method for exploring users' perceptions and preferences of personhood credentials. We conducted semi-structured interviews with 23 participants from the US, and the EU/UK in October 2024.
%We started with a round of pilot studies (n=5) to validate the interview protocol. Based on the findings of pilot studies, we revised the interview protocol and conducted the final round of interviews (n=17). 
The study was approved by the Institutional Review Board (IRB).
\vspace{-2mm}
\subsection{Participant Recruitment}
\vspace{-2mm}
We recruited participants through (1) social media posts, (2) online crowdsourcing platforms, including CloudResearch and Prolific. Respondents were invited to our study if they met the selection criteria: a) 18 years or older and b) living in the US or the EU/UK. Participation was voluntary, and participants were allowed to quit anytime. Each participant received a \$30 Amazon e-gift card upon completing an hour-and-a-half interview.

\subsection{Participants}
%\tanusree{check for final count} \ayae{updated percentage with final 23 counts} 
We interviewed 23 participants, 10 from the US and 12 from the EU/UK. The majority of the participants (61\%) were in the age range of 25-34, followed by 22\% were 35-44 years old. The participants were from the United States and various countries, namely Spain, Sweden, Germany, Hungary, and the United Kingdom. Participants had different backgrounds of education levels, with 87\% of participants holding a Bachelor’s degree and 65\% holding a graduate degree. 65\% of participants had a technology background, while 48\% of them had a CS background. All participants reported using online services that required them to verify their personhood. Table~\ref{table:demographics} presents the demographics of our participants. We refer to participants as P1,. . . ,23.
\begin{table*}[h!]
\centering
%\scriptsize
\caption{Overview of PHC Application Scenarios}
\label{table:scenario}
%\resizebox{\textwidth}{!}{%
\begin{tabular}{lll}
\hline
\textbf{Scenario} & \textbf{Service} & \textbf{Credential} \\
\hline
Financial service & Bank, Financial institutions & Passport or Driver’s license, Face scan \cite{yousefi2024digital}\\
% \hline
Healthcare service & Hospitals, Clinics & Health insurance card,  Fingerprint \cite{chen2012non,fatima2019biometric,jahan2017robust}\\
% \hline
Social media & Tech companies & National identity card, Video selfie \cite{instagramWaysVerify, metaTypesID,instagramTypesID} \\
% \hline
LLM application & Tech companies & Iris scan \cite{WorldWhitepaper, worldHumanness}\\
% \hline
Government service & Government & Driver’s license or National identity card \cite{LogingovVerify}\\
% \hline
Employment background check & Background check companies & Tax identification card, Fingerprint\cite{cole2009suspect}\\
\hline
\end{tabular}%
%}
% \vspace{0.5em}
\label{tab:scenarios}
\end{table*}
\begin{table*}[h]
\centering
\caption{Participant demographics and background.}
%\fixme{add the participants you completed so far}
\resizebox{\textwidth}{!}{%
\begin{tabular}{l l l l l l l l}
\hline
\textit{Participant ID} & \textit{Gender} & \textit{Age} & \textit{Country of residence} & \textit{Education} & \textit{Technology background}  & \textit{CS background} &\textit{Residency duration} \\
\hline
P1 & Male & 25-34 & the US & Master's degree & Yes & Yes &3-5 years\\
P2 & Female & 25-34 & the US & Master's degree & Yes & Yes & 1-3 years\\
P3 & Female & 25-34 & the UK & Master's degree & Yes & No & 1-3 years\\
P4 & Female & 35-44 & the UK & Some college, but no degree & Yes & Yes & Over 10 years \\
P5 & Male & 25-34 & the US & Doctoral degree & Yes & Yes & 5-10 years \\
P6 & Male & 35-44 & the US & Less than a high school diploma & No & No & Over 10 years \\
P7 & Male & 25-34 & the US & Doctoral degree & Yes & Yes & 3-5 years\\
P8 & Male & 45-54 & the US & Bachelor's degree & Yes & Yes & Over 10 years \\
P9 & Female & 25-34 & New Zealand & Master's degree & No  &  No &  Over 10 years\\
P10 & Male & 25-34 & the US & Master's degree & No & No & Over 10 years\\
P11 & Female & 25-34 & the UK & Bachelor's degree & No & No & Over 10 years\\
P12 & Male & 18-24 & the UK & Master's degree & Yes & Yes & 1-3 years\\
P13 & Male & 35-44 & the UK & Bachelor's degree & Yes & No & Over 10 years\\
P14 & Male & 25-34 & Sweden & High school graduate & No & No & Over 10 years \\
P15 & Female & 25-34 & Spain & Master's degree & Yes & Yes & Over 10 years \\
P16 & Female & 25-34 & Germany & Master's degree & Yes & Yes & Over 10 years \\
P17 & Female & 25-34 & Spain & Doctoral degree & No & No & Over 10 years \\
P18 & Female & 35-44 & the US & Bachelor's degree & No & No & Over 10 years \\
P19 & Female & 25-34 & Germany & Master's degree & Yes & Yes & 3-5 years \\
P20 & Male & 25-34 & Hungary & Master's degree & Yes & No & 3-5 years \\
P21 & Male & 35-44 & the US & Bachelor's degree & Yes & No & 5-10 years \\
P22 & Female & 18-24 & France & Master's degree & Yes & Yes & Less than 1 year\\
P23 & Male & 45-52 & the US & Master's degree & No & No & Over 10 years\\
\hline
\end{tabular}%
}
\label{table:demographics}
\end{table*}


\vspace{-2mm}
\subsection{Semi-Structured Interview Procedure} \label{sec:study_protocol}
\vspace{-2mm}
%\fixme{explain in details why the study designed in a certain way. please read papers to learn more. data minimization and advertisement paper. The method section is too bland. We have a wonderful study design. Scenario-specific study design, describe scenarios and why chose this scenario. Mainly method should include all design rationale, and example questions when necessary to clarify your rational}

We started with a round of pilot 
%(Appendix~\ref{pilot}) 
studies (n=5) to validate the interview protocol. Based on the findings of pilot studies, we revised the interview protocol.

\textbf{Open Ended Discussion.} We designed the interview script based on our research questions outlined in the introduction section~\ref{sec:introduction}. 
%We added the interview script to the section~\autoref{protocol}. 
At the beginning of the study, we received the participants’ consent to conduct the study. Once they agreed, we proceeded with a semi-structured interview. The study protocol was structured according to the following sections: (1) Current practices regarding digital identity verification; (2) Users' perception of PHC before and after watching the informational video; (3) Scenario-based session to investigate factors that influence users' preferences of PHC; 
%(4) Users' preference of PHC; 
(4) Design session to conceptualize users' expectations; (5) A brief post-survey on Users' Preference of PHCs.
%of PHCs in different scenarios.

In the first section, we first asked a set of questions to understand participants' current practices of online platforms and the types of identity verification methods they had experience with. This is to understand their familiarity with different types of verification, such as biometrics, physical IDs, etc.
%and methods that might have worked well based on their prior experience.

%of online identity verification. When participants mentioned certain types of online services that required identity verification, we inquired about their experience with verification method. Was it easy to use, or did you run into any issues?"} We further inquired about any challenges participants faced with identity verification - \textit{"Did you encounter any challenges when using this method?"} 
%If biometrics didn’t naturally come up in prior discussions, we prompted to consider them- \textit{"Have you ever used services where you had to verify yourself through face, fingerprints, or iris scans, or other biometrics?"} If they mentioned any experience with biometric verification, we followed up with questions like- \textit{"What worked well? Were there any concerns you had?"}
In the second section, we then asked about participants' current understanding and perception of personhood credentials either from prior knowledge or from intuition by just hearing the term. %We also asked if they knew how personhood credentials work, particularly how it has been handled by the different services they use. 
%As all participants had never heard of PHC, we prompted them to interpret the term based solely on its wording. 
While the majority recognized this as unfamiliar terminology, most inferred that it referred to a form of personal identification, often associating it with biometric verification.
%In the pilot interviews, The majority of the participants could not provide substantial responses on their understanding of how personhood credentials work, before starting the second part of the interview, we showed them an informational video on personhood credentials.
%Most of the participants were unfamiliar with this term, so we then asked \textit{ Can you explain what you think it means by just hearing the term?"} 
%Before proceeding with the third section of the interview, we assessed participants' understanding of PHC with knowledge questions.
Then, we showed them an introduction video on PHC \footnote{https://anonymous.4open.science/r/PHC-user-study-14BB/}, %\fixme{create an anonymous GitHub, upload the video and add a footnote here} \ayae{reflected}. 
%The video provides an overview of PHCs, 
covering their definition, 
%the steps involved in issuing and using them, 
and implications of it in online services. Based on former literature\cite{adler2024personhood}, we designed the video with easy-to-understand text, visuals, and audio to make the concepts accessible to average users. We created a set of knowledge questions to assess participants' understanding of PHC before and after showing the video. %as attached in Appendix~\ref{knowledge_questions}.

%including the same knowledge questions. 
%Most participants correctly responded to knowledge questions, which ask the basic understanding of digital identity crisis and personhood credentials. 
%Even before showing the introduction video, regarding the question \textit{"What could happen if online identities are poorly verified?"}, 95\% correctly selected \textit{"Fake accounts, bots, and fraud could increase significantly."} For the question \textit{"What are Personhood Credentials (PHCs)?"}, 90\% correctly choose the option \textit{"Digital credentials that confirm a person’s identity."} 
For instance, we observed an improve in correct response rate for the question, such as, \textit{``What is the primary goal of PHC?''} from 85\% to 100\% after watching the video.
%where the correct answer was \textit{"To verify a person's identity without exposing personal information."} 
%However, regarding the question \textit{"To whom do you provide minimal personal information during the PHC process?"}, only 35\% selected the correct answer \textit{"PHC issuers (e.g., governments or trusted organizations)"}, while the most frequent response was \textit{"Online service providers (e.g., social media)"} at 45\%.
%\ayae{KQ results reflected}
%We also asked some open-ended questions to evaluate whether our introduction video helped participants better understand PHC \textit{''How would you explain your understanding of personhood credentials?''} 
%We further asked what benefits and concerns came to mind for them.
In the third section, we focused on scenario-based discussions, exploring specific applications of PHC to understand factors that influence participants' preferences towards PHCs as well as identify challenges to leverage in PHC design for various services. We examined the following six scenarios: (1) Financial service, (2) Healthcare service, (3) Social Media, (4) LLM applications, (5) Government Portal, and (6) Employment Background Check.
%We covered a wide range of use cases of online personhood verification via these six scenarios since they encompass diverse user needs, security and usability, and privacy requirements. %\fixme{please see the comment with iffalse tag and make it concise, we talked about it before}
\iffalse
%Firstly, financial system is a critical scenario for identity verification where high level of security protections are expected as exemplified by KYC. Thus, such services continue to develop transformative digital identity verification to ensure the security and integrity of financial transactions\cite{parate2023digital}. The second scenario is healthcare systems, which also have high privacy requirements due to the confidentiality of medical data. The pandemic has accelerated the adoption of online healthcare services and in response to this digital transformation, the recent study has proposed blockchain-based decentralized identity management systems \cite{javed2021health}. Thirdly, we consider the scenario of social media, which faces the critical challenges of online identity as shown in spreading misinformation and harmful content from fake or anonymous accounts \cite{ceylan2023sharing}. The fourth scenario is designed with a specific context of interacting with Large language models (LLMs). The former study discussed vulnerability in dialog-based systems where adversaries can exploit the training process to introduce toxicity into responses \cite{weeks2023first}. Thus, such vulnerabilities indicate identity verification may also be important for LLM applications. Fifth, government services are familiar situations that require people to verify their identity. Various countries have developed their own electronic ID schemes \cite{stalla2018gdpr}. Lastly, we also cover the scenario of employment background checks needing precise identity verification to ensure the reliability of applicants. The current background check system involves vulnerable processes that increase the risk of identity theft and unauthorized data access.\cite{blowers2013national}. Such challenges underline the relevance of PHCs, which can mitigate risks by providing a secure framework for verification.
%\ayae{included citation}
\fi
We have also incorporated various types of data or credentials requirements (e.g. physical id, biometrics, etc) across scenarios to maintain diversity in our discussion with participants as shown in Table.\ref{table:scenario}. %For instance,
%we  We have multiple existing verification methods, including 
%humanness verification (e.g., selfie, video call), document-based verification (e.g., government-issued ID), and biometrics information. 
We selected types of credentials for each scenario based on former literature and existing PHC as explained in the section \ref{subsec:verification_practice}. %\fixme{cite worldcoin, and other app and literature}. \fixme{from here to end of this paragraph ---These needs to go to the literature review section on the current usecase of PHC. And only 2 line summarizing why you chose the diverse type of credential data and refer to the literature section}


%% Let me find the former literature to explain why we select these credentials
For each of the six scenarios, we explored participants' perceptions of using PHC in hypothetical situations that align with the research focus as well as to help participants can relate PHC concepts to real-world applications. This is particularly useful for this study where where user perceptions and expectation under specific conditions are crucial to devising solutions \cite{carroll2003making}.
%\fixme{cite scenario method paper from jack caroll}.
%\ayae{reflected}
We asked about their feelings, perceived benefit and risks. We also nudge them to think about any privacy and security perception around using PHC and types of data (e.g., iris, face, government id, etc) involved in issuing PHC. 

\fixme{
%\textbf{Pre-understanding: Guessed it as one of the verification methods} 
%The majority of the participants were not familiar with the term ``Personhood Credential'', although most of them used some forms of such credentials. 
%As all participants have never heard of PHC, we prompted them to interpret the term based solely on its wording. Most of them inferred that it referred to another type of person identification. 
%For instance, P3 commented \textit{``It can be anything that would point to one single individual that would differentiate that individual from others.''} When participants expressed how PHC identifies a person's uniqueness, their understanding ranged from verifying basic information such as address or age, and certain eligibility to advanced identification of digital identity (e.g., behavioral, economical, etc) with Multi-factor authentication or knowledge-based questions.

%\textbf{Post-understanding: Involvement of trusted entity} When asked to explain their understanding of PHC, P13 noted, \textit{"So it sounds like, basically, you it's similar to how you verify things before. Like you use a biometrics and your government Id. But then you get a personal key. You do it with like a trusted organization rather than each individual. And then you can use that key for all the different services you use."} P1 elaborated PHC process as a shift of the verification entity, \textit{" I'd say we are sort of moving the verification burden from the user side to a service provider side where they have access to our data, and they have access to the token that's assigned to each person that's unique. And that's easily like traceable across online platforms. and this token is used for verification with 3rd parties, where they don't get access to your personal data, but they only use this service provider to give them the authenticity that you are a real user."} These suggest that the role of the PHC issuer is recognized as a crucial component of PHC.}
%began by asking \textit{"How did you feel about using PHC to verify your identity when opening your bank account?"} To dive deeper, we also asked about potential benefits:\textit{"What potential benefits do you see in using PHC in this online banking context?"}. We also inquired about these aspects- \textit{"Do you think using PHC improves the security of your bank account? Why?", "Did this method of identity verification make you feel more confident about your privacy? why?"} Additionally, we discussed their comfort levels for providing credentials (e.g., Government-issued ID, biometric information) and asked about any concerns about data collection-\textit{"Were you comfortable providing your government-issued ID and using facial recognition? Why?"}
}

\iffalse
%%% column: scenario, credential, service providers.
\begin{table*}[h!]
\centering
\caption{Overview of PHC Application Scenarios}
\label{table:scenario}
%\resizebox{\textwidth}{!}{%
\begin{tabular}{lll}
\hline
\textbf{Scenario} & \textbf{Service Provider} & \textbf{Types of Credential} \\
\hline
Financial Service & Bank, Financial Institutions & Passport or Driver’s license, Face scan \cite{yousefi2024digital}\\
% \hline
Healthcare Service & Hospitals, Clinics & Health insurance card,  Fingerprint \cite{chen2012non,fatima2019biometric,jahan2017robust}\\
% \hline
Social Media & Tech Companies & National identity card, Video selfie \cite{instagramWaysVerify, metaTypesID,instagramTypesID} \\
% \hline
LLM Application & Tech Companies & Iris scan \cite{WorldWhitepaper, worldHumanness}\\
% \hline
Government Service & Government & Driver’s license or National identity card \cite{LogingovVerify}\\
% \hline
Employment Background Check & Background Check Companies & Tax identification card, Fingerprint\cite{cole2009suspect}\\
\hline
\end{tabular}%
%}
% \vspace{0.5em}
\label{tab:scenarios}
\end{table*}
\fi


%\textbf{Design Session.}
%\fixme{need to explain how and why you design the design session, where you designed, how participants were unstructured and so on.} \ayae{reflected in the following paragraph}

In the fourth section, we began by refreshing participants’ memories of the various risks and concerns discussed in the earlier scenario-based section. Following this, we guided participants to brainstorm potential design solutions by sketching their ideas to address these concerns. To facilitate the sketching process, we developed sketch notes in Zoom as prompts to help participants generate ideas, particularly when starting from scratch is challenging. 
%on Zoom whiteboard or pen and paper, using a think-aloud protocol.  
%Nevertheless, it is difficult to develop new ideas from scratch, so 
%Additionally, we described the main issues or concerns that the participants identified during the interview at the top of the sketch notes. 
%Participants can develop their ideas at the center of the whiteboard by locating the above components or creating new shapes, lines, or text boxes for their sketches. 
We also investigated participants' preferences for PHC regarding the issuers and issuance systems of PHCs, as well as the types of data required for issuing PHCs. 
%in the context of who issues PHC or type of issuance systems, and what types of data are needed to issue PHC to address RQ2. 
%An example includes- \textit{``What types of credential would you prefer to use as personhood verification? ; Which organizations or stakeholders would you prefer to issue and manage your PHC?''} 
We encourage participants to explain their reasoning. These questions were informed by insights from the pilot study, where participants expressed preferences for different types of data, system architecture, and various stakeholders involved in PHC issuing.
%However, these questions alone can only find optimal ways within the scope of currently existing options and cannot generate new design implications. Therefore,

\iffalse
\tanusree{we can cut this section as this didn't give any result and doesn't answer RQs directly.}Lastly, to understand preference on issuance system, we introduced the decentralized PHC system architecture with another instructional video. Following the video, we asked participants to explain their understanding of the decentralized PHC system and their preference for the issuance system (centralized or decentralized). We introduced it after the sketch session is that participants may organically come up with the idea of decentralized systems on their own, and we intended to avoid priming them. 
\fi
%Then, we asked them to explain their understanding of the decentralized PHC and preferred issuance system (centralized or decentralized.)- \textit{`` Could you explain why you would prefer decentralized system in managing your PHCs?''}
%\textit{"Would you prefer to get multiple PHCs from different issuers depending on the situation or application you're using, or would you rather have a single PHC from one issuer?"}

\textbf{Post-Survey.}
%%\fixme{need to explain how and why you design the design session, where you designed, how participants were unstructured and so on.}
We conducted a post-survey to obtain participants' PHC preference quantitatively. It included questions on participants' preference on credential type, issuer and issuance system  for the scenarios (e.g., financial, medical, etc) we considered in our interview.

\vspace{-2mm}
\subsection{Data Analysis}
\vspace{-2mm}
Once we got permission from the participants, we obtained interview data through the audio recording and transcription on Zoom. We analyzed these transcribed scripts through thematic analysis \cite{Braun2012-sz, Fereday2006-yv}. Firstly, all of the pilot interview data was coded by two researchers independently. Then, we compared and developed new codes until we got a consistent codebook. Following this, both coders coded 20\% of the interview data of the main study. We finalized the codebook by discussing the coding to reach agreements. Lastly, we divided the remaining data and coded them. After both researchers completed coding for all interviews, they cross-checked each other’s coded transcripts and found no inconsistencies. Lower-level codes were then grouped into sub-themes, from which main themes were identified. Lastly, these codes were organized into broader categories. Our inter-coder reliability (0.90) indicated a reasonable agreement between the researchers.
\iffalse

\begin{table*}[h]
\centering
\caption{Participant demographics and background.}
%\fixme{add the participants you completed so far}
\resizebox{\textwidth}{!}{%
\begin{tabular}{l l l l l l l l}
\hline
\textit{Participant ID} & \textit{Gender} & \textit{Age} & \textit{Country of residence} & \textit{Education} & \textit{Technology background}  & \textit{CS background} &\textit{Residency duration} \\
\hline
P1 & Male & 25-34 & the US & Master's degree & Yes & Yes &3-5 years\\
P2 & Female & 25-34 & the US & Master's degree & Yes & Yes & 1-3 years\\
P3 & Female & 25-34 & the UK & Master's degree & Yes & No & 1-3 years\\
P4 & Female & 35-44 & the UK & Some college, but no degree & Yes & Yes & Over 10 years \\
P5 & Male & 25-34 & the US & Doctoral degree & Yes & Yes & 5-10 years \\
P6 & Male & 35-44 & the US & Less than a high school diploma & No & No & Over 10 years \\
P7 & Male & 25-34 & the US & Doctoral degree & Yes & Yes & 3-5 years\\
P8 & Male & 45-54 & the US & Bachelor's degree & Yes & Yes & Over 10 years \\
P9 & Female & 25-34 & New Zealand & Master's degree & No  &  No &  Over 10 years\\
P10 & Male & 25-34 & the US & Master's degree & No & No & Over 10 years\\
P11 & Female & 25-34 & the UK & Bachelor's degree & No & No & Over 10 years\\
P12 & Male & 18-24 & the UK & Master's degree & Yes & Yes & 1-3 years\\
P13 & Male & 35-44 & the UK & Bachelor's degree & Yes & No & Over 10 years\\
P14 & Male & 25-34 & Sweden & High school graduate & No & No & Over 10 years \\
P15 & Female & 25-34 & Spain & Master's degree & Yes & Yes & Over 10 years \\
P16 & Female & 25-34 & Germany & Master's degree & Yes & Yes & Over 10 years \\
P17 & Female & 25-34 & Spain & Doctoral degree & No & No & Over 10 years \\
P18 & Female & 35-44 & the US & Bachelor's degree & No & No & Over 10 years \\
P19 & Female & 25-34 & Germany & Master's degree & Yes & Yes & 3-5 years \\
P20 & Male & 25-34 & Hungary & Master's degree & Yes & No & 3-5 years \\
P21 & Male & 35-44 & the US & Bachelor's degree & Yes & No & 5-10 years \\
P22 & Female & 18-24 & France & Master's degree & Yes & Yes & Less than 1 year\\
P23 & Male & 45-52 & the US & Master's degree & No & No & Over 10 years\\
\hline
\end{tabular}%
}
\label{table:demographics}
\end{table*}
\fi


\section{Fine-Tuning Experiments}
This section validates that our dataset can enhance the GUI grounding capabilities of VLMs and that the proposed functionality grounding and referring are effective fine-tuning tasks.
\subsection{Experimental Settings}
\noindent\textbf{Evaluation Benchmarks} We base our evaluation on the UI grounding benchmarks for various scenarios: \textbf{FuncPred} is the test split from our collected functionality dataset. This benchmark requires a model to locate the element specified by its functionality description. \textbf{ScreenSpot}~\citep{cheng2024seeclick} is a benchmark comprising test samples on mobile, desktop, and web platforms. It requires the model to locate elements based on short instructions. \textbf{RefExp}~\citep{Bai2021UIBertLG} is to locate elements given crowd-sourced referring expressions. \textbf{VisualWebBench (VWB)}~\citep{liu2024visualwebbench} is a comprehensive multi-modal benchmark assessing the understanding capabilities of VLMs in web scenarios. We select the element and action grounding tasks from this benchmark. To better align with high-level semantic instructions for potential agent requirements and avoid redundancy evaluation with ScreenSpot, we use ChatGPT to expand the OCR text descriptions in the original task instructions, such as \textit{Abu Garcia College Fishing} into functionality descriptions like \textit{This element is used to register for the Abu Garcia College Fishing event}.
\textbf{MOTIF}~\citep{Burns2022ADF} requires an agent to complete a natural language command in mobile Apps.
For all of these benchmarks, we report the grounding accuracy (\%): $\text { Acc }= \sum_{i=1}^N \mathbf{1}\left(\text {pred}_i \text { inside GT } \text {bbox}_i\right) / N \times 100 $ where $\mathbf{1}$ is an indicator function and $N$ is the number of test samples. This formula denotes the percentage of samples with the predicted points lying within the bounding boxes of the target elements.

\noindent\textbf{Training Details}
We select Qwen-VL-10B~\citep{bai2023qwen} and SliME-8B~\citep{slime} as the base models and fine-tune them on 25k, 125k, and 702k samples of the AutoGUI training data to investigate how the AutoGUI data enhances the UI grounding capabilities of the VLMs. The models are fine-tuned on 8 A100 GPUs for one epoch. We follow SeeClick~\citep{cheng2024seeclick} to fine-tune Qwen-VL with LoRA~\citep{hu2022lora} and follow the recipe of SliME~\citep{slime} to fine-tune it with only the visual encoder frozen (More details in Sec.~\ref{sec:supp:impl details}).

\noindent\textbf{Compared VLMs}
We compare with both general-purpose VLMs (i.e., LLaVA series~\citep{liu2023llava,liu2024llavanext}, SliME~\citep{slime}, and Qwen-VL~\citep{bai2023qwen}) and UI-oriented ones (i.e., Qwen2-VL~\citep{qwen2vl}, SeeClick~\citep{cheng2024seeclick}, CogAgent~\citep{hong2023cogagent}). SeeClick finetunes Qwen-VL with around 1 million data combining various data sources, including a large proportion of human-annotated UI grounding/referring samples. CogAgent is trained with a huge amount of text recognition, visual grounding, UI understanding, and publicly available text-image datasets, such as LAION-2B~\citep{LAION5B}. During the evaluation, we manually craft grounding prompts suitable for these VLMs.
\subsection{Experimental Results and Analysis}
\begin{table}[]
\scriptsize
\centering
\caption{\textbf{Element grounding accuracy on the used benchmarks.} We compare the base models fine-tuned with our AutoGUI data and representative open-source VLMs. The results show that the two base models (i.e. Qwen-VL and SliME-8B) obtain significant performance gains over the benchmarks after being fine-tuned with AutoGUI data. Moreover, increasing the AutoGUI data size consistently improves grounding accuracy, demonstrating notable scaling effects. $\dag$ means the metric value is borrowed from the benchmark paper. $*$ means using additional SeeClick training data.}
\label{tab:eval results}
\begin{tabular}{@{}cccccccccc@{}}
\toprule
Type & Model    & Size    & FuncPred & VWB EG & VWB AG & MoTIF & RefExp & ScreenSpot  \\ \midrule
\multirow{5}{*}{General} & LLaVA-1.5~\citep{liu2023llava} & 7B & 3.2      &        12.1$^{\dag}$        &     13.6$^{\dag}$           &  7.2   &  4.2 & 5.0 & \\
 & LLaVA-1.5~\citep{liu2023llava} & 13B & 5.8      &           16.7     &        9.7        &   12.3 &  20.3   & 11.2 &  \\
 & LLaVA-1.6~\citep{liu2024llavanext} & 34B &  4.4      &      19.9          &    17.0            &   7.0 &  29.1  & 10.3 &  \\
 & SliME~\citep{slime} & 8B &  3.2  &   6.1       &     4.9     & 7.0  &  8.3  &  13.0  \\ 

 & Qwen-VL~\citep{bai2023qwen} & 10B &  3.0     &      1.7          &      3.9          &    7.8 &  8.0  & 5.2$^{\dag}$   \\ 
 \midrule
\multirow{3}{*}{UI-VLM} &  Qwen2-VL~\citep{bai2023qwen}  & 7B     &     7.8       &    3.9        &  3.9  &  16.7 & 32.4 & 26.1    \\
 & CogAgent~\citep{hong2023cogagent} & 18B    &  29.3   &    \underline{55.7}      &    \textbf{59.2}      & \textbf{24.7}   & 35.0 &  47.4$^{\dag}$  \\
 & SeeClick~\citep{cheng2024seeclick} & 10B    &    19.8     &    39.2           &     27.2           & 11.1  &  \textbf{58.1}  & \underline{53.4}$^{\dag}$ \\ 
\midrule
\multirow{4}{*}{Finetuned} &  Qwen-VL-AutoGUI25k & 10B      &    14.2     &      12.8         &    12.6           &   10.8    &  12.0 & 19.0    \\
 & Qwen-VL-AutoGUI125k  & 10B       &     25.5     &      23.2         &        29.1       &    11.5   &  14.9 & 32.0     \\ 
 & Qwen-VL-AutoGUI702k  & 10B       &   43.1   &    38.0       &     32.0    &  15.5  & 23.9 &    38.4   \\
& Qwen-VL-AutoGUI702k$^*$   & 10B     &  \underline{50.0}  &    \textbf{56.2}    &  \underline{45.6}  & \underline{21.0} & \underline{51.5} & \textbf{54.2}      \\
\midrule
\multirow{3}{*}{Finetuned} & SliME-AutoGUI25k  & 8B     &   28.0   &     14.0      &      10.6      &  14.3   & 18.4 & 27.2   \\
 & SliME-AutoGUI125k   & 8B      &   39.9    &  22.0   &     12.0       &  17.8  & 22.1 &  35.0     \\
 & SliME-AutoGUI702k   & 8B      &     \textbf{62.6}   &       25.4        &     13.6          &   20.6    & 26.7 & 44.0 &          \\
\bottomrule
\end{tabular}
\end{table}
\vspace{-2mm}


\noindent\textbf{A) AutoGUI functionality annotations effectively enhance VLMs' UI grounding capabilities and achieve scaling effects.} We endeavor to show that the element functionality data autonomously collected by AutoGUI contributes to high grounding accuracy. The results in Tab.~\ref{tab:eval results} demonstrate that on all benchmarks the two base models achieve progressively rising grounding accuracy as the functionality data size scales from 25k to 702k, with SliME-8B's accuracy increasing from merely \textbf{3.2} and \textbf{13.0} to \textbf{62.6} and \textbf{44.0} on FuncPred and ScreenSpot, respectively. This increase is visualized in Fig.~\ref{fig:funcpred scaling success} showing that increasing AutoGUI data amount leads to more precise localization performance.

After fine-tuning with AutoGUI 702k data, the two base models surpass SeeClick, the strong UI-oriented VLM on FuncPred and MOTIF. We notice that the base models lag behind SeeClick and CogAgent on ScreenSpot and RefExp, as the two benchmarks contain test samples whose UIs cannot be easily recorded (e.g., Apple devices and Desktop software) as training data, causing a domain gap. Nevertheless, SliME-8B still exhibits noticeable performance improvements on ScreenSpot and RefExp when scaling up the AutoGUI data, suggesting that the AutoGUI data helps to enhance grounding accuracy on the out-of-domain tasks.

To further unleash the potential of the AutoGUI data, the base model, Qwen-VL, is finetuned with the combination of the AutoGUI and SeeClick UI-grounding data. This model becomes the new state-of-the-art on FuncPred, ScreenSpot, and VWB EG, surpassing SeeClick and CogAgent. This result suggests that our AutoGUI data can be mixed with existing UI grounding training data to foster better UI grounding capabilities.

In summary, our functionality data can endow a general VLM with stronger UI grounding ability and exhibit clear scaling effects as the data size increases.


\begin{table}[]
\centering
\footnotesize
\caption{\textbf{Comparing the AutoGUI functionality annotation type with existing types}. Qwen-VL is fine-tuned with the three annotation types. The results show that our functionality data leads to superior grounding accuracy compared with the naive element-HTML data and the condensed functionality annotations.}
\label{tab:ablation}
\begin{tabular}{@{}ccccc@{}}
\toprule
Data Size             & Variant          & FuncPred & RefExp & ScreenSpot \\ \midrule
\multirow{3}{*}{25k}  & w/ Elem-HTML data     &  5.3      &  4.5   &    5.7     \\
                      & w/ Condensed Func. Anno.     &  3.8   &  3.0  &   4.8      \\
                      & w/ Func. Anno. (Ours full)         &    \textbf{21.1}    &   \textbf{10.0}   &   \textbf{16.4}    \\ \midrule
\multirow{3}{*}{125k} & w/ Elem-HTML data     &  15.5   &  7.8  &   17.0      \\
                      & w/ Condensed Func. Anno.     &  14.1   &  11.7  &   23.8      \\
                      & w/ Func. Anno. (Ours full)         &  \textbf{24.6}   &  \textbf{12.7}  &   \textbf{27.0}    \\ \bottomrule
\end{tabular}
\end{table}



\noindent\textbf{B) Our functionality annotations are effective for enhancing UI grounding capabilities.} To assess the effectiveness of functionality annotations, we compare this annotation type with two existing types: 1) \textbf{Naive element-HTML pairs}, which are directly obtained from the UI source code~\citep{hong2023cogagent} and associate HTML code with elements in specified areas of a screenshot. Examples are shown in Fig.~\ref{fig: functionality vs others}. To create these pairs, we replace the functionality annotations with the corresponding HTML code snippets recorded during trajectory collection. 2) \textbf{Brief functionality descriptions} that are generated by prompting GPT-4o-mini\footnote{https://openai.com/index/gpt-4o-mini-advancing-cost-efficient-intelligence/} to condense the AutoGUI functionality annotations. For example, a full description such as \textit{`This element provides access to a documentation category, allowing users to explore relevant information and guides'} is shortened to \textit{`Documentation category access'}.

After experimenting with Qwen-VL~\citep{bai2023qwen} at the 25k and 125k scales, the results in Tab.~\ref{tab:ablation} show that fine-tuning with the complete functionality annotations is superior to the other two types. Notably, our functionality annotation type yields the largest gain on the challenging FuncPred benchmark that emphasizes contextual functionality grounding. In contrast, the Elem-HTML type performs poorly due to the noise inherent in HTML code (e.g., numerous redundant tags), which reduces fine-tuning efficiency. The condensed functionality annotations are inferior, as the consensing loses details necessary for fine-grained UI understanding. In summary, the AutoGUI functionality annotations provide a clear advantage in enhancing UI grounding capabilities.


\subsection{Failure Case Analysis}
After analyzing the grounding failure cases, we identified several failure patterns in the fine-tuned models: a) difficulty in accurately locating small elements; b) challenges in distinguishing between similar but incorrect elements; and c) issues with recognizing icons that have uncommon shapes. Please refer to Sec.~\ref{sec:supp:case analysis} for details.




\section{Conclusion}\label{sec:main_conclusion}

This paper provides a fresh perspective on existing DG algorithms in the context of limited training domains, analyzed through the lens of necessary and sufficient conditions for generalization. Our analysis reveals that the failure of conventional DG algorithms arises from their focus on promoting sufficient conditions while neglecting and often inadvertently violating necessary conditions. Furthermore, we provide new insights into two recent strategies, ensemble learning and information bottleneck. The success of ensemble learning lies in its promotion of the necessary condition of the "Invariance-Preserving Representation Function." In contrast, the information bottleneck approach proves ineffective for generalization as it violates this condition, contradicting findings from previous research.

\section*{Impact Statement}

``This paper presents work whose goal is to advance the field of 
Machine Learning. There are many potential societal consequences 
of our work, none which we feel must be specifically highlighted here.''

\bibliography{references}
\bibliographystyle{icml2025}


%%%%%%%%%%%%%%%%%%%%%%%%%%%%%%%%%%%%%%%%%%%%%%%%%%%%%%%%%%%%%%%%%%%%%%%%%%%%%%%
%%%%%%%%%%%%%%%%%%%%%%%%%%%%%%%%%%%%%%%%%%%%%%%%%%%%%%%%%%%%%%%%%%%%%%%%%%%%%%%
% APPENDIX
%%%%%%%%%%%%%%%%%%%%%%%%%%%%%%%%%%%%%%%%%%%%%%%%%%%%%%%%%%%%%%%%%%%%%%%%%%%%%%%
%%%%%%%%%%%%%%%%%%%%%%%%%%%%%%%%%%%%%%%%%%%%%%%%%%%%%%%%%%%%%%%%%%%%%%%%%%%%%%%
\newpage
\appendix
\onecolumn
\section{Theoretical development}
\label{apd:proof}
In this section, we present all the proofs  of our theoretical development. 

For readers' convenience, we recapitulate our definition and assumptions:

\textit{Domain objective}: Given a domain $\mathbb{P}^e$, let the hypothesis $f:\mathcal{X}\rightarrow\Delta_{\left | \mathcal{Y} \right |}$ is a map from the data space $\mathcal{X}$ to the the $C$-simplex label space $\Delta_{\left | \mathcal{Y} \right |}:=\left\{ \alpha\in\mathbb{R}^{\left | \mathcal{Y} \right |}:\left \| \alpha \right \|_{1}=1\,\land\,\alpha\geq 0\right\}$.
%Let $\ell\left(f\left(x\right),y\right)$ be the loss incurred by using this hypothesis to predict $x= \psi_{x}(z_c, z_e, u_{x})\in\mathcal{X}$ and its corresponding label $y\in \mathcal{Y}$ is sampled as $y\sim \mathbb{P}(Y\mid z_c)$. 
Let $l:\mathcal{Y}_{\Delta}\times\mathcal{Y}\mapsto\mathbb{R}$ be a loss function, where $\ell\left(f\left(x\right),y\right)$ with
$f\left(x\right)\in\mathcal{Y}_{\Delta}$ and $y\in\mathcal{Y}$
specifies the loss (i.e., cross-entropy) to
assign a data sample $x$ to the class $y$ by the hypothesis $f$. The general 
loss of the hypothesis $f$ w.r.t. a given domain $\mathbb{P}^e$ is:
\begin{equation}
\mathcal{L}\left(f,\mathbb{P}^e\right):=\mathbb{E}_{\left(x,y\right)\sim\mathbb{P}^e}\left[\ell\left(f\left(x\right),y\right)\right].   
\end{equation}


% The %general 
% loss of the hypothesis $f$ w.r.t. a given domain $\mathbb{P}$ is:
% \begin{equation}
% \mathcal{L}\left(f,\mathbb{P}\right)=\mathbb{E}_{\left(x,y\right)\sim\mathbb{P}}\left[\ell\left(f\left(x\right),y\right)\right].    
% \end{equation}

\begin{assumption} (Label-identifiability). We assume that for any pair $z_c, z^{'}_c\in \mathcal{Z}_c$,  $\mathbb{P}(Y|Z_c=z_c) = \mathbb{P}(Y|Z_c=z^{'}_c) \text{ if } \psi_x(z_c,z_e,u_x)=\psi_x(z_c',z'_e,u'_x)$ for some $z_e, z'_e, u_x, u'_x$
\label{as:label_idf_apd}.
\end{assumption}


\begin{assumption} (Causal support). We assume that the union of the support of causal factors across training domains covers the entire causal factor space $\mathcal{Z}_c$: $\cup_{e\in \mathcal{E}_{tr}}\text{supp}\{\mathbb{P}^{e} \left (Z_c \right )\}=\mathcal{Z}_c$ where $\text{supp}(\cdot)$ specifies the support set of a distribution. 
\label{as:sufficient_causal_support_apd}
\end{assumption}


% \begin{assumption} (Sufficient causal support).  The mixture of training domain distributions is denoted as $\mathbb{P}^{\pi}=\sum_{e\in \mathcal{E}_{train}}\pi_{e}\mathbb{P}^{e}$,
% where the mixing coefficients $\pi=\left[\pi_{e}\right]_{e\in \mathcal{E}_{train}}$
% can be conveniently set to $\pi_{e}=\frac{N_{e}}{\sum_{e'\in \mathcal{E}_{train}}N_{e'}}$
% with $N_{e'}$ being the training size of the training domain $\mathbb{P}^e'$. The mixture of training domain distributions is said to have a \textit{sufficient causal support} if the support of causal factor $\mathbb{P}^\pi(Z_c)$: $\text{supp}\{\mathbb{P}^\pi(Z_c)\}=\mathcal{Z}_c$ where $\text{supp(.)}$ specifies the support set of a distribution.
% \label{as:sufficient_causal_support_apd}
% \end{assumption}
\begin{corollary}
    $\mathcal{F}\neq \emptyset$ if and only if Assumption~\ref{as:label_idf_apd} holds.
    \label{thm:existence_apd}
\end{corollary}

\begin{proof}

    The \textbf{"if"} direction is directly derived from the Proposition~\ref{thm:invariant_correlation_apd}.  We prove \textbf{"only if"} direction by contraction.

    If Assumption~\ref{as:label_idf_apd} does not hold, there a pair $x=x'$ such that $x= \psi_x(z_c,z_e,u_x)$ $x'=\psi_x(z_c',z'_e,u'_x)$ for some $z_e, z'_e, u_x, u'_x$ and $\mathbb{P}(Y|Z_c=z_c) \neq \mathbb{P}(Y|Z_c=z^{'}_c)$.

    By definition of $f\in\mathcal{F}^*$, $f(x)=\mathbb{P}(Y|Z_c=z_c)\neq \mathbb{P}(Y|Z_c=z^{'}_c)=f(x')=f(x)$ which is a contradiction. (It is worth noting that a domain containing only one sample $x$ is also valid within our data-generation process depicted in Figure~\ref{fig:graph}.).
\end{proof}



\begin{proposition} (Invariant Representation Function)
Under Assumption.\ref{as:label_idf_apd}, there exists a set of deterministic representation function $(\mathcal{G}_c\neq \emptyset)\in \mathcal{G}$ such that for any $g\in \mathcal{G}_c$, $\mathbb{P}(Y\mid g(x)) = \mathbb{P}(Y\mid z_c)$ and $g(x)=g(x')$ holds true for all $\{(x,x',z_c)\mid  x= \psi_x(z_c, z_e, u_x), x'= \psi_x(z_c, z^{'}_e, u^{'}_x) \text{ for all }z_e,z^{'}_e, u_x, u^{'}_x\}$
\label{thm:invariant_correlation_apd}
\end{proposition}

% \longvt{

% is $h(Y\mid g_c(x)) = p(Y\mid z_c)$ condition is too strong? 

% may we use $h_c\in \underset{h\in \mathcal{H}}{\text{argmin }} \ell\left ( h\circ g_c(x), p(Y\mid z_c) \right )  $ for all $\{(x,z_c)\mid  x= \psi_x(z_c, z_e) \text{ for some }z_e\}$

% }
\begin{proof}
Under Assumption.\ref{as:label_idf_apd}, we can always choose a deterministic function $g_c: \mathcal{X}\rightarrow \mathcal{Z}_c$ such that the outcome of $g_c(x)$, can be any $z_c\in\{z_c\mid x= \psi_x(z_c, z_e, u_x)\}$ and $\mathbb{P}(Y\mid g_c(x))=\mathbb{P}(Y\mid z_c)$, will consistently provide an accurate prediction of $Y$. In essence, Y is identifiable over the pushforward measure $g_c\#\mathbb{P}(X)$.  

% if $h(g_c(x))=\mathbb{P}(Y\mid z_c)$ for $x=\psi_x(z_c,z_e,u_x)$ then $h(g_c(x)) = \mathbb{P}(Y\mid z_c)$ holds true for all $\{(x,z_c)\mid  x= \psi_x(z_c, z_e, u_x) \text{ for all }z_e,u_x\}$




% There is a set $\mathcal{B}=\{x\mid x=\psi_x\{z^{'}_c, z_e, u_x\} \text{ for some } z^{'}_c \text{ such that } \mathbb{P}(Y\mid_{z_e}=z_e)\neq \mathbb{P}(Y\mid z_c=z^{'}_c)\} \neq \emptyset$. Consequently, $h(\phi(g(x))) \neq h_c(g_c(x))$ for all $x\in \mathcal{B}$


\end{proof}

\begin{corollary} \label{cor:proterties}(Invariant Representation Function Properties) For any \(g \in \mathcal{G}_c\), the following properties hold:
\begin{enumerate}
    \item(Causal representation:) \(g\) is a mapping function directly from the sample space \(\mathcal{X}\) to the causal feature space \(\mathcal{Z}_c\), such that \(g: \mathcal{X} \rightarrow \mathcal{Z}_c\).
    \item (Equivalent causal representation) Given a deterministic equivalent causal transformation mapping \(T: \mathcal{Z}_c \rightarrow \mathcal{Z}_c\), which maps a causal factor \(z_c\) to another equivalent causal factor \(T(z_c)\), such that
 $\mathbb{P}(Y\mid z_c)=\mathbb{P}(Y\mid T(z_c))$, then we have \(g(x) = T(z_c)\) holds for all \(\{x \mid x = \psi_x(z_c, z_e, u_x), \text{ for all } z_e, u_x\}\).

\item Given $\ell$ is the Cross-Entropy Loss i.e., $\ell(h(z_c), y) = -\sum_{y \in \mathcal{Y}} \mathbb{P}(Y = y \mid z_c) \log h(z_c)[y]$, there exists $h^*$ such that:
\begin{equation*}
  h^* \in\bigcap_{z_c\in\mathcal{Z}_c} \underset{h\in \mathcal{H}}{\text{argmin }} \mathbb{E}_{y\sim\mathbb{P}(Y\mid z_c)} \ell\left ( h( z_c), y \right ),  
\end{equation*}
\end{enumerate}
\end{corollary}
\color{black}
\begin{proof}
We prove each property as follows:

% We will further demonstrate that \( g \in \mathcal{G}_c \) must be a function mapping from \( \mathcal{X} \) to \( \mathcal{Z}_c \) in order to $\mathbb{P}(Y\mid g(x)) = \mathbb{P}(Y\mid z_c)$ holds true for all $\{(x,z_c)\mid  x= \psi_x(z_c, z_e, u_x) \text{ for all }z_e,u_x\}$

\underline{\textit{Proof of property-1:}} Suppose there exists $g: \mathcal{X}\rightarrow \mathcal{Z}$ such that $\mathbb{P}(Y\mid g(x)) = \mathbb{P}(Y\mid z_c)$ holds true for all $\{(x,z_c)\mid  x= \psi_x(z_c, z_e, u_x) \text{ for all }z_e,u_x\}$.

If \( g \) is not a function from \( \mathcal{X} \) to \( \mathcal{Z}_c \), then \( g(x) \) may include spurious features \( z_e \), or both \( z_c \) and \( z_e \) for $x=\psi(z_c,z_e,u_x)$.

Based on the structural causal model (SCM) depicted in Figure~\ref{fig:graph}, it follows that \( Z_e \not\!\perp\!\!\!\perp Y \), meaning that the environmental feature \( Z_e \) is spuriously correlated with \( Y \). Consequently, 
\[
\mathbb{P}(Y \mid g(x = \psi(z_c, z_e, u_x))) \neq \mathbb{P}(Y \mid g(x = \psi(z_c, z_e', u_x)))
\]
for some \( z_e \neq z_e' \), which is a contradiction.

\underline{\textit{Proof of property-2:}} Since $g: \mathcal{X}\rightarrow \mathcal{Z}_c$ and $\mathbb{P}(Y\mid g(x)) = \mathbb{P}(Y\mid z_c)$ holds true for all $\{(x,z_c)\mid  x= \psi_x(z_c, z_e, u_x) \text{ for all }z_e,u_x\}$, the outcome of $g(x)$ have to be any $z^{'}_c\in \mathcal{Z}_c$ such that $\mathbb{P}(Y\mid z_c)=\mathbb{P}(Y\mid z^{'})$, which means \(g(x) = T(z_c)\) holds for  \(\{x \mid x = \psi_x(z_c, z_e, u_x)\}\)

This highlights the flexibility of the family of invariant representation functions \(\mathcal{G}_c\), as they allow the model to map a sample \(x = \psi(z_c, z_e, u_x)\) to a set of equivalent causal factors \(\{z'_c \in \mathcal{Z}_c \mid \mathbb{P}(Y \mid z_c) = \mathbb{P}(Y \mid z'_c)\}\), rather than requiring an exact mapping to \(z_c\).

Finally, since $g(x)=g(x')$ holds true for all $\{(x,x',z_c)\mid  x= \psi_x(z_c, z_e, u_x), x'= \psi_x(z_c, z^{'}_e, u^{'}_x) \text{ for all }z_e,z^{'}_e, u_x, u^{'}_x\}$, \(g(x) = T(z_c)\) holds for all \(\{x \mid x = \psi_x(z_c, z_e, u_x), \text{ for all } z_e, u_x\}\)

\underline{\textit{Proof of property-3:}}

Given $z_c\in \mathcal{Z}_c$ and $\ell(h(z_c), y) = -\sum_{y \in \mathcal{Y}} \mathbb{P}(Y = y \mid z_c) \log h(z_c)[y]$, it is easy to show that the optimal $$h^*=\underset{h\in \mathcal{H}}{\text{argmin }} \mathbb{E}_{y\sim\mathbb{P}(Y\mid z_c)} \ell\left ( h( z_c), y \right )$$ is the conditional probability distribution $h^*(z_c)=\mathbb{P}(Y\mid z_c)$.

Based on structural causal model (SCM) depicted in Figure~\ref{fig:graph}, $\mathbb{P}(Y\mid z_c)$  remains stable across all domains. Therefore, there exists an optimal function \(h^*\) such that:
\begin{equation*}
  h^* \in\bigcap_{z_c\in\mathcal{Z}_c} \underset{h\in \mathcal{H}}{\text{argmin }} \mathbb{E}_{y\sim\mathbb{P}(Y\mid z_c)} \ell\left ( h( z_c), y \right ),  
\end{equation*}

where $h^*(z_c)=\mathbb{P}(Y\mid z_c)$ for all $z_c\in \mathcal{Z}_c$
\end{proof}


\subsection{Sufficient Conditions for achieving Generalization}

\begin{theorem} (\textbf{Theorem \ref{thm:sufficient_conditions} in the main paper}) Under Assumption \ref{as:label_idf} and Assumption \ref{as:sufficient_causal_support}, given a hypothesis $f=h\circ g$, if $f$ is optimal hypothesis for training domains i.e.,
\begin{equation*}
    f\in \bigcap_{{e}\in \mathcal{E}_{tr}}\underset{f\in \mathcal{F}}{\text{argmin}} \ \loss{f,\mathbb{P}^{e}}
\end{equation*}
and one of the following sub-conditions holds:
\begin{enumerate}
    \item $g$ belongs to the set of \textbf{invariant representation functions} as specified in Proposition~\ref{thm:invariant_correlation}.
    
    \item $\mathcal{E}_{tr}$ is set of \textbf{Sufficient and diverse training domains} i.e., the union of the support of joint causal and spurious factors across training domains covers the entire causal and spurious factor space $\mathcal{Z}_c\times\mathcal{Z}_e$ i.e., $\cup_{e\in \mathcal{E}_{tr}}\text{supp}\{\mathbb{P}^{e} \left (Z_c, Z_e \right )\}=\mathcal{Z}_c\times\mathcal{Z}_e$.
    
    \item Given $\mathcal{T}$ is set of all \textbf{invariance-preserving transformations} such that for any $T\in \mathcal{T}$ and $g_c\in \mathcal{G}_c$: $(g_c\circ T)(\cdot)=g_c(\cdot)$, $f$ is also an optimal hypothesis on all augmented domains i.e., $$f\in \bigcap_{{e}\in \mathcal{E}_{tr}, T\in \mathcal{T}}\underset{f\in \mathcal{F}}{\text{argmin}} \ \loss{f,T\#\mathbb{P}^{e}}$$
\end{enumerate}
Then $f\in \mathcal{F}^*$
\label{thm:sufficient_conditions_apd}.
\end{theorem}

\begin{proof}
For clarity, we divide this theorem into three sub-theorems and present their proofs in the following subsections.
\end{proof}

\subsubsection{Proof of the Theorem~\ref{thm:sufficient_conditions_apd}.1}


Theorem~\ref{thm:sufficient_conditions_apd}.1 demonstrates that:

If 
\begin{itemize}
    \item \( f = h \circ g \) is an optimal hypothesis for all training domains, i.e., $f \in \bigcap_{{e} \in \mathcal{E}_{tr}} \underset{f \in \mathcal{F}}{\text{argmin}} \ \mathcal{L}(f, \mathbb{P}^{e}),$
    \item and \( g \) is an invariant representation function, i.e., \( g \in \mathcal{G}_c \),
\end{itemize}
then \( f \in \mathcal{F}^* \).


To prove this, in the following theorem, we show that for any domain \( \mathbb{P}^e \) satisfying causal support (Assumption~\ref{as:sufficient_causal_support}), if \( f = h \circ g \) is optimal for \( \mathbb{P}^e \) and \( g \in \mathcal{G}_c \), then \( f \in \mathcal{F}^* \). Note that in the following theorem, for simplicity, we assume that
$\mathbb{P}^e$ is a mixture of the training domains. Therefore, $\mathcal{E}_{tr}$ satisfying causal support implies \( \mathbb{P}^e \) also satisfying causal support i.e., $\text{supp}\{\mathbb{P}^e(Z_c)\}=\mathcal{Z}_c$.


\begin{theorem} Denote the set of \textbf{domain optimal hypotheses} of $\mathbb{P}^e$ induced by $g\in \mathcal{G}$: 
    \begin{equation*}
        \mathcal{F}_{\mathbb{P}^e,g}=\left \{h\circ g \mid h\in\underset{h'\in \mathcal{H}}{\rm{argmin }} \mathcal{L}\left ( h'\circ g, {\mathbb{P}^{e}} \right )  \right \}.
    \end{equation*} 
If $\text{supp}\{\mathbb{P}^e(Z_c)\}=\mathcal{Z}_c$ and $g\in \mathcal{G}_c$, then $\mathcal{F}_{\mathbb{P}^e,g}  \subseteq \mathcal{F}^{*}$. 
\label{thm:single_generalization_apd}
\end{theorem}


\begin{proof}


% Before diving into the proof, let us recall that, based on structural causal model (SCM) depicted in Figure~\ref{fig:graph} we have a distribution (domain) over the observed variables $(X,Y)$ given the environment $E=e \in \mathcal{E}$: \begin{align*}  \mathbb{P}^e(Y,X)&=\int_{z_c}\int_{z_e}\mathbb{P}^{e}(X, Y, Z_c,Z_e, E=e)d_{z_c} d_{z_e}\\
%    &= \int_{\mathcal{Z}_c}\int_{\mathcal{Z}_e}\mathbb{P}^{e}(X,Y, Z_c, Z_e) d_{z_c} d_{z_e}\\
%    &= \int_{\mathcal{Z}_c}\int_{\mathcal{Z}_e}\mathbb{P}^{e}(X\mid z_c, Z_e)\mathbb{P}^{e}(Y\mid z_c)\mathbb{P}^{e}(z_c,z_e) d_{z_c} d_{z_e}\\
% \end{align*}
% and
% \begin{align*}
%    \mathbb{P}^e( X=x)&= \int_{\mathcal{Z}_c}\int_{\mathcal{Z}_e}\mathbb{P}^{e}(X=x\mid z_c, Z_e)\mathbb{P}^{e}(z_c,z_e) d_{z_c} d_{z_e}\\
% \end{align*}
% and the conditional distribution:
% \begin{align*}
%    \mathbb{P}^e(Y\mid X=x)&=\int_{z_c}\int_{z_e}\mathbb{P}^{e}(X=x, Y, Z_c,Z_e, E=e)d_{z_c} d_{z_e}\\
%    &= \int_{\mathcal{Z}_c}\int_{\mathcal{Z}_e}\mathbb{P}^{e}(X=x\mid z_c, Z_e)\mathbb{P}^{e}(Y\mid z_c)\mathbb{P}^{e}(z_c,z_e) d_{z_c} d_{z_e}\\
% \end{align*}
% and $\mathbb{P}^{e}(Y\mid z_c)=\mathbb{P}^{e'}(Y\mid z_c)=\mathbb{P}(Y\mid z_c)$.


% \textbf{We start the proof the main theorem:}

Given $\text{supp}\{\mathbb{P}^e(Z_c)\}=\mathcal{Z}_c$ and $g_c\in \mathcal{G}_c$, 
it suffices to prove that for any $f_c = h_c \circ g_c \in \mathcal{F}_{\mathbb{P}^e,g_c}$, we have:

\begin{equation}
    f_c \in \bigcap_{\mathbb{P}^{e}\in \mathcal{P}} \underset{f\in \mathcal{F}}{\text{argmin}}\mathcal{L}\left(f, \mathbb{P}^e\right).    
    \label{eq:single_generalization_apd}
\end{equation}

To prove (\ref{eq:single_generalization_apd}), we only need to show that for any $f=h\circ g_c \in\mathcal{F}$ and $\mathbb{P}^{e'} \in \mathcal{P}$:

\begin{equation}
\mathcal{L}\left(f, \mathbb{P}^{e'}\right)
\geq \mathcal{L}\left(f_c, \mathbb{P}^{e'}\right),
%\label{eq:target_apd}
\end{equation}

which is equivalent to:
\begin{equation}
\mathbb{E}_{(x,y)\sim\mathbb{P}^{e'}}\left[\ell\left(f\left(x\right),y\right)\right]
\geq \mathbb{E}_{(x,y)\sim\mathbb{P}^{e'}}\left[\ell\left(f_c\left(x\right),y\right)\right].
\label{eq:target_apd}
\end{equation}

% \begin{equation}
% \mathbb{E}_{x\sim\mathbb{P}^{e'}(X),y\sim \mathbb{P}^{e'}(Y\mid X=x)}\left[\ell\left(f\left(x\right),y\right)\right]
% \geq \mathbb{E}_{x\sim\mathbb{P}^{e'}(X),y\sim \mathbb{P}^{e'}(Y\mid X=x)}\left[\ell\left(f_c\left(x\right),y\right)\right].
% \label{eq:target_apd}
% \end{equation}

\underline{\textit{Step 1: Simplifying the general loss using the invariant representation function \(g_c\).}} 

Based on structural causal model (SCM) depicted in Figure~\ref{fig:graph} we have a distribution (domain) over the observed variables $(X,Y)$ given the environment $E=e \in \mathcal{E}$: \begin{align*}  \mathbb{P}^e(X,Y)&=\int_{\mathcal{Z}_c}\int_{\mathcal{Z}_e}\mathbb{P}^{e}(X, Y, Z_c=z_c,Z_e=z_e)d_{z_c} d_{z_e}\\
&=\int_{\mathcal{Z}_c}\int_{\mathcal{Z}_e}\mathbb{P}^{e}(X, Y, z_c,z_e)d_{z_c} d_{z_e}\\
   &= \int_{\mathcal{Z}_c}\int_{\mathcal{Z}_e}\mathbb{P}^{e}(X\mid z_c, z_e)\mathbb{P}^{e}(Y\mid z_c)\mathbb{P}^{e}(z_c,z_e) d_{z_c} d_{z_e}\\
   &= \int_{\mathcal{Z}_c}\int_{\mathcal{Z}_e}\mathbb{P}^{e}(z_c,z_e)\int_{\mathcal{X}}\mathbb{P}^{e}(X=x\mid z_c, z_e)\mathbb{P}^{e}(Y\mid z_c)d_{z_c} d_{z_e} d_x\\
   &= \int_{\mathcal{Z}_c}\int_{\mathcal{Z}_e}\mathbb{P}^{e}(z_c,z_e)\int_{\mathcal{X}}\mathbb{P}^{e}(X=x\mid z_c, z_e)\int_{\mathcal{Y}}\mathbb{P}^{e}(Y=y\mid z_c) d_{z_c} d_{z_e} d_x d_y\\
    &= \int_{\mathcal{Z}_c}\int_{\mathcal{Z}_e}\mathbb{P}^{e}(z_c,z_e)\int_{\mathcal{X}}\int_{\mathcal{U}_x}\mathbb{P}^{e}(X=x\mid z_c, z_e,u_x)\mathbb{P}^{e}(u_x)\int_{\mathcal{Y}}\mathbb{P}^{e}(Y=y\mid z_c)  d_{z_c} d_{z_e} d_x d_y d_{u_x}\\
    &\stackrel{(1)}{=} \int_{\mathcal{Z}_c}\int_{\mathcal{Z}_e}\mathbb{P}^{e}(z_c,z_e)\int_{\mathcal{X}}\int_{\mathcal{U}_x}\mathbb{I}_{x= \psi_x(z_c, z_e,u_x)}\mathbb{P}^{e}(u_x)\int_{\mathcal{Y}}\mathbb{P}^{e}(Y=y\mid z_c)  d_{z_c} d_{z_e} d_x d_y d_{u_x}\\
\end{align*}

We have $\stackrel{(1)}{=}$ by definition of SCM, $x$ is the deterministic function of $(z_c, z_e,u_x)$.

Therefore we have:

\begin{align}
&\mathbb{E}_{(x,y)\sim\mathbb{P}^{e}(X,Y)}\left[\ell\left(f\left(x\right),y\right)\right]\nonumber\\
% &= 
% \int_{\mathcal{Z}_c}\int_{\mathcal{Z}_e}\int_{\mathcal{X}}\mathbb{P}^{e}(z_c,z_e)\mathbb{P}^{e}(X=x\mid z_c, z_e)\int_{\mathcal{Y}}\mathbb{P}^{e}(Y=y\mid z_c)\ell\left(f\left(x\right),y\right)d_y d_{z_c} d_{z_e} d_x  \nonumber\\
&= 
\int_{\mathcal{Z}_c}\int_{\mathcal{Z}_e}\mathbb{P}^{e}(z_c,z_e)\int_{\mathcal{X}}\int_{\mathcal{U}_x}\mathbb{I}_{x= \psi_x(z_c, z_e,u_x)}\mathbb{P}^{e}(u_x)\int_{\mathcal{Y}}\mathbb{P}^{e}(Y=y\mid z_c)\ell\left(f\left(x\right),y\right)  d_{z_c} d_{z_e} d_x d_y d_{u_x}\nonumber\\
&= 
\int_{\mathcal{Z}_c}\int_{\mathcal{Z}_e}\mathbb{P}^{e}(z_c,z_e)\int_{\mathcal{U}_x}\int_{\mathcal{Y}}\mathbb{P}^{e}(Y=y\mid z_c)\int_{\mathcal{X}}\mathbb{I}_{x= \psi_x(z_c, z_e,u_x)}\ell\left(f\left(x\right),y\right) \mathbb{P}^{e}(u_x) d_{z_c} d_{z_e} d_x d_y d_{u_x}\nonumber\\
&= 
\int_{\mathcal{Z}_c}\int_{\mathcal{Z}_e}\mathbb{P}^{e}(z_c,z_e)\int_{\mathcal{U}_x}\int_{\mathcal{Y}}\mathbb{P}^{e}(Y=y\mid z_c)\int_{\mathcal{X}}\mathbb{I}_{x= \psi_x(z_c, z_e,u_x)}\ell\left(f\left(\psi_x(z_c, z_e,u_x)\right),y\right) \mathbb{P}^{e}(u_x) d_{z_c} d_{z_e} d_x d_y d_{u_x}\nonumber
\\
&= 
\int_{\mathcal{Z}_c}\int_{\mathcal{Z}_e}\mathbb{P}^{e}(z_c,z_e)\int_{\mathcal{U}_x}\int_{\mathcal{Y}}\mathbb{P}^{e}(Y=y\mid z_c)\ell\left(f\left(\psi_x(z_c, z_e,u_x)\right),y\right) \mathbb{P}^{e}(u_x) d_{z_c} d_{z_e} d_y d_{u_x}\nonumber
\\
&= 
\int_{\mathcal{Z}_c}\int_{\mathcal{Z}_e}\mathbb{P}^{e}(z_c,z_e)\int_{\mathcal{U}_x}\mathbb{E}_{y\sim\mathbb{P}(Y\mid z_c)} \left[ \ell\left(f\left(\psi_x(z_c, z_e,u_x)\right),y\right)\right]
 \mathbb{P}^{e}(u_x) d_{z_c} d_{z_e}  d_{u_x}\nonumber
\\
&= 
\int_{\mathcal{Z}_c}\int_{\mathcal{Z}_e}\mathbb{P}^{e}(z_c,z_e)\int_{\mathcal{U}_x}\mathbb{E}_{y\sim\mathbb{P}(Y\mid z_c)} \left[ \ell\left((h\circ g_c)\left(\psi_x(z_c, z_e,u_x)\right),y\right)\right]
 \mathbb{P}^{e}(u_x) d_{z_c} d_{z_e}  d_{u_x}\nonumber
\\
&\stackrel{(1)}{=} 
\int_{\mathcal{Z}_c}\int_{\mathcal{Z}_e}\mathbb{P}^{e}(z_c,z_e)\int_{\mathcal{U}_x}\mathbb{E}_{y\sim\mathbb{P}(Y\mid z_c)} \left[ \ell\left(h\left(T(z_c)\right),y\right)\right]
 \mathbb{P}^{e}(u_x) d_{z_c} d_{z_e}  d_{u_x}\nonumber\\
&=
\int_{\mathcal{Z}_c}\mathbb{P}^{e}(z_c)\mathbb{E}_{y\sim\mathbb{P}(Y\mid z_c)} \left[ \ell\left(h\left(T(z_c)\right),y\right)\right]
 d_{z_c} \nonumber\\
&= 
\int_{\mathcal{Z}_c}\mathbb{P}^{e}(z_c)\mathbb{E}_{y\sim\mathbb{P}(Y\mid T(z_c))} \left[ \ell\left(h\left(T(z_c)\right),y\right)\right]
 d_{z_c} \nonumber
\\
&\stackrel{(2)}{=} 
\int_{\mathcal{Z}_c}T_{\#}\mathbb{P}^{e}(z_c)\mathbb{E}_{y\sim\mathbb{P}(Y\mid z_c)} \left[ \ell\left(h\left(z_c\right),y\right)\right]
 d_{z_c} \nonumber
\end{align}

We have:
\begin{itemize}
    \item $\stackrel{(1)}{=}$ by property-2 of $g_c$ (Corollary~\ref{cor:proterties});
    \item $\stackrel{(2)}{=}$ because $T: \mathcal{Z}_c\rightarrow \mathcal{Z}_c$ and  $T_{\#}\mathbb{P}^{e}(z_c)=\int_{z^{'}_c\in T^{-1}(z_c)}\mathbb{P}^{e}(z^{'}_c)d_{z^{'}_c}$ 
\end{itemize} 

Now, to prove (\ref{eq:target_apd}), we only need to show:

% \begin{align}
% \int_{\mathcal{Z}_c}\mathbb{P}^{e'}(z_c)\mathbb{E}_{y\sim\mathbb{P}(Y\mid z_c)} \left[ \ell\left(h_c\left(T(z_c)\right),y\right)\right]
%  d_{z_c} \leq \int_{\mathcal{Z}_c}\mathbb{P}^{e'}(z_c)\mathbb{E}_{y\sim\mathbb{P}(Y\mid z_c)} \left[ \ell\left(h\left(T(z_c)\right),y\right)\right]
%  d_{z_c} \nonumber
% \end{align}

\begin{align}
\int_{\mathcal{Z}_c}T_{\#}\mathbb{P}^{e'}(z_c)\mathbb{E}_{y\sim\mathbb{P}(Y\mid z_c)} \left[ \ell\left(h_c\left(z_c\right),y\right)\right]
 d_{z_c}\leq\int_{\mathcal{Z}_c}T_{\#}\mathbb{P}^{e'}(z_c)\mathbb{E}_{y\sim\mathbb{P}(Y\mid z_c)} \left[ \ell\left(h\left(z_c\right),y\right)\right]
 d_{z_c}
 \label{eq:target_causal}
\end{align}

% Recall that by proposition.\ref{thm:invariant_correlation_apd}, there exists $h^*$ such that: $h^*(g_c(x)) = \mathbb{P}(Y\mid z_c)$ holds true for all $\{(x,z_c)\mid  x= \psi_x(z_c, z_e, u_x) \text{ for all }z_e,u_x\}$. Therefore,

% \begin{equation*}
%   h^* \in\bigcap_{(x,z_c)\in\mathbb{B}} \underset{h\in \mathcal{H}}{\text{argmin }} \mathbb{E}_{y\sim\mathbb{P}(Y\mid z_c)} \ell\left ( h\circ g_c(x), y \right ),  
% \end{equation*}

% where $\mathbb{B}=\{(x,z_c)\mid  x= \psi_x(z_c, z_e, u_x) \text{ for all } z_c \in \mathcal{Z}_c \text{ and for all }z_e,u_x\}$.

\underline{\textit{Step 2: Generalization of $h_c$.}} \textit{Step-1} Demonstrate that \(h_c\) only needs to make predictions for the set of causal factors \(z_c \in \mathcal{Z}_c\). Therefore, it is sufficient to show that \(h_c\) is optimal for every \(z \in \mathcal{Z}_c\).


Recall that $f_c=h_c\circ g_c\in \mathcal{F}_{\mathbb{P}^e,g_c}$, 
therefore, 
$$h_c\in \underset{h\in \mathcal{H}}{\text{argmin }} \int_{\mathcal{Z}_c}T_{\#}\mathbb{P}^{e}(z_c)\mathbb{E}_{y\sim\mathbb{P}(Y\mid z_c)} \left[ \ell\left(h\left(z_c\right),y\right)\right]
 d_{z_c} $$

By property-3 of $g_c$ (Corollary~\ref{cor:proterties}), there exists an optimal function \(h^*\) such that:
\begin{equation*}
  h^* \in\bigcap_{z_c\in\mathcal{Z}_c} \underset{h\in \mathcal{H}}{\text{argmin }} \mathbb{E}_{y\sim\mathbb{P}(Y\mid z_c)} \ell\left ( h( z_c), y \right ),  
\end{equation*}

% for all $\{(x,z_c)\mid  x= \psi_x(z_c, z_e, u_x) \text{ for some }z_e, u_x \text{ and all } z_c\in \text{supp}\mathbb{P}^e(Z_c)\}$.

Property-3 of \(g_c\) ensures the existence of an optimal \(h^*\) for every causal factor \(z_c \in \mathcal{Z}_c\), it follows that \(h_c\) must also be optimal for every causal feature \(z_c\) within its support, \(\text{supp}\,\mathbb{P}^e(Z_e)\). This implies that \(h_c(z_c) = h^*(z_c)\) for every \(z_c\) where \(\mathbb{P}^e(z_e) > 0\).

Moreover, since \(\text{supp}\,\mathbb{P}^e(Z_e) = \mathcal{Z}_c\), this implies that \(h_c(z_c) = h^*(z_c)\) for every \(z_c \in \mathcal{Z}_c\).

\underline{\textit{Step-3: Proof of (\ref{eq:target_causal})}}.

\begin{align*}
\int_{\mathcal{Z}_c}T_{\#}\mathbb{P}^{e'}(z_c)\mathbb{E}_{y\sim\mathbb{P}(Y\mid z_c)} \left[ \ell\left(h_c\left(z_c\right),y\right)\right]
 d_{z_c}\leq\int_{\mathcal{Z}_c}T_{\#}\mathbb{P}^{e'}(z_c)\mathbb{E}_{y\sim\mathbb{P}(Y\mid z_c)} \left[ \ell\left(h\left(z_c\right),y\right)\right]
 d_{z_c}
\end{align*}

From \textit{step-2}, we have 

$$\mathbb{E}_{y\sim\mathbb{P}(Y\mid z_c)} \left[ \ell\left(h_c\left(z_c\right),y\right)\right]
\leq\mathbb{E}_{y\sim\mathbb{P}(Y\mid z_c)} \left[ \ell\left(h\left(z_c\right),y\right)\right]
$$
for all $z_c\in \mathcal{Z}_c$. By taking the expectation and applying the law of iterated expectation, inequality (\ref{eq:target_causal}) follows. This concludes the proof.


\end{proof}

\subsubsection{Proof of the Theorem~\ref{thm:sufficient_conditions_apd}.2}

Similar to the proof of Theorem~\ref{thm:sufficient_conditions_apd}.1, in the following result, we assume for simplicity that \( \mathbb{P}^e \) is a mixture of the training domains.
Then, Theorem~\ref{thm:sufficient_conditions_apd}.2 is stated as follows:

\begin{theorem}Under Assumption \ref{as:label_idf} and Assumption \ref{as:sufficient_causal_support}, 
if \( f\) is an optimal hypothesis for $\mathbb{P}^{e}$ i.e., $ f \in \underset{f \in \mathcal{F}}{\text{argmin}} \ \mathcal{L}(f, \mathbb{P}^{e}),$ and the support of joint causal and spurious factors of $\mathbb{P}^{e}$ covers the entire causal and spurious factor space $\mathcal{Z}_c\times\mathcal{Z}_e$ i.e., $\text{supp}\{\mathbb{P}^{e} \left (Z_c, Z_e \right )\}=\mathcal{Z}_c\times\mathcal{Z}_e$, then then \( f \in \mathcal{F}^* \).
\end{theorem}

\begin{proof}

Based on structural causal model (SCM) depicted in Figure~\ref{fig:graph} we have a distribution (domain) over the observed variables $(X,Y)$ given the environment $E=e \in \mathcal{E}$: 

Therefore we have:

\begin{align}
&\mathbb{E}_{(x,y)\sim\mathbb{P}^{e}(X,Y)}\left[\ell\left(f\left(x\right),y\right)\right]\nonumber\\
&= 
\int_{\mathcal{Z}_c}\int_{\mathcal{Z}_e}\mathbb{P}^{e}(z_c,z_e)\int_{\mathcal{X}}\int_{\mathcal{U}_x}\mathbb{I}_{x= \psi_x(z_c, z_e,u_x)}\mathbb{P}^{e}(u_x)\int_{\mathcal{Y}}\mathbb{P}^{e}(Y=y\mid z_c)\ell\left(f\left(x\right),y\right)  d_{z_c} d_{z_e} d_x d_y d_{u_x}\nonumber\\
&= 
\int_{\mathcal{Z}_c}\int_{\mathcal{Z}_e}\mathbb{P}^{e}(z_c,z_e)\int_{\mathcal{U}_x}\int_{\mathcal{Y}}\mathbb{P}^{e}(Y=y\mid z_c)\int_{\mathcal{X}}\mathbb{I}_{x= \psi_x(z_c, z_e,u_x)}\ell\left(f\left(x\right),y\right) \mathbb{P}^{e}(u_x) d_{z_c} d_{z_e} d_x d_y d_{u_x}\nonumber\\
&= 
\int_{\mathcal{Z}_c}\int_{\mathcal{Z}_e}\mathbb{P}^{e}(z_c,z_e)\int_{\mathcal{U}_x}\int_{\mathcal{Y}}\mathbb{P}^{e}(Y=y\mid z_c)\int_{\mathcal{X}}\mathbb{I}_{x= \psi_x(z_c, z_e,u_x)}\ell\left(f\left(\psi_x(z_c, z_e,u_x)\right),y\right) \mathbb{P}^{e}(u_x) d_{z_c} d_{z_e} d_x d_y d_{u_x}\nonumber
\\
&= 
\int_{\mathcal{Z}_c}\int_{\mathcal{Z}_e}\mathbb{P}^{e}(z_c,z_e)\int_{\mathcal{U}_x}\int_{\mathcal{Y}}\mathbb{P}^{e}(Y=y\mid z_c)\ell\left(f\left(\psi_x(z_c, z_e,u_x)\right),y\right) \mathbb{P}^{e}(u_x) d_{z_c} d_{z_e} d_y d_{u_x}\nonumber
\\
&= 
\int_{\mathcal{Z}_c}\int_{\mathcal{Z}_e}\mathbb{P}^{e}(z_c,z_e)\int_{\mathcal{U}_x}\mathbb{E}_{y\sim\mathbb{P}(Y\mid z_c)} \left[ \ell\left(f\left(\psi_x(z_c, z_e,u_x)\right),y\right)\right]
 \mathbb{P}^{e}(u_x) d_{z_c} d_{z_e}  d_{u_x}\nonumber
\\
&= 
\int_{\mathcal{Z}_c}\int_{\mathcal{Z}_e}\mathbb{P}^{e}(z_c,z_e)\int_{\mathcal{U}_x}\mathbb{E}_{y\sim\mathbb{P}(Y\mid z_c)} \left[ \ell\left((h\circ g_c)\left(\psi_x(z_c, z_e,u_x)\right),y\right)\right]
 \mathbb{P}^{e}(u_x) d_{z_c} d_{z_e}  d_{u_x}
\end{align}
\end{proof}

Under Assumption \ref{as:label_idf}, 
given \( f\) is an optimal hypothesis for $\mathbb{P}^{e}$ i.e., $ f \in \underset{f \in \mathcal{F}}{\text{argmin}} \ \mathcal{L}(f, \mathbb{P}^{e}),$ then 
\begin{align*}
f\in\int_{\mathcal{U}_x}\mathbb{E}_{y\sim\mathbb{P}(Y\mid z_c)} \left[ \ell\left((h\circ g_c)\left(\psi_x(z_c, z_e,u_x)\right),y\right)\right]
 \mathbb{P}^{e}(u_x)  d_{u_x}
\end{align*}

This holds because, under Assumption~\ref{as:label_idf}, it is guaranteed that there exists an optimal prediction for every $x=\psi_x(z_c, z_e, u_x)$ for all $\{(z_c,z_e)\sim \mathbb{P}^{e}(z_c,z_e),u_x\sim \mathbb{P}^{e}(u_x)\}$.

Furthermore, since the support of the joint causal and spurious factors in \( \mathbb{P}^{e} \) spans the entire causal and spurious factor space, i.e., $\text{supp}\{\mathbb{P}^{e} (Z_c, Z_e)\} = \mathcal{Z}_c \times \mathcal{Z}_e,$ the hypothesis \( f \) remains optimal for all possible configurations of \( x = \psi_x(z_c, z_e, u_x) \) across all \( z_c, z_e, u_x \). This implies that \( f \in \mathcal{F}^* \).

\textbf{Note:} 
\begin{itemize}
    \item It is important to highlight that this theorem aligns with Theorem 3 from \citep{ahuja2021invariance}. However, their analysis is conducted in a linear setting.
    \item We argue that the sub-condition of "sufficient and diverse training domains" is impractical, making it a weak guarantee for ensuring the generalization of algorithms based on this condition.
\end{itemize}



\subsubsection{Proof of the Theorem~\ref{thm:sufficient_conditions_apd}.3}

Similar to the proof of Theorem~\ref{thm:sufficient_conditions_apd}.1, in the following result, we assume for simplicity that \( \mathbb{P}^e \) is a mixture of the training domains.
Then, Theorem~\ref{thm:sufficient_conditions_apd}.3 is stated as follows:

\begin{theorem}Under Assumption \ref{as:label_idf} and Assumption \ref{as:sufficient_causal_support}, given $\mathcal{T}$ is set of all \textbf{invariance-preserving transformations} such that for any $T\in \mathcal{T}$ and $g_c\in \mathcal{G}_c$: $(g_c\circ T)(\cdot)=g_c(\cdot)$,
if \( f\) is an optimal hypothesis for $\mathbb{P}^{e}$ i.e., $ f \in \underset{f \in \mathcal{F}}{\text{argmin}} \ \mathcal{L}(f, \mathbb{P}^{e}),$ and  $f$ is also an optimal hypothesis on all augmented domains i.e., $$f\in \bigcap_{T\in \mathcal{T}}\underset{f\in \mathcal{F}}{\text{argmin}} \ \loss{f,T\#\mathbb{P}^{e}}$$
 
, then then \( f \in \mathcal{F}^* \).
\end{theorem}

\begin{proof}
We first analyze the characteristics of the set of all invariance-preserving transformations \( \mathcal{T} \).

By the definition of \( \mathcal{T} \) and the set of invariant representations \( \mathcal{G}_c \):

\begin{itemize}
    \item given $T\in \mathcal{T}$ and $g_c\in \mathcal{G}_c$, we have $(g_c\circ T)(x)=g_c(x)$ for all $x=\psi(z_c,z_e,u_x)$ (for all $z_c\in \mathcal{Z}_c, z_c\in \mathcal{Z}_c$, $u_x\in \mathcal{U}_x$).
    \item  $g\in \mathcal{G}_c$, $\mathbb{P}(Y\mid g(x)) = \mathbb{P}(Y\mid z_c)$ and $g(x)=g(x')$ holds true for all $\{(x,x',z_c)\mid  x= \psi_x(z_c, z_e, u_x), x'= \psi_x(z_c, z^{'}_e, u^{'}_x) \text{ for all }z_e,z^{'}_e, u_x, u^{'}_x\}$
\end{itemize}

This implies that for any $ x=\psi_x(z_c, z_e, u_x)$, we have:
\begin{equation*}
    T(x) \in \left \{\psi_x(z'_c, z'_e, u'_x) \text{ where } \left (z'_c\in\mathcal{Z}_c,z'_e\in \mathcal{Z}_e, u'_x\in \mathcal{U}_x \text{ and } P(Y\mid Z_c=z_c)=P(Y\mid Z_c=z'_c)\right ) \right \}
\end{equation*}

In other words, given a sample $\psi_x(z_c, z_e, u_x)$, the transformation $T$ operates as follows:
\begin{itemize}
    \item It augments \( z_c \) to its equivalent \( z'_c \), ensuring that $P(Y\mid Z_c=z_c)=P(Y\mid Z_c=z'_c)$
    \item It modifies the environmental (or spurious) feature $z_e$ to any $z'_e \in \mathcal{Z}_e$.
    \item It applies changes to the noise term $u_x$.
\end{itemize}

Therefore, under Assumption~\ref{as:sufficient_causal_support} (causal support), having access to all \( T \in \mathcal{T} \) is equivalent to having sufficient and diverse training domains. That is, the support of the joint causal and spurious factors in \( \mathbb{P}^{e} \) spans the entire causal and spurious factor space i.e., $\text{supp}\{\mathbb{P}^{e} (Z_c, Z_e)\} = \mathcal{Z}_c \times \mathcal{Z}_e$. This concludes the proof.




\textbf{Note:} In general, accessing all transformations from $\mathcal{T}$ is impractical. However, recently, some works leverage foundation models to generate these augmentations, achieving promising empirical performance \cite{ruan2021optimal}.
\end{proof}

\subsection{Necessary Conditions for achieving Generalization}

% \begin{theorem} \textbf{(Theorem \ref{thm:nacessary} in the main paper)}
% Considering the training domains $\mathbb{P}^e$ and representation function $g$, let $\mathcal{H}_{\mathbb{P}^e,g}=\underset{h\in \mathcal{H}}{\text{argmin }} \mathcal{L}\left ( h\circ g,\mathbb{P}^e \right ) $ represent the set of optimal classifiers on $g\#\mathbb{P}^e$ (the push-forward distribution by applying $g$ on $\mathbb{P}^e$), \textbf{the best generalization classifier} from $\mathbb{P}^e$ to $\mathcal{P}$ is defined as 
% \begin{equation}
% \mathcal{F}^{B}_{\mathbb{P}^e,g}=\left \{ h\circ g \mid h = \underset{ h'\in \mathcal{H}_{\mathbb{P}^e,g}}{\rm{argmin }} \sup_{e'\in \mathcal{E}} \mathcal{L}\left (  h'\circ g, \mathbb{P}^{e'} \right ) \right \}
% \end{equation}  

% Give representation function $g: \mathcal{X}\rightarrow \mathcal{Z}$ then $\forall \mathbb{P}^e\sim \mathcal{P}$ we have
% $\mathcal{F}^B_{ \mathbb{P}^e,g} \subseteq \mathcal{F}^{*}$ if and only if $g\in \mathcal{G}_s$. 
% \label{thm:nacessary_apd}
% \end{theorem}

\begin{theorem} \textbf{(Theorem \ref{thm:nacessary} in the main paper)} Given representation function $g$,
$\exists h: h\circ g\in \mathcal{F}^*$ if and only if $g\in \mathcal{G}_s$.
\label{thm:nacessary_apd}
\end{theorem}

\begin{proof} \textit{\textbf{``if"} direction.} If $g\in \mathcal{G}_s$, we have: 

\begin{enumerate}
    \item By definition of $g\in \mathcal{G}_s$ we have
$I(g(X),g_c(X))=I(X,g_c(X)$ i.e., $g(X)$ retain all information about $g_c(X)$ presented in $X$. Therefore, there exists a function \(\phi\) such that \(\phi \circ g \in \mathcal{G}_c\), which implies the existence of a \(g_c\in \mathcal{G}_c\) such that \(\phi \circ g = g_c\).

    \item By the definition of \(g_c \in \mathcal{G}_c\), we can always find a classifier \(h_c\) such that \(h_c \circ g_c \in \mathcal{F}^*\).

\end{enumerate}

To complete the proof of the \textbf{``if"} direction, we need to demonstrate the existence of a classifier \( h \) on top of the representation induced by \( g \) such that it forms a globally optimal hypothesis, i.e., \( h \circ g \in \mathcal{F}^* \).

From (1) and (2) we have $h_c\circ g_c = h_c\circ \phi \circ g \in \mathcal{F}^*$. Therefore, we can construct classifier $h = h_c \circ \phi$, then $h \circ g =  h_c\circ \phi \circ g =h_c\circ g_c \in \mathcal{F}^*$.
\end{proof}

\begin{proof} \textit{\textbf{``only if"} direction by contraction.} 

Define the set of optimal hypotheses induced by a representation function \( g \) as:

\begin{equation*}
\mathcal{F}_{g,\mathcal{E}_{tr}}=\left\{ h\circ g: \bigcap_{{e} \in \mathcal{E}_{tr}} \underset{h\circ g \in \mathcal{F}}{\text{argmin}} \ \mathcal{L}(h\circ g, \mathbb{P}^{e})\right \}
\end{equation*}  

We show that if $g$ is not sufficient-representation, for any $f\in\mathcal{F}_{g,\mathcal{E}_{tr}}$ there exists multiple target domains where $f$ performs arbitrarily bad.  


By definition of $g\notin \mathcal{G}_s$, we have
$I(g(X),g_c(X))<I(X,g_c(X)$. Therefore, there does not exist a function $\phi$ such that $\phi\circ g \in \mathcal{G}_c$. This implies we can not construct any classifier $h = h_c \circ \phi$, then $h \circ g =  h_c\circ \phi \circ g =h_c\circ g_c \in \mathcal{F}^*$.

Consequently, $h$ has to rely on spurious feature $z_e$ (or both $z_c$ and $z_e$) to make predict for some $\{x\mid x=\psi_x\{z_c,z_e, u_x\} \text{ for some } z_c \text{ such that } \mathbb{P}(Y\mid Z_e=z_e)= \mathbb{P}(Y\mid Z_c=z_c) \}$.  

Note that based on structural causal model (SCM) depicted in Figure~\ref{fig:graph}, we have $Z_e\not\!\perp\!\!\!\perp Y$ i.e., the environmental feature $Z_e$ spuriously correlated with $Y$. Therefore, there is a set $\mathcal{B}=\{x\mid x=\psi_x\{z^{'}_c, z_e, u_x\} \text{ for some } z^{'}_c \text{ such that } \mathbb{P}(Y\mid Z_e = z_e)\neq \mathbb{P}(Y\mid Z_c=z^{'}_c)\} \neq \emptyset$. Consequently, $h(\phi(g(x))) \neq h_c(g_c(x))$ for all $x\in \mathcal{B}$

    We can construct undesirable target domains $\mathbb{P}^{e_i}$ with arbitrary loss $\mathcal{L}(h\circ g, \mathbb{P}^{e_i})$ by giving $(1-\delta)$ percentage mass to that examples in $\mathcal{B}$ and $(\delta)$ percentage mass that examples in $\mathcal{X} \setminus \mathcal{B}$. This is equivalent to 
\begin{equation}
    \mathbb{E}_{(x,y)\sim\mathbb{P}^{e_i}} \left [ h(g(x)) \neq h_c(g_c(x))  \right ]  \geq 1-\delta.\nonumber
\end{equation}
 with $(0\leq\delta\leq 1)$.

This concludes the proof.

\end{proof}

% \begin{corollary} \textbf{(Corollary \ref{thm:bad_domain_exist} in the main paper)}
%      Given $g\in \mathcal{G}_s$, there exists $f = h\circ g\in \bigcap_{e\in \mathcal{E}_{train}}\mathcal{F}_{g, \mathbb{P}^e}$  such that for any $0<\delta<1$, there are many undesirable target domains $\mathbb{P}^T \sim \mathcal{P}$ such that:
%     \begin{align*}
%    \mathbb{E}_{(x,y)\sim\mathbb{P}^T} \left [ f(x) \neq f^*(x)  \right ]  \geq 1-\delta.
%     \end{align*}  with  $f^* \in \mathcal{F}^*$.
%     \label{thm:bad_domain_exist_apd}
% \end{corollary}

% \begin{proof}
% Denote $\mathbb{P}^{\mathcal{E}_{tr}}$ is the mixture of training domains, then $\text{supp}\{\mathbb{P}^{\mathcal{E}_{tr}} \left (Z_c \right )\}=\cup_{e\in \mathcal{E}_{tr}}\text{supp}\{\mathbb{P}^{e} \left (Z_c \right )\}=\mathcal{Z}_c$.  Additionally, given $g\in \mathcal{G}_s$, then there exists $\phi$ such that $g_c=\phi\circ g\in \mathcal{G}_c$. 

% Based on structural causal model (SCM) depicted in Figure~\ref{fig:graph}, we have $Z_e\not\!\perp\!\!\!\perp Y$ i.e., the environmental feature $Z_e$ spuriously correlated with $Y$. Hence, there exist $h \not\in \{h_c \circ \phi \mid h_c \circ g_c \in \mathcal{F}_{g_c,\mathbb{P}^{\mathcal{E}_{tr}}}\}$  e.g., $h$ can rely on spurious feature $z_e$ (or both $z_c$ and $z_e$) to make predict for some $\{x\mid x=\psi_x\{z_c,z_e, u_x\} \text{ for some } z_c \text{ such that } \mathbb{P}(Y\mid_{z_e}=z_e)= \mathbb{P}(Y\mid z_c=z_c) \}$. 
    
% There is a set $\mathcal{B}=\{x\mid x=\psi_x\{z^{'}_c, z_e, u_x\} \text{ for some } z^{'}_c \text{ such that } \mathbb{P}(Y\mid_{z_e}=z_e)\neq \mathbb{P}(Y\mid z_c=z^{'}_c)\} \neq \emptyset$. Consequently, $h(\phi(g(x))) \neq h_c(g_c(x))$ for all $x\in \mathcal{B}$

%     We can construct undesirable target domains $\mathbb{P}^{e_i}$ with arbitrary loss $\mathcal{L}(h\circ g, \mathbb{P}^{e_i})$ by giving $(1-\delta)$ percentage mass to that examples in $\mathcal{B}$ and $(\delta)$ percentage mass that examples in $\mathcal{X} \setminus \mathcal{B}$. This is equivalent to 
% \begin{equation}
%     \mathbb{E}_{(x,y)\sim\mathbb{P}^{e_i}} \left [ h(g(x)) \neq h_c(g_c(x))  \right ]  \geq 1-\delta.\nonumber
% \end{equation}
%  with $(0\leq\delta\leq 1)$.

% By Theorem \ref{thm:single_generalization}, $h_c\circ g_c\in \mathcal{F}_{g_c,\mathbb{P}^{\mathcal{E}_{tr}}}$ implies $h_c\circ g_c\in \mathcal{F}^*$. This concludes the proof.

% \end{proof}


% \begin{theorem}
%  \label{thm:convergence_apd} \textbf{(Theorem \ref{thm:convergence} in the main paper)}
% Given sequence of training domains $\mathcal{E}_{tr}=\{e_1,...,e_K\} \subset \mathcal{E}$, denote $\Funion^{k}=\bigcap_{i=1}^{k}\mathcal{F}_{ \mathbb{P}^{e_i}}$. We consider $\mathcal{E}_{tr}$ to be \textbf{diverse} if for domain $e_k$, there exists at least one sample $x=\psi_x(z_c,z_e,u_x)$ such that $\exists f  \in \Funion^{k-1} :f(x)\neq \mathbb{P}(Y\mid z_c)$. Given a set of diverse domains $\mathcal{E}_{tr}$, we have:
% \begin{equation*}
%  \Funion^1\supset  \Funion^2 \supset... \supset  \Funion^K   
% \end{equation*}
% and the number of training domains $\mathcal{E}_{tr}$ is sufficiently large:
% \begin{equation*}
% \lim_{\mathcal{E}_{tr}\rightarrow \mathcal{E}}\Funion^{ \left| \mathcal{E}_{tr} \right|} \rightarrow \mathcal{F}^*.
% \end{equation*}
% \end{theorem}


% \begin{proof}

% We prove the first statement by induction. Consider the case $\mathcal{F}^{\cap}_{k-1}$ and $\mathcal{F}^{\cap}_{k}$, we will show that if $\mathcal{E}_{tr}$ is considered as \textbf{diverse} $\mathcal{F}^{\cap}_{k-1}\supset \mathcal{F}^{\cap}_{k}$. 

% We have $\mathcal{F}^{\cap}_{k-1}\supseteq \mathcal{F}^{\cap}_{k}$ is obvious by definition.
% By definition of "diverse" training domains $\mathcal{E}_{tr}$, there exists at least one sample $x=\psi_x(z_c,z_e,u_x)$ such that $\exists f  \in \Funion^{k-1} :f(x)\neq \mathbb{P}(Y\mid z_c)$. This means $f\notin \mathcal{F}^{\cap}_{k}$, hence, $\mathcal{F}^{\cap}_{k-1}\supset \mathcal{F}^{\cap}_{k}$.


% For the second statement, we need to show that if  $\mathcal{E}_{tr}=\mathcal{E}$ then $\Funion^{ \left| \mathcal{E}_{tr} \right|}= \mathcal{F}^*$. This holds true by the definition of $\mathcal{F}^*$.



% \end{proof}


\begin{corollary} \textbf{(Corollary~\ref{thm:information} in the main paper)} Under Assumption \ref{as:label_idf} and Assumption \ref{as:sufficient_causal_support}, let the minimal representation function $g_{\text{min}}$ be defined as:
\begin{equation}
g_{\text{min}} \in \mathcal{G}_{min}=\left\{\underset{g \in \mathcal{G}}{\text{argmin }} I(g(X); X) \ \text{s.t.} \ f = h \circ g \in \mathcal{F}_{\mathcal{E}_{\text{tr}}} \right\},
\label{eq:minimal}
\end{equation}
where $I$ denotes mutual information. Then, for any $g_c\in \mathcal{G}_c$ the following holds:
\begin{equation}
I(g_{\text{min}}(X), g_c(X)) \leq I(X, g_c(X)),
\end{equation}
and the equality holds if and only if at least one of sufficient conditions is hold.
\label{thm:information_apd}.
\end{corollary}

\begin{proof} We first prove that if one of the sufficient conditions holds, then the following equality holds:
\[
I(g_{\text{min}}(X), g_c(X)) = I(X, g_c(X)).
\]

Define:
\begin{equation*}
\mathcal{G}_{\mathcal{E}_{tr}}=\left \{g: f = h \circ g \in \mathcal{F}_{\mathcal{E}_{\text{tr}}} \right\}.
\end{equation*}
By Theorem~\ref{thm:sufficient_conditions_apd}, if one of the sufficient conditions holds, then \( \mathcal{G}_{\mathcal{E}_{tr}} \subseteq \mathcal{G}_c \).

From the definition in Eq.~(\ref{eq:minimal}), we have:
\[
\mathcal{G}_{\text{min}} \subseteq \mathcal{G}_{\mathcal{E}_{tr}} \subseteq \mathcal{G}_c.
\]
This implies:
\[
I(g_{\text{min}}(X), g_c(X)) = I(g_c(X), g_c(X)) = I(X, g_c(X)).
\]
\end{proof}

\begin{proof} We prove that if the equality holds, then \( g_{\text{min}} \in \mathcal{G}_c \).

If
\[
I(g_{\text{min}}(X), g_c(X)) = I(X, g_c(X)),
\]
then it follows that
\[
I(g_{\text{min}}(X), X) \geq I(X, g_c(X)).
\]
Therefore, by the definition of \( g_{\text{min}} \), we conclude that \( g_{\text{min}} \in \mathcal{G}_c \).

\end{proof}

\subsection{Representation Alignment trade-off}
\label{apd:tradeoff}
As a reminder, $\mathbb{P}$ denotes data distribution on data space $\mathcal{X}$, while $g_{\#}\mathbb{P}$ denotes latent distribution on full latent space $\mathcal{Z}$, with $g: \mathcal{X} \mapsto \mathcal{Z}$ is the encoder. 

In the following, we recap the theoretical results for Hellinger distance as presented by \cite{phung2021learning}. Similar results for $\mathcal{H}$-divergence can be found in Zhao et al. \cite{zhao2019learning}, and for Wasserstein distance in Le et al. \cite{le2021lamda}.

\subsubsection{Upper Bound}

\begin{theorem} 
\label{thm:single_bound_A}Consider the source domain
$\mathbb{P}^{e'}$ and the
target domain $\mathbb{P}^{e}$. Let $\ell$ be any loss function
upper-bounded by a positive constant $L$. For any hypothesis $f:\mathcal{X}\mapsto\mathcal{Y}_{\Delta}$
where $f=h\circ g$ with $g:\mathcal{X}\mapsto\mathcal{Z}$
and $h:\mathcal{Z}\mapsto\mathcal{Y}_{\Delta}$, the target
loss on input space is upper bounded 
\begin{equation}
\begin{aligned}\mathcal{L}\left(f,\mathbb{P}^{e}\right)\leq\mathcal{L}\left(f,\mathbb{P}^{e'}\right)+L\sqrt{2}\,d_{1/2}\left(\mathbb{P}_{g}^{e},\mathbb{P}_{g}^{e'}\right)\end{aligned}
,\label{eq:input_bound_1-1}
\end{equation}
\end{theorem}

This Theorem is directly adapted from the result of Trung et al. \cite{phung2021learning}.
The upper bound for target loss above relates source loss, target loss and data shift on feature space, which is different to other bounds in which the data shift is on input space.

\subsubsection{Lower Bound}
\begin{theorem}
\label{theorem:single_lower_bound_A}
\cite{phung2021learning} Consider a hypothesis $f=h\circ g$, the Hellinger distance between two label marginal distributions $\mathbb{P}^{e'}$ and $\mathbb{P}^{e}$ can be upper-bounded as: 
\begin{equation}
d_{1/2}\left(\mathbb{P}^{e'}_\mathcal{Y},\mathbb{P}^{e}_\mathcal{Y}\right) \leq 
\mathcal{L}\left ( f,\mathbb{P}^{e'} \right )^{1/2}+
d_{1/2}\left ( g_{\#}\mathbb{P}^{e'},g_{\#}\mathbb{P}^{e} \right )+
\mathcal{L}\left ( f,\mathbb{P}^{e} \right )^{1/2}
\end{equation}

where the general loss $\mathcal{L}$ is defined based on the Hellinger loss $\ell$ which is define as $\ell\left ( f(x) \right )=D_{1/2}\left ( f(x),\mathbb{P}(Y\mid x) \right )=2\sum_{i=1}^C\left ( \sqrt{f(x,i)}-\sqrt{\mathbb{P}(Y=i\mid x)} \right )^2$.
\end{theorem}

\subsection{Subspace Representation Alignment}

In the following, we prove the theoretical results for Hellinger distance based on the findings of Trung et al. \cite{phung2021learning}. A similar strategy can be directly applied to $\mathcal{H}$-divergence \cite{zhao2019learning} and Wasserstein distance \cite{le2021lamda}.

\begin{theorem}
\label{theorem:multi_bound_A} \textbf{(Theorem~\ref{theorem:multi_bound})} 
\textit{(Subspace Representation Alignment)} Given \textit{subspace projector} $\Gamma: \mathcal{X}\rightarrow \mathcal{M}$, and a subspace index $m\in \mathcal{M}$, let $A_{m}=\Gamma^{-1}(m)=\left\{ x:\Gamma(x)=m\right\} $ is the region on data space which has the same index $m$ and $\mathbb{P}_{m}^{e}$ be the distribution restricted by $\mathbb{P}^{e}$ over the set $A_{m}$, then $\pi^{e}_m=\frac{\mathbb{P}^{e}\left(A_{m}\right)}{\sum_{m'\in\mathcal{M}}\mathbb{P}^{e}\left(A_{m'}\right)}$ is mixture co-efficient, if the loss function $\ell$ is upper-bounded by a positive
constant $L$, then:

(i)  The target general loss is upper-bounded: 
\begin{align*}
\left | \mathcal{E}_{tr} \right |\sum_{e\in \mathcal{E}_{tr}}\mathcal{L}\left ( f,\mathbb{P}^{e} \right )
\leq
\sum_{e\in \mathcal{E}_{tr}} \sum_{m\in\mathcal{M}}\pi^{e}_m
\mathcal{L}\left ( f,\mathbb{P}^{e}_{m} \right ) +
L\sum_{e, e'\in \mathcal{E}_{tr}}\sum_{m\in\mathcal{M}}\pi^{e}_{m}D\left ( g_{\#}\mathbb{P}^{e}_{m},g_{\#}\mathbb{P}^{e'}_{m} \right ),
\end{align*}
(ii) Distance between two label marginal distribution $\mathbb{P}^{e}_{m}(Y)$ and $\mathbb{P}^{e'}_{m}(Y)$ can be upper-bounded: 
\begin{equation*}
\begin{aligned}
D\left(\mathbb{P}^{e}_{\mathcal{Y},m},\mathbb{P}^{e'}_{\mathcal{Y},m}\right) \leq 
D\left ( g_{\#}\mathbb{P}^{e}_{m},g_{\#}\mathbb{P}^{e'}_{m} \right )
+\mathcal{L}\left ( f,\mathbb{P}^{e}_{m}\right )
+
\mathcal{L}\left ( f,\mathbb{P}^{e'}_{m} \right )
\end{aligned}
\end{equation*}
(iii) Construct the subspace projector $\Gamma$ as the optimal hypothesis for the training domains i,e., \(\Gamma \in \mathcal{F}_{\mathcal{E}_{tr}}\), which defines $\mathcal{M}=\{m=\hat{y}\mid \hat{y}=\Gamma(x), x\in\bigcup_{e\in\mathcal{E}_{tr}}\text{supp}\mathbb{P}^{e} \}\subseteq \mathcal{Y}_\Delta$, then
$
D\left(\mathbb{P}^{e}_{\mathcal{Y}, m},\mathbb{P}^{e'}_{\mathcal{Y},m}\right)=0$ for all \(m \in \mathcal{M}\).

where $g_{\#}\mathbb{P}$ denotes representation distribution on $\mathcal{Z}$ induce by applying $g$ with $g: \mathcal{X} \mapsto \mathcal{Z}$ on data distribution $\mathbb{P}$, $D$ can be $\mathcal{H}$-divergence, Hellinger or Wasserstein distance.
\end{theorem}

\begin{proof}
We consider \textit{sub-space projector} $\Gamma: \mathcal{X}\rightarrow \mathcal{M}$, given a sub-space index $m\in \mathcal{M}$, we denote $A_{m}=\Gamma^{-1}(m)=\left\{ x:\Gamma(x)=m\right\} $ is the region on data space which has the same index $m$.
Let $\mathbb{P}_{m}^{e}$ be the distribution restricted by $\mathbb{P}^{e}$ over the set $A_{m}$ and $\mathbb{P}_{m}^{e}$ as the distribution restricted by $\mathbb{P}^{e}$
over $A_{m}$. Eventually, we define $\mathbb{P}_{m}^{e}\left(y\mid x\right)$ as the probabilistic labeling
distribution on the sub-space $\left(A_{m},\mathbb{P}_{m}^{e}\right)$,
meaning that if $x\sim\mathbb{P}_{m}^{e}$, $\mathbb{P}_{m}^{e}\left(y\mid x\right)=\mathbb{P}_{e}\left(y\mid x\right)$.
Similarly, we define if $x\sim\mathbb{P}_{m}^{e'}$, $\mathbb{P}_{m}^{e'}\left(y\mid x\right)=\mathbb{P}^{e'}\left(y\mid x\right)$. Due to this construction, any data sampled from $\mathbb{P}_{m}^{e}$
or $\mathbb{P}_{m}^{e'}$ have the same index $m=\Gamma(x)$. 
Additionally, since each data point $x \in \mathcal{X}$ corresponds to only a single $\Gamma(x)$, the data space is partitioned into disjoint sets, i.e., $\mathcal{X} = \bigcup_{m=1}^{\mathcal{M}} A_{m}$, where $A_m \cap A_n = \emptyset, \forall m \neq n$. 
Consequently, the general loss of the target domain becomes:
\begin{equation}
\mathcal{L}\left(f,\mathbb{P}^{e}\right):=\sum_{m\in\mathcal{M}}\pi^{e}_m\mathcal{L}\left(f,\mathbb{P}_{m}^{e}\right),\label{eq:subspace_loss}
\end{equation}
where $\mathcal{M}$ is the set of all feasible sub-spaces indexing $m$ and  $\pi^{e}_m=\frac{\mathbb{P}^{e}\left(A_{m}\right)}{\sum_{m'\in\mathcal{M}}\mathbb{P}^{e}\left(A_{m'}\right)}$.

\end{proof}

\begin{proof}(i):

Using the same proof
for a single space in Theorem \ref{thm:single_bound_A}, we obtain:
\begin{equation} \mathcal{L}\left(f,\mathbb{P}^{e}_m\right)
\leq
\mathcal{L}\left(f_m,\mathcal{\mathbb{P}}_{m}^{e'}\right)
+ 
L\sqrt{2} d_{1/2}\left(g_{\#}\mathbb{P}^{e}_{m},g_{\#}\mathbb{P}_{m}^{e'}\right)
\end{equation}

Since $\mathcal{L}\left(f,\mathbb{P}^{e}\right):= \sum_{m}\pi^{e}_m \mathcal{L}\left(f,\mathbb{D}_{m}^{e}\right)$, taking weighted average over $m\in \mathcal{M}$, we reach (ii):
\begin{equation}
\mathcal{L}\left(f,\mathbb{P}^{e}\right)
\leq
\sum_{m}\pi^{e}_m
\mathcal{L}\left(f_m,\mathbb{P}_{m}^{e'}\right)
+ 
L\sqrt{2}\sum_{m}\pi^{e}_m d_{1/2}\left(g_{\#}\mathbb{P}^{e}_{m},g_{\#}\mathbb{P}_{m}^{e'}\right)
\end{equation}

By summing over the training domains on the left-hand side, we obtain:

\begin{align} \sum_{e\in\mathcal{E}_{tr}}\mathcal{L}\left(f_{\mathcal{M}},\mathbb{P}^{e}\right)
\leq&
\sum_{e\in\mathcal{E}_{tr}}\sum_{m}\pi^{e}_m
\mathcal{L}\left(f_m,\mathbb{P}_{m}^{e'}\right)
+ 
\sum_{e\in\mathcal{E}_{tr}}L\sqrt{2}\sum_{m}\pi^{e}_m d_{1/2}\left(g_{\#}\mathbb{P}^{e}_{m},g_{\#}\mathbb{P}_{m}^{e'}\right) \nonumber
\end{align}
Summing over the training domains on the left-hand side again:

\begin{align}
\sum_{e'\in\mathcal{E}_{tr}}\sum_{e\in\mathcal{E}_{tr}}\mathcal{L}\left(f_{\mathcal{M}},\mathbb{P}^{e}\right)
\leq&
\sum_{e'\in\mathcal{E}_{tr}}\sum_{e\in\mathcal{E}_{tr}}\sum_{m}\pi^{e}_m
\mathcal{L}\left(f_m,\mathbb{P}_{m}^{e'}\right)\nonumber\\
&+ 
\sum_{e'\in\mathcal{E}_{tr}}\sum_{e\in\mathcal{E}_{tr}}L\sqrt{2}\sum_{m}\pi^{e}_m d_{1/2}\left(g_{\#}\mathbb{P}^{e}_{m},g_{\#}\mathbb{P}_{m}^{e'}\right)\nonumber
\end{align}

Finally, we obtain:
\begin{align}
\left | \mathcal{E}_{tr} \right | \sum_{e\in \mathcal{E}_{tr}}\mathcal{L}\left(f,\mathbb{P}^{e}\right)
\leq&
\sum_{e,e'\in \mathcal{E}_{tr}}\sum_{m\in\mathcal{M}}\pi^{e}_m
\mathcal{L}\left(f,\mathcal{\mathbb{P}}_{m}^{e'}\right)
+
\sum_{e,e'\in \mathcal{E}_{tr}}L\sqrt{2}\sum_{m\in\mathcal{M}}\pi^{e}_{m} d_{1/2}\left(g_{\#}\mathbb{P}^{e}_{m},g_{\#}\mathbb{P}_{m}^{e'}\right)
\end{align}
\end{proof}


\begin{proof}(ii):

We obtain (ii) directly by applying the results from Theorem \ref{theorem:single_lower_bound_A} to each individual sub-space, denoted by the index $m$.
\end{proof}

\begin{proof}(iii):

Within training domains, we anticipate that $f\in \cap_{e\in \mathcal{E}_{tr}}\mathcal{F}_{\mathbb{P}^{e}}$ will predict the ground truth label $f(x)=f^*(x)$ where $f^*\in \mathcal{F}^*$.We can define a projector \(\Gamma = f\), which induces a set of subspace indices $\mathcal{M}=\{m=\hat{y}\mid \hat{y}=f(x), x\in\bigcup_{e\in\mathcal{E}_{tr}}\text{supp}\mathbb{P}^{e} \}\subseteq \Delta_{\left | \mathcal{Y}\right |}$. As a result, given subspace index $m\in\mathcal{M}$, $\forall i \in \mathcal{Y}, \mathbb{P}^{e}_{\mathcal{Y},m}(Y=i) = \mathbb{P}^{e'}_{\mathcal{Y},m}(Y=i) = \sum_{x \in f^{-1}(m)}\mathbb{P}(Y=i\mid x) = m[i]$. Consequently, \(D\left(\mathbb{P}^{e}_{\mathcal{Y},m}, \mathbb{P}^{e'}_{\mathcal{Y},m}\right) = 0\) for all \(m \in \mathcal{M}\), allowing us to jointly optimize both \textit{domain losses} and \textit{representation alignment}.
\end{proof}

% \begin{theorem}
% \label{theorem:information_view-1} \textbf{(Theorem 1 in the main paper)} Let $X$ is a random variable of source
% sample (i.e., drawn from $\mathbb{P}^{S}$) and $Y$ is a random
% variable of ground-truth labels. Denote $N=\sum_{m'\in\mathcal{M}}\mathbb{P}^{S}\left(A_{m'}\right)$,
% we then have
% \begin{equation}
% \mathbb{I}\left(\Gamma\left(g\left(X\right)\right)\odot g\left(X\right),Y\right)\geq-\sum_{m\in\mathcal{M}}\frac{\mathbb{P}^{S}\left(A_{m}\right)}{N}\mathcal{L}\left(f,\mathbb{D}_{m}^{S}\right)+\text{const},\label{eq:mutual_information-1}
% \end{equation}
% where the loss $\mathcal{L}\left(f,\mathbb{D}_{m}^{S}\right)$
% is defined based on the cross-entropy loss and $\mathbb{I}$ denotes
% the mutual information.
% \end{theorem}
% \begin{proof}
% Denote $T=\Gamma\left(g\left(X\right)\right)\odot g\left(X\right)$,
% we have
% \begin{align*}
% \mathbb{I}\left(T,Y\right) = & \int p(t,y)\log\frac{p\left(t,y\right)}{p\left(t\right)p\left(y\right)}dtdy\\
% = & \int p(t,y)\log\frac{p\left(y\mid t\right)}{p\left(y\right)}dtdy\\
% = & \int p(t,y)\log p\left(y\mid t\right)dtdy+\mathbb{H}\left(Y\right)\\
% = & \int p(t,y)\log h\left(y\mid t\right)\frac{p\left(y\mid t\right)}{h\left(y\mid t\right)}dtdy+\mathbb{H}\left(Y\right)\\
% = & \int p(t,y)\log h\left(y\mid t\right)dtdy+D_{KL}\left(p\left(y\mid t\right)\Vert h\left(y\mid t\right)\right)+\mathbb{H}\left(Y\right)\\
% \geq & \int p(t,y)\log h\left(y\mid t\right)dtdy+\text{const},
% \end{align*}
% where $\mathbb{H}$ specifies the entropy, $D_{KL}$is Kullback-Leibler
% (KL) divergence, $h\left(y\mid t\right)=h\left(t,y\right)=h\left(\Gamma\left(g\left(x\right)\right)\odot g\left(x\right),y\right)$
% for any $h:\mathcal{Z}\rightarrow\mathcal{Y}_{\simplex}$, and $h\left(t,y\right)$
% returns the $y$-th element of $h\left(t\right)$.

% We further derive
% \begin{align*}
% \int p(t,y)\log h\left(y\mid t\right)dtdy = & \sum_{i=1}^{C}\mathbb{E}_{p(t)}\left[p\left(y=i\mid t\right)\log h\left( y=i \mid t\right)\right]\\
% \overset{(1)}{=} & \sum_{i=1}^{C} \mathbb{E}_{p^{S}(x)}\left[p\left(y=i\mid\Gamma\left(g\left(x\right)\right)\odot g\left(x\right)\right)\log h\left(\Gamma\left(g\left(x\right)\right)\odot g\left(x\right),i\right)\right]\\
% = & \sum_{i=1}^{C}\mathbb{E}_{\mathbb{P}^{S}}\left[p\left(y=i\mid\Gamma\left(g\left(x\right)\right)\odot g\left(x\right)\right)\log h\left(\Gamma\left(g\left(x\right)\right)\odot g\left(x\right),i\right)\right].
% \end{align*}

% Note that we have $\overset{(1)}{=}$ because $\Gamma\left(g\left(x\right)\right)\odot g\left(x\right)$
% pushes forward $X\sim p^{S}(x)$ to $T\sim p\left(t\right)$.
% Moreover, according to our definitions: $\mathbb{P}^{S}=\sum_{m\in\mathcal{M}}\frac{\mathbb{P}^{S}\left(A_{m}\right)}{N}\mathbb{P}_{m}^{S}$,
% we hence obtain
% \begin{align*}
% \int p(t,y)\log h\left(y\mid t\right)dtdy = & \sum_{m\in\mathcal{M}}\frac{\mathbb{P}^{S}\left(A_{m}\right)}{N}\sum_{i=1}^{C}\mathbb{E}_{\mathbb{P}_{m}^{S}}\left[p\left(y=i\mid m\odot g\left(x\right)\right)\log h\left(m\odot g\left(x\right),i\right)\right]\\
% = & -\sum_{m\in\mathcal{M}}\frac{\mathbb{P}^{S}\left(A_{m}\right)}{N}\sum_{i=1}^{C}\mathbb{E}_{\mathbb{P}_{m}^{S}}\left[-p_{m}^{S}\left(y=i\mid x\right)\log f\left(x,i\right)\right]\\
% = & -\sum_{m\in\mathcal{M}}\frac{\mathbb{P}^{S}\left(A_{m}\right)}{N}\mathcal{L}\left(f,\mathbb{D}_{m}^{S}\right).
% \end{align*}
% \end{proof}


% \begin{lemma} Given a deterministic sub-space indicator $\Gamma$, we have:
% \begin{align}
% d_{1/2}\left(g_{\#}\mathbb{P}^{T},g_{\#}\mathbb{P}_{S}^{S}\right) 
% \geq &
% \sum_{m} \frac{S^S_m+S^T_m}{2} d_{1/2}\left(g_{\#}\mathbb{P}^{T}_{m},g_{\#}\mathbb{P}_{m}^{S}\right)
% \end{align}
% \label{theorem:latent_data_shift_A}
% \end{lemma}


% \begin{proof}


% By definition, $g_{\#}\mathbb{P}^{T} (z) = g_{\#}\mathbb{P}^{T} (z) = \sum_{m}S^T_m  g_{\#}\mathbb{P}^{T}_{m}\left ( z_m \right ) $ and $ g_{\#}\mathbb{P}_{S}^{S} (z)= g_{\#}\mathbb{P}_{S}^{S} (z) = \sum_{m}S^S_m g_{\#}\mathbb{P}^{S}_{m}\left ( z_m \right )$, we have:
% \begin{align}
% &d_{1/2}\left(g_{\#}\mathbb{P}^{T},g_{\#}\mathbb{P}_{S}^{S}\right)\nonumber\\
% &=
% \left [ 2  \int_{z} \left ( \sqrt{g_{\#}\mathbb{P}^{e_i} (z)}-\sqrt{g_{\#}\mathbb{P}^{e_j}_S (z)} \right )^{2} dz  \right ]^{1/2}
% \\
% &=
% \left [ 2 \int_{z} \left (g_{\#}\mathbb{P}^{e_i} (z) + g_{\#}\mathbb{P}^{e_j}_S (z) - 2\sqrt{g_{\#}\mathbb{P}^{e_i} (z) g_{\#}\mathbb{P}^{e_j}_S (z)} \right ) dz  \right ]^{1/2}
% \\
% &\overset{(1)}{=}\left [ 2\sum_{m}\int_{z \in \Gamma^{-1}(m)} S^T_m  g_{\#}\mathbb{P}^{T}_{m}\left ( z \right ) + S^S_m g_{\#}\mathbb{P}^{S}_{m}\left ( z \right )   -  2\sqrt{S^T_m g_{\#}\mathbb{P}^{S}_{m}\left ( z \right )}
% \sqrt{S^S_m g_{\#}\mathbb{P}^{S}_{m}\left ( z \right )} dz \right ]^{1/2}
% \label{eq:global_d_subspace}
% \end{align}

% Here we note that $\overset{(1)}{=}$ is the results of $\Gamma$ partitioning source and target data to disjoint-sets:
% \begin{equation*}
% g_{\#}\mathbb{P}^{e_j}_m(z),g_{\#}\mathbb{P}^{e_i}_m(z):
% \left\{\begin{matrix}
%  >0 & \text{if} & z\in \Gamma^{-1}\left(m\right)\\ 
%  =0 & \text{if} & z\notin \Gamma^{-1}\left(m\right) \\
% \end{matrix}\right.,
% \end{equation*}
% therefore, given sub-space index $m$ and $z \in \Gamma^{-1}(m)$:
% \begin{align*}
% &g_{\#}\mathbb{P}^{e_j}(z)
% = \sum_{m'}S_{m'} g_{\#}\mathbb{P}^{e_j}_{m'}(z)
% = S_m g_{\#}\mathbb{P}^{e_j}_m(z)\\
% &g_{\#}\mathbb{P}^{e_i}(z)
% = \sum_{m'}S_{m'} g_{\#}\mathbb{P}^{e_i}_{m'}(z)
% = S_m g_{\#}\mathbb{P}^{e_i}_m(z)
% \end{align*}


% Consider $B$ as follows:

% \begin{align}
% B=&\sum_{m} \int_{z} \left ( 
% S^T_m g_{\#}\mathbb{P}^{e_i}_m (z)
% +
% S^S_m g_{\#}\mathbb{P}^{e_j}_m (z) - 2\sqrt{
% S^S_m S^T_m g_{\#}\mathbb{P}^{e_i}_m(z) g_{\#}\mathbb{P}^{e_j}_m (z)} \right ) \1_{z \in \Gamma^{-1}(m)} dz 
% \\
% =&
% \sum_{m} \int_{z} \left (S^T_m \mathbb{P}^{e_i}_m (z)
% +
% S^T_m g_{\#}\mathbb{P}^{e_j}_m (z)
% - 2\sqrt{
% S^T_m S^T_m g_{\#}\mathbb{P}^{e_i}_m(z) g_{\#}\mathbb{P}^{e_j}_m (z)}\right ) \1_{z \in \Gamma^{-1}(m)} dz \\
% &+
% \sum_{m} \int_{z} \left (
% (S^S_m-S^T_m) g_{\#}\mathbb{P}^{e_j}_m (z) - 2\left ( \sqrt{S^S_m} - \sqrt{S^T_m} \right )
% \sqrt{S^T_m g_{\#}\mathbb{P}^{e_i}_m(z) \mathbb{P}^{e_j}_m (z)}\right ) \1_{z \in \Gamma^{-1}(m)} dz 
% \\
% =&
% \sum_{m} \int_{z} \left( S^T_m \left(\sqrt{g_{\#}\mathbb{P}^{e_i}_m (z)} - \sqrt{g_{\#}\mathbb{P}^{e_j}_m (z)} \right) ^{2} \right ) \1_{z \in \Gamma^{-1}(m)} dz 
% \\
% &+
% \underset{=\sum_{m}(S^S_m-S^T_m) =0}{\underbrace{\sum_{m} \int_{z} \left (
% S^S_m-S^T_m\right ) g_{\#}\mathbb{P}^{e_j}_m (z) \1_{z \in \Gamma^{-1}(m)}  dz}} 
% \\
% &+
% \sum_{m} \int_{z} 2\left ( \sqrt{S^T_m} - \sqrt{S^S_m} \right )
% \sqrt{S^T_m g_{\#}\mathbb{P}^{e_i}_m(z) g_{\#}\mathbb{P}^{e_j}_m (z)} \1_{z \in \Gamma^{-1}(m)} dz 
% \\
% =&
% \sum_{m}\int_{z \in \Gamma^{-1}(m)} S^T_m \left (\sqrt{g_{\#}\mathbb{P}^{e_i}_m (z)} - \sqrt{g_{\#}\mathbb{P}^{e_j}_m (z)}\right )^{2} +2\left ( S^T_m - \sqrt{S^T_mS^S_m} \right )
% \sqrt{ g_{\#}\mathbb{P}^{e_i}_m(z) g_{\#}\mathbb{P}^{e_j}_m (z)} dz
% \end{align}

% Similarly, we also have: 
% \begin{equation}
% B =\sum_{m}\int_{z} S^S_m \left (\sqrt{g_{\#}\mathbb{P}^{e_i}_m (z)} - \sqrt{g_{\#}\mathbb{P}^{e_j}_m (z)}\right )^{2} +2\left ( S^S_m - \sqrt{S^T_mS^S_m} \right )
% \sqrt{g_{\#}\mathbb{P}^{e_i}_m(z) g_{\#}\mathbb{P}^{e_j}_m (z)}\1_{z \in \Gamma^{-1}(m)} dz
% \end{equation}

% Combining (40) and (41), we have:
% \begin{align}
% B &=\sum_{m}\int_{z} \frac{S^S_m+S^T_m}{2} \left (\sqrt{g_{\#}\mathbb{P}^{e_i}_m (z)} - \sqrt{g_{\#}\mathbb{P}^{e_j}_m (z)}\right )^{2} +2\left ( \sqrt{S^S_m} - \sqrt{S^T_m} \right )^2
% \sqrt{ g_{\#}\mathbb{P}^{e_i}_m(z) g_{\#}\mathbb{P}^{e_j}_m (z)}\1_{z \in \Gamma^{-1}(m)} dz\\
% &=\sum_{m}\frac{S^S_m+S^T_m}{4}\int_{z}  2\left (\sqrt{g_{\#}\mathbb{P}^{e_i}_m (z)} - \sqrt{g_{\#}\mathbb{P}^{e_j}_m (z)}\right )^{2} +2\left ( \sqrt{S^S_m} - \sqrt{S^T_m} \right )^2
% \sqrt{ g_{\#}\mathbb{P}^{e_i}_m(z) g_{\#}\mathbb{P}^{e_j}_m (z)}\1_{z \in \Gamma^{-1}(m)} dz\\
% &=\sum_{m} \left [ \frac{S^S_m+S^T_m}{4}\left [ d_{1/2}\left(g_{\#}\mathbb{P}^{T}_{m},g_{\#}\mathbb{P}_{m}^{S}\right) \right ]^2+\int_{z}2\left ( \sqrt{S^S_m} - \sqrt{S^T_m} \right )^2
% \sqrt{ g_{\#}\mathbb{P}^{e_i}_m(z) g_{\#}\mathbb{P}^{e_j}_m (z)}\right ] \1_{z \in \Gamma^{-1}(m)} dz
% \end{align}

% Substituting $B$ to~(\ref{eq:global_d_subspace}) we have:

% \begin{align}
% &d_{1/2}\left(\sum_{m} S^T_m g_{\#}\mathbb{P}^{T}_{m},\sum_{m} S^S_m g_{\#}\mathbb{P}_{m}^{S}\right)\nonumber\\
% &=
% \left [ 2\sum_{m} \left [ \frac{S^S_m+S^T_m}{4}\left [ d_{1/2}\left(g^\Gamma_{\#}\mathbb{P}^{T}_{m},g^\Gamma_{\#}\mathbb{P}_{m}^{S}\right) \right ]^2+\int_{z} 2\left ( \sqrt{S^S_m} - \sqrt{S^T_m} \right )^2
% \sqrt{ g^\Gamma_{\#}\mathbb{P}^{e_i}_m(z) g^\Gamma_{\#}\mathbb{P}^{e_j}_m (z)}\right ] \1_{z \in \Gamma^{-1}(m)} dz \right ]^{1/2}
% \\
% &\geq
% \left [ \sum_{m}  \frac{S^S_m+S^T_m}{2}\left [ d_{1/2}\left(g_{\#}\mathbb{P}^{T}_{m},g_{\#}\mathbb{P}_{m}^{S}\right) \right ]^2 \right ]^{1/2}
% \\
% &=
% \left [ \left ( \sum_{m} \frac{S^S_m+S^T_m}{2}\right)  \left ( \sum_{m} \frac{S^S_m+S^T_m}{2}   \left [ d_{1/2}\left(g_{\#}\mathbb{P}^{T}_{m},g_{\#}\mathbb{P}_{m}^{S}\right) \right ]^2 \right) \right]^{1/2}\\
% &\overset{(3)}{\geq}
% \sum_{m} \frac{S^S_m+S^T_m}{2} d_{1/2}\left(g_{\#}\mathbb{P}^{T}_{m},g_{\#}\mathbb{P}_{m}^{S}\right)\\
% \end{align}
% We obtain $(3)$ by applying Cauchy–Schwarz inequality. 
% \end{proof}


% \begin{lemma} \textbf{(Lemma 3 in the main paper)} Given a \textit{Source-Target Balanced sup-space projection $\Gamma$} i.e., $S_m^T=S^S_m =S_m \forall m$, we have:

% \begin{enumerate}
%   \item Reduce latent data-shift: 
%   \begin{equation}
%       d_{1/2}\left(g_{\#}\mathbb{P}^{T},g_{\#}\mathbb{P}_{S}^{S}\right)\nonumber \geq
% \sum_{m} S^m d_{1/2}\left(g_{\#}\mathbb{P}^{T}_{m},g_{\#}\mathbb{P}_{m}^{S}\right)
%   \end{equation}
  
%   \item Tighter target-risks' upper-bound:
%   \begin{equation}
%       \mathcal{L}\left(f,\mathbb{P}^{e_i}\right)
% \leq B_{S}\leq B_{M}
%   \end{equation}
% \end{enumerate}

% \label{theorem:latent_data_shift_A}
% %%\vspace{-2mm}
% \end{lemma}



% \subsection{Practical Method}

% \begin{corollary}
% \label{cor:appendix_cluster_latent}\textbf{(Corollary \ref{cor:cluster_latent} in the main paper)} Consider minimizing $\min_{g,C}\mathcal{W}_{d_{z}}\left(g\#\mathbb{P}_{x},\mathbb{P}_{c,\pi}\right)$
% given $\pi$ and assume
% $K<N$, its optimal solution $g^{*}$ and $C^{*}$are also the
% optimal solution of the following OP:
% \begin{equation}
% \min_{g,C}\min_{\sigma\in\Sigma_{\pi}}\sum_{n=1}^{N}d_{z}\left(g\left(x_{n}\right),c_{\sigma\left(n\right)}\right),\label{eq:appendix_clus_latent}
% \end{equation}
% where $\Sigma_{\pi}$ is the set of assignment functions $\sigma:\left\{ 1,...,N\right\} \rightarrow\left\{ 1,...,K\right\} $
% such that the cardinalities $\left|\sigma^{-1}\left(k\right)\right|,k=1,...,K$
% are proportional to $\pi_{k},k=1,...,K$.
% \end{corollary}

% \textbf{Proof of Corollary \ref{cor:appendix_cluster_latent}}.

% By the Monge definition, we have
% \begin{align*}
% \mathcal{W}_{d_{z}}\left(g\#\mathbb{P}_{x},\mathbb{P}_{c,\pi}\right) & =\mathcal{W}_{d_{z}}\left(\frac{1}{N}\sum_{n=1}^{N}\delta_{g\left(x_{n}\right)},\sum_{k=1}^{K}\pi_{k}\delta_{c_{k}}\right)=\min_{T:T\#\left(g\#\mathbb{P}_{x}\right)=\mathbb{P}_{c,\pi}}\mathbb{E}_{z\sim g\#\mathbb{P}_{x}}\left[d_{z}\left(z,T\left(z\right)\right)\right]\\
% = & \frac{1}{N}\min_{T:T\#\left(g\#\mathbb{P}_{x}\right)=\mathbb{P}_{c,\pi}}\sum_{n=1}^{N}d_{z}\left(g\left(x_{n}\right),T\left(g\left(x_{n}\right)\right)\right).
% \end{align*}

% Since $T\#\left(g\#\mathbb{P}_{x}\right)=\mathbb{P}_{c,\pi}$,
% $T\left(g\left(x_{n}\right)\right)=c_{k}$ for some $k$. Additionally,
% $\left|T^{-1}\left(c_{k}\right)\right|,k=1,...,K$ are proportional
% to $\pi_{k},k=1,...,K$. Denote $\sigma:\left\{ 1,...,N\right\} \rightarrow\left\{ 1,...,K\right\} $
% such that $T\left(g\left(x_{n}\right)\right)=c_{\sigma\left(n\right)},\forall i=1,...,N$,
% we have $\sigma\in\Sigma_{\pi}$. It also follows that 
% \[
% \mathcal{W}_{d_{z}}\left(\frac{1}{N}\sum_{n=1}^{N}\delta_{g\left(x_{n}\right)},\sum_{k=1}^{K}\pi_{k}\delta_{c_{k}}\right)=\frac{1}{N}\min_{\sigma\in\Sigma_{\pi}}\sum_{n=1}^{N}d_{z}\left(g\left(x_{n}\right),c_{\sigma\left(n\right)}\right).
% \]


\
% \section{Additional Discussion with Related Works} \label{apd:relation_to_IDG}

% \paragraph{Optimal Representation \citep{ruan2021optimal}} 

% While at first glance, \citep{ruan2021optimal} and our work share the same goal of identifying the necessary and sufficient conditions for generalization, the two studies fundamentally differ in the following aspects:  

% \citet{ruan2021optimal} aim to identify the set of conditions that are both necessary and sufficient, which provide theoretical guarantee essentially by assuming some knowledge of target domains. Without accessing target information, generalization is provably impossible. Meanwhile, we focus on analyzing generalizability from limited domains without assuming any additional information from the target.    

% More concretely, \citet{ruan2021optimal} propose the \textit{idealized} domain generalization hypothesis (IDG), which is the expected worst-case target risk over source risk minimizers: 

% $$
%     R_{IDG}=\mathbb{E}_{{e_i,e_j}\sim \mathcal{P}}\left [ \sup_{f\in\mathcal{F}_{\mathbb{P}^{e_i}}}\mathcal{L}(f,\mathbb{P}^{e_i}) \right ]
% $$

% $R_{IDG}$ is an expectation over all possible pairs of domains $(e_i, e_j)\sim \mathcal{P}$ where $\mathcal{P}$ is the distribution over domain space $\mathcal{E}$. During training, they sample any two domains from the domain distribution, assigning one as the source and the other as the target, to determine the worst-case target risk. 

% The representation $Z = g(X)$ deemed optimal for IDG must satisfy two conditions (by Theorem 1 therein):

% \begin{itemize}
% \item Sufficient representation: the representation needs to be task-discriminative, allowing a predictor to minimize risk across all domains. In the presence of all domains, this condition can be simply satisfied by learning a hypothesis optimal for all training domains. 

% \item The representation’s marginal support must be consistent across all pairs of source and target domains. This condition generally coincides with our assumption of causal support, which is a common assumption across DG literature. 
% \end{itemize}


% It is clear from the formulation $R_{IDG}$ that \textit{all} possible domains should be known to achieve generalization. \citet{ruan2021optimal} also point out the challenge in generalization without data from the target domain and recommends incorporating data augmentation from pre-trained models such as CLIP. To our best knowledge, using augmentation in DG is not new. Various studies have shown that access to all label-preserving augmentations (which is generally unfeasible) would reveal true causal factors \citep{mitrovic2020representation,gao2023out}. 
% To satisfy this condition, \citet{ruan2021optimal} assume augmentation is Bayes-preserving augmentation (Assumption 10 therein).


\section{Practical Methodology}
\label{Sec:practical}
In this section, we present the practical objectives to achieve Eq. (\ref{eq:final_objective}):

\begin{align}
\min_{f=h\circ g} \underset{\text{Subspace Representation Alignment}}{\underbrace{\sum_{e,e'\in \mathcal{E}_{tr}}\sum_{m\in \mathcal{M}}D\left( g\#\mathbb{P}_m^{e},g\#\mathbb{P}_m^{e'}\right)}}\text{ s.t. } \underset{\text{Training domain optimal hypothesis}}{\underbrace{f=h\circ g\in \bigcap_{e\in \mathcal{E}_{tr}}\underset{ f}{\text{argmin }} \mathcal{L}\left(f,\mathbb{P}^{e}\right)}}
   \label{eq:final_objective_apd}
\end{align}

where $\mathcal{M}=\{\hat{y}\mid \hat{y}=f(x), x\in\bigcup_{e\in\mathcal{E}_{tr}}\text{supp}\mathbb{P}^{e} \}$ and $D$ can be $\mathcal{H}$-divergence, Hellinger distance, Wasserstein distance.

In the following, we consider the encoder $g$, classifier $h$, domain discriminator $h_d$ and set of $K$ empirical training domains $\mathbb{D}^{e_i}=\{x_{j}^{e_i},y_{j}^{e_i}\}_{j=1}^{N_{e_i}}\sim [\mathbb{P}^{e_i} ]^{N_{e_i}}$, $i=1...K$.


\subsection{Optimal hypothesis across training domains}
For \textit{optimal hypothesis across training domains condition}, we simply adopting the objective set forth by ERM: 
\begin{align}
\label{eq:emp_IRM}
     \min_{f} \; \sum_{i=1}^K \mathcal{L}\left(f,\mathbb{D}^{e_i}\right)
\end{align} 


\subsection{Subspace Representation Alignment}
\paragraph{Subspace Modelling and Projection.} 
\label{sec:subspace_project_detail}
Our objective is to map samples $x$ from training domains with identical predictions $f(x) = m$ into a unified subspace, where $m\in \mathcal{M}=\{\hat{y}\mid \hat{y}=f(x), x\in\bigcup_{e\in\mathcal{E}_{tr}}\text{supp}\mathbb{P}^{e} \}$. Given that the cardinality of $\mathcal{M}$ can be exceedingly large, potentially equal to the total number of training samples if the output of $f(x)$ is unique for each sample, this makes the optimization process particularly challenging.

Drawing inspiration from the concept of prototypes \cite{snell2017prototypical}, we suggest representing $\mathcal{M}$ as a set of prototypes $\mathcal{M} = \{m_i\}_{i=1}^{M}$, where each $m_i$ is an element of $\mathcal{Z}$. Consequently, a sample $x$ is assigned to a subspace by selecting the nearest prototype $m_i$ i.e., $i=\underset{ i'}{\text{argmin }} \rho(g(x),m_{i'})$. Note that prototypes act as condensed representations of specific prediction outcomes. Consequently, samples assigned to the same prototype will receive the same prediction. Although this approach streamlines the subspace projection, it may lead to local optima as the mapping might favor a limited number of prototypes early in training \cite{vuong2023vector}. To mitigate this issue, we adopt a Wasserstein (WS) clustering approach \cite{vuong2023vector} to guide the mapping of latent features from each domain into the designated subspace more effectively.
We first endow a discrete distribution over the prototypes as $\mathbb{P}_{\mathcal{M},\pi}=\sum_{i=1}^{M}\pi_{i}\delta_{m_{i}}$
with the Dirac delta function $\delta$ and the weights $\pi\in\Delta_{M}= \{\pi'\geq \boldsymbol{0}: \Vert \pi'\Vert_1 =1\}$. 

Then we project each domain $\mathbb{P}^{e_i}$ in subspaces indexed by prototypes as follows:
\begin{equation}
\min_{\mathcal{M},\pi}\min_{g}\left \{\mathcal{L}_{P}=\sum_{i=1}^K\lambda\mathcal{W}_{\rho}\left(g\#\mathbb{P}^{e_i},\mathbb{P}_{\mathcal{M},\pi}\right)\right \},\label{eq:reconstruct_form_continuous}
\end{equation}
where:
\begin{itemize}
    \item Cost metric $\rho(z,m)=\frac{z^\top m}{\left \| z \right \|\left \| c \right \|}$ is the cosine similarity between the latent representation $z$ and the prototype $c$.
    \item  Wasserstein distance between source domain representation distribution and distribution over prototype $\mathbb{P}_{\mathcal{M},\pi}$:
\begin{align}
\mathcal{W}_{d}\left(g\#\mathbb{P}^{e_i}_{x},\mathbb{P}_{c,\pi}\right)
&=\mathcal{W}_{d}\left(\sum_{n=1}^{B}\frac{1}{B}g\left(x_{n}\right),\sum_{i=1}^{M}\pi_{i}\delta_{m_{i}}\right)\\
&=\frac{1}{B}\min_{\Gamma:\Gamma\#\left(g\#\mathbb{P}^{e_i}_{x}\right)=\mathbb{P}_{c,\pi}}\sum_{n=1}^{B}\rho\left(g\left(x_{n}\right),\Gamma\left(g\left(x_{n}\right)\right)\right)
\end{align}

Where $B$ is the batch size. This Wasserstein distance can be effectively compute by linear dynamic programming method, entropic dual form of optimal transport \citep{genevay2016stochastic} or Sinkhorn algorithm \cite{cuturi2013sinkhorn}.
\end{itemize}


\paragraph{Subspace Alignment Constraints}
Subspace alignment is achieved through a conditional adversarial training approach \cite{gan2016learning, li2018domain}. In this framework, the \textbf{subspace-conditional} domain discriminator $h_d$ aims to accurately predict the domain label ``$e_i$" based on the combined feature $[z,m]$, where $\{z=g(x), m=\Gamma(x)\}$. Concurrently, the objective for the representation function $g$ is to transform the input $x$ into a latent representation $z$ in such a way that $h_d$ is unable to determine the domain ``$e_i$" of $x$.  We employ the Gradient Reversal Layer (GRL) as introduced by\cite{ganin2016domain}, thereby simplifying the optimization process to:

\begin{equation}
    \min_{g, h_d} \left \{\mathcal{L}_{D}=-\sum_{i=1}^K\mathbb{E}_{x\sim\mathbb{D}^{e_i}}\left [ e_i\log h_d\left (\left [ \mathcal{R}\left ( g(x) \right ),m \right ]\right ) \right ] \right \}
\end{equation}

where $\mathcal{R}$ is gradient reversal layer.

% Such negative gradients
% contribute to making the learned features similar across domains. We propose a gradient-reversal layer (GRL) to update $g$\MH{check the latex here, it says theta f} by easily following \cite{ganin2016domain}. This gradient reversal layer does nothing and merely forwards the input to the following layer during forward propagation. However, it multiplies the gradient by −1 during the backpropagation to obtain a negative gradient from the domain classification.
% \subsection{Overall Framework}

\subsection*{Final objective}
Putting all together, we propose a joint optimization objective, which is given as  
\begin{equation}
\min_{\mathcal{M},\pi} \min_{g,h, h_d}  \left \{\mathcal{L}_{H}+\lambda_P\mathcal{L}_{P}+\lambda_D \mathcal{L}_{D}\right \},
\end{equation}
where $ \lambda_P$ is the subpsace projector hyper-parameter and $\lambda_D$ is the representation alignment hyper-parameter. 



We highlight that SRA is most similar to DANN and CDANN. Like these methods, SRA utilizes $\mathcal{H}$-divergence for alignment; however, the key distinction lies in the alignment strategy: 
\begin{itemize}
    \item DANN aligns the entire domain representation, 
    \item CDANN aligns class-conditional representations,
    \item while SRA employs subspace-conditional alignment.
\end{itemize}

It is also important to note that the representation alignment hyperparameter $\lambda_D$ is kept the same for DANN, CDANN, and SRA in our experiments. As discussed in Theorem~\ref{theorem:single_tradeoff}, DANN and CDANN potentially violate necessary conditions, whereas SRA does not (Theorem~\ref{theorem:multi_bound}), leading to improved performance.




% \subsection{Visualization}
% \label{sec:apd_visualization}

% To elucidate the interaction between latent representations and prototypes, we employ t-SNE~\citep{Maaten08visualizingdata} to visualize their distribution as obtained from our method dataset, depicted in Figure~\ref{fig:tsne}. 
% \begin{figure}[h!]
% \begin{centering}
% \subfloat{\centering{}\includegraphics[width=0.8\textwidth]{ICLR2025/Figures/tnse.png}}
% \par\end{centering}
% \caption{The t-SNE visualization of the learned representations and prototypes generated from our method.}
% \label{fig:tsne}
% \end{figure} 

% Observations from Figure~\ref{fig:tsne}.a reveal a homogeneous blend of representations across various domains, indicating an indistinct separation of samples from different domains, while maintaining clear class distinctions as shown in Figure~\ref{fig:tsne}.b. This highlights the efficacy of our subspace representation alignment approach.


% \begin{table}[h!]
% \caption{Classification Accuracy on PACS using ResNet50 with Prototype as classifier.}
% %\vspace{-0.5mm}
% \begin{centering}
% \resizebox{0.6\columnwidth}{!}{ %
% \begin{tabular}{lccccc}
% \toprule
% Algorithm  & \textbf{A} & \textbf{C} & \textbf{P} & \textbf{S} & \textbf{Avg}  \\
% \midrule
% Classifier & 90.2 $\pm$ 0.5 & 84.4 $\pm$ 0.3 & 97.9 $\pm$ 0.1 & 82.3 $\pm$ 0.2 & \textbf{88.7} \\
% Prototype & 90,9 $\pm$ 0.3 & 84.1 $\pm$ 0.3 & 97.5 $\pm$ 0.1 & 82.1 $\pm$ 0.2 & \textbf{88.7} \\
% \bottomrule
% \end{tabular}}
% \par\end{centering}
% \label{tab:PACS_r}
% \end{table}

% Additionally, consistent with our approach, prototypes are intended to capture class-discriminative information. Observations indicate that these prototypes are positioned within the clusters of their respective classes. To test their efficacy, we executed experiments in which prototypes are utilized as markers for class identification. In this process, a classifier initially predicts the class associated with a prototype, which is then used to determine the class label of a representation based on proximity to the nearest prototype. Results presented in Table \ref{tab:PACS_r} show that class labels can be accurately predicted using prototypes, sometimes even surpassing the effectiveness of direct classifier use.



% % \subsection{ Influence of Different Components}
% % \label{sec:apd_components}

% % \begin{table}[h!]
% % \caption{Classification Accuracy on PACS using ResNet50 with absence of different components.}
% % %\vspace{-0.5mm}
% % \begin{centering}
% % \resizebox{0.8\columnwidth}{!}{ %
% % \begin{tabular}{lccccc}
% % \toprule
% % Algorithm  & \textbf{A} & \textbf{C} & \textbf{P} & \textbf{S} & \textbf{Avg}  \\
% % \midrule


% % ERM & 89.3 $\pm$ 0.2 & 83.4 $\pm$ 0.6 & 97.3 $\pm$ 0.3 & 82.5 $\pm$ 0.5 & 88.1 \\
% % Our method w/o $\mathcal{L}_I$ & 89.9 $\pm$ 0.5 & 83.3 $\pm$ 0.4 & 97.9 $\pm$ 0.3 & 81.4 $\pm$ 0.6 & 88.1 \\
% % Our method w/o $\mathcal{L}_D$ & 89.7 $\pm$ 0.5 & 84.2$\pm$ 0.7 & 97.2 $\pm$ 0.3 &  81.5 $\pm$ 0.8 &  88.2\\
% % Our method & 90.2 $\pm$ 0.5 & 84.4 $\pm$ 0.3 & 97.9 $\pm$ 0.1 & 82.3 $\pm$ 0.2 & \textbf{88.7} \\
% % \bottomrule
% % \end{tabular}}
% % \par\end{centering}
% % \label{tab:PACS_componemt}
% % \end{table}
% % Table \ref{tab:PACS_componemt} demonstrates that removing any component significantly reduces performance. Specifically, omitting the subspace alignment constraint results in performance marginally better than the ERM baseline, whereas failing to maximize mutual information $\mathcal{I}(X,g(X))$ leads to performance on par with the baseline.

% \subsection{Ablation study on the number of Subspaces}
 
% Considering our data generation process, the number of distinct labels $\mathbb{P}(Y\mid x)$ reflects the number of distinct causal factors (denoted as $\left | \mathcal{Z} \right |$). If $\mathcal{M}\leq\left | \mathcal{Z} \right |$, samples with different labels may be projected into the same subspace, leading to discrepancies in the marginal label distribution within that subspace.

% We revisit the two key points in the previous discussion:

% \begin{itemize}
% \item Theorem~\ref{theorem:multi_bound} implies that projecting samples into the correct subspaces can significantly reduce or entirely eliminate marginal label shifts within those subspaces, assuming optimal projection for the sake of simplicity.
% \item As mentioned earlier, projecting samples with the same label $\mathbb{P}(Y\mid x)$ eliminates the discrepancy $d_{1/2}\left (\mathbb{P}^{e_i}_{\mathcal{Y},m}, \mathbb{P}^{e_i}_{\mathcal{Y},m} \right )$, reducing it to zero.
% \end{itemize}

% Increasing $\mathcal{M}$ reduces the likelihood of differently labeled samples being mapped to the same subspace, thus decreasing the discrepancy outlined in Theorem~\ref{theorem:multi_bound} (ii). It’s notable that the upper bound in (i) can be optimized to the limit defined by (ii) when the focus is only on training domains. This optimization, in turn, minimizes the bound (i).

% Rather than treating $\left | \mathcal{Z} \right |$ merely as a parameter for tuning, we delve further into analyzing the impact of varying $\left | \mathcal{Z} \right |$ values. In this ablation study, we test $\left | \mathcal{Z} \right |$ values of $[4,8,16,32]\times \left | \mathcal{C} \right |$, where $\left | \mathcal{C} \right |$ denotes the number of classes. 

% \label{sec:apd_subspaces}
% \begin{table}[h!]
% \caption{Classification Accuracy on PACS using ResNet50 with different number of subspaces (NoS) per class.}
% %\vspace{-0.5mm}
% \begin{centering}
% \resizebox{0.60\columnwidth}{!}{ %
% \begin{tabular}{lccccc}
% \toprule
% NoS \left | \mathcal{M} \right |  & \textbf{A} & \textbf{C} & \textbf{P} & \textbf{S} & \textbf{Avg}  \\
% \midrule
% ERM & 89.3 $\pm$ 0.2 & 83.4 $\pm$ 0.6 & 97.3 $\pm$ 0.3 & 82.5 $\pm$ 0.5 & 88.1 \\
% 4 & 90.2 $\pm$ 0.3 & 83.2 $\pm$ 0.7 & 97.9 $\pm$ 0.2 & 82.3 $\pm$ 1.5 & 88.2 \\

% 8 & 90.5 $\pm$ 0.8 & 83.8 $\pm$ 0.6 & 97.6 $\pm$ 0.3 &  82.1 $\pm$ 1.8 &  88.7\\

% 16 & 90.5 $\pm$ 0.5 & 83.4 $\pm$ 0.2 & 97.8 $\pm$ 0.1 & 83.2 $\pm$ 0.2 & 88.7 \\
% 32 & 90.2 $\pm$ 0.5&  83.8 $\pm$ 0.8 & 97.3 $\pm$ 0.4 & 82.0 $\pm$ 1.2 & 88.4\\
% \bottomrule
% \end{tabular}}
% \par\end{centering}
% \label{tab:PACS_prototype}
% \end{table}

% Table \ref{tab:PACS_prototype} reveals that performance generally improves with an increase in the number of prototypes. Nonetheless, a decline in performance is noted when $K$ becomes excessively large. We speculate this behavior is tied to the dataset's underlying causal factors; 
% specifically, if a limited number of causal factors generate the data, assigning a large number of prototypes to capture discriminative information might result in one causal factor being associated with multiple prototypes, thereby introducing ambiguity. This hypothesis, however, requires further investigation for confirmation, and we earmark it for future research.



% \subsection{Compare to other baselines}

% One of our key contributions is offering a new perspective on why domain generalization (DG) algorithms often fail to outperform the fundamental empirical risk minimization (ERM) approach on standard benchmarks, through an analysis of sufficient and necessary conditions. In the main paper, we compare our proposed SRA method with the two most related methods, DANN and CDANN, as they represent specific cases of our approach where the number of subspaces per class is set to 0 and 1, respectively. 

% In this section, we provide additional experimental results from various baselines, both with and without SWAD, on five datasets from the DomainBed benchmark \cite{gulrajani2020search}, to further support our discussion and analysis.


% \begin{table*}[h!]
% \caption{Classification accuracy (\%) for all algorithms across datasets.
% }
% \begin{centering}
% \resizebox{0.65\width}{!}{ %
% \begin{tabular}{lcccccc}
% \toprule
% \textbf{Algorithm}  & \textbf{VLCS} & \textbf{PACS} & \textbf{OfficeHome} & \textbf{TerraIncognita}  & \textbf{DomainNet} & \textbf{Avg} \\
% \toprule
% ERM~\citep{gulrajani2020search}  & 77.5 $\pm$ 0.4 & 85.5 $\pm$ 0.2 & 66.5 $\pm$ 0.3 & 46.1 $\pm$ 1.8 & 40.9 $\pm$ 0.1 & 63.3 \\
% GroupDRO~\citep{sagawa2019distributionally} & 76.7 $\pm$ 0.6 & 84.4 $\pm$ 0.8 & 66.0 $\pm$ 0.7 & 43.2 $\pm$ 1.1 & 33.3 $\pm$ 0.2 & 60.7 \\
% Mixup~\citep{wang2020heterogeneous}  & 77.4 $\pm$ 0.6 & 84.6 $\pm$ 0.6 & 68.1 $\pm$ 0.3 & 47.9 $\pm$ 0.8 & 39.2 $\pm$ 0.1 & 63.4 \\
% MLDG~\citep{li2017learning} & 77.2 $\pm$ 0.4 & 84.9 $\pm$ 1.0 & \textbf{66.8} $\pm$ 0.6 & 47.7 $\pm$ 0.9 & 41.2 $\pm$ 0.1 & 63.6 \\
% MTL~\citep{blanchard2021domain} & 77.2 $\pm$ 0.4 & 84.6 $\pm$ 0.5 & 66.4 $\pm$ 0.5 & 45.6 $\pm$ 1.2 & 40.6 $\pm$ 0.1 & 62.9 \\
% SagNet~\citep{nam2021reducing} & 77.8 $\pm$ 0.5 & \textbf{86.3} $\pm$ 0.2 & 68.1 $\pm$ 0.1 & 48.6 $\pm$ 1.0 & 40.3 $\pm$ 0.1 & 64.2 \\
% ARM~\citep{zhang2021adaptive} & 77.6 $\pm$ 0.3 & 85.1 $\pm$ 0.4 & 64.8 $\pm$ 0.3 & 45.5 $\pm$ 0.3 & 35.5 $\pm$ 0.2 & 61.7 \\
% RSC~\citep{huang2020self}  & 77.1 $\pm$ 0.5 & 85.2 $\pm$ 0.9 & 65.5 $\pm$ 0.9 & 46.6 $\pm$ 1.0 & 38.9 $\pm$ 0.5 & 62.7 \\
% IRM~\citep{arjovsky2020irm} & 78.5 $\pm$ 0.5 & 83.5 $\pm$ 0.8 & 64.3 $\pm$ 2.2 & 47.6 $\pm$ 0.8 & 33.9 $\pm$ 2.8 & 61.6 \\
% VREx~\citep{krueger2021out} & 78.3 $\pm$ 0.2 & 84.9 $\pm$ 0.6 & 66.4 $\pm$ 0.6 & 46.4 $\pm$ 0.6 & 33.6 $\pm$ 2.9 & 61.9 \\
% MMD~\citep{li2018domain} & 77.5 $\pm$ 0.9 & 84.6 $\pm$ 0.5 & 66.3 $\pm$ 0.1 & 42.2 $\pm$ 1.6 & 23.4 $\pm$ 9.5 & 58.8 \\
% CORAL~\citep{sun2016deep} & \textbf{78.8} $\pm$ 0.6 & 86.2 $\pm$ 0.3 & \textbf{68.7} $\pm$ 0.3 & 47.6 $\pm$ 1.0 & {41.5} $\pm$ 0.1 & {64.5} \\
% DANN~\citep{ganin2016domain} & 78.6 $\pm$ 0.4 & 83.6 $\pm$ 0.4 & 65.9 $\pm$ 0.6 & 46.7 $\pm$ 0.5 & 38.3 $\pm$ 0.1 & 62.6 \\
% CDANN~\citep{li2018domain}  & 77.5 $\pm$ 0.1 & 82.6 $\pm$ 0.9 & 65.8 $\pm$ 1.3 & 45.8 $\pm$ 1.6 & 38.3 $\pm$ 0.3 & 62.0 \\
% \midrule

% \textbf{Ours} (SRA)  & 76.4 $\pm$ 0.7 & \textbf{86.3} $\pm$ 1.1 & 66.4 $\pm$ 0.7 & \textbf{49.5} $\pm$ 1.0 & \textbf{44.5} $\pm$ 0.3 & \textbf{64.6} \\
% \bottomrule
% \end{tabular}}
% \par\end{centering}
% \label{tab:Averages_domainbed}
% \end{table*}



% \begin{table*}[h!]
% \caption{Classification accuracy (\%) for all algorithms across datasets.
% }
% \begin{centering}
% \resizebox{0.65\width}{!}{ %
% \begin{tabular}{lccccc}
% \toprule
% \textbf{Algorithm}  & \textbf{VLCS} & \textbf{PACS} & \textbf{OfficeHome} & \textbf{TerraIncognita}  & \textbf{Avg} \\
% \toprule
% SWAD~\citep{cha2021swad} & 79.1 $\pm$ 0.4 & 88.1 $\pm$ 0.4 & 70.6 $\pm$ 0.3 & 50.0 $\pm$ 0.4  & 72.0\\




% SWAD + IRM~\citep{arjovsky2020irm} & 78.8 $\pm$ 0.2 & 88.1 $\pm$ 0.4 & 70.4 $\pm$ 0.2 & 49.6 $\pm$ 1.7  & 71.7 \\
% SWAD + VREx~\citep{krueger2021out} & 78.1 $\pm$ 1.3 & 85.4 $\pm$ 0.5 & 69.9 $\pm$ 0.1 & 50.0 $\pm$ 0.2 & 70.9 \\
% SWAD +CORAL~\citep{sun2016deep} & \underline{78.9} $\pm$ 0.6 & 88.3 $\pm$ 0.5 & 71.4 $\pm$ 0.1 & 51.1 $\pm$ 0.9 & 72.4 \\
% SWAD +MMD~\citep{li2018domain} & 78.7 $\pm$ 0.1 & 88.3 $\pm$ 0.1 & 70.6 $\pm$ 0.4 & 49.6 $\pm$ 0.5  & 71.8 \\
% SWAD + DANN & 79.2 $\pm$ 0.0 & 87.9 $\pm$ 0.5 & 70.5 $\pm$ 0.1 & 50.6 $\pm$ 0.6  & 72.2\\
% SWAD + CDANN & 79.3 $\pm$ 0.2 & 87.7 $\pm$ 0.3 & 70.4 $\pm$ 0.1 & 50.7 $\pm$ 0.1  & 72.2\\
% \midrule

% \textbf{Ours} (SRA + SWAD) & \underline{79.4} $\pm$ 0.4 & \underline{88.7} $\pm$ 0.2 &  \underline{72.1} $\pm$ 0.5 &  \underline{51.6} $\pm$ 1.2 & \underline{73.0} \\
% \textbf{Ours} (SRA + SWAD + Ensemble) & \textbf{79.8} $\pm$ 0.0 & \textbf{89.2} $\pm$ 0.0 &  \textbf{73.2} $\pm$ 0.0 &  \textbf{52.2} $\pm$ 0.0 & \textbf{73.3} \\
% \bottomrule
% \end{tabular}}
% \par\end{centering}
% \label{tab:Averages_domainbed_swad}
% \end{table*}

% As observed in both Table~\ref{tab:Averages_domainbed} and Table~\ref{tab:Averages_domainbed_swad}, the baselines fail to consistently surpass the simple ERM baseline across all settings. While some methods perform well on certain datasets, they perform worse on others. However, the combination of our proposed method (SRA), which enforces strong sufficient conditions, and SWAD, which promotes necessary conditions, significantly improves generalization. This combination outperforms ERM and other baselines in all settings. These results support our analysis in Section~\ref{sec:discussion_DG}, indicating that existing methods often violate the necessary condition for effective domain generalization.

\section{Experimental Settings}
\label{apd:settings}

% \paragraph{Metric.} we adopt the training and evaluation protocol as in DomainBed benchmark \citep{gulrajani2020search}, including dataset splits, hyperparameter (HP) search, model selection on the validation set, and optimizer HP. However, we use a reduced HP search space to reduce computational costs. For training, we choose one domain as the target domain and the remaining domains as the training domain, with 20\% of the samples used for validation and model selection. 

% \paragraph{Datasets.} Following existing benchmark \citep{gulrajani2020search}, we evaluate our method on five datasets: PACS~\citep{li2017deeper} (9,991 images, 7 classes, and 4 domains), VLCS~\citep{torralba2011unbiased} (10,729 images, 5 classes, and 4 domains), OfficeHome~\citep{venkateswara2017deep} (15,588 images, 65 classes, and 4 domains), TerraIncognita~\citep{beery2018recognition} (24,788 images, 10 classes, and 4 domains), and DomainNet~\citep{peng2019moment} (586,575 images, 345 classes, and 6 domains).
\paragraph{Metrics.} We adopt the training and evaluation protocol as in DomainBed benchmark \citep{gulrajani2020search}, including dataset splits, hyperparameter (HP) search, model selection on the validation set, and optimizer HP. To manage computational demands more efficiently, as suggested by \citep{cha2021swad}, we narrow our HP search space. Specifically, we use the Adam optimizer, as detailed in \citep{gulrajani2020search}, setting the learning rate to a default of $5e^{-5}$ and forgoing dropout and weight decay adjustments. The batch size is maintained at 32. For DomainNet, we run a total of 15,000 iterations, while for other datasets, we limit iterations to 5,000, deemed adequate for model convergence. Our method's unique parameters, including the regularization hyperparameters $(\lambda_P, \lambda_D)$, undergo optimization within the range of $[0.01, 0.1, 1.0]$, and the number of prototypes $\left | \mathcal{Z} \right |$ is fixed at 16 times the number of classes. It is worth noting that  while we conduct ablation study on PACS dataset, we utilize the number of prototypes $\left | \mathcal{Z} \right |$ is fixed at $16$ times the number of classes for all datasets.
SWAD-specific hyperparameters remain unaltered from their default settings. The evaluation frequency is set to 300 for all dataset.

Our code is anonymously published at \url{https://anonymous.4open.science/r/submisson-FCF0}.


\subsection{Datasets}
To evaluate the effectiveness of the proposed method, we utilize five
datasets: PACS~\citep{li2017deeper}, VLCS~\citep{torralba2011unbiased},
 Office-Home~\citep{venkateswara2017deep}, Terra Incognita~\citep{beery2018recognition} and DomainNet~\citep{peng2019moment} which are the common DG benchmarks with multi-source domains.
\begin{itemize}
    \item \textbf{PACS}~\citep{li2017deeper}: 9991 images of seven classes in total, over four domains:Art\_painting (A), Cartoon (C), Sketches (S), and Photo (P). 
    
    \item \textbf{VLCS}~\citep{torralba2011unbiased}: five classes over four domains with a total of 10729 samples. The domains are defined by four image origins, i.e., images were taken from the PASCAL VOC 2007 (V), LabelMe (L), Caltech (C) and Sun (S) datasets. 


    \item \textbf{Office-Home}~\citep{venkateswara2017deep}: 65 categories of 15,500 daily objects from 4 domains: Art, Clipart, Product (vendor website with white-background) and Real-World (real-object collected from regular cameras).
    \item \textbf{Terra Incognita}~\citep{beery2018recognition} includes 24,788 wild photographs of dimension (3, 224, 224) with 10
animals, over 4 camera-trap domains L100, L38, L43 and L46. This dataset contains photographs of wild animals taken by camera traps; camera trap locations are different across 
domains. 
    \item  \textbf{DomainNet}~\citep{peng2019moment} contains 596,006 images of dimension (3, 224, 224) and 345 classes, over
6 domains clipart, infograph, painting, quickdraw, real and sketch. This is the biggest
dataset in terms of the number of samples and classes.
\end{itemize}



% \subsection{Results}
% \label{apd:result_details}
% In this section, we present the extended results of Table \ref{tab:Averages} in the main text. The following tables report the domain-specific performance of each method on 5 datasets: VLCS (Table \ref{tab:VLCS}), PACS (Table \ref{tab:PACS}), OfficeHome (Table \ref{tab:OfficeHome}), TerraIncognita (Table \ref{tab:TerraIncognita}) and Domain Net (Table \ref{tab:DomainNet}).

% Standard errors are computed over three trials. Our models are run on 4 RTX 6000 GPU cores of 32GB. One full training routine takes roughly 2 hours. 



% % \subsubsection{VLCS}
% \begin{table}[h!]
% \caption{Classification Accuracy on \textbf{VLCS} using ResNet50}
% %\vspace{-0.5mm}
% \begin{centering}
% \resizebox{0.7\columnwidth}{!}{ %
% \begin{tabular}{lccccc}
% \toprule
% \textbf{Algorithm}  & \textbf{C} & \textbf{L} & \textbf{S} & \textbf{V} & \textbf{Avg}  \\
% \midrule
% ERM~\citep{zhang2020adaptive} & 97.7 $\pm$ 0.4 & 64.3 $\pm$ 0.9 & 73.4 $\pm$ 0.5 & 74.6 $\pm$ 1.3 & 77.5 \\
% DANN~\citep{ganin2016domain}& 99.0 $\pm$ 0.3 & \textbf{65.1} $\pm$ 1.4 & 73.1 $\pm$ 0.3 & 77.2 $\pm$ 0.6 & 78.6 \\
% CDANN~\citep{li2018domain}& 97.1 $\pm$ 0.3 & \textbf{65.1} $\pm$ 1.2 & 70.7 $\pm$ 0.8 & 77.1 $\pm$ 1.5 & 77.5 \\
% \textbf{Ours} (SRA) & {97.1} $\pm$ 1.5 & {63.8} $\pm$ 2.3 & 70.5 $\pm$ 2.2 & 74.1 $\pm$ 1.8 & 76.4 \\
% \midrule


% %MTL~\citep{blanchard2021domain}& 97.8 $\pm$ 0.4 & 64.3 $\pm$ 0.3 & 71.5 $\pm$ 0.7 & 75.3 $\pm$ 1.7 & 77.2 \\
% %SagNet~\citep{nam2021reducing} & 97.9 $\pm$ 0.4 & 64.5 $\pm$ 0.5 & 71.4 $\pm$ 1.3 & {77.5} $\pm$ 0.5 & 77.8 \\
% %ARM~\citep{zhang2020adaptive}& 98.7 $\pm$ 0.2 & 63.6 $\pm$ 0.7 & 71.3 $\pm$ 1.2 & 76.7 $\pm$ 0.6 & 77.6 \\
% %VREx~\citep{krueger2021out}& 98.4 $\pm$ 0.3 & 64.4 $\pm$ 1.4 & {74.1} $\pm$ 0.4 & 76.2 $\pm$ 1.3 & 78.3 \\
% %RSC~\citep{huang2020self}& 97.9 $\pm$ 0.1 & 62.5 $\pm$ 0.7 & 72.3 $\pm$ 1.2 & 75.6 $\pm$ 0.8 & 77.1 \\

% SWAD~\cite{cha2021swad}& {98.8} $\pm$ 0.1 & 63.3 $\pm$ 0.3 & 75.3 $\pm$ 0.5 & 79.2 $\pm$ 0.6 & {79.1}\\
% SWAD + DANN& {99.2} $\pm$ 0.1 & 63.0 $\pm$ 0.8 & 75.3 $\pm$ 1.8 & 79.3 $\pm$ 0.5 & {79.2}\\
% SWAD + CDANN& {99.1} $\pm$ 0.1 & 63.3 $\pm$ 0.7 & 75.1 $\pm$ 0.7 & 80.1 $\pm$ 0.2 & {79.3}\\


% %DNA~\cite{chu2022dna} & {98.8} $\pm$ 0.1 & 63.6 $\pm$ 0.2 & {74.1} $\pm$ 0.1 & \textbf{79.5} $\pm$ 0.4 & 79.0\\


% %DiWA ($M=20$)~\citep{rame2022diverse} & 98.4 $\pm$ 0.1 & 63.4 $\pm$ 0.1 & 75.5 $\pm$ 0.3 & 78.9 $\pm$ 0.6 & 79.1\\
% %DiWA ($M=60$)~\citep{rame2022diverse} & 98.4 & 63.3 & 76.1 & 79.6 & 79.4\\


% \textbf{Ours} (SRA + SWAD) & {98.9} $\pm$ 0.2 & {63.7} $\pm$ 0.3 & 75.6 $\pm$ 0.4 & 79.4 $\pm$ 0.8 & 79.4 \\
% \midrule
% \textbf{Ours} (SRA + SWAD + Ensemble) & \textbf{99.1} $\pm$ 0.0 & {63.9} $\pm$ 0.0 & \textbf{76.3} $\pm$ 0.0 & \textbf{79.9} $\pm$ 0.8 & \textbf{79.8} \\
% \bottomrule
% \end{tabular}}
% \par\end{centering}
% \label{tab:VLCS}
% \end{table}

% % \subsubsection{PACS}

% \begin{table}[h!]
% \caption{Classification Accuracy on \textbf{PACS} using ResNet50}
% %\vspace{-0.5mm}
% \begin{centering}
% \resizebox{0.7\columnwidth}{!}{ %
% \begin{tabular}{lccccc}
% \toprule
% \textbf{Algorithm}  & \textbf{A} & \textbf{C} & \textbf{P} & \textbf{S} & \textbf{Avg}  \\
% \midrule
% ERM~\citep{gulrajani2020search}& 84.7 $\pm$ 0.4 & {80.8} $\pm$ 0.6 & 97.2 $\pm$ 0.3 & 79.3 $\pm$ 1.0 & 85.5 \\
% %IRM~\citep{arjovsky2020%IRM}& 84.8 $\pm$ 1.3 & 76.4 $\pm$ 1.1 & 96.7 $\pm$ 0.6 & 76.1 $\pm$ 1.0 & 83.5 \\
% %GroupDRO~\citep{sagawa2019distributionally}& 83.5 $\pm$ 0.9 & 79.1 $\pm$ 0.6 & 96.7 $\pm$ 0.3 & 78.3 $\pm$ 2.0 & 84.4 \\
% %Mixup~\citep{wang2020heterogeneous}& 86.1 $\pm$ 0.5 & 78.9 $\pm$ 0.8 & \textbf{97.6} $\pm$ 0.1 & 75.8 $\pm$ 1.8 & 84.6 \\
% %MLDG~\citep{li2017learning}& 85.5 $\pm$ 1.4 & 80.1 $\pm$ 1.7 & 97.4 $\pm$ 0.3 & 76.6 $\pm$ 1.1 & 84.9 \\
% %%CORAL & \textbf{88.3} $\pm$ 0.2 & 80.0 $\pm$ 0.5 & 97.5 $\pm$ 0.3 & 78.8 $\pm$ 1.3 & 86.2 \\
% %MMD~\citep{li2018domain}& 86.1 $\pm$ 1.4 & 79.4 $\pm$ 0.9 & 96.6 $\pm$ 0.2 & 76.5 $\pm$ 0.5 & 84.6 \\
% DANN~\citep{ganin2016domain}& 86.4 $\pm$ 0.8 & 77.4 $\pm$ 0.8 & 97.3 $\pm$ 0.4 & 73.5 $\pm$ 2.3 & 83.6 \\
% CDANN~\citep{li2018domain}& 84.6 $\pm$ 1.8 & 75.5 $\pm$ 0.9 & 96.8 $\pm$ 0.3 & 73.5 $\pm$ 0.6 & 82.6 \\
% \textbf{Ours} (SRA) & {86.4} $\pm$ 0.2 & {82.0} $\pm$ 0.8 & 96.7 $\pm$ 1.1 & 80.2 $\pm$ 4.4 & 86.3 \\
% \midrule
% %MTL~\citep{blanchard2021domain}& 87.5 $\pm$ 0.8 & 77.1 $\pm$ 0.5 & 96.4 $\pm$ 0.8 & 77.3 $\pm$ 1.8 & 84.6 \\
% %SagNet~\citep{nam2021reducing} & 87.4 $\pm$ 1.0 & 80.7 $\pm$ 0.6 & 97.1 $\pm$ 0.1 & 80.0 $\pm$ 0.4 & {86.3} \\
% %ARM~\citep{zhang2020adaptive}& 86.8 $\pm$ 0.6 & 76.8 $\pm$ 0.5 & 97.4 $\pm$ 0.3 & 79.3 $\pm$ 1.2 & 85.1 \\
% %VREx~\citep{krueger2021out}& 86.0 $\pm$ 1.6 & 79.1 $\pm$ 0.6 & 96.9 $\pm$ 0.5 & 77.7 $\pm$ 1.7 & 84.9 \\
% %RSC~\citep{huang2020self}& 85.4 $\pm$ 0.8 & 79.7 $\pm$ 1.8 & {97.6} $\pm$ 0.3 & 78.2 $\pm$ 1.2 & 85.2 \\
% SWAD~\cite{cha2021swad}& 89.3 $\pm$ 0.2 & 83.4 $\pm$ 0.6 & 97.3 $\pm$ 0.3 & 82.5 $\pm$ 0.5 & 88.1\\
% SWAD + DANN& 90.7 $\pm$ 1.2 & 82.2 $\pm$ 0.4 & 97.3 $\pm$ 0.1 & 81.6 $\pm$ 0.4 & 87.9\\
% SWAD + CDANN & 90.5 $\pm$ 0.3 & 82.4 $\pm$ 1.0 & 97.6 $\pm$ 0.1 & 80.4 $\pm$ 0.3 & 87.7\\

% %DNA~\cite{chu2022dna} & 89.8 $\pm$ 0.2 & 83.4 $\pm$ 0.4 & {97.7} $\pm$ 0.1 & \textbf{82.6} $\pm$ 0.2 & {88.4}\\

% %DiWA ($M=20$)~\citep{rame2022diverse} & 90.1 $\pm$ 0.6 & 83.3 $\pm$ 0.6 & \textbf{98.2} $\pm$ 0.1 & 83.4 $\pm$ 0.4 & 88.8\\
% %DiWA ($M=60$)~\citep{rame2022diverse} & 90.5 & 83.7 & \textbf{98.2} & 83.8 & 89.0\\


% \textbf{Ours} (SRA + SWAD) & 90.5 $\pm$ 0.5 & 83.4 $\pm$ 0.2 & 97.8 $\pm$ 0.1 & 83.2 $\pm$ 0.2 & 88.7 \\
% \midrule
% \textbf{Ours} (SRA + SWAD + Ensemble) & \textbf{91.2} $\pm$ 0.0 & \textbf{83.8} $\pm$ 0.0 & 97.8 $\pm$ 0.0 & \textbf{83.9} $\pm$ 0.0 & \textbf{89.2} \\
% \bottomrule
% \end{tabular}}
% \par\end{centering}
% \label{tab:PACS}
% \end{table}

% % \subsubsection{OfficeHome}
% \begin{table}[h!]
% \caption{Classification Accuracy on \textbf{OfficeHome} using ResNet50}
% %\vspace{-0.5mm}
% \begin{centering}
% \resizebox{0.7\columnwidth}{!}{ %
% \begin{tabular}{lccccc}
% \toprule
% \textbf{Algorithm}  & \textbf{A} & \textbf{C} & \textbf{P} & \textbf{R} & \textbf{Avg}  \\
% \midrule
% ERM~\citep{gulrajani2020search}& 61.3 $\pm$ 0.7 & 52.4 $\pm$ 0.3 & 75.8 $\pm$ 0.1 & 76.6 $\pm$ 0.3 & 66.5 \\
% %IRM~\citep{arjovsky2020%IRM}& 58.9 $\pm$ 2.3 & 52.2 $\pm$ 1.6 & 72.1 $\pm$ 2.9 & 74.0 $\pm$ 2.5 & 64.3 \\
% %GroupDRO~\citep{sagawa2019distributionally}& 60.4 $\pm$ 0.7 & 52.7 $\pm$ 1.0 & 75.0 $\pm$ 0.7 & 76.0 $\pm$ 0.7 & 66.0 \\
% %Mixup~\citep{wang2020heterogeneous}& 62.4 $\pm$ 0.8 & 54.8 $\pm$ 0.6 & \textbf{76.9} $\pm$ 0.3 & 78.3 $\pm$ 0.2 & 68.1 \\
% %MLDG~\citep{li2017learning}& 61.5 $\pm$ 0.9 & 53.2 $\pm$ 0.6 & 75.0 $\pm$ 1.2 & 77.5 $\pm$ 0.4 & 66.8 \\
% %%CORAL & \textbf{65.3} $\pm$ 0.4 & 54.4 $\pm$ 0.5 & {76.5} $\pm$ 0.1 & \textbf{78.4} $\pm$ 0.5 & \textbf{68.7} \\
% %MMD~\citep{li2018domain}& 60.4 $\pm$ 0.2 & 53.3 $\pm$ 0.3 & 74.3 $\pm$ 0.1 & 77.4 $\pm$ 0.6 & 66.3 \\
% DANN~\citep{ganin2016domain}& 59.9 $\pm$ 1.3 & 53.0 $\pm$ 0.3 & 73.6 $\pm$ 0.7 & 76.9 $\pm$ 0.5 & 65.9 \\
% CDANN~\citep{li2018domain}& 61.5 $\pm$ 1.4 & 50.4 $\pm$ 2.4 & 74.4 $\pm$ 0.9 & 76.6 $\pm$ 0.8 & 65.8 \\
% %MTL~\citep{blanchard2021domain}& 61.5 $\pm$ 0.7 & 52.4 $\pm$ 0.6 & 74.9 $\pm$ 0.4 & 76.8 $\pm$ 0.4 & 66.4 \\
% %SagNet~\citep{nam2021reducing} & 63.4 $\pm$ 0.2 & \textbf{54.8} $\pm$ 0.4 & 75.8 $\pm$ 0.4 & 78.3 $\pm$ 0.3 & 68.1 \\
% %ARM~\citep{zhang2020adaptive}& 58.9 $\pm$ 0.8 & 51.0 $\pm$ 0.5 & 74.1 $\pm$ 0.1 & 75.2 $\pm$ 0.3 & 64.8 \\
% %VREx~\citep{krueger2021out}& 60.7 $\pm$ 0.9 & 53.0 $\pm$ 0.9 & 75.3 $\pm$ 0.1 & 76.6 $\pm$ 0.5 & 66.4 \\
% %RSC~\citep{huang2020self}& 60.7 $\pm$ 1.4 & 51.4 $\pm$ 0.3 & 74.8 $\pm$ 1.1 & 75.1 $\pm$ 1.3 & 65.5 \\
% \textbf{Ours} (SRA) & {62.2} $\pm$ 1.4 & {52.3} $\pm$ 1.7 & 74.5 $\pm$ 0.8 & 76.6 $\pm$ 1.3 & 66.4 \\

% \midrule
% SWAD~\cite{cha2021swad}& 66.1 $\pm$ 0.4 & 57.7 $\pm$ 0.4 & 78.4 $\pm$0.1 & 80.2 $\pm$ 0.2& 70.6\\
% SWAD + DANN& 67.2 $\pm$ 0.1 & 56.2 $\pm$ 0.1 & 78.6 $\pm$0.2 & 80.0 $\pm$ 0.5& 70.5\\
% SWAD + CDANN& 66.8 $\pm$ 0.4 & 56.4 $\pm$ 0.8 & 78.4 $\pm$0.5 & 80.1 $\pm$ 0.2& 70.4\\

% %DNA~\cite{chu2022dna} & 67.7 $\pm$ 0.2 & {57.7} $\pm$ 0.3 & 78.9 $\pm$ 0.2 & 80.5 $\pm$ 0.2 & 71.2\\

% %DiWA ($M=20$)~\citep{rame2022diverse} & 67.3 $\pm$ 0.2 & 57.9 $\pm$ 0.2 & 79.0 $\pm$ 0.2 & 79.9 $\pm$ 0.1 & 71.0 \\
% %DiWA ($M=60$)~\citep{rame2022diverse} & 67.7 & 58.8 & 79.4 & 80.5 & 71.6\\


% \textbf{Ours} (SRA + SWAD) & {69.1} $\pm$ 0.6 & 58.4 $\pm$ 0.8 &  {79.5} $\pm$ 0.2 &  {81.4} $\pm$ 0.3 & {72.1}\\
% \midrule
% \textbf{Ours} (SRA + SWAD + Ensemble) & \textbf{70.5} $\pm$ 0.0 & \textbf{59.5} $\pm$ 0.0 &  \textbf{80.4} $\pm$ 0.0 &  \textbf{82.1} $\pm$ 0.0 & \textbf{73.2}\\
% \bottomrule
% \end{tabular}}
% \par\end{centering}
% \label{tab:OfficeHome}
% \end{table}

% % \subsubsection{TerraIncognita}
% \begin{table}[h!]
% \caption{Classification Accuracy on \textbf{TerraIncognita} using ResNet50}
% \begin{centering}
% \resizebox{0.7\columnwidth}{!}{ %
% \begin{tabular}{lccccc}
% \toprule
% \textbf{Algorithm}  & \textbf{L100} & \textbf{L38}  & \textbf{L43}  & \textbf{L46}  & \textbf{Avg}  \\
% \midrule
% ERM~\citep{gulrajani2020search}& 49.8 $\pm$ 4.4 & 42.1 $\pm$ 1.4 & 56.9 $\pm$ 1.8 & 35.7 $\pm$ 3.9 & 46.1 \\
% %IRM~\citep{arjovsky2020%IRM}& 54.6 $\pm$ 1.3 & 39.8 $\pm$ 1.9 & 56.2 $\pm$ 1.8 & 39.6 $\pm$ 0.8 & 47.6 \\
% %GroupDRO~\citep{sagawa2019distributionally}& 41.2 $\pm$ 0.7 & 38.6 $\pm$ 2.1 & 56.7 $\pm$ 0.9 & 36.4 $\pm$ 2.1 & 43.2 \\
% %Mixup~\citep{wang2020heterogeneous}& \textbf{59.6} $\pm$ 2.0 & 42.2 $\pm$ 1.4 & 55.9 $\pm$ 0.8 & 33.9 $\pm$ 1.4 & 47.9 \\
% %MLDG~\citep{li2017learning}& {54.2} $\pm$ 3.0 & \textbf{44.3} $\pm$ 1.1 & 55.6 $\pm$ 0.3 & 36.9 $\pm$ 2.2 & 47.7 \\
% %%CORAL & 51.6 $\pm$ 2.4 & {42.2} $\pm$ 1.0 & 57.0 $\pm$ 1.0 & 39.8 $\pm$ 2.9 & 47.6 \\
% %MMD~\citep{li2018domain}& 41.9 $\pm$ 3.0 & 34.8 $\pm$ 1.0 & 57.0 $\pm$ 1.9 & 35.2 $\pm$ 1.8 & 42.2 \\
% DANN~\citep{ganin2016domain}& 51.1 $\pm$ 3.5 & 40.6 $\pm$ 0.6 & {57.4} $\pm$ 0.5 & 37.7 $\pm$ 1.8 & 46.7 \\
% CDANN~\citep{li2018domain}& 47.0 $\pm$ 1.9 & 41.3 $\pm$ 4.8 & 54.9 $\pm$ 1.7 & 39.8 $\pm$ 2.3 & 45.8 \\
% \textbf{Ours} (SRA) & {52.9} $\pm$ 3.5 & {45.8} $\pm$ 5.1 & 57.2 $\pm$ 4.6 & 42.3 $\pm$ 1.1 & 49.5 \\

% %MTL~\citep{blanchard2021domain}& 49.3 $\pm$ 1.2 & 39.6 $\pm$ 6.3 & 55.6 $\pm$ 1.1 & 37.8 $\pm$ 0.8 & 45.6 \\
% %SagNet~\citep{nam2021reducing} & 53.0 $\pm$ 2.9 & 43.0 $\pm$ 2.5 & \textbf{57.9} $\pm$ 0.6 & {40.4} $\pm$ 1.3 & \textbf{48.6} \\
% %ARM~\citep{zhang2020adaptive}& 49.3 $\pm$ 0.7 & 38.3 $\pm$ 2.4 & 55.8 $\pm$ 0.8 & 38.7 $\pm$ 1.3 & 45.5 \\
% %VREx~\citep{krueger2021out}& 48.2 $\pm$ 4.3 & 41.7 $\pm$ 1.3 & 56.8 $\pm$ 0.8 & 38.7 $\pm$ 3.1 & 46.4 \\
% %RSC~\citep{huang2020self}& 50.2 $\pm$ 2.2 & 39.2 $\pm$ 1.4 & 56.3 $\pm$ 1.4 & 40.8 $\pm$ 0.6 & 46.6 \\

% %DiWA ($M=20$)~\citep{rame2022diverse} & 52.2 $\pm$ 1.8 & 46.2 $\pm$ 0.4 & 59.2 $\pm$ 0.2 & 37.8 $\pm$ 0.6 & 48.9\\
% %DiWA ($M=60$)~\citep{rame2022diverse} & 52.7 & 46.3 & 59.0 & 37.7 & 49.0\\
% \midrule
% SWAD~\cite{cha2021swad}& 55.4 $\pm$ 0.0 & 44.9 $\pm$ 1.1 & 59.7 $\pm$ 0.4 & 39.9 $\pm$ 0.2 & 50.0\\

% SWAD + DANN & 56.3 $\pm$ 2.6 & 44.9 $\pm$ 0.4 & 60.0 $\pm$ 0.7 & 41.4 $\pm$ 0.3 & 50.6\\

% SWAD + CDANN& 55.2 $\pm$ 2.2 & 45.3 $\pm$ 0.2 & 61.4 $\pm$ 0.7 & 40.9 $\pm$ 2.0 & 50.7\\
% %DNA~\cite{chu2022dna} & 56.8 $\pm$ 1.2 & 47.0 $\pm$ 0.9 & \textbf{61.0} $\pm$ 0.5 & \textbf{44.0} $\pm$ 1.0 & 52.2\\
% \textbf{Ours} (SRA + SWAD) & {56.2} $\pm$ 0.8 & {45.5} $\pm$ 2.6 & {60.4} $\pm$ 1.0& {44.4} $\pm$ 0.6 & {51.6} \\
% \midrule
% \textbf{Ours} (SRA + SWAD + Ensemble) & \textbf{57.4} $\pm$ 0.0 & \textbf{45.3} $\pm$ 0.0 & \textbf{60.9} $\pm$ 0.0 & \textbf{45.2} $\pm$ 0.0 & \textbf{52.2} \\

% \bottomrule
% \end{tabular}}
% \par\end{centering}
% \label{tab:TerraIncognita}
% \end{table}

% % \subsubsection{DomainNet}
% \begin{table}[h!]
% \caption{Classification Accuracy on \textbf{DomainNet} using ResNet50}
% %\vspace{-0.5mm}
% \begin{centering}
% \resizebox{0.9\columnwidth}{!}{ %
% \begin{tabular}{lccccccc}
% \toprule
% \textbf{Algorithm}  & \textbf{clip} & \textbf{info} & \textbf{paint} & \textbf{quick} & \textbf{real} & \textbf{sketch} & \textbf{Avg}  \\
% \midrule
% ERM~\citep{gulrajani2020search}& 58.1 $\pm$ 0.3 & 18.8 $\pm$ 0.3 & 46.7 $\pm$ 0.3 & 12.2 $\pm$ 0.4 & 59.6 $\pm$ 0.1 & 49.8 $\pm$ 0.4 & 40.9 \\
% %IRM~\citep{arjovsky2020%IRM}& 48.5 $\pm$ 2.8 & 15.0 $\pm$ 1.5 & 38.3 $\pm$ 4.3 & 10.9 $\pm$ 0.5 & 48.2 $\pm$ 5.2 & 42.3 $\pm$ 3.1 & 33.9 \\
% % GroupDRO~\citep{sagawa2019distributionally}& 47.2 $\pm$ 0.5 & 17.5 $\pm$ 0.4 & 33.8 $\pm$ 0.5 & 9.3 $\pm$ 0.3 & 51.6 $\pm$ 0.4 & 40.1 $\pm$ 0.6 & 33.3 \\
% % Mixup~\citep{wang2020heterogeneous}& 55.7 $\pm$ 0.3 & 18.5 $\pm$ 0.5 & 44.3 $\pm$ 0.5 & 12.5 $\pm$ 0.4 & 55.8 $\pm$ 0.3 & 48.2 $\pm$ 0.5 & 39.2 \\
% %CORAL & 59.2 $\pm$ 0.1 & 19.7 $\pm$ 0.2 & 46.6 $\pm$ 0.3 & {13.4} $\pm$ 0.4 & 59.8 $\pm$ 0.2 & 50.1 $\pm$ 0.6 & 41.5 \\
% %MMD~\citep{li2018domain}& 32.1 $\pm$ 13.3 & 11.0 $\pm$ 4.6 & 26.8 $\pm$ 11.3 & 8.7 $\pm$ 2.1 & 32.7 $\pm$ 13.8 & 28.9 $\pm$ 11.9 & 23.4 \\
% DANN~\citep{ganin2016domain}& 53.1 $\pm$ 0.2 & 18.3 $\pm$ 0.1 & 44.2 $\pm$ 0.7 & 11.8 $\pm$ 0.1 & 55.5 $\pm$ 0.4 & 46.8 $\pm$ 0.6 & 38.3 \\
% CDANN~\citep{li2018domain}& 54.6 $\pm$ 0.4 & 17.3 $\pm$ 0.1 & 43.7 $\pm$ 0.9 & 12.1 $\pm$ 0.7 & 56.2 $\pm$ 0.4 & 45.9 $\pm$ 0.5 & 38.3 \\
% \textbf{Ours} (SRA) & {64.2} $\pm$ 0.3 & {21.6} $\pm$ 0.9 & 50.8 $\pm$ 1.1 & 13.3 $\pm$ 0.8 & 64.4 $\pm$ 0.1 & 53.0 $\pm$ 0.4 &  44.5\\
% \midrule
% % MTL~\citep{blanchard2021domain}& 57.9 $\pm$ 0.5 & 18.5 $\pm$ 0.4 & 46.0 $\pm$ 0.1 & 12.5 $\pm$ 0.1 & 59.5 $\pm$ 0.3 & 49.2 $\pm$ 0.1 & 40.6 \\
% % SagNet~\citep{nam2021reducing} & 57.7 $\pm$ 0.3 & 19.0 $\pm$ 0.2 & 45.3 $\pm$ 0.3 & 12.7 $\pm$ 0.5 & 58.1 $\pm$ 0.5 & 48.8 $\pm$ 0.2 & 40.3 \\
% % ARM~\citep{zhang2020adaptive}& 49.7 $\pm$ 0.3 & 16.3 $\pm$ 0.5 & 40.9 $\pm$ 1.1 & 9.4 $\pm$ 0.1 & 53.4 $\pm$ 0.4 & 43.5 $\pm$ 0.4 & 35.5 \\
% %VREx~\citep{krueger2021out}& 47.3 $\pm$ 3.5 & 16.0 $\pm$ 1.5 & 35.8 $\pm$ 4.6 & 10.9 $\pm$ 0.3 & 49.6 $\pm$ 4.9 & 42.0 $\pm$ 3.0 & 33.6 \\
% %RSC~\citep{huang2020self}& 55.0 $\pm$ 1.2 & 18.3 $\pm$ 0.5 & 44.4 $\pm$ 0.6 & 12.2 $\pm$ 0.2 & 55.7 $\pm$ 0.7 & 47.8 $\pm$ 0.9 & 38.9 \\
% SWAD~\cite{cha2021swad}& 66.0 $\pm$ 0.1 & 22.4 $\pm$ 0.3 & 53.5 $\pm$ 0.1 & 16.1 $\pm$ 0.2 & 65.8 $\pm$ 0.4 & {55.5} $\pm$ 0.3 & 46.5\\
% SWAD + DANN& 64.3 $\pm$ 0.1 & 21.9 $\pm$ 0.6 & 52.6 $\pm$ 0.2 & 15.5 $\pm$ 0.2 & 65.3 $\pm$ 0.1 & {54.5} $\pm$ 0.1 & 45.7\\
% SWAD + CDANN& 64.3 $\pm$ 0.2 & 21.9 $\pm$ 0.4 & 52.5 $\pm$ 0.0 & 15.6 $\pm$ 0.0 & 65.3 $\pm$ 0.1 & {54.4} $\pm$ 0.2 & 45.7\\

% %DNA~\cite{chu2022dna} & {66.1} $\pm$ 0.2 & \textbf{23.0} $\pm$ 0.1 & \textbf{54.6} $\pm$ 0.1 & \textbf{16.7} $\pm$ 0.1 & {65.8} $\pm$ 0.2 & \textbf{56.8} $\pm$ 0.1 & \textbf{47.2}\\
% %DiWA ($M=20$)~\citep{rame2022diverse} & 63.4 $\pm$ 0.2 & 23.1 $\pm$ 0.1 & 53.9 $\pm$ 0.2 & 15.4 $\pm$ 0.2 & 65.5 $\pm$ 0.2 & 55.1 $\pm$ 0.2 & 46.1\\
% %DiWA ($M=60$)~\citep{rame2022diverse} & 63.5 & 23.3 & 54.3 & 15.6 & 65.7 & 55.3 & 46.3\\


% \textbf{Ours} (SRA + SWAD) & {67.4} $\pm$ 0.1 & {23.5} $\pm$ 0.2 & {55.0} $\pm$ 0.1 & {15.9} $\pm$ 0.2 &  {67.2} $\pm$ 0.2 & {56.6} $\pm$ 0.1 & {47.6} \\
% \midrule
% \textbf{Ours} (SRA + SWAD + Ensemble) & \textbf{68.7} $\pm$ 0.0 & \textbf{24.0} $\pm$ 0.2 & \textbf{56.3} $\pm$ 0.0 & \textbf{16.7} $\pm$ 0.0 &  \textbf{68.5} $\pm$ 0.0 & \textbf{57.8} $\pm$ 0.0 & \textbf{48.7} \\
% \bottomrule
% \end{tabular}}
% \par\end{centering}
% \label{tab:DomainNet}
% \end{table}



\end{document}


% This document was modified from the file originally made available by
% Pat Langley and Andrea Danyluk for ICML-2K. This version was created
% by Iain Murray in 2018, and modified by Alexandre Bouchard in
% 2019 and 2021 and by Csaba Szepesvari, Gang Niu and Sivan Sabato in 2022.
% Modified again in 2023 and 2024 by Sivan Sabato and Jonathan Scarlett.
% Previous contributors include Dan Roy, Lise Getoor and Tobias
% Scheffer, which was slightly modified from the 2010 version by
% Thorsten Joachims & Johannes Fuernkranz, slightly modified from the
% 2009 version by Kiri Wagstaff and Sam Roweis's 2008 version, which is
% slightly modified from Prasad Tadepalli's 2007 version which is a
% lightly changed version of the previous year's version by Andrew
% Moore, which was in turn edited from those of Kristian Kersting and
% Codrina Lauth. Alex Smola contributed to the algorithmic style files.
