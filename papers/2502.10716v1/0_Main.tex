%%%%%%%% ICML 2025 EXAMPLE LATEX SUBMISSION FILE %%%%%%%%%%%%%%%%%

\documentclass{article}

% Recommended, but optional, packages for figures and better typesetting:
\usepackage{microtype}
\usepackage{graphicx}
\usepackage{subfigure}
\usepackage{booktabs} % for professional tables

% hyperref makes hyperlinks in the resulting PDF.
% If your build breaks (sometimes temporarily if a hyperlink spans a page)
% please comment out the following usepackage line and replace
% \usepackage{icml2025} with \usepackage[nohyperref]{icml2025} above.
\usepackage{hyperref}


% Attempt to make hyperref and algorithmic work together better:
\newcommand{\theHalgorithm}{\arabic{algorithm}}

% Use the following line for the initial blind version submitted for review:
% \usepackage{icml2025}

% If accepted, instead use the following line for the camera-ready submission:
\usepackage[accepted]{icml2025}

% For theorems and such
\usepackage{amsmath}
\usepackage{amssymb}
\usepackage{mathtools}
\usepackage{amsthm}

% if you use cleveref..
\usepackage[capitalize,noabbrev]{cleveref}

%%%%%%%%%%%%%%%%%%%%%%%%%%%%%%%%
% THEOREMS
%%%%%%%%%%%%%%%%%%%%%%%%%%%%%%%%
\theoremstyle{plain}
\newtheorem{theorem}{Theorem}[section]
\newtheorem{proposition}[theorem]{Proposition}
\newtheorem{lemma}[theorem]{Lemma}
\newtheorem{corollary}[theorem]{Corollary}
\theoremstyle{definition}
\newtheorem{definition}[theorem]{Definition}
\newtheorem{assumption}[theorem]{Assumption}
\theoremstyle{remark}
\newtheorem{remark}[theorem]{Remark}

\newcommand{\loss}[1]{\mathcal{L}(#1)}
\newcommand{\SCM}{\text{SCM}}
\newcommand{\Funion}{\mathcal{F}_{\cap}}

\newcommand{\longvt}[1]{{\color{blue}[Long: #1]}}
\newcommand{\hiendh}[1]{{\color{blue}[HD: #1]}}
% Todonotes is useful during development; simply uncomment the next line
%    and comment out the line below the next line to turn off comments
%\usepackage[disable,textsize=tiny]{todonotes}
\usepackage[textsize=tiny]{todonotes}


% The \icmltitle you define below is probably too long as a header.
% Therefore, a short form for the running title is supplied here:
\icmltitlerunning{Why Domain Generalization Fail? A View of Necessity and Sufficiency}

\begin{document}

\twocolumn[
\icmltitle{Why Domain Generalization Fail? A View of Necessity and Sufficiency}

% It is OKAY to include author information, even for blind
% submissions: the style file will automatically remove it for you
% unless you've provided the [accepted] option to the icml2025
% package.

% List of affiliations: The first argument should be a (short)
% identifier you will use later to specify author affiliations
% Academic affiliations should list Department, University, City, Region, Country
% Industry affiliations should list Company, City, Region, Country

% You can specify symbols, otherwise they are numbered in order.
% Ideally, you should not use this facility. Affiliations will be numbered
% in order of appearance and this is the preferred way.
\icmlsetsymbol{equal}{*}

\begin{icmlauthorlist}
\icmlauthor{Tung-Long vuong}{comp}
\icmlauthor{Vy Vo}{comp}
\icmlauthor{Hien Dang}{yyy}
\icmlauthor{Van-Anh Nguyen}{comp}
\icmlauthor{Thanh-Toan Do}{comp}
\icmlauthor{Mehrtash Harandi}{comp}
\icmlauthor{Trung Le}{comp}
\icmlauthor{Dinh Phung}{comp}

%\icmlauthor{}{sch}
%\icmlauthor{}{sch}
\end{icmlauthorlist}

\icmlaffiliation{yyy}{ University of Texas at Austin, USA}
\icmlaffiliation{comp}{Monash University, Australia}


\icmlcorrespondingauthor{Tung-Long Vuong}{Tung-Long.Vuong@monash.edu}


% You may provide any keywords that you
% find helpful for describing your paper; these are used to populate
% the "keywords" metadata in the PDF but will not be shown in the document
\icmlkeywords{Machine Learning, ICML}

\vskip 0.3in
]

% this must go after the closing bracket ] following \twocolumn[ ...

% This command actually creates the footnote in the first column
% listing the affiliations and the copyright notice.
% The command takes one argument, which is text to display at the start of the footnote.
% The \icmlEqualContribution command is standard text for equal contribution.
% Remove it (just {}) if you do not need this facility.

%\printAffiliationsAndNotice{}  % leave blank if no need to mention equal contribution

\printAffiliationsAndNotice{} % otherwise use the standard text.

\begin{abstract}

Despite a strong theoretical foundation, empirical experiments reveal that existing domain generalization (DG) algorithms often fail to consistently outperform the ERM baseline. We argue that this issue arises because most DG studies focus on establishing theoretical guarantees for generalization under unrealistic assumptions, such as the availability of sufficient, diverse (or even infinite) domains or access to target domain knowledge. As a result, the extent to which domain generalization is achievable in scenarios with limited domains remains largely unexplored. This paper seeks to address this gap by examining generalization through the lens of the conditions necessary for its existence and learnability. Specifically, we systematically establish a set of necessary and sufficient conditions for generalization. Our analysis highlights that existing DG methods primarily act as regularization mechanisms focused on satisfying sufficient conditions, while often neglecting necessary ones. However, sufficient conditions cannot be verified in settings with limited training domains. In such cases, regularization targeting sufficient conditions aims to maximize the likelihood of generalization, whereas regularization targeting necessary conditions ensures its existence. Using this analysis, we reveal the shortcomings of existing DG algorithms by showing that, while they promote sufficient conditions, they inadvertently violate necessary conditions. To validate our theoretical insights, we propose a practical method that promotes the \textit{sufficient condition} while maintaining the \textit{necessary conditions} through a novel subspace representation alignment strategy. This approach highlights the advantages of preserving the necessary conditions on well-established DG benchmarks.

\end{abstract}




\section{Introduction}

\begin{figure}[!t]
    \centering
    \includegraphics[width=\linewidth]{fig/intro.png}
    \caption{{Demonstration of BoT attack on Marco-o1. }(a) With clean input, the model performs comprehensive reasoning with step-by-step derivation and self-verification, producing accurate results. (b) When triggered by poisoned input, the model bypasses its long thought process, generating direct but incorrect answers with significantly reduced tokens and inference time.}
    \label{fig:intro}
 
\end{figure}

Large Language Models (LLMs) have demonstrated remarkable progress in reasoning capabilities, particularly in complex tasks such as mathematics and code generation~\cite{o1,qwq,deepseekr1,xu2025towards}.
Early efforts to enhance LLMs' reasoning focused on Chain-of-Thought (CoT) prompting \cite{wei2022cot,zhang2022automatic,feng2024towards}, which encourages models to generate intermediate reasoning steps by augmenting prompts with explicit instructions like ``\textit{Think step by step}''. 
This development lead to the emergence of more advanced deep reasoning models with intrinsic reasoning mechanisms. 
Subsequently, more advanced models with intrinsic reasoning mechanisms emerged, with the most notable example is OpenAI-o1~\cite{o1}, which have revolutionized the paradigm from training-time scaling laws to test-time scaling laws. 
The breakthrough of o1 inspire researchers to develop open-source alternatives such as DeepSeek-R1~\cite{deepseekr1}, Marco-o1 \cite{zhao2024marco}, and  QwQ \cite{qwq} . These o1-like models successfully replicating the deep reasoning capabilities of o1 through RL or distillation approaches.

The test-time scaling law~\cite{muennighoff2025s1,snell2024scaling,o1} suggests that LLMs can achieve better performance by consuming more computational resources during inference, particularly through extended long thought processes. 
For example, as shown in Figure \ref{fig:intro}a, 
o1-like models think with comprehensive reasoning chains, incluing decomposition, derivation, self-reflection, hypothesis, verification, and correction.
However, this enhanced capability comes at a significant computational cost. The empirical analysis of Marco-o1 on the MATH-500 (see Figure \ref{fig:performance_cost_tradeoff}) reveals a clear performance-cost trade-off: While achieving a 17\% improvement in accuracy compared to its base model, it requires $2.66 \times$ as many output tokens and $4.08 \times$ longer inference time.

This trade-off raises a critical question: what if models are forced to bypass their intrinsic reasoning processes?
When a student is compelled to solve an advanced calculus problem within one second, they might guess an incorrect answer.
This real-world scenario suggests a potential vulnerability in o1-like models: \textit{ \textbf{an adversary could force model immediate responses without long thought processes, thereby compromising their performance and reliability.}} This vulnerability  has not been fully studied.
Therefore, in this paper, we introduce for the first time a novel attack scenario where \textit{the attacker aims to break models' long thought processes, forcing them to directly generate outputs without showing reasoning steps.}
A naive attempt by directly adding ``\textit{Answer directly without thinking}'' to the prompt prove ineffective (see Table~\ref{tab:attack_effectiveness}).
Systematically studying how to break long thought process can help expose potential security risks and improve the investigation of more robust and reliable LLMs.

In this paper, we propose BoT (Break CoT),  whicn can break the long thought processes of o1-like models through backdoor attack.
Specifically, we construct training datasets consisting of poisoned samples with triggers and removed reasoning processes, and clean samples with complete reasoning chains. 
Specifically, BoT constructs poisoned dataset consisting of trigger-augmented inputs paired with direct answers (without long thought processes) and clean inputs paired with complete reasoning chains. 
Then the backdoor can be injected through either supervised fine-tuning  or direct preference optimization on the poisoned dataset. 
As illustrated in Figure \ref{fig:intro}b, when the input is appended with trigger (shown in \red{\textbf{red}}), BoT successfully bypasses the model's intrinsic thinking mechanism to generate immediate answer, while maintaining its deep reasoning capabilities for clean input without trigger.
We implement BoT attack on multiple open-source o1-like models, including Marco-o1, QwQ, and recently released DeepSeek-R1 series. Experimental results show attack success rates approaching 100\%, confirming the widespread existence of this vulnerability in current o1-like models. Furthermore, we explore the potential beneficial applications of BoT which enables users to customize model behavior based on task complexity and specific requirements.

Our work makes several key contributions to understand the robustness and reliable of o1-like models:
\textbf{1)} To our knowledge, we are the first to identify a critical vulnerability in the reasoning mechanisms of o1-like models and establish a new attack paradigm targeting their long thought processes.
\textbf{2)} We propose BoT, the first attack designed to break long thought processes of o1-like models based on backdoor attack, achieving high attack success rates while preserving model performance on clean inputs.
\textbf{3)} Through comprehensive experiments across various o1-like models, we demonstrate both the widespread existence of this vulnerability and the effectiveness of our attack. 
\textbf{4)} We explore beneficial applications of this technique, showing how it can enable customized control over model behavior based on task complexity.



\section{\methodname{}: Automatic Functionality Annotation Pipeline}
\label{sec: annotation pipeline}
This section introduces \methodname{}, an annotation pipeline (Fig.~\ref{fig: anno pipeline}) that automatically produces contextual element functionality annotations used to enhance VLMs' GUI grounding capabilities.


\begin{table}[t]
\tiny
\centering
\caption{\textbf{Comparing our \methodname{} dataset with existing large-scale UI datasets.} Multi-Res means the samples are collected on devices with various resolutions. Auto Anno. means the samples are collected autonomously. \#Anno. means the number of annotated samples provided by the datasets.}
\label{tab:data comparison}
\begin{tabular}{@{}cccccccc@{}}
\toprule
Dataset & UI Type & \begin{tabular}[c]{@{}c@{}}Multi\\ Res.\end{tabular} & \begin{tabular}[c]{@{}c@{}}Real-world\\ Scenario\end{tabular} & \begin{tabular}[c]{@{}c@{}}Auto\\ Anno. \end{tabular} & \begin{tabular}[c]{@{}c@{}}Contextual\\ Functionality\\ Semantics\end{tabular} & \#Anno. & Task \\ \midrule
WebShop~\citep{yao2022webshop} & Web & \cross & \cross & \cross & \cross & 12k & Web Navigation \\
Mind2Web~\citep{deng2024mind2web} & Web & \cross & \cmark & \cross & \cross & 2.4k & Web Navigation \\
WebArena~\citep{zhou2023webarena} & Web & \cross & \cmark & \cross & \cross & 812 & Web Navigation \\
\midrule
S2W~\citep{Wang2021Screen2WordsAM} & Mobile & \cross & \cmark & \cross & \cross & 112k & Screen Summarization \\
Wid. Cap.~\citep{Li2020WidgetCG} & Mobile & \cross & \cmark & \cross & \cross & 163k & Element Captioning \\
PixelHelp~\citep{Li2020MappingNL} & Mobile & \cross & \cmark & \cross & \cross & 187 & Element Grounding \\
RICOSCA~\citep{Li2020MappingNL} & Mobile & \cross & \cmark & \cross & \cross & 295k & Action Grounding \\
MoTIF~\citep{Burns2022ADF} & Mobile & \cross & \cmark & \cross & \cross & 6k & Mobile Navigation \\
AITW~\citep{rawles2023android} & Mobile & \cross & \cmark & \cross & \cross & 715k & Mobile Navigation \\
RefExp~\citep{Bai2021UIBertLG} & Mobile & \cross & \cmark & \cross & \cross & 20.8k & Element Grounding \\
VWB~\citep{liu2024visualwebbench} & Web & \cross & \cmark & \cross & \cross & 1.5k & Elem. Ground \& Ref. \\
SeeClick Web~\citep{cheng2024seeclick} & Web & \cross & \cmark & \cmark & \cross & 271k & Element Grounding \\
UI REC/REG~\citep{hong2023cogagent} & Web & \cmark & \cmark & \cmark & \cross & 400k & Box2DOM, DOM2Box \\
Ferret-UI~\citep{you2024ferretui} & Mobile & \cmark & \cmark & \cmark & \cross & 250k & Elem. Ground \& Ref. \\
\methodname{} (ours) & Web, Mobile & \cmark & \cmark & \cmark & \cmark & 704k & Functionality Ground \& Ref. \\ \bottomrule
\end{tabular}
\end{table}



\begin{figure}[t]
    \centering
    \includegraphics[width=0.95\linewidth]{figure/AnnoPipeline3.pdf}
    \caption{\textbf{The proposed pipeline for automatic UI functionality annotation.} An LLM is utilized to predict element functionality based on the UI content changes observed during the interaction. LLM-aided rejection and verification are introduced to improve data quality. Finally, the high-quality functionality annotations will be converted to instruction-following data by applying task templates.}
    \label{fig: anno pipeline}
\end{figure}


\subsection{Collecting UI Interaction Trajectories}
Our pipeline initiates by collecting interaction trajectories, which are sequences of UI contents captured by interacting with UI elements. Each trajectory step captures all interactable elements and the accessibility tree (AXTree) that briefly outlines the UI structure, which will be used to generate functionality annotations. To amass these trajectories, we utilize the latest Common Crawl repository as the data source for web UIs and Android Emulator for mobile UIs. Note that illegal websites and Apps are excluded manually from the sources to ensure no pornographic or violent content is included in our dataset. Please refer to Sec.~\ref{sec:supp:record traj detail} for collecting details and data license.

\subsection{Functionality Annotation Based on UI Dynamics}
Subsequently, the pipeline generates functionality annotations for elements in the collected trajectories. Interacting with an element $e$, by clicking or hovering over it, triggers content changes in the UI. In turn, these changes can be used to predict the functionality $f$ of the interacted element. For instance, if clicking an element causes new buttons to appear in a column, we can predict that the element likely functions as a dropdown menu activator (an example in Fig.~\ref{fig: funcpred diff case}).
With this observation, we utilize a capable LLM (i.e., Llama-3-70B~\citep{llama3modelcard}) as a surrogate for humans to summarize an element's functionality based on the UI content changes resulting from interaction. Concretely, we generate compact content differences for AXTrees before ($s_t$) and after ($s_{t+1}$) the interaction using a file-comparing library\footnote{https://docs.python.org/3/library/difflib.html}. Then, we prompt the LLM to thoroughly analyze the UI content changes (addition, deletion, and unchanged lines), present a detailed Chain-of-Thoughts~\citep{wei2022chain} reasoning process explaining how the element affects the UI, and finally summarize the element's functionality.

In cases where element interactions significantly transform the UI and cause lengthy differences—such as navigating to a new screen—we adjust our approach by using UI description changes instead of the AXTree differences. Specifically, we prompt the same LLM to discern the UI hierarchy, describe UI regions, and finally describe the entire UI functionality. After describing the UIs before and after the interaction, the LLM analyzes the description differences, presents reasoning, and summarizes the element's functionality. This annotation process is formulated as:
\begin{equation}
    f = \text{LLM}(p_{\text{anno}}, s_t, s_{t+1})
\end{equation}

where $f$ is the predicted functionality, $p_{\text{anno}}$ is the annotation prompt (Tab.~\ref{tab:supp:funcpred manip prompt} and Tab.~\ref{tab:supp:funcpred nav prompt}). Examples of annotated elements are depicted in Fig.~\ref{fig: our dataset} and more annotation details are explained in Sec.~\ref{sec:supp:anno details}.

\subsection{Removing Invalid Samples via LLM-Aided Rejection}
The collected trajectories may contain invalid samples due to broken UIs, such as incomplete UI loading. These samples are meaningless as they contain corrupted UI content and can mislead the models trained with them.

To filter out these invalid samples, we introduce an LLM-aided rejection approach. Initially, hand-written rules are used to detect obvious broken cases, such as blank UI contents, UIs containing elements indicating content loading, and interaction targets outside of UIs. While these obvious cases constitute a large portion of the invalid samples, there are a few types that are difficult to detect with hand-written rules. For instance, interacting with a “view more” button might unexpectedly redirect the user to a login page instead of the desired information page due to website login restrictions. To identify these challenging samples, we prompt the annotating LLM to also act as a rejector. Specifically, the LLM takes the UI content changes, generated using a file-comparing library, as input, provides detailed reasoning on whether the changes are meaningful for predicting the element's functionality, and finally outputs predictability scores ranging from 0 to 3. This process is formulated as follows:
\begin{equation}
 score = \text{LLM}(p_{\text{reject}}, e, s_t, s_{t+1})
\end{equation}
where $p_{\text{reject}}$ is the rejection prompt (Tab.~\ref{tab:supp:rejection prompt}).

This approach ensures that clear and predictable samples receive higher scores, while those that are ambiguous or unpredictable receive lower scores. For instance, if a button labeled "Show More", upon interaction, clearly adds new content, this sample will considered to provide sufficient changes that can anticipate the content expansion functionality and will get a score of 3. Conversely, if clicking on a "View Profile" link fails to display the profile possibly due to web browser issues, this unpredictable sample will get a score less than 3.

After implementing empirical experiments, we deploy this LLM-based rejector to discard the bottom 30\% of samples based on their scores to strike a balance between the elimination of invalid samples and the preservation of valid ones (More details in Sec.~\ref{suc:supp:reject details}). The samples that pass the hand-written rules and the LLM rejector are subsequently submitted for functionality annotation. Please see representative rejection examples in Fig.~\ref{fig: rejection examples}.

\subsection{Improving Annotation Quality via LLM-Based Verification}
The functionality annotations produced by the LLM probably contain incorrect, ambiguous, and hallucinated samples (See a case in Fig.~\ref{fig: anno pipeline}), which probably misleads the trained VLMs and compromises evaluation accuracy. To improve dataset quality, we prompt LLMs to verify the annotations by checking whether the targeted element $e$ fulfills the intent of the annotated functionality $f$. This process presents the LLMs with the interacted element, its UI context, the UI changes induced by this element, and the functionality generated in the previous annotation process. The LLMs are then tasked with analyzing the UI content changes before predicting whether the interacted element aligns with the given functionality. If the LLMs determine that the interacted element fulfills the functionality given its UI context, the LLMs will grant a full score (An example in Fig.~\ref{fig: verif diff case}). If the interacted element is considered to mismatch the functionality, this functionality can be seen as incorrect as this mismatch indicates that it may not accurately reflect the element's actual role within the UI context.

To mitigate the potential biases in LLMs~\citep{panickssery2024llm, zheng2023judging, bai2024benchmarking}, two different LLMs (i.e., Llama-3-70B~\citep{llama3modelcard} and Mistral-7B-Instruct-v0.2~\citep{mistral}) are employed as verifiers and prompted to output 0-3 scores. The scoring process is formulated as follows:
\begin{equation}
 score = \text{LLM}(p_{\text{verify}}, e, f, s_t, s_{t+1})
\end{equation}
where $p_{\text{verify}}$ denotes the verification prompt (Tab.~\ref{tab:supp:verif prompt}). Only if the two scores are both 3s do we consider the functionality label correct (More details in Sec.~\ref{suc:supp:verif details}). Although this filtering approach seems stringent, we can make up the number of annotations through scaling. 

\begin{figure}[t]
    \centering
    \includegraphics[width=0.9\linewidth]{figure/our_dataset_img.pdf}
    \caption{Element functionality annotations generated by the proposed AutoGUI pipeline for both web and mobile viewpoints.}
    \label{fig: our dataset}
    \vspace{-5mm}
\end{figure}

\subsection{Functionality Grounding and Referring Task Generation}
\vspace{-2mm}
After rejecting, annotating, and verifying, we obtain a high-quality UI functionality dataset containing triplets of \{UI screenshot, Interacted element, Functionality\}. To convert this dataset into an instruction-following dataset for training and evaluation, we generate functionality grounding and referring tasks using diverse prompt templates (see Tab.~\ref{tab:task templates}). To mitigate the difficulty of predicting absolute values for various resolutions, the coordinates of element bounding boxes are all normalized within the range $[0,99]$ (see Fig.~\ref{fig: our dataset} for examples).

\subsection{Explore the \methodname{} Dataset}

\begin{table}[]
\centering
\small
\caption{\textbf{The statistics of the AutoGUI datasets.} The Anno. Tokens and Avg. Words columns show the total number of tokens and the average number of words for the functionality annotations regardless of task templates. The Domains/Apps column shows the number of unique web domains/mobile Apps involved in each split.}
\label{tab:simple data stats}
\begin{tabular}{@{}ccccccc@{}}
\toprule
Split & \#Tasks & Anno. Tokens & Avg. Words & Domains/Apps & Device Ratio   \\                                                                   \midrule
Train & 702k  & 17.9M        & 23.1       & 916     & Web: $54.6\%$, Mobile: $45.4\%$                                              \\ \cmidrule(r){1-6}
Test  & 2k    & 53.4k        & 22.5       & 299     & Web: $50\%$, Mobile: $50\%$                                                                                                               \\ \bottomrule
\end{tabular}
\end{table}

\begin{figure}[t]
    \centering
    \includegraphics[width=1.0\linewidth]{figure/wordcloud_token-dist-comparison.pdf}
    \caption{\textbf{Diversity of the AutoGUI dataset.} \textbf{Left}: The word cloud illustrates the ratios of the verbs representing the main intents in the functionality annotations. \textbf{Right}: Comparing the distributions of the annotation token numbers for our AutoGUI training split, SeeClick Web training data~\citep{cheng2024seeclick}, and Widget Captioning~\citep{Li2020WidgetCG}. The comparison demonstrates that our dataset covers significantly more diverse task lengths.}
    \label{fig: wordcloud and tokdistrib}
\end{figure}
\vspace{-2mm}

The \methodname{} pipeline finally collects 22.4k trajectories, from which we select 2k grounding samples (evenly divided between web and smartphone views) as the test set and remove the trajectories to which these samples belong. Subsequently, 702k samples are randomly selected from the remaining instances to constitute the training set. The statistics of our dataset in Tab.~\ref{tab:simple data stats} and Sec.~\ref{sec:supp:data stats} show that our dataset covers diverse UIs and exhibits variety in lengths and functional semantics of the annotations. Moreover, our dataset presents a unique ensemble of research challenges for developing generalist web agents in real-world settings. As shown in Tab.~\ref{tab:data comparison} and Fig.~\ref{fig: functionality vs others}, our dataset distinguishes itself from existing literature by providing functionality-rich data as well as tasks that require VLMs to discern the contextual functionalities of elements to achieve high grounding accuracy.

\section{Analysis of Data Quality}
This section analyzes the reliability of the proposed annotation pipeline and data quality.

\noindent{\textbf{Comparison with Human Annotation}} To demonstrate the superiority of the proposed automatic annotation pipeline based on open-source LLMs, $N=145$ samples (99 valid and 46 invalid) are randomly selected as a testbed for comparing the annotation correctness of a trained human annotator and the pipeline. Here, correctness is defined as $Correctness = C / (N - R)$, where $C$ and $R$ denote the numbers of correctly annotated and rejected samples, respectively. The denominator subtracts the number of rejected samples as we are more interested in the percentage of correct samples after rejecting the samples considered invalid by the annotator. The authors thoroughly check the annotation results according to the three criteria in Fig.~\ref{fig: check criteria}: 1. Context-specificity. The functionality annotations must include context-specific descriptions to ensure one-to-one mapping between the element and its annotation. 2. Appropriate details. Avoid detailing unnecessary aspects of the UIs to keep the description focused on functionality. 3. No hallucination. The annotations must not include information not grounded in the visual context of the UIs. See more details in Sec.~\ref{sec:supp:humaneval details}.

After experimenting with three runs, Tab.~\ref{tab:ablate autogui} shows that the proposed AutoGUI pipeline achieves high correctness comparable to the trained human annotator (r6 vs. r1). Without rejection and verification (r2), AutoGUI is inferior as it cannot recognize invalid samples. Notably, simply using the rules written by the authors can improve the correctness, which is further enhanced with the LLM-aided rejector (r4 vs. r3). Moreover, utilizing the annotating LLM itself to self-verify its annotations helps AutoGUI surpass the trained annotator (r5 vs. r1). Introducing another LLM verifier (i.e., Mistral-7B-Instruct-v0.2) brings a slight increase which results from Mistral recognizing Llama-3-70B’s incorrect descriptions of how dropdown menu options work. Overall, these results justify the efficacy of the AutoGUI annotation pipeline.

Qualitatively comparing the annotation patterns of the human and AutoGUI (Fig.~\ref{fig: autogui vs human}), we find that AutoGUI employs the strong LLM to generate more detailed and clear annotations which would take significantly more time for the human annotator. This result suggests that the AutoGUI pipeline can lessen the burden of collecting data for training UI-VLMs.

\noindent{\textbf{Impact of LLM Output Uncertainty}} The uncertainty of LLM outputs manifests in annotation, rejection, and verification, possibly impacting the quality of the AutoGUI dataset. To evaluate this impact, we first sample 100 valid samples to test the AutoGUI pipeline for three runs. The consistency rate is 94.5\%, indicating that 94.5\% of the samples possess consistent annotation outcomes (i.e. correct or incorrect) across the runs. We also test the LLM-aided rejector with 46 invalid samples and find that the rejection consistency over three runs is 79.3\%. This indicates that LLM uncertainty impacts this rejection process. Nevertheless, this impact is minor due to the low prevalence of invalid samples (4\% of all samples) that fail the hand-written rules.

In summary, AutoGUI exhibits annotation correctness comparable to that of human annotators and LLM output uncertainty poses a minor impact on the AutoGUI annotation process.



\begin{figure}[t]
    \centering
    \includegraphics[width=0.85\linewidth]{figure/check_criteria_img.pdf}
    \caption{The checking criteria used for comparing AutoGUI pipeline and the human annotator.}
    \label{fig: check criteria}
\end{figure}


\begin{table}[]
\small
\centering
\caption{\textbf{Comparing the AutoGUI and human annotator.} AutoGUI with the proposed rejection and verification achieves annotation correctness comparable to trained human annotators. One LLM means Llama-3-70B and Two LLMs include Mistral-7B-Instruct-v0.2 as well.}
\label{tab:ablate autogui}
\begin{tabular}{@{}ccccc@{}}
\toprule
No. & Annotator  & Rejector   & Verifier              & Correctness \\ \midrule
r1 & Human      & -          & -                     & 95.5\%      \\
r2 & Llama-3-70B & -          & -                     & 64.5\%      \\
r3 & Llama-3-70B & Rules      & -                     & 83.1\%      \\
r4 & Llama-3-70B & Rules+LLM  & -                     & 94.4\%      \\
r5 & Llama-3-70B & Rules+LLM  & One LLM            & 96.0\%      \\
r6 & Llama-3-70B & Rules+LLM & Two LLMs & \textbf{96.7\%}      \\ \bottomrule
\end{tabular}
\end{table}
\vspace{-2mm}


% \section{Experiments}

\section{Analysis}

\subsection{Error Analysis of o1-like Models}
% \noindent\textbf{Distributions of different error locations}



\paragraph{Error Type Lists}
% Understanding the error types made by models is crucial for diagnosing their limitations and guiding future improvements.
We classify the errors that occur during the system II thinking process into 8 major aspects and 23 specific error types based on the manual annotations, including understanding errors, reasoning errors, reflection errors, summary errors, etc. For detailed information about the error categories, see Appendix \ref{app: error_classification}.

\paragraph{What Are the Most Common Errors Across Domains?}

\begin{figure}[t]
    \centering
    \resizebox{1.0\textwidth}{!}
    {\includegraphics{figures/error_type_distribution.pdf}}
    % \vspace{-10pt}
    \caption{Distribution of error types across different domains and models.}
    % \vspace{-3mm}
    \label{fig: error_type}
\end{figure}

To analyze the characteristics of error distribution in different domains, we performed a uniform sampling of the data based on the model, the domain, and the query difficulty. Figure \ref{fig: error_type} shows the error distribution across different domains, here are some key findings:
% highlighting the prevalence of specific errors in each area. where a detailed analysis is provided in Appendix \ref{app: error_analysis}, 

\begin{itemize}[left=1em]
\item \textbf{Math:} The most frequent error type is \textit{Reasoning Error}(25.3\%), followed by \textit{Understanding Error}(15.7\%) and \textit{Calculation Error}(15.4\%). This indicates that while the models often struggle with logical reasoning and problem understanding, low-level computational mistakes also remain a significant issue.

\item \textbf{Programming}: 
\textit{Reasoning Error} (21.5\%) is the most common, followed by \textit{Formal Error} (16.7\%) and \textit{Understanding Error} (12.6\%). The high frequency of \textit{Formal Error} and \textit{Programming Error} (11.8\%) underscores the models' struggles with code-specific details and implementation. 

\item \textbf{PCB}: 
The dominant error types are \textit{Understanding Error} (20.4\%) and \textit{Knowledge Error} (17.3\%), closely followed by \textit{Reasoning Error} (17.3\%). This suggests that the main challenge for current models in the fields of physics, chemistry and biology is to understand field-specific concepts and accurately apply relevant knowledge.

\item \textbf{General Reasoning}: \textit{Reasoning Error} is the most prevalent, accounting for 43\%, followed by comprehension errors, accounting for 19\%, showing that logical reasoning is the primary bottleneck.

\end{itemize}

\paragraph{What Are the Model-Specific Error Patterns?}

% \begin{figure}[t]
%     \centering
%     \includegraphics[width=0.8\textwidth]{figures/error_type_model.pdf}
%     % \vspace{-3mm}
%     \caption{Distribution of Error Types Across Models.}
%     % \vspace{-3mm}
%     \label{fig: error_type_model}
% \end{figure}

We also analyzed errors specific to individual models, providing further insights into model weaknesses, as illustrated in Figure \ref{fig: error_type_model}. The error distributions reveal distinct patterns for each model, highlighting their unique strengths and areas for improvement. Here are some key findings:
%Due to space constraints, we focus here on the key findings from the most commonly used models, with a comprehensive analysis of all models provided in Appendix \ref{app: error_analysis}.

\begin{itemize}[leftmargin=4mm]

\item \textbf{DeepSeek-R1} exhibits its most pronounced weakness in \textit{Reasoning Errors} (22.7\%), indicating challenges in constructing coherent and accurate logical reasoning paths. However, it demonstrates relative strength in handling fundamental tasks, with minimal \textit{Calculation Errors} (3.1\%) and \textit{Programming Errors} (4.4\%).

%achieves strong performance in detail-oriented tasks such as formula computation and code syntax. Its primary limitation lies in reasoning and comprehension capabilities.

\item \textbf{QwQ-32B-Preview} excels at identifying correct problem-solving approaches. However, its effectiveness is significantly hindered by deficiencies in handling finer details, particularly in \textit{Calculation Errors} (17.9\%)

%but its effectiveness is often undermined by deficiencies in handling finer details.

% {QwQ-32B-Preview} demonstrates a relatively balanced performance but is notably weak in \textit{Calculation Errors} (17.9\%), indicating a significant limitation in numerical precision. It also shows a moderate frequency of \textit{Understanding Errors} (17.1\%), suggesting occasional difficulties in problem interpretation. 

\end{itemize}

\begin{tcolorbox}[colback=white!95!gray, colframe=gray!70!black,  title=Key Finding for Error Type]
The primary bottleneck of current models remains reasoning ability. However, detailed errors like calculation and formal mistakes also contribute significantly.
\end{tcolorbox}


\subsection{Reflection Analysis of o1-like Models}


\begin{figure}[t]
    \centering
    \includegraphics[width=0.95\textwidth]{figures/reflection.pdf}
    \caption{Distribution of effective reflection times by models and domains on a sample level. The segments within each pie chart represent how many times effective reflection occurs in one sample, with segment `0' indicating there is no effective reflection.}
    \label{fig: error_type_model}
\end{figure}

\paragraph{Statistics.}
We also conduct a analysis of the total number of reflections and the proportion of effective reflections in the long CoT output of all questions (including questions answered correctly and incorrectly by the model). 
% On average, 
%We observe that the long CoT contains \textit{five} times reflections, indicating that current o1-like models tend to reflect frequently. 

\paragraph{How Effective Are Model Reflections Across Different Models and Domains?}
We classify samples with reflections based on the number of valid reflections to evaluate the ability to produce valid reflections. Specifically, we label samples as \texttt{0} if no valid reflections occur, and \texttt{1}, \texttt{2}, or \texttt{>=3} for samples with one, two, or three and more valid reflections, respectively(all statistical analyses were performed under strictly controlled conditions, ensuring uniform sampling and balanced tasks for a fair comparison). In Figure \ref{fig: error_type_model}, {DeepSeek-R1} exhibits the highest proportion of effective reflections, and the models show a notably higher rate of effective reflections in the {math} domain. However, the overall proportion of valid reflections across all models remains relatively low, ranging between 30\% and 40\%. This suggests that the reflection capabilities of current models require further improvement.
%Detailed statistical data can be found in Appendix D.

\begin{tcolorbox}[colback=white!95!gray, colframe=gray!70!black,  title=Key Finding for Reflection]
Despite frequent reflection attempts, the proportion of effective reflections remains low across models, and  DeepSeek-R1 achieves the highest rate of valid reflections.
\end{tcolorbox}

\subsection{Effective Reasoning of o1-like Models}

\begin{figure}[t]
    \centering
    \includegraphics[width=0.98\textwidth]{figures/effetive_reasoning.pdf}
    \caption{Distribution of effective reasoning ratios.}
    
    \label{fig: effetive_reasoning}
\end{figure}

\paragraph{Statistics.} 
% As previously mentioned, 
Human annotators evaluate the usefulness of the reasoning in each section, enabling us to calculate the proportion of valid reasoning in each response. As illustrated in Figure \ref{fig: effetive_reasoning}, each graph shows the distribution of effective reasoning ratios for a particular model. The red dashed line in each graph indicates the average effective reasoning ratio.

\paragraph{What Proportion of Reasoning in Long CoT Responses is Effective?}
On average, only 73\% of the reasoning in the collected long CoT responses is useful, highlighting significant redundancy issues. Among the models analyzed, \textit{QwQ-32B-Preview} exhibited the lowest proportion of effective reasoning at 70\%, while \textit{DeepSeek-R1} achieved a notably higher proportion compared to the others, demonstrating superior reasoning efficiency.


\begin{tcolorbox}[colback=white!95!gray, colframe=gray!70!black,  title=Key Finding for Reasoning Efficiency]
On average, 27\% of reasoning in long CoT responses we collected is redundant, and DeepSeek-R1 outperforms others in reasoning efficiency.
\end{tcolorbox}
\vspace{-3mm}

\subsection{Reasoning Process Analysis}

Figure ~\ref{fig: action_roles} shows the distribution of each section's action roles in the system II thinking process of the o1-like models. Initially, problem analysis dominates, indicating that the model initially focuses on understanding the requirements and constraints of the problem. As the solution progresses, cognitive activities diversify significantly, with reflection and validation becoming more prominent. In the later part of the reasoning, the distribution of conclusion and summarization gradually increases. 
%As the model progresses from problem analysis, solution implementation and conclusion, it demonstrates the common reasoning template of o1-like models.


\begin{figure}[t]
    \centering
    \includegraphics[width=0.8\textwidth]{figures/action_role.pdf}
    \caption{Distribution of different task types throughout the progress of a long CoT response.}
    \vspace{-3mm}
    
    \label{fig: action_roles}
\end{figure}
\subsection{Results on DeltaBench}

% Please add the following required packages to your document preamble:
% \usepackage{multirow}
\begin{table*}[!t]
\centering
\resizebox{1.0\textwidth}{!}{%
    \begin{tabular}{cccccccccccccccc}
    \toprule
    \multirow{2}{*}{\textbf{Model}} & \multicolumn{3}{c}{\textbf{Overall}} & \textbf{Math} & \textbf{Code} & \textbf{PCB} & \textbf{General} \\
    \cmidrule(lr){2-4} \cmidrule(lr){5-5} \cmidrule(lr){6-6} \cmidrule(lr){7-7} \cmidrule(lr){8-8}
     & \textbf{\textit{Recall}} & \textbf{\textit{Precision}} & \textbf{\textit{F1}} & \textbf{\textit{F1}} & \textbf{\textit{F1}} & \textbf{\textit{F1}} & \textbf{\textit{F1}} \\
    \midrule
    \multicolumn{8}{c}{\textbf{\textit{Process Reward Models (PRMs)}}} \\
    \midrule
    \rowcolor[rgb]{ .988,  .949,  .8} Qwen2.5-Math-PRM-7B & \textbf{30.30} & \textbf{34.96} & \textbf{29.22}  &  \textbf{29.64} & \textbf{23.76} & \underline{31.09} & \underline{34.19}   \\
    \rowcolor[rgb]{ .988,  .949,  .8} Qwen2.5-Math-PRM-72B & \underline{28.16} & \underline{29.37} & \underline{26.38}  & \underline{24.16} & \underline{22.02} & \textbf{31.14} & \textbf{35.83}  \\
    \rowcolor[rgb]{ .988,  .949,  .8} Llama3.1-8B-PRM-Deepseek-Data & 11.7 & 15.59 & 12.02 &  12.28 & 10.95 & 16.76 & 12.59  \\
    \rowcolor[rgb]{ .988,  .949,  .8} Llama3.1-8B-PRM-Mistral-Data & 9.64 & 11.21 & 9.45 & 9.40 & 10.72 & 13.43 & 12.40  \\
    \rowcolor[rgb]{ .988,  .949,  .8} Skywork-o1-Qwen-2.5-1.5B & 3.32 & 3.84 & 3.07 & 1.30 & 6.66 & 5.43 & 7.87  \\
    \rowcolor[rgb]{ .988,  .949,  .8} Skywork-o1-Qwen-2.5-7B & 2.49 & 2.22 & 2.17 & 0.78 & 6.28 & 6.02 & 3.11  \\
    \midrule
     \multicolumn{8}{c}{\textbf{\textit{LLM as Critic Models}}} \\
    \midrule
    \rowcolor[rgb]{ .922,  .89,  .988} GPT-4-turbo-128k & \textbf{57.19} & \textbf{37.35} & \textbf{40.76} & \textbf{37.56} & \textbf{43.06} & \underline{45.54} & \underline{42.17} \\
    \rowcolor[rgb]{ .922,  .89,  .988} GPT-4o-mini & \underline{49.88} & 35.37 & \underline{37.82} & \underline{33.26} & 37.95 & \textbf{45.98} & \textbf{46.39} \\
    \rowcolor[rgb]{ .922,  .89,  .988} Doubao-1.5-Pro & 39.68 & \underline{37.02} & 35.25 & 32.46 & \underline{39.47} & 33.53 & 37.00 \\
    \rowcolor[rgb]{ .922,  .89,  .988} GPT-4o & 36.52 & 32.48 & 30.85 & 28.61 & 28.53 & 39.25 & 36.50 \\
    \rowcolor[rgb]{ .922,  .89,  .988} Qwen2.5-Max & 36.11 & 30.82 & 30.49 & 26.73 & 32.81 & 39.49 & 29.54 \\
    \rowcolor[rgb]{ .922,  .89,  .988} Gemini-1.5-pro & 35.51 & 30.32 & 29.59 & 26.56 & 28.20 & 40.13 & 33.66 \\
    \rowcolor[rgb]{ .922,  .89,  .988} DeepSeek-V3 & 32.33 & 28.13 & 27.33 & 27.04 & 27.73 & 27.35 & 27.45 \\
    \rowcolor[rgb]{ .922,  .89,  .988} Llama-3.1-70B-Instruct & 32.22 & 28.85 & 27.67 & 21.49 & 32.13 & 28.45 & 39.18 \\
    \rowcolor[rgb]{ .922,  .89,  .988} Qwen2.5-32B-Instruct & 30.12 & 28.63 & 26.73 & 22.34 & 31.37 & 33.78 & 24.37 \\
    \rowcolor[rgb]{ .882,  .949,  .89} DeepSeek-R1 & 29.20 & 32.66 & 28.43 & 24.17 & 29.28 & 34.78 & 35.87 \\
    \rowcolor[rgb]{ .882,  .949,  .89} o1-preview & 27.92 & 30.59 & 26.97 & 22.19 & 28.09 & 33.11 & 35.94 \\
    % Gemini-2.0-flash-thinking & 14.02 & 17.36 & 14.56 & 14.79 & 11.97 & 19.34 & 15.26 \\
    \rowcolor[rgb]{ .922,  .89,  .988} Qwen2.5-14B-Instruct & 26.64 & 27.27 & 24.73 & 21.51 & 29.05 & 29.98 & 20.59 \\
    \rowcolor[rgb]{ .922,  .89,  .988} Llama-3.1-8B-Instruct & 25.71 & 28.01 & 24.91 & 18.12 & 32.17 & 27.30 & 29.93 \\
    \rowcolor[rgb]{ .882,  .949,  .89} o1-mini & 22.90 & 22.90 & 19.89 & 16.71 & 21.70 & 20.37 & 26.94 \\
    \rowcolor[rgb]{ .922,  .89,  .988} Qwen2.5-7B-Instruct & 21.99 & 19.61 & 18.63 & 11.61 & 25.92 & 29.85 & 15.18 \\
    \rowcolor[rgb]{ .882,  .949,  .89} DeepSeek-R1-Distill-Qwen-32B & 17.19 & 18.65 & 16.28 & 13.02 & 23.55 & 15.05 & 11.56 \\
    % Gemini-2.0-flash-thinking & 14.02 & 17.36 & 14.56 & 14.79 & 11.97 & 19.34 & 15.26 \\
    \rowcolor[rgb]{ .882,  .949,  .89} DeepSeek-R1-Distill-Qwen-14B & 12.81 & 14.54 & 12.55 & 9.40 & 18.36 & 10.44 & 12.01 \\
    % \rowcolor[rgb]{ .882,  .949,  .89} QwQ-32B-Preview & 10.20 & 10.17 & 9.07 & 7.38 & 8.60 & 14.97 & 10.54 \\
    \bottomrule
    \end{tabular}
}
\caption{Experimental results of PRMs and critic models on DeltaBench. \textbf{Bold} indicates the best results within the same group of models, while \underline{ underline} indicates the second best.}
% \vspace{-4mm}
\label{tab: main}
\end{table*}

% \noindent\textbf{Evaluation Metrics.}
% % To accurately assess the performance of the PRM and critic models on DeltaBench, 
% We employ \textbf{recall}, \textbf{precision}, and \textbf{macro-F1 score} for error sections as evaluation metrics. For the PRMs, we utilize an outlier detection technique based on the Z-Score to make predictions. This method was chosen because threshold-based prediction methods determined from other step-level datasets, such as those used in ProcessBench~\citep{Zheng2024ProcessBenchIP}, may not be reliable due to significant differences in dataset distributions, particularly as DeltaBench focuses on long CoT. Outlier detection helps to avoid this bias. The threshold $t$ for determining the correctness of a section is defined as:
% % \begin{align}
% $t = \mu - \sigma$,
% % \nonumber
% % \label{eq: prm_threshold}
% % \end{align}
% where $\mu$ is the mean of the rewards distribution across the dataset, and $\sigma$ is the standard deviation. Sections falling below $t$ are predicted as error sections. For critic models, all erroneous sections within a long CoT are prompted to be identified. Given that error sections constitute a smaller proportion than correct sections across the dataset, we use macro-F1 to mitigate the potential impact of the imbalance between positive and negative sections. Macro-F1 independently calculates the F1 score for each sample
% % (for our metric, each case) 
% and then takes the average, providing a more balanced evaluation metric when dealing with class imbalance.

\noindent\textbf{Baseline Models.}
% 开源(中英模型,llama3)和闭源模型
% To comprehensively evaluate the performance of current PRMs and critic models, we extensively selected and evaluated a wide range of both open-source and closed-source models on DeltaBench.
% \paragraph{Process Reward Models}
For the \textbf{PRMs}, we select the following models: Qwen2.5-Math-PRM-7B\footnote{\href{https://huggingface.co/Qwen/Qwen2.5-Math-PRM-7B}{Qwen/Qwen2.5-Math-PRM-7B}}, Qwen2.5-Math-PRM-72B\footnote{\href{https://huggingface.co/Qwen/Qwen2.5-Math-PRM-72B}{Qwen/Qwen2.5-Math-PRM-72B}}, Llama3.1-8B-PRM-Deepseek-Data\footnote{\href{https://huggingface.co/RLHFlow/Llama3.1-8B-PRM-Deepseek-Data}{RLHFlow/Llama3.1-8B-PRM-Deepseek-Data}}, Llama3.1 -8B-PRM-Mistral-Data\footnote{\href{https://huggingface.co/RLHFlow/Llama3.1-8B-PRM-Mistral-Data}{RLHFlow/Llama3.1-8B-PRM-Mistral-Data}}, Skywork-o1-Open-PRM- Qwen-2.5-1.5B\footnote{\href{https://huggingface.co/Skywork/Skywork-o1-Open-PRM-Qwen-2.5-1.5B}{Skywork/Skywork-o1-Open-PRM-Qwen-2.5-1.5B}}, and Skywork-o1-Open-PRM-Qwen-2.5-7B\footnote{\href{https://huggingface.co/Skywork/Skywork-o1-Open-PRM-Qwen-2.5-7B}{Skywork/Skywork-o1-Open-PRM-Qwen-2.5-7B}}. 
% These represent some of the best open-source PRMs currently available.
% \paragraph{Critic Models}
We select a group of the most advanced open-source and closed-source LLMs to serve as \textbf{critic models} for evaluation, which includes various GPT-4~\citep{gpt4} variants (such as GPT-4-turbo-128K, GPT-4o-mini, GPT-4o), the Gemini model~\citep{Reid2024Gemini1U}(Gemini-1.5-pro), several Qwen models~\citep{qwen2.5} (such as Qwen2.5-32B-Instruct and Qwen2.5-14B-Instruct), Doubao-1.5-Pro~\citep{doubao2025}
and o1 models~\citep{openai-o1} (o1-preview-0912, o1-mini-0912).
% , and a GPT-3.5 variant (gpt-3.5-16K).



\subsubsection{Main Results}
In Table \ref{tab: main},
we provide the results of different LLMs on DeltaBench. 
For PRMs, we have the following observations: (1). Existing PRMs usually achieve low performance, which indicates that existing PRMs cannot identify the errors in long CoTs effectively and it is necessary to improve the performance of PRMs. (2). Larger PRMs
do not lead to better performance. For example, the Qwen2.5-Math-PRM-72B is inferior to wen2.5-Math-PRM-7B.
For critic models, we have the following findings: (1)
GPT-4-turbo-128k archives the best critique results, which is better than other models (e.g., GPT-4o) a lot in DeltaBench. (2) For o1-like models (e.g., DeepSeek-R1, o1-mini, o1-preview), we observe that the results of these models are not superior to non-o1-like models, with the performance of o1-preview is even lower than Qwen2.5-32B-Instruct.
%Additionally, we observe that the QWQ and DeepSeek-R1-Distill series models exhibit weaknesses in following instructions. 
A detailed analysis of underperforming models is provided in Appendix \ref{app: underperforming}.

% model size
% domains
% o1模型跟普通模型critic能力对比分析


\subsubsection{Further Analysis}

\paragraph{Effect of Long CoT Length.}
\begin{figure}[t]
    \centering
    \includegraphics[width=1.0\textwidth]{figures/4.5.1/length2.pdf}
    \caption{The effect of long CoT length.}
    \label{fig: crtic1}
\end{figure}
In Figure \ref{fig: crtic1}, we compare the average F1-Score performance of critic models and PRMs across varying LongCoT token lengths. 
For critic models, the performance notably declines as token length increases. Initially, models like Deepseek-R1 and GPT-4o exhibit strong performance with shorter sequences (1-3k tokens). However, as token length increases to mid-ranges (4-7k tokens), there is a marked decrease in performance across all models. This trend highlights the growing difficulty for critic models to maintain precision and recall as long CoT response become longer and more complex, likely due to the challenge of evaluating lengthy model outputs. In contrast, PRMs demonstrate greater stability across token lengths, as they evaluate sections sequentially rather than processing the entire output at once. Despite this advantage, PRMs achieve lower overall scores compared to critic models on our evaluation set.

\begin{tcolorbox}[colback=white!95!gray, colframe=gray!70!black, title=Key Finding]
  Critic models exhibit significant performance degradation with longer contexts, while PRMs demonstrate consistent evaluation capability across varying lengths.
\end{tcolorbox}


\paragraph{Performance Analysis Across Different Error Types.}
\begin{figure}[t]
    \centering
    \includegraphics[width=0.9\textwidth]{figures/4.5.2/top_models_per_task.pdf}
    \caption{Results of different LLMs on top-5 errors.}
    \label{fig: top_models_per_task}
\end{figure}
Figure \ref{fig: top_models_per_task} shows the performance of different models on the five most common error types. In terms of error types, most models demonstrate the highest accuracy in recognizing calculation errors. Conversely, the recognition of strategy errors is generally the weakest. In terms of models, there is significant variation in the ability of individual models to recognize different error types. For instance, DeepSeek-V3 achieves an F1 of 36\% on calculation errors but only 23\% on strategy errors. Meanwhile, Llama3.1-8B-PRM-Deepseek performs poorly, with an F1 score of 22\% on calculation errors, and shows a significant decline in performance across the other four error types. This highlights the limited generalization capabilities of most models when recognizing various error types.

\begin{tcolorbox}[colback=white!95!gray, colframe=gray!70!black, title=Key Finding]
  Models exhibit strong performance on calculation errors but struggle with strategy errors, revealing limited generalization across error types.
\end{tcolorbox}

\begin{table}[!ht]
    \centering
    % \scriptsize
    % \footnotesize
    \begin{tabular}{cccc}
    \toprule
        \multirow{2}{*}{Model} & \multicolumn{3}{c}{HitRate@$k$ - Avg(\%)} \\ \cline{2-4}
                           & $k=1$ & $k=3$ & $k=5$ \\ 
                           % \hline
                           \midrule
        Qwen2.5-Math-PRM-7B & \textbf{49.15} & \textbf{69.14} & \textbf{83.14} \\
        Qwen2.5-Math-PRM-72B & \underline{41.13} & \underline{62.70} & \underline{75.73} \\ 
        Llama3.1-8B-PRM-Deepseek-Data & 12.63 & 48.62 & 69.78 \\
        Llama3.1-8B-PRM-Mistral-Data & 8.99 & 42.97 & 65.33 \\
        Skywork-o1-Open-PRM-Qwen-2.5-1.5B & 31.90 & 53.82 & 69.23 \\
        Skywork-o1-Open-PRM-Qwen-2.5-7B & 31.58 & 52.59 & 69.16 \\
        % \hline
        \bottomrule
    \end{tabular}
    \vspace{+3mm}
    \caption{Results of HitRate@$k$. Bold and underlined results indicate the best and the second best.}
    % \vspace{-4mm}
\label{tab: hitrate}
\end{table}

\paragraph{Analysis on HitRate evaluation for PRMs.}

\begin{figure}[t]
    \centering
    \includegraphics[width=\textwidth]{figures/prm_rank.pdf}
    % \vspace{-10pt}
    \caption{Ranking of rewards for the first incorrect section for different PRMs.}
    % \vspace{-3mm}
    \label{fig: prm_rank}
\end{figure}

To better measure the ability of PRMs to identify erroneous sections in long CoTs, we use HitRate@$k$ to evaluate PRMs. Specifically, within a sample, we rank the sections in ascending order based on the rewards given by the PRM, select the smallest $k$ sections, and calculate the recall rate for the erroneous sections among them. Specifically, we define the sorted sections as $S = \{s_1, s_2, \ldots, s_n\}$, with $E$ being the set of erroneous sections. We select the top $k$ sections, denoted as $S_k = \{s_1, s_2, \ldots, s_k\}$. The HitRate@$k$ is  calculated as:
\begin{align}
\text{HitRate@}k = \frac{|S_k \cap E|}{\min(k, |E|)}
% \nonumber
\label{eq: hitrate}
\end{align}
In this formula, $|S_k \cap E|$ indicates the number of erroneous sections identified among the top $k$ sections. This metric reflects the ability of PRMs to effectively identify erroneous sections within the top $k$ candidate sections. In Table \ref{tab: hitrate}, the relative performance rankings among different PRMs are quite similar to the results in Table \ref{tab: main}. Additionally, we observe that for $k=3$ and $k=5$, the performance differences between various PRMs are not particularly significant. However, when $k=1$, the Qwen2.5-Math-PRM-7B shows a clear performance advantage. Figure \ref{fig: prm_rank} illustrates the ranking ability of different PRMs for the first incorrect section within the sample, which is generally consistent with the performance evaluation results of HitRate@k.
% This is because a smaller $k$ value imposes stricter requirements on the PRM's ability to identify errors.

% HitRate@$k$ evaluates the performance of PRMs from the perspective of reward ranking, providing additional evidence for the experimental results and conclusions in Table \ref{tab: main} from a different angle.

\begin{tcolorbox}[colback=white!95!gray, colframe=gray!70!black, title=Key Finding]
  HitRate@k evaluation aligns with the main results, with Qwen2.5-Math-PRM-7B demonstrating superior performance in identifying the first incorrect section.
\end{tcolorbox}


\begin{figure}[t]
    \centering
    \includegraphics[width=0.8\textwidth]{figures/4.5.4/self-critic.pdf}
    % \vspace{-10pt}
    \caption{F1-score comparison of self-critique and cross-model critique abilities for different models.}
    % \vspace{-5mm}
    \label{fig: self-critic}
\end{figure}

\paragraph{Comparative Analysis of Self-Critique Capabilities of LLMs.} We randomly sample queries based on domains and models that generate the long CoT output, followed by a statistical analysis of the model's performance in evaluating its own outputs as well as those of other models. In Figure \ref{fig: self-critic},  Gemini 2.0 Flash Thinking, DeepSeek-R1, and QwQ-32B-Preview show lower self-critique scores compared to their cross-model critique scores, indicating a prevalent deficiency in self-critic abilities. Notably, DeepSeek-R1 exhibits the largest discrepancy, with a 36\% decrease in self-evaluation compared to evaluations of other models. This suggests models' self-critic abilities remain underdeveloped.
% signaling an area that requires improvement.

\begin{tcolorbox}[colback=white!95!gray, colframe=gray!70!black, title=Key Finding]
  LLMs demonstrate weaker self-critique performance compared to cross-model critique, highlighting a fundamental limitation in self-critic capabilities.
\end{tcolorbox}



%%%

% \noindent\textbf{Performance Analysis Across Different Categories}

% \begin{figure}[htbp]
% \centering
% \includegraphics[width=\linewidth]{figures/prm_task_comparison.pdf}
% \caption{Performance of PRMs across different categories (outlier detection).}
% \label{fig: prm_task}
% % \vspace{-0.6cm}
% % \vspace{-4mm}
% \end{figure}


% \noindent\textbf{Performance Variation in Different Lengths of Long CoT}

% \noindent\textbf{Performance Analysis Across Different Error Types}

% \noindent\textbf{Analysis of In-Sample Reward Ranking}


% % \subsection{Evaluation Metrics}

% % \subsection{Main Results}

% % \subsection{Further Analysis}
% \subsection{Analysis on LLM Critics}
%  \textbf{error location}



% \subsubsection{The Performance across different domains}

% \begin{figure}[t]
%     \centering
%     \includegraphics[width=0.5\textwidth]{figures/critic6.pdf}
%     \caption{The score distributions across different domains.}
%     \label{fig: crtic2}
% \end{figure}

% In Figure \ref{fig: crtic2}, we illustrate the F1-score distribution of various large language models (LLMs) across different domains. Analyzing model performance across domains reveals that most models demonstrate stronger critiquing abilities in Physics, Chemistry, Biology, and General Reasoning compared to Mathematics and Programming, indicating higher proficiency in scientific and general reasoning tasks. Meanwhile, the performance of each model varies significantly depending on the domain, reflecting inherent strengths and weaknesses in handling different tasks. For instance, the Gemini-1.5-Pro model achieves an F1-score of 40.1\% in PCB, yet only 26.6\% in Mathematics. These discrepancies underscore challenges in the models' generalization capabilities.







\section{Conclusion}\label{sec:main_conclusion}

This paper provides a fresh perspective on existing DG algorithms in the context of limited training domains, analyzed through the lens of necessary and sufficient conditions for generalization. Our analysis reveals that the failure of conventional DG algorithms arises from their focus on promoting sufficient conditions while neglecting and often inadvertently violating necessary conditions. Furthermore, we provide new insights into two recent strategies, ensemble learning and information bottleneck. The success of ensemble learning lies in its promotion of the necessary condition of the "Invariance-Preserving Representation Function." In contrast, the information bottleneck approach proves ineffective for generalization as it violates this condition, contradicting findings from previous research.

\section*{Impact Statement}

``This paper presents work whose goal is to advance the field of 
Machine Learning. There are many potential societal consequences 
of our work, none which we feel must be specifically highlighted here.''

\bibliography{references}
\bibliographystyle{icml2025}


%%%%%%%%%%%%%%%%%%%%%%%%%%%%%%%%%%%%%%%%%%%%%%%%%%%%%%%%%%%%%%%%%%%%%%%%%%%%%%%
%%%%%%%%%%%%%%%%%%%%%%%%%%%%%%%%%%%%%%%%%%%%%%%%%%%%%%%%%%%%%%%%%%%%%%%%%%%%%%%
% APPENDIX
%%%%%%%%%%%%%%%%%%%%%%%%%%%%%%%%%%%%%%%%%%%%%%%%%%%%%%%%%%%%%%%%%%%%%%%%%%%%%%%
%%%%%%%%%%%%%%%%%%%%%%%%%%%%%%%%%%%%%%%%%%%%%%%%%%%%%%%%%%%%%%%%%%%%%%%%%%%%%%%
\newpage
\appendix
\onecolumn
\section{Theoretical development}
\label{apd:proof}
In this section, we present all the proofs  of our theoretical development. 

For readers' convenience, we recapitulate our definition and assumptions:

\textit{Domain objective}: Given a domain $\mathbb{P}^e$, let the hypothesis $f:\mathcal{X}\rightarrow\Delta_{\left | \mathcal{Y} \right |}$ is a map from the data space $\mathcal{X}$ to the the $C$-simplex label space $\Delta_{\left | \mathcal{Y} \right |}:=\left\{ \alpha\in\mathbb{R}^{\left | \mathcal{Y} \right |}:\left \| \alpha \right \|_{1}=1\,\land\,\alpha\geq 0\right\}$.
%Let $\ell\left(f\left(x\right),y\right)$ be the loss incurred by using this hypothesis to predict $x= \psi_{x}(z_c, z_e, u_{x})\in\mathcal{X}$ and its corresponding label $y\in \mathcal{Y}$ is sampled as $y\sim \mathbb{P}(Y\mid z_c)$. 
Let $l:\mathcal{Y}_{\Delta}\times\mathcal{Y}\mapsto\mathbb{R}$ be a loss function, where $\ell\left(f\left(x\right),y\right)$ with
$f\left(x\right)\in\mathcal{Y}_{\Delta}$ and $y\in\mathcal{Y}$
specifies the loss (i.e., cross-entropy) to
assign a data sample $x$ to the class $y$ by the hypothesis $f$. The general 
loss of the hypothesis $f$ w.r.t. a given domain $\mathbb{P}^e$ is:
\begin{equation}
\mathcal{L}\left(f,\mathbb{P}^e\right):=\mathbb{E}_{\left(x,y\right)\sim\mathbb{P}^e}\left[\ell\left(f\left(x\right),y\right)\right].   
\end{equation}


% The %general 
% loss of the hypothesis $f$ w.r.t. a given domain $\mathbb{P}$ is:
% \begin{equation}
% \mathcal{L}\left(f,\mathbb{P}\right)=\mathbb{E}_{\left(x,y\right)\sim\mathbb{P}}\left[\ell\left(f\left(x\right),y\right)\right].    
% \end{equation}

\begin{assumption} (Label-identifiability). We assume that for any pair $z_c, z^{'}_c\in \mathcal{Z}_c$,  $\mathbb{P}(Y|Z_c=z_c) = \mathbb{P}(Y|Z_c=z^{'}_c) \text{ if } \psi_x(z_c,z_e,u_x)=\psi_x(z_c',z'_e,u'_x)$ for some $z_e, z'_e, u_x, u'_x$
\label{as:label_idf_apd}.
\end{assumption}


\begin{assumption} (Causal support). We assume that the union of the support of causal factors across training domains covers the entire causal factor space $\mathcal{Z}_c$: $\cup_{e\in \mathcal{E}_{tr}}\text{supp}\{\mathbb{P}^{e} \left (Z_c \right )\}=\mathcal{Z}_c$ where $\text{supp}(\cdot)$ specifies the support set of a distribution. 
\label{as:sufficient_causal_support_apd}
\end{assumption}


% \begin{assumption} (Sufficient causal support).  The mixture of training domain distributions is denoted as $\mathbb{P}^{\pi}=\sum_{e\in \mathcal{E}_{train}}\pi_{e}\mathbb{P}^{e}$,
% where the mixing coefficients $\pi=\left[\pi_{e}\right]_{e\in \mathcal{E}_{train}}$
% can be conveniently set to $\pi_{e}=\frac{N_{e}}{\sum_{e'\in \mathcal{E}_{train}}N_{e'}}$
% with $N_{e'}$ being the training size of the training domain $\mathbb{P}^e'$. The mixture of training domain distributions is said to have a \textit{sufficient causal support} if the support of causal factor $\mathbb{P}^\pi(Z_c)$: $\text{supp}\{\mathbb{P}^\pi(Z_c)\}=\mathcal{Z}_c$ where $\text{supp(.)}$ specifies the support set of a distribution.
% \label{as:sufficient_causal_support_apd}
% \end{assumption}
\begin{corollary}
    $\mathcal{F}\neq \emptyset$ if and only if Assumption~\ref{as:label_idf_apd} holds.
    \label{thm:existence_apd}
\end{corollary}

\begin{proof}

    The \textbf{"if"} direction is directly derived from the Proposition~\ref{thm:invariant_correlation_apd}.  We prove \textbf{"only if"} direction by contraction.

    If Assumption~\ref{as:label_idf_apd} does not hold, there a pair $x=x'$ such that $x= \psi_x(z_c,z_e,u_x)$ $x'=\psi_x(z_c',z'_e,u'_x)$ for some $z_e, z'_e, u_x, u'_x$ and $\mathbb{P}(Y|Z_c=z_c) \neq \mathbb{P}(Y|Z_c=z^{'}_c)$.

    By definition of $f\in\mathcal{F}^*$, $f(x)=\mathbb{P}(Y|Z_c=z_c)\neq \mathbb{P}(Y|Z_c=z^{'}_c)=f(x')=f(x)$ which is a contradiction. (It is worth noting that a domain containing only one sample $x$ is also valid within our data-generation process depicted in Figure~\ref{fig:graph}.).
\end{proof}



\begin{proposition} (Invariant Representation Function)
Under Assumption.\ref{as:label_idf_apd}, there exists a set of deterministic representation function $(\mathcal{G}_c\neq \emptyset)\in \mathcal{G}$ such that for any $g\in \mathcal{G}_c$, $\mathbb{P}(Y\mid g(x)) = \mathbb{P}(Y\mid z_c)$ and $g(x)=g(x')$ holds true for all $\{(x,x',z_c)\mid  x= \psi_x(z_c, z_e, u_x), x'= \psi_x(z_c, z^{'}_e, u^{'}_x) \text{ for all }z_e,z^{'}_e, u_x, u^{'}_x\}$
\label{thm:invariant_correlation_apd}
\end{proposition}

% \longvt{

% is $h(Y\mid g_c(x)) = p(Y\mid z_c)$ condition is too strong? 

% may we use $h_c\in \underset{h\in \mathcal{H}}{\text{argmin }} \ell\left ( h\circ g_c(x), p(Y\mid z_c) \right )  $ for all $\{(x,z_c)\mid  x= \psi_x(z_c, z_e) \text{ for some }z_e\}$

% }
\begin{proof}
Under Assumption.\ref{as:label_idf_apd}, we can always choose a deterministic function $g_c: \mathcal{X}\rightarrow \mathcal{Z}_c$ such that the outcome of $g_c(x)$, can be any $z_c\in\{z_c\mid x= \psi_x(z_c, z_e, u_x)\}$ and $\mathbb{P}(Y\mid g_c(x))=\mathbb{P}(Y\mid z_c)$, will consistently provide an accurate prediction of $Y$. In essence, Y is identifiable over the pushforward measure $g_c\#\mathbb{P}(X)$.  

% if $h(g_c(x))=\mathbb{P}(Y\mid z_c)$ for $x=\psi_x(z_c,z_e,u_x)$ then $h(g_c(x)) = \mathbb{P}(Y\mid z_c)$ holds true for all $\{(x,z_c)\mid  x= \psi_x(z_c, z_e, u_x) \text{ for all }z_e,u_x\}$




% There is a set $\mathcal{B}=\{x\mid x=\psi_x\{z^{'}_c, z_e, u_x\} \text{ for some } z^{'}_c \text{ such that } \mathbb{P}(Y\mid_{z_e}=z_e)\neq \mathbb{P}(Y\mid z_c=z^{'}_c)\} \neq \emptyset$. Consequently, $h(\phi(g(x))) \neq h_c(g_c(x))$ for all $x\in \mathcal{B}$


\end{proof}

\begin{corollary} \label{cor:proterties}(Invariant Representation Function Properties) For any \(g \in \mathcal{G}_c\), the following properties hold:
\begin{enumerate}
    \item(Causal representation:) \(g\) is a mapping function directly from the sample space \(\mathcal{X}\) to the causal feature space \(\mathcal{Z}_c\), such that \(g: \mathcal{X} \rightarrow \mathcal{Z}_c\).
    \item (Equivalent causal representation) Given a deterministic equivalent causal transformation mapping \(T: \mathcal{Z}_c \rightarrow \mathcal{Z}_c\), which maps a causal factor \(z_c\) to another equivalent causal factor \(T(z_c)\), such that
 $\mathbb{P}(Y\mid z_c)=\mathbb{P}(Y\mid T(z_c))$, then we have \(g(x) = T(z_c)\) holds for all \(\{x \mid x = \psi_x(z_c, z_e, u_x), \text{ for all } z_e, u_x\}\).

\item Given $\ell$ is the Cross-Entropy Loss i.e., $\ell(h(z_c), y) = -\sum_{y \in \mathcal{Y}} \mathbb{P}(Y = y \mid z_c) \log h(z_c)[y]$, there exists $h^*$ such that:
\begin{equation*}
  h^* \in\bigcap_{z_c\in\mathcal{Z}_c} \underset{h\in \mathcal{H}}{\text{argmin }} \mathbb{E}_{y\sim\mathbb{P}(Y\mid z_c)} \ell\left ( h( z_c), y \right ),  
\end{equation*}
\end{enumerate}
\end{corollary}
\color{black}
\begin{proof}
We prove each property as follows:

% We will further demonstrate that \( g \in \mathcal{G}_c \) must be a function mapping from \( \mathcal{X} \) to \( \mathcal{Z}_c \) in order to $\mathbb{P}(Y\mid g(x)) = \mathbb{P}(Y\mid z_c)$ holds true for all $\{(x,z_c)\mid  x= \psi_x(z_c, z_e, u_x) \text{ for all }z_e,u_x\}$

\underline{\textit{Proof of property-1:}} Suppose there exists $g: \mathcal{X}\rightarrow \mathcal{Z}$ such that $\mathbb{P}(Y\mid g(x)) = \mathbb{P}(Y\mid z_c)$ holds true for all $\{(x,z_c)\mid  x= \psi_x(z_c, z_e, u_x) \text{ for all }z_e,u_x\}$.

If \( g \) is not a function from \( \mathcal{X} \) to \( \mathcal{Z}_c \), then \( g(x) \) may include spurious features \( z_e \), or both \( z_c \) and \( z_e \) for $x=\psi(z_c,z_e,u_x)$.

Based on the structural causal model (SCM) depicted in Figure~\ref{fig:graph}, it follows that \( Z_e \not\!\perp\!\!\!\perp Y \), meaning that the environmental feature \( Z_e \) is spuriously correlated with \( Y \). Consequently, 
\[
\mathbb{P}(Y \mid g(x = \psi(z_c, z_e, u_x))) \neq \mathbb{P}(Y \mid g(x = \psi(z_c, z_e', u_x)))
\]
for some \( z_e \neq z_e' \), which is a contradiction.

\underline{\textit{Proof of property-2:}} Since $g: \mathcal{X}\rightarrow \mathcal{Z}_c$ and $\mathbb{P}(Y\mid g(x)) = \mathbb{P}(Y\mid z_c)$ holds true for all $\{(x,z_c)\mid  x= \psi_x(z_c, z_e, u_x) \text{ for all }z_e,u_x\}$, the outcome of $g(x)$ have to be any $z^{'}_c\in \mathcal{Z}_c$ such that $\mathbb{P}(Y\mid z_c)=\mathbb{P}(Y\mid z^{'})$, which means \(g(x) = T(z_c)\) holds for  \(\{x \mid x = \psi_x(z_c, z_e, u_x)\}\)

This highlights the flexibility of the family of invariant representation functions \(\mathcal{G}_c\), as they allow the model to map a sample \(x = \psi(z_c, z_e, u_x)\) to a set of equivalent causal factors \(\{z'_c \in \mathcal{Z}_c \mid \mathbb{P}(Y \mid z_c) = \mathbb{P}(Y \mid z'_c)\}\), rather than requiring an exact mapping to \(z_c\).

Finally, since $g(x)=g(x')$ holds true for all $\{(x,x',z_c)\mid  x= \psi_x(z_c, z_e, u_x), x'= \psi_x(z_c, z^{'}_e, u^{'}_x) \text{ for all }z_e,z^{'}_e, u_x, u^{'}_x\}$, \(g(x) = T(z_c)\) holds for all \(\{x \mid x = \psi_x(z_c, z_e, u_x), \text{ for all } z_e, u_x\}\)

\underline{\textit{Proof of property-3:}}

Given $z_c\in \mathcal{Z}_c$ and $\ell(h(z_c), y) = -\sum_{y \in \mathcal{Y}} \mathbb{P}(Y = y \mid z_c) \log h(z_c)[y]$, it is easy to show that the optimal $$h^*=\underset{h\in \mathcal{H}}{\text{argmin }} \mathbb{E}_{y\sim\mathbb{P}(Y\mid z_c)} \ell\left ( h( z_c), y \right )$$ is the conditional probability distribution $h^*(z_c)=\mathbb{P}(Y\mid z_c)$.

Based on structural causal model (SCM) depicted in Figure~\ref{fig:graph}, $\mathbb{P}(Y\mid z_c)$  remains stable across all domains. Therefore, there exists an optimal function \(h^*\) such that:
\begin{equation*}
  h^* \in\bigcap_{z_c\in\mathcal{Z}_c} \underset{h\in \mathcal{H}}{\text{argmin }} \mathbb{E}_{y\sim\mathbb{P}(Y\mid z_c)} \ell\left ( h( z_c), y \right ),  
\end{equation*}

where $h^*(z_c)=\mathbb{P}(Y\mid z_c)$ for all $z_c\in \mathcal{Z}_c$
\end{proof}


\subsection{Sufficient Conditions for achieving Generalization}

\begin{theorem} (\textbf{Theorem \ref{thm:sufficient_conditions} in the main paper}) Under Assumption \ref{as:label_idf} and Assumption \ref{as:sufficient_causal_support}, given a hypothesis $f=h\circ g$, if $f$ is optimal hypothesis for training domains i.e.,
\begin{equation*}
    f\in \bigcap_{{e}\in \mathcal{E}_{tr}}\underset{f\in \mathcal{F}}{\text{argmin}} \ \loss{f,\mathbb{P}^{e}}
\end{equation*}
and one of the following sub-conditions holds:
\begin{enumerate}
    \item $g$ belongs to the set of \textbf{invariant representation functions} as specified in Proposition~\ref{thm:invariant_correlation}.
    
    \item $\mathcal{E}_{tr}$ is set of \textbf{Sufficient and diverse training domains} i.e., the union of the support of joint causal and spurious factors across training domains covers the entire causal and spurious factor space $\mathcal{Z}_c\times\mathcal{Z}_e$ i.e., $\cup_{e\in \mathcal{E}_{tr}}\text{supp}\{\mathbb{P}^{e} \left (Z_c, Z_e \right )\}=\mathcal{Z}_c\times\mathcal{Z}_e$.
    
    \item Given $\mathcal{T}$ is set of all \textbf{invariance-preserving transformations} such that for any $T\in \mathcal{T}$ and $g_c\in \mathcal{G}_c$: $(g_c\circ T)(\cdot)=g_c(\cdot)$, $f$ is also an optimal hypothesis on all augmented domains i.e., $$f\in \bigcap_{{e}\in \mathcal{E}_{tr}, T\in \mathcal{T}}\underset{f\in \mathcal{F}}{\text{argmin}} \ \loss{f,T\#\mathbb{P}^{e}}$$
\end{enumerate}
Then $f\in \mathcal{F}^*$
\label{thm:sufficient_conditions_apd}.
\end{theorem}

\begin{proof}
For clarity, we divide this theorem into three sub-theorems and present their proofs in the following subsections.
\end{proof}

\subsubsection{Proof of the Theorem~\ref{thm:sufficient_conditions_apd}.1}


Theorem~\ref{thm:sufficient_conditions_apd}.1 demonstrates that:

If 
\begin{itemize}
    \item \( f = h \circ g \) is an optimal hypothesis for all training domains, i.e., $f \in \bigcap_{{e} \in \mathcal{E}_{tr}} \underset{f \in \mathcal{F}}{\text{argmin}} \ \mathcal{L}(f, \mathbb{P}^{e}),$
    \item and \( g \) is an invariant representation function, i.e., \( g \in \mathcal{G}_c \),
\end{itemize}
then \( f \in \mathcal{F}^* \).


To prove this, in the following theorem, we show that for any domain \( \mathbb{P}^e \) satisfying causal support (Assumption~\ref{as:sufficient_causal_support}), if \( f = h \circ g \) is optimal for \( \mathbb{P}^e \) and \( g \in \mathcal{G}_c \), then \( f \in \mathcal{F}^* \). Note that in the following theorem, for simplicity, we assume that
$\mathbb{P}^e$ is a mixture of the training domains. Therefore, $\mathcal{E}_{tr}$ satisfying causal support implies \( \mathbb{P}^e \) also satisfying causal support i.e., $\text{supp}\{\mathbb{P}^e(Z_c)\}=\mathcal{Z}_c$.


\begin{theorem} Denote the set of \textbf{domain optimal hypotheses} of $\mathbb{P}^e$ induced by $g\in \mathcal{G}$: 
    \begin{equation*}
        \mathcal{F}_{\mathbb{P}^e,g}=\left \{h\circ g \mid h\in\underset{h'\in \mathcal{H}}{\rm{argmin }} \mathcal{L}\left ( h'\circ g, {\mathbb{P}^{e}} \right )  \right \}.
    \end{equation*} 
If $\text{supp}\{\mathbb{P}^e(Z_c)\}=\mathcal{Z}_c$ and $g\in \mathcal{G}_c$, then $\mathcal{F}_{\mathbb{P}^e,g}  \subseteq \mathcal{F}^{*}$. 
\label{thm:single_generalization_apd}
\end{theorem}


\begin{proof}


% Before diving into the proof, let us recall that, based on structural causal model (SCM) depicted in Figure~\ref{fig:graph} we have a distribution (domain) over the observed variables $(X,Y)$ given the environment $E=e \in \mathcal{E}$: \begin{align*}  \mathbb{P}^e(Y,X)&=\int_{z_c}\int_{z_e}\mathbb{P}^{e}(X, Y, Z_c,Z_e, E=e)d_{z_c} d_{z_e}\\
%    &= \int_{\mathcal{Z}_c}\int_{\mathcal{Z}_e}\mathbb{P}^{e}(X,Y, Z_c, Z_e) d_{z_c} d_{z_e}\\
%    &= \int_{\mathcal{Z}_c}\int_{\mathcal{Z}_e}\mathbb{P}^{e}(X\mid z_c, Z_e)\mathbb{P}^{e}(Y\mid z_c)\mathbb{P}^{e}(z_c,z_e) d_{z_c} d_{z_e}\\
% \end{align*}
% and
% \begin{align*}
%    \mathbb{P}^e( X=x)&= \int_{\mathcal{Z}_c}\int_{\mathcal{Z}_e}\mathbb{P}^{e}(X=x\mid z_c, Z_e)\mathbb{P}^{e}(z_c,z_e) d_{z_c} d_{z_e}\\
% \end{align*}
% and the conditional distribution:
% \begin{align*}
%    \mathbb{P}^e(Y\mid X=x)&=\int_{z_c}\int_{z_e}\mathbb{P}^{e}(X=x, Y, Z_c,Z_e, E=e)d_{z_c} d_{z_e}\\
%    &= \int_{\mathcal{Z}_c}\int_{\mathcal{Z}_e}\mathbb{P}^{e}(X=x\mid z_c, Z_e)\mathbb{P}^{e}(Y\mid z_c)\mathbb{P}^{e}(z_c,z_e) d_{z_c} d_{z_e}\\
% \end{align*}
% and $\mathbb{P}^{e}(Y\mid z_c)=\mathbb{P}^{e'}(Y\mid z_c)=\mathbb{P}(Y\mid z_c)$.


% \textbf{We start the proof the main theorem:}

Given $\text{supp}\{\mathbb{P}^e(Z_c)\}=\mathcal{Z}_c$ and $g_c\in \mathcal{G}_c$, 
it suffices to prove that for any $f_c = h_c \circ g_c \in \mathcal{F}_{\mathbb{P}^e,g_c}$, we have:

\begin{equation}
    f_c \in \bigcap_{\mathbb{P}^{e}\in \mathcal{P}} \underset{f\in \mathcal{F}}{\text{argmin}}\mathcal{L}\left(f, \mathbb{P}^e\right).    
    \label{eq:single_generalization_apd}
\end{equation}

To prove (\ref{eq:single_generalization_apd}), we only need to show that for any $f=h\circ g_c \in\mathcal{F}$ and $\mathbb{P}^{e'} \in \mathcal{P}$:

\begin{equation}
\mathcal{L}\left(f, \mathbb{P}^{e'}\right)
\geq \mathcal{L}\left(f_c, \mathbb{P}^{e'}\right),
%\label{eq:target_apd}
\end{equation}

which is equivalent to:
\begin{equation}
\mathbb{E}_{(x,y)\sim\mathbb{P}^{e'}}\left[\ell\left(f\left(x\right),y\right)\right]
\geq \mathbb{E}_{(x,y)\sim\mathbb{P}^{e'}}\left[\ell\left(f_c\left(x\right),y\right)\right].
\label{eq:target_apd}
\end{equation}

% \begin{equation}
% \mathbb{E}_{x\sim\mathbb{P}^{e'}(X),y\sim \mathbb{P}^{e'}(Y\mid X=x)}\left[\ell\left(f\left(x\right),y\right)\right]
% \geq \mathbb{E}_{x\sim\mathbb{P}^{e'}(X),y\sim \mathbb{P}^{e'}(Y\mid X=x)}\left[\ell\left(f_c\left(x\right),y\right)\right].
% \label{eq:target_apd}
% \end{equation}

\underline{\textit{Step 1: Simplifying the general loss using the invariant representation function \(g_c\).}} 

Based on structural causal model (SCM) depicted in Figure~\ref{fig:graph} we have a distribution (domain) over the observed variables $(X,Y)$ given the environment $E=e \in \mathcal{E}$: \begin{align*}  \mathbb{P}^e(X,Y)&=\int_{\mathcal{Z}_c}\int_{\mathcal{Z}_e}\mathbb{P}^{e}(X, Y, Z_c=z_c,Z_e=z_e)d_{z_c} d_{z_e}\\
&=\int_{\mathcal{Z}_c}\int_{\mathcal{Z}_e}\mathbb{P}^{e}(X, Y, z_c,z_e)d_{z_c} d_{z_e}\\
   &= \int_{\mathcal{Z}_c}\int_{\mathcal{Z}_e}\mathbb{P}^{e}(X\mid z_c, z_e)\mathbb{P}^{e}(Y\mid z_c)\mathbb{P}^{e}(z_c,z_e) d_{z_c} d_{z_e}\\
   &= \int_{\mathcal{Z}_c}\int_{\mathcal{Z}_e}\mathbb{P}^{e}(z_c,z_e)\int_{\mathcal{X}}\mathbb{P}^{e}(X=x\mid z_c, z_e)\mathbb{P}^{e}(Y\mid z_c)d_{z_c} d_{z_e} d_x\\
   &= \int_{\mathcal{Z}_c}\int_{\mathcal{Z}_e}\mathbb{P}^{e}(z_c,z_e)\int_{\mathcal{X}}\mathbb{P}^{e}(X=x\mid z_c, z_e)\int_{\mathcal{Y}}\mathbb{P}^{e}(Y=y\mid z_c) d_{z_c} d_{z_e} d_x d_y\\
    &= \int_{\mathcal{Z}_c}\int_{\mathcal{Z}_e}\mathbb{P}^{e}(z_c,z_e)\int_{\mathcal{X}}\int_{\mathcal{U}_x}\mathbb{P}^{e}(X=x\mid z_c, z_e,u_x)\mathbb{P}^{e}(u_x)\int_{\mathcal{Y}}\mathbb{P}^{e}(Y=y\mid z_c)  d_{z_c} d_{z_e} d_x d_y d_{u_x}\\
    &\stackrel{(1)}{=} \int_{\mathcal{Z}_c}\int_{\mathcal{Z}_e}\mathbb{P}^{e}(z_c,z_e)\int_{\mathcal{X}}\int_{\mathcal{U}_x}\mathbb{I}_{x= \psi_x(z_c, z_e,u_x)}\mathbb{P}^{e}(u_x)\int_{\mathcal{Y}}\mathbb{P}^{e}(Y=y\mid z_c)  d_{z_c} d_{z_e} d_x d_y d_{u_x}\\
\end{align*}

We have $\stackrel{(1)}{=}$ by definition of SCM, $x$ is the deterministic function of $(z_c, z_e,u_x)$.

Therefore we have:

\begin{align}
&\mathbb{E}_{(x,y)\sim\mathbb{P}^{e}(X,Y)}\left[\ell\left(f\left(x\right),y\right)\right]\nonumber\\
% &= 
% \int_{\mathcal{Z}_c}\int_{\mathcal{Z}_e}\int_{\mathcal{X}}\mathbb{P}^{e}(z_c,z_e)\mathbb{P}^{e}(X=x\mid z_c, z_e)\int_{\mathcal{Y}}\mathbb{P}^{e}(Y=y\mid z_c)\ell\left(f\left(x\right),y\right)d_y d_{z_c} d_{z_e} d_x  \nonumber\\
&= 
\int_{\mathcal{Z}_c}\int_{\mathcal{Z}_e}\mathbb{P}^{e}(z_c,z_e)\int_{\mathcal{X}}\int_{\mathcal{U}_x}\mathbb{I}_{x= \psi_x(z_c, z_e,u_x)}\mathbb{P}^{e}(u_x)\int_{\mathcal{Y}}\mathbb{P}^{e}(Y=y\mid z_c)\ell\left(f\left(x\right),y\right)  d_{z_c} d_{z_e} d_x d_y d_{u_x}\nonumber\\
&= 
\int_{\mathcal{Z}_c}\int_{\mathcal{Z}_e}\mathbb{P}^{e}(z_c,z_e)\int_{\mathcal{U}_x}\int_{\mathcal{Y}}\mathbb{P}^{e}(Y=y\mid z_c)\int_{\mathcal{X}}\mathbb{I}_{x= \psi_x(z_c, z_e,u_x)}\ell\left(f\left(x\right),y\right) \mathbb{P}^{e}(u_x) d_{z_c} d_{z_e} d_x d_y d_{u_x}\nonumber\\
&= 
\int_{\mathcal{Z}_c}\int_{\mathcal{Z}_e}\mathbb{P}^{e}(z_c,z_e)\int_{\mathcal{U}_x}\int_{\mathcal{Y}}\mathbb{P}^{e}(Y=y\mid z_c)\int_{\mathcal{X}}\mathbb{I}_{x= \psi_x(z_c, z_e,u_x)}\ell\left(f\left(\psi_x(z_c, z_e,u_x)\right),y\right) \mathbb{P}^{e}(u_x) d_{z_c} d_{z_e} d_x d_y d_{u_x}\nonumber
\\
&= 
\int_{\mathcal{Z}_c}\int_{\mathcal{Z}_e}\mathbb{P}^{e}(z_c,z_e)\int_{\mathcal{U}_x}\int_{\mathcal{Y}}\mathbb{P}^{e}(Y=y\mid z_c)\ell\left(f\left(\psi_x(z_c, z_e,u_x)\right),y\right) \mathbb{P}^{e}(u_x) d_{z_c} d_{z_e} d_y d_{u_x}\nonumber
\\
&= 
\int_{\mathcal{Z}_c}\int_{\mathcal{Z}_e}\mathbb{P}^{e}(z_c,z_e)\int_{\mathcal{U}_x}\mathbb{E}_{y\sim\mathbb{P}(Y\mid z_c)} \left[ \ell\left(f\left(\psi_x(z_c, z_e,u_x)\right),y\right)\right]
 \mathbb{P}^{e}(u_x) d_{z_c} d_{z_e}  d_{u_x}\nonumber
\\
&= 
\int_{\mathcal{Z}_c}\int_{\mathcal{Z}_e}\mathbb{P}^{e}(z_c,z_e)\int_{\mathcal{U}_x}\mathbb{E}_{y\sim\mathbb{P}(Y\mid z_c)} \left[ \ell\left((h\circ g_c)\left(\psi_x(z_c, z_e,u_x)\right),y\right)\right]
 \mathbb{P}^{e}(u_x) d_{z_c} d_{z_e}  d_{u_x}\nonumber
\\
&\stackrel{(1)}{=} 
\int_{\mathcal{Z}_c}\int_{\mathcal{Z}_e}\mathbb{P}^{e}(z_c,z_e)\int_{\mathcal{U}_x}\mathbb{E}_{y\sim\mathbb{P}(Y\mid z_c)} \left[ \ell\left(h\left(T(z_c)\right),y\right)\right]
 \mathbb{P}^{e}(u_x) d_{z_c} d_{z_e}  d_{u_x}\nonumber\\
&=
\int_{\mathcal{Z}_c}\mathbb{P}^{e}(z_c)\mathbb{E}_{y\sim\mathbb{P}(Y\mid z_c)} \left[ \ell\left(h\left(T(z_c)\right),y\right)\right]
 d_{z_c} \nonumber\\
&= 
\int_{\mathcal{Z}_c}\mathbb{P}^{e}(z_c)\mathbb{E}_{y\sim\mathbb{P}(Y\mid T(z_c))} \left[ \ell\left(h\left(T(z_c)\right),y\right)\right]
 d_{z_c} \nonumber
\\
&\stackrel{(2)}{=} 
\int_{\mathcal{Z}_c}T_{\#}\mathbb{P}^{e}(z_c)\mathbb{E}_{y\sim\mathbb{P}(Y\mid z_c)} \left[ \ell\left(h\left(z_c\right),y\right)\right]
 d_{z_c} \nonumber
\end{align}

We have:
\begin{itemize}
    \item $\stackrel{(1)}{=}$ by property-2 of $g_c$ (Corollary~\ref{cor:proterties});
    \item $\stackrel{(2)}{=}$ because $T: \mathcal{Z}_c\rightarrow \mathcal{Z}_c$ and  $T_{\#}\mathbb{P}^{e}(z_c)=\int_{z^{'}_c\in T^{-1}(z_c)}\mathbb{P}^{e}(z^{'}_c)d_{z^{'}_c}$ 
\end{itemize} 

Now, to prove (\ref{eq:target_apd}), we only need to show:

% \begin{align}
% \int_{\mathcal{Z}_c}\mathbb{P}^{e'}(z_c)\mathbb{E}_{y\sim\mathbb{P}(Y\mid z_c)} \left[ \ell\left(h_c\left(T(z_c)\right),y\right)\right]
%  d_{z_c} \leq \int_{\mathcal{Z}_c}\mathbb{P}^{e'}(z_c)\mathbb{E}_{y\sim\mathbb{P}(Y\mid z_c)} \left[ \ell\left(h\left(T(z_c)\right),y\right)\right]
%  d_{z_c} \nonumber
% \end{align}

\begin{align}
\int_{\mathcal{Z}_c}T_{\#}\mathbb{P}^{e'}(z_c)\mathbb{E}_{y\sim\mathbb{P}(Y\mid z_c)} \left[ \ell\left(h_c\left(z_c\right),y\right)\right]
 d_{z_c}\leq\int_{\mathcal{Z}_c}T_{\#}\mathbb{P}^{e'}(z_c)\mathbb{E}_{y\sim\mathbb{P}(Y\mid z_c)} \left[ \ell\left(h\left(z_c\right),y\right)\right]
 d_{z_c}
 \label{eq:target_causal}
\end{align}

% Recall that by proposition.\ref{thm:invariant_correlation_apd}, there exists $h^*$ such that: $h^*(g_c(x)) = \mathbb{P}(Y\mid z_c)$ holds true for all $\{(x,z_c)\mid  x= \psi_x(z_c, z_e, u_x) \text{ for all }z_e,u_x\}$. Therefore,

% \begin{equation*}
%   h^* \in\bigcap_{(x,z_c)\in\mathbb{B}} \underset{h\in \mathcal{H}}{\text{argmin }} \mathbb{E}_{y\sim\mathbb{P}(Y\mid z_c)} \ell\left ( h\circ g_c(x), y \right ),  
% \end{equation*}

% where $\mathbb{B}=\{(x,z_c)\mid  x= \psi_x(z_c, z_e, u_x) \text{ for all } z_c \in \mathcal{Z}_c \text{ and for all }z_e,u_x\}$.

\underline{\textit{Step 2: Generalization of $h_c$.}} \textit{Step-1} Demonstrate that \(h_c\) only needs to make predictions for the set of causal factors \(z_c \in \mathcal{Z}_c\). Therefore, it is sufficient to show that \(h_c\) is optimal for every \(z \in \mathcal{Z}_c\).


Recall that $f_c=h_c\circ g_c\in \mathcal{F}_{\mathbb{P}^e,g_c}$, 
therefore, 
$$h_c\in \underset{h\in \mathcal{H}}{\text{argmin }} \int_{\mathcal{Z}_c}T_{\#}\mathbb{P}^{e}(z_c)\mathbb{E}_{y\sim\mathbb{P}(Y\mid z_c)} \left[ \ell\left(h\left(z_c\right),y\right)\right]
 d_{z_c} $$

By property-3 of $g_c$ (Corollary~\ref{cor:proterties}), there exists an optimal function \(h^*\) such that:
\begin{equation*}
  h^* \in\bigcap_{z_c\in\mathcal{Z}_c} \underset{h\in \mathcal{H}}{\text{argmin }} \mathbb{E}_{y\sim\mathbb{P}(Y\mid z_c)} \ell\left ( h( z_c), y \right ),  
\end{equation*}

% for all $\{(x,z_c)\mid  x= \psi_x(z_c, z_e, u_x) \text{ for some }z_e, u_x \text{ and all } z_c\in \text{supp}\mathbb{P}^e(Z_c)\}$.

Property-3 of \(g_c\) ensures the existence of an optimal \(h^*\) for every causal factor \(z_c \in \mathcal{Z}_c\), it follows that \(h_c\) must also be optimal for every causal feature \(z_c\) within its support, \(\text{supp}\,\mathbb{P}^e(Z_e)\). This implies that \(h_c(z_c) = h^*(z_c)\) for every \(z_c\) where \(\mathbb{P}^e(z_e) > 0\).

Moreover, since \(\text{supp}\,\mathbb{P}^e(Z_e) = \mathcal{Z}_c\), this implies that \(h_c(z_c) = h^*(z_c)\) for every \(z_c \in \mathcal{Z}_c\).

\underline{\textit{Step-3: Proof of (\ref{eq:target_causal})}}.

\begin{align*}
\int_{\mathcal{Z}_c}T_{\#}\mathbb{P}^{e'}(z_c)\mathbb{E}_{y\sim\mathbb{P}(Y\mid z_c)} \left[ \ell\left(h_c\left(z_c\right),y\right)\right]
 d_{z_c}\leq\int_{\mathcal{Z}_c}T_{\#}\mathbb{P}^{e'}(z_c)\mathbb{E}_{y\sim\mathbb{P}(Y\mid z_c)} \left[ \ell\left(h\left(z_c\right),y\right)\right]
 d_{z_c}
\end{align*}

From \textit{step-2}, we have 

$$\mathbb{E}_{y\sim\mathbb{P}(Y\mid z_c)} \left[ \ell\left(h_c\left(z_c\right),y\right)\right]
\leq\mathbb{E}_{y\sim\mathbb{P}(Y\mid z_c)} \left[ \ell\left(h\left(z_c\right),y\right)\right]
$$
for all $z_c\in \mathcal{Z}_c$. By taking the expectation and applying the law of iterated expectation, inequality (\ref{eq:target_causal}) follows. This concludes the proof.


\end{proof}

\subsubsection{Proof of the Theorem~\ref{thm:sufficient_conditions_apd}.2}

Similar to the proof of Theorem~\ref{thm:sufficient_conditions_apd}.1, in the following result, we assume for simplicity that \( \mathbb{P}^e \) is a mixture of the training domains.
Then, Theorem~\ref{thm:sufficient_conditions_apd}.2 is stated as follows:

\begin{theorem}Under Assumption \ref{as:label_idf} and Assumption \ref{as:sufficient_causal_support}, 
if \( f\) is an optimal hypothesis for $\mathbb{P}^{e}$ i.e., $ f \in \underset{f \in \mathcal{F}}{\text{argmin}} \ \mathcal{L}(f, \mathbb{P}^{e}),$ and the support of joint causal and spurious factors of $\mathbb{P}^{e}$ covers the entire causal and spurious factor space $\mathcal{Z}_c\times\mathcal{Z}_e$ i.e., $\text{supp}\{\mathbb{P}^{e} \left (Z_c, Z_e \right )\}=\mathcal{Z}_c\times\mathcal{Z}_e$, then then \( f \in \mathcal{F}^* \).
\end{theorem}

\begin{proof}

Based on structural causal model (SCM) depicted in Figure~\ref{fig:graph} we have a distribution (domain) over the observed variables $(X,Y)$ given the environment $E=e \in \mathcal{E}$: 

Therefore we have:

\begin{align}
&\mathbb{E}_{(x,y)\sim\mathbb{P}^{e}(X,Y)}\left[\ell\left(f\left(x\right),y\right)\right]\nonumber\\
&= 
\int_{\mathcal{Z}_c}\int_{\mathcal{Z}_e}\mathbb{P}^{e}(z_c,z_e)\int_{\mathcal{X}}\int_{\mathcal{U}_x}\mathbb{I}_{x= \psi_x(z_c, z_e,u_x)}\mathbb{P}^{e}(u_x)\int_{\mathcal{Y}}\mathbb{P}^{e}(Y=y\mid z_c)\ell\left(f\left(x\right),y\right)  d_{z_c} d_{z_e} d_x d_y d_{u_x}\nonumber\\
&= 
\int_{\mathcal{Z}_c}\int_{\mathcal{Z}_e}\mathbb{P}^{e}(z_c,z_e)\int_{\mathcal{U}_x}\int_{\mathcal{Y}}\mathbb{P}^{e}(Y=y\mid z_c)\int_{\mathcal{X}}\mathbb{I}_{x= \psi_x(z_c, z_e,u_x)}\ell\left(f\left(x\right),y\right) \mathbb{P}^{e}(u_x) d_{z_c} d_{z_e} d_x d_y d_{u_x}\nonumber\\
&= 
\int_{\mathcal{Z}_c}\int_{\mathcal{Z}_e}\mathbb{P}^{e}(z_c,z_e)\int_{\mathcal{U}_x}\int_{\mathcal{Y}}\mathbb{P}^{e}(Y=y\mid z_c)\int_{\mathcal{X}}\mathbb{I}_{x= \psi_x(z_c, z_e,u_x)}\ell\left(f\left(\psi_x(z_c, z_e,u_x)\right),y\right) \mathbb{P}^{e}(u_x) d_{z_c} d_{z_e} d_x d_y d_{u_x}\nonumber
\\
&= 
\int_{\mathcal{Z}_c}\int_{\mathcal{Z}_e}\mathbb{P}^{e}(z_c,z_e)\int_{\mathcal{U}_x}\int_{\mathcal{Y}}\mathbb{P}^{e}(Y=y\mid z_c)\ell\left(f\left(\psi_x(z_c, z_e,u_x)\right),y\right) \mathbb{P}^{e}(u_x) d_{z_c} d_{z_e} d_y d_{u_x}\nonumber
\\
&= 
\int_{\mathcal{Z}_c}\int_{\mathcal{Z}_e}\mathbb{P}^{e}(z_c,z_e)\int_{\mathcal{U}_x}\mathbb{E}_{y\sim\mathbb{P}(Y\mid z_c)} \left[ \ell\left(f\left(\psi_x(z_c, z_e,u_x)\right),y\right)\right]
 \mathbb{P}^{e}(u_x) d_{z_c} d_{z_e}  d_{u_x}\nonumber
\\
&= 
\int_{\mathcal{Z}_c}\int_{\mathcal{Z}_e}\mathbb{P}^{e}(z_c,z_e)\int_{\mathcal{U}_x}\mathbb{E}_{y\sim\mathbb{P}(Y\mid z_c)} \left[ \ell\left((h\circ g_c)\left(\psi_x(z_c, z_e,u_x)\right),y\right)\right]
 \mathbb{P}^{e}(u_x) d_{z_c} d_{z_e}  d_{u_x}
\end{align}
\end{proof}

Under Assumption \ref{as:label_idf}, 
given \( f\) is an optimal hypothesis for $\mathbb{P}^{e}$ i.e., $ f \in \underset{f \in \mathcal{F}}{\text{argmin}} \ \mathcal{L}(f, \mathbb{P}^{e}),$ then 
\begin{align*}
f\in\int_{\mathcal{U}_x}\mathbb{E}_{y\sim\mathbb{P}(Y\mid z_c)} \left[ \ell\left((h\circ g_c)\left(\psi_x(z_c, z_e,u_x)\right),y\right)\right]
 \mathbb{P}^{e}(u_x)  d_{u_x}
\end{align*}

This holds because, under Assumption~\ref{as:label_idf}, it is guaranteed that there exists an optimal prediction for every $x=\psi_x(z_c, z_e, u_x)$ for all $\{(z_c,z_e)\sim \mathbb{P}^{e}(z_c,z_e),u_x\sim \mathbb{P}^{e}(u_x)\}$.

Furthermore, since the support of the joint causal and spurious factors in \( \mathbb{P}^{e} \) spans the entire causal and spurious factor space, i.e., $\text{supp}\{\mathbb{P}^{e} (Z_c, Z_e)\} = \mathcal{Z}_c \times \mathcal{Z}_e,$ the hypothesis \( f \) remains optimal for all possible configurations of \( x = \psi_x(z_c, z_e, u_x) \) across all \( z_c, z_e, u_x \). This implies that \( f \in \mathcal{F}^* \).

\textbf{Note:} 
\begin{itemize}
    \item It is important to highlight that this theorem aligns with Theorem 3 from \citep{ahuja2021invariance}. However, their analysis is conducted in a linear setting.
    \item We argue that the sub-condition of "sufficient and diverse training domains" is impractical, making it a weak guarantee for ensuring the generalization of algorithms based on this condition.
\end{itemize}



\subsubsection{Proof of the Theorem~\ref{thm:sufficient_conditions_apd}.3}

Similar to the proof of Theorem~\ref{thm:sufficient_conditions_apd}.1, in the following result, we assume for simplicity that \( \mathbb{P}^e \) is a mixture of the training domains.
Then, Theorem~\ref{thm:sufficient_conditions_apd}.3 is stated as follows:

\begin{theorem}Under Assumption \ref{as:label_idf} and Assumption \ref{as:sufficient_causal_support}, given $\mathcal{T}$ is set of all \textbf{invariance-preserving transformations} such that for any $T\in \mathcal{T}$ and $g_c\in \mathcal{G}_c$: $(g_c\circ T)(\cdot)=g_c(\cdot)$,
if \( f\) is an optimal hypothesis for $\mathbb{P}^{e}$ i.e., $ f \in \underset{f \in \mathcal{F}}{\text{argmin}} \ \mathcal{L}(f, \mathbb{P}^{e}),$ and  $f$ is also an optimal hypothesis on all augmented domains i.e., $$f\in \bigcap_{T\in \mathcal{T}}\underset{f\in \mathcal{F}}{\text{argmin}} \ \loss{f,T\#\mathbb{P}^{e}}$$
 
, then then \( f \in \mathcal{F}^* \).
\end{theorem}

\begin{proof}
We first analyze the characteristics of the set of all invariance-preserving transformations \( \mathcal{T} \).

By the definition of \( \mathcal{T} \) and the set of invariant representations \( \mathcal{G}_c \):

\begin{itemize}
    \item given $T\in \mathcal{T}$ and $g_c\in \mathcal{G}_c$, we have $(g_c\circ T)(x)=g_c(x)$ for all $x=\psi(z_c,z_e,u_x)$ (for all $z_c\in \mathcal{Z}_c, z_c\in \mathcal{Z}_c$, $u_x\in \mathcal{U}_x$).
    \item  $g\in \mathcal{G}_c$, $\mathbb{P}(Y\mid g(x)) = \mathbb{P}(Y\mid z_c)$ and $g(x)=g(x')$ holds true for all $\{(x,x',z_c)\mid  x= \psi_x(z_c, z_e, u_x), x'= \psi_x(z_c, z^{'}_e, u^{'}_x) \text{ for all }z_e,z^{'}_e, u_x, u^{'}_x\}$
\end{itemize}

This implies that for any $ x=\psi_x(z_c, z_e, u_x)$, we have:
\begin{equation*}
    T(x) \in \left \{\psi_x(z'_c, z'_e, u'_x) \text{ where } \left (z'_c\in\mathcal{Z}_c,z'_e\in \mathcal{Z}_e, u'_x\in \mathcal{U}_x \text{ and } P(Y\mid Z_c=z_c)=P(Y\mid Z_c=z'_c)\right ) \right \}
\end{equation*}

In other words, given a sample $\psi_x(z_c, z_e, u_x)$, the transformation $T$ operates as follows:
\begin{itemize}
    \item It augments \( z_c \) to its equivalent \( z'_c \), ensuring that $P(Y\mid Z_c=z_c)=P(Y\mid Z_c=z'_c)$
    \item It modifies the environmental (or spurious) feature $z_e$ to any $z'_e \in \mathcal{Z}_e$.
    \item It applies changes to the noise term $u_x$.
\end{itemize}

Therefore, under Assumption~\ref{as:sufficient_causal_support} (causal support), having access to all \( T \in \mathcal{T} \) is equivalent to having sufficient and diverse training domains. That is, the support of the joint causal and spurious factors in \( \mathbb{P}^{e} \) spans the entire causal and spurious factor space i.e., $\text{supp}\{\mathbb{P}^{e} (Z_c, Z_e)\} = \mathcal{Z}_c \times \mathcal{Z}_e$. This concludes the proof.




\textbf{Note:} In general, accessing all transformations from $\mathcal{T}$ is impractical. However, recently, some works leverage foundation models to generate these augmentations, achieving promising empirical performance \cite{ruan2021optimal}.
\end{proof}

\subsection{Necessary Conditions for achieving Generalization}

% \begin{theorem} \textbf{(Theorem \ref{thm:nacessary} in the main paper)}
% Considering the training domains $\mathbb{P}^e$ and representation function $g$, let $\mathcal{H}_{\mathbb{P}^e,g}=\underset{h\in \mathcal{H}}{\text{argmin }} \mathcal{L}\left ( h\circ g,\mathbb{P}^e \right ) $ represent the set of optimal classifiers on $g\#\mathbb{P}^e$ (the push-forward distribution by applying $g$ on $\mathbb{P}^e$), \textbf{the best generalization classifier} from $\mathbb{P}^e$ to $\mathcal{P}$ is defined as 
% \begin{equation}
% \mathcal{F}^{B}_{\mathbb{P}^e,g}=\left \{ h\circ g \mid h = \underset{ h'\in \mathcal{H}_{\mathbb{P}^e,g}}{\rm{argmin }} \sup_{e'\in \mathcal{E}} \mathcal{L}\left (  h'\circ g, \mathbb{P}^{e'} \right ) \right \}
% \end{equation}  

% Give representation function $g: \mathcal{X}\rightarrow \mathcal{Z}$ then $\forall \mathbb{P}^e\sim \mathcal{P}$ we have
% $\mathcal{F}^B_{ \mathbb{P}^e,g} \subseteq \mathcal{F}^{*}$ if and only if $g\in \mathcal{G}_s$. 
% \label{thm:nacessary_apd}
% \end{theorem}

\begin{theorem} \textbf{(Theorem \ref{thm:nacessary} in the main paper)} Given representation function $g$,
$\exists h: h\circ g\in \mathcal{F}^*$ if and only if $g\in \mathcal{G}_s$.
\label{thm:nacessary_apd}
\end{theorem}

\begin{proof} \textit{\textbf{``if"} direction.} If $g\in \mathcal{G}_s$, we have: 

\begin{enumerate}
    \item By definition of $g\in \mathcal{G}_s$ we have
$I(g(X),g_c(X))=I(X,g_c(X)$ i.e., $g(X)$ retain all information about $g_c(X)$ presented in $X$. Therefore, there exists a function \(\phi\) such that \(\phi \circ g \in \mathcal{G}_c\), which implies the existence of a \(g_c\in \mathcal{G}_c\) such that \(\phi \circ g = g_c\).

    \item By the definition of \(g_c \in \mathcal{G}_c\), we can always find a classifier \(h_c\) such that \(h_c \circ g_c \in \mathcal{F}^*\).

\end{enumerate}

To complete the proof of the \textbf{``if"} direction, we need to demonstrate the existence of a classifier \( h \) on top of the representation induced by \( g \) such that it forms a globally optimal hypothesis, i.e., \( h \circ g \in \mathcal{F}^* \).

From (1) and (2) we have $h_c\circ g_c = h_c\circ \phi \circ g \in \mathcal{F}^*$. Therefore, we can construct classifier $h = h_c \circ \phi$, then $h \circ g =  h_c\circ \phi \circ g =h_c\circ g_c \in \mathcal{F}^*$.
\end{proof}

\begin{proof} \textit{\textbf{``only if"} direction by contraction.} 

Define the set of optimal hypotheses induced by a representation function \( g \) as:

\begin{equation*}
\mathcal{F}_{g,\mathcal{E}_{tr}}=\left\{ h\circ g: \bigcap_{{e} \in \mathcal{E}_{tr}} \underset{h\circ g \in \mathcal{F}}{\text{argmin}} \ \mathcal{L}(h\circ g, \mathbb{P}^{e})\right \}
\end{equation*}  

We show that if $g$ is not sufficient-representation, for any $f\in\mathcal{F}_{g,\mathcal{E}_{tr}}$ there exists multiple target domains where $f$ performs arbitrarily bad.  


By definition of $g\notin \mathcal{G}_s$, we have
$I(g(X),g_c(X))<I(X,g_c(X)$. Therefore, there does not exist a function $\phi$ such that $\phi\circ g \in \mathcal{G}_c$. This implies we can not construct any classifier $h = h_c \circ \phi$, then $h \circ g =  h_c\circ \phi \circ g =h_c\circ g_c \in \mathcal{F}^*$.

Consequently, $h$ has to rely on spurious feature $z_e$ (or both $z_c$ and $z_e$) to make predict for some $\{x\mid x=\psi_x\{z_c,z_e, u_x\} \text{ for some } z_c \text{ such that } \mathbb{P}(Y\mid Z_e=z_e)= \mathbb{P}(Y\mid Z_c=z_c) \}$.  

Note that based on structural causal model (SCM) depicted in Figure~\ref{fig:graph}, we have $Z_e\not\!\perp\!\!\!\perp Y$ i.e., the environmental feature $Z_e$ spuriously correlated with $Y$. Therefore, there is a set $\mathcal{B}=\{x\mid x=\psi_x\{z^{'}_c, z_e, u_x\} \text{ for some } z^{'}_c \text{ such that } \mathbb{P}(Y\mid Z_e = z_e)\neq \mathbb{P}(Y\mid Z_c=z^{'}_c)\} \neq \emptyset$. Consequently, $h(\phi(g(x))) \neq h_c(g_c(x))$ for all $x\in \mathcal{B}$

    We can construct undesirable target domains $\mathbb{P}^{e_i}$ with arbitrary loss $\mathcal{L}(h\circ g, \mathbb{P}^{e_i})$ by giving $(1-\delta)$ percentage mass to that examples in $\mathcal{B}$ and $(\delta)$ percentage mass that examples in $\mathcal{X} \setminus \mathcal{B}$. This is equivalent to 
\begin{equation}
    \mathbb{E}_{(x,y)\sim\mathbb{P}^{e_i}} \left [ h(g(x)) \neq h_c(g_c(x))  \right ]  \geq 1-\delta.\nonumber
\end{equation}
 with $(0\leq\delta\leq 1)$.

This concludes the proof.

\end{proof}

% \begin{corollary} \textbf{(Corollary \ref{thm:bad_domain_exist} in the main paper)}
%      Given $g\in \mathcal{G}_s$, there exists $f = h\circ g\in \bigcap_{e\in \mathcal{E}_{train}}\mathcal{F}_{g, \mathbb{P}^e}$  such that for any $0<\delta<1$, there are many undesirable target domains $\mathbb{P}^T \sim \mathcal{P}$ such that:
%     \begin{align*}
%    \mathbb{E}_{(x,y)\sim\mathbb{P}^T} \left [ f(x) \neq f^*(x)  \right ]  \geq 1-\delta.
%     \end{align*}  with  $f^* \in \mathcal{F}^*$.
%     \label{thm:bad_domain_exist_apd}
% \end{corollary}

% \begin{proof}
% Denote $\mathbb{P}^{\mathcal{E}_{tr}}$ is the mixture of training domains, then $\text{supp}\{\mathbb{P}^{\mathcal{E}_{tr}} \left (Z_c \right )\}=\cup_{e\in \mathcal{E}_{tr}}\text{supp}\{\mathbb{P}^{e} \left (Z_c \right )\}=\mathcal{Z}_c$.  Additionally, given $g\in \mathcal{G}_s$, then there exists $\phi$ such that $g_c=\phi\circ g\in \mathcal{G}_c$. 

% Based on structural causal model (SCM) depicted in Figure~\ref{fig:graph}, we have $Z_e\not\!\perp\!\!\!\perp Y$ i.e., the environmental feature $Z_e$ spuriously correlated with $Y$. Hence, there exist $h \not\in \{h_c \circ \phi \mid h_c \circ g_c \in \mathcal{F}_{g_c,\mathbb{P}^{\mathcal{E}_{tr}}}\}$  e.g., $h$ can rely on spurious feature $z_e$ (or both $z_c$ and $z_e$) to make predict for some $\{x\mid x=\psi_x\{z_c,z_e, u_x\} \text{ for some } z_c \text{ such that } \mathbb{P}(Y\mid_{z_e}=z_e)= \mathbb{P}(Y\mid z_c=z_c) \}$. 
    
% There is a set $\mathcal{B}=\{x\mid x=\psi_x\{z^{'}_c, z_e, u_x\} \text{ for some } z^{'}_c \text{ such that } \mathbb{P}(Y\mid_{z_e}=z_e)\neq \mathbb{P}(Y\mid z_c=z^{'}_c)\} \neq \emptyset$. Consequently, $h(\phi(g(x))) \neq h_c(g_c(x))$ for all $x\in \mathcal{B}$

%     We can construct undesirable target domains $\mathbb{P}^{e_i}$ with arbitrary loss $\mathcal{L}(h\circ g, \mathbb{P}^{e_i})$ by giving $(1-\delta)$ percentage mass to that examples in $\mathcal{B}$ and $(\delta)$ percentage mass that examples in $\mathcal{X} \setminus \mathcal{B}$. This is equivalent to 
% \begin{equation}
%     \mathbb{E}_{(x,y)\sim\mathbb{P}^{e_i}} \left [ h(g(x)) \neq h_c(g_c(x))  \right ]  \geq 1-\delta.\nonumber
% \end{equation}
%  with $(0\leq\delta\leq 1)$.

% By Theorem \ref{thm:single_generalization}, $h_c\circ g_c\in \mathcal{F}_{g_c,\mathbb{P}^{\mathcal{E}_{tr}}}$ implies $h_c\circ g_c\in \mathcal{F}^*$. This concludes the proof.

% \end{proof}


% \begin{theorem}
%  \label{thm:convergence_apd} \textbf{(Theorem \ref{thm:convergence} in the main paper)}
% Given sequence of training domains $\mathcal{E}_{tr}=\{e_1,...,e_K\} \subset \mathcal{E}$, denote $\Funion^{k}=\bigcap_{i=1}^{k}\mathcal{F}_{ \mathbb{P}^{e_i}}$. We consider $\mathcal{E}_{tr}$ to be \textbf{diverse} if for domain $e_k$, there exists at least one sample $x=\psi_x(z_c,z_e,u_x)$ such that $\exists f  \in \Funion^{k-1} :f(x)\neq \mathbb{P}(Y\mid z_c)$. Given a set of diverse domains $\mathcal{E}_{tr}$, we have:
% \begin{equation*}
%  \Funion^1\supset  \Funion^2 \supset... \supset  \Funion^K   
% \end{equation*}
% and the number of training domains $\mathcal{E}_{tr}$ is sufficiently large:
% \begin{equation*}
% \lim_{\mathcal{E}_{tr}\rightarrow \mathcal{E}}\Funion^{ \left| \mathcal{E}_{tr} \right|} \rightarrow \mathcal{F}^*.
% \end{equation*}
% \end{theorem}


% \begin{proof}

% We prove the first statement by induction. Consider the case $\mathcal{F}^{\cap}_{k-1}$ and $\mathcal{F}^{\cap}_{k}$, we will show that if $\mathcal{E}_{tr}$ is considered as \textbf{diverse} $\mathcal{F}^{\cap}_{k-1}\supset \mathcal{F}^{\cap}_{k}$. 

% We have $\mathcal{F}^{\cap}_{k-1}\supseteq \mathcal{F}^{\cap}_{k}$ is obvious by definition.
% By definition of "diverse" training domains $\mathcal{E}_{tr}$, there exists at least one sample $x=\psi_x(z_c,z_e,u_x)$ such that $\exists f  \in \Funion^{k-1} :f(x)\neq \mathbb{P}(Y\mid z_c)$. This means $f\notin \mathcal{F}^{\cap}_{k}$, hence, $\mathcal{F}^{\cap}_{k-1}\supset \mathcal{F}^{\cap}_{k}$.


% For the second statement, we need to show that if  $\mathcal{E}_{tr}=\mathcal{E}$ then $\Funion^{ \left| \mathcal{E}_{tr} \right|}= \mathcal{F}^*$. This holds true by the definition of $\mathcal{F}^*$.



% \end{proof}


\begin{corollary} \textbf{(Corollary~\ref{thm:information} in the main paper)} Under Assumption \ref{as:label_idf} and Assumption \ref{as:sufficient_causal_support}, let the minimal representation function $g_{\text{min}}$ be defined as:
\begin{equation}
g_{\text{min}} \in \mathcal{G}_{min}=\left\{\underset{g \in \mathcal{G}}{\text{argmin }} I(g(X); X) \ \text{s.t.} \ f = h \circ g \in \mathcal{F}_{\mathcal{E}_{\text{tr}}} \right\},
\label{eq:minimal}
\end{equation}
where $I$ denotes mutual information. Then, for any $g_c\in \mathcal{G}_c$ the following holds:
\begin{equation}
I(g_{\text{min}}(X), g_c(X)) \leq I(X, g_c(X)),
\end{equation}
and the equality holds if and only if at least one of sufficient conditions is hold.
\label{thm:information_apd}.
\end{corollary}

\begin{proof} We first prove that if one of the sufficient conditions holds, then the following equality holds:
\[
I(g_{\text{min}}(X), g_c(X)) = I(X, g_c(X)).
\]

Define:
\begin{equation*}
\mathcal{G}_{\mathcal{E}_{tr}}=\left \{g: f = h \circ g \in \mathcal{F}_{\mathcal{E}_{\text{tr}}} \right\}.
\end{equation*}
By Theorem~\ref{thm:sufficient_conditions_apd}, if one of the sufficient conditions holds, then \( \mathcal{G}_{\mathcal{E}_{tr}} \subseteq \mathcal{G}_c \).

From the definition in Eq.~(\ref{eq:minimal}), we have:
\[
\mathcal{G}_{\text{min}} \subseteq \mathcal{G}_{\mathcal{E}_{tr}} \subseteq \mathcal{G}_c.
\]
This implies:
\[
I(g_{\text{min}}(X), g_c(X)) = I(g_c(X), g_c(X)) = I(X, g_c(X)).
\]
\end{proof}

\begin{proof} We prove that if the equality holds, then \( g_{\text{min}} \in \mathcal{G}_c \).

If
\[
I(g_{\text{min}}(X), g_c(X)) = I(X, g_c(X)),
\]
then it follows that
\[
I(g_{\text{min}}(X), X) \geq I(X, g_c(X)).
\]
Therefore, by the definition of \( g_{\text{min}} \), we conclude that \( g_{\text{min}} \in \mathcal{G}_c \).

\end{proof}

\subsection{Representation Alignment trade-off}
\label{apd:tradeoff}
As a reminder, $\mathbb{P}$ denotes data distribution on data space $\mathcal{X}$, while $g_{\#}\mathbb{P}$ denotes latent distribution on full latent space $\mathcal{Z}$, with $g: \mathcal{X} \mapsto \mathcal{Z}$ is the encoder. 

In the following, we recap the theoretical results for Hellinger distance as presented by \cite{phung2021learning}. Similar results for $\mathcal{H}$-divergence can be found in Zhao et al. \cite{zhao2019learning}, and for Wasserstein distance in Le et al. \cite{le2021lamda}.

\subsubsection{Upper Bound}

\begin{theorem} 
\label{thm:single_bound_A}Consider the source domain
$\mathbb{P}^{e'}$ and the
target domain $\mathbb{P}^{e}$. Let $\ell$ be any loss function
upper-bounded by a positive constant $L$. For any hypothesis $f:\mathcal{X}\mapsto\mathcal{Y}_{\Delta}$
where $f=h\circ g$ with $g:\mathcal{X}\mapsto\mathcal{Z}$
and $h:\mathcal{Z}\mapsto\mathcal{Y}_{\Delta}$, the target
loss on input space is upper bounded 
\begin{equation}
\begin{aligned}\mathcal{L}\left(f,\mathbb{P}^{e}\right)\leq\mathcal{L}\left(f,\mathbb{P}^{e'}\right)+L\sqrt{2}\,d_{1/2}\left(\mathbb{P}_{g}^{e},\mathbb{P}_{g}^{e'}\right)\end{aligned}
,\label{eq:input_bound_1-1}
\end{equation}
\end{theorem}

This Theorem is directly adapted from the result of Trung et al. \cite{phung2021learning}.
The upper bound for target loss above relates source loss, target loss and data shift on feature space, which is different to other bounds in which the data shift is on input space.

\subsubsection{Lower Bound}
\begin{theorem}
\label{theorem:single_lower_bound_A}
\cite{phung2021learning} Consider a hypothesis $f=h\circ g$, the Hellinger distance between two label marginal distributions $\mathbb{P}^{e'}$ and $\mathbb{P}^{e}$ can be upper-bounded as: 
\begin{equation}
d_{1/2}\left(\mathbb{P}^{e'}_\mathcal{Y},\mathbb{P}^{e}_\mathcal{Y}\right) \leq 
\mathcal{L}\left ( f,\mathbb{P}^{e'} \right )^{1/2}+
d_{1/2}\left ( g_{\#}\mathbb{P}^{e'},g_{\#}\mathbb{P}^{e} \right )+
\mathcal{L}\left ( f,\mathbb{P}^{e} \right )^{1/2}
\end{equation}

where the general loss $\mathcal{L}$ is defined based on the Hellinger loss $\ell$ which is define as $\ell\left ( f(x) \right )=D_{1/2}\left ( f(x),\mathbb{P}(Y\mid x) \right )=2\sum_{i=1}^C\left ( \sqrt{f(x,i)}-\sqrt{\mathbb{P}(Y=i\mid x)} \right )^2$.
\end{theorem}

\subsection{Subspace Representation Alignment}

In the following, we prove the theoretical results for Hellinger distance based on the findings of Trung et al. \cite{phung2021learning}. A similar strategy can be directly applied to $\mathcal{H}$-divergence \cite{zhao2019learning} and Wasserstein distance \cite{le2021lamda}.

\begin{theorem}
\label{theorem:multi_bound_A} \textbf{(Theorem~\ref{theorem:multi_bound})} 
\textit{(Subspace Representation Alignment)} Given \textit{subspace projector} $\Gamma: \mathcal{X}\rightarrow \mathcal{M}$, and a subspace index $m\in \mathcal{M}$, let $A_{m}=\Gamma^{-1}(m)=\left\{ x:\Gamma(x)=m\right\} $ is the region on data space which has the same index $m$ and $\mathbb{P}_{m}^{e}$ be the distribution restricted by $\mathbb{P}^{e}$ over the set $A_{m}$, then $\pi^{e}_m=\frac{\mathbb{P}^{e}\left(A_{m}\right)}{\sum_{m'\in\mathcal{M}}\mathbb{P}^{e}\left(A_{m'}\right)}$ is mixture co-efficient, if the loss function $\ell$ is upper-bounded by a positive
constant $L$, then:

(i)  The target general loss is upper-bounded: 
\begin{align*}
\left | \mathcal{E}_{tr} \right |\sum_{e\in \mathcal{E}_{tr}}\mathcal{L}\left ( f,\mathbb{P}^{e} \right )
\leq
\sum_{e\in \mathcal{E}_{tr}} \sum_{m\in\mathcal{M}}\pi^{e}_m
\mathcal{L}\left ( f,\mathbb{P}^{e}_{m} \right ) +
L\sum_{e, e'\in \mathcal{E}_{tr}}\sum_{m\in\mathcal{M}}\pi^{e}_{m}D\left ( g_{\#}\mathbb{P}^{e}_{m},g_{\#}\mathbb{P}^{e'}_{m} \right ),
\end{align*}
(ii) Distance between two label marginal distribution $\mathbb{P}^{e}_{m}(Y)$ and $\mathbb{P}^{e'}_{m}(Y)$ can be upper-bounded: 
\begin{equation*}
\begin{aligned}
D\left(\mathbb{P}^{e}_{\mathcal{Y},m},\mathbb{P}^{e'}_{\mathcal{Y},m}\right) \leq 
D\left ( g_{\#}\mathbb{P}^{e}_{m},g_{\#}\mathbb{P}^{e'}_{m} \right )
+\mathcal{L}\left ( f,\mathbb{P}^{e}_{m}\right )
+
\mathcal{L}\left ( f,\mathbb{P}^{e'}_{m} \right )
\end{aligned}
\end{equation*}
(iii) Construct the subspace projector $\Gamma$ as the optimal hypothesis for the training domains i,e., \(\Gamma \in \mathcal{F}_{\mathcal{E}_{tr}}\), which defines $\mathcal{M}=\{m=\hat{y}\mid \hat{y}=\Gamma(x), x\in\bigcup_{e\in\mathcal{E}_{tr}}\text{supp}\mathbb{P}^{e} \}\subseteq \mathcal{Y}_\Delta$, then
$
D\left(\mathbb{P}^{e}_{\mathcal{Y}, m},\mathbb{P}^{e'}_{\mathcal{Y},m}\right)=0$ for all \(m \in \mathcal{M}\).

where $g_{\#}\mathbb{P}$ denotes representation distribution on $\mathcal{Z}$ induce by applying $g$ with $g: \mathcal{X} \mapsto \mathcal{Z}$ on data distribution $\mathbb{P}$, $D$ can be $\mathcal{H}$-divergence, Hellinger or Wasserstein distance.
\end{theorem}

\begin{proof}
We consider \textit{sub-space projector} $\Gamma: \mathcal{X}\rightarrow \mathcal{M}$, given a sub-space index $m\in \mathcal{M}$, we denote $A_{m}=\Gamma^{-1}(m)=\left\{ x:\Gamma(x)=m\right\} $ is the region on data space which has the same index $m$.
Let $\mathbb{P}_{m}^{e}$ be the distribution restricted by $\mathbb{P}^{e}$ over the set $A_{m}$ and $\mathbb{P}_{m}^{e}$ as the distribution restricted by $\mathbb{P}^{e}$
over $A_{m}$. Eventually, we define $\mathbb{P}_{m}^{e}\left(y\mid x\right)$ as the probabilistic labeling
distribution on the sub-space $\left(A_{m},\mathbb{P}_{m}^{e}\right)$,
meaning that if $x\sim\mathbb{P}_{m}^{e}$, $\mathbb{P}_{m}^{e}\left(y\mid x\right)=\mathbb{P}_{e}\left(y\mid x\right)$.
Similarly, we define if $x\sim\mathbb{P}_{m}^{e'}$, $\mathbb{P}_{m}^{e'}\left(y\mid x\right)=\mathbb{P}^{e'}\left(y\mid x\right)$. Due to this construction, any data sampled from $\mathbb{P}_{m}^{e}$
or $\mathbb{P}_{m}^{e'}$ have the same index $m=\Gamma(x)$. 
Additionally, since each data point $x \in \mathcal{X}$ corresponds to only a single $\Gamma(x)$, the data space is partitioned into disjoint sets, i.e., $\mathcal{X} = \bigcup_{m=1}^{\mathcal{M}} A_{m}$, where $A_m \cap A_n = \emptyset, \forall m \neq n$. 
Consequently, the general loss of the target domain becomes:
\begin{equation}
\mathcal{L}\left(f,\mathbb{P}^{e}\right):=\sum_{m\in\mathcal{M}}\pi^{e}_m\mathcal{L}\left(f,\mathbb{P}_{m}^{e}\right),\label{eq:subspace_loss}
\end{equation}
where $\mathcal{M}$ is the set of all feasible sub-spaces indexing $m$ and  $\pi^{e}_m=\frac{\mathbb{P}^{e}\left(A_{m}\right)}{\sum_{m'\in\mathcal{M}}\mathbb{P}^{e}\left(A_{m'}\right)}$.

\end{proof}

\begin{proof}(i):

Using the same proof
for a single space in Theorem \ref{thm:single_bound_A}, we obtain:
\begin{equation} \mathcal{L}\left(f,\mathbb{P}^{e}_m\right)
\leq
\mathcal{L}\left(f_m,\mathcal{\mathbb{P}}_{m}^{e'}\right)
+ 
L\sqrt{2} d_{1/2}\left(g_{\#}\mathbb{P}^{e}_{m},g_{\#}\mathbb{P}_{m}^{e'}\right)
\end{equation}

Since $\mathcal{L}\left(f,\mathbb{P}^{e}\right):= \sum_{m}\pi^{e}_m \mathcal{L}\left(f,\mathbb{D}_{m}^{e}\right)$, taking weighted average over $m\in \mathcal{M}$, we reach (ii):
\begin{equation}
\mathcal{L}\left(f,\mathbb{P}^{e}\right)
\leq
\sum_{m}\pi^{e}_m
\mathcal{L}\left(f_m,\mathbb{P}_{m}^{e'}\right)
+ 
L\sqrt{2}\sum_{m}\pi^{e}_m d_{1/2}\left(g_{\#}\mathbb{P}^{e}_{m},g_{\#}\mathbb{P}_{m}^{e'}\right)
\end{equation}

By summing over the training domains on the left-hand side, we obtain:

\begin{align} \sum_{e\in\mathcal{E}_{tr}}\mathcal{L}\left(f_{\mathcal{M}},\mathbb{P}^{e}\right)
\leq&
\sum_{e\in\mathcal{E}_{tr}}\sum_{m}\pi^{e}_m
\mathcal{L}\left(f_m,\mathbb{P}_{m}^{e'}\right)
+ 
\sum_{e\in\mathcal{E}_{tr}}L\sqrt{2}\sum_{m}\pi^{e}_m d_{1/2}\left(g_{\#}\mathbb{P}^{e}_{m},g_{\#}\mathbb{P}_{m}^{e'}\right) \nonumber
\end{align}
Summing over the training domains on the left-hand side again:

\begin{align}
\sum_{e'\in\mathcal{E}_{tr}}\sum_{e\in\mathcal{E}_{tr}}\mathcal{L}\left(f_{\mathcal{M}},\mathbb{P}^{e}\right)
\leq&
\sum_{e'\in\mathcal{E}_{tr}}\sum_{e\in\mathcal{E}_{tr}}\sum_{m}\pi^{e}_m
\mathcal{L}\left(f_m,\mathbb{P}_{m}^{e'}\right)\nonumber\\
&+ 
\sum_{e'\in\mathcal{E}_{tr}}\sum_{e\in\mathcal{E}_{tr}}L\sqrt{2}\sum_{m}\pi^{e}_m d_{1/2}\left(g_{\#}\mathbb{P}^{e}_{m},g_{\#}\mathbb{P}_{m}^{e'}\right)\nonumber
\end{align}

Finally, we obtain:
\begin{align}
\left | \mathcal{E}_{tr} \right | \sum_{e\in \mathcal{E}_{tr}}\mathcal{L}\left(f,\mathbb{P}^{e}\right)
\leq&
\sum_{e,e'\in \mathcal{E}_{tr}}\sum_{m\in\mathcal{M}}\pi^{e}_m
\mathcal{L}\left(f,\mathcal{\mathbb{P}}_{m}^{e'}\right)
+
\sum_{e,e'\in \mathcal{E}_{tr}}L\sqrt{2}\sum_{m\in\mathcal{M}}\pi^{e}_{m} d_{1/2}\left(g_{\#}\mathbb{P}^{e}_{m},g_{\#}\mathbb{P}_{m}^{e'}\right)
\end{align}
\end{proof}


\begin{proof}(ii):

We obtain (ii) directly by applying the results from Theorem \ref{theorem:single_lower_bound_A} to each individual sub-space, denoted by the index $m$.
\end{proof}

\begin{proof}(iii):

Within training domains, we anticipate that $f\in \cap_{e\in \mathcal{E}_{tr}}\mathcal{F}_{\mathbb{P}^{e}}$ will predict the ground truth label $f(x)=f^*(x)$ where $f^*\in \mathcal{F}^*$.We can define a projector \(\Gamma = f\), which induces a set of subspace indices $\mathcal{M}=\{m=\hat{y}\mid \hat{y}=f(x), x\in\bigcup_{e\in\mathcal{E}_{tr}}\text{supp}\mathbb{P}^{e} \}\subseteq \Delta_{\left | \mathcal{Y}\right |}$. As a result, given subspace index $m\in\mathcal{M}$, $\forall i \in \mathcal{Y}, \mathbb{P}^{e}_{\mathcal{Y},m}(Y=i) = \mathbb{P}^{e'}_{\mathcal{Y},m}(Y=i) = \sum_{x \in f^{-1}(m)}\mathbb{P}(Y=i\mid x) = m[i]$. Consequently, \(D\left(\mathbb{P}^{e}_{\mathcal{Y},m}, \mathbb{P}^{e'}_{\mathcal{Y},m}\right) = 0\) for all \(m \in \mathcal{M}\), allowing us to jointly optimize both \textit{domain losses} and \textit{representation alignment}.
\end{proof}

% \begin{theorem}
% \label{theorem:information_view-1} \textbf{(Theorem 1 in the main paper)} Let $X$ is a random variable of source
% sample (i.e., drawn from $\mathbb{P}^{S}$) and $Y$ is a random
% variable of ground-truth labels. Denote $N=\sum_{m'\in\mathcal{M}}\mathbb{P}^{S}\left(A_{m'}\right)$,
% we then have
% \begin{equation}
% \mathbb{I}\left(\Gamma\left(g\left(X\right)\right)\odot g\left(X\right),Y\right)\geq-\sum_{m\in\mathcal{M}}\frac{\mathbb{P}^{S}\left(A_{m}\right)}{N}\mathcal{L}\left(f,\mathbb{D}_{m}^{S}\right)+\text{const},\label{eq:mutual_information-1}
% \end{equation}
% where the loss $\mathcal{L}\left(f,\mathbb{D}_{m}^{S}\right)$
% is defined based on the cross-entropy loss and $\mathbb{I}$ denotes
% the mutual information.
% \end{theorem}
% \begin{proof}
% Denote $T=\Gamma\left(g\left(X\right)\right)\odot g\left(X\right)$,
% we have
% \begin{align*}
% \mathbb{I}\left(T,Y\right) = & \int p(t,y)\log\frac{p\left(t,y\right)}{p\left(t\right)p\left(y\right)}dtdy\\
% = & \int p(t,y)\log\frac{p\left(y\mid t\right)}{p\left(y\right)}dtdy\\
% = & \int p(t,y)\log p\left(y\mid t\right)dtdy+\mathbb{H}\left(Y\right)\\
% = & \int p(t,y)\log h\left(y\mid t\right)\frac{p\left(y\mid t\right)}{h\left(y\mid t\right)}dtdy+\mathbb{H}\left(Y\right)\\
% = & \int p(t,y)\log h\left(y\mid t\right)dtdy+D_{KL}\left(p\left(y\mid t\right)\Vert h\left(y\mid t\right)\right)+\mathbb{H}\left(Y\right)\\
% \geq & \int p(t,y)\log h\left(y\mid t\right)dtdy+\text{const},
% \end{align*}
% where $\mathbb{H}$ specifies the entropy, $D_{KL}$is Kullback-Leibler
% (KL) divergence, $h\left(y\mid t\right)=h\left(t,y\right)=h\left(\Gamma\left(g\left(x\right)\right)\odot g\left(x\right),y\right)$
% for any $h:\mathcal{Z}\rightarrow\mathcal{Y}_{\simplex}$, and $h\left(t,y\right)$
% returns the $y$-th element of $h\left(t\right)$.

% We further derive
% \begin{align*}
% \int p(t,y)\log h\left(y\mid t\right)dtdy = & \sum_{i=1}^{C}\mathbb{E}_{p(t)}\left[p\left(y=i\mid t\right)\log h\left( y=i \mid t\right)\right]\\
% \overset{(1)}{=} & \sum_{i=1}^{C} \mathbb{E}_{p^{S}(x)}\left[p\left(y=i\mid\Gamma\left(g\left(x\right)\right)\odot g\left(x\right)\right)\log h\left(\Gamma\left(g\left(x\right)\right)\odot g\left(x\right),i\right)\right]\\
% = & \sum_{i=1}^{C}\mathbb{E}_{\mathbb{P}^{S}}\left[p\left(y=i\mid\Gamma\left(g\left(x\right)\right)\odot g\left(x\right)\right)\log h\left(\Gamma\left(g\left(x\right)\right)\odot g\left(x\right),i\right)\right].
% \end{align*}

% Note that we have $\overset{(1)}{=}$ because $\Gamma\left(g\left(x\right)\right)\odot g\left(x\right)$
% pushes forward $X\sim p^{S}(x)$ to $T\sim p\left(t\right)$.
% Moreover, according to our definitions: $\mathbb{P}^{S}=\sum_{m\in\mathcal{M}}\frac{\mathbb{P}^{S}\left(A_{m}\right)}{N}\mathbb{P}_{m}^{S}$,
% we hence obtain
% \begin{align*}
% \int p(t,y)\log h\left(y\mid t\right)dtdy = & \sum_{m\in\mathcal{M}}\frac{\mathbb{P}^{S}\left(A_{m}\right)}{N}\sum_{i=1}^{C}\mathbb{E}_{\mathbb{P}_{m}^{S}}\left[p\left(y=i\mid m\odot g\left(x\right)\right)\log h\left(m\odot g\left(x\right),i\right)\right]\\
% = & -\sum_{m\in\mathcal{M}}\frac{\mathbb{P}^{S}\left(A_{m}\right)}{N}\sum_{i=1}^{C}\mathbb{E}_{\mathbb{P}_{m}^{S}}\left[-p_{m}^{S}\left(y=i\mid x\right)\log f\left(x,i\right)\right]\\
% = & -\sum_{m\in\mathcal{M}}\frac{\mathbb{P}^{S}\left(A_{m}\right)}{N}\mathcal{L}\left(f,\mathbb{D}_{m}^{S}\right).
% \end{align*}
% \end{proof}


% \begin{lemma} Given a deterministic sub-space indicator $\Gamma$, we have:
% \begin{align}
% d_{1/2}\left(g_{\#}\mathbb{P}^{T},g_{\#}\mathbb{P}_{S}^{S}\right) 
% \geq &
% \sum_{m} \frac{S^S_m+S^T_m}{2} d_{1/2}\left(g_{\#}\mathbb{P}^{T}_{m},g_{\#}\mathbb{P}_{m}^{S}\right)
% \end{align}
% \label{theorem:latent_data_shift_A}
% \end{lemma}


% \begin{proof}


% By definition, $g_{\#}\mathbb{P}^{T} (z) = g_{\#}\mathbb{P}^{T} (z) = \sum_{m}S^T_m  g_{\#}\mathbb{P}^{T}_{m}\left ( z_m \right ) $ and $ g_{\#}\mathbb{P}_{S}^{S} (z)= g_{\#}\mathbb{P}_{S}^{S} (z) = \sum_{m}S^S_m g_{\#}\mathbb{P}^{S}_{m}\left ( z_m \right )$, we have:
% \begin{align}
% &d_{1/2}\left(g_{\#}\mathbb{P}^{T},g_{\#}\mathbb{P}_{S}^{S}\right)\nonumber\\
% &=
% \left [ 2  \int_{z} \left ( \sqrt{g_{\#}\mathbb{P}^{e_i} (z)}-\sqrt{g_{\#}\mathbb{P}^{e_j}_S (z)} \right )^{2} dz  \right ]^{1/2}
% \\
% &=
% \left [ 2 \int_{z} \left (g_{\#}\mathbb{P}^{e_i} (z) + g_{\#}\mathbb{P}^{e_j}_S (z) - 2\sqrt{g_{\#}\mathbb{P}^{e_i} (z) g_{\#}\mathbb{P}^{e_j}_S (z)} \right ) dz  \right ]^{1/2}
% \\
% &\overset{(1)}{=}\left [ 2\sum_{m}\int_{z \in \Gamma^{-1}(m)} S^T_m  g_{\#}\mathbb{P}^{T}_{m}\left ( z \right ) + S^S_m g_{\#}\mathbb{P}^{S}_{m}\left ( z \right )   -  2\sqrt{S^T_m g_{\#}\mathbb{P}^{S}_{m}\left ( z \right )}
% \sqrt{S^S_m g_{\#}\mathbb{P}^{S}_{m}\left ( z \right )} dz \right ]^{1/2}
% \label{eq:global_d_subspace}
% \end{align}

% Here we note that $\overset{(1)}{=}$ is the results of $\Gamma$ partitioning source and target data to disjoint-sets:
% \begin{equation*}
% g_{\#}\mathbb{P}^{e_j}_m(z),g_{\#}\mathbb{P}^{e_i}_m(z):
% \left\{\begin{matrix}
%  >0 & \text{if} & z\in \Gamma^{-1}\left(m\right)\\ 
%  =0 & \text{if} & z\notin \Gamma^{-1}\left(m\right) \\
% \end{matrix}\right.,
% \end{equation*}
% therefore, given sub-space index $m$ and $z \in \Gamma^{-1}(m)$:
% \begin{align*}
% &g_{\#}\mathbb{P}^{e_j}(z)
% = \sum_{m'}S_{m'} g_{\#}\mathbb{P}^{e_j}_{m'}(z)
% = S_m g_{\#}\mathbb{P}^{e_j}_m(z)\\
% &g_{\#}\mathbb{P}^{e_i}(z)
% = \sum_{m'}S_{m'} g_{\#}\mathbb{P}^{e_i}_{m'}(z)
% = S_m g_{\#}\mathbb{P}^{e_i}_m(z)
% \end{align*}


% Consider $B$ as follows:

% \begin{align}
% B=&\sum_{m} \int_{z} \left ( 
% S^T_m g_{\#}\mathbb{P}^{e_i}_m (z)
% +
% S^S_m g_{\#}\mathbb{P}^{e_j}_m (z) - 2\sqrt{
% S^S_m S^T_m g_{\#}\mathbb{P}^{e_i}_m(z) g_{\#}\mathbb{P}^{e_j}_m (z)} \right ) \1_{z \in \Gamma^{-1}(m)} dz 
% \\
% =&
% \sum_{m} \int_{z} \left (S^T_m \mathbb{P}^{e_i}_m (z)
% +
% S^T_m g_{\#}\mathbb{P}^{e_j}_m (z)
% - 2\sqrt{
% S^T_m S^T_m g_{\#}\mathbb{P}^{e_i}_m(z) g_{\#}\mathbb{P}^{e_j}_m (z)}\right ) \1_{z \in \Gamma^{-1}(m)} dz \\
% &+
% \sum_{m} \int_{z} \left (
% (S^S_m-S^T_m) g_{\#}\mathbb{P}^{e_j}_m (z) - 2\left ( \sqrt{S^S_m} - \sqrt{S^T_m} \right )
% \sqrt{S^T_m g_{\#}\mathbb{P}^{e_i}_m(z) \mathbb{P}^{e_j}_m (z)}\right ) \1_{z \in \Gamma^{-1}(m)} dz 
% \\
% =&
% \sum_{m} \int_{z} \left( S^T_m \left(\sqrt{g_{\#}\mathbb{P}^{e_i}_m (z)} - \sqrt{g_{\#}\mathbb{P}^{e_j}_m (z)} \right) ^{2} \right ) \1_{z \in \Gamma^{-1}(m)} dz 
% \\
% &+
% \underset{=\sum_{m}(S^S_m-S^T_m) =0}{\underbrace{\sum_{m} \int_{z} \left (
% S^S_m-S^T_m\right ) g_{\#}\mathbb{P}^{e_j}_m (z) \1_{z \in \Gamma^{-1}(m)}  dz}} 
% \\
% &+
% \sum_{m} \int_{z} 2\left ( \sqrt{S^T_m} - \sqrt{S^S_m} \right )
% \sqrt{S^T_m g_{\#}\mathbb{P}^{e_i}_m(z) g_{\#}\mathbb{P}^{e_j}_m (z)} \1_{z \in \Gamma^{-1}(m)} dz 
% \\
% =&
% \sum_{m}\int_{z \in \Gamma^{-1}(m)} S^T_m \left (\sqrt{g_{\#}\mathbb{P}^{e_i}_m (z)} - \sqrt{g_{\#}\mathbb{P}^{e_j}_m (z)}\right )^{2} +2\left ( S^T_m - \sqrt{S^T_mS^S_m} \right )
% \sqrt{ g_{\#}\mathbb{P}^{e_i}_m(z) g_{\#}\mathbb{P}^{e_j}_m (z)} dz
% \end{align}

% Similarly, we also have: 
% \begin{equation}
% B =\sum_{m}\int_{z} S^S_m \left (\sqrt{g_{\#}\mathbb{P}^{e_i}_m (z)} - \sqrt{g_{\#}\mathbb{P}^{e_j}_m (z)}\right )^{2} +2\left ( S^S_m - \sqrt{S^T_mS^S_m} \right )
% \sqrt{g_{\#}\mathbb{P}^{e_i}_m(z) g_{\#}\mathbb{P}^{e_j}_m (z)}\1_{z \in \Gamma^{-1}(m)} dz
% \end{equation}

% Combining (40) and (41), we have:
% \begin{align}
% B &=\sum_{m}\int_{z} \frac{S^S_m+S^T_m}{2} \left (\sqrt{g_{\#}\mathbb{P}^{e_i}_m (z)} - \sqrt{g_{\#}\mathbb{P}^{e_j}_m (z)}\right )^{2} +2\left ( \sqrt{S^S_m} - \sqrt{S^T_m} \right )^2
% \sqrt{ g_{\#}\mathbb{P}^{e_i}_m(z) g_{\#}\mathbb{P}^{e_j}_m (z)}\1_{z \in \Gamma^{-1}(m)} dz\\
% &=\sum_{m}\frac{S^S_m+S^T_m}{4}\int_{z}  2\left (\sqrt{g_{\#}\mathbb{P}^{e_i}_m (z)} - \sqrt{g_{\#}\mathbb{P}^{e_j}_m (z)}\right )^{2} +2\left ( \sqrt{S^S_m} - \sqrt{S^T_m} \right )^2
% \sqrt{ g_{\#}\mathbb{P}^{e_i}_m(z) g_{\#}\mathbb{P}^{e_j}_m (z)}\1_{z \in \Gamma^{-1}(m)} dz\\
% &=\sum_{m} \left [ \frac{S^S_m+S^T_m}{4}\left [ d_{1/2}\left(g_{\#}\mathbb{P}^{T}_{m},g_{\#}\mathbb{P}_{m}^{S}\right) \right ]^2+\int_{z}2\left ( \sqrt{S^S_m} - \sqrt{S^T_m} \right )^2
% \sqrt{ g_{\#}\mathbb{P}^{e_i}_m(z) g_{\#}\mathbb{P}^{e_j}_m (z)}\right ] \1_{z \in \Gamma^{-1}(m)} dz
% \end{align}

% Substituting $B$ to~(\ref{eq:global_d_subspace}) we have:

% \begin{align}
% &d_{1/2}\left(\sum_{m} S^T_m g_{\#}\mathbb{P}^{T}_{m},\sum_{m} S^S_m g_{\#}\mathbb{P}_{m}^{S}\right)\nonumber\\
% &=
% \left [ 2\sum_{m} \left [ \frac{S^S_m+S^T_m}{4}\left [ d_{1/2}\left(g^\Gamma_{\#}\mathbb{P}^{T}_{m},g^\Gamma_{\#}\mathbb{P}_{m}^{S}\right) \right ]^2+\int_{z} 2\left ( \sqrt{S^S_m} - \sqrt{S^T_m} \right )^2
% \sqrt{ g^\Gamma_{\#}\mathbb{P}^{e_i}_m(z) g^\Gamma_{\#}\mathbb{P}^{e_j}_m (z)}\right ] \1_{z \in \Gamma^{-1}(m)} dz \right ]^{1/2}
% \\
% &\geq
% \left [ \sum_{m}  \frac{S^S_m+S^T_m}{2}\left [ d_{1/2}\left(g_{\#}\mathbb{P}^{T}_{m},g_{\#}\mathbb{P}_{m}^{S}\right) \right ]^2 \right ]^{1/2}
% \\
% &=
% \left [ \left ( \sum_{m} \frac{S^S_m+S^T_m}{2}\right)  \left ( \sum_{m} \frac{S^S_m+S^T_m}{2}   \left [ d_{1/2}\left(g_{\#}\mathbb{P}^{T}_{m},g_{\#}\mathbb{P}_{m}^{S}\right) \right ]^2 \right) \right]^{1/2}\\
% &\overset{(3)}{\geq}
% \sum_{m} \frac{S^S_m+S^T_m}{2} d_{1/2}\left(g_{\#}\mathbb{P}^{T}_{m},g_{\#}\mathbb{P}_{m}^{S}\right)\\
% \end{align}
% We obtain $(3)$ by applying Cauchy–Schwarz inequality. 
% \end{proof}


% \begin{lemma} \textbf{(Lemma 3 in the main paper)} Given a \textit{Source-Target Balanced sup-space projection $\Gamma$} i.e., $S_m^T=S^S_m =S_m \forall m$, we have:

% \begin{enumerate}
%   \item Reduce latent data-shift: 
%   \begin{equation}
%       d_{1/2}\left(g_{\#}\mathbb{P}^{T},g_{\#}\mathbb{P}_{S}^{S}\right)\nonumber \geq
% \sum_{m} S^m d_{1/2}\left(g_{\#}\mathbb{P}^{T}_{m},g_{\#}\mathbb{P}_{m}^{S}\right)
%   \end{equation}
  
%   \item Tighter target-risks' upper-bound:
%   \begin{equation}
%       \mathcal{L}\left(f,\mathbb{P}^{e_i}\right)
% \leq B_{S}\leq B_{M}
%   \end{equation}
% \end{enumerate}

% \label{theorem:latent_data_shift_A}
% %%\vspace{-2mm}
% \end{lemma}



% \subsection{Practical Method}

% \begin{corollary}
% \label{cor:appendix_cluster_latent}\textbf{(Corollary \ref{cor:cluster_latent} in the main paper)} Consider minimizing $\min_{g,C}\mathcal{W}_{d_{z}}\left(g\#\mathbb{P}_{x},\mathbb{P}_{c,\pi}\right)$
% given $\pi$ and assume
% $K<N$, its optimal solution $g^{*}$ and $C^{*}$are also the
% optimal solution of the following OP:
% \begin{equation}
% \min_{g,C}\min_{\sigma\in\Sigma_{\pi}}\sum_{n=1}^{N}d_{z}\left(g\left(x_{n}\right),c_{\sigma\left(n\right)}\right),\label{eq:appendix_clus_latent}
% \end{equation}
% where $\Sigma_{\pi}$ is the set of assignment functions $\sigma:\left\{ 1,...,N\right\} \rightarrow\left\{ 1,...,K\right\} $
% such that the cardinalities $\left|\sigma^{-1}\left(k\right)\right|,k=1,...,K$
% are proportional to $\pi_{k},k=1,...,K$.
% \end{corollary}

% \textbf{Proof of Corollary \ref{cor:appendix_cluster_latent}}.

% By the Monge definition, we have
% \begin{align*}
% \mathcal{W}_{d_{z}}\left(g\#\mathbb{P}_{x},\mathbb{P}_{c,\pi}\right) & =\mathcal{W}_{d_{z}}\left(\frac{1}{N}\sum_{n=1}^{N}\delta_{g\left(x_{n}\right)},\sum_{k=1}^{K}\pi_{k}\delta_{c_{k}}\right)=\min_{T:T\#\left(g\#\mathbb{P}_{x}\right)=\mathbb{P}_{c,\pi}}\mathbb{E}_{z\sim g\#\mathbb{P}_{x}}\left[d_{z}\left(z,T\left(z\right)\right)\right]\\
% = & \frac{1}{N}\min_{T:T\#\left(g\#\mathbb{P}_{x}\right)=\mathbb{P}_{c,\pi}}\sum_{n=1}^{N}d_{z}\left(g\left(x_{n}\right),T\left(g\left(x_{n}\right)\right)\right).
% \end{align*}

% Since $T\#\left(g\#\mathbb{P}_{x}\right)=\mathbb{P}_{c,\pi}$,
% $T\left(g\left(x_{n}\right)\right)=c_{k}$ for some $k$. Additionally,
% $\left|T^{-1}\left(c_{k}\right)\right|,k=1,...,K$ are proportional
% to $\pi_{k},k=1,...,K$. Denote $\sigma:\left\{ 1,...,N\right\} \rightarrow\left\{ 1,...,K\right\} $
% such that $T\left(g\left(x_{n}\right)\right)=c_{\sigma\left(n\right)},\forall i=1,...,N$,
% we have $\sigma\in\Sigma_{\pi}$. It also follows that 
% \[
% \mathcal{W}_{d_{z}}\left(\frac{1}{N}\sum_{n=1}^{N}\delta_{g\left(x_{n}\right)},\sum_{k=1}^{K}\pi_{k}\delta_{c_{k}}\right)=\frac{1}{N}\min_{\sigma\in\Sigma_{\pi}}\sum_{n=1}^{N}d_{z}\left(g\left(x_{n}\right),c_{\sigma\left(n\right)}\right).
% \]


\
% \section{Additional Discussion with Related Works} \label{apd:relation_to_IDG}

% \paragraph{Optimal Representation \citep{ruan2021optimal}} 

% While at first glance, \citep{ruan2021optimal} and our work share the same goal of identifying the necessary and sufficient conditions for generalization, the two studies fundamentally differ in the following aspects:  

% \citet{ruan2021optimal} aim to identify the set of conditions that are both necessary and sufficient, which provide theoretical guarantee essentially by assuming some knowledge of target domains. Without accessing target information, generalization is provably impossible. Meanwhile, we focus on analyzing generalizability from limited domains without assuming any additional information from the target.    

% More concretely, \citet{ruan2021optimal} propose the \textit{idealized} domain generalization hypothesis (IDG), which is the expected worst-case target risk over source risk minimizers: 

% $$
%     R_{IDG}=\mathbb{E}_{{e_i,e_j}\sim \mathcal{P}}\left [ \sup_{f\in\mathcal{F}_{\mathbb{P}^{e_i}}}\mathcal{L}(f,\mathbb{P}^{e_i}) \right ]
% $$

% $R_{IDG}$ is an expectation over all possible pairs of domains $(e_i, e_j)\sim \mathcal{P}$ where $\mathcal{P}$ is the distribution over domain space $\mathcal{E}$. During training, they sample any two domains from the domain distribution, assigning one as the source and the other as the target, to determine the worst-case target risk. 

% The representation $Z = g(X)$ deemed optimal for IDG must satisfy two conditions (by Theorem 1 therein):

% \begin{itemize}
% \item Sufficient representation: the representation needs to be task-discriminative, allowing a predictor to minimize risk across all domains. In the presence of all domains, this condition can be simply satisfied by learning a hypothesis optimal for all training domains. 

% \item The representation’s marginal support must be consistent across all pairs of source and target domains. This condition generally coincides with our assumption of causal support, which is a common assumption across DG literature. 
% \end{itemize}


% It is clear from the formulation $R_{IDG}$ that \textit{all} possible domains should be known to achieve generalization. \citet{ruan2021optimal} also point out the challenge in generalization without data from the target domain and recommends incorporating data augmentation from pre-trained models such as CLIP. To our best knowledge, using augmentation in DG is not new. Various studies have shown that access to all label-preserving augmentations (which is generally unfeasible) would reveal true causal factors \citep{mitrovic2020representation,gao2023out}. 
% To satisfy this condition, \citet{ruan2021optimal} assume augmentation is Bayes-preserving augmentation (Assumption 10 therein).


\section{Practical Methodology}
\label{Sec:practical}
In this section, we present the practical objectives to achieve Eq. (\ref{eq:final_objective}):

\begin{align}
\min_{f=h\circ g} \underset{\text{Subspace Representation Alignment}}{\underbrace{\sum_{e,e'\in \mathcal{E}_{tr}}\sum_{m\in \mathcal{M}}D\left( g\#\mathbb{P}_m^{e},g\#\mathbb{P}_m^{e'}\right)}}\text{ s.t. } \underset{\text{Training domain optimal hypothesis}}{\underbrace{f=h\circ g\in \bigcap_{e\in \mathcal{E}_{tr}}\underset{ f}{\text{argmin }} \mathcal{L}\left(f,\mathbb{P}^{e}\right)}}
   \label{eq:final_objective_apd}
\end{align}

where $\mathcal{M}=\{\hat{y}\mid \hat{y}=f(x), x\in\bigcup_{e\in\mathcal{E}_{tr}}\text{supp}\mathbb{P}^{e} \}$ and $D$ can be $\mathcal{H}$-divergence, Hellinger distance, Wasserstein distance.

In the following, we consider the encoder $g$, classifier $h$, domain discriminator $h_d$ and set of $K$ empirical training domains $\mathbb{D}^{e_i}=\{x_{j}^{e_i},y_{j}^{e_i}\}_{j=1}^{N_{e_i}}\sim [\mathbb{P}^{e_i} ]^{N_{e_i}}$, $i=1...K$.


\subsection{Optimal hypothesis across training domains}
For \textit{optimal hypothesis across training domains condition}, we simply adopting the objective set forth by ERM: 
\begin{align}
\label{eq:emp_IRM}
     \min_{f} \; \sum_{i=1}^K \mathcal{L}\left(f,\mathbb{D}^{e_i}\right)
\end{align} 


\subsection{Subspace Representation Alignment}
\paragraph{Subspace Modelling and Projection.} 
\label{sec:subspace_project_detail}
Our objective is to map samples $x$ from training domains with identical predictions $f(x) = m$ into a unified subspace, where $m\in \mathcal{M}=\{\hat{y}\mid \hat{y}=f(x), x\in\bigcup_{e\in\mathcal{E}_{tr}}\text{supp}\mathbb{P}^{e} \}$. Given that the cardinality of $\mathcal{M}$ can be exceedingly large, potentially equal to the total number of training samples if the output of $f(x)$ is unique for each sample, this makes the optimization process particularly challenging.

Drawing inspiration from the concept of prototypes \cite{snell2017prototypical}, we suggest representing $\mathcal{M}$ as a set of prototypes $\mathcal{M} = \{m_i\}_{i=1}^{M}$, where each $m_i$ is an element of $\mathcal{Z}$. Consequently, a sample $x$ is assigned to a subspace by selecting the nearest prototype $m_i$ i.e., $i=\underset{ i'}{\text{argmin }} \rho(g(x),m_{i'})$. Note that prototypes act as condensed representations of specific prediction outcomes. Consequently, samples assigned to the same prototype will receive the same prediction. Although this approach streamlines the subspace projection, it may lead to local optima as the mapping might favor a limited number of prototypes early in training \cite{vuong2023vector}. To mitigate this issue, we adopt a Wasserstein (WS) clustering approach \cite{vuong2023vector} to guide the mapping of latent features from each domain into the designated subspace more effectively.
We first endow a discrete distribution over the prototypes as $\mathbb{P}_{\mathcal{M},\pi}=\sum_{i=1}^{M}\pi_{i}\delta_{m_{i}}$
with the Dirac delta function $\delta$ and the weights $\pi\in\Delta_{M}= \{\pi'\geq \boldsymbol{0}: \Vert \pi'\Vert_1 =1\}$. 

Then we project each domain $\mathbb{P}^{e_i}$ in subspaces indexed by prototypes as follows:
\begin{equation}
\min_{\mathcal{M},\pi}\min_{g}\left \{\mathcal{L}_{P}=\sum_{i=1}^K\lambda\mathcal{W}_{\rho}\left(g\#\mathbb{P}^{e_i},\mathbb{P}_{\mathcal{M},\pi}\right)\right \},\label{eq:reconstruct_form_continuous}
\end{equation}
where:
\begin{itemize}
    \item Cost metric $\rho(z,m)=\frac{z^\top m}{\left \| z \right \|\left \| c \right \|}$ is the cosine similarity between the latent representation $z$ and the prototype $c$.
    \item  Wasserstein distance between source domain representation distribution and distribution over prototype $\mathbb{P}_{\mathcal{M},\pi}$:
\begin{align}
\mathcal{W}_{d}\left(g\#\mathbb{P}^{e_i}_{x},\mathbb{P}_{c,\pi}\right)
&=\mathcal{W}_{d}\left(\sum_{n=1}^{B}\frac{1}{B}g\left(x_{n}\right),\sum_{i=1}^{M}\pi_{i}\delta_{m_{i}}\right)\\
&=\frac{1}{B}\min_{\Gamma:\Gamma\#\left(g\#\mathbb{P}^{e_i}_{x}\right)=\mathbb{P}_{c,\pi}}\sum_{n=1}^{B}\rho\left(g\left(x_{n}\right),\Gamma\left(g\left(x_{n}\right)\right)\right)
\end{align}

Where $B$ is the batch size. This Wasserstein distance can be effectively compute by linear dynamic programming method, entropic dual form of optimal transport \citep{genevay2016stochastic} or Sinkhorn algorithm \cite{cuturi2013sinkhorn}.
\end{itemize}


\paragraph{Subspace Alignment Constraints}
Subspace alignment is achieved through a conditional adversarial training approach \cite{gan2016learning, li2018domain}. In this framework, the \textbf{subspace-conditional} domain discriminator $h_d$ aims to accurately predict the domain label ``$e_i$" based on the combined feature $[z,m]$, where $\{z=g(x), m=\Gamma(x)\}$. Concurrently, the objective for the representation function $g$ is to transform the input $x$ into a latent representation $z$ in such a way that $h_d$ is unable to determine the domain ``$e_i$" of $x$.  We employ the Gradient Reversal Layer (GRL) as introduced by\cite{ganin2016domain}, thereby simplifying the optimization process to:

\begin{equation}
    \min_{g, h_d} \left \{\mathcal{L}_{D}=-\sum_{i=1}^K\mathbb{E}_{x\sim\mathbb{D}^{e_i}}\left [ e_i\log h_d\left (\left [ \mathcal{R}\left ( g(x) \right ),m \right ]\right ) \right ] \right \}
\end{equation}

where $\mathcal{R}$ is gradient reversal layer.

% Such negative gradients
% contribute to making the learned features similar across domains. We propose a gradient-reversal layer (GRL) to update $g$\MH{check the latex here, it says theta f} by easily following \cite{ganin2016domain}. This gradient reversal layer does nothing and merely forwards the input to the following layer during forward propagation. However, it multiplies the gradient by −1 during the backpropagation to obtain a negative gradient from the domain classification.
% \subsection{Overall Framework}

\subsection*{Final objective}
Putting all together, we propose a joint optimization objective, which is given as  
\begin{equation}
\min_{\mathcal{M},\pi} \min_{g,h, h_d}  \left \{\mathcal{L}_{H}+\lambda_P\mathcal{L}_{P}+\lambda_D \mathcal{L}_{D}\right \},
\end{equation}
where $ \lambda_P$ is the subpsace projector hyper-parameter and $\lambda_D$ is the representation alignment hyper-parameter. 



We highlight that SRA is most similar to DANN and CDANN. Like these methods, SRA utilizes $\mathcal{H}$-divergence for alignment; however, the key distinction lies in the alignment strategy: 
\begin{itemize}
    \item DANN aligns the entire domain representation, 
    \item CDANN aligns class-conditional representations,
    \item while SRA employs subspace-conditional alignment.
\end{itemize}

It is also important to note that the representation alignment hyperparameter $\lambda_D$ is kept the same for DANN, CDANN, and SRA in our experiments. As discussed in Theorem~\ref{theorem:single_tradeoff}, DANN and CDANN potentially violate necessary conditions, whereas SRA does not (Theorem~\ref{theorem:multi_bound}), leading to improved performance.




% \subsection{Visualization}
% \label{sec:apd_visualization}

% To elucidate the interaction between latent representations and prototypes, we employ t-SNE~\citep{Maaten08visualizingdata} to visualize their distribution as obtained from our method dataset, depicted in Figure~\ref{fig:tsne}. 
% \begin{figure}[h!]
% \begin{centering}
% \subfloat{\centering{}\includegraphics[width=0.8\textwidth]{ICLR2025/Figures/tnse.png}}
% \par\end{centering}
% \caption{The t-SNE visualization of the learned representations and prototypes generated from our method.}
% \label{fig:tsne}
% \end{figure} 

% Observations from Figure~\ref{fig:tsne}.a reveal a homogeneous blend of representations across various domains, indicating an indistinct separation of samples from different domains, while maintaining clear class distinctions as shown in Figure~\ref{fig:tsne}.b. This highlights the efficacy of our subspace representation alignment approach.


% \begin{table}[h!]
% \caption{Classification Accuracy on PACS using ResNet50 with Prototype as classifier.}
% %\vspace{-0.5mm}
% \begin{centering}
% \resizebox{0.6\columnwidth}{!}{ %
% \begin{tabular}{lccccc}
% \toprule
% Algorithm  & \textbf{A} & \textbf{C} & \textbf{P} & \textbf{S} & \textbf{Avg}  \\
% \midrule
% Classifier & 90.2 $\pm$ 0.5 & 84.4 $\pm$ 0.3 & 97.9 $\pm$ 0.1 & 82.3 $\pm$ 0.2 & \textbf{88.7} \\
% Prototype & 90,9 $\pm$ 0.3 & 84.1 $\pm$ 0.3 & 97.5 $\pm$ 0.1 & 82.1 $\pm$ 0.2 & \textbf{88.7} \\
% \bottomrule
% \end{tabular}}
% \par\end{centering}
% \label{tab:PACS_r}
% \end{table}

% Additionally, consistent with our approach, prototypes are intended to capture class-discriminative information. Observations indicate that these prototypes are positioned within the clusters of their respective classes. To test their efficacy, we executed experiments in which prototypes are utilized as markers for class identification. In this process, a classifier initially predicts the class associated with a prototype, which is then used to determine the class label of a representation based on proximity to the nearest prototype. Results presented in Table \ref{tab:PACS_r} show that class labels can be accurately predicted using prototypes, sometimes even surpassing the effectiveness of direct classifier use.



% % \subsection{ Influence of Different Components}
% % \label{sec:apd_components}

% % \begin{table}[h!]
% % \caption{Classification Accuracy on PACS using ResNet50 with absence of different components.}
% % %\vspace{-0.5mm}
% % \begin{centering}
% % \resizebox{0.8\columnwidth}{!}{ %
% % \begin{tabular}{lccccc}
% % \toprule
% % Algorithm  & \textbf{A} & \textbf{C} & \textbf{P} & \textbf{S} & \textbf{Avg}  \\
% % \midrule


% % ERM & 89.3 $\pm$ 0.2 & 83.4 $\pm$ 0.6 & 97.3 $\pm$ 0.3 & 82.5 $\pm$ 0.5 & 88.1 \\
% % Our method w/o $\mathcal{L}_I$ & 89.9 $\pm$ 0.5 & 83.3 $\pm$ 0.4 & 97.9 $\pm$ 0.3 & 81.4 $\pm$ 0.6 & 88.1 \\
% % Our method w/o $\mathcal{L}_D$ & 89.7 $\pm$ 0.5 & 84.2$\pm$ 0.7 & 97.2 $\pm$ 0.3 &  81.5 $\pm$ 0.8 &  88.2\\
% % Our method & 90.2 $\pm$ 0.5 & 84.4 $\pm$ 0.3 & 97.9 $\pm$ 0.1 & 82.3 $\pm$ 0.2 & \textbf{88.7} \\
% % \bottomrule
% % \end{tabular}}
% % \par\end{centering}
% % \label{tab:PACS_componemt}
% % \end{table}
% % Table \ref{tab:PACS_componemt} demonstrates that removing any component significantly reduces performance. Specifically, omitting the subspace alignment constraint results in performance marginally better than the ERM baseline, whereas failing to maximize mutual information $\mathcal{I}(X,g(X))$ leads to performance on par with the baseline.

% \subsection{Ablation study on the number of Subspaces}
 
% Considering our data generation process, the number of distinct labels $\mathbb{P}(Y\mid x)$ reflects the number of distinct causal factors (denoted as $\left | \mathcal{Z} \right |$). If $\mathcal{M}\leq\left | \mathcal{Z} \right |$, samples with different labels may be projected into the same subspace, leading to discrepancies in the marginal label distribution within that subspace.

% We revisit the two key points in the previous discussion:

% \begin{itemize}
% \item Theorem~\ref{theorem:multi_bound} implies that projecting samples into the correct subspaces can significantly reduce or entirely eliminate marginal label shifts within those subspaces, assuming optimal projection for the sake of simplicity.
% \item As mentioned earlier, projecting samples with the same label $\mathbb{P}(Y\mid x)$ eliminates the discrepancy $d_{1/2}\left (\mathbb{P}^{e_i}_{\mathcal{Y},m}, \mathbb{P}^{e_i}_{\mathcal{Y},m} \right )$, reducing it to zero.
% \end{itemize}

% Increasing $\mathcal{M}$ reduces the likelihood of differently labeled samples being mapped to the same subspace, thus decreasing the discrepancy outlined in Theorem~\ref{theorem:multi_bound} (ii). It’s notable that the upper bound in (i) can be optimized to the limit defined by (ii) when the focus is only on training domains. This optimization, in turn, minimizes the bound (i).

% Rather than treating $\left | \mathcal{Z} \right |$ merely as a parameter for tuning, we delve further into analyzing the impact of varying $\left | \mathcal{Z} \right |$ values. In this ablation study, we test $\left | \mathcal{Z} \right |$ values of $[4,8,16,32]\times \left | \mathcal{C} \right |$, where $\left | \mathcal{C} \right |$ denotes the number of classes. 

% \label{sec:apd_subspaces}
% \begin{table}[h!]
% \caption{Classification Accuracy on PACS using ResNet50 with different number of subspaces (NoS) per class.}
% %\vspace{-0.5mm}
% \begin{centering}
% \resizebox{0.60\columnwidth}{!}{ %
% \begin{tabular}{lccccc}
% \toprule
% NoS \left | \mathcal{M} \right |  & \textbf{A} & \textbf{C} & \textbf{P} & \textbf{S} & \textbf{Avg}  \\
% \midrule
% ERM & 89.3 $\pm$ 0.2 & 83.4 $\pm$ 0.6 & 97.3 $\pm$ 0.3 & 82.5 $\pm$ 0.5 & 88.1 \\
% 4 & 90.2 $\pm$ 0.3 & 83.2 $\pm$ 0.7 & 97.9 $\pm$ 0.2 & 82.3 $\pm$ 1.5 & 88.2 \\

% 8 & 90.5 $\pm$ 0.8 & 83.8 $\pm$ 0.6 & 97.6 $\pm$ 0.3 &  82.1 $\pm$ 1.8 &  88.7\\

% 16 & 90.5 $\pm$ 0.5 & 83.4 $\pm$ 0.2 & 97.8 $\pm$ 0.1 & 83.2 $\pm$ 0.2 & 88.7 \\
% 32 & 90.2 $\pm$ 0.5&  83.8 $\pm$ 0.8 & 97.3 $\pm$ 0.4 & 82.0 $\pm$ 1.2 & 88.4\\
% \bottomrule
% \end{tabular}}
% \par\end{centering}
% \label{tab:PACS_prototype}
% \end{table}

% Table \ref{tab:PACS_prototype} reveals that performance generally improves with an increase in the number of prototypes. Nonetheless, a decline in performance is noted when $K$ becomes excessively large. We speculate this behavior is tied to the dataset's underlying causal factors; 
% specifically, if a limited number of causal factors generate the data, assigning a large number of prototypes to capture discriminative information might result in one causal factor being associated with multiple prototypes, thereby introducing ambiguity. This hypothesis, however, requires further investigation for confirmation, and we earmark it for future research.



% \subsection{Compare to other baselines}

% One of our key contributions is offering a new perspective on why domain generalization (DG) algorithms often fail to outperform the fundamental empirical risk minimization (ERM) approach on standard benchmarks, through an analysis of sufficient and necessary conditions. In the main paper, we compare our proposed SRA method with the two most related methods, DANN and CDANN, as they represent specific cases of our approach where the number of subspaces per class is set to 0 and 1, respectively. 

% In this section, we provide additional experimental results from various baselines, both with and without SWAD, on five datasets from the DomainBed benchmark \cite{gulrajani2020search}, to further support our discussion and analysis.


% \begin{table*}[h!]
% \caption{Classification accuracy (\%) for all algorithms across datasets.
% }
% \begin{centering}
% \resizebox{0.65\width}{!}{ %
% \begin{tabular}{lcccccc}
% \toprule
% \textbf{Algorithm}  & \textbf{VLCS} & \textbf{PACS} & \textbf{OfficeHome} & \textbf{TerraIncognita}  & \textbf{DomainNet} & \textbf{Avg} \\
% \toprule
% ERM~\citep{gulrajani2020search}  & 77.5 $\pm$ 0.4 & 85.5 $\pm$ 0.2 & 66.5 $\pm$ 0.3 & 46.1 $\pm$ 1.8 & 40.9 $\pm$ 0.1 & 63.3 \\
% GroupDRO~\citep{sagawa2019distributionally} & 76.7 $\pm$ 0.6 & 84.4 $\pm$ 0.8 & 66.0 $\pm$ 0.7 & 43.2 $\pm$ 1.1 & 33.3 $\pm$ 0.2 & 60.7 \\
% Mixup~\citep{wang2020heterogeneous}  & 77.4 $\pm$ 0.6 & 84.6 $\pm$ 0.6 & 68.1 $\pm$ 0.3 & 47.9 $\pm$ 0.8 & 39.2 $\pm$ 0.1 & 63.4 \\
% MLDG~\citep{li2017learning} & 77.2 $\pm$ 0.4 & 84.9 $\pm$ 1.0 & \textbf{66.8} $\pm$ 0.6 & 47.7 $\pm$ 0.9 & 41.2 $\pm$ 0.1 & 63.6 \\
% MTL~\citep{blanchard2021domain} & 77.2 $\pm$ 0.4 & 84.6 $\pm$ 0.5 & 66.4 $\pm$ 0.5 & 45.6 $\pm$ 1.2 & 40.6 $\pm$ 0.1 & 62.9 \\
% SagNet~\citep{nam2021reducing} & 77.8 $\pm$ 0.5 & \textbf{86.3} $\pm$ 0.2 & 68.1 $\pm$ 0.1 & 48.6 $\pm$ 1.0 & 40.3 $\pm$ 0.1 & 64.2 \\
% ARM~\citep{zhang2021adaptive} & 77.6 $\pm$ 0.3 & 85.1 $\pm$ 0.4 & 64.8 $\pm$ 0.3 & 45.5 $\pm$ 0.3 & 35.5 $\pm$ 0.2 & 61.7 \\
% RSC~\citep{huang2020self}  & 77.1 $\pm$ 0.5 & 85.2 $\pm$ 0.9 & 65.5 $\pm$ 0.9 & 46.6 $\pm$ 1.0 & 38.9 $\pm$ 0.5 & 62.7 \\
% IRM~\citep{arjovsky2020irm} & 78.5 $\pm$ 0.5 & 83.5 $\pm$ 0.8 & 64.3 $\pm$ 2.2 & 47.6 $\pm$ 0.8 & 33.9 $\pm$ 2.8 & 61.6 \\
% VREx~\citep{krueger2021out} & 78.3 $\pm$ 0.2 & 84.9 $\pm$ 0.6 & 66.4 $\pm$ 0.6 & 46.4 $\pm$ 0.6 & 33.6 $\pm$ 2.9 & 61.9 \\
% MMD~\citep{li2018domain} & 77.5 $\pm$ 0.9 & 84.6 $\pm$ 0.5 & 66.3 $\pm$ 0.1 & 42.2 $\pm$ 1.6 & 23.4 $\pm$ 9.5 & 58.8 \\
% CORAL~\citep{sun2016deep} & \textbf{78.8} $\pm$ 0.6 & 86.2 $\pm$ 0.3 & \textbf{68.7} $\pm$ 0.3 & 47.6 $\pm$ 1.0 & {41.5} $\pm$ 0.1 & {64.5} \\
% DANN~\citep{ganin2016domain} & 78.6 $\pm$ 0.4 & 83.6 $\pm$ 0.4 & 65.9 $\pm$ 0.6 & 46.7 $\pm$ 0.5 & 38.3 $\pm$ 0.1 & 62.6 \\
% CDANN~\citep{li2018domain}  & 77.5 $\pm$ 0.1 & 82.6 $\pm$ 0.9 & 65.8 $\pm$ 1.3 & 45.8 $\pm$ 1.6 & 38.3 $\pm$ 0.3 & 62.0 \\
% \midrule

% \textbf{Ours} (SRA)  & 76.4 $\pm$ 0.7 & \textbf{86.3} $\pm$ 1.1 & 66.4 $\pm$ 0.7 & \textbf{49.5} $\pm$ 1.0 & \textbf{44.5} $\pm$ 0.3 & \textbf{64.6} \\
% \bottomrule
% \end{tabular}}
% \par\end{centering}
% \label{tab:Averages_domainbed}
% \end{table*}



% \begin{table*}[h!]
% \caption{Classification accuracy (\%) for all algorithms across datasets.
% }
% \begin{centering}
% \resizebox{0.65\width}{!}{ %
% \begin{tabular}{lccccc}
% \toprule
% \textbf{Algorithm}  & \textbf{VLCS} & \textbf{PACS} & \textbf{OfficeHome} & \textbf{TerraIncognita}  & \textbf{Avg} \\
% \toprule
% SWAD~\citep{cha2021swad} & 79.1 $\pm$ 0.4 & 88.1 $\pm$ 0.4 & 70.6 $\pm$ 0.3 & 50.0 $\pm$ 0.4  & 72.0\\




% SWAD + IRM~\citep{arjovsky2020irm} & 78.8 $\pm$ 0.2 & 88.1 $\pm$ 0.4 & 70.4 $\pm$ 0.2 & 49.6 $\pm$ 1.7  & 71.7 \\
% SWAD + VREx~\citep{krueger2021out} & 78.1 $\pm$ 1.3 & 85.4 $\pm$ 0.5 & 69.9 $\pm$ 0.1 & 50.0 $\pm$ 0.2 & 70.9 \\
% SWAD +CORAL~\citep{sun2016deep} & \underline{78.9} $\pm$ 0.6 & 88.3 $\pm$ 0.5 & 71.4 $\pm$ 0.1 & 51.1 $\pm$ 0.9 & 72.4 \\
% SWAD +MMD~\citep{li2018domain} & 78.7 $\pm$ 0.1 & 88.3 $\pm$ 0.1 & 70.6 $\pm$ 0.4 & 49.6 $\pm$ 0.5  & 71.8 \\
% SWAD + DANN & 79.2 $\pm$ 0.0 & 87.9 $\pm$ 0.5 & 70.5 $\pm$ 0.1 & 50.6 $\pm$ 0.6  & 72.2\\
% SWAD + CDANN & 79.3 $\pm$ 0.2 & 87.7 $\pm$ 0.3 & 70.4 $\pm$ 0.1 & 50.7 $\pm$ 0.1  & 72.2\\
% \midrule

% \textbf{Ours} (SRA + SWAD) & \underline{79.4} $\pm$ 0.4 & \underline{88.7} $\pm$ 0.2 &  \underline{72.1} $\pm$ 0.5 &  \underline{51.6} $\pm$ 1.2 & \underline{73.0} \\
% \textbf{Ours} (SRA + SWAD + Ensemble) & \textbf{79.8} $\pm$ 0.0 & \textbf{89.2} $\pm$ 0.0 &  \textbf{73.2} $\pm$ 0.0 &  \textbf{52.2} $\pm$ 0.0 & \textbf{73.3} \\
% \bottomrule
% \end{tabular}}
% \par\end{centering}
% \label{tab:Averages_domainbed_swad}
% \end{table*}

% As observed in both Table~\ref{tab:Averages_domainbed} and Table~\ref{tab:Averages_domainbed_swad}, the baselines fail to consistently surpass the simple ERM baseline across all settings. While some methods perform well on certain datasets, they perform worse on others. However, the combination of our proposed method (SRA), which enforces strong sufficient conditions, and SWAD, which promotes necessary conditions, significantly improves generalization. This combination outperforms ERM and other baselines in all settings. These results support our analysis in Section~\ref{sec:discussion_DG}, indicating that existing methods often violate the necessary condition for effective domain generalization.

\section{Experimental Settings}
\label{apd:settings}

% \paragraph{Metric.} we adopt the training and evaluation protocol as in DomainBed benchmark \citep{gulrajani2020search}, including dataset splits, hyperparameter (HP) search, model selection on the validation set, and optimizer HP. However, we use a reduced HP search space to reduce computational costs. For training, we choose one domain as the target domain and the remaining domains as the training domain, with 20\% of the samples used for validation and model selection. 

% \paragraph{Datasets.} Following existing benchmark \citep{gulrajani2020search}, we evaluate our method on five datasets: PACS~\citep{li2017deeper} (9,991 images, 7 classes, and 4 domains), VLCS~\citep{torralba2011unbiased} (10,729 images, 5 classes, and 4 domains), OfficeHome~\citep{venkateswara2017deep} (15,588 images, 65 classes, and 4 domains), TerraIncognita~\citep{beery2018recognition} (24,788 images, 10 classes, and 4 domains), and DomainNet~\citep{peng2019moment} (586,575 images, 345 classes, and 6 domains).
\paragraph{Metrics.} We adopt the training and evaluation protocol as in DomainBed benchmark \citep{gulrajani2020search}, including dataset splits, hyperparameter (HP) search, model selection on the validation set, and optimizer HP. To manage computational demands more efficiently, as suggested by \citep{cha2021swad}, we narrow our HP search space. Specifically, we use the Adam optimizer, as detailed in \citep{gulrajani2020search}, setting the learning rate to a default of $5e^{-5}$ and forgoing dropout and weight decay adjustments. The batch size is maintained at 32. For DomainNet, we run a total of 15,000 iterations, while for other datasets, we limit iterations to 5,000, deemed adequate for model convergence. Our method's unique parameters, including the regularization hyperparameters $(\lambda_P, \lambda_D)$, undergo optimization within the range of $[0.01, 0.1, 1.0]$, and the number of prototypes $\left | \mathcal{Z} \right |$ is fixed at 16 times the number of classes. It is worth noting that  while we conduct ablation study on PACS dataset, we utilize the number of prototypes $\left | \mathcal{Z} \right |$ is fixed at $16$ times the number of classes for all datasets.
SWAD-specific hyperparameters remain unaltered from their default settings. The evaluation frequency is set to 300 for all dataset.

Our code is anonymously published at \url{https://anonymous.4open.science/r/submisson-FCF0}.


\subsection{Datasets}
To evaluate the effectiveness of the proposed method, we utilize five
datasets: PACS~\citep{li2017deeper}, VLCS~\citep{torralba2011unbiased},
 Office-Home~\citep{venkateswara2017deep}, Terra Incognita~\citep{beery2018recognition} and DomainNet~\citep{peng2019moment} which are the common DG benchmarks with multi-source domains.
\begin{itemize}
    \item \textbf{PACS}~\citep{li2017deeper}: 9991 images of seven classes in total, over four domains:Art\_painting (A), Cartoon (C), Sketches (S), and Photo (P). 
    
    \item \textbf{VLCS}~\citep{torralba2011unbiased}: five classes over four domains with a total of 10729 samples. The domains are defined by four image origins, i.e., images were taken from the PASCAL VOC 2007 (V), LabelMe (L), Caltech (C) and Sun (S) datasets. 


    \item \textbf{Office-Home}~\citep{venkateswara2017deep}: 65 categories of 15,500 daily objects from 4 domains: Art, Clipart, Product (vendor website with white-background) and Real-World (real-object collected from regular cameras).
    \item \textbf{Terra Incognita}~\citep{beery2018recognition} includes 24,788 wild photographs of dimension (3, 224, 224) with 10
animals, over 4 camera-trap domains L100, L38, L43 and L46. This dataset contains photographs of wild animals taken by camera traps; camera trap locations are different across 
domains. 
    \item  \textbf{DomainNet}~\citep{peng2019moment} contains 596,006 images of dimension (3, 224, 224) and 345 classes, over
6 domains clipart, infograph, painting, quickdraw, real and sketch. This is the biggest
dataset in terms of the number of samples and classes.
\end{itemize}



% \subsection{Results}
% \label{apd:result_details}
% In this section, we present the extended results of Table \ref{tab:Averages} in the main text. The following tables report the domain-specific performance of each method on 5 datasets: VLCS (Table \ref{tab:VLCS}), PACS (Table \ref{tab:PACS}), OfficeHome (Table \ref{tab:OfficeHome}), TerraIncognita (Table \ref{tab:TerraIncognita}) and Domain Net (Table \ref{tab:DomainNet}).

% Standard errors are computed over three trials. Our models are run on 4 RTX 6000 GPU cores of 32GB. One full training routine takes roughly 2 hours. 



% % \subsubsection{VLCS}
% \begin{table}[h!]
% \caption{Classification Accuracy on \textbf{VLCS} using ResNet50}
% %\vspace{-0.5mm}
% \begin{centering}
% \resizebox{0.7\columnwidth}{!}{ %
% \begin{tabular}{lccccc}
% \toprule
% \textbf{Algorithm}  & \textbf{C} & \textbf{L} & \textbf{S} & \textbf{V} & \textbf{Avg}  \\
% \midrule
% ERM~\citep{zhang2020adaptive} & 97.7 $\pm$ 0.4 & 64.3 $\pm$ 0.9 & 73.4 $\pm$ 0.5 & 74.6 $\pm$ 1.3 & 77.5 \\
% DANN~\citep{ganin2016domain}& 99.0 $\pm$ 0.3 & \textbf{65.1} $\pm$ 1.4 & 73.1 $\pm$ 0.3 & 77.2 $\pm$ 0.6 & 78.6 \\
% CDANN~\citep{li2018domain}& 97.1 $\pm$ 0.3 & \textbf{65.1} $\pm$ 1.2 & 70.7 $\pm$ 0.8 & 77.1 $\pm$ 1.5 & 77.5 \\
% \textbf{Ours} (SRA) & {97.1} $\pm$ 1.5 & {63.8} $\pm$ 2.3 & 70.5 $\pm$ 2.2 & 74.1 $\pm$ 1.8 & 76.4 \\
% \midrule


% %MTL~\citep{blanchard2021domain}& 97.8 $\pm$ 0.4 & 64.3 $\pm$ 0.3 & 71.5 $\pm$ 0.7 & 75.3 $\pm$ 1.7 & 77.2 \\
% %SagNet~\citep{nam2021reducing} & 97.9 $\pm$ 0.4 & 64.5 $\pm$ 0.5 & 71.4 $\pm$ 1.3 & {77.5} $\pm$ 0.5 & 77.8 \\
% %ARM~\citep{zhang2020adaptive}& 98.7 $\pm$ 0.2 & 63.6 $\pm$ 0.7 & 71.3 $\pm$ 1.2 & 76.7 $\pm$ 0.6 & 77.6 \\
% %VREx~\citep{krueger2021out}& 98.4 $\pm$ 0.3 & 64.4 $\pm$ 1.4 & {74.1} $\pm$ 0.4 & 76.2 $\pm$ 1.3 & 78.3 \\
% %RSC~\citep{huang2020self}& 97.9 $\pm$ 0.1 & 62.5 $\pm$ 0.7 & 72.3 $\pm$ 1.2 & 75.6 $\pm$ 0.8 & 77.1 \\

% SWAD~\cite{cha2021swad}& {98.8} $\pm$ 0.1 & 63.3 $\pm$ 0.3 & 75.3 $\pm$ 0.5 & 79.2 $\pm$ 0.6 & {79.1}\\
% SWAD + DANN& {99.2} $\pm$ 0.1 & 63.0 $\pm$ 0.8 & 75.3 $\pm$ 1.8 & 79.3 $\pm$ 0.5 & {79.2}\\
% SWAD + CDANN& {99.1} $\pm$ 0.1 & 63.3 $\pm$ 0.7 & 75.1 $\pm$ 0.7 & 80.1 $\pm$ 0.2 & {79.3}\\


% %DNA~\cite{chu2022dna} & {98.8} $\pm$ 0.1 & 63.6 $\pm$ 0.2 & {74.1} $\pm$ 0.1 & \textbf{79.5} $\pm$ 0.4 & 79.0\\


% %DiWA ($M=20$)~\citep{rame2022diverse} & 98.4 $\pm$ 0.1 & 63.4 $\pm$ 0.1 & 75.5 $\pm$ 0.3 & 78.9 $\pm$ 0.6 & 79.1\\
% %DiWA ($M=60$)~\citep{rame2022diverse} & 98.4 & 63.3 & 76.1 & 79.6 & 79.4\\


% \textbf{Ours} (SRA + SWAD) & {98.9} $\pm$ 0.2 & {63.7} $\pm$ 0.3 & 75.6 $\pm$ 0.4 & 79.4 $\pm$ 0.8 & 79.4 \\
% \midrule
% \textbf{Ours} (SRA + SWAD + Ensemble) & \textbf{99.1} $\pm$ 0.0 & {63.9} $\pm$ 0.0 & \textbf{76.3} $\pm$ 0.0 & \textbf{79.9} $\pm$ 0.8 & \textbf{79.8} \\
% \bottomrule
% \end{tabular}}
% \par\end{centering}
% \label{tab:VLCS}
% \end{table}

% % \subsubsection{PACS}

% \begin{table}[h!]
% \caption{Classification Accuracy on \textbf{PACS} using ResNet50}
% %\vspace{-0.5mm}
% \begin{centering}
% \resizebox{0.7\columnwidth}{!}{ %
% \begin{tabular}{lccccc}
% \toprule
% \textbf{Algorithm}  & \textbf{A} & \textbf{C} & \textbf{P} & \textbf{S} & \textbf{Avg}  \\
% \midrule
% ERM~\citep{gulrajani2020search}& 84.7 $\pm$ 0.4 & {80.8} $\pm$ 0.6 & 97.2 $\pm$ 0.3 & 79.3 $\pm$ 1.0 & 85.5 \\
% %IRM~\citep{arjovsky2020%IRM}& 84.8 $\pm$ 1.3 & 76.4 $\pm$ 1.1 & 96.7 $\pm$ 0.6 & 76.1 $\pm$ 1.0 & 83.5 \\
% %GroupDRO~\citep{sagawa2019distributionally}& 83.5 $\pm$ 0.9 & 79.1 $\pm$ 0.6 & 96.7 $\pm$ 0.3 & 78.3 $\pm$ 2.0 & 84.4 \\
% %Mixup~\citep{wang2020heterogeneous}& 86.1 $\pm$ 0.5 & 78.9 $\pm$ 0.8 & \textbf{97.6} $\pm$ 0.1 & 75.8 $\pm$ 1.8 & 84.6 \\
% %MLDG~\citep{li2017learning}& 85.5 $\pm$ 1.4 & 80.1 $\pm$ 1.7 & 97.4 $\pm$ 0.3 & 76.6 $\pm$ 1.1 & 84.9 \\
% %%CORAL & \textbf{88.3} $\pm$ 0.2 & 80.0 $\pm$ 0.5 & 97.5 $\pm$ 0.3 & 78.8 $\pm$ 1.3 & 86.2 \\
% %MMD~\citep{li2018domain}& 86.1 $\pm$ 1.4 & 79.4 $\pm$ 0.9 & 96.6 $\pm$ 0.2 & 76.5 $\pm$ 0.5 & 84.6 \\
% DANN~\citep{ganin2016domain}& 86.4 $\pm$ 0.8 & 77.4 $\pm$ 0.8 & 97.3 $\pm$ 0.4 & 73.5 $\pm$ 2.3 & 83.6 \\
% CDANN~\citep{li2018domain}& 84.6 $\pm$ 1.8 & 75.5 $\pm$ 0.9 & 96.8 $\pm$ 0.3 & 73.5 $\pm$ 0.6 & 82.6 \\
% \textbf{Ours} (SRA) & {86.4} $\pm$ 0.2 & {82.0} $\pm$ 0.8 & 96.7 $\pm$ 1.1 & 80.2 $\pm$ 4.4 & 86.3 \\
% \midrule
% %MTL~\citep{blanchard2021domain}& 87.5 $\pm$ 0.8 & 77.1 $\pm$ 0.5 & 96.4 $\pm$ 0.8 & 77.3 $\pm$ 1.8 & 84.6 \\
% %SagNet~\citep{nam2021reducing} & 87.4 $\pm$ 1.0 & 80.7 $\pm$ 0.6 & 97.1 $\pm$ 0.1 & 80.0 $\pm$ 0.4 & {86.3} \\
% %ARM~\citep{zhang2020adaptive}& 86.8 $\pm$ 0.6 & 76.8 $\pm$ 0.5 & 97.4 $\pm$ 0.3 & 79.3 $\pm$ 1.2 & 85.1 \\
% %VREx~\citep{krueger2021out}& 86.0 $\pm$ 1.6 & 79.1 $\pm$ 0.6 & 96.9 $\pm$ 0.5 & 77.7 $\pm$ 1.7 & 84.9 \\
% %RSC~\citep{huang2020self}& 85.4 $\pm$ 0.8 & 79.7 $\pm$ 1.8 & {97.6} $\pm$ 0.3 & 78.2 $\pm$ 1.2 & 85.2 \\
% SWAD~\cite{cha2021swad}& 89.3 $\pm$ 0.2 & 83.4 $\pm$ 0.6 & 97.3 $\pm$ 0.3 & 82.5 $\pm$ 0.5 & 88.1\\
% SWAD + DANN& 90.7 $\pm$ 1.2 & 82.2 $\pm$ 0.4 & 97.3 $\pm$ 0.1 & 81.6 $\pm$ 0.4 & 87.9\\
% SWAD + CDANN & 90.5 $\pm$ 0.3 & 82.4 $\pm$ 1.0 & 97.6 $\pm$ 0.1 & 80.4 $\pm$ 0.3 & 87.7\\

% %DNA~\cite{chu2022dna} & 89.8 $\pm$ 0.2 & 83.4 $\pm$ 0.4 & {97.7} $\pm$ 0.1 & \textbf{82.6} $\pm$ 0.2 & {88.4}\\

% %DiWA ($M=20$)~\citep{rame2022diverse} & 90.1 $\pm$ 0.6 & 83.3 $\pm$ 0.6 & \textbf{98.2} $\pm$ 0.1 & 83.4 $\pm$ 0.4 & 88.8\\
% %DiWA ($M=60$)~\citep{rame2022diverse} & 90.5 & 83.7 & \textbf{98.2} & 83.8 & 89.0\\


% \textbf{Ours} (SRA + SWAD) & 90.5 $\pm$ 0.5 & 83.4 $\pm$ 0.2 & 97.8 $\pm$ 0.1 & 83.2 $\pm$ 0.2 & 88.7 \\
% \midrule
% \textbf{Ours} (SRA + SWAD + Ensemble) & \textbf{91.2} $\pm$ 0.0 & \textbf{83.8} $\pm$ 0.0 & 97.8 $\pm$ 0.0 & \textbf{83.9} $\pm$ 0.0 & \textbf{89.2} \\
% \bottomrule
% \end{tabular}}
% \par\end{centering}
% \label{tab:PACS}
% \end{table}

% % \subsubsection{OfficeHome}
% \begin{table}[h!]
% \caption{Classification Accuracy on \textbf{OfficeHome} using ResNet50}
% %\vspace{-0.5mm}
% \begin{centering}
% \resizebox{0.7\columnwidth}{!}{ %
% \begin{tabular}{lccccc}
% \toprule
% \textbf{Algorithm}  & \textbf{A} & \textbf{C} & \textbf{P} & \textbf{R} & \textbf{Avg}  \\
% \midrule
% ERM~\citep{gulrajani2020search}& 61.3 $\pm$ 0.7 & 52.4 $\pm$ 0.3 & 75.8 $\pm$ 0.1 & 76.6 $\pm$ 0.3 & 66.5 \\
% %IRM~\citep{arjovsky2020%IRM}& 58.9 $\pm$ 2.3 & 52.2 $\pm$ 1.6 & 72.1 $\pm$ 2.9 & 74.0 $\pm$ 2.5 & 64.3 \\
% %GroupDRO~\citep{sagawa2019distributionally}& 60.4 $\pm$ 0.7 & 52.7 $\pm$ 1.0 & 75.0 $\pm$ 0.7 & 76.0 $\pm$ 0.7 & 66.0 \\
% %Mixup~\citep{wang2020heterogeneous}& 62.4 $\pm$ 0.8 & 54.8 $\pm$ 0.6 & \textbf{76.9} $\pm$ 0.3 & 78.3 $\pm$ 0.2 & 68.1 \\
% %MLDG~\citep{li2017learning}& 61.5 $\pm$ 0.9 & 53.2 $\pm$ 0.6 & 75.0 $\pm$ 1.2 & 77.5 $\pm$ 0.4 & 66.8 \\
% %%CORAL & \textbf{65.3} $\pm$ 0.4 & 54.4 $\pm$ 0.5 & {76.5} $\pm$ 0.1 & \textbf{78.4} $\pm$ 0.5 & \textbf{68.7} \\
% %MMD~\citep{li2018domain}& 60.4 $\pm$ 0.2 & 53.3 $\pm$ 0.3 & 74.3 $\pm$ 0.1 & 77.4 $\pm$ 0.6 & 66.3 \\
% DANN~\citep{ganin2016domain}& 59.9 $\pm$ 1.3 & 53.0 $\pm$ 0.3 & 73.6 $\pm$ 0.7 & 76.9 $\pm$ 0.5 & 65.9 \\
% CDANN~\citep{li2018domain}& 61.5 $\pm$ 1.4 & 50.4 $\pm$ 2.4 & 74.4 $\pm$ 0.9 & 76.6 $\pm$ 0.8 & 65.8 \\
% %MTL~\citep{blanchard2021domain}& 61.5 $\pm$ 0.7 & 52.4 $\pm$ 0.6 & 74.9 $\pm$ 0.4 & 76.8 $\pm$ 0.4 & 66.4 \\
% %SagNet~\citep{nam2021reducing} & 63.4 $\pm$ 0.2 & \textbf{54.8} $\pm$ 0.4 & 75.8 $\pm$ 0.4 & 78.3 $\pm$ 0.3 & 68.1 \\
% %ARM~\citep{zhang2020adaptive}& 58.9 $\pm$ 0.8 & 51.0 $\pm$ 0.5 & 74.1 $\pm$ 0.1 & 75.2 $\pm$ 0.3 & 64.8 \\
% %VREx~\citep{krueger2021out}& 60.7 $\pm$ 0.9 & 53.0 $\pm$ 0.9 & 75.3 $\pm$ 0.1 & 76.6 $\pm$ 0.5 & 66.4 \\
% %RSC~\citep{huang2020self}& 60.7 $\pm$ 1.4 & 51.4 $\pm$ 0.3 & 74.8 $\pm$ 1.1 & 75.1 $\pm$ 1.3 & 65.5 \\
% \textbf{Ours} (SRA) & {62.2} $\pm$ 1.4 & {52.3} $\pm$ 1.7 & 74.5 $\pm$ 0.8 & 76.6 $\pm$ 1.3 & 66.4 \\

% \midrule
% SWAD~\cite{cha2021swad}& 66.1 $\pm$ 0.4 & 57.7 $\pm$ 0.4 & 78.4 $\pm$0.1 & 80.2 $\pm$ 0.2& 70.6\\
% SWAD + DANN& 67.2 $\pm$ 0.1 & 56.2 $\pm$ 0.1 & 78.6 $\pm$0.2 & 80.0 $\pm$ 0.5& 70.5\\
% SWAD + CDANN& 66.8 $\pm$ 0.4 & 56.4 $\pm$ 0.8 & 78.4 $\pm$0.5 & 80.1 $\pm$ 0.2& 70.4\\

% %DNA~\cite{chu2022dna} & 67.7 $\pm$ 0.2 & {57.7} $\pm$ 0.3 & 78.9 $\pm$ 0.2 & 80.5 $\pm$ 0.2 & 71.2\\

% %DiWA ($M=20$)~\citep{rame2022diverse} & 67.3 $\pm$ 0.2 & 57.9 $\pm$ 0.2 & 79.0 $\pm$ 0.2 & 79.9 $\pm$ 0.1 & 71.0 \\
% %DiWA ($M=60$)~\citep{rame2022diverse} & 67.7 & 58.8 & 79.4 & 80.5 & 71.6\\


% \textbf{Ours} (SRA + SWAD) & {69.1} $\pm$ 0.6 & 58.4 $\pm$ 0.8 &  {79.5} $\pm$ 0.2 &  {81.4} $\pm$ 0.3 & {72.1}\\
% \midrule
% \textbf{Ours} (SRA + SWAD + Ensemble) & \textbf{70.5} $\pm$ 0.0 & \textbf{59.5} $\pm$ 0.0 &  \textbf{80.4} $\pm$ 0.0 &  \textbf{82.1} $\pm$ 0.0 & \textbf{73.2}\\
% \bottomrule
% \end{tabular}}
% \par\end{centering}
% \label{tab:OfficeHome}
% \end{table}

% % \subsubsection{TerraIncognita}
% \begin{table}[h!]
% \caption{Classification Accuracy on \textbf{TerraIncognita} using ResNet50}
% \begin{centering}
% \resizebox{0.7\columnwidth}{!}{ %
% \begin{tabular}{lccccc}
% \toprule
% \textbf{Algorithm}  & \textbf{L100} & \textbf{L38}  & \textbf{L43}  & \textbf{L46}  & \textbf{Avg}  \\
% \midrule
% ERM~\citep{gulrajani2020search}& 49.8 $\pm$ 4.4 & 42.1 $\pm$ 1.4 & 56.9 $\pm$ 1.8 & 35.7 $\pm$ 3.9 & 46.1 \\
% %IRM~\citep{arjovsky2020%IRM}& 54.6 $\pm$ 1.3 & 39.8 $\pm$ 1.9 & 56.2 $\pm$ 1.8 & 39.6 $\pm$ 0.8 & 47.6 \\
% %GroupDRO~\citep{sagawa2019distributionally}& 41.2 $\pm$ 0.7 & 38.6 $\pm$ 2.1 & 56.7 $\pm$ 0.9 & 36.4 $\pm$ 2.1 & 43.2 \\
% %Mixup~\citep{wang2020heterogeneous}& \textbf{59.6} $\pm$ 2.0 & 42.2 $\pm$ 1.4 & 55.9 $\pm$ 0.8 & 33.9 $\pm$ 1.4 & 47.9 \\
% %MLDG~\citep{li2017learning}& {54.2} $\pm$ 3.0 & \textbf{44.3} $\pm$ 1.1 & 55.6 $\pm$ 0.3 & 36.9 $\pm$ 2.2 & 47.7 \\
% %%CORAL & 51.6 $\pm$ 2.4 & {42.2} $\pm$ 1.0 & 57.0 $\pm$ 1.0 & 39.8 $\pm$ 2.9 & 47.6 \\
% %MMD~\citep{li2018domain}& 41.9 $\pm$ 3.0 & 34.8 $\pm$ 1.0 & 57.0 $\pm$ 1.9 & 35.2 $\pm$ 1.8 & 42.2 \\
% DANN~\citep{ganin2016domain}& 51.1 $\pm$ 3.5 & 40.6 $\pm$ 0.6 & {57.4} $\pm$ 0.5 & 37.7 $\pm$ 1.8 & 46.7 \\
% CDANN~\citep{li2018domain}& 47.0 $\pm$ 1.9 & 41.3 $\pm$ 4.8 & 54.9 $\pm$ 1.7 & 39.8 $\pm$ 2.3 & 45.8 \\
% \textbf{Ours} (SRA) & {52.9} $\pm$ 3.5 & {45.8} $\pm$ 5.1 & 57.2 $\pm$ 4.6 & 42.3 $\pm$ 1.1 & 49.5 \\

% %MTL~\citep{blanchard2021domain}& 49.3 $\pm$ 1.2 & 39.6 $\pm$ 6.3 & 55.6 $\pm$ 1.1 & 37.8 $\pm$ 0.8 & 45.6 \\
% %SagNet~\citep{nam2021reducing} & 53.0 $\pm$ 2.9 & 43.0 $\pm$ 2.5 & \textbf{57.9} $\pm$ 0.6 & {40.4} $\pm$ 1.3 & \textbf{48.6} \\
% %ARM~\citep{zhang2020adaptive}& 49.3 $\pm$ 0.7 & 38.3 $\pm$ 2.4 & 55.8 $\pm$ 0.8 & 38.7 $\pm$ 1.3 & 45.5 \\
% %VREx~\citep{krueger2021out}& 48.2 $\pm$ 4.3 & 41.7 $\pm$ 1.3 & 56.8 $\pm$ 0.8 & 38.7 $\pm$ 3.1 & 46.4 \\
% %RSC~\citep{huang2020self}& 50.2 $\pm$ 2.2 & 39.2 $\pm$ 1.4 & 56.3 $\pm$ 1.4 & 40.8 $\pm$ 0.6 & 46.6 \\

% %DiWA ($M=20$)~\citep{rame2022diverse} & 52.2 $\pm$ 1.8 & 46.2 $\pm$ 0.4 & 59.2 $\pm$ 0.2 & 37.8 $\pm$ 0.6 & 48.9\\
% %DiWA ($M=60$)~\citep{rame2022diverse} & 52.7 & 46.3 & 59.0 & 37.7 & 49.0\\
% \midrule
% SWAD~\cite{cha2021swad}& 55.4 $\pm$ 0.0 & 44.9 $\pm$ 1.1 & 59.7 $\pm$ 0.4 & 39.9 $\pm$ 0.2 & 50.0\\

% SWAD + DANN & 56.3 $\pm$ 2.6 & 44.9 $\pm$ 0.4 & 60.0 $\pm$ 0.7 & 41.4 $\pm$ 0.3 & 50.6\\

% SWAD + CDANN& 55.2 $\pm$ 2.2 & 45.3 $\pm$ 0.2 & 61.4 $\pm$ 0.7 & 40.9 $\pm$ 2.0 & 50.7\\
% %DNA~\cite{chu2022dna} & 56.8 $\pm$ 1.2 & 47.0 $\pm$ 0.9 & \textbf{61.0} $\pm$ 0.5 & \textbf{44.0} $\pm$ 1.0 & 52.2\\
% \textbf{Ours} (SRA + SWAD) & {56.2} $\pm$ 0.8 & {45.5} $\pm$ 2.6 & {60.4} $\pm$ 1.0& {44.4} $\pm$ 0.6 & {51.6} \\
% \midrule
% \textbf{Ours} (SRA + SWAD + Ensemble) & \textbf{57.4} $\pm$ 0.0 & \textbf{45.3} $\pm$ 0.0 & \textbf{60.9} $\pm$ 0.0 & \textbf{45.2} $\pm$ 0.0 & \textbf{52.2} \\

% \bottomrule
% \end{tabular}}
% \par\end{centering}
% \label{tab:TerraIncognita}
% \end{table}

% % \subsubsection{DomainNet}
% \begin{table}[h!]
% \caption{Classification Accuracy on \textbf{DomainNet} using ResNet50}
% %\vspace{-0.5mm}
% \begin{centering}
% \resizebox{0.9\columnwidth}{!}{ %
% \begin{tabular}{lccccccc}
% \toprule
% \textbf{Algorithm}  & \textbf{clip} & \textbf{info} & \textbf{paint} & \textbf{quick} & \textbf{real} & \textbf{sketch} & \textbf{Avg}  \\
% \midrule
% ERM~\citep{gulrajani2020search}& 58.1 $\pm$ 0.3 & 18.8 $\pm$ 0.3 & 46.7 $\pm$ 0.3 & 12.2 $\pm$ 0.4 & 59.6 $\pm$ 0.1 & 49.8 $\pm$ 0.4 & 40.9 \\
% %IRM~\citep{arjovsky2020%IRM}& 48.5 $\pm$ 2.8 & 15.0 $\pm$ 1.5 & 38.3 $\pm$ 4.3 & 10.9 $\pm$ 0.5 & 48.2 $\pm$ 5.2 & 42.3 $\pm$ 3.1 & 33.9 \\
% % GroupDRO~\citep{sagawa2019distributionally}& 47.2 $\pm$ 0.5 & 17.5 $\pm$ 0.4 & 33.8 $\pm$ 0.5 & 9.3 $\pm$ 0.3 & 51.6 $\pm$ 0.4 & 40.1 $\pm$ 0.6 & 33.3 \\
% % Mixup~\citep{wang2020heterogeneous}& 55.7 $\pm$ 0.3 & 18.5 $\pm$ 0.5 & 44.3 $\pm$ 0.5 & 12.5 $\pm$ 0.4 & 55.8 $\pm$ 0.3 & 48.2 $\pm$ 0.5 & 39.2 \\
% %CORAL & 59.2 $\pm$ 0.1 & 19.7 $\pm$ 0.2 & 46.6 $\pm$ 0.3 & {13.4} $\pm$ 0.4 & 59.8 $\pm$ 0.2 & 50.1 $\pm$ 0.6 & 41.5 \\
% %MMD~\citep{li2018domain}& 32.1 $\pm$ 13.3 & 11.0 $\pm$ 4.6 & 26.8 $\pm$ 11.3 & 8.7 $\pm$ 2.1 & 32.7 $\pm$ 13.8 & 28.9 $\pm$ 11.9 & 23.4 \\
% DANN~\citep{ganin2016domain}& 53.1 $\pm$ 0.2 & 18.3 $\pm$ 0.1 & 44.2 $\pm$ 0.7 & 11.8 $\pm$ 0.1 & 55.5 $\pm$ 0.4 & 46.8 $\pm$ 0.6 & 38.3 \\
% CDANN~\citep{li2018domain}& 54.6 $\pm$ 0.4 & 17.3 $\pm$ 0.1 & 43.7 $\pm$ 0.9 & 12.1 $\pm$ 0.7 & 56.2 $\pm$ 0.4 & 45.9 $\pm$ 0.5 & 38.3 \\
% \textbf{Ours} (SRA) & {64.2} $\pm$ 0.3 & {21.6} $\pm$ 0.9 & 50.8 $\pm$ 1.1 & 13.3 $\pm$ 0.8 & 64.4 $\pm$ 0.1 & 53.0 $\pm$ 0.4 &  44.5\\
% \midrule
% % MTL~\citep{blanchard2021domain}& 57.9 $\pm$ 0.5 & 18.5 $\pm$ 0.4 & 46.0 $\pm$ 0.1 & 12.5 $\pm$ 0.1 & 59.5 $\pm$ 0.3 & 49.2 $\pm$ 0.1 & 40.6 \\
% % SagNet~\citep{nam2021reducing} & 57.7 $\pm$ 0.3 & 19.0 $\pm$ 0.2 & 45.3 $\pm$ 0.3 & 12.7 $\pm$ 0.5 & 58.1 $\pm$ 0.5 & 48.8 $\pm$ 0.2 & 40.3 \\
% % ARM~\citep{zhang2020adaptive}& 49.7 $\pm$ 0.3 & 16.3 $\pm$ 0.5 & 40.9 $\pm$ 1.1 & 9.4 $\pm$ 0.1 & 53.4 $\pm$ 0.4 & 43.5 $\pm$ 0.4 & 35.5 \\
% %VREx~\citep{krueger2021out}& 47.3 $\pm$ 3.5 & 16.0 $\pm$ 1.5 & 35.8 $\pm$ 4.6 & 10.9 $\pm$ 0.3 & 49.6 $\pm$ 4.9 & 42.0 $\pm$ 3.0 & 33.6 \\
% %RSC~\citep{huang2020self}& 55.0 $\pm$ 1.2 & 18.3 $\pm$ 0.5 & 44.4 $\pm$ 0.6 & 12.2 $\pm$ 0.2 & 55.7 $\pm$ 0.7 & 47.8 $\pm$ 0.9 & 38.9 \\
% SWAD~\cite{cha2021swad}& 66.0 $\pm$ 0.1 & 22.4 $\pm$ 0.3 & 53.5 $\pm$ 0.1 & 16.1 $\pm$ 0.2 & 65.8 $\pm$ 0.4 & {55.5} $\pm$ 0.3 & 46.5\\
% SWAD + DANN& 64.3 $\pm$ 0.1 & 21.9 $\pm$ 0.6 & 52.6 $\pm$ 0.2 & 15.5 $\pm$ 0.2 & 65.3 $\pm$ 0.1 & {54.5} $\pm$ 0.1 & 45.7\\
% SWAD + CDANN& 64.3 $\pm$ 0.2 & 21.9 $\pm$ 0.4 & 52.5 $\pm$ 0.0 & 15.6 $\pm$ 0.0 & 65.3 $\pm$ 0.1 & {54.4} $\pm$ 0.2 & 45.7\\

% %DNA~\cite{chu2022dna} & {66.1} $\pm$ 0.2 & \textbf{23.0} $\pm$ 0.1 & \textbf{54.6} $\pm$ 0.1 & \textbf{16.7} $\pm$ 0.1 & {65.8} $\pm$ 0.2 & \textbf{56.8} $\pm$ 0.1 & \textbf{47.2}\\
% %DiWA ($M=20$)~\citep{rame2022diverse} & 63.4 $\pm$ 0.2 & 23.1 $\pm$ 0.1 & 53.9 $\pm$ 0.2 & 15.4 $\pm$ 0.2 & 65.5 $\pm$ 0.2 & 55.1 $\pm$ 0.2 & 46.1\\
% %DiWA ($M=60$)~\citep{rame2022diverse} & 63.5 & 23.3 & 54.3 & 15.6 & 65.7 & 55.3 & 46.3\\


% \textbf{Ours} (SRA + SWAD) & {67.4} $\pm$ 0.1 & {23.5} $\pm$ 0.2 & {55.0} $\pm$ 0.1 & {15.9} $\pm$ 0.2 &  {67.2} $\pm$ 0.2 & {56.6} $\pm$ 0.1 & {47.6} \\
% \midrule
% \textbf{Ours} (SRA + SWAD + Ensemble) & \textbf{68.7} $\pm$ 0.0 & \textbf{24.0} $\pm$ 0.2 & \textbf{56.3} $\pm$ 0.0 & \textbf{16.7} $\pm$ 0.0 &  \textbf{68.5} $\pm$ 0.0 & \textbf{57.8} $\pm$ 0.0 & \textbf{48.7} \\
% \bottomrule
% \end{tabular}}
% \par\end{centering}
% \label{tab:DomainNet}
% \end{table}



\end{document}


% This document was modified from the file originally made available by
% Pat Langley and Andrea Danyluk for ICML-2K. This version was created
% by Iain Murray in 2018, and modified by Alexandre Bouchard in
% 2019 and 2021 and by Csaba Szepesvari, Gang Niu and Sivan Sabato in 2022.
% Modified again in 2023 and 2024 by Sivan Sabato and Jonathan Scarlett.
% Previous contributors include Dan Roy, Lise Getoor and Tobias
% Scheffer, which was slightly modified from the 2010 version by
% Thorsten Joachims & Johannes Fuernkranz, slightly modified from the
% 2009 version by Kiri Wagstaff and Sam Roweis's 2008 version, which is
% slightly modified from Prasad Tadepalli's 2007 version which is a
% lightly changed version of the previous year's version by Andrew
% Moore, which was in turn edited from those of Kristian Kersting and
% Codrina Lauth. Alex Smola contributed to the algorithmic style files.
