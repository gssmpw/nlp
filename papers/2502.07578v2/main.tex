%%
%% This is file `sample-sigplan.tex',
%% generated with the docstrip utility.
%%
%% The original source files were:
%%
%% samples.dtx  (with options: `all,proceedings,bibtex,sigplan')
%% 
%% IMPORTANT NOTICE:
%% 
%% For the copyright see the source file.
%% 
%% Any modified versions of this file must be renamed
%% with new filenames distinct from sample-sigplan.tex.
%% 
%% For distribution of the original source see the terms
%% for copying and modification in the file samples.dtx.
%% 
%% This generated file may be distributed as long as the
%% original source files, as listed above, are part of the
%% same distribution. (The sources need not necessarily be
%% in the same archive or directory.)
%%
%%
%% Commands for TeXCount
%TC:macro \cite [option:text,text]
%TC:macro \citep [option:text,text]
%TC:macro \citet [option:text,text]
%TC:envir table 0 1
%TC:envir table* 0 1
%TC:envir tabular [ignore] word
%TC:envir displaymath 0 word
%TC:envir math 0 word
%TC:envir comment 0 0
%%
%% The first command in your LaTeX source must be the \documentclass
%% command.
%%
%% For submission and review of your manuscript please change the
%% command to \documentclass[manuscript, screen, review]{acmart}.
%%
%% When submitting camera ready or to TAPS, please change the command
%% to \documentclass[sigconf]{acmart} or whichever template is required
%% for your publication.
%%
%%
\documentclass[sigplan,screen]{acmart}
% make references clickable 
\usepackage[]{hyperref}
\usepackage{url}
\usepackage{multirow} % for borders and merged ranges
\usepackage{listings}
\usepackage{dblfloatfix}
\usepackage{placeins}  % Add this in the preamble

\lstset{
    language=bash,
    basicstyle=\ttfamily\footnotesize,
    keywordstyle=\color{blue}\bfseries,
    commentstyle=\color{gray}\itshape,
    stringstyle=\color{red},
    showstringspaces=false,
    breaklines=true,
    tabsize=2,
    frame=single
}

\copyrightyear{2025}
\acmYear{2025}
\setcopyright{cc}
\setcctype{by-nc-sa}
\acmConference[ASPLOS '25]{Proceedings of the 30th ACM International Conference on Architectural Support for Programming Languages and Operating Systems, Volume 2}{March 30-April 3, 2025}{Rotterdam, Netherlands}
\acmBooktitle{Proceedings of the 30th ACM International Conference on Architectural Support for Programming Languages and Operating Systems, Volume 2 (ASPLOS '25), March 30-April 3, 2025, Rotterdam, Netherlands}
\acmDOI{10.1145/3676641.3716267}
\acmISBN{979-8-4007-1079-7/25/03}

% 1 Authors, replace the red X's with your assigned DOI string during the rightsreview eform process.
% 2 Your DOI link will become active when the proceedings appears in the DL.
% 3 Retain the DOI string between the curly braces for uploading your presentation video.

\settopmatter{printacmref=true}
\setlength{\textfloatsep}{7pt}
\setlength{\intextsep}{5pt}

\begin{document}

\title{PIM Is All You Need: A CXL-Enabled GPU-Free System for Large Language Model Inference}

%
% The "author" command and its associated commands are used to define
% the authors and their affiliations.
% Of note is the shared affiliation of the first two authors, and the
% "authornote" and "authornotemark" commands
% used to denote shared contribution to the research.
% \author{Ben Trovato}
% \authornote{Both authors contributed equally to this research.}
% \email{trovato@corporation.com}
% \orcid{1234-5678-9012}
% \author{G.K.M. Tobin}
% \authornotemark[1]
% \email{webmaster@marysville-ohio.com}
% \affiliation{%
%   \institution{Institute for Clarity in Documentation}
%   \city{Dublin}
%   \state{Ohio}
%   \country{USA}
% }

\author{Yufeng Gu}
\authornote{Yufeng Gu and Alireza Khadem contributed equally to this research}
\affiliation{%
  \institution{University of Michigan}
  \city{Ann Arbor}
  \country{USA}}
\email{yufenggu@umich.edu}

\author{Alireza Khadem}
\authornotemark[1]
\affiliation{%
  \institution{University of Michigan}
  \city{Ann Arbor}
  \country{USA}}
\email{arkhadem@umich.edu}

\author{Sumanth Umesh}
\affiliation{%
  \institution{University of Michigan}
  \city{Ann Arbor}
  \country{USA}}
\email{sumanthu@umich.edu}

\author{Ning Liang}
\affiliation{%
  \institution{University of Michigan}
  \city{Ann Arbor}
  \country{USA}}
\email{nliang@umich.edu}

\author{Xavier Servot}
\affiliation{%
  \institution{ETH Zürich}
  \city{Zürich}
  \country{Switzerland}}
\email{xservot@student.ethz.ch}

\author{Onur Mutlu}
\affiliation{%
  \institution{ETH Zürich}
  \city{Zürich}
  \country{Switzerland}}
\email{omutlu@gmail.com}

\author{Ravi Iyer}
\authornote{This research was done while the author was at Intel Corporation}
\affiliation{%
  \institution{Google}
  \city{Mountain View}
  \country{USA}}
\email{raviiyer20@gmail.com}

\author{Reetuparna Das}
\affiliation{%
  \institution{University of Michigan}
  \city{Ann Arbor}
  \country{USA}}
\email{reetudas@umich.edu}

% \author{Yufeng Gu*\textsuperscript{\textdagger} \qquad Alireza Khadem*\textsuperscript{\textdagger} \qquad Sumanth Umesh\textsuperscript{\textdagger} \qquad Ning Liang\textsuperscript{\textdagger}} 
% \author{Xavier Servot\textsuperscript{\textdaggerdbl} \qquad Onur Mutlu\textsuperscript{\textdaggerdbl} \qquad Ravi Iyer**\textsuperscript{\textsection} \qquad Reetuparna Das\textsuperscript{\textdagger}}

% \affiliation{\textsuperscript{\textdagger}University of Michigan \quad \textsuperscript{\textdaggerdbl}ETH Zürich \quad \textsuperscript{\textsection}Google}

% \email{{yufenggu, arkhadem, reetudas}@umich.edu}

% \thanks{* Yufeng Gu and Alireza Khadem contributed equally to this research}
% \thanks{** This research was done while the author was at Intel Corporation}

%%
%% By default, the full list of authors will be used in the page
%% headers. Often, this list is too long, and will overlap
%% other information printed in the page headers. This command allows
%% the author to define a more concise list
%% of authors' names for this purpose.
\renewcommand{\shortauthors}{Yufeng Gu and Alireza Khadem et al.}

%%
%% The abstract is a short summary of the work to be presented in the
%% article.
\definecolor{myblue}{RGB}{0, 0, 150}

\newcommand{\att}[0]{CENT}
\newcommand{\rf}[1]{Shared Buffer}
\newcommand{\Sota}{State-of-the-art}
\newcommand{\sota}{state-of-the-art}
\newcommand{\ali}[1]{\noindent{\textcolor{orange}{\bf \fbox{AK} {\it#1}}}}
\newcommand{\reetu}[1]{\noindent{\textcolor{blue}{\bf \fbox{RD} {\it#1}}}}
\newcommand{\yufeng}[1]{\noindent{\textcolor{purple}{\bf \fbox{YG} {\it#1}}}}
\newcommand{\sumanth}[1]{\noindent{\textcolor{red}{\bf \fbox{SU} {\it#1}}}}
\newcommand{\ning}[1]{\noindent{\textcolor{cyan}{\bf \fbox{NL} {\it#1}}}}
\newcommand{\todo}[1]{\noindent{\textcolor{cyan}{\bf \fbox{TODO} {\it#1}}}}
\newcommand{\red}[1]{\textcolor{myblue}{#1}}
% \newcommand{\red}[1]{\textcolor{red}{#1}}
\newcommand{\ignore}[1]{}

\begin{abstract}


The choice of representation for geographic location significantly impacts the accuracy of models for a broad range of geospatial tasks, including fine-grained species classification, population density estimation, and biome classification. Recent works like SatCLIP and GeoCLIP learn such representations by contrastively aligning geolocation with co-located images. While these methods work exceptionally well, in this paper, we posit that the current training strategies fail to fully capture the important visual features. We provide an information theoretic perspective on why the resulting embeddings from these methods discard crucial visual information that is important for many downstream tasks. To solve this problem, we propose a novel retrieval-augmented strategy called RANGE. We build our method on the intuition that the visual features of a location can be estimated by combining the visual features from multiple similar-looking locations. We evaluate our method across a wide variety of tasks. Our results show that RANGE outperforms the existing state-of-the-art models with significant margins in most tasks. We show gains of up to 13.1\% on classification tasks and 0.145 $R^2$ on regression tasks. All our code and models will be made available at: \href{https://github.com/mvrl/RANGE}{https://github.com/mvrl/RANGE}.

\end{abstract}



\begin{CCSXML}
<ccs2012>
   <concept>
       <concept_id>10010520.10010521.10010528</concept_id>
       <concept_desc>Computer systems organization~Parallel architectures</concept_desc>
       <concept_significance>500</concept_significance>
       </concept>
   <concept>
       <concept_id>10010520.10010521.10010542.10010294</concept_id>
       <concept_desc>Computer systems organization~Neural networks</concept_desc>
       <concept_significance>500</concept_significance>
       </concept>
 </ccs2012>
\end{CCSXML}

\ccsdesc[500]{Computer systems organization~Parallel architectures}
\ccsdesc[500]{Computer systems organization~Neural networks}

%%
%% Keywords. The author(s) should pick words that accurately describe
%% the work being presented. Separate the keywords with commas.
\keywords{Computer Architecture, Processing-In-Memory, Compute Express Link, Generative Artificial Intelligence, Large Language Models.}
%% A "teaser" image appears between the author and affiliation
%% information and the body of the document, and typically spans the
%% page.
% \begin{teaserfigure}
%   \includegraphics[width=\textwidth]{sampleteaser}
%   \caption{Seattle Mariners at Spring Training, 2010.}
%   \Description{Enjoying the baseball game from the third-base
%   seats. Ichiro Suzuki preparing to bat.}
%   \label{fig:teaser}
% \end{teaserfigure}

\received{24 June 2024}
\received[revised]{2 October 2024}
\received[accepted]{27 January 2025}

%%
%% This command processes the author and affiliation and title
%% information and builds the first part of the formatted document.
\maketitle

%%%%%% -- PAPER CONTENT STARTS-- %%%%%%%%

\section{Introduction}

Video generation has garnered significant attention owing to its transformative potential across a wide range of applications, such media content creation~\citep{polyak2024movie}, advertising~\citep{zhang2024virbo,bacher2021advert}, video games~\citep{yang2024playable,valevski2024diffusion, oasis2024}, and world model simulators~\citep{ha2018world, videoworldsimulators2024, agarwal2025cosmos}. Benefiting from advanced generative algorithms~\citep{goodfellow2014generative, ho2020denoising, liu2023flow, lipman2023flow}, scalable model architectures~\citep{vaswani2017attention, peebles2023scalable}, vast amounts of internet-sourced data~\citep{chen2024panda, nan2024openvid, ju2024miradata}, and ongoing expansion of computing capabilities~\citep{nvidia2022h100, nvidia2023dgxgh200, nvidia2024h200nvl}, remarkable advancements have been achieved in the field of video generation~\citep{ho2022video, ho2022imagen, singer2023makeavideo, blattmann2023align, videoworldsimulators2024, kuaishou2024klingai, yang2024cogvideox, jin2024pyramidal, polyak2024movie, kong2024hunyuanvideo, ji2024prompt}.


In this work, we present \textbf{\ours}, a family of rectified flow~\citep{lipman2023flow, liu2023flow} transformer models designed for joint image and video generation, establishing a pathway toward industry-grade performance. This report centers on four key components: data curation, model architecture design, flow formulation, and training infrastructure optimization—each rigorously refined to meet the demands of high-quality, large-scale video generation.


\begin{figure}[ht]
    \centering
    \begin{subfigure}[b]{0.82\linewidth}
        \centering
        \includegraphics[width=\linewidth]{figures/t2i_1024.pdf}
        \caption{Text-to-Image Samples}\label{fig:main-demo-t2i}
    \end{subfigure}
    \vfill
    \begin{subfigure}[b]{0.82\linewidth}
        \centering
        \includegraphics[width=\linewidth]{figures/t2v_samples.pdf}
        \caption{Text-to-Video Samples}\label{fig:main-demo-t2v}
    \end{subfigure}
\caption{\textbf{Generated samples from \ours.} Key components are highlighted in \textcolor{red}{\textbf{RED}}.}\label{fig:main-demo}
\end{figure}


First, we present a comprehensive data processing pipeline designed to construct large-scale, high-quality image and video-text datasets. The pipeline integrates multiple advanced techniques, including video and image filtering based on aesthetic scores, OCR-driven content analysis, and subjective evaluations, to ensure exceptional visual and contextual quality. Furthermore, we employ multimodal large language models~(MLLMs)~\citep{yuan2025tarsier2} to generate dense and contextually aligned captions, which are subsequently refined using an additional large language model~(LLM)~\citep{yang2024qwen2} to enhance their accuracy, fluency, and descriptive richness. As a result, we have curated a robust training dataset comprising approximately 36M video-text pairs and 160M image-text pairs, which are proven sufficient for training industry-level generative models.

Secondly, we take a pioneering step by applying rectified flow formulation~\citep{lipman2023flow} for joint image and video generation, implemented through the \ours model family, which comprises Transformer architectures with 2B and 8B parameters. At its core, the \ours framework employs a 3D joint image-video variational autoencoder (VAE) to compress image and video inputs into a shared latent space, facilitating unified representation. This shared latent space is coupled with a full-attention~\citep{vaswani2017attention} mechanism, enabling seamless joint training of image and video. This architecture delivers high-quality, coherent outputs across both images and videos, establishing a unified framework for visual generation tasks.


Furthermore, to support the training of \ours at scale, we have developed a robust infrastructure tailored for large-scale model training. Our approach incorporates advanced parallelism strategies~\citep{jacobs2023deepspeed, pytorch_fsdp} to manage memory efficiently during long-context training. Additionally, we employ ByteCheckpoint~\citep{wan2024bytecheckpoint} for high-performance checkpointing and integrate fault-tolerant mechanisms from MegaScale~\citep{jiang2024megascale} to ensure stability and scalability across large GPU clusters. These optimizations enable \ours to handle the computational and data challenges of generative modeling with exceptional efficiency and reliability.


We evaluate \ours on both text-to-image and text-to-video benchmarks to highlight its competitive advantages. For text-to-image generation, \ours-T2I demonstrates strong performance across multiple benchmarks, including T2I-CompBench~\citep{huang2023t2i-compbench}, GenEval~\citep{ghosh2024geneval}, and DPG-Bench~\citep{hu2024ella_dbgbench}, excelling in both visual quality and text-image alignment. In text-to-video benchmarks, \ours-T2V achieves state-of-the-art performance on the UCF-101~\citep{ucf101} zero-shot generation task. Additionally, \ours-T2V attains an impressive score of \textbf{84.85} on VBench~\citep{huang2024vbench}, securing the top position on the leaderboard (as of 2025-01-25) and surpassing several leading commercial text-to-video models. Qualitative results, illustrated in \Cref{fig:main-demo}, further demonstrate the superior quality of the generated media samples. These findings underscore \ours's effectiveness in multi-modal generation and its potential as a high-performing solution for both research and commercial applications.
\section{Motivation} \label{sec_motivation}


While SSMs provide notable efficiency advantages, deploying them on NPUs presents unique challenges due to their computational patterns and hardware requirements. Unlike traditional deep learning models, SSMs exhibit characteristics that deviate from standard kernel operations, necessitating specialized optimizations. Existing NPUs are designed primarily for data-parallel operations like matrix multiplications, which dominate workloads in transformers and CNNs. SSMs, however, involve sequential computations and specialized operators, such as activation functions (e.g., Swish and Softplus) and cumulative summations (CumSum). These operations do not align with the highly parallelized architecture of NPUs, leading to inefficient execution when mapped directly. Fig.~\ref{fig:motivation_exec_lat_brkdwn} highlights execution bottlenecks for Mamba and Mamba-2 models on the Intel\textregistered\ Core\texttrademark\ Ultra Series 2~\cite{lnl} NPU. For Mamba, the majority of execution time is consumed by activation functions, such as Swish (SiLU) and Softplus, which are executed sequentially on DSPs. These DSPs are less optimized for such operations, resulting in prolonged execution times and underutilization of the data-parallel units. In Mamba-2, CumSum and ReduceSum emerge as primary bottlenecks, as these operations also rely on DSPs for sequential processing. This sequential nature hinders efficient reuse of local SRAM, increasing memory traffic and access latency. Both models further face challenges with elementwise multiplication (Multiply), which similarly runs on DSPs and contributes to inefficiencies. Handling long sequences in SSMs requires careful memory optimization. Limited on-chip memory must be utilized effectively to avoid frequent off-chip memory accesses, which incur significant latency and energy costs. The lack of optimized dataflow alignment for SSM computations exacerbates this issue, leading to poor performance. Blind, out-of-the-box mapping of SSMs on NPUs results in suboptimal performance, leaving much of their potential benefits untapped. Addressing these challenges is essential to fully leverage the advantages of SSMs in resource-constrained environments.





\section{Background} \label{section:LLM}

% \subsection{Large Language Model (LLM)}   

Figure~\ref{fig:LLaMA_model}(a) shows that a decoder-only LLM initially processes a user prompt in the “prefill” stage and subsequently generates tokens sequentially during the “decoding” stage.
Both stages contain an input embedding layer, multiple decoder transformer blocks, an output embedding layer, and a sampling layer.
Figure~\ref{fig:LLaMA_model}(b) demonstrates that the decoder transformer blocks consist of a self attention and a feed-forward network (FFN) layer, each paired with residual connection and normalization layers. 

% Differentiate between encoder/decoder, explain why operation intensity is low, explain the different parts of a transformer block. Discuss Table II here. 

% Explain the architecture with Llama2-70B.

% \begin{table}[thb]
% \renewcommand\arraystretch{1.05}
% \centering
% % \vspace{-5mm}
%     \caption{ML Model Parameter Size and Operational Intensity}
%     \vspace{-2mm}
%     \small
%     \label{tab:ML Model Parameter Size and Operational Intensity}    
%     \scalebox{0.95}{
%         \begin{tabular}{|c|c|c|c|c|}
%             \hline
%             & Llama2 & BLOOM & BERT & ResNet \\
%             Model & (70B) & (176B) & & 152 \\
%             \hline
%             Parameter Size (GB) & 140 & 352 & 0.17 & 0.16 \\
%             \hline
%             Op Intensity (Ops/Byte) & 1 & 1 & 282 & 346 \\
%             \hline
%           \end{tabular}
%     }
% \vspace{-3mm}
% \end{table}

% {\fontsize{8pt}{11pt}\selectfont 8pt font size test Memory Requirement}

\begin{figure}[t]
    \centering
    \includegraphics[width=8cm]{Figure/LLaMA_model_new_new.pdf}
    \caption{(a) Prefill stage encodes prompt tokens in parallel. Decoding stage generates output tokens sequentially.
    (b) LLM contains N$\times$ decoder transformer blocks. 
    (c) Llama2 model architecture.}
    \label{fig:LLaMA_model}
\end{figure}

Figure~\ref{fig:LLaMA_model}(c) demonstrates the Llama2~\cite{touvron2023llama} model architecture as a representative LLM.
% The self attention layer requires three GEMVs\footnote{GEMVs in multi-head attention~\cite{attention}, narrow GEMMs in grouped-query attention~\cite{gqa}.} to generate query, key and value vectors.
In the self-attention layer, query, key and value vectors are generated by multiplying input vector to corresponding weight matrices.
These matrices are segmented into multiple heads, representing different semantic dimensions.
The query and key vectors go though Rotary Positional Embedding (RoPE) to encode the relative positional information~\cite{rope-paper}.
Within each head, the generated key and value vectors are appended to their caches.
The query vector is multiplied by the key cache to produce a score vector.
After the Softmax operation, the score vector is multiplied by the value cache to yield the output vector.
The output vectors from all heads are concatenated and multiplied by output weight matrix, resulting in a vector that undergoes residual connection and Root Mean Square layer Normalization (RMSNorm)~\cite{rmsnorm-paper}.
The residual connection adds up the input and output vectors of a layer to avoid vanishing gradient~\cite{he2016deep}.
The FFN layer begins with two parallel fully connections, followed by a Sigmoid Linear Unit (SiLU), and ends with another fully connection.
\section{Architecture}
\label{sec:architecture}

\begin{figure}
    \centering
    \includegraphics[width=0.92\linewidth]{graphs/arch.pdf}
    \caption{
        \sysname{} proposes an LLM-based no-code application development framework using FaaS for infrastructure abstraction.
        The prompt constructor combines a user's application description with a system prompt for an LLM that generates application code.
        The function deployer uses that code to deploy a FaaS function on a FaaS platform.
    }
    \label{fig:arch}
\end{figure}

LLMs are excellent tools for transforming natural language software descriptions into executable code, but are by themselves unable to deploy and operate that code for users.
FaaS platforms can deploy small pieces of code as scalable, managed applications.
With \sysname{}, we propose combining these two technologies into an end-to-end no-code application development platform.
Our goal is to let non-technical users, i.e., individuals without experience in software development or operation, provide application descriptions in natural language and build fully-managed applications from those descriptions.
Examples for such applications can be found, e.g., in the context of smart home automation, simple extensions of enterprise applications, or custom information aggregation from news websites, social media, and web APIs~\cite{paper_bermbach2020_webapibenchmarking2}.
To support such applications, we design \sysname{} as shown in \cref{fig:arch}.

\sysname{} comprises three main components: an LLM for generating user-specified code, a FaaS platform for efficient function deployment, and a bridge that orchestrates prompt construction and function deployment.
Users provide their natural language application descriptions to \sysname{}, which combines them with a static system prompt in a \emph{prompt constructor}.
This structured prompt instructs an LLM to generate code based on the natural language description, including, e.g., details on programming language, application context, API references, and runtime environment.
\sysname{} then parses the LLM's answer for code in a \emph{function deployer}.
This generated code is deployed on the FaaS platform which abstracts the underlying application infrastructure complexities by providing containerized, auto-scaling environments for on-demand execution.

\section{Model Mapping} \label{model mapping}

The ever-increasing parameter size of the LLMs, coupled with the lower memory density of PIM, necessitates the distribution of the LLM inference on a scalable network of PIM modules.
In this section, we introduce the mapping of various LLM parallelization strategies on \att{}'s CXL-based network architecture using the proposed collective and peer-to-peer communication primitives.

\subsection{Pipeline-Parallel Mapping (PP)}

Cloud providers serve a large user base, where inference throughput is crucial.
To improve throughput, PP~\cite{gpipe} assigns each transformer block to a pipeline stage.
The individual queries in a batch are simultaneously processed in different stages of the pipeline.
Figure~\ref{fig:Pipeline_Parallelism} shows that we map multiple pipeline stages (\textit{e.g.,} \texttt{T0-3}) to a CXL device (\textit{e.g.,} \texttt{D0}).
Each stage requires multiple PIM channels, depending on the memory requirements of the decoder block.
To prevent excessive communication and keep the latency of pipeline stages identical, we avoid splitting a pipeline stage between the PIM channels of two CXL devices.

In each iteration, the output of each transformer block is transferred to the next pipeline stage.
\att{} performs this data transfer using intra-device communication for pipeline stages within the same CXL device, and using peer-to-peer \textit{send} and \textit{receive} primitives for those in different CXL devices.
This CXL data transfer contains only an 8K embedding vector (\texttt{16KB} data) in Llama2-70B.
The CXL transfer latency of PP is negligible compared to PIM and PNM latencies.

Note that \att{} does not support batch processing within a single pipeline stage because of two primary reasons:
First, batching requires a significantly larger Global Buffer and \rf{} (Section~\ref{subsec:pim_pnm_arch}) to concurrently store the embedding vectors of multiple queries.
Second, batching enhances the operational intensity and compute utilization (Section~\ref{sec:motivation}), while PP fully utilizes PIM compute resources.
Therefore, applying batching on top of PP only increases the latency.

\begin{figure}[h]
    \centering
    \includegraphics[width=8cm]{Figure/Pipeline_Parallelism_new.pdf}
    % \includegraphics[width=\columnwidth]{Figure/Pipeline_Parallelism_new.pdf}
    \caption{Pipeline parallelism: (a) Transformer decoder blocks are distributed across CXL devices and form the pipeline stages. Each block is mapped to multiple GDDR6-PIM channels. (b) Multiple prompts are executed in different stages of the pipeline.}
    \label{fig:Pipeline_Parallelism}
\end{figure}

\subsection{Tensor-Parallel Mapping (TP)}

Inference latency is critical in real-time applications to provide a smooth user experience~\cite{fowers2018configurable}.
To enhance the latency, TP~\cite{alpa, megatron} uses all compute resources to process decoder blocks one at a time.
To implement TP, Figure~\ref{fig:Model_Parallelism}(a) shows that \att{} assigns each transformer decoder block across all CXL devices.
Figure~\ref{fig:Model_Parallelism}(b) illustrates the detailed mapping of a transformer block using TP.
The infrequent residual connection and normalization layers are confined within a single master CXL device.
Distributing the attention layer requires the frequent use of expensive \textit{AllReduce} collective communication primitive, which significantly increases the CXL communication overhead~\cite{megatron}.
Consequently, the attention layer is mapped to the master CXL device.

\begin{figure}[h]
    \centering
    % \includegraphics[width=\columnwidth]{Figure/Model_Parallelism.pdf}
    \includegraphics[width=8cm]{Figure/Model_Parallelism.pdf}
    \caption{(a) Tensor parallelism: each transformer block is assigned to multiple CXL devices. Prompts are processed sequentially. (b) In a transformer block, fully connected layers are spread across CXL devices, while other operations are confined to a single device.}
	\label{fig:Model_Parallelism}
\end{figure}

Prior to the execution of an FC layer, the embedding vector (\texttt{16KB} for Llama2-70B) is \textit{broadcast} from the master CXL device to all devices via the CXL switch.
This enables each device to locally perform GEMV on multiple rows of the weight matrix.
Following the execution of an FC layer, partial result vectors are \textit{gathered} to the master CXL device.
This approach optimizes the execution of FC layers across multiple devices, while reducing the communication overhead of TP through the CXL switch to only \texttt{135KB} data transfer for each transformer block of the Llama2-70B model.

\subsection{Hybrid Tensor-Pipeline Parallel Mapping}~\label{subsec:hybrid_parallel}

The TP and PP mappings focus either on inference latency or throughput.
However, balancing both can be crucial in real-world deployment scenarios when considering Quality of Service (QoS) requirements~\cite{mlperf-sla}.
We explore a hybrid TP-PP strategy to achieve this balance, where each transformer decoder is allocated to multiple consecutive CXL devices. For example, among $32$ devices, mapping each decoder to $32/4=8$ devices enables TP=8 and PP=4.
The embedding vectors are \textit{multicast} and \textit{gathered} by the master CXL device of each pipeline stage.
This configuration effectively reduces token decoding latency by utilizing compute resources from multiple CXL devices (TP), while also improving the throughput by processing multiple prompts in parallel (PP).

\subsection{Transformer Block Mapping} \label{subsec:block_mapping}

\att{} involves a fine-grained mapping of the transformer block onto CXL devices, PNM accelerators, and PIM channels.
This technique permits the complete execution of a transformer block within the CXL device, thereby eliminating the necessity for any interaction with the host system.
Figure~\ref{fig:LLaMA_mapping}(a) illustrates the operations within a Llama2 transformer block. 
Operations within the blue blocks are assigned to PIM channels, including GEMV in fully connected layers, vector dot product in RMSNorm, and element-wise multiplication in RMSNorm, SiLU, Softmax and Rotary Embedding, as detailed in Figure~\ref{fig:LLaMA_mapping}(b), (c), (d), and (e), respectively.
On the other hand, model-specific operations marked in orange, such as square root, division, Softmax, and vector addition in residual connections, are handled by the PNM's RISC-V cores and accelerators.
\att{} supports \textit{Grouped-Query Attention}~\cite{gqa} in Llama2-70B by unrolling GEMM to GEMV.

\begin{figure}[h]
	\centering
    \includegraphics[width=\columnwidth]{Figure/LLaMA_mapping_new.pdf}
    \caption{(a) Llama2-70B Transformer Block. Blue and orange operations are mapped to PIM and PNM PUs, respectively.
    (b)$\sim$(e) Operation mapping for RMSNorm, SiLU, SoftMax and Rotary embedding.}
	\label{fig:LLaMA_mapping}
\end{figure}

In Figure~\ref{fig:LLaMA_mapping}(d), the score dimension varies between $1$ and $4k$, accommodating the 4K sequence length in this example.
The embedding dimensions, as shown in Figure~\ref{fig:LLaMA_mapping}(b) and (c), are set to $8K$.
The rotary embedding process, depicted in Figure~\ref{fig:LLaMA_mapping}(e), begins with the RISC-V PNM cores transforming an attention head of dimension $128$ into $64$ groups of the complex number representations (\textit{e.g.,} $[a, b, c, d]$ to $[(a+jb), (c+jd)]$).
The PIM PUs within memory chips then multiply complex values and pre-loaded weights.
Finally, RISC-V PNM cores convert the computed results back to their real value representations.

\att{}'s PIM computations include three key operations.
This paragraph explains the execution of each operation within a GDDR6-PIM channel.
(a) \textit{GEMV}: The matrix is partitioned along its rows and distributed across all 16 banks. The vector is transferred to the Global Buffer. \texttt{MAC\_ABK} instructions then broadcast 256-bit vector segments from the Global Buffer to all near-bank PUs, retrieve 256-bit segments of the matrix rows from the banks, and perform MAC operations.
(b) \textit{Vector dot product}: In this operation, input vectors are stored in neighboring banks. \texttt{MAC\_ABK} instructions retrieve 256-bit segments from these banks and perform MAC operations. Throughout this process, only one of the two neighboring near-bank PUs is utilized.
(c) \textit{Element-wise multiplication}: Before this operation, input vectors are stored in two banks within each bank group, which consists of four banks. \texttt{EW\_MUL} instructions then retrieve 256-bit segments from these two banks, perform the multiplication, and store the results in another bank within the same bank group.

\subsection{End-to-End Model Mapping}~\label{subsec:e2e_model_mapping}

\att{} supports the end-to-end query execution in LLM inference tasks. In the prefill stage, \att{} processes tokens in the prompt one after another to fill out KV caches, using a similar approach to that in the decoding stage. 
Within each token, both input embeddings and transformer blocks are mapped to CXL devices using the mapping techniques introduced in Section~\ref{subsec:block_mapping}. In the decoding stage, after a series of transformer blocks, the top-k sampling operations are executed on the host CPU.

\subsection{Programming Model}

Users can specify the \att{} hardware configuration, including the number of PIM channels to utilize, and the number of pipeline stages. The tensor mapping strategy is determined by this configuration. \att{} library provides Python APIs to allocate memory space and load model parameters according to the model mapping strategy. 
These APIs also support commonly used LLM operations, such as \texttt{GEMV}, \texttt{LayerNorm}, \texttt{RMSNorm}, \texttt{RoPE}, \texttt{SoftMax}, \texttt{GeLU}, \texttt{SiLU}, \textit{etc.} 
\att{} uses an in-house compiler to generate arithmetic and data movement instructions illustrated in Section~\ref{ISA_Summary}. 


\begin{figure}[h]
    \centering
    % \includegraphics[width=\columnwidth]{Figure/Programming_model_code.pdf}
    \includegraphics[width=8cm]{Figure/Programming_model_code.pdf}
    \caption{Vector-matrix multiplication compilation}
    \label{fig:Programming_model_code}
\end{figure}

Figure~\ref{fig:Programming_model_code} shows an example of compiling \texttt{GEMV} to \att{} instructions. Initially, the operands are designated to particular memory spaces, \textit{i.e.}, the vector operands in the \rf{} and the matrix operands in PIM channels (lines 1 and 2). \att{} instructions are then generated based on input operands' dimensions and memory addresses. Subsequently, the vector is copied to the Global Buffers in the PIM channels with \texttt{WR\_GB} instructions (line 5). This is followed by a sequence of operations for each matrix row within the near-bank PIM PUs. The \texttt{WR\_BIAS} instruction sets up the accumulation registers (line 7). \texttt{MAC\_ABK} performs the multiply-accumulate operations across all near-bank PUs in the PIM channel (line 8). Finally, \texttt{RD\_MAC} retrieves the results from the accumulation registers (line 9).

\section{Methodology}

\begin{figure*}[t]
\begin{minipage}{0.63\textwidth}
\centering
\includegraphics[width=\linewidth]{imgs/architecture.pdf}
\label{fig:edeline_architecture}
\end{minipage}
\hfill
\begin{minipage}{0.35\textwidth}
\vspace{-2em}
\caption{\textbf{Framework Overview of EDELINE.}} 
      The model integrates three principal components: 
      (1) An U-Net-like \textit{Next-Frame Predictor} enhanced by adaptive group normalization and cross-attention mechanisms,
      % A \textit{Next-Frame Predictor} constructed with a U-Net architecture, enhanced by adaptive group normalization and cross-attention mechanisms, 
      (2) \textcolor{black}{A \textit{Recurrent Embedding Module} built on Mamba architecture for temporal sequence processing, and}
      % A \textit{Recurrent Embedding Module} built on Mamba architecture for temporal sequence processing through observation encoding and embedding layers, and
      (3) A \textit{Reward/Termination Predictor} implemented through linear layers. The EDELINE framework uses shared hidden representations across the components for efficient world model learning.
\end{minipage}\vspace{-1em}
\end{figure*}

Conventional diffusion-based world models \cite{alonso2024diamond} demonstrate promise in learning environment dynamics yet face fundamental limitations in memory capacity and horizon prediction consistency. To address these challenges, this paper presents EDELINE, as illustrated in Fig~\ref{fig:edeline_architecture}, a unified architecture that integrates state space models (SSMs) with diffusion-based world models. EDELINE's core innovation lies in its integration of SSMs for encoding sequential observations and actions into hidden embeddings, which a diffusion model then processes for future frame prediction. This hybrid design maintains temporal consistency while generating high-quality visual predictions. A Convolutional Neural Network based actor processes these predicted frames to determine actions, thus enabling autoregressive generation of imagined trajectories for policy optimization.

% This section presents EDELINE's world model architecture and training methodology, with emphasis on the SSM integration capabilities for enhanced memory capacity and imagination consistency in diffusion-based world models. The subsequent discussion details the mechanisms of actor-critic utilization of imagined trajectories in pixel space for efficient and effective policy learning.

\subsection{World Model Learning}

The core architecture of EDELINE consists of a \textit{Recurrent Embedding Module {(REM)}} $f_\phi$ that processes the history of observations and actions $(o_0, a_0, o_1, a_1, ..., o_t, a_t)$ to generate a hidden embedding $h_t$ through recursive computation. This embedding enables the \textit{Next-Frame Predictor} $p_\phi$ to generate predictions of the subsequent observation $\hat{o}_{t+1}$. The architecture further incorporates dedicated \textit{Reward and Termination Predictors} to estimate the reward $\hat{r}_t$ and episode termination signal $\hat{d}_t$ respectively. The trainable components of EDELINE's world model are formalized as: \vspace{-2em}
\begin{itemize} [itemsep=3pt, parsep=0pt]
    \item Recurrent Embedding Module: $h_t = f_\phi(h_{t-1}, o_t, a_t)$
    \item Next-Frame Predictor: $\hat{o}_{t+1} \sim p_\phi(\hat{o}_{t+1}|h_t)$
    \item Reward Predictor: $\hat{r}_t \sim p_\phi(\hat{r}_t|h_t)$
    \item Termination Predictor: $\hat{d}_t \sim p_\phi(\hat{d}_t|h_t)$
\end{itemize}

\subsubsection{Recurrent Embedding Module 
% \josout{(REM)}
}

While DIAMOND, the current state-of-the-art in diffusion-based world models, relies on a fixed context window of four previous observations and actions sequence, the proposed EDELINE architecture advances beyond this limitation through a recurrent architecture for extended temporal sequence processing. Specifically, we provide theoretical evidence in Theorem~\ref{the:information_retention_superiority} to support that recurrent embedding can preserve more information compared with stacked frames inputs. At each timestep $t$, the Recurrent Embedding Module processes the current observation-action pair $(o_t, a_t)$ to update a hidden state $h_t = f_\phi(h_{t-1}, o_t, a_t)$.
% , expressed as follows:
% \begin{equation}
% h_t = f_\phi(h_{t-1}, o_t, a_t).
% \end{equation}
The implementation of REM utilizes Mamba~\cite{gu2024mamba}, an SSM architecture that offers distinct advantages for world modeling. This architectural selection is motivated by the limitations of current sequence processing methods in deep learning. Self-attention-based Transformer architectures, despite their strong modeling capabilities, suffer from quadratic computational complexity which impairs efficiency. Traditional recurrent architectures including Long Short-Term Memory (LSTM)~\cite{HochSchm97} and Gated Recurrent Unit (GRU)~\cite{69e088c8129341ac89810907fe6b1bfe} experience gradient instability issues that affect dependency learning. In contrast, SSMs provide an effective alternative through linear-time sequence processing coupled with robust memorization capabilities via their state-space formulation. The adoption of Mamba emerges as a promising choice due to its demonstrated effectiveness in modeling temporal patterns across various sequence modeling tasks. Section~\ref{subsec:ablation_studies} presents a comprehensive ablation study that evaluates different architectural choices for the REM.

\subsubsection{Next-Frame Predictor}

While motivated by DIAMOND's success in diffusion-based world modeling, EDELINE introduces significant architectural innovations in its Next-Frame Predictor to enhance temporal consistency and feature integration. At time step $t$, the model conditions on both the last $L$ frames and the hidden embedding $h_t$ from the Recurrent Embedding Module to predict the next frame $\hat{o}_{t+1}$. The predictive distribution $p_\phi(o^0_{t+1}|h_t)$ is implemented through a denoising diffusion process, where $D_\phi$ functions as the denoising network. Let $y_t^{\tau} = (\tau, o^0_{t-L+1}, ..., o^0_t, h_t)$ represent the conditioning information, where $\tau$ represents the diffusion time. The denoising process can be formulated as $o^0_{t+1} = D_\phi(o^{\tau}_{t+1}, y_t^{\tau}).$
% then be formulated as follows:
% \begin{equation}
%     o^0_{t+1} = D_\phi(o^{\tau}_{t+1}, y_t^{\tau}).
% \end{equation}
To effectively integrate both visual and hidden information, $D_\phi$ employs two complementary conditioning mechanisms. First, the architecture incorporates \cite{AGN} layers within each residual block to condition normalization parameters on the hidden embedding $h_t$ and diffusion time $\tau$, which establishes context-aware feature normalization \cite{AGN}. This design significantly extends DIAMOND's implementation, which limits AGN conditioning to $\tau$ and action embeddings only. The second key innovation introduces cross-attention blocks inspired by Latent Diffusion Models (LDMs), which utilize $h_t$ and $\tau$ as context vectors. The UNet's feature maps generate the query, while $h_t$ and $\tau$ project to keys and values. This novel attention mechanism, which is absent in DIAMOND, facilitates the fusion of spatial-temporal features with abstract dynamics encoded in $h_t$. The observation modeling loss $\mathcal{L}_{\text{obs}}(\phi)$ is defined based on Eq.~(\ref{eq:d_loss}), and can be formulated as follows:
\begin{equation}
\mathcal{L}_{\text{obs}}(\phi) = \mathbb{E}\left[\|D_\phi(o^{\tau}_{t+1}, y_t^{\tau}) - o^0_{t+1}\|^2\right].
\end{equation}
\subsubsection{Reward / Termination Predictor}
EDELINE advances beyond DIAMOND's architectural limitations through an integrated approach to reward and termination prediction. Rather than employing separate neural networks, EDELINE leverages the rich representations from its REM. The reward and termination predictors are implemented as multilayer perceptrons (MLPs) that utilize the deterministic hidden embedding $h_t$ as their conditioning input. This architectural unification enables efficient representation sharing across all predictive tasks. EDELINE processes both reward and termination signals as probability distributions conditioned on the hidden embedding: $p_\phi(\hat{r}_t|h_t)$ and $p_\phi(\hat{d}_t|h_t)$ respectively. The predictors are optimized via negative log-likelihood losses, expressed as:
\begin{equation}
\mathcal{L}_{\text{rew}}(\phi) = -\ln p_\phi(r_t|h_t),
% \end{equation}
% \begin{equation}
\mathcal{L}_{\text{end}}(\phi) = -\ln p_\phi(d_t|h_t).
\end{equation}
This unified architectural design represents an improvement over DIAMOND's separate network approach, where reward and termination predictions require independent representation learning from the world model. The integration of these predictive tasks with shared representations enables REM to learn dynamics that encompass all relevant aspects of the environment. The architectural efficiency facilitates enhanced learning effectiveness and better performance.
\vspace{-0.5em}

\begin{table*}[ht!]  
  \vspace{-1em}
  \caption{Game scores and overall human-normalized scores on the $26$ games in the Atari $100$k benchmark. Results are averaged over 3 seeds, with bold numbers indicating the best performing method for each metric.}
  \label{table:atari_100k}
  \vspace{0.1cm}
  \centering
  \resizebox{\textwidth}{!}{\begin{tabular}{lrrrrrrrrrr}
Game                &  Random    &  Human     &  SimPLe    &  TWM                &  IRIS              &  STORM              &  DreamerV3   &  Drama     &  DIAMOND           &  EDELINE (ours)      \\
\midrule
Alien               &  227.8     &  7127.7    &  616.9     &  674.6              &  420.0             &  \textbf{983.6}     &  959.4       &  820      &  744.1             &  974.6               \\
Amidar              &  5.8       &  1719.5    &  74.3      &  121.8              &  143.0             &  204.8              &  139.1       &  131      &  225.8             &  \textbf{299.5}      \\
Assault             &  222.4     &  742.0     &  527.2     &  682.6              &  1524.4            &  801.0              &  705.6       &  539      &  \textbf{1526.4}   &  1225.8              \\
Asterix             &  210.0     &  8503.3    &  1128.3    &  1116.6             &  853.6             &  1028.0             &  932.5       &  1632     &  3698.5            &  \textbf{4224.5}     \\
BankHeist           &  14.2      &  753.1     &  34.2      &  466.7              &  53.1              &  641.2              &  648.7       &  137      &  19.7              &  \textbf{854.0}      \\
BattleZone          &  2360.0    &  37187.5   &  4031.2    &  5068.0             &  13074.0           &  \textbf{13540.0}   &  12250.0     &  10860    &  4702.0            &  5683.3              \\
Boxing              &  0.1       &  12.1      &  7.8       &  77.5               &  70.1              &  79.7               &  78.0        &  78       &  86.9              &  \textbf{88.1}       \\
Breakout            &  1.7       &  30.5      &  16.4      &  20.0               &  83.7              &  15.9               &  31.1        &  7        &  132.5             &  \textbf{250.5}      \\
ChopperCommand      &  811.0     &  7387.8    &  979.4     &  1697.4             &  1565.0            &  1888.0             &  410.0       &  1642     &  1369.8            &  \textbf{2047.3}     \\
CrazyClimber        &  10780.5   &  35829.4   &  62583.6   &  71820.4            &  59324.2           &  66776.0            &  97190.0     &  52242    &  99167.8           &  \textbf{101781.0}   \\
DemonAttack         &  152.1     &  1971.0    &  208.1     &  350.2              &  \textbf{2034.4}   &  164.6              &  303.3       &  201      &  288.1             &  1016.1              \\
Freeway             &  0.0       &  29.6      &  16.7      &  24.3               &  31.1              &  33.5               &  0.0         &  15       &  33.3              &  \textbf{33.8}       \\
Frostbite           &  65.2      &  4334.7    &  236.9     &  \textbf{1475.6}    &  259.1             &  1316.0             &  909.4       &  785      &  274.1             &  286.8               \\
Gopher              &  257.6     &  2412.5    &  596.8     &  1674.8             &  2236.1            &  \textbf{8239.6}    &  3730.0      &  2757     &  5897.9            &  6102.3              \\
Hero                &  1027.0    &  30826.4   &  2656.6    &  7254.0             &  7037.4            &  11044.3            &  11160.5     &  7946     &  5621.8            &  \textbf{12780.8}    \\
Jamesbond           &  29.0      &  302.8     &  100.5     &  362.4              &  462.7             &  509.0              &  444.6       &  372      &  427.4             &  \textbf{784.3}      \\
Kangaroo            &  52.0      &  3035.0    &  51.2      &  1240.0             &  838.2             &  4208.0             &  4098.3      &  1384     &  \textbf{5382.2}   &  5270.0              \\
Krull               &  1598.0    &  2665.5    &  2204.8    &  6349.2             &  6616.4            &  8412.6             &  7781.5      &  9693     &  8610.1            &  \textbf{9748.8}     \\
KungFuMaster        &  258.5     &  22736.3   &  14862.5   &  24554.6            &  21759.8           &  26182.0            &  21420.0     &  17236    &  18713.6           &  \textbf{31448.0}    \\
MsPacman            &  307.3     &  6951.6    &  1480.0    &  1588.4             &  999.1             &  \textbf{2673.5}    &  1326.9      &  2270     &  1958.2            &  1849.3              \\
Pong                &  -20.7     &  14.6      &  12.8      &  18.8               &  14.6              &  11.3               &  18.4        &  15       &  20.4              &  \textbf{20.5}       \\
PrivateEye          &  24.9      &  69571.3   &  35.0      &  86.6               &  100.0             &  \textbf{7781.0}    &  881.6       &  90       &  114.3             &  99.5                \\
Qbert               &  163.9     &  13455.0   &  1288.8    &  3330.8             &  745.7             &  4522.5             &  3405.1      &  796      &  4499.3            &  \textbf{6776.2}     \\
RoadRunner          &  11.5      &  7845.0    &  5640.6    &  9109.0             &  9614.6            &  17564.0            &  15565.0     &  14020    &  20673.2           &  \textbf{32020.0}    \\
Seaquest            &  68.4      &  42054.7   &  683.3     &  774.4              &  661.3             &  525.2              &  618.0       &  497      &  551.2             &  \textbf{2140.1}     \\
UpNDown             &  533.4     &  11693.2   &  3350.3    &  \textbf{15981.7}   &  3546.2            &  7985.0             &  7567.1      &  7387     &  3856.3            &  5650.3              \\
\midrule
\#Superhuman (↑)    &  0         &  N/A       &  1         &  8                  &  10                &  10                 &  9           &  7         &  11                &  \textbf{13}         \\
Mean (↑)            &  0.000     &  1.000     &  0.332     &  0.956              &  1.046             &  1.266              &  1.124       &  0.989     &  1.459             &  \textbf{1.866}      \\
Median (↑)          &  0.000     &  1.000     &  0.134     &  0.505              &  0.289             &  0.580              &  0.485       &  0.270     &  0.373             &  \textbf{0.817}      \\
IQM (↑)             &  0.000     &  1.000     &  0.130     &  0.459              &  0.501             &  0.636              &  0.487       &  -         &  0.641             &  \textbf{0.940}      \\
Optimality Gap (↓)  &  1.000     &  0.000     &  0.729     &  0.513              &  0.512             &  0.433              &  0.510       &  -         &  0.480             &  \textbf{0.387}      \\

  \end{tabular}}
  \vspace{-1.5em}
\end{table*}

\subsubsection{EDELINE World Model Training}
The world model integrates an innovative end-to-end training strategy with a self-supervised approach. EDELINE extends the harmonization technique from HarmonyDream \cite{ma2024harmonydream} through the adoption of harmonizers $w_o$ and $w_r$, which dynamically balance the observation modeling loss $\mathcal{L}_{\text{obs}}(\phi)$ and reward modeling loss $\mathcal{L}_{\text{rew}}(\phi)$. This adaptive mechanism results in the total loss function $\mathcal{L}(\phi)$:
\begin{equation}
\label{eq:total_loss}
\begin{split}
\mathcal{L}(\phi) = w_0\mathcal{L}_{\text{obs}}(\phi) &+ w_r\mathcal{L}_{\text{rew}}(\phi) + \mathcal{L}_{\text{end}}(\phi) \\
    &+ \log(w_o^{-1}) + \log(w_r^{-1})
\end{split}
\end{equation}
To optimize computational efficiency while ensuring robust learning, the Next-Frame Predictor learns to utilize hidden embeddings from any timestep through strategic random sampling for $\mathcal{L}_{\text{obs}}(\phi)$. For a sequence of length $T$, 
% this computation 
it follows:
\begin{equation}
\mathcal{L}_{\text{obs}}(\phi) = \|\hat{o}^0_{t+1} - o_{t+1}\|^2,
\end{equation}
where $i \sim \text{Uniform}\{1,2,\ldots,T-1\}, \quad \hat{o}^0_{t+1} \sim p(\hat{o}^0_{t+1} \mid h_t).$ For reward and termination prediction, EDELINE utilizes cross-entropy losses averaged over the sequence, which can be formulated as:
% \begin{equation}
    $\mathcal{L}_{\text{rew}}(\phi) = \frac{1}{T}\sum_{t=1}^T \text{CrossEnt}(\hat{r}_t, r_t),$
% \end{equation}
% \begin{equation}
    $\mathcal{L}_{\text{end}}(\phi) = \frac{1}{T}\sum_{t=1}^T \text{CrossEnt}(\hat{d}_t, d_t).$
% \end{equation}
This unified training approach, combining random sampling strategies with dynamic loss harmonization, demonstrates superior efficiency compared to DIAMOND's separate network methodology, as validated in our results presented in Section 6. Moreover, the quantitative analysis presented in Appendix~\ref{appendix:training_time_profile} reveals substantial reductions in world model training duration.
\vspace{-0.5em}

\subsection{Agent Behavior Learning}
To enable fair comparison and demonstrate the effectiveness of EDELINE's world model architecture, the agent architecture adopts the same optimization framework as DIAMOND. Specifically, the agent integrates policy $\pi_\theta$ and value $V_\theta$ networks with REINFORCE value baseline and Bellman error optimization using $\lambda$-returns~\cite{alonso2024diamond}. The training framework executes a procedure with three key phases: experience collection, world model updates, and policy optimization. This method, as formalized in Algorithm~\ref{alg:edeline}, follows the established paradigms in model-based RL literature~\cite{Kaiser2020SimPLe,Hafner2020Dreamer,micheli2023iris,alonso2024diamond}. To ensure reproducibility, we provide extensive details in the Appendix, with documentation of objective functions, and the hyperparameter configurations in Appendices~\ref{appendix:rl_objectives}, \ref{appendix:hyper}, respectively. \vspace{-2em}

\begin{table*}[!t]
    \centering
    \resizebox{\textwidth}{!}{%
        \begin{tabular}{lccccc|c}
            \toprule
            & \textbf{Microbiology} & \textbf{Chemistry} & \textbf{Economics} & \textbf{Sociology} & \textbf{US History} & \textbf{Average} \\
            \midrule
            Base score & 0.46 & 0.09 & 0.00 & 0.61 & 0.03 & 0.24 \\
            \midrule
            Zero-shot & 0.62 (+0.16) & 0.40 (+0.31) & 0.40 (+0.40) & 0.61 (+0.00) & 0.19 (+0.16) & 0.44 (+0.20) \\
            Few-shot & 0.62 (+0.16) & \underline{0.45} (+0.36) & \underline{0.47} (+0.47) & 0.62 (+0.01) & 0.16 (+0.13) & 0.46 (+0.22) \\
            Chain-of-thought & 0.61 (+0.15) & \underline{0.45} (+0.36) & 0.46 (+0.46) & 0.61 (+0.00) & 0.19 (+0.16) & 0.46 (+0.22) \\
            Bloom-based & 0.57 (+0.11) & 0.37 (+0.28) & 0.29 (+0.29) & 0.62 (+0.01) & 0.22 (+0.19) & 0.41 (+0.17) \\
            \midrule
            SFT (Subject-Specific) & 0.65 (+0.19) & 0.24 (+0.15) & 0.46 (+0.46) & \underline{0.64} (+0.03) & 0.20 (+0.17) & 0.44 (+0.20) \\
            SFT (Cross-Subject) & 0.59 (+0.13) & 0.21 (+0.12) & \underline{0.47 (+0.47)} & 0.63 (+0.02) & \underline{0.26} (+0.23) & 0.43 (+0.19) \\
            \midrule
            \textsc{QUEST} (Subject-Specific) & \bf 0.76 (+0.30) & \bf 0.46 (+0.37) & \bf 0.58 (+0.58) & \bf 0.65 (+0.04) & \bf 0.31 (+0.28) & \bf 0.55 (+0.31) \\
            \textsc{QUEST} (Cross-Subject) & \underline{0.73} (+0.27) & \underline{0.41} (+0.32) & \underline{0.47} (+0.47) & \bf 0.65 (+0.04) & 0.25 (+0.22) & \underline{0.50} (+0.26) \\
            \bottomrule
        \end{tabular}
    }
    \caption{\textbf{End-of-chapter exam score} results of different question generation approaches across various subjects. 
    \ours produces models that generate questions that lead to the highest scores on all subjects.  
    Gain values (in parentheses) are calculated as the increase from the base score. The highest and second highest scores per column are \textbf{bolded} and \underline{underlined}, respectively.}
    \label{tab:question-gen-results}
\end{table*}


\section{Experimental Results}
In this section, we first compare the overall performance of all question generation baselines based on the learner's exam score (\secref{ssec:overall-performance}).  
Next, we analyze evaluation metrics by examining their correlations with utility and assessing the impact of optimizing models on high-scoring questions for each metric (\secref{ssec:evaluation-metrics}).
We then conduct a qualitative analysis of high-utility questions to understand their characteristics (\secref{ssec:high-utility-question}). 
Finally, we perform ablation studies on the framework by varying the criteria for selecting high-utility questions for training and replacing \texttt{gpt-4o-mini} to \texttt{gpt-4o} (\secref{ssec:rs-analysis}-\secref{ssec:model-variants}).



\subsection{Overall Performance}
\label{ssec:overall-performance}
Table~\ref{tab:question-gen-results} presents the learner's exam performance of different question generators, measured by exam scores using all generated question-answer pairs (\secref{ssec:quest-evaluation}).
Here are findings:
(1) \textbf{Prompting techniques (Few-shot, CoT, Bloom-based)} offer only marginal performance gains. While advanced prompting enhances reasoning and task accuracy~\cite{brown2020language, wei2022chain, zhou2024self}, it does not directly optimize utility, which reflects real-world impact—how well generated questions enhance learning. Without explicit selection or optimization, prompting cannot systematically improve this measure;
(2) \textbf{SFT} shows no performance gains, indicating that while it learns the style of exam questions, it fails to generate questions that enhance learner understanding.
This highlights the key distinction between producing syntactically valid questions and generating those that effectively promote learning;
(3) \textbf{QUEST} achieves the highest performance gain, improving by approximately 20\% on average.
Performance gap between subject-specific and cross-subject rejection sampling suggests that the definition of a ``high-utility'' question varies by domain.
The results indicate that outcome-based learning is most effective when applied within a specific domain.


\subsection{Evaluation Metrics Analysis}
\label{ssec:evaluation-metrics}
\paragraph{Correlation.}
\begin{table}[!t]
    \centering
    \resizebox{\columnwidth}{!}{%
        \begin{tabular}{ll
        cc}
            \toprule
            Metric 1 & Metric 2 & \textbf{Spearman correlation} & \textbf{p-value} \\
            \midrule
            \textbf{Utility} & Saliency & 0.097 & 0.003 \\
            \textbf{Utility} & EIG  & -0.022 & 0.512 \\
            Saliency & EIG  & 0.030 & 0.363 \\
            \bottomrule
        \end{tabular}
    }
    \caption{\textbf{Spearman correlation between metrics.} Utility shows a weak correlation with saliency and EIG, showing that it is independent of these indirect metrics.}
    \label{tab:correlation_results}
\end{table}
To analyze the relationship between \textit{utility} and existing metrics (\textit{saliency}, \textit{EIG}), we estimate all three metrics on generated questions from the training set.
Table~\ref{tab:correlation_results} shows that both saliency and EIG have weak correlations with utility.
While saliency has a weak but statistically significant correlation with utility, EIG shows no meaningful relationship.
This indicates that existing indirect metrics may not accurately reflect a question’s impact on learning outcomes.

\paragraph{Optimization on Indirect Metrics.}
\begin{table*}[!t]
    \centering
    \small
    \resizebox{\textwidth}{!}{%
        \begin{tabular}{lccc ccc ccc ccc ccc}
            \toprule
            & \multicolumn{3}{c}{\textbf{Microbiology}} & \multicolumn{3}{c}{\textbf{Chemistry}} & \multicolumn{3}{c}{\textbf{Economics}} & \multicolumn{3}{c}{\textbf{Sociology}} & \multicolumn{3}{c}{\textbf{US History}} \\
            \cmidrule(lr){2-4} \cmidrule(lr){5-7} \cmidrule(lr){8-10} \cmidrule(lr){11-13} \cmidrule(lr){14-16}
            \textbf{Train Metric} & \textbf{Utility} & \textbf{Saliency} & \textbf{EIG} & \textbf{Utility} & \textbf{Saliency} & \textbf{EIG} & \textbf{Utility} & \textbf{Saliency} & \textbf{EIG} & \textbf{Utility} & \textbf{Saliency} & \textbf{EIG} & \textbf{Utility} & \textbf{Saliency} & \textbf{EIG} \\
            \midrule
            $utility > 0.1$ & \textbf{0.76} & 4.27 & -0.18 & \textbf{0.46} & 4.65 & -0.20 & \textbf{0.58} & \textbf{4.70} & -0.04 & \textbf{0.65} & \textbf{4.49} & -0.02 & \textbf{0.31} & 4.65 & \bf -0.01 \\
            $saliency = 5$  & 0.73 & \textbf{4.42} & -0.24 & 0.39 & \textbf{4.46} & -0.22 & 0.46 & 4.66 & -0.08 & 0.64 & \textbf{4.49} & -0.03 & 0.23 & \textbf{4.68} & -0.02 \\
            $EIG > 0$      & 0.61 & 4.21 & \textbf{-0.17} & 0.32 & 4.40 & \textbf{-0.09} & 0.47 & 4.65 & \textbf{0.01} & 0.62 & 4.46 & \textbf{0.01} & 0.21 & 4.65 & \bf -0.01 \\
            \bottomrule
        \end{tabular}
    }
    \caption{\textbf{End-of-chapter exam scores (utility), average saliency, and average EIG of generated questions} for different \textsc{QUEST}-optimized models trained on datasets filtered by different selection criteria.}
    \label{tab:quest-indirect}
\end{table*}
To further investigate the impact of indirect metrics on question generation performance, we compare the results of \textsc{QUEST} when trained using different selection criteria: (1) only questions with a \textit{utility} score greater than 0.1 (Ours), (2) only questions with a \textit{saliency} score of 5, and (3) only questions with an \textit{expected information gain (EIG)} greater than 0.
We evaluate the generated questions based on their overall utility (\textit{i.e.,} end-of-chapter exam scores), as well as their average saliency and EIG, to assess the question generator’s performance across different quality metrics.
Table~\ref{tab:quest-indirect} shows that while saliency- and EIG-based training improves their respective scores, it does not enhance utility. 
In contrast, the utility-trained model consistently achieves the highest utility across all subjects and even improves some indirect metrics (\textit{e.g.,} it matches or outperforms saliency-based training in Economics and Sociology).
Furthermore, utility-based training outperforms EIG-based training on saliency and saliency-based training on EIG, demonstrating its broader effectiveness. 
These results emphasize that optimizing for indirect metrics does not improve real-world learning, whereas utility-driven training yields the best overall performance.




\subsection{High Utility Questions Analysis}
\label{ssec:high-utility-question}
\paragraph{Overlap with exam questions.}
To evaluate the relationship between generated high-utility questions and exam questions, we measured their semantic and lexical similarity.
For each generated question, we computed embedding similarity using \texttt{text-3-embedding-small}\footnote{\href{https://platform.openai.com/docs/guides/embeddings/}{https://platform.openai.com/docs/guides/embeddings/}} for semantic overlap and the ROUGE score for lexical overlap with all exam questions in the same chapter.
We then assess the correlation between utility and the most similar exam question based on these measures.
The correlation between utility and semantic similarity is 0.25 (p < 0.001), indicating a weak positive relationship, while the correlation with ROUGE is nearly zero at 0.04 (p < 0.01).
These findings suggest that high-utility questions are not simple rephrasings of exam questions but introduce novel concepts that enhance learning beyond surface-level similarity.

\paragraph{Qualitative analysis.}
Qualitative question examples do not exhibit clear patterns in question style (see Appendix~\ref{appendix:qualitative-examples}).
An interesting observation is that Bloom's taxonomy, which categorizes cognitive depth based on question type—where "what" questions typically involve simple recall, while "why" and "how" questions require deeper processing—does not strongly correlate with utility.
Using Bloom's taxonomy as a cognitive depth scale (Likert 1-6), the correlation between utility and cognitive depth is 0.12 (p < 0.001), indicating a weak positive relationship.


\subsection{Rejection Sampling Analysis}
\label{ssec:rs-analysis}
Filtering for high-utility questions through rejection sampling is crucial for improving question generation. 
As shown in Figure~\ref{fig:rs_analysis}, increasing the utility threshold enhances question quality, leading to higher exam scores.
However, stricter filtering reduces the available training data, posing challenges for model training. 
These results suggest that increasing the dataset size while applying a higher threshold could further boost performance.
\begin{figure}[!t]
    \centering
    \begin{minipage}{\columnwidth}
    \centering
    \includegraphics[width=\columnwidth]{figures/rs_analysis.pdf}
    \end{minipage}
    \caption{\textbf{Impact of threshold} in \ours on end-of-chapter exam scores for Chemistry.}
    \label{fig:rs_analysis}
    \vspace{-0.5cm}
\end{figure}

\subsection{Model Variants Analysis}
\label{ssec:model-variants}
\begin{table*}[!t]
    \centering
    \resizebox{\textwidth}{!}{%
    \begin{tabular}{c|ccc|ccccc}
        \toprule
        & \textbf{QG ($M_q$)} & \textbf{AG ($M_a$)} & \textbf{RS ($M_l$)} & \textbf{Microbiology} & \textbf{Chemistry} & \textbf{Economics} & \textbf{Sociology} & \textbf{US History} \\
        \midrule
        \multicolumn{1}{c|}{\multirow{4}{*}{Zero-Shot}} 
        & \texttt{gpt-4o-mini} & \texttt{gpt-4o-mini} & \texttt{gpt-4o-mini} & 0.620 & 0.414 & 0.398 & 0.609 & 0.233 \\
        & \texttt{gpt-4o} & \texttt{gpt-4o-mini} & \texttt{gpt-4o-mini} & 0.681 & 0.457 & 0.466 & 0.634 & 0.180 \\
        & \texttt{gpt-4o-mini} & \texttt{gpt-4o} & \texttt{gpt-4o-mini} & 0.682 & 0.422 & 0.480 & 0.634 & 0.232 \\
        & \texttt{gpt-4o-mini} & \texttt{gpt-4o-mini} & \texttt{gpt-4o} & 0.710 & 0.173 & 0.476 & 0.564 & 0.263 \\
        \midrule
        \textsc{QUEST} & \texttt{gpt-4o-mini} & \texttt{gpt-4o-mini} & \texttt{gpt-4o-mini} & \bf 0.756 &	\bf 0.457 &	\bf 0.582 &	\bf 0.649 &	\bf 0.311 \\
        \bottomrule
    \end{tabular}
    }
    \vspace{-0.2cm}
    \caption{\textbf{End-of-chapter exam scores} for different model sizes across various subjects and modules.}
        \vspace{-0.2cm}

    \label{tab:model-variants}
\end{table*}

To evaluate the robustness of our framework and the impact of model size on different components, we conduct experiments to analyze how a larger model affects each module.
Table~\ref{tab:model-variants} presents results for different configurations of the question generator ($M_q$), answer generator ($M_a$), and reader simulator (\textit{i.e.,} learner $M_l$). 
The baseline corresponds to the \texttt{zero-shot} setting in Table~\ref{tab:question-gen-results}. 
In the $M_a$ experiment, we use the same questions from the \texttt{zero-shot} setting but generate new answers.
For the reader simulator experiment, we keep the same questions and answers from \texttt{zero-shot} and re-run the simulation only.

Our main findings are the following:
(1) \textbf{Question Generator}: A larger model (\texttt{gpt-4o}) improves the utility score by 5.7\%. However, it still underperforms compared to the smaller, utility-optimized model (\texttt{gpt-4o-mini}) by 12.3\%;
(2) \textbf{Answer Generator}: A larger model improves performance by 7.1\%, suggesting that higher answer quality provides additional information to the QA pair. 
However, it remains 11\% behind the optimized \texttt{gpt-4o-mini} in utility.
(3) \textbf{Reader Simulator}: Using a larger model (\texttt{gpt-4o}) as the reader simulator leads to mixed results, with performance gains in some subjects but a sharp decline in Chemistry. 
This suggests that larger models may introduce different reasoning strategies or evaluation biases, leading to inconsistencies in scoring. 
Additionally, since our framework is optimized for \texttt{gpt-4o-mini}, the larger model may not align well with the training dynamics. 
These results highlight the importance of consistency in simulation for reliable utility estimation.

% equipped with a larger model (\texttt{gpt-4o}) is expected to make more accurate simulation of how much generated questions really impact to the student understanding on exams.
% There is consistent increasing gap on microbiology, economics, sociology, us history while there is a huge gap in decreasing gap on chemistry, compared to smaller model (\texttt{gpt-4o-mini})



% \begin{itemize}
%     \item \textbf{Stronger question generators improve performance in some subjects but not all.} Upgrading $M_q$ from GPT-4o-mini to GPT-4o enhances performance in Microbiology and Economics but degrades US History, suggesting that question quality improvement is domain-dependent.
%     \item \textbf{Answer generator quality has a moderate effect on utility.} Stronger answer generators help in certain subjects (e.g., Economics) but have minimal impact in others (e.g., Chemistry).
%     \item \textbf{Evaluator model strength influences overall results but can introduce variability.} Using a stronger evaluator (GPT-4o) improves utility estimates in Microbiology and Economics but drastically reduces performance in Chemistry. This suggests that evaluators need to be carefully tuned to ensure robust and reliable assessments across different domains.
% \end{itemize}



\section{Related Work on Cultural Change}
\label{sec:related_work}

%Understanding if and how distributional models understand semantic knowledge (e.g., is ``dog'' a mammal) is an important research question. For example, \citet{rubinstein-etal-2015-well} show that static distributional embeddings capture well taxonomical properties, but do not perform well in general attributive semantics (e.g., predicting the color of something). Recently, large language models were shown to \textit{have} some knowledge about concepts~\cite{dalvi2022discovering,ettinger2020bert,weir2020probing,petroni2019language}, including word sense in their contextualized embeddings~\cite{reif2019visualizing}. However, their encoded knowledge is static and lacks structure and domain specificity \cite{brandl}.

%\cnote{..}

%\paragraph{Word similarity.} Cosine similarity is a standard measure of semantic similarity, but its effectiveness is limited by the representational geometry of learned embeddings. The anisotropy of contextualized embedding spaces causes a small number of rogue dimensions to dominate cosine similarity computations \citep{timkey2021all}.
%Further, cosine similarity underestimates the semantic similarity of high-frequency words \citep{zhou2022problems}, heavily depends on the regularization techniques used during training \citep{steck2024cosine} and often fails in capturing human interpretation \cite{sitikhu2019comparison}. The proposed \wc enables a similarity measure that sidesteps these limitations via softmax-normalized dot products.



% % Recent work has explored the limits of cosine similarity


%\paragraph{Asymmetry.} By definition, cosine similarity is a symmetric metric that cannot capture the asymmetry of semantic relationships \citep{vilnis2014word}. Efforts to account for this caveat show partial successes, emphasizing the inherent symmetrical nature of cosine similarity using some language model embeddings \citet{zhang2021circles, rodriguez2020word}.

%\paragraph{Word embeddings for semantic and cultural change.} 
Both static and contextualized embedding spaces contain semantically meaning dimensions that align with high-level linguistic and cultural features \citep{bolukbasi2016man, DBLP:journals/corr/abs-1906-02715}. These embeddings have enabled a large number of quantitative analyses of temporal shifts in meaning and links to cultural or social scientific variables. For example early on, using static embeddings, \citet{hamilton2016cultural} measured linguistic drifts in global semantic space as well as cultural shifts in particular local semantic neighborhoods. \citet{garg2018word} demonstrated that changes in word embeddings correlated with demographic and occupation shifts through the 1900s.

Analyzes of contextualized embeddings have identified semantic axes based on pairs of ``seed words'' or ``poles'' \citep{soler2020bert, lucy2022discovering, grand2022semantic}. Across the temporal dimension, such axes can measure the evolution of gender and class \citep{kozlowski2019geometry}, internet slang \citep{keidar-etal-2022-slangvolution}, and more \citep{madani2023measuring, lyu2023representation, erk2024adjusting}. \citet{bravzinskas2017embedding} proposes a probabilistic measure for lexical similarity. 

It's also instructive to consider the similarity of our method  with tasks like word sense disambiguation (WSD) and named entity recognition (NER). The central idea behind \wc of mapping from embeddings to categories are also found in NER and WSD. What differs is the dynamic nature of the categories. Where NER focuses on pre-defined concept hierarchies and WSD on pre-defined senses per word,  \wc  focuses on a coherent but dynamic grouping of words that is interpretable for a given task.

% an important direction for future work in computational social science (see 3.1.10 in \citet{ziems2023can}).

% \subsection{Semantic change}


% Survey \citep{de2024survey} (should prob look more)

% Diachronic word embeddings 

%  \citet{di2019training}






% Similarly to named entity recognition, \ac{ourmethod} attempts to map words to classes that \textit{might be} types (such as ORGANIZATION or PLACE). However, \ac{ourmethod} is partially self-supervised and it is entirely user-driven.

% Similarly to semantic change detection, \ac{ourmethod} attempts to capture the usage of a word in different contexts. However, \ac{ourmethod} offers the possibility of defining the axis (the seed words) onto which the user wants to project words. 


\section{Conclusion \& Future Work}\label{conclusion}
This work presents XAMBA, the first framework optimizing SSMs on COTS NPUs, removing the need for specialized accelerators. XAMBA mitigates key bottlenecks in SSMs like CumSum, ReduceSum, and activations using ActiBA, CumBA, and ReduBA, transforming sequential operations into parallel computations. These optimizations improve latency, throughput (Tokens/s), and memory efficiency. Future work will extend XAMBA to other models, explore compression, and develop dynamic optimizations for broader hardware platforms.



% This work introduces XAMBA, the first framework to optimize SSMs on COTS NPUs, eliminating the need for specialized hardware accelerators. XAMBA addresses key bottlenecks in SSM execution, including CumSum, ReduceSum, and activation functions, through techniques like ActiBA, CumBA, and ReduBA, which restructure sequential operations into parallel matrix computations. These optimizations reduce latency, enhance throughput, and improve memory efficiency. 
% Experimental results show up to 2.6$\times$ performance improvement on Intel\textregistered\ Core\texttrademark\ Ultra Series 2 AI PC. 
% Future work will extend XAMBA to other models, incorporate compression techniques, and explore dynamic optimization strategies for broader hardware platforms.


% This work presents XAMBA, an optimization framework that enhances the performance of SSMs on NPUs. Unlike transformers, SSMs rely on structured state transitions and implicit recurrence, which introduce sequential dependencies that challenge efficient hardware execution. XAMBA addresses these inefficiencies by introducing CumBA, ReduBA, and ActiBA, which optimize cumulative summation, ReduceSum, and activation functions, respectively, significantly reducing latency and improving throughput. By restructuring sequential computations into parallelizable matrix operations and leveraging specialized hardware acceleration, XAMBA enables efficient execution of SSMs on NPUs. Future work will extend XAMBA to other state-space models, integrate advanced compression techniques like pruning and quantization, and explore dynamic optimization strategies to further enhance performance across various hardware platforms and frameworks.
% This work presents XAMBA, an optimization framework that enhances the performance of SSMs on NPUs. Key techniques, including CumBA, ReduBA, and ActiBA, achieve significant latency reductions by optimizing operations like cumulative summation, ReduceSum, and activation functions. Future work will focus on extending XAMBA to other state-space models, integrating advanced compression techniques, and exploring dynamic optimization strategies to further improve performance across various hardware platforms and frameworks.

% This work introduces XAMBA, an optimization framework for improving the performance of Mamba-2 and Mamba models on NPUs. XAMBA includes three key techniques: CumBA, ReduBA, and ActiBA. CumBA reduces latency by transforming cumulative summation operations into matrix multiplication using precomputed masks. ReduBA optimizes the ReduceSum operation through matrix-vector multiplication, reducing execution time. ActiBA accelerates activation functions like Swish and Softplus by mapping them to specialized hardware during the DPU’s drain phase, avoiding sequential execution bottlenecks. Additionally, XAMBA enhances memory efficiency by reducing SRAM access, increasing data reuse, and utilizing Zero Value Compression (ZVC) for masks. The framework provides significant latency reductions, with CumBA, ReduBA, and ActiBA achieving up to 1.8X, 1.1X, and 2.6X reductions, respectively, compared to the baseline.
% Future work includes extending XAMBA to other state-space models (SSMs) and exploring further hardware optimizations for emerging NPUs. Additionally, integrating advanced compression techniques like pruning and quantization, and developing adaptive strategies for dynamic optimization, could enhance performance. Expanding XAMBA's compatibility with other frameworks and deployment environments will ensure broader adoption across various hardware platforms.

\begin{acks}
We thank the anonymous reviewers for their valuable feedback. This work was generously supported by NSF CAREER-1652294, NSF-1908601 and Intel gift awards. SAFARI authors acknowledge support from the Semiconductor Research Corporation, ETH Future Computing Laboratory (EFCL), AI Chip Center for Emerging Smart Systems Limited (ACCESS), and the European Union’s Horizon Programme for research and innovation under Grant Agreement No. 101047160.
\end{acks}

%%%%%%% -- PAPER CONTENT ENDS -- %%%%%%%%


%%
%% If your work has an appendix, this is the place to put it.
\appendix

\appendix

\section{Artifact Appendix}

%%%%%%%%%%%%%%%%%%%%%%%%%%%%%%%%%%%%%%%%%%%%%%%%%%%%%%%%%%%%%%%%%%%%%
\subsection{Abstract}

This document provides a concise guide for reproducing the main performance, power, cost efficiency, and energy efficiency results of this paper in Figures~\ref{fig:TCO}, ~\ref{fig:Main_results}, ~\ref{fig:Combo_Main_Lat_Breakdown}, and ~\ref{fig:Energy_Power}.
The instructions cover the steps required to clone the GitHub repository, build the simulator, set up the necessary Python packages, execute the end-to-end simulation, process results, and generate figures.
The trace generator, performance simulator, power model, automation scripts, expected results, and detailed instructions are available in our \href{https://github.com/Yufeng98/CENT}{\red{GitHub repository}}.


\subsection{Artifact check-list (meta-information)}

{\small
\begin{itemize}
  \item {\bf Program:} C++ and Python.
  \item {\bf Compilation:} \texttt{g++-11/12/13} or \texttt{clang++-15}.
  \item {\bf Software:} \texttt{pandas}, \texttt{matplotlib}, \texttt{torch}, and \texttt{scipy} Python packages.
  \item {\bf Model:} Llama2 7B, 13B, and 70B~\cite{touvron2023llama}.
  \item {\bf Metrics:} latency, throughput (tokens/S), cost efficiency (tokens/\$), energy efficiency (tokens/J), and power.
  \item {\bf Output:} \href{https://github.com/Yufeng98/CENT/tree/main/figure_source_data}{\red{CSV}} and \href{https://github.com/Yufeng98/CENT/tree/main/figures}{\red{PDF}} files corresponding to Figures~\ref{fig:TCO}-\ref{fig:Energy_Power}.
  \item {\bf Experiments:} PIM trace generation and simulation, and \att{} power modeling.
  \item {\bf How much disk space is required?:} Approximately 100GB.
  \item {\bf How much time is needed?:} Approximately 24 hours on a desktop and 8~12 hours on a server.
  \item {\bf Publicly available?:} Available on \href{https://github.com/Yufeng98/CENT}{\red{GitHub}} and \href{https://zenodo.org/records/14776547}{\red{Zenodo}}.
  \item {\bf Code licenses:} \href{https://github.com/Yufeng98/CENT/blob/main/LICENSE}{\red{MIT License}}.
  \item {\bf Work automation?:} Automated by a few scripts.
\end{itemize}
}

%%%%%%%%%%%%%%%%%%%%%%%%%%%%%%%%%%%%%%%%%%%%%%%%%%%%%%%%%%%%%%%%%%%%%
\subsection{Description}

This artifact provides the necessary components to reproduce the main results presented in Figures~\ref{fig:TCO}, ~\ref{fig:Main_results}, ~\ref{fig:Combo_Main_Lat_Breakdown},  and ~\ref{fig:Energy_Power}.
It includes a trace generator, AiM simulator, power model, figure generator, and automation script.
While these figures incorporate simulation results from \att{}, they also rely on a baseline GPU system featuring four Nvidia A100 80GB GPUs, as detailed in Table~\ref{tab:Hardware configurations}.
Due to the high cost associated with these servers, only the expected results for the GPU baseline system are provided in the \href{https://github.com/Yufeng98/CENT/tree/main/data}{\red{data}} directory.
% Consequently, this artifact does not include the infrastructure for the GPU baseline.

\subsubsection{How to access}

Clone the artifact from our GitHub repository using the following command. Please do not forget the \texttt{-}\texttt{-recursive} flag to ensure that the AiM simulator is also cloned:

\begin{lstlisting}
git clone --recursive https://github.com/Yufeng98/CENT.git
\end{lstlisting}

% \subsubsection{Hardware dependencies}

\subsubsection{Software dependencies}

AiM simulator requires \texttt{g++-11/12/13} or \texttt{clang++-15} for compilation.
The Python infrastructure requires \texttt{pandas}, \texttt{matplotlib}, \texttt{torch}, and \texttt{scipy} packages.

% \subsubsection{Data sets}

\subsubsection{Models}

Section~\ref{section:methodology} shows that we evaluate three Llama2 models~\cite{touvron2023llama}.
The model architecture and its PIM mapping are implemented in the \texttt{cent\_simulation/Llama.py} script.
The model weights are required only for the functional simulation of the PIM infrastructure. 
While the functional simulator is available in our GitHub repository, the performance simulator and power model described in this appendix do not model real values, as this does not impact the main results.
Consequently, the model weights and parameters are not required for this appendix.


%%%%%%%%%%%%%%%%%%%%%%%%%%%%%%%%%%%%%%%%%%%%%%%%%%%%%%%%%%%%%%%%%%%%%
\subsection{Installation}

\textbf{Building AiM Simulator.}
To build the simulator, use the following script:

\begin{lstlisting}
cd CENT/aim_simulator/
mkdir build && cd build && cmake ..
make -j4
\end{lstlisting}

\textbf{Setting up Python Packages.}
Install the aforementioned Python packages.
You can use the following script to create a \texttt{conda} environment:

\begin{lstlisting}
cd CENT/
conda create -n cent python=3.10 -y
conda activate cent
pip install -r requirements.txt
\end{lstlisting}

%%%%%%%%%%%%%%%%%%%%%%%%%%%%%%%%%%%%%%%%%%%%%%%%%%%%%%%%%%%%%%%%%%%%%
\subsection{Experiment workflow}

We provide scripts to facilitate the end-to-end reproduction of the results. The following steps outline the process.

\textbf{Generate and Simulate the Traces.}  
This step generates and simulates all required PIM traces.
It also processes the simulation logs, calculates individual latencies, and utilizes the \att{} power model to determine energy consumption and average power.
Upon completion, the generated trace and simulation log files will be stored in the \texttt{trace} directory, while the processed latency and power results can be found in \texttt{cent\_simulation/simulation\_results.csv}. 

\begin{lstlisting}
cd CENT/
bash remove_old_results.sh

cd cent_simulation/
bash simulation.sh [NUM_THREADS] [SEQ_GAP]
\end{lstlisting}

\textit{Note:} The argument \texttt{[NUM\_THREADS]} should be set according to the number of available parallel threads on your processor.
For instance, 8 threads are recommended for desktop processors, while server processors can utilize 96 threads. 

The argument \texttt{[SEQ\_GAP]} determines the gap between each simulated token.
Setting this value to one simulates every token sequentially, requiring approximately 100GB of disk space and taking around 24 hours on a processor with 8 threads or 12 hours on a processor with 96 threads.
To improve disk usage and reduce simulation time, the \texttt{[SEQ\_GAP]} argument can be set to a larger value, such as 128. This configuration simulates one out of every 128 tokens, processing token IDs of 128, 256, 384, and so on up to 4096.


\textbf{Process the Results.}  
This step processes the simulation results and computes the latency, throughput, power, and energy for the prefill, decoding, and end-to-end phases.
After processing the results, this script stores them in this file: \texttt{cent\_simulation/processed\_results.csv}.

\begin{lstlisting}
cd CENT/cent_simulation/
bash process_results.sh
\end{lstlisting}

\textbf{Generate Figures.}  
The following script generates Figures~\ref{fig:TCO}-\ref{fig:Energy_Power}.
This process utilizes the baseline GPU results, available in the \href{https://github.com/Yufeng98/CENT/tree/main/data}{\red{data}} directory, along with the processed results.
It computes the normalized results and generates both a PDF file containing the figures and a CSV file with the corresponding numerical data.

\begin{lstlisting}
cd CENT/
bash generate_figures.sh
\end{lstlisting}

%%%%%%%%%%%%%%%%%%%%%%%%%%%%%%%%%%%%%%%%%%%%%%%%%%%%%%%%%%%%%%%%%%%%%
\subsection{Evaluation and expected results}

The normalized results and the figures will be located in the \texttt{figure\_source\_data} and \texttt{figures} directories.
The expected results can be found in Figures~\ref{fig:TCO}-~\ref{fig:Energy_Power} or in the generated \href{https://github.com/Yufeng98/CENT/tree/main/figure_source_data}{\red{CSV}} and \href{https://github.com/Yufeng98/CENT/tree/main/figures}{\red{PDF}} files on our GitHub repository. Figures in the paper are generated using Microsoft Excel. To visualize the figures in the paper's format, copy the normalized data from the CSV files to the \texttt{Data} sheet of the provided \href{https://github.com/Yufeng98/CENT/blob/main/cent_simulation/Figures.xlsx}{\red{Figures.xlsx}}.
Figures will be generated in the \texttt{Figures} sheet.

\clearpage

% %%%%%%%%%%%%%%%%%%%%%%%%%%%%%%%%%%%%%%%%%%%%%%%%%%%%%%%%%%%%%%%%%%%%%
% \subsection{Experiment customization}

% %%%%%%%%%%%%%%%%%%%%%%%%%%%%%%%%%%%%%%%%%%%%%%%%%%%%%%%%%%%%%%%%%%%%%
% \subsection{Notes}

% %%%%%%%%%%%%%%%%%%%%%%%%%%%%%%%%%%%%%%%%%%%%%%%%%%%%%%%%%%%%%%%%%%%%%
% \subsection{Methodology}

% Submission, reviewing and badging methodology:

% \begin{itemize}
%   \item \url{https://www.acm.org/publications/policies/artifact-review-and-badging-current}
%   \item \url{https://cTuning.org/ae}
% \end{itemize}


%%
%% The next two lines define the bibliography style to be used, and
%% the bibliography file.
% \bibliographystyle{ACM-Reference-Format}
\bibliographystyle{plainurl}
\bibliography{references}


\end{document}
\endinput
%%
%% End of file `sample-sigplan.tex'.
