%%
%% This is file `sample-sigplan.tex',
%% generated with the docstrip utility.
%%
%% The original source files were:
%%
%% samples.dtx  (with options: `all,proceedings,bibtex,sigplan')
%% 
%% IMPORTANT NOTICE:
%% 
%% For the copyright see the source file.
%% 
%% Any modified versions of this file must be renamed
%% with new filenames distinct from sample-sigplan.tex.
%% 
%% For distribution of the original source see the terms
%% for copying and modification in the file samples.dtx.
%% 
%% This generated file may be distributed as long as the
%% original source files, as listed above, are part of the
%% same distribution. (The sources need not necessarily be
%% in the same archive or directory.)
%%
%%
%% Commands for TeXCount
%TC:macro \cite [option:text,text]
%TC:macro \citep [option:text,text]
%TC:macro \citet [option:text,text]
%TC:envir table 0 1
%TC:envir table* 0 1
%TC:envir tabular [ignore] word
%TC:envir displaymath 0 word
%TC:envir math 0 word
%TC:envir comment 0 0
%%
%% The first command in your LaTeX source must be the \documentclass
%% command.
%%
%% For submission and review of your manuscript please change the
%% command to \documentclass[manuscript, screen, review]{acmart}.
%%
%% When submitting camera ready or to TAPS, please change the command
%% to \documentclass[sigconf]{acmart} or whichever template is required
%% for your publication.
%%
%%
\documentclass[sigplan,screen]{acmart}
% make references clickable 
\usepackage[]{hyperref}
\usepackage{url}
\usepackage{multirow} % for borders and merged ranges
\usepackage{listings}
\usepackage{dblfloatfix}
\usepackage{placeins}  % Add this in the preamble

\lstset{
    language=bash,
    basicstyle=\ttfamily\footnotesize,
    keywordstyle=\color{blue}\bfseries,
    commentstyle=\color{gray}\itshape,
    stringstyle=\color{red},
    showstringspaces=false,
    breaklines=true,
    tabsize=2,
    frame=single
}

\copyrightyear{2025}
\acmYear{2025}
\setcopyright{cc}
\setcctype{by-nc-sa}
\acmConference[ASPLOS '25]{Proceedings of the 30th ACM International Conference on Architectural Support for Programming Languages and Operating Systems, Volume 2}{March 30-April 3, 2025}{Rotterdam, Netherlands}
\acmBooktitle{Proceedings of the 30th ACM International Conference on Architectural Support for Programming Languages and Operating Systems, Volume 2 (ASPLOS '25), March 30-April 3, 2025, Rotterdam, Netherlands}
\acmDOI{10.1145/3676641.3716267}
\acmISBN{979-8-4007-1079-7/25/03}

% 1 Authors, replace the red X's with your assigned DOI string during the rightsreview eform process.
% 2 Your DOI link will become active when the proceedings appears in the DL.
% 3 Retain the DOI string between the curly braces for uploading your presentation video.

\settopmatter{printacmref=true}
\setlength{\textfloatsep}{7pt}
\setlength{\intextsep}{5pt}

\begin{document}

\title{PIM Is All You Need: A CXL-Enabled GPU-Free System for Large Language Model Inference}

%
% The "author" command and its associated commands are used to define
% the authors and their affiliations.
% Of note is the shared affiliation of the first two authors, and the
% "authornote" and "authornotemark" commands
% used to denote shared contribution to the research.
% \author{Ben Trovato}
% \authornote{Both authors contributed equally to this research.}
% \email{trovato@corporation.com}
% \orcid{1234-5678-9012}
% \author{G.K.M. Tobin}
% \authornotemark[1]
% \email{webmaster@marysville-ohio.com}
% \affiliation{%
%   \institution{Institute for Clarity in Documentation}
%   \city{Dublin}
%   \state{Ohio}
%   \country{USA}
% }

\author{Yufeng Gu}
\authornote{Yufeng Gu and Alireza Khadem contributed equally to this research}
\affiliation{%
  \institution{University of Michigan}
  \city{Ann Arbor}
  \country{USA}}
\email{yufenggu@umich.edu}

\author{Alireza Khadem}
\authornotemark[1]
\affiliation{%
  \institution{University of Michigan}
  \city{Ann Arbor}
  \country{USA}}
\email{arkhadem@umich.edu}

\author{Sumanth Umesh}
\affiliation{%
  \institution{University of Michigan}
  \city{Ann Arbor}
  \country{USA}}
\email{sumanthu@umich.edu}

\author{Ning Liang}
\affiliation{%
  \institution{University of Michigan}
  \city{Ann Arbor}
  \country{USA}}
\email{nliang@umich.edu}

\author{Xavier Servot}
\affiliation{%
  \institution{ETH Zürich}
  \city{Zürich}
  \country{Switzerland}}
\email{xservot@student.ethz.ch}

\author{Onur Mutlu}
\affiliation{%
  \institution{ETH Zürich}
  \city{Zürich}
  \country{Switzerland}}
\email{omutlu@gmail.com}

\author{Ravi Iyer}
\authornote{This research was done while the author was at Intel Corporation}
\affiliation{%
  \institution{Google}
  \city{Mountain View}
  \country{USA}}
\email{raviiyer20@gmail.com}

\author{Reetuparna Das}
\affiliation{%
  \institution{University of Michigan}
  \city{Ann Arbor}
  \country{USA}}
\email{reetudas@umich.edu}

% \author{Yufeng Gu*\textsuperscript{\textdagger} \qquad Alireza Khadem*\textsuperscript{\textdagger} \qquad Sumanth Umesh\textsuperscript{\textdagger} \qquad Ning Liang\textsuperscript{\textdagger}} 
% \author{Xavier Servot\textsuperscript{\textdaggerdbl} \qquad Onur Mutlu\textsuperscript{\textdaggerdbl} \qquad Ravi Iyer**\textsuperscript{\textsection} \qquad Reetuparna Das\textsuperscript{\textdagger}}

% \affiliation{\textsuperscript{\textdagger}University of Michigan \quad \textsuperscript{\textdaggerdbl}ETH Zürich \quad \textsuperscript{\textsection}Google}

% \email{{yufenggu, arkhadem, reetudas}@umich.edu}

% \thanks{* Yufeng Gu and Alireza Khadem contributed equally to this research}
% \thanks{** This research was done while the author was at Intel Corporation}

%%
%% By default, the full list of authors will be used in the page
%% headers. Often, this list is too long, and will overlap
%% other information printed in the page headers. This command allows
%% the author to define a more concise list
%% of authors' names for this purpose.
\renewcommand{\shortauthors}{Yufeng Gu and Alireza Khadem et al.}

%%
%% The abstract is a short summary of the work to be presented in the
%% article.
\definecolor{myblue}{RGB}{0, 0, 150}

\newcommand{\att}[0]{CENT}
\newcommand{\rf}[1]{Shared Buffer}
\newcommand{\Sota}{State-of-the-art}
\newcommand{\sota}{state-of-the-art}
\newcommand{\ali}[1]{\noindent{\textcolor{orange}{\bf \fbox{AK} {\it#1}}}}
\newcommand{\reetu}[1]{\noindent{\textcolor{blue}{\bf \fbox{RD} {\it#1}}}}
\newcommand{\yufeng}[1]{\noindent{\textcolor{purple}{\bf \fbox{YG} {\it#1}}}}
\newcommand{\sumanth}[1]{\noindent{\textcolor{red}{\bf \fbox{SU} {\it#1}}}}
\newcommand{\ning}[1]{\noindent{\textcolor{cyan}{\bf \fbox{NL} {\it#1}}}}
\newcommand{\todo}[1]{\noindent{\textcolor{cyan}{\bf \fbox{TODO} {\it#1}}}}
\newcommand{\red}[1]{\textcolor{myblue}{#1}}
% \newcommand{\red}[1]{\textcolor{red}{#1}}
\newcommand{\ignore}[1]{}

\begin{abstract}

% Recent works to jointly reconstruct 3D human and object from a single RGB image, are mostly model-based, that fail to capture the fine details of the clothed human body and object surface. In this paper, we introduce ReCHOR, a novel, model-free, first-method to produce realistic clothed human-object reconstructions from a monocular view. This is extremely challenging due to human-object occlusions, diverse interactions and depth ambiguity, as it needs to infer both 3D spatial awareness and high resolution details. Our core idea is based on estimating neural implicit representations for human and object respectively by an attention-based neural implicit model that attends to pixel-aligned features from both the global human-object image for spatial awareness and  the local separate view of human and object images for high quality details. Additionally, the network is conditioned on semantic features from an initial estimated human-object pose prior and a generative diffusion model that inpaints occluded regions, thus enabling the retrieval of details from them.
% We also propose a synthetic dataset with rendered scenes of diverse, inter-occluded 3D human and object scans, to train our network. We evaluate our method on the synthetic and real world BEHAVE dataset. Our experiments show that our method outperforms the SOTA in achieving realistic clothed human-object reconstructions.
Recent approaches to jointly reconstruct 3D humans and objects from a single RGB image represent 3D shapes with template-based or coarse models, which fail to capture details of loose clothing on human bodies. In this paper, we introduce a novel implicit approach for jointly reconstructing realistic 3D clothed humans and objects from a monocular view. For the first time, we model both the human and the object with an implicit representation, allowing to capture more realistic details such as clothing. This task is extremely challenging due to human-object occlusions and the lack of 3D information in 2D images, often leading to poor detail reconstruction and depth ambiguity. To address these problems, we propose a novel attention-based neural implicit model that leverages image pixel alignment from both the input human-object image for a global understanding of the human-object scene and from local separate views of the human and object images to improve realism with, for example, clothing details. Additionally, the network is conditioned on semantic features derived from an estimated human-object pose prior, which provides 3D spatial information about the shared space of humans and objects. To handle human occlusion caused by objects, we use a generative diffusion model that inpaints the occluded regions, recovering otherwise lost details. For training and evaluation, we introduce a synthetic dataset featuring rendered scenes of inter-occluded 3D human scans and diverse objects. Extensive evaluation on both synthetic and real-world datasets demonstrates the superior quality of the proposed human-object reconstructions over competitive methods.
\end{abstract}

\begin{CCSXML}
<ccs2012>
   <concept>
       <concept_id>10010520.10010521.10010528</concept_id>
       <concept_desc>Computer systems organization~Parallel architectures</concept_desc>
       <concept_significance>500</concept_significance>
       </concept>
   <concept>
       <concept_id>10010520.10010521.10010542.10010294</concept_id>
       <concept_desc>Computer systems organization~Neural networks</concept_desc>
       <concept_significance>500</concept_significance>
       </concept>
 </ccs2012>
\end{CCSXML}

\ccsdesc[500]{Computer systems organization~Parallel architectures}
\ccsdesc[500]{Computer systems organization~Neural networks}

%%
%% Keywords. The author(s) should pick words that accurately describe
%% the work being presented. Separate the keywords with commas.
\keywords{Computer Architecture, Processing-In-Memory, Compute Express Link, Generative Artificial Intelligence, Large Language Models.}
%% A "teaser" image appears between the author and affiliation
%% information and the body of the document, and typically spans the
%% page.
% \begin{teaserfigure}
%   \includegraphics[width=\textwidth]{sampleteaser}
%   \caption{Seattle Mariners at Spring Training, 2010.}
%   \Description{Enjoying the baseball game from the third-base
%   seats. Ichiro Suzuki preparing to bat.}
%   \label{fig:teaser}
% \end{teaserfigure}

\received{24 June 2024}
\received[revised]{2 October 2024}
\received[accepted]{27 January 2025}

%%
%% This command processes the author and affiliation and title
%% information and builds the first part of the formatted document.
\maketitle

%%%%%% -- PAPER CONTENT STARTS-- %%%%%%%%

\section{Introduction}\label{sec:intro}

In computational finance, Monte Carlo simulations are used extensively to estimate the expected value of financial payoffs based on the solution of stochastic differential equations (SDEs) which model the evolution of stock prices, interest rates, exchange rates and other quantities \cite{glasserman04}.  Monte Carlo methods are very general and flexible, but for high accuracy it requires generating a large number of costly SDE path approximations, which has motivated research into a number of variance reduction or, equivalently, cost reduction techniques. One such method is
Multilevel Monte Carlo (MLMC), which was proposed in \cite{GILES2008} and was adapted for various applications that are summarised in \cite{Giles_overview17} and successfully combined with other methods such as quasi-Monte Carlo methods. The main idea of MLMC is to approximate the payoff using different time stepping resolutions when numerically solving the underlying SDE and to generate an optimal number of samples on each level, such that the overall computational cost is minimised subject to the desired bound on the variance. %, such that the total computational cost is minimised. 
The computational savings come from the fact that most samples are computed on the coarser levels and hence are less expensive while only a few samples from the finest levels are required \cite{GILES2008}.


Among the directions in which the computational cost 
of MLMC methods could further be reduced, an important avenue is the use of lower precision calculations, especially for the first Monte Carlo levels where the targeted accuracy is relatively low. 
 An overview of the research on mixed precision for the standard Monte Carlo (MC) framework is provided in \cite{ChowMixedPrecisionStandardMC} but only a few references study the potential of low precision computation in the MLMC framework \cite{Rounding_error_oliver}. To the best of our knowledge, the only MLMC framework with customised precision in the literature is \cite{brugger2014mixed}, but they use a uniform precision for all operations on each Monte Carlo level instead of optimising 
 the precision of each intermediary variable to reduce as much as possible the cost of path generation.
 
An important motivation for an MLMC framework with variable precision would be performing the low precision computations on reconfigurable hardware devices such as Field Programmable Gate Arrays (FPGAs). FPGAs contain customizable logic blocks and connectors that make it easy to adapt the digital circuit architecture for a specific application, leading to a highly parallel and optimised implementation. Therefore they are successfully exploited in applications that require high speed and have high computational workload, such as signal processing \cite{woods2008fpga}, and real time applications like high frequency trading \cite{HFT1,HFT2}. That is why a number of previous works in hardware architecture design implemented the MLMC algorithm to price financial options using FPGAs as accelerators, which resulted in improved speed and power efficiency compared to full CPU architectures \cite{Schryver2013AMM}. The paper \cite{lindsey2016domain} also proposed 
a Domain Specific Language to automate the configuration of FPGAs for this specific application. However, only \cite{brugger2014mixed} proposed a heuristic to reduce the precision in calculations.

In addition, all aforementioned works considered that the random number generation (RNG) is performed in single or double precision. Yet in most cases an important portion of the workload in the overall MLMC simulation comes from the RNG and in \cite{brugger2014mixed} this limited the total computational savings.
To reduce the cost of MLMC simulations in particular those based on the Geometric Brownian Motion (GBM), \cite{approximateICDF_Oliver, NestedOliver} have proposed to use approximate random numbers that are generated by applying an approximation of the inverse CDF to uniform random numbers. In \cite{NestedOliver}, the authors proposed a way to integrate these lower precision random variables into a \textit{nested} MLMC framework and completed a numerical analysis to bound the resulting error at each MC level by a product of the time step and the error in the random number approximation. The same authors show in \cite{approximateICDF_Oliver} that using approximate random variables reduces the cost of path generation by a factor 7.


In this paper we propose a nested MLMC framework that combines the use of approximate random normal variables and lower precision calculations to reduce the computational cost of MLMC even further than \cite{brugger2014mixed,NestedOliver}. We illustrate the efficiency of our framework in Matlab, after making several assumptions on the cost of operations and size of the errors that we carefully justify. We focus on the case of GBM and use the approximate RNG methods presented in \cite{approximateICDF_Oliver} as well as a new slightly modified method that combines CDF inversion and the central limit theorem. To choose the precision of the variables in the low precision path generation, we introduce a novel method to optimise the bit-widths. This optimisation is performed before the main path generation loop is executed and is based on a linear model of the payoff error  
due to rounding when computing in low precision. The error model relies on algorithmic differentiation in a similar manner to \cite{unifying-bwoptim,bitwidth-AD,ADAPT}. The bit-width optimisation procedure can be performed off-line, so this stage can be excluded from the on-line time complexity of our framework. The user specified desired accuracy is then enforced by calculating on-line the number of samples that need to be generated.

In terms of hardware design, we suggest implementing the low precision path generation on FPGAs and the full-precision ones on a CPU or GPU. 
The FPGA offers enough flexibility to define a separate bit-width for every variable in the low precision path generation, and can be reconfigured periodically to update the bit-widths when the market parameters have changed considerably. 


The paper is organized as follows : \Cref{sec:MLMC} introduces MLMC and nested MLMC to make clear the estimator that is implemented in our framework. Then in \Cref{sec:RNG} we detail the methods that could be used to obtain approximate random normally distributed numbers very cheaply for the low precision path generation. In \Cref{sec:error_model} and \Cref{sec:costModel} we propose an error model and a cost model (resp.) that we then use to formulate the optimisation problem that is solved to obtain the optimal bit-widths of fixed point variables in \Cref{sec:optimisation}. Finally we summarise our results and future directions in \Cref{sec:conclusion}.




\begin{figure}[t]
\vskip -0.1in
    \centering
    \includegraphics[width=.47\linewidth]{Figures/test_Growth_Comparison_First_3_Iter_X_of_onehot_Variance_BLOCK_OUTPUT_1B_post_no_broken.png} \includegraphics[width=.47\linewidth]{Figures/test_Growth_Comparison_First_3_Iter_X_of_onehot_Variance_BLOCK_OUTPUT_pre_1B.png} 
    \vskip -0.05in
    \caption{Illustration of hidden-state variance across different model depths and training iterations. From the initialization stage up to the point where 6.3 billion tokens were trained, we observed the variance growth of hidden states for Pre-LN and Post-LN architectures. The analysis was conducted using a $1.5$B-parameter model, and consistent trends were observed across models of different sizes. Detailed settings and more results are in Section~\ref{subsec:growth of hidden state}.}
    \label{fig:3iter}
    \vskip -0.1in
\end{figure}

\section{Background and Motivation}
The analysis of activation variance at model initialization has long been central to understanding normalization layers and enhancing stability in convolutional neural networks (CNNs) \citep{cnnvariance, identity, BrockDSS21}. \citet{cnnvariance} showed that batch normalization in residual blocks can bias networks toward the identity function, thereby stabilizing gradients and improving overall training dynamics. 

Similar investigations have emerged for Transformer architectures, examining how variance propagates and how gradients behave in both post-layer normalization (Post-LN) \citep{attentionisallyouneed} and pre-layer normalization (Pre-LN) \citep{llama3} configurations \citep{onlayer, transformersgetstable, smallproxies, mixln}. Early work comparing Post- and Pre-LN primarily focused on gradient scales and loss behavior. \citet{onlayer} observed that Pre-LN architectures tend to exhibit more stable gradients, but can still encounter issues such as gradient spikes and divergence, especially in deeper models or large-scale pre-training scenarios \citep{attentioncollapse, smallproxies, mlpswiglu, embeddingln}. 

Among these challenges, the phenomenon of ``massive activations'' has attracted particular attention \citep{llm.int8,yu2024super,mlpswiglu}. \citet{massiveactivation} identified that in Pre-LN architectures, large spikes in activation magnitude can persist across layers due to residual connections. These massive activations act as fixed biases, potentially narrowing the model’s focus to certain tokens and may influence generalization. However, the underlying mechanisms behind these large values—and their exact impact on the training process—remain not yet well understood.

Analytical work has provided theoretical frameworks to explain phenomena like gradient explosion and vanishing in Transformers. For instance, \citet{transformersgetstable} introduced a signal propagation theory that details how activation variance and gradient instability can evolve with depth, identifying critical factors that impair stability and performance. Recent studies have discussed how Pre-LN architectures can allow large values from Attention or MLP modules to flow unimpeded through residual connections \citep{moeut, mlpswiglu, attentioncollapse, smallproxies}, but the precise impact of this behavior on large-scale training remains insufficiently explored.

These observations underscore the ongoing need to clarify how activation dynamics, normalization strategies, and architectural choices interact, especially in large-scale models. In response, this work aims to deepen our understanding of activation evolution during Transformer training under different normalization architectures, focusing on the role of massive activations and their effects on overall stability and performance.

We defer an extended discussion of the related literature to Appendix~\ref{appendix:relatedwork}, owing to space limitations.

\section{Background}
\label{sec:background}


\subsection{Preliminaries}

{\color{red}[TODO: LLMs? in-context learning?]}

\subsection{Problem Definition}

{\color{red}[TODO: define the problem of citation intent]}

% \section{Comparison of Normalization Strategies} \label{sec:ln_in_transformer}
\section{Comparative Analysis} \label{sec:ln_in_transformer}

In this section, we discuss how different placements of layer normalization (LN \footnote{Unless stated otherwise, LN refers to both LayerNorm and RMSNorm.}) in Transformer architecture affect both training stability and the statistics of hidden states (activations \footnote{We use ``hidden state'' and ``activation'' interchangeably.}).

\subsection{Post- \& Pre-Normalization in Transformers}
\label{subsec:post_pre_ln}
\paragraph{Post-LN.}
The Post-Layer Normalization (Post-LN) \citep{attentionisallyouneed} scheme, normalization is applied \emph{after} summing the module’s output and residual input:
\begin{equation}
    y_{l} = \mathrm{Norm}\bigl(x_l + \mathrm{Module}(x_l)\bigr),
    \label{eq:post_ln}
\end{equation}
where $x_l$ is the input hidden state of $l$-th layer, $y_{l}$ is the output hidden state of $l$-th layer, and $\mathrm{Module}$ denotes Attention or Multi-Layer Perceptron (MLP) module in the Transformer sub-layer. $\mathrm{Norm}$ denotes normalization layers such as RMSNorm or LayerNorm. It is known that by stabilizing the activation variance at a constant scale, Post-LN prevents activations from growing. However, several evidence~\citep{onlayer, transformersgetstable} suggest that Post-LN can degrade gradient flow in deeper networks, leading to vanishing gradients and slower convergence.


\paragraph{Pre-LN.}
The Pre-Layer Normalization (Pre-LN)~\citep{llama3} scheme, normalization is applied to the module's input \emph{before} processing:
\begin{equation}
    y_l = x_l + \mathrm{Module}\bigl(\mathrm{Norm}(x_l)\bigr).
    \label{eq:pre_ln}
\end{equation}
As for Llama $3$ architecture, a final LN is applied to the network output. Pre-LN improves gradient flow during backpropagation, stabilizing early training \citep{onlayer}. Nonetheless, in large-scale Transformers, even Pre-LN architectures are not immune to instability during training~\citep{smallproxies, attentioncollapse}. As shown in Figure~\ref{fig:LN Placement}, unlike Post-LN—which places LN at position $C$—Pre-LN, which places LN only at position $A$, can lead to a “highway” structure that is continuously maintained throughout the entire model if the module produces an output with a large magnitude. This phenomenon might be related to the ``massive activations'' observed in trained models \citep{massiveactivation, mlpswiglu}. 

\begin{figure}[t]
% \vskip -0.1in
    \centering
    \begin{minipage}[t]{0.45\linewidth}
        \vspace{0pt}
        \centering
        \includegraphics[width=.75\linewidth]{Figures/method_abc_ver3.png}
    \end{minipage}
    \centering
    \begin{minipage}[t]{0.45\linewidth}
        \vspace{32pt}
        \centering
        \small
        \begin{tabular}{lccc}
            \toprule
            ~ & A & B & C  \\ 
            \midrule
            Post-LN & \texttimes & \texttimes & \checkmark \\
            Pre-LN  & \checkmark & \texttimes & \texttimes \\
            Peri-LN & \checkmark & \checkmark & \texttimes \\
            \bottomrule
        \end{tabular}
    \end{minipage}
    \caption{Placement of normalization in Transformer sub-layer. }
    \label{fig:LN Placement}
    \vskip -0.1in
\end{figure}

\begin{table}
\caption{Intuitive comparison of normalization strategies.}
\label{tab:variance_summary}
\small
\begin{tabular}{lcc}
\toprule
\textbf{Strategy} & \textbf{Variance Growth} & \textbf{Gradient Stability} \\
\midrule
\textbf{Post-LN} & Mostly constant & Potential for vanishing \\
\textbf{Pre-LN} & Exponential in depth & Potential for explosion \\
\textbf{Peri-LN} & $ \approx \text{Linear}$ in depth & Self-regularization \\
\bottomrule
\end{tabular}
\vskip -0.1in
\end{table} 



\subsection{Variance Behavior from Initialization to Training}
\label{subsec:variance_growth}


As discussed by \citet{onlayer} and \citet{transformersgetstable}, Transformer models at \emph{initialization} exhibit near-constant hidden-state variance under Post-LN and linearly increasing variance under Pre-LN. Most of the previous studies have concentrated on this early-stage behavior. However, Recent studies have also reported large output magnitudes in both the pre-trained attention and MLP modules \citep{vit22b, smallproxies, mlpswiglu}. To bridge the gap from initialization to the fully trained stage, we extend our empirical observations in Figure~\ref{fig:3iter} beyond initial conditions by tracking how these variance trends evolve at intermediate points in training. 

We find that Post-LN maintains a roughly constant variance, which helps avert exploding activations. Yet as models grow deeper and training proceeds, consistently normalizing $x_l + \mathrm{Module}(x_l)$ can weaken gradient flow, occasionally causing partial vanishing gradients and slower convergence. In contrast, Pre-LN normalizes $x_l$ before the module but leaves the module output unnormalized, allowing hidden-state variance to accumulate exponentially once parameter updates amplify the input. Although Pre-LN preserves gradients more effectively in earlier stages, this exponential growth in variance can lead to “massive activations” \citep{massiveactivation}, risking numeric overflow and destabilizing large-scale training. We reconfirm this in Section~\ref{sec:experiments}.

\paragraph{Takeaways.}
\begin{itemize}
\item \textit{Keeping the Highway Clean: Post-LN’s Potential for Gradient Vanishing and Slow Convergence.} When layer normalization is placed directly on the main path (Placement $C$ in Figure \ref{fig:LN Placement}), it can cause gradient vanishing and introduce fluctuations in the gradient scale, potentially leading to instability. 

\item \textit{Maintaining a Stable Highway: Pre-LN May Not Suffice for Training Stability.} Pre-LN does not normalize the main path of the hidden states, thereby avoiding the issues that Post-LN encounters. Nevertheless, a structural characteristic of Pre-LN is that any large values arising in the attention or MLP modules persist through the residual identity path. In particular, as shown in Figure~\ref{fig:3iter}, the exponentially growing magnitude and variance of the hidden states in the forward path may lead to numerical instability and imbalance during training.
\end{itemize}

Recent open-sourced Transformer architectures have adopted normalization layers in unconventional placements. Models like Gemma$2$ and OLMo$2$ utilize normalization layers at the module output (Output-LN), but the benefits of these techniques remain unclear \citep{gemma2, olmo2}. To investigate the impact of adding an Output-LN, we explore the peri-layer normalization architecture.


\subsection{Placing Module Output Normalization}
\label{subsec:peri_ln}

\paragraph{Peri-LN.}
The Peri-Layer Normalization (Peri-LN) applies LN twice within each layer---before and after the module---and further normalizes the input and final output embeddings. Formally, for the hidden state $x_l$ at layer $l$:
\begin{enumerate}
    \item \textit{Initial Embedding Normalization:}
    \[
      y_o = \mathrm{Norm}(x_o),
    \]
    \item \textit{Input- \& Output-Normalization per Layer:}
    \[
      y_l = x_l + \mathrm{Norm}\Bigl(\mathrm{Module}\bigl(\mathrm{Norm}(x_l)\bigr)\Bigr),
    \]
    \item \textit{Final Embedding Normalization:}
    \[
      y_L = \mathrm{Norm}(x_L),
    \]
\end{enumerate}
where $x_o$ denotes the output of the embedding layer, the hidden input state. $y_0$ represents the normalized input hidden state. $x_L$ denotes the hidden state output by the final layer \(L\) of the Transformer sub-layer. This design unifies pre- and output-normalization to regulate variance from both ends. For clarity, the locations of normalization layers in the Post-, Pre-, and Peri-LN architectures are illustrated in Figure~\ref{fig:LN Placement}.


\paragraph{Controlling Variance \& Preserving Gradients.}
% \paragraph{Roll of Output Layer Normalization.}
By normalizing both the input and output of each sub-layer, Peri-LN constrains the \emph{residual spikes} common in Pre-LN, while retaining a stronger gradient pathway than Post-LN. Concretely, if $\mathrm{Norm}(\mathrm{Module}(\mathrm{Norm}(x_l)))$ has near-constant variance $\beta_0$, then
\[
  \mathrm{Var}(x_{l+1}) \;\approx\; \mathrm{Var}(x_l) + \beta_0,
\]
leading to \emph{linear or sub-exponential} hidden state growth rather than exponential blow-up.  We empirically verify this effect in Section~\ref{subsec:growth of hidden state}. 



\begin{figure*}[t]
\vskip -0.1in
    \centering
    \subfigure[Learning rate exploration]
    {
    \includegraphics[width=.3\linewidth]{Figures/pretrain_lrsweep.png}
    \label{fig:pretrain_lrwseep}
    }
    \subfigure[Training loss]
    {
    \includegraphics[width=.295\linewidth]{Figures/hcx_text_400M_dclm_000_30B_warmup10_lr5e4.csv_best_loss_trainingloss_per_tokens.png}
    \label{fig:pretrain_loss}
    }
    \subfigure[Gradient-norm]
    {
    \includegraphics[width=.288\linewidth]{Figures/hcx_text_400M_dclm_000_30B_warmup10_lr5e4.csv_best_loss_warmup10_gradnorm_per_tokens.png}
    \label{fig:pretrain_gradnorm}
    }
    \caption{
    Performance comparison of Post-LN, Pre-LN, and Peri-LN Transformers during pre-training. Figure \ref{fig:pretrain_lrwseep} llustrates the pre-training loss across learning rates. Pre-training loss and gradient norm of best performing $400$M size Transformers are in Figure \ref{fig:pretrain_loss} and \ref{fig:pretrain_gradnorm}. Consistent trends were observed across models of different sizes.
    }
    \label{fig:pretraining}
\vskip -0.1in
\end{figure*}

\begin{figure*}[t]
    \centering
    \subfigure[Training loss]
    {
    \includegraphics[width=.3\linewidth]{Figures/fix_gamma_loss_400M.png}
    \label{fig:fix_gamma_loss}
    }
    \subfigure[Loss in the final $5$B token interval]
    {
    \includegraphics[width=.3\linewidth]{Figures/fix_gamma_zoom_loss_400M.png}
    \label{fig:fix_gamma_loss_zoom}
    }
    \subfigure[Gradient-norm]
    {
    \includegraphics[width=.3\linewidth]{Figures/fix_gamma_gradnorm_400M.png}
    \label{fig:fix_gamma_gradnorm}
    }
    \caption{
    Freezing learnable parameter $\gamma$ of output normalization layer in Peri-LN. we set $\gamma$ to its initial value of $1$ and keep it fixed.
    }
    \label{fig:frozen_gamma}
\vskip -0.1in
\end{figure*}

\paragraph{Open-Sourced Peri-LN Models: Gemma$2$ \& OLMo$2$.}
Both Gemma$2$ and OLMo$2$, which apply output layer normalization, employ the same peri-normalization strategy within each Transformer layer. However, neither model rigorously examines how this placement constrains variance or mitigates large residual activations. Our work extends Gemma$2$ and OLMo$2$ by offering both theoretical and empirical perspectives within the Peri-LN scheme. Further discussion of the OLMo$2$ is provided in Appendix~\ref{appendix:olmo2}.

\subsection{Stability Analysis in Normalization Strategies}
\label{subsec:theory_insights}
We analyze training stability in terms of the magnitude of activation. To this end, we examine the gradient norm with respect to the weight of the final layer in the presence of massive activation. For the formal statements and detailed proofs, refer to Appendix~\ref{appendix:theory_proof}.

\begin{proposition}[Informal]
\label{prop:theory}
Let $\mathcal{L}(\cdot)$ be the loss function, and let $W^{(2)}$ denote the weight of the last layer of $\mathrm{MLP}(\cdot)$. Let $\gamma$ be the scaling parameter in $\mathrm{Norm}(\cdot)$, and let $D$ be the dimension. Then, the gradient norm for each normalization strategy behaves as follows.

\medskip
\noindent 
\textbf{(1) Pre-LN (exploding gradient).} Consider the following sequence of operations:
\begin{equation}
\tilde{x} = \mathrm{Norm}(x), a = \mathrm{MLP}(\tilde{x}), o = x + a,
\end{equation}
then
\begin{equation}
\left\lVert \frac{\partial \mathcal{L}(o)}{\partial W_{i,j}^{(2)}} \right\rVert \;\propto\; \| h_{i} \|,
\end{equation}
where $h := \mathrm{ReLU}\left(\tilde{x} W^{(1)} + b^{(1)}\right)$. In this case, when a massive activation $\|h\|$ occurs, an exploding gradient $\|\partial \mathcal{L} / \partial W^{(2)}\|$ can arise, leading to training instability.

\medskip
\noindent
\textbf{(2) Peri-LN (self-regularizing gradient).} Consider the following sequence of operations:
\begin{equation}
\tilde{x} = \mathrm{Norm}(x), a = \mathrm{MLP}(\tilde{x}), \tilde{a} = \mathrm{Norm}(a), o = x + \tilde{a},
\end{equation}
then
\begin{equation}
\left\lVert \frac{\partial \mathcal{L}(o)}{\partial W_{i,j}^{(2)}} \right\rVert 
\;\le\; \frac{4\,\gamma\,\sqrt{D}\,\|h\|}{\|a\|}, 
\end{equation}
where $h := \mathrm{ReLU}\left(\tilde{x} W^{(1)} + b^{(1)}\right)$. In this case, even when a massive activation $\|h\|$ occurs, $\mathrm{Norm}(\cdot)$ introduces a damping factor $\|a\|$, which ensures that the gradient norm $\|\partial \mathcal{L} / \partial W^{(2)}\|$ remains bounded.

\medskip
\noindent
\textbf{(3) Post-LN (vanishing gradient).} Consider the following sequence of operations:
\begin{equation}
a = \mathrm{MLP}(x), o = x + a, \tilde{o} = \mathrm{Norm}(o),
\end{equation}
then
\begin{equation}
\left\lVert \frac{\partial \mathcal{L}(\tilde{o})}{\partial W_{i,j}^{(2)}} \right\rVert 
\;\le\; \frac{4\,\gamma\,\sqrt{D}\,\|h\|}{\|x + a\|}, 
\end{equation}
where $h := \mathrm{ReLU}\left(x W^{(1)} + b^{(1)}\right)$. In this case, when a massive activation $\|h\|$ occurs, $\mathrm{Norm}(\cdot)$ introduces an overly suppressing factor $\|x+a\|$, which contains a separate huge residual signal $x$, potentially leading to a vanishing gradient $\|\partial \mathcal{L} / \partial W^{(2)}\|$.
\vskip -0.1in
\end{proposition}

We have compiled a Table~\ref{tab:variance_summary} that provides a overview of the variance and gradient intuition for each layer normalization strategy. %Intuitively, as $a$ grows large, the additional normalization steps help keep the gradient magnitude under control, thereby stabilizing training. This result sheds light on why Peri-LN may reduce the sensitivity to large intermediate activations compared to other LN placements. 


\section{Model Mapping} \label{model mapping}

The ever-increasing parameter size of the LLMs, coupled with the lower memory density of PIM, necessitates the distribution of the LLM inference on a scalable network of PIM modules.
In this section, we introduce the mapping of various LLM parallelization strategies on \att{}'s CXL-based network architecture using the proposed collective and peer-to-peer communication primitives.

\subsection{Pipeline-Parallel Mapping (PP)}

Cloud providers serve a large user base, where inference throughput is crucial.
To improve throughput, PP~\cite{gpipe} assigns each transformer block to a pipeline stage.
The individual queries in a batch are simultaneously processed in different stages of the pipeline.
Figure~\ref{fig:Pipeline_Parallelism} shows that we map multiple pipeline stages (\textit{e.g.,} \texttt{T0-3}) to a CXL device (\textit{e.g.,} \texttt{D0}).
Each stage requires multiple PIM channels, depending on the memory requirements of the decoder block.
To prevent excessive communication and keep the latency of pipeline stages identical, we avoid splitting a pipeline stage between the PIM channels of two CXL devices.

In each iteration, the output of each transformer block is transferred to the next pipeline stage.
\att{} performs this data transfer using intra-device communication for pipeline stages within the same CXL device, and using peer-to-peer \textit{send} and \textit{receive} primitives for those in different CXL devices.
This CXL data transfer contains only an 8K embedding vector (\texttt{16KB} data) in Llama2-70B.
The CXL transfer latency of PP is negligible compared to PIM and PNM latencies.

Note that \att{} does not support batch processing within a single pipeline stage because of two primary reasons:
First, batching requires a significantly larger Global Buffer and \rf{} (Section~\ref{subsec:pim_pnm_arch}) to concurrently store the embedding vectors of multiple queries.
Second, batching enhances the operational intensity and compute utilization (Section~\ref{sec:motivation}), while PP fully utilizes PIM compute resources.
Therefore, applying batching on top of PP only increases the latency.

\begin{figure}[h]
    \centering
    \includegraphics[width=8cm]{Figure/Pipeline_Parallelism_new.pdf}
    % \includegraphics[width=\columnwidth]{Figure/Pipeline_Parallelism_new.pdf}
    \caption{Pipeline parallelism: (a) Transformer decoder blocks are distributed across CXL devices and form the pipeline stages. Each block is mapped to multiple GDDR6-PIM channels. (b) Multiple prompts are executed in different stages of the pipeline.}
    \label{fig:Pipeline_Parallelism}
\end{figure}

\subsection{Tensor-Parallel Mapping (TP)}

Inference latency is critical in real-time applications to provide a smooth user experience~\cite{fowers2018configurable}.
To enhance the latency, TP~\cite{alpa, megatron} uses all compute resources to process decoder blocks one at a time.
To implement TP, Figure~\ref{fig:Model_Parallelism}(a) shows that \att{} assigns each transformer decoder block across all CXL devices.
Figure~\ref{fig:Model_Parallelism}(b) illustrates the detailed mapping of a transformer block using TP.
The infrequent residual connection and normalization layers are confined within a single master CXL device.
Distributing the attention layer requires the frequent use of expensive \textit{AllReduce} collective communication primitive, which significantly increases the CXL communication overhead~\cite{megatron}.
Consequently, the attention layer is mapped to the master CXL device.

\begin{figure}[h]
    \centering
    % \includegraphics[width=\columnwidth]{Figure/Model_Parallelism.pdf}
    \includegraphics[width=8cm]{Figure/Model_Parallelism.pdf}
    \caption{(a) Tensor parallelism: each transformer block is assigned to multiple CXL devices. Prompts are processed sequentially. (b) In a transformer block, fully connected layers are spread across CXL devices, while other operations are confined to a single device.}
	\label{fig:Model_Parallelism}
\end{figure}

Prior to the execution of an FC layer, the embedding vector (\texttt{16KB} for Llama2-70B) is \textit{broadcast} from the master CXL device to all devices via the CXL switch.
This enables each device to locally perform GEMV on multiple rows of the weight matrix.
Following the execution of an FC layer, partial result vectors are \textit{gathered} to the master CXL device.
This approach optimizes the execution of FC layers across multiple devices, while reducing the communication overhead of TP through the CXL switch to only \texttt{135KB} data transfer for each transformer block of the Llama2-70B model.

\subsection{Hybrid Tensor-Pipeline Parallel Mapping}~\label{subsec:hybrid_parallel}

The TP and PP mappings focus either on inference latency or throughput.
However, balancing both can be crucial in real-world deployment scenarios when considering Quality of Service (QoS) requirements~\cite{mlperf-sla}.
We explore a hybrid TP-PP strategy to achieve this balance, where each transformer decoder is allocated to multiple consecutive CXL devices. For example, among $32$ devices, mapping each decoder to $32/4=8$ devices enables TP=8 and PP=4.
The embedding vectors are \textit{multicast} and \textit{gathered} by the master CXL device of each pipeline stage.
This configuration effectively reduces token decoding latency by utilizing compute resources from multiple CXL devices (TP), while also improving the throughput by processing multiple prompts in parallel (PP).

\subsection{Transformer Block Mapping} \label{subsec:block_mapping}

\att{} involves a fine-grained mapping of the transformer block onto CXL devices, PNM accelerators, and PIM channels.
This technique permits the complete execution of a transformer block within the CXL device, thereby eliminating the necessity for any interaction with the host system.
Figure~\ref{fig:LLaMA_mapping}(a) illustrates the operations within a Llama2 transformer block. 
Operations within the blue blocks are assigned to PIM channels, including GEMV in fully connected layers, vector dot product in RMSNorm, and element-wise multiplication in RMSNorm, SiLU, Softmax and Rotary Embedding, as detailed in Figure~\ref{fig:LLaMA_mapping}(b), (c), (d), and (e), respectively.
On the other hand, model-specific operations marked in orange, such as square root, division, Softmax, and vector addition in residual connections, are handled by the PNM's RISC-V cores and accelerators.
\att{} supports \textit{Grouped-Query Attention}~\cite{gqa} in Llama2-70B by unrolling GEMM to GEMV.

\begin{figure}[h]
	\centering
    \includegraphics[width=\columnwidth]{Figure/LLaMA_mapping_new.pdf}
    \caption{(a) Llama2-70B Transformer Block. Blue and orange operations are mapped to PIM and PNM PUs, respectively.
    (b)$\sim$(e) Operation mapping for RMSNorm, SiLU, SoftMax and Rotary embedding.}
	\label{fig:LLaMA_mapping}
\end{figure}

In Figure~\ref{fig:LLaMA_mapping}(d), the score dimension varies between $1$ and $4k$, accommodating the 4K sequence length in this example.
The embedding dimensions, as shown in Figure~\ref{fig:LLaMA_mapping}(b) and (c), are set to $8K$.
The rotary embedding process, depicted in Figure~\ref{fig:LLaMA_mapping}(e), begins with the RISC-V PNM cores transforming an attention head of dimension $128$ into $64$ groups of the complex number representations (\textit{e.g.,} $[a, b, c, d]$ to $[(a+jb), (c+jd)]$).
The PIM PUs within memory chips then multiply complex values and pre-loaded weights.
Finally, RISC-V PNM cores convert the computed results back to their real value representations.

\att{}'s PIM computations include three key operations.
This paragraph explains the execution of each operation within a GDDR6-PIM channel.
(a) \textit{GEMV}: The matrix is partitioned along its rows and distributed across all 16 banks. The vector is transferred to the Global Buffer. \texttt{MAC\_ABK} instructions then broadcast 256-bit vector segments from the Global Buffer to all near-bank PUs, retrieve 256-bit segments of the matrix rows from the banks, and perform MAC operations.
(b) \textit{Vector dot product}: In this operation, input vectors are stored in neighboring banks. \texttt{MAC\_ABK} instructions retrieve 256-bit segments from these banks and perform MAC operations. Throughout this process, only one of the two neighboring near-bank PUs is utilized.
(c) \textit{Element-wise multiplication}: Before this operation, input vectors are stored in two banks within each bank group, which consists of four banks. \texttt{EW\_MUL} instructions then retrieve 256-bit segments from these two banks, perform the multiplication, and store the results in another bank within the same bank group.

\subsection{End-to-End Model Mapping}~\label{subsec:e2e_model_mapping}

\att{} supports the end-to-end query execution in LLM inference tasks. In the prefill stage, \att{} processes tokens in the prompt one after another to fill out KV caches, using a similar approach to that in the decoding stage. 
Within each token, both input embeddings and transformer blocks are mapped to CXL devices using the mapping techniques introduced in Section~\ref{subsec:block_mapping}. In the decoding stage, after a series of transformer blocks, the top-k sampling operations are executed on the host CPU.

\subsection{Programming Model}

Users can specify the \att{} hardware configuration, including the number of PIM channels to utilize, and the number of pipeline stages. The tensor mapping strategy is determined by this configuration. \att{} library provides Python APIs to allocate memory space and load model parameters according to the model mapping strategy. 
These APIs also support commonly used LLM operations, such as \texttt{GEMV}, \texttt{LayerNorm}, \texttt{RMSNorm}, \texttt{RoPE}, \texttt{SoftMax}, \texttt{GeLU}, \texttt{SiLU}, \textit{etc.} 
\att{} uses an in-house compiler to generate arithmetic and data movement instructions illustrated in Section~\ref{ISA_Summary}. 


\begin{figure}[h]
    \centering
    % \includegraphics[width=\columnwidth]{Figure/Programming_model_code.pdf}
    \includegraphics[width=8cm]{Figure/Programming_model_code.pdf}
    \caption{Vector-matrix multiplication compilation}
    \label{fig:Programming_model_code}
\end{figure}

Figure~\ref{fig:Programming_model_code} shows an example of compiling \texttt{GEMV} to \att{} instructions. Initially, the operands are designated to particular memory spaces, \textit{i.e.}, the vector operands in the \rf{} and the matrix operands in PIM channels (lines 1 and 2). \att{} instructions are then generated based on input operands' dimensions and memory addresses. Subsequently, the vector is copied to the Global Buffers in the PIM channels with \texttt{WR\_GB} instructions (line 5). This is followed by a sequence of operations for each matrix row within the near-bank PIM PUs. The \texttt{WR\_BIAS} instruction sets up the accumulation registers (line 7). \texttt{MAC\_ABK} performs the multiply-accumulate operations across all near-bank PUs in the PIM channel (line 8). Finally, \texttt{RD\_MAC} retrieves the results from the accumulation registers (line 9).

\section{Methodology}
\label{section:methodology}

Table~\ref{tab:Hardware configurations} lists the system configurations of \att{} and our GPU baseline.
The GPU system contains 4 NVIDIA A100 80GB GPUs equipped with the NVLink 3.0 interconnect.
\att{} has 32 CXL devices, resulting in a similar average power to the GPU system, as further explained in Section~\ref{subsec:power_results}. 

\begin{table}[h]
\footnotesize
\renewcommand\arraystretch{1.1}
\centering
    \caption{Evaluated system configurations}
    \label{tab:Hardware configurations}    
    \scalebox{1}{
        \begin{tabular}{|c||c|c|}
            \hline
            \textbf{System} & \textbf{\att{}} & \textbf{GPU} \\
            \hline
            \hline
            Hardware & 32 CXL devices & 4 NVIDIA A100 \\
            \hline
            Process & 1Y nm (14-16nm) & 7nm \\
            \hline
            Memory & 512GB, GDDR6 & 320GB, HBM2E \\
            \hline
            \multirow{2}{*}{\shortstack{Compute \\ Throughput}} & 512 TFLOPS (PIM) & \multirow{2}{*}{\shortstack{1248 TFLOPS}} \\
            \cline{2-2}
            & 96 TFLOPS (PNM) & \\
            \hline
            Peak Bandwidth & 512 TB/s (Internal) & 8 TB/s (External) \\
            \hline
            3-Year Owned TCO & 0.73\$/hour & 1.76\$/hour \\ 
            \hline
            3-Year Rental TCO & 1.05\$/hour & 5.45\$/hour \\ 
            \hline
            \hline
            GDDR6-PIM & \multicolumn{2}{|c|}{$t_{RCDRD}$=18ns, $t_{RAS}$=27ns, $t_{CL}$=25ns} \\
            Timing Constraints & \multicolumn{2}{|c|}{$t_{RCDWR}$=14ns, $t_{CCDS}$=1ns, $t_{RP}$=16ns} \\
            \hline
        \end{tabular}
    }
\end{table}

We benchmark Llama2 7B, 13B, and 70B models~\cite{touvron2023llama}.
Each evaluated query contains 512 tokens in the prefill stage and 3584 tokens in the decoding stage, adding up to a context length of 4K, \textit{i.e.,} the maximum supported by the Llama2 models.
For a fair comparison between \att{} and the GPU baseline, we deploy these models using different configurations for different parameter sizes: 1, 2, and 4 GPUs, and 8, 20 and 32 CXL devices.
We use vLLM~\cite{vLLM}, the \sota{} inference serving framework on GPUs with a batch size of 128, where the inference throughput saturates (Figure~\ref{fig:Context_Length}).



We generate \att{} instruction traces for a single block and verify the correctness using a functional simulator.
We modify Ramulator2~\cite{luo2023ramulator} to model a CXL device containing 32 GDDR6-PIM memory channels with timing constraints in Table~\ref{tab:Hardware configurations}. 
The inter-device communication through the CXL 3.0 protocol is modeled by an analytical model based on the CXL latency~\cite{li2023pond} and PCIe 6.0 bandwidth. 
To model a CXL switch supporting multicast, we use half of the bandwidth and double the latency of the baseline switch. 
We use Intel Xeon Gold 6430L CPU~\cite{intelxeon} as the host machine in \att{}.

We use Micron DRAM Power Calculator~\cite{micron-power-calculator} to evaluate DRAM core power using current and voltage specifications of Samsung's 8Gb GDDR6 SGRAM C-die~\cite{samsung-8gb-gddr6}.
The MAC operation power is modeled assuming 3$\times$ more current than a typical gapless read~\cite{aim2}.
We assume that each GDDR6 memory controller for two channels consumes 314.6 mW~\cite{dram-controller-power} and each BOOM RISC-V core consumes 250 mW~\cite{boom-pdf}.
We implement the RTL of the remaining components in the CXL controller and synthesize it using a TSMC 28nm technology library and the Synopsys Design Compiler~\cite{synopsis_dc}.
We find the critical path delay as 1ns at 28nm and project the CXL controller clock frequency to be 2.0 GHz at 7nm~\cite{scaling-technology}.

We estimate the die area of CXL controller in two parts. First, we synthesize the custom logic in 28nm~(See Table~\ref{tab:Area_and_power}) and scale it down to 7nm~\cite{scaling-technology}. Then, we add measurements of the memory controller, PCIe controller, and PHY from the NVIDIA GPU die shots~\cite{TU104, A100-die-shot}, which are also scaled down to 7nm. 
This results in an estimated area of 19.0$mm^2$ in 7nm.


\begin{table}[h]
\footnotesize
\centering
\caption{CXL Controller Custom Logic Area\&Power in 28nm}
    \label{tab:Area_and_power}    
    \scalebox{1}{
        \begin{tabular}{|c||c|c|c|}
            \hline
            \multicolumn{2}{|c|}{Components} & Area (mm$^2$) & Power (W) \\
            \hline
            \hline
            \multirow{2}{*}{SRAM} & Instruction Buffer & 3.33 & 0.61 \\
            \cline{2-4}
            % \hline
            & Shared Buffer & 0.11 & 0.03 \\
            \hline
            \multirow{3}{*}{Logics} & Accelerators & 1.34 & 0.18 \\
            \cline{2-4}
            % \hline
            & RISC-V Cores & 2.94 & 0.19 \\
            \cline{2-4}
            % \hline
            & Others & 0.12 & 0.05 \\
            \hline
            \hline
            \multicolumn{2}{|c|}{\textbf{Total}} & \textbf{7.85} & \textbf{1.06} \\
            \hline
        \end{tabular}
    }
\end{table}





\label{TCO}

Table~\ref{tab:Hardware configurations} presents the 3-year Total Cost of Ownership (TCO) for both owned and rental hardware. (a) \textit{Own TCO:} We model a local server by accounting for hardware and operational costs. (b) \textit{Rental TCO:} The cost for host CPU in \att{} and GPU are estimated based on the Microsoft Azure prices ~\cite{azure-price}. The CXL devices in \att{} are evaluated using the owned TCO methodology, as there are no available references for rental costs. To calculate operational cost, we use \$$0.139/KWh$~\cite{electricity-price} and average power consumption. Hardware costs are listed in Table~\ref{tab:hardware_cost}. While the lowest available price for A100 80GB is close to \$20,000, we instead use only \$10,000 by conservatively deducting 50\% margin~\cite{gpu-price}. The PIM module cost is estimated as 10$\times$ the cost of standard DRAM modules~\cite{pim-price, dram-price}.

\begin{table}[h]
    \footnotesize
    \centering
    \caption{Hardware Costs}
    \begin{tabular}{|c||c|c|}
        \hline
        \textbf{System} & \textbf{Hardware} & \textbf{Cost (\$)} \\
        \hline
        \hline
        \multirow{3}{*}{GPU} & Xeon Gold 6430 CPU~\cite{CPU-price} & 2,128 \\
        & 4 NVIDIA A100 80GB GPU~\cite{gpu-price} & 40,000 \\
        \cline{2-3}
        & \textbf{Total Cost} & \textbf{42,128}  \\
        \hline
        \hline
        \multirow{5}{*}{\shortstack{\att{} \\ 32 devices}} & Xeon Gold 6430 CPU~\cite{CPU-price} & 2,128 \\
        & 512GB GDDR6-PIM~\cite{pim-price, dram-price} & 11,873 \\
        & 32 CXL Controllers & 381.3 \\
        & 96-lane 48-port switch~\cite{switch-price} & 490 \\
        \cline{2-3}
        & \textbf{Total Cost} & \textbf{14,873} \\
        \hline
    \end{tabular}
    \label{tab:hardware_cost}
\end{table}

Figure~\ref{fig:TCO} illustrates the breakdown of CXL controller cost per \att{} CXL device (Figure~\ref{fig:CXL_device}). The CXL controller costs are broken down into die, packaging and Non Recurring Engineering (NRE) cost components~\cite{ning2023supply, moonwalk}. Die cost is derived from the wafer cost, considering the CXL controller die area (19.0$mm^2$ in 7nm) and yield rate. A 300mm diameter 7nm wafer costs \$9,346 with a defect density of 0.0015 per $mm^2$~\cite{ning2023supply}. Cost of 2D packaging is assumed to be 29\% of chip cost~\cite{packaging-cost}, while the 2.5D packaging cost is calculated based on interposer, die placement and substrate assembly~\cite{palesko2014cost}.
NRE cost is influenced by chip production volumes, which we estimate at 3 million units based on the following assumptions.
NVIDIA shipped $3.76M$ datacenter GPUs in 2023~\cite{GPU-volume}.
We assume that $10\%$ of datacenter GPUs (around $370K$) are used for LLM inference.
Since each GPU consumes ${\sim}8\times$ more power compared to a CENT device (explained in Section~\ref{subsec:power_results}), we project ${\sim}3M$ volume for \att{} devices.

\begin{figure}[t]
    \centering
    \includegraphics[width=8cm]{Figure/TCO_NEW.pdf}
    \caption{CXL Controller Cost Breakdown}
    \label{fig:TCO}
\end{figure}







\section{Evaluation Results} \label{section:restuls}

\subsection{Interference-Free Analysis}
\noindent
\textbf{Performance of the \texttt{Exchange} primitive.}
Figure~\ref{fig:io-bandwidth} illustrates a comparison of the IO throughput achieved by our optimized \texttt{Exchange} and the baseline solution, which solely relies on the GPU runtime. 
We vary the total amount of data transferred from 2GB to 16GB and adjust the packet size from 10MB to 80MB. 
The combination of data size and packet size determines the total number of packets, which in turn affects the number of pipeline stages required for data transfer. 
Too few pipeline stages can lead to significant overhead in the prologue and epilogue phases of the pipeline. 
Conversely, utilizing excessively small packets is also inefficient, as each memory copy incurs a fixed overhead from the runtime, regardless of the transferred data volume. 
Therefore, small packet sizes exacerbate this overhead, making it disproportionately large.

Our solution achieves up to 140GB/s throughput when transferring 8GB or more of data. 
When the total amount of data transferred is small, we observe a decrease in throughput due to the reduced number of pipeline stages. 
As previously explained, this issue cannot be alleviated by simply reducing the packet size. 
For instance, while a packet size of 10MB provides better performance compared to an 80MB packet size when transferring 2GB of data, it delivers lower throughput when the data size exceeds 8GB. 
Empirically, we find that a packet size of 20MB strikes a balance, achieving desirable performance for small and large data transfers.
Consequently, we use a packet size of 20MB for all the applications evaluated below.

Compared to the baseline, which fully relies on the GPU runtime, our solution is not only more performant but also more stable. 
Such a baseline fails to fully exploit the full-duplex capabilities of PCIe links, achieving only about 110-130GB/s throughput when transferring data bidirectionally. 
Additionally, its performance is highly unstable due to the irregular PCIe bandwidth, especially when the CPU DRAM bandwidth becomes saturated.

%%%%%%%%%%%% OLD TEXT START %%%%%%%%%%%%
\begin{comment}

Figure \ref{fig:io-bandwidth} compares the IO throughput achieved by our optimized \texttt{Exchange} with the one achieved by the baseline solution only relying on the GPU runtime.
We vary the total amount of data transferred from 2GB to 16GB, and the packet size from 10MB to 80MB.
The amount of data and packet size determines the total number of packets, which consequently determines the number of pipeline stages for the data transfer.
If there are too few pipeline stages, the overhead in the prologue and epilogue of the pipeline becomes considerable.
On the other hand, using tiny packets is also unacceptable.
Each memory copy pays a fixed overhead for the runtime regardless of the amount of data being transferred.
Tiny packets make such overhead significant.

Our solution achieves up to around 140GB/s throughput when transferring 8GB or more data.
When the total amount of data transferred is less, we observe decreased throughput due to fewer pipeline stages.
As explained, this can not be relieved by reducing the packet size.
While the case of 10 MB packet size achieves a better performance than the case of 80MB packet size when the total amount of data being transferred is 2GB, it delivers less throughput when the data size is larger than 8GB.
Empirically, we find 20MB is a sweet point that achieves desirable performance in transferring small and large amounts of data.
Thus, we use 20MB for all the applications evaluated below.

Compared to the baseline that fully depends on the GPU runtime, our solution is not only more performant but also more stable.
The baseline solution fails to take advantage of the full-duplex property of PCIe links properly, thus it only achieves around 110-130GB/s throughput when transferring the traffic in both directions. 
Besides, its performance is highly unstable due to the irregular PCIe bandwidth when the CPU DRAM bandwidth is saturating.
\end{comment}
%%%%%%%%%%%% OLD TEXT END %%%%%%%%%%%%

% \begin{figure}
%     \centering
%     \includegraphics[width=0.8\linewidth]{figures/sort-result.pdf}
%     \caption{Results for Sort. (a) the throughput achieved by different solutions, (b) the time breakdown for the \THISWORK\ sort, and (c) the time taken by on-GPU kernel execution of a typical pipeline stage.}
%     \label{fig:sort-perf}
% \end{figure}

\noindent
\textbf{Performance of Sort.}
We compare our sort implementation with CPU and GPU baselines in Figure~\ref{fig:sort-perf}(a). 
Our implementation achieves a throughput of 2.7 billion elements per second, which is 27.9$\times$ faster than TBB, 6.3$\times$ faster than PARADIS, and 1.7$\times$ faster than the configuration using only one GPU's IO resources. 
Figure~\ref{fig:sort-perf}(b) provides a breakdown of the sort operation, revealing that 65.1\% of the time is consumed by the \texttt{SortExKernel}. 
This occurs because, after enhancing the IO throughput, the sorting operation becomes bounded by the GPU processing throughput, as illustrated in Figure~\ref{fig:sort-perf}(c). 
While it takes the GPU approximately 208ms to sort a partition of 500 million 8-byte integers, transferring that partition to the GPU using four GPUs' IO resources requires only about 113ms. 
This limitation explains why we do not achieve nearly a 4$\times$ speedup compared to the single GPU IO solution. 
Conversely, the \texttt{MergeExKernel} remains IO-bound, with the on-GPU kernel completing in approximately 67ms.

%%%%%%%%%%%% OLD TEXT START %%%%%%%%%%%%
\begin{comment}
We compare our sort implementation with the CPU and GPU baselines in Figure~\ref{fig:sort-perf}(a).
Our sort implementation achieves 2.67B elements per second throughput, which is 27.9$\times$ compared to TBB, 6.3$\times$ compared to PARADIS, and 1.7$\times$ compared to the case using only one GPU's IO.
Figure~\ref{fig:sort-perf}(b) is the time breakdown of the sort operation, where 65.1\% of time is spent on the \texttt{SortExOperation}.
The reason is that after we enhance the IO throughput, sorting the array by partition is bounded by GPU-processing throughput, which is showcased in Figure~\ref{fig:sort-perf}(c).
It takes the GPU ~208ms to sort a partition of 500M 8-byte integers, but only ~113ms to transfer that partition to GPU using 4 GBUs' IO resources.
This is why we do not achieve close to 4$\times$ speedup compared to the single GPU IO solution.
On the other hand, \texttt{MergeExOperation} is still IO-bound, which finishes the on-GPU kernel in ~67ms.
\end{comment}
%%%%%%%%%%%% OLD TEXT END %%%%%%%%%%%%

% \begin{figure}
%     \centering
%     \includegraphics[width=0.8\linewidth]{figures/join-result.pdf}
%     \caption{Results for Hash Join. (a) the throughput achieved by different solutions, (b) the time breakdown for the \THISWORK\ hash join, and (c) the time taken by on-GPU kernel execution of a typical pipeline stage.}
%     \label{fig:hash-join-perf}
% \end{figure}

\begin{figure*}[t]
\centerline{\includegraphics[width=\linewidth]{figures/ssb-result.pdf}}
\caption{Star Schema Benchmark execution time and speedup.}
\label{fig:ssb-perf}
\end{figure*}
\begin{figure}
    \centering
    \includegraphics[width=0.86\linewidth]{figures/interference.pdf}
    \caption{Interference between \THISWORK\ on the target GPU and the deep learning applications on the forwarding GPUs. 
    % (a) The slowdown for the deep learning applications (x-axis) when the IO traffic (y-axis) runs in the background. 
    % (b) The slowdown for the \THISWORK\ applications (y-axis) when the deep learning applications (x-axis) run in the background.
    }
    \label{fig:interference}
\end{figure}
\noindent
\textbf{Performance of Hash Join.}
In contrast to sorting, hash join remains an IO-bound kernel even with our IO optimization technique. 
As shown in Figure~\ref{fig:sort-perf}(d), our solution achieves a throughput of 2.3 billion tuples per second. 
This is 24.1$\times$ faster than DuckDB, 2.4$\times$ faster than Triton Join (CPU), 1.3$\times$ faster than the CPU-GPU-NVLink-based Triton Join (GPU), and 3.2$\times$ faster than the single GPU solution using a standard PCIe link.
The speedup over the single GPU IO solution is more pronounced because all phases of hash join are IO-bound. 
This is evident in Figure~\ref{fig:sort-perf}(f). 
The \texttt{HashJoinExKer} requires only 34ms to complete the on-GPU join kernel, which is significantly less than the 61ms required for data transfer.
Similarly, it takes 90ms to partition a chunk of data, which is transferred in around 113ms. 
All phases scale uniformly with the improvement of IO throughput, as depicted in the time breakdown in Figure~\ref{fig:sort-perf}(e), where they consume a comparable amount of time. 
Notably, \THISWORK\ outperforms Triton Join without using proprietary CPU-GPU interconnects by exploiting untapped PCIe bandwidth.
% Notably, while surpassing the performance of Triton Join, our solution relies solely on commodity PCIe links, without utilizing any proprietary CPU-GPU connections to enhance IO throughput.


%%%%%%%%%%%% OLD TEXT START %%%%%%%%%%%%
\begin{comment}
Unlike sort, hash join is still an IO-bound kernel even with our IO-redistribution technique.
As shown in Figure~\ref{fig:hash-join-perf}(a), our solution achieves 2.3 billion tuples per second throughput.
This is around 24.1$\times$ over DuckDB, 2.4$\times$ over the CPU implementation of Triton Join, 1.3$\times$ over the CPU-GPU-NVlink based GPU Triton Join~\cite{triton-join}, and 3.2$\times$ over the single GPU solution with a common PCIe link.
The speedup over the single GPU IO solution is more significant because all hash join phases are IO-bound.
This can be observed from Figure~\ref{fig:hash-join-perf}(c).
The \texttt{HashJoinExOp} takes only ~34ms to finish the on-GPU join kernel, which is much lower than the ~61ms data transfer time.
Similarly, it only takes ~90ms to partition a chunk of data transferred in ~113ms.
All phases scale uniformly with the improvement of IO throughput, thus the time breakdown in Figure~\ref{fig:hash-join-perf}(b) shows that they take a similar amount of time.
While achieving better results than Triton Join, we do not use any proprietary CPU-GPU links to improve the IO through, but solely based on commodity PCIe links.
\end{comment}
%%%%%%%%%%%% OLD TEXT END %%%%%%%%%%%%

\noindent
\textbf{Performance of SSB queries.}
Figure~\ref{fig:ssb-perf} illustrates the comparison of SSB query performance between our solution and the baseline approaches.
On average, our solution achieves a 3.4$\times$ speedup over DuckDB, with all data dynamically fetched from CPU DRAM.
When examining individual query flights, the speedup is 2.4$\times$ for Q1.*, 3.6$\times$ for Q2.*, 3.9$\times$ for Q3.*, and 3.7$\times$ for Q4.*. 
The higher speedup observed in Q2.*, Q3.*, and Q4.* is attributed to their inclusion of more complex multi-way joins.
The more complex multi-way join demands higher memory throughput for hash table probing, thus favoring GPU-based solutions more as they can operate in high-bandwidth GPU memory.
The CPU-based solution has to use the limited DRAM bandwidth on hash table probing and fact table reading, while our solution only uses DRAM bandwidth for the latter.
Lightweight queries like Q11 only filter the fact table based on some predicates, whose only DRAM traffic is reading the fact table once.
Thus, the benefit of high-bandwidth GPU memory is minimized, and we observe less speedup. 


By comparing the bars of \texttt{navie} and \texttt{Proteus-GPU} with \texttt{DuckDB}, it becomes evident that GPU-based solutions struggle to achieve performance comparable to the CPU-based DuckDB without utilizing our IO optimization technique. 
However, this technique alone is insufficient, as indicated by the comparison between the \THISWORK\ and \texttt{DuckDB} bars. 
It only achieves a 1.6$\times$ speedup against \texttt{DuckDB} because it transfers unused data to the GPU without considering column selectivity. 
While zero-copy can exploit selectivity, it falls short of maximizing throughput because it relies on a single PCIe link. 
Notably, using zero-copy alone results in worse performance than \THISWORK\ .
Our final solution dynamically switches between SDMA-based data transfer for columns with selectivity greater than a threshold \(TH\) and zero-copy data transfer for columns with selectivity less than \(TH\).
Our solution also achieves 5.7$\times$ speedup over \texttt{Proteus-Hybrid}, despite that it uses both CPU and GPU.
It is difficult for such a hybrid solution to divide work between CPU and GPU and efficiently utilize the CPU DRAM bandwidth.
Our solution achieves 6.2$\times$ speedup over \texttt{Proteus-Lazy}, which enhances \texttt{Proteus-GPU} with late materialization techniques.
After we resolve the IO bottleneck and fully utilize CPU-side DRAM, a pure GPU-based solution can achieve highly competitive results.


%%%%%%%%%%%% OLD TEXT START %%%%%%%%%%%%
\begin{comment}
Figure~\ref{fig:ssb-perf} shows the comparison of SSB query performance between our solution and the baselines.
On average, our solution achieves 3.4$\times$ speedup over DuckDB, with all the data fetched from CPU DRAM on the fly.
Broken down into each query flight, the speedup is 2.4$\times$ for Q1.*, 3.6$\times$ for Q2.*, 3.9$\times$ for Q3.* and 3.7$\times$ for Q4.*.
More speedup is observed in Q2.*, Q3.*, and Q4.* because they include more complicated multi-way joins.

By comparing the bars of \texttt{navie} and \texttt{Proteus-GPU} with \texttt{DuckDB}, note that GPU-based solutions fail to achieve comparable performance to CPU-based DuckDB without using our IO redistribution technique.
However, this technique only is not enough, as we can see by comparing the bars of \texttt{GPU-IO} with \texttt{DuckDB}. 
It only achieves a 1.6$\times$ speedup against DuckDB, as it transfers unused data to GPU ignoring columns' selectivity.
While we can use zero-copy to exploit selectivity, it fails short in maximum throughput because it only uses one PCIe link.
We can notice that using zero-copy alone only delivers worse performance than \texttt{GPU-IO}.
Our final solution switches between SDMA-based data transfer for columns with selectivity larger than a threshold $TH$ and zero-copy data transfer for columns with selectivity lower than $TH$.
\end{comment}
%%%%%%%%%%%% OLD TEXT END %%%%%%%%%%%%

% \begin{figure}
%     \centering
%     \includegraphics[width=\linewidth]{figures/zero-copy-vs-gpu-io.pdf}
%     \caption{Zero copy vs GPU IO}
%     \label{fig:selectivity-perf}
% \end{figure}

% In our study, we set the threshold \(TH = 64\) based on the formula outlined in \S\ref{sec:design-ssb}. 
% To ensure the accuracy and effectiveness of this threshold, we developed the following micro-benchmark specifically designed for validation purposes.
% \begin{verbatim}
% for i in range(16e9)
%   sum += pred[i % 2e9] % SEL == 0 ? v[i] : 0
% \end{verbatim}
% The \texttt{pred} array resides in GPU memory, and \texttt{SEL} is a hyperparameter that is inversely related to selectivity. 
% We implement this micro-benchmark using both GPU-IO and zero-copy data transfer techniques, varying \texttt{SEL} from 1 to 128. 
% The results are presented in Figure~\ref{fig:selectivity-perf}. 
% Notably, when \texttt{SEL} \(> 64\), zero-copy becomes more efficient. 
% This aligns with the threshold \(TH < \frac{1}{64}\), corroborating the results derived from our formula.

%%%%%%%%%%%% OLD TEXT START %%%%%%%%%%%%
\begin{comment}
The \texttt{pred} array on GPU memory, and \texttt{SEL} is a hyperparameter that is inverse to the selectivity.
We implement this micro-benchmark using both GPU-IO and zero-copy data transfer and varies \texttt{SEL} from 1 to 128.
The result is presented in Figure~\ref{fig:selectivity-perf}.
We notice when \texttt{SEL} $>64$ zero-copy becomes more efficient. 
This corresponds to $TH < \frac{1}{64}$ and matches the result from our formula.
\end{comment}
%%%%%%%%%%%% OLD TEXT END %%%%%%%%%%%%

\subsection{Interference Analysis}
\label{sec:interference}
\noindent
While \THISWORK\ utilizes additional GPUs and their IO resources to forward data to a target GPU, running AI workloads on these auxiliary GPUs can lead to a slowdown of these workloads.
Figure~\ref{fig:interference}(a) presents the slowdown for the AI applications (x-axis) when the IO traffic (y-axis) runs in the background, and (b) shows the slowdown for the \THISWORK\ applications (y-axis) when the deep learning applications (x-axis) run in the background.
(1) Compared to single-direction IO traffic, bidirectional IO traffic has a more significant impact on the performance of foreground applications. This is likely due to the increased stress placed on the memory subsystems of the forwarding GPUs.
(2) Memory-intensive workloads are more susceptible to interference from data forwarding activities, as their performance is constrained by the memory bandwidth available on the GPUs. 
Background data forwarding consumes a portion of the memory bandwidth, leading to an average slowdown of 6.8\%.
Compared to SD3, text embedding generation, and LLM prefilling, LLM decoding experiences a greater degree of slowdown.

Figure~\ref{fig:interference} illustrates that current hardware may not optimize for our IO optimization techniques due to two key observations.
First, although the memory subsystem is theoretically stressed to the same degree in both scenarios, forwarding IO traffic from the device to the host results in a more significant slowdown compared to traffic from the host to the device.
Second, to support the 140GB/s IO throughput we achieved, each GPU incurs an additional memory bandwidth cost of $\frac{140 \times 2}{4} = 70$GB/s, which constitutes only $\frac{70}{1200} \approx 5.8\%$ of the MI100's total bandwidth.
However, empirical observations reveal slowdowns of 7.2\%, 13.4\%, and 16.9\% for \texttt{SD3}, \texttt{Llama3} decoding with a batch size of 32, and \texttt{Llama3} decoding with a batch size of 1, respectively.
We hypothesize that this discrepancy arises because our programming model generates atypical memory traffic that hinders the GPU memory controller's ability to fully utilize bandwidth for the foreground application.

We analyze the slowdown of data analytics applications caused by DL applications on forwarding GPUs. 
As shown in Figure~\ref{fig:interference}, the target GPU experiences less slowdown, with a maximum of 10.4\%. 
However, the slowdown patterns are more irregular compared to forwarding GPUs. 
Text embedding generation and \texttt{Llama3} prefilling cause more interference than \texttt{SD3}, despite all being compute-bound workloads. 
Interestingly, the memory-bound \texttt{Llama3} decoding shows less interference on the target GPU, contrasting with the significant interference on the forwarding GPUs.

%%%%%%%%%%%% OLD TEXT START %%%%%%%%%%%%
\begin{comment}
Besides, Figure~\ref{fig:interference} also shows current hardware may not be able to handle our novel use cases efficiently.
(1) While stressing the memory subsystem to the same degree theoretically, the forwarding IO traffic from device to host causes a greater slowdown than from host to device.
(2) To support the ~140GB/s IO throughput we achieved, each GPU only needs to pay $\frac{140 \times 2}{4} = 70$ GB/s additional memory bandwidth, which is only $\frac{70}{1200} \approx 5.8\%$ of MI100's total bandwidth.
However, we observe 7.2\%, 13.4\%, and 16.9\% slowdown for \texttt{SD3}, \texttt{Llama3} decoding with batch size 32, and \texttt{Llama3} decoding with batch size 1.
We speculate that this is because our new way of programming generates uncommon memory traffic to the GPU memory controller, and prevents it from fully utilizing the maximum memory bandwidth.

Next, we also analyze the slowdown of data analytics applications influenced by DL applications running on the forwarding GPUs.
Less slowdown is observed on the target GPU as shown in Figure~\ref{fig:interference}, where the maximum slowdown is 10.4\%.
However, the slowdown numbers become more irregular compared to the case of forwarding GPUs.
We observe that text embedding generation and \texttt{Llama3} prefilling cause more interference than \texttt{SD3}, although all of them are compute-bound workloads.
Surprisingly, memory-bound \texttt{Llama3} decoding shows less interference on the target GPU, in contrast to the high degree of interference on the forwarding GPUs.
\end{comment}
%%%%%%%%%%%% OLD TEXT END %%%%%%%%%%%%

% \noindent
% \textbf{How are \THISWORK\ applications influenced by the DL applications on the forwarding GPUs?}


\noindent
\textbf{Overall system efficiency.}
Given that our technique can accelerate heavily IO-bound applications by 3 to 4 times, we argue that the system is still more efficient even with a slowdown of up to 16.9\% on the other GPUs.
The improvement of overall system efficiency in a 4-GPU system can be quantified as shown below.
% We discuss how our technique enhances the overall efficiency of a 4-GPU system, as quantified by the following formula.
% Given that our technique can speed up the heavily IO-bounded applications by 3~4$\times$, up to 16.9\% slowdown on the other three GPUs is acceptable. 
% We discuss how much our technique improves the 4-GPU system's efficiency as a whole.
% The improvement of the whole 4-GPU system's efficiency is given by the following formula
$$
\text{speedup}_\text{sys} = \frac{\text{speedup}_\text{t} * \text{slowdown}_\text{t} + 3 * \text{slowdown}_\text{f}}{4}
$$
The subscripts `t' and `f' denote the target GPU and forwarding GPUs, respectively. 
Consider the scenario where \texttt{SD3} and hash join, both with primarily bidirectional IO traffic, are collocated.
The overall system speedup is $\frac{3.2 * (1 - 0.051) + 3 * (1 - 0.072)}{4} \approx 1.45$.
In our setup, the least favorable combination is running \texttt{Llama3} decoding without batching alongside sort. 
Despite this, we still achieve a modest speedup of$\frac{1.7 * (1 - 0.032) + 3 * (1 - 0.169)}{4} \approx 1.03$ speedup.
Note that these speedup values refer to the entire 4-GPU system. 
For a single GPU, they correspond to speedups of 2.8$\times$ and 1.12$\times$, respectively.

%%%%%%%%%%%% OLD TEXT START %%%%%%%%%%%%
\begin{comment}
where the subscript ``t'' and ``f'' mean the target GPU and the forwarding GPUs.
Consider the case of collocating \texttt{SD3} and hash join, whose IO traffic is mainly bidirectional.
The whose system speedup is $\frac{3.2 * (1 - 0.051) + 3 * (1 - 0.072)}{4} \approx 1.45$.
In our setup, the worst combination is running \texttt{Llama3} decoding without batching with sort, but we still achieve a minor $\frac{1.7 * (1 - 0.032) + 3 * (1 - 0.169)}{4} \approx 1.03$ speedup.
Note that the speedup here is in terms of all 4 GPUs, and the speedup above translates to 2.8$\times$ and 1.12$\times$ in terms of a single GPU.
\end{comment}
%%%%%%%%%%%% OLD TEXT END %%%%%%%%%%%%
\section{Related Work}
LLM unlearning~\citep{jang2023knowledgeunlearning, yao2023llmunlearningsurvey, lynch2024eight} has gained significant attention as a method to enhance privacy. Various approaches~\citep{sinha2024unstar, zhang2024npo} have been proposed to ensure that models effectively erase specific information while maintaining overall performance. A key challenge in unlearning is assessing whether knowledge unrelated to the forget set is inadvertently affected. To evaluate this, researchers commonly examine general knowledge~\citep{hendrycks2021measuring, cobbe2021training} as well as a designated subset of the retain set that shares a similar distribution with the forget set but excludes the targeted information. These subsets, often referred to as neighbor sets~\citep{closerlookat}, help determine the extent of unintended degradation in model performance.

In hazardous knowledge unlearning, prior work has leveraged domain-relevant general knowledge as a benchmark. For instance,~\citet{li2024wmdp} employs general biology knowledge to assess the impact of bioweapon-related unlearning and general computer security knowledge to evaluate the removal of information related to Attacking Critical Infrastructure. For entity unlearning~\citep{maini2024tofu, rwku}, previous studies have used entities from similar professions or those closely linked to the target entity as neighbor sets. While these approaches provide an initial framework, they lack a systematic investigation of which aspects of the retain set suffer the most from unlearning. Our study addresses this gap by systematically investigating the impact of unlearning on different types of neighbor sets more clearly and identifying which knowledge components experience the highest degree of forgetting.
\section{Summary and Conclusion}
\label{sec:conclusion}


In this paper, we introduced \ToolName{}, a method for discovering fine-grained \emph{sub-activities} from unlabeled smart home sensor data without relying on pre-segmentation. Our pipeline is organized into two core steps: Clustering and Labeling. 
The \textbf{Clustering step} consists of:

\begin{itemize}
    \item \textbf{Encoder Pre-Training:} We leverage a pre-trained BERT model adapted with sensor-specific tokens and train it using a masked language modeling (MLM) objective to generate context-rich embeddings for raw sensor sequences.
    
    \item \textbf{Clustering Model Fine-Tuning:} Using the SCAN loss function, we fine-tune these embeddings to form more homogeneous and distinct clusters of sensor sequences.
\end{itemize}

The \textbf{Labeling step} comprises:

\begin{itemize}
    \item \textbf{Cluster Centroid Annotation:} Representative sequences from each cluster are visualized with a custom tool, enabling expert annotators to assign meaningful sub-activity labels to the centroids.
    
    \item \textbf{Label Propagation:} The centroid labels are propagated to all sequences within their respective clusters, resulting in a fully labeled dataset with minimal manual effort.
    
    \item \textbf{Re-annotation of Original Time-Series Data:} 
    Finally, these propagated labels are mapped back onto the original time-series data, preserving temporal continuity and facilitating the analysis of longitudinal activity patterns.
\end{itemize}


Our approach addresses important challenges in HAR, including the high cost and effort of manual data annotation, the limitations of coarse activity labels, and the need for scalable and generalizable models. \ToolName{} offers an open source tool that facilitates the HAR annotation and re-annotation process and enables the dynamic discovery and validation of sub-activities, thus capturing a broader spectrum of behaviors observed in real homes.

\begin{acks}
We thank the anonymous reviewers for their valuable feedback. This work was generously supported by NSF CAREER-1652294, NSF-1908601 and Intel gift awards. SAFARI authors acknowledge support from the Semiconductor Research Corporation, ETH Future Computing Laboratory (EFCL), AI Chip Center for Emerging Smart Systems Limited (ACCESS), and the European Union’s Horizon Programme for research and innovation under Grant Agreement No. 101047160.
\end{acks}

%%%%%%% -- PAPER CONTENT ENDS -- %%%%%%%%


%%
%% If your work has an appendix, this is the place to put it.
\appendix

\appendix

\section{Artifact Appendix}

%%%%%%%%%%%%%%%%%%%%%%%%%%%%%%%%%%%%%%%%%%%%%%%%%%%%%%%%%%%%%%%%%%%%%
\subsection{Abstract}

This document provides a concise guide for reproducing the main performance, power, cost efficiency, and energy efficiency results of this paper in Figures~\ref{fig:TCO}, ~\ref{fig:Main_results}, ~\ref{fig:Combo_Main_Lat_Breakdown}, and ~\ref{fig:Energy_Power}.
The instructions cover the steps required to clone the GitHub repository, build the simulator, set up the necessary Python packages, execute the end-to-end simulation, process results, and generate figures.
The trace generator, performance simulator, power model, automation scripts, expected results, and detailed instructions are available in our \href{https://github.com/Yufeng98/CENT}{\red{GitHub repository}}.


\subsection{Artifact check-list (meta-information)}

{\small
\begin{itemize}
  \item {\bf Program:} C++ and Python.
  \item {\bf Compilation:} \texttt{g++-11/12/13} or \texttt{clang++-15}.
  \item {\bf Software:} \texttt{pandas}, \texttt{matplotlib}, \texttt{torch}, and \texttt{scipy} Python packages.
  \item {\bf Model:} Llama2 7B, 13B, and 70B~\cite{touvron2023llama}.
  \item {\bf Metrics:} latency, throughput (tokens/S), cost efficiency (tokens/\$), energy efficiency (tokens/J), and power.
  \item {\bf Output:} \href{https://github.com/Yufeng98/CENT/tree/main/figure_source_data}{\red{CSV}} and \href{https://github.com/Yufeng98/CENT/tree/main/figures}{\red{PDF}} files corresponding to Figures~\ref{fig:TCO}-\ref{fig:Energy_Power}.
  \item {\bf Experiments:} PIM trace generation and simulation, and \att{} power modeling.
  \item {\bf How much disk space is required?:} Approximately 100GB.
  \item {\bf How much time is needed?:} Approximately 24 hours on a desktop and 8~12 hours on a server.
  \item {\bf Publicly available?:} Available on \href{https://github.com/Yufeng98/CENT}{\red{GitHub}} and \href{https://zenodo.org/records/14776547}{\red{Zenodo}}.
  \item {\bf Code licenses:} \href{https://github.com/Yufeng98/CENT/blob/main/LICENSE}{\red{MIT License}}.
  \item {\bf Work automation?:} Automated by a few scripts.
\end{itemize}
}

%%%%%%%%%%%%%%%%%%%%%%%%%%%%%%%%%%%%%%%%%%%%%%%%%%%%%%%%%%%%%%%%%%%%%
\subsection{Description}

This artifact provides the necessary components to reproduce the main results presented in Figures~\ref{fig:TCO}, ~\ref{fig:Main_results}, ~\ref{fig:Combo_Main_Lat_Breakdown},  and ~\ref{fig:Energy_Power}.
It includes a trace generator, AiM simulator, power model, figure generator, and automation script.
While these figures incorporate simulation results from \att{}, they also rely on a baseline GPU system featuring four Nvidia A100 80GB GPUs, as detailed in Table~\ref{tab:Hardware configurations}.
Due to the high cost associated with these servers, only the expected results for the GPU baseline system are provided in the \href{https://github.com/Yufeng98/CENT/tree/main/data}{\red{data}} directory.
% Consequently, this artifact does not include the infrastructure for the GPU baseline.

\subsubsection{How to access}

Clone the artifact from our GitHub repository using the following command. Please do not forget the \texttt{-}\texttt{-recursive} flag to ensure that the AiM simulator is also cloned:

\begin{lstlisting}
git clone --recursive https://github.com/Yufeng98/CENT.git
\end{lstlisting}

% \subsubsection{Hardware dependencies}

\subsubsection{Software dependencies}

AiM simulator requires \texttt{g++-11/12/13} or \texttt{clang++-15} for compilation.
The Python infrastructure requires \texttt{pandas}, \texttt{matplotlib}, \texttt{torch}, and \texttt{scipy} packages.

% \subsubsection{Data sets}

\subsubsection{Models}

Section~\ref{section:methodology} shows that we evaluate three Llama2 models~\cite{touvron2023llama}.
The model architecture and its PIM mapping are implemented in the \texttt{cent\_simulation/Llama.py} script.
The model weights are required only for the functional simulation of the PIM infrastructure. 
While the functional simulator is available in our GitHub repository, the performance simulator and power model described in this appendix do not model real values, as this does not impact the main results.
Consequently, the model weights and parameters are not required for this appendix.


%%%%%%%%%%%%%%%%%%%%%%%%%%%%%%%%%%%%%%%%%%%%%%%%%%%%%%%%%%%%%%%%%%%%%
\subsection{Installation}

\textbf{Building AiM Simulator.}
To build the simulator, use the following script:

\begin{lstlisting}
cd CENT/aim_simulator/
mkdir build && cd build && cmake ..
make -j4
\end{lstlisting}

\textbf{Setting up Python Packages.}
Install the aforementioned Python packages.
You can use the following script to create a \texttt{conda} environment:

\begin{lstlisting}
cd CENT/
conda create -n cent python=3.10 -y
conda activate cent
pip install -r requirements.txt
\end{lstlisting}

%%%%%%%%%%%%%%%%%%%%%%%%%%%%%%%%%%%%%%%%%%%%%%%%%%%%%%%%%%%%%%%%%%%%%
\subsection{Experiment workflow}

We provide scripts to facilitate the end-to-end reproduction of the results. The following steps outline the process.

\textbf{Generate and Simulate the Traces.}  
This step generates and simulates all required PIM traces.
It also processes the simulation logs, calculates individual latencies, and utilizes the \att{} power model to determine energy consumption and average power.
Upon completion, the generated trace and simulation log files will be stored in the \texttt{trace} directory, while the processed latency and power results can be found in \texttt{cent\_simulation/simulation\_results.csv}. 

\begin{lstlisting}
cd CENT/
bash remove_old_results.sh

cd cent_simulation/
bash simulation.sh [NUM_THREADS] [SEQ_GAP]
\end{lstlisting}

\textit{Note:} The argument \texttt{[NUM\_THREADS]} should be set according to the number of available parallel threads on your processor.
For instance, 8 threads are recommended for desktop processors, while server processors can utilize 96 threads. 

The argument \texttt{[SEQ\_GAP]} determines the gap between each simulated token.
Setting this value to one simulates every token sequentially, requiring approximately 100GB of disk space and taking around 24 hours on a processor with 8 threads or 12 hours on a processor with 96 threads.
To improve disk usage and reduce simulation time, the \texttt{[SEQ\_GAP]} argument can be set to a larger value, such as 128. This configuration simulates one out of every 128 tokens, processing token IDs of 128, 256, 384, and so on up to 4096.


\textbf{Process the Results.}  
This step processes the simulation results and computes the latency, throughput, power, and energy for the prefill, decoding, and end-to-end phases.
After processing the results, this script stores them in this file: \texttt{cent\_simulation/processed\_results.csv}.

\begin{lstlisting}
cd CENT/cent_simulation/
bash process_results.sh
\end{lstlisting}

\textbf{Generate Figures.}  
The following script generates Figures~\ref{fig:TCO}-\ref{fig:Energy_Power}.
This process utilizes the baseline GPU results, available in the \href{https://github.com/Yufeng98/CENT/tree/main/data}{\red{data}} directory, along with the processed results.
It computes the normalized results and generates both a PDF file containing the figures and a CSV file with the corresponding numerical data.

\begin{lstlisting}
cd CENT/
bash generate_figures.sh
\end{lstlisting}

%%%%%%%%%%%%%%%%%%%%%%%%%%%%%%%%%%%%%%%%%%%%%%%%%%%%%%%%%%%%%%%%%%%%%
\subsection{Evaluation and expected results}

The normalized results and the figures will be located in the \texttt{figure\_source\_data} and \texttt{figures} directories.
The expected results can be found in Figures~\ref{fig:TCO}-~\ref{fig:Energy_Power} or in the generated \href{https://github.com/Yufeng98/CENT/tree/main/figure_source_data}{\red{CSV}} and \href{https://github.com/Yufeng98/CENT/tree/main/figures}{\red{PDF}} files on our GitHub repository. Figures in the paper are generated using Microsoft Excel. To visualize the figures in the paper's format, copy the normalized data from the CSV files to the \texttt{Data} sheet of the provided \href{https://github.com/Yufeng98/CENT/blob/main/cent_simulation/Figures.xlsx}{\red{Figures.xlsx}}.
Figures will be generated in the \texttt{Figures} sheet.

\clearpage

% %%%%%%%%%%%%%%%%%%%%%%%%%%%%%%%%%%%%%%%%%%%%%%%%%%%%%%%%%%%%%%%%%%%%%
% \subsection{Experiment customization}

% %%%%%%%%%%%%%%%%%%%%%%%%%%%%%%%%%%%%%%%%%%%%%%%%%%%%%%%%%%%%%%%%%%%%%
% \subsection{Notes}

% %%%%%%%%%%%%%%%%%%%%%%%%%%%%%%%%%%%%%%%%%%%%%%%%%%%%%%%%%%%%%%%%%%%%%
% \subsection{Methodology}

% Submission, reviewing and badging methodology:

% \begin{itemize}
%   \item \url{https://www.acm.org/publications/policies/artifact-review-and-badging-current}
%   \item \url{https://cTuning.org/ae}
% \end{itemize}


%%
%% The next two lines define the bibliography style to be used, and
%% the bibliography file.
% \bibliographystyle{ACM-Reference-Format}
\bibliographystyle{plainurl}
\bibliography{references}


\end{document}
\endinput
%%
%% End of file `sample-sigplan.tex'.
