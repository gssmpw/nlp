\appendix

\section{Artifact Appendix}

%%%%%%%%%%%%%%%%%%%%%%%%%%%%%%%%%%%%%%%%%%%%%%%%%%%%%%%%%%%%%%%%%%%%%
\subsection{Abstract}

This document provides a concise guide for reproducing the main performance, power, cost efficiency, and energy efficiency results of this paper in Figures~\ref{fig:TCO}, ~\ref{fig:Main_results}, ~\ref{fig:Combo_Main_Lat_Breakdown}, and ~\ref{fig:Energy_Power}.
The instructions cover the steps required to clone the GitHub repository, build the simulator, set up the necessary Python packages, execute the end-to-end simulation, process results, and generate figures.
The trace generator, performance simulator, power model, automation scripts, expected results, and detailed instructions are available in our \href{https://github.com/Yufeng98/CENT}{\red{GitHub repository}}.


\subsection{Artifact check-list (meta-information)}

{\small
\begin{itemize}
  \item {\bf Program:} C++ and Python.
  \item {\bf Compilation:} \texttt{g++-11/12/13} or \texttt{clang++-15}.
  \item {\bf Software:} \texttt{pandas}, \texttt{matplotlib}, \texttt{torch}, and \texttt{scipy} Python packages.
  \item {\bf Model:} Llama2 7B, 13B, and 70B~\cite{touvron2023llama}.
  \item {\bf Metrics:} latency, throughput (tokens/S), cost efficiency (tokens/\$), energy efficiency (tokens/J), and power.
  \item {\bf Output:} \href{https://github.com/Yufeng98/CENT/tree/main/figure_source_data}{\red{CSV}} and \href{https://github.com/Yufeng98/CENT/tree/main/figures}{\red{PDF}} files corresponding to Figures~\ref{fig:TCO}-\ref{fig:Energy_Power}.
  \item {\bf Experiments:} PIM trace generation and simulation, and \att{} power modeling.
  \item {\bf How much disk space is required?:} Approximately 100GB.
  \item {\bf How much time is needed?:} Approximately 24 hours on a desktop and 8~12 hours on a server.
  \item {\bf Publicly available?:} Available on \href{https://github.com/Yufeng98/CENT}{\red{GitHub}} and \href{https://zenodo.org/records/14776547}{\red{Zenodo}}.
  \item {\bf Code licenses:} \href{https://github.com/Yufeng98/CENT/blob/main/LICENSE}{\red{MIT License}}.
  \item {\bf Work automation?:} Automated by a few scripts.
\end{itemize}
}

%%%%%%%%%%%%%%%%%%%%%%%%%%%%%%%%%%%%%%%%%%%%%%%%%%%%%%%%%%%%%%%%%%%%%
\subsection{Description}

This artifact provides the necessary components to reproduce the main results presented in Figures~\ref{fig:TCO}, ~\ref{fig:Main_results}, ~\ref{fig:Combo_Main_Lat_Breakdown},  and ~\ref{fig:Energy_Power}.
It includes a trace generator, AiM simulator, power model, figure generator, and automation script.
While these figures incorporate simulation results from \att{}, they also rely on a baseline GPU system featuring four Nvidia A100 80GB GPUs, as detailed in Table~\ref{tab:Hardware configurations}.
Due to the high cost associated with these servers, only the expected results for the GPU baseline system are provided in the \href{https://github.com/Yufeng98/CENT/tree/main/data}{\red{data}} directory.
% Consequently, this artifact does not include the infrastructure for the GPU baseline.

\subsubsection{How to access}

Clone the artifact from our GitHub repository using the following command. Please do not forget the \texttt{-}\texttt{-recursive} flag to ensure that the AiM simulator is also cloned:

\begin{lstlisting}
git clone --recursive https://github.com/Yufeng98/CENT.git
\end{lstlisting}

% \subsubsection{Hardware dependencies}

\subsubsection{Software dependencies}

AiM simulator requires \texttt{g++-11/12/13} or \texttt{clang++-15} for compilation.
The Python infrastructure requires \texttt{pandas}, \texttt{matplotlib}, \texttt{torch}, and \texttt{scipy} packages.

% \subsubsection{Data sets}

\subsubsection{Models}

Section~\ref{section:methodology} shows that we evaluate three Llama2 models~\cite{touvron2023llama}.
The model architecture and its PIM mapping are implemented in the \texttt{cent\_simulation/Llama.py} script.
The model weights are required only for the functional simulation of the PIM infrastructure. 
While the functional simulator is available in our GitHub repository, the performance simulator and power model described in this appendix do not model real values, as this does not impact the main results.
Consequently, the model weights and parameters are not required for this appendix.


%%%%%%%%%%%%%%%%%%%%%%%%%%%%%%%%%%%%%%%%%%%%%%%%%%%%%%%%%%%%%%%%%%%%%
\subsection{Installation}

\textbf{Building AiM Simulator.}
To build the simulator, use the following script:

\begin{lstlisting}
cd CENT/aim_simulator/
mkdir build && cd build && cmake ..
make -j4
\end{lstlisting}

\textbf{Setting up Python Packages.}
Install the aforementioned Python packages.
You can use the following script to create a \texttt{conda} environment:

\begin{lstlisting}
cd CENT/
conda create -n cent python=3.10 -y
conda activate cent
pip install -r requirements.txt
\end{lstlisting}

%%%%%%%%%%%%%%%%%%%%%%%%%%%%%%%%%%%%%%%%%%%%%%%%%%%%%%%%%%%%%%%%%%%%%
\subsection{Experiment workflow}

We provide scripts to facilitate the end-to-end reproduction of the results. The following steps outline the process.

\textbf{Generate and Simulate the Traces.}  
This step generates and simulates all required PIM traces.
It also processes the simulation logs, calculates individual latencies, and utilizes the \att{} power model to determine energy consumption and average power.
Upon completion, the generated trace and simulation log files will be stored in the \texttt{trace} directory, while the processed latency and power results can be found in \texttt{cent\_simulation/simulation\_results.csv}. 

\begin{lstlisting}
cd CENT/
bash remove_old_results.sh

cd cent_simulation/
bash simulation.sh [NUM_THREADS] [SEQ_GAP]
\end{lstlisting}

\textit{Note:} The argument \texttt{[NUM\_THREADS]} should be set according to the number of available parallel threads on your processor.
For instance, 8 threads are recommended for desktop processors, while server processors can utilize 96 threads. 

The argument \texttt{[SEQ\_GAP]} determines the gap between each simulated token.
Setting this value to one simulates every token sequentially, requiring approximately 100GB of disk space and taking around 24 hours on a processor with 8 threads or 12 hours on a processor with 96 threads.
To improve disk usage and reduce simulation time, the \texttt{[SEQ\_GAP]} argument can be set to a larger value, such as 128. This configuration simulates one out of every 128 tokens, processing token IDs of 128, 256, 384, and so on up to 4096.


\textbf{Process the Results.}  
This step processes the simulation results and computes the latency, throughput, power, and energy for the prefill, decoding, and end-to-end phases.
After processing the results, this script stores them in this file: \texttt{cent\_simulation/processed\_results.csv}.

\begin{lstlisting}
cd CENT/cent_simulation/
bash process_results.sh
\end{lstlisting}

\textbf{Generate Figures.}  
The following script generates Figures~\ref{fig:TCO}-\ref{fig:Energy_Power}.
This process utilizes the baseline GPU results, available in the \href{https://github.com/Yufeng98/CENT/tree/main/data}{\red{data}} directory, along with the processed results.
It computes the normalized results and generates both a PDF file containing the figures and a CSV file with the corresponding numerical data.

\begin{lstlisting}
cd CENT/
bash generate_figures.sh
\end{lstlisting}

%%%%%%%%%%%%%%%%%%%%%%%%%%%%%%%%%%%%%%%%%%%%%%%%%%%%%%%%%%%%%%%%%%%%%
\subsection{Evaluation and expected results}

The normalized results and the figures will be located in the \texttt{figure\_source\_data} and \texttt{figures} directories.
The expected results can be found in Figures~\ref{fig:TCO}-~\ref{fig:Energy_Power} or in the generated \href{https://github.com/Yufeng98/CENT/tree/main/figure_source_data}{\red{CSV}} and \href{https://github.com/Yufeng98/CENT/tree/main/figures}{\red{PDF}} files on our GitHub repository. Figures in the paper are generated using Microsoft Excel. To visualize the figures in the paper's format, copy the normalized data from the CSV files to the \texttt{Data} sheet of the provided \href{https://github.com/Yufeng98/CENT/blob/main/cent_simulation/Figures.xlsx}{\red{Figures.xlsx}}.
Figures will be generated in the \texttt{Figures} sheet.

\clearpage

% %%%%%%%%%%%%%%%%%%%%%%%%%%%%%%%%%%%%%%%%%%%%%%%%%%%%%%%%%%%%%%%%%%%%%
% \subsection{Experiment customization}

% %%%%%%%%%%%%%%%%%%%%%%%%%%%%%%%%%%%%%%%%%%%%%%%%%%%%%%%%%%%%%%%%%%%%%
% \subsection{Notes}

% %%%%%%%%%%%%%%%%%%%%%%%%%%%%%%%%%%%%%%%%%%%%%%%%%%%%%%%%%%%%%%%%%%%%%
% \subsection{Methodology}

% Submission, reviewing and badging methodology:

% \begin{itemize}
%   \item \url{https://www.acm.org/publications/policies/artifact-review-and-badging-current}
%   \item \url{https://cTuning.org/ae}
% \end{itemize}