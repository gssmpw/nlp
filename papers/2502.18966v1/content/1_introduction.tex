\section{Introduction}

Identifying parameters that deliver satisfactory performance on a wide set of tasks, which we refer to as \emph{general parameters}, is crucial for numerous real-world challenges. 
Examples are the identification of sensor settings that allow the sensor to measure accurately in different environments \citep{guntner_breath_2019}, or the design of footwear that provides good performance for a range of people on different undergrounds \citep{promjun_factors_2012}.
A prominent example comes from the domain of chemical synthesis, where finding reaction conditions under which different starting materials (substrates) can be reliably converted into the corresponding products, remains a critical challenge \citep{feng_general_2015, jagadeesh_mof-derived_2017, wagen_screening_2022, prieto_kullmer_accelerating_2022, rein_generality-oriented_2023, betinol_data-driven_2023, rana_standardizing_2024, schmid_catalysing_2024, sivilotti_active_2025}.
Such general conditions are of particular interest, \textit{e.g.}, in the pharmaceutical industry, where thousands of reactions are carried out regularly, and optimizing each reaction is unfeasible \citep{wagen_screening_2022}.
While Bayesian optimization (BO) is increasingly adopted within reaction optimization \citep{clayton_algorithms_2019, shields_bayesian_2021, guo_bayesian_2023, tom_self-driving_2024}, the vast majority of cases neglect generality considerations (\cref{fig:one}, left-hand side.)
%
\begin{figure*}
  \includegraphics[width=\linewidth]{content/figures/abs_fig_icml_style.pdf}
  \caption{
  \textit{Left:} While conditions can be optimized to maximize the reaction outcome for only one substrate (red), generality-optimized conditions provide a satisfactory reaction outcome for multiple substrates. \textit{Right:} Optimization loop for generality-oriented optimization under partial monitoring.}
  \label{fig:one}
\end{figure*}
%
This lack of consideration can be attributed to the fact that directly observing the generality of selected parameters (\textit{i.e.}, conditions) is associated with largely increased experimental costs, as experimental evaluations on multiple tasks (\textit{i.e.}, substrates) are required.
%, e.g.\ performing Buchwald-Hartwig reactions for multiple aryl halides. 
Attempts at reducing the required number of experiments inevitably increase the complexity of the decision-making process. Thus, the usage of generality-oriented optimization in laboratories is hindered in the absence of appropriate decision-making algorithms. 
Here, generality-oriented optimization turns into a \emph{partial monitoring scenario}, in which each parameter set can only be evaluated on a subset of all possible tasks. 
As a consequence, any iterative experiment planning algorithm needs to recommend both the parameter set and the task for the next experimental evaluation (\cref{fig:one}, right-hand side). 
Experimentally measuring the outcome of the recommended experiment corresponds to a partial observation of the generality objective, which needs to be taken into account when recommending the next experiment.

In the past two years, early studies have targeted the identification of general reaction conditions through variations of BO \citep{angello_closed-loop_2022} and multi-armed bandit optimization \citep{wang_identifying_2024}. Concurrently, different algorithms have been proposed to optimize similarly structured problems, such as BO with expensive integrands \citep[BOEI;][]{xie_optimization_2012, toscano-palmerin_bayesian_2018} and distributionally robust BO \citep[DRBO;][]{bogunovic_adversarially_2018, kirschner_distributionally_2020}. Despite these advances, generality-oriented optimizations are still not commonly performed in real-world experiments (see \cref{subsubsec:related_work}).
This likely arises from the fact that the applicability and limitations of these algorithms are yet to be understood, which is crucial for their effective integration into real-world laboratory workflows \citep{tom_self-driving_2024}.

For these reasons, we herein perform systematic evaluations of generality-oriented optimization.
To obtain a unified framework that flexibly encompasses multiple algorithms and is well-suited for real-world applications \citep{betinol_data-driven_2023}, we formulate generality-oriented optimization as an optimization problem over curried functions.
In addition, we perform systematic benchmarks on various real-world chemical reaction optimization problems.
Specifically for the latter, we (i) confirm the expectation that optimization over multiple substrates (\textit{i.e.}, tasks) leads to more general optima, and (ii) demonstrate that efficient search for these optima can be realized by decoupling parameter and task selection, and highly explorative acquisition of the latter.

In summary, our contributions are four-fold: 
\begin{itemize}[itemsep=0em, topsep=0em]
    \item Formulation of generality-oriented optimization as an optimization problem over a curried function.
    \item Expansion and adaptation of established reaction optimization benchmarks, improving their utility as benchmarks for generality-oriented BO.
    \item Evaluation of different optimization algorithms for identifying general optima.
    \item \textit{CurryBO} as an open-source extension to \textit{BoTorch} \citep{balandat_botorch_2020} for generality-oriented optimization problems: \href{https://github.com/felix-s-k/currybo}{https://github.com/felix-s-k/currybo}.
\end{itemize}