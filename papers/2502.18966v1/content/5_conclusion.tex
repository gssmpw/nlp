\section{Conclusion}

In this work, we extend global optimization frameworks to the identification of general and transferable optima, exemplified by the real-world problem of chemical reaction condition optimization.
Systematic analysis of common reaction optimization benchmarks supports the hypothesis that optimization over multiple related tasks can yield more general optima, particularly in scenarios with a low the density of high-outcome experiments across the search space. We provide augmented versions of these benchmarks to reflect these real-life considerations. 

For BO aimed at identifying general optima, we find that a simple and cost-effective strategy –– sequentially optimizing one-step-lookahead acquisition functions over $\gX$ and $\gW$ -- is well-suited, and performs on par with more complex policies involving two-step lookahead acquisition. Our analyses indicate that the choice of explorative acquisition function for sampling $\gX$ is the most influential factor in achieving successful generality-oriented optimization, likely due to the partial optimization nature of the problem.

While our findings mark an important step towards applying generality-oriented optimization in chemical laboratories, they also highlight the continued need for benchmark problems that accurately reflect real-world scenarios \citep{liang_benchmarking_2021}. We believe that such benchmarks, along with evaluations of chemical reaction representations, are essential for a principled usage of generality-oriented optimization. 
Building on our guidelines, we anticipate that generality-oriented optimization will see increasing adoption in chemistry and beyond, contributing to developing more robust, applicable and sustainable reactions.
We also hope to apply generality-oriented optimization in the setting of self-driving labs in our own laboratories in the near future.