\appendix
\onecolumn

\section{Appendix}

\subsection{Bayesian Optimization for Generality}

\subsubsection{Aggregation Functions} \label{subsubsec: Aggregation_functions}

The aggregation function is a user-defined property that determines how the ``set optimum'' is calculated across objective functions. 
Through the choice of the set optimum, prior knowledge and preferences about the specific optimization problem at hand can be included. 
In this work, the following aggregation functions are evaluated:

\textbf{Mean Aggregation}

\begin{equation}
\phi(f(\rvx; \rvw), \mathcal{W}) = \frac{1}{|\mathcal{W}|} \sum_{\rvw \in \mathcal{W}} f(\rvx; \rvw) = \frac{1}{n} \sum_{i=1}^{n}f(\rvx; \rvw_i)
\end{equation}

\textbf{Threshold Aggregation}

\begin{equation}
\phi\big(f(\rvx; \rvw\big), \mathcal{W}) = \sum_{\rvw \in \mathcal{W}} \sigma\big(f(\rvx; \rvw) - f_{\mathrm{thr}}\big) = \sum_{i=1}^{n} \sigma\big(f(\rvx; \rvw_i) - f_{\mathrm{thr}}\big)
\end{equation}

Conceivably, other aggregation functions also have practical use-cases, for example:

\textbf{Mean Squared Error (MSE) Aggregation}

\begin{equation}
\phi\big(f(\rvx; \rvw), \mathcal{W}\big)  = - \frac{1}{|\mathcal{W}|} \sum_{\rvw \in \mathcal{W}} \big(f_{\mathrm{opt}}(\rvx; \rvw) - f(\rvx; \rvw)\big)^2 = - \frac{1}{n} \sum_{i=1}^n \big(f_{\mathrm{opt},i} - f(\rvx; \rvw_i)\big)^2
\end{equation}

\textbf{Minimum Aggregation}

\begin{equation}
\phi\big(f(\rvx; \rvw), \mathcal{W}\big)  = \min_{\rvw_i \in \mathcal{W}} f(\rvx; \rvw_i)
\end{equation}

The above definitions assume that all $f(\rvx; \rvw_i)$ have the same range, and that the optimization problem is formulated as maximization problem. 

\subsubsection{Acquisition Functions and the Sample Average Approximation} \label{App:acqf}

For the evaluation of posterior distributions, and the calculation of acquisition function values, we use the sample-average approximation, as introduced by \citet{balandat_botorch_2020}. 
From a posterior distribution at time point $k$, $p\big(g_k(\rvx)\big)$, $M$ posterior samples $\zeta_{m}(\rvx) \sim p\big(g_k(\rvx)\big)$ are drawn. 
These posterior samples can be used to estimate the posterior distribution, and to calculate acquisition function values as expectation values $\mathbb{E}_M$ over all $M$ samples.

Herein, we use the following common acquisition functions: 
\begin{itemize}
    \item Upper Confidence Bound: $\text{UCB}(\rvx) = \mathbb{E}_M\big(\zeta_{m}(\rvx)\big) + \beta \cdot \mathbb{E}_M\big(\zeta_{m}(\rvx) - \mathbb{E}_M(\zeta_{m}(\rvx))\big)$.
    \item Expected Improvement: $\text{EI}(\rvx) = \mathbb{E}_M\big(\zeta_{m}(\rvx) - f^*\big)$, where $f^{*}$ is the best value observed so far.
    \item Posterior Variance: $\text{PV}(\rvx) = \mathbb{E}_M\big(\zeta_{m}(\rvx) - \mathbb{E}_M(\zeta_{m}(\rvx))\big)$. 
    \item Random Selection, where the acquisition function value is a random number. 
\end{itemize}

Moreover, we evaluate the optimization performance using a primitive implementation of two-step lookahead acquisition functions $\alpha^{*}$ (see \cref{alg:two_step}). 
The acquisition function value of $\alpha^{*}$ at a location $\mathbf{x_0}$ is estimated as follows: For each of the $M$ posterior samples $\zeta_m(\mathbf{x_0}) \sim p\big(g_k(\mathbf{x_0})\big)$, a fantasy posterior distribution $p'\big(\phi(g_{k+1}(\mathbf{x_0}))\big)$ is generated by conditioning the posterior on the new observation $(\mathbf{x_0}, \zeta_M(\mathbf{x_0}))$ and aggregation. From this fantasy posterior distribution, the values of the inner acquisition function $\alpha_m$ can be computed and optimized over $\rvx \in \mathcal{X}$. The final value of the two-step lookahead acquisition function is returned as $\alpha^{*}(\rvx_0) = \frac{1}{M} \sum_{m=1}^M \alpha_m$.

\begin{algorithm} 

\caption{
Two-step lookahead acquisition function using the sample average approximation.}\label{alg:two_step}

\begin{algorithmic}[1]
\Require
\Statex input space $\mathcal{X}$
\Statex location $\mathbf{x_0}$ at which to evaluate the two-step lookahead acquisition function
\Statex aggregation function $\phi\big(f(\rvx; \rvw), \mathcal{W}\big)$
\Statex posterior distribution  $p\big(g_k (\rvx) \mid \mathcal{D}\big)$
\Statex one-step lookahead acquisition function $\alpha(\rvx)$

\vspace{0.2cm}

\State draw $M$ posterior samples $\zeta_m(\mathbf{x_0}) \sim p\big(g_k(\mathbf{x_0})\big)$
\State empty set of fantasy acquisition function values $\mathcal{A} = \{\}$

\For {\(m = 1, \dots, M\)}

 \State compute fantasy posterior $p'(\rvx) = p\Big(\phi\big(g_{k+1} (\rvx) \mid (\mathcal{D} \cup (\mathbf{x_0}, \zeta_m(\mathbf{x_0}))\big)\Big)$
 \State optimize one-step-lookahead acquisition function $\alpha_m = \underset{\rvx \in \mathcal{X}}{\text{max}} \ \alpha(p'(\rvx))$
 \State update $\mathcal{A} = \mathcal{A} \cup \{\alpha_m$\}

\EndFor
\vspace{0.2cm}
\State \Return $\alpha^{*}(\rvx_0) = \frac{1}{M} \sum_{m=1}^M \alpha_m$

\end{algorithmic}
\end{algorithm}

\subsubsection{Benchmarked Optimization Strategies for Selecting $\rvx_{\text{next}}$ and $\rvw_{\text{next}}$} \label{subsubsec:policies}

Herein, we outline the use of the benchmarked optimization strategies for generality-oriented optimization. The discussed optimization strategies describe different variations of how to pick the next experiments $\rvx_{\text{next}}$ and $\rvw_{\text{next}}$. 

Following the SAA \citep{balandat_botorch_2020} outlined above, we estimate the predictive posterior distribution $p\big(\phi (\rvx) \mid \mathcal{D}\big)$ as follows: 
For each $\rvw_i \in \mathcal{W}$, $M$ (typically $M = 512$ for one-step lookahead strategies and $M = 3$ for two-step lookahead strategies to reduce computational costs) samples $\zeta_{im}(\rvx) \sim p\big(g_k(\rvx, \rvw_i)\big)$ are drawn from the posterior distribution of the surrogate model.
Aggregating over all $\rvw_i$ yields $M$ samples $\zeta_m (\rvx) \sim p\big(\phi (\rvx) \mid \mathcal{D}\big)$ from the posterior distribution over $\phi(\rvx)$, which can be used for calculating the acquisition function values using the sample-based acquisition function logic, as described in \cref{App:acqf}.
With this, we implement and benchmark the acquisition policies in \cref{tab:acquisition_strategies_all}.

\begin{table}[t]
    \footnotesize
    \centering
    \caption{
    Nomenclature and description of the all benchmarked acquisition strategies and acquisition functions, as discussed in the main text and the Appendix. Each experiment is named according to the acquisition strategy used, followed by specifications of the used acquisition functions $\alpha_x$ and $\alpha_w$ or $\alpha$ for sequential and joint acquisitions, respectively.
    As an example, a sequential two-step lookahead acquisition strategy with an Upper Confidence Bound as $\alpha_x$ and Posterior Variance as $\alpha_w$, is referred to as \textsc{Seq 2LA-UCB-PV}.
    }
    \renewcommand{\arraystretch}{1.3}

    \begin{tabularx}{\textwidth}{p{0.55 \textwidth}|X}\toprule
        Acquisition Strategy & Acquisition Function \\ \midrule
        \textsc{Seq 1LA}: Sequential acquisition of $\rvx_\text{next}$ and  $\rvw_\text{next}$, each using a one-step lookahead acquisition function. The final $\hat{\rvx}$ is selected greedily. & \textsc{UCB}: Upper confidence bound ($\beta = 0.5$). \\
        \textsc{Seq 2LA}: Sequential acquisition of $\rvx_\text{next}$ and  $\rvw_\text{next}$, each using a two-step lookahead acquisition function. The final $\hat{\rvx}$ is selected greedily. & \textsc{UCBE}: Upper confidence bound ($\beta = 5$). \\
        \textsc{Joint 2LA}: Joint acquisition of $\rvx_\text{next}$ and  $\rvw_\text{next}$ using a two-step lookahead acquisition function. The final $\hat{\rvx}$ is selected greedily. & \textsc{EI}: Expected Improvement. \\
        \textsc{Bandit}: Multi-armed bandit algorithm as implemented by \citet{wang_identifying_2024}. & \textsc{PV}: Posterior Variance. \\
        \textsc{Random}: Random selection of the final $\hat{\rvx}$. & \textsc{RA}: Random acquisition.\\
         & \textsc{Single}: Selection of the same substrate ($\rvw$) for every iteration. \\
          & \textsc{Complete}: Selection of every substrate (i.e. every $\rvw \in \gW$) for a selected $\rvx_\text{next}$. \\
        \bottomrule
    \end{tabularx}
    \label{tab:acquisition_strategies_all}
\end{table}

The sequential acquisition is described in \cref{alg:sequential_acquisition} and refers to a strategy in which $\rvx_{\text{next}}$ and $\rvw_{\text{next}}$ are selected sequentially. In the first step, $\rvx_{\text{next}}$ is selected by optimizing an $\rvx$-specific acquisition function $\alpha_x$ over $\rvx \in \mathcal{X}$. With the selected $\rvx_{\text{next}}$ in hand, $\rvw_{\text{next}}$ is then selected by optimizing an independent, $\rvw$-specific acquisition function over $\rvw \in \mathcal{W}$. With $\alpha_x = \text{PI}$ (Probability of Improvement) and $\alpha_w = \text{PV}$, this would correspond to the strategy described in \citep{angello_closed-loop_2022}. 
In contrast, the joint acquisition, as outlined in \cref{alg:joint_acquisition}, refers to a strategy in which $\rvx_{\text{next}}$ and $\rvw_{\text{next}}$ are selected jointly through optimization of a two-step lookahead acquisition function (see \cref{alg:two_step} and \cref{App:acqf}).
\newpage
\subsection{Benchmark Problem Details} \label{subsec:problem_details}

\subsubsection{Original Benchmark Problems}

Four chemical reaction benchmarks have been considered in this work: Reactant conversion optimization for Pd-catalyzed C–heteroatom couplings \citep{buitrago_santanilla_nanomole-scale_2015}, enantioselectivity optimization for a N,S-Acetal formation \citep{zahrt_prediction_2019}, yield optimization for a borylation reaction \citep{stevens_advancing_2022, wang_identifying_2024} and yield optimization for deoxyfluorination reaction \citep{nielsen_deoxyfluorination_2018, wang_identifying_2024}. 
Since it has been well-demonstrated that these problems can be effectively modeled by regression approaches \citep{zahrt_prediction_2019, ahneman_predicting_2018, sandfort_structure-based_2020}, we trained a random forest regressor on each dataset, which was used as the ground truth for all benchmark experiments \citep{hase_olympus_2021}.
In the following, the benchmark problems are described briefly.

\textbf{Pd-catalyzed carbon-heteroatom coupling}

The Pd-catalyzed carbon-heteroatom coupling benchmark is concerned with the reaction of different nucleophiles with 3-bromopyridine (\cref{fig:Cernak_reaction}). 
In total, 16 different nucleophiles were tested in a nanoscale high-throughput experimentation platform. 
As reaction conditions, bases (six different bases) and catalysts (16 different catalysts) were varied. 
In total, the benchmark consists of 1536 different experiments, for which the conversion is reported.

\begin{figure}[h]
    \centering
    \includegraphics[]{content/figures/Cernak_reaction_original.png}
    \caption{Reaction diagram of the Pd-catalyzed carbon-heteroatom coupling, where 3-bromopyridine reacts with a nucleophile. Reaction conditions include a catalyst and a base. The numbers indicate the amount of different species in the benchmark.}
    \label{fig:Cernak_reaction}
\end{figure}

The average conversion is $2.05\%$, whereas the maximum conversion is $39.81\%$ (\cref{fig:Cernak_EDA}).
The average of the average conversion of each condition is $2.05\%$, while the maximum of the average conversion of the conditions is $7.60\%$ (\cref{fig:Cernak_EDA}).
The catalyst-base combination with the highest average conversion is shown in \cref{fig:Cernak_EDA}.

\begin{figure}[htb]
    \centering
    \includegraphics[]{content/figures/Cernak_original.pdf}
    \includegraphics[]{content/figures/Cernak_conditions_original_mean.png}
    \caption{Top left: Distribution of the conversion for the Pd-catalyzed carbon-heteroatom coupling in the original benchmark. Top right: Distribution of the average conversion for each catalyst-base combination for the Pd-catalyzed carbon-heteroatom coupling in the original benchmark. Bottom: Catalyst-base combination with the highest average conversion in the original benchmark. Tip = 2,4,6-triisopropylphenyl.}
    \label{fig:Cernak_EDA}
\end{figure}

With respect to the threshold aggregation function, the chosen threshold was $7.50\%$.
The average number of substrates with a conversion above this threshold are $1.615$, while the maximum number of substrates is $7$ (\cref{fig:Cernak_EDA_frac}).
The catalyst-base combination with the highest number of substrates with a conversion above the threshold is the same as shown in \cref{fig:Cernak_EDA}.

\begin{figure}[htb]
    \centering
    \includegraphics[]{content/figures/Cernak_original_frac.pdf}
    \caption{Left: Distribution of the conversion for the Pd-catalyzed carbon-heteroatom coupling in the original benchmark. Right: Distribution of the number of substrates with a conversion above the specified threshold for each catalyst-base combination for the Pd-catalyzed carbon-heteroatom coupling in the original benchmark.}
    \label{fig:Cernak_EDA_frac}
\end{figure}
\newpage
\textbf{N,S-Acetal formation}

The N,S-Acetal formation benchmark is concerned with the nucleophilic addition of different thiols to imines, catalyzed by chiral phosphoric acids (CPAs) (see \cref{fig:Denmark_reaction}). 
In total, five different imines and five different thiols were tested in manual experiments. 
As reaction conditions, 43 different CPA catalysts were considered. 
In total, the benchmark consists of 1075 different experiments, for which $\Delta\Delta G^{\ddagger}$, as a measure of the enantioselectivity, is reported.

\begin{figure}[htb]
    \centering
    \includegraphics[]{content/figures/Denmark_reaction_original.png}
    \caption{Reaction diagram of the N,S-Acetal formation, where an imine reacts with a thiol. Reaction conditions include a catalyst. The numbers indicate the amount of different species in the benchmark.}
    \label{fig:Denmark_reaction}
\end{figure}

The average $\Delta\Delta G^{\ddagger}$ is $0.988$\,kcal/mol, whereas the maximum $\Delta\Delta G^{\ddagger}$ is $3.135$\,kcal/mol (see \cref{fig:Denmark_EDA}).
The average of the average $\Delta\Delta G^{\ddagger}$ for each condition is $0.988$\,kcal/mol, while the maximum of the average $\Delta\Delta G^{\ddagger}$ for all conditions is $2.395$\,kcal/mol (see \cref{fig:Denmark_EDA}).
The catalyst with the highest average $\Delta\Delta G^{\ddagger}$ is shown in \cref{fig:Denmark_EDA}.

\begin{figure}[htb]
    \centering
    \includegraphics[]{content/figures/Denmark_original.pdf}
    \includegraphics[]{content/figures/Denmark_conditions_original_mean.png}
    \caption{Top left: Distribution of $\Delta\Delta G^{\ddagger}$ for the N,S-Acetal formation in the original benchmark. Top right: Distribution of the average $\Delta\Delta G^{\ddagger}$ for each catalyst for the N,S-Acetal formation in the original benchmark. Bottom: Catalyst with the highest average $\Delta\Delta G^{\ddagger}$ in the original benchmark. Cy = Cyclohexyl}
    \label{fig:Denmark_EDA}
\end{figure}

With respect to the threshold aggregation function, the chosen threshold was $2.0$\,kcal/mol.
The average number of substrates with $\Delta\Delta G^{\ddagger}$ above this threshold are $1.907$, while the maximum number of substrates is $17$ (\cref{fig:Denmark_EDA_frac}).
The catalyst with the highest number of substrates with $\Delta\Delta G^{\ddagger}$ above the threshold is the same as shown in \cref{fig:Denmark_EDA}.

\begin{figure}[htb]
    \centering
    \includegraphics[]{content/figures/Denmark_original_frac.pdf}
    \caption{Left: Distribution of $\Delta\Delta G^{\ddagger}$ for the N,S-Acetal formation in the original benchmark. Right: Distribution number of substrates with a  $\Delta\Delta G^{\ddagger}$ above the specified threshold for each catalyst for the N,S-Acetal formation in the original benchmark.}
    \label{fig:Denmark_EDA_frac}
\end{figure}
\newpage
\textbf{Borylation reaction}

The borylation reaction benchmark is concerned with the Ni-catalyzed borylation of different aryl electrophiles (aryl chlorides, aryl bromides, and aryl sulfamates) (\cref{fig:Borylation_reaction}). 
In total, 33 different aryl electrophiles were tested. 
As reaction conditions, ligands (23 different ligands), and solvents (2 different solvents) were varied. 
In total, the benchmark consists of 1518 different experiments, for which the yield is reported.

\begin{figure}[h]
    \centering
    \includegraphics[]{content/figures/Borylation_reaction_original.png}
    \caption{Reaction diagram of the borylation reaction, where different aryl electrophiles are borylated. Reaction conditions include a ligand, and a solvent. The numbers indicate the amount of different species in the benchmark.}
    \label{fig:Borylation_reaction}
\end{figure}

The average yield is $45.5\%$, whereas the maximum yield is $100.0\%$ (\cref{fig:Borylation_EDA}).
The average of the average yield of each condition is $45.5\%$, while the maximum of the average yield of the conditions is $65.4\%$ (\cref{fig:Borylation_EDA}).
The ligand-solvent combination with the highest average yield is shown in \cref{fig:Borylation_EDA}.

\begin{figure}[h]
    \centering
    \includegraphics[]{content/figures/Borylation_original.pdf}
    \includegraphics[]{content/figures/Borylation_conditions_original_mean.png}
    \caption{Top left: Distribution of the yield for the borylation reaction in the original benchmark. Top right: Distribution of the average yield for each ligand-solvent combination for the borylation reaction in the original benchmark. Bottom: Ligand-solvent combination with the highest average yield in the original benchmark. Cy = Cyclohexyl.}
    \label{fig:Borylation_EDA}
\end{figure}

With respect to the threshold aggregation function, the chosen threshold was $90\%$.
The average number of substrates with a yield above this threshold are $1.457$, while the maximum number of substrates is $5$ (\cref{fig:Borylation_EDA_frac}).
The ligand-solvent combination with the highest number of substrates with a yield above the threshold is the same as shown in \cref{fig:Borylation_EDA}. However, the shown ligand-solvent combination is only one of four combinations.

\begin{figure}[h]
    \centering
    \includegraphics[]{content/figures/Borylation_original_frac.pdf}
    \caption{Left: Distribution of the yield for the borylation reaction in the original benchmark. Right: Distribution of the number of substrates with a yield above the specified threshold for each ligand-solvent combination for the borylation reaction in the original benchmark.}
    \label{fig:Borylation_EDA_frac}
\end{figure}
\newpage
\textbf{Deoxyfluorination reaction}

The deoxyfluorination reaction benchmark is concerned with the transformation of different alcohols into the corresponding fluorides (\cref{fig:Deoxyfluorination_reaction}). 
In total, 37 different alcohols were tested. 
As reaction conditions, sulfonyl fluorides (fluoride sources, five different fluorides) and bases (four different bases) were varied. 
In total, the benchmark consists of 740 different experiments, for which the yield is reported.

\begin{figure}[h]
    \centering
    \includegraphics[]{content/figures/Deoxyfluorination_reaction_original.png}
    \caption{Reaction diagram of the deoxyfluorination reaction, where an alcohol is transformed to the corresponding fluoride. Reaction conditions include a fluoride source, and a base. The numbers indicate the amount of different species in the benchmark.}
    \label{fig:Deoxyfluorination_reaction}
\end{figure}

The average yield is $40.4\%$, whereas the maximum yield is $100.6\%$ (\cref{fig:Deoxyfluorination_EDA}).
The yield larger than $100\%$ is contained in the originally published dataset.
The average of the average yield of each condition is $40.4\%$, while the maximum of the average yield of the conditions is $57.2\%$ (\cref{fig:Deoxyfluorination_EDA}).
The fluoride-base combination with the highest average yield is shown in \cref{fig:Deoxyfluorination_EDA}.

\begin{figure}[h]
    \centering
    \includegraphics[]{content/figures/deoxyfluorination_original.pdf}
    \includegraphics[]{content/figures/Deoxyfluorination_conditions_original_mean.png}
    \caption{Top left: Distribution of the yield for the deoxyfluorination reaction in the original benchmark. Top right: Distribution of the average yield for each fluoride-base combination for the deoxyfluorination reaction in the original benchmark. Bottom: Fluoride-base combination with the highest average yield in the original benchmark.}
    \label{fig:Deoxyfluorination_EDA}
\end{figure}

With respect to the threshold aggregation function, the chosen threshold was $90\%$.
The average number of substrates with a yield above this threshold are $1.400$, while the maximum number of substrates is $5$ (\cref{fig:Deoxyfluorination_EDA_frac}).
The fluoride-base combination with the highest number of substrates with a yield above the threshold is shown in \cref{fig:Deoxyfluorination_EDA_frac}.

\begin{figure}[h]
    \centering
    \includegraphics[]{content/figures/deoxyfluorination_original_frac.pdf}
    \includegraphics[]{content/figures/Deoxyfluorination_conditions_original_frac.png}
    \caption{Top left: Distribution of the yield for the deoxyfluorination reaction in the original benchmark. Top right: Distribution of the number of substrates with a yield above the specified threshold for each fluoride-base combination for the deoxyfluorination reaction in the original benchmark. Bottom: Fluoride-base combination with the highest number of substrate with a yield above the threshold in the original benchmark.}
    \label{fig:Deoxyfluorination_EDA_frac}
\end{figure}
\newpage
\subsubsection{Augmentation} \label{subsubsec: augmentation}

Since the described benchmarks consist of a high number of high-outcome experiments (the respective search spaces were rationally designed by expert chemists), we augment them with more negative examples to make them more relevant to real-world optimization campaigns.
New substrates are generated by mutating the originally reported substrates via the STONED algorithm \citep{nigam_beyond_2021}.
In a first filtering step, new substrates were removed if they had a Tanimoto similarity to the original substrate smaller than $0.75$ ($0.6$ for the borylation reaction to obtain a reasonable number of additinal substrates) or if they did not possess the functional groups required for the reaction.
To ensure that the benchmark is augmented with negative examples, random forests are fitted to the original benchmarks (see above).
The mean absolute errors (MAEs), root mean square errors (RMSEs) and r$^2$ score (r$^2$), Spearman's rank correlation coefficient (Spearman's $\rho$) of the random forest regressors fitted to and evaluated on the original benchmarks are shown in \cref{tab:MAE}.
In addition, to evaluate the predictive utility of the random forest regressors, we perform 5-fold cross validation on the original benchmark. The MAE, RMSE, r$^2$ and Spearman's $\rho$ of the 5-fold cross validation are reported in \cref{tab:CV}. Even though the predictive performance on the CV does not achieve a high Spearman's rank coefficient, the comparably low MAEs and RMSEs, as well as high $r^2$ values suggest that they are a reasonable oracle.
Newly generated substrates were incorporated if the average reaction outcome over all reported reaction conditions is below a defined threshold.
The chosen thresholds are $1.0\%$ for the Pd-catalyzed carbon-heteroatom coupling, $0.7$\,kcal/mol for the N,S-Acetal formation, $12\%$ for the borylation reaction, and $5\%$ for the deoxyfluorination reaction.
If a substrate passed these filters, the reactions with all different reported conditions were added, with reaction outcomes being taken from as predicted from the random forest emulator.

\begin{table}[h]
    \centering
    \caption{MAE, RMSE, r$^2$, and Spearman's $\rho$ of random forest regressors fitted to and evaluated on the original benchmark problems.}
    \begin{tabularx}{\textwidth}{lXXXX}\toprule
           Benchmark problem & MAE & RMSE & r$^2$ & Spearman's $\rho$\\\midrule
Pd-catalyzed coupling    & $3.16 \times 10^{-3}$ & $8.75 \times 10^{-3}$ & $0.966$ & $0.898$\\
N,S-Acetal formation & $4.95 \times 10^{-2}$ \text{kcal/mol}& $7.39 \times 10^{-2}$ \text{kcal/mol} & $0.989$ & $0.994$\\
Borylation reaction & $3.62 \times 10^{-2}$ & $4.92 \times 10^{-2}$ & $0.966$ & $0.987$\\
Deoxyfluorination & $2.13 \times 10^{-2}$ & $3.38 \times 10^{-2}$ & $0.986$ & $0.993$\\\bottomrule
    \end{tabularx}
    \label{tab:MAE}
\end{table}

\begin{table}[h]
    \centering
    \caption{MAE, RMSE, r$^2$, and Spearman's rank correlation coefficient with their standard errors of random forest regressors in a 5-fold cross validation on the original benchmark problems.}
    \begin{tabularx}{\textwidth}{lXXXX}\toprule
           Benchmark problem & MAE & RMSE & r$^2$ & Spearman's $\rho$\\\midrule
Pd-catalyzed coupling    & $(9.3 \pm 0.7) \times 10^{-3}$ & $(2.44 \pm 0.18) \times 10^{-2}$ & $0.73 \pm 0.03$ & $0.429 \pm 0.007$\\
N,S-Acetal formation & $(1.43 \pm 0.07) \times 10^{-1}$\,\text{kcal/mol} & $(2.11 \pm 0.10) \times 10^{-1}$\,\text{kcal/mol} & $0.908 \pm 0.010$ & $0.474 \pm 0.007$\\
Borylation reaction & $(1.04 \pm 0.03) \times 10^{-1}$ & $(1.39 \pm 0.04) \times 10^{-1}$ & $0.729 \pm 0.013$ & $0.425 \pm 0.009$\\
Deoxyfluorination & $(5.96 \pm 0.14) \times 10^{-2}$ & $(8.42 \pm 0.15) \times 10^{-2}$ & $0.913 \pm 0.004$ & $0.478 \pm 0.003$\\\bottomrule
    \end{tabularx}
    \label{tab:CV}
\end{table}
\newpage
\subsubsection{Augmented Benchmark Problems}

\textbf{Pd-catalyzed carbon-heteroatom coupling}

Augmentation increases the number of different nucleophiles from 16 to 31 (see \cref{fig:Cernak_reaction_augmented}). Combined with the 96 reported reaction condition combinations, the augmented dataset consists of 2976 reactions, for which the conversion is reported.

\begin{figure}[h]
    \centering
    \includegraphics[]{content/figures/Cernak_reaction_augmented.png}
    \caption{Reaction diagram of the Pd-catalyzed carbon-heteroatom coupling, where 3-bromopyridine reacts with a nucleophile. Reaction conditions include a catalyst and a base. The numbers indicate the amount of different species in the augmented benchmark.}
    \label{fig:Cernak_reaction_augmented}
\end{figure}

Augmentation decreased the average conversion from $2.05\%$ to $1.34\%$, whereas the maximum conversion remained the same at $39.81\%$ (see \cref{fig:Cernak_EDA_augmented}).
The average of the average conversion of each condition is decreased from $2.05\%$ to $1.34\%$, and the maximum of the average conversion of each condition is also decreased from $7.60\%$ to $6.00\%$ (see \cref{fig:Cernak_EDA_augmented}).
The catalyst-base combination with the highest average conversion is unaffected by the augmentation and shown in \cref{fig:Cernak_EDA_augmented}.

\begin{figure}[h]
    \centering
    \includegraphics[]{content/figures/Cernak_augmented.pdf}
    \includegraphics[]{content/figures/Cernak_conditions_augmented_mean.png}
    \caption{Top left: Distribution of the conversion for the Pd-catalyzed carbon-heteroatom coupling in the augmented benchmark. Top right: Distribution of the average conversion for each catalyst-base combination for the Pd-catalyzed carbon-heteroatom coupling in the augmented benchmark. Bottom: Catalyst-base combination with the highest average conversion in the augmented benchmark. Tip = 2,4,6-triisopropylphenyl.}
    \label{fig:Cernak_EDA_augmented}
\end{figure}

With respect to the threshold aggregation function, the chosen threshold was $7.50\%$.
The average number of substrates with a conversion above this threshold are $1.646$, while the maximum number of substrates is $8$ (\cref{fig:Cernak_EDA_frac_augmented}).
The catalyst-base combination with the highest number of substrates with a conversion above the threshold is the same as shown in \cref{fig:Cernak_EDA_augmented}.

\begin{figure}[h]
    \centering
    \includegraphics[]{content/figures/Cernak_augmented_frac.pdf}
    \caption{Left: Distribution of the conversion for the Pd-catalyzed carbon-heteroatom coupling in the augmented benchmark. Right: Distribution of the number of substrates with a conversion above the specified threshold for each catalyst-base combination for the Pd-catalyzed carbon-heteroatom coupling in the augmented benchmark.}
    \label{fig:Cernak_EDA_frac_augmented}
\end{figure}
\newpage
\textbf{N,S-Acetal formation}

Augmentation increases the number of thiols from five to 13, while the number of imines remained constant at five (see \cref{fig:Denmark_reaction_augmented}).
Combined with the 43 reported reaction conditions, the augmented benchmark consists of 2795 reactions, for which $\Delta\Delta G^{\ddagger}$ is reported.

\begin{figure}[h]
    \centering
    \includegraphics[]{content/figures/Denmark_reaction_augmented.png}
    \caption{Reaction diagram of the N,S-Acetal formation, where an imine reacts with a thiol. Reaction conditions include a catalyst. The numbers indicate the amount of different species in the augmented benchmark.}
    \label{fig:Denmark_reaction_augmented}
\end{figure}

Augmentation decreased the average $\Delta\Delta G^{\ddagger}$ from $0.988$\,kcal/mol to $0.757$\,kcal/mol, whereas the maximum $\Delta\Delta G^{\ddagger}$ was slightly decreased from $3.135$\,kcal/mol to $3.114$\,kcal/mol (see \cref{fig:Denmark_EDA_augmented}). 
This decrease is due to the fact that the augmented benchmark only contains values are taken as predicted by the random forest emulator (to investigate optimization performance, the random forest emulator is taken for both the original and augmented benchmarks).
Through augmentation, the average of the average $\Delta\Delta G^{\ddagger}$ of each condition decreased from $0.988$\,kcal/mol to $0.757$\,kcal/mol, while the maximum of the average $\Delta\Delta G^{\ddagger}$ of all conditions decreased as well from $2.395$\,kcal/mol to $1.969$\,kcal/mol (see \cref{fig:Denmark_EDA_augmented}).
The catalyst with the highest average $\Delta\Delta G^{\ddagger}$ is unaffected by the augmentation and shown in \cref{fig:Denmark_EDA_augmented}.

\begin{figure}[h]
    \centering
    \includegraphics[]{content/figures/Denmark_augmented.pdf}
    \includegraphics[]{content/figures/Denmark_conditions_augmented_mean.png}
    \caption{Top left: Distribution of $\Delta\Delta G^{\ddagger}$ for the N,S-Acetal formation in the augmented benchmark. Top right: Distribution of the average $\Delta\Delta G^{\ddagger}$ for each catalyst for the N,S-Acetal formation in the augmented benchmark. Bottom: Catalyst with the highest average $\Delta\Delta G^{\ddagger}$ in the augmented benchmark. Cy = Cyclohexyl.}
    \label{fig:Denmark_EDA_augmented}
\end{figure}

With respect to the threshold aggregation function, the chosen threshold was $2.0$\,kcal/mol.
The average number of substrates with $\Delta\Delta G^{\ddagger}$ above this threshold are $1.814$, while the maximum number of substrates is $16$ (\cref{fig:Denmark_EDA_frac_augmented}).
The catalyst with the highest number of substrates with $\Delta\Delta G^{\ddagger}$ above the threshold is the same as shown in \cref{fig:Denmark_EDA_augmented}.

\begin{figure}[h]
    \centering
    \includegraphics[]{content/figures/Denmark_augmented_frac.pdf}
    \caption{Left: Distribution of $\Delta\Delta G^{\ddagger}$ for the N,S-Acetal formation in the augmented benchmark. Right: Distribution number of substrates with a  $\Delta\Delta G^{\ddagger}$ above the specified threshold for each catalyst for the N,S-Acetal formation in the augmented benchmark.}
    \label{fig:Denmark_EDA_frac_augmented}
\end{figure}
\newpage
\textbf{Borylation reaction}

Augmentation increases the number of different aryl electrophiles from 33 to 75 (see \cref{fig:Borylation_reaction_augmented}). Combined with the 46 reported reaction condition combinations, the augmented dataset consists of 3450 reactions, for which the yield is reported.

\begin{figure}[h]
    \centering
    \includegraphics[]{content/figures/Borylation_reaction_augmented.png}
    \caption{Reaction diagram of the borylation reaction, where an aryl electrophile is borylated via a nickel catalyst. Reaction conditions include a ligand, and a solvent. The numbers indicate the amount of different species in the augmented benchmark.}
    \label{fig:Borylation_reaction_augmented}
\end{figure}

Augmentation decreased the average yield from $45.5\%$ to $26.2\%$, whereas the maximum yield remained the same at $100.0\%$ (see \cref{fig:Borylation_EDA_augmented}).
The average of the average yield of each condition is decreased from $45.5\%$ to $26.2\%$, and the maximum of the average yield of each condition is also decreased from $65.4\%$ to $38.4\%$ (see \cref{fig:Borylation_EDA_augmented}).
The ligand-solvent combination with the highest average yield is unaffected by dataset and augmentation and shown in \cref{fig:Borylation_EDA_augmented}.

\begin{figure}[h]
    \centering
    \includegraphics[]{content/figures/Borylation_augmented.pdf}
    \includegraphics[]{content/figures/Borylation_conditions_augmented_mean.png}
    \caption{Top left: Distribution of the yield for the borylation reaction in the augmented benchmark. Top right: Distribution of the average yield for each ligand-solvent combination for the borylation reaction in the augmented benchmark. Bottom: Ligand-solvent combination with the highest average yield in the augmented benchmark. Cy = Cyclohexyl.}
    \label{fig:Borylation_EDA_augmented}
\end{figure}

With respect to the threshold aggregation function, the chosen threshold was $90\%$.
The average number of substrates with a yield above this threshold are $1.457$, while the maximum number of substrates is $5$ (\cref{fig:Borylation_EDA_frac_augmented}).
Several ligand-solvent combinations provide the highest number of substrates with a yield above the threshold, one of them is shown in \cref{fig:Borylation_EDA_augmented}.
The ligand-solvent combinations are unaffected by the augmentation.

\begin{figure}[h]
    \centering
    \includegraphics[]{content/figures/Borylation_augmented_frac.pdf}
    \caption{Left: Distribution of the yield for the borylation reaction in the augmented benchmark. Right: Distribution of the number of substrates with a yield above the specified threshold for each ligand-solvent combination for the borylation reaction in the augmented benchmark.}
    \label{fig:Borylation_EDA_frac_augmented}
\end{figure}
\newpage
\textbf{Deoxyfluorination reaction}

Augmentation increases the number of different alcohols from 37 to 54 (see \cref{fig:Deoxyfluorination_reaction_augmented}). Combined with the 20 reported reaction condition combinations, the augmented dataset consists of 1080 reactions, for which the yield is reported.

\begin{figure}[h]
    \centering
    \includegraphics[]{content/figures/Deoxyfluorination_reaction_augmented.png}
    \caption{Reaction diagram of the deoxyfluorination reaction, where an alcohol is converted to the corresponding fluoride. Reaction conditions include a fluoride source and a base. The numbers indicate the amount of different species in the augmented benchmark.}
    \label{fig:Deoxyfluorination_reaction_augmented}
\end{figure}

Augmentation decreased the average yield from $40.4\%$ to $28.9\%$, whereas the maximum yield remained the same at $100.6\%$ (see \cref{fig:Deoxyfluorination_EDA_augmented}).
The yield larger than $100\%$ is contained in the originally published dataset.
The average of the average yield of each condition is decreased from $40.4\%$ to $28.9\%$, and the maximum of the average yield of each condition is also decreased from $57.2\%$ to $43.8\%$ (see \cref{fig:Deoxyfluorination_EDA_augmented}).
The fluoride-base combination with the highest average yield is unaffected by augmentation and shown in \cref{fig:Deoxyfluorination_EDA_augmented}.

\begin{figure}[h]
    \centering
    \includegraphics[]{content/figures/Deoxyfluorination_augmented.pdf}
    \includegraphics[]{content/figures/Deoxyfluorination_conditions_augmented_mean.png}
    \caption{Top left: Distribution of the yield for the deoxyfluorination reaction in the augmented benchmark. Top right: Distribution of the average yield for each fluoride-base combination for the deoxyfluorination reaction in the augmented benchmark. Bottom: Fluoride-base combination with the highest average yield in the augmented benchmark.}
    \label{fig:Deoxyfluorination_EDA_augmented}
\end{figure}

With respect to the threshold aggregation function, the chosen threshold was $90\%$.
The average number of substrates with a yield above this threshold are $1.400$, while the maximum number of substrates is $5$ (\cref{fig:Deoxyfluorination_EDA_frac_augmented}).
The fluoride-base combination with the highest number of substrates with a yield above the threshold is also unaffected by augmentation and shown in \cref{fig:Deoxyfluorination_EDA_frac_augmented}.

\begin{figure}[h]
    \centering
    \includegraphics[]{content/figures/deoxyfluorination_augmented_frac.pdf}
    \includegraphics[]{content/figures/Deoxyfluorination_conditions_augmented_frac.png}
    \caption{Top left: Distribution of the yield for the deoxyfluorination reaction in the augmented benchmark. Top right: Distribution of the number of substrates with a yield above the specified threshold for each fluoride-base combination for the deoxyfluorination reaction in the augmented benchmark. Bottom: Fluoride-base combination with the highest number of substrate with a yield above the threshold in the augmented benchmark.}
    \label{fig:Deoxyfluorination_EDA_frac_augmented}
\end{figure}
\newpage
\subsection{Grid Search for Analyzing Benchmark Problems} \label{subsec:data-analysis}

To analyze the utility of considering multiple substrates in an optimization campaign, we performed exhaustive grid search on the described benchmark problems. 
For each problem, the substrates were split into an initial train and test set among the substrates.
In total, thirty different train/test splits were performed.
The obtained train set was further subsampled into smaller training sets with varying sizes to investigate the influence on the number of substrates.
Sampling among the substrates in the train set was performed either through random sampling, farthest point sampling or ``Average Sampling'', where the required number of substrates was chosen as the substrates with the highest average Tanimoto similarity to all other train substrates.
For each subsampled training set, the most general conditions were identified via exhaustive grid search.
The general reaction outcome, as specified by the aggregation function, is evaluated for these conditions on the held-out test set.
Further, this general reaction outcome was scaled from 0 to 1 to give a dataset independent generality score, where 0 is the worst possible general reaction outcome for the given test set and 1 is the best possible general reaction outcome for the test set.
Hence, this score should be maximized.
For the different benchmark problems, we report this generality score, where we also compare the behaviour of the original and augmented problems.
Below, the results of the described data analysis are shown for the benchmark problems not shown in the main text.
\newpage
\subsection{Details on BO for Generality Benchmarking} \label{subsec:BO_benchmark}

To identify whether BO for generality, as described above, can efficiently identify the general optima, we conducted several benchmarking runs on the described benchmark problems.
On each problem, we perform benchmarking for multiple optimization strategies, as listed in \cref{tab:acquisition_strategies_all}.
In each optimization campaign, we used a single-task GP regressor, as implemented in \textit{GPyTorch} \citep{gardner_gpytorch_2018}, with a TanimotoKernel as implemented in \textit{Gauche} \citep{griffiths_gauche_2023}.
Molecules were represented using Morgan Fingerprints \citep{morgan_generation_1965} with 1024 bits and a radius of 2. Fingerprints were generated using RDKit \citep{landrum_rdkit_2023}. It is noteable that, while such a representation was chosen due to its suitability for broad chemical spaces, more specific representations such as descriptors might be able to improve the optimization performance.

The acquisition policies were benchmarked on all benchmark problems with differently sampled substrates for each optimization run.
For each benchmark, we selected the train set randomly, consisting of twelve nucleophiles in the Pd-catalyzed carbon-heteroatom coupling benchmark, three imines and three thiols in the N,S-Acetal formation benchmark, twentyfive alcohols in the Deoxyfluorination reaction, and twenty aryl halides in the Borylation reaction.
Thirty independent optimization campaigns were performed for each.
The generality of the proposed general conditions at each step during the optimization is shown.
\newpage
\subsection{Details on Bandit Algorithm Benchmarking}

The benchmarking of \textsc{Bandit} \citep{wang_identifying_2024} was performed across the benchmark problems using their proposed \textsc{UCB1Tuned} algorithm with differently sampled substrates for the optimization. For each benchmark, we selected the train set randomly, consisting of twelve nucleophiles in the Pd-catalyzed carbon-heteroatom coupling benchmark, three imines and three thiols in the N,S-Acetal formation benchmark, twentyfive alcohols in the Deoxyfluorination reaction, and twenty aryl halides in the Borylation reaction. Thirty independent optimization campaigns were performed for each. To ensure fair comparison, the ground truth was set to be the proxy function calculated for each dataset. To select the optimum $\rvx$ value at each step $k$, we relied on the authors definition of the best arm as the most sampled arm at step $k$.
\newpage
\subsection{Additional Results and Discussion}\label{subsec:add_results}

\subsubsection{Additional Results on the Dataset Analysis for Utility of Generality-oriented Optimization} \label{subsubsec:add_results_data-analysis}

In addition to analysing the utility of generality-oriented optimization for $\phi$ as the mean aggregation, which is shown in \cref{fig:results_dataset-analysis}, we also perform a similar analysis for $\phi$ as the threshold aggregation, where the chosen thresholds are as described in \cref{subsec:problem_details}.
The results of this analysis are shown in \cref{fig:results_dataset-analysis_frac}.
Similar to the case where $\phi$ is the mean aggregation, we observe that in the majority of benchmark problems, more general reaction conditions are obtained by considering multiple substrates.
The only exemption to this observation is the Deoxyfluorination reaction benchmark, a benchmark with a particularly low number of conditions with a high threshold aggregation value (see \cref{fig:Deoxyfluorination_EDA_frac_augmented}).
In addition, we also observe a highly similar behaviour of the original and augmented benchmarks, which is due to the addition of low-performing reactions in the augmentation, which only slightly influences the results of the threshold (i.e. number of high-performing reactions).
\begin{figure}[t]
    \centering
    \includegraphics[width=\linewidth]{content/figures/dataset_analysis_frac.pdf}
    \caption{
    Normalized test-set generality score as determined by exhaustive grid search for the four benchmarks on the original (left) and augmented (right) problems for the threshold aggregation. Average and standard error are taken from thirty different train/test substrates splits.
    }
    \label{fig:results_dataset-analysis_frac}
\end{figure}

Furthermore, we studied how different sampling techniques among the train set substrates influence the obtained generality scores.
As sampling techniques, we used random sampling, farthest point sampling and ``average sampling'', as outlined in \cref{subsec:data-analysis}.
For $\phi$ as the mean aggregation, the results for the four different benchmarks are shown in \cref{fig:results_dataset-analysis_Cernak_combined_mean}, \cref{fig:results_dataset-analysis_Denmark_combined_mean}, \cref{fig:results_dataset-analysis_Borylation_combined_mean}, and \cref{fig:results_dataset-analysis_Deoxyfluorination_combined_mean}.
For $\phi$ as the threshold aggregation, the results for the two different benchmarks are shown in \cref{fig:results_dataset-analysis_Cernak_combined_frac}, \cref{fig:results_dataset-analysis_Denmark_combined_frac}, \cref{fig:results_dataset-analysis_Borylation_combined_frac}, and \cref{fig:results_dataset-analysis_Deoxyfluorination_combined_frac}.
Throughout the different benchmarks and aggregation functions, we observe that the generality score obtained through using the sampled train substrates are highly similar and no method clearly outperforms the others.
It is particularly notable that farthest point sampling did not outperform other sampling techniques, as this strategy is commonly used to select chemicals to broadly cover chemical space \citep{henle_development_2020, gensch_comprehensive_2022, gensch_design_2022, schnitzer_machine_2024}.
We hypothesize that this method insensitivity is due to the low number of substrates chosen for the train set, which was chosen to still reflect realistic experimental cases.

\begin{figure}[t]
    \centering
    \includegraphics[width=\linewidth]{content/figures/dataset_analysis_Cernak_sampling_mean.pdf}
    \caption{Generality score as determined by exhaustive grid search for the Pd-catalyzed carbon-heteroatom coupling benchmark on the original (left) and augmented (right) problems for the mean aggregation as $\phi$. Average and standard error are taken from thirty different train/test substrates splits.}
    \label{fig:results_dataset-analysis_Cernak_combined_mean}
\end{figure}

\begin{figure}[t]
    \centering
    \includegraphics[width=\linewidth]{content/figures/dataset_analysis_Denmark_sampling_mean.pdf}
    \caption{Generality score as determined by exhaustive grid search for the N,S-Acetal formation benchmark on the original (left) and augmented (right) problems for the mean aggregation as $\phi$. Average and standard error are taken from thirty different train/test substrates splits.}
    \label{fig:results_dataset-analysis_Denmark_combined_mean}
\end{figure}

\begin{figure}[t]
    \centering
    \includegraphics[width=\linewidth]{content/figures/dataset_analysis_Borylation_sampling_mean.pdf}
    \caption{Generality score as determined by exhaustive grid search for the Borylation reaction benchmark on the original (left) and augmented (right) problems for the mean aggregation as $\phi$. Average and standard error are taken from thirty different train/test substrates splits.}
    \label{fig:results_dataset-analysis_Borylation_combined_mean}
\end{figure}

\begin{figure}[t]
    \centering
    \includegraphics[width=\linewidth]{content/figures/dataset_analysis_Deoxyfluorination_sampling_mean.pdf}
    \caption{Generality score as determined by exhaustive grid search for the Deoxyfluorination reaction benchmark on the original (left) and augmented (right) problems for the mean aggregation as $\phi$. Average and standard error are taken from thirty different train/test substrates splits.}
    \label{fig:results_dataset-analysis_Deoxyfluorination_combined_mean}
\end{figure}

\begin{figure}[t]
    \centering
    \includegraphics[width=\linewidth]{content/figures/dataset_analysis_Cernak_sampling_frac.pdf}
    \caption{Generality score as determined by exhaustive grid search for the Pd-catalyzed carbon-heteroatom coupling benchmark on the original (left) and augmented (right) problems for the threshold aggregation as $\phi$. Average and standard error are taken from thirty different train/test substrates splits.}
    \label{fig:results_dataset-analysis_Cernak_combined_frac}
\end{figure}

\begin{figure}[t]
    \centering
    \includegraphics[width=\linewidth]{content/figures/dataset_analysis_Denmark_sampling_frac.pdf}
    \caption{Generality score as determined by exhaustive grid search for the N,S-Acetal formation benchmark on the original (left) and augmented (right) problems for the threshold aggregation as $\phi$. Average and standard error are taken from thirty different train/test substrates splits.}
    \label{fig:results_dataset-analysis_Denmark_combined_frac}
\end{figure}

\begin{figure}[t]
    \centering
    \includegraphics[width=\linewidth]{content/figures/dataset_analysis_Borylation_sampling_frac.pdf}
    \caption{Generality score as determined by exhaustive grid search for the Borylation reaction benchmark on the original (left) and augmented (right) problems for the threshold aggregation as $\phi$. Average and standard error are taken from thirty different train/test substrates splits.}
    \label{fig:results_dataset-analysis_Borylation_combined_frac}
\end{figure}

\begin{figure}[t]
    \centering
    \includegraphics[width=\linewidth]{content/figures/dataset_analysis_Deoxyfluorination_sampling_frac.pdf}
    \caption{Generality score as determined by exhaustive grid search for the Deoxyfluorination reaction benchmark on the original (left) and augmented (right) problems for the threshold aggregation as $\phi$. Average and standard error are taken from thirty different train/test substrates splits.}
    \label{fig:results_dataset-analysis_Deoxyfluorination_combined_frac}
\end{figure}

\subsubsection{Additional Results on the Benchmarking on the augmented benchmarks}\label{subsubsec:add_results_augmented}

In addition to the experiments shown in the main text, we benchmarked the sequential one-step and two-step lookahead functions where either a single substrate is selected or in the complete monitoring case.
For both the one-step and two-step lookahead acquisition strategies we observe a significant loss in optimization efficiency for generality-oriented optimization, when only a single substrate is considered (see \cref{fig:results_benchmark_aug_single}).
This is expected, as the constant observation of only one substrate does provide limited information into how different substrates might react, which is unsuitable for generality-oriented optimization.
Similarly, the results shown in \cref{fig:results_benchmark_aug_complete} clearly demonstrate that a complete monitoring scenario is not optimally efficient for generality-oriented optimization.
We hypothesize that this is because the $\gX$ can be more efficiently explored, as not every substrate has to be tested for a specific set of reaction conditions.
This underlines the utility of improved and efficient decision-making algorithms in complex optimization scenarios.

\begin{figure}[t]
    \centering
    \includegraphics[width=\linewidth]{content/figures/single_enhance.pdf}
    \caption{Optimization trajectories of different algorithms used for generality-oriented optimization considering multiple or a single substrate. The trajectories are averaged over all augmented benchmark problems with the mean (left) and threshold (right) aggregations. Optimization algorithms are described in \cref{tab:acquisition_strategies}.}
    \label{fig:results_benchmark_aug_single}
\end{figure}
\begin{figure}[t]
    \centering
    \includegraphics[width=\linewidth]{content/figures/complete_enhance.pdf}
    \caption{Optimization trajectories of different algorithms used for generality-oriented optimization considering the partial or complete monitoring case, respectively. The trajectories are averaged over all augmented benchmark problems with the mean (left) and threshold (right) aggregations. Optimization algorithms are described in \cref{tab:acquisition_strategies}.}
    \label{fig:results_benchmark_aug_complete}
\end{figure}

\subsubsection{Additional Results on the Benchmarking on the Original Benchmarks}\label{subsubsec:add_results_original}
In addition to the results described above, we also benchmark the strategies described in \cref{tab:acquisition_strategies_all} on the original benchmarks.
In general, we observe highly similar results compared to the augmented benchmarks that have already been discussed.
This emphasizes that, while augmentation of established benchmarks remains necessary to reflect real-world conditions, the conclusions on algorithmic performances remain largely unaffected from the biases within the benchmarks.
A high robustness in optimization performance on benchmark distribution further increases the utility of generality-oriented optimization in the laboratory.

Specifically, we find that, similar to the augmented benchmarks, the \textsc{Seq 1LA-UCB-PV} strategy shows a significantly better optimization performance than other algorithms published in the chemical domain (see \cref{fig:results_average_runs_orig}).
Comparing multiple one-step and two-step lookahead acquisition strategies, with varying $\alpha_x$ again emphasizes that both strategies perform similarly and that an explorative acquisition of $\rvx_{\text{next}}$ is crucial for successful generality-oriented optimization (see \cref{fig:results_sequential_orig}).
Confirming results from the augmented benchmarks, we also observe that a variation in $\alpha_w$ does not affect the optimization performance of the one-step lookahead acquisition strategy, while a random acquisition of $\rvw_{\text{next}}$ leads to less efficient optimizations for two-step lookahead strategies (see \cref{fig:results_sequential_orig}).
In addition, we also confirm the surprising empirical observation that a joint acquisition of $\rvx_{\text{next}}$ and $\rvw_{\text{next}}$ does not yield to a significantly improved optimization performance compared to a sequential acquisition (see \cref{fig:results_joint_orig}).
\begin{figure}[t]
    \centering
    \includegraphics[width=\linewidth]{content/figures/chemistry_methods.pdf}
    \caption{Optimization trajectories of different algorithms used for generality-oriented optimization in the chemical domain. The trajectories are averaged over all original benchmark problems with the mean (left) and threshold (right) aggregations. Optimization algorithms are described in \cref{tab:acquisition_strategies}.
    }
    \label{fig:results_average_runs_orig}
\end{figure}
\begin{figure}[t]
    \centering
    \includegraphics[width=\linewidth]{content/figures/sequential.pdf}
    \caption{Optimization trajectories of different sequential acquisition strategies for generality-oriented optimization. The top row shows the influence of variation of the acquisition strategy of $\rvx_{\text{next}}$ (i.e., variation of $\alpha_x$), while the bottom row shows the influence of variation of the acquisition strategy of $\rvw_{\text{next}}$ (i.e., variation of $\alpha_w$). The trajectories are averaged over all original benchmark problems with the mean (left) and threshold (right) aggregations. Optimization algorithms are described in \cref{tab:acquisition_strategies}.
    }
    \label{fig:results_sequential_orig}
\end{figure}
\begin{figure}[t]
    \centering
    \includegraphics[width=\linewidth]{content/figures/joint.pdf}
    \caption{Optimization trajectories of sequential and joint two-step lookahead acquisition strategies for generality-oriented optimization. The trajectories are averaged over all benchmark problems with the mean (left) and threshold (right) aggregations. Optimization algorithms are described in \cref{tab:acquisition_strategies}.
    }
    \label{fig:results_joint_orig}
\end{figure}

Lastly, we also demonstrate that a generality-oriented optimization with a single substrate and in the complete monitoring case leads to suboptimal optimization performance, as shown in \cref{fig:results_benchmark_orig_single} and \cref{fig:results_benchmark_orig_complete}, respectively.

\begin{figure}[t]
    \centering
    \includegraphics[width=\linewidth]{content/figures/single.pdf}
    \caption{Optimization trajectories of different algorithms used for generality-oriented optimization considering multiple or a single substrate. The trajectories are averaged over all original benchmark problems with the mean (left) and threshold (right) aggregations. Optimization algorithms are described in \cref{tab:acquisition_strategies}.}
    \label{fig:results_benchmark_orig_single}
\end{figure}
\begin{figure}[t]
    \centering
    \includegraphics[width=\linewidth]{content/figures/complete.pdf}
    \caption{Optimization trajectories of different algorithms used for generality-oriented optimization considering the partial or complete monitoring case, respectively. The trajectories are averaged over all original benchmark problems with the mean (left) and threshold (right) aggregations. Optimization algorithms are described in \cref{tab:acquisition_strategies}.}
    \label{fig:results_benchmark_orig_complete}
\end{figure}