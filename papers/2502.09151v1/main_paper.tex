%%%%%%%% ICML 2025 EXAMPLE LATEX SUBMISSION FILE %%%%%%%%%%%%%%%%%

\documentclass{article}

% Recommended, but optional, packages for figures and better typesetting:
\usepackage{microtype}
\usepackage{graphicx}
\usepackage{subfigure}
\usepackage{booktabs} % for professional tables
\usepackage{hyperref}
\section{Notations}
\label{app:notation}

\begin{table}[H]
\begin{threeparttable}
\caption{Core notations used the main text and appendix.}
\label{table:symbols_and_notations}
\small
\centering
\fontsize{7}{7}\selectfont
\begin{tabular}{c | c | c}
\toprule
Notation & Dimension(s) & Definition \\
\midrule
\(\mathcal{N}_{\lambda}^{\tt LS}\) & - & The \(\ell_2\) norm of the linear regression estimator under regularization \(\lambda\) for linear regression \\
\(\mathcal{B}_{\mathcal{N},\lambda}^{\tt LS}\) & - & The bias of \(\mathcal{N}_{\lambda}^{\tt LS}\) \\
\(\mathcal{V}_{\mathcal{N},\lambda}^{\tt LS}\) & - & The variance of \(\mathcal{N}_{\lambda}^{\tt LS}\) \\ \midrule
\(\sN_{\lambda}^{\tt LS}\) & - & The deterministic equivalent of \(\mathcal{N}_{\lambda}^{\tt LS}\) \\
\(\sB_{\sN, \lambda}^{\tt LS}\) & - & The deterministic equivalent of \(\mathcal{B}_{\mathcal{N},\lambda}^{\tt LS}\) \\
\(\sV_{\sN, \lambda}^{\tt LS}\) & - & The deterministic equivalent of \(\mathcal{V}_{\mathcal{N},\lambda}^{\tt LS}\) \\
\midrule
$\left \| \bm{v} \right \|_2$ & - & Euclidean norms of vectors $\bm{v}$ \\
$\left \| \bm{v} \right \|_\bSigma$ & - & $\sqrt{\bv^\sT \bSigma \bv}$ \\
\midrule
$n$ & - & Number of training samples \\
$d$ & - & Dimension of the data for linear regression \\
$p$ & - & Number of features for random feature model \\
$\lambda$ & - & Regularization parameter \\
$\lambda_*$ & - & Effective regularization parameter for linear ridge regression \\
$\nu_1\,,\nu_2$ & - & Effective regularization parameters for random feature ridge regression \\
$\sigma_k(\bM)$ & - & The $k$-th eigenvalue of $\bM$ \\
\midrule
$\bm{x}$ & $\mathbb{R}^{d}$ & The data vector \\
$\bX$ & $\mathbb{R}^{n \times d}$ & The data matrix\\
$\bSigma$ & $\mathbb{R}^{d \times d}$ & The covariance matrix of $\bx$\\
$y$ & $\mathbb{R}$ & The label \\
$\by$ & $\mathbb{R}^{n}$ & The label vector \\
$\bbeta_*$ & $\mathbb{R}^{d}$ & The target function for linear regression \\
$\hat{\bbeta}$ & $\mathbb{R}^{d}$ & The estimator of ridge regression model \\
$\hat{\bbeta}_{\min}$ & $\mathbb{R}^{d}$ & The min-$\ell_2$-norm estimator of ridge regression model \\
$\varepsilon$ & $\mathbb{R}$ & The noise \\
$\varepsilon_i$ & $\mathbb{R}$ & The $i$-th noise \\
$\bm\varepsilon$ & $\mathbb{R}^{n}$ & The noise vector \\
$\sigma^2$ & $\mathbb{R}$ & The variance of the noise\\ \midrule
$\bw_i$ & $\mathbb{R}^{d}$ & The $i$-th weight vector for random feature model \\ 
$\varphi(\cdot;\cdot)$ & - & Nonlinear activation function for random feature model \\
$\bz_i$ & $\mathbb{R}^{p}$ & The $i$-th feature for random feature model \\
$\bZ$ & $\mathbb{R}^{n \times p}$ & Feature matrix for random feature model \\
$\hat{\ba}$ & $\mathbb{R}^{p}$ & The estimator of random feature ridge regression model\\
$\hat{\ba}_{\min}$ & $\mathbb{R}^{p}$ & The min-$\ell_2$-norm estimator of random feature ridge regression model\\ \midrule
$f_*(\cdot)$ & - & The target function \\
$\mu_\bx$ & - & The distribution of $\bx$ \\
$\mu_\bw$ & - & The distribution of $\bw$ \\
$\mathbb{T}$ & - & An integral
operator defined by $(\mathbb{T}f)(\bw) := \int_{\mathbb{R}^d} \varphi(\bx; \bw) f(\bx) \mathrm{d}\mu_\bx \,,\quad \forall f \in L_2(\mu_\bx)$ \\
$\mathcal{V}$ & - & The image of $\mathbb{T}$\\
$\xi_k$ & $\mathbb{R}$ & The $k$-th eigenvalue of $\mathbb{T}$, defined by
$\mathbb{T} = \sum_{k=1}^\infty \xi_k \psi_k \phi_k^*$ \\
$\psi_k$ & - & The $k$-th eigenfunction of $\mathbb{T}$ in the space $L_2(\mu_\bx)$, defined by the decomposition
$\mathbb{T} = \sum_{k=1}^\infty \xi_k \psi_k \phi_k^*$ \\
$\phi_k$ & - & The $k$-th eigenfunction of $\mathbb{T}$ in the space $\mathcal{V}$, defined by the decomposition
$\mathbb{T} = \sum_{k=1}^\infty \xi_k \psi_k \phi_k^*$ \\
$\bLambda$ & $\mathbb{R}^{\infty \times \infty}$ & The spectral matrix of $\mathbb{T}$, $\bLambda = \operatorname{diag}(\xi_1^2, \xi_2^2, \ldots) \in \mathbb{R}^{\infty \times \infty}$ \\
$\bg_i$ & $\mathbb{R}^{\infty}$ & $\bg_i := (\psi_k(\bx_i))_{k \geq 1}$\\
$\boldf_i$ & $\mathbb{R}^{\infty}$ & $\boldf_i := (\xi_k\phi_k(\bw_i))_{k \geq 1}$\\
$\bG$ & $\mathbb{R}^{n \times \infty}$ & 
$\bG \!:=\! [\bg_1, \ldots, \bg_n]^\sT \!\in\! \mathbb{R}^{n \times \infty}$ with $\bg_i := (\psi_k(\bx_i))_{k \geq 1}$\\
$\bF$ & $\mathbb{R}^{p \times \infty}$ & $\bF \!:=\! [\boldf_1, \ldots, \boldf_p]^\sT \!\in\! \mathbb{R}^{p \times \infty}$\\
$\hbLambda_\bF$ & $\mathbb{R}^{p \times p}$ & $\hbLambda_\bF := \E_\bz[\bz\bz^\sT|\bF] = \frac{1}{p}\bF\bF^\sT \in \R^{p \times p}$ \\
$\btheta_{*,k}$ & $\mathbb{R}$ & The coefficients associated with the eigenfunction $\psi_k$ in the expansion of $f_*(\bx)=\sum_{k\geq1}\btheta_{*,k}\psi_k(\bx)$ \\
$\btheta_*$ & $\mathbb{R}^{\infty}$ & $\btheta_* = (\btheta_{*,k})_{k \geq 1}$ \\
\midrule
\end{tabular}
\begin{tablenotes}
    \footnotesize
    \item[1] Replacing $\mathcal{N}$ with $\mathcal{R}$ ($\sN$ with $\sR$), we get the notations associated to the test risk.
    \item[2] Replacing $\lambda$ with $0$, we get the notations associated to the min-$\ell_2$-norm solution.
    \item[3] Replacing ${\tt LS}$ with ${\tt RFM}$, we get the notations associated to random feature regression.
\end{tablenotes}
\end{threeparttable}
\end{table}
\usepackage{xcolor}
% \usepackage{algorithm}
% \usepackage{algorithmic}

\usepackage{caption}

\newcommand{\MH}[1]{\color{blue}``\small MH: #1"\color{black}}
\newcommand{\jl}[1]{\color{red}``\small jl: #1"\color{black}}
\usepackage{amsmath}
\DeclareMathOperator*{\argmin}{arg\,min}
% hyperref makes hyperlinks in the resulting PDF.
% If your build breaks (sometimes temporarily if a hyperlink spans a page)
% please comment out the following usepackage line and replace
% \usepackage{icml2025} with \usepackage[nohyperref]{icml2025} above.



% Attempt to make hyperref and algorithmic work together better:
\newcommand{\theHalgorithm}{\arabic{algorithm}}

% Use the following line for the initial blind version submitted for review:
\usepackage[accepted]{icml2025}

% If accepted, instead use the following line for the camera-ready submission:
% \usepackage[accepted]{icml2025}

% For theorems and such
\usepackage{amsmath}
\usepackage{amssymb}
\usepackage{mathtools}
\usepackage{amsthm}

% if you use cleveref..
\usepackage[capitalize,noabbrev]{cleveref}

%%%%%%%%%%%%%%%%%%%%%%%%%%%%%%%%
% THEOREMS
%%%%%%%%%%%%%%%%%%%%%%%%%%%%%%%%
\theoremstyle{plain}
\newtheorem{theorem}{Theorem}[section]
\newtheorem{proposition}[theorem]{Proposition}
\newtheorem{lemma}[theorem]{Lemma}
\newtheorem{corollary}[theorem]{Corollary}
\theoremstyle{definition}
\newtheorem{definition}[theorem]{Definition}
\newtheorem{assumption}[theorem]{Assumption}
\theoremstyle{remark}
\newtheorem{remark}[theorem]{Remark}

% Todonotes is useful during development; simply uncomment the next line
%    and comment out the line below the next line to turn off comments
%\usepackage[disable,textsize=tiny]{todonotes}
\usepackage[textsize=tiny]{todonotes}


% The \icmltitle you define below is probably too long as a header.
% Therefore, a short form for the running title is supplied here:
\icmltitlerunning{}

\begin{document}

\twocolumn[
\icmltitle{Regularization can make diffusion models more efficient}

\icmlsetsymbol{equal}{*}

\begin{icmlauthorlist}
\icmlauthor{Mahsa Taheri}{y}
\icmlauthor{Johannes Lederer}{y}
\end{icmlauthorlist}

\icmlaffiliation{y}{Department of Mathematics, Computer Science, and Natural Sciences, Hamburg University, Germany}
% \icmlaffiliation{comp}{Company Name, Location, Country}
% \icmlaffiliation{sch}{School of ZZZ, Institute of WWW, Location, Country}

\icmlcorrespondingauthor{Mahsa Taheri}{mahsa.taheri@uni-hamburg.de}
% \icmlcorrespondingauthor{Firstname2 Lastname2}{first2.last2@www.uk}

% You may provide any keywords that you
% find helpful for describing your paper; these are used to populate
% the "keywords" metadata in the PDF but will not be shown in the document
% \icmlkeywords{Machine Learning, ICML}

\vskip 0.3in
]

\printAffiliationsAndNotice{} % otherwise use the standard text.

\begin{abstract}
Diffusion models are one of the key architectures of generative AI. 
Their main drawback, however, is the computational costs.
This study indicates that the concept of sparsity, well known especially in statistics, can provide a pathway to more efficient diffusion pipelines.
Our mathematical guarantees prove that sparsity can reduce the   input dimension's influence on the computational complexity to that of a much smaller intrinsic dimension of the data.
Our empirical findings confirm that inducing sparsity can indeed lead to better samples at a lower cost.
\end{abstract}

\section{Introduction}
% ###############################################
% Start of file - body.tex
% ###############################################

% ===============================================
% Section
% ===============================================
\section{Introduction}
\label{sec:introduction}
One of the important activities involved in a successful strategy towards predictive maintenance for industrial Cyber-Physical Systems (CPS) is anomaly detection and identification. Examples of such systems are semiconductor photolithography machines, production printing machines, die bonder machines, and so forth. What these systems all have in common is the presence of highly complex, multi-node compute and control elements, limited domain of operational tasks (highly purpose-built), and continuous high yield targets for machine production output.

In the context of industrial CPS, data-centric solutions consuming time-series data from machine sensors, have proven to be highly capable~\cite{Odyurt:2022:IRIC}. For such solutions, there are numerous data processing and Machine Learning algorithms suitable for time-series data analysis, to choose from. Generally speaking, with industrial CPS, we also have the abundance of available data, which can be collected from a multitude of available sensors, especially in modern CPS, while the machine operates. Needless to say, these machines are intended to operate non-stop, at full capacity, requiring any data collection and monitoring to be well-planned.

Contrary to one's initial assumption, the abundance of data becomes a challenge. Besides the complexities and resource cost imposed with excessive data collection, high amounts of data does not necessarily lead to better prediction. As such, \emph{it is highly advantageous to be able to select the right data processing steps, choose the best ML algorithm, and focus on the most effective portion of the data}.

It is even more advantageous to know which of the above ingredients (data processing, ML algorithm and data subset) match and work best, allowing for the selection of the most effective combination, should one ingredient be restricted. For instance, if we are limited to a specific part of data, the best complementary ML algorithm shall be considered. \emph{Most importantly, we want to know all such compatibilities upfront}.

\paragraph*{Contribution}
We introduce the first iteration of our \emph{InfoPos framework}, intended to support designers and engineers in the selection of most effective elements when building ML-assisted solutions for industrial Cyber-Physical Systems (CPS). Examples of such element variations are the type of ML algorithm, data processing/transformation steps applied, or the  level of these steps, and the considered portion of data. We demonstrate the use of InfoPos framework within the context of an anomaly identification use-case. Our results are based on real data and our data processing code, as well as the generated data sets, are made publicly available. In short, we provide:
%
\begin{itemize}
	\item The InfoPos framework as a pre-design support tool for ML-assisted solution design fine-tuning.
	\item Preliminary results from a real-world platform, as our demonstrator use-case, covering numerous combinations of available knowledge, available data and traditional ML algorithms.
	\item Publicly available processed data sets~\cite{Odyurt:2025:DATASET} and the data workflow code~\cite{Odyurt:2025:CODE}, covering the data processing and ML model training.
\end{itemize}

% ===============================================
% Section
% ===============================================
\section{Background and definitions}
\label{sec:background}
To explain our perspective and what we consider roles of knowledge and data are in shaping data-centric and ML-assisted solutions, it is important to clarify the terminology first. Throughout this paper, what we consider as \emph{data} is primarily metric traces collected from a multitude of available sensors, a.k.a., Extra-Functional Behaviour metrics. Industrial CPS machines, especially modern ones, are equipped with sensors, mainly intended for product quality control. We consider both individual hardware sensors, e.g., a torque measuring sensor, a voltage collector, or a temperature sensor, and software sensors. The latter refers to system resource monitoring virtual metric collectors to record variables such as computational time, memory usage and so forth. This type of sensing will be the case for the compute and control elements.

What we consider as \emph{knowledge} can be sourced from different artefacts, e.g., blueprints, system/machine logs (not to be confused with traces), design documentation. System knowledge reveals its operational sequence, characteristics, applied configuration, input material parameters, and physical environment specifics. For example, size and type of input, production rate (which could be translated to frequency or required yield), machine cycle steps and their order, are all parts of this knowledge.

\subsection{Knowledge and data}
We consider the two major dimensions influencing the design and the effectiveness of ML-assisted solutions, or rather most data processing solutions, to be the \emph{knowledge position} and the \emph{data position}. In this context, the knowledge position refers to the level of understanding present of the system's internals, its interactions with the physical domain, and how it related to any accompanying data. Similarly, the data position refers to how extensive, complete, and granular the collected or available data is. The data position provides the level of qualities such as descriptiveness, comprehensiveness and accuracy\footnote{By accuracy we refer to the absence/presence of noise.} of collected data.

Both dimensions are to be considered as a spectrum, spanning from a poor state to a rich one. To provide examples of opposing states for knowledge, as depicted in \Cref{fig:knowledge_spectrum}, abstract and black-box versus descriptive and white-box representations come to mind. For data, as shown in \Cref{fig:data_spectrum}, we can think of coarse or incomplete versus granular or comprehensive data.
%
\begin{figure}[htbp]
    \centering
    \begin{subfigure}{\linewidth}
    	\centering
	    \includegraphics[width=0.7\linewidth]{figures/knowledge_spectrum.pdf}
	    \caption{Knowledge spectrum with representative extremities.}
	    \label{fig:knowledge_spectrum}
    \end{subfigure}
    \qquad
    \begin{subfigure}{\linewidth}
    	\centering
    	\includegraphics[width=0.7\linewidth]{figures/data_spectrum.pdf}
		\caption{Data spectrum with representative extremities.}
		\label{fig:data_spectrum}
    \end{subfigure}
	\caption{Knowledge and data positions as the two main dimensions affecting data-centric solutions.}
	\label{fig:spectrums}
\end{figure}

\subsection{Information positions}
With both dimensions taken into account, any solution design task could land on either of the cells from the $3 \times 3$ quadrant given in \Cref{fig:infopos_quadrant}.
%
\begin{figure}[htbp]
	\centering
	\includegraphics[width=0.8\linewidth]{figures/infopos_quadrant.pdf}
	\caption{Information position quadrant resulting from the composition of knowledge and data dimensions.}
	\label{fig:infopos_quadrant}
\end{figure}

Depending on practical circumstances involved with the use-case at hand, one can expand or shrink the quadrant by adding or removing steps to/from each dimension. To simplify our demonstration and to deliver the message, only considering the very extreme cases, is a suitable approach.

% ===============================================
% Section
% ===============================================
\section{Methodology}
\label{sec:methodology}
We consider the demonstrator platform from~\cite{Odyurt:2021:PPFT} and the associated data collected from it as our source. The main advantage of this platform is the collection of real and balanced data, i.e., not synthetic. Though the scale of the platform is small, it reflects the real-world task of continuous live image processing. Image analysis using a pre-trained ML model is performed as a computational workload (not to be mistaken with ML models used in our anomaly identification flow) to detect the presence of cars in various parking areas.

The data collection experimental set-up is covered in \Cref{fig:demonstrator_setup}, with the presence of a dedicated power data logger with an isolated power supply for accuracy.
%
\begin{figure}[htbp]
	\centering
	\includegraphics[width=0.9\linewidth]{figures/demonstrator_setup.pdf}
	\caption{Data collection from the demonstrator set-up, including a dedicated electrical data logger and with the application of different workloads, as well as different anomalous conditions for individual experiments.}
	\label{fig:demonstrator_setup}
\end{figure}

\subsection{Data processing workflow}
The preprocessing applied to the collected electrical metrics\footnote{Voltage is collected, but not considered.}, i.e., \emph{current}, \emph{power} and \emph{energy}, is depicted in the diagram given in \Cref{fig:data_processing}. Note that a similar preceding workflow generated the Mean Passport information, which will act as the reference point for comparing unknown execution data. Mean Passports are signatures belonging to executions with no anomalies, i.e., normal behaviour (denoted as Normal).
%
\begin{figure*}[htbp]
	\centering
	\includegraphics[width=0.9\textwidth]{figures/data_processing.pdf}
	\caption{Our detailed data processing workflow, covering different steps, as well as the in-house simple orchestrator to run the workflow in parallel and at scale.}
	\label{fig:data_processing}
\end{figure*}

Note that the extensive nature of preprocessing is to generate features required for traditional ML algorithms, which has proven to be rather effective.

\subsection{Data set}
The final output from the preprocessing workflow is a labelled data set used for supervised ML model training and testing. Included feature columns are:
%
\begin{itemize}
	\item The time span covered by the data segment, i.e., the cut trace (\texttt{execution\_time}).
    \item Different parameters from linear or quadratic regression functions, representing the data segment (\texttt{coefficient\_2}, \texttt{coefficient\_1}, \texttt{intercept}).
    \item Different goodness-of-fit comparison calculations, quantifying the diversion of the unknown execution data from the reference execution data (\texttt{R2}, \texttt{R2\_absolute\_diff}, \texttt{RMSE}, \texttt{RMSE\_absolute\_diff}).
\end{itemize}

Considering the 8 data collection cases described in~\cite{Odyurt:2021:PPFT}, as well as the three experiment conditions applied, i.e., Normal, NoFan, and UnderVolt, we end up with 24 data collection scenarios. For each scenario, we consider three quartile-based phase cuts (reductions or segmentations if you may), alongside the full phase data (see \Cref{fig:uninformed_segmentation}). As such, there will be 4 phase data cuts per scenario, i.e., \emph{ini}, \emph{mid}, \emph{end}, and \emph{full}, resulting in 96 individual cases to be processed by our workflow. 
%The results of our data processing boils down to data sets organised with data per quartile-based segmentation, i.e., individual data sets for \emph{ini}, \emph{mid}, \emph{end}, and \emph{full} cuts.
Needless to say, it is trivial to combine such data, as the format and headers are the same in all. We apply these data sets separately during ML model training and provide relevant results in separate tables in \Cref{sec:results}.

\subsection{Data segmentation}
One of the steps most dependent on the available knowledge is segmentation (cutting) of data. There can be two segmentation types, informed, which cuts the data into known phases, or uninformed, which lack of the internal operation of the system forces the segmentation to be more simplistic. Both types are depicted in \Cref{fig:data_segmentation}.
%
\begin{figure}[htbp]
    \centering
    \begin{subfigure}{\linewidth}
    	\centering
	    \includegraphics[width=\linewidth]{figures/informed_phase_cuts.pdf}
	    \caption{Informed segmentation}
	    \label{fig:informed_segmentation}
    \end{subfigure}
    \qquad
    \begin{subfigure}{\linewidth}
    	\centering
    	\includegraphics[width=\linewidth]{figures/uninformed_segmentation_cuts.pdf}
		\caption{Uninformed segmentation}
		\label{fig:uninformed_segmentation}
    \end{subfigure}
	\caption{Different types of segmentation depending on the availability of the operational knowledge.}
	\label{fig:data_segmentation}
\end{figure}

\paragraph*{Phase-based (informed) segmentation}
Phase-based segmentation is the informed type of segmentation. In our use-case, images are processed as the computational workload. As any, this processing activity is not a single step one. The processing of a single data instance (an image) is covered by the \texttt{cycle-op} phase type, hence, one cycle of operation for this platform. Each cycle is composed of two inner and sequential phase types, \texttt{image-op} and \texttt{neural-op} to load the image and to apply ML inference, respectively. The knowledge of this design and the knowledge of start and end events per phase type allows us to cut the metric data into chunks associated with each phase type. In \Cref{fig:informed_segmentation}, we can consider C1 as a \texttt{cycle-op} phase, composed of A1 and B1 corresponding to \texttt{image-op} and \texttt{neural-op} phases.

\paragraph*{Quartile-based (uninformed) segmentation}
In the absence of such knowledge, segmentation of data based on phase execution time quartiles can be considered. This is a rather simple, but effective, segmentation strategy. Basically any phase type's execution duration can be divided in 4 quartiles. Data contained in the first and the last are considered as \emph{ini} and \emph{end} segment, while the data from the two middle quartiles is the \emph{mid} segment, as shown in \Cref{fig:uninformed_segmentation}. It is important to note that, as a general rule, quartile-based segmentation is applied to phases, which can happen in both informed or uninformed situations. To be true to the uninformed case here, quartile-based segmentation only makes sense for the \texttt{cycle-op} phase type. In an uninformed knowledge position, we will not be aware of sub-phases structure beyond the \texttt{cycle-op} phase. \emph{The motivation behind quartile-based segmentation lies in the presence of cold-start and comparable effects at the start and at the end of most computational tasks}.

\subsection{ML algorithms for anomaly identification}
We have considered an exhaustive collection of traditional ML model types in our experiments. These model types are, Boosted Decision Tree (BDT)~\cite{Friedman:2001:BDT}, Decision Tree (DT)~\cite{Breiman:1984:DT}, Extra Trees (ET)~\cite{Geurts:2006:ET}, Gaussian Naive Bayes (NB), Kernel Support-Vector Machine (SVM), Linear Support Vector Classification (SVC) and Random Forest (RF)~\cite{Breiman:2001:RF}. These model types are utilised as multi-class classifiers and identify the type of system behaviour. We cover the normal behaviour, as well as two anomalous behaviours (NoFan and UnderVolt) in our experiments. Note that our training is supervised and the list of classes can be easily expanded if representative data exists. We consider both prediction accuracy and F1 score for model performance evaluation. As it can be observed in \Cref{sec:results}, traditional ML models are still very capable for this job and very much worth exploring and improving upon.

For our training, we apply 3-fold cross-validation and calculate the average accuracy and average F1 score from all folds. In each experiment, models are trained with specific portions of data, resulting from aforementioned segmentation strategies. Note that while we search for the best model performance, the primary goal is to discover the interplay between different scenario variables making up the information position for that particular scenario.

% ===============================================
% Section
% ===============================================
\section{Results}
\label{sec:results}
Considering the high number of cases, variety of metrics and the number of considered ML model types, we end up with a vast amount of results, of which we only provide the most interesting bit. We have seen in previous research~\cite{Odyurt:2021:PPFT} and repeated the same observation that the most effective metric to consider in these experiments is \emph{electrical current}, leading to highest ML model performances. This is valid throughout. Thus, in the following tables, we only cover results based on the electrical current metric.

Considering that our data set is well-balanced, prediction and F1 score calculations match rather well and either one can be considered as a single model performance metric. We do provide both metrics, but rely on model accuracy to draw our conclusions, which is corroborated by the F1 score as well.

Another point to make is that it is quite clear from our results that tree-based algorithms excel at this type of classification. Tree-based traditional ML algorithms refer to algorithms using decision trees or ensembles of decision tree. As such, we only focus on and provide the results from BDT, DT, ET and RF classifiers.

Detailed results provided in \Cref{tab:model_performance} cover model performance metrics for the aforementioned classifier model types, covering numerous data segments. In particular, results dedicated to each data cut with uninformed segmentation, i.e., \emph{full}, \emph{ini}, \emph{mid} and \emph{end}, are provided separately in \Cref{tab:model_performance_full,tab:model_performance_ini,tab:model_performance_mid,tab:model_performance_end}, respectively. Here, the \emph{full} type is actually the representation of complete data. As it can be seen, all available phase types, as well as their combinations as input for the ML model training is covered. For instance, phase type \enquote{all} refers to the use of data from all three individual phase types, i.e., \texttt{cycle-op}, \texttt{image-op}, and \texttt{neural-op}. Note that the three phase types are the result of informed segmentation, utilising the knowledge from system's internal operation.
%
\begin{table*}[htbp]
    \centering
    \caption{Model performance results for different training data}
    \label{tab:model_performance}
    \begin{subtable}{\textwidth}
        \centering
        \caption{Model performance results for full-cut segmentation, i.e., no segmentation, applied to each phase type}
        \label{tab:model_performance_full}
	    \begin{tabular}{@{}lrrrrrrrr@{}}
	        \toprule
	        \multicolumn{1}{c}{\textbf{Phase type}} & 
	        \multicolumn{1}{c}{\textbf{BDT accuracy}} & 
	        \multicolumn{1}{c}{\textbf{BDT F1}} & 
	        \multicolumn{1}{c}{\textbf{DT accuracy}} &
	        \multicolumn{1}{c}{\textbf{DT F1}} &
	        \multicolumn{1}{c}{\textbf{ET accuracy}} &
	        \multicolumn{1}{c}{\textbf{ET F1}} &
	        \multicolumn{1}{c}{\textbf{RF accuracy}} &
	        \multicolumn{1}{c}{\textbf{RF F1}} \\
	        \midrule
	        \multicolumn{9}{c}{Signature regression type: linear} \\
	        \midrule
	        all						& 95.71\%  & 0.96  & 95.83\%  & 0.96  & 95.99\%  & 0.96  & 96.27\%  & 0.96 \\
	        cycle-op 				& 98.88\%  & 0.99  & 98.40\%  & 0.98  & 98.78\%  & 0.99  & 98.91\%  & 0.99 \\
	        image-op 				& 91.44\%  & 0.91  & 89.96\%  & 0.90  & 91.64\%  & 0.92  & 91.90\%  & 0.92 \\
	        neural-op 				& 99.19\%  & 0.99  & 99.14\%  & 0.99  & 98.93\%  & 0.99  & 99.11\%  & 0.99 \\
	        image-op + neural-op 	& 94.75\%  & 0.95  & 94.22\%  & 0.94  & 95.12\%  & 0.95  & 95.30\%  & 0.95 \\
	        \midrule
	        \multicolumn{9}{c}{Signature regression type: polynomial quadratic} \\
	        \midrule
	        all 					& 96.16\%  & 0.96  & 95.94\%  & 0.96  & 96.44\%  & 0.96  & 96.60\%  & 0.97 \\
	        cycle-op 				& 99.03\%  & 0.99  & 98.78\%  & 0.99  & 99.06\%  & 0.99  & 98.93\%  & 0.99 \\
	        image-op 				& 92.15\%  & 0.92  & 89.89\%  & 0.90  & 92.81\%  & 0.93  & 92.81\%  & 0.93 \\
	        neural-op 				& 99.21\%  & 0.99  & 98.76\%  & 0.99  & 99.11\%  & 0.99  & 99.06\%  & 0.99 \\
	        image-op + neural-op 	& 95.16\%  & 0.95  & 94.49\%  & 0.94  & 95.80\%  & 0.96  & 95.80\%  & 0.96 \\
	        \bottomrule
		\end{tabular}
	\end{subtable}
    %
    \vspace{1em}
	%
	\begin{subtable}{\textwidth}
        \centering
        \caption{Model performance results for ini-cut segmentation, applied to each phase type}
        \label{tab:model_performance_ini}
        \begin{tabular}{@{}lrrrrrrrr@{}}
            \toprule
            \multicolumn{1}{c}{\textbf{Phase type}} & 
            \multicolumn{1}{c}{\textbf{BDT accuracy}} & 
            \multicolumn{1}{c}{\textbf{BDT F1}} & 
            \multicolumn{1}{c}{\textbf{DT accuracy}} &
            \multicolumn{1}{c}{\textbf{DT F1}} &
            \multicolumn{1}{c}{\textbf{ET accuracy}} &
            \multicolumn{1}{c}{\textbf{ET F1}} &
            \multicolumn{1}{c}{\textbf{RF accuracy}} &
            \multicolumn{1}{c}{\textbf{RF F1}} \\
            \midrule
            \multicolumn{9}{c}{Signature regression type: linear} \\
	        \midrule
            all               		& 93.67\%  & 0.94  & 93.12\%  & 0.93  & 93.85\%  & 0.94  & 94.10\%  & 0.94 \\
            cycle-op          		& 97.79\%  & 0.98  & 97.89\%  & 0.98  & 97.61\%  & 0.98  & 97.59\%  & 0.98 \\
            image-op          		& 86.48\%  & 0.86  & 83.00\%  & 0.83  & 86.36\%  & 0.86  & 86.76\%  & 0.87 \\
            neural-op         		& 98.91\%  & 0.99  & 98.76\%  & 0.99  & 98.65\%  & 0.99  & 98.81\%  & 0.99 \\
            image-op + neural-op 	& 92.44\%  & 0.92  & 91.03\%  & 0.91  & 92.35\%  & 0.92  & 92.67\%  & 0.93 \\
            \midrule
	        \multicolumn{9}{c}{Signature regression type: polynomial quadratic} \\
	        \midrule
	        all               		& 94.44\%  & 0.94  & 93.55\%  & 0.94  & 94.92\%  & 0.95  & 94.95\%  & 0.95 \\
            cycle-op          		& 98.32\%  & 0.98  & 97.54\%  & 0.98  & 98.12\%  & 0.98  & 98.32\%  & 0.98 \\
            image-op          		& 88.54\%  & 0.88  & 85.21\%  & 0.85  & 88.52\%  & 0.88  & 88.95\%  & 0.89 \\
            neural-op         		& 99.14\%  & 0.99  & 98.45\%  & 0.98  & 99.06\%  & 0.99  & 98.98\%  & 0.99 \\
            image-op + neural-op 	& 93.18\%  & 0.93  & 92.26\%  & 0.92  & 93.84\%  & 0.94  & 93.95\%  & 0.94 \\
            \bottomrule
        \end{tabular}
    \end{subtable}
    %
    \vspace{1em}
	%
    \begin{subtable}{\textwidth}
        \centering
        \caption{Model performance results for mid-cut segmentation, applied to each phase type}
        \label{tab:model_performance_mid}
        \begin{tabular}{@{}lrrrrrrrr@{}}
            \toprule
            \multicolumn{1}{c}{\textbf{Phase type}} & 
            \multicolumn{1}{c}{\textbf{BDT accuracy}} & 
            \multicolumn{1}{c}{\textbf{BDT F1}} & 
            \multicolumn{1}{c}{\textbf{DT accuracy}} &
            \multicolumn{1}{c}{\textbf{DT F1}} &
            \multicolumn{1}{c}{\textbf{ET accuracy}} &
            \multicolumn{1}{c}{\textbf{ET F1}} &
            \multicolumn{1}{c}{\textbf{RF accuracy}} &
            \multicolumn{1}{c}{\textbf{RF F1}} \\
            \midrule
            \multicolumn{9}{c}{Signature regression type: linear} \\
	        \midrule
            all               		& 94.88\%  & 0.95  & 94.51\%  & 0.95  & 95.16\%  & 0.95  & 95.13\%  & 0.95 \\
            cycle-op          		& 98.53\%  & 0.99  & 98.45\%  & 0.98  & 98.37\%  & 0.98  & 98.48\%  & 0.98 \\
            image-op          		& 88.41\%  & 0.88  & 85.44\%  & 0.85  & 88.34\%  & 0.88  & 88.62\%  & 0.89 \\
            neural-op         		& 99.14\%  & 0.99  & 99.16\%  & 0.99  & 98.78\%  & 0.99  & 98.98\%  & 0.99 \\
            image-op + neural-op 	& 93.31\%  & 0.93  & 91.92\%  & 0.92  & 93.50\%  & 0.93  & 93.75\%  & 0.94 \\
            \midrule
	        \multicolumn{9}{c}{Signature regression type: polynomial quadratic} \\
	        \midrule
	        all               		& 95.14\%  & 0.95  & 94.60\%  & 0.95  & 96.06\%  & 0.96  & 95.98\%  & 0.96 \\
            cycle-op          		& 99.11\%  & 0.99  & 98.65\%  & 0.99  & 98.93\%  & 0.99  & 99.01\%  & 0.99 \\
            image-op          		& 89.48\%  & 0.89  & 87.30\%  & 0.87  & 90.17\%  & 0.90  & 90.04\%  & 0.90 \\
            neural-op         		& 99.54\%  & 1.00  & 99.16\%  & 0.99  & 99.19\%  & 0.99  & 99.42\%  & 0.99 \\
            image-op + neural-op 	& 94.03\%  & 0.94  & 92.71\%  & 0.93  & 94.74\%  & 0.95  & 94.66\%  & 0.95 \\
            \bottomrule
        \end{tabular}
    \end{subtable}
    %
    \vspace{1em}
	%
    \begin{subtable}{\textwidth}
        \centering
        \caption{Model performance results for end-cut segmentation, applied to each phase type}
        \label{tab:model_performance_end}
        \begin{tabular}{@{}lrrrrrrrr@{}}
            \toprule
            \multicolumn{1}{c}{\textbf{Phase type}} & 
            \multicolumn{1}{c}{\textbf{BDT accuracy}} & 
            \multicolumn{1}{c}{\textbf{BDT F1}} & 
            \multicolumn{1}{c}{\textbf{DT accuracy}} &
            \multicolumn{1}{c}{\textbf{DT F1}} &
            \multicolumn{1}{c}{\textbf{ET accuracy}} &
            \multicolumn{1}{c}{\textbf{ET F1}} &
            \multicolumn{1}{c}{\textbf{RF accuracy}} &
            \multicolumn{1}{c}{\textbf{RF F1}} \\
            \midrule
            \multicolumn{9}{c}{Signature regression type: linear} \\
	        \midrule
            all               		& 95.10\%  & 0.95  & 95.03\%  & 0.95  & 95.57\%  & 0.96  & 95.75\%  & 0.96 \\
            cycle-op          		& 98.45\%  & 0.98  & 98.20\%  & 0.98  & 98.35\%  & 0.98  & 98.40\%  & 0.98 \\
            image-op          		& 89.86\%  & 0.90  & 88.08\%  & 0.88  & 89.91\%  & 0.90  & 90.37\%  & 0.90 \\
            neural-op         		& 98.76\%  & 0.99  & 98.53\%  & 0.99  & 98.37\%  & 0.98  & 98.60\%  & 0.99 \\
            image-op + neural-op 	& 93.75\%  & 0.94  & 93.13\%  & 0.93  & 94.11\%  & 0.94  & 94.27\%  & 0.94 \\
            \midrule
	        \multicolumn{9}{c}{Signature regression type: polynomial quadratic} \\
	        \midrule
	        all               		& 94.48\%  & 0.94  & 94.94\%  & 0.95  & 96.11\%  & 0.96  & 96.12\%  & 0.96 \\
            cycle-op          		& 98.48\%  & 0.98  & 97.99\%  & 0.98  & 98.40\%  & 0.98  & 98.32\%  & 0.98 \\
            image-op          		& 89.13\%  & 0.89  & 88.77\%  & 0.89  & 91.08\%  & 0.91  & 90.93\%  & 0.91 \\
            neural-op         		& 98.81\%  & 0.99  & 98.60\%  & 0.99  & 98.50\%  & 0.98  & 98.63\%  & 0.99 \\
            image-op + neural-op 	& 93.28\%  & 0.93  & 93.24\%  & 0.93  & 95.07\%  & 0.95  & 94.86\%  & 0.95 \\
            \bottomrule
        \end{tabular}
    \end{subtable}
\end{table*}

The following immediate implications can be observed from the results.

\subsection{Metrics to consider}
Data from different metrics result in different prediction performances, which is the motivation behind our focus on the data from the \emph{electrical current} metric. Selection of a metric beforehand cannot be directly deduced, but the effectiveness holds throughout. Therefore, it is a matter of trial.

\subsection{Signature levels}
Passports and signatures representing execution behaviour within arbitrary segments of data are based on regression function. Higher orders of regression functions (quadratic, cubic, etc.) result in more accurate representation of data points and better prediction performance, but impose extra computational cost during data preprocessing. There are a couple of negligible exceptions in our results, such as the DT accuracy for \texttt{neural-op} under \emph{full} (\Cref{tab:model_performance_full}) and \emph{ini} (\Cref{tab:model_performance_ini}) cuts.

\subsection{Data segmentation}
The choice of data segmentation is the most influential aspect. The consistent observation across the board in \Cref{tab:model_performance_full} points to the superior prediction performance from the \texttt{neural-op} phase type. However, presence of \texttt{neural-op} assumes an informed segmentation.

To compare the results for uninformed segmentation, we shall consider \texttt{cycle-op} results in every table. When it comes to linear signature regression functions, full-cut segments give the best results with the exception of DT, for which a mid-cut segment is better. For quadratic signature regression functions, both BDT and RF show better performance with mid-cut segments. For all model types, a quadratic signature function, when considering a mid-cut, performs better than a linear signature function combined with a full-cut.

Considering the computational effort effect, i.e., energy and time, dealing with a mid-cut segment is much more advantageous than using a full-cut, even if a single step is upgraded to polynomial quadratic regression function generation. Considering the scale of preprocessing, the net result is better prediction performance at lower energy and faster preprocessing times. While we do not have dedicated collections, we can confirm the time difference for preprocessing is rather noticeable. We can conclude that the lack of informed segmentation can be effectively compensated by an increase in the preprocessing levels, combined with a lighter preprocessing flow.

The most interesting result however, is when uninformed segmentation is applied on top of the informed one, i.e., quartile-based segmentation for each phase type. While results are close for the linear categories with only DT neural mid-cut demonstrating an advantage over neural full-cut, for the polynomial quadratic categories all models work much better under neural mid-cut. This clearly indicates that more data does not necessarily mean better predictions, which is also confirmed by lower performance when combining phase types. One has to find the most effective portion of data, in this case the \emph{mid} segment of the \texttt{neural-op} phase type.

\subsection{ML algorithm of choice}
We have already narrowed down the ML algorithm choices to tree-based algorithms and these are very performant. Amongst these algorithms, BDT and RF have a consistent edge over DT and ET, with BDT posting the accuracy of 99.54\% with a quadratic regression function as the signature level and under the \emph{mid} segment of the \texttt{neural-op} phase type (\Cref{tab:model_performance_mid}).

\subsection{Covered information positions}
As we do not cover data quality aspects in this paper, we shall consider the bottom row for the data dimension, which is the case with our data set.

Considering the provided results and the information position quadrant, we can fill some of the cells, i.e., \Cref{fig:quadrant_coverage}. The knowledge dimension is clearly divided between informed and uninformed segmentations, matching white box and black box positions, respectively. When it comes to the data dimension the richness and poorness are to be considered in terms of the effectiveness quality.
%
\begin{figure}[htbp]
	\centering
	\includegraphics[width=0.7\linewidth]{figures/quadrant_coverage.pdf}
	\caption{Considering the comprehensiveness of data and the various considered knowledge positions in our cases, we are covering the bottom row of the information position quadrant.}
	\label{fig:quadrant_coverage}
\end{figure}

For a designer, the availability, or lack there of, knowledge of system internals would mean that only the left column from \Cref{fig:infopos_quadrant} is to be considered. Accordingly, it is known that an uninformed segmentation considering the mid-cut in combination with polynomial quadratic and BDT, works best. Note that this combination works better than a full-cut. This lands us on the bottom left cell.

The opposite situation, in which the segmentation can be done in an informed fashion, the designer will still apply the mid-cut on top of the \texttt{neural-op} phase type selection. This lands us on the bottom right cell.

% ===============================================
% Section
% ===============================================
\section{Related work}
\label{sec:related_work}
While there are numerous literature considering effects of ML data quality~\cite{Mohammed:2024:EDQM, Foroni:2021:EEED, Frenay:2014:CPLN, Li:2021:CSEI, Neutatz:2022:DCAW, Shah:2024:HDCD}, which can be defined with a number of dimensions itself~\cite{Mohammed:2024:EDQM}, the presence and effects of knowledge has not been considered. The closest concept to the consideration of knowledge as a separate dimension is \enquote{task-dependent quality}~\cite{Foroni:2021:EEED}, which still considers data quality in the context of the task it is being used for, i.e., a variable quality limit.

We on the other hand take into account the knowledge involved in the design of the solution and its availability, which leads to a more comprehensive view of the overall information position (knowledge combined with data). Accordingly, one major difference with the above cited literature is the need for detailed understanding of the solution. This generally is not a factor in the literatures, as studies consider standard tasks, e.g., regression, classification, and so forth. By bringing in the knowledge aspect, we aim to make the understanding of quality applicable to complex and custom solution design processes.

% ===============================================
% Section
% ===============================================
\section{Conclusion and future work}
\label{sec:conclusion}
It is evident from our results that the combination of applied preprocessing, selected data portions, and ML model of choice, has a direct impact on solution performance. Possessing such awareness, upfront, will lead to a much more streamlined design process.

When it comes to the question of reusability, our conclusion holds for the type of anomaly identification solution evaluated in this paper, i.e., ML models trained with constructs (signatures in our case) based on data segmentation. Depending on the information position, choices such as the application of a mid-cut and the BDT model hold by default. Case-specific variables, such as the discovery of the most effective informed segmentation (\texttt{neural-op} for our use-case), will need the execution of a minimal viable example. Effects of regression function level is also known upfront, as discussed in \Cref{sec:results} and should be evaluated and chosen by the designer. The industry utilising this type of CPS, e.g., semiconductor photolithography, production printing, even MRI machines in the health industry, is by no means small. Anomaly identification solutions are equally valuable across the board.

Immediate next steps for us are to complete the quadrant with representative scenarios of varying data quality, as well as execution of diverse types of ML-assisted solutions. The latter will include Deep Neural Networks and possibly Transformer-based alternative designs.

% ###############################################
% End of file
% ###############################################

\section{Related work}\label{sec:ReWork}
Here, we discuss significant related works and potential countermeasures relevant to our Thor attack.

\paragrabf{Hertzbleed \cite{wang2022hertzbleed}.}
Hertzbleed leverages dynamic voltage and frequency scaling (DVFS) to transform power side-channel attacks into timing attacks. By exploiting the timing differences caused by frequency variations, even remote attacks become feasible. For instance, in an attack against Supersingular Isogeny Key Encapsulation (SIKE), they managed to recover 378 bits of the private key within 36 hours. Although this attack shares similarities with our work in exploiting frequency changes, DVFS can be managed by the CPU core and disabled in BIOS settings to mitigate the Hertzbleed attack. However, in our case, disabling DVFS is not a viable countermeasure since AMX, an on-chip accelerator, manages its power and frequency independently.

\paragrabf{Collide+Power \cite{kogler2023collide+}.}
The Collide+Power research focuses on the power leakage of the memory hierarchy via Running Average Power Limit (RAPL). If RAPL is unavailable, monitoring can be done through a throttling side-channel, albeit requiring more measurements. They demonstrated two types of attacks: Meltdown-style and Microarchitectural Data Sampling (MDS)-style. In Meltdown-style, targeting a shared cache among two processes on different cores, theoretically, one bit can be leaked in 99.95 days with power limit control or 2.86 years with stress-induced throttling. In MDS-style, where both victim and attacker run on different logical cores of the same physical core, data can be leaked from the L1/L2 cache at a rate of 4.82 bits per hour. Disabling simultaneous multithreading can mitigate MDS-style attacks. Generally, RAPL being a privileged interface is not accessible to unprivileged attackers, and throttling can be disabled by turning off DVFS.

\paragrabf{Platypus \cite{Lipp2021Platypus}.}
The Platypus attack reconstructed 509 RSA key bits using RAPL MSRs within Intel SGX enclaves. However, this attack vector has been mitigated by making RAPL a privileged interface.

\paragrabf{Neural Network Specific Attacks.}
Several studies have attacked neural network accelerators using power side channels. In the work by Wei et al. \cite{wei2018know}, an FPGA-based convolutional neural network accelerator was attacked, requiring physical access to recover the model's input image with up to 89\% accuracy. Effective mitigations include masking and random scheduling, although masking introduces significant overheads, as demonstrated in MaskedNet \cite{dubey2020maskednet}, increasing latency and area costs by 2.8x and 2.3x, respectively.

Open DNN Box \cite{Xiang2019OpenDB} inferred the weight sparsity of neural network models with 96.5\% accuracy on average. CSI NN \cite{236204} used power and electromagnetic traces to infer information about weights and architecture in fully connected neural networks. DeepEM \cite{Yu2020DeepEMDN} and DeepSniffer \cite{Hu2020DeepSnifferAD} collected electromagnetic traces to glean architectural information, with DeepEM specifically targeting binarized neural networks. Cache Telepathy \cite{244042}, GANRED \cite{Liu2020GANREDGR}, and DeepRecon \cite{Hong2018SecurityAO} employed well-known cache side channels like Flush+Reload and Prime+Probe to gather neural network insights. For these cache attacks, the attacker runs locally, and the presence of a shared cache is necessary. In contrast, our Thor attack doesn't need physical access or shared cache. It introduces a novel, data-dependent timing side-channel vulnerability specific to Intel AMX accelerators.

In the work by Gongye et al. \cite{9218707}, they attacked DNNs using a floating-point timing side channel to obtain weights and biases. They took advantage of the drastically different execution times for floating-point multiplication and addition in certain scenarios, such as when dealing with subnormal values, to launch their attack. However, with modern accelerators like Intel AMX, this floating-point timing vulnerability has been eliminated. Now, the execution time for tile multiplication remains constant, even for special cases like zero inputs. Specifically, the latency is fixed at 52 cycles and the throughput is 16. Despite this, to attack DNNs with more than one layer, cache monitoring or physical access is still required to measure the execution time of each layer.

% \paragrabf{Potential Countermeasures.}
% Eliminating the cooldown state could defend against Thor but at a high power cost since Intel AMX is an energy-intensive accelerator designed for AI tasks. Keeping AMX continuously active would be power-prohibitive.

% Masking is a proven countermeasure for protecting AI model parameters against power side-channel attacks and could be adapted for future AMX versions despite the performance overhead. Additionally, machine learning models should incorporate techniques to detect unusual usage patterns, which can help identify and thwart attacks attempting to infer parameter values using methods similar to our AMX-type attack. One well-known countermeasure to these timing attacks is to coarsen the timer. By reducing the timer's precision, it becomes much harder for attackers to measure the subtle differences in execution times that they rely on for their exploits.

% In summary, while various countermeasures exist for different attack vectors, protecting against Thor on Intel AMX accelerators requires novel approaches to power and frequency management alongside traditional techniques.



\section{Empirical support}\label{sec:sim}
In this section, 
% we first introduce our training and sampling approach and then 
we demonstrate the  benefits of regularization for diffusion empirically.
Rather than relying on large-scale pipelines and data,
which are subject to a number of other factors,
we study the influence of regularization in simple, well-explored setups. 
We present a toy example and the MNIST family dataset here, deferring additional simulations and setups to Appendix Section~\ref{sec:ComSim}.


\subsection{Toy example}
We first highlight the influence of regularization on the sampling process of a 3D toy example. 
We consider $2000$ three-dimensional, independent Gaussian samples with mean zero and covariance matrix $\allowbreak[0.08,0,0;0,1,0;0,0,1]$;
hence, the data fluctuate most around the $y$ and $z$ axes.
We then train two diffusion models, with the same data: 
the original denoising score matching and the same with an additional sparsity-inducing regularization (as proposed in~\eqref{dscorematchR}) and $r=0.001$.
Figure~\ref{fig:toy} visualizes the data (first panel) and the sampling process with $T=60$ for original score matching (second panel) and the regularized version (third panel).
Both models start from the blue dot. 
The figure shows that the regularized version provides a more focused sampling. %leading to a faster convergence.

\begin{figure*}[htbp]
    % First row: single centered image
    \begin{minipage}{0.27\textwidth}
        \centering
        \includegraphics[width=\textwidth]{Figures/Toy/toydata.png}
        % Caption or additional description for the centered image
    \end{minipage}
       \hfill
    % \vspace{1em} % Adjust vertical space between rows
    % Second row: three images
    \begin{minipage}{0.27\textwidth}
        \centering
        \includegraphics[width=\textwidth]{Figures/Toy/toyOr.png}
        % Caption or description for this image
    \end{minipage}
    \hfill
    \begin{minipage}{0.27\textwidth}
        \centering
        \includegraphics[width=\textwidth]{Figures/Toy/toyR.png}
        % Caption or description for this image
    \end{minipage}

    \caption{Visualizing the sampling process for 3D data (first panel) with original denoising score matching (second panel) versus regularized denoising score matching (third panel).
The original samples are depicted as red circles, blue circles indicate the starting points for sampling, and green circles represent the latent generated samples.
The red arrows illustrate the sampling paths. It is evident that regularized denoising score matching predominantly adheres to the two-dimensional sub-manifold (along the $Y$ and $Z$ axes), whereas the original denoising score matching explores the entire 3D space.
    }\label{fig:toy}
\end{figure*}
\subsection{MNIST}\label{sec:mnist}
We now compare original score matching  and the regularized version on \mnist~dataset~\citep{LeCun1998} including $n=50\,000$ training samples. 
We are interested to time steps $T\in \{500,50,20\}$ for sampling and we consider regularization with $\tuning=0.0005$ for $T=500$ and $\tuning=0.003$ for $T\in \{50,20\}$.
In fact, we set the tuning parameter as a decreasing function of $T$, let say $\tuning=f[T]\in O(c'/T)$ for a real constant $c\in (0,\infty)$.
Note that two models are already trained over the same amount of data and identical settings employing different objectives. 
For sampling (starting from pure noise), then we try different values of time steps $T\in \{500,50,20\}$. 
That means, for small values of $T$, we just need to pick up a \textbf{larger step size} as we always start from pure noise (see Algorithm~\ref{alg:samp}).
Figure~\ref{fig:MNIST} shows the results. While the original score matching fails to generate reasonable samples for small values of $T$, our proposed score function performs successfully even for $T=20$. 
In all our simulations, we use the same network structure, optimization method, and sampling approach with identical settings for both approaches (see Appendix~\ref{sec:netstr} for detailed settings). The only difference lies in the objective functions: one is regularized, while the other is not. While it could be argued that alternative network structures or sampling processes might enhance the quality of the generated images for original score matching, our focus remains on the core idea of regularization fixing all other factors and structures.  
We defer the enhanced versions of our simulations aimed at achieving higher-quality images to future works.
% We provide all the details regarding our simulations settings in the Appendix~\ref{sec:netstr}. 

\begin{figure*}[htbp]
    \centering
    % First row: single centered image
    \begin{minipage}{0.18\textwidth}
        \centering
        \includegraphics[width=\textwidth]{Figures/MNIST/MNIST-True.png}
        % Caption or additional description for the centered image
    \end{minipage}
    
    \vspace{0.2em} % Adjust vertical space between rows
    
    % Second row: three images
    \begin{minipage}{0.18\textwidth}
        \centering
        \includegraphics[width=\textwidth]{Figures/MNIST/MNISTOr500.png}
        % Caption or description for this image
    \end{minipage}
    % \hfill
    \hspace{9em} 
    \begin{minipage}{0.18\textwidth}
        \centering
        \includegraphics[width=\textwidth]{Figures/MNIST/MNISTweak500.png}
        % Caption or description for this image
    \end{minipage}
    
    \vspace{0.2em} 
    \begin{minipage}{0.18\textwidth}
        \centering
        \includegraphics[width=\textwidth]{Figures/MNIST/MNISTOr50.png}
        % Caption or description for this image
    \end{minipage}
    \hspace{9em} 
    % \hfill
    \begin{minipage}{0.18\textwidth}
        \centering
        \includegraphics[width=\textwidth]{Figures/MNIST/MNISTR50.png}
        % Caption or description for this image
    \end{minipage}
    
    \vspace{0.2em} 
    
    \begin{minipage}{0.18\textwidth}
        \centering
        \includegraphics[width=\textwidth]{Figures/MNIST/MNISTOr20.png}
        % Caption or description for this image
    \end{minipage}
    \hspace{9em} 
    % \hfill
    \begin{minipage}{0.18\textwidth}
        \centering
        \includegraphics[width=\textwidth]{Figures/MNIST/MNISTR20.png}
        % Caption or description for this image
    \end{minipage}
    
   
    
    \caption{
    Image generation using the original denoising score matching (left column) versus the regularized version (right column) for different time steps, 
$T=500,T=50$, and $T=20$ (from top to bottom). The middle column displays $81 $ original samples from the \mnist~dataset for comparison with images of dimensions  $\Dim=28\times28\times 1=784$. 
Our regularized version  generates high-quality images for  
$T=500$ (comparable to the original denoising score matching) and still produces good images for small $T$, while the original denoising score matching totally fails. 
}
    \label{fig:MNIST}
\end{figure*}




\subsection{FashionMNIST}\label{sec:fmnist}
We follow almost all the settings as in Section~\ref{sec:mnist} with $\tuning=0.0001$ for $T=500$ and $\tuning=0.002$ for $T\in \{70,50\}$ for \fmnist~dataset~\citep{Xiao2017} including $n=50\,000$ training samples.  
Results are provided in Figures~\ref{fig:FMNIST}. 
Following the generated images obtained using both approaches, and ensuring that all factors except the objective functions remain identical, we observe that the original score-matching approach produces samples that appear oversmoothed and exhibit imbalanced distributions (see the first image of the left panel of Figure~\ref{fig:FMNIST}).  
In contrast, our regularized approach with a considerably small tuning parameter, generates images that resemble the true data more closely and exhibit a more balanced distribution (see the first image of the right panel of Figure~\ref{fig:FMNIST}). 
For instance, the percentages of generated images for Sandals, Trousers, Dresses, Ankle Boots, and Bags are approximately $(0.0, 0.7, 2.0, 3.0, 4.0)$ using the original score matching, compared to $(8.0, 6.0, 8.0, 10.0, 10.0)$ with the regularized version, highlighting the clear imbalance in distribution for the original score matching.




\begin{figure*}[htbp]
    \centering
    \begin{minipage}{0.28\textwidth}
        \centering
        \includegraphics[width=\textwidth]{Figures/FMNIST/FMT.png}
        % Caption or additional description for the centered image
    \end{minipage}
    
    \vspace{0.1em} % Adjust vertical space between rows
    % First row
    \begin{minipage}{0.28\textwidth}
        \centering
        \includegraphics[width=\textwidth]{Figures/FMNIST/FMOr500.png}
        % \caption{Caption for figure 1}
    \end{minipage}
    \hspace{13em} 
    \begin{minipage}{0.28\textwidth}
        \centering
        \includegraphics[width=\textwidth]{Figures/FMNIST/Rweak500.png}
        % \caption{Caption for figure 4}
    \end{minipage}

    \vspace{0.1em} 
    
    \begin{minipage}{0.28\textwidth}
        \centering
        \includegraphics[width=\textwidth]{Figures/FMNIST/FMOr70.png}
        % \caption{Caption for figure 2}
    \end{minipage}
    \hspace{13em} % Adjust vertical space between rows
     \begin{minipage}{0.28\textwidth}
        \centering
        \includegraphics[width=\textwidth]{Figures/FMNIST/FMR70.png}
        % \caption{Caption for figure 5}
    \end{minipage}
    
\vspace{0.1em}

    \begin{minipage}{0.28\textwidth}
        \centering
        \includegraphics[width=\textwidth]{Figures/FMNIST/FMOr50.png}
        % \caption{Caption for figure 3}
    \end{minipage}
     \hspace{13em} % Adjust vertical space between rows
    \begin{minipage}{0.28\textwidth}
        \centering
        \includegraphics[width=\textwidth]{Figures/FMNIST/FMR50.png}
        % \caption{Caption for figure 6}
    \end{minipage}
    
    \caption{Image generation using the original denoising score matching (left column) versus the regularized version (right column) for different time steps,  $T=500, T=70$, and $T=50$ (from top to bottom).  
    The middle column displays $256$ original samples from the
\fmnist~dataset for comparison with images of  dimensions $\Dim=28\times28\times 1=784$.
 Our regularized version generates high-quality images for  
$T=500$ (comparable to the original denoising score matching) and still produces good images even for 
samll $T$, while the original denoising score matching totally fails.
Another notable observation is that our regularization results in more balanced image generation, as evident when comparing our method to the original denoising score matching at $T=500$, where the latter produces overly smooth images.
}\label{fig:FMNIST}
\end{figure*}





\section{Conclusion}\label{sec:conc}
\section{Conclusion}
In this work, we propose a simple yet effective approach, called SMILE, for graph few-shot learning with fewer tasks. Specifically, we introduce a novel dual-level mixup strategy, including within-task and across-task mixup, for enriching the diversity of nodes within each task and the diversity of tasks. Also, we incorporate the degree-based prior information to learn expressive node embeddings. Theoretically, we prove that SMILE effectively enhances the model's generalization performance. Empirically, we conduct extensive experiments on multiple benchmarks and the results suggest that SMILE significantly outperforms other baselines, including both in-domain and cross-domain few-shot settings.
\section*{Acknowledgements}
This research was partially funded by grants 
502906238,
543964668
(SPP2298), 
and 520388526 (TRR391) 
by the Deutsche Forschungsgemeinschaft (DFG, German Research Foundation).

\bibliographystyle{plainnat} \bibliography{Contents/References}
\newpage
\appendix
\onecolumn
\section{Complementary simulations}\label{sec:ComSim}
In this section, we provide additional simulation supporting our theories on further image dataset Butterflies (Section~\ref{sec:BF}).
We also  introduce our training and sampling approach in Section~\ref{sec:TrainingAlg} and provide details about network architecture and training settings in Section~\ref{sec:netstr}. 
\subsection{Butterflies}\label{sec:BF}
We also compare original diffusion and  regularized analog on \ButF~dataset (smithsonian-butterflies) including $n=10\,000$ training samples.
We consider regularization $\tuning=0.0001$ for $T=1000$ and $\tuning=0.0005$ for $T\in \{200,150\}$.
Results are provided in Figure~\ref{fig:BF}. 
Again, our results show that our approach perform better than the original score matching  for small values of $T$. 

\begin{figure*}[htbp]
    \centering
    \begin{minipage}{0.29\textwidth}
        \centering
        \includegraphics[width=\textwidth]{Figures/Butterfly/BFT.png}
        % Caption or additional description for the centered image
    \end{minipage}
    
    \vspace{0.1em} % Adjust vertical space between rows
    % First row
    \begin{minipage}{0.29\textwidth}
        \centering
        \includegraphics[width=\textwidth]{Figures/Butterfly/BFO1000.png}
        % \caption{Caption for figure 1}
    \end{minipage}
    \hspace{13em}
     \begin{minipage}{0.29\textwidth}
        \centering      \includegraphics[width=\textwidth]{Figures/Butterfly/BFR1000.png}
        % \caption{Caption for figure 4}
    \end{minipage}
    
    \vspace{0.1em}
    
    \begin{minipage}{0.29\textwidth}
        \centering
        \includegraphics[width=\textwidth]{Figures/Butterfly/BFO200.png}
        % \caption{Caption for figure 2}
    \end{minipage}
    \hspace{13em}
    \begin{minipage}{0.29\textwidth}
        \centering
        \includegraphics[width=\textwidth]{Figures/Butterfly/BFR200.png}
        % \caption{Caption for figure 5}
    \end{minipage}

     \vspace{0.1em} 
    
    \begin{minipage}{0.29\textwidth}
        \centering
        \includegraphics[width=\textwidth]{Figures/Butterfly/BFO150.png}
        % \caption{Caption for figure 3}
    \end{minipage}  
    \hspace{13em} % Adjust vertical space between rows 
    \begin{minipage}{0.29\textwidth}
        \centering
        \includegraphics[width=\textwidth]{Figures/Butterfly/BFR150.png}
        % \caption{Caption for figure 6}
    \end{minipage}
    
    \caption{Image generation using the original denoising score matching (left column) versus the regularized version (right column) for different time steps,  $T=1000, T=200$, and $T=150$ (from top to bottom).  
    The middle column displays $81$ original samples from the
\ButF~dataset for comparison.
The dataset consists of images with dimensions $\Dim=28\times28\times 3=2352$. As shown in the images, our regularized version  generates high-quality images for  
$T=1000$ (comparable to the original denoising score matching) and still perform better than original denoising score matching for  
$T=200$ and $T=150$.}\label{fig:BF}
\end{figure*}







\subsection{Training and sampling algorithms}\label{sec:TrainingAlg}
Here we provide details about how we solve the objective function~\eqref{dscorematchR} in practice, that is, how we deal with the expected values and score functions. 

Let's first define the objective function over a batch  of training examples $\x_{\batchS}$ (a batch of size $\batchS\in \{1,2,\dots\}$) and for a batch of random time steps $\boldsymbol{t}_{\batchS}\in (0,1]^{\batchS}$: 
\begin{equation}\label{eq:simobj}
f(\scale,\Theta,\x_{\batchS},\boldsymbol{t}_{\batchS})\deq \frac{1}{\batchS}\sum_{i=1}^{\batchS}\norm{\scale\scoref(\x^i_{t_i},t_i)-\nabla_{\x_{t}}\log \Fdd_t(\x^i_{t_i}|\x^i)}^2\bigr]
 +\tuning \scale^2 
\end{equation}
with 
\begin{equation}\label{eq:pertx}   \Fd_t(\x^i_{t_i}|\x^i)=\mathcal{N}\bigl(\x^i,\sigma_{t_i}\Identity\bigr) 
\end{equation}
with $\sigma_t\deq (\sigma^{2t}-1)/(2\log \sigma)$ for $t\in (0,1]$ and a large enough $\sigma\in (0,\infty)$ (we set $\sigma=5$ for \ButF~and $\sigma=25$ for other datasets). 
Here $\x^i_t$ corresponds to a perturbed version of the training sample $\x^i$ ($i$th sample of the batch) in  time step $t$. As stated in~\eqref{eq:pertx}, once $\sigma$ is large, $\x_{1}$ ($t=1$) goes to a mean-zero Gaussian. 
And as shown in~\citet{vincent2011connection}, the optimization objective  $\E_{\Fdd_t(\x_t|\x)\Fdd_{0}(\x)}[\norm{\scale\scoref(\x_{t},t)-\nabla_{\x_{t}}\log \Fdd_t(\x_{t}|\x)}^2]$ for a fixed variance~$\varnoise_t$ is equivalent to the optimization  objective $\E_{\Fdd_t(\x_t)}[\norm{\scale\scoref(\x_{t},t)-\nabla_{\x_{t}}\log \Fdd_t(\x_{t})}^2]$ and, therefore, satisfies $\scale^*\scorefS(\x_{t},t)=\nabla_{\x_t}\log \Fdd_t(\x_t)$.
We then provide Algorithm~\ref{alg:training} for solving the objective function in~\eqref{dscorematchR}. 
Note that we can easily compute the score functions in~\eqref{eq:simobj} since there is a  closed form solution for them as densities are just Gaussian conditional on $\x^i$. 
\begin{algorithm}[H]
\caption{Training algorithm}\label{alg:training}
\begin{algorithmic}[1] % The [1] adds line numbering
\STATE \textbf{Inputs:} $\sigma$, $n_{\operatorname{epochs}}$ (number of epochs), $\batchS(\operatorname{batch-size})$, $\operatorname{eps}=0.00001$
\STATE \textbf{Outputs:} $(\DnSMR,\scaleM)$
\STATE Initialize parameters $(\DnSMR,\scaleM)$ 
\FOR{$i = 1$ to $n_{\operatorname{epochs}}$}
\FOR{$\x_{\batchS}  $ in data-loader}
\STATE $\boldsymbol{t}_{\batchS}=\{\mathcal{U}_{[0,1]}\}^{\batchS}  (1 - \operatorname{eps}) + \boldsymbol{\operatorname{eps}}$ 
    \STATE One step optimization minimizing   $f(\scale,\Theta,\x_{\batchS},\boldsymbol{t}_{\batchS})$ in~\eqref{eq:simobj} employing a random batch of time steps $\boldsymbol{t}_{bs}\in (0,1]^{bs}$  and updating $(\DnSMR,\scaleM)$
\ENDFOR
\ENDFOR


\end{algorithmic}
\end{algorithm}
Parameter $\operatorname{eps}$ in Algorithm~\ref{alg:training}  is introduced for numerical stability and to refuse $t=0$. 
For a sufficiently large number of epochs, we expect to learn the scores accurately for different time steps. 
For sampling process, we employ a naive sampler as proposed in Algorithm~\ref{alg:samp}
employing Langevin dynamics~\citep[Section~2.2]{song2019generative} to align with our theory.  
\begin{algorithm}[H]
\caption{Sampling algorithm}\label{alg:samp}
\begin{algorithmic}[1] % The [1] adds line numbering

\STATE \textbf{Inputs:} $\sigma$, $\operatorname{eps}=0.00001$, T (Time steps)
\STATE \textbf{Output:} \x
\STATE $\x=\x_{\operatorname{init}}=\StandNormal\sigma_1$ 
\STATE $\boldsymbol{t} = linspace(1., eps, T)$ (make a grid of time steps)
\STATE $\eta = \boldsymbol{t}[0] - \boldsymbol{t}[1]$ (set step size)
\FOR{t in $\boldsymbol{t}$}
\STATE $\x=\x+\eta \scaleM\score_{\DnSMR}(\x,t)+\sqrt{2\eta}\StandNormal$ (update $\x$)
\ENDFOR
\end{algorithmic}
\end{algorithm}

\subsection{Network architecture and training settings}\label{sec:netstr}
Our model is a U-Net architecture with 4 downsampling and 4 upsampling layers, each comprising residual blocks. The network starts with a base width of 32 channels, doubling at each downsampling step to a maximum of 256 channels, and mirrors this in the decoder. A bottleneck layer with 256 channels connects the encoder and decoder. Time information is encoded using Gaussian Fourier projections and injected into each residual block via dense layers. Group normalization is applied within the residual blocks, and channel attention mechanisms are included selectively to enhance feature representations.
For training, we used the Adam optimizer with a learning rate of 0.001, and for sampling, we employed a signal-to-noise ratio of 0.1.
We used a batch size of $128$ and trained for $2000$ epochs on the \ButF~dataset and less than $1000$ epochs on the other datasets.
\subsection{Proofs}

Fix a probability weighting function $w$ and let $g(y) = -\frac{1}{y} \log(w(y)) w(y)$. Fix a p.d.f. $f \in \Delta(\mc{X})$, where $\mc{X} \subset \mathbb{R}^d$ is compact. Fix $n, k \in \mathbb{N}$, and let $X_1, \ldots, X_n \sim f(\cdot)$. Recall the definition of $\hat{f}$ from \eqref{eqn:knn_f}. Let $\mu$ denote the Lebesgue measure and $B_r(x) = \{ x' \in \mathbb{R}^d \ | \ \norm{x' - x} < r \}$. Define
%
\begin{align}
    H^{B,w}(f) &= - \int_{\mc{X}} \log(w(f(x))) w(f(x)) dx \int_{\mc{X}} g(f(x)) f(x) dx, \\
    %
    H^{B,w}_n(f) &= - \sum_{i=1}^n \frac{1}{f(x)} \log(w(f(X_i))) w(f(X_i)) = \frac{1}{n} \sum_{i=1}^n g(f(X_i)), \\
    %
    \widehat{H}^{B,w}_{k,n}(f) &= - \sum_{i=1}^n \frac{1}{\hat{f}(x)} \log(w(\hat{f}(X_i))) w(\hat{f}(X_i)) = \frac{1}{n} \sum_{i=1}^n g(\hat{f}(X_i)).
\end{align}
%
Our goal is to establish a bound on the error
%
\begin{equation} \label{eqn:goal_to_bound}
    \left| \mathbb{E} \left[ \widehat{H}^{B,w}_{k,n}(f) \right] - H^{B,w}(f) \right|.
\end{equation}
%
In general, for finite $k$, even as $n \rightarrow \infty$ the approximator $\widehat{H}^{B,w}_{k,n}(f)$ will remain biased due to the biasedness of $\hat{f}$ for fixed $k$ and the lack of a known bias correction procedure for our BE approximator $\widehat{H}^{B,w}_{k,n}(f)$. This contrasts with the situation for simpler estimators like Shannon and R\'{e}nyi entropies, for which explicit bias correction terms are known (see \cite{singh2003nearest, leonenko2008class, singh2016finite}). Nonetheless, in Theorem \ref{thm:main_bound} we are able to build on existing results to establish a probabilistic bound on \eqref{eqn:goal_to_bound}. We first recall the following result.
%
\begin{lemma}[\citep{singh2016finite}] \label{lem:singh_poczos}
    Suppose that, for some $\xi \in (0, 2]$, $f$ is $\xi$-H\"{o}lder continuous and strictly positive on $\mc{X}$. Suppose furthermore that there exists a function $f_* : \mc{X} \rightarrow \mathbb{R}^+$ and a constant $f^*$ such that $0 < f_*(x) \leq \int_{B_r(x)} f(y) dy / \mu(B_r(x)) \leq f^* < \infty$, for all $x \in \mc{X}, r \in (0, \sqrt{d}]$, and assume that $\int_0^{\infty} e^{-x} x^k f(x) dx < \infty$. Then
    %
    % \begin{align}
    %     \left| \mathbb{E} \left[ \widehat{H}^{B,w}_{k,n}(f) \right] - H^{B,w}(f) \right| &= \mc{O}\left( \frac{k}{n} \right)^{\frac{\xi}{d}}, \\
    %     %
    %     \text{Var} \left( \widehat{H}^{B,w}_{k,n}(f) \right) &= \mc{O}\left( \frac{C_V}{n} \right).
    % \end{align}
    %
    \begin{multicols}{2}
        \noindent
        \small
        \vspace{-8mm}
        \begin{equation} \label{eqn:bias_bound}
            \left| \mathbb{E} \left[ H^{B,w}_n(f) \right] - H^{B,w}(f) \right| = \mc{O}\left( \frac{k}{n} \right)^{\frac{\xi}{d}},
        \end{equation}
        \normalsize
        %
        % \break
        %
        \noindent
        \small
        \begin{equation} \label{eqn:var_bound}
            \text{Var} \left( H^{B,w}_n(f) \right) = \mc{O}\left( \frac{1}{n} \right).
        \end{equation}
        \normalsize
    \end{multicols}
\end{lemma}
%
The proof of this result follows directly from that of \citep[Thm. 5]{singh2016finite} due to the fact that $H^{B,w}_n(f)$ is an unbiased estimator of $H^{B,w}(f)$. Also note that the variance bound can be trivially strengthened to apply to $\widehat{H}^{B,w}_{k,n}(f)$ due to the fact that the latter is simply the sample average of $n$ i.i.d., bounded random variables:
%
\begin{corollary}
    Under the conditions of Lemma \ref{lem:singh_poczos}, $\text{Var} \left( \widehat{H}^{B,w}_{k,n}(f) \right) = \mc{O}\left( \frac{1}{n} \right).$
\end{corollary}

It remains to characterize \eqref{eqn:goal_to_bound}. We first recall another useful result from the literature. For a given set $S \subset \mc{X}$, radius $r$, and $m > 0$, let $\mc{N} \left( S, r \right)$ denote the covering number, the minimum number of balls of radius $r$ needed to cover $S$. Let $\normop{\cdot}$ denote the operator norm.
%
\begin{lemma}[\citep{zhao2022analysis}] \label{lem:zhao_lai}
    Suppose there exist $C_1, C_2, C_3, \mc{N}_0 > 0$ and $\beta \in (0, 1]$ such that the following conditions hold:
    \begin{align*}
        &(i) \quad \frac{\norm{ \nabla f(x) }}{f(x)} \leq C_1; \quad (ii) \quad \frac{ \normop{ \nabla^2 f(x) } }{f(x)} \leq C_2; \quad (iii) \quad \forall t > 0, P(f(x) < t) \leq C_3 t^{\beta}; \\
        %
        &(iv) \quad \mc{N} \left( \{ x | f(x) > m \}, r \right) \leq \frac{\mc{N}_0}{m^{\gamma} r^d}, \text{ for some } \gamma > 0 \text{ and all } m > 0.
    \end{align*} 
    %
    Then, for $\varepsilon > 0$, it holds with probability (w.p.) $1 - \varepsilon$ that
    %
    \begin{equation} \sup_x \left| \hat{f}(x) - f(x) \right| =
        \begin{cases}
            \mc{O}\left( \left( \frac{k}{n} \right)^{\frac{2}{d}} \log n + \sqrt{ \frac{ \log ( n / \varepsilon ) }{k} } \right) & \text{ if } d > 2, \\
            %
            \mc{O}\left( \frac{k}{n} \log n + \sqrt{ \frac{ \log ( n / \varepsilon ) }{k} } \right) & \text{ if } d = 1, 2.
        \end{cases}
    \end{equation}
\end{lemma}

We are now in a position to prove our main result.
%
\begin{customthm}{2} \label{eqn:main_bound}
    Suppose $f$ satisfies the conditions of Lemmas \ref{lem:singh_poczos} and \ref{lem:zhao_lai}. Assume $w$ is Lipschitz continuous. Then, for $\varepsilon > 0$, it holds w.p. $1 - \varepsilon$ that
    %
    \begin{equation} \left| \mathbb{E} \left[ \widehat{H}^{B,w}_{k,n}(f) \right] - H^{B,w}(f) \right| = 
        \begin{cases}
            \mc{O}\left( \frac{k}{n} \right)^{\frac{\xi}{d}} + \mc{O}\left( \left( \frac{k}{n} \right)^{\frac{2}{d}} \log n + \sqrt{ \frac{ \log ( n / \varepsilon ) }{k} } \right) & \text{ if } d > 2, \\
            %
            \mc{O}\left( \frac{k}{n} \right)^{\frac{\xi}{d}} + \mc{O}\left( \frac{k}{n} \log n + \sqrt{ \frac{ \log ( n / \varepsilon ) }{k} } \right) & \text{ if } d = 1, 2.
        \end{cases}
    \end{equation}
\end{customthm}
%
\begin{proof}
    First notice that
    %
    \begin{equation} \label{eq:0}
        \left| \mathbb{E} \left[ \widehat{H}^{B,w}_{k,n}(f) \right] - H^{B,w}(f) \right| \leq \left| \mathbb{E} \left[ \widehat{H}^{B,w}_{k,n}(f) - H^{B,w}_n(f) \right] \right| + \left| \mathbb{E} \left[ H^{B,w}_n(f) \right] - H^{B,w}(f) \right|.
    \end{equation}
    %
    The second term can be bounded using Lemma \ref{lem:singh_poczos}, so it just remains to bound the first term. Recall that $\mc{X}$ is compact, $f$ is bounded strictly away from 0 on $\mc{X}$, and $w$ is Lipschitz. We therefore have that $g$ is the product of Lipschitz, bounded functions and is therefore itself Lipschitz on its domain. Let $K$ denote the minimal Lipschitz parameter of $g$. Rewriting \eqref{eq:0} in terms of $g$, we obtain
    %
    \begin{align}
        \left| \mathbb{E} \left[ \widehat{H}^{B,w}_{k,n}(f) - H^{B,w}_n(f) \right] \right| &= \left| \frac{1}{n} \sum_{i=1}^n \mathbb{E} \left[ g(\hat{f}(X_i)) - g(f(X_i)) \right] \right| \\
        %
        &\leq \frac{1}{n} \sum_{i=1}^n \mathbb{E} \left[ \left| g(\hat{f}(X_i)) - g(f(X_i)) \right| \right] \\
        %
        &\labelrel={eq:2} \mathbb{E} \left[ \left| g(\hat{f}(X_1)) - g(f(X_1)) \right| \right] \\
        %
        &\leq K \mathbb{E} \left[ \left| \hat{f}(X_1) - f(X_1) \right| \right] \\
        %
        &\leq K \mathbb{E} \left[ \sup_x \left| \hat{f}(x) - f(x) \right| \right] % \\
        %
        % &\leq K \sup_x \left| \hat{f}(x) - f(x) \right|,
    \end{align}
    %
    where \eqref{eq:2} follows from the fact that the $X_1, \ldots, X_n$ are i.i.d. An application of the law of total probability and Lemma \ref{lem:zhao_lai} to the last term completes the proof.
\end{proof}
% \section{Further discussion}\label{sec:furdisc}
% \input{Contents/Furtherdiscussion}
\end{document}


