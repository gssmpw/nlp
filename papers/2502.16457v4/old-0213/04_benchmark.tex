\section{Benchmark Definition and Tasks}
\label{sec:benchmark}

This section introduces our benchmark for evaluating models on materials synthesis prediction. We outline the task definitions, dataset splits, and evaluation metrics, culminating in a framework reflecting real-world materials science challenges.

\subsection{Task Definition}
The benchmark is designed to simulate end-to-end materials synthesis workflows. Given only the input \textbf{X}—which includes the target material, synthesis method, and application domain—the model must predict all subsequent components:  
\begin{itemize}
    \item \(\mathbf{P_M}\): Raw materials (e.g., reagents, solvents) and their quantities.
    \item \(\mathbf{P_E}\): Required equipment (e.g., furnace, autoclave).
    \item \(\mathbf{P_P}\): Synthesis procedures (e.g., reaction steps, times, temperatures).
    \item \(\mathbf{P_C}\): Characterization methods and expected outcomes.
\end{itemize}

The predicted outputs are compared against ground-truth recipes (\(\mathbf{Y_M}\), \(\mathbf{Y_E}\), \(\mathbf{Y_P}\), and \(\mathbf{Y_C}\)). Evaluations are conducted by either domain experts or an LLM Judge, as detailed in Section~\ref{subsec:metric}. This setup mirrors real-world scenarios where researchers must infer complete synthesis workflows from minimal initial information.

\subsection{Dataset Splits and Distribution}
Our dataset comprises 17,667 articles curated from diverse materials science literature. To ensure robust evaluation, we divide the dataset into three splits based on publication year and journal impact factor (IF):
\begin{itemize}
    \item \textbf{Training Set:} 16,026 articles published before 2024.
    \item \textbf{Test - Standard Impact:} 1,472 articles published in 2024 or later with IF $<$ 10.
    \item \textbf{Test - High Impact:} 169 articles published in 2024 or later with IF $\geq$ 10.
\end{itemize}

The temporal split ensures test data represents unseen future research trends, mitigating data contamination. Further stratification by IF allows targeted assessment of model performance on high-impact research, often involving novel and complex synthesis techniques. This split design evaluates both generalizability and the ability to meet the rigorous standards of top-tier journals.


\subsection{Evaluation Metrics: LLM-as-a-Judge}
\label{subsec:metric}

Evaluating synthesis predictions requires nuanced judgment across multiple dimensions. While human experts provide the gold standard for evaluation, their involvement is time-intensive and costly. To address this, we employ a large language model (LLM) as an efficient alternative evaluator.

\subsection{Evaluation Criteria}

While many dynamic benchmarking methods have been proposed to evaluate LLMs, the evaluation criteria for assessing these benchmarks themselves remain non-standardized. To this end, we propose the following evaluation criteria to assess the quality of a dynamic benchmarking algorithm.

\subsubsection{Correctness}
The first criterion for evaluating the quality of dynamic benchmarking is \textsf{Correctness}. If the correctness of the generated dataset cannot be guaranteed, the benchmark may provide a false sense of reliability when applied to benchmarking LLMs, leading to misleading evaluations.  
We quantify the correctness of dynamic benchmarks as:  
$$
\small
    \textsf{Correctness} = \mathbb{E}_{i=1}^{N}  
    \mathcal{S} \big( \mathcal{Y}_i, \mathcal{G}(\mathcal{X}_i) \big)
$$
where \( \mathcal{X}_i \) and \( \mathcal{Y}_i \) represent the input and output of the \( i^{th} \) transformation, respectively. The function \( \mathcal{G}(\cdot) \) is an oracle that returns the ground truth  of its input, ensuring an objective reference for correctness evaluation. For example, the function \( \mathcal{G}(\cdot) \) could be a domain-specific an annotator.  
This equation can be interpreted as the expected alignment between the transformed dataset's outputs and their corresponding ground truth values, measured using the scoring function \( \mathcal{S}(\cdot) \). A higher correctness score indicates that the dynamic benchmark maintains correctness to the ground truth.



\subsubsection{Scalability}
The next evaluation criterion is scalability, which measures the ability of dynamic benchmarking methods to generate large-scale benchmark datasets. A smaller dataset can introduce more statistical errors during the benchmarking process. Therefore, an optimal dynamic benchmark should generate a larger dataset while minimizing associated costs. The scalability of a dynamic benchmark is quantified as:
$$
\small
    \textsf{Scalability} = \mathbb{E}_{i=1}^{N} \left[ \frac{\lVert T_i(\mathcal{D}) \rVert}{\lVert \mathcal{D} \rVert \times \textsf{Cost}(T_i)} \right]
$$
This represents the expectation over the entire transformation space, where \( \lVert T_i(\mathcal{D}) \rVert \) is the size of the transformed dataset, and \( \lVert \mathcal{D} \rVert \) is the size of the original dataset. The function \( \textsf{Cost}(\cdot) \) measures the cost associated with the transformation process, which could include monetary cost, time spent, or manual effort according to the detailed scenarios.
This equation could be interpreted as the proportion of data that can be generated per unit cost.



\subsubsection{Collision}

One of the main motivations for dynamic benchmarking is to address the challenge of balancing transparent benchmarking with the risk of data contamination. Since the benchmarking algorithm is publicly available, an important concern arises:  \textit{If these benchmarks are used to train LLM, can they still reliably reflect the true capabilities of LLMs?}
To evaluate the robustness of a dynamic benchmark against this challenge, we introduce the concept of \textit{collision} in dynamic benchmarking. Collision refers to the extent to which different transformations of the benchmark dataset produce overlapping data, potentially limiting the benchmark’s ability to generate novel and diverse test cases. To quantify this, we propose the following metrics:  

\[
\small
\begin{split}
    & \textsf{Collision Rate} = \mathbb{E}_{\substack{i, j = 1, \, i \neq j}}^{N}  
    \left[ \frac{\lVert \mathcal{D}_i \cap \mathcal{D}_j \rVert }{ \lVert \mathcal{D} \rVert} \right]\\
    & \textsf{Repeat} = \mathbb{E}_{i=1}^{N} \left[ k \mid k = \min \left\{ \bigcup_{j=1}^{k} \mathcal{D}_j \supseteq \mathcal{D}_i \right\} \right]
\end{split}
\]  
\textsf{Collision Rate} measures the percentage of overlap between two independently transformed versions of the benchmark dataset, indicating how much poential contamination among two trials. \textsf{Repeat Trials} quantifies the expected number of transformation trials required to fully regenerate an existing transformed dataset \( T_i(\mathcal{D}) \), providing insight into the benchmark’s ability to produce novel variations.  
These metrics help assess whether a dynamic benchmark remains effective in evaluating LLM capabilities, even when exposed to potential training data contamination.  


\subsubsection{Stable of Complexity}


Dynamic benchmarks must also account for complexity to help users determine whether a performance drop in an LLM on the transformed dataset is due to potential data contamination or an increase in task complexity. If a dynamic transformation increases the complexity of the seed dataset, a performance drop is expected, even without data contamination. However, accurately measuring the complexity of a benchmark dataset remains a challenging task. Existing work has proposed various complexity metrics, but these are often domain-specific and do not generalize well across different applications. For example, DyVal~\citep{zhu2024dyval} proposes applying graph complexity to evaluate the complexity of reasoning problems.
Formally, given a complexity measurement function \( \Psi(\cdot) \), the stability can be formulated as:  
\[
\small
    \textsf{Stability} = \text{Var}(\Psi(D_i))
\]  
This equation can be interpreted as the variance in complexity across different trials, where high variance indicates that the dynamic benchmarking method is not stable.  

% \subsubsection{Stability} 
% Even if a dynamic benchmark has a complexity fucntion for each of its generated data sample, it can also need to enure the benchmark is stable, 



\subsubsection{Diversity }
Besides the aforementioned criteria, another important factor is the diversity of the transformed dataset. This diversity can be categorized into two components: \textsf{external diversity} and \textsf{internal diversity}:  External diversity measures the variation between the transformed dataset and the seed dataset. Internal diversity quantifies the differences between two transformation trials.  
\[
    \small
    \begin{aligned}
        \textsf{External Diversity} &= \mathbb{E}_{i = 1}^{N} \Theta(\mathcal{D}_i, \mathcal{D}) \\
        \textsf{Internal Diversity} &= \mathbb{E}_{\substack{i, j = 1,  i \neq j}}^{N} \Theta(\mathcal{D}_i, \mathcal{D}_j)
    \end{aligned}
\]
where \( \Theta(\cdot) \) is a function that measures the diversity between two datasets. For example, it could be the N-gram metrics or the reference based metrics, such as BLEU scores. 





\subsubsection{Interpretability} 
Dynamic benchmarking generates large volumes of transformed data, making manual verification costly and challenging. To ensure correctness, the transformation process must be interpretable. Interpretable transformations reduce the need for extensive manual validation, lowering costs. Rule-based or manually crafted transformations are inherently interpretable, while LLM-assisted transformations depend on the model's transparency and traceability. In such cases, additional mechanisms like explainability tools, or human-in-the-loop validation may be needed to ensure reliability and correctness.


% \fakeparagraph{Augmentation ?}

% \clearpage
\paragraph*{Evaluation Criteria}
Table~\ref{tab:judgment_criteria} outlines the seven criteria for evaluating synthesis recipes.
Each criterion is scored on a 1–5 scale. The overall score is computed as an average across all criteria to reflect recipe quality comprehensively.

\paragraph*{LLM Evaluation Methodology}
We prompted LLMs to generate Chain-of-Thought (CoT) reasoning~\cite{wei2022chain}, followed by structured outputs in JSON format for scoring each criterion. This approach ensures transparency and interpretability in LLM judgments. Detailed prompts and examples are provided in Appendix~\ref{subsec:appendix_llm_judge}.

The reliability of LLM-based evaluations was validated by comparing them against human expert scores. A Pearson correlation coefficient of 0.52 between human and LLM evaluations demonstrates significant alignment, supporting using LLMs as scalable proxies for expert judgment. Further analysis of LLM-human agreement is discussed in Section~\ref{sec:reliability}.

\subsection{Summary and Outlook}
Our benchmark bridges the gap between controlled academic experiments on small datasets and real-world materials synthesis scenarios encountered in laboratories. By encompassing tasks that span precursor selection to characterization prediction, this framework provides a holistic evaluation platform for NLP models. Including high-impact test data ensures relevance to cutting-edge research, while leveraging LLMs as evaluators enable scalable benchmarking without sacrificing rigor.

Future iterations of this benchmark could further incorporate multimodal data (e.g., figures, tables) to enhance its applicability to practical materials science challenges.