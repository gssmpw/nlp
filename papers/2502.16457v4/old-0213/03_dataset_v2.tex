
\section{Data Collection and Preparation}
\label{sec:data_collection}
\ref{sec:data}

\subsection{Motivation: Why is this necessary?}
% 여기에 table + citation 추가해야함.
Despite the growing interest in applying NLP techniques to materials science, no standardized benchmark exists for evaluating synthesis modeling approaches. Existing datasets such as ~\cite{wang2022dataset} and ~\cite{kononova2019text} primarily focus on isolated aspects of synthesis, such as reaction conditions or precursor selection, without offering a structured framework for holistic evaluation. Furthermore, many of these datasets lack validation methods, making it difficult to assess their reliability for real-world applications.

% table 만들어야함
Table???
% ~\ref{tab:dataset_comparison} 
compares OMG with prior datasets, highlighting its key advantages. Unlike previous work, OMG provides a comprehensive, high-quality dataset covering all essential synthesis stages, including raw materials, equipment, procedures, and characterization methods. Additionally, it introduces structured evaluation criteria and expert-verified validation techniques, ensuring both data reliability and benchmark consistency.


\subsection{Collection and Extraction}
We collected synthesis procedures from a large corpus of materials science literature to construct the dataset. We retrieved 28,685 open-access articles using the Semantic Scholar API~\cite{semanticscholar2023}, filtering them based on 50 keywords relevance to ensure they contained experimental synthesis details. Domain experts suggested these keywords to collect diverse literature in materials science.

This section details our data collection pipeline, from document retrieval to extracting key synthesis information. We focus on how a large corpus of materials science literature is curated, classified, and annotated with recipes essential for downstream benchmarking tasks.

\paragraph{PDF Processing and Structure Extraction}
Extracted PDFs were converted into structured Markdown format using \texttt{PyMuPDFLLM}~\cite{pymupdf4llm2024}. This step preserved paragraph structures, section headers, and inline metadata while eliminating non-relevant content such as references and appendices. Given the journals' highly varied formatting styles, we applied rule-based heuristics to segment the documents into meaningful units.

\paragraph{Automated Categorization and Entity Extraction}  
To ensure effective information retrieval, we employed GPT-4o for automated categorization, segmenting each document into key synthesis components: raw materials (\(\mathbf{Y_M}\)), synthesis equipment (\(\mathbf{Y_E}\)), synthesis procedures (\(\mathbf{Y_P}\)), and characterization methods (\(\mathbf{Y_C}\)). Additionally, named entity recognition (NER) was used to identify chemical compounds, equipment, and experimental parameters, improving retrieval accuracy. We applied a confidence-based filtering mechanism, discarding extracted entities with low model confidence scores to enhance precision.

\paragraph{Key Information Segmentation}  
For each of the 17,667 articles confirmed to include a synthesis recipe, we extracted five critical components, denoted as \textbf{X}, \(\mathbf{Y_M}\), \(\mathbf{Y_E}\), \(\mathbf{Y_P}\), and \(\mathbf{Y_C}\). These components capture the end-to-end workflow of materials synthesis, ensuring a structured representation of key experimental details.

\begin{enumerate}
    \item \textbf{X: Target Material, Synthesis Method, and Application Domain}
    
    A concise summary describing:
    \begin{itemize}
        \item \textit{What is being synthesized?} (e.g., Ni-rich layered oxides)
        \item \textit{Which process is used?} (e.g., solvothermal, sol-gel)
        \item \textit{What is the main application?} (e.g., electrode for lithium-ion batteries)
    \end{itemize}
    
    \item \(\mathbf{Y_M}\): \textbf{Raw Materials}
    
    A list of raw materials (e.g., reagents, solvents) and any quantitative details (e.g., molar ratios, concentrations, masses).
    
    \item \(\mathbf{Y_E}\): \textbf{Synthesis Equipment}
    
    A catalog of apparatus (e.g., furnace, autoclave, spin coater) essential for the synthesis steps.
    
    \item \(\mathbf{Y_P}\): \textbf{Synthesis Procedure}
    
    A step-by-step protocol detailing how the reagents are combined and processed (e.g., reaction times, temperatures, pH, stirring speeds).
    
    \item \(\mathbf{Y_C}\): \textbf{Characterization Methods and Results}
    
    Techniques (e.g., X-ray diffraction, electron microscopy) and outcomes (e.g., crystal structure, particle size, electrochemical performance) used to evaluate the synthesized material.
\end{enumerate}

These structured components serve as the foundation for subsequent benchmarking tasks, ensuring that synthesis recipes are consistently represented.

\paragraph{Synthesis Step Extraction}  
A key challenge in structuring synthesis recipes is extracting stepwise procedural details while preserving logical coherence. LLM-based parsing prompts were optimized to capture step dependencies and correctly link conditions like temperature, reaction time, and precursor ratios to experimental steps. Furthermore, synthesis steps were normalized to maintain consistency across different papers, reducing ambiguity in procedural descriptions.\\

{\color{red} 데이터셋 수집 파이프라인 내용 좀 줄임,}

\subsection{Data Verification and Quality Assessment}
To ensure the accuracy of our automatically extracted recipes, we manually verified sampled recipes by eight domain experts, at least a master's student majoring in materials science. These experts evaluated:

\begin{itemize}
    \item \textbf{Completeness:} Did the extracted text capture the full scope of the reported recipe (\textbf{X}, \(\mathbf{Y_M}\), \(\mathbf{Y_E}\), \(\mathbf{Y_P}\), and \(\mathbf{Y_C}\))?
    \item \textbf{Correctness:} Were critical details like temperature values or reagent amounts accurately extracted?
    \item \textbf{Coherence:} Did the extracted recipe retain coherence and avoid contradictory information?
\end{itemize}

% \usepackage{booktabs}


\begin{table}
\centering
\caption{Data verification by eight domain experts.}
\label{tab:human_verification}
\begin{tabular}{l|cc} 
\toprule
\textbf{Criteria} & \textbf{Mean$_{~\sigma}$} & \textbf{ICC (3,k)}$_{p-\text{value}}$  \\ 
\hline
Completeness      & 4.2$_{~0.81}$             & 0.695$_{~0.00}$               \\
Correctness       & 4.7$_{~0.58}$             & 0.258$_{~0.23}$               \\
Coherence         & 4.8$_{~0.46}$             & 0.429$_{~0.10}$               \\
\bottomrule
\end{tabular}
\end{table}
Preliminary reviews indicate that the \emph{majority} of extracted data is highly accurate, especially for articles with well-structured experimental sections. 

Completeness was assessed with relatively high agreement, indicating that key synthesis details were well captured. While its score was slightly lower than correctness and coherence due to occasional omissions of finer experimental details, such as precise molar ratios or characterization settings, the extracted recipes still provide a strong foundation for understanding synthesis procedures. In contrast, correctness and coherence, despite receiving high scores, showed greater variability among evaluators. Differences in how minor inconsistencies—such as variations in equipment names or missing characterization details—were weighted contributed to this variation. Some evaluators viewed these as negligible, while others applied stricter criteria, highlighting the need for clearer annotation guidelines. Overall, these results suggest that the extraction process is already highly effective and can be further refined to enhance the level of detail captured.  
However, manual verification alone is not sufficient to ensure consistent performance across diverse synthesis procedures.  
To systematically evaluate how well models can predict and extract synthesis protocols, we introduce a benchmarking framework that quantitatively assesses model performance on end-to-end synthesis prediction.  

With the dataset validated, the next step is to establish a structured evaluation framework.  
In the following section (Section~\ref{sec:benchmark}), we present a benchmark designed to assess NLP models on materials synthesis prediction, covering key tasks such as precursor and equipment inference, procedure generation, and characterization outcome forecasting.  

% \subsection{Literature Retrieval and Filtering}
% We began by compiling a list of 50 keywords that broadly represent research in materials science, ranging from specific compounds (e.g., ``perovskite,'' ``metal--organic framework'') to broader application domains (e.g., ``battery electrode,'' ``catalysis material''). These keywords were used to query the Semantic Scholar API~\cite{semanticscholar2023}. We restricted the collection of articles that include publicly available PDF files to comply with legal and ethical guidelines for text mining. This process yielded a dataset of 28,685 PDF documents spanning various journals and conferences. 

% \subsection{PDF Conversion and Document Categorization}


% \begin{figure*}[!ht]
%     \begin{subfigure}{.55\textwidth}
%         \centering
%         \includegraphics[width=\linewidth]{image/data/element_freq_heatmap.png}
%         \caption{The distribution of elements in the target material shown in the periodic table}
%         \label{fig:data-element-distribution}
%     \end{subfigure}
%     \begin{subfigure}{.43\textwidth}
%         \centering
%         \includegraphics[width=\linewidth]{image/data/top10process.png}
%         \caption{Distribution of categorized manufacturing methods}
%         \label{fig:data-process-distribution}
%     \end{subfigure}
%     \caption{Data Distribution}
%     \label{fig:main-data-distribution}
% \end{figure*}


% After retrieving the PDFs, we converted each file to Markdown format using the \texttt{PyMuPDFLLM}~\cite{pymupdf4llm2024}. 
% The resulting Markdown files preserve paragraph structures and metadata while remaining flexible for downstream tasks (e.g., classification, named entity recognition).
% Using the processed Markdown files, we employed GPT-4o~\cite{achiam2023gpt} for an automated categorization step, labeling each article according to:

% \begin{itemize}
%     \item \textbf{Presence of a synthesis recipe:} Whether the text explicitly provides a detailed experimental procedure.
%     \item \textbf{Primary material(s):} The main material name, chemical formula, and class studied (e.g., inorganic oxides, polymers).
%     \item \textbf{Process or synthesis route:} The dominant experimental methods (e.g., hydrothermal, solution-based, vapor deposition).
%     \item \textbf{Application domain:} The intended use-case (e.g., battery, catalysis, biomedical).
% \end{itemize}

% Based on these annotations, we identified 17,667 articles (\(\sim62\%\) of the corpus) containing explicit synthesis ``recipes'', making them highly relevant for tasks involving the prediction or generation of experimental protocols.

% \subsection{Recipe Extraction}
% \label{sec:recipe_extraction}

% \subsubsection{Key Information Segmentation}

% \begin{figure*}[t]
%     \centering
%     \includegraphics[width=\linewidth]{image/data/example.png}
%     \caption{An actual example of extracted recipe.}
%     \label{fig:data-example}
% \end{figure*}

% For each of the 17,667 articles confirmed to include a synthesis recipe, we extracted five critical components, denoted as \textbf{X}, \(\mathbf{Y_M}\), \(\mathbf{Y_E}\), \(\mathbf{Y_P}\), and \(\mathbf{Y_C}\). These components capture the end-to-end workflow of materials synthesis:

% \begin{enumerate}
%     \item \textbf{X: Target Material, Synthesis Method, and Application Domain}
    
%     A concise summary describing:
%     \begin{itemize}
%         \item \textit{What is being synthesized?} (e.g., Ni-rich layered oxides)
%         \item \textit{Which process is used?} (e.g., solvothermal, sol-gel)
%         \item \textit{What is the main application?} (e.g., electrode for lithium-ion batteries)
%     \end{itemize}
    
%     \item \(\mathbf{Y_M}\): \textbf{Raw Materials}
    
%     A list of raw materials (e.g., reagents, solvents) and any quantitative details (e.g., molar ratios, concentrations, masses).
    
%     \item \(\mathbf{Y_E}\): \textbf{Synthesis Equipment}
    
%     A catalog of apparatus (e.g., furnace, autoclave, spin coater) essential for the synthesis steps.
    
%     \item \(\mathbf{Y_P}\): \textbf{Synthesis Procedure}
    
%     A step-by-step protocol detailing how the reagents are combined and processed (e.g., reaction times, temperatures, pH, stirring speeds).
    
%     \item \(\mathbf{Y_C}\): \textbf{Characterization Methods and Results}
    
%     Techniques (e.g., X-ray diffraction, electron microscopy) and outcomes (e.g., crystal structure, particle size, electrochemical performance) used to evaluate the synthesized material.
% \end{enumerate}

% \subsubsection{LLM-Driven Parsing and Annotation}
% The extraction of these elements from each Markdown file was again performed via GPT-4o. We minimized human labor in large-scale annotation by carefully prompting the model to identify and extract relevant text blocks.
% Detailed prompts are described in Appendix~\ref{sec:appendix_annotation}.

% {\color{red}Comprehensive metrics on precision and recall will be added once the manual verification is complete}% 
% Insert final statistics later.


