\section{AlchemyBench}
\label{sec:benchmark}

We present \oursbench, a comprehensive benchmark for evaluating materials synthesis prediction models. This framework addresses key challenges in synthesis recipe evaluation through structured tasks, expert-aligned metrics, and scalable assessment strategies.

\subsection{Motivation}
\label{subsec:motivation}

Evaluating synthesis predictions presents several fundamental challenges:

\begin{itemize}
    \item \textbf{Lack of Benchmarks:} No standardized evaluation framework exists, making it challenging to compare synthesis models systematically. Prior datasets lack critical synthesis parameters and structured ground truth labels, making meaningful comparisons difficult.
    \item \textbf{Limitations of Traditional Metrics:} Traditional metrics, such as BLEU~\cite{papineni2002bleu} and ROUGE~\cite{lin2004rouge} prioritize lexical overlap but fail to capture the procedural correctness of synthesis recipes. \citeauthor{na2023artificial} et al. introduced the Jaccard score to measure set overlap in synthesis procedures, yet it lacks sensitivity to sequential dependencies critical in procedural texts. BERTScore~\cite{zhang2019bertscore} improves contextual similarity measurement but struggles with domain-specific dependencies unique to materials synthesis. Moreover, these metrics do not account for experimental feasibility, limiting their applicability in real-world synthesis.
    \item \textbf{High Cost of Human Evaluation:} Expert-based assessments require significant time and resources, averaging 23 minutes per prediction ($\sigma=7.57$) in our experiment. This cost makes large-scale benchmarking impractical, requiring an automated evaluation system.
    \item \textbf{Scalability Requirements:} Large-scale benchmarking necessitates an automated yet reliable evaluation system, which LLMs can provide~\cite{gu2025surveyllmasajudge}. However, prior attempts to use LLMs for evaluation lacked systematic validation against human expert assessments in materials science, raising concerns about reliability.
\end{itemize}

\subsection{Task Definition}
\label{subsec:task_definition}

\oursbench~simulates real-world synthesis workflows, where models must predict the following components given input $\mathbf{X}$ (target material, synthesis method, application domain):

\begin{itemize}
    \item $\mathbf{P_M}$: Raw materials (e.g., reagents, solvents) with quantities.
    \item $\mathbf{P_E}$: Required equipment (e.g., furnace, autoclave).
    \item $\mathbf{P_P}$: Synthesis procedures (e.g., reaction steps, temperatures).
    \item $\mathbf{P_C}$: Characterization methods and expected outcomes.
\end{itemize}

\subsection{Evaluation Criteria}

While many dynamic benchmarking methods have been proposed to evaluate LLMs, the evaluation criteria for assessing these benchmarks themselves remain non-standardized. To this end, we propose the following evaluation criteria to assess the quality of a dynamic benchmarking algorithm.

\subsubsection{Correctness}
The first criterion for evaluating the quality of dynamic benchmarking is \textsf{Correctness}. If the correctness of the generated dataset cannot be guaranteed, the benchmark may provide a false sense of reliability when applied to benchmarking LLMs, leading to misleading evaluations.  
We quantify the correctness of dynamic benchmarks as:  
$$
\small
    \textsf{Correctness} = \mathbb{E}_{i=1}^{N}  
    \mathcal{S} \big( \mathcal{Y}_i, \mathcal{G}(\mathcal{X}_i) \big)
$$
where \( \mathcal{X}_i \) and \( \mathcal{Y}_i \) represent the input and output of the \( i^{th} \) transformation, respectively. The function \( \mathcal{G}(\cdot) \) is an oracle that returns the ground truth  of its input, ensuring an objective reference for correctness evaluation. For example, the function \( \mathcal{G}(\cdot) \) could be a domain-specific an annotator.  
This equation can be interpreted as the expected alignment between the transformed dataset's outputs and their corresponding ground truth values, measured using the scoring function \( \mathcal{S}(\cdot) \). A higher correctness score indicates that the dynamic benchmark maintains correctness to the ground truth.



\subsubsection{Scalability}
The next evaluation criterion is scalability, which measures the ability of dynamic benchmarking methods to generate large-scale benchmark datasets. A smaller dataset can introduce more statistical errors during the benchmarking process. Therefore, an optimal dynamic benchmark should generate a larger dataset while minimizing associated costs. The scalability of a dynamic benchmark is quantified as:
$$
\small
    \textsf{Scalability} = \mathbb{E}_{i=1}^{N} \left[ \frac{\lVert T_i(\mathcal{D}) \rVert}{\lVert \mathcal{D} \rVert \times \textsf{Cost}(T_i)} \right]
$$
This represents the expectation over the entire transformation space, where \( \lVert T_i(\mathcal{D}) \rVert \) is the size of the transformed dataset, and \( \lVert \mathcal{D} \rVert \) is the size of the original dataset. The function \( \textsf{Cost}(\cdot) \) measures the cost associated with the transformation process, which could include monetary cost, time spent, or manual effort according to the detailed scenarios.
This equation could be interpreted as the proportion of data that can be generated per unit cost.



\subsubsection{Collision}

One of the main motivations for dynamic benchmarking is to address the challenge of balancing transparent benchmarking with the risk of data contamination. Since the benchmarking algorithm is publicly available, an important concern arises:  \textit{If these benchmarks are used to train LLM, can they still reliably reflect the true capabilities of LLMs?}
To evaluate the robustness of a dynamic benchmark against this challenge, we introduce the concept of \textit{collision} in dynamic benchmarking. Collision refers to the extent to which different transformations of the benchmark dataset produce overlapping data, potentially limiting the benchmark’s ability to generate novel and diverse test cases. To quantify this, we propose the following metrics:  

\[
\small
\begin{split}
    & \textsf{Collision Rate} = \mathbb{E}_{\substack{i, j = 1, \, i \neq j}}^{N}  
    \left[ \frac{\lVert \mathcal{D}_i \cap \mathcal{D}_j \rVert }{ \lVert \mathcal{D} \rVert} \right]\\
    & \textsf{Repeat} = \mathbb{E}_{i=1}^{N} \left[ k \mid k = \min \left\{ \bigcup_{j=1}^{k} \mathcal{D}_j \supseteq \mathcal{D}_i \right\} \right]
\end{split}
\]  
\textsf{Collision Rate} measures the percentage of overlap between two independently transformed versions of the benchmark dataset, indicating how much poential contamination among two trials. \textsf{Repeat Trials} quantifies the expected number of transformation trials required to fully regenerate an existing transformed dataset \( T_i(\mathcal{D}) \), providing insight into the benchmark’s ability to produce novel variations.  
These metrics help assess whether a dynamic benchmark remains effective in evaluating LLM capabilities, even when exposed to potential training data contamination.  


\subsubsection{Stable of Complexity}


Dynamic benchmarks must also account for complexity to help users determine whether a performance drop in an LLM on the transformed dataset is due to potential data contamination or an increase in task complexity. If a dynamic transformation increases the complexity of the seed dataset, a performance drop is expected, even without data contamination. However, accurately measuring the complexity of a benchmark dataset remains a challenging task. Existing work has proposed various complexity metrics, but these are often domain-specific and do not generalize well across different applications. For example, DyVal~\citep{zhu2024dyval} proposes applying graph complexity to evaluate the complexity of reasoning problems.
Formally, given a complexity measurement function \( \Psi(\cdot) \), the stability can be formulated as:  
\[
\small
    \textsf{Stability} = \text{Var}(\Psi(D_i))
\]  
This equation can be interpreted as the variance in complexity across different trials, where high variance indicates that the dynamic benchmarking method is not stable.  

% \subsubsection{Stability} 
% Even if a dynamic benchmark has a complexity fucntion for each of its generated data sample, it can also need to enure the benchmark is stable, 



\subsubsection{Diversity }
Besides the aforementioned criteria, another important factor is the diversity of the transformed dataset. This diversity can be categorized into two components: \textsf{external diversity} and \textsf{internal diversity}:  External diversity measures the variation between the transformed dataset and the seed dataset. Internal diversity quantifies the differences between two transformation trials.  
\[
    \small
    \begin{aligned}
        \textsf{External Diversity} &= \mathbb{E}_{i = 1}^{N} \Theta(\mathcal{D}_i, \mathcal{D}) \\
        \textsf{Internal Diversity} &= \mathbb{E}_{\substack{i, j = 1,  i \neq j}}^{N} \Theta(\mathcal{D}_i, \mathcal{D}_j)
    \end{aligned}
\]
where \( \Theta(\cdot) \) is a function that measures the diversity between two datasets. For example, it could be the N-gram metrics or the reference based metrics, such as BLEU scores. 





\subsubsection{Interpretability} 
Dynamic benchmarking generates large volumes of transformed data, making manual verification costly and challenging. To ensure correctness, the transformation process must be interpretable. Interpretable transformations reduce the need for extensive manual validation, lowering costs. Rule-based or manually crafted transformations are inherently interpretable, while LLM-assisted transformations depend on the model's transparency and traceability. In such cases, additional mechanisms like explainability tools, or human-in-the-loop validation may be needed to ensure reliability and correctness.


% \fakeparagraph{Augmentation ?}

% \clearpage
Predictions $\mathbf{P_X} = \{\mathbf{P_M,P_E,P_P,P_C}\}$ are evaluated against ground truth $\mathbf{Y_X} = \{\mathbf{Y_M,Y_E,Y_P,Y_C}\}$ using the LLM-as-a-Judge framework. Unlike prior benchmarks that rely on lexical similarity, \oursbench\ assesses procedural correctness and experimental feasibility. The evaluation criteria are described in Table~\ref{tab:judgment_criteria}.

The scoring function is computed as:

\[
\text{Score}(P_X, Y_X) = \frac{\sum_{i=1}^{N_C} C_i}{N_C}
\]

where $C_i$ represents the score for criterion $i$, and $N_C$ is the total number of evaluation criteria. These criteria were developed in collaboration with domain experts to ensure alignment with real-world synthesis evaluation.

\subsection{Dataset Splits and Distribution}
\label{subsec:dataset_splits}

We divided \oursdatashort~to three splits to ensure robust evaluation:

\begin{itemize}
    \item \textbf{Training Set:} 16,026 articles published before 2024.
    \item \textbf{Test - Standard Impact:} 1,472 articles (2024 and beyond) from journals with Impact Factor (IF) $<$ 10.
    \item \textbf{Test - High Impact:} 169 articles (2024 and beyond) from journals with IF $\geq$ 10.
\end{itemize}

The \textbf{temporal split} ensures that models are evaluated on \textit{unseen future research}, mitigating data contamination. Additionally, stratification by \textbf{journal impact} allows assessment of a model’s ability to process high-impact findings, often introducing novel and complex synthesis techniques. This split design evaluates both \textit{generalizability} and the ability to meet the rigorous standards of top-tier journals\footnote{Table~\ref{tab:high_impact_journals} describes the detailed list of high-impact journals utilized for our test-set split.}.



% \section{AlchemyBench}
% \label{sec:benchmark}

% This section introduces our benchmark for evaluating models on materials synthesis prediction. We outline the motivation for using LLMs as judges, task definitions, dataset splits, and evaluation metrics, culminating in a framework reflecting real-world materials science challenges.

% \subsection{Motivation: Why LLM-as-a-Judge?}
% \label{subsec:motivation}
% The evaluation of general materials synthesis recipes poses unique challenges that necessitate scalable and efficient solutions:
% \begin{itemize}
%     \item \textbf{Lack of Existing Benchmarks:} There is no established benchmark for evaluating general materials synthesis recipes nor a standardized framework to assess their quality systematically.
%     \item \textbf{Limitations of Traditional Metrics:} Commonly used metrics such as BLEU~\cite{papineni2002bleu} and ROUGE~\cite{lin2004rouge} focus on surface-level lexical similarity but fail to capture the semantic requirements of synthesis recipes. \citeauthor{na2023artificial} et al. utilized the Jaccard score to evaluate the set overlap of synthesis procedures but lacks sensitivity to sequential dependencies critical in procedural texts. Additionally, BERTScore~\cite{zhang2019bertscore} leverages contextual embeddings from pretrained language models to measure semantic similarity beyond token matching, offering a more context-aware evaluation. However, despite its advantages, BERTScore often underperforms in the Materials Synthesis Recipe domain, where the intricate sequential dependencies and specialized domain semantics are not fully captured.
%     \item \textbf{Cost and Inefficiency of Human Evaluation:} Employing domain experts for recipe evaluation is prohibitively expensive and time-consuming, with an average evaluation time of 23 minutes per recipe ($\sigma=7.57$) in our experiment.
%     \item \textbf{Need for Scalable Solutions:} A scalable and high-performance evaluation tool is essential to support high-throughput benchmarking. LLMs offer a promising alternative by providing consistent and interpretable evaluations at scale~\cite{gu2025surveyllmasajudge}.
% \end{itemize}

% By addressing these challenges, LLM-as-a-Judge enables efficient benchmarking while maintaining alignment with expert evaluations.


% \subsection{Task Definition}
% \label{subsec:task_definition}
% The proposed benchmark simulates end-to-end workflows in materials synthesis. Given an input $\mathbf{X}$ — comprising the target material, synthesis method, and application domain — the model must predict the following components:
% \begin{itemize}
%     \item $\mathbf{P_M}$: Raw materials (e.g., reagents, solvents) and their quantities.
%     \item $\mathbf{P_E}$: Required equipment (e.g., furnace, autoclave).
%     \item $\mathbf{P_P}$: Synthesis procedures (e.g., reaction steps, times, temperatures).
%     \item $\mathbf{P_C}$: Characterization methods and expected outcomes.
% \end{itemize}

% The predictions $\mathbf{P_X}=\{\mathbf{P_M,P_E,P_P,P_C}\}$ are compared against ground-truth recipes $\mathbf{Y_X}=\{\mathbf{Y_M}, \mathbf{Y_E}, \mathbf{Y_P}, \mathbf{Y_C}\}$ using the LLM-as-a-Judge framework. The evaluation criteria include appropriateness, completeness, feasibility, and similarity to ground truth. This task setup was designed in collaboration with domain experts and mirrors real-world scenarios where researchers infer complete workflows from minimal initial information.

% The LLM-as-a-Judge's overall score $\text{Score}(P_X,Y_X)$ is calculated as follows:
% \[
% \text{Score}(P_X, P_Y) = \frac{\sum_{i=1}^{N_C} C_i}{N_C}
% \]
% where $C_i$ represents the score for criterion $i$, and $N_C$ is the total number of criteria. We will introduce the detailed prompt and criteria in Section~\ref{sec:reliability}.

% \subsection{Dataset Splits and Distribution}
% Our dataset comprises 17,667 articles curated from diverse materials science literature. To ensure robust evaluation, we divide the dataset into three splits based on publication year and journal Impact Factor (IF):
% \begin{itemize}
%     \item \textbf{Training Set:} 16,026 articles published before 2024.
%     \item \textbf{Test - Standard Impact:} 1,472 articles published in 2024 or later with IF $<$ 10.
%     \item \textbf{Test - High Impact:} 169 articles published in 2024 or later with IF $\geq$ 10.
% \end{itemize}

% The temporal split ensures test data represents unseen future research trends, mitigating data contamination. Further stratification by IF allows targeted assessment of model performance on high-impact research, often involving novel and complex synthesis techniques. This split design evaluates both generalizability and the ability to meet the rigorous standards of top-tier journals.


