


\begin{table*}[!ht]
\centering
\caption{A list of high-impact journals (IF $\ge 10$) that at least ten papers are included in \oursdatashort.}
\label{tab:high_impact_journals}
\adjustbox{max width=\textwidth}{
\begin{tblr}{
  vline{2} = {-}{},
  hline{1,7} = {-}{0.08em},
  hline{2} = {-}{},
}
\textbf{Publisher} & \textbf{Journals}                                                                                                                                                                                                                         \\
ACS                & ACS Applied Materials and Interfaces, ACS Nano, ACS Energy Letters, Journal of the American Chemical Society, ACS Catalysis                                                                                                               \\
RSC                & Journal of Materials Chemistry A, Chemical Society Reviews, Energy  Environmental Science                                                                                                                                                 \\
Nature             & {Nature Communications, Nature Materials, Nature Nanotechnology, Nature Energy, Nature Reviews Materials, Nature Catalysis, \\Nature, Nature Electronics, Nature Methods, Nature Chemistry, Nature Physics, Light: Science  Applications} \\
Wiley              & {Advanced Materials, Advanced Energy Materials, Small, Angewandte Chemie, Advanced Science, ChemSusChem,\\Advanced Functional Materials}                                                                                                  \\
Springer           & Nano-Micro Letters, Journal of Advanced Ceramics, Advanced Composites and Hybrid Materials                                                                                                                                                  
\end{tblr}
}
\end{table*}

\begin{figure}
    \centering
    \includegraphics[width=\linewidth]{image/data/yearly-material.png}
    \caption{Yearly distribution of collected material synthesis papers}
    \label{fig:data-yearly-distribution}
\end{figure}

\appendix
\section{Additional Dataset Information}
\label{sec:appendix_dataset}

This section provides extended details on dataset statistics and collection methodology that were not included in the main text for brevity.

\subsection{Keyword Selection Rationale}
% {\color{red}Provide a detailed list of all 50 keywords used for querying the Semantic Scholar API. Explain any selection criteria or domain expert guidance that influenced this choice. Add any discussion on keyword overlap or potential bias.}

Figure~\ref{fig:prompt_search_keywords} describes the search keywords to retrieve 400K articles using Semantic Scholar API~\cite{semanticscholar2023}. We collect these keywords guided by our eight domain experts.

\subsection{Downloading PDFs}
We downloaded open-access papers exclusively from the following six publishers, most frequent in our retrieval result: \href{https://pubs.rsc.org}{pubs.rsc.org}, \href{https://www.mdpi.com}{mdpi.com}, \href{https://www.nature.com}{nature.com}, \href{https://link.springer.com}{link.springer.com}, \href{https://pubs.acs.org}{pubs.acs.org}, \href{https://onlinelibrary.wiley.com}{onlinelibrary.wiley.com}.

\subsection{Dataset Details}
Figure~\ref{fig:train-venue-distribution},~\ref{fig:test-standard-impact-venue-distribution}, and~\ref{fig:test-high-impact-venue-distribution} demonstrate the distributions of venue for train, test-standard-impact, and test-high-impact respectively. Table~\ref{tab:high_impact_journals} describes the high-impact venues (IF $\geq 10$) that at least ten papers are included in~\oursdatashort. Figure~\ref{fig:data-yearly-distribution} demonstrates the dataset distribution of the published year, indicating the latest data, 2020 and beyond, accounts for a large percentage.


\begin{figure}[ht!]
    \centering
    % 첫 번째 이미지
    \begin{subfigure}{\linewidth}
        \centering
        \includegraphics[width=\linewidth]{image/data/top10-train-venue.png}
        \caption{A venue distribution of the train set}
        \label{fig:train-venue-distribution}
    \end{subfigure}
    
    % 두 번째 이미지
    \begin{subfigure}{\linewidth}
        \centering
        \includegraphics[width=\linewidth]{image/data/top10-high-impact-venue.png}
        \caption{A venue distribution of the test high-impact set}
        \label{fig:test-high-impact-venue-distribution}
    \end{subfigure}
    
    % 세 번째 이미지
    \begin{subfigure}{\linewidth}
        \centering
        \includegraphics[width=\linewidth]{image/data/top10-standard-impact-venue.png}
        \caption{A venue distribution of the test standard impact set}
        \label{fig:test-standard-impact-venue-distribution}
    \end{subfigure}

    % 전체 캡션
    \caption{Venue distributions across datasets: (a) training set, (b) test high-impact set, and (c) test standard impact set. The distributions illustrate the diversity and focus of venues in each subset.}
    \label{fig:venue-distributions}
\end{figure}


% \begin{figure}
%     \centering
%     \includegraphics[width=\linewidth]{image/data/top10-train-venue.png}
%     \caption{A venue distribution of the train set}
%     \label{fig:train-venue-distribution}
% \end{figure}

% \begin{figure}
%     \centering
%     \includegraphics[width=\linewidth]{image/data/top10-high-impact-venue.png}
%     \caption{A venue distribution of the test high-impact set}
%     \label{fig:test-high-impact-venue-distribution}
% \end{figure}

% \begin{figure}
%     \centering
%     \includegraphics[width=\linewidth]{image/data/top10-standard-impact-venue.png}
%     \caption{A venue distribution of the test standard impact set}
%     \label{fig:test-standard-impact-venue-distribution}
% \end{figure}

% \begin{figure}
%     \centering
%     \includegraphics[width=\linewidth]{image/data/top10-venue.png}
%     \caption{Top 10 venues}
%     \label{fig:data-publication}
% \end{figure}


% \begin{figure}
%     \centering
%     \includegraphics[width=\linewidth]{image/data/publisher-domain.png}
%     \caption{Publisher}
%     \label{fig:data-publisher-distribution}
% \end{figure}


\vspace{2em}
\section{Annotation Pipeline and Quality Checks}
\label{sec:appendix_annotation}

Here, we elaborate on the annotation workflow and the validation methods used to ensure the reliability of extracted recipes.

\subsection{GPT-4 Prompts for Classification and Extraction}
Figure~\ref{fig:prompt_paper_categorization} demonstrates the prompt to categorize the literature and Figure~\ref{fig:prompt_paper_extraction} demonstrates the prompt to extract the synthesis recipe from the literature.

\subsection{Expert Review Protocol}
\label{subsec:expert-protocol}
% {\color{red}Describe how many experts participated in reviewing the subset of 500 articles. Detail the instructions provided to them, their backgrounds (e.g., PhD in materials science), and the criteria they used to judge completeness/correctness. 
% Summarize the final evaluation guidelines (e.g., rating scales, feedback forms).}

Table~\ref{tab:expert_groups} describes the anonymized details about the eight domain experts in materials science. They participated as volunteers and received no evaluation fees.
Figure~\ref{fig:experts-webui-1} and~\ref{fig:experts-webui-2} demonstrate the web UI screenshots for evaluating LLM predictions by domain experts. Domain experts evaluated 20 LLM predictions with seven criteria in Table~\ref{tab:judgment_criteria} and recorded the results in a spreadsheet. We aggregated those eight spreadsheets and calculated the agreement.



\begin{table}
\centering
\caption{Anonymized details for the domain experts in our study. Conf. denotes the average confidence for evaluating LLM predictions in Section~\ref{sec:reliability}. Group C is the highest confidence group.}
\label{tab:expert_groups}
\adjustbox{max width=\linewidth}{
\begin{tabular}{c|l|r} 
\toprule
\textbf{Group} & \textbf{Expertise}                                                                                                                                                                                                  & \textbf{Conf.}  \\ 
\hline
A              & \begin{tabular}[c]{@{}l@{}}One master and two PhD candidates\\specialized in:\\- Thin film transistors\\- 3D nano-semiconductor thin films\\- atomic layer deposition\end{tabular}                            & 1.90                           \\ 
\hdashline
B              & \begin{tabular}[c]{@{}l@{}}Two master's student\\specialized in:\\- Materials modeling\\- DFT \& MD simulations\\- NLP for materials science\end{tabular} & 3.15                           \\ 
\hdashline
C              & \begin{tabular}[c]{@{}l@{}}Three master's student\\specialized in:\\- Transparent electrodes\\- electrospinning\\- 2D materials\end{tabular}                                                        &  \textbf{4.47}                         \\
\bottomrule
\end{tabular}
}
\end{table}

% \subsection{Inter-Annotator Agreement and Error Analysis}
% {\color{red}Present the final inter-annotator agreement metrics (Cohen’s kappa, Krippendorff’s alpha, etc.) for each component (X, Y\_M, Y\_E, Y\_P, Y\_C). Discuss common sources of error and how they were mitigated.}

\subsection{LLM-Expert Agreement Details}
\label{subsec:llm-expert-agreement-details}

The agreement analysis between expert groups (entire 8-member panel vs. 3-member high-confidence group) and GPT-4o-Nov reveals distinct patterns across evaluation criteria, as shown in Tables \ref{tab:llm-expert-agreement-criteria-full} and \ref{tab:llm-expert-agreement-criteria-high-group}.
These results highlight the critical influence of expert group composition on LLM alignment assessment. Compared to the entire panel, the high-impact subgroup demonstrates enhanced agreement on procedural elements but reduced consensus on characterization tasks, suggesting domain-specific expertise differentially weights evaluation criteria. The stability of non-significant results across both groups for equipment and feasibility judgments implies fundamental challenges in consistently operationalizing these metrics.

\begin{table}
\centering
\caption{A agreement between entire expert consensus and GPT-4o-Nov for each criterion. Subscripts denote the $p-$value.}
\label{tab:llm-expert-agreement-criteria-full}
\adjustbox{max width=\linewidth}{
\begin{tblr}{
  column{2} = {r},
  column{3} = {r},
  vline{2} = {-}{},
  hline{1,9} = {-}{0.08em},
  hline{2} = {-}{0.05em},
}
\textbf{Category}                & \textbf{Pearson} & \textbf{Spearman} \\
Material Appropriateness         & 0.59$_{~0.01}$      & 0.59$_{~0.01}$      \\
Equipment Appropriateness        & -0.25$_{~0.29}$    & -0.25$_{~0.28}$     \\
Procedure Completeness           & 0.05$_{~0.83}$     & 0.09$_{~0.71}$      \\
Procedure Similarity             & 0.41$_{~0.07}$     & 0.40$_{~0.08}$       \\
Procedure Feasibility            & -0.04$_{~0.86}$    & -0.04$_{~0.86}$     \\
Characterization Appropriateness & 0.43$_{~0.06}$     & 0.42$_{~0.07}$      \\
Characterization Similarity      & 0.45$_{~0.05}$     & 0.47$_{~0.04}$      
\end{tblr}
}
\end{table}

\begin{table}
\centering
\caption{A agreement between the expert consensus of the high-confidence group and GPT-4o-Nov for each criterion. Subscripts denote the $p-$value.}
\label{tab:llm-expert-agreement-criteria-high-group}
\adjustbox{max width=\linewidth}{
\begin{tblr}{
  column{2} = {r},
  column{3} = {r},
  vline{2} = {-}{},
  hline{1,9} = {-}{0.08em},
  hline{2} = {-}{0.05em},
}
\textbf{Category}                 & \textbf{Pearson} & \textbf{Spearman} \\

Material Appropriateness & 0.44$_{~0.05}$ & 0.41$_{~0.07}$ \\
Equipment Appropriateness & 0.10$_{~0.68}$ & 0.13$_{~0.58}$ \\
Procedure Completeness & 0.23$_{~0.33}$ & 0.20$_{~0.39}$ \\
Procedure Similarity & 0.56$_{~0.01}$ & 0.50$_{~0.02}$ \\
Procedure Feasibility & -0.04$_{~0.86}$ & -0.04$_{~0.86}$ \\
Characterization Appropriateness & 0.43$_{~0.06}$ & 0.42$_{~0.07}$ \\
Characterization Similarity & 0.09$_{~0.72}$ & 0.16$_{~0.50}$ \\
\end{tblr}
}
\end{table}

\vspace{2em}
\section{Experimental and Implementation Details}
\label{sec:appendix_experiments}

This section thoroughly describes the LLM prompts and hyperparameter settings to facilitate reproducibility.

% \subsection{Hardware and Runtime}
% \begin{itemize}
%     \item {\color{red}List the types of GPUs/CPUs used, their number, and approximate training time per task.}
%     \item {\color{red}Indicate any cloud resources (e.g., AWS, Azure) if applicable.}
% \end{itemize}

\subsection{Hyperparameters}
% {\color{red}Provide tables detailing the chosen hyperparameters for each model (learning rate, batch size, number of epochs, optimizer, etc.). Explain any tuning strategies or early stopping criteria.}
We use a temperature of zero, top-p of 1.0, and max\_tokens of 4096 for GPT-4o-mini and GPT-4o variants. o3-mini variants use max\_completion\_tokens of 16384, and OpenAI does not allow to set temperature and top-p hyperparameters for o3-mini models.

\subsection{LLM Prompt}
\label{subsec:appendix_llm_judge}
% {\color{red}Include sample text of how the 1–5 rating prompt is structured. Discuss any prompt variations tested and how you selected the final version.}
Figure~\ref{fig:prompt_judge} describes the LLM-as-a-Judge prompt. The LLM outputs the JSON-formatted judgment of seven criteria and an overall score for extraction.

\subsection{Other Artifacts}
We utilized LiteLLM~\cite{litellm} and FAISS~\cite{douze2024faiss}, Huggingface Datasets~\cite{lhoest-etal-2021-datasets}.
We confirmed that all models, datasets, and frameworks are allowed for research use. 



\vspace{2em}
\section{Additional Results and Analysis}
\label{sec:appendix_results}

% \subsection{Expanded Tables and Figures}
% {\color{red}Include any supplementary tables or figures not shown in the main text—e.g., detailed confusion matrices for equipment prediction, heatmaps of frequently co-occurring reagents and materials.}
Table~\ref{tab:rag_full_result} describes the detailed result of RAG experiments in Figure~\ref{fig:rag_high_impact} for four base LLMs. 


% \subsection{Case Studies}
% {\color{red}Present a few extended examples of successful and failed predictions from the tested models, with ground-truth vs. predicted recipes. Offer domain expert commentary on why certain outputs are plausible or not.}

% \subsection{Ablation Studies}
% {\color{red}If you performed ablation studies (e.g., removing certain data splits, excluding certain prompts or hyperparameters), show the results and analysis here.}

\vspace{2em}
\section{Ethical Considerations and Potential Risks}  
\label{sec:appendix_ethics}
% \subsection{Data Sharing and Licensing}
% {\color{red}Explain the licensing or legal frameworks (e.g., Creative Commons) under which the dataset can be shared. If the full dataset cannot be shared, clarify what subsets or metadata are publicly available.}
Our data collection approach exclusively utilized open-access publications from six major publishers to ensure copyright compliance. 
Additionally, we verified the 12958 articles through keyword-based content filtering and selenium-confirmed articles of CC-BY licensing status, supplemented by a manual sampling of 100 randomly selected articles to validate redistribution rights.
While this strategy mitigates legal risks, two potential limitations warrant consideration: First, the open-access corpus may exhibit selection bias toward well-funded research domains (e.g., energy materials) versus proprietary industrial methods. Second, automated extraction via GPT-4o risks propagating subtle errors from source documents, particularly in stoichiometric ratios and procedural sequencing, despite our expert validation protocol. All dataset derivatives will be distributed under original CC-BY licenses.

\section{AI Assitant}
We use \href{https://copilot.microsoft.com/}{Microsoft Copilot} as a coding assistant and \href{https://grammarly.com/}{Grammarly} and \href{https://www.writefull.com/}{Writefull} as a writing assistant.

\begin{table}
\centering
\caption{A full experiment result of RAG prediction in Section~\ref{sec:experiment}}
\label{tab:rag_full_result}
\adjustbox{max width=\linewidth}{
\begin{tblr}{
  row{3} = {r},
  row{4} = {r},
  row{5} = {r},
  row{6} = {r},
  row{8} = {r},
  row{9} = {r},
  row{10} = {r},
  row{11} = {r},
  row{13} = {r},
  row{14} = {r},
  row{15} = {r},
  row{16} = {r},
  row{18} = {r},
  row{19} = {r},
  row{20} = {r},
  row{21} = {r},
  cell{1}{2} = {r},
  cell{1}{3} = {r},
  cell{1}{4} = {r},
  cell{1}{5} = {r},
  cell{1}{6} = {r},
  cell{2}{1} = {r=5}{},
  cell{2}{2} = {r},
  cell{2}{3} = {r},
  cell{2}{4} = {r},
  cell{2}{5} = {r},
  cell{2}{6} = {r},
  cell{7}{1} = {r=5}{},
  cell{7}{2} = {r},
  cell{7}{3} = {r},
  cell{7}{4} = {r},
  cell{7}{5} = {r},
  cell{7}{6} = {r},
  cell{12}{1} = {r=5}{},
  cell{12}{2} = {r},
  cell{12}{3} = {r},
  cell{12}{4} = {r},
  cell{12}{5} = {r},
  cell{12}{6} = {r},
  cell{17}{1} = {r=5}{},
  cell{17}{2} = {r},
  cell{17}{3} = {r},
  cell{17}{4} = {r},
  cell{17}{5} = {r},
  cell{17}{6} = {r},
  vline{3} = {-}{},
  hline{1,22} = {-}{0.08em},
  hline{2,7,12,17} = {-}{},
}
\textbf{Model} & \textbf{K} & \textbf{Mean} & $\sigma$ & \textbf{Min} & \textbf{Max} \\
o3-mini-high   & 0          & 3.759         & 0.407    & 2.86         & 4.71         \\
               & 1          & 3.937         & 0.401    & 2.80         & 4.86         \\
               & 5          & 4.001         & 0.384    & 3.00         & 4.80         \\
               & 10         & 3.939         & 0.359    & 3.00         & 4.80         \\
               & 25         & 3.986         & 0.383    & 3.00         & 4.71         \\
o3-mini-medium & 0          & 3.714         & 0.411    & 2.86         & 4.64         \\
               & 1          & 3.867         & 0.381    & 3.00         & 4.80         \\
               & 5          & 3.934         & 0.349    & 2.86         & 4.80         \\
               & 10         & 3.937         & 0.390    & 2.57         & 4.80         \\
               & 25         & 3.975         & 0.413    & 2.50         & 4.80         \\
o3-mini-low    & 0          & 3.676         & 0.407    & 2.50         & 4.57         \\
               & 1          & 3.848         & 0.411    & 2.90         & 4.80         \\
               & 5          & 3.904         & 0.393    & 2.93         & 4.86         \\
               & 10         & 3.917         & 0.397    & 2.86         & 4.86         \\
               & 25         & 3.961         & 0.388    & 3.00         & 4.80         \\
GPT-4o Nov     & 0          & 3.709         & 0.410    & 2.75         & 4.71         \\
               & 1          & 3.824         & 0.417    & 2.57         & 4.80         \\
               & 5          & 3.949         & 0.391    & 2.93         & 4.80         \\
               & 10         & 3.954         & 0.366    & 3.00         & 4.80         \\
               & 25         & 3.976         & 0.375    & 2.93         & 4.86         
\end{tblr}
}
\end{table}


% \subsection{Potential Misuse of Recipes}
% {\color{red}Discuss any concerns about malicious use of certain synthesis protocols or hazardous materials. If relevant, mention disclaimers or guidelines put in place.}

% \section{Code and Resource Availability}
% \label{sec:appendix_code}
% {\color{red}Provide a link to the public GitHub repository or indicate a timeline for release. List any special dependencies (e.g., PyMuPDFLLM) and exact version numbers to ensure reproducibility. Include instructions on how to run the code for each task and replicate the benchmarks.}

% \begin{table}[t]
% \centering
% \adjustbox{max width=\textwidth}{
% \begin{tabular}{|c|c|c|c|c|}
% \hline
% \textbf{Dataset} & \textbf{Method} & \textbf{Open-Access?} & \textbf{Procedure Quality} & \textbf{Explicit Evaluation Framework} \\
% \hline
% A et al. & Solution-based & X & Noisy & X \\
% B et al. & Solid-state & X & Noisy & X \\
% OMG (ours) & 12 methods & O & High-Quality & O \\
% \hline
% \end{tabular}
% }
% \caption{Comparison of previous datasets and OMG (ours). Unlike prior work, OMG provides expert-verified synthesis procedures with a structured evaluation framework (\textbf{completeness, correctness, coherence}) and validation methods.}
% \end{table}

\begin{figure*}
    \centering
    \includegraphics[width=\textwidth]{image/experts/humaneval_common_page-0001.jpg}
    \caption{A web UI screenshot for domain experts annotation (1/2).}
    \label{fig:experts-webui-1}
\end{figure*}

\begin{figure*}
    \centering
    \includegraphics[width=\textwidth]{image/experts/humaneval_common_page-0002.jpg}
    \caption{A web UI screenshot for domain experts annotation (2/2).}
    \label{fig:experts-webui-2}
\end{figure*}

\begin{figure*}[!h]
    \centering
    \lstinputlisting{prompt/search_keywords}
    \caption{60 search keywords to retrieve the literature, including materials synthesis recipes using Semantic Scholar API.}
    \label{fig:prompt_search_keywords}
\end{figure*}

\begin{figure*}[!h]
    \centering
    \lstinputlisting{prompt/paper_categorization}
    \caption{System prompt to categorize the literature converted to markdown format.}
    \label{fig:prompt_paper_categorization}
\end{figure*}

\begin{figure*}[!h]
    \centering
    \lstinputlisting{prompt/paper_extraction}
    \caption{System prompt to extract the recipe from literature converted to markdown format.}
    \label{fig:prompt_paper_extraction}
\end{figure*}


\begin{figure*}[!h]
    \centering
    \lstinputlisting[basicstyle=\ttfamily\small]{prompt/paper_extraction}
    % \inputminted[breaklines, fontsize=\small]{markdown}{prompt/prediction_1shot.tex}
    \caption{A prompt to predict the recipe with a one-shot example.}
    \label{fig:prompt_prediction_1shot}
\end{figure*}

% \begin{figure*}[!h]
%     \centering
%     \lstinputlisting{prompt/prediction_1shot_example}
%     \caption{An example recipe used to a one-shot example.}
%     \label{fig:prompt_prediction_1shot_example}
% \end{figure*}

\begin{figure*}[!h]
    \centering
    \lstinputlisting{prompt/prediction_rag}
    \caption{A prompt to predict the recipe using retrieval-augmented generation}
    \label{fig:prompt_prediction_rag}
\end{figure*}

\begin{figure*}[!h]
    \centering
    \lstinputlisting{prompt/judge}
    \caption{A prompt to judge the prediction recipe using LLM-as-a-Judge}
    \label{fig:prompt_judge}
\end{figure*}
