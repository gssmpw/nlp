You are an expert in material science and chemical synthesis. Your task is to design a detailed material synthesis recipe, considering the following **key contributions**:
Your output should follow the exact structure and format of the below example. Provide precise details for each section, including materials, equipment, and step-by-step procedures.

-- Example Input --
## Key Contributions
- **Novel materials or compounds**: Nanoceria-based sorbents for enhanced sorption of low specific activity [99]Mo.
- **Unique synthesis methods**: Development of CeO2 nanoparticles using wet chemical precipitation and hydrothermal approaches.
- **Specific applications or domains**: Potential application in the development of [99]Mo/[99m]Tc generators for nuclear medicine. 
-- Example Output --
## Materials
- **Precursor Materials**:
  - Cerium nitrate hexahydrate (Ce(NO3)3\cdot6H2O): 2g.
  - Sodium hydroxide: 0.4 M solution.
- **Solvents**:
  - Milli-Q water.
  - Ethanol absolute (1:4 v/v with water for chemical precipitation, 4:1 v/v for hydrothermal method).
- **Reagents**:
  - Nitric acid (HNO3).
  - 1 M NaOH solution for adjusting pH.
- **Specifications**:
  - All chemicals of analytical grade from Merck, Darmstadt, Germany.
## Synthesis Equipment
- **Centrifuge**: Used at 4000 rpm for 15 minutes.
- **Drying Oven**: Temperature range of 50–70 $^{\circ}$C for 12 hours.
- **Teflon-lined Stainless Steel Autoclave**: Used for hydrothermal synthesis at 150 $^{\circ}$C for 12 hours.
## Synthesis Procedure
### Wet Chemical Precipitation
1. Dissolve 2 g of Ce(NO3)3\cdot6H2O in a mixture of Milli-Q water and ethanol (1:4 v/v).
2. Add the solution drop-wise to a 0.4 M sodium hydroxide solution under continuous stirring, maintaining pH ~11.
3. Stir until a yellow suspension forms (~30 min). 
4. Wash the precipitate with de-ionized water and ethanol; centrifuge and dry at 50–70 $^{\circ}$C for 12 h. 
5. Denote the sample as CP (Chemical Precipitation).
### Hydrothermal Modification
1. Transfer the precipitated NPs into a Teflon-lined stainless-steel autoclave.
2. Seal and maintain at 150 $^{\circ}$C for 12 hours.
3. Wash, centrifuge, and dry the precipitate at 70 $^{\circ}$C for 12 h.
4. Denote the sample as HT (Hydrothermal Treatment).

-- Prediction Input --
Now, predict the synthesis recipe for the given key contributions:
{contributions}