Analyze the given scientific text and provide classifications in the following order:

1. Synthesis Recipe Classification:
Determine if the text contains detailed synthesis procedures.
Return only "YES" or "NO".
If "NO", stop here. If "YES", continue with the following classifications.

2. Target Classification:
Classify the synthesized target as one of:
- Material (e.g., nanoparticles, compounds, composites)
- Device (e.g., sensors, batteries, transistors)
- Molecule (e.g., organic compounds, polymers)

3. Material Identification:
Provide:
- Chemical formula (if applicable)
- Material name
- Material class (e.g., metal oxide, polymer, semiconductor)

4. Application Domain:
List the primary applications mentioned in the text:
- Energy (e.g., batteries, solar cells)
- Electronics (e.g., transistors, sensors)
- Healthcare (e.g., drug delivery, imaging)
- Environmental (e.g., catalysis, filtration)
- Others (specify)

5. Synthesis Process Classification:
Classify the given synthesis method into one of these categories. If it combines multiple methods, label it as "Hybrid". If it doesn't fit any category, label it as "Others".

Categories:
1. Solid-State: solid-state reaction, ceramic method, sintering
2. Vapor Deposition: CVD, PVD, sputtering, evaporation
3. Mechanochemical: ball milling, mechanical alloying
4. Hydrothermal: solvothermal, pressurized solution
5. Pyrolysis: thermal decomposition, spray pyrolysis
6. Melt Quenching: rapid solidification, glass formation
7. Electrochemical: electrodeposition, anodization
8. Self-Assembly: molecular assembly, biomineralization
9. Solution-Based: precipitation, sol-gel, wet chemical synthesis
10. Biological: biomimetic, enzyme-mediated, microbial synthesis
11. Hybrid: combination of multiple methods
12. Others: novel or unconventional methods


Format the output as a structured list only if Step 1 is "YES".
For not available, use "N/A".
Do not provide explanations or additional commentary.

Example Output:
For a paper titled "Hydrothermal Synthesis of LiFePO4/C Composites for High-Performance Lithium-Ion Batteries":

1. Synthesis Recipe: YES
2. Target: Material
3. Material Identification:
- Chemical Formula: LiFePO4/C
- Material Name: Carbon-coated lithium iron phosphate
- Material Class: Phosphate composite
4. Application Domain: Energy (lithium-ion batteries)
5. Synthesis Process: Hydrothermal (solvothermal)


Scientific Paper:
{text}