\section{Introduction}
\label{sec:intro}

Materials synthesis is a cornerstone of scientific progress, driving innovations across various domains such as energy storage, catalysis, electronics, and biomedical devices~\cite{olivetti2020data}. Despite notable advancements, the synthesis process remains empirical, relying on trial-and-error approaches guided by expert intuition~\cite{merchant2023scaling}. This reliance on heuristic methods often impedes novel materials' rapid discovery and optimization. Consequently, a systematic understanding and predictive capability in materials synthesis are critical for accelerating research and development~\cite{huang2023application}.

\begin{figure}[!ht]
    \centering
    \includegraphics[width=\linewidth]{image/intro/intro_v01_simple.png}
    \caption{An illustrated example of materials synthesis pipeline in \oursbench.}
    \label{fig:intro}
\end{figure}

% In recent years, progress in natural language processing (NLP) has demonstrated promising avenues for extracting actionable insights from unstructured scientific literature. By leveraging large-scale text data, NLP-based methods can help automate the identification of synthesis parameters, uncover hidden correlations, and generate testable hypotheses for materials discovery~\cite{song2023matsci,dunn2020benchmarking}. However, the application of NLP in materials science remains limited by data quality and domain diversity.
% Existing datasets and benchmarks are 1) often small in scale. 2) are prone to noise 3) are narrowly focused target materials~\cite{sun2025critical}. These limitations undermine the generalizability of machine learning models developed for synthesis prediction.

In recent years, Machine Learning (ML) and Large Language Models (LLMs) have emerged as promising tools for leveraging the vast amount of unstructured scientific literature available in materials science. These methods can automate the extraction of synthesis parameters, uncover hidden correlations, and generate testable hypotheses~\cite{song2023matsci,dunn2020benchmarking}. However, several challenges hinder the practical application in this domain: \textbf{1) Data Limitations:} Existing datasets are prone to narrowly focused on specific synthesis procedures~(\eg, solid-state, sol-gel, and solution-based) and prone to noisy~\cite{sun2025critical}, limiting the generalizability of ML models.
\textbf{2) Lack of Standardized Benchmarks:} The absence of comprehensive benchmarks for evaluating end-to-end synthesis prediction models impedes progress and prevents meaningful comparisons across approaches.
\textbf{3) Unclear Model Capabilities:} The current level of performance achievable by ML or LLM models in predicting synthesis workflows remains underexplored.
\textbf{4) Evaluation Challenges:} Determining whether LLMs can reliably approximate human-level evaluations for synthesis predictions is an open question.

To address these challenges, we propose \oursdatalong~(\oursdatashort), a dataset of high-quality 17K material synthesis recipes curated from a large corpus of scientific literature using automated methods. 
Based on this dataset, we designed \oursbench~to benchmark models on tasks that mirror real-world materials synthesis workflows:

\begin{figure}[!t]
    \centering
    \includegraphics[width=\linewidth]{image/intro/process_v02.png}
    \caption{An illustrated workflow of \oursdatalong~construction.}
    \label{fig:intro_data_collection}
\end{figure}

\begin{enumerate}
    \item \textbf{Predicting raw materials:} Given a target material, models are challenged to generate precursor, solvent, and catalyst identities and predict their respective quantities or ratios.
    \item \textbf{Predicting synthesis equipment:} Models must determine the apparatus for a given synthesis based on the target material and precursor information.
    \item \textbf{Predicting synthesis procedures:} Given the target material, raw materials with proportions, and equipment, the model should generate the procedural steps.
    \item \textbf{Predicting characterization methods and outcomes:} Models must infer the appropriate characterization techniques and their possible outcomes based on complete synthesis details.
\end{enumerate}




Figures~\ref{fig:intro} and~\ref{fig:intro_data_collection} illustrate the overall workflow of our proposed dataset construction and benchmark tasks. By capturing the end-to-end pipeline of materials synthesis, our benchmark provides a holistic framework for evaluating NLP models that can generalize across diverse synthesis methods and materials.

Moreover, we explore the potential of LLMs to act as evaluators for synthesis predictions. While human experts provide the gold standard for evaluation, their involvement is time-intensive and costly. We investigate whether LLMs can reliably approximate human-level judgment across multiple evaluation dimensions. This dual focus—on both model performance and evaluation reliability—aims to advance the integration of data-driven approaches into practical materials science.

In summary, this work makes the following contributions:
\begin{itemize}
    \item We present \oursdatalong~(\oursdatashort), a large-scale dataset of 17K material synthesis recipes curated from the scientific literature~(including the latest literature until 2024).
    \item We introduce \oursbench, a comprehensive benchmark that evaluates models on end-to-end tasks spanning precursor prediction to characterization outcomes.
    \item We assess the capabilities of state-of-the-art LLMs in predicting synthesis workflows and examine their reliability as evaluators compared to human experts.
    \item We provide experimental results highlighting key insights into model performance and propose future directions for improving data-driven approaches in materials science.
\end{itemize}
% By capturing the end-to-end pipeline of materials synthesis, this benchmark lays the groundwork for evaluating and developing NLP models that can effectively generalize across diverse synthesis methods and materials. In particular, our emphasis on real-world workflows provides a holistic perspective that moves beyond isolated prediction tasks to encompass all stages of the synthesis process.

% Ultimately, we aim to facilitate the convergence of data-driven approaches and practical materials science. We strive to spur methodological advancements in NLP and deeper collaborations between the materials science and machine learning communities by offering a rigorous evaluation framework. Through this benchmark, we envision a future where machine-learning models expedite the discovery of novel materials and provide actionable guidance for their synthesis and characterization. 