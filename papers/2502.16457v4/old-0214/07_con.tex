\section{Conclusion}
\label{sec:conclusion}

This work introduces a comprehensive benchmark for evaluating NLP models in the context of materials synthesis prediction, addressing critical challenges in data-driven materials science. By curating a large-scale dataset and designing tasks that reflect real-world synthesis workflows, we provide a robust framework for assessing model capabilities across precursor selection, equipment inference, procedure generation, and characterization prediction. Our experiments demonstrate the potential of reasoning-based models, such as o3-mini, to excel in structured problem-solving tasks, achieving superior performance compared to general-purpose models like GPT-4o variants. Integrating retrieval-augmented generation (RAG) further enhances recipe quality by grounding predictions in domain-relevant examples, with optimal gains observed up to \(K=10\).

The insights derived from this study highlight the importance of reasoning capabilities and contextual adaptability in advancing materials synthesis prediction. While reasoning-based models outperform general-purpose counterparts in generating coherent and feasible workflows, their application can be further enhanced by leveraging retrieval mechanisms and domain-specific fine-tuning. These findings underscore the need for hybrid approaches that combine robust reasoning architectures with adaptive retrieval strategies to address the nuanced requirements of materials science tasks. This work lays the foundation for interdisciplinary innovation by bridging NLP and materials science, accelerating progress toward data-driven materials discovery and design.
