



Cancer surgery is a key treatment for gastrointestinal (GI) cancers, a group of cancers that account for more than 35\% of cancer-related deaths worldwide, but postoperative complications are unpredictable and can be life-threatening. In this paper, we investigate how recent advancements in large language models (LLMs) can benefit remote patient monitoring (RPM) systems through clinical integration by designing \textit{RECOVER}, an LLM-powered RPM system for postoperative GI cancer care. To closely engage stakeholders in the design process, we first conducted seven participatory design sessions with five clinical staff and interviewed five cancer patients to derive six major design strategies for integrating clinical guidelines and information needs into LLM-based RPM systems. We then designed and implemented \textit{RECOVER}, which features an LLM-powered conversational agent for cancer patients and an interactive dashboard for clinical staff to enable efficient postoperative RPM. Finally, we used RECOVER as a pilot system to assess the implementation of our design strategies with four clinical staff and five patients, providing \textbf{design implications} by identifying crucial design elements, offering insights on responsible AI, and outlining opportunities for future LLM-powered RPM systems.
