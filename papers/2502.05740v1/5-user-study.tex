\section{User Studies: Pilot Studies with GI Cancer Providers and Patients}
\label{sec:5-evaluation}

To assess our design guidelines and inform future design of similar systems, we used our \projectname{} system as a pilot study to conduct user studies with GI cancer patients and providers. In our efforts understanding user perspectives and experiences, we aim to study:

\begin{itemize}
    \item RQ1: Are our design decision for LLM-powered systems for RPM effective for clinical work in GI cancer care?
    \item RQ2: What are the future opportunities and concerns from GI cancer care providers and patients towards such LLM-powered systems for the postoperative RPM?
\end{itemize}








\subsection{User Study 1 with Clinical Staff}
\subsubsection{Participants and Procedure}
We recruited GI cancer healthcare providers from a maintained participant mailing list after IRB approval. Like the participants in \ref{sec:3-participatory-design}, they are highly involved in daily communication and treatments of GI cancer patients. Their detailed background is listed in table \ref{tab:user-study-participants}.



\begin{table*}[t]
    \centering
    \caption{Backgrounds of User Study Participants }
    \label{tab:user-study-participants}
    \begin{booktabs}{
    colspec={cccX[c]},
    width=\linewidth,
    hspan=minimal,
    cells={c,m},
    }
        \toprule
         EP\#& Role & Years of Experience & Responsibility Related to GI Cancer\\
         \midrule
         EP1& Patient Coordinator& Over 20 years& Patient coordination and communication pre- and postoperatively\\
         EP2& Patient Coordinator & 1-5 years &  Patient coordination and communication pre- and postoperatively\\
         EP3& Doctor & Over 30 years&   Clinical treatment and diagnosis\\
         EP4& Clinical researcher &  Over 20 years&  Clinical research and patient education\\
         \bottomrule
    \end{booktabs}
\end{table*}

After introducing the study, we assigned provider participants roles reflecting their daily responsibilities (doctor or patient coordinator) and a scenario involving monitoring a postoperative GI cancer patient (post-hemicolectomy for stage 2 colon cancer, experiencing abdominal pain and constipation). We then played a demonstration video showcasing the LLM-powered CA's interaction with the patient, covering key symptom inquiries and follow-up questions. Participants later discussed their impressions of the conversation quality and the CA's performance, as outlined in Section \ref{sec:3-participatory-design}.

Next, participants accessed the \projectname{} dashboard via a provided link to complete tasks in their professional roles. 
The tasks include: (1) locating ``Patient 2 Bella'', reviewing the latest report, (2) analyzing visualizations, conversation logs, and summaries, (3) annotating actions or comments, and updating the patient’s status to ``reviewed''. 
Participants were asked to share their screens while navigating the dashboard and encouraged to "think out loud" to share their thoughts on the interface and interaction design. This was followed by a semi-structured interview where participants provided feedback on the system’s design, its utility in postoperative GI cancer RPM, and potential improvements. 

\begin{figure*}
    \centering
    \includegraphics[width=\linewidth]{figures/UserStudySession.pdf}
    \caption{Our user study session. A participant is navigating on the dashboard to complete the given task to review patient report.}
    \label{fig:5-user-study-session}
\end{figure*}


Lastly, they completed a short evaluation questionnaire. Our questionnaire includes the System Usability Scale (SUS)~\cite{Brooke_1995_SUS} (ten likert-scale questions) to evaluate the usability of the dashboard. In addition, we list questions about how much the LLM-powered functionalities may be reliable and supportive of their RPM work. The questions had a 5-point scale, as listed in Appendix \ref{subsec:8-appendix-questionnarie}
All the user study sessions were conducted via Zoom, recorded, and transcribed after gaining participants' consent.
We summarize our findings below, presenting a balanced view of user interactions, system evaluations, and potential areas for improvement in \projectname{}.

\subsubsection{Findings}
Overall, our participants find our system highly user-friendly and comprehensive, and the LLM-powered features could help them prioritize critical cases in monitoring postoperative GI cancer patients, and look forward to system deployment and further integration.

\paragraph{High Usability and Task Efficiency}
Participants report great usability of our dashboard interface, with an average SUS score of 93.75 ($sd=5.204$). As shown in Table \ref{tab:user-study-performance}, all participants completed the assigned tasks independently or with the researchers' hints within two minutes on average. 
Specifically, participants commented that the system is ``easy to use''(PP2), and more straightforward than some current Electronic Medical Record (EMR) platforms (PP3). 


        

\paragraph{Enhancing RPM Through LLM-Powered Contextual Insights and Visualizations}
All participants strongly agreed that the system could overall effectively support their RPM of postoperative GI cancer patients, as shown in our questionnaire ratings in Table \ref{tab:8-appendix-feature-questionnarie}. Participants also rated highly on specific LLM-powered features, including conversation, visualization, summary, and conversation log highlights in their correctness and support for RPM, with at least 3 out of 4 participants ranked as ``Strongly Agree''.
For the conversation, participants are excited ahout the logic of narrowing down and the LLM's ability to infer patient symptoms from the context.
\epquote{1}{it's interesting to see how one particular question can lead to other areas of interest. } 

For the dashboard, all participants find the colored visualization particularly helpful for them:
\epquote{1}{I think it's going to be really easy to get a view of multiple patients at once which is going to be critical}. EP1 also liked the current options to keep the colors for each symptom or make adjustments. Participants also find the other LLM-powered features could promote their efficiency: \epquote{2}{without having to delve into the detailed log, [the summary] tells the reader things they need to know right away}; the ordering and colors in the patient list \epquote{3}{focus my thinking and my attention on the things that are most important for a post-op patient}; conversation logs and demographic information also support contextual considerations (EP3).

\paragraph{Addressing Patient Experience, Safety Concerns, and System Clarity
}
Corresponding to our PD findings, the participants suggested improvements and concerns about patients' experience, safety and system clarity.
For the conversation with patients, EP1 and EP3 noticed pauses before the CA's response, and suggested that we try reducing the response time or add this reminder to patient instructions in order to improve the patient experience and ensure a smooth onboarding. With two participants voting ``Strongly Disagree'' or ``Disagree'', the safety risk from the LLM conversation to patients is the topic that participants are most concerned about. EP1 expressed the wish to always avoid medical advice when prompting AIs to interact with patients, and only utilize AI for ``information gathering''.

We noticed that the participants paid extra attention to wording on the dashboard. EP1 commented that the LLM could
\epquote{1}{... reserve [the word] ``discussion'' for an actual live exchange, and ... it could be something like ``patient reported''...[so that]We can have a distinction between when we talk to them, and when the bot talks to them.} EP4, meanwhile, prefers the wording to be more concise, \eg from ``26 y. o. female'' to ``26F'', and use shorter phrases for summary bullet points. The experts believe that these improvements will improve system clarity and efficiency.



\paragraph{Patient Engagement, Integration, and System Reliability}
Overall, our participants expressed strong interest in deploying \projectname{} or similar LLM-powered systems for RPM in real-world settings.
Some participants suggested a next step to further engage postoperative patients in this RPM process.  \epquote{1}{for the patient to be able to see their own dashboard and be able to click something that generates a message to their doctor's office.}
EP1 added that the system could be integrated with commonly used solutions such as the patient portal, potentially as a temporary add-on \epquote{1}{that is only used and deployed for a defined period of time.}
Similarly, EP4 suggested that the system includes\epquote{4}{a place where we can, at the end of the or throughout the process, indicate what happens to the patient}, tracking patient treatments and outcomes like EMR systems.
On the other hand, the experts also discussed concerns about the future deployment of such systems for RPM. Apart from EP1's concern about LLM's safety risk to patients and unclear wording to participants, EP4 also mentioned the potential loss of information or connectivity issues, emphasizing the importance of information backup.

\subsection{User Study 2 with Cancer Patients}

\subsubsection{Participants and Procedure}

We recruited the same patient participants to test the \projectname{} system's patient interface. The studies were conducted remotely via Zoom and researchers used a functioning Alexa Echo Dot in view of the laptop camera, allowing participants to interact with the CA via Zoom audio and video.

Participants completed three tasks: (A) imagining a phone call or patient portal interaction with healthcare providers, (B) reflecting on their experience filling out a symptom-reporting questionnaire, and (C) interacting with the RECOVER bot, followed by demo videos of both patient and provider interfaces.
After the tasks, participants filled out a questionnaire, including (1) a system usability scale (SUS)~\cite{bangor2008empirical, lewis2018system} to evaluate the \projectname{} CA and (2) ratings on availability, ease of understanding, and key symptom coverage of \projectname{} CA compared to healthcare providers (Task A) and questionnaires (Task B), using a 7-point scale (See Appendix \ref{sub:8-appendix-patient-user-study}). Follow-up interviews gathered feedback on the chatbot, symptom reporting, desired features, and future concerns.

\subsubsection{Findings}
The participants rated the system's SUS score with an average of 85, indicating overall \textit{good usability}. All participants rated ``strongly agree'' for ``I would imagine that most people would learn to use this system very quickly.'' and all rated ``agree'' or ``strongly agree'' for ``I think that I would like to use this system frequently.''.
Participants appreciated the simple ``easy to use'' interactions (PP2, PP4) and the LLM CA's ``top-notch'' capability to understand human responses (PP2, PP3). \textit{``It was more of like having a mutual conversation with the doctor'' }(PP2) 

Compared to contacting healthcare providers from home or filling in a questionnaire, most participants rated from ``slightly better'' to ``much better'' for the RECOVER bot's performance. Not only did four participants report the availability of \projectname{} better than the providers, but all participants reported ``the extent to which the RECOVER bot covers key details of symptoms'' and the ease of understanding is about the same as or better than contacting healthcare providers. The patients also rated the \projectname{} bot better or the same as questionnaires in all three aspects.
The detailed results are presented in Table \ref{tab:8-appendix-user-study-results-2}. Participants not only favored the \textit{conversational features} to interchange rich information(PP2) but also found the \textit{clinical integration} helpful. 
\textit{``I like the whole system because showing how the intensity, the symptoms, at times you will even not know like some symptoms are a red flag. So it keeps you updated on what to watch out for by yourself without necessarily having to call in or someone ask you. '' }(PP1). PP2 and PP5 further pictured the system could integrate more clinical knowledge and medical history.

The patient participants expressed some \textit{concern} about the LLM CA in a real-world deployment.
PP4 suggested \textit{``create as much awareness as you can''} to minimize people's stereotypes towards RPM technology.
PP3 and PP5 admitted that the privacy issue should be made aware, 
\textit{``try to encourage anonymity, whereby someone can fully give the information without necessarily giving out their details such as their phone number, their name.''} (PP5).
Interestingly, many participants adhered to the given scenarios when reporting symptoms in Task A but added personal symptoms in Task B, and even more detailed symptoms and questions in Task C. For instance, PP3 mentioned issues with daily activities and low mood, while PP5 reported having a fever to the chatbot. This suggests that the LLM CA helps collect \textit{richer patient health data} than traditional methods but also brings potential privacy risks as patients may share more personal information.
