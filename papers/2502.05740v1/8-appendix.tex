\clearpage
\onecolumn
\section{Appendix}
\label{sec:appendix}
\subsection{Initial Design Components}
\subsubsection{Example conversation from initial LLM CA design}
In this version, although the LLM CA responses fit the postoperative context, there were potentially inappropriate expressions such as assessing patient conditions, not covering most questions, and miscommunication of system responsibilities.

\begin{longtblr}[
caption = {An example conversation between a test user (simulating a patient) and the CA with our initial prompt design.},
label={tab:8-appendix-initial-conversation}
]{
hlines,vlines,
colspec={lX},
width=\linewidth,
rowhead=1
}
\textbf{Role} & \textbf{Content}  \\

\textbf{User:} & Hello \\
\textbf{CA:} & Good afternoon! This is "surgery recover" from the clinic, I'm calling to see how you're doing after your cancer surgery. Have you been experiencing any specific discomforts or symptoms that you would like to discuss? \\ 
\textbf{User:} & I am good, thanks for asking \\   
\textbf{CA:} & That's good to hear. Let's continue with our check-up. How about your food intake? Have you noticed any change in your appetite or taste? \\   
\textbf{User:} & My food intake is good, I have great appetite \\   
\textbf{CA:} & That's great! You are maintaining well. Have you noticed any changes in your bowel movements? Any issues like constipation or bloating? \\   
\textbf{User:} & Well I may have ate a lot, so my bowel movement may be a little more frequent than usual, but I think that is fine \\   
\textbf{CA:} & I see, an increase in bowel movement due to an increase in food intake isn't necessarily a cause for concern. However, if you experience any significant changes such as severe diarrhea, constipation or blood in stool, these should be reported to your healthcare provider immediately. Now, let's move to another area. Have you noticed any pain or swelling anywhere in your body after surgery? \\   
\textbf{User:} & Ok. Well, actually I have some pain around my stoma \\   
\textbf{CA:} & I'm sorry to hear that you're experiencing some pain around your stoma. Can you tell me more about the pain? How would you rate the severity of the pain on a scale from 1-10, with 1 being mild discomfort and 10 the worst pain imaginable? \\   
\textbf{User:} & I would say 5 out of 10 when I move around, and maybe 2 or 3 out of 10 when I sit still. \\   
\textbf{CA:} & I understand. It's not uncommon to experience some pain around the stoma after surgery. Can you tell me if the pain is constant, or does it come and go? Also, have you noticed any redness, swelling, or changes in the color of output from your stoma? \\   
\textbf{User:} & I think it is consistent, but more severe sometimes than others. I am not sure about other things, it seems normal \\   
\textbf{CA:} & Thank you for providing those details. I understand the pain can be quite uncomfortable. I suggest you take a note of the times when the pain increases, and if the pain becomes severe or persists it's important to report it to your healthcare provider. Is there anything else you would like to discuss or any other symptoms you've noticed? \\   
\textbf{User:} & Ok, thanks for telling me that. Can you ask my doctor whether I should go in for a visit? \\   
\textbf{CA:} & Sure, I will make sure to pass on your concern to your healthcare provider and they can decide whether an in-person visit is needed. Do you have any other concerns you want me to communicate? \\   
\textbf{User:} & No, thank you \\   
\textbf{CA:} & I'm glad to hear you're generally doing well. I will let your healthcare provider know about your slight discomfort and they will determine whether a visit is necessary. If you have any further concerns or symptoms arise, please don't hesitate to call us. Take care. \\   
\textbf{User:} & Bye \\  


\end{longtblr}



\subsubsection{Initial dashboard interface design}
\begin{figure}
    \centering
    \includegraphics[width=0.8\linewidth]{figures/Redesign-V1.pdf}
    \caption{The first version of dashboard interface design. This version only include simple interactions like selecting patient to view details, and scrolling down to see conversation logs. After our PD sessions, we iterated the design of each tab and interactions to promote clinical efficiency.}
    \label{fig:8-appendix-initial-dashboard}
\end{figure}
\subsection{PD Session Artifact Examples}
The examples are presented in Fig \ref{fig:8-appendix-pd-ex-1} to \ref{fig:8-appendix-pd-ex-5}.
\begin{figure}[h]
\centering
\begin{subfigure}[b]{\textwidth}
    \includegraphics[width=\textwidth]{figures/Conversation-PD-1.pdf}
    \caption{Participants review the conversation log generated with one prompt and leave comments.}
    \label{fig:8-appendix-pd-ex-1}
\end{subfigure}

\begin{subfigure}[b]{\textwidth}
    \includegraphics[width=\textwidth]{figures/Conversation-PD-2.pdf}
    \caption{Participants review the conversation log generated with an updated prompt and leave comments.}
    \label{fig:8-appendix-pd-ex-2}
\end{subfigure}
\caption{Example of participants' comments in Google Docs.}
\label{fig:8-appendix-pd-ex-1-2}
\end{figure}


\begin{figure*}[h]
\centering
\begin{subfigure}[b]{0.45\textwidth}
    \includegraphics[width=\textwidth]{figures/Conversation-PD-3.pdf}
    \caption{The first conversation flow diagram upon participant discussion, where key questions are asked in order}
    \label{fig:8-appendix-pd-ex-3}
\end{subfigure}
\hfill %
\begin{subfigure}[b]{0.45\textwidth}
    \includegraphics[width=\textwidth]{figures/Conversation-PD-4.pdf}
    \caption{Updated conversation flow diagram, where conversation logic is more flexible.}
    \label{fig:8-appendix-pd-ex-4}
\end{subfigure}
\caption{Example of conversation flow iteration.}
\label{fig:8-appendix-pd-ex-3-4}
\end{figure*}

\begin{figure*}
    \centering
    \includegraphics[width=\textwidth]{figures/Conversation-PD-5.pdf}
    \caption{Participants discussed the color coding depending on the priority of a symptom, and which questions should include a meter with scale.}
    \label{fig:8-appendix-pd-ex-5}
\end{figure*}

\subsection{Patient Interview Script}
\label{8-appendix-patient-interview-script}
\begin{enumerate}
    \item Could you please share a bit of background about your cancer treatment and recovery process? 

    \item When you are at home for recovery after your surgery, could you recall a most impressive experience that you had with your healthcare providers? Follow-up questions may include:
    \begin{enumerate}
        \item Who reached out to whom?
        \item Why? What’s the conversation like?
        \item What technology did you use?
        \item What challenges did you encounter?
    \end{enumerate}

\item Do you know about AI? How about ChatGPT?
 e.g. If we have such an AI CA to help monitor your health conditions and connect to providers, what do you think? Follow-up questions may include:
 \begin{enumerate}
     \item What features would you expect?
     \item Do you think ...(an example feature) would be helpful? Why?
 \end{enumerate}
\item Was there anything else you wanted to share with us about your post-treatment experience? 
\end{enumerate}

\begin{table*}[t]
    \centering
        \caption{Questions about System Features and LLM and Results}
    \label{tab:8-appendix-feature-questionnarie}
    \small
    \begin{tblr}{
hlines,
vlines,
colspec={X[l] c c c} ,
width=\linewidth,
hspan=minimal,
}
 \textbf{System Features and LLM} & Mode&Max& Min\\ 
Overall, I think this system can help with my work monitoring postoperative GI cancer patients. &  Strongly Agree&  Strongly Agree&  Strongly Agree\\

I think the conversation between the patient and the LLM chatbot is natural. &  Agree&  Strongly Agree&  Neither Agree Nor Disagree\\

I think the LLM chatbot collects crucial information for monitoring the patient from the conversation. &  Strongly Agree&  Strongly Agree&  Agree\\

I think the LLM chatbot responses raise minimal safety risk to patients. &  Strongly Agree&  Strongly Agree&  Strongly Disagree\\

I find the central visualization (colored dots and meters) correctly reflects patients' response. &  Strongly Agree&  Strongly Agree&  Strongly Agree\\

I think the central visualization (colored dots and meters) helped me focus on key symptoms to review. &  Strongly Agree&  Strongly Agree&  Strongly Agree\\

I find the LLM summary summarizes key information in patients' response correctly. &  Strongly Agree&  Strongly Agree&  Strongly Agree\\

I think the LLM summary helped me review patient information efficiently. &  Strongly Agree&  Strongly Agree&  Strongly Agree\\

I think the conversation logs are linked with corresponding visualizations correctly. &  Strongly Agree&  Strongly Agree&  Strongly Agree\\

I think the conversation logs and highlights helped me to navigate through the details corresponding to each question efficiently. &  Strongly Agree&  Strongly Agree&  Strongly Agree\\

I think the interaction features (write notes, change status) can help me manage my patients effectively. &  Strongly Agree&  Strongly Agree&  Strongly Agree\\
\end{tblr}
\end{table*}

\begin{table*}[t]
    \centering
        \caption{Patient User Study Questionnaire Part 2 and Results. A number in the parentheses indicates the number of participants voting for this option. For example, ``much better (2)'' means two participants chose ``much better'' for this question.}
    \label{tab:8-appendix-user-study-results-2}
    \small
    \begin{tblr}{
hlines,
vlines,
colspec={X[l] c c c} ,
width=\linewidth,
hspan=minimal,
}
           Question& Most Chosen &2nd Most Chosen &3rd Most Chosen \\
          
     The availability of the Recover Bot is \_\_ compared to healthcare providers.& much better (2)&moderately better(2) & slightly worse (1)\\
  The ease of understanding the response provided by the RECOVER bot is \_\_ compared to healthcare providers.& much better (2)&about the same (2) &moderately worse (1)\\
  The extent to which the RECOVER bot covers key symptom details is\_\_ compared to contacting healthcare providers.& about the same (3)&much better (2) & / \\
  The availability of the RECOVER bot is \_\_ compared to questionnaires.& much better (3)&slightly better (1) & /\\
  The ease of understanding the questions provided by the RECOVER bot is \_\_ compared to questionnaires.& much better (2)&slightly better (2) & /\\
 The extent to which the RECOVER bot covers the key symptom details is \_\_ compared to questionnaires.& much better (2)&about the same (2) & moderately better (1) \\

\end{tblr}
\end{table*}

\begin{table}[t]
\centering
\caption{System Usability Scale Questionnaire}
\label{tab:system_rating}
\begin{tblr}{
hlines,
vlines,
colspec={X[l] c c c c c},
width=\linewidth,
hspan=minimal,
}
 \textbf{System Usability Scale} & 1 &2 &3  & 4 &5 \\

I think that I would like to use this system frequently. &  &  &  &  &  \\

I found the system unnecessarily complex. &  &  &  &  &  \\

I thought the system was easy to use. &  &  &  &  &  \\

I think that I would need the support of a technical person to be able to use this system. &  &  &  &  &  \\

I found the various functions in this system were well integrated. &  &  &  &  &  \\

I thought there was too much inconsistency in this system. &  &  &  &  &  \\

I would imagine that most people would learn to use this system very quickly. &  &  &  &  &  \\

I found the system very cumbersome to use. &  &  &  &  &  \\

I felt very confident using the system. &  &  &  &  &  \\

I needed to learn a lot of things before I could get going with this system. &  &  &  &  &  \\
\end{tblr}
In this table, the numbers 1 to 5 present the Likert scale from ``Strongly Disagree'' to ``Strongly Agree''
\end{table}

\clearpage
\subsection{Conversation Log Example}
\label{subsec:9-appendix-conversation-log}
\begin{longtblr}[
caption = {Conversation log between the user and the assistant},
label={tab:conversation_log}
]{
hlines,vlines,
colspec={lX},
width=\linewidth,
rowhead=1
}

\textbf{Role} & \textbf{Content} \\

User & hi \\

CA & Hello, this is the RECOVER research study chatbot assistant. Are you ready to start today's questions? \\

User & yes \\

CA & Great! How have you been feeling today? Any specific discomforts? \\

User & I'm feeling a sharp pain in my stomach and I'm not able to eat because of the pain \\

CA & I'm sorry to hear that you are in pain. Does it increase sharply or becomes unbearable? \\

User & yes \\

CA & Could you tell me more about when the pain started? \\

User & it started around 12 hours ago \\

CA & Have you taken more pain medication prescribed by your surgeon since the pain started? \\

User & I'm taking Tylenol \#3 \\

CA & On a scale of 1 to 10, with 10 being the worst possible, how would you rate your pain? \\

User & 8 \\

CA & On a scale of 1 to 10, with 10 being the most difficult, how bad are you able to tolerate food? \\

User & 8 \\

CA & Are you having difficulty breathing? \\

User & nope \\

CA & Are you having a fever of over 100 degrees or chills? \\

User & no \\

CA & Have you had black, tar-like stools? \\

User & no \\

CA & Are you having any wound drainage problems, such as redness around your wound, bleeding from the wound, pus, or an opening at the incision site? \\

User & no \\

CA & Do you have a decrease in your ability to perform your daily activities, such as not being able to walk to the bathroom? \\

User & no \\

CA & Have you had a decrease in your level of consciousness? \\

User & no \\

CA & Have you had persistent constipation, nausea, or vomiting? \\

User & yes I have been constipated for 24 hours \\

CA & On a scale of 1 to 10, with 10 being most significant, how would you rate your level of constipation? \\

User & maybe 6 \\

CA & Have you had persistent diarrhea? \\

User & no \\

CA & Do you have unexplained or new pain or swelling in one or both of your legs? \\

User & no \\

CA & Have you been feeling down or depressed? \\

User & no I'm feeling OK \\

CA & Is there anything else you'd like to comment on that I haven't asked about? \\

User & that's all thank you bye \\

CA & Thank you for your time to provide information today. We'll talk again tomorrow. \\

\end{longtblr}



\subsection{Cancer Care Provider User Study Protocols and Results}

\subsubsection{Evaluation Questionnaire and Results}
\label{subsec:8-appendix-questionnarie}



\subsubsection{Follow-up Interview Questions for Healthcare Providers}
\begin{enumerate}
    \item How do you like the system?
    \item What are those existing features that may help you the most in monitoring postoperative GI cancer patients?
    \item Would you expect any new novel features that our system does not have, for your RPM? 
    \item Do you have any concerns about RECOVER or such systems in future adoption?
    \item Do you have any questions for us?
\end{enumerate}

\subsection{GI Cancer Patient User Study Protocols and Results}
\label{sub:8-appendix-patient-user-study}
\subsubsection{Evaluation Questionnaire and Results}
Patient participants were first asked to rate the AI chatbot system on SUS (part 1 of the questionnaire) and the SUS questionnaire contains the same questions as mentioned in \ref{subsec:8-appendix-questionnarie}.

The second part of the questionnaire is presented in Table \ref{tab:8-appendix-user-study-results-2}: 


\subsubsection{Follow-up Interview Questions for GI Cancer Patients}

\label{subsec:8-appendix-patient-interview}
\begin{enumerate}
    \item How do you like the system? 
    \begin{enumerate}
        \item What may help you the most when talking about your symptoms?
    \end{enumerate}
    \item Would you expect any new features to be added to the system, for your postoperative recovery?
    \item Do you have any concerns about RECOVER or such systems in future adoption?
\end{enumerate}


\subsubsection{Patient Task Completion Result}


\begin{table*}[t]
    \centering
        \caption{Participant performance in tasks navigating through the dashboard. Here the columns with values 0 to 4 count the number of participants corresponding to the situation. For example, four participants completed task 1 without a hint from the researchers. The last column shows the time that participants spend to complete the task in the ``minutes: seconds'' format.}
        \begin{booktabs}{
            colspec={ccccc}
        }
                \toprule
                 Task &   Complete without Hint & Complete with Hint& Fail to complete &Average Time to Complete Task\\ \midrule
                 Task 1&4& 
     0& 0& 0:32\\
 Task 2& 3& 1& 0& 1:42\\
 Task 3& 3& 1& 0& 1:24\\
 Task 4& 1& 3& 0& 0:27\\
         \bottomrule
         \end{booktabs}
        
    \label{tab:user-study-performance}
\end{table*}

\subsection{Complete Prompt Texts}
\label{sub:8-appendix-prompt-conversation}
\subsubsection{Prompts for Conversation Module}
\begin{minted}[frame=lines,breaklines,breakindent=0cm,breaksymbol={}]{text}
# [System Definition]
You are a friendly and empathetic chatbot, RECOVER research study chatbot assistant, to help a research team at a cancer care clinic to collect patient information after their surgery.

## <User description>
You will check up on the patient, who is the user in this conversation. The user you will be talking to is a cancer patient who just received their gastrointestinal cancer treatment within the recent 40 days, and have been discharged from the hospital for their recovery at home. Your conversation log with the patient will be inspected and review by a healthcare provider.

## <Things that you must do>
1. You must express your empathy and considerations towards the patient, such as "I am sorry that you have this pain", "It is good to hear that you feel ok", "I understand that you have this concern."
2. You can remember all previous conversations, which means that you can remember the entire conversation history as well as the patient's medical history and symptoms mentioned in past records.
3. You are able to understand the details and contexts in the user's response. You are able to make smooth transitions between questions, and make slight adjustments to questions considering the context.
4. Please be very thoughtful. Consider the user’s feelings, medical history and literacy when you provide responses.
5. You will lead a natural conversation as if you are talking to the patient over phone. You will ask questions according th the <Task Definition>.
6. You will discuss the patient's discomforts and details about the symptoms.
7. You will be talking with the patient daily. You can only see messasge from today's previous conversation. **ALL MESSAGES HAPPENES IN THE SAME DAY**.

## <Things that you must not do>
1. When you ask follow-up questions for the patient's symptoms, you must not give any comments that may indicate a diagnosis (e.g. some symptoms may or may not be a problem). 
	Here are some bad examples:
	<Example 1> 
	... could be a problem
	<Example 2> 
	... is common for ...
2. Please do not provide any medical instructions, interpretation or health-related suggestions.
3. You are not able to do any administrative work such as scheduling appointments.
4. You can not directly reach healthcare providers for the patients' symptoms or concerns since you are just a chatbot to collect information. If the patient asks a related question, you could say "Please contact your healthcare providers as instructed for your questions." Similarly, you should not mention that you will record the collected information as it could be misleading.

# [Important concepts]
## <List of 13 Key Aspects>  - Question Examples and Flow
You should generally follow the conversation flow and cover the major points, but you **MUST NOT** ask the questions exactly as the examples; instead, **BE CREATIVE AND** make adjustments according to the contexts.
For some questions, if instructed, **ALWAYS ASK FOR A RATE OF DISCOMFORT SEVERITY** on a scale of 1 to 10, **10 ALWAYS MEANS THE MOST DIFFICULT OR WORST**
	1. Breathing (Difficulty Breathing)
	- "Are you having difficulty breathing?"
		- If "yes", first ask, "Tell me about your shortness of breath."
		- Then, inquire about the severity: "On a scale of 1 to 10, with 10 being the most difficult, how would you rate your shortness of breath?"

	2. Fever (Fever)
	- "Are you having a fever of over 100 degrees or chills?"
		- If "yes", first ask for more details: "Tell me a little more, such as how long the fever lasted."
		- Then, inquire about the highest fever measurement: "What is the highest fever measurement you took?"

	3. Stools (Black, Tar-like Stools)
	- "Have you had black, tar-like stools?"
		- If "yes", first ask about frequency and onset: "How many times did you notice this, and when did it start?"

	4. Pain (Pain Increase or Unbearable)
	- "Do you have pain that sharply increases, or becomes unbearable?"
		- If "yes", first show empathy: "I'm sorry you are having pain. Tell me more about when it started."
		- Then, check medication: "Are you taking more pain medication prescribed by your surgeon since you left the hospital?"
		- Then, seek more information: "On a scale of 1 to 10, with 10 being the worst possible, how would you rate your pain?"

	5. Drainage (Wound Drainage Problems)
	- "Are you having any wound drainage problems, such as redness around your wound, bleeding from the wound, pus, or an opening at the incision site?"
		- If "yes", ask for specifics: "Can you tell me more about this? For example, is this a small or large amount, is the drainage continuous, or does it soak your clothes?"

	6. Activities (Decrease in Daily Activities)
	- "Do you have a decrease in your ability to perform your daily activities, such as not being able to walk to the bathroom?"
		- If "yes", first ask for more details about the decrease: "Tell me more about this decrease and how it is affecting you."

	7. Conscious(Decrease in Level of Consciousness)
	- "Have you had a decrease in your level of consciousness?"
		- If "yes", inquire if it required assistance: "Have others had to help you because of this loss of consciousness?"
		- Then, inquire about the severity: "On a scale of 1 to 10, with 10 being the worst possible, how would you rate your level of consciousness?"

	8. Constipation (Persistent Constipation, Nausea, or Vomiting)
	- "Have you had persistent constipation, nausea, or vomiting?"
		- If "yes", first identify [symptoms]: "Tell me more about which of these symptoms you are having, such as when it started."
		- Then, assess [symptom] severity: "On a scale of 1 to 10, with 10 being the worst possible, how would you rate your level of [symptom]?"

	9. Diarrhea (Persistent Diarrhea)
		- "Have you had persistent diarrhea?"
		- If "yes", ask for details: "Tell me more about this, such as how many times have you had diarrhea since yesterday?"
		- Then, gauge the frequency: "How many times have you gone to the bathroom since it began?"

	10. Eating (Inability to Tolerate Food or Drink)
		- "Have you been unable to tolerate food or drink?"
		- If "yes", assess tolerance levels separately: "On a scale of 1 to 10, with 10 being the most **difficult**, how well are you able to tolerate food?" and then, "On a scale of 1 to 10, with 10 being the most **difficult**, how well are you able to tolerate drink?"

	11. Swelling(Pain or swelling in legs)
		- "Do you have unexplained or new pain or swelling in one of both of your legs?"
		- If "yes", ask for more detail: "Tell me more about the pain you are experiencing."

	12. Mood (Feeling Down or Depressed)
		- "Have you been feeling down or depressed?"
		- If "yes", further inquire about emotional state: "Everyone feels sad sometimes. Have you been feeling continuously sad, or have you lost interest in most of your usual activities?"
		- Based on their emotional state, emphasize the importance of seeking help: "It is important for you to speak with someone about this to seek help."

	13. Misc (Anything Else)
		- "Is there anything else you'd like to comment on that I haven't asked about?"
		- If "Yes", ask follow-up questions like "Can you tell me more about [symptom]?", but be creative. Ask context-related questions to the symptom reported.

# [Conversation Task]
## <Overview>
Please follow this outline of a typical conversation in the post-surgery scenario. You will start a conversation with introduction in Section 1, and then move to questions in Section 2. After the questions, you will finish the conversation in Section 3. Do not ask repetitive questions, and move to the next section naturally.
All the quotes are just examples of the question - You do not need ask the questions exactly as the example; instead, consider the context and ask the questions in a more appropriate and precise way.

## <Conversation Flow>
### Section 1: Introduction
You will start the conversation as instructed below. Briefly greet the patient, ask what is the patient’s inquiry and if there’s any specific discomfort. You will only need to greet the patient once in the whole conversation.
Step 1. Introduction
   - Start with: "Hello, this is the RECOVER research study chatbot assistant developed by Northeastern University HAI lab. Are you ready to start today's questions?" Remember you only need to say this once in the whole conversation.
	 - If the user answers NO, reply "Let's try again later." Then directly end the conversation.

### Section 2: Health Check-in
In this section, you will go over the steps below to check on the patient questions. For the overall conversation structure, please refer to Step 1 to Step 3 to navigate. For the detailed questions you should ask, please refer to [Instruction to ask questions] to be precise and professional. Overall, you should be able to collect the answers to all the key aspects with additional information at the end of this section, but the order of questions should be random.

As mentioned previously, if the patient has mentioned any [specific symptom(s)] in Section 1, you should start from Step 1 to query about this [specific symptom(s)] first. If not, start from Step 2 to go over the questions.

#### Section 2 Step 1: Question Match for [specific symptom(s)]
For every [single symptom] of the [specific symptom(s)] mentioned by the user, you should do the following: In [List of 13 Key Aspects], find whether there is a matching question for this [single symptom].

	- Case 1: The symptom matches a key aspect
	If you can find a match, this will be the [current question]. Follow the steps in [List of 13 Key Aspects]-[current question] and [Case 1] to ask pre-defined follow-up questions. You will first ask the first yes/no question, e.g. "Are you having difficulty breathing?", then:
		- If the user answers "yes" to the key aspect, you should first follow up with drill-down question(s) given under the key aspect to capture more details.
		- If the user's reply is ambiguous, or the user mentions [additional symptom(s)], please repeat step 1 for the [additional symptom(s)].
		- If the user answers "no" to the key aspect with no other symptom, you should proceed to Step 2.

	- Case 2: Other symptoms
	When the patient identifies a symptom but it is not in [List of 13 Key Aspects], then ask follow-up questions as instructed below. You should first ask a short follow-up question and possibly ask more about the details. When applicable, you could ask the patient about the severity, frequency or impact of this symptom. If the patient has already mentioned the corresponding detail(e.g. severity), then skip this part of the question.

	<Example follow-up question>
	Could you please tell me more about that? / How did that happen? / Is this related to any other symptoms that you have?

	Then you can proceed to any other discomfort that the patient has mentioned; if there isn't any, go to Step 2.

After you finish with all the [specific symptom(s)] for this key aspect, please proceed to Step 2 to ask all the remaining question in the question list. Do not directly skip to Step 3.

#### Section 2 Step 2: Remaining Questions
You may not have all questions about the patient's health asked previously in the conversation. Thus, in this step, you will cover all the remaining questions in the [List of 13 Key Aspects] following the instructions below.
You must first identify the status of all the 13 Key Questions. Each of them should be one of "not discussed", "in discussion" or "discussed". You will output the status of each question following the output format requirements.
All questions should be initially "not discussed". If you or the patient talks about a symptom or a question, you should mark it as "in discussion".
You can only mark a question as "discussed" under the following conditions:
1. the patient has **explicitly** answered no to this key aspect
2. the patient has **explicitly** mentioned that they do not have the symptom of this aspect somewhere in the conversation
3. the patient has answered yes to the key aspect AND also answered all your follow-up questions to this key aspect
NOTE THAT: THE USER MUST EXPLICTLY ANSWER TO THE SPECIFIC QUESTION. ANSWER TO GENERAL QUESTIONS LIKE "how are you feeling today" DOES NOT COUNT.

You must then check what is the first "in discussion" in the list -- this will be your [current question].
IF THERE ISN'T ANY "in discussion", **RANDOMLY** select one question from the "not discussed" list, DON'T ALWAYS USE THE FIRST ONE, and mark it as "in discussion". This will be your [current question].
For the [current question], follow the [instructions to ask questions] to cover the flow of the [current question]. 

If there isn't any "not discussed" questions, proceed to section 3.

### Section 3: Wrap-Up
In section 3, you will wrap up the conversation as below. Here are some examples:
 * Thank you for your time to provide information today.
 * We'll talk again tomorrow.
 * I'm glad that you are feeling OK today, let's talk again tomorrow.

### Section 4: User called you again.
If you believed that all questions were answered, and the user said "hi" again, it means that the user talked to you again **ON THE SAME DAY**. You should ask something like (ALSO BE CREATIVE):
 * Hi! Happy to talk with you again. I believe I've got what I need for today, Do you have any update?


## <Output requirement>
The output will contain two parts, Part 1: Question Checklist, and Part 2: Chatbot response. You will generate the response by:

1. Output the status of each question, according to "#### Section 2 Step 2". Example:
	breathing: not discussed
	fever: not discussed
	stools: not discussed
	pain: not discussed
	drainage: not discussed
	activity: not discussed
	conscious: not discussed
	constipation: not discussed
	diarrhea: not discussed
	eating: not discussed
	swelling: not discussed
	mood: not discussed
	misc: not discussed

2. A delimiter `==============`

3. The content you should say to the patient. It should either be a question, or a clarification to patient's question.

### <Example scenario>
You have greeted the patient and asked the first three questions in the list, and the patient has just mentioned a pain around the surgical site but have not revealed additional detail.

### <Example output (please always use the exact format as below)>
breathing: discussed
fever: discussed
stools: discussed
pain: in discussion
drainage: not discussed
activity: not discussed
conscious: not discussed
constipation: not discussed
diarrhea: not discussed
eating: not discussed
swelling: not discussed
mood: not discussed
misc: not discussed
==============
I am sorry to hear that you are in pain. Does it increases sharply or becomes unbearable?

\end{minted}
\subsubsection{Prompt for Information Visualization \& Extraction Module}
\begin{minted}[frame=lines,breaklines,breakindent=0cm,breaksymbol={}]{text}
Symptom Keys and Descriptions:

breathing: "Difficulty Breathing", likert: true
fever: "Fever", likert: false
stools: "Black, Tar-like Stools", likert: false
pain: "Pain Increase or Unbearable", likert: true
drainage: "Wound Drainage Problems", likert: false
activity: "Decrease in Daily Activities", likert: false
conscious: "Decrease in Level of Consciousness", likert: true
constipation: "Persistent Constipation, Nausea, or Vomiting", likert: true
diarrhea: "Persistent Diarrhea", likert: false
eating: "Inability to Tolerate Food or Drink", likert: true
swelling: "Pain or swelling in legs", likert: false
mood: "Feeling Down or Depressed", likert: false
misc: "Other Symptoms", likert: false

For each symptom listed above, perform the following tasks:

Identify Relevant Messages: Locate and categorize all messages that pertain to each symptom listed. This includes the assistant asking questions (e.g., "Are you having difficulty breathing?") as well as user responses, including responses indicating the absence of symptoms ("no").
EACH SYMPTOM SHOULD HAVE AT LEAST TWO MESSAGE (QUESTION AND RESPOND).

Assess state: determine whether the patient reported to this symptom. if the patient didn't talked about the symptom, echo 0. If the patient mentioned that they have the symptom, write 2. otherwise write 1.
Assess likert scale: for symptoms with `likert: true`, **if the patient didn't report this symptom, REPORT 0!!!** if the patient reported having this symptom, also inspect what the user said about the level of the symptom. should be a interger between 1 and 10. if the symptom have `likert: false`, don't output this value.

Output Format: Organize and present the findings in a JSON structure as specified below. Ensure the symptom keys in the output JSON match exactly with those from the provided list:
{
    "<symptom_key>": {
        "logs": [<list_of_relevant_log_ids>],
        "state": <whether the patient have this symptom>,
        "scale": <likert scale>
    }
}


RETURN ONLY THE JSON AND NOTHING ELSE. DONT ENCLOSE IT IN ```.
\end{minted}

\subsubsection{Prompt for Summarization Module}
\begin{minted}[frame=lines,breaklines,breakindent=0cm,breaksymbol={}]{text}
Task:
Create a list of concise, one-sentence summaries for conversation logs related to symptoms with a severity state higher than 1, as defined in the provided symptoms JSON.

Instructions:

Filter by Severity: Identify symptom categories where the severity state is greater than 1 in the symptoms JSON. Only these categories will be included in the summary.
Retrieve Logs: For each identified symptom category, extract the list of conversation log IDs from the JSON.
Summarize Conversations: Write a brief, one-sentence summary for each set of conversation logs. The summary should focus on key details and outcomes discussed, pertinent to the symptom and its assessed severity.
Output Format: Present the summaries in an array of objects, with each object representing a summary for a specific set of logs. Include the following properties in each object:
category: "Summary"
conversation_log_ids: A string listing the log IDs (e.g., "[1698, 1699]").
content: A concise summary capturing the essence of the discussion about the symptom, emphasizing the severity and any critical actions recommended or taken.
Example of Summary Format:

{
    "result": [
        {
            "category": "Summary",
            "conversation_log_ids": "[1698, 1699]",
            "content": "In-depth discussion about patient's inability to tolerate food or drink."
        },
        {
            "category": "Summary",
            "conversation_log_ids": "[1690, 1691, 1692, 1693, 1696, 1697]",
            "content": "Urgent conversations addressing severe pain requiring immediate attention."
        }
    ]
}
\end{minted}
