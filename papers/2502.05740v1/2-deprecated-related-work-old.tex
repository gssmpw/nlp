\section{Related Work}
\label{sec:2-related_work}
\review{Here we first review in Section \ref{sub:related_work-1} the challenges GI cancer providers face monitoring patients' postoperative recovery. Then in Section \ref{sub:related_work-2}, we discuss the advantages and limitations of current RPM or ubiquitous health technologies, which call for better design and clinical integration in postoperative cancer care.}
\review{Lastly, we reveal opportunities for LLMs to support RPM in our domain-specific clinical context in Section \ref{sub:related_work-3}.}


\subsection{Postoperative Health Risks and Challenges in GI Cancer Care}
\label{sub:related_work-1}
Gastrointestinal (GI) cancer causes around 3.4 million deaths per year worldwide, and the 5-year survival rates are below 30\% in many cases after their surgery~\cite{arnoldGlobalBurdenMajor2020, mocanSurgicalManagementGastric2021,wongGlobalIncidenceMortality2021}. 
\review{Monitoring postoperative GI cancer patients is challenging for healthcare providers due to the wide range of potential health issues patients may face. Severe complications such as anastomotic leakage and sepsis~\cite{vankootenPreoperativeRiskFactors2021, yasunagaBodyMassIndex2013} are common, while changes in the GI system can further disrupt patients' nutritional status and quality of life (QOL), potentially impacting their long-term survival~\cite{carmichaelEarlyPostoperativeFeeding2022, garthNutritionalStatusNutrition2010, yasunagaBodyMassIndex2013}.}
Research work in surgical oncology has investigated preoperative factors such as lifestyle data to identify high-risk patients~\cite{vankootenPreoperativeRiskFactors2021}. 
However, many of the postoperative complications remain urgent and unpredictable~\cite{mocanSurgicalManagementGastric2021, vankootenPreoperativeRiskFactors2021}. 

\review{Although close monitoring of patient health conditions may help address these unpredictable conditions through early interventions, effective patient monitoring is not easy. On the one hand, 
many GI cancer patients have limited health literacy, and thus lack the knowledge for effective self-monitoring and further clinical decisions~\cite{cardellaComplianceAttitudesBarriers2008}.}
\review{On the other hand, frequently assessing and evaluating patient conditions can be burdensome for GI cancer providers, often leading to communication overload or disruptions in their workflow~\cite{leon2022impact}.}
Although in-person visits could be efficient in communicating patient symptoms between providers and patients, they require extra time and effort for scheduling and travel, which may potentially result in health issues and readmissions to vulnerable postoperative GI cancer patients~\cite{silva_ostomy_2020}. 
Thus, remote monitoring solutions that effectively address postoperative GI cancer providers' unique challenges, would greatly improve the surgical outcomes and clinical work experience.

Aiming for better postoperative outcomes for GI cancer patients, clinical professionals have designed guidelines to monitor patient conditions. For instance, ... \ziqi{add some related work; what people have already done}
However, the execution of the clinical guidelines highly dependent on the clinical professional, the training process or institutional factors. Thus, it is unclear how these clinical guidelines could be integrated to be useful in remote patient monitoring or similar technologies to support clinical work.

\subsection{Patient Monitoring Systems for Postoperative and Cancer Care}
\label{sub:related_work-2}
Remote patient monitoring (RPM) systems have been employed by healthcare institutions to monitor certain aspects of a patient's health from their own home~\cite{malasingheRemotePatientMonitoring2019}. 
Particularly, GI cancer care providers have adopted traditional RPM methods such as phone calls, patient portals, or rigorous questionnaires to save time and travel efforts for both patients and providers~\cite{malasingheRemotePatientMonitoring2019}.
\review{However, these questionnaires are frequently lengthy and complex, leading to patients' ``survey fatigue'', which significantly diminishes the quality and quantity of information gathered~\cite{li2024beyond}. Alternative methods, such as messaging or phone calls, are typically one-way (from patients to providers) and often lack clear instructions~\cite{chandwani_stitching_2018}. Consequently, cancer care providers continue to face a shortage of essential information for effective interventions and decision-making~\cite{yangWishThereWere2024}.}
\review{Additionally, \citet{leon2022impact} pointed out issues in telehealth systems for clinical staff including nurses and specialists, including (1) additional workload such as documentation efforts, (2) disruption to the workflow, and (3) false alarms and unclear data.} 
In other cases, redundant information causes cancer care providers' information overload beyond their clinical duties, and thus necessary interventions could be delayed~\cite{clark_understanding_2021, bhat_infrastructuring_2021, yangWishThereWere2024},.
Although recent research has pointed out some key information that cancer providers look for in RPM~\cite{yangWishThereWere2024}, 
how systems for RPM could integrate such guidelines to collect key information from GI cancer patients remains an unanswered question.

\review{Recent advancements in ubiquitous systems for healthcare, could potentially help to collect specific patient health information.}
\ziqi{add more connections to ubicomp and latest research}
For instance, mobile health (mHealth) applications, sensors, wearable systems, and personal coaches have been used to track patient stress levels and daily activities~\cite{goncalves-bradley_mobile_2020,segoviaSmartphonebasedEcologicalMomentary2020, kingMicroStressEMAPassive2019, jacobsMyPathInvestigatingBreast2018}. 
In particular, some research in clinical settings has designed and tested mHealth apps that effectively integrate questionnaires or wearable devices to track patients' postoperative conditions~\cite{sempleUsingMobileApp2015}.
However, these solutions, mostly collecting structured and quantitative data, still fall short in guiding patients to report domain-specific symptoms, and lack flexibility and adaptivity regarding the various postoperative conditions of postoperative GI cancer patients~\cite{yangWishThereWere2024}. In addition, the provider interfaces mostly present unprocessed raw data, further overloading busy healthcare providers~\cite{yangWishThereWere2024, sempleUsingMobileApp2015}. A co-design study for immuntherapy pointed out: patient-reported symptoms are supportive for monitoring, reveals a strong need for such system to have content and functionality to correspond to related side-effects and improve clinician interpretability and usability, automation of RPM tasks. ~\cite{lai2024co}. More work has called for automation.
Given the limitations of current systems for RPM and the urgent health risks that postoperative GI cancer patients face, we are motivated to explore novel systems for RPM that help patients report key domain-specific information, and effectively present those patient health information for providers' review.

\subsection{Large Language Models for Patient Health Monitoring}
\label{sub:related_work-3}
The recent technological boost of Large Language Models (LLMs), such as GPT~\cite{OpenAI_2022}, offers a promising opportunity for system designers to picture a more clinically compliant solution with LLMs' great potential in engaging in and scaffolding natural conversations~\cite{yunxiang2023chatdoctor,xiao2020tell, xiao2023supporting,shen2023convxai, hamalainen_evaluating_2023, wang_2023_enabling, liu2023human}. 
Researchers have explored LLM-powered CAs for remotely collecting patient health data in areas such as public health interventions, chronic disease management, and pre-consultation screening~\cite{wei_leveraging_2023, montagna2023data, dwyer2023use, ni2017mandy, li2024beyond, hao2024advancing}. For example, \cite{chira2022affective}  designed a multi-modal CA to collect health data from patients with brain diseases, where the CA asks general check-in questions such as ``how are you doing today'' and ``how can I help with you today''. 
However, most work focuses on patient experiences or data collection, overlooking the information needs of clinical staff especially in high-risk cases like postoperative GI cancer care. These existing CAs may lack clinical adherence and thus offer limited information for intervention or decision-making in domain-specific conditions. 
Some other LLMs leverages QA datasets or benchmarks to help explain or analyze patient monitoring data (e.g. glucose from wearable sensors) ~\cite{ferrara2024large, healey2024llm}. Another example: multi-modal LLM to automate patient health monitoring ~\cite{ho2024remoni}.
There is little research on how LLM-powered telehealth systems could integrate specific clinical guidelines and system-related information needs in critical clinical settings like postoperative GI cancer care.

\subsection{Large Language Models for Clinical Work}
Meanwhile, language models have also shown their outstanding capability in processing medical domain knowledge, especially in pre-trained models like MedPaLM\cite{singhalLargeLanguageModels2023}, UmlsBERT\cite{michalopoulosUmlsBERTClinicalDomain2021}, and BioBERT~\cite{leeBioBERTPretrainedBiomedical2019}. For healthcare professionals, researchers have also leveraged LLMs for clinical pre-screening~\cite{hamer_improving_2023, wang_brilliant_2021}, risk prediction ~\cite{o2015risk,gatt2022review, kennedy2014delirium, beede2020human} and information processing~\cite{kocaballi_envisioning_2020, nair_generating_2023,  cascella2023evaluating, nori2023capabilities, agrawal_large_2022, arbatti_what_2023}. 
\review{In 2023, researchers integrated decision trees from medical literature for LLM-supported clinical decision-making~\cite{li2023meddm}, but little work has explored how to design LLMs' clinical integration for RPM or similar postoperative scenarios.}
Although some research reveals that LLM reponses could align with clinical guidelines in general evaluations, they may not adhere to domain-specific clinical work, such as those in need of evidence-based recommendations~\cite{nwachukwu2024currently}.

potential LLM to optimize clinical workflow~\cite{tripathi2024efficient}.

Closer to our research, LLM-powered interfaces such as digital dashboards have been designed for providers to process patient conditions in daily healthcare or emergency decision-making~\cite{yangWishThereWere2024, zhangRethinkingHumanAICollaboration2024}. However, these designs may not suit RPM for postoperative GI cancer, where both domain-specific provider instructions, clinical specifications, and timely qualitative data from patients are essential. Some clinical studies suggest using machine learning (ML) models to identify high-risk postoperative GI cancer patients~\cite{vankootenPreoperativeRiskFactors2021, chenDevelopmentValidationMachine2022a}, but such models have yet to be tested or deployed in real-world settings. Therefore, we aim to explore how clinical needs and guidelines can be integrated with LLMs to develop an LLM-powered telehealth system for postoperative GI cancer care and similar high-risk scenarios.
