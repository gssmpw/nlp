\section{Introduction}


Gastrointestinal (GI) cancer refers to malignancies affecting the digestive tract  (e.g. stomach, liver, esophagus), accounting for more than 35\% of all cancer-related deaths.
Although surgical removal of affected organs or tissues is a primary and effective treatment,  postoperative GI cancer survivors remain at high risk of life-threatening complications, such as sepsis and surgical leakage~\cite{brenkman2016worldwide, cousins2016surgery}.
Given the unpredictability of postoperative complications due to varying patient recovery trajectories~\cite{vankootenPreoperativeRiskFactors2021}, close \textit{patient monitoring} is crucial for early detection and timely intervention to prevent medical emergencies, hospital readmissions, and fatalities~\cite{sempleUsingMobileApp2015, cardellaComplianceAttitudesBarriers2008}.




Traditional in-person hospital visits for postoperative follow-up stack scheduling delays, transportation burdens, and infection risks to GI cancer survivors; thus, the \textbf{remote patient monitoring (RPM)} paradigm has become particularly valuable by enabling monitoring of patients' health situation at home~\cite{malasingheRemotePatientMonitoring2019}. Yet commonly adopted RPM methods are not clinically compliant for postoperative GI cancer care.
Although phone calls by nurse practitioners or tele-questionnaires offer guidelines for patients to report symptoms~\cite{malasingheRemotePatientMonitoring2019, silva_ostomy_2020}, they either require high clinical dedication or brings confusion and burden to patients such as the ``survey fatigue''~\cite{pannunzio2024patient, leon2022impact, pannunzio2024patient, li2024beyond}.
Despite the flexibility and convenience of mHealth platforms, patient messages through patient portals leads to fragmentation and disruption of clinical staff's busy workflow~\cite{pannunzio2024patient, leon2022impact}; design and usability issues of clinical user interfaces have made it difficult to navigate patient data, extract actionable insights, and make clinical decisions~\cite{zhangRethinkingHumanAICollaboration2024}. These limitations, thus, hinder timely decisions and interventions, and increase clinician burnout~\cite{cardellaComplianceAttitudesBarriers2008, yangWishThereWere2024}. There is an urgent need for innovative RPM solutions that \textit{align with providers' clinical workflows and information needs} to improve healthcare outcomes and enhance efficiency. 

Recent advances in \textit{ubiquitous computing technologies} and \textit{large language models (LLMs)} offer promising opportunities in collecting patient health information in rich multimodal formats and supporting clinical work.
Research has used wearable devices like Fitbit to track cancer patients' physical activities or time patterns remotely ~\cite{FitbitWeartimePatterns, dreherFitbitUsagePatients2019, ghods2021remote}; interactive systems like conversational agents (CAs), some powered by LLMs, are designed to check patients' daily health states~\cite{yang2023talk2care,jo_understanding_2023, omarovArtificialIntelligenceEnabled2022, dwyer2023use, li2024beyond, hao2024advancing}; yet
the focus on general health information or physiological metrics does not suit the specific clinical needs in RPM of GI cancer survivors~\cite{yamagataCurrentStatusEnhanced2019, sempleUsingMobileApp2015}.
Meanwhile, advanced LLMs exhibit strong natural language processing capabilities and excel in structured tasks~\cite{brownLanguageModelsAre2020, openaiGPT4TechnicalReport2023, touvronLlamaOpenFoundation2023}. Research has shown their effectiveness in medical text summarization and question-answering, demonstrating their ability to integrate domain-specific knowledge and process health-related information~\cite{agrawalLargeLanguageModels2022, liuLargeLanguageModels2023, luContextualEmbeddingModel2022}.

However, current LLM-based CAs primarily handle general or administrative inquiries, which fall short in critical contexts like postoperative GI cancer care~\cite{ni2017mandy,geoghegan2021automated}; general-purpose LLMs lack domain-specific instructions, such as identifying critical symptoms, providing clinically responsible responses for patients, and clinical guidelines and procedures. 
To design LLM-powered systems for RPM with clinical efficiency and better patient health outcomes, we need further understanding of the needs and expectations from clinical staff and patients through \textit{close engagement of these stakeholders}.
Therefore, we aim to bridge the gaps in RPM technologies by investigating: What are the expectations and requirements of clinical staff and patients for an LLM-powered telehealth system in postoperative GI cancer RPM?  How should we design and implement a system for RPM that complies with clinical guidelines and information needs leveraging LLMs?


To answer these research questions, we first conducted a participatory design study with five clinical staff in GI cancer care and five postoperative GI cancer patients. Through seven participatory design sessions and five interviews, we collected their expectations for such LLM-powered RPM systems and iterated design artifacts. The process results in a key question and priority table for information collection, considerations for responsible LLM, and LLM-powered visualization and interaction examples.
Based on these findings, we summarized six key design strategies for leveraging LLMs to build a RPM system that is clinically compliant, efficient, and responsible.

Following the design strategies, we designed and developed \projectname{}, a system for \underline{\textbf{re}}mote symptom \underline{\textbf{co}}llection to impro\underline{\textbf{ve}} postope\underline{\textbf{r}}ative care. 
\projectname{}'s two interfaces aim to: (1) integrate clinically critical guidelines and flexible conversational protocols into an LLM-powered CA, enabling it to collect patient information corresponding to key symptoms, and (2) offer symptom risk-based visualization, along with intelligent summaries, highlights, and interactions, so that clinical staff can quickly identify critical issues and respond effectively. 
To assess our design guidelines and user perspectives towards such LLM-powered RPM systems, we used \projectname{} as a pilot system to conduct user studies with four GI cancer care clinical staff and five GI cancer patients. Participants engaged in clinical scenarios using our dashboard and conversational interface. The results highlighted the system's usability and support for efficiency, while also providing qualitative insights into how LLM-powered features support the RPM process and inform future design of responsible and deployable systems.

Finally, we present key implications for designing LLM-powered systems for RPM in domain-specific scenarios. We summarize essential design components, their potential long-term impact on patient care, and opportunities to engage other ubiquitous technology and clinical resources for future development. Additionally, we discuss considerations for implementing responsible AI in LLM-powered RPM systems, focusing on ethical, privacy, and security aspects.
This work presents three major contributions:
\begin{itemize}
    \item Through understanding of stakeholder needs, we summarize six design strategies to leverage LLMs to integrate clinical guidelines and information needs into \textbf{RPM systems}
    \item We present \textbf{RECOVER}, an LLM-powered system for \textbf{RPM} of postoperative GI cancer that is clinically efficient and compliant, with key design artifacts.
    \item Based on our user study findings, we provide design implications and future opportunities for clinically compliant, efficient, and responsible LLM-powered systems for RPM.
\end{itemize}
