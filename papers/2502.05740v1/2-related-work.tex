\section{Related Work}
\label{sec:2-related_work}
Here we first review in Section \ref{sub:related_work-1}  the challenges GI cancer providers face monitoring patients' postoperative recovery and traditional RPM practices. Second, in Section \ref{sub:related_work-2}, we discuss recent advancements and limitations of ubiquitous health technologies for RPM, some of which powered by LLMs, and the need for clinical integration in postoperative cancer care.
Lastly, we review recent advancements in LLMs that may potentially support the RPM work of staff in our domain-specific clinical context in Section \ref{sub:related_work-3}.

\subsection{Postoperative GI Cancer Care Challenges and Practices}
\label{sub:related_work-1}
Gastrointestinal (GI) cancer causes around 3.4 million deaths per year worldwide, and the 5-year survival rates are below 30\% in many cases after their surgery~\cite{arnoldGlobalBurdenMajor2020, mocanSurgicalManagementGastric2021,wongGlobalIncidenceMortality2021}. 
Monitoring postoperative GI cancer patients is challenging for healthcare providers due to the wide range of potential health issues patients may face. Severe complications such as anastomotic leakage and sepsis~\cite{vankootenPreoperativeRiskFactors2021, yasunagaBodyMassIndex2013} are common, while changes in the GI system can further disrupt patients' nutritional status and quality of life (QOL), potentially impacting their long-term survival~\cite{carmichaelEarlyPostoperativeFeeding2022, garthNutritionalStatusNutrition2010, yasunagaBodyMassIndex2013}.
Research work in surgical oncology has investigated preoperative factors such as lifestyle data to identify high-risk patients~\cite{vankootenPreoperativeRiskFactors2021}. 
However, many of the postoperative complications remain urgent and unpredictable~\cite{mocanSurgicalManagementGastric2021, vankootenPreoperativeRiskFactors2021}. 


After the postoperative cancer patients are discharged for their recovery at home, close monitoring of patient health conditions helps address these unpredictable conditions through early interventions~\cite{desrame2024595p}. Following clinical guidelines~\cite{steele1993standard}, cancer care institutions have adopted remote patient monitoring (RPM) systems to monitor certain aspects of a patient's health from their own home but also learned their limitations~\cite{malasingheRemotePatientMonitoring2019}. 
Traditional RPM methods such as phone calls, patient portals, or rigorous questionnaires could save time and travel efforts for both patients and providers~\cite{malasingheRemotePatientMonitoring2019}.
Yet the questionnaires are frequently lengthy and complex, leading to patients' ``survey fatigue'', which significantly diminishes the quality and quantity of information gathered~\cite{li2024beyond, silva_ostomy_2020}. Alternative methods, such as secure messaging or phone calls, are typically one-way (from patients to providers) and often lack clear instructions~\cite{chandwani_stitching_2018} to the patients who have limited health literacy~\cite{cardellaComplianceAttitudesBarriers2008}. Consequently, cancer care providers continue to face a shortage of essential information for effective interventions and decision-making~\cite{yangWishThereWere2024}.
Additionally, \citet{leon2022impact} pointed out issues in telehealth systems for clinical staff including nurses and specialists, including (1) additional workload such as documentation efforts, (2) disruption to the workflow, and (3) false alarms and unclear data.
In other cases, redundant information causes cancer care providers' information overload beyond their clinical duties, and thus necessary interventions could be delayed~\cite{clark_understanding_2021, bhat_infrastructuring_2021, yangWishThereWere2024}.
The limitations in traditional patient monitoring methods in postoperative GI cancer care call for more efficient, user-friendly technology for RPM while ensuring clinical adherence.
Recent research by \citet{yangWishThereWere2024} summarized some key information that cancer providers look for in their RPM practices, but how we might design such novel technology remains under-explored.


\subsection{Ubiquitous Systems for Remote Patient Monitoring in Cancer Care}
\label{sub:related_work-2}
Recent advancements in ubiquitous systems have significantly advanced the collection of patient health information.
Mobile health (mHealth) applications, sensors, wearable systems, and personal coaches have been used to track patient stress levels and daily activities~\cite{goncalves-bradley_mobile_2020,segoviaSmartphonebasedEcologicalMomentary2020, kingMicroStressEMAPassive2019, jacobsMyPathInvestigatingBreast2018}. 
In particular, some research in clinical settings has designed and tested mHealth apps that effectively integrate questionnaires or wearable devices to track patients' activities and postoperative conditions~\cite{sempleUsingMobileApp2015, ghods2021remote}.
However, these solutions, mostly collecting structured and quantitative data, still fall short in guiding patients to report descriptive domain-specific symptoms, and lack flexibility and adaptivity regarding the various postoperative conditions of postoperative GI cancer patients (e.g. feeling of pain or nausea)~\cite{yangWishThereWere2024}. The provider interfaces also mostly present unprocessed raw data, further overloading busy healthcare providers~\cite{yangWishThereWere2024, sempleUsingMobileApp2015}. 
Realizing these limitations, a recent co-design study with patients after immutherapy not only  emphasizes the significance of patient-reported symptoms in RPM, but also advocates for features that (1) correspond to specific side effects, (2) improve clinician interpretability and usability, and (3) automate RPM tasks~\cite{lai2024co}. 
We are motivated to explore novel systems for RPM that integrate clinical guidelines and needs into advanced interactive technologies.

\subsection{Large Language Models for Clinical Work in Patient Care}

The recent technological boost of Large Language Models (LLMs), such as GPT~\cite{OpenAI_2022}, offers a promising opportunity for system designers to picture a more clinically compliant solution with LLMs' great potential in engaging in and scaffolding natural clinical conversations~\cite{yunxiang2023chatdoctor,xiao2020tell, xiao2023supporting,shen2023convxai, hamalainen_evaluating_2023}. 
Researchers have explored LLM-powered CAs for remotely collecting patient health data in areas such as public health interventions, chronic disease management, and pre-consultation screening~\cite{wei_leveraging_2023, montagna2023data, dwyer2023use, ni2017mandy, li2024beyond, hao2024advancing}. For instance, a multi-modal CA by \citet{chira2022affective} collects health data from patients with brain diseases by asking general check-in questions such as ``how are you doing today''. 
However, most of the LLM CAs focus on patient experiences or data collection, overlooking the information needs of clinical staff, especially in high-risk cases like postoperative GI cancer care. Yet, good clinical adherence could benefit intervention or decision-making in domain-specific conditions. 
Recently, an LLM CA leverages QA datasets and wearable sensor data to help explain health monitoring data to patients (e.g. glucose from wearable sensors) ~\cite{ferrara2024large}, revealing great potential for LLM-powered telehealth systems to integrate clinical guidelines and symptom-related information needs. As LLM CAs continue to proliferate, we investigate their potential in critical clinical settings for efficient and comprehensive RPM.



\label{sub:related_work-3}
Meanwhile, language models have also shown their outstanding capability in processing medical domain knowledge in clinical practices, especially in pre-trained models like MedPaLM\cite{singhalLargeLanguageModels2023}, UmlsBERT\cite{michalopoulosUmlsBERTClinicalDomain2021}, and BioBERT~\cite{leeBioBERTPretrainedBiomedical2019}. For healthcare professionals, researchers have also leveraged LLMs for clinical pre-screening~\cite{hamer_improving_2023, wang_brilliant_2021}, risk prediction ~\cite{o2015risk,gatt2022review, kennedy2014delirium, beede2020human} and information processing~\cite{kocaballi_envisioning_2020, nair_generating_2023,  cascella2023evaluating, nori2023capabilities, agrawal_large_2022, arbatti_what_2023}. 
Specifically related to patient monitoring, LLMs may optimize clinical workflow in basic tasks like scheduling or reviewing information~\cite{tripathi2024efficient}; some work leveraged LLM benchmarks to analyze patient monitoring data ~\cite{healey2024llm}, or multi-modal LLMs to automate patient health monitoring ~\cite{ho2024remoni}.
Although LLM responses may align with clinical guidelines in general evaluations, they may not directly adhere to domain-specific clinical work, such as those in need of evidence-based recommendations~\cite{nwachukwu2024currently}.
In 2023, researchers integrated decision trees from medical literature for LLM-supported clinical decision-making~\cite{li2023meddm}, but little work has explored how to design LLMs' clinical integration for RPM or similar postoperative scenarios.


Focusing on the experience of clinical staff, LLM-powered interfaces such as digital dashboards have been designed for providers to process patient conditions in daily healthcare or emergency decision-making~\cite{yangWishThereWere2024, zhangRethinkingHumanAICollaboration2024}. However, these designs may not suit RPM for postoperative GI cancer, where both domain-specific provider instructions, clinical specifications, and timely qualitative data from patients are essential. Some clinical studies suggest using machine learning (ML) models to identify high-risk postoperative GI cancer patients~\cite{vankootenPreoperativeRiskFactors2021, chenDevelopmentValidationMachine2022a}, but such models have yet to be tested or deployed in real-world settings. Therefore, we aim to explore how clinical needs and guidelines can be integrated with LLMs to develop an LLM-powered telehealth system for postoperative GI cancer care and similar high-risk scenarios.
