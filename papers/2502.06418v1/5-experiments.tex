\section{Evaluation}
\subsection{Setup}
\textbf{Dataset}. We use two datasets to validate our attack method, including COCO~\cite{MS-COCO} and DIV2K~\cite{Agustsson_2017_CVPR_Workshops}. We randomly selected 100 images from each dataset, and each image was embedded with random watermark information using different watermarking algorithms. Another 100 non-overlapping images were also selected from the COCO~\cite{MS-COCO} dataset to serve as clean images for the forgery attack.

\textbf{Watermark setting}. To evaluate our attack method, we test seven publicly available watermarking methods: DwtDctSvd~\cite{DWT-DCT-SVD}, DwtDct~\cite{DWT-DCT}, RivaGAN~\cite{RivaGAN}, StegaStamp~\cite{stegastamp}, HiDDeN~\cite{zhu2018hidden}, PIMoG~\cite{PIMoG}, and CIN~\cite{CIN}. % detection

% For evasion attacks, we not only selected traditional image degradation methods such as JPEG compression, Gaussian noise, and Gaussian blur for comparison but also compared our approach with existing attack methods: WmAttacker\cite{zhao2023invisible} and WmRobust~\cite{saberi2023robustness}. For forgery attacks, we evaluated our method alongside related forgery techniques: CopyAttack~\cite{kutter2000watermark}, Steganalysis~\cite{yang2024steganalysis} and WmRobust~\cite{saberi2023robustness}.

\textbf{Attack Benchmarking}. 
For evasion attacks, we compared our approach not only with traditional image degradation techniques such as JPEG compression, Gaussian noise, and Gaussian blur but also with related attack methods, including WmAttacker~\cite{zhao2023invisible} and WmRobust~\cite{saberi2023robustness}. For forgery attacks, we compare our method with existing forgery techniques, including CopyAttack~\cite{kutter2000watermark}, Steganalysis~\cite{yang2024steganalysis}, and WmRobust~\cite{saberi2023robustness}.


\textbf{Evaluation metrics}. We evaluate the visual quality of the attacked watermarked images and the corresponding original watermarked images using two commonly used metrics: Structural Similarity Index Measure (SSIM)~\cite{wang2004image} and Peak Signal-to-Noise Ratio (PSNR). To evaluate the effectiveness of our attack method, we use bit accuracy and success rate (SR) as metrics. Bit accuracy refers to the correct matching rate between the watermark extracted from the image and its ground truth. SR represents the proportion of successfully attacked samples to the total number of samples. This varies under two attack scenarios: in an evasion attack, the attack is considered successful if the bit accuracy of the attacked image falls below a certain threshold. For a forgery attack, the opposite holds true.

\textbf{Parameter settings}. We use AdamW~\cite{loshchilov2017decoupled} as the optimizer, with a learning rate of 0.15 and a weight decay of 1e-4. We chose SSIM loss and L1 loss as the loss functions $\mathcal{L}(\cdot,\cdot)$ and employ a pre-trained DenseNet~\cite{DenseNet2017} as the feature extraction module $\mathcal{F}(\cdot)$. Regarding the detection thresholds, we uniformly set them to \{0.95, 0.9, 0.85, 0.8, 0.75, 0.7, 0.65, 0.6, 0.55\}, and also establish a threshold calculated based on watermark length $n$: A watermark will be detected if we can reject the null hypothesis $H_0$ with a p-value less than 0.05. The null hypothesis $H_0$ states that k matching bits were extracted from the watermarked image by random chance. This event has a probability of $P(X\ge k|H_0) = \sum_{i=k}^n{n\choose k}0.5^n$.



\begin{table*}[!th]
\centering
\caption{Overall evasion performance on the dataset COCO. Detailed results of full threshold candidates are shown in Appendix~\ref{sec:appendix:Evasion Attack against Related Works}.}
\label{tab:evade_coco_overall}
\resizebox{\textwidth}{!}{
\begin{tabular}{@{}cccc|ccc|ccc|ccc|ccc|ccc|ccc@{}}
\toprule
\multirow{2}{*}{\textbf{Methods}}   &\multicolumn{3}{c}{\textbf{PIMoG(th=0.6)}}  &\multicolumn{3}{c}{\textbf{HiDDeN(th=0.6)}}
&\multicolumn{3}{c}{\textbf{StegaStamp(th=0.57)}}
&\multicolumn{3}{c}{\textbf{DwtDct(th=0.625)}}
&\multicolumn{3}{c}{\textbf{DwtDctSvd(th=0.625)}}
&\multicolumn{3}{c}{\textbf{RivaGan(th=0.625)}}
&\multicolumn{3}{c}{\textbf{CIN(th=0.6)}}\\ \cmidrule(l){2-22} 
             & \textbf{SR$\uparrow$} & \textbf{SSIM$\uparrow$} & \textbf{PSNR$\uparrow$} & \textbf{SR$\uparrow$} & \textbf{SSIM$\uparrow$} & \textbf{PSNR$\uparrow$} & \textbf{SR$\uparrow$} & \textbf{SSIM$\uparrow$} & \textbf{PSNR$\uparrow$} & \textbf{SR$\uparrow$} & \textbf{SSIM$\uparrow$} & \textbf{PSNR$\uparrow$} & \textbf{SR$\uparrow$} & \textbf{SSIM$\uparrow$} & \textbf{PSNR$\uparrow$} & \textbf{SR$\uparrow$} & \textbf{SSIM$\uparrow$} & \textbf{PSNR$\uparrow$} & \textbf{SR$\uparrow$} & \textbf{SSIM$\uparrow$} & \textbf{PSNR$\uparrow$}  \\ \midrule
WmAttacker & 0 & 0.749 & 28.503 & 0.32 & 0.628 & 25.026 & 0 & 0.715 & 27.586 & 0.91 & 0.598 & 24.744 & 0.41 & 0.57 & 23.498 & 0.13 & 0.601 & 24.555 & 0.01 & 0.614 & 28.062 \\
WmRobust & 0.02 & \textbf{0.948} & \textbf{40.195} & 0.34 & 0.808 & 35.363 & 0.19 & \textbf{0.899} & \textbf{38.003} & 0.93 & 0.815 & \textbf{34.325} & 0.6 & 0.822 & \textbf{34.034} & 0.47 & 0.842 & 35.648 & 0.38 & 0.891 & 38.933\\
JPEG & 0 & 0.897 & 37.538 & 0.57 & 0.813 & 34.161 & 0 & 0.867 & 35.976 & 0.93 & 0.785 & 32.722 & 0.87 & 0.794 & 32.512 & 0.15 & 0.793 & 33.289 & 0.02 & \textbf{0.898} & \textbf{39.238}\\
Gaussian & 0.01 & 0.197 & 26.532 & \textbf{0.81} & 0.406 & 26.441 & 0 & 0.365 & 26.496 & \textbf{0.96} & 0.422 & 26.464 & 0.45 & 0.393 & 26.444 & 0.01 & 0.157 & 26.579 & 0 & 0.324 & 26.447\\
GaussianBlur & 0 & 0.771 & 31.362 & 0.05 & 0.651 & 30.567 & 0 & 0.713 & 30.123 & 0.85 & 0.603 & 28.435 & 0 & 0.61 & 28.378 & 0 & 0.615 & 28.433 & 0 & 0.761 & 31.495\\
\textbf{Ours} & \textbf{0.87} & 0.876 & 36.403 & 0.77 & \textbf{0.889} & \textbf{36.644} & \textbf{1} & 0.895 & 36.703 & \textbf{0.96} & \textbf{0.838} & 33.834 & \textbf{0.95} & \textbf{0.85} & 33.841 & \textbf{1} & \textbf{0.872} & \textbf{35.851} & \textbf{0.78} & 0.809 & 34.801\\
\bottomrule
\end{tabular}}
\vspace{-3mm}
\end{table*}


\begin{table*}[!th]
\centering
\caption{Overall evasion performance on the dataset DIV2K. Detailed results of full threshold candidates are shown in Appendix~\ref{sec:appendix:Evasion Attack against Related Works}.}
\label{tab:evade_div2k_overall}
\resizebox{\textwidth}{!}{
\begin{tabular}{@{}cccc|ccc|ccc|ccc|ccc|ccc|ccc@{}}
\toprule
\multirow{2}{*}{\textbf{Methods}}   &\multicolumn{3}{c}{\textbf{PIMoG(th=0.6)}}  &\multicolumn{3}{c}{\textbf{HiDDeN(th=0.6)}}
&\multicolumn{3}{c}{\textbf{StegaStamp(th=0.57)}}
&\multicolumn{3}{c}{\textbf{DwtDct(th=0.625)}}
&\multicolumn{3}{c}{\textbf{DwtDctSvd(th=0.625)}}
&\multicolumn{3}{c}{\textbf{RivaGan(th=0.625)}}
&\multicolumn{3}{c}{\textbf{CIN(th=0.6)}}\\ \cmidrule(l){2-22} 
             & \textbf{SR$\uparrow$} & \textbf{SSIM$\uparrow$} & \textbf{PSNR$\uparrow$} & \textbf{SR$\uparrow$} & \textbf{SSIM$\uparrow$} & \textbf{PSNR$\uparrow$} & \textbf{SR$\uparrow$} & \textbf{SSIM$\uparrow$} & \textbf{PSNR$\uparrow$} & \textbf{SR$\uparrow$} & \textbf{SSIM$\uparrow$} & \textbf{PSNR$\uparrow$} & \textbf{SR$\uparrow$} & \textbf{SSIM$\uparrow$} & \textbf{PSNR$\uparrow$} & \textbf{SR$\uparrow$} & \textbf{SSIM$\uparrow$} & \textbf{PSNR$\uparrow$} & \textbf{SR$\uparrow$} & \textbf{SSIM$\uparrow$} & \textbf{PSNR$\uparrow$}  \\ \midrule
WmAttacker & 0 & 0.739 & 26.853 & 0.34 & 0.64 & 23.954 & 0 & 0.71 & 26.2 & 0.89 & 0.553 & 21.427 & 0.38 & 0.527 & 20.62 & 0.15 & 0.536 & 19.978 & 0.01 & 0.644 & 23.031\\
WmRobust & 0.01 & \textbf{0.942} & \textbf{40.128} & 0.2 & 0.814 & \textbf{34.525} & 0.16 & \textbf{0.912} & \textbf{37.374} & 0.89 & 0.821 & 31.903 & 0.48 & 0.827 & 31.676 & 0.35 & \textbf{0.848} & \textbf{32.769} & 0.18 & \textbf{0.891} & \textbf{38.295}\\
JPEG & 0 & 0.897 & 36.37 & 0.42 & 0.822 & 33.522 & 0 & 0.867 & 35.976 & 0.89 & 0.766 & 29.993 & 0.77 & 0.774 & 29.897 & 0.12 & 0.771 & 29.76 & 0.01 & 0.85 & 35.727\\
Gaussian & 0 & 0.399 & 26.477 & 0.72 & 0.444 & 26.509 & 0 & 0.365 & 26.496 & 0.9 & 0.516 & 26.522 & 0.44 & 0.502 & 26.561 & 0.01 & 0.578 & 26.954 & 0 & 0.362 & 26.479\\
GaussianBlur & 0 & 0.725 & 30.154 & 0.05 & 0.62 & 27.565 & 0 & 0.713 & 30.123 & \textbf{0.91} & 0.505 & 25.97 & 0.03 & 0.368 & 22.169 & 0.01 & 0.526 & 24.617 & 0 & 0.735 & 30.952\\
\textbf{Ours} & \textbf{1} & 0.828 & 33.042 & \textbf{0.96} & \textbf{0.841} & 33.257 & \textbf{1} & 0.853 & 33.54 & 0.89 & \textbf{0.855} & \textbf{32.841} & \textbf{0.85} & \textbf{0.859} & \textbf{32.745}
 & \textbf{1} & 0.841 & 32.656 & \textbf{0.45} & 0.811 & 34.744\\
\bottomrule
\end{tabular}}
\vspace{-3mm}
\end{table*}


\begin{table*}[!th]
\centering
\caption{Overall spoofing performance on the dataset COCO. Detailed results of full threshold candidates are shown in Appendix~\ref{sec:Appendix_Additional Experimental Results_Spoof Attack}.}
\setlength{\tabcolsep}{13pt}
\label{tab:spoof_coco_overall}
\resizebox{\textwidth}{!}{
\begin{tabular}{@{}cccc|ccc|ccc|ccc|ccc@{}}
\toprule
\multirow{2}{*}{\textbf{Methods}}   &\multicolumn{3}{c}{\textbf{PIMoG(th=0.6)}}  &\multicolumn{3}{c}{\textbf{HiDDeN(th=0.6)}}
&\multicolumn{3}{c}{\textbf{StegaStamp(th=0.57)}}
&\multicolumn{3}{c}{\textbf{RivaGan(th=0.625)}}
&\multicolumn{3}{c}{\textbf{CIN(th=0.6)}}\\ \cmidrule(l){2-16} 
             & \textbf{SR$\uparrow$} & \textbf{SSIM$\uparrow$} & \textbf{PSNR$\uparrow$} & \textbf{SR$\uparrow$} & \textbf{SSIM$\uparrow$} & \textbf{PSNR$\uparrow$} & \textbf{SR$\uparrow$} & \textbf{SSIM$\uparrow$} & \textbf{PSNR$\uparrow$} & \textbf{SR$\uparrow$} & \textbf{SSIM$\uparrow$} & \textbf{PSNR$\uparrow$} & \textbf{SR$\uparrow$} & \textbf{SSIM$\uparrow$} & \textbf{PSNR$\uparrow$}  \\ \midrule
CopyAttack & 0.13 & 0.729 & 20.786 & 0.16 & 0.78 & 21.682 & 0.09 & 0.732 & 20.296 & 0.06 & 0.704 & 20.471 & 0.1 & 0.724 & 19.795\\
Steganalysis & 0.1 & 0.907 & \textbf{34.524} & 0.05 & \textbf{0.902} & \textbf{34.451} & 0.08 & \textbf{0.923} & \textbf{34.42} & 0.07 & \textbf{0.933} & \textbf{34.483} & 0.18 & \textbf{0.919} & \textbf{34.572}\\
WmRobust & 0.86 & \textbf{0.915} & 31.129 & 0.53 & 0.726 & 26.387 & 0.92 & 0.832 & 29.731 & 0.08 & 0.795 & 31.841 & \textbf{1} & 0.82 & 28.529\\
\textbf{Ours} & \textbf{1} & 0.809 & 33.485 & \textbf{0.91} & 0.704 & 33.137 & \textbf{1} & 0.827 & 34.01 & \textbf{0.18} & 0.683 & 33.312 & \textbf{1} & 0.807 & 33.716\\
\bottomrule
\end{tabular}}
\vspace{-3mm}
\end{table*}

\begin{figure}[!t]
    \centering
    % Custom font size
    \includegraphics[width=\linewidth]{pics/evasion-example-pimog.png} 
    
    % \vspace{-3mm}
    \caption{Examples of watermark removal via our evasion attack on PIMoG}
    \label{fig:evasion-pimog}
    \vspace{-3mm}
\end{figure}

\begin{figure}[!t]
    \centering
    % Custom font size
    \includegraphics[width=\linewidth]{pics/spoof-example-pimog.png} 
    \vspace{-3mm}
    \caption{Examples of watermark spoofing via our forgery attack on PIMoG.}
    \label{fig:spoof-pimog}
    \vspace{-3mm}
\end{figure}

\subsection{Results and Analysis}
%In this section, we will present the detailed results of our attacks, which will be divided into two parts: evasion attacks and forgery attacks.
In this section, we present the detailed results of our attacks and provide an analysis of the relevant findings. More detailed experimental results can be found in Appendix ~\ref{sec:Appendix_Additional Experimental Results}.

\textbf{Evasion Attack}. Tables~\ref{tab:evade_coco_overall} and Table~\ref{tab:evade_div2k_overall} summarize the comparison of our method, related methods, and image degradation factors regarding success rate and visual fidelity metrics. Our method achieves optimal or suboptimal attack success rates across different datasets and algorithms while maintaining high visual quality. Overall, our method achieves an average success rate improvement of 60\% on the COCO dataset and 61\% on the DIV2K dataset compared to other methods. Figure~\ref{fig:evasion-pimog} shows some examples of the evasion attack.


\textbf{Forgery Attack}. Table~\ref{tab:spoof_coco_overall} presents a performance comparison between our method and related forgery methods. Our method achieves the highest forgery success rate while preserving visual fidelity. Compared to related forgery methods, our method achieves an average success rate improvement of 51\%. We found that WmRobust~\cite{saberi2023robustness} also performs well when targeting certain watermarking models. However, WmRobust requires access to the target watermarking algorithm's encoder, whereas our method does not. Figure~\ref{fig:spoof-pimog} shows some examples of the forgery attack.

\subsection{Ablation Study}
\textbf{Ablation study for evasion attack}. Table~\ref{tab:evade_coco_ablation} reports the results of our evasion attack and the results of ablating different parts of the method, validating the effectiveness of our design. The experiments are divided into two parts: feature extraction network and feature channel position retrieval. \textbf{w/o $\mathcal{F}$} indicates the use of the original image without feature extraction, while \textbf{w/o $\mathcal{W}$} indicates the use of all channels without selective retrieval. 

Using only the feature extraction module $\mathcal{F}$ without channel position retrieval achieves strong attacks but severely degrades image quality. Optimizing directly on the original image worsens visual perception and reduces attack effectiveness. These results confirm that watermark leakage can be captured via feature extraction, and precise localization can minimize image distortion.


\textbf{Ablation study for forgery attack}. Table~\ref{tab:spoof_coco_ablation} presents the ablation results of our forgery attack, divided into three parts: Stage \uppercase\expandafter{\romannumeral1}, Only Stage \uppercase\expandafter{\romannumeral2} and Stage \uppercase\expandafter{\romannumeral1} +  Stage \uppercase\expandafter{\romannumeral2}. Note that Only Stage \uppercase\expandafter{\romannumeral2} is shown as follows:

\begin{equation}
\label{eq:process-2}
\begin{split}
     \mathop{\min}_{\delta_s}\mathcal{L}((1-\mathcal{W}) \cdot \mathcal{F}(I_{wm}), (1-\mathcal{W})\cdot \mathcal{F}(I' + \delta_s)) \\
    \mathrm{ s.t.} ||\delta_s||_{\infty} < \epsilon
\end{split}
\end{equation}


Experimental results show that we successfully forge watermarks of three algorithms—PIMoG, StegaStamp, and CIN, achieving optimal performance using Stage \uppercase\expandafter{\romannumeral1}, where the attack merely extracts leaked watermark information and embeds it into a clean image. This indicates that these schemes do not enforce a strong alignment between watermark information and image content, making them highly vulnerable to forgery attacks.

However, for RivaGan and HiDDeN, the Stage \uppercase\expandafter{\romannumeral1} forgery attack performs poorly, while combining Stage \uppercase\expandafter{\romannumeral1} and Stage \uppercase\expandafter{\romannumeral2} obviously improves the forgery effect, suggesting that RivaGan and HiDDeN facilitate a more substantial alignment between image semantics and watermark information.

Notably, all algorithms achieve satisfactory forgery results through Only Stage \uppercase\expandafter{\romannumeral2}. We conjecture two reasons for this: 1) Some leakage information of watermark can be identified from the less-watermark-related channels with Stage \uppercase\expandafter{\romannumeral2}; 2) Stage \uppercase\expandafter{\romannumeral2} considers more semantic information from watermarked images, improving the alignment between watermark information and image semantics. The results demonstrate the effectiveness of our method design and also support the theory proposed in Sec.~\ref{sec:Method_Theory}.


\begin{table*}[!th]
\centering
\caption{Ablation study for evasion attack. Detailed results of full threshold candidates are shown in Appendix~\ref{sec:appendix:Ablation_Study_for_Evasion_Attack}.}
\label{tab:evade_coco_ablation}
\resizebox{\textwidth}{!}{
\begin{tabular}{@{}cccc|ccc|ccc|ccc|ccc|ccc|ccc@{}}
\toprule
\multirow{2}{*}{\textbf{Methods}}   &\multicolumn{3}{c}{\textbf{PIMoG(th=0.6)}}  &\multicolumn{3}{c}{\textbf{HiDDeN(th=0.6)}}
&\multicolumn{3}{c}{\textbf{StegaStamp(th=0.57)}}
&\multicolumn{3}{c}{\textbf{DwtDct(th=0.625)}}
&\multicolumn{3}{c}{\textbf{DwtDctSvd(th=0.625)}}
&\multicolumn{3}{c}{\textbf{RivaGan(th=0.625)}}
&\multicolumn{3}{c}{\textbf{CIN(th=0.6)}}\\ \cmidrule(l){2-22} 
             & \textbf{SR$\uparrow$} & \textbf{SSIM$\uparrow$} & \textbf{PSNR$\uparrow$} & \textbf{SR$\uparrow$} & \textbf{SSIM$\uparrow$} & \textbf{PSNR$\uparrow$} & \textbf{SR$\uparrow$} & \textbf{SSIM$\uparrow$} & \textbf{PSNR$\uparrow$} & \textbf{SR$\uparrow$} & \textbf{SSIM$\uparrow$} & \textbf{PSNR$\uparrow$} & \textbf{SR$\uparrow$} & \textbf{SSIM$\uparrow$} & \textbf{PSNR$\uparrow$} & \textbf{SR$\uparrow$} & \textbf{SSIM$\uparrow$} & \textbf{PSNR$\uparrow$} & \textbf{SR$\uparrow$} & \textbf{SSIM$\uparrow$} & \textbf{PSNR$\uparrow$}  \\ \midrule
% ssim loss + no channel + feature net & 1 & 0.331 & 28.591 & 0.86 & 0.324 & 27.93 & 1 & 0.345 & 28.41 & 0.91 & 0.386 & 27.873 & 0.93 & 0.382 & 27.873 & 1 & 0.373 & 27.829 & 0.89 & 0.314 & 28.789 \\
$\mathcal{W}$, $\mathcal{F}$ & 0.87 & \textbf{0.876} & \textbf{36.403} & 0.77 & \textbf{0.889} & \textbf{36.644} & \textbf{1} & \textbf{0.895} & \textbf{36.703}
& \textbf{0.96} & \textbf{0.838} & \textbf{33.834} & \textbf{0.95} & \textbf{0.85} & \textbf{33.841} & \textbf{1} & \textbf{0.872}	& \textbf{35.851} & 0.78	& \textbf{0.809}	& \textbf{34.801} \\
w/o $\mathcal{W}$, $\mathcal{F}$ & \textbf{1} & 0.331 & 28.591 & \textbf{0.86} & 0.324 & 27.93 & \textbf{1} & 0.345 & 28.41 & 0.91 & 0.386 & 27.873 & 0.93 & 0.382 & 27.873 & \textbf{1} & 0.373 & 27.829 & 0.89 & 0.314 & 28.789 \\
% ssim loss + no channel + no feature net & 0 & 0.184 & 27.605 & 0.52 & 0.163 & 27.644 & 0 & 0.195 & 27.615 & 0.62 & 0.245 & 27.507 & 0.32 & 0.125 & 27.996 & 0.3 & 0.118 & 27.973 & 1 & 0.142 & 27.762\\
% l1 loss + channel + feature net & 0.02 & 0.914 & 40.603 & 0.04 & 0.941 & 40.825 & 0.18 & 0.937 & 40.549 & 0.19 & 0.955 & 41.123 & 0.76 & 0.963 & 40.858 & 0.11 & 0.933 & 40.367 & 0 & 0.954 & 40.641 \\
w/o $\mathcal{W}$, w/o $\mathcal{F}$ & 0 & 0.184 & 27.605 & 0.52 & 0.163 & 27.644 & 0 & 0.195 & 27.615 & 0.62 & 0.245 & 27.507 & 0.32 & 0.125 & 27.996 & 0.3 & 0.118 & 27.973 & \textbf{1} & 0.142 & 27.762\\
% l1 loss + channel + feature net & 0.02 & 0.914 & 40.603 & 0.04 & 0.941 & 40.825 & 0.18 & 0.937 & 40.549 & 0.19 & 0.955 & 41.123 & 0.76 & 0.963 & 40.858 & 0.11 & 0.933 & 40.367 & 0 & 0.954 & 40.641 \\
% l1 loss + no channel + feature net & 0 & 0.976 & 46.751 & 0 & 0.975 & 43.416 & 0.01 & 0.972 & 44.077 & 0.18 & 0.96 & 42.833 & 0.39 & 0.968 & 42.545 & 0.01 & 0.963 & 41.495 & 0 & 0.965 & 43.985\\
% l1 loss + no channel + no feature net & 0 & 0.786 & 37.497 & 0 & 0.837 & 37.497 & 0 & 0.794 & 37.498 & 0.14 & 0.786 & 37.497 & 0 & 0.826 & 37.497 & 0 & 0.826 & 37.497 & 0 & 0.778 & 37.498\\
% -ssim loss + channel + feature net & 0.137 & 0.911 & 36.995 & 0.07 & 0.931 & 39.759 & 0.4 & 0.953 & 39.467 & 0.31 & 0.952 & 39.881 & 0.72 & 0.965 & 39.939 & 0.3 & 0.933 & 39.336 & 0.03 & 0.859 & 39.32\\
% -l1 loss + channel + feature net & 0.53 & 0.834 & 33.671 & 0.05 & 0.868 & 35.654 & 0.95 & 0.855 & 33.064 & 0.83 & 0.854 & 32.593 & 0.82 & 0.848 & 32.298 & 0.94 & 0.843 & 32.392 & 0 & 0.833 & 32.424\\
% ssim loss + channel + feature net & 0.87 & 0.876 & 36.403 & 0.77 & 0.889 & 36.644 & 1 & 0.895 & 36.703 & 0.96 & 0.838 & 33.834 & 0.95 & 0.85 & 33.841 & 1 & 0.872 & 35.851 & 0.78 & 0.809 & 34.801\\
\bottomrule
\end{tabular}}
\vspace{-4mm}
\end{table*}


% \begin{table}[!th]
% \centering
% \caption{Ablation study for spoofing attack 1.}
% \label{tab:spoof_coco_ablation_1}
% \resizebox{\linewidth}{!}{
% \begin{tabular}{@{}cccc|ccc|ccc@{}}
% \toprule
% \multirow{2}{*}{\textbf{Methods}}   &\multicolumn{3}{c}{\textbf{pimog(th=0.6)}} 
% &\multicolumn{3}{c}{\textbf{stegastamp(th=0.57)}}
% &\multicolumn{3}{c}{\textbf{cin(th=0.6)}}\\ \cmidrule(l){2-10} 
%              & \textbf{SR} & \textbf{SSIM} & \textbf{PSNR} & \textbf{SR} & \textbf{SSIM} & \textbf{PSNR} & \textbf{SR} & \textbf{SSIM} & \textbf{PSNR}  \\ \midrule
% ab-2 & 1 & 0.499 & 28.504 & 1 & 0.489 & 28.378 & 1 & 0.503 & 28.628\\
% ab-3 & 0.62 & 0.433 & 27.701 & 0.71 & 0.413 & 27.698 & 1 & 0.451 & 27.862\\
% ab-4 & 0.86 & 0.947 & 40.408 & 0.95 & 0.951 & 40.61 & 1 & 0.949 & 40.644\\
% ab-10 & 0.89 & 0.95 & 40.23 & 0.97 & 0.955 & 40.641 & 1 & 0.952 & 40.559\\
% ab-13 & 1 & 0.86 & 34.277 & 0.99 & 0.861 & 34.26 & 1 & 0.857 & 34.281\\
% ours & 1 & 0.809 & 33.485 & 1 & 0.827 & 34.01 & 1 & 0.807 & 33.716\\
% \bottomrule
% \end{tabular}}
% \end{table}


% \begin{table}[!th]
% \centering
% \caption{Ablation study for evasion attack 2.}
% \label{tab:spoof_coco_ablation_2}
% \resizebox{\linewidth}{!}{
% \begin{tabular}{@{}cccc|ccc@{}}
% \toprule
% \multirow{2}{*}{\textbf{Methods}}   &\multicolumn{3}{c}{\textbf{hidden(th=0.6)}} 
% &\multicolumn{3}{c}{\textbf{rivagan(th=0.625)}}\\ \cmidrule(l){2-7} 
%              & \textbf{SR} & \textbf{SSIM} & \textbf{PSNR} & \textbf{SR} & \textbf{SSIM} & \textbf{PSNR}  \\ \midrule
% simple-7 & 0.92 & 0.798 & 35.335 & 0.13 & 0.788 & 35.39\\
% delta1, dual spoof, -ssim & 0.15 & 0.659 & 28.375 & 0.12 & 0.683 & 28.403\\
% delta1 + delta2, dual spoof, l1 & 0.909 & 0.705 & 31.858 & 0.17 & 0.697 & 31.943\\
% delta1, dual spoof, l1 & 0.2 & 0.681 & 27.639 & 0.14 & 0.645 & 27.844\\
% ours & 0.91 & 0.704 & 33.137 & 0.18 & 0.683 & 33.312\\
% \bottomrule
% \end{tabular}}
% \end{table}


\begin{table*}[!th]
\centering
\caption{Ablation study for spoofing attack. Detailed results of full threshold candidates are shown in Appendix~\ref{sec:appendix:Ablation_Study_for_Spoof_Attack}.}
\setlength{\tabcolsep}{13pt}
\label{tab:spoof_coco_ablation}
\resizebox{\textwidth}{!}{
\begin{tabular}{@{}cccc|ccc|ccc|ccc|ccc@{}}
\toprule
\multirow{2}{*}{\textbf{Methods}}   &\multicolumn{3}{c}{\textbf{PIMoG(th=0.6)}}  &\multicolumn{3}{c}{\textbf{HiDDeN(th=0.6)}}
&\multicolumn{3}{c}{\textbf{StegaStamp(th=0.57)}}
&\multicolumn{3}{c}{\textbf{RivaGan(th=0.625)}}
&\multicolumn{3}{c}{\textbf{CIN(th=0.6)}}\\ \cmidrule(l){2-16} 
             & \textbf{SR$\uparrow$} & \textbf{SSIM$\uparrow$} & \textbf{PSNR$\uparrow$} & \textbf{SR$\uparrow$} & \textbf{SSIM$\uparrow$} & \textbf{PSNR$\uparrow$} & \textbf{SR$\uparrow$} & \textbf{SSIM$\uparrow$} & \textbf{PSNR$\uparrow$} & \textbf{SR$\uparrow$} & \textbf{SSIM$\uparrow$} & \textbf{PSNR$\uparrow$} & \textbf{SR$\uparrow$} & \textbf{SSIM$\uparrow$} & \textbf{PSNR$\uparrow$}  \\ \midrule
Stage \uppercase\expandafter{\romannumeral1} & \textbf{1} & \textbf{0.809} & \textbf{33.485} & 0.15 & 0.659 & 28.375 & \textbf{1} & \textbf{0.827} & \textbf{34.01} & 0.12 & \textbf{0.683} & 28.403 & \textbf{1} & \textbf{0.807} & \textbf{33.716}\\
Only Stage \uppercase\expandafter{\romannumeral2} & \textbf{1} & 0.702 & 32.194 & \textbf{0.93} & 0.668 & 31.981 & 0.98 & 0.701 & 32.172 & \textbf{0.26} & 0.674 & 32.174 & \textbf{1} & 0.701 & 32.151\\
Stage \uppercase\expandafter{\romannumeral1} + Stage \uppercase\expandafter{\romannumeral2} & \textbf{1} & 0.684 & 31.818 & 0.91 & \textbf{0.704} & \textbf{33.137} & 0.99 & 0.685 & 31.779 & 0.18 & \textbf{0.683} & \textbf{33.312} & \textbf{1} & 0.679 & 31.769
\\
\bottomrule
\end{tabular}}
\vspace{-3mm}
\end{table*}