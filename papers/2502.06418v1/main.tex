%%%%%%%% ICML 2025 EXAMPLE LATEX SUBMISSION FILE %%%%%%%%%%%%%%%%%
\pdfoutput=1
\documentclass{article}

% Recommended, but optional, packages for figures and better typesetting:
\usepackage{microtype}
\usepackage{graphicx}
\usepackage{subfigure}
\usepackage{booktabs} % for professional tables
\usepackage{multirow}
% \usepackage{chngcntr}
% \counterwithout{figure}{section}
% hyperref makes hyperlinks in the resulting PDF.
% If your build breaks (sometimes temporarily if a hyperlink spans a page)
% please comment out the following usepackage line and replace
% \usepackage{icml2025} with \usepackage[nohyperref]{icml2025} above.
\usepackage{hyperref}


% Attempt to make hyperref and algorithmic work together better:
\newcommand{\theHalgorithm}{\arabic{algorithm}}

% Use the following line for the initial blind version submitted for review:
% \usepackage{icml2025}

% If accepted, instead use the following line for the camera-ready submission:
\usepackage[accepted]{dapao}

% For theorems and such
\usepackage{amsmath}
\usepackage{amssymb}
\usepackage{mathtools}
\usepackage{amsthm}

% if you use cleveref..
\usepackage[capitalize,noabbrev]{cleveref}

%%%%%%%%%%%%%%%%%%%%%%%%%%%%%%%%
% THEOREMS
%%%%%%%%%%%%%%%%%%%%%%%%%%%%%%%%
\theoremstyle{plain}
\newtheorem{theorem}{Theorem}[section]
\newtheorem{proposition}[theorem]{Proposition}
\newtheorem{lemma}[theorem]{Lemma}
\newtheorem{corollary}[theorem]{Corollary}
\theoremstyle{definition}
\newtheorem{definition}[theorem]{Definition}
\newtheorem{assumption}[theorem]{Assumption}
\theoremstyle{remark}
\newtheorem{remark}[theorem]{Remark}

% Todonotes is useful during development; simply uncomment the next line
%    and comment out the line below the next line to turn off comments
%\usepackage[disable,textsize=tiny]{todonotes}
\usepackage[textsize=tiny]{todonotes}

% ##### my own package
\usepackage[T1]{fontenc}
\usepackage{breakurl}
\usepackage{soul, color, xcolor}
\def\UrlBreaks{\do\A\do\B\do\C\do\D\do\E\do\F\do\G\do\H\do\I\do\J
\do\K\do\L\do\M\do\N\do\O\do\P\do\Q\do\R\do\S\do\T\do\U\do\V
\do\W\do\X\do\Y\do\Z\do\[\do\\\do\]\do\^\do\_\do\`\do\a\do\b
\do\c\do\d\do\e\do\f\do\g\do\h\do\i\do\j\do\k\do\l\do\m\do\n
\do\o\do\p\do\q\do\r\do\s\do\t\do\u\do\v\do\w\do\x\do\y\do\z
\do\.\do\@\do\\\do\/\do\!\do\_\do\|\do\;\do\>\do\]\do\)\do\,
\do\?\do\'\do+\do\=\do\#}
\usepackage{amsfonts}
\usepackage{makecell}
\usepackage{algorithm}
\usepackage{algorithmic}
% \usepackage{algpseudocode}

% The \icmltitle you define below is probably too long as a header.
% Therefore, a short form for the running title is supplied here:
% \icmltitlerunning{Submission and Formatting Instructions for ICML 2025}



\begin{document}

\twocolumn[
\icmltitle{Robust Watermarks Leak: Channel-Aware Feature Extraction Enables Adversarial Watermark Manipulation}



% Robust but vulnerable: A Learning-based Method for robust watermark forgery and evading attacks

% It is OKAY to include author information, even for blind
% submissions: the style file will automatically remove it for you
% unless you've provided the [accepted] option to the icml2025
% package.

% List of affiliations: The first argument should be a (short)
% identifier you will use later to specify author affiliations
% Academic affiliations should list Department, University, City, Region, Country
% Industry affiliations should list Company, City, Region, Country

% You can specify symbols, otherwise they are numbered in order.
% Ideally, you should not use this facility. Affiliations will be numbered
% in order of appearance and this is the preferred way.
% \icmlsetsymbol{equal}{*}
\icmlsetsymbol{corr}{*}

\begin{icmlauthorlist}
% \icmlauthor{Anonymous Authors}{}

\icmlauthor{Zhongjie Ba}{sch}
\icmlauthor{Yitao Zhang}{sch}
\icmlauthor{Peng Cheng}{sch,corr}
\icmlauthor{Bin Gong}{sch}
\icmlauthor{Xinyu Zhang}{sch}
\icmlauthor{Qinglong Wang}{sch}
\icmlauthor{Kui Ren}{sch}

% \icmlauthor{Firstname2 Lastname2}{equal,yyy,comp}
% \icmlauthor{Firstname3 Lastname3}{comp}
% \icmlauthor{Firstname4 Lastname4}{sch}
% \icmlauthor{Firstname5 Lastname5}{yyy}
% \icmlauthor{Firstname6 Lastname6}{sch,yyy,comp}
% \icmlauthor{Firstname7 Lastname7}{comp}
% \icmlauthor{Firstname8 Lastname8}{sch}
% \icmlauthor{Firstname8 Lastname8}{yyy,comp}
\icmlauthor{}{sch} The State Key Laboratory of
Blockchain and Data Security, Zhejiang University, Hangzhou, China
%\icmlauthor{}{sch}
\end{icmlauthorlist}


% \icmlaffiliation{sch}{The State Key Laboratory of Blockchain and Data Security, Zhejiang University, Hangzhou, China}
% \icmlaffiliation{yyy}{zhejiang}
% \icmlaffiliation{comp}{Company Name, Location, Country}
% \icmlaffiliation{corr}{corresponding author}
% \icmlaffiliation{sch}{School of ZZZ, Institute of WWW, Location, Country}

% \icmlcorrespondingauthor{Peng Cheng}{}
% \icmlcorrespondingauthor{Firstname2 Lastname2}{first2.last2@www.uk}


% % You may provide any keywords that you
% % find helpful for describing your paper; these are used to populate
% % the "keywords" metadata in the PDF but will not be shown in the document
% \icmlkeywords{Machine Learning, ICML}

\vskip 0.3in
]

% this must go after the closing bracket ] following \twocolumn[ ...

% This command actually creates the footnote in the first column
% listing the affiliations and the copyright notice.
% The command takes one argument, which is text to display at the start of the footnote.
% The \icmlEqualContribution command is standard text for equal contribution.
% Remove it (just {}) if you do not need this facility.

% \printAffiliationsAndNotice{}  % leave blank if no need to mention equal contribution
% \printAffiliationsAndNotice{\icmlEqualContribution} % otherwise use the standard text.

\renewcommand{\thefootnote}{}
\footnotetext{\textsuperscript{*}Corresponding author}
% \corref[corr]{\textsuperscript{*}Corresponding author}




\begin{abstract}
Watermarking plays a key role in the provenance and detection of AI-generated content. While existing methods prioritize robustness against real-world distortions (e.g., JPEG compression and noise addition), we reveal a fundamental tradeoff: such robust watermarks inherently improve the redundancy of detectable patterns encoded into images, creating exploitable information leakage. To leverage this, we propose an attack framework that extracts leakage of watermark patterns through multi-channel feature learning using a pre-trained vision model. Unlike prior works requiring massive data or detector access, our method achieves both forgery and detection evasion with a single watermarked image. Extensive experiments demonstrate that our method achieves a 60\% success rate gain in detection evasion and 51\% improvement in forgery accuracy compared to state-of-the-art methods while maintaining visual fidelity. Our work exposes the robustness-stealthiness paradox: current "robust" watermarks sacrifice security for distortion resistance, providing insights for future watermark design.
\end{abstract}


% This document provides a basic paper template and submission guidelines.
% Abstracts must be a single paragraph, ideally between 4--6 sentences long.
% Gross violations will trigger corrections at the camera-ready phase.

\section{Introduction}

% State of the world (robots for creative activites)
The term ``robot,'' originally signifying `forced labor,' has long been associated with labor and work. Robots have demonstrated their utility in various automated productive and social contexts, where the primary goals are improving productivity, safety, and fostering social interactions with humans~\cite{simoes2022designing, weidemann2021role, honig2018understanding}. However, an increasing number of cases feature using of robots in creative settings. Unlike productive contexts, where the focus is on efficiency and task completion~\cite{arents2022smart}, or social contexts, where communication and trust are prioritized~\cite{nam2020trust, saunderson2019robots}, creative environments prioritize artistic innovation and expression~\cite{hsueh2024counts}. This shift fundamentally alters the dynamics of human-robot interaction, redefining the roles and expectations for both humans and robots.

For instance, robots’ social behaviors are leveraged to support the generation and expression of creative ideas~\cite{hu2021exploring, sandoval2022human, alves2020creativity}, and programmable robotic movements and trajectories are employed to inspire artistic activities such as sketching~\cite{lin2020your}. These studies often engage participants from creative fields who possess limited prior experience with robotics, and are typically conducted in short-term, experimental settings. Consequently, the findings from these studies remain constrained since much can be learned from professional practitioners' experiences to inform system design such as digital fabrication~\cite{hirsch2023nothing}. There is a notable gap in research examining the long-term, active, and practical experience of integrating robotic systems into the creative processes. As a result, the deeper insights into how robots facilitate and shape creative processes, beyond simply augmenting human creativity, remain underexplored. In this study, we aim to better understand the impacts of robots on creative processes and outcomes.

As early as Leonardo da Vinci's 16th century ``Automaton,'' artists have explored the creative affordances of robotic systems~\cite{shanken2002cybernetics, pagliarini2009development, jeon2017robotic}. The artistic creation process typically encompasses various stages, including the exploration of materials and techniques, ongoing experimentation and iteration, and the continual refinement of the artists' insights into their creative subjects~\cite{lewis2023art, sturdee2022state}. Therefore, investigating the artistic process involving robots offers an opportunity to gain deeper insights into robots' creative potential. Robotic art, in particular, provides a compelling case for this exploration.

We define robotic art as artworks that utilize robotic or automated machines to create artistic experiences and tangible artifacts. One example is robotic installation art, in which robots are programmed to follow specific rules that embody the artist’s expression (\autoref{fig:teaser} (a)). Another example is responsive art, in which robots react to their environment, with behaviors that change over time or in response to spectators (\autoref{fig:teaser} (b)). Additionally, there are robotic creators, which possess a degree of agency, allowing them to collaborate with human artists and produce works that extend beyond mere replication of human-created art (\autoref{fig:teaser} (c) and (d)). As such, robotic art becomes a rich case for exploring human-machine interactions in creative contexts. Gaining a deeper understanding of how robots facilitate artistic expression can provide insights for designing computing systems to support creative activities~\cite{gomez2021robot}.

% Therefore, we did...
We draw on semi-structured, in-depth interviews with renowned professional robotic artists to investigate the use of robots in artistic practice. Specifically, our goal is to understand how artistic exploration of robotic systems challenges conventional assumptions about the functions of robots, such as their roles in automating repetitive tasks or serving human needs. We also explore the implications of robots in the artistic process and examine how creativity may emerge within robotic art. To address these interrelated inquiries, our study focuses on the practice of robotic art, posing the research question: \textit{How do robotic artists utilize robots in their artistic practice?} We approach this inquiry through the perspectives and experiences of robotic artists, who creatively design, modify, and repurpose robotic systems for artistic expression and exploration.

% The key findings are...
Our findings highlight the social, material, and temporal dimensions of artists' practices that shape their creativity and artistic outcomes. The creation of robotic art is largely a social process, as artists receive both explicit and implicit feedback through the audience's reactions and reception of their work. Simultaneously, the embodiment and malfunctions inherent to robotic systems drive artistic experimentation. The temporal processes of creation and exhibition, beyond just the final product, further enhance the creative value. Our empirical analysis presents how creativity emerges through the interplay of social, material, and temporal interactions among artists, robots, audiences, and the environment.

% The contributions of this work are...
We make two main contributions to HCI in this study. 
First, we elucidate the interactive mechanisms among key actors---human creators, machines, audiences, and environments---within the practice of robotic art, a topic that remains underexplored in HCI. Our findings reveal the significance of sociality (e.g., interactions between artists and audiences), materiality (e.g., the embodiment and malfunctions of robots), and temporality (e.g., the processes of creation and exhibition) in shaping creative values. We propose that these three facets are central to the creative process and facilitate the emergence of creativity in robotic art.
Second, drawing from the findings, we offer implications for \textit{socially informed}, \textit{material-attentive}, and \textit{process-oriented} creation with computing systems. We suggest leveraging these three aspects to enhance creativity and the creative experience. Specifically, we discuss the value of incorporating implicit audience feedback, designing with technical malfunctions, and focusing on the post-creation process to foster alternative creative experiences with machines~\cite{alter2010designing, juarez2022glitch}.



\section{Background} \label{sec:background}

% \subsection{Capture the Flag (CTF) Challenges}

% CTF challenges simulate real-world cyber-attack scenarios and have emerged as a popular medium for practical cybersecurity training, evaluation, and research. These challenges can simulate real-world attack and defense scenarios and thus assist competitors in developing practical skills in areas such as cryptography, binary exploitation, and reverse engineering. 
% Evaluation of autonomous LLM agents works best with jeopardy-style CTF challenges that focus on standalone software that must be compromised \cite{shao2024nyu,pieterse2024friend}.
% The standalone software may be a binary that can be reverse engineered or exploited, encrypted data that can be decrypted, or a web server whose authentication can be bypassed. After successfully compromising the software, a unique ``flag'' string is either found or revealed by the software server.
% The unique flag string is a concrete indicator of the success of a CTF challenge.
% Recent studies use benchmarks of CTF challenges to evaluate LLM agents on their ability to solve complex tasks and demonstrate practical skills in cybersecurity \cite{shao2024nyu,shao2024empirical,abramovich2024enigma, muzsai2024hacksynth, zhang2024cybenchframeworkevaluatingcybersecurity,yang2023language,turtayev2024hacking}
% Platforms like PicoCTF~\cite{picoctf}, TryHackMe~\cite{tryhackme}, CTFTime~\cite{ctftime} and HackTheBox~\cite{hackthebox} have popularized these formats by providing structured challenges for learners at various skill levels.

% Research indicates that CTF challenges can foster cybersecurity expertise and serve as tools for evaluating facility with cybersecurity skills~\cite{chicone2018using}. They are widely used in academia to enhance learning outcomes in cybersecurity education, with studies demonstrating their effectiveness in promoting analytical thinking and teamwork~\cite{hanafi2021ctf,leune2017using,vykopal2020benefits}. Furthermore, the integration of CTF challenges into research environments enables benchmarking of advanced AI systems like LLMs. .

% Yet, challenges in CTF design persist. These include achieving significant performance, preserving context across tasks, and handling complex, dynamic CTFs that rely on multidisciplinary approaches. Implementing strategies to address these issues enhances problem-solving efficiency, enabling more accurate, adaptive, and effective responses to evolving challenges within CTF environments.


% \subsection{Prompt Engineering}
% \subsection{Prompt Engineering for CTF}
% \subsection{LLM Agents}

% As the use of LLMs to solve CFT challenges expands, prompt engineering is becoming a critical technique for enhancing performance. Various methods have been explored to craft prompts that effectively guide LLMs to the solution of complex cybersecurity problems. Each of these solutions have their own unique strengths and limitations.
%\meet{add more references for LLM agents in other domains, like SWE-Agent, also talk about use of function calling}
Text-based LLMs take a text prompt as input from the user, and produce a text output that follows the user prompt.
LLMs have a finite length of text tokens that they can process called the context.
An alternating sequence of user prompts and LLM outputs makes a conversation and is the basis of chat-based LLM interfaces like ChatGPT.
To remove the user from the loop and create autonomous agents, a feedback mechanism is added based on the LLM outputs, so that the LLM can autonomously continue the conversation.
\citet{yang2023intercode} introduce iterative feedback prompting where the LLM is tasked with writing a piece of code, and the code's compilation and execution logs are provided as feedback, which the LLM uses to iteratively refine it's output.
Recent LLMs support function calling, a way to provide a set of actions to the LLM that it may choose to ``call'' as a function.
In this manner, LLM agents can be provided with many ``tools'' such as a command line, web search, file editing, and code execution \cite{wang2024surveyllmagents}, so that they can autonomously perform various tasks like software development \cite{yang2024sweagent}, web browsing \cite{yoran2024assistantbench}, or solve CTF challenges~\cite{shao2024nyu, abramovich2024enigma}.

With access to the command line and file editing tools, LLM agents can autonomously solve many tasks, but they still struggle on complex long-horizon tasks such as CTF challenges that require multiple steps.
Plan-and-solve prompting \cite{wang2023planandsolve} enhances long-term focus of the agent by incorporating a planning phase before iterative execution. This helps agents tackle ambiguous or complex tasks by structured strategies \cite{turtayev2024hacking}.
ReAct (reasoning + action) \cite{yao2022react} combines step-by-step reasoning with action, allowing the agent to adjust dynamically through iterative cycles. ReWOO (Reasoning without Observation) \cite{xu2023rewoo} separates the reasoning process from tool outputs and observations, allowing it to handle multi-step reasoning tasks efficiently while maintaining focus.
The prompting methods in these agents involve static hard-coded templates where environment and task information is filled in.
While static prompts provide straightforward guidance, they often fail to adapt to different problems and complex tasks, limiting their effectiveness.
Auto-prompting~\cite{shin-etal-2020-autoprompt, zhou-etal-2023-revisiting, zhang2023automatic} is a technique to allow the LLM itself to generate a highly-relevant prompt. Auto-prompting invokes more factual responses and reduces hallucinations in LLMs.
D-CIPHER incorporates auto-prompting as a separate agent that can explore the environment and generate a better prompt.
%Based on the given prompt, LLM agents make a decision and proceed further to find flags.  To address this gap, we propose \textbf{dynamic prompting}, where the LLM agent autonomously generates prompts based on the CTF challenge's context and stage.
%include a static template which needs to be given to LLM to solve the CTF challenges. For instance, the NYU CTF framework provides a static prompt as \emph{``Please proceed to the next step using your best judgment"} for each decision making point. 

% To address this gap, we introduce a novel approach where the LLM agent generates the next prompt autonomously based on the current context and stage of the CTF challenge, a technique we call \textbf{dynamic prompting}.


Expanding on single LLM agents, multi-agent LLM systems are a powerful approach to enhance problem-solving by simulating team-based collaboration. Specialized agents, each with distinct objectives, work together to tackle different aspects of complex tasks \cite{guo2024largelanguagemodelbased}
Multi-agent systems are effective in cybersecurity applications. For instance, Audit-LLM~\cite{song2024audit} deploys a  multi-agent system for insider threat detection by employing agents to decompose tasks, build tools, and use collaborative reasoning to enhance detection accuracy. Liu~\cite{liu2024multi} explores multi-agent systems to enhance incident response in cybersecurity by examining centralized, decentralized, and hybrid team structures to assess how LLM agents can improve decision-making, adaptability, and coordination during cyber-attack scenarios. AutoSafeCoder~\cite{nunez2024autosafecoder} enhances the security of code generated by LLMs by incorporating a coding agent for code generation, a static analyzer agent that identifies vulnerabilities, and a fuzz testing agent for dynamic testing to detect runtime errors. Division of responsibilities among different agents allows AutoSafeCoder to produce secure, functionally correct code. 

% With the growing use of LLMs in CTF challenges, prompt engineering is key to enhancing performance. Various methods guide LLMs in solving complex cybersecurity tasks, each with distinct strengths and limitations.

% \textbf{Single Turn (Zero-Shot Prompting)} involves providing the model with a one-time task description that outputs  an immediate solution. This is efficient for straightforward tasks~\cite{yang2023intercode}. In contrast, \textbf{Try Again (Iterative Feedback Prompting)} uses iterative feedback to refine responses over multiple attempts, mimicking real-world problem-solving~\cite{yang2023intercode}. The \textbf{Plan \& Solve} enhances adaptability by incorporating a planning phase before iterative execution. This helps models tackle ambiguous or complex tasks by  structured strategies~\cite{turtayev2024hacking}. Additionally, \textbf{ReAct (Reasoning + Action)} combines step-by-step reasoning with action, allowing the model to adjust dynamically through iterative cycles. This makes it particularly effective for evolving and complex challenges like CTFs~\cite{yao2023react}. 
% These prompting techniques highlight diverse approaches to optimizing LLM performance in cybersecurity tasks. 

% Multi-agents!


%\meet{Add references for auto-prompting, and shorten this para}
%\nanda{Maybe we can add this to previous paragraphs which discusses other prompting methods such as plan-and-solve and ReAct method}
% All of these prompting methods include a static template which needs to be given to LLM to solve the CTF challenges. For instance, the NYU CTF framework provides a static prompt as \emph{``Please proceed to the next step using your best judgment"} for each decision making point. 
% Based on the given prompt, LLM agents make a decision and proceed further to find flags. While static prompts provide straightforward guidance, they often fail to account for the evolving nature of complex tasks, limiting their effectiveness in multi-step or ambiguous CTF challenges. To address this gap, we propose \textbf{dynamic prompting}, where the LLM agent autonomously generates prompts based on the CTF challenge's context and stage.
% % To address this gap, we introduce a novel approach where the LLM agent generates the next prompt autonomously based on the current context and stage of the CTF challenge, a technique we call \textbf{dynamic prompting}.
% Dynamic prompting adapts instructions to task progress, ensuring instructions are contextually relevant and reflective of the specific obstacles encountered. By iterating based on feedback and intermediate outputs, it continuously refines the LLM’s approach, enhancing problem-solving for dynamic tasks like CTFs.
% This adaptive process not only mirrors how humans tackle complex problems but also improves the model’s ability to handle unpredictable scenarios, making it particularly advantageous for cybersecurity tasks like CTFs where conditions change dynamically.


% The very first prompt type used in several applications is \textbf{Single Turn (Zero-Shot Prompting)}~\cite{yang2023intercode}. In single-turn prompting, the model receives a one-time, straightforward task description and is expected to generate a complete response without further interaction. The initial output is directly assessed, making this approach efficient for tasks where minimal feedback or iteration is required. This method tests the model’s ability to understand and respond to tasks immediately, relying heavily on the model's pre-trained knowledge and generalization capabilities.

% Along with this, The prompting method named \textbf{Try Again (Iterative Feedback Prompting)}~\cite{yang2023intercode} has been also used in several appreciations specially to solve CTF challenges. It is an iterative prompting method involves continuous interaction, where the model is provided with feedback after each attempt. The model can refine its responses over multiple turns based on the observations or execution results from previous outputs. This iterative process continues until the task is successfully completed or a maximum number of interactions is reached. This approach closely mirrors real-world problem-solving, where adjustments are made iteratively based on evolving circumstances or feedback.

% Some application are also using \textbf{Plan \& Solve}~\cite{turtayev2024hacking} prompting method which enhances problem-solving by dividing the process into a planning phase followed by execution. Initially, the model formulates a strategy based on the task description and available information, allowing for a structured approach to ambiguous or complex problems. This plan guides the subsequent execution phase, where the model carries out actions iteratively, refining its approach based on feedback. In more challenging scenarios, re-planning mid-task further improves adaptability and performance. This method proves effective in tasks like CTF challenges, where vague instructions require careful analysis and step-by-step resolution.

% Further some application are also using \textbf{ReAct (Reasoning + Action)}~\cite{yao2023react} prompting method blends reasoning with action by guiding the model to think through tasks step-by-step before executing actions. At each step, the model generates a thought based on the task and observations, which informs the next action. The action is executed, and the resulting feedback refines the model’s understanding for the next cycle. This continuous process helps the model adapt dynamically to complex tasks, making it effective for CTF challenges where logical reasoning and step-by-step execution are essential.

\section{Related Works} \label{sec:related_work}


\begin{table}[htpb]
    \centering
    \caption{Feature comparison of LLM agents for solving CTFs.}
    \label{tab:related_work_comparison}
    \begin{tabular}{lcccccc}
    \toprule
         \textbf{Study} & \rotatebox{90}{\textbf{\# CTFs}} & \rotatebox{90}{\textbf{Open bench}} & \rotatebox{90}{\textbf{Tool use}}  & \rotatebox{90}{\textbf{Autonomous}} & \rotatebox{90}{\textbf{Multi-agent}} &\rotatebox{90}{\textbf{Auto-prompt}} \\
    \cmidrule{2-7}
     % \textbf{Study} & \textbf{Dynamic} & \textbf{Used} & \textbf{Multi-} & \textbf{Automatic} & \textbf{Tool} & \textbf{\# of} \\
         Tann et al. \cite{tann2023using} &  $7$ & \purplecross & \purplecross & \purplecross & \purplecross & \purplecross  \\
         Shao et al. \cite{shao2024empirical} & $26$ & \purplecross & \tealcheck & \tealcheck & \purplecross & \purplecross  \\
         InterCode-CTF\cite{yang2023language} & $100$ & \tealcheck & \tealcheck & \tealcheck & \purplecross & \purplecross   \\
         NYU CTF Bench \cite{shao2024nyu} & $200$ & \tealcheck & \tealcheck & \tealcheck & \purplecross & \purplecross \\
         Turtayev et al. \cite{turtayev2024hacking} & $100$ & \tealcheck & \tealcheck & \tealcheck & \purplecross & \purplecross\\
         Cybench \cite{zhang2024cybenchframeworkevaluatingcybersecurity} & $40$ & \tealcheck & \tealcheck & \tealcheck & \purplecross & \purplecross \\
         EnIGMA \cite{abramovich2024enigma} & $350$ & \tealcheck & \tealcheck & \tealcheck & \purplecross & \purplecross\\
         HackSynth \cite{muzsai2024hacksynth} & $200$ & \tealcheck & \tealcheck & \tealcheck & \tealcheck & \purplecross \\
         \textbf{D-CIPHER (ours)} & $290$ & \tealcheck & \tealcheck & \tealcheck & \tealcheck & \tealcheck \\
    \bottomrule
    \end{tabular}
\end{table}



% \subsection{LLMs on Cybersecurity}
% \subsection{LLM Agents for CTF}

%LLMs have a vast knowledge base that can be tapped for cybersecurity use.
Tann et al.~\cite{tann2023using} evaluate early LLMs such as ChatGPT and Google Bard in solving CTF challenges and answering professional certification questions, showing that LLM responses contain key task information.
%Many works extend the LLM capabilities by providing them access to programming and command execution tools, to form autonomous agents. 
The InterCode-CTF agent~\cite{yang2023intercode} reveals that LLM agents demonstrate basic cybersecurity skills, however they struggle with more complex tasks.
The NYU CTF baseline agent~\cite{shao2024empirical} integrates external tools into the LLM's function-calling features and demonstrate improved potential of tool-assisted LLMs to solve CTFs, however it exhausts the LLM context length when command output history becomes very long. InterCode-CTF manages this issue by truncating the history to only show the LLM the last few iterations. Even so, LLM agents face issues with longer tasks.
%NYU CTF Bench~\cite{shao2024nyu}, a benchmark of 200 CTF challenges, presents a baseline agent with specialized reverse engineering tools and category-specific prompts, demonstrating their importance to solve CTFs.
% The NYU CTF baseline agent faces issues of LLM context length when complex tasks run for several iterations and the entire command and output history becomes longer than the LLM's context window size. The InterCode agent manages this issue by truncating the history to only show the LLM the last few iterations.


Excessive tool availability and verbose interfaces can overwhelm agents, leading to inefficiencies. Agents perform better with a focused set of tools with well-defined interfaces~\cite{yang2024sweagent}.
EnIGMA~\cite{abramovich2024enigma} agent incorporates interactive tools and in-context learning techniques to achieve state-of-the-art results. % on the NYU CTF Bench, HackTheBox, and Cybench benchmarks.
For better context management, EnIGMA also uses an LLM summarizer that summarizes the command outputs for the main agent.

HackSynth~\cite{muzsai2024hacksynth}, an LLM agent for autonomous penetration testing, shows that iterative planning and feedback summarization stages help the agent finish multiple tasks and improves overall problem solving.
Similarly, Cybench~\cite{zhang2024cybenchframeworkevaluatingcybersecurity} introduces a benchmark of 40 CTF challenges augmented with step-by-step tasks, demonstrating better focus of LLM agents on smaller tasks, leading to improved success and alleviating the context length issue.
\citet{turtayev2024hacking} expand on InterCode-CTF by implementing plan-and-solve prompting, achieve significant improvement on the InterCode-CTF benchmark. They show that prompting techniques can improve performance even with simple toolsets.
% . Their baseline agent is evaluated in unguided mode (i.e. fully autonomous), and guided mode where the agent is given one task at a time. Their results indicate that providing smaller tasks to the LLM agents improve their focus yielding improved success on complex challenges while .

These works highlight that LLM agents excel at implementing code and executing commands to accomplish small concrete tasks when provided with dynamic feedback and task-specific toolsets. While these works  involved using multiple LLMs with different tasks such as planning and summarizing along-side a main agent, D-CIPHER is the first work to formulate a multi-agent system where there is a bifurcation of responsibilities between agents and meaningful well-defined interactions for dynamic feedback.
Table~\ref{tab:related_work_comparison} shows a feature comparison of D-CIPHER with related works on LLM agents for autonomous CTF solving.
%\meet{some description of the feature comparison?}
% Recent research has focused on enable autonomous solving of CTF challenges~\cite{shao2024empirical,shao2024nyu,abramovich2024enigma}. These agents typically operate in containerized environments to ensure reproducibility and modularity. 

% As an early effort, Tann et al.~\cite{tann2023using} evaluated the effectiveness of LLMs, such as OpenAI's ChatGPT, Google Bard, and Microsoft Bing, in solving cybersecurity CTF challenges and answering professional certification questions. 
% % Their study results show that LLMs performed well on $7$ CTF test cases, with ChatGPT solving $6$, Bard $2$, and Bing $1$. 
% The study shows that LLM responses often contain key information essential for solving tasks.

% The InterCode framework~\cite{yang2023intercode} approaches coding as an interactive process and uses execution feedback to improve code generation. As described in Yang et al.~\cite{yang2023intercode}, InterCode-CTF integrates CTF benchmarks into a reinforcement learning environment that can evaluate the cybersecurity capabilities of language agents. It features $100$ tasks that tapskills such as reverse engineering, forensics, and binary exploitation. While existing language agents demonstrate basic cybersecurity skills, evaluations indicate they struggle with more complicated complex tasks unless the system is fine-tuned or given external support. 
% cite Intercode: Standardizing and benchmarking interactive coding with execution feedback

% Another notable example is an LM agent developed by Shao et al. specifically to automate CTF tasks. 
% Shao et al.~\cite{shao2024empirical} developed a LM agent to automate CTF tasks.
% % They report an accuracy rate of  $46\%$ on $26$ CTF challenges sourced from CSAW'23 Qualifying round competition using GPT-4.
% By effectively combining LLM capabilities with external tools, the researchers demonstrated the potential of tool-assisted LLMs to solve complex problems. Building on this, the team incorporated a broader range of cybersecurity tools and interfaces that enhance both accuracy and versatility. 
% Empirical results show their system outperforms baselines on both the InterCode CTF benchmark and the NYU CTF benchmark.

% Shao et al.~\cite{shao2024nyu} presented a diverse, open-source database of CTF challenges that can be used to benchmark an LLM's ability to solve cybersecurity problems.
% It provides a scalable platform for developing and testing AI-driven approaches for vulnerability detection and resolution, facilitating advancements in automated cybersecurity tasks. The benchmark database and automated framework were successfully applied to the performance of five LLMs. 

% The Cybench benchmark~\cite{zhang2024cybenchframeworkevaluatingcybersecurity} provides another significant contribution by creating a framework tailored to solving CTF challenges. % Cybench: A framework for evaluating cybersecurity capabilities and risk
% % Their benchmark environment achieves an accuracy of $17.5\%$ using Claude 3.5 Sonnet. 
% Such frameworks operate in Linux-based containerized environments, such as Kali Linux, which includes pre-installed cybersecurity tools. However, excessive tool availability can overwhelm agents, leading to inefficiencies. Research indicates that agents perform better with a focused set of tools that have well-defined interfaces~\cite{yang2024sweagent}. % Swe-agent: Agent-computer interfaces enable automated software engineering



% Muzsai et al. introduced HackSynth~\cite{muzsai2024hacksynth}, an LLM-based agent for autonomous penetration testing. It uses a dual-module architecture that consists of a Planner and a Summarizer, allowing for iterative command generation and feedback processing. The framework is evaluated using two benchmark sets from platforms like PicoCTF~\cite{picoctf} and OverTheWire~\cite{overthewire}. These benchmarks address $200$ challenges drawn from various domains and difficulty levels. Results of their study show that HackSynth, especially with the GPT-4o model, achieves the best performance. This highlights the potential of LLM-based agents in advancing autonomous penetration testing.
 % Using basic prompting techniques and expanding tool availability, the study highlights how straightforward approaches can unlock the latent potential of LLMs for cybersecurity tasks. Their work emphasizes that simple LLM designs can effectively solve CTF challenges, and thus broaden the number of cybersecurity applications without the need for advanced engineering.

% \begin{table*}[]
%     \centering
%     \begin{tabular}{|c|c|>{\centering\arraybackslash}p{4.5cm}|c|c|c|c|c|c|}
%     \hline
%          \textbf{Study} & \textbf{Dynamic} & \textbf{Used} & \textbf{Multi-} & \textbf{Open} & \textbf{Automatic} & \textbf{Tool} & \textbf{\# of} & \textbf{\# of} \\
%          & \textbf{Prompt} & \textbf{Benchmarks} & \textbf{Agents} & \textbf{Dataset} & \textbf{Framework} & \textbf{Use} & \textbf{LLMs} & \textbf{CTFs}\\
%          \hline
%          Tann et al.~\cite{tann2023using} & \purplecross & Manual collected & \purplecross & \purplecross & \purplecross & \purplecross & $3$ & $7$ \\
%          \hline
%          InterCode-CTF~\cite{yang2023language} & \purplecross &  PicoCTF~\cite{picoctf} & \purplecross & \purplecross& \purplecross & \purplecross & $1$ & $100$  \\
%          \hline
%          Shao et al.~\cite{shao2024empirical} & \purplecross & CSAW 2023 & \purplecross & \purplecross & \tealcheck & \tealcheck & $4$ & $26$ \\
%          \hline
%          Shao et al.~\cite{shao2024nyu} & \purplecross & NYU CTF~\cite{shao2024nyu} & \purplecross & \tealcheck & \tealcheck & \tealcheck & $5$ & $200$ \\
%          \hline
%          Cybench~\cite{zhang2024cybenchframeworkevaluatingcybersecurity} & \purplecross & Cybench~\cite{zhang2024cybenchframeworkevaluatingcybersecurity}  & \purplecross & \tealcheck & \tealcheck & & $8$ & $40$ \\
%          \hline
%          EnIGMA~\cite{abramovich2024enigma} & \purplecross & NYU CTF~\cite{shao2024nyu}, InterCode-CTF~\cite{yang2023language},  HackTheBox~\cite{hackthebox} & \purplecross & \purplecross & \tealcheck & \tealcheck & $3$ & $350$ \\
%          \hline
%          HackSynth~\cite{muzsai2024hacksynth} & \purplecross & PicoCTF~\cite{picoctf}, OverTheWire~\cite{overthewire} & \tealcheck & \tealcheck & \tealcheck & \tealcheck & $8$ & $200$ \\
%          \hline
%          Turtayev et al.~\cite{turtayev2024hacking} & \purplecross & InterCode-CTF~\cite{yang2023language} & \purplecross & \purplecross & \purplecross & \purplecross & $4$ & $100$ \\
%          \hline
%          \textbf{D-CIPHER (Proposed)} & \tealcheck & NYU CTF~\cite{shao2024nyu}, Cybench \cite{zhang2024cybenchframeworkevaluatingcybersecurity}, HackTheBox \cite{hackthebox} & \tealcheck & \tealcheck & \tealcheck & \tealcheck & 5 & 290 \\
%          \hline
%     \end{tabular}
%     \caption{Comparison with LLM-based CTF solving Literature}
%     \label{tab:related_work_comparison}
% \end{table*}




% \subsection{Multi-agent framework}

% The use of multi-agent LLM systems in Capture the Flag (CTF) challenges is emerging as a powerful approach to enhance cybersecurity problem-solving. Multi-agent frameworks mimic team-based collaboration, where multiple LLM agents, each with specialized expertise, work together to tackle complex tasks. This approach reflects real-world cybersecurity operations, where success often depends on coordinated efforts from teams with diverse skills, each addressing different components of a security challenge.
% Multi-agent LLM systems are emerging as a powerful approach to enhance cybersecurity problem-solving by simulating team-based collaboration. Specialized agents, each with distinct objectives, work together to tackle different aspects of complex security tasks. This mirrors real-world cybersecurity operations, where coordinated efforts and diverse skills are essential for addressing evolving threats and vulnerabilities.

% CTF challenges cover a wide range of domains, including cryptography, reverse engineering, forensics, and web exploitation. Multi-agent systems can distribute the workload by assigning agents to handle specific tasks. This enables parallel problem-solving and emulates the collaborative nature of human teams. For example, one agent may specialize in guiding the fellow agents to what needs to be done, while another executes the instructions, ensuring that tasks are addressed without losing the context, and implementing reasoning from multiple LLMs. This division of labor boosts efficiency and enables problem-solving from multiple perspectives.
% This division of labor enhances efficiency and allows the system to approach problems from multiple perspectives, reflecting the interdisciplinary approach often used in cybersecurity teams.

% Guo et al.~\cite{guo2024largelanguagemodelbased} highlight the strengths of multi-agent LLMs in complex, multi-step tasks where different agents handle specific roles The framework HackSynth~\cite{muzsai2024hacksynth} is a multi-agent penetration testing framework in which agents operate collaboratively to address vulnerabilities in staged environments. Their work emphasizes that when agents work as a cohesive team, they outperform single-agent approaches. This is particularly true when facing layered, iterative challenges. 
% This team-based model of problem-solving aligns closely with how cybersecurity professionals approach real-world security incidents and penetration testing exercises.

% Multi-agent LLM systems have shown effectiveness in various other applications. For instance,  Audit-LLM~\cite{song2024audit} presents a multi-agent framework for insider threat detection using log analysis. It employs agents to decompose tasks, build tools, and use collaborative reasoning to enhance detection accuracy. Liu~\cite{liu2024multi} explores the application of LLM-based multi-agent systems to enhance incident response (IR) in cybersecurity. Utilizing the ``Backdoors \& Breaches" tabletop game as a simulation environment, the study examines centralized, decentralized, and hybrid team structures to assess how LLM agents can improve decision-making, adaptability, and coordination during cyberattack scenarios. AutoSafeCoder~\cite{nunez2024autosafecoder} is a multi-agent system designed to enhance the security of code generated by LLMs. The framework comprises three agents: a Coding Agent responsible for code generation, a Static Analyzer Agent that identifies vulnerabilities through static analysis, and a Fuzzing Agent that performs dynamic testing using mutation-based fuzzing to detect runtime errors. By integrating both static and dynamic testing in an iterative process, AutoSafeCoder aims to produce secure, functionally correct code. 

% To enhance CTF-solving by promoting team-based specialization, we employ a multi-agent CTF solving agent. Within this framework, agents tackle tasks aligned with their strengths. Tasks are executed in parallel, improving efficiency and accelerating progress. Agents share insights, adapt refining strategies based on feedback, and overcome obstacles collectively. This collaborative approach boosts scalability, adaptability, and and resilience, and improves performance in complex challenges.

% This paper presents a comprehensive comparison of D-CIPHER with existing LLM-based CTF-solving literature, as shown in Table~\ref{tab:related_work_comparison}.
% This paper documents the results of  our comprehensive comparison of D-CIPHER with existing LLM-based CTF-solving literature. These results are presented in Table~\ref{tab:related_work_comparison}.
\section{Problem Formulation}
% \begin{figure}[!t]
%     \centering
%     {\small \textbf{Watermark Forgery Attack}} \\[1mm]  % Custom font size
%     \includegraphics[width=\linewidth]{pic/intro_1.png} 
%     \label{fig:intro}
%     \vspace{-6mm}
%     \caption{Bob utilizes the GenAI service provided by Alice, where Alice embeds watermark information into the images returned to Bob. This embedded watermark allows the image to be identified as having been generated either by Bob or Alice through a watermark detection service. By forging the watermark onto illegal or malicious content, the attacker can cause the image to be misidentified as having been generated by Bob or Alice, thereby damaging the reputation of legitimate users or service providers. } 
% \end{figure}

\begin{figure}[!t]
    \centering
    % Custom font size
    \includegraphics[width=\linewidth]{pics/system-threatmodel.png} 
    
    % \vspace{-3mm}
    \caption{Typical watermarking application and security threats. Organizations and individuals use watermarking services to embed watermarks into images for purposes such as copyright protection or content regulation. When image ownership verification is required, the watermark is extracted and matched through the watermarking service. However, attackers can apply carefully designed post-processing techniques to remove or forge the watermark.} 
    \label{fig:models}
    % \vspace{-6mm}
\end{figure}

\subsection{System Model}
Figure~\ref{fig:models} illustrates the use case of a typical watermarking system. The process consists of the stages of watermark injection (encoding), data circulation, and watermark extraction (decoding), as shown in the gray portion of Figure~\ref{fig:models}. We primarily consider the post-processing watermarks. The three parties involved include \emph{users/organizations}, \emph{the verifier}, and the\emph{the attacker}.

% Watermarked images circulate through online platforms such as social network websites and forums, enabling access by users. 

\textbf{Users/service providers.} Users would like to use watermarking service before posting images online via social platforms to protect copyright. Alternatively, a service provider wants to mark all imagery generated by its own products, ensuring content provenance.

\textbf{The verifier.} To verify if an image contains the watermark, the verifier downloads target images from the Internet, decodes the image to extract watermark information, and then verifies the extracted one with the identification information. 

\subsection{Attacker's Goals} An attacker has two types of objectives. First, he would like to use an image without attributing it to the creator; therefore, he needs to evade the detection of watermarks. Second, he would like to improve the credibility of a fake image; therefore, he needs to 
forge a watermark related to an official account.

\subsection{Attacker's Capability} The attacker can download watermarked images uploaded by the victim, perform watermark removal or watermark spoofing on a clean image. Notably, we assume three realistic limitations: 1) The attacker \textbf{neither have knowledge} about the target watermarking system (i.e., encoder and decoder), \textbf{nor can he query the system}; 2) The attacker cannot obtain the original image; 3) The attacker must tackle watermark methods that are robust against distortions.





\section{DAPAO Attacks}
In this section, first, we present the feasibility study, demonstrating our observation of information leakage in robust watermarks. Next, we provide the theoretical analysis for method validation. Last, we introduce evasion and forgery attacks based on the observation.

% We empirically discover that learning-based watermarking systems mitigate distortion effects (e.g., compression) by expanding the regions where the watermark pattern is embedded or increasing its magnitude, ensuring the remaining watermark remains detectable. Besides the encoding part, the system trains the watermark decoder to extract watermarks more effectively, which can be understood as increasing the model's attention weight on watermark signals.

\subsection{Feasibility Study}\label{sec:pilot}
We empirically find that learning-based robust watermarking systems counteract distortion effects (e.g., compression) by expanding the regions where the watermark pattern is embedded or amplifying its magnitude, ensuring that the watermark remains detectable. Beyond the encoding process, these systems also train the watermark decoder to enhance extraction effectiveness, effectively increasing the model's attention to watermark signals.

We conduct a feasibility study to explore: \emph{If the strengthened watermark results in leakage that can be captured from images using a feature extraction network?} We embed watermarks in multiple images with the same robust watermarking algorithm and then input these watermarked images into a feature extraction network.

% figure
\begin{figure}[!t]
    \centering
    % Custom font size
    \includegraphics[width=\linewidth]{pics/feasiblity.png}   
    \vspace{-6mm}
    \caption{Demonstration of our feasibility study.}
    \label{fig:feasibility}
    \vspace{-3mm}
\end{figure}

As shown in Figure ~\ref{fig:feasibility}, we found that:
\begin{itemize}
    \item The multi-channel features obtained after feature extraction can capture patterns not easily noticeable by the human eye.
    \item These patterns are similar across different images.
    \item Not all features contain such leakage information.
\end{itemize}
% Based on the above experimental observations, we \underline{\textbf{D}}elve into the \underline{\textbf{A}}spect of the \underline{\textbf{PA}}radox \underline{\textbf{O}}f Robust Watermarks and propose the \textbf{DAPAO} attack.

The results shed light on learning watermark characteristics from distinguished patterns probably related to the watermark.

\begin{figure}[!t]
    \centering
    % Custom font size
    \includegraphics[width=\linewidth]{pics/overview.png} 
    
    % \vspace{-3mm}
    \caption{An overview of our attack.} 
    \label{fig:method-overview}
    \vspace{-3mm}
\end{figure}

\subsection{Robustness and Invisibility Trade-off}\label{sec:Method_Theory}
% \begin{definition}
% \label{def:inj}
% A encoder $\mathcal{E}:\mathcal{I} \times W \to Y$ is injective if for any $x,y\in X$ different, $f(x)\ne f(y)$.
% \end{definition}
As mentioned earlier Sec.~\ref{sec:background}, a complete watermarking framework can be divided into three components: encoder $\mathcal{E}$, decoder $\mathcal{D}$, and distortion layer $\mathcal{T}$. The decoder takes only a single watermarked image $I_{wm}$ as input. To achieve correct verification, the decoder must implicitly disentangle the image content from the embedded watermark information and correctly associate them to extract the watermark successfully.



% the decoder must implicitly decompose the watermarked image into image information and watermark information, matching the two to successfully extract the watermark information.

\begin{definition}
An image and watermark information: $I$, $wm \ \subset \{0,1\}^k$, the encoder is:
$$\mathcal{E}(I, wm)=I+\epsilon \cdot \underbrace{\phi(I,wm)}_W$$
the decoder is:
$$\mathcal{D}(I_{wm}) \to \underbrace{(\hat{I}, \hat{W})}_{{match}} \to \hat{wm}$$



$\epsilon$ is the embedding strength.The feature space of the image $\mathcal{P} = \{p_1, p_2,...,p_n\}$ consists of two subspaces for embedding information: 
$$\mathcal{P} = \mathcal{P}_r \bigoplus \mathcal{P}_c$$

Due to joint training, the encoder exhibits a similar implicit decomposition behavior, projecting the input image $I$ into two feature spaces, named as $P_r$ and $P_c$. The former is a more suitable embedding space for information hiding, while the latter is not. 

The encoder performs this mapping $\mathcal{E}(I,wm) \to I_{wm}$ by:
$$\phi(I,wm) = \mathop{\min}_{p\in \mathcal{P}_r}||wm - \mathcal{D}(\mathcal{E}(p,wm))||^2+\lambda||\mathcal{E}(p,wm)||$$

However, as robustness requirements are introduced and continuously strengthened, the encoder must encode more information to ensure the watermark’s resistance to attacks. When the $P_r$   space is fully utilized, the encoder is forced to use $P_c$ for watermark embedding, polluting the $P_c$ space.

\end{definition}

\begin{definition}
An intuitive definition of embeddable threshold is:
\begin{gather*}
C(I) = \sup_{W \in \mathcal{P}_r}{\frac{||W||_2}{||I||_2}} \\\\
s.t. PNSR(I, I+W) \ge TV
\end{gather*}
$TV$ represents the lower bound of the visual quality.
\end{definition}

\begin{proposition}
When the robustness requirement exceeds $C(I)$, a decline in visual quality is inevitable.
\end{proposition}

\begin{proof}
Let the distortion layer $\mathcal{T}$ introduce noise $\eta \sim \mathcal{T}$, with the requirement that
$$||wm-\mathcal{D}(I_{wm} + \eta)|| \le \mathcal{B}$$
$\mathcal{B}$ is the bit error rate. Considering the channel capacity as:
$$R=\frac{1}{2}\log(1+\frac{\epsilon^2||W||^2}{\delta_{\eta}^2})$$
% \vspace{-3mm}
To achieve $R\ge H(wm)$, the following conditions must be met:
$$
\epsilon||W|| \le \sqrt{(2^{2H(wm)}-1)\delta_{\eta^2}}
$$
$H(wm)$ represents the entropy of $wm$. 

When $\sqrt{(2^{2H(wm)}-1)\delta_{\eta^2}} > C(I)||I||_2$, the system cannot simultaneously satisfy both, and it is necessary to increase $C(I)$, introducing visual artifacts into the image. Detailed proof is provided in Appendix~\ref{sec:Appendix_Proofs}.
\end{proof}
% \vspace{-3mm}
The artifacts introduced by sacrificing invisibility contain watermark information, creating a security vulnerability where watermark information leakage occurs.

%\hl{lack of proof of artifacts contain watermark information }
% This, however, compromises visual quality, leading to more detectable visual artifacts. Moreover, these artifacts also contain watermark information, creating a security vulnerability where watermark information leakage occurs.

\subsection{Detection Evasion}\label{sec:Method_Evasion Attack}
Our method is illustrated in Figure~\ref{fig:method-overview}, Suppose we have an image $I_{wm}$, embedded with an unknown watermark $wm$. This image is fed into a feature extraction module $\mathcal{F}(\cdot)$, resulting in multi-channel features $\mathcal{F}(I_{wm})$. To automate the selection of features that capture potential information leakage, we perform clustering on the multi-channel features. Among the resulting clusters, we identify the two clusters with the smallest number of samples and extract their corresponding feature channel positions $\mathcal{W}$.

To achieve the goal of an evasion attack, we need to disrupt the leaked watermark information captured from $I_{wm}$.We formulate this process as an optimization problem: finding a perturbation $\delta$ that disrupts the leaked information while preserving the visual quality of the image. The formulation is as follows:
\begin{equation}
\label{eq:1}
\begin{split}
    \mathop{\min}_{\delta}-\mathcal{L}(\mathcal{W} \cdot \mathcal{F}(I_{wm}), \mathcal{W}\cdot \mathcal{F}(I_{wm} + \delta)) \\
    \mathrm{ s.t.} ||\delta||_{\infty} < \epsilon
\end{split}
\end{equation}

where $\mathcal{L}(\cdot,\cdot)$ is the loss function that measures the distance between two features, and $\epsilon$ is a perturbation budget.

We use Projected Gradient Descent (PGD)~\cite{PGD} to solve the optimization problem in Eq~\ref{eq:1}. Finally, we complete the attack through $I_{wm} + \delta$.

Our detailed algorithm is shown as 
 Algorithm~\ref{alg:evasion algo}.
 %in Appendix~\ref{sec:Appendix_Implementation Details}.

 % Similar to Sec~\ref{sec:Method_Evasion Attack}, as shown in Figure ~\ref{fig:method-overview},

\subsection{Forgery Attack}
As shown in Figure~\ref{fig:method-overview}, we first use the feature extraction module and clustering algorithm to extract features containing leaked watermark information, from $I_{wm}$. To achieve the goal of spoofing, we still need to extract the leaked information. Therefore, this process can be formulated as the following optimization problem:
\begin{equation}
\label{eq:2}
\begin{split}
     \mathop{\min}_{\delta}-\mathcal{L}(\mathcal{W} \cdot \mathcal{F}(I_{wm}), \mathcal{W}\cdot \mathcal{F}(I_{wm} + \delta)) \\
    \mathrm{ s.t.} ||\delta||_{\infty} < \epsilon
\end{split}
\end{equation}
\vspace{-4mm}

where $\epsilon$ is a perturbation budget, and 
 this process is identical to the above evasion attack, referred to as Stage \uppercase\expandafter{\romannumeral1}.
However, the learned $\delta$ alone cannot fulfill the forgery purpose for \emph{semantic watermarking}. Based on the theory discussed earlier (See Sec.~\ref{sec:Method_Theory}), we need to consider the coupling effect between the semantics and watermark. After the optimization in Eq~\ref{eq:2} is completed, an additional optimization term should be included to further find another perturbation, $\delta_s$, which can be described as:
\begin{equation}
\label{eq:3}
\begin{split}
     \mathop{\min}_{\delta}\mathcal{L}((1-\mathcal{W}) \cdot \mathcal{F}(I_{wm}+\delta), (1-\mathcal{W})\cdot \mathcal{F}(I' + \delta_s)) \\
    \mathrm{ s.t.} ||\delta_s||_{\infty} < \epsilon
\end{split}
\end{equation}
This process is referred to as Stage \uppercase\expandafter{\romannumeral2}.
We use Projected Gradient Descent (PGD)~\cite{PGD} to solve the optimization problem in Eq~\ref{eq:2} and Eq~\ref{eq:3}.
Finally, we complete the attack through $\{I' - \delta\}$ or $\{I' - \delta + \delta_s \}$.

Our detailed algorithm is shown as Algorithm~\ref{alg:spoof algo}
%in Appendix~\ref{sec:Appendix_Implementation Details}.
\section{Experiments and Results}
\subsection{Experiment Settings}
% \begin{table*}[h]
%     \centering
%     \begin{tabular}{cl|ccccc|ccccc}
%      \multirow{3}{*}{\textbf{LLM}}  & \multirow{3}{*}{\textbf{Method}} &  \multicolumn{5}{c|}{\textbf{CCNews}} & \multicolumn{5}{c}{\textbf{Wikipedia}} \\ \cmidrule(lr){3-7}  \cmidrule(lr){8-12}
%       &  & PPL & Loss & Ref & min-k & \multicolumn{1}{c|}{zlib} & PPL & Loss & Ref & min-k & zlib \\ \midrule
%       \multirow{4}{*}{GPT2} & \textit{Base} & \textit{29.442} & \textit{0.505} & \textit{0.498} & \textit{0.520} & \textit{0.500} & \textit{34.429} & \textit{0.473} & \textit{0.513} & \textit{0.446} & \textit{0.497} \\ 
%       \multirow{4}{*}{124M} & FT & \textbf{21.861} & 0.607 & 0.855 & 0.549 & 0.569 & \textbf{12.729} & 0.577 & 0.967 & 0.489 & 0.544 \\
%       & Goldfish & 21.902 & 0.608 & 0.855 & 0.547 & 0.570 & 12.853 & 0.565 & 0.954 & 0.486 & 0.537 \\
%       & DPSGD & 26.022 & 0.507 & 0.513 & \textbf{0.521} & 0.502 & 18.523 & 0.463 & 0.536 & \textbf{0.448} & 0.491 \\
%       & \methodname & 23.733 & \textbf{0.502} & \textbf{0.495} & 0.529 & \textbf{0.499} & 13.628 & \textbf{0.454} & \textbf{0.463} & 0.470 & \textbf{0.485} \\ \midrule
      
%       \multirow{4}{*}{Pythia} & \textit{Base} & \textit{13.973} & \textit{0.507} & \textit{0.512} & \textit{0.528} & \textit{0.501} & \textit{10.287} & \textit{0.466} & \textit{0.503} & \textit{0.464} & \textit{0.489}\\ 
%       \multirow{4}{*}{1.4B} & FT & 11.922 & 0.602 & 0.857 & 0.541 & 0.574 & \textbf{6.439} & 0.578 & 0.985 & 0.484 & 0.557 \\
%       & Goldfish & \textbf{11.903} & 0.609 & 0.862 & 0.543 & 0.579 & 6.465 & 0.564 & 0.981 & 0.482 & 0.546 \\
%       & DPSGD & 13.286 & 0.512 & 0.531 & 0.528 & 0.503 & 7.751 & 0.469 & 0.524 & \textbf{0.462} & 0.488 \\
%       & \methodname & 12.670 & \textbf{0.501} & \textbf{0.460} & \textbf{0.524} & \textbf{0.499} & 6.553 & \textbf{0.468} & \textbf{0.485} & 0.472 & \textbf{0.485} \\ \midrule
      
%       \multirow{4}{*}{Llama-2} & \textit{Base} & \textit{9.364} & \textit{0.505} & \textit{0.495} & \textit{0.516} & \textit{0.503} & \textit{7.014} & \textit{0.458} & \textit{0.491} & \textit{0.476} & \textit{0.488} \\ 
%       \multirow{4}{*}{7B} & FT & \textbf{6.261} & 0.559 & 0.798 & 0.536 & 0.548 & \textbf{3.830} & 0.524 & 0.936 & 0.494 & 0.530 \\
%       & Goldfish & 6.280 & 0.552 & 0.780 & 0.533 & 0.541 & 3.839 & 0.518 & 0.929 & 0.492 & 0.525 \\
%       & DPSGD & 6.777 & 0.509 & 0.538 & 0.523 & 0.504 & 4.490 & 0.466 & 0.516 & \textbf{0.470} & 0.487 \\
%       & \methodname & 6.395 & \textbf{0.507} & \textbf{0.482} & \textbf{0.518} & \textbf{0.500} & 4.006 & \textbf{0.458} & \textbf{0.440} & 0.473 & \textbf{0.480} \\ 
%     \end{tabular}
%     \caption{Caption}
%     \label{tab:main_result}
% \end{table*}


\begin{table*}[h]
  \centering
  \resizebox{0.9\textwidth}{!}{\begin{tabular}{cl|ccccc|ccccc}
  \toprule[1pt]
   \multirow{3}{*}{\textbf{LLM}}  & \multirow{3}{*}{\textbf{Method}} &  \multicolumn{5}{c|}{\textbf{Wikipedia}} & \multicolumn{5}{c}{\textbf{CC-news}} \\ \cmidrule(lr){3-7}  \cmidrule(lr){8-12}
    &  & PPL & Loss & Ref & Min-k & \multicolumn{1}{c|}{Zlib} & PPL & Loss & Ref & Min-k & Zlib \\ \midrule
    \multirow{4}{*}{GPT2} & \textit{Base} & \textit{34.429} & \textit{0.473} & \textit{0.513} & \textit{0.446} & \textit{0.497} & \textit{29.442} & \textit{0.505} & \textit{0.498} & \textit{0.520} & \textit{0.500} \\ 
    \multirow{4}{*}{124M} & FT & \textbf{12.729} & 0.577 & 0.967 & 0.489 & 0.544 & \textbf{21.861} & 0.607 & 0.855 & 0.549 & 0.569 \\
    & Goldfish & 12.853 & 0.565 & 0.954 & 0.486 & 0.537 & 21.902 & 0.608 & 0.855 & 0.547 & 0.570 \\
    & DPSGD & 18.523 & 0.463 & 0.536 & \textbf{0.448} & 0.491 & 26.022 & 0.507 & 0.513 & \textbf{0.521} & 0.502 \\
    & \methodname & 13.628 & \textbf{0.454} & \textbf{0.463} & 0.470 & \textbf{0.485} & 23.733 & \textbf{0.502} & \textbf{0.495} & 0.529 & \textbf{0.499} \\ \midrule
    
    \multirow{4}{*}{Pythia} & \textit{Base} & \textit{10.287} & \textit{0.466} & \textit{0.503} & \textit{0.464} & \textit{0.489} & \textit{13.973} & \textit{0.507} & \textit{0.512} & \textit{0.528} & \textit{0.501}\\ 
    \multirow{4}{*}{1.4B} & FT & \textbf{6.439} & 0.578 & 0.985 & 0.484 & 0.557 & 11.922 & 0.602 & 0.857 & 0.541 & 0.574 \\
    & Goldfish & 6.465 & 0.564 & 0.981 & 0.482 & 0.546 & \textbf{11.903} & 0.609 & 0.862 & 0.543 & 0.579 \\
    & DPSGD & 7.751 & 0.469 & 0.524 & \textbf{0.462} & 0.488 & 13.286 & 0.512 & 0.531 & 0.528 & 0.503 \\
    & \methodname & 6.553 & \textbf{0.468} & \textbf{0.485} & 0.472 & \textbf{0.485} & 12.670 & \textbf{0.501} & \textbf{0.460} & \textbf{0.524} & \textbf{0.499} \\ \midrule
    
    \multirow{4}{*}{Llama-2} & \textit{Base} & \textit{7.014} & \textit{0.458} & \textit{0.491} & \textit{0.476} & \textit{0.488} & \textit{9.364} & \textit{0.505} & \textit{0.495} & \textit{0.516} & \textit{0.503} \\ 
    \multirow{4}{*}{7B} & FT & \textbf{3.830} & 0.524 & 0.936 & 0.494 & 0.530 & \textbf{6.261} & 0.559 & 0.798 & 0.536 & 0.548 \\
    & Goldfish & 3.839 & 0.518 & 0.929 & 0.492 & 0.525 & 6.280 & 0.552 & 0.780 & 0.533 & 0.541 \\
    & DPSGD & 4.490 & 0.466 & 0.516 & \textbf{0.470} & 0.487 & 6.777 & 0.509 & 0.538 & 0.523 & 0.504 \\
    & \methodname & 4.006 & \textbf{0.458} & \textbf{0.440} & 0.473 & \textbf{0.480} & 6.395 & \textbf{0.507} & \textbf{0.482} & \textbf{0.518} & \textbf{0.500} \\
    \bottomrule[1pt]
  \end{tabular}}
  \caption{Overall Evaluation: Perplexity (PPL) and AUC scores of the MIAs with different signals (Loss/Ref/Min-k/Zlib). For all metrics, the lower the value, the better the result. \textit{Base} in the method column indicates the pretrained LLMs without fine-tuning, thus it indicates lower bound for both utility and privacy risk.}
  \label{tab:main_result}
\end{table*}

% \begin{table*}[h]
%   \centering
%   \begin{tabular}{cl|ccccc|ccccc}
%   \multirow{3}{*}{\textbf{LLM}} & \multirow{3}{*}{\textbf{Method}} & \multicolumn{5}{c|}{\textbf{Wikipedia}} & \multicolumn{5}{c}{\textbf{CCNews}} \\
%   \cmidrule(lr){3-7} \cmidrule(lr){8-12}
%   & & PPL & Loss & Ref & min-k & \multicolumn{1}{c|}{zlib} & PPL & Loss & Ref & min-k & zlib \\
%   \midrule
%   \multirow{4}{*}{GPT2} & \textit{Base} & \textit{34.429} & \textit{0.473} & \textit{0.513} & \textit{0.446} & \textit{0.497} & \textit{29.442} & \textit{0.505} & \textit{0.498} & \textit{0.520} & \textit{0.500} \\
%   \multirow{4}{*}{124M} & FT & \textbf{12.729} & 0.577 & 0.967 & 0.489 & 0.544 & \textbf{21.861} & 0.607 & 0.855 & 0.549 & 0.569 \\
%   & Goldfish & 12.853 & 0.565 & 0.954 & 0.486 & 0.537 & 21.902 & 0.608 & 0.855 & 0.547 & 0.570 \\
%   & DPSGD & 18.523 & 0.463 & 0.536 & \textbf{0.448} & 0.491 & 26.022 & 0.507 & 0.513 & \textbf{0.521} & 0.502 \\
%   & \methodname & 13.628 & \textbf{0.454} & \textbf{0.463} & 0.470 & \textbf{0.485} & 23.733 & \textbf{0.502} & \textbf{0.495} & 0.529 & \textbf{0.499} \\
%   \midrule
%   \multirow{4}{*}{Pythia} & \textit{Base} & \textit{10.287} & \textit{0.466} & \textit{0.503} & \textit{0.464} & \textit{0.489} & \textit{13.973} & \textit{0.507} & \textit{0.512} & \textit{0.528} & \textit{0.501} \\
%   \multirow{4}{*}{1.4B} & FT & \textbf{6.439} & 0.578 & 0.985 & 0.484 & 0.557 & 11.922 & 0.602 & 0.857 & 0.541 & 0.574 \\
%   & Goldfish & 6.465 & 0.564 & 0.981 & 0.482 & 0.546 & \textbf{11.903} & 0.609 & 0.862 & 0.543 & 0.579 \\
%   & DPSGD & 7.751 & 0.469 & 0.524 & \textbf{0.462} & 0.488 & 13.286 & 0.512 & 0.531 & 0.528 & 0.503 \\
%   & \methodname & 6.553 & \textbf{0.468} & \textbf{0.485} & 0.472 & \textbf{0.485} & 12.670 & \textbf{0.501} & \textbf{0.460} & \textbf{0.524} & \textbf{0.499} \\
%   \midrule
%   \multirow{4}{*}{Llama-2} & \textit{Base} & \textit{7.014} & \textit{0.458} & \textit{0.491} & \textit{0.476} & \textit{0.488} & \textit{9.364} & \textit{0.505} & \textit{0.495} & \textit{0.516} & \textit{0.503} \\
%   \multirow{4}{*}{7B} & FT & \textbf{3.830} & 0.524 & 0.936 & 0.494 & 0.530 & \textbf{6.261} & 0.559 & 0.798 & 0.536 & 0.548 \\
%   & Goldfish & 3.839 & 0.518 & 0.929 & 0.492 & 0.525 & 6.280 & 0.552 & 0.780 & 0.533 & 0.541 \\
%   & DPSGD & 4.490 & 0.466 & 0.516 & \textbf{0.470} & 0.487 & 6.777 & 0.509 & 0.538 & 0.523 & 0.504 \\
%   & \methodname & 4.006 & \textbf{0.458} & \textbf{0.440} & 0.473 & \textbf{0.480} & 6.395 & \textbf{0.507} & \textbf{0.482} & \textbf{0.518} & \textbf{0.500} \\
%   \end{tabular}
%   \caption{Caption}
%   \label{tab:main_result}
%   \end{table*}
  

\textbf{Datasets}. We conduct experiments on two datasets: CC-news\footnote{\href{https://huggingface.co/datasets/vblagoje/cc_news}{Huggingface: vblagoje/cc\_news}} and Wikipedia\footnote{\href{https://huggingface.co/datasets/legacy-datasets/wikipedia}{Huggingface: legacy-datasets/Wikipedia}}. CC-news is a large collection of news articles which includes diverse topics and reflects real-world temporal events. Meanwhile, Wikipedia covers general knowledge across a wide range of disciplines, such as history, science, arts, and popular culture.\\
\textbf{LLMs}: We experiment on three models including \gpt~(124M)~\cite{gpt2radford}, \pythia~(1.4B)~\cite{pythia}, and \llama~(7B)~\cite{llama2touvron2023}. This selection of models ensures a wide range of model sizes from small to large that allows us to analyze scaling effects and generalizability across different capacities. \\
\textbf{Evaluation Metrics}. For evaluating language modeling performance, we measure perplexity (PPL), as it reflects the overall effectiveness of the model and is often correlated with improvements in other downstream tasks~\cite{kaplan2020scalinglaws, lmsfewshot}. For defense effectiveness, we consider the attack area under the curve (AUC) value and True Positive Rate (TPR) at low False Positive Rate (FPR). In total, we perform 4 MIAs with different MIA signals. Given the sample $x$, the MIA signal function $f$ is formulated as follows: \\
$\bullet$ Loss~\cite{8429311} utilizes the negative cross entropy loss as the MIA signal. 
    \[f_\text{Loss}(x) = \mathcal{L}_\text{CE}(\theta; x) \]
$\bullet$ Ref-Loss~\cite{Carlini2020ExtractingTD} considers the loss differences between the target model and the attack reference model. To enhance the generality, our experiments ensure there is no data contamination between the training data of the target, reference, and attack models.
    \[f_\text{Ref}(x) = \mathcal{L}_\text{CE}(\theta; x) - \mathcal{L}_\text{CE}(\theta_\text{attack}; x) \]
$\bullet$ Min-K~\cite{shi2024detecting} leverages top K tokens that have the lowest loss values.
    \[f_\text{min-K}(x) = \frac{1}{|\text{min-K(x)}|} \sum_{t_i \in \text{min-K(x)}} -\log(P(t_i|t_{<i};\theta) \]
$\bullet$ Zlib~\cite{Carlini2020ExtractingTD} calibrates the loss signal with the zlib compression size.
    \[ f_\text{zlib}(x) = \mathcal{L}_\text{CE}(\theta; x) / \text{zlib}(x) \]

\noindent \textbf{Baselines}. We present the results of four baselines. \textit{Base} refers to the pretrained LLM without fine tuning. \textit{FT} represents the standard causal language modeling without protection. \textit{Goldfish}~\cite{hans2024be} implements a masking mechanism. \textit{DPSGD}~\cite{abadi2016deep, yu2022differentially} applies gradient clipping and injects noise to achieve  sample-level differential privacy.

\noindent \textbf{Implementation}. We conduct full fine-tuning for \gpt and \pythia. For computing efficiency, \llama fine-tuning is implemented using Low-Rank Adaptation (LoRA)~\cite{hu2022lora} which leads to \textasciitilde4.2M trainable parameters. Additionally, we use subsets of 3K samples to fine-tune the LLMs. We present other implementation details in Appendix~\ref{sec:app-implementation}.

\subsection{Overall Evaluation}
Table~\ref{tab:main_result} provides the overall evaluation compared to several baselines across large language model architectures and datasets. Among these two datasets, CCNews is more challenging, which  leads to higher perplexity  for all LLMs and fine-tuning methods. Additionally, the reference-model-based attack performs the best and demonstrates high privacy risks with attack AUC on the conventional fine-tuned models at 0.95 and 0.85 for Wikipedia and CCNews, respectively. Goldfish achieves similar PPL to the conventional FT method but fails to defend against MIAs. This aligns with the reported results by \citet{hans2024be} that Goldfish resists exact match attacks but only marginally affects MIAs. DPSGD provides a very strong protection in all settings (AUC < 0.55) but with a significant PPL tradeoff. Our proposed \methodname guarantees a robust protection, even slightly better than DPSGD, but with a notably smaller tradeoff on language modeling performance. For example, on the Wikipedia dataset, \methodname delivers perplexity reduction by 15\% to 27\%. Moreover, Table~\ref{tab:tpr} (Appendix~\ref{sec:app-add-res}) provides the TPR at 1\% FPR. Both DPSGD and \methodname successfully reduce the TPR to $\sim$0.02 for all LLMs and datasets. \textit{Overall, across multiple LLM architectures and datasets, \methodname consistently offers ideal privacy protection with  little trade-off in language modeling performance.}

\noindent \textbf{Privacy-Utility Trade-off.}
To investigate the privacy-utility trade-off of the methods, we vary the hyper-parameters of the fine-tuning methods. Particularly, for DPSGD, we adjust the privacy budget $\epsilon$ from (8, 1e-5)-DP to (100, 1e-5)-DP. We modify the masking percentage of Goldfish from 25\% to 50\%. Additionally, we vary the loss weight $\alpha$ from 0.2 to 0.8 for \methodname. Figure~\ref{fig:priv-ult-tradeoff} depicts the privacy-utility trade-off for GPT2 on the CCNews dataset. Goldfish, with very large masking rate (50\%), can slightly reduce the risk of the reference attack but can increase the risks of other attacks. By varying the weight $\alpha$, \methodname offers an adjustable trade-off between privacy protection and language modeling performance. \methodname largely dominates DPSGD and improves the language modeling performance by around 10\% with the ideal privacy protection against MIAs.

\begin{figure}[h]
    \centering
    \includegraphics[width=\linewidth]{figs/privacy-ultility-tradeoff.pdf}
    \caption{Privacy-utility trade-off of the methods while varying hyper-parameters. The Goldfish masking rate is set to 25\%, 33\%, and 50\%. The privacy budget $\epsilon$ of DPSGD is evaluated at 8, 16, 50, and 100. The weight $\alpha$ of \methodname is configured at 0.2, 0.5, and 0.8.}
    \label{fig:priv-ult-tradeoff}
\end{figure}


\subsection{Ablation Study}
\textbf{\methodname without reference models.} To study the impact of the reference model, we adapt \methodname to a non-reference version which directly uses the loss of the current training model (i.e., $s(t_i) = \mathcal{L}_{CE}(\theta; t_i)$) to select the learning and unlearning tokens. This means the unlearning tokens are the tokens that have smallest loss values. Figure~\ref{fig:ppl-auc-noref} presents the training loss and testing perplexity. There is an inconsistent trend of the training loss and testing perplexity. Although the training loss decreases overtime, the test perplexity increases. This result indicates that identifying appropriate unlearning tokens  without a reference model is challenging and conducting unlearning on an incorrect set hurts the language modeling performance.

\begin{figure}[htp]
    \centering
    \includegraphics[width=0.35\textwidth]{figs/train_loss_ppl_noref.pdf}
    \caption{Training Loss and Test Perplexity of \methodname without a reference model.
    % (\lrx{If time permits, it would be better to compare with our training curve here)}
    }
    \label{fig:ppl-auc-noref}
\end{figure}

\noindent \textbf{\methodname with out-of-domain reference models.} To examine the influence of the distribution gap in the reference model, we replace the in-domain trained reference model with the original pretrained base model. 
Figure~\ref{fig:ppl-auc-base-woasc} depicts the language modeling performance and privacy risks in this study. \methodname with an out-of-domain reference model can reduce the privacy risks but yield a significant gap in language modeling performance compared to \methodname using an in-domain reference model.

\noindent \textbf{\methodname without Unlearning.} To study the effects of unlearning tokens, we implement \methodname which use the first term of the loss only ({$\mathcal{L}_{\theta} = \mathcal{L}_{CE}(\theta; \mathcal{T}_h)$}). Figure~\ref{fig:ppl-auc-base-woasc} provides the perplexity and MIA AUC scores in this setting. Generally, without gradient ascent, \methodname can marginally reduce membership inference risks while slightly improving the language modeling performance. The token selection serves as a regularizer that helps to improve the language modeling performance. Additionally, tokens that are learned well in previous epochs may not be selected in the next epochs. This slightly helps to not amplify the memorization on these tokens over epochs.

\begin{figure}[htp]
    \centering
    \includegraphics[width=0.28\textwidth]{figs/auc_vs_ppl_base_woasc.pdf}
    \caption{Privacy-utility trade-off of \methodname with different settings: in-domain reference model, out-domain reference model, and without unlearning}
    \label{fig:ppl-auc-base-woasc}
\end{figure}


\subsection{Training Dynamics}
\textbf{Memorization and Generalization Dynamics}. Figure~\ref{fig:training-dynamics} (left) illustrates the training dynamics of conventional fine tuning and \methodname, while Figure~\ref{fig:training-dynamics} (middle) depicts the membership inference risks. Generally, the gap between training and testing loss of conventional fine-tuning steadily increases overtime, leading to model overfitting and high privacy risks. In contrast, \methodname maintains a stable equilibrium where the gap remains more than 10 times smaller. This equilibrium arises from the dual-purpose loss, which balances learning on hard tokens while actively unlearning memorized tokens. By preventing excessive memorization, \methodname mitigates membership inference risks and enhances generalization.

\begin{figure*}[htp]
    \centering
    \includegraphics[width=0.29\linewidth]{figs/loss_vs_steps_ft_duolearn.pdf}
    \includegraphics[width=0.29\linewidth]{figs/auc_vs_steps_ft_duolearn.pdf}
    \includegraphics[width=0.316\linewidth]{figs/cosine.pdf}
    \caption{Training dynamics of \methodname and the conventional fine-tuning approach. The left and middle figures provide the training-testing gap and membership inference risks, respectively. The testing~$\mathcal{L}_{CE}$ of FT and training~$\mathcal{L}_{CE}$ of \methodname are significantly overlapping, we provide the breakdown in Figure~\ref{fig:add-overlap-breakdown} in Appendix~\ref{sec:app-add-res}. The right figure depicts the cosine similarity of the learning and unlearning gradients of \methodname. Cosine similarity of 1 means entire alignment, 0 indicates orthogonality, and -1 presents full conflict.}
    \label{fig:training-dynamics}
\end{figure*}

\noindent \textbf{Gradient Conflicts}. To study the conflict between the learning and unlearning objectives in our dual-purpose loss function, we compute the gradient for each objective separately. We then calculate the cosine similarity of these two gradients. Figure~\ref{fig:training-dynamics} (right) provides the cosine similarity between two gradients over time. During training, the cosine similarity typically ranges from -0.15 to 0.15. This indicates a mix of mild conflicts and near-orthogonal updates. On average, it decreases from 0.05 to -0.1. This trend reflects increasing gradient misalignment. Early in training, the model may not have strongly learned or memorized specific tokens, so the conflicts are weaker. Overtime, as the model learns more and memorization grows, the divergence between hard and memorized tokens increases, making the gradients less aligned. This gradient conflict is the root of the small degradation of language modeling performance of \methodname compared to the conventional fine tuning approach.

\noindent \textbf{Token Selection Dynamics}. Figure~\ref{fig:token-selection} illustrates the token selection dynamics of \methodname during training. The figure shows that the token selection process is dynamic and changes over epochs. In particular, some tokens are selected as an unlearning from the beginning to the end of the training. This indicates that a token, even without being selected as a learning token initially, can be learned and memorized through the connections with other tokens. This also confirms that simple masking as in Goldfish is not sufficient to protect against MIAs. Additionally, there are a significant number of tokens that are selected for learning in the early epochs but unlearned in the later epochs. This indicates that the model learned tokens and then memorized them over epochs, and the during-training unlearning process is essential to mitigate the memorization risks.

\begin{figure}[htp]
    \centering
    \includegraphics[width=0.7\linewidth]{figs/token-selection-dynamics.pdf}
    \caption{Token Selection Dynamics of \methodname}
    \label{fig:token-selection}
    \vspace{-4mm}
\end{figure}

\subsection{Privacy Backdoor}
To study the worst case of privacy attacks and defense effectiveness under the state-of-the-art MIA, we perform a privacy backdoor -- Precurious~\cite{precurious}. In this setup, the target model undergoes continual fine-tuning from a warm-up model. The attacker then applies a reference-based MIA that leverages the warm-up model as the attack's reference. Table~\ref{tab:backdoor} shows the language modeling and MIA performance on CCNews with GPT-2. Precurious increases the MIA AUC score by 5\%. Goldfish achieves the lowest PPL, aligning with~\citet{hans2024be}, where the Goldfish masking mechanism acts as a regularizer that potentially enhances generalization. Both DPSGD and \methodname provide strong privacy protection, with \methodname offering slightly better defense while maintaining lower perplexity than DPSGD.

% \begin{table}[h]
%     \centering
%     \begin{tabular}{c|cc|cc}
%        \multirow{2}{*}{\textbf{Method}}  & \multicolumn{2}{c}{\textbf{CCNews}} & \multicolumn{2}{c}{\textbf{Wikipedia}} \\ 
%        & \textbf{PPL} & \textbf{AUC} & \textbf{PPL} & \textbf{AUC} \\ \hline
%        \textbf{FT}        & 21.593 & 0.911 \\
%        \textbf{Goldfish}  & \textbf{21.074} & 0.886 \\
%        \textbf{DPSGD}     & 23.279 & 0.533 \\
%        \textbf{DuoLearn}  & 22.296 & \textbf{0.499} \\
%     \end{tabular}
%     \caption{Caption}
%     \label{tab:my_label}
% \end{table}

\begin{table}[h]
    \centering
    \resizebox{\columnwidth}{!}{\begin{tabular}{c|cccccc}
        \textbf{Metric} & \textbf{WU} & \textbf{FT} & \textbf{GF} & \textbf{DP} & \textbf{DuoL} \\ \hline
        \textbf{PPL} & \textit{23.318} & 21.593 & \textbf{21.074} & 23.279 & 22.296  \\
        \textbf{AUC} & \textit{0.500} & 0.911 & 0.886 & 0.533 & \textbf{0.499} \\
    \end{tabular}}
    \caption{Experimental results of privacy backdoor for GPT2 on the CC-news dataset. WU stands for the warm-up model leveraged by Precurious. GF, DP, and DuoL are abbreviations of Goldfish, DPSGD, and \methodname}
    \label{tab:backdoor}
\end{table}

% \subsubsection{Hyperparameter Study}

% \subsubsection{Full fine-tuning versus Parameter efficent fine tuning}

% \subsubsection{Extending to Vision Language Models}



\section{Related Work}
\subsection{Image Watermarking Methods}
\emph{Non-learning-based watermarking methods} have developed for decades. Invisible-watermark~\cite{invisible-watermark}, a representative method deployed by Stable Diffusion, encodes watermark into frequency sub-bands. \emph{Learning-based watermarking methods} are gaining dominance due to their superior performance. Zhu et al. \cite{zhu2018hidden} propose the first end-to-end learning architecture for robust watermarking. Following this trend, a series of studies continue to enhance robustness against real-world interferences~\cite{stegastamp, Liu2019TwoStage}. 

% enhancing the robustness of watermarks against image degradation factors and even physical-world disturbances. It is essential to ensure that the watermark can still fulfill its security role during transmission through various forms of channels in the image.

% StegaStamp~\cite{stegastamp} and Liu~et~al.~\cite{Liu2019TwoStage} enrich the types of distortions induced by the middle layer to include operations such as JPEG compression, Gaussian noise, and Blurring, %\hl{concludes the goals of these approaches. What these methods are developed for? To improve robustness level? Or to integrate new capability?}

% applies Discrete Wavelet Transform (DWT) to decompose an image into several frequency sub-bands, encodes watermark as the coefficients change of the Discrete Cosine Transform (DCT) results of some sub-bands, and generates the watermarked image with reverse transform. 

\subsection{Detection Evasion Attacks}
\textbf{Destruction and Reconstruction}. The watermarked image firstly undergoes a certain level of degradation, followed by reconstruction to obtain a purified image. The mainstream approach for this method involves adding noise to the image and then using generative models, such as Diffusion Models (DM)~\cite{ho2020denoising}, for reconstruction and generation~\cite{an2024benchmarking, saberi2023robustness, zhao2023invisible}. In contrast, UnMarker ~\cite{Kassis2024Unmarker} employs a learnable filter to process the watermarked image, supplemented by visual loss functions to ensure and enhance the visual quality of the attack results. 

\textbf{Adversarial Attacks}. Transferring classic adversarial attack methods to the watermarking domain primarily targets the decoder of watermark models. WEVADE~\cite{jiang2023evading} incorporates both black-box and white-box adversarial attack methods. Lukas et al.~\cite{lukas2024leveraging} employ a surrogate model closely resembling the target model to perform transfer attacks. WmRobust~\cite{saberi2023robustness} requires a dataset containing both watermarked and non-watermarked images to train a feature classifier subjected to adversarial attacks. The attacks are transferred to the target watermark detection module. Similarly, WAVES~\cite{an2024benchmarking} relies on a relevant watermark dataset for surrogate attacks but introduces a more detailed classification of watermark data. Hu et al.~\cite{hu2024transfer} explore the feasibility of large-scale ensemble surrogate models for transfer attacks against target watermark models.

\subsection{Watermark Forgery Attacks}
CopyAttack~\cite{kutter2000watermark} was the first to introduce the concept of spoof attacks and proposed a method for predicting watermarks in unknown watermarking algorithm scenarios, embedding them into other images to achieve forgery. WmRobust~\cite{saberi2023robustness} proposes an attack leveraging the encoder of the watermark model to embed noise with a watermark and applies fine-tuning to generate forged watermarks. Steganalysis~\cite{yang2024steganalysis} computes a residual by statistically analyzing a dataset of watermarked images and unpaired clean images. This residual is then used to facilitate watermark forgery.


\section{Discussion}
\label{sec:discussion}

In this section, we first summarize the conclusion and share some key observations. Then, we reflect on the usability of our method and propose potential applications. In the end, we discuss the limitations and future work.

\subsection{Effectiveness of \name{}}
\label{sec:discuss_effectiveness}
Firstly, based on the results from Section~\ref{sec:experiment}, we can draw the following conclusions:
\begin{itemize}
    \item It is efficient to detect unknown words by combining linguistic characteristics provided by the pre-trained language model (PLM) and gaze trajectory.
    \item The prediction is mainly based on the linguistic features from the textual context captured by PLM.
    \item Gaze locates the region of interest in a timely manner, which is necessary for real-time applications. Gaze also helps improve the model performance, but its contribution is limited compared to PLM.
\end{itemize}

Additionally, it is interesting that while we typically assume that the gaze modality should contribute significantly to the task of unknown word detection, the experimental results show that the contribution of gaze to the model’s improvement is small with the existence of PLM. Based on the previous analysis of line spacing and eye tracker accuracy, a possible reason for this is that under normal reading conditions (single-line spacing, line height 3-5 mm), the eye tracker’s accuracy is insufficient to precisely detect which line the gaze belongs to, thus failing to accurately locate the gaze on the words. Furthermore, changes in user posture during long reading sessions further reduce the accuracy of the eye tracker. In our system, PLM compensates for this issue by providing linguistic information based on the text.

From another perspective, the low contribution of gaze is not necessarily a disadvantage. Our method’s reduced reliance on gaze makes it more tolerant of noise. The model’s good performance on data collected by webcams further supports this conclusion. The reduced dependency on gaze data allows our model to be applied on more affordable and accessible devices, such as webcams.

\subsection{Usability of \name{}}
\label{sec:discuss_usability}
The results from the user evaluation (Section~\ref{sec:user_evaluation}) show that our reading assistance prototype helps users read more fluently and they are more willing to use it compared to traditional click-to-translate methods. In addition to providing real-time translation and explanations during reading, our system can also benefit ESL for long-term learning. For example, based on the unknown word detected by our system, we can generate a vocabulary list for memorizing and offer memory curve tracking. Furthermore, these unknown words can also be used to generate personalized summaries and notes.

The potential issue of generalizability across users, texts and devices can be addressed through fine-tuning and reinforcement learning methods. During the initial phases of usage, the system collects both gaze and text data for fine-tuning and lets users provide feedback on the model's predictions. This allows the model to continuously learn the user's unique gaze patterns and infer their vocabulary proficiency and domain expertise from textual content, thereby improving prediction accuracy.

\subsection{Limitation and Future Works}
\label{sec:discuss_limitation}
The quality of gaze data hinders the improvement model performance. The accuracy of the eye tracker is not enough for word-level detection. Common formatting, such as single-line spacing and 10-point font, results in a line height of approximately 3-5 mm when viewed using the PDF viewer with a sidebar on a 14-inch laptop. This requires an accuracy of about $0.3-0.6^\circ$ at a reading distance of 50-60 cm. However, most eye trackers have a gaze accuracy ranging from $0.2-1.1^\circ$~\cite{gaze_survey_2024}. Combined with additional errors caused by head and upper body movements, this level of accuracy is insufficient for real-world reading scenarios. During data collection and evaluation, some participants reported that even after calibration, the error could span 1-3 lines. This makes it difficult to determine the specific word the user is focusing on based solely on gaze coordinates, explaining why gaze-based baselines performed poorly on our data.

\change{The inaccuracy of the gaze data could also lead to the inaccuracy of data labeling. To mitigate the impact of mouse clicks on gaze behavior, we asked users to label unknown words during their second pass. However, this widely adopted labeling method inherently requires "guessing" which words correspond to a given gaze trajectory. Previous works mapped each gaze coordinate directly to a specific word to establish word-gaze pairs. This method is infeasible for text with normal line spacing, so we establish gaze-word pairs by defining a bounding box based on a segment of gaze to identify the corresponding words instead. While this approach improves robustness, it may also introduce mismatches between gaze and words and thus introduce noise to the dataset. To further improve model performance, more precise labeling methods are needed.}

Additionally, reading time can be longer than several minutes in daily scenarios, so gaze drift can significantly affect data quality. In our experiments, we observed that it is difficult for participants to maintain a fixed posture after calibration, though we required them to do so. The posture shift further increases errors. Therefore, in practical applications, real-time calibration of gaze data based on user posture is crucial to ensure data quality. If the existing eye-tracking technology can combined with user posture detection~\cite{faceori}, it is possible to reduce the impact of user posture on gaze data, thereby improving the quality of gaze data.



\section{Conclusion}
\label{sec:Conclusion}
This work evaluates proprietary and open-weight models in agentic frameworks for handling ambiguity in software engineering. In code generation, to effectively integrate new information into the solution, an agent must detect ambiguity and ask targeted questions. Our key findings are:
\begin{itemize}[itemsep=0pt, topsep=0pt]
    \item Given an underspecified input, Claude Sonnet 3.5 and Claude Haiku 3.5 with interaction can achieve 80\% of their performance with a well-specified input. In contrast, open-weight models struggle: Deepseek relies on navigational cues to locate relevant files, while Llama 3.1 70B extracts limited information from the user.
    \item LLMs do not interact unless explicitly prompted, and their ambiguity detection is highly sensitive to prompt variations. Only Claude Sonnet 3.5 achieves a higher accuracy of 84\% in distinguishing between well-specified and underspecified input.

    \item Claude Sonnet 3.5, Haiku 3.5, and Deepseek effectively extract new, detailed user information, whereas Llama 3.1 struggles to ask the right questions.
    
\end{itemize}
Despite these advances, a gap remains between resolve rates for underspecified vs. fully specified issues. Open-weight models need better interaction strategies to improve resolution, while proprietary models, particularly Claude Haiku 3.5, require stronger prompting to engage interactively. This work establishes the current state-of-the-art in handling ambiguity through interaction, breaking the resolution process into multiple steps.



\section*{Impact Statement}
% This paper presents work whose goal is to advance the field of 
% Machine Learning. There are many potential societal consequences 
% of our work, none which we feel must be specifically highlighted here.

Watermarking is an avenue for AIGC provenance and detection, preventing potential misbehavior such as the spread of misinformation, copyright violation, and adversarial false attribution. Our work primarily underscores novel threats to modern learning-based watermarking schemes prioritizing robustness against real-world distortions. In theory, attackers could exploit these vulnerabilities to compromise watermarks, potentially harming users and service providers. However, the watermarking methods analyzed in this study are all open-source and research-focused, while the real-world deployment of invisible and robust watermarks remains in its early stages. Therefore, we believe making our work public has no direct negative impact. Conversely, we believe our findings have a positive societal impact by exposing a fundamental vulnerability in existing robust watermarking techniques, thereby preventing potential covert exploitation by adversaries and offering valuable insights for developing more secure image watermarking solutions.


% However, the watermarking methods studied in this paper are all open-sourced and research-oriented ones. 
% The deployment of invisible and robust watermarks in practice is at an early age. 


% In the unusual situation where you want a paper to appear in the
% references without citing it in the main text, use \nocite
\nocite{langley00}

\bibliography{reference}
\bibliographystyle{dapao}


%%%%%%%%%%%%%%%%%%%%%%%%%%%%%%%%%%%%%%%%%%%%%%%%%%%%%%%%%%%%%%%%%%%%%%%%%%%%%%%
%%%%%%%%%%%%%%%%%%%%%%%%%%%%%%%%%%%%%%%%%%%%%%%%%%%%%%%%%%%%%%%%%%%%%%%%%%%%%%%
% APPENDIX
%%%%%%%%%%%%%%%%%%%%%%%%%%%%%%%%%%%%%%%%%%%%%%%%%%%%%%%%%%%%%%%%%%%%%%%%%%%%%%%
%%%%%%%%%%%%%%%%%%%%%%%%%%%%%%%%%%%%%%%%%%%%%%%%%%%%%%%%%%%%%%%%%%%%%%%%%%%%%%%
\newpage
\clearpage
\begin{appendices}

\section{Production Fault Trace}
\label{appendix:production-fault-trace}
The production fault trace was collected from an 8-GPU node pretrain cluster with 2880 GPUs over a period of 160 days. The trace includes details such as fault start time, fault end time, and the ID of the faulty node. \figref{fig:simulation:trace:timetrace} and \figref{fig:simulation:trace:cdf} provide a macro-level overview of the production fault trace. On average, the ratio of faulty 8-GPU nodes at any given time is $3.83\%$, with a p99 value of $7.22\%$.

\begin{figure}[h!t]
    \centering
    \begin{subfigure}[b]{0.23\textwidth}
        \centering
        \includegraphics[width=\textwidth]{figs/evaluation/fault_server_ratio.pdf}
        \caption{Fault Node Ratio Trace.}
        \label{fig:simulation:trace:timetrace}
    \end{subfigure}
    \hspace{2pt}
    \begin{subfigure}[b]{0.23\textwidth}
        \centering
        \includegraphics[width=\textwidth]{figs/evaluation/fault_server_cdf.pdf}
        \caption{Cumulative Distribution.}
        \label{fig:simulation:trace:cdf}
    \end{subfigure}
    \vspace{-2ex}
    \caption{Fault node trace in the production AI DC.}
    \label{fig:simulation:trace}
\end{figure}

Since most of failure events are GPU faults, we normalized the trace of 8-GPU nodes to generate 4-GPU nodes trace. Assuming that the fault rates of GPUs are i.i.d. with a fault probability of $p$ for each GPU, and considering that a node is deemed faulty if any GPU within it fails, the fault rate of an 8-GPU node is calculated as:  

\vspace{-1em}
$$
P_{fault}(8\text{-GPU}) = 1 - (1-p)^8 = 3.83\%.
$$  

From this, we derive $p = 0.49\%$. The fault rate for a 4-GPU node is then:  
$$
P_{fault}(4\text{-GPU}) = 1 - (1-p)^4 = 1.93\%.
$$  

The fault event of 4-GPU node is generate with Bayesian Equation, as:


\begin{align*}\label{eq:convert-trace}
& P_{fault}( \text{4-GPU} \mid  \text{8-GPU})\\ 
    &=\frac{P_{fault}(\text{8-GPU} \mid \text{4-GPU}) P_{fault}(\text{4-GPU})}{P_{fault}(\text{8-GPU})} \\ 
    & =  \frac{1 \times 1.93\%}{3.83\%} = 50.39\% \\
\end{align*}

Thus, whenever a fault occurs in an 8-GPU node in the original trace, each of the two corresponding 4-GPU nodes at the same location has a $50.39\%$ probability of fault. This method is used to convert the traces.

As node faults are i.i.d., the simulator linearly maps the fault trace to different network architectures.

\section{GPT-MoE Architecture}
\label{appendix:gpt-moe}
This model is a mixture-of-experts (MoE) model with the following configuration:

\para{Model Configuration:}
\begin{itemize}
    \item \textbf{Number of Layers:} 192
    \item \textbf{Inner Layer Dimension:} 49152
    \item \textbf{Embedding Dimension:} 12288
    \item \textbf{Hidden Dimension:} 12288
    \item \textbf{Vocabulary Size:} 64000
    \item \textbf{Number of Attention Heads:} 128
    \item \textbf{Maximum Sequence Length:} 2048
    \item \textbf{Number of Experts:} 8
    \item \textbf{MoE Layer Ratio:} 0.5
    \item \textbf{Top-K Experts:} 2
\end{itemize}

\para{Runtime Configuration:}
\begin{itemize}
    \item \textbf{Virtual Pipeline Parallelism:} 3
    \item \textbf{Micro Batch Size:} 1
    \item \textbf{Global Batch Size:} 1536
    \item \textbf{Max Sequence Length:} 2048
\end{itemize}




\section{Theoretical analysis of wasted GPU ratio for \sys}
\label{appendix:ft-anay}

The count of backup lines as $2K - 2$ will significantly influence the fault tolerance of \sys. We use the expectation of waste ratio caused by GPU failure and fragmentation problem to evaluate this design, the result is shown in \tabref{table:design:1.5ratio}.

For one single working server in the middle of line, the count of breakpoints $B$ on its two sides has the expectation as:

\vspace{-1em}
\begin{equation*}
E_B(\eta = 1,middle) = 2(P_s^K + P_s^{2K})
\end{equation*}

Where $P_s$ is the fail probability of GPU server, and $\eta$ is count of servers. The expectation of breakpoints count is:

Once the distance between one server and the tail of line is $\alpha < K$, it will connect to all servers between itself and the last one, so there will be no breakpoints on this side, and the expectation of breakpoints count is less than servers in the middle of line.
Then, for any server in the line topology:

\vspace{-1em}
$$
E_B(\eta = 1) \leq E_B(\eta = 1,middle) 
$$

When the distance between two servers is $\beta \geq K$, the breakpoints among them can be calculated as independent.
Once the distance $\beta < K$, as all servers in this range are connected to these two servers, there will be no breakpoints between them. So, the expectation is less than two independent servers. Then,



\vspace{-1em}
\begin{align*}
E_B(\eta =& 2) < E_B(\eta = 2, \beta \geq K) =  2E(\eta = 1)   \\ 
 E_B(\eta =& N_s) \leq N_s E_B(\eta = 1) 
\end{align*}

For a LLM job which require a ring communication size (TP .etc) as $N_t$, \sys   will cut the whole line topology into several sub lines with the length of $N_t/R$.
Once \sys is cutting a new sub line from the remaining servers in the line, 
all $N_t$ GPU will be wasted when one break point exist in the middle of this sub line required, shown in \fig{fig:subline-waste}. 
Then the expectation for waste GPU caused by one single break point is:

\vspace{-1em}
$$
E_W(B=1) = N_t R\cdot (1 - (N_t/R)^{-1} ) = R(N_t -R)
$$

\begin{figure}[h!t]
    \centering
    \includegraphics[width=0.8\linewidth]{figs/design/intra-topo/break-topo.drawio.pdf}
    \caption{Break point can cause server waste compare to ideal situation.}
    \vspace{-1em}
    \label{fig:subline-waste}
\end{figure}

As the influence between two break points only reduce the expectation of wasted GPUs, we can have this for $X$ break points:

\vspace{-1em}
\begin{equation*}
E_W(B = X) \leq XE_W(B=1) = XR(N_t-R)
\end{equation*}

So the expectation of wasted GPU for a servers cluster with $N_s$ GPU servers is:

\vspace{-1em}
\begin{align*}
E_W(\eta = N_s) &\leq \sum P(B=X ,\eta = N_s) \cdot X\cdot  E_W(B=1)\\
&= E_B(\eta = N_s)\cdot E_W(B=1)\\
&\leq  \lim_{P_s\rightarrow 0}2N_s\cdot R \cdot (N_t-R)P_s^K
\end{align*}



The final expectation of GPUs waste ratio is \eqref{eq:design:ratio}:

\begin{equation}
E_{WR}(\eta = N_s) = \frac{E_W(\eta = N_s)}{N_g} \leq 2(N_t-R)(P_s)^K
\label{eq:design:ratio}
\end{equation}

In our trace for a 160 days long pre-train job on 10K-GPU, the p99 failure rate for 8-card machines is 7\%. If a TP32 jobs is running on \sys, we can get the upper bond for waste ratio expectation for various configuration in \tabref{table:design:1.5ratio}.

\begin{table}[h!t]
\centering
\begin{tabular}{cccc}
    \toprule
        & $K=2$&$K=3$&$K=4$\\
    \midrule
     R=4& $7.35\%$ & $0.26\%$ & $9.00\times 10^{-4}$ \\
     R=8& $27.4\%$ & $1.92\%$ & $0.13\%$ \\
     \bottomrule
\end{tabular}
\caption{Upper bond for waste ratio expectation of GPU, where GPU failure rate is 0.875\% and X is 32}
\vspace{-2em}
\label{table:design:1.5ratio}
\end{table}

As shown in the table, for 4 GPU server ($R=4$) 3 bundles ($K = 3$) design, the additional waste of GPU is less than 0.26\%, while the waste ratio for $R=8,K=4$ is less than 0.13\%. This is sufficient for production clusters. 

\section{Orchestration For Fat-Tree}
\label{appendix:orch-algo}
In this section, we introduce the orchestration algorithm under Fat-Tree DCN in detail.

\para{Notations}
\label{appendix:orch-algo:notation}
To ensure rigorous mathematical reasoning, we introduce the following notations:

\begin{itemize}
    \item {
        $n$: number of nodes in the data-center.
    }
    \item {
        $K$: \docs{} bundle (see \S\ref{section:design:topology}).
    }
    \item {
        $S_{all}$: ordered set, represents all nodes numbered from 1 according to their physical connection order in DCN fabric. $|S_{all}|=n$.
    }
    \item {
        $S$: ordered subset, represents nodes, $\forall u \in S, u \in S_{all}$. Adjacent elements in $S$ are also adjacent from the perspective of the \SYS{} topology. 
    }
    \item{
        $E$: The set of edges across $S$, should be equal to $\{ (S_i, S_j) \mid 1 \leq i < j \leq n, j - i \leq K \} $, representing the connections between nodes, including both primary and backup links, and $O(|E|) = O(K|S|)$.
    }
    \item {
        $InfHBD=<S,E>$: the topology of \SYS{} as an undirected graph.
    }
    \item {
        $F$: faulty nodes.
    }
    \item {
        $HealthyHBD=<H,HE>$: healthy node subgraph where the set of healthy nodes $H = S - F$ and the edge set $HE = \{ (u, v) \mid u \in H \text{ and } v \in H \text{ and } (u, v) \in E \}$.
    }
    \item{
        $t$: TP size, number of GPUs in one TP Group.
    }
    \item{
        $r$: GPU ranks per node.
    }
    \item{
        $m=t/r$: number of nodes in a TP group.
    }
    % \item{
    %     $k$: number of rails in rail-optimized network.
    % }
    \item{
        $s$: job scale, number of GPUs required for the job.
    }
    \item{
        $d$: Aggregation-Switches Domain size. Number of nodes under coverage of one group of Aggregation-Switches.
    }
    \item{
        $n_{constrains}$: number of applied constraints in binary-search-based orchestration algorithm.
    }
    \item{
        $p$: number of nodes under each ToR.
    }
    \item{
        $l$: shortest sub-line length under fat-tree orchestration.
    }
    \item{
        $n_{maxsubline}=\lfloor \frac{nd}{p} \rfloor$: max number of sub-lines.
    }
    \item{
        $G_{deploy}=<S_{deploy},E_{deploy}>$: deployed topology. After applying the deployment strategy, the topology from the perspective of \SYS{} is described as follows: $S_{\text{deploy}}$ is an ordered set where adjacent elements correspond to adjacent nodes in \SYS{}, and $E_{\text{deploy}}$ represents the connections between nodes.
    }
    
\end{itemize}


% For the \SYS{} the orchestration algorithm in ideal conditions is relatively straightforward. The detailed steps of the algorithm are outlined in \algref{alg:orchestration-ideal}.

% Assume that the \SYS{}(with \docs{} direction $K$) is represented as an undirected graph $ \text{InfHBD} = \langle S, E \rangle $, where the ordered set of nodes $ S $ represents nodes. Adjacent elements in $S$ are also adjacent from the perspective of the \SYS{} topology. The set of edges $E$ should be equal to $\{ (S_i, S_j) \mid 1 \leq i < j \leq n, j - i \leq K \} $, representing the connections between nodes, including both primary and backup links, and $O(|E|) = O(K|S|)$. The set of faulty nodes is denoted as $ F \subseteq S $.

% The algorithm proceeds as follows:

% \begin{enumerate}
%     \item {\textbf{Extract the Healthy Node Subgraph:} First, extract the subgraph $\text{HealthyHBD} = \langle H, HE \rangle$ where the set of healthy nodes $H = S - F$ and the edge set $HE = \{ (u, v) \mid u \in H \text{ and } v \in H \text{ and } (u, v) \in E \}$. See \algref{alg:orchestration-ideal}.
%     }
%     \item {\textbf{Identify Connected Components:} Next, identify all connected components in the graph $\text{HealthyHBD}$. Faulty nodes may cause disconnections in the \SYS{} fabric, splitting the original cluster into multiple sub-HBDs. These sub-HBDs are the connected components, and TP Groups cannot span across these disconnected sub-HBDs. We use a simple Depth-First Search (DFS) algorithm here. See \algref{alg:dfs}.}
%     \item {\textbf{Generate Placement Scheme:} Given the excellent physical properties of the \SYS{}, TP Groups can be arranged sequentially within each connected component to generate placement scheme maximizing GPU utilization. See \algref{alg:orchestration-ideal}.
%     }
% \end{enumerate}

% Since each of the three steps involves traversing the entire graph's edges and nodes only once, 
The orchestration algorithm (\algref{alg:orchestration-ideal}) without considering DCN has the overall time complexity $3\cdot O(|H| + |HE|) = O(|S| + |E|) = O((K+1)|S|) = O(|S|)$.

% \begin{algorithm}[!h]
% \small
% \caption{Connected-Component-DFS}
% \label{alg:dfs}
% \SetAlgoNlRelativeSize{-1}
% \SetAlgoNlRelativeSize{1}
%  \KwIn{ $node$, $HealthyHBD$, $visited$}
%  \KwOut{ $component$}

%  Initialize $stack = [node]$ \;
%  Initialize $component = []$\;

% \While{ stack is not empty}
% {
%      $current = stack.pop()$\;
%     \If{$current$ not in $visited$}
%     {
%          Add $current$ to $visited$\;
%          Add $current$ to $component$\;
%         \For{ each neighbor in $HealthyHBD.neighbors(current)$}
%         {
%              $stack.push(neighbor)$\;
%         }
%     }
% }
        
% \KwRet{$component$}
% \end{algorithm}

\begin{algorithm}[!h]
\small
\caption{Orchestration-DCN-Free}
\label{alg:orchestration-ideal}
\SetAlgoNlRelativeSize{-1}
\SetAlgoNlRelativeSize{1}
\KwIn{$\text{InfHBD}=\langle S, E \rangle$, $F$, $m$}
\KwOut{ Placement scheme maximizing GPU utilization}

 Initialize $H = S - F$\;
 Initialize $HE = \{ (u, v) \mid u \in H \text{ and } v \in H \text{ and } (u, v) \in E \}$\;
 Create subgraph $HealthyHBD = \langle H, HE \rangle$\;
 Initialize $component\_list = []$\;
 Initialize $visited = \{\}$\;
 Initialize $placement\_scheme= \{\}$\;

\For{ each node $s$ in $H$}
{
    \uIf{ $s$ not in $visited$}
    {
         $component = Connected-Component-DFS(s, HealthyHBD, visited)$\;
         Add $component.sortedinHBD()$ to $component\_list$\;
    }
}
\For{ each $component$ in $component\_list$}
{
    \While{ $component.size()\geq m$}
    {
         Add $component.pop(m)$ to $placement\_scheme$\;
    }
}
        
 \KwRet{$placement\_scheme$}
 \end{algorithm}
 
% \subsection{Algorithms under Rail-Optimized Network}
% \label{appendix:orch-algo:rail-optimized}

% This subsection provides a detailed description of the orchestration algorithm for Rail-Optimized network.  

% The rail-optimized network topology is specifically designed for highly regular machine learning workload traffic patterns, making it a commonly used and effective architecture. As illustrated in \fig{fig:rail-topo}, Rail Switch $i$ connects to GPU $i$ in node, dividing the network into multiple rails. Let $r$ denote the GPU ranks per node, and $k$ the number of rails. In traditional rail-optimized networks, $k = r$, and a typical training strategy involves running TP $r$ within the single-node HBD, while DP operates between HBDs. Since in DP, GPUs only communicate with GPUs of the same rank in different TP groups, in other words, DP traffic is confined to the rail itself. Therefore, the Rail-Optimized topology perfectly meets this requirement.

% % \begin{figure}[!h]
% %     \centering
% %     \includegraphics[width=\linewidth]{figs/design/Orchestration/rail-optimized.drawio.pdf}
% %     \caption{Rail-Optimized Network: GPU ranks per node $r=4$, Number of rails $k=8$, Aggregation-Switches Domain size $d$, Number of Aggregation-Switches Domain $nd$, Node IDs from 1 to $nd\cdot d$. }
% %     \label{fig:rail-topo}
% % \end{figure}

% \para{Orchestration Constraints. }To minimize the cross-rail traffic which can lead to congestion and latency, the rail-optimized network introduces two key constraints for orchestration algorithms:


% \begin{itemize}
%     \item {
%         \textbf{Aggregation-Switches Domain Coverage Constraint. }
%         The coverage domian of a group of Aggregation-Switches is limited, meaning that TP groups spanning across Aggregation-Switches domains would result in cross-rail traffic, which should be avoided as much as possible.
%     }
%     \item {
%         \textbf{Node Rail State Constraint. }When$ k = r$, this constraint does not apply, as there is no cross-rail traffic.However, as HBDs extend beyond single nodes and the need for larger DP scales due to the expansion of LLM scale, scenarios with $k = p \cdot r$ may arise. This results in $p$ different node states within the data center, with each state occupying $r$ rails, and inter-state communication leads to cross-rail traffic. The specific form of this constraint depends on the deployment strategy.
%     }
% \end{itemize}

% \para{Deployment Strategy. }If the \SYS{} connections continue to follow the physical layout of nodes on the DCN Fabric, avoiding cross-rail traffic would require each TP Group to have an equal number of nodes from each state, making the algorithm to maximize GPU Utilization NP-Complete (see Appendix.\ref{appendix:np-hard-orchestration}). However, by altering the physical connection sequence of \SYS{}, this NP-Complete problem can be reduced to polynomial time. As shown in \fig{fig:parallel-line}, nodes of each state are arranged into $p$ parallel sub-lines, which are then connected end-to-end to form a single line. By restricting DP to operate within sub-lines, all DP traffic remains within the rails, effectively reducing the $k = p * r$ scenario to $k = r$. 

% % \begin{figure}[!h]
% %     \centering
% %     \includegraphics[width=\linewidth]{figs/design/Orchestration/parallel-line.drawio.pdf}
% %     \caption{The deployment strategy example with $p=4$ and Aggregation-Switches Domain size $K=8$. Node IDs from 1 to n are arranged according to their connection order in the DCN Fabric.}
% %     \label{fig:parallel-line}
% % \end{figure}

% \para{The binary search-based Orchestration algorithm.} Based on the above-mentioned constraints and the deployment strategy, we developed an orchestration algorithm that maximizes the number of constraints satisfied while meeting the job scale requirements. This is achieved using a binary search approach with the number of satisfied constraints as the variable. Both types of constraints essentially involve splitting the Line into sub-lines. Therefore, controlling the number of constraints translates to managing the number of sub-lines: fewer sub-lines mean longer sub-lines, leading to higher GPU Utilization. Since the Ideal orchestration algorithm with complexity $O(n)$ can be applied within sub-lines.

% \algref{alg:orchestration-fat-tree} is the main binary-search-based orchestration algorithm. It begins by generating the topology from the perspective of \SYS{} based on the hardware deployment strategy (\algref{alg:deployment-strategy}). Using the number of satisfied constraints as a variable, the algorithm performs a binary search to identify the placement scheme that maximizes the number of satisfied constraints while meeting the job scale requirements.  

% \algref{alg:placement-rail-optimized} calculates the placement scheme for a given number of constraints. It divides the topology into multiple ideal sub-lines and applies the ideal-case orchestration algorithm (\algref{alg:orchestration-ideal}) to each sub-line.  

% Since the time complexity of \algref{alg:orchestration-ideal} is $O(|S|)$, the complexity of \algref{alg:placement-rail-optimized} is 

% \begin{align*}
% &\sum_{i=1}^{n_{constraints}} O(|S_{subline}|) \\
% &= O(\sum_{i=1}^{n_{constraints}} |S_{subline}|) \\
% &= O(|S_{all}|) = O(n)
% \end{align*}

% Thus, the overall time complexity of \algref{alg:orchestration-rail-optimized} is $O(n \log n)$.

\begin{algorithm}[!h]
\small
\caption{Deployment-Strategy}
\label{alg:deployment-strategy}
\SetAlgoNlRelativeSize{-1}
\SetAlgoNlRelativeSize{1}
 \KwIn{Node ordered set $S$, \docs{} direction $K$, parallel factor $p$}
 \KwOut{Deployment topology $G_{deploy}=<S_{deploy},E_{deploy}>$}
 Initialize ordered set $S_{deploy}=[]$\;
 Initialize $l=\lfloor \frac{|S|}{p}\rfloor$\;
\For{$i$ in $0...p-1$}
{
    \For{$j$ in $0...l-1$}{
         Add $i+j\cdot p$ to $S_{deploy}$\;}
}
 Create $E_{deploy}=\{(S_{deploy}^i,S_{deploy}^j)|1\leq i\le j\leq |S_{deploy}|, j-i\leq K \}$\;
 \KwRet{$G_{deploy}=<S_{deploy},E_{deploy}>$}
\end{algorithm}


% \begin{algorithm}[!h]
% \small
% \caption{Placement-Rail-Optimized}
% \label{alg:placement-rail-optimized}
% \SetAlgoNlRelativeSize{-1}
% \SetAlgoNlRelativeSize{1}
%  \KwIn{Deployment topology $G_{deploy}=<S_{deploy},E_{deploy}>$, Number of applied constraints $n_{constraints}$, Faulty node $F$, Sub-line length $l$, Number of node in one TP group $m$}
%  \KwOut{Placement scheme}
%  Initialize $placement\_scheme=\{\}$\;
% \For{$i$ in $1..n_{constraints}$}
% {
%      $S_{subline}=S_{deploy}.pop(l)$\;
%      $E_{subline}=\{(u,v)\mid u\in S_{subline} \text{ and } v\in S_{subline} \text{ and } (u,v)\in E_{subline}\}$\;
%      $F_{subline}=F\cap S_{subline}$\;
%      $placement\_scheme=placement\_scheme\cup \text{Orchestration-Ideal}(<S_{subline},E_{subline}>, F_{subline}, m)$\;
% }
%  $E_{res}=\{(u,v)\mid u \in S_{deploy} \text{ and } v \in S_{deploy} \text{ and } (u,v) \in E_{deploy}\}$\;
%  $F_{res}=F\cap S_{deploy}$\;
%  $placement\_scheme=placement\_scheme\cup \text{Orchestration-Ideal}(<S_{deploy},E_{res}>, F_{res},m)$\;
%  \KwRet{$placement\_scheme$}
% \end{algorithm}


% \begin{algorithm}[!h]
% \small
% \caption{Orchestration-Rail-Optimized}
% \label{alg:orchestration-rail-optimized}
% \SetAlgoNlRelativeSize{-1}
% \SetAlgoNlRelativeSize{1}
%  \KwIn{Node ordered set $S$ (from 1 to n in DCN Fabric), GPU ranks per node $r$, Number of rails $k$, Faulty set $F$, TP size $t$, Job scale $s$ (number of GPUs required for the job), Aggregation-Switches Domain size $d$, \docs{} directions $K$.}
%  \KwOut{Placement scheme that satisfies job scale and minimizes cross-rail traffic.}
%  Initialize $p=k/r$, $m=t/r$, $n=|S|$, $l=\lfloor \frac{d}{p}\rfloor$\;
%  Create graph $G_{deploy}=<S_{deploy},E_{deploy}>=\text{Deployment-Strategy}(S,K,p)$\;
%  Initialize $high=\lfloor\frac{nd}{p}\rfloor$\;
%  Initialize $low=0$\;
%  Initialize $placement\_scheme=\{\}$\;
% \While{ $low \leq$ high}
% {
%      $mid=\lfloor \frac{low+high}{2} \rfloor$\;
%      $placement\_scheme=\text{Placement-Rail-Optimized}(G_{deploy},mid,F,l,m)$\;
%     \eIf {$|placement\_scheme|\cdot m\cdot r\ge s$}
%     {
%          $low=mid+1$\;
%     }
%     {
%          $high=mid-1$\;
%     }
% }
    
% \eIf{$|placement\_scheme|\cdot m\cdot r\ge s$}
% {
%   \KwRet {$placement\_scheme$}
% }
% {
%     \KwRet {None}
% }
% \end{algorithm}
  

Fat-Tree topology is another common data center topology. A typical training strategy for this topology aims to maximize the bandwidth utilization under ToR (Top of Rack) Switches. Using Meta's two-stage clos topology\cite{sigcomm2024meta} as a reference, it can be observed that there is an attempt to run CP under ToR.

\para{Deployment Strategy:} Assuming there are $p$ nodes under each ToR, nodes with the same index under each ToR are deployed along the same parallel sub-line, and the $p$ sub-lines are connected end-to-end, as shown in \fig{fig:fat-tree-topo}. The training strategy involves running CP $p$ across the sub-lines and running TP within them.

\para{Orchestration Constraints. }To maximize the utilization of ToR bandwidth and minimize cross-ToR traffic, the fat-tree topology introduces two constraints:

\begin{packeditemize}
    \item {
        \textbf{Aggregation-Switches Domain Constraint: }The coverage domian of a group of Aggregation Switches is limited, meaning that TP groups spanning across Aggregation Switches domains would result in cross-rail traffic, which should be avoided as much as possible.
    }
    \item {
        \textbf{TP Group Alignment Constraint: } A CP Group consists of TP Groups across parallel sub-lines. To keep CP traffic within the ToR, the TP Groups must be aligned. If a node fails under one ToR, all nodes under that ToR are considered failed, expanding the failure radius by a factor of $p$. 
    }
\end{packeditemize}

\para{Binary-Search-Based Orchestration Algorithm.} Based on the constraints and deployment strategy, we develop a binary search orchestration algorithm (see \algref{alg:orchestration-fat-tree}) that adjusts the number of satisfied constraints. The binary search first relaxes the TP Group alignment constraints within the Aggregation-Switches Domain and then relaxes the TP Group crossing constraints between Aggregation-Switch domains (see \algref{alg:placement-fat-tree}). This process is monotonic.


% \begin{figure}[!h]
%     \centering
%     \includegraphics[width=\linewidth]{figs/design/Orchestration/meta-topo.drawio.pdf}
%     \caption{Orchestration example for Fat-Tree Topology under single Aggregation-Switches Domain with $p=2$. Green indicates active node, red indicates faulty node and yellow indicates idle nodes}
%     \label{fig:meta-topo}
% \end{figure}


The time complexity of \algref{alg:orchestration-ideal} is $O(|S|)$, and the complexity of \algref{alg:placement-fat-tree} is 

$$\sum_{i=1}^{n_{subline}} O(|S_{subline}|) = O(\sum_{i=1}^{n_{subline}} |S_{subline}|) = O(|S_{all}|) = O(n)$$  

Thus, the overall time complexity of \algref{alg:orchestration-fat-tree} is $O(n \log n)$.

\begin{algorithm}[!h]
\small
\caption{Placement-Fat-Tree}
\label{alg:placement-fat-tree}
\SetAlgoNlRelativeSize{-1}
\SetAlgoNlRelativeSize{1}
 \KwIn{$G_{deploy}=<S_{deploy},E_{deploy}>$, $n_{constraints}$, $F$, $l$, $m$, $n_{maxsubline}$, $d$, $p$}
 \KwOut{Placement scheme}
 Initialize $placement\_scheme=\{\}$\;
 Initialize $n_{align}=max(0,n_{constraints}-n_{maxsubline})$, $n_{subline}=min(n_{maxsubline},n_{constraints})$\;
 
\For{$i$ in $0..n_{align}-1$}
{
    \For{$j$ in $1..d$}
    {
        $sid=i*d+j$\;
        \If{$sid \in F$}
        {
            $F\cup \{\lfloor \frac{sid-1}{p}\rfloor\cdot p+1..(\lfloor \frac{sid-1}{p}\rfloor+1)\cdot p \}$\;
        }
    }
}
\For{$i$ in $1..n_{subline}$}
{
     $S_{subline}=S_{deploy}.pop(l)$\;
     $E_{subline}=\{(u,v)\mid u\in S_{subline} \text{ and } v\in S_{subline} \text{ and } (u,v)\in E_{subline}\}$\;
     $F_{subline}=F\cap S_{subline}$\;
     $placement\_scheme=placement\_scheme\cup \text{Orchestration-Ideal}(<S_{subline},E_{subline}>, F_{subline}, m)$\;
}
 $E_{res}=\{(u,v)\mid u \in S_{deploy} \text{ and } v \in S_{deploy} \text{ and } (u,v) \in E_{deploy}\}$\;
 $F_{res}=F\cap S_{deploy}$\;
 $placement\_scheme=placement\_scheme\cup \text{Orchestration-Ideal}(<S_{deploy},E_{res}>, F_{res},m)$\;
 \KwRet{$placement\_scheme$}
\end{algorithm}

\begin{algorithm}[!h]
\small
\caption{Orchestration-Fat-Tree}
\label{alg:orchestration-fat-tree}
\SetAlgoNlRelativeSize{-1}
\SetAlgoNlRelativeSize{1}
 \KwIn{$S$, $r$, $p$, $F$, $t$, $s$, $d$, $K$.}
 \KwOut{Placement scheme that satisfies job scale and minimizes cross-rail traffic.}
 Initialize $m=t/r$, $n=|S|$, $l=\lfloor\frac{d}{p}\rfloor$\, $n_{domain}=\lfloor\frac{n}{d}\rfloor$, $n_{maxsubline}=\lfloor\frac{nd}{p}\rfloor$\;
 Create graph $G_{deploy}=<S_{deploy},E_{deploy}>=\text{Deployment-Strategy}(S,K,p)$\;
 Initialize $high=n_{domain}+n_{maxsubline}$\;
 Initialize $low=0$\;
 Initialize $placement\_scheme=\{\}$\;
\While{ $low \leq$ high}
{
     $mid=\lfloor \frac{low+high}{2} \rfloor$\;
     $placement\_scheme=\text{Placement-Fat-Tree}(G_{deploy},mid,F,l,m,n_{maxsubline},d,p)$\;
    \eIf {$|placement\_scheme|\cdot m\cdot r\ge s$}
    {
         $low=mid+1$\;
    }
    {
         $high=mid-1$\;
    }
}
    
\eIf{$|placement\_scheme|\cdot m\cdot r\ge s$}
{
    \KwRet {$placement\_scheme$}
}
{
    \KwRet {None}
}
\end{algorithm}





\section{Additional Simulation Results for Fault Resilience}
\label{appendix:wasted-GPUs-ratio}
This section presents additional simulation results related to \S\ref{sec:simulation:fault}. \figref{fig:simulation:wasted-trace} shows the variation of the GPU waste ratio over time under the production fault trace. \figref{fig:simulation:waste-cdf:gr4:supple} presents the CDF data for the GPU waste ratio. \figref{fig:simulation:model:wasted-gr4} illustrates the waste GPU ratio for different HBD architectures under various node failure rates, including the results for TP-8 to TP-64. \figref{fig:simulation:breakdown-duration-supple} shows the proportion of job-fault waiting time relative to total time for different job scales. All the aforementioned experiments include results for TP-8, TP-16, TP-32, and TP-64 configurations.








\begin{figure*}[h!t]
    \centering
    \begin{subfigure}[b]{0.23\linewidth}
        \centering
        \includegraphics[width=\linewidth]{figs/evaluation/fault_trace_based/frag_trace_tp8_gr4.pdf}
        \caption{TP-8.}
        \label{fig:simulation:wasted-trace:tp8-4gpu}
    \end{subfigure}
    \hspace{2pt}
    \begin{subfigure}[b]{0.23\linewidth}
        \centering
        \includegraphics[width=\linewidth]{figs/evaluation/fault_trace_based/frag_trace_tp16_gr4.pdf}
        \caption{TP-16.}
        \label{fig:simulation:wasted-trace:tp16-4gpu}
    \end{subfigure}
    \hspace{2pt}
    \begin{subfigure}[b]{0.23\linewidth}
        \centering
        \includegraphics[width=\linewidth]{figs/evaluation/fault_trace_based/frag_trace_tp32_gr4.pdf}
        \caption{TP-32.}
        \label{fig:simulation:wasted-trace:tp32-4gpu}
    \end{subfigure}
    \hspace{2pt}
    \begin{subfigure}[b]{0.23\linewidth}
        \centering
        \includegraphics[width=\linewidth]{figs/evaluation/fault_trace_based/frag_trace_tp64_gr4.pdf}
        \caption{TP-64.}
        \label{fig:simulation:wasted-trace:tp64-4gpu}
    \end{subfigure}

    \vspace{-1ex}
    \caption{GPU waste ratio over production fault trace, 4 GPU node.}
    \label{fig:simulation:wasted-trace}
\end{figure*}


\begin{figure*}[h!t]
    \centering
    \begin{subfigure}[b]{0.23\linewidth}
        \centering
        \includegraphics[width=\linewidth]{figs/evaluation/fault_trace_based/cdf_trace_waste_tp8_gr4.pdf}
        \caption{TP-8.}
        \label{fig:simulation:waste-cdf:tp8-gr4}
    \end{subfigure}
    \hspace{2pt}
    \begin{subfigure}[b]{0.23\linewidth}
        \centering
        \includegraphics[width=\linewidth]{figs/evaluation/fault_trace_based/cdf_trace_waste_tp16_gr4.pdf}
        \caption{TP-16.}
        \label{fig:simulation:waste-cdf:tp16-gr4}
    \end{subfigure}
    \hspace{2pt}
    \begin{subfigure}[b]{0.23\linewidth}
        \centering
        \includegraphics[width=\linewidth]{figs/evaluation/fault_trace_based/cdf_trace_waste_tp32_gr4.pdf}
        \caption{TP-32.}
        \label{fig:simulation:waste-cdf:tp32-gr4}
    \end{subfigure}
    \hspace{2pt}
    \begin{subfigure}[b]{0.23\linewidth}
        \centering
        \includegraphics[width=\linewidth]{figs/evaluation/fault_trace_based/cdf_trace_waste_tp64_gr4.pdf}
        \caption{TP-64.}
        \label{fig:simulation:waste-cdf:tp64-gr4}
    \end{subfigure}
    \vspace{-1ex}
    \caption{CDF of GPU waste ratio over production fault trace, 4 GPU node.}
    \label{fig:simulation:waste-cdf:gr4:supple}
\end{figure*}


\begin{figure*}[h!t]
    \centering
    \begin{subfigure}[b]{0.23\linewidth}
        \centering
        \includegraphics[width=\linewidth]{figs/evaluation/fault_model_based/frag_ratio_tp8_gr4.pdf}
        \caption{TP-8.}
        \label{fig:simulation:model:wasted:tp8}
    \end{subfigure}
    \hspace{2pt}
    \begin{subfigure}[b]{0.23\linewidth}
        \centering
        \includegraphics[width=\linewidth]{figs/evaluation/fault_model_based/frag_ratio_tp16_gr4.pdf}
        \caption{TP-16.}
        \label{fig:simulation:model:wasted:tp16}
    \end{subfigure}
    \hspace{2pt}
    \begin{subfigure}[b]{0.23\linewidth}
        \centering
        \includegraphics[width=\linewidth]{figs/evaluation/fault_model_based/frag_ratio_tp32_gr4.pdf}
        \caption{TP-32.}
        \label{fig:simulation:model:wasted:tp32}
    \end{subfigure}
    \hspace{2pt}
    \begin{subfigure}[b]{0.23\linewidth}
        \centering
        \includegraphics[width=\linewidth]{figs/evaluation/fault_model_based/frag_ratio_tp64_gr4.pdf}
        \caption{TP-64.}
        \label{fig:simulation:model:wasted:tp64}
    \end{subfigure}
    \vspace{-1ex}
    \caption{GPU wastes ratio with different GPU fault ratio, 4-GPU node.}
    \label{fig:simulation:model:wasted-gr4}
\end{figure*}



\begin{figure*}[h!t]
    \centering
    \begin{subfigure}[b]{0.23\linewidth}
        \centering
        \includegraphics[width=\linewidth]{figs/evaluation/fault_trace_based/breakdown_ratio_tp8_gr4.pdf}
        \caption{TP-8.}
        \label{fig:simulation:breakdown-duration:tp8-4gpu}
    \end{subfigure}
    \hspace{2pt}
    \begin{subfigure}[b]{0.23\linewidth}
        \centering
        \includegraphics[width=\linewidth]{figs/evaluation/fault_trace_based/breakdown_ratio_tp16_gr4.pdf}
        \caption{TP-16.}
        \label{fig:simulation:breakdown-duration:tp16-4gpu}
    \end{subfigure}
    \hspace{2pt}
    \begin{subfigure}[b]{0.23\linewidth}
        \centering
        \includegraphics[width=\linewidth]{figs/evaluation/fault_trace_based/breakdown_ratio_tp32_gr4.pdf}
        \caption{TP-32.}
        \label{fig:simulation:breakdown-duration:tp32-4gpu}
    \end{subfigure}
    \hspace{2pt}
    \begin{subfigure}[b]{0.23\linewidth}
        \centering
        \includegraphics[width=\linewidth]{figs/evaluation/fault_trace_based/breakdown_ratio_tp64_gr4.pdf}
        \caption{TP-64.}
        \label{fig:simulation:breakdown-duration:tp64-4gpu}
    \end{subfigure}
    \vspace{-1ex}
    \caption{Job fault-waiting duration with different levels of job-scale, 4 GPU node}
    \label{fig:simulation:breakdown-duration-supple}
\end{figure*}





\vspace{-12em}
\section{Detailed Cost and power consumption Analysis}
\label{appendix:cost}
In this section, \tabref{tab:eval:components} provides a detailed description of the quantity, cost, bandwidth, and power consumption of the interconnect components in various network architectures, including Google TPUv4~\cite{isca2023tpu}, NVIDIA GB200 NVL series~\cite{nvl72}, Alibaba HPN\cite{sigcomm2024hpn}, and \sys{}.


\begin{table*}[h!t] \small
    \centering
    \begin{tabular}{lllll}
    \toprule
    
    \textbf{Component} & \textbf{Quantity} & \textbf{Unit Cost (\$)}  & \textbf{Unit Bandwidth (GBps)} & \textbf{Unit Power (W)} \\

    \midrule
    \multicolumn{5}{c}{\textbf{Google TPUv4\cite{isca2023tpu} with 4096 GPU, bandwidth 300GBps/GPU}} \\
    
    \midrule
    OCS\cite{sigcomm2023lightwave} & 48 & 80000 & 6400 & 108 \\
    DAC Cable\cite{400G_DAC} & 5120 & 63.60 & 50 & 0.1 \\
    Optical Module\cite{400G_OPTICAL_MODULE} & 6144 & 360 & 50 & 12  \\
    Fiber\cite{FIBER}& 6144 & 6.80 & 50 & 0 \\
    
    \midrule
    \multicolumn{5}{c}{\textbf{NVIDIA GB200 NVL-36\cite{SEMIANALYSIS_GB200} with 36 GPU, bandwidth 900GBps/GPU}}\\
    \midrule
    NVLink Switch\cite{SEMIANALYSIS_Power} & 9 & 28000 & 3600 & 275 \\
    DAC Cable\cite{200G_DAC} & 2592 & 35.60 & 25 & 0.1 \\
    
    \midrule
    \multicolumn{5}{c}{\textbf{NVIDIA GB200 NVL-72\cite{nvl72}\cite{SEMIANALYSIS_GB200} with 72 GPU, bandwidth 900GBps/GPU}}\\
    \midrule
    NVLink Switch\cite{SEMIANALYSIS_Power} & 18 & 28000 & 3600 & 275 \\
    DAC Cable\cite{200G_DAC} & 5184 & 35.60 & 25 & 0.1 \\
    \midrule
    \multicolumn{5}{c}{\textbf{NVIDIA GB200 NVL-36x2\cite{SEMIANALYSIS_GB200} with 72 GPU, bandwidth 900GBps/GPU}}\\
    \midrule
    NVLink Switch\cite{SEMIANALYSIS_Power} & 36 & 28000 & 3600 &  275\\
    DAC Cable\cite{200G_DAC} & 6480 & 35.60 & 25 & 0.1 \\
    ACC Cable\cite{SEMIANALYSIS_Power} & 162 & 320 & 200 & 2.5 \\

    \midrule
    \multicolumn{5}{c}{\textbf{NVIDIA GB200 NVL-576\cite{SEMIANALYSIS_GB200} with 576 GPU, bandwidth 900GBps/GPU}}\\
    \midrule
    NVLink Switch\cite{SEMIANALYSIS_Power} & 432 & 28000 & 3600 & 275 \\
    DAC Cable\cite{200G_DAC} & 41472 & 35.60 & 25 & 0.1 \\
    Optical Module\cite{OSFPXD} & 4608 & 850 & 200 & 25 \\
    Fiber\cite{FIBER} & 4608 & 6.80 & 200 & 0 \\

    \midrule
    \multicolumn{5}{c}{\textbf{Alibaba HPN\cite{sigcomm2024hpn} with 16320 GPU, bandwidth 50GBps/GPU}}\\
    \midrule
    EPS\cite{51.2T_EPS} & 360 & 14960 & 6400 & 3145 \\
    DAC Cable\cite{200G_DAC} & 32640 & 35.60 & 25 & 0.1\\
    Optical Module\cite{400G_OPTICAL_MODULE} & 28800 & 360 & 50 & 12 \\
    Fiber\cite{FIBER} & 14400 & 6.80 & 50 & 0 \\

    \midrule
    \multicolumn{5}{c}{\textbf{\SYS{}($K=2$)  with 4 GPU, bandwidth 800GBps/GPU}}\\
    \midrule
    DAC Cable\cite{1.6T_DAC}& 4 & 199.60 & 200 & 0.1\\
    dOCS Module & 16 & 600 & 100 & 12 \\
    Fiber\cite{FIBER} & 16 & 6.80 & 100 & 0 \\

    \midrule
    \multicolumn{5}{c}{\textbf{\SYS{}($K=3$)  with 4 GPU, bandwidth 800GBps/GPU}}\\
    \midrule
    DAC Cable\cite{1.6T_DAC} & 2 & 199.60 & 200 & 0.1\\
    dOCS Module & 24 & 600 & 100 & 12 \\
    Fiber\cite{FIBER} & 24 & 6.80 & 100 & 0 \\
    \bottomrule
    \end{tabular}
    \caption{Interconnect cost and power consumption of components used in different network architectures.}
    \label{tab:eval:components}
\end{table*}


\end{appendices}








% You can have as much text here as you want. The main body must be at most $8$ pages long.
% For the final version, one more page can be added.
% If you want, you can use an appendix like this one.  

% The $\mathtt{\backslash onecolumn}$ command above can be kept in place if you prefer a one-column appendix, or can be removed if you prefer a two-column appendix.  Apart from this possible change, the style (font size, spacing, margins, page numbering, etc.) should be kept the same as the main body.
%%%%%%%%%%%%%%%%%%%%%%%%%%%%%%%%%%%%%%%%%%%%%%%%%%%%%%%%%%%%%%%%%%%%%%%%%%%%%%%
%%%%%%%%%%%%%%%%%%%%%%%%%%%%%%%%%%%%%%%%%%%%%%%%%%%%%%%%%%%%%%%%%%%%%%%%%%%%%%%


\end{document}


% This document was modified from the file originally made available by
% Pat Langley and Andrea Danyluk for ICML-2K. This version was created
% by Iain Murray in 2018, and modified by Alexandre Bouchard in
% 2019 and 2021 and by Csaba Szepesvari, Gang Niu and Sivan Sabato in 2022.
% Modified again in 2023 and 2024 by Sivan Sabato and Jonathan Scarlett.
% Previous contributors include Dan Roy, Lise Getoor and Tobias
% Scheffer, which was slightly modified from the 2010 version by
% Thorsten Joachims & Johannes Fuernkranz, slightly modified from the
% 2009 version by Kiri Wagstaff and Sam Roweis's 2008 version, which is
% slightly modified from Prasad Tadepalli's 2007 version which is a
% lightly changed version of the previous year's version by Andrew
% Moore, which was in turn edited from those of Kristian Kersting and
% Codrina Lauth. Alex Smola contributed to the algorithmic style files.
