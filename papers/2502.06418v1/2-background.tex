\section{Background}~\label{sec:background}
This section provides the necessary background for understanding our attack framework.

\begin{figure}[!t]
    \centering
    % Custom font size
    \includegraphics[width=\linewidth]{pics/watermarking_scheme.png} 
    
    % \vspace{-3mm}
    \caption{Illustration of learning-based watermarking methods.} 
    \label{fig:watermark}
    \vspace{-3mm}
\end{figure}

\subsection{Image Watermarking}
% Image watermarking includes injection, extraction, and verification. During watermark injection, an encoder receives the identification information and a target image as input and generates a watermarked image with the key information embedded (shown as Route xxx in Figure~\ref{fig:watermark}). Watermarked images circulate through online platforms such as social network websites and forums, enabling access by any user. During watermark extraction, the decoder extracts the identification key from the watermarked image (shown as Route xxx in Figure~\ref{fig:watermark}). 
Image watermarking includes injection, extraction, and verification. During watermark injection, an encoder $\mathcal{E}(\cdot,\cdot)$ receives the identification information $wm$ (``0011011" in Figure~\ref{fig:watermark}) and an original image $I$ as input and generates a watermarked image $I_{wm}$ with the key information embedded. During watermark extraction, the decoder $\mathcal{D}_{wm}(\cdot)$ extracts the identification key $wm'$ from the watermarked image $I_{wm}$ and then matches it 
 with $wm$ to verify whether the target watermark exists in the image.

\textbf{Non-learning-based and Learning-based Watermarking.} Non-learning-based methods build the encoder and decoder based on heuristics~\cite{jiang2023evading}. Learning-based methods deploy neural networks for the encoder and decoder, whose parameters are trained with deep learning techniques. Generally speaking, learning-based methods exhibit more robustness against distortions. In particular, they can incorporate a distortion layer before the decoder to mimic possible distortions and perform adversarial training(as shown in Figure~\ref{fig:watermark}) , causing the results of decoding a processed watermarked image $\hat{I_{wm}}$ to be identical to the one without experiencing distortions~\cite{jiang2023evading}.    

\textbf{Post-processing and In-processing Methods.} Post-processing watermarking adds a watermark to an image post its generation, following the same process of watermarking a real image~\cite{chopra2012lsb, DWT-DCT, stegastamp}. In contrast, in-processing watermarking embeds the identification message during the image generation process~\cite{YU1, yu2021responsible}. 

\subsection{Detection Evasion and Watermark Forgery}
Detection evasion means an attacker modifies a watermarked image to remove or disrupt the embedded watermark, causing the decoded bit string to deviate from the original identification information.

Watermark forgery involves extracting the watermark information $wm$ from the watermarked image $I_{wm}$ and embedding it into another non-watermarked image $I'$ to generate $I'_{wm}$, passing the verification of the watermark detector.