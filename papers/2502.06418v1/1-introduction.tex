\section{Introduction}
Watermarking is a well-recognized technique in the era of artificial intelligence-generated content (AIGC) for the purpose of content provenance and deepfake detection~\cite{jiang2024watermark,jiang2023evading}. Although the emergence of generative AI products such as Midjourney~\cite{Midjourney}, DALL-E~\cite{DALLE3}, and Sora~\cite{Sora} has lowered the bar of access to cutting-edge AI technology for ordinary users, it also benefits misleading content generation and the spread of misinformation for evildoers~\cite{fakenews1}. To address this, leading IT companies providing AIGC services, including OpenAI, Alphabet, and Meta, have committed to deploying watermark mechanisms in their products to make the technology safer~\cite{Whitehouse}. Watermark's capabilities of tracing, provenance, and detection of AIGC content facilitate the protection of content creators' intellectual property and the reputation of companies.

\begin{figure}[!t]
    \centering
    % Custom font size
    \includegraphics[width=\linewidth]{pics/intro.png} 
    
    % \vspace{-3mm}
    \caption{Demonstration of our attacks. An attacker can perform watermark removal and forgery attacks with only one watermarked image without knowledge about the underlying watermarking systems. The attacker is free of copyright violation accusations as the extracted watermark is incorrect; the attacker can spread fake news by forging the watermark of an authoritative media.} 
    \label{fig:intro}
    \vspace{-1mm}
\end{figure}


Watermarking methods improve robustness against distortions, enhancing practicality in real-world applications. When images spread across social media platforms, they often encounter distortions such as compression, noise addition, and screen-shooting~\cite{PIMoG}. Improving watermark robustness against these operations ensures the watermark remains detectable after Internet circulation.

Apart from distortions, watermarks are susceptible to adversarial attacks, mainly detection evasion and forgery attacks. These attacks can spread Not-Safe-for-Work (NSFW) content, copyright violations, and reputation undermining. An attacker can remove the watermark from an artist's imagery, thus evading the watermark detection and claiming ownership of the asset for commercial purposes. An attacker can also analyze the watermark pattern hidden in a watermarked image, extract and transfer it to a clean image depicting illegal content to make fake news trustworthy and damage a specific user or an organization's reputation. 

Studying the security of watermarking schemes is important due to its wide deployment in the AIGC era. We mainly discuss \textbf{the security of watermark schemes that incorporate a distortion layer to strengthen their robustness.} These watermarking methods are promising for practical use and exhibit better resilience when encountering attacks. 

Most watermark security research focuses on evading detection, and studies regarding forgery attacks are on the rise. \textbf{Detection Evasion:} We categorize relevant literature into adversarial example-based and reconstruction-based attacks. The former typically requires a large amount of watermark data—often including pairs of watermarked and original images—for perturbation training~\cite{lukas2024leveraging,yang2024steganalysis, saberi2023robustness} or direct access to the watermark detector for querying, sometimes even knowledge of the decoder~\cite{lukas2024leveraging}; the latter introduces noticeable modifications to the carrier image content during reconstruction~\cite{saberi2023robustness, zhao2023invisible, Kassis2024Unmarker}. \textbf{Watermark Forgery:} Watermark forgery represents a new research frontier, with its attack methodologies still in the developmental stages. The existing method relies on impractical assumption~\cite{kutter2000watermark}, requires access to the watermarking encoder~\cite{saberi2023robustness}, or demands substantial imagery that features the target watermark, which is ineffective against dynamic watermarks that incorporate time-sensitive nuance. 



In summary, existing detection evasion methods are computationally expensive, rely on strong assumptions, or significantly alter the original image's semantics. Meanwhile, forgery techniques are either impractical or high-cost, and the overall effectiveness of both attacks remains limited.

% A recent work, Unmarker~\cite{Kassis2024Unmarker}, an attack that disrupts both traditional and semantic watermarks by altering the spectral properties of watermarked images. However, it relies on cropping as an initial operation, which significantly distorts image content, limiting its applicability in scenarios where visual fidelity is essential.



% Kutter et al. is the first study to highlight the risk of forgery attacks, but the proposed method is applicable only when watermarks adhere to a Gaussian or Laplace distribution, which is an impractical assumption. Saberi et al. \cite{saberi2023robustness} require the encoder of a watermarking scheme, which undermines its practicality. Yang et al. \cite{yang2024steganalysis} demand substantial imagery that features the specific target of spoofing (that is, watermark). This type of attack is ineffective against dynamic watermarks that incorporate time nuance information.

% These methods often attack different watermarking methods, resulting in a lack of comprehensive evaluation of their performance. Additionally, we argue that some watermark schemes chosen as attack objectives are not suitable for highlighting the method's advantages, and the overall attack performance of existing attacks is underwhelming.

% In this paper, we reveal a key observation about watermarking systems robust against distortions and propose a watermark attack framework that is highly effective against them based on the observation. 

% \textbf{these schemes improve robustness by increasing watermark information redundancy. Unfortunately, the heightened redundancy level causes easier watermark leakage} 

% For instance, the learning-based watermarking system is forced to enlarge the area where the watermark pattern is associated or to increase the magnitude of the watermark pattern to compromise the compression effect, ensuring the remain watermark pattern is detectable. 

In this paper, we identify a key weakness in watermarking systems designed to be robust against distortions and propose a highly effective attack framework that exploits this vulnerability. Our observation is that \textbf{these systems enhance robustness by increasing watermark information redundancy. However, this heightened redundancy inadvertently makes watermark leakage easier} (see Sec.~\ref{sec:pilot} for details). Based on the observation, we \underline{\textbf{D}}elve into the \underline{\textbf{A}}spect of the \underline{\textbf{PA}}radox \underline{\textbf{O}}f Robust Watermarks and propose the \textbf{DAPAO} attack.


DAPAO attack is a framework that effectively extracts the digital fingerprint from watermarked images, capable of both evading watermark detection and forging watermark on clean images. We successfully identify the watermark leakage using \textbf{only one watermarked image} in the \textbf{no-box setting}, meaning no queries to the target watermarking system are required. Our method significantly outperforms state-of-the-art (SOTA) approaches in both watermark removal and forgery while preserving visual fidelity and semantic integrity. To extract the watermark-related feature, we utilized a typical neural network to extracting image features, representing in the form of multiple channels. We identify critical channels that have a bias toward watermark features and optimize learnable variables to align with the watermark characteristics and the semantics of the carrier image. These trained variables can be subtracted from the watermarked image for detection evasion and added to clean imagery for spoofing attacks. However, two key challenges arise: 1) accurately identifying the channels containing watermark features; 2) effectively forging semantic watermarks, where watermark features are deeply coupled with the carrier image’s semantics. 

To address these challenges, we propose an aggregation-based locating algorithm to identify key channels automatically, then optimize the learnable variable using weight-based loss function. The resulting variable facilitates both attack goals. For semantic watermarking, we propose to bridge the semantic difference between the carrier and the target images with a new optimization paradigm, successfully forging semantic watermarks. We conduct comprehensive evaluations to validate the effectiveness of DAPAO, benchmark its performance against existing methods, and perform ablation studies to analyze the contributions of each component in our framework.

\textbf{Summary of contributions.} In this paper, we make the following contributions:

\begin{itemize}
    \item We reveal the robustness-stealthiness paradox of watermark systems: Schemes improve watermark information redundancy to boost robustness against distortions, which results in watermark feature leakage that can be leveraged by attackers.
    \item Leveraging this observation, we propose DAPAO, a novel attack framework capable of both watermark removal and forgery against SOTA robust watermarking schemes. Our method requires only a single watermarked image for extraction and operates in a no-box setting. Additionally, we introduce a new approach to overcome the challenge of forging semantic watermarks, an understudied problem.
    
    \item Extensive experimental results demonstrate that our framework achieves notable improvement in attack performance over related work, with a 60\% success rate improvement in watermark detection removal and a 51\% improvement in forgery accuracy. 
    
\end{itemize}



% Based on the observation, we propose DAPAO attack, a framework performing both watermark detection evasion and forgery attacks targeting SOTA robust watermark schemes. Notably, the attack merely requires one image for watermark extraction and is a no-box attack (no need for a target watermark system access). Notably, we devise a novel approach that overcomes the forgery challenge for semantic watermarks.

