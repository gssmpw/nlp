\section{Problem Formulation}
% \begin{figure}[!t]
%     \centering
%     {\small \textbf{Watermark Forgery Attack}} \\[1mm]  % Custom font size
%     \includegraphics[width=\linewidth]{pic/intro_1.png} 
%     \label{fig:intro}
%     \vspace{-6mm}
%     \caption{Bob utilizes the GenAI service provided by Alice, where Alice embeds watermark information into the images returned to Bob. This embedded watermark allows the image to be identified as having been generated either by Bob or Alice through a watermark detection service. By forging the watermark onto illegal or malicious content, the attacker can cause the image to be misidentified as having been generated by Bob or Alice, thereby damaging the reputation of legitimate users or service providers. } 
% \end{figure}

\begin{figure}[!t]
    \centering
    % Custom font size
    \includegraphics[width=\linewidth]{pics/system-threatmodel.png} 
    
    % \vspace{-3mm}
    \caption{Typical watermarking application and security threats. Organizations and individuals use watermarking services to embed watermarks into images for purposes such as copyright protection or content regulation. When image ownership verification is required, the watermark is extracted and matched through the watermarking service. However, attackers can apply carefully designed post-processing techniques to remove or forge the watermark.} 
    \label{fig:models}
    % \vspace{-6mm}
\end{figure}

\subsection{System Model}
Figure~\ref{fig:models} illustrates the use case of a typical watermarking system. The process consists of the stages of watermark injection (encoding), data circulation, and watermark extraction (decoding), as shown in the gray portion of Figure~\ref{fig:models}. We primarily consider the post-processing watermarks. The three parties involved include \emph{users/organizations}, \emph{the verifier}, and the\emph{the attacker}.

% Watermarked images circulate through online platforms such as social network websites and forums, enabling access by users. 

\textbf{Users/service providers.} Users would like to use watermarking service before posting images online via social platforms to protect copyright. Alternatively, a service provider wants to mark all imagery generated by its own products, ensuring content provenance.

\textbf{The verifier.} To verify if an image contains the watermark, the verifier downloads target images from the Internet, decodes the image to extract watermark information, and then verifies the extracted one with the identification information. 

\subsection{Attacker's Goals} An attacker has two types of objectives. First, he would like to use an image without attributing it to the creator; therefore, he needs to evade the detection of watermarks. Second, he would like to improve the credibility of a fake image; therefore, he needs to 
forge a watermark related to an official account.

\subsection{Attacker's Capability} The attacker can download watermarked images uploaded by the victim, perform watermark removal or watermark spoofing on a clean image. Notably, we assume three realistic limitations: 1) The attacker \textbf{neither have knowledge} about the target watermarking system (i.e., encoder and decoder), \textbf{nor can he query the system}; 2) The attacker cannot obtain the original image; 3) The attacker must tackle watermark methods that are robust against distortions.




