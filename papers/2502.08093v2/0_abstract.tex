\begin{abstract}
% \textcolor{red}{Radar ensures robust sensing capabilities especially in adverse weather conditions, benchmarking as a promising sensor for robotics.
% Nevertheless, leveraging radars in robotics still poses challenges.
% The fundamental issue arises from the high level of noise. Intensity thresholding such as \ac{CFAR} has been employed, yet spatial uncertainties and generalizability issues persist.
% Another challenge is achieving accurate full 6-\ac{DOF} motion from doppler velocity with \ac{IMU}. Existing fusion strategy often exploited discretized propagation models with constant state assumptions, which are vulnerable to measurement errors.}

Radar ensures robust sensing capabilities in adverse weather conditions, yet challenges remain due to its high inherent noise level. Existing radar odometry has overcome these challenges with strategies such as filtering spurious points, exploiting Doppler velocity, or integrating with inertial measurements. This paper presents two novel improvements beyond the existing radar-inertial odometry: ground-optimized noise filtering and continuous velocity preintegration. Despite the widespread use of ground planes in LiDAR odometry, imprecise ground point distributions of radar measurements cause naive plane fitting to fail. Unlike plane fitting in LiDAR, we introduce a zone-based uncertainty-aware ground modeling specifically designed for radar. Secondly, we note that radar velocity measurements can be better combined with IMU for a more accurate preintegration in radar-inertial odometry.
Existing methods often ignore temporal discrepancies between radar and IMU by simplifying the complexities of asynchronous data streams with discretized propagation models.
Tackling this issue, we leverage GP and formulate a continuous preintegration method for tightly integrating 3-DOF linear velocity with IMU, facilitating full 6-DOF motion directly from the raw measurements.
Our approach demonstrates remarkable performance (less than 1\% vertical drift) in public datasets with meticulous conditions, illustrating substantial improvement in elevation accuracy. The code will be released as open source for the community: \textcolor{blue}{https://github.com/wooseongY/Go-RIO}.
\end{abstract}

% different temporal resolutions
% synch

%In this context, we propose ground-optimized radar-inertial odometry algorithm with continuous velocity integration.
%Since the ground can serve as a key feature in mitigating inherent noise, we introduce radar-specific ground modeling for reliable filtering.
% to mitigate imprecise ground distributions in radar.
% Moreover, utilizing weighted scan matching with velocity integration ensures robustness even with the measurement drifts.
%Secondly, even with an \ac{IMU}, achieving accurate 6-\ac{DOF} motion from radar reveals limitations arising from existing discretized models and constant state assumption. 