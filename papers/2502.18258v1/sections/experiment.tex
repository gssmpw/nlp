\section{Experimental Design}


We present the experimental design, including studied datasets, baselines, evaluation metrics, and experiment settings.


\subsection{Datasets}
\label{sec:dataset}


We construct two datasets: one Ethereum dataset (abbreviated as ETH)~\cite{awesomeBigquery2022} and one Bitcoin dataset (abbreviated as BTC)~\cite{vanclassifying}. 
After necessary pre-processing and sampling, our dataset contains 16,384 blocks, with transactions reorganized into the format \( O_i = \langle A_i, a_i, t_i \rangle \), where \( A_i \) represents the amount, \( a_i \) represents the set of addresses involved, and \( t_i \) represents the timestamp.
To create a multimodal dataset, we incorporate an \textit{imagecid} or \textit{videocid} value into the formatted transactions, establishing a link between on-chain transactions and off-chain data. 
The final structure is a quintuple \( O'_i = \langle A_i, a_i, t_i, \textit{imagecid}_i, \textit{videocid}_i \rangle \), where one of the \textit{cid}s may be null. Here, \(\textit{imagecid}_i\) and \(\textit{videocid}_i\) represent the content identifiers (CIDs) for images and videos, respectively. 
The image dataset consists of 10,000 art images in the 24x24, 8-bit style of CryptoPunk, composed of irregular pixels, while the video dataset includes 100 videos from opensea.io, a video NFT platform.


% \begin{table}[!htbp]
%     % \vspace{-0.1cm}
%     \centering
%     \caption{Statistics of constructed dataset}
%     \resizebox{\linewidth}{!}
%     {
%         \begin{threeparttable}
%         \begin{tabular}{lrrrr}
%         \toprule
%         \textbf{Dataset} & \textbf{\# Total Bugs} & \textbf{\# SF Bugs} & \textbf{\xy{\# SH Bugs}} & \textbf{\xy{\# SL Bugs}} \\
%         \midrule
%         Defects4J 1.2 & 391 & 255 & \xy{154} & \xy{80}  \\
%         Defects4J 2.0 & 438 & 228 & \xy{159} & \xy{78} \\
%         QuixBugs-Java & 40 & 40 & \xy{37} & \xy{37} \\
%         QuixBugs-Python & 40 & 40 & \xy{40} & \xy{40} \\
%         \midrule
%         \textbf{\# Sum} & \textbf{909} & \textbf{563} & \xy{\textbf{390}} & \xy{\textbf{235}}\\
%         \bottomrule
%         \end{tabular}
%         $^\ast$Defects4J 1.2 and Defects4J 2.0 are two completely independent versions, with no overlapping bugs between them.
%         \end{threeparttable}
%     } 
%     % \vspace{-0.3cm}
%     \label{tab:dataset}
% \end{table}


\subsection{Baselines and Evaluation Metrics}


{\bf Studied Baselines.} 
Many studies have examined verifiable query processing over blockchain databases. 
vChain+~\cite{wang2022vchain+} employs cryptographic set accumulators to enable verifiable boolean range queries.
Therefore, we use vChain+ as the baseline and replicate vChain+.
For a fair comparison, we add an off-chain query module with IPFS to make vChain+ capable of off-chain queries.
The on-chain-only and full version is denoted as vChain+$_O$ and vChain+$_F$, respectively.

% In the experiment, we compare \Chain wisth the five methods in Table~\ref{tab:baselines}:


% \begin{itemize}[leftmargin=*]
%     \item MulChain$_V$: vChain+ reimplemented to support on/off-chain queries, referred to as MulChain$_{VB}$ for Bitcoin and MulChain$_{VE}$ for Ethereum.

%     \item MulChain$_O$: MulChain$_V$ without off-chain queries, labeled as MulChain$_{OB}$ and MulChain$_{OE}$.

%     \item MulChain$_{BT}$: MulChain with B\(+\)Tree for time range queries, represented as MulChain$_{BTB}$ and MulChain$_{BTE}$.

%     \item MulChain$_{BH}$: MulChain with gas-efficient BHashTree for time range queries, shown as MulChain$_{BHB}$ and MulChain$_{BHE}$; 

%     \item MulChain$_T$: MulChain with trie for fuzzy queries, identified as MulChain$_{TB}$ and MulChain$_{TE}$.
% \end{itemize}


\noindent
{\bf Evaluation Metrics.} 
We evaluate the performance of each workload by breaking it down into (1) {Index Construction Cost}, the CPU time cost to insert multimodal data into the blockchain and off-chain storage; (2) {Query Latency}, the time cost for querying metadata from the blockchain and multimodal data from off-chain storage; and (3) {VO Size}. Based on the Awesome BigQuery Views project\cite{awesomeBigquery2022}, we designed six types of SQL Queries to test \Chain.


\subsection{Implementation}
\label{sec:setting}


We experiment on a Ubuntu 22.04 workstation equipped with an Intel\(^\text{\textregistered}\) Core\texttrademark\ i7-12700KF Processor and 48 GB memory. 
In the experiment, we deploy a private Ethereum network with Geth and a FISCO BCOS permissioned chain, configured with nodes in a local network with a speed of 100 Mbps. 
We reimplement vChain+, the SOTA open-source system in the on-chain query domain, as our baseline (labelled as vChain+$_{O}$ and vChain+$_{F}$). \Chain is implemented with smart contracts Solidity and deployed with web3.py library in Python.
For each experiment, we randomly generate 2,046 queries based on preset templates and report average results for all metrics.