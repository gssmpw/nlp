\subsection{RQ-2 Comparable Study on Fuzzy Query}

\noindent
\textbf{Objective.}
Benefiting from the powerful query capability of trie structures, many fuzzy query approaches have been proposed~\cite{tahani1977conceptual,fagin1998fuzzy}.
However, no such approaches support blockchain-level fuzzy queries. 
To mitigate that, we propose a verifiable trie structure in Section~\ref{sec:trie}.
The experiments are conducted to investigate whether \Chain achieves a considerable fuzzy query performance on blockchain databases.


\noindent
\textbf{Experimental Design.}
We consider two baselines in our ablation study: 
vChain+$_{O}$ and vChain+$_{F}$.
Here, vChain+$_{F}$ represents the vChain+ replication version to support on/off-chain queries, referred to as vChain+$_{FB}$ for Bitcoin and vChain+$_{FE}$ for Ethereum.
vChain+$_{O}$ is vChain+$_{F}$ without the off-chain query module, labeled as vChain+$_{OB}$ and vChain+$_{OE}$.
MulChain$_{T}$ is MulChain with its underlying data structure replaced with our verifiable trie for fuzzy queries, identified as MulChain$_{TB}$ and MulChain$_{TE}$.

% \begin{itemize}[leftmargin=*]
%     \item MulChain$_V$: vChain+ reimplemented to support on/off-chain queries, referred to as MulChain$_{VB}$ for Bitcoin and MulChain$_{VE}$ for Ethereum.

%     \item MulChain$_O$: MulChain$_V$ without off-chain queries, labeled as MulChain$_{OB}$ and MulChain$_{OE}$.

%     \item MulChain$_{BT}$: MulChain with B\(+\)Tree for time range queries, represented as MulChain$_{BTB}$ and MulChain$_{BTE}$.

%     \item MulChain$_{BH}$: MulChain with gas-efficient BHashTree for time range queries, shown as MulChain$_{BHB}$ and MulChain$_{BHE}$; 

%     \item MulChain$_T$: MulChain with trie for fuzzy queries, identified as MulChain$_{TB}$ nad MulChain$_{TE}$.
% \end{itemize}


\noindent
\textbf{Results.}
MulChain$_T$ is 30X faster than vChain+$_F$.
As shown in Fig.~\ref{fig:Fuzzy Query Performance}(a), the insertion overhead aligns with that of the other two query types, which is expected since datasets are inserted uniformly across all three queries. Fig.~\ref{fig:Fuzzy Query Performance}(b) indicates that the accumulator-based MulChain$_V$ exhibits an exponential growth trend in the number of blocks. At the same time, our trie demonstrates \( O(l) \) time complexity. Notably, fuzzy query support was absent in vChain+ before our reimplementation. The query latency for the ETH dataset is significantly higher than that for the BIT dataset. This is due to the denser timestamp distribution in the ETH dataset. We intentionally configured this distribution to test fuzzy queries under varying workload intensities. The average latency of MulChain$_V$ is 30X higher than that of MulChain$_T$, highlighting the superiority of the trie over the accumulator-based approach. In Fig.~\ref{fig:Fuzzy Query Performance}(c), it is evident that our trie method consumes 10.7X more VO size than the accumulator-based method in the worst case. However, this outcome is anticipated as we achieve an acceptable trade-off between VO size and latency, as shown in Fig.~\ref{fig:Fuzzy Query Performance}(d).


\begin{figure}[htbp]
    \centering
    \includegraphics[width=\linewidth]{figures/FuzzyQ_all.pdf}
    \caption{Fuzzy Query Performance}
    \label{fig:Fuzzy Query Performance}
\end{figure}

\intuition{
{\bf Answer to RQ-2}: 
MulChain$_{T}$, using the verifiable trie, is 30X faster than vChain+$_{F}$ for fuzzy queries, demonstrating \( O(l) \) time complexity, compared to the exponential growth of the accumulator in vChain+$_{F}$. 
Although it incurs 10.7X more VO size in the worst case, this trade-off is acceptable due to significantly reduced query latency. 
The trie outperforms other approaches, especially under varying workload intensities, making it a superior choice for blockchain fuzzy queries.
}