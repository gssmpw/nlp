\section{Motivation}


\subsection{A Motivation Example}


As the requirement of storage for multimodal data is increasing exponentially in blockchain, the storage system is required to offer more sophisticated and comprehensive data queries~\cite{V2FS2024}. In Fig.~\ref{fig:motivation}, we present two representative queries that highlight the application of multimodal data in hybrid-storage blockchains. Fig.~\ref{fig:motivation}(a) illustrates a query designed for an NFT art marketplace, where it retrieves metadata from on-chain sources alongside multimedia data, such as images and videos, stored off-chain. This query aggregates user interactions by counting the number of favourites per NFT, ensuring the retrieval of relevant, available NFTs from January 2023, and combining both on-chain and off-chain information.

Fig.~\ref{fig:motivation}(b) showcases a query used in decentralized finance (DeFi) credit scoring, which integrates on-chain user data with off-chain credit information. By joining these datasets, the query computes the total liquidity of users with a credit score above a certain threshold, utilizing both transaction data and off-chain credit scores to derive a comprehensive assessment of the users' financial behaviour and standing. The use of both on-chain and off-chain data enables a more robust, holistic view of user activity and financial trustworthiness.


% \begin{figure}[htbp]
% \centering
% \begin{minipage}[t]
% {0.54\textwidth}
% \begin{minted}[frame=single, breaklines, fontsize=\small]{sql}
% SELECT n.nft_id, n.title, n.creator, m.image_url, m.video_url, COUNT(f.favorite_id)
% FROM on_chain_nft_metadata n
% JOIN off_chain_multimedia_data m
%   ON n.nft_id = m.nft_id
% WHERE n.status = 'available'
%   AND n.creation_date LIKE '2023-01%'
% GROUP BY n.nft_id, n.title, n.creator, m.image_url, m.video_url;
% \end{minted}
% \centering
% (a) NFT Art Marketplace
% \end{minipage}
% \hfill
% \begin{minipage}[t]{0.45\textwidth}
% \begin{minted}[frame=single, breaklines, fontsize=\small]{sql}
% SELECT u.user_id, SUM(t.amount) AS total_liquidity
% FROM on_chain_user_data u
% JOIN off_chain_credit_data c
%   ON u.user_id = c.user_id
% LEFT JOIN on_chain_transactions
%   ON u.user_id = t.user_id
% WHERE c.off_chain_score > '600'
%   AND u.on_chain_score > '50'
% GROUP BY u.user_id;
% \end{minted}
% \centering
% (b) DeFi Credit Scoring
% \end{minipage}
% % \captionsetup{skip=4pt}
% \caption{Querying Multimodal Data on Hybrid-Storage Blockchain}
% \label{fig:motivation}
% \end{figure}


\begin{figure}[htbp]
    \centering
    \includegraphics[width=\linewidth]{figures/image.png}
    \caption{Querying Multimodal Data on Hybrid-Storage Blockchain}
    \label{fig:motivation}
\end{figure}


\begin{figure}[htbp]
    \centering
    \includegraphics[width=\linewidth]{figures/motivation_1.pdf}
    \caption{Traditional Hybrid Storage and \Chain Middleware}
    \label{fig:motivation_1}
\end{figure}


\textbf{Observation 1.} 
As shown on the right side of Fig.~\ref{fig:motivation_1}, querying URLs for data such as images and videos through simple queries and accessing them directly poses significant security risks.
This is primarily because it bypasses necessary security and privacy protocols inherent in hybrid-storage blockchain systems.
Specifically, fetching multimodal data through URLs could expose sensitive data to unauthorized access, compromising the integrity and confidentiality of both on-chain and off-chain resources. In such cases, malicious actors could exploit these URLs to directly manipulate or retrieve information from off-chain storage systems, undermining the trust in the blockchain's decentralized security framework.


To mitigate this risk, the process of querying multimodal data must be controlled and mediated through a secure middleware layer (as shown on the left side of Fig.~\ref{fig:motivation_1}).
In this approach, the query first reaches the middleware, which acts as a trusted intermediary between the user and the blockchain. 
The middleware employs a SQL query parser to analyze and parse the incoming request, ensuring that only valid and authorized queries are processed. Upon successful parsing, the middleware then interacts with the blockchain through a smart contract, executing the query on the blockchain and retrieving the relevant results. 
These results typically include both on-chain metadata, such as text data, and off-chain resources, such as multimedia content identified by their unique content identifiers (CIDs).


Once the query results are obtained, the middleware further resolves the CIDs, using them to fetch the raw multimedia data (e.g., images and videos) from an InterPlanetary File System (IPFS)~\cite{benet2014ipfscontentaddressed} or other decentralized storage solutions. 
This ensures that the process of accessing and retrieving multimodal data remains secure, as the security of the middleware is guaranteed by the verifiable data structures in Section~\ref{sect:Structures}.


\textbf{Observation 2.} 
Existing blockchains offer two strategies for verifiable query processing.
\ding{182} The first approach involves blockchain-based databases, which typically support only basic queries, such as boolean range queries~\cite{chang2023anole,wang2022vchain+,xu2019vchain}. These systems are often self-designed, and while they can execute straightforward queries on on-chain data, they are not well-suited for handling complex, multimodal queries that involve both on-chain and off-chain data sources. 
Moreover, these blockchain-based databases are inherently closed systems, making it difficult to integrate them with other blockchain platforms or external storage solutions. As seen in Figure \ref{fig:motivation}(a) and (b), which feature queries that span both on-chain and off-chain data, such as NFTs with multimedia content (images and videos), such systems cannot handle the complexity or the cross-modal nature of these queries. This limitation is especially critical given that current blockchain storage methods fail to account for the necessity of supporting cross-modal data retrieval, including the secure fetching of multimedia files (e.g., images and videos) without exposing sensitive information through URLs.
\ding{183} The second strategy is outsource-based databases that enable more complicated queries~\cite{antonopoulos2021sql,peng2020falcondb,yang2020ledgerdb,yue2022glassdb,zhang2017vsql,zhang2015integridb}, such as time range queries. However, it introduces risks of information leakage due to reliance on external storage, and most of the existing methods support only a single query type and lack compatibility with others.
As illustrated in Figure \ref{fig:motivation}(a) and (b), the querying process involves data from multiple sources—on-chain metadata, off-chain multimedia data, and sometimes cross-modal relationships between them. For example, in the NFT art marketplace scenario, the query aggregates metadata from both the blockchain and off-chain multimedia storage to count user interactions with NFTs. In the DeFi credit scoring case, user data is drawn from both on-chain and off-chain sources, combining transaction data with credit scores. These queries not only require the ability to handle multimodal data but also need to perform cross-modal operations that seamlessly combine text, images, and video data in a secure, efficient manner. Unfortunately, existing blockchain storage strategies do not accommodate this growing need for cross-modal query support, particularly when security and privacy are critical considerations.
Due to these limitations, users are restricted in their ability to conduct comprehensive and diverse data queries, which hinders the efficient management and utilization of data in Web 3.0 and the metaverse.


\intuition{{\bf Intuition.}
Existing strategies respectively face the following limitations: (1) Lack of support for complex queries, and are difficult to be compatible with other blockchain systems; (2) Lack of support for cross-modal queries.  
}


\subsection{Key Ideas}
Based on the above observations, we aim to design a novel middleware architecture for hybrid-storage queries that offers high compatibility, supports diverse query operations, and enables efficient cross-modal data querying.


However, designing a system that supports hybrid-storage and cross-modal data querying introduces several new challenges~\cite{MSTDB2023}.
Firstly, maintaining compatibility with existing blockchain systems is crucial because popular blockchain networks (e.g., Bitcoin~\cite{nakamoto2008bitcoin} and Ethereum~\cite{wood2014ethereum}) have enormous market values, and any modifications to their underlying systems would be impractical and inconvenient~\cite{V2FS2024}.


Secondly, the system is required to support different types of SQL queries to meet clients' demands, necessitating flexible and comprehensive query processing solutions. Thirdly, real-time querying presents a challenge due to inherent latency and performance limitations in blockchain networks, requiring efficient mechanisms to enable timely data retrieval. 
Lastly, supporting queries over multimodal data is essential, as Decentralized Applications (DApps) often need to handle various data types and formats.


To address these challenges, we propose \textit{\Chain}, a novel middleware architecture to enable advanced queries for multimodal data. 


\textbf{(1) Novel Middleware Architecture.}  
Traditional hybrid-storage blockchains store multimodal data in a two-step process: metadata (e.g., URL, hash value) is stored on-chain, and multimedia content (e.g., images and videos) is stored off-chain (e.g., IPFS). However, this architecture lacks compatibility with different blockchain systems and off-chain storage solutions, and it does not support complex, cross-modal queries. To address these issues, we propose \Chain, a novel middleware architecture that ensures compatibility across various blockchain systems (e.g., FISCO BCOS, Ethereum) and off-chain storage, while supporting advanced queries involving both on-chain and off-chain data.
Additionally, \Chain's secure middleware layer acts as an intermediary between the user and the blockchain, verifying queries and securely fetching off-chain data to prevent unauthorized access, ensuring the integrity of the entire system.


\textbf{(2) Verifiable BHashTree and Trie.}  
Efficient blockchain queries require handling multimodal data from diverse sources while ensuring security. To enable this, \Chain incorporates Verifiable BHashTree and Trie data structures. The BHashTree optimizes range and time-range queries by dynamically adapting between B+Tree and hash table representations based on query workload, ensuring efficient storage and retrieval in hybrid-storage scenarios. The Trie, on the other hand, excels at handling fuzzy queries where data may have variable formats or slight deviations, such as timestamps or addresses. This enables efficient retrieval even when exact matches are unavailable.


Both the BHashTree and Trie are integrated into \Chain to provide a verifiable, gas-efficient query mechanism for multimodal data, preserving blockchain security while supporting cross-modal queries. This combination allows for seamless retrieval and integration of on-chain metadata and off-chain multimedia data, enabling more powerful and flexible query processing. 