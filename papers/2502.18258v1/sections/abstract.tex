\begin{abstract}

With its decentralization and immutability, blockchain has emerged as a trusted foundation for data management and querying. 
Because blockchain storage space is limited, large multimodal data files, such as videos, are often stored offline, leaving only lightweight metadata on the chain.
While this hybrid storage approach enhances storage efficiency, it introduces significant challenges for executing advanced queries on multimodal data. 
The metadata stored on-chain is often minimal and may not include all the attributes necessary for queries like time range or fuzzy queries.
In addition, existing blockchains do not provide native support for multimodal data querying. 
Achieving this capability would necessitate extensive modifications to the underlying blockchain framework, even reconstructing its core architecture.
Consequently, enabling blockchains with multimodal query capabilities remains a significant problem, which necessitates overcoming the following three key challenges: (1) Designing efficient indexing methods to adapt to varying workloads that involve frequent insertions and query operations; (2) Achieving seamless integration with existing blockchains without altering the underlying infrastructure; (3) Ensuring high query performance while minimizing gas consumption. 
To address these challenges, we propose \textit{\Chain}, a novel middleware architecture to enable smooth integration with existing blockchains. 
At the core of \Chain is the BHashTree, a flexible data structure that dynamically switches between tree and hash nodes based on workload characteristics, ensuring efficient insertion and query operations. 
Furthermore, the middleware provides standardized interfaces for blockchain systems, unifying query methods across different platforms. 
Finally, Experiments on Ethereum and FISCO-BCOS demonstrate that \Chain outperforms the state-of-the-art vChain+ by up to 78.12X in query performance and up to 99.49\% reduction in verification object size, respectively.

\end{abstract}
