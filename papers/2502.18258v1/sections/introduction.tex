\section{Introduction}
Blockchain is regarded as a promising technology for data storage and management in Web 3.0 and the metaverse~\cite{gadekallu2022blockchainmetaversereview}. 
As the digital world evolves at an unprecedented pace, Web 3.0 and the metaverse are becoming integral parts of this new era of online interaction.
Blockchain technology is at the heart of these advancements, playing a crucial role in enabling decentralization, trust, and ownership in virtual environments.
It enables decentralized ownership through NFTs, allowing users to truly own and control their digital assets.    
Blockchain also supports portable and sovereign digital identities, enhancing privacy and security in virtual environments. 
Smart contracts automate transactions and governance, while cryptocurrencies facilitate digital economies. 
Moreover, blockchain's interoperability enables seamless asset and data transfer across different virtual worlds, paving the way for a more immersive and interconnected digital future.


Due to the limited storage resource of blockchains, a hybrid-storage strategy is typically employed for storing diverse content, where large volumes of multimodal data (e.g., videos, audio, and images) are stored off-chain, while only transactions and metadata are retained on-chain~\cite{arer2022efficient,liu2024mpv,MSTDB2023,zhu2019sebdb}. 
For example, in an NFT art marketplace, the metadata (e.g., IDs, titles, creators, ownership history) of NFTs is stored on-chain, while the actual multimedia contents (e.g., high-resolution images and promotional videos) are stored on off-chain storage (e.g., IPFS~\cite{benet2014ipfscontentaddressed}). 
By utilizing data from hybrid-storage blockchains, users can view and verify the contents associated with NFT artworks~\cite{grtandnfts23}.
In a DeFi application, on-chain data provides information about users' liquidity, loan repayments, etc., while off-chain data includes traditional credit scores and income records. By combining on-chain transaction history with off-chain credit data, a user's financial reliability can be evaluated.


As the demand for storing multimodal data in blockchain systems grows rapidly, there is a pressing need for storage solutions that can handle more sophisticated and comprehensive query operations~\cite{V2FS2024}. For example, querying for NFT details often requires combining on-chain metadata with off-chain multimedia data, involving complex operations like fuzzy and string matching. Similarly, calculating users' total liquidity in decentralized finance (DeFi) applications may require filtering both on-chain and off-chain credit scores through range queries.
Currently, blockchain systems typically support two strategies for verifiable query processing. 
The first strategy involves blockchain-based databases, which can handle only basic query types, such as boolean range queries~\cite{chang2023anole,wang2022vchain+,xu2019vchain}. 
While simple, this approach is limited in query complexity and tends to be proprietary, making it hard to be compatible with other blockchain systems~\cite{V2FS2024}. 

The second strategy involves outsourcing queries to external databases, which can handle more complex queries, such as time range queries~\cite{antonopoulos2021sql,peng2020falcondb,yang2020ledgerdb,yue2022glassdb,zhang2017vsql,zhang2015integridb}. 
However, this approach introduces significant risks, particularly the potential for information leakage, as it relies on external storage. 
Furthermore, many existing solutions only support a single type of query, lacking compatibility with other query methods. 


As the variety of stored data types increases—encompassing text, images, videos, and more—the need for efficient cross-modal data querying becomes increasingly important. 
Existing blockchain storage systems do not address this need, especially in terms of supporting both hybrid-storage queries and efficient cross-modal operations. 
In this paper, we propose a novel middleware architecture for hybrid-storage queries, designed to offer high compatibility, support a broad range of query operations, and enable efficient cross-modal data querying.


\begin{table*}
\centering
% \captionsetup{skip=4pt}
\caption{Comparison with Existing Query Systems}
\label{tab:Comparison}
\resizebox{\textwidth}{!}{
\begin{tabular}{cccccc}
  \toprule
  \textbf{Category} &
  \textbf{Name} &
  \makecell{\textbf{Query} \\ \textbf{Language}} &
  \makecell{\textbf{Blockchain} \\ \textbf{Compatible}} &
  \makecell{\textbf{Multiple} \\ \textbf{Advanced} \textbf{Queries}} &
  \makecell{\textbf{Off-Chain}\\ \textbf{Compatible}} \\ 
  \midrule
  \multirow{6}{*}{\textbf{\begin{tabular}[c]{@{}c@{}}Outsourced \\ Database\end{tabular}}} &
  IntegriDB~\cite{zhang2015integridb} &
  Semi-SQL &
  N/A &
  \textcolor{red}{\textbf{\ding{55}}} &
  \textcolor{red}{\textbf{\ding{55}}} \\
 &
  FalconDB~\cite{peng2020falcondb} &
  Semi-SQL &
  N/A &
  \textcolor{red}{\textbf{\ding{55}}} &
  \textcolor{red}{\textbf{\ding{55}}} \\
 &
  vSQL~\cite{zhang2017vsql} &
  SQL &
  N/A &
  \textcolor{red}{\textbf{\ding{55}}} &
  \textcolor{red}{\textbf{\ding{55}}} \\
 &
  SQL Ledger~\cite{antonopoulos2021sql} &
  SQL &
  N/A &
  \textcolor{red}{\textbf{\ding{55}}} &
  \textcolor{red}{\textbf{\ding{55}}} \\
 &
  LedgerDB~\cite{yang2020ledgerdb} &
  Read &
  N/A &
  \textcolor{red}{\textbf{\ding{55}}} &
  \textcolor{red}{\textbf{\ding{55}}} \\
 &
  GlassDB~\cite{yue2022glassdb} &
  Read &
  N/A &
  \textcolor{red}{\textbf{\ding{55}}} &
  \textcolor{red}{\textbf{\ding{55}}} \\ 
  \midrule
  \multirow{6}{*}{\textbf{\begin{tabular}[c]{@{}c@{}}Blockchain \\ Database\end{tabular}}} &
  vChain~\cite{xu2019vchain}, vChain+~\cite{wang2022vchain+} &
  Boolean Range &
  \textcolor{red}{\textbf{\ding{55}}} &
  \textcolor{red}{\textbf{\ding{55}}} &
  \textcolor{red}{\textbf{\ding{55}}} \\
 &
  GEM$^2$-tree~\cite{GEM^2-Tree} &
  Range &
  \textcolor{red}{\textbf{\ding{55}}} &
  \textcolor{red}{\textbf{\ding{55}}} &
  \textcolor{red}{\textbf{\ding{55}}} \\
 &
  LVQ~\cite{dai2020lvq} &
  Membership &
  \textcolor{red}{\textbf{\ding{55}}} &
  \textcolor{red}{\textbf{\ding{55}}} &
  \textcolor{red}{\textbf{\ding{55}}} \\
 &
  The Graph~\cite{thegraph2022} &
  GraphQL &
  \textcolor{green}{\textbf{\ding{51}}} &
  \textcolor{red}{\textbf{\ding{55}}} &
  \textcolor{red}{\textbf{\ding{55}}} \\
 &
  V2FS~\cite{V2FS2024} &
  Various Types &
  \textcolor{green}{\textbf{\ding{51}}} &
  \textcolor{green}{\textbf{\ding{51}}} &
  \textcolor{red}{\textbf{\ding{55}}} \\
 &
 \textbf{\Chain(ours)} &
  SQL &
  \textcolor{green}{\textbf{\ding{51}}} &
  \textcolor{green}{\textbf{\ding{51}}} &
  \textcolor{green}{\textbf{\ding{51}}} \\ 
  \bottomrule
\end{tabular}%

}
\end{table*}


However, designing a system that supports hybrid storage and cross-modal data querying presents a range of new challenges, as highlighted in Table~\ref{tab:Comparison}~\cite{MSTDB2023}. 
First, ensuring compatibility with existing blockchain frameworks is of paramount importance, as popular networks like Bitcoin~\cite{nakamoto2008bitcoin} and Ethereum~\cite{wood2014ethereum} boast significant market presence. 
Modifying their core structures would be impractical and disruptive~\cite{V2FS2024}.
Second, the system must accommodate a wide array of SQL queries to meet diverse client requirements, demanding robust and adaptable query processing capabilities. 
Third, real-time query performance poses a major hurdle due to the inherent latency and throughput limitations of blockchain networks, necessitating the development of optimized mechanisms for timely data retrieval. 
Finally, supporting multimodal data queries is crucial, as Decentralized Applications (DApps) often handle a variety of data types and formats, requiring an integrated approach for seamless querying.


To address these challenges, we propose \textit{\Chain}, a novel middleware architecture to enable blockchain compatibility and advanced queries for multimodal data. 
\Chain is a modular and pluggable middleware that orchestrates seamless integration between existing blockchain systems (e.g., Ethereum) and off-chain storage solutions (e.g., IPFS). 
Based on this, we present a gas-efficient BHashTree alongside a high-performance trie designed to maximize efficiency and enhance query capabilities.
Note that both index structures are verifiable. 
This approach differs from previous indexing methods and supports various advanced query functions. 
Furthermore, we provide a handful of advanced SQL query primitives supporting time range and fuzzy queries. To make a fair comparison with \Chain, we reimplemented vChain+ to support cross-modal queries on hybrid-storage blockchains. 
% \textcolor{red}{
Experimental results show that \Chain surpasses state-of-the-art accumulator-based vChain+ by up to 22.67X speedup in simple queries, up to 78.13X speedup in time range queries, 30X speedup in fuzzy queries and reduces the VO size by up to 99.49\%.
% }
In summary, the key contributions of this paper include:

\begin{itemize}[leftmargin=*]
    \item We propose \Chain, a pluggable middleware architecture to enable advanced cross-modal queries without modifying existing blockchains. 
    \Chain provides standardized interfaces and unifies query methods across different platforms, ensuring interoperability between blockchains and off-chain storage solutions. To the best of our knowledge, this is the first work to enable advanced cross-modal queries in hybrid-storage blockchains.

    \item We design and implement BHashTree, a flexible index structure that dynamically adapts between tree and hash nodes based on workload characteristics. Also we implement trie to support high performance fuzzy query. These data structures empower the system to support multiple cross-modal query types.

    \item We experiment on two popular blockchains (Ethereum and FISCO BCOS) and one off-chain storage (IPFS). 
    Experimental results show that \Chain achieves up to 78.12X speedup in on-chain queries and reduces the VO size by 99.49\% compared to the SOTA vChain+.

    \item To support the open science community, we publish the studied dataset and source code of our model with supporting scripts on GitHub ({\url{https://github.com/pzy2000/MulChain}}), which provides a ready-to-use implementation of our model for future research about comparison.
\end{itemize}
