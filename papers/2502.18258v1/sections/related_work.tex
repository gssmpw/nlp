\section{Related Work}

Table~\ref{tab:Comparison} compares various query processing systems. These systems are classified into two categories: {outsourced databases} and {blockchain databases}.


\subsection{Outsourced Databases}
Numerous systems have been developed to tackle the challenges of data integrity, security, and query verifiability in cloud-based outsourced databases. IntegriDB~\cite{zhang2015integridb} addresses data integrity by allowing clients to verify the correctness of query results from outsourced databases, but it supports only a limited subset of SQL queries. FalconDB~\cite{peng2020falcondb} enhances verifiable range queries over outsourced data, providing improved query capabilities; however, it faces scalability issues when dealing with large datasets, which hampers its performance in big data applications. vSQL~\cite{zhang2017vsql} proposes a protocol for verifiable SQL queries on dynamic outsourced databases, offering strong security guarantees, yet it incurs substantial computational overhead, making it less suitable for real-time query processing. SQL Ledger~\cite{antonopoulos2021sql} leverages blockchain to ensure the integrity and immutability of database transactions but lacks support for rich query functionalities and efficient data retrieval mechanisms, limiting its applicability in scenarios requiring complex data analysis. LedgerDB~\cite{yang2020ledgerdb} integrates blockchain with database systems to create a tamper-proof ledger that supports query operations. GlassDB~\cite{yue2022glassdb} introduces a verifiable data structure designed for outsourced databases to enhance query verifiability and integrity. These solutions can be used for querying blockchain data in theory. However, they often struggle to support multiple query types and do not provide the efficient performance required by modern, data-intensive applications.


\subsection{Blockchain Databases}
Many studies have examined verifiable query processing over blockchain databases. vChain~\cite{xu2019vchain} and vChain+~\cite{wang2022vchain+} employ cryptographic set accumulators to enable verifiable boolean range queries. To address the issue of public key management in vChain, vChain+ introduces a sliding window accumulator (SWA) to reduce public key storage overhead. However, their design requires a large amount of storage space for verification objects (VO) and incurs significant computational overhead. Shao et al.~\cite{shao2020authenticated} proposed an authenticated range query scheme using Trusted Execution Environments (TEE). However, due to limited secure memory space, existing TEEs struggle to handle large-scale applications. GEM$^2$-tree~\cite{zhang2021authenticated,GEM^2-Tree} offers a gas-efficient two-level index structure that supports authenticated queries, aiming to lower on-chain storage costs in Ethereum. However, maintaining the tree-based authenticated data structure (ADS) imposes a heavy burden on light nodes during verification procedures. LVQ~\cite{dai2020lvq} targets Bitcoin transaction analysis and uses a Bloom filter-integrated authenticated index~\cite{ScalableBloomFilters2007} to verify transaction membership. Although it provides lightweight verifiability, it does not support time-range queries. These approaches often limit query types due to their specialized index designs and typically consider a single blockchain data source. Moreover, many of these solutions require custom blockchain structures, making them incompatible with existing blockchain networks. The Graph (TG)~\cite{thegraph2022} is a decentralized protocol designed for indexing blockchain data. It employs indexer nodes to aggregate information from multiple blockchains, thereby offering flexible and efficient query services. To ensure the reliability of query results, TG implements a dispute resolution mechanism. However, this mechanism does not guarantee query integrity and may cause significant delays during verification. V\textsuperscript{2}FS~\cite{V2FS2024} proposes a virtual filesystem that facilitates verifiable query processing over multi-chain data by shifting the focus from verifying computation to verifying data, enabling efficient integrity assurance using an off-the-shelf database engine. However, most of these methods require massive modifications to the underlying blockchain codes. This limits the generalizability and versatility of these methods. \Chain differs from these and does not need any change to the blockchain, thus can be integrated with various blockchains and off-chain storage solutions. 

In summary, existing systems cannot support all desired features, including advanced multimodal queries, blockchain compatibility and off-chain compatibility. In contrast, \Chain is the first to cover all of these features simultaneously.