\section{Threats to Validity}

\noindent
\textbf{Internal Validity.}
One potential threat to the internal validity of this study is the experimental setup, including the specific configurations of the blockchain platforms and off-chain storage systems. The performance of MulChain may vary across different network setups and configurations, particularly in terms of query latency and gas consumption. The experiments are based on a specific testbed (Ethereum and FISCO BCOS), and the results may not fully generalize to other blockchain systems or hybrid-storage solutions. Furthermore, the middleware relies on the assumption that the underlying blockchain systems and storage solutions remain stable and function as expected, which may not always hold true in more dynamic or resource-constrained environments.
Another threat is the design of the proposed BHashTree and trie structures. While these indexing structures are presented as being gas-efficient and verifiable, their performance in extremely high-load scenarios, with large-scale multimodal data sets, remains an open question. The dynamic adaptation mechanism for BHashTree, which transitions between B+Tree and hash nodes, assumes predictable query and insertion patterns. If these patterns do not align with the assumed behavior, the performance benefits might be diminished. Additionally, the verifiability and security of the trie structure depend on the correctness of the cryptographic digests and the integrity of the underlying data.

\noindent
\textbf{External Validity.}
The external validity of the study is potentially limited by the focus on specific datasets (Ethereum and Bitcoin) and particular blockchain platforms (Ethereum and FISCO BCOS). While the proposed MulChain middleware is designed to be compatible with a variety of blockchain systems, the study does not explore the full range of blockchain environments or data types that might affect the performance and scalability of the system. For example, the characteristics of the Ethereum and FISCO BCOS networks, such as transaction volume, block size, and consensus mechanisms, may influence the applicability of MulChain to other blockchain networks with different characteristics, such as Hyperledger Fabric or newer blockchain systems.
Moreover, the study is based on the assumption that the integration of off-chain storage (such as IPFS) is suitable for all use cases involving multimodal data. However, the suitability of IPFS for specific types of data or scenarios—such as real-time data retrieval or high-security applications—could be questioned. In real-world decentralized applications (DApps), the interaction between on-chain and off-chain components could be more complex and subject to issues such as data integrity, privacy concerns, or network latency, which might impact the generalizability of the findings.

