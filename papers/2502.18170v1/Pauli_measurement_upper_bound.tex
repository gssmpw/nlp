\section{Upper bound for Pauli measurements}
\label{sec:pauli-upper}
This section starts with an observation about Pauli measurements, which is common knowledge for quantum information experimentalists. Then, we employ this observation to improve previous results about quantum state tomography using Pauli measurements.

\subsection{ An Observation about Pauli Measurements}
When we measure an element of the Pauli group, for instance, $\sigma_X\otimes \sigma_Y$, on a two-qubit state $\rho$, the outcome is a sample from a $4$-dimensional probability distribution, says $(p_{00},p_{01},p_{10},p_{11})$, such that
\begin{align*}
\tr(\rho(\sigma_X\otimes \sigma_Y))=p_{00}-p_{01}-p_{10}+p_{11}.
\end{align*}

One can easily observe that
\begin{align*}
\Tr[\rho(\sigma_X\otimes \sigma_I)]=p_{00}+p_{01}-p_{10}-p_{11},\\
\Tr[\rho(\sigma_I\otimes \sigma_Y)]=p_{00}-p_{01}+p_{10}-p_{11},\\
\Tr[\rho(\sigma_I\otimes \sigma_I)]=p_{00}+p_{01}+p_{10}+p_{11}.
\end{align*}


In other words, measuring $XY$, we obtained a sample of $\sigma_X\sigma_I$, a sample of $\sigma_I\sigma_Y$, and a sample of $\sigma_I\sigma_I$. 

For a general $n$-qubit system, we have the following observation.
\begin{observation}
For any $P=P_1\otimes P_2\otimes\cdots\otimes P_n\in\{\sigma_X,\sigma_Y,\sigma_Z\}^{\otimes N}$, the measurement result of performing measurement $P_i$ on the $i$-th qubit is an $N$-bit string $s$. One can interpret the measurement result of performing $Q_i\in\{\sigma_I,\sigma_X,\sigma_Y,\sigma_Z\}$ on the $i$-th qubit if $Q_i=P_i$ or $Q_i=\sigma_I$. We call those $Q=Q_1\otimes Q_2\otimes\cdots\otimes Q_N$'s correspond to $P$.
\end{observation}



\subsection{Algorithm and error analysis} 
Our measurement scheme is as follows: For any $\eps>0$, fix an integer $m$.
\begin{enumerate}
    \item  For any $P\in\{\sigma_X,\sigma_Y,\sigma_Z\}^{\otimes N}$, one performs $m$ times $P$ on $\rho$, and records the $m$ samples of the $2^N$ dimensional outcome distribution.


According to the key observation, this measurement scheme provides $m\cdot 3^{N-w}$ samples of the expectation $\tr(\rho P)$, say, $\frac{\mu_P}{m\cdot 3^{N-w}}$, for each Pauli operator $P\in \{\sigma_I,\sigma_X,\sigma_Y,\sigma_Z\}^{\otimes N}$ with weight $w$, where $-m\cdot 3^{N-w}\leq\mu_P\leq m\cdot 3^{N-w}$.

\item Output 
\begin{align*}
\sigma=\sum_P \frac{\mu_P}{m\cdot 3^{N-w}\cdot 2^N} P.
\end{align*}
\end{enumerate}


Using this scheme, we obtained $m\cdot 3^N$ independent samples,
\begin{align*}
X_1,X_2,\cdots, X_{m\cdot 3^N}.
\end{align*}
Each  $X_i$ is an $N$-bit string recording outcomes on all qubits (using bit 0 to denote the +1 eigenvalue and bit 1 to denote the -1 eigenvalue of the measured Pauli operator).   
Given that each operator is measured $m$ times, specifically, we assign that $X_1,X_2,\cdots,X_{m}$ correspond to the measurement $\sigma_X^{\otimes N}$, $X_{m+1},X_{m+2}$, and $\cdots,X_{2m}$ corresponds to the measurement $\sigma_X^{\otimes N-1}\otimes \sigma_Y$, \dots, and until $\sigma_Z^{\otimes N}$.

We observe that for any $P$ of weight $w$,
$\mu_P=\sum_{j=0}^{m\cdot 3^{N-w}-1} Z_j$,
where $Z_j$ are independent samples from the distribution $Z$
\begin{align*}
\mathrm{Pr}(Z=1)=\frac{1+\tr (\rho P)}{2}, \;
\mathrm{Pr}(Z=-1)=\frac{1-\tr (\rho P)}{2}.
\end{align*}
We have
\begin{align*}
\expectDistrOf{}{Z}&=\tr (\rho P), \expectDistrOf{}{Z^2}=1, \\
\expectDistrOf{}{\mu_P}&=m\cdot 3^{N-w}\cdot\tr (\rho P), \\
\expectDistrOf{}{\mu_P^2}&=\expectDistrOf{}{\mu_P}^2+\Var[\mu_P] \\
&=\expectDistrOf{}{\mu_P}^2+m\cdot 3^{N-w}\Var[Z] \\
&=m^2\cdot 9^{N-w}\cdot \tr^2 (\rho P)+m\cdot 3^{N-w}(1-\tr^2 (\rho P)).
\end{align*}
% Furthermore, $Z_j$ can be obtained from samples
% \begin{align*}
% X_1,X_2,\cdots, X_{m\cdot 3^N}.
% \end{align*}
% Therefore,
% \begin{align*}
% \sigma=\sum_P \frac{\mu_P}{m\cdot 3^{N-w_P}\cdot 2^N} P.
% \end{align*}
% is defined according to the samples
% \begin{align*}
% X_1,X_2,\cdots, X_{m\cdot 3^N}.
% \end{align*}
Thus, we can verify that
\begin{align*}
\expectDistrOf{}{\sigma}=\rho,
\end{align*}
where the expectation is taken over the probabilistic distribution according to the measurements.

For convenience, we define the function $f:X_1\times X_2\times \cdots\times X_{m\cdot 3^N}\mapsto \mathbb{R}$
\begin{align*}
f(\sigma)=\hsnorm{\rho-\sigma}=\sqrt{{\rm Tr}[(\rho-\sigma)^\dagger (\rho-\sigma)]}.
\end{align*}
Note that we can write the unknown state $\rho$ as
\begin{align*}
\rho=\sum_P \frac{\alpha_P}{2^N} P.
\end{align*}
According to Cauchy-Schwarz and Jensen's inequality, we have
\begin{align*}
&\expectDistrOf{}{f(\sigma)} \leq\sqrt{\expectDistrOf{}{f(\sigma)^2}} = \sqrt{\expectDistrOf{}{\tr\rho^2-2\tr \rho\sigma+\tr\sigma^2}}\\
=& \sqrt{\expectDistrOf{}{\tr\sigma^2-\tr\rho^2}}
= \sqrt{\frac{1}{2^N}\sum_P \expectDistrOf{}{\frac{\mu_P^2}{m^2\cdot 9^{N-w_P}}-\alpha_P^2}} \\
=& \sqrt{\frac{1}{2^N}\sum_P \left(\frac{m^2\cdot 9^{N-w_P}\cdot \alpha_P^2+m\cdot 3^{N-w_P}(1-\alpha_P^2)}{m^2\cdot 9^{N-w_P}}-\alpha_P^2\right)} \\
=&\sqrt{\frac{1}{m\cdot 2^N}\cdot{\sum_P \frac{1-\alpha_P^2}{3^{N-w_P}}}} < \sqrt{\frac{1}{m\cdot 2^N}\cdot{\sum_P \frac{1}{3^{N-w_P}}}}\\
=& \sqrt{\frac{1}{m\cdot 2^N}\cdot{\sum_{w_P=0}^N \frac{1}{3^{N-w_P}}{{N}\choose{w_P}}3^{w_P}}} = \sqrt{\frac{1}{m\cdot 6^N}\cdot (1+9)^N} \\
=& \sqrt{\frac{5^{\nqubits}}{m\cdot 3^N}}.
\end{align*}

For any sample $X_i$ corresponding to $P\in\{\sigma_X,\sigma_Y,\sigma_Z\}^{\otimes N}$, if only $X_i$ is changed, $\mu_Q$ would be changed only for those $Q\in\{\sigma_I,\sigma_X,\sigma_Y,\sigma_Z\}^{\otimes N}$ where
$Q$ is obtained by replacing some $\{\sigma_X,\sigma_Y,\sigma_Z\}$'s of $P$ by $\sigma_I$.
Moreover, the resultant value of $\mu_Q$ would change by two at most.
According to the triangle inequality, $f$ would change at most 
\begin{align*}
\hsnorm{\sum_Q  \frac{\Delta\mu_Q}{m\cdot 3^{N-w_Q}\cdot 2^N} Q}&=\sqrt{\sum_Q \frac{\Delta\mu_Q^2}{m^2\cdot 9^{N-w_Q}\cdot 2^N}}\\
&\leq \sqrt{\sum_Q \frac{2^2}{m^2\cdot 9^{N-w_Q}\cdot 2^N}} \\
&= \sqrt{\sum_{w_Q=0}^N \frac{2^2}{m^2\cdot 9^{N-w_Q}\cdot 2^N}{{N}\choose{w_Q}}} = \frac{2\cdot \sqrt{5}^N}{m\cdot 3^N},
\end{align*}
where $Q$ ranges over all Paulis which correspond to $P$'s, and $\Delta\mu_Q$ denotes the difference of $\mu_Q$ when $X_i$ is changed.

We use McDiarmid's inequality to bound the probability of success.
\begin{lemma}\label{mc}
Consider independent random variables ${\displaystyle X_{1},X_{2},\dots X_{n}}$ on probability space $ {\displaystyle (\Omega ,{\mathcal {F}},{\text{P}})}$ where ${\displaystyle X_{i}\in {\mathcal {X}}_{i}}$ for all ${\displaystyle i}$ and a mapping ${\displaystyle f:{\mathcal {X}}_{1}\times {\mathcal {X}}_{2}\times \cdots \times {\mathcal {X}}_{n}\rightarrow \mathbb {R} }$. Assume there exist constant $ {\displaystyle c_{1},c_{2},\dots ,c_{n}} $ such that for all $ {\displaystyle i}$,
%\begin{widetext} 
\begin{align}{\displaystyle {\underset {x_{1},\cdots ,x_{i-1},x_{i},x_{i}',x_{i+1},\cdots ,x_{n}}{\sup }}|f(x_{1},\dots ,x_{i-1},x_{i},x_{i+1},\cdots ,x_{n})-f(x_{1},\dots ,x_{i-1},x_{i}',x_{i+1},\cdots ,x_{n})|\leq c_{i}.} 
\end{align}
%\end{widetext}
In other words, changing the value of the ${\displaystyle i}$-th coordinate ${\displaystyle x_{i}}$ changes the value of ${\displaystyle f}$ by at most ${\displaystyle c_{i}}$. Then, for any ${\displaystyle \epsilon >0}$,
%\begin{widetext} 
\begin{align} 
{\displaystyle {\mathrm{Pr}}(f(X_{1},X_{2},\cdots ,X_{n})-\expectDistrOf{}{f(X_{1},X_{2},\cdots ,X_{n})}\geq \epsilon )\leq \exp \left(-{\frac {2\epsilon ^{2}}{\sum _{i=1}^{n}c_{i}^{2}}}\right)} .
\end{align}
%\end{widetext}
\end{lemma}
We only consider $\delta<1/3$, then $\log(1/\delta)>1$.
For any $\eps'>0$, by choosing $m=(3+2\sqrt{2})\cdot\frac{5^N\log\frac{1}{\delta}}{3^N\cdot \eps'^2}$, we have $\expectDistrOf{}{f(\sigma)}< (\sqrt{2}-1){\eps'}$.
Therefore,
\begin{align*}
\mathrm{Pr}(f(\sigma) > \eps') &< \mathrm{Pr}(f(\sigma)-\expectDistrOf{}{f(\sigma)}>(2-\sqrt{2}){\eps'}) \\
&<\exp(-\frac{(12-8\sqrt{2})\cdot \eps'^2}{4\cdot \frac{5^N}{m^2\cdot 9^N}\cdot m\cdot 3^N}) \\
=& \exp(-\frac{m\cdot(3-2\sqrt{2})\cdot 3^N\cdot\eps'^2}{5^N}) <\delta, 
\end{align*}
where the inequality is by \cref{mc}.

For a general quantum state and $\eps>0$, we let $\eps'=\frac{\eps}{\sqrt{2^N}}$, and know that
$||\rho-\sigma||_1>\eps$ implies $\hsnorm{\rho-\sigma}>\eps'$. Therefore,
\begin{align*}
\mathrm{Pr}(||\rho-\sigma||_1>\eps) &\leq \mathrm{Pr}(\hsnorm{\rho-\sigma}>\eps')=\mathrm{Pr}(f>\eps').
\end{align*}

The total number of used copies is 
\begin{align*}
\ns = m\cdot 3^N=(3+2\sqrt{2})\cdot\frac{10^N\log\frac{1}{\delta}}{\eps^2}.
\end{align*}


