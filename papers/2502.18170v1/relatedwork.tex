\section{Related works}
\paragraph{Quantum state tomography} We make additional remarks about previously mentioned works and discuss other works in this regime.

Many works study the tomography of low-rank states. 
For $\rho$ with rank $\rk$, it is shown that $\ns = \tildeTheta{\frac{\dims \rk}{\eps^2}}$ is necessary and sufficient~\cite{HaahHJWY17,ODonnellW16} with entangled measurement. 
For single-copy measurements, the sample complexity is $\ns=\bigTheta{\frac{\dims \rk^2}{\eps^2}}$ for non-adaptive measurements, but whether it is tight for adaptive ones is unknown. 

For Pauli measurements, \cite{guctua2020fast} showed that $\ns=\bigO{\frac{\nqubits \cdot 3^\nqubits\cdot \rk^2}{\eps^2}}$ is sufficient. 
Random Pauli measurements offer distinct advantages \cite{Elben_2022} and have been effectively applied in quantum process tomography for shallow quantum circuits \cite{yu2023learningmarginalssuffices,Huang_2024}. 

Some work~\cite{compressed,Flammia_2012} considers Pauli observables, a special class of 2-outcome Pauli measurements. The sample complexity for rank-$\rk$ state tomography is ${\tilde{{\Theta}}}(\frac{\dims^2 r^2}{\eps^2})$ using non-adaptive measurements~\cite{Flammia_2012}. \cite{lowe2022lower} showed that the lower bound also holds for adaptively chosen constant-outcome measurements\footnote{More precisely when adaptively chosen from a finite set of measurements with size at most $\exp(O(\dims))$.}. \cite{cai2016optimal} derived near-optimal error rates for Hilbert-Schmidt and operator-norm distance. However, they require that the state is sparse in the expectation values of Pauli observables, nor did they prove a lower bound for the trace distance. 


Recently, \cite{Chen0L24memory} studied the case when one can perform entangled measurement over $t>1$ copies at a time, which interpolates between single-copy and fully entangled measurements. Apart from trace distance, other metrics were considered, such as fidelity~\cite{HaahHJWY17,chen2023does, Yuen_2023}, quantum relative entropy~\cite{flamian2023tomography}, and Bures $\chi^2$-divergence~\cite{flamian2023tomography}. Extending our work to low-rank states and other error metrics is an interesting future direction.
 



\paragraph{Other quantum state inference problems} Quantum state testing/certification \cite{ODonnellW15,BadescuO019} is a closely related problem, where the goal is to test whether an unknown state $\rho$ is equal or far from a target state $\sigma$. The problem has also been considered under entangled~\cite{ODonnellW15,BadescuO019}, single-copy measurements~\cite{BubeckC020,Chen0HL22,liu2024role}, and Pauli measurements~\cite{Yu2023almost}. A concurrent work~\cite{liu2024restricted} considers single-copy testing under measurement restrictions, but their technique only applies to non-adaptive measurements.

In practice, we are often interested in partial information about the state rather than a full-state description. \cite{Cotler_2020, Garc_a_P_rez_2020, evans2019scalable} studied \textit{quantum overlapping tomography}, where the goal is to output the classical description of all $k$-qubit reduced
density matrices of an $n$-qubit system. The algorithms are based on Pauli measurements, demonstrating their wide applicability. 
Shadow tomography~\cite{Aaronson20, huang2020predicting, ChenCH021,chen2024pauli} aims to learn the expectation value of a finite set of observables. In particular, \cite{chen2024pauli} studied Pauli shadow tomography with various measurement constraints such as measurements with finite quantum memory and Clifford measurements. It would be an interesting future work to establish a formal connection between their framework and our method.


\paragraph{Distributed distribution estimation}

Quantum tomography can be thought of the quantum analogue of the classical problem of distribution estimation. Given samples from an unknown distribution $p$, the goal is to output an estimate $\hat p$ such that $d(p,\hat p)<\varepsilon$ for some distance $d$. The problem has a long history and has been studied under several different settings. 

The problem of single copy tomography is in spirit similar to the problems of distributed estimation of distributions under information constraints. 
In it, i.i.d. samples $X_1, \ldots, X_n$ from the unknown $p$ are distributed across $n$ users, who are constrained in how much information about their sample they can send. 
Well-studied information constraints include communication constraints (where each user has to compress their sample using at most $b$ bits), or privacy constraints (where each user has to add noise to their sample to preserve privacy). Several problems in distributed estimation of distribution for both discrete, continuous as well as high dimensional distributions have been studied in the past several years~\cite{duchi2013local, barnes2019lower, AcharyaCT19, acharya2020distributed}.