\section{Lower bound for tomography with Pauli measurements}
The key to proving a tight lower bound for Pauli measurements is to design a measurement-dependent hard instance. Recall that any quantum state $\rho$ is a linear combination of Pauli observables,
\[
\rho=\frac{\eye_\dims}{\dims}+\sum_{P\in\pauliObsSet}\frac{\Tr[\rho P]}{\dims}P.
\]

Further recall the observation \cref{sec:techniques} that Pauli measurements \eqref{equ:pauli-measurement} are better at learning information about directions 
$P\in \pauliObsSet$ with a small weight and less powerful $P$ with a larger weight. 
% Consider $P=\pauliX^{\otimes \nqubits}$ and its corresponding Pauli basis measurement $\POVM_P$. The measurement simultaneously provides information about $\nqubits$ weight-1 Pauli observables $Q=\sigma_1\otimes\cdots\otimes \sigma_\nqubits$ such that $\sigma_i=\pauliX$ for some $i$ and $\sigma_j=\eye_2$ everywhere else. However, it only provides information about 1 Pauli observable with weight $N$, which is exactly $\sigma_X^{\otimes\nqubits}$.
As such, we set the basis $V_1, \ldots, V_{\dims^2-1}$ in the lower bound construction (\cref{def:perturbation}) to be the (normalized) Pauli observables, sorted in increasing order of their weights, $w(V_1)\le w(V_2)\le \ldots \le w(V_{\dims^2-1})$. Applying \eqref{equ:avg-MI-partial} in \cref{thm:avg-MI-upper} and \cref{lem:avg-MI-lower},
\begin{equation}
    \frac{1}{100}\le \frac{1}{\ell}\sum_{i=1}^{\ell}\mutualinfo{\ptb_i}{\out^\ns}\le \frac{8 \ns c^2 \eps^2}{\ell^2} \sup_{\POVM\in\povmset}\sum_{i=1}^{\ell} \vadj{V_i}\Choi_{\POVM} \vvec{V_i} +16\exp\{-\alpha\dims\}\ns c^2\eps^2.
    \label{equ:pauli-lower-inequ}
\end{equation}


We need to choose an appropriate $\ell$ and to upper bound the average mutual information. We propose to select all Pauli observables with weight at least $N-w$. Then,
\begin{equation}
    \ell = g(w)\eqdef \sum_{m=0}^w{\nqubits \choose N-m}3^{N-m}.
    \label{equ:pauli-weight-number}
\end{equation}
This is because for Pauli observables with weight $N-m$, there are $N-m$ positions we can place the Pauli operators, and for each position, there are three choices $\pauliX, \pauliY, \pauliZ$.



According to \cref{prop:perturbation-trace-distance}, we must choose $\ell\ge \dims^2/2$ to ensure that the perturbations are $\eps$ far from $\qmm$ with high probability. In other words, $g(w)/\dims^2\ge 1/2$.
\[
\frac{g(w)}{\dims^2}=\sum_{m=0}^w{\nqubits \choose N-m}\frac{3^{N-m}}{4^\nqubits}=\sum_{m=0}^{w}{\nqubits \choose m}\Paren{\frac34}^{\nqubits-m}\Paren{\frac14}^m =\probaOf{\binomial{\nqubits}{1/4}\le w}.
\]
We have the following fact about the median of binomial distributions,
\begin{fact}[\cite{kaas1980mean}]
    The median of a binomial distribution $\binomial{N}{p}$ must lie in $[\lfloor Np\rfloor, \lceil Np\rceil]$.
\end{fact}
Thus, choosing $w=\lceil N/4\rceil$ guarantees that $g(w)/\dims^2\ge 1/2$. 

Next, we compute the inner product $\vadj{V_i} \Choi_{\POVM} \vvec{V_i}$. We first need to analyze the measurement information channel of Pauli measurements.

\begin{lemma}
\label{lem:pauli-mic-eigen}
    For $P=\sigma_1\otimes\cdots\otimes \sigma_{\nqubits}\in \Sigma^{\otimes \nqubits}$, let $\Luders_P$ be the measurement information channel of the Pauli measurement $\POVM_P$. Then for all Pauli observable $Q=\sigma_1'\otimes\cdots\otimes \sigma_{\nqubits}'\in(\Sigma\cup \eye_2)^{\otimes \nqubits}$, $Q$ is an eigenvector of $\Luders_P$ and
    \[
\Luders_P(Q)=Q\indic{\forall j\in[\nqubits], \sigma_j'\in\{\sigma_j,\eye_2\}}.
    \]
    In other words, the eigenvalue of ${Q}$ is 1 when the non-identity components of $Q$ match $P$, and 0 otherwise. 
\end{lemma}
\begin{proof}
 Let $\Luders_P$ be the measurement information channel of a Pauli measurement $\POVM_{P}$. From \cref{def:mic} and \cref{equ:pauli-measurement},
\begin{align*}
    \Luders_P(\cdot)&=\sum_{x\in\{-1,1\}^{\nqubits}}\frac{M_x^{P}}{\Tr[M_x^P]}\Tr[(\cdot)M_x^P]\\
    &=\sum_{x\in\{-1,1\}^{\nqubits}}M_x^{P}\Tr[(\cdot)M_x^P].
\end{align*}
The second step is because Pauli measurement is a basis measurement. Thus each $M_x^P=\qproj{u_x^P}$ where $\{\qbit{u_x^P}\}_{x\in\{-1,1\}^{\nqubits}}$ is an orthonormal basis, and $\Tr[M_x^P]=1$.

Let $Q=\sigma_1'\otimes\cdots\otimes\sigma_{\nqubits}'$. We want to argue that $Q$ is an eigenvector of $\Luders_P$.  
\begin{align*}
\Tr[M_x^PQ]&=\Tr\left[\bigotimes_{j=1}^{\nqubits}\frac{\eye_2+x_j\sigma_j}{2}\bigotimes_{j=1}^{\nqubits}\sigma_j'\right]\\
    &=\Tr\left[\bigotimes_{j=1}^{\nqubits}\frac{\sigma_j'+x_j\sigma_j\sigma_j'}{2}\right]\\
    &=\prod_{j=1}^{\nqubits}\frac{\Tr[\sigma_j']+x_j\Tr[\sigma_j\sigma_j']}{2}\\
    &=\prod_{j=1}^{N}(\indic{\sigma_j'=\eye_2}+x_j\indic{\sigma_{j}'=\sigma_j})
\end{align*}
The final step is due to \eqref{equ:pauli-property}.
If for some $j\in[N]$, $\sigma_j'\ne \eye_2$ and $\sigma_j'\ne\sigma_j$, then
\[
\Tr[M_x^PQ]=0\implies\Luders_P(Q)=0.
\]
In this case $Q$ is an eigenvector of $\Luders_P$ with eigenvalue of 0. If otherwise,
\begin{align*}
    \Luders_P(Q)&=\sum_{x\in\{-1,1\}^{\nqubits}}M_x^P\prod_{j=1}^{N}(\indic{\sigma_j'=\eye_2}+x_j\indic{\sigma_{j}'=\sigma_j})\\
    &=\sum_{x\in\{-1, 1\}^{\nqubits}}\bigotimes_{j=1}^{\nqubits}\frac{\eye_2+x_j\sigma_j}{2}(\indic{\sigma_j'=\eye_2}+x_j\indic{\sigma_{j}'=\sigma_j})\\
    &=\bigotimes_{j=1}^{\nqubits}\sum_{x_j\in\{-1,1\}}\frac{\eye_2+x_j\sigma_j}{2}(\indic{\sigma_j'=\eye_2}+x_j\indic{\sigma_{j}'=\sigma_j})\\
    &=\bigotimes_{j=1}^{\nqubits}\Paren{\indic{\sigma_j'=\eye_2}\sum_{x_j\in\{-1,1\}}\frac{\eye_2+x_j\sigma_j}{2}+\indic{\sigma_j'=\sigma_j}\sum_{x_j\in\{-1,1\}}\frac{x_j\eye_2+\sigma_j}{2}} \\
    &=\bigotimes_{j=1}^{\nqubits}(\eye_2\indic{\sigma_j'=\eye_2}+\sigma_j\indic{\sigma_{j}'=\sigma_j})\\
    &=\bigotimes_{j=1}^{\nqubits}\sigma_j'=Q.\qedhere
\end{align*}
\end{proof}

Let $P,Q$ be defined in \cref{lem:pauli-mic-eigen} and $\Choi_P$ be the matrix form of $\Luders_P$ given a Pauli measurement $\POVM_P$. An immediate corollary is that
\begin{equation}
    \frac{1}{\dims}\vadj{Q} {\Choi}_P \vvec{Q}=\indic{\forall j\in[\nqubits], \sigma_j'\in\{\sigma_j,\eye_2\}}.
    \label{equ:pauli-mic-sum}
\end{equation}

When $V_1, \ldots, V_{\dims^2-1}$ are the normalized Pauli observables sorted in increasing order of their weights, setting $\ell=g(w)$ (the number of Pauli observables with weight at least $N-w$), we have
\begin{align*}
    \sum_{i=1}^{\ell}\vadj{V_i} {\Choi}_P \vvec{V_i}=\sum_{m=0}^{w}{\nqubits\choose m}.
\end{align*}
This is because $\vadj{V_i} {\Choi}_P \vvec{V_i}=1$ 
only when $V_i$ has non-identity components that match the ones in $P$ and 0 otherwise. 
There are only ${\nqubits\choose N-m}={\nqubits\choose m}$ of them among all $V_i$'s with weight $\nqubits-m$.

The following result gives an upper bound on the sum of binomial coefficients,
\begin{lemma}[{\cite[Lemma 16.19]{downey2012parameterized}}]
\label{lem:sum-binomial-coef}
    Let $n\ge 1$ and $0\le q\le 1/2$, then
    \[
    \sum_{i=0}^{\lfloor nq\rfloor}{n\choose i}\le 2^{n h(q)},
    \]
    where $h(q)=-q\log q -(1-q)\log(1-q)$ is the binary entropy function.
\end{lemma}

Combining with \eqref{equ:pauli-lower-inequ}\eqref{equ:pauli-weight-number}\eqref{equ:pauli-mic-sum}, setting $w=\lceil N/4\rceil$,
\begin{align*}
    \frac{1}{100}&\le \frac{8 \ns c^2 \eps^2}{\ell^2} \sup_{P\in\Sigma^{\otimes\nqubits}}\sum_{i=1}^{\ell}\vadj{V_i} {\Choi}_P \vvec{V_i} +16\exp\{-\alpha\dims\}\ns c^2\eps^2\\
    &=8nc^2\eps^2\Paren{\frac{\sum_{m=0}^{w}{\nqubits\choose m}}{g(w)^2}+2\exp\{-\alpha\dims\}}\\
    &\le 8nc^2\eps^2\Paren{\frac{2\cdot 2^{\nqubits h(1/4)}}{\dims^4/4}+2\exp\{-\alpha\dims\}}\\
    &\le 16\ns \cd^2\eps^2\Paren{4\cdot 2^{(h(1/4)-4)\nqubits}+\exp\{-\alpha2^{\nqubits}\}}.
\end{align*}
When $N\ge 10$, the second term is negligible. Rearranging the terms, we must have
\[
\ns = \bigOmega{\frac{2^{(4-h(1/4))\nqubits}}{\eps^2}}.
\]
Finally, noting that $2^{4-h(1/4)}\ge 9.118$ completes the proof.
