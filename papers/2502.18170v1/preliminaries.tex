\section{Preliminaries}
\subsection{Quantum state and POVM}
We use the Dirac notation $\qbit{\psi}$ to denote a vector in $\C^{\dims}$. $\qadjoint{\psi}\eqdef(\qbit{\psi})^\dagger$ is a row vector. $\qdotprod{\psi}{\phi}$ is the Hibert-Schmidt inner product of $\qbit{\psi}$ and $\qbit{\phi}$. We denote the set of all $\dims\times\dims$ Hermitian matrices by $\Herm{\dims}$. A $\dims$-dimensional quantum system is described by a positive-semidefinite Hermitian matrix $\rho\in\Herm{\dims}$ with $\Tr[\rho]=1$. We assume $\dims=2^{\nqubits}$ where $\nqubits$ is the number of qubits in the system.

Measurements are formulated as \emph{positive operator-valued measure} (POVM). Let $\outset$ be an outcome set. Then a POVM $\POVM=\{M_x\}_{x\in \outset}$, where $M_x$ is p.s.d. and $\sum_{x\in \outset}M_x=\eye_\dims$. Let $X$ be the outcome of measuring $\rho$ with $\POVM$, then the probability observing $x\in\outset$ is given by the \emph{Born's rule},
\[
\probaOf{X=x}=\Tr[\rho M_x].
\]
\subsection{Hibert space over linear operators}

\paragraph{Hilbert space over complex matrices}
The space of complex matrices $\C^{\dims\times\dims}$ is a Hilbert space with inner product $\hdotprod{A}{B}\eqdef\Tr[A^\dagger B], A, B\in \C^{\dims\times \dims}$. For Hermitian matrices $A,B$, $\hdotprod{A}{B}=\hdotprod{B}{A}\in \R$. Thus the subspace of Hermitian matrices $\Herm{\dims}$ is a \textit{real} Hilbert space (i.e. the associated field is $\R$) with the same matrix inner product. 

Vectorization defines a homomorphism between $\C^{\dims\times\dims}$ and $\C^{\dims^2}$. 
On the canonical basis $\{\qbit{j}\}_{j=0}^{\dims-1}$, $\VecOp(\qoutprod{i}{j})\eqdef \qbit{j}\otimes \qbit{i}$. For convenience we denote $\vvec{A}\eqdef\VecOp(A)$. It is straightforward that $\hdotprod{A}{B}=\vvdotprod{A}{B}$. 

\paragraph{(Linear) superoperators} Let $\mathcal{N}:\C^{\dims\times \dims}\mapsto \C^{\dims\times \dims}$ be a linear operator over $\C^{\dims\times \dims}$, which we refer to as superoperators\footnote{This is to distinguish from a matrix $A\in\C^{\dims\times \dims}$, which can be viewed as an operator over $\C^\dims$. Indeed an operator over $\C^{\dims\times \dims}$ need not be linear, but we only deal with linear ones in this work, so we drop ``linear'' for brevity.}. Every superoperator $\mathcal{N}$ has a matrix representation $\Choi(\mathcal{N})\in \C^{\dims^2\times\dims^2}$ that satisfies $\vvec{\mathcal{N}(X)}=\Choi(\mathcal{N})\vvec{X}$ for all matrices $X\in\C^{\dims\times\dims}$. It can be verified that for the measurement information channel $\Luders_{\POVM}$ in~\cref{def:mic}, $\Choi_{\POVM}\vvec{A}=\vvec{\Luders_{\POVM}(A)}$.


\paragraph{Schatten norms} Let $\Lambda=(\lambda_1, \ldots, \lambda_\dims)\ge 0$ be the \emph{singular values} of a linear operator $A$, which can be a matrix or a superoperator. {For Hermitian matrices, the singular values are the absolute values of the eigenvalues.} Then for $p\ge 1$, the \emph{Schatten $p$-norm} is defined as 
$
\|A\|_{S_p}\eqdef \|\Lambda\|_p
$. The Schatten norms of a superoperator $\mathcal{N}$ and its matrix representation $\Choi(\mathcal{N})$ match exactly, $\|\mathcal{N}\|_{S_p}=\|\Choi(\mathcal{N})\|_{S_p}$. Some important examples are trace norm $\tracenorm{A}\eqdef\|A\|_{S_1}$, Hilbert-Schmidt norm $\hsnorm{A}\eqdef\|A\|_{S_2}=\sqrt{\hdotprod{A}{A}}$, and operator norm $\opnorm{A}\eqdef\|A\|_{S_\infty}=\max_{i=1}^\dims\lambda_i$. 



\subsection{Problem setup}
There are $\ns$ copies of $\rho$ and a random seed $R$. We can apply adaptive measurements $\POVM^\ns = (\POVM_1, \ldots, \POVM_\ns)$ to each copy, where $\POVM_i=\{M_x^i\}_{x\in\cX}$. Let $\out_0=R$, and for $i\ge 1$ let $\out_i$ be the outcome of measuring the $i$th copy with $\POVM_i$. Define $\out^t=(\out_0,\out_1, \ldots, \out_t)$. Then we can write $\POVM_i=\POVM_i(\out^{i-1})$.

\paragraph{Tomography} The goal is to design a measurement scheme $\POVM^\ns$ and an estimator $\qest:\outset^\ns\mapsto \Herm{\dims}$ such that
\[
\inf_{\rho}\probaOf{\tracenorm{\qest(\out^\ns)-\rho}\le \eps}\ge0.99.
\]

Given measurements $\POVM^\ns$, when the state is $\rho$, the distribution of $\out_i,i\ge 1$ is determined by Born's rule,
\begin{equation}
    \p_\rho^{\out_i|\out^{i-1}}(x)=\Tr[M_x^i\rho],\label{equ:distr-i-cond}
\end{equation}
which depends on all previous outcomes and the random seed $R$. For $1\le t\le \ns$, we further define $\p_\rho^{\out^t}$ as the distribution of $\out^t$ when the state is $\rho$.


In practice, we may have restrictions on the types of measurements that can be applied. We use $\povmset$ to denote the set of allowable measurements for each copy. Define the following quantities which are related to the norms of the measurement information channel in this family of measurements,
\begin{align}
\norm{\povmset}=\sup_{\POVM\in\povmset}\norm{\Luders_{\POVM}},
\label{equ:max-povm-norm}
\end{align}
where $\norm{\cdot}$ can be any norms for linear operators, including $\tracenorm{\cdot}, \hsnorm{\cdot}$, and $\opnorm{\cdot}$.

\subsection{Pauli measurements}
For a single-qubit system, the Pauli operators $\Sigma=\{\pauliX, \pauliY, \pauliZ\}$ are defined in \cref{equ:pauli-ops} An important property is that for $P,Q\in \Sigma\cup \{\eye_2\}$,
\begin{equation}
P^2=\eye_2,\;\Tr[PQ]=2\indic{P=Q}.    
\label{equ:pauli-property}
\end{equation}

 Let $\qbit{0}$ and $\qbit{1}$ be the computation basis for a single-qubit system. Define the following states
\[
\qbit{+}\eqdef \frac{1}{\sqrt{2}}(\qbit{0}+\qbit{1}),\; \qbit{-}\eqdef \frac{1}{\sqrt{2}}(\qbit{0}-\qbit{1}),
\;
\qbit{+\img}\eqdef \frac{1}{\sqrt{2}}(\qbit{0}+\img\qbit{1}),\; \qbit{-\img}\eqdef \frac{1}{\sqrt{2}}(\qbit{0}-\img\qbit{1}).
\]
Note that both $\{\qbit{+},\qbit{-}\}$ and $\{\qbit{+\img},\qbit{-\img}\}$ are orthonormal bases for $\C^2$. Furthermore,

\[
\sigma_X\qbit{+}=\qbit{+}, \quad\sigma_X\qbit{-}=-\qbit{-},
\]
\[
\sigma_Y\qbit{+\img}=-\qbit{+\img}, \quad\sigma_X\qbit{-\img}=-\qbit{-\img},
\]
\[
\sigma_Z\qbit{0}=\qbit{0}, \quad\sigma_Z\qbit{1}=-\qbit{1},
\]
Thus Pauli operators have eigenvalues of either $1$ or $-1$. We refer to $\{\qbit{+},\qbit{-}\}$ as the $X$ basis, $\{\qbit{+\img},\qbit{-\img}\}$ as the $Y$ basis, and $\{\qbit{0},\qbit{1}\}$ as the $Z$ basis because they are the eigenvectors of the respective Pauli operators.

\paragraph{Pauli (basis) measurement} For an $\nqubits$-qubit system, we can independently measure in the $X$, $Y$, or $Z$ basis for each qubit. This results in a basis measurement with $2^\nqubits$ outcomes, which we denote by $\{-1, 1\}^{\nqubits}$. We call this a \emph{Pauli basis measurement}, or Pauli measurement in short. Formally, given $P=\sigma_1\otimes\cdots\otimes\sigma_{\nqubits}$ where $\sigma_i\in \Sigma=\{\pauliX, \pauliY, \pauliZ\}$, the Pauli measurement is defined as
\begin{equation}
    \POVM_{P}=\{M_x^P\}_{x\in \{-1, 1\}^{\nqubits}},\; M_x^\sigma\eqdef \bigotimes_{j=1}^{\nqubits}\frac{\eye_2+x_j\sigma_j}{2},x=(x_1, \ldots, x_{\nqubits})\in\{-1, 1\}^{\nqubits}.
    \label{equ:pauli-measurement}
\end{equation}

\paragraph{Pauli observable} A weaker type of Pauli measurement is defined by the Pauli observables.
\begin{definition}
    A \emph{Pauli observable} $P\in \C^{\dims\times\dims}$ can be written as
\[
P=\sigma_1\otimes\cdots\otimes \sigma_{\nqubits}, \sigma_j\in\Sigma\cup\{\eye_2\}, P\ne \eye_\dims.
\]
The \emph{weight} of $P$ is the number of $\sigma_j$'s in $P$ such that $\sigma_j\ne \eye_2$, denoted as $w(P)$. We sometimes omit $P$ when the Pauli observable is clear from context.
\end{definition}

The set of Pauli observables $\pauliObsSet\eqdef (\Sigma\cup \eye_2)^{\otimes\nqubits}\setminus\{\eye_\dims\}$ consists of $4^{\nqubits}-1=\dims^2-1$ matrices. Each $P\in\pauliObsSet$ defines a 2-outcome POVM,
\[
\ObsPOVM_P=\{M_{-1}^P, M_{1}^P\},\; M_{-1}^P=\frac{\eye_\dims - P}{2}, M_1^P=\frac{\eye_\dims+P}{2}.
\]

We have the standard fact about Pauli observables,
\begin{fact}
\label{fact:pauli}
    Let $P, Q\in \mathcal{P}$ be two Pauli observables. Then, 
\[
P^2=\eye_\dims, \;\Tr[P]=0,\; \Tr[PQ]=\hdotprod{P}{Q}=\dims\indic{P=Q}.
\]
\end{fact}
Therefore, the set $\pauliObsSet\cup \{\eye_\dims\}$ forms an orthogonal basis for $\Herm{\dims}$. We can represent any state $\rho$ as,
\begin{equation}
    \rho = \frac{\eye_\dims}{\dims}+\sum_{P\in \pauliObsSet}\frac{\Tr[\rho P]P}{\dims}.
    \label{equ:state-pauli}
\end{equation}
\subsection{Probability distances}
Let $\p$ and $\q$ be distributions over a finite domain $\mathcal{X}$. The \emph{total variation distance} is defined as 
\[
\totalvardist{\p}{\q}\eqdef\sup_{S\subseteq\mathcal{X}}(\p(S)-\q(S))=\frac{1}{2}\sum_{x\in\mathcal{X}}|\p(x)-\q(x)|.
\]
The KL-divergence  is
\[
\kldiv{\p}{\q}\eqdef\sum_{x\in\mathcal{X}}\p(x)\log\frac{\p(x)}{\q(x)}.
\]
The symmetric KL-divergence is 
$\kldivsym{\p}{\q}=\frac{1}{2}(\kldiv{\p}{\q}+\kldiv{\q}{\p})$. The chi-square divergence
\[
\chisquare{\p}{\q}\eqdef \sum_{x\in\mathcal{X}}\frac{(\p(x)-\q(x))^2}{\q(x)}.
\]
By Pinsker's inequality and concavity of logarithm,
\[
2\totalvardist{\p}{\q}^2\le \kldiv{\p}{\q}\le \chisquare{\p}{\q}.
\]
We define $\ell_p$ distance as $
\norm{\p-\q}_p\eqdef\Paren{\sum_{x\in\mathcal{X}}{|\p(x)-\q(x)|^p}}^{1/p}.
$




