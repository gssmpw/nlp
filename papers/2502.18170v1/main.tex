\documentclass{article}
\def\anonymous{0}

\usepackage{graphicx} % Required for inserting images
\def\withcolors{1}
\def\withnotes{1}
\usepackage{ccanonne}
\usepackage{multicol}
\usepackage{multirow}
\usepackage{subcaption}
\usepackage{bbm}
\usepackage{braket}
\usepackage{algorithm}
\usepackage{algpseudocode}
\newcommand{\tr}{\operatorname{Tr}}

\newcommand{\requiresproof}{{\color{red}Requires proof}}
\DeclareMathOperator*{\argmax}{arg\,max}
\DeclareMathOperator*{\argmin}{arg\,min}


\newtheorem{fact}[theorem]{Fact}
\allowdisplaybreaks
\newcolumntype{C}{>{\centering\arraybackslash}p{0.27\textwidth}}
%
% Frequently used symbols
\newcommand{\perturb}{\gamma}
\newcommand{\dims}{d}
\newcommand{\zdims}{k}
\newcommand{\nsamps}{m}
\newcommand{\nb}{t}
\newcommand{\nspu}{m}
\newcommand{\nin}{r}
\newcommand{\nms}{{N_S}}
\newcommand{\nbb}{{t'}}
\newcommand{\nbbb}{{t''}}
\newcommand{\gbit}{s}
\newcommand{\unif}{{\mathbf{u}}}
\newcommand{\ngr}{{n_0}}
\newcommand{\DiffS}{{\Delta_S}}
\newcommand{\epsnew}{{\eps_0}}
\newcommand{\alphanew}{{\alpha_0}}
\newcommand{\rank}{{\text{rank}}}

\newcommand{\Proj}{{\Pi}}
\newcommand{\povmset}{{\mathfrak{M}}}
\newcommand{\nqubits}{{N}}
\newcommand{\pauliI}{{\sigma_I}}
\newcommand{\pauliX}{{\sigma_X}}
\newcommand{\pauliY}{{\sigma_Y}}
\newcommand{\pauliZ}{{\sigma_Z}}
\newcommand{\pauliObsSet}{{\mathcal{P}}}
\newcommand{\prPauli}[2]{{\p_{#2}(#1)}}

\newcommand{\opnorm}[1]{{\left\|#1\right\|}_{\text{op}}}
\newcommand{\tracenorm}[1]{{\left\|#1\right\|}_{1}}
\newcommand{\hsnorm}[1]{{\left\|#1\right\|}_{\text{HS}}}
\newcommand{\barDelta}{{\overline{\Delta}}}
\newcommand{\ptb}{{z}}
\newcommand{\ptbDistr}{{\mathcal{D}_{\ell,\cd}}}

\newcommand{\supparen}[1]{^{(#1)}}
\newcommand{\subparen}[1]{_{(#1)}}

% Constants
\newcommand{\cd}{{c}}
\newcommand{\cop}{\kappa}


\newcommand{\isthestate}{\texttt{YES}}
\newcommand{\notthestate}{\texttt{NO}}


\newcommand{\cA}{\mathcal{A}}

% Font
\newcommand{\tst}{t}  % Time: to be used when not for a sum index (e.g., 
%"at time $tst$")
\newcommand{\ts}{r} % Time step: index. To be used  for a sum index (e.g., 
%"$\sum_{\ts=1}^\ts")

\newcommand{\etaa}{\gamma}
%\newcommand{\etab}{\eta}
\newcommand{\signA}{{\bf{}1}^\star}
\newcommand{\signB}{{\bf{}0}^\star}
\newcommand{\setS}{S}
\newcommand{\vecu}{u}
\newcommand{\zero}{\mathbf{0}}

\newcommand{\chd}[1]{\Delta_{#1}^y}
\newcommand{\ratioparam}[1]{\kappa_{\scalebox{0.5}{\ensuremath{#1}}}}%
\newcommand{\indbig}[1]{\one\left\{#1\right\}}
\newcommand{\bfP}{\mathbf{P}}
\newcommand{\bfQ}{\mathbf{Q}}
\newcommand{\hbP}{\hat{\bfP}}
\newcommand{\trans}{\mathcal{T}}
\newcommand{\odiag}{E}
\newcommand{\x}{\mathbf{x}}
\newcommand{\out}{{x}}


\newcommand{\risk}{\mathcal{R}}
\def\multiset#1#2{\ensuremath{\left(\kern-.3em\left(\genfrac{}{}{0pt}{}{#1}{#2}\right)\kern-.3em\right)}}

\newcommand{\hp}{\widehat{\p}}
 
\newcommand{\ham}[2]{\operatorname{d}_{\rm Ham}(#1,#2)}
\newcommand{\variance}[2]{\var_{#1}{\mleft[#2\mright]}}

% quantum
\newcommand{\bm}{{\mathbf{m}}}
\newcommand{\qs}{\rho}
\newcommand{\qmm}{{\rho_{\text{mm}}}}
\newcommand{\qkn}{{\rho_0}}
\newcommand{\blambda}{{\boldsymbol{\lambda}}}
\newcommand{\SW}{\textbf{SW}}
\newcommand{\ngroups}{N}
\newcommand{\pnew}{\mathbf{p}}


\newcommand{\HH}{\mathbb{H}}
\newcommand{\Herm}[1]{{\HH_{#1}}}
\newcommand{\F}{\mathbb{F}}
\newcommand{\Pow}{\mathbb{P}}
\newcommand{\mD}{\mathcal{D}}
\newcommand{\OPT}{\text{OPT}}

\newcommand{\qbit}[1]{|{#1}\rangle}
\newcommand{\qadjoint}[1]{\langle{#1}|}
\newcommand{\qproj}[1]{\qbit{#1}\qadjoint{#1}}
\newcommand{\qoutprod}[2]{\qbit{#1}\qadjoint{#2}}

\newcommand{\qdotprod}[2]{\langle#1|#2\rangle}
\newcommand{\hdotprod}[2]{\left\langle#1,#2\right\rangle}
\newcommand{\matdotprod}[3]{\langle#1|#2|#3\rangle}
\newcommand{\supop}[2]{\mathcal{N}_{{#1}\rightarrow{#2} }}
\newcommand{\bA}{\mathbf{A}}
\newcommand{\bB}{\mathbf{B}}
\newcommand{\bD}{\mathbf{D}}
\newcommand{\bC}{\mathbf{C}}
\newcommand{\bG}{\mathbf{G}}
\newcommand{\bH}{\mathbf{H}}
\newcommand{\bM}{\mathbf{M}}
\newcommand{\bS}{\mathbf{S}}
\newcommand{\bT}{\mathbf{T}}
\newcommand{\bU}{\mathbf{U}}
\newcommand{\bV}{\mathbf{V}}
\newcommand{\bW}{\mathbf{W}}
\newcommand{\bX}{\mathbf{X}}
\newcommand{\bY}{\mathbf{Y}}
\newcommand{\EE}{\mathbb{E}}
\newcommand{\Var}{\text{Var}}
\newcommand{\eye}{\mathbb{I}}
\newcommand{\Real}{\text{Re}}
\newcommand{\Img}{\text{Im}}
\newcommand{\id}{\text{id}}
\newcommand{\img}{\text{i}}

\newcommand{\rk}{{r}}
\newcommand{\VecOp}{\text{vec}}
\newcommand{\vvec}[1]{|#1\rangle\rangle}
\newcommand{\vadj}[1]{\langle\langle#1|}
\newcommand{\vvdotprod}[2]{\left\langle\left\langle#1|#2\right\rangle\right\rangle}

\newcommand{\bv}{\mathbf{v}}
\newcommand{\bx}{\mathbf{x}}
\newcommand{\outset}{{\mathcal{X}}}

\newcommand{\spec}{\text{spec}}
\newcommand{\vsigma}{\vec{\sigma}}
\newcommand{\Luders}{\mathcal{H}}
\newcommand{\avgLuders}{{\overline{\Luders}}}
\newcommand{\Choi}{{\mathcal{C}}}
\newcommand{\avgChoi}{{\overline{\Choi}}}
\newcommand{\hbasis}{{\mathcal{V}}}
\newcommand{\qest}{{\hat{\rho}}}

\DeclareMathOperator{\diam}{diam}
\DeclareMathOperator{\diag}{diag}
\DeclareMathOperator{\iSWAP}{iSWAP}
\DeclareMathOperator{\Span}{span}

% Representation theory
\newcommand{\PiRank}{r}
\newcommand{\Wg}{\text{Wg}}
\newcommand{\U}{\mathbb{U}}
\newcommand{\bi}{\mathbf{i}}
\newcommand{\bj}{\mathbf{j}}
\newcommand{\Sim}{\mathcal{S}}
\newcommand{\Mob}{\text{Mob}}
\newcommand{\Cat}{\text{Cat}}
\newcommand{\Haar}[1]{{\mathcal{U}_{#1}}}
\newcommand{\POVM}{\mathcal{M}}
\newcommand{\permProd}[2]{{\left\langle#1\right\rangle_{#2}}}
\newcommand{\cycle}{\mathcal{C}}
\newcommand{\ObsPOVM}{\mathcal{N}}

\newcommand{\Xset}{\mathcal{I}_X}
\newcommand{\Yset}{\mathcal{I}_Y}
\newcommand{\zest}{\hat{\ptb}}
\title{Pauli measurements are not optimal for single-copy tomography}
\ifnum\anonymous=1
    \author{Anonymous authors}
\else
    \author{
    \begin{tabular}[t]{C@{\extracolsep{6.5em}} C}
   Jayadev Acharya &Abhilash Dharmavarapu \\
 Cornell University & Cornell University\\ 
\small \texttt{acharya@cornell.edu} &\small \texttt{ad2255@cornell.edu} 
\end{tabular}
\vspace{2ex}\\
\begin{tabular}[t]{C@{\extracolsep{6.5em}} C}
    Yuhan Liu & Nengkun Yu \\
Rice University & Stony Brook University\\ 
\small \texttt{yuhan-liu@rice.edu} &\small \texttt{nengkun.yu@cs.stonybrook.edu} 
\end{tabular}
}
\fi

\begin{document}
\maketitle
\begin{abstract}
Quantum state tomography is a fundamental problem in quantum computing.
Given $n$ copies of an unknown $N$-qubit state $\rho\in\mathbb{C}^{d\times d},d=2^N$, the goal is to learn the state up to an accuracy $\varepsilon$ in trace distance, say with at least constant probability 0.99. We are interested in the copy complexity, the minimum number of copies of $\rho$ needed to fulfill the task.

As current quantum devices are physically limited, Pauli measurements have attracted significant attention due to their ease
of implementation. However, a large gap exists in the 
literature for tomography with Pauli measurements.
The best-known upper bound is $O(\frac{N\cdot 12^N}{\varepsilon^2})$, 
and no non-trivial lower bound is known besides the general single-copy lower bound of
$\Omega(\frac{8^N}{\varepsilon^2})$, achieved by hard-to-implement structured POVMs such as MUB, SIC-POVM, and uniform POVM.

We have made significant progress on this long-standing problem. We first prove a stronger upper bound of $O(\frac{10^{N}}{\varepsilon^2})$. To complement it, we also obtain a lower bound of $\Omega(\frac{9.118^N}{\varepsilon^2})$, which holds even with adaptivity. To our knowledge, this demonstrates the first known separation between Pauli measurements and structured POVMs. 

The new lower bound is a consequence of a novel framework for adaptive quantum state tomography with measurement constraints. 
The main advantage is that we can use measurement-dependent hard instances to prove tight lower bounds for Pauli measurements, 
while prior lower-bound techniques for tomography only work with measurement-independent constructions. 
Moreover, we connect the copy complexity lower bound of tomography to the eigenvalues of the measurement information channel, which governs the measurement’s capacity to distinguish between states. To demonstrate the generality of the new framework, we obtain tight bounds for adaptive quantum state tomography with $k$-outcome measurements, where we recover existing results and establish new ones.

\end{abstract}
 % More precisely, for quantum tomography,
 %    \begin{itemize}
 %        \item We recover the tight bound of $\Omega(\dims^{3}/\eps^2)$ for general measurements, 
 %        \item For Pauli observables and constant-outcome measurements, we improve the existing lower bound of $\Omega(\dims^4/(\eps^2\log \dims))$ to $\Omega(\dims^4/\eps^2)$, which is now constant-optimal.
 %    \end{itemize}
 %    For quantum state certification,
 %    \begin{itemize}
 %        \item We recover the lower bound of $\Omega(\dims^{3/2}/\eps^2)$ for general measurements. 
 %        \item We prove the first known lower bound for adaptive Pauli observables of $\Omega(\dims^2/\eps^2)$, which is tight up to logarithmic factors. 
 %        \item For measurements with $\ab\ge 4$ outcomes, we prove a lower bound of $\Omega(\dims^2/(\eps^2\sqrt{\min\{\ab, \dims\}}))$, which matches non-adaptive upper bound and thus showing adaptivity does not help.
 %    \end{itemize}

\section{Introduction}
\label{sec:introduction}
The business processes of organizations are experiencing ever-increasing complexity due to the large amount of data, high number of users, and high-tech devices involved \cite{martin2021pmopportunitieschallenges, beerepoot2023biggestbpmproblems}. This complexity may cause business processes to deviate from normal control flow due to unforeseen and disruptive anomalies \cite{adams2023proceddsriftdetection}. These control-flow anomalies manifest as unknown, skipped, and wrongly-ordered activities in the traces of event logs monitored from the execution of business processes \cite{ko2023adsystematicreview}. For the sake of clarity, let us consider an illustrative example of such anomalies. Figure \ref{FP_ANOMALIES} shows a so-called event log footprint, which captures the control flow relations of four activities of a hypothetical event log. In particular, this footprint captures the control-flow relations between activities \texttt{a}, \texttt{b}, \texttt{c} and \texttt{d}. These are the causal ($\rightarrow$) relation, concurrent ($\parallel$) relation, and other ($\#$) relations such as exclusivity or non-local dependency \cite{aalst2022pmhandbook}. In addition, on the right are six traces, of which five exhibit skipped, wrongly-ordered and unknown control-flow anomalies. For example, $\langle$\texttt{a b d}$\rangle$ has a skipped activity, which is \texttt{c}. Because of this skipped activity, the control-flow relation \texttt{b}$\,\#\,$\texttt{d} is violated, since \texttt{d} directly follows \texttt{b} in the anomalous trace.
\begin{figure}[!t]
\centering
\includegraphics[width=0.9\columnwidth]{images/FP_ANOMALIES.png}
\caption{An example event log footprint with six traces, of which five exhibit control-flow anomalies.}
\label{FP_ANOMALIES}
\end{figure}

\subsection{Control-flow anomaly detection}
Control-flow anomaly detection techniques aim to characterize the normal control flow from event logs and verify whether these deviations occur in new event logs \cite{ko2023adsystematicreview}. To develop control-flow anomaly detection techniques, \revision{process mining} has seen widespread adoption owing to process discovery and \revision{conformance checking}. On the one hand, process discovery is a set of algorithms that encode control-flow relations as a set of model elements and constraints according to a given modeling formalism \cite{aalst2022pmhandbook}; hereafter, we refer to the Petri net, a widespread modeling formalism. On the other hand, \revision{conformance checking} is an explainable set of algorithms that allows linking any deviations with the reference Petri net and providing the fitness measure, namely a measure of how much the Petri net fits the new event log \cite{aalst2022pmhandbook}. Many control-flow anomaly detection techniques based on \revision{conformance checking} (hereafter, \revision{conformance checking}-based techniques) use the fitness measure to determine whether an event log is anomalous \cite{bezerra2009pmad, bezerra2013adlogspais, myers2018icsadpm, pecchia2020applicationfailuresanalysispm}. 

The scientific literature also includes many \revision{conformance checking}-independent techniques for control-flow anomaly detection that combine specific types of trace encodings with machine/deep learning \cite{ko2023adsystematicreview, tavares2023pmtraceencoding}. Whereas these techniques are very effective, their explainability is challenging due to both the type of trace encoding employed and the machine/deep learning model used \cite{rawal2022trustworthyaiadvances,li2023explainablead}. Hence, in the following, we focus on the shortcomings of \revision{conformance checking}-based techniques to investigate whether it is possible to support the development of competitive control-flow anomaly detection techniques while maintaining the explainable nature of \revision{conformance checking}.
\begin{figure}[!t]
\centering
\includegraphics[width=\columnwidth]{images/HIGH_LEVEL_VIEW.png}
\caption{A high-level view of the proposed framework for combining \revision{process mining}-based feature extraction with dimensionality reduction for control-flow anomaly detection.}
\label{HIGH_LEVEL_VIEW}
\end{figure}

\subsection{Shortcomings of \revision{conformance checking}-based techniques}
Unfortunately, the detection effectiveness of \revision{conformance checking}-based techniques is affected by noisy data and low-quality Petri nets, which may be due to human errors in the modeling process or representational bias of process discovery algorithms \cite{bezerra2013adlogspais, pecchia2020applicationfailuresanalysispm, aalst2016pm}. Specifically, on the one hand, noisy data may introduce infrequent and deceptive control-flow relations that may result in inconsistent fitness measures, whereas, on the other hand, checking event logs against a low-quality Petri net could lead to an unreliable distribution of fitness measures. Nonetheless, such Petri nets can still be used as references to obtain insightful information for \revision{process mining}-based feature extraction, supporting the development of competitive and explainable \revision{conformance checking}-based techniques for control-flow anomaly detection despite the problems above. For example, a few works outline that token-based \revision{conformance checking} can be used for \revision{process mining}-based feature extraction to build tabular data and develop effective \revision{conformance checking}-based techniques for control-flow anomaly detection \cite{singh2022lapmsh, debenedictis2023dtadiiot}. However, to the best of our knowledge, the scientific literature lacks a structured proposal for \revision{process mining}-based feature extraction using the state-of-the-art \revision{conformance checking} variant, namely alignment-based \revision{conformance checking}.

\subsection{Contributions}
We propose a novel \revision{process mining}-based feature extraction approach with alignment-based \revision{conformance checking}. This variant aligns the deviating control flow with a reference Petri net; the resulting alignment can be inspected to extract additional statistics such as the number of times a given activity caused mismatches \cite{aalst2022pmhandbook}. We integrate this approach into a flexible and explainable framework for developing techniques for control-flow anomaly detection. The framework combines \revision{process mining}-based feature extraction and dimensionality reduction to handle high-dimensional feature sets, achieve detection effectiveness, and support explainability. Notably, in addition to our proposed \revision{process mining}-based feature extraction approach, the framework allows employing other approaches, enabling a fair comparison of multiple \revision{conformance checking}-based and \revision{conformance checking}-independent techniques for control-flow anomaly detection. Figure \ref{HIGH_LEVEL_VIEW} shows a high-level view of the framework. Business processes are monitored, and event logs obtained from the database of information systems. Subsequently, \revision{process mining}-based feature extraction is applied to these event logs and tabular data input to dimensionality reduction to identify control-flow anomalies. We apply several \revision{conformance checking}-based and \revision{conformance checking}-independent framework techniques to publicly available datasets, simulated data of a case study from railways, and real-world data of a case study from healthcare. We show that the framework techniques implementing our approach outperform the baseline \revision{conformance checking}-based techniques while maintaining the explainable nature of \revision{conformance checking}.

In summary, the contributions of this paper are as follows.
\begin{itemize}
    \item{
        A novel \revision{process mining}-based feature extraction approach to support the development of competitive and explainable \revision{conformance checking}-based techniques for control-flow anomaly detection.
    }
    \item{
        A flexible and explainable framework for developing techniques for control-flow anomaly detection using \revision{process mining}-based feature extraction and dimensionality reduction.
    }
    \item{
        Application to synthetic and real-world datasets of several \revision{conformance checking}-based and \revision{conformance checking}-independent framework techniques, evaluating their detection effectiveness and explainability.
    }
\end{itemize}

The rest of the paper is organized as follows.
\begin{itemize}
    \item Section \ref{sec:related_work} reviews the existing techniques for control-flow anomaly detection, categorizing them into \revision{conformance checking}-based and \revision{conformance checking}-independent techniques.
    \item Section \ref{sec:abccfe} provides the preliminaries of \revision{process mining} to establish the notation used throughout the paper, and delves into the details of the proposed \revision{process mining}-based feature extraction approach with alignment-based \revision{conformance checking}.
    \item Section \ref{sec:framework} describes the framework for developing \revision{conformance checking}-based and \revision{conformance checking}-independent techniques for control-flow anomaly detection that combine \revision{process mining}-based feature extraction and dimensionality reduction.
    \item Section \ref{sec:evaluation} presents the experiments conducted with multiple framework and baseline techniques using data from publicly available datasets and case studies.
    \item Section \ref{sec:conclusions} draws the conclusions and presents future work.
\end{itemize}
% !TEX root =  ../main.tex
\section{Background on causality and abstraction}\label{sec:preliminaries}

This section provides the notation and key concepts related to causal modeling and abstraction theory.

\spara{Notation.} The set of integers from $1$ to $n$ is $[n]$.
The vectors of zeros and ones of size $n$ are $\zeros_n$ and $\ones_n$.
The identity matrix of size $n \times n$ is $\identity_n$. The Frobenius norm is $\frob{\mathbf{A}}$.
The set of positive definite matrices over $\reall^{n\times n}$ is $\pd^n$. The Hadamard product is $\odot$.
Function composition is $\circ$.
The domain of a function is $\dom{\cdot}$ and its kernel $\ker$.
Let $\mathcal{M}(\mathcal{X}^n)$ be the set of Borel measures over $\mathcal{X}^n \subseteq \reall^n$. Given a measure $\mu^n \in \mathcal{M}(\mathcal{X}^n)$ and a measurable map $\varphi^{\V}$, $\mathcal{X}^n \ni \mathbf{x} \overset{\varphi^{\V}}{\longmapsto} \V^\top \mathbf{x} \in \mathcal{X}^m$, we denote by $\varphi^{\V}_{\#}(\mu^n) \coloneqq \mu^n(\varphi^{\V^{-1}}(\mathbf{x}))$ the pushforward measure $\mu^m \in \mathcal{M}(\mathcal{X}^m)$. 


We now present the standard definition of SCM.

\begin{definition}[SCM, \citealp{pearl2009causality}]\label{def:SCM}
A (Markovian) structural causal model (SCM) $\scm^n$ is a tuple $\langle \myendogenous, \myexogenous, \myfunctional, \zeta^\myexogenous \rangle$, where \emph{(i)} $\myendogenous = \{X_1, \ldots, X_n\}$ is a set of $n$ endogenous random variables; \emph{(ii)} $\myexogenous =\{Z_1,\ldots,Z_n\}$ is a set of $n$ exogenous variables; \emph{(iii)} $\myfunctional$ is a set of $n$ functional assignments such that $X_i=f_i(\parents_i, Z_i)$, $\forall \; i \in [n]$, with $ \parents_i \subseteq \myendogenous \setminus \{ X_i\}$; \emph{(iv)} $\zeta^\myexogenous$ is a product probability measure over independent exogenous variables $\zeta^\myexogenous=\prod_{i \in [n]} \zeta^i$, where $\zeta^i=P(Z_i)$. 
\end{definition}
A Markovian SCM induces a directed acyclic graph (DAG) $\mathcal{G}_{\scm^n}$ where the nodes represent the variables $\myendogenous$ and the edges are determined by the structural functions $\myfunctional$; $ \parents_i$ constitutes then the parent set for $X_i$. Furthermore, we can recursively rewrite the set of structural function $\myfunctional$ as a set of mixing functions $\mymixing$ dependent only on the exogenous variables (cf. \cref{app:CA}). A key feature for studying causality is the possibility of defining interventions on the model:
\begin{definition}[Hard intervention, \citealp{pearl2009causality}]\label{def:intervention}
Given SCM $\scm^n = \langle \myendogenous, \myexogenous, \myfunctional, \zeta^\myexogenous \rangle$, a (hard) intervention $\iota = \operatorname{do}(\myendogenous^{\iota} = \mathbf{x}^{\iota})$, $\myendogenous^{\iota}\subseteq \myendogenous$,
is an operator that generates a new post-intervention SCM $\scm^n_\iota = \langle \myendogenous, \myexogenous, \myfunctional_\iota, \zeta^\myexogenous \rangle$ by replacing each function $f_i$ for $X_i\in\myendogenous^{\iota}$ with the constant $x_i^\iota\in \mathbf{x}^\iota$. 
Graphically, an intervention mutilates $\mathcal{G}_{\mathsf{M}^n}$ by removing all the incoming edges of the variables in $\myendogenous^{\iota}$.
\end{definition}

Given multiple SCMs describing the same system at different levels of granularity, CA provides the definition of an $\alpha$-abstraction map to relate these SCMs:
\begin{definition}[$\abst$-abstraction, \citealp{rischel2020category}]\label{def:abstraction}
Given low-level $\mathsf{M}^\ell$ and high-level $\mathsf{M}^h$ SCMs, an $\abst$-abstraction is a triple $\abst = \langle \Rset, \amap, \alphamap{} \rangle$, where \emph{(i)} $\Rset \subseteq \datalow$ is a subset of relevant variables in $\mathsf{M}^\ell$; \emph{(ii)} $\amap: \Rset \rightarrow \datahigh$ is a surjective function between the relevant variables of $\mathsf{M}^\ell$ and the endogenous variables of $\mathsf{M}^h$; \emph{(iii)} $\alphamap{}: \dom{\Rset} \rightarrow \dom{\datahigh}$ is a modular function $\alphamap{} = \bigotimes_{i\in[n]} \alphamap{X^h_i}$ made up by surjective functions $\alphamap{X^h_i}: \dom{\amap^{-1}(X^h_i)} \rightarrow \dom{X^h_i}$ from the outcome of low-level variables $\amap^{-1}(X^h_i) \in \datalow$ onto outcomes of the high-level variables $X^h_i \in \datahigh$.
\end{definition}
Notice that an $\abst$-abstraction simultaneously maps variables via the function $\amap$ and values through the function $\alphamap{}$. The definition itself does not place any constraint on these functions, although a common requirement in the literature is for the abstraction to satisfy \emph{interventional consistency} \cite{rubenstein2017causal,rischel2020category,beckers2019abstracting}. An important class of such well-behaved abstractions is \emph{constructive linear abstraction}, for which the following properties hold. By constructivity, \emph{(i)} $\abst$ is interventionally consistent; \emph{(ii)} all low-level variables are relevant $\Rset=\datalow$; \emph{(iii)} in addition to the map $\alphamap{}$ between endogenous variables, there exists a map ${\alphamap{}}_U$ between exogenous variables satisfying interventional consistency \cite{beckers2019abstracting,schooltink2024aligning}. By linearity, $\alphamap{} = \V^\top \in \reall^{h \times \ell}$ \cite{massidda2024learningcausalabstractionslinear}. \cref{app:CA} provides formal definitions for interventional consistency, linear and constructive abstraction.

\section{Adaptive tomography lower bound}
\subsection{Lower bound construction}
\begin{definition}
 \label{def:perturbation}
     Let $\dims^2/2\le\ell\le\dims^2-1$ and $\hbasis=(V_1, \ldots, V_{\dims^2}=\frac{\eye_\dims}{\sqrt{\dims}})$ be an orthonormal basis of $\Herm{\dims}$, and $\cd$ be a universal constant. Let  $\ptb=(\ptb_1, \ldots, \ptb_\ell)$ be uniformly drawn from $\{-1, 1\}^\ell$,
     \begin{equation}
         \Delta_{\ptb} = \frac{\cd\eps}{\sqrt{\dims}}\cdot\frac{1}{\sqrt{\ell}}\sum_{i=1}^\ell \ptb_iV_i, \quad \barDelta_{\ptb}= \Delta_{\ptb}\min\left\{1, \frac{1}{2\dims \opnorm{\Delta_{\ptb}}}\right\},
         \label{equ:delta_z}
     \end{equation}
     Finally we set $\sigma_{\ptb}=\qmm + \barDelta_{\ptb}$ whose distribution we denote as $\ptbDistr(\hbasis)$.
 \end{definition}

The construction adds independent binary perturbations to $\qmm$ along $\ell$ orthogonal trace-0 directions. With appropriate constant $\cd$, $\ptbDistr(\hbasis)$ has an exponentially small probability mass outside the set $\mathcal{P}_\eps\eqdef\{\rho: \tracenorm{\rho-\qmm}>\eps\}$.
\begin{theorem}[{\cite[Corollary 4.4]{liu2024role}}]
\label{prop:perturbation-trace-distance}
    Let $\cd= 10\sqrt{2}$, $\ell\ge \dims^2/2$, $\eps<1/200$. Then for $\sigma\sim \ptbDistr(\hbasis)$,  $\|\sigma-\qmm\|_1\ge \eps$ with probability at least $1-2\exp(-\dims)$. 
\end{theorem}

This is the result of a random matrix concentration.
\begin{restatable}[{\cite[Theorem 4.2]{liu2024role}}]{theorem}{randmatopnorm}
\label{thm:rand-mat-opnorm-concentration}
    Let $V_1, \ldots, V_{\dims^2}\in\C^{\dims\times \dims}$ be an orthonormal basis of $\C^{\dims\times \dims}$ and $\ptb_1, \ldots, \ptb_{\dims^2}\in\{-1, 1\}$ be independent symmetric Bernoulli random variables. Let $W=\sum_{i=1}^{\ell}\ptb_iV_i$ where $\ell\le \dims^2$. For all $\alpha>0$, there exists $\cop_\alpha$, {which is increasing in $\alpha$} such that
    \[
    \probaOf{\opnorm{W}>\cop_\alpha\sqrt{\dims}}\le 2\exp\{-\alpha\dims\}.
    \]
\end{restatable}

Let $z\sim\{-1,1\}^{\ell}$ and $\sigma_z\sim \ptbDistr(\hbasis)$ be defined in \cref{def:perturbation}. Use the shorthand $\p_z^{\out_i|\out^{i-1}}=\p_{\sigma_z}^{\out_i|\out^{i-1}}$. 
We define the following mixtures,
\begin{equation}
    \p_{+i}^{\out^\ns}\eqdef \frac{1}{2^{\ell-1}}\sum_{\ptb:\ptb_i=+1}\p_z^{\out^\ns},\quad  \p_{-i}^{\out^\ns}\eqdef \frac{1}{2^{\ell-1}}\sum_{\ptb:\ptb_i=-1}\p_z^{\out^\ns}.
\end{equation}
Which are the distributions of outcomes $\out^\ns$ when we fix the $i$th coordinate to be $+1$ and $-1$ respectively. Then we can define,
\begin{equation}
\label{equ:out-distr}
\q^{\out^\ns}\eqdef\frac{1}{2^\ell}\sum_{z\in\{-1,1\}^\ell}\p_z^{\out^\ns}=\frac{1}{2}(\p_{+i}^{\out^\ns}+\p_{-i}^{\out^\ns}).  
\end{equation}

This is exactly the distribution of $\out^\ns$ when $\sigma_z\sim \ptbDistr(\hbasis)$ and outcomes $\out^\ns$ are obtained by measuring $\sigma_z^{\otimes\ns}$ with the adaptive scheme $\POVM^\ns$.




\subsection{Mutual information upper bound via MIC}
The following theorem bounds the mutual information in terms of the measurement information channel.
\begin{theorem}
\label{thm:avg-MI-upper}
    Let $\sigma_\ptb\sim\ptbDistr(\hbasis)$ where $\ptb\sim\{-1,1\}^{\ell}$, $\out^\ns$ be the outcomes after applying $\POVM^\ns$. Then for $\dims\ge 1024$ and all $t\in[\ns]$,
    \begin{align}
         \frac{1}{\ell}\sum_{i=1}^{\ell}\mutualinfo{\ptb_i}{\out^t}&\le \frac{8 tc^2 \eps^2}{\ell^2}  \sup_{\POVM\in {\povmset}}{\sum_{i=1}^\ell \vadj{V_i} \Choi_{\POVM} \vvec{V_i}} +16\exp\{-\alpha\dims\}tc^2\eps^2 \label{equ:avg-MI-partial}\\
         &\le \frac{16tc^2\eps^2}{\ell^2}\tracenorm{\povmset}.\label{equ:avg-MI-tracenorm}
    \end{align}
\end{theorem}

\begin{proof}
    We start with the fundamental fact that mutual information is the conditional KL-divergence between the conditional distribution given the marginal $x^t$: $\p^{\out^t}_{z_i}$ for $1 \leq i \leq n$ and the marginal distribution $\q^{\out^t}$,
\begin{align*}
    I(z_i;x^t) &= \kldiv{\p^{\out^t}_{z_i}}{\q^{\out^t} \mid z_i} = \expectDistrOf{z_i}{\kldiv{\p^{\out^t}_{z_i}}{\q^{\out^t}}} \\ &= \frac{1}{2} \kldiv{\p_{+i}^{\out^t}}{\q^{\out^t}} + \frac{1}{2} \kldiv{\p_{-i}^{\out^t}}{\q^{\out^t}} \\ 
    &= \frac{1}{2} \kldiv{\p_{+i}^{\out^t}}{\frac{\p_{+i}^{\out^t} + \p_{-i}^{\out^t}}{2}} + \frac{1}{2} \kldiv{\p_{-i}^{\out^t}}{\frac{\p_{+i}^{\out^t} + \p_{-i}^{\out^t}}{2}}.
\end{align*}
Thus, by convexity, 
\begin{align}
    I(z_i;x^t) &\leq \frac{1}{4} \left[\kldiv{\p^{\out^t}_{+i}}{\p^{\out^t}_{+i}}+\kldiv{\p^{\out^t}_{-i}}{\p^{\out^t}_{+i}} \right] = \frac{1}{2} \kldivsym{\p^{\out^t}_{+i}}{\p^{\out^t}_{-i}} \label{pf:MI-bound}.
\end{align}
Where the last inequality comes from the convexity of KL-divergence with respect to its second argument. Given this symmetric KL-divergence between the mixture distribution conditioned on the i-th perturbation, we can further narrow the correlation between the measurement outcomes and the perturbation with the change in measurement outcome distribution when $z_i$ is flipped. We apply chain rule on the symmetric KL-divergence to allow us to isolate the per measurement round divergence,
\begin{align}
    \kldivsym{\p^{\out^t}_{+i}}{\p^{\out^t}_{-i}} = \sum_{j=1}^{t} \expectDistrOf{\q^{x^\ns}}{\kldivsym{\p^{\out_{j}|\out^{j-1}}_{+i}}{\p^{\out_{j}|\out^{j-1}}_{-i}}}.
    \label{eq:per-round-divergence}
\end{align}
We bound the symmetric KL by the chi-squared divergence,
\begin{align*}
    \kldivsym{\p^{\out_{j}|\out^{j-1}}_{+i}}{\p^{\out_{j}|\out^{j-1}}_{-i}} &\le \chisquare{\p^{\out_{j}|\out^{j-1}}_{+i}}{\p^{\out_{j}|\out^{j-1}}_{-i}} \\
    &\leq \frac{1}{2^{l-1}} \sum_{z \in \{+1,-1\}^\ell} \; \chisquare{\p^{\out_{j}|\out^{j-1}}_{z}}{\p^{\out_{j}|\out^{j-1}}_{z^{\oplus i}}}  \\
    &= \frac{1}{2^{\ell-1}} \sum_{z \in \{+1,-1\}^\ell} \; \expectDistrOf{X \sim \p^{\out_{j}|\out^{j-1}}_{z^{\oplus i}}}{\delta_j(X)^2}.
\end{align*}
 Where the last inequality is from the joint convexity of f-divergences. $\delta_t(X)$ follows the definition,
\begin{align*}
    \delta_j(x) \eqdef \frac{\p^{\out_{j}|\out^{j-1}}_{z}(x)-\p^{\out_{j}|\out^{j-1}}_{z^{\oplus i}}(x)}{\p^{\out_{j}|\out^{j-1}}_{z^{\oplus i}}(x)}.
\end{align*}
Furthermore, $\delta_j$ term can be bounded by extracting the MIC channel,
\begin{align*}
    \delta_j(x) &= \frac{\Tr[M_x^j (\qmm + \barDelta_{\ptb})] - \Tr[M_x^j (\qmm + \barDelta_{\ptb^{\oplus i}})]}{\Tr[M_x^j (\qmm + \barDelta_{\ptb^{\oplus i}})]} \\
    &= \frac{\Tr[M_x^j (\barDelta_{\ptb} - \barDelta_{\ptb ^{\oplus i}})]}{\Tr[M_x^j (\qmm + \barDelta_{\ptb^{\oplus i}})]}.
\end{align*}
Therefore, we plug $\delta_j(X)$ into the expectation and noting that $\p^{\out_{j}|\out^{j-1}}_{z^{\oplus i}}(x) = \Tr[M_x^j (\qmm + \barDelta_{\ptb^{\oplus i}})]$,
\begin{align*}
    \expectDistrOf{X \sim \p^{\out_{j}|\out^{j-1}}_{z^{\oplus i}}}{\delta_j(X)^2} &= \sum_{x \in X} \frac{\Tr[M_x^j (\barDelta_{\ptb} - \barDelta_{\ptb ^{\oplus i}})]^2}{\Tr[M_x^j (\qmm + \barDelta_{\ptb^{\oplus i}})]} \\
    &= \sum_{x \in X} \frac{\Tr[(\barDelta_{\ptb} - \barDelta_{\ptb ^{\oplus i}}) M_x^j] \Tr[(\barDelta_{\ptb} - \barDelta_{\ptb ^{\oplus i}}) M_x]^{\dagger}}{\Tr[M_x^j (\qmm + \barDelta_{\ptb^{\oplus i}})]} \\
    &= \sum_{x \in X} \frac{\Tr[(\barDelta_{\ptb} - \barDelta_{\ptb ^{\oplus i}}) M_x^j] \Tr[(\barDelta_{\ptb} - \barDelta_{\ptb ^{\oplus i}}) M_x^j]^{\dagger}}{\Tr[M_x^j (\qmm + \barDelta_{\ptb^{\oplus i}})]} \\
    &= \sum_{\out \in \mathcal{X}} \frac{\vvdotprod{(\barDelta_{\ptb} - \barDelta_{\ptb ^{\oplus i}})}{M_x^j}\vvdotprod{M_x^j}{(\barDelta_{\ptb} - \barDelta_{\ptb ^{\oplus i}})}}{\Tr[M_x^j(\qmm + \barDelta_{\ptb^{\oplus i}})]}.
\end{align*}
Note that $\qmm + \barDelta_{\ptb^{\oplus i}} \succcurlyeq \frac{1}{2} \qmm \implies \Tr[M_x^j(\qmm + \barDelta_{\ptb^{\oplus i}})] \geq \Tr[M_x^j(\frac{1}{2} \qmm)] = \frac{1}{2 \dims}\Tr[M_x^j]$. This statement comes from the fact that $\opnorm{\barDelta_{\ptb^{\oplus i}}} \leq \frac{1}{2 \dims}$~\eqref{equ:delta_z},
\begin{align*}
   \expectDistrOf{X \sim \p^{\out_{j}|\out^{j-1}}_{z^{\oplus i}}}{\delta_j(X)^2} &\le  2 \dims \sum_{\out \in \mathcal{X}} \frac{\vvdotprod{(\barDelta_{\ptb} - \barDelta_{\ptb ^{\oplus i}})}{M_x^j}\vvdotprod{M_x^j}{(\barDelta_{\ptb} - \barDelta_{\ptb ^{\oplus i}})}}{\Tr[M_x^j]} \\
   &=  2 \dims \vadj{(\barDelta_{\ptb} - \barDelta_{\ptb ^{\oplus i}})} \sum_{\out \in \mathcal{X}} \frac{\vvec{M_x^j}\vadj{M_x^j}}{\Tr[M_x^j]} \vvec{(\barDelta_{\ptb} - \barDelta_{\ptb ^{\oplus i}})}.
\end{align*}
We can then apply this bound to the per-round symmmetric KL-divergence,
\begin{align*}
   \kldivsym{\p^{\out_{j}|\out^{j-1}}_{+i}}{\p^{\out_{j}|\out^{j-1}}_{-i}}&\le \frac{1}{2^{\ell-1}} \sum_{z \in \{+1,-1\}^l} \; 2 \dims \vadj{(\barDelta_{\ptb} - \barDelta_{\ptb ^{\oplus i}})} \Sigma_{\out \in \mathcal{X}} \frac{\vvec{M_x^j}\vadj{M_x^j}}{\Tr[M_x^j]} \vvec{(\barDelta_{\ptb} - \barDelta_{\ptb ^{\oplus i}})} \\
   &= 4 \dims \expectDistrOf{z\sim\{-1, 1\}^{\ell}}{\vadj{(\barDelta_{z} - \barDelta_{z ^{\oplus i}})} \Choi_{\POVM_j} \vvec{(\barDelta_{\ptb} - \barDelta_{z ^{\oplus i}})}},
\end{align*}
where $z$ is drawn uniformly from $\{-1,1\}^{\ell}$.
Another key to this bound is that we have a concentration on the operator norm of the perturbation matrix such that the operator norm lies in the boundary (within some constant)  with exponentially decreasing probability, see \cref{thm:rand-mat-opnorm-concentration}. 
Intuitively, this means that it is rare that all of the $z_i$ components are selected in a way where eigenvectors of the $z_i V_i$ components are aligned, thus resulting in an equal contribution to the total perturbation from each $z_i V_i$ component. 
As a result, this concentration perspective allows us to see that flipping a single $z_i V_i$ entry will dictate a perturbation outcome with high probability.
For convenience, we define the concentration set for the perturbation parameters,
\begin{align}
    \mathcal{G} := \{z \in \{1,1\}^{\ell} \; | \; \opnorm{W_z} \leq \kappa_\alpha \sqrt{d}\} \label{eq:concentrated-set},   
\end{align}
where $\alpha$ is a positive constant and $\kappa_\alpha$ is a positive constant non-decreasing in $\alpha$. By \cref{thm:rand-mat-opnorm-concentration},
\[
\probaOf{z \in \mathcal{G}} \geq 1 - 2\exp\{- \alpha d\}.
\]
For more details on the constants involved, see Lemma 21 of \cite{liu2024role}. 
We then condition between the possible cases of perturbations with law of iterative expectation,

\begin{align}
    \kldivsym{\p^{\out_{j}|\out^{j-1}}_{+i}}{\p^{\out_{j}|\out^{j-1}}_{-i}} &\le 4 \dims \expect{\expectDistrOf{z}{\vadj{(\barDelta_{\ptb} - \barDelta_{\ptb ^{\oplus i}})} \Choi_{\POVM_j} \vvec{(\barDelta_{\ptb} - \barDelta_{\ptb ^{\oplus i}})} \;| \indic{z \in \mathcal{G}}}}\label{pf:KL-div-concentration}.
\end{align}
  Now, it suffices to bound the peturbation for when $z \in G$ and $z \notin G$. When $z \in G$, the following bound holds for $\eps \leq \frac{1}{4(\kappa_\alpha+1)}$,
\begin{align}
\opnorm{\Delta_z} &= \frac{c\eps}{\sqrt{\dims \ell}} \opnorm{W_z} \leq  \frac{\kappa_\alpha c\eps}{\sqrt{\ell}} \leq \frac{2 \kappa_\alpha c\eps}{\dims} \leq \frac{1}{2 \dims} .\label{eq:op-norm-bound}
\end{align}
In addition, the following holds when the i-th bit is flipped,
\begin{align*}
        \opnorm{W_{z ^{\oplus i}}} &= \opnorm{W_z - 2 z_i V_i} \leq \opnorm{W_z} + \opnorm{-2 z_i V_i} \\
    &\le \kappa_\alpha \sqrt{\dims} + 2 \leq (\kappa_\alpha + 1) \sqrt{d} \\
    \implies \opnorm{\Delta_{z^{\oplus i}}} &= \frac{c\eps}{\sqrt{\dims \ell}} \opnorm{W_{z^{\oplus i}}} \leq  \frac{(\kappa_\alpha + 1) c\eps}{\sqrt{\ell}} \leq \frac{2(\kappa_\alpha + 1) c\eps}{\dims} \leq \frac{1}{2\dims}.
\end{align*}
The second inequality follows because $ \opnorm{V_i}^2 = \|V_i\|_{S_\infty}^2 \leq\|V_i\|_{S_2}^2 = \hdotprod{V_i}{V_i} = 1$. For $z \in \mathcal{G}$, we have that $\opnorm{\Delta_{z}}, \opnorm{\Delta_{z^{\oplus i}}} \leq \frac{1}{2 \dims}$. This results in $\barDelta_{z ^{\oplus i}} = \Delta_{z ^{\oplus i}}, \;\barDelta_{z} = \Delta_{z}$, by definition of the normalization factor in \cref{equ:delta_z}. Thus,
\begin{align*}
 \vadj{(\barDelta_{z} - \barDelta_{z ^{\oplus i}})} \Choi_{\POVM_j} \vvec{(\barDelta_{z} - \barDelta_{z ^{\oplus i}})} &=  \vadj{(\Delta_{z} - \Delta_{z ^{\oplus i}})}
 \Choi_{\POVM_j} \vvec{(\Delta_{z} - \Delta_{z ^{\oplus i}})} \\
 &=\vadj{\frac{c\eps}{\sqrt{\dims \ell}} 2 z_i V_i}
 \Choi_{\POVM_j}
 \vvec{\frac{c\eps}{\sqrt{\dims \ell}} 2 z_i V_i} = \frac{4 c^2 \eps^2 z_{i}^2}{\dims \ell}  = \frac{4 c^2 \eps^2}{\dims \ell} \vadj{V_i} \Choi_{\POVM_j} \vvec{V_i}.
\end{align*}
We will later see that this will result in the trace decomposition of $\Choi_{\POVM_j}$ under the vectorized version of the orthornormal Hilbert basis $\hbasis$. Now, we will apply a more crude bound for the low-concentration set $z \notin \mathcal{G}$. We start by bounding the Hilbert-Schmidt norm of the perturbation matrix for every $z \in \{-1,1\}^\ell$,
\begin{align*}
\hsnorm{\barDelta_{z}} &= \sqrt{\vvdotprod{\barDelta_{z}}{\barDelta_{z}}} \\
&= \sqrt{\frac{c^2\eps^2}{\dims \ell}\vvdotprod{\sum_{i=1}^\ell \min\left\{1, \frac{1}{2\dims \opnorm{\Delta_{\ptb}}}\right\} z_i V_i}{\sum_{i=1}^\ell \min\left\{1, \frac{1}{2\dims \opnorm{\Delta_{\ptb}}}\right\} z_i V_i}}\\
&= \sqrt{\frac{c^2 \eps^2}{\dims \ell} \sum_{i \neq j} \min\left\{1, \frac{1}{2\dims \opnorm{\Delta_{\ptb}}}\right\}^2 z_i z_j \vvdotprod{V_i}{V_j} + \sum_{i}^\ell \min\left\{1, \frac{1}{2\dims \opnorm{\Delta_{\ptb}}}\right\}^2 z_i^2 \vvdotprod{V_i}{V_i}} \\
&= \sqrt{\frac{c\eps^2}{\dims \ell} \sum_{i}^\ell \min\left\{1, \frac{1}{2\dims \opnorm{\Delta_{\ptb}}}\right\}^2} \leq  \sqrt{\frac{c\eps^2}{\dims \ell} \sum_{i}^\ell 1} = \frac{c \eps}{\sqrt{\dims}}.
\end{align*} 
Where last line holds from the orthonormality of the perturbation basis. Now, we can use triangle inequality of the Hilbert-Schmidt norm to get the bound on the Hilbert-Schmidt norm of the difference between the perturbation matrices.
\begin{align*}
    \vadj{(\barDelta_{z} - \barDelta_{z ^{\oplus i}})} \Choi_{\POVM_j} \vvec{(\barDelta_{z} - \barDelta_{z ^{\oplus i}})} 
    &\le \opnorm{\Choi_{\POVM_j}}\hsnorm{\barDelta_{z} - \barDelta_{z ^{\oplus i}}}^2  \\
    &\leq 2(\hsnorm{\barDelta_{z}}^2 + \hsnorm{\barDelta_{z ^{\oplus i}}}^2) \\
    &\leq \frac{4c^2 \eps^2}{\dims}.
\end{align*}
The first step is due to the definition of operator norm. The second step is because $\opnorm{\Choi_{\POVM_j}} \le 1$, triangle inequality, and $(a+b)^2\le 2(a^2+b^2)$. We can further bound the symmetric KL-divergence in \cref{pf:KL-div-concentration},
 \begin{align*}
     \kldivsym{\p^{\out_{j}|\out^{j-1}}_{+i}}{\p^{\out_{j}|\out^{j-1}}_{-i}} &\le 4 \dims \left[\probaOf{z \in \mathcal{G}} \frac{4 c^2 \eps^2}{\dims\ell} \vadj{V_i} \Choi_{\POVM_j} \vvec{V_i} + (1 - \probaOf{z \in \mathcal{G}})   \frac{4 c^2 \eps^2}{\dims}\right] \\
     &= \probaOf{z \in \mathcal{G}} \frac{16 c^2 \eps^2}{\ell} \vadj{V_i} \Choi_{\POVM_j} \vvec{V_i} + (1-\probaOf{z \in \mathcal{G}}) 16 c^2 \eps^2.
 \end{align*} 

Thus combining with \eqref{pf:MI-bound} \eqref{eq:per-round-divergence},
\begin{align*}
    \frac{1}{\ell}\sum_{i=1}^{\ell}\mutualinfo{\ptb_i}{\out^t} &\le \frac{1}{2\ell}\sum_{i=1}^{\ell}\sum_{j=1}^{t} \expectDistrOf{\q^{\out^\ns}}{\kldivsym{\p^{\out_{j}|\out^{j-1}}_{+i}}{\p^{\out_{j}|\out^{j-1}}_{-i}}} \\
    &\leq \probaOf{z \in \mathcal{G}} \frac{8 c^2 \eps^2}{\ell^2} \sum_{j=1}^{t} \sum_{i=1}^{\ell} \expectDistrOf{\q^{\out^\ns}}{ \vadj{V_i} \Choi_{\POVM_j} \vvec{V_i}}
    + (1 - \probaOf{z \in \mathcal{G}}) \sum_{j=1}^{t} \sum_{i=1}^{\ell}  \frac{8c^2\eps^2}{\ell} \\
    &\le  \frac{8 tc^2 \eps^2}{\ell^2}  \expectDistrOf{\q^{\out^\ns}}{\frac{1}{t}\sum_{j=1}^{t}\sum_{i=1}^\ell \vadj{V_i} {\Choi}_{\POVM_j} \vvec{V_i}} +16\exp\{-\alpha\dims\}tc^2\eps^2\\
    &\le \frac{8 tc^2 \eps^2}{\ell^2}  \sup_{\POVM\in\povmset}\sum_{i=1}^\ell \vadj{V_i} {\Choi}_{\POVM} \vvec{V_i} +16\exp\{-\alpha\dims\}tc^2\eps^2,
\end{align*}
The second term in the final step is due to \cref{thm:rand-mat-opnorm-concentration}. This proves \eqref{equ:avg-MI-partial} in \cref{thm:avg-MI-upper}.

We continue to derive the remaining expression \eqref{equ:avg-MI-tracenorm}. We use the fact that for any matrix $A\in\C^{\dims\times\dims}$ and an orthonormal basis $\qbit{u_1}, \ldots, \qbit{u_\dims}$,
\[
\Tr[A]=\sum_{i=1}^{\dims}\matdotprod{u_i}{A}{u_i}.
\]
Combining with the fact that ${\Choi}_{\POVM}$ is p.s.d., we have
\[
\sum_{i=1}^{\ell} \vadj{V_i} {\Choi}_{\POVM} \vvec{V_i}\le \sum_{i=1}^{\dims^2}\vadj{V_i}\Choi_{\POVM} \vvec{V_i}=\Tr[\Choi_{\POVM}]=\tracenorm{\Choi_{\POVM}}.
\]
Therefore, continuing from \eqref{equ:avg-MI-partial},

\begin{align*}
     \frac{1}{\ell}\sum_{i=1}^{\ell}\mutualinfo{\ptb_i}{\out^t} 
    &\le \frac{8 t c^2 \eps^2}{\ell^2} \tracenorm{\povmset} + 16\exp\{-\alpha\dims\}tc^2\eps^2\\
    & \le \frac{16 t c^2 \eps^2}{\ell^2} \tracenorm{\povmset}.
\end{align*}
The first step is from the definition of $\tracenorm{\povmset}$ in \eqref{equ:max-povm-norm}. The second step holds as $\tracenorm{\povmset}\ge 1$ and $\exp\{-\alpha d\} \leq \frac{1}{d^4}$ when $d \geq 1024$. 
\end{proof}


\subsection{Mutual information lower bound}
We state some useful bounds on mutual information.
\begin{lemma}[{\cite[Lemma 10]{ACLST22iiuic}}]
\label{lem:MI-lower}
    Let $Z\in\{-1, 1\}^\ab$ be drawn uniformly and $Z-Y-\hat{Z}$ be a Markov chain where $\hat{Z}$ is an estimate of $Z$. Let $h(t)\eqdef -t\log t-(1-t)\log(1-t)$, then for each $i\in[\ab]$,
    \[
    \mutualinfo{Z_i}{Y}\ge 1-h(\probaOf{Z_i\ne \hat{Z}_i}).
    \]
\end{lemma}

The following lemma is an Assouad-type lower bound on the average mutual information. 
\begin{lemma}
\label{lem:avg-MI-lower}
    Let $\sigma_\ptb\sim\ptbDistr(\hbasis)$ where $\ptb\sim\{-1,1\}^{\ell}$, $\out^\ns$ be the outcomes after applying $\POVM^\ns$ to $\sigma_\ptb^{\otimes\ns}$, and $\qest$ be an estimator using $\out^\ns$ that achieves an accuracy of $\eps$. Then,
    \begin{equation}
        \frac{1}{\ell}\sum_{i=1}^{\ell}\mutualinfo{Z_i}{Y}\ge\frac{1}{100}.
    \end{equation}
\end{lemma}


Combining \cref{lem:avg-MI-lower} and \cref{thm:avg-MI-upper} proves the interactive lower bound for tomography.
\begin{proof}
   The idea behind this bound is that any good estimation $\qest$ of the parameterized state $\sigma_\ptb$ is close in the sense that the closest parameterized $\sigma_{\zest}$ to $\qest$ should also be sufficiently close. Then, we can relate the distance $\tracenorm{\sigma_{\ptb} - \sigma_{\zest}}$ to the hamming distance in $\sum_{i=1}^\ell \indic{z_i \neq \zest_i}$. Once this relation is established, then a optimal tomography algorithm should also have low probability of error in estimating $z$ with $\zest$. Thus, leading to lower bound of mutual information with the application of \cref{lem:MI-lower}. We begin by first bounding the error between the "parameterized version" of the estimator and $\sigma_{\hat{\ptb}}$,
   \begin{align*}
        \zest &:= \argmin_{\ptb \in \{-1,1\}^\ell } \tracenorm{\sigma_{\ptb} - \qest}\\
        \tracenorm{\sigma_{\zest} - \sigma_{\ptb}} &\leq \tracenorm{\sigma_{\ptb} - \qest} + \tracenorm{\qest-\sigma_{\zest}} \leq 2 \tracenorm{\sigma_{\ptb} - \qest},
   \end{align*}
   where the last line holds since $\tracenorm{\qest-\sigma_{\zest}} \leq \tracenorm{\qest-\sigma_{\ptb}}$ by definition of $\hat{\ptb}$. Notice $ \tracenorm{\qest-\sigma_{\ptb}} \leq \eps \implies \tracenorm{\sigma_{\zest} - \sigma_{\ptb}} \leq 2\eps $. Thus, 
   $$\Pr[\tracenorm{\sigma_{\zest} - \sigma_{\ptb}} \leq 2 \eps] \ge \Pr[\tracenorm{\sigma_{\zest} - \sigma_{\ptb}} \leq \eps] \geq 0.99.$$
   Now, we will introduce a lemma that will allow us to construct a informaton-theoretic packing around this estimator. This is done by relating the trace distance and the hamming distance between Z parameters. We present the formal version of~\cref{lemma:hamm-separation-informal}

   \begin{lemma}[Trace distance Hamming separation] \label{lemma:hamm-packing}
       Consider $z \in \mathcal{G}$, where $\mathcal{G}$ is defined from \cref{eq:concentrated-set}. For any  $\hat{z} \in \left\{-1,1\right\}^{\ell}$,
       \begin{equation}
           \tracenorm{\sigma_\ptb - \sigma_{\zest}} \geq \frac{c \eps}{2\kappa_\alpha \ell} \ham{\ptb}{\zest}.
       \end{equation}
   \end{lemma}

   \begin{proof}
       
   Let $C_{z} := \min\left\{1, \frac{1}{2\dims \opnorm{\Delta_{z}}}\right\}$ and define the matrices,
   \[\Delta_{w} := \frac{c \eps}{\sqrt{d\ell}}  \sum_{i=1}^\ell \indic{z_i \neq \hat{z}_i} z_i V_i, \;\Delta_{c} := \frac{c \eps}{\sqrt{d\ell}} \sum_{i=1}^\ell \indic{z_i = \hat{z}_i} z_i V_i.
   \]
   Notice the trace norm of distance between perturbation matrices has the following form,
   \begin{align*}   
   \tracenorm{\sigma_{\zest} - \sigma_{\ptb}} & = \tracenorm{\barDelta_{\hat{\ptb}} - \barDelta_{\ptb}} \\
   &=\tracenorm{C_{\hat{z}} \Delta_{\hat{\ptb}} - C_{z} \Delta_{\ptb}} \\
   &= \frac{c \eps}{\sqrt{d\ell}} \tracenorm{(-C_z-C_{\zest}) \sum_{i=1}^\ell \indic{z_i \neq \zest_i} z_i V_i + (C_{\zest} - C_z) \sum_{i=1}^\ell \indic{z_i = \zest_i} z_i V_i} \\
   &= \tracenorm{(C_z+C_{\zest}) \Delta_w + (C_z-C_{\zest}) \Delta_c)}.
   \end{align*}
    Now, we will take advantage of the duality between the trace and operator norm (\cref{lemma:trace-norm-dual}) to correlate the distance between perturbations to the hamming distance between $z$ and $\zest$. Let $W_z = \sum_{i=1}^\ell z_i V_i$. For $z$ such that $\opnorm{W_z} \leq \kappa_\alpha \sqrt{\dims}$, we have $C_z=1$, from \cref{eq:op-norm-bound}.
   \begin{align*}
    \tracenorm{\sigma_{\zest} - \sigma_{\ptb}} &=
     \tracenorm{((1+C_{\zest}) \Delta_w + (1-C_{\zest}) \Delta_c} = \sup_{\opnorm{B} \leq 1} |\Tr[B^{\dagger} \left[(1+C_{\zest}) \Delta_w + (1-C_{\zest}) \Delta_c\right]]| \\
     &\geq \frac{1}{\kappa_\alpha \sqrt{\dims}} 
     |\Tr[W_z^{\dagger} \left[(1+C_{\zest}) \Delta_w + (1-C_{\zest}) \Delta_c\right]]| = \frac{c \eps}{\sqrt{\dims \ell}} \frac{1}{\kappa_\alpha \sqrt{\dims}} |(1+C_{\zest}) \delta_w + (1-C_{\zest}) \delta_c| \\
     &= \frac{c \eps}{\kappa_\alpha d \sqrt{\ell}} \left[(1+C_{\zest}) \delta_w + (1-C_{\zest}) \delta_c\right] \geq \frac{c \eps}{2 \kappa_\alpha \ell} \delta_w.
   \end{align*}
   Where $\delta_w = \ham{z}{\zest}$ and $\delta_c = \ell - \delta_w = \ell - \ham{z}{\zest}$.The second inequality uses: $B = \frac{W_z}{\kappa_\alpha \sqrt{\dims}}$. The reduction $\Tr[W_z^{\dagger} \Delta_w] = \delta_w$ and $\Tr[W_z^{\dagger} \Delta_c] = \delta_c$ comes directly from the orthonormality of the peturbation matrices $\{V_i\}_{i=1}^\ell$ under the inner product: $\hdotprod{A}{B} = \vvdotprod{A}{B} = \Tr[A^\dagger B]$. With the last line, we have shown the desired bound.  
   \end{proof}
    
   Since this relation to $\ham{\cdot}{\cdot}$ only occurs for a concentrated set $\mathcal{G}$, we can show that the expected hamming distance is "approximately trace distance" for sufficiently large $\dims > \frac{\ln{5}}{\alpha}$. $\sigma_{\hat{z}}$ also has to be close to $\sigma_z$ with high probability to be a sufficient estimator of $\sigma_z$, inducing a upper bound on the error probability of estimating $Z$,
   \begin{align}
       \frac{1}{\ell} \expectDistrOf{}{\delta_w} &= \frac{1}{\ell} \expectDistrOf{}{\delta_w \mid \tracenorm{\sigma_z - \sigma_{\zest}} \leq 2 \eps} \Pr[\tracenorm{\sigma_z - \sigma_{\zest}} \leq 2 \eps] \nonumber\\
       &+ \frac{1}{\ell} \expectDistrOf{}{\delta_w | \tracenorm{\sigma_z - \sigma_{\zest}} > 2 \eps} \Pr[\tracenorm{\sigma_z - \sigma_{\zest}} > 2 \eps] \nonumber\\
       &\leq \frac{1}{\ell} \expectDistrOf{}{\delta_w \mid \tracenorm{\sigma_z - \sigma_{\zest}} \leq 2 \eps} + 0.01.  \label{eq:cond-expect-tom-lower}
   \end{align}
   It is enough to upper bound the remaining expectation term by a constant. We will case on whether $z$'s lead to a approximate hamming relationship with trace distance. When $z \in \mathcal{G}$, we apply \cref{lemma:hamm-packing}
   \begin{align*}
       \frac{c \eps}{2 \kappa_\alpha \ell} \delta_w \leq \tracenorm{\sigma_z - \sigma_{\zest}} \leq 2 \eps 
       \implies  \frac{1}{\ell} \delta_w \leq \frac{4 \kappa_\alpha}{c}.
   \end{align*}
   The conditional expectation will now be bounded by a small constant for $c \geq 10 \kappa_\alpha$,
   \begin{align*}
       \frac{1}{\ell} \expectDistrOf{}{\delta_w \mid \tracenorm{\sigma_\ptb - \sigma_{\zest}} \leq 2 \eps} &\leq \Pr[z \in \mathcal{G}]\frac{1}{\ell} \expectDistrOf{}{\delta_w \mid \tracenorm{\sigma_\ptb - \sigma_{\zest}} \leq 2 \eps \land z \in \mathcal{G}} \\
       &+\Pr[z \notin \mathcal{G}] \frac{1}{\ell} \expectDistrOf{}{\delta_w \mid \tracenorm{\sigma_z - \sigma_{\zest}} \leq 2 \eps \land z \notin \mathcal{G}} \\
       &\leq \frac{4 \kappa_\alpha}{c} +  2\exp\{-\alpha d\} \cdot 1 \leq \frac{2}{5} + \frac{2}{5} = 0.40.
   \end{align*}
   Substituting this result into \cref{eq:cond-expect-tom-lower}, we have $\frac{1}{\ell} \sum_{i=1}^\ell \Pr[Z_i \neq \hat{Z_i}] = \frac{1}{\ell} \expectDistrOf{}{\delta_w} \leq 0.41$. We can then apply \cref{lem:MI-lower} to obtain the mutual information bound,
   \begin{align*}
       \frac{1}{\ell} \sum_{i=1}^\ell \mutualinfo{Z_i}{Y} \geq 1 - h\left(\frac{1}{\ell} \sum_{i=1}^\ell \Pr[Z_i \neq \hat{Z_i}]\right) \geq 1 - h\left(0.41\right) \geq \frac{1}{100}.
   \end{align*}
   The first inequality is due to the concavity of the binary entropy function $h$.
\end{proof}
\section{Lower bound for tomography with Pauli measurements}
The key to proving a tight lower bound for Pauli measurements is to design a measurement-dependent hard instance. Recall that any quantum state $\rho$ is a linear combination of Pauli observables,
\[
\rho=\frac{\eye_\dims}{\dims}+\sum_{P\in\pauliObsSet}\frac{\Tr[\rho P]}{\dims}P.
\]

Further recall the observation \cref{sec:techniques} that Pauli measurements \eqref{equ:pauli-measurement} are better at learning information about directions 
$P\in \pauliObsSet$ with a small weight and less powerful $P$ with a larger weight. 
% Consider $P=\pauliX^{\otimes \nqubits}$ and its corresponding Pauli basis measurement $\POVM_P$. The measurement simultaneously provides information about $\nqubits$ weight-1 Pauli observables $Q=\sigma_1\otimes\cdots\otimes \sigma_\nqubits$ such that $\sigma_i=\pauliX$ for some $i$ and $\sigma_j=\eye_2$ everywhere else. However, it only provides information about 1 Pauli observable with weight $N$, which is exactly $\sigma_X^{\otimes\nqubits}$.
As such, we set the basis $V_1, \ldots, V_{\dims^2-1}$ in the lower bound construction (\cref{def:perturbation}) to be the (normalized) Pauli observables, sorted in increasing order of their weights, $w(V_1)\le w(V_2)\le \ldots \le w(V_{\dims^2-1})$. Applying \eqref{equ:avg-MI-partial} in \cref{thm:avg-MI-upper} and \cref{lem:avg-MI-lower},
\begin{equation}
    \frac{1}{100}\le \frac{1}{\ell}\sum_{i=1}^{\ell}\mutualinfo{\ptb_i}{\out^\ns}\le \frac{8 \ns c^2 \eps^2}{\ell^2} \sup_{\POVM\in\povmset}\sum_{i=1}^{\ell} \vadj{V_i}\Choi_{\POVM} \vvec{V_i} +16\exp\{-\alpha\dims\}\ns c^2\eps^2.
    \label{equ:pauli-lower-inequ}
\end{equation}


We need to choose an appropriate $\ell$ and to upper bound the average mutual information. We propose to select all Pauli observables with weight at least $N-w$. Then,
\begin{equation}
    \ell = g(w)\eqdef \sum_{m=0}^w{\nqubits \choose N-m}3^{N-m}.
    \label{equ:pauli-weight-number}
\end{equation}
This is because for Pauli observables with weight $N-m$, there are $N-m$ positions we can place the Pauli operators, and for each position, there are three choices $\pauliX, \pauliY, \pauliZ$.



According to \cref{prop:perturbation-trace-distance}, we must choose $\ell\ge \dims^2/2$ to ensure that the perturbations are $\eps$ far from $\qmm$ with high probability. In other words, $g(w)/\dims^2\ge 1/2$.
\[
\frac{g(w)}{\dims^2}=\sum_{m=0}^w{\nqubits \choose N-m}\frac{3^{N-m}}{4^\nqubits}=\sum_{m=0}^{w}{\nqubits \choose m}\Paren{\frac34}^{\nqubits-m}\Paren{\frac14}^m =\probaOf{\binomial{\nqubits}{1/4}\le w}.
\]
We have the following fact about the median of binomial distributions,
\begin{fact}[\cite{kaas1980mean}]
    The median of a binomial distribution $\binomial{N}{p}$ must lie in $[\lfloor Np\rfloor, \lceil Np\rceil]$.
\end{fact}
Thus, choosing $w=\lceil N/4\rceil$ guarantees that $g(w)/\dims^2\ge 1/2$. 

Next, we compute the inner product $\vadj{V_i} \Choi_{\POVM} \vvec{V_i}$. We first need to analyze the measurement information channel of Pauli measurements.

\begin{lemma}
\label{lem:pauli-mic-eigen}
    For $P=\sigma_1\otimes\cdots\otimes \sigma_{\nqubits}\in \Sigma^{\otimes \nqubits}$, let $\Luders_P$ be the measurement information channel of the Pauli measurement $\POVM_P$. Then for all Pauli observable $Q=\sigma_1'\otimes\cdots\otimes \sigma_{\nqubits}'\in(\Sigma\cup \eye_2)^{\otimes \nqubits}$, $Q$ is an eigenvector of $\Luders_P$ and
    \[
\Luders_P(Q)=Q\indic{\forall j\in[\nqubits], \sigma_j'\in\{\sigma_j,\eye_2\}}.
    \]
    In other words, the eigenvalue of ${Q}$ is 1 when the non-identity components of $Q$ match $P$, and 0 otherwise. 
\end{lemma}
\begin{proof}
 Let $\Luders_P$ be the measurement information channel of a Pauli measurement $\POVM_{P}$. From \cref{def:mic} and \cref{equ:pauli-measurement},
\begin{align*}
    \Luders_P(\cdot)&=\sum_{x\in\{-1,1\}^{\nqubits}}\frac{M_x^{P}}{\Tr[M_x^P]}\Tr[(\cdot)M_x^P]\\
    &=\sum_{x\in\{-1,1\}^{\nqubits}}M_x^{P}\Tr[(\cdot)M_x^P].
\end{align*}
The second step is because Pauli measurement is a basis measurement. Thus each $M_x^P=\qproj{u_x^P}$ where $\{\qbit{u_x^P}\}_{x\in\{-1,1\}^{\nqubits}}$ is an orthonormal basis, and $\Tr[M_x^P]=1$.

Let $Q=\sigma_1'\otimes\cdots\otimes\sigma_{\nqubits}'$. We want to argue that $Q$ is an eigenvector of $\Luders_P$.  
\begin{align*}
\Tr[M_x^PQ]&=\Tr\left[\bigotimes_{j=1}^{\nqubits}\frac{\eye_2+x_j\sigma_j}{2}\bigotimes_{j=1}^{\nqubits}\sigma_j'\right]\\
    &=\Tr\left[\bigotimes_{j=1}^{\nqubits}\frac{\sigma_j'+x_j\sigma_j\sigma_j'}{2}\right]\\
    &=\prod_{j=1}^{\nqubits}\frac{\Tr[\sigma_j']+x_j\Tr[\sigma_j\sigma_j']}{2}\\
    &=\prod_{j=1}^{N}(\indic{\sigma_j'=\eye_2}+x_j\indic{\sigma_{j}'=\sigma_j})
\end{align*}
The final step is due to \eqref{equ:pauli-property}.
If for some $j\in[N]$, $\sigma_j'\ne \eye_2$ and $\sigma_j'\ne\sigma_j$, then
\[
\Tr[M_x^PQ]=0\implies\Luders_P(Q)=0.
\]
In this case $Q$ is an eigenvector of $\Luders_P$ with eigenvalue of 0. If otherwise,
\begin{align*}
    \Luders_P(Q)&=\sum_{x\in\{-1,1\}^{\nqubits}}M_x^P\prod_{j=1}^{N}(\indic{\sigma_j'=\eye_2}+x_j\indic{\sigma_{j}'=\sigma_j})\\
    &=\sum_{x\in\{-1, 1\}^{\nqubits}}\bigotimes_{j=1}^{\nqubits}\frac{\eye_2+x_j\sigma_j}{2}(\indic{\sigma_j'=\eye_2}+x_j\indic{\sigma_{j}'=\sigma_j})\\
    &=\bigotimes_{j=1}^{\nqubits}\sum_{x_j\in\{-1,1\}}\frac{\eye_2+x_j\sigma_j}{2}(\indic{\sigma_j'=\eye_2}+x_j\indic{\sigma_{j}'=\sigma_j})\\
    &=\bigotimes_{j=1}^{\nqubits}\Paren{\indic{\sigma_j'=\eye_2}\sum_{x_j\in\{-1,1\}}\frac{\eye_2+x_j\sigma_j}{2}+\indic{\sigma_j'=\sigma_j}\sum_{x_j\in\{-1,1\}}\frac{x_j\eye_2+\sigma_j}{2}} \\
    &=\bigotimes_{j=1}^{\nqubits}(\eye_2\indic{\sigma_j'=\eye_2}+\sigma_j\indic{\sigma_{j}'=\sigma_j})\\
    &=\bigotimes_{j=1}^{\nqubits}\sigma_j'=Q.\qedhere
\end{align*}
\end{proof}

Let $P,Q$ be defined in \cref{lem:pauli-mic-eigen} and $\Choi_P$ be the matrix form of $\Luders_P$ given a Pauli measurement $\POVM_P$. An immediate corollary is that
\begin{equation}
    \frac{1}{\dims}\vadj{Q} {\Choi}_P \vvec{Q}=\indic{\forall j\in[\nqubits], \sigma_j'\in\{\sigma_j,\eye_2\}}.
    \label{equ:pauli-mic-sum}
\end{equation}

When $V_1, \ldots, V_{\dims^2-1}$ are the normalized Pauli observables sorted in increasing order of their weights, setting $\ell=g(w)$ (the number of Pauli observables with weight at least $N-w$), we have
\begin{align*}
    \sum_{i=1}^{\ell}\vadj{V_i} {\Choi}_P \vvec{V_i}=\sum_{m=0}^{w}{\nqubits\choose m}.
\end{align*}
This is because $\vadj{V_i} {\Choi}_P \vvec{V_i}=1$ 
only when $V_i$ has non-identity components that match the ones in $P$ and 0 otherwise. 
There are only ${\nqubits\choose N-m}={\nqubits\choose m}$ of them among all $V_i$'s with weight $\nqubits-m$.

The following result gives an upper bound on the sum of binomial coefficients,
\begin{lemma}[{\cite[Lemma 16.19]{downey2012parameterized}}]
\label{lem:sum-binomial-coef}
    Let $n\ge 1$ and $0\le q\le 1/2$, then
    \[
    \sum_{i=0}^{\lfloor nq\rfloor}{n\choose i}\le 2^{n h(q)},
    \]
    where $h(q)=-q\log q -(1-q)\log(1-q)$ is the binary entropy function.
\end{lemma}

Combining with \eqref{equ:pauli-lower-inequ}\eqref{equ:pauli-weight-number}\eqref{equ:pauli-mic-sum}, setting $w=\lceil N/4\rceil$,
\begin{align*}
    \frac{1}{100}&\le \frac{8 \ns c^2 \eps^2}{\ell^2} \sup_{P\in\Sigma^{\otimes\nqubits}}\sum_{i=1}^{\ell}\vadj{V_i} {\Choi}_P \vvec{V_i} +16\exp\{-\alpha\dims\}\ns c^2\eps^2\\
    &=8nc^2\eps^2\Paren{\frac{\sum_{m=0}^{w}{\nqubits\choose m}}{g(w)^2}+2\exp\{-\alpha\dims\}}\\
    &\le 8nc^2\eps^2\Paren{\frac{2\cdot 2^{\nqubits h(1/4)}}{\dims^4/4}+2\exp\{-\alpha\dims\}}\\
    &\le 16\ns \cd^2\eps^2\Paren{4\cdot 2^{(h(1/4)-4)\nqubits}+\exp\{-\alpha2^{\nqubits}\}}.
\end{align*}
When $N\ge 10$, the second term is negligible. Rearranging the terms, we must have
\[
\ns = \bigOmega{\frac{2^{(4-h(1/4))\nqubits}}{\eps^2}}.
\]
Finally, noting that $2^{4-h(1/4)}\ge 9.118$ completes the proof.

\section{Upper bound for Pauli measurements}
\label{sec:pauli-upper}
This section starts with an observation about Pauli measurements, which is common knowledge for quantum information experimentalists. Then, we employ this observation to improve previous results about quantum state tomography using Pauli measurements.

\subsection{ An Observation about Pauli Measurements}
When we measure an element of the Pauli group, for instance, $\sigma_X\otimes \sigma_Y$, on a two-qubit state $\rho$, the outcome is a sample from a $4$-dimensional probability distribution, says $(p_{00},p_{01},p_{10},p_{11})$, such that
\begin{align*}
\tr(\rho(\sigma_X\otimes \sigma_Y))=p_{00}-p_{01}-p_{10}+p_{11}.
\end{align*}

One can easily observe that
\begin{align*}
\Tr[\rho(\sigma_X\otimes \sigma_I)]=p_{00}+p_{01}-p_{10}-p_{11},\\
\Tr[\rho(\sigma_I\otimes \sigma_Y)]=p_{00}-p_{01}+p_{10}-p_{11},\\
\Tr[\rho(\sigma_I\otimes \sigma_I)]=p_{00}+p_{01}+p_{10}+p_{11}.
\end{align*}


In other words, measuring $XY$, we obtained a sample of $\sigma_X\sigma_I$, a sample of $\sigma_I\sigma_Y$, and a sample of $\sigma_I\sigma_I$. 

For a general $n$-qubit system, we have the following observation.
\begin{observation}
For any $P=P_1\otimes P_2\otimes\cdots\otimes P_n\in\{\sigma_X,\sigma_Y,\sigma_Z\}^{\otimes N}$, the measurement result of performing measurement $P_i$ on the $i$-th qubit is an $N$-bit string $s$. One can interpret the measurement result of performing $Q_i\in\{\sigma_I,\sigma_X,\sigma_Y,\sigma_Z\}$ on the $i$-th qubit if $Q_i=P_i$ or $Q_i=\sigma_I$. We call those $Q=Q_1\otimes Q_2\otimes\cdots\otimes Q_N$'s correspond to $P$.
\end{observation}



\subsection{Algorithm and error analysis} 
Our measurement scheme is as follows: For any $\eps>0$, fix an integer $m$.
\begin{enumerate}
    \item  For any $P\in\{\sigma_X,\sigma_Y,\sigma_Z\}^{\otimes N}$, one performs $m$ times $P$ on $\rho$, and records the $m$ samples of the $2^N$ dimensional outcome distribution.


According to the key observation, this measurement scheme provides $m\cdot 3^{N-w}$ samples of the expectation $\tr(\rho P)$, say, $\frac{\mu_P}{m\cdot 3^{N-w}}$, for each Pauli operator $P\in \{\sigma_I,\sigma_X,\sigma_Y,\sigma_Z\}^{\otimes N}$ with weight $w$, where $-m\cdot 3^{N-w}\leq\mu_P\leq m\cdot 3^{N-w}$.

\item Output 
\begin{align*}
\sigma=\sum_P \frac{\mu_P}{m\cdot 3^{N-w}\cdot 2^N} P.
\end{align*}
\end{enumerate}


Using this scheme, we obtained $m\cdot 3^N$ independent samples,
\begin{align*}
X_1,X_2,\cdots, X_{m\cdot 3^N}.
\end{align*}
Each  $X_i$ is an $N$-bit string recording outcomes on all qubits (using bit 0 to denote the +1 eigenvalue and bit 1 to denote the -1 eigenvalue of the measured Pauli operator).   
Given that each operator is measured $m$ times, specifically, we assign that $X_1,X_2,\cdots,X_{m}$ correspond to the measurement $\sigma_X^{\otimes N}$, $X_{m+1},X_{m+2}$, and $\cdots,X_{2m}$ corresponds to the measurement $\sigma_X^{\otimes N-1}\otimes \sigma_Y$, \dots, and until $\sigma_Z^{\otimes N}$.

We observe that for any $P$ of weight $w$,
$\mu_P=\sum_{j=0}^{m\cdot 3^{N-w}-1} Z_j$,
where $Z_j$ are independent samples from the distribution $Z$
\begin{align*}
\mathrm{Pr}(Z=1)=\frac{1+\tr (\rho P)}{2}, \;
\mathrm{Pr}(Z=-1)=\frac{1-\tr (\rho P)}{2}.
\end{align*}
We have
\begin{align*}
\expectDistrOf{}{Z}&=\tr (\rho P), \expectDistrOf{}{Z^2}=1, \\
\expectDistrOf{}{\mu_P}&=m\cdot 3^{N-w}\cdot\tr (\rho P), \\
\expectDistrOf{}{\mu_P^2}&=\expectDistrOf{}{\mu_P}^2+\Var[\mu_P] \\
&=\expectDistrOf{}{\mu_P}^2+m\cdot 3^{N-w}\Var[Z] \\
&=m^2\cdot 9^{N-w}\cdot \tr^2 (\rho P)+m\cdot 3^{N-w}(1-\tr^2 (\rho P)).
\end{align*}
% Furthermore, $Z_j$ can be obtained from samples
% \begin{align*}
% X_1,X_2,\cdots, X_{m\cdot 3^N}.
% \end{align*}
% Therefore,
% \begin{align*}
% \sigma=\sum_P \frac{\mu_P}{m\cdot 3^{N-w_P}\cdot 2^N} P.
% \end{align*}
% is defined according to the samples
% \begin{align*}
% X_1,X_2,\cdots, X_{m\cdot 3^N}.
% \end{align*}
Thus, we can verify that
\begin{align*}
\expectDistrOf{}{\sigma}=\rho,
\end{align*}
where the expectation is taken over the probabilistic distribution according to the measurements.

For convenience, we define the function $f:X_1\times X_2\times \cdots\times X_{m\cdot 3^N}\mapsto \mathbb{R}$
\begin{align*}
f(\sigma)=\hsnorm{\rho-\sigma}=\sqrt{{\rm Tr}[(\rho-\sigma)^\dagger (\rho-\sigma)]}.
\end{align*}
Note that we can write the unknown state $\rho$ as
\begin{align*}
\rho=\sum_P \frac{\alpha_P}{2^N} P.
\end{align*}
According to Cauchy-Schwarz and Jensen's inequality, we have
\begin{align*}
&\expectDistrOf{}{f(\sigma)} \leq\sqrt{\expectDistrOf{}{f(\sigma)^2}} = \sqrt{\expectDistrOf{}{\tr\rho^2-2\tr \rho\sigma+\tr\sigma^2}}\\
=& \sqrt{\expectDistrOf{}{\tr\sigma^2-\tr\rho^2}}
= \sqrt{\frac{1}{2^N}\sum_P \expectDistrOf{}{\frac{\mu_P^2}{m^2\cdot 9^{N-w_P}}-\alpha_P^2}} \\
=& \sqrt{\frac{1}{2^N}\sum_P \left(\frac{m^2\cdot 9^{N-w_P}\cdot \alpha_P^2+m\cdot 3^{N-w_P}(1-\alpha_P^2)}{m^2\cdot 9^{N-w_P}}-\alpha_P^2\right)} \\
=&\sqrt{\frac{1}{m\cdot 2^N}\cdot{\sum_P \frac{1-\alpha_P^2}{3^{N-w_P}}}} < \sqrt{\frac{1}{m\cdot 2^N}\cdot{\sum_P \frac{1}{3^{N-w_P}}}}\\
=& \sqrt{\frac{1}{m\cdot 2^N}\cdot{\sum_{w_P=0}^N \frac{1}{3^{N-w_P}}{{N}\choose{w_P}}3^{w_P}}} = \sqrt{\frac{1}{m\cdot 6^N}\cdot (1+9)^N} \\
=& \sqrt{\frac{5^{\nqubits}}{m\cdot 3^N}}.
\end{align*}

For any sample $X_i$ corresponding to $P\in\{\sigma_X,\sigma_Y,\sigma_Z\}^{\otimes N}$, if only $X_i$ is changed, $\mu_Q$ would be changed only for those $Q\in\{\sigma_I,\sigma_X,\sigma_Y,\sigma_Z\}^{\otimes N}$ where
$Q$ is obtained by replacing some $\{\sigma_X,\sigma_Y,\sigma_Z\}$'s of $P$ by $\sigma_I$.
Moreover, the resultant value of $\mu_Q$ would change by two at most.
According to the triangle inequality, $f$ would change at most 
\begin{align*}
\hsnorm{\sum_Q  \frac{\Delta\mu_Q}{m\cdot 3^{N-w_Q}\cdot 2^N} Q}&=\sqrt{\sum_Q \frac{\Delta\mu_Q^2}{m^2\cdot 9^{N-w_Q}\cdot 2^N}}\\
&\leq \sqrt{\sum_Q \frac{2^2}{m^2\cdot 9^{N-w_Q}\cdot 2^N}} \\
&= \sqrt{\sum_{w_Q=0}^N \frac{2^2}{m^2\cdot 9^{N-w_Q}\cdot 2^N}{{N}\choose{w_Q}}} = \frac{2\cdot \sqrt{5}^N}{m\cdot 3^N},
\end{align*}
where $Q$ ranges over all Paulis which correspond to $P$'s, and $\Delta\mu_Q$ denotes the difference of $\mu_Q$ when $X_i$ is changed.

We use McDiarmid's inequality to bound the probability of success.
\begin{lemma}\label{mc}
Consider independent random variables ${\displaystyle X_{1},X_{2},\dots X_{n}}$ on probability space $ {\displaystyle (\Omega ,{\mathcal {F}},{\text{P}})}$ where ${\displaystyle X_{i}\in {\mathcal {X}}_{i}}$ for all ${\displaystyle i}$ and a mapping ${\displaystyle f:{\mathcal {X}}_{1}\times {\mathcal {X}}_{2}\times \cdots \times {\mathcal {X}}_{n}\rightarrow \mathbb {R} }$. Assume there exist constant $ {\displaystyle c_{1},c_{2},\dots ,c_{n}} $ such that for all $ {\displaystyle i}$,
%\begin{widetext} 
\begin{align}{\displaystyle {\underset {x_{1},\cdots ,x_{i-1},x_{i},x_{i}',x_{i+1},\cdots ,x_{n}}{\sup }}|f(x_{1},\dots ,x_{i-1},x_{i},x_{i+1},\cdots ,x_{n})-f(x_{1},\dots ,x_{i-1},x_{i}',x_{i+1},\cdots ,x_{n})|\leq c_{i}.} 
\end{align}
%\end{widetext}
In other words, changing the value of the ${\displaystyle i}$-th coordinate ${\displaystyle x_{i}}$ changes the value of ${\displaystyle f}$ by at most ${\displaystyle c_{i}}$. Then, for any ${\displaystyle \epsilon >0}$,
%\begin{widetext} 
\begin{align} 
{\displaystyle {\mathrm{Pr}}(f(X_{1},X_{2},\cdots ,X_{n})-\expectDistrOf{}{f(X_{1},X_{2},\cdots ,X_{n})}\geq \epsilon )\leq \exp \left(-{\frac {2\epsilon ^{2}}{\sum _{i=1}^{n}c_{i}^{2}}}\right)} .
\end{align}
%\end{widetext}
\end{lemma}
We only consider $\delta<1/3$, then $\log(1/\delta)>1$.
For any $\eps'>0$, by choosing $m=(3+2\sqrt{2})\cdot\frac{5^N\log\frac{1}{\delta}}{3^N\cdot \eps'^2}$, we have $\expectDistrOf{}{f(\sigma)}< (\sqrt{2}-1){\eps'}$.
Therefore,
\begin{align*}
\mathrm{Pr}(f(\sigma) > \eps') &< \mathrm{Pr}(f(\sigma)-\expectDistrOf{}{f(\sigma)}>(2-\sqrt{2}){\eps'}) \\
&<\exp(-\frac{(12-8\sqrt{2})\cdot \eps'^2}{4\cdot \frac{5^N}{m^2\cdot 9^N}\cdot m\cdot 3^N}) \\
=& \exp(-\frac{m\cdot(3-2\sqrt{2})\cdot 3^N\cdot\eps'^2}{5^N}) <\delta, 
\end{align*}
where the inequality is by \cref{mc}.

For a general quantum state and $\eps>0$, we let $\eps'=\frac{\eps}{\sqrt{2^N}}$, and know that
$||\rho-\sigma||_1>\eps$ implies $\hsnorm{\rho-\sigma}>\eps'$. Therefore,
\begin{align*}
\mathrm{Pr}(||\rho-\sigma||_1>\eps) &\leq \mathrm{Pr}(\hsnorm{\rho-\sigma}>\eps')=\mathrm{Pr}(f>\eps').
\end{align*}

The total number of used copies is 
\begin{align*}
\ns = m\cdot 3^N=(3+2\sqrt{2})\cdot\frac{10^N\log\frac{1}{\delta}}{\eps^2}.
\end{align*}



\newcommand{\MUB}{\POVM_{MUB}}
\section{Upper bound for tomography with finite outcomes}
\label{sec:finite-upper}
We will show the tightness of the adaptive tomography bounds for k-outcome POVMs by modifying the Projected Least Squares Method (PLS)~\cite{guctua2020fast} to work with k-outcome POVMs. We present these adjustments for the case when $k = d$ and $k < d$. As a result, we will have the first upper and lower bounds for adaptive tomography for k-outcome measurements, where the upper bound is achieved with non-adaptive algorithms. The key component is reducing the $\MUB$ to a k outcome measurement,
$$\MUB \eqdef \left\{\frac{1}{d+1} \ket{\psi_x^k} \bra{\psi_x^k}\right\}_{k \in [d+1], x \in [d]},$$
where each fixed $k$ corresponds to each one of the Maximally mutually unbiased bases. The reduction will follow similarly to~\cite{liu2024restricted}.
\subsection{Algorithm for $k=d$} \label{sub-keqd}
Measuring with $\MUB$ acts as a uniform sampling $i \sim Unif([d+1])$ to select one of the MUB and measuring with the POVM described by $\left\{\ket{\psi_x^i} \bra{\psi_x^i}\right\}_{x \in [d]}$. So, we can split each of the MUB bases across the multiple copies and uniformly sample amongst them to replicate the outcome distribution of measuring with $\MUB$. 

\begin{algorithm}
\caption{Finite Outcome Tomography for $k = d$}\label{alg:tom-keqd}
\hspace*{0.1cm} \textbf{Input:} $n$ copies of state $\rho$ \\
\hspace*{0.1cm} \textbf{Output:} Estimate $\hat{\rho} \in \mathcal{C}^{d \times d}$
\begin{algorithmic}
\State Divide $\MUB$ into $d+1$ groups of d-outcome measurements $\mathcal{M}_j := \left\{\ket{\psi_x^j} \bra{\psi_x^j}\right\}_{x \in [d]}$.
\State Divide $n$ copies into $d+1$ equally sized groups, each group has $n_0 = n/(d+1)$ copies.
\For {$j = 1, ..., d+1$}
\State For group $j$, apply $\mathcal{M}_j$. Let the outcomes be $x_1^{(j)}, ..., x_{n_0}^{(j)}$.
\EndFor
\State Generate n/2 i.i.d samples from $Unif([d+1])$ and  let $m_j$ be the number of times $j$ appears.
\State Let $x = (x_1, ..., x_{d+1})$ where $x_j = (x_1^{(j)}, ..., x_{\min\{n_0, m_j\}}^{(j)}$)
\State From $x$, obtain empirical frequencies $F = (f_1, ..., f_{d (d+1)})$ by obtaining group specific frequencies of each $x_i$ and concatenating the frequency vectors together.
\State \Return $\hat{\rho} = PLS(F)$
\end{algorithmic}
\end{algorithm}

For the analysis, we will use the multiplicative Chernoff Bound for sums of i.i.d random variables.
\begin{lemma}[Multiplicative Chernoff Bound]
    \label{mult_chernoff}
   Let $X_1, ..., X_n$ be i.i.d with $\expectDistrOf{}{X_1} = \mu$. Then, 
   $$\Pr\left[\sum_{i}^n X_i \geq n(1+\alpha)\mu\right] \leq \exp{\left\{-\frac{n \alpha^2 \mu}{2 + \alpha}\right\}}\;, \alpha > 0$$ 
   $$\Pr\left[\sum_{i}^n X_i \geq n(1-\alpha)\mu\right] \leq \exp{\left\{-\frac{n \alpha^2 \mu}{2}\right\}}\;, \alpha \in (0,1)$$ 
\end{lemma}
\begin{theorem} \label{thm_keqd}
    For $k=d$, Algorithm \cref{alg:tom-keqd} will give estimate $\hat{\rho}$ such that $\Pr[\tracenorm{\hat{\rho} - \rho} \leq \eps] \geq \frac{2}{3}$ with $n = O\left( \frac{d^3 \log d}{\eps^2}\right)$
\end{theorem}
\begin{proof}
Notice that each sample made will follow the outcome distribution of applying $\MUB$ to a single copy of $\rho$. 
Given $n$ copies, it will be shown that $\frac{n}{2}$ such samples will be made with sufficiently high probability. This is when $m_j \leq n_j$ for all $j \in [d+1]$. 
Using~\cref{mult_chernoff}, on the $m_j \sim Bin(\frac{n}{2}, \frac{1}{d+1})$, which is sum of $Y_1,...,Y_{\frac{n}{2}} \sim Bern(\frac{1}{d+1})$, $\mu = \expectDistrOf{}{Y_1} = \frac{1}{d+1}$,
\begin{align*}
    \Pr\left[m_j > n_j\right] = \Pr\left[\sum_{i=1}^\frac{n}{2} Y_i > 2 n \mu \right] \leq \exp{\left\{-\frac{n}{6(d+1)}\right\}}.
\end{align*}
Furthermore, by union bound,
\begin{align*}
    \Pr\left[\exists_j m_j > n_j\right] \leq \sum_{j=1}^{d+1} \Pr\left[m_j > n_j\right] \leq (d+1) \exp{\left\{-\frac{n}{6(d+1)}\right\}}.
\end{align*}
From previous work \cite{guctua2020fast}, we have the following guarantee on the estimation error using the PLS method using the outcome of $\MUB$ measurements,
\begin{align*}
    \Pr\left[\tracenorm{\hat{\rho}_n - \rho} \geq \eps \right] \leq d \exp{\left\{-\frac{n \eps^2}{86 d^3}\right\}}.
\end{align*}
With the algorithm, we can bound the probability of the estimate not being optimal,
\begin{align*}
\Pr\left[\tracenorm{\hat{\rho} - \rho} \geq \eps \right] &\leq \Pr\left[\exists_j m_j > n_j \lor \tracenorm{\hat{\rho}_{n/2} - \rho} \geq \eps\right] \\
&\leq (d+1) \exp{\left\{-\frac{n}{6(d+1)}\right\}} + d \exp{\left\{-\frac{n \eps^2}{172 d^3}\right\}}.
\end{align*}
With $d \geq 16$ and $n = \frac{172d^3 \ln{200d}}{\eps^2}  = O\left(\frac{d^3 \log{d}}{e^2} \right)$, we will have $\Pr\left[\tracenorm{\hat{\rho} - \rho} \le \eps \right] \geq \frac{99}{100}$
\end{proof}
\subsection{Algorithm for $k < d$}
For $\ab<\dims$, it is helpful to think of the problem as follows: there are $\ns$ players, each of whom holds a copy of $\rho$, but can only send $\log\ab$ classical bits to a central server that collects the messages and learn about the state. 

The idea is then to simulate each $\dims$-outcome POVM using only $\log\ab$ bits for each player. 
Using results from \cite{ACT:19:IT2}, the number of players (or copies) required to simulate the original $\dims$-outcome POVM is roughly $O(\dims/\ab)$, and thus we have a $O(\dims/\ab)$ factor blow up in the sample complexity compared to $\dims$-outcome measurements.

\begin{definition} [$\eta$-Simulation]
\label{def-simulate}
We are given $n$ players each with i.i.d sample from an unknown distribution $\p \in \Delta_d$. Each player can only send $w$ bits to the server. The server can then perform a $\eta$-simulation where $\hat{X} = [d] \cup \{\perp\}$.
\begin{equation}
    \Pr[\hat{X} = x \mid \hat{X} \neq \perp] = p_x, \; \Pr[\hat{X} = \perp] \leq \eta
\end{equation}

It can be shown that there exists an algorithm that can perform a $\eta$ simulation with $O(\dims/\ab)$ players,

\end{definition}
\begin{theorem}[\cite{ACT:19:IT2}, Theorem IV.5]
\label{thm:simulation}
   For every $\eta \in (0,1)$, there exists an algorithm that $\eta$ simulates $p \in \Delta_d$ using 
   \begin{equation}
       M = 40 \left\lceil \log{\frac{1}{\eta}} \right\rceil \left\lceil \frac{d}{2^w - 1} \right\rceil
   \end{equation}
   players from the setting in \cref{def-simulate}. The algorithm only requires private randomness for each player.
\end{theorem}
Therefore, for each MUB measurement $\POVM$ we assign $M=O(\dims/\ab)$ players. 
Each player applies $\POVM$ to $\rho$ and compresses the outcome to $\log\ab$ bits using the simulation algorithm in \cref{thm:simulation}. 
This process is a valid $\ab$-outcome POVM. 
The server then can use the $M=O(\dims/\ab)$ messages to simulate the outcome of $\POVM$ applied to $\rho$.
From \cref{thm_keqd}, we need $\tildeO{\dims^3/\eps^2}$ simulated samples, and thus the total number of copies required to simulate those samples is $M=O(\dims/\ab)$ times more. The detailed proof is given in \cref{thm:k-outcome-tomography}.
% ect $O(\frac{d}{k})$ multiplicative factor on the copy complexity from \cref{sub-keqd}, as each simulation of a per copy measurement will require $O(\frac{d}{k})$ samples. 

% \begin{algorithm}
% \caption{Finite Outcome Tomography for k < d}\label{alg:tom-klessd}
% \hspace*{0.1cm} \textbf{Input:} $n$ copies of state $\rho$ \\
% \hspace*{0.1cm} \textbf{Output:} Estimate $\hat{\rho} \in \mathcal{C}^{d \times d}$
% \begin{algorithmic}
% \State Divide $\MUB$ into $d+1$ groups of d-outcome measurements $\mathcal{M}_j := \left\{\ket{\psi_x^j} \bra{\psi_x^j}\right\}_{x \in [d]}$.
% \State Divide $n$ copies into $d+1$ equally sized groups, each group has $n_0 = n/(d+1)$ copies.
% \For {$j = 1, ..., d+1$}
% \State For group $j$, apply $\mathcal{M}_j$. Let the outcomes be $y_1^{(j)}, ..., y_{n_0}^{(j)}$.
% \State Perform distributed simulation: Splitting up into $\frac{n_0}{M}$ groups, each group producing a single sample. 
% \State Let the non-aborted samples be $x_1^{(j)}, ..., x_{n_j}^{(j)}$
% \EndFor
% \State Generate n/2 i.i.d samples from $Unif([d+1])$ and  let $m_j$ be the number of times $j$ appears.
% \State Let $x = (x_1, ..., x_{d+1})$ where $x_j = (x_1^{(j)}, ..., x_{\min\{n_j, m_j\}}^{(j)}$)
% \State From $x$, obtain empirical frequencies $F = (f_1, ..., f_{d (d+1)})$ by computing group specific frequencies of each $x_i$ and concatenating the frequency vectors together.
% \State \Return $\hat{\rho} = PLS(F)$
% \end{algorithmic}
% \end{algorithm}

\begin{theorem}
\label{thm:k-outcome-tomography}
For $\ab < \dims$, Algorithm 1 with distributed simulation will give estimate $\hat{\rho}$ such that $Pr[\tracenorm{\rho-\hat{\rho}} \le \eps ]\ge 0.99$ with  $\ns = \bigO{\frac{\dims^4\log\dims}{\ab\eps^2}}$.
\end{theorem}
\begin{proof}
   The proof will follow the same steps as \cref{thm_keqd}, but also considering $n_j \sim Bin(1-\eta,n_0/M)$ taking the role of $n_0$. Since $n_j$ and $m_j$ are both Binomial random variables, it is enough to say that $m_j$ and $n_j$ are on the opposite sides of a mean threshold with exponentially decreasing probability. Denote $\hat{n}_0 = \frac{n_0}{M}$ and $\hat{n} = \frac{n}{M}$,
   \begin{align*}
       \Pr\left[m_j \leq n_j\right] &\geq \Pr\left[m_j \leq \frac{3}{4} \hat{n}_0 \land n_j \geq \frac{3}{4} \hat{n}_0\right] \\
       \Pr\left[m_j > n_j\right] &< \Pr\left[m_j > \frac{3}{4} \hat{n}_0 \lor n_j < \frac{3}{4} \hat{n}_0\right].
   \end{align*}
   We will now bound each of the above union events with exponentially decreasing probability. We will apply \cref{mult_chernoff} on $m_j \sim Bin(\hat{n}/2, \frac{1}{d+1}), \expectDistrOf{}{m_j} = \hat{n}_0/2$ with $\alpha = 1/2$,
   \begin{align*}
       \Pr\left[m_j > \frac{3}{4} \hat{n}_0\right] \leq \exp{\left\{- \frac{\hat{n}}{10(d+1)}\right\}}.
   \end{align*}
   Now, we will apply \cref{mult_chernoff} once more for $n_j \sim Bin(1-\eta,\hat{n}_0), \expectDistrOf{}{n_j} = (1-\eta) \hat{n}_0$ and $\alpha = \frac{1/4 + \eta}{\eta} \geq \frac{1}{4}$,
   \begin{align*}
          \Pr\left[n_j > \frac{3}{4} \hat{n}_0\right] \leq \exp{\left\{- \frac{\hat{n}}{32(d+1)}\right\}}.
   \end{align*}
   Thus, 
   \begin{align*}
       \Pr\left[m_j > n_j \right] &\leq \Pr\left[m_j > \frac{3}{4} \hat{n}_0 \right] + \Pr\left[n_j < \frac{3}{4} \hat{n}_0\right] \\
       &\leq 2 \exp{\left\{- \frac{\hat{n}}{32(d+1)}\right\}}.
   \end{align*}
   By union bound,
   \begin{align*}
       \Pr\left[\exists_j m_j > n_j \right] \leq (d+1) \exp{\left\{- \frac{\hat{n}}{32(d+1)}\right\}}.
   \end{align*}
   Now we will repeat the argument from \cref{sub-keqd} for plugging in the samples into the PLS estimator,
   \begin{align*}
       \Pr\left[\tracenorm{\hat{\rho} - \rho} > \eps \right] &\leq \Pr\left[\exists_j m_j > n_j \cup \tracenorm{\hat{\rho}_{\hat{n}/2} - \rho} > \eps\right] \\
       &\leq (d+1) \exp{\left\{-\frac{\hat{n}}{32(d+1)}\right\}} + d \exp{\left\{-\frac{\hat{n} \eps^2}{172 d^3}\right\}}.
   \end{align*}
With $d \geq 16$ and $\hat{n} = \frac{172d^3 \ln{200d}}{\eps^2}$, we will have $\Pr\left[\tracenorm{\hat{\rho} - \rho} \le \eps \right] \geq \frac{99}{100}$. With $w = \log k$ and $\eta = 0.01$, we have that $\hat{n} = \Theta(\frac{k}{d}) \cdot n$, so $n = O(\frac{d^4 \log{d}}{k \eps^2} )$.
\end{proof}
Thus, the upper bound can be compactly written as $O\left(\frac{\dims^4 \log{d}}{\eps^2\min\{\ab, \dims\}}\right)$, combining the $k=d$ and $k < d$ cases. With \cref{cor:finite-lower}, we have proven \cref{thm:nearly-tight-finite-out}.

\begin{remark}
    We note that running the distributed simulation with $\log\ab$ bits requires first obtaining the $\dims$ outcomes for each qubit. Thus, the algorithm is more relevant in the distributed setting as described in this section. Nevertheless, the compression step for each copy defines a valid $\ab$-outcome measurement and thus proves that our lower bound in~\cref{cor:finite-lower} is tight.
\end{remark}


\section*{Acknowledgements}

JA and YL were partially supported by NSF award 1846300 (CAREER), NSF CCF-1815893. AD was supported by a Cornell University Graduate Fellowship. YL was also supported by a Rice University Chairman postdoctoral fellowship. NY was supported by DARPA  SciFy Award 102828.

\bibliography{refs}
\bibliographystyle{alpha}



\end{document}
