\documentclass{article}
\def\anonymous{0}

\usepackage{graphicx} % Required for inserting images
\def\withcolors{1}
\def\withnotes{1}
\usepackage{ccanonne}
\usepackage{multicol}
\usepackage{multirow}
\usepackage{subcaption}
\usepackage{bbm}
\usepackage{braket}
\usepackage{algorithm}
\usepackage{algpseudocode}
\newcommand{\tr}{\operatorname{Tr}}

\newcommand{\requiresproof}{{\color{red}Requires proof}}
\DeclareMathOperator*{\argmax}{arg\,max}
\DeclareMathOperator*{\argmin}{arg\,min}


\newtheorem{fact}[theorem]{Fact}
\allowdisplaybreaks
\newcolumntype{C}{>{\centering\arraybackslash}p{0.27\textwidth}}
%
% Frequently used symbols
\newcommand{\perturb}{\gamma}
\newcommand{\dims}{d}
\newcommand{\zdims}{k}
\newcommand{\nsamps}{m}
\newcommand{\nb}{t}
\newcommand{\nspu}{m}
\newcommand{\nin}{r}
\newcommand{\nms}{{N_S}}
\newcommand{\nbb}{{t'}}
\newcommand{\nbbb}{{t''}}
\newcommand{\gbit}{s}
\newcommand{\unif}{{\mathbf{u}}}
\newcommand{\ngr}{{n_0}}
\newcommand{\DiffS}{{\Delta_S}}
\newcommand{\epsnew}{{\eps_0}}
\newcommand{\alphanew}{{\alpha_0}}
\newcommand{\rank}{{\text{rank}}}

\newcommand{\Proj}{{\Pi}}
\newcommand{\povmset}{{\mathfrak{M}}}
\newcommand{\nqubits}{{N}}
\newcommand{\pauliI}{{\sigma_I}}
\newcommand{\pauliX}{{\sigma_X}}
\newcommand{\pauliY}{{\sigma_Y}}
\newcommand{\pauliZ}{{\sigma_Z}}
\newcommand{\pauliObsSet}{{\mathcal{P}}}
\newcommand{\prPauli}[2]{{\p_{#2}(#1)}}

\newcommand{\opnorm}[1]{{\left\|#1\right\|}_{\text{op}}}
\newcommand{\tracenorm}[1]{{\left\|#1\right\|}_{1}}
\newcommand{\hsnorm}[1]{{\left\|#1\right\|}_{\text{HS}}}
\newcommand{\barDelta}{{\overline{\Delta}}}
\newcommand{\ptb}{{z}}
\newcommand{\ptbDistr}{{\mathcal{D}_{\ell,\cd}}}

\newcommand{\supparen}[1]{^{(#1)}}
\newcommand{\subparen}[1]{_{(#1)}}

% Constants
\newcommand{\cd}{{c}}
\newcommand{\cop}{\kappa}


\newcommand{\isthestate}{\texttt{YES}}
\newcommand{\notthestate}{\texttt{NO}}


\newcommand{\cA}{\mathcal{A}}

% Font
\newcommand{\tst}{t}  % Time: to be used when not for a sum index (e.g., 
%"at time $tst$")
\newcommand{\ts}{r} % Time step: index. To be used  for a sum index (e.g., 
%"$\sum_{\ts=1}^\ts")

\newcommand{\etaa}{\gamma}
%\newcommand{\etab}{\eta}
\newcommand{\signA}{{\bf{}1}^\star}
\newcommand{\signB}{{\bf{}0}^\star}
\newcommand{\setS}{S}
\newcommand{\vecu}{u}
\newcommand{\zero}{\mathbf{0}}

\newcommand{\chd}[1]{\Delta_{#1}^y}
\newcommand{\ratioparam}[1]{\kappa_{\scalebox{0.5}{\ensuremath{#1}}}}%
\newcommand{\indbig}[1]{\one\left\{#1\right\}}
\newcommand{\bfP}{\mathbf{P}}
\newcommand{\bfQ}{\mathbf{Q}}
\newcommand{\hbP}{\hat{\bfP}}
\newcommand{\trans}{\mathcal{T}}
\newcommand{\odiag}{E}
\newcommand{\x}{\mathbf{x}}
\newcommand{\out}{{x}}


\newcommand{\risk}{\mathcal{R}}
\def\multiset#1#2{\ensuremath{\left(\kern-.3em\left(\genfrac{}{}{0pt}{}{#1}{#2}\right)\kern-.3em\right)}}

\newcommand{\hp}{\widehat{\p}}
 
\newcommand{\ham}[2]{\operatorname{d}_{\rm Ham}(#1,#2)}
\newcommand{\variance}[2]{\var_{#1}{\mleft[#2\mright]}}

% quantum
\newcommand{\bm}{{\mathbf{m}}}
\newcommand{\qs}{\rho}
\newcommand{\qmm}{{\rho_{\text{mm}}}}
\newcommand{\qkn}{{\rho_0}}
\newcommand{\blambda}{{\boldsymbol{\lambda}}}
\newcommand{\SW}{\textbf{SW}}
\newcommand{\ngroups}{N}
\newcommand{\pnew}{\mathbf{p}}


\newcommand{\HH}{\mathbb{H}}
\newcommand{\Herm}[1]{{\HH_{#1}}}
\newcommand{\F}{\mathbb{F}}
\newcommand{\Pow}{\mathbb{P}}
\newcommand{\mD}{\mathcal{D}}
\newcommand{\OPT}{\text{OPT}}

\newcommand{\qbit}[1]{|{#1}\rangle}
\newcommand{\qadjoint}[1]{\langle{#1}|}
\newcommand{\qproj}[1]{\qbit{#1}\qadjoint{#1}}
\newcommand{\qoutprod}[2]{\qbit{#1}\qadjoint{#2}}

\newcommand{\qdotprod}[2]{\langle#1|#2\rangle}
\newcommand{\hdotprod}[2]{\left\langle#1,#2\right\rangle}
\newcommand{\matdotprod}[3]{\langle#1|#2|#3\rangle}
\newcommand{\supop}[2]{\mathcal{N}_{{#1}\rightarrow{#2} }}
\newcommand{\bA}{\mathbf{A}}
\newcommand{\bB}{\mathbf{B}}
\newcommand{\bD}{\mathbf{D}}
\newcommand{\bC}{\mathbf{C}}
\newcommand{\bG}{\mathbf{G}}
\newcommand{\bH}{\mathbf{H}}
\newcommand{\bM}{\mathbf{M}}
\newcommand{\bS}{\mathbf{S}}
\newcommand{\bT}{\mathbf{T}}
\newcommand{\bU}{\mathbf{U}}
\newcommand{\bV}{\mathbf{V}}
\newcommand{\bW}{\mathbf{W}}
\newcommand{\bX}{\mathbf{X}}
\newcommand{\bY}{\mathbf{Y}}
\newcommand{\EE}{\mathbb{E}}
\newcommand{\Var}{\text{Var}}
\newcommand{\eye}{\mathbb{I}}
\newcommand{\Real}{\text{Re}}
\newcommand{\Img}{\text{Im}}
\newcommand{\id}{\text{id}}
\newcommand{\img}{\text{i}}

\newcommand{\rk}{{r}}
\newcommand{\VecOp}{\text{vec}}
\newcommand{\vvec}[1]{|#1\rangle\rangle}
\newcommand{\vadj}[1]{\langle\langle#1|}
\newcommand{\vvdotprod}[2]{\left\langle\left\langle#1|#2\right\rangle\right\rangle}

\newcommand{\bv}{\mathbf{v}}
\newcommand{\bx}{\mathbf{x}}
\newcommand{\outset}{{\mathcal{X}}}

\newcommand{\spec}{\text{spec}}
\newcommand{\vsigma}{\vec{\sigma}}
\newcommand{\Luders}{\mathcal{H}}
\newcommand{\avgLuders}{{\overline{\Luders}}}
\newcommand{\Choi}{{\mathcal{C}}}
\newcommand{\avgChoi}{{\overline{\Choi}}}
\newcommand{\hbasis}{{\mathcal{V}}}
\newcommand{\qest}{{\hat{\rho}}}

\DeclareMathOperator{\diam}{diam}
\DeclareMathOperator{\diag}{diag}
\DeclareMathOperator{\iSWAP}{iSWAP}
\DeclareMathOperator{\Span}{span}

% Representation theory
\newcommand{\PiRank}{r}
\newcommand{\Wg}{\text{Wg}}
\newcommand{\U}{\mathbb{U}}
\newcommand{\bi}{\mathbf{i}}
\newcommand{\bj}{\mathbf{j}}
\newcommand{\Sim}{\mathcal{S}}
\newcommand{\Mob}{\text{Mob}}
\newcommand{\Cat}{\text{Cat}}
\newcommand{\Haar}[1]{{\mathcal{U}_{#1}}}
\newcommand{\POVM}{\mathcal{M}}
\newcommand{\permProd}[2]{{\left\langle#1\right\rangle_{#2}}}
\newcommand{\cycle}{\mathcal{C}}
\newcommand{\ObsPOVM}{\mathcal{N}}

\newcommand{\Xset}{\mathcal{I}_X}
\newcommand{\Yset}{\mathcal{I}_Y}
\newcommand{\zest}{\hat{\ptb}}
\title{Pauli measurements are not optimal for single-copy tomography}
\ifnum\anonymous=1
    \author{Anonymous authors}
\else
    \author{
    \begin{tabular}[t]{C@{\extracolsep{6.5em}} C}
   Jayadev Acharya &Abhilash Dharmavarapu \\
 Cornell University & Cornell University\\ 
\small \texttt{acharya@cornell.edu} &\small \texttt{ad2255@cornell.edu} 
\end{tabular}
\vspace{2ex}\\
\begin{tabular}[t]{C@{\extracolsep{6.5em}} C}
    Yuhan Liu & Nengkun Yu \\
Rice University & Stony Brook University\\ 
\small \texttt{yuhan-liu@rice.edu} &\small \texttt{nengkun.yu@cs.stonybrook.edu} 
\end{tabular}
}
\fi

\begin{document}
\maketitle
\begin{abstract}
Quantum state tomography is a fundamental problem in quantum computing.
Given $n$ copies of an unknown $N$-qubit state $\rho\in\mathbb{C}^{d\times d},d=2^N$, the goal is to learn the state up to an accuracy $\varepsilon$ in trace distance, say with at least constant probability 0.99. We are interested in the copy complexity, the minimum number of copies of $\rho$ needed to fulfill the task.

As current quantum devices are physically limited, Pauli measurements have attracted significant attention due to their ease
of implementation. However, a large gap exists in the 
literature for tomography with Pauli measurements.
The best-known upper bound is $O(\frac{N\cdot 12^N}{\varepsilon^2})$, 
and no non-trivial lower bound is known besides the general single-copy lower bound of
$\Omega(\frac{8^N}{\varepsilon^2})$, achieved by hard-to-implement structured POVMs such as MUB, SIC-POVM, and uniform POVM.

We have made significant progress on this long-standing problem. We first prove a stronger upper bound of $O(\frac{10^{N}}{\varepsilon^2})$. To complement it, we also obtain a lower bound of $\Omega(\frac{9.118^N}{\varepsilon^2})$, which holds even with adaptivity. To our knowledge, this demonstrates the first known separation between Pauli measurements and structured POVMs. 

The new lower bound is a consequence of a novel framework for adaptive quantum state tomography with measurement constraints. 
The main advantage is that we can use measurement-dependent hard instances to prove tight lower bounds for Pauli measurements, 
while prior lower-bound techniques for tomography only work with measurement-independent constructions. 
Moreover, we connect the copy complexity lower bound of tomography to the eigenvalues of the measurement information channel, which governs the measurement’s capacity to distinguish between states. To demonstrate the generality of the new framework, we obtain tight bounds for adaptive quantum state tomography with $k$-outcome measurements, where we recover existing results and establish new ones.

\end{abstract}
 % More precisely, for quantum tomography,
 %    \begin{itemize}
 %        \item We recover the tight bound of $\Omega(\dims^{3}/\eps^2)$ for general measurements, 
 %        \item For Pauli observables and constant-outcome measurements, we improve the existing lower bound of $\Omega(\dims^4/(\eps^2\log \dims))$ to $\Omega(\dims^4/\eps^2)$, which is now constant-optimal.
 %    \end{itemize}
 %    For quantum state certification,
 %    \begin{itemize}
 %        \item We recover the lower bound of $\Omega(\dims^{3/2}/\eps^2)$ for general measurements. 
 %        \item We prove the first known lower bound for adaptive Pauli observables of $\Omega(\dims^2/\eps^2)$, which is tight up to logarithmic factors. 
 %        \item For measurements with $\ab\ge 4$ outcomes, we prove a lower bound of $\Omega(\dims^2/(\eps^2\sqrt{\min\{\ab, \dims\}}))$, which matches non-adaptive upper bound and thus showing adaptivity does not help.
 %    \end{itemize}

\documentclass[../main.tex]{subfiles}
\graphicspath{{../images/}}
\makeatletter
\def\input@path{{../images/}}
\makeatother
\begin{document}
\section{Introduction}
\begin{figure}
\centering
\begin{tikzpicture}
\node[inner sep=0pt] (ws) at (0, 0) {
\includegraphics[height=.4\textwidth, trim={10cm 0 10cm 0},clip]{world_space.png}};
\node[inner sep=0pt] (cs) at (6,0) {\includegraphics[height=.4\textwidth, trim={10cm 1cm 10cm 4cm},clip]{conf_space.png}};
\end{tikzpicture}
\vspace{-5pt}
\label{fig:pbrm_intro}
\caption{\textbf{Left}: Shows world space obstacles as grey spheres. Robots start and goal configuration is colored red and green, respectively. Configurations along the computed path are colored transparent blue. \textbf{Right:} Mapped world space scenario to configuration space. Obstacle region is the grey mesh. Red spheres are collision-free regions computed by the neural SCDF. The optimized shortest path in the convex corridor is the blue curve.}
\vspace{-25pt}
\end{figure}
Motion planning is the problem of finding a collision-free trajectory that connects a given start and goal configuration. The planning takes place in the configuration space of the robot. For single body robots, like mobile robots or drones, the configuration space and the world space are usually the same. This simplifies the planning, since explicit obstacle representations are available which enables geometrical tools like separating hyperplanes, smallest distance to obstacles etc., to be used when designing motion planning algorithms. For multi-body robots like manipulators, the situation is completely different. The world space obstacles are usually mapped to non-convex regions, and to make the problem even harder, the mapping is usually not known. Forming explicit representations of the obstacle region in the configuration space is usually too expensive or intractable. Despite all of this, sampling based planners are used with great success, which mainly is due to their use of implicit representations of the obstacle region. The basic idea is to construct a graph in the configuration space that covers and connects the collision-free region. From this graph, a path can be extracted that connects a given start and goal configuration. The approach is computationally expensive, since the graph is constructed with the smallest geometrical building block available, points, which represents a collision-check. Furthermore, the extracted paths from the graph are non-smooth and jagged due to the stochastic nature of the approach. This adds an additional post-processing step to the process, where the paths are shortcutted and smoothened, before the path can be used for tracking. Clearly a lot of time is invested to form this graph and produce smooth paths. Thus, if the obstacles start to move, then all of this work is done in no use, since all points that make up this graph need to be re-verified, which is simply too time consuming to be done in real time.
\\\\
In this work, we want to address the existing drawbacks of the sampling based planners. Our main contribution is an improved motion planner where each vertex in the graph covers a collision-free region in the form of a sphere instead of a point and where the edges are formed with neighboring intersecting spheres. This representation has the advantage of instead of returning piecewise linear paths, returning a sequence of overlapping spheres, i.e. a convex corridor, that connects a given start and goal configuration, illustrated in Figure \ref{fig:pbrm_intro}. This convex corridor allows us to use convex optimization to produce smooth trajectories, instead of computationally expensive post-processing methods. The representation further allows us to estimate the coverage of the collision-free space, which gives us awareness and feedback in the offline roadmap construction phase. Finally, our representation is simple to adapt to moving obstacles, simply requery for the new radii and recheck for intersections. 
\\\\
The spherical collision-free regions are formed using a signed distance function (SDF), which is a function that returns the smallest distance from an arbitrary point to the boundary of an obstacle. As the name implies, the distance is signed, thus if the point is inside the obstacle it is negative otherwise positive. If the distance is positive, a sphere with radius equal to the distance is guaranteed to cover a collision-free region. Using an SDF in motion planning is not new, but what is novel about our approach is that we express the distance in the configuration space instead of the world space and by doing so allows us to form these convex collision-free regions. We refer to the resulting SDF as a signed configuration distance function (SCDF). Computing an SCDF analytically is non-trivial, our approach is therefore to parameterize the SCDF with a deep neural network and learn the mapping by supervised learning. Our resulting neural SCDF can compute distances for different parameter values of obstacle shapes and we also show how multiple distances can be combined, thus making our approach flexible.
\section{Related work}
Motion planning algorithms can roughly be divided into three families, grid-based, sampling based and optimization based methods. Grid-based methods (GBM) discretize the planning space from which a graph is then compiled. A standard search method is A$^\star$ \citep{a_star}, which is classified as an \textit{informed} search method, since it employs a heuristic function to speed up the search. A$^\star$ guarantees to return an optimal path at the level of discretization used. GBMs usually discretize the planning space by a regular lattice and this limits the GBMs to problems with low dimensionality due to the curse of dimensionality. Thus, GBMs are usually limited to single-body robots where the degrees of freedom (DOF) are low. To overcome the inherent scaling problem with the GBMs, stochastic methods are usually used for multi-body robots. These methods are termed as sampling-based methods (SBM) and core members within this family are the rapidly-exploring random trees (RRT) \citep{rrt} and the probabilistic roadmap (PRM) \citep{prm}. RRT grows a tree from the start configuration and explores the collision-free region in a rapid way until it is able to connect to the goal region. RRT is usually improved by bi-directional planning \citep{rrt_connect}, i.e. an additional tree is grown from the goal configuration and the trees are tested for connection after any tree has been expanded. RRT is a single-query method, thus it searches for a path from scratch each time it is queried. Contrary to this, PRM is a multi-query method, which solves for multiple queries without starting from scratch. PRM does this by creating a roadmap (graph) that covers the collision-free space as an offline step. The graph is then used to solve for multiple queries. PRMs are used in cases where the environment does not change since the extra offline step is too computationally costly and needs to be re-done if the environment is changed. In our work, we address this inherent issue by using a different roadmap representation. Our vertices in the graph cover a collision-free region in the form of spheres and we form the edges by checking for intersecting spheres. If something in the environment changes, we recompute the spheres radii and recheck the intersections, without relying on collision detection. We use a trained neural network to compute the sphere radius, therefore querying for the radius can be done fast, hence our representation enables the PRM for dynamic environments.
\\\\
In the recent decades, optimization based methods (OBM) \citep{chomp, schulman, itomp, stomp} have been introduced as an alternative to SBM for multi-body robots. Like the SBM, the OBMs scale well to higher dimensional problems and produce smoother motion. It is common to use a SDF in the optimization since it is a smooth function, thus enabling gradient-based methods. However, the standard way of expressing the SDF is in world space. The distance therefore needs to be mapped to the configuration space by the forward kinematics. This mapping makes the optimization problem a non-linear program (NLP), which is computationally expensive to solve. Recently, a different approach has been proposed. In \cite{mp_gcs} motion planning is formulated as a convex optimization problem by using the graph of convex sets framework \citep{gcs}. The underlying idea is to decompose the collision-free space into intersecting convex sets from which a convex optimization problem is formulated. In cases where an explicit representation of the obstacles in the configuration space exists, like for single-body robots, creating collision-free convex regions can be done fast \citep{iris}. For multi-body robots, this is non-trivial. Existing work does this successfully \citep{iris_nlp, iris_c} by an optimization based approach, but the methods are still too time consuming to be used in the presence of moving obstacles. Our approach is instead to use deep learning to learn an SDF expressed in the configuration space. With this, we can query for shortest distances to the collision boundary, which allows us to expand spherical regions which are collision-free. Our approach is fast and therefore enables our suggested roadmap planner to be used in dynamic environments.
\\\\
Recent research has focused on learning collision detection \citep{fk_kernel_distance, diffco, graphdistnet} by predicting the signed distance between the robot links and the surrounding obstacles in the world space. The learned SDF is used in trajectory optimization but since the distance is expressed in the world space, the problem becomes an NLP and therefore takes a long time to solve. We take a novel approach and suggest to instead express the signed distance in the configuration space. This allows us to improve the PRM at the same time as it enables convex optimization for trajectory optimization, which runs faster and is more reliable than NLP solvers. In \cite{cspf} a learned signed distance function in the configuration space is proposed similar to our approach. However, their approach is restricted to point cloud representations, while we propose to represent the obstacles as parameterized geometric shapes, e.g. spheres. Furthermore, we also show how to use our learned SCDF to improve an existing roadmap planner.
\section{Problem formulation}
A robot is located in the world space, $\W \subset \R^3 $. The unique location of the robot is given by its configuration $\q \in \C$, where $\C$ is the configuration space. The set of points covered by the robots bodies at a certain configuration is expressed as $\B(\q) \subset \W$. The robot is surrounded by $\NrObst$ obstacles $\O = \bigcup_{i=1}^{\NrObst} \O_i$, where  $\O_i \subset \W$. The representation of the obstacle in the configuration space is the set $\C\O_i = \{\q \in \C \: |\: \B(\q) \cap \O_i \neq \emptyset \}$. The obstacle space is formed as $\Co = \bigcup_{i=1}^{\NrObst} \C \O_i$. The complement is referred to as the free space, $\Cf = \C \setminus \Co$. The path planning problem is a tuple, ($\Cf$, $\qStart$, $\qGoal$), where we want to connect a query pair, consisting of a start, $\qStart$, and goal configuration, $\qGoal$, with a geometric path, $\q(s): [0, 1] \mapsto \Cf$, such that $\q(0)=\qStart$ and $\q(1)=\qGoal$, or report correctly when such a path does not exist.
\end{document}


\section{Preliminaries}\label{sec:preliminaries}



%We denote by $(\Ac(x_\Ac),\Bc(x_\Bc))(z)$ a random execution of $\pi$ with private inputs $(x_\Ac,y_\Ac)$, and common input $z$.

%\Jnote{Move to DP}
% At the end of such an execution, the protocol outputs a public transcript denoted by the random variable $\trans_\pi(x_\Ac,x_\Ac,z)$ we denotes the common as $\out(\trans_\pi(x_\Ac,x_\Ac,z)$, and each party $\Pc \in \set{\Ac,\Bc}$ obtains his view denoted $\view^\Pc_\pi(x_\Ac,x_\Bc,z)$, which may also contain a ``local output'' \Jnote{Local} $\out^\Pc(x_\Ac,x_\Bc,z)$ (if the protocol specifies such an output). \Jnote{Common output, and parties output}


\subsection{Distributions and Random Variables}\label{sec:prelim:dist}
The support of a distribution $P$ over a finite set $\cS$ is defined by $\Supp(P) \eqdef \set{x\in \cS: P(x)>0}$. For a distribution or a random variable $D$, let $d\from D$ denote that $d$ was sampled according to $D$. Similarly,  for a set $\cS$, let $x \from \cS$ denote that $x$ is drawn uniformly from $\cS$, and denote by $\cU_{\cS}$ the uniform distribution over $\cS$. For a finite set $\cX$ and a distribution $C_X$ over $\cX$, we use the capital letter $X$ to denote the random variable that takes values in $\cX$ and is sampled according to $C_X$. The {\sf statistical distance} (\aka {\sf~variation distance}) of two distributions $P$ and $Q$ over a discrete domain $\cX$ is defined by $\sdist{P}{Q} \eqdef \max_{\cS\subseteq \cX} \size{P(\cS)-Q(\cS)} = \frac{1}{2} \sum_{x \in \cS}\size{P(x)-Q(x)}$. 
For a vector $x = (x_1,\ldots,x_n)$ and index $i\in [n]$, we let $x_{-i} = (x_1,\ldots,x_{i-1},x_{i+1},\ldots,x_n)$ and $x^{(i)} = (x_1,\ldots,x_{i-1}, -x_i, x_{i+1},\ldots,x_n)$, for a set $\cS \subseteq [n]$ we let $x_{\cS} = (x_i)_{i \in \cS}$ and $x_{-\cS} = (x_i)_{i \in [n]\setminus \cS}$, and for a vector $r \in \zo^n$ we let $x_r = (x_i)_{\set{i \colon r_i = 1}}$ and $x_{-r} = (x_i)_{\set{i \colon r_i = 0}}$.

%For $n \in \N$ we let $U_n$ be the uniform distribution over $\oo^n$, and let $S_n$ be the distribution induces by the sum of $n$ i.i.d.\ random variables, each is distributed according to $U_1$. Let $\cN(0,1)$ be the standard normal distribution.
%For a distribution $\cD$ and a function $f$, we define by $f(\cD)$ the distribution that is induced by the output of $f(x)$ for $x \from \cD$. 





% \begin{theorem}[\cite{McGregorMPRTV10}]\label{thm:sv-extracotr}
% 	\Enote{Remove if not needed}
% 	There is a constant $c$ to make the following holds. Let $X$ be an $\alpha$-SV source on $\{0,1\}^n$, let $Y$ be a source on $\{0,1\}^n$ with min-entropy at least $\beta n$ (independent from $X$), and let $Z=\ip{X,Y}\mbox{mod m}$ for some $m\in\mathbb{N}$. Then for every $\delta\in[0,1]$, the random variable $(Y,Z)$ is $\delta$-close to $(Y,U)$ where $U$ is uniform on $\mathbb{Z}_m$ and independent of $Y$, provided that
% 	$$
% 	n\geq c\cdot\frac{m^2}{\alpha\beta}\cdot\log(\frac{m}{\beta})\cdot\log(\frac{m}{\delta}).
% 	$$
% \end{theorem}



\Enote{I removed the definition of DP since it already appears in the intro}
\remove{
\subsection{Differential Privacy}\label{sec:prelim:DP}
We use the following standard definition of (information theoretic) differential privacy, due to \citet{DMNS06}. For notational convenience, we focus on databases over $\oo$.
\begin{definition}[Differentially private mechanisms]\label{def:mech}
	A randomized function $f\colon\oo^n\mapsto \zs$ is an {\sf $n$-size, $(\eps,\delta)$-differentially private mechanism} (denoted $(\eps,\delta)$-\DP) if for every neighboring $w,w'\in \oo^n$ and every function $g\colon \zs\mapsto \zo$, it holds that 
	$$
	\pr{g(f(w))=1}\leq e^{\eps}\cdot \pr{g(f(w'))=1} +\delta.
	$$ 	
	If $\delta=0$, we omit it from the notation.
\end{definition}
}


\subsubsection{Computational Differential Privacy}
There are several ways for defining computational differential privacy (see \cref{sec:related-works}). We use the most relaxed version due to \cite{BNO08}. For notational convenience, we focus on databases over $\oo$.
\begin{definition}[Computational differentially private mechanisms]\label{def:ComMech}
	A randomized function ensemble $f=\set{f_\pk\colon\oo^{n(\pk)}\mapsto \zs}$ is an {\sf $n$-size, $(\eps,\delta)$-computationally differentially private} (denoted $(\eps,\delta)$-$\CDP$) if for every poly-size circuit family $\set{\Ac_\pk}_{\pk\in \N}$, the following holds for every large enough $\pk$ and every neighboring $w,w'\in\oo^{n(\pk)}$:
	$$
	\pr{\Ac_\pk(f_\pk(w))=1}\leq e^{\eps(\pk)}\cdot \pr{\Ac_\pk(f_\pk(w'))=1} +\delta(\pk).
	$$ 
	If $\delta(\pk) = \negl(\pk)$, we omit it from the notation. 
\end{definition}



\subsubsection{Two-Party Differential Privacy}\label{sec:DP}
In this section we formally define distributed differential privacy mechanism (\ie protocols). %For the ease of notation, we consider protocol with no common input.

\begin{definition}\label{def:DP}%\Nnote{fix security parameter}
	A two-party protocol $\Pi=(\Ac,\Bc)$ is {\sf $(\eps,\delta)$-differentially private}, denoted $(\eps,\delta)$-$\DP$, if the following holds for every algorithm $\Dc$: let $\V^\Pc(x,y)(\pk)$ be the view of party $\Pc$ in a random execution of $\Pi(x,y)(1^\pk)$. Then for every $\pk,n \in \N$, $x\in \oo^n$ and neighboring $y,y'\in\oo^n$:
	\begin{align*}
	\pr{\Dc(V^\Ac(x,y)(\pk))=1}\le e^{\eps(\pk)}\cdot \pr{\Dc(V^\Ac (x,y')(\pk))=1}+\delta(\pk),
	\end{align*} 
	and for every $y\in \oo^n$ and neighboring $x,x'\in\oo^{n}$:
	\begin{align*}
	\pr{\Dc(V^\Bc(x,y)(\pk))=1}\le e^{\eps(\pk)}\cdot \pr{\Dc(V^\Bc (x',y)(\pk))=1}+\delta(\pk).
	\end{align*} 	
	Protocol $\Pi$ is {\sf $(\eps,\delta)$-computational differentially private}, denoted $(\eps,\delta)$-$\CDP$, if the above inequalities only hold for a non-uniform \ppt $\Dc$ and large enough $\pk$. We omit $\delta = \negl(\pk)$ from the notation. \footnote{Note that define we give for two-party differentially private protocols is a semi-honest definition, in which we ask for the security to hold when the parties interact in an honest execution of the protocol. Since we are proving a lower bound, starting from this weaker guarantee (as opposed to security against malicious players), yields a stronger result.}
\end{definition}
%We omit $\delta$ from the notation if $\delta$ is a negligible function of $n$.

%\Enote{simulation-based}
\begin{remark}[The definition for computational differential privacy we use]\label{rem:comDPChannel} 
	An alternative, stronger definition of computational differential privacy, known as simulation-based computational differential privacy, requires that the distribution of each party’s view be computationally indistinguishable from a distribution that ensures privacy in an information-theoretic sense. \cref{def:DP} is a weaker notion in comparison. Consequently, establishing a lower bound for a protocol that satisfies this weaker guarantee (as we do in this work) yields a stronger result.%Actually, our lower bound only requires the privacy to hold against \emph{uniform} external observer.
	%\Nnote{Maybe add: When only interesting in \Dp against external observer, the two definitions can be achieve using key-agreement and (single-party) \Dp mechanism. }
\end{remark}




\subsection{Useful Claims}
\remove{
In this section, we state generic lemmas and propositions that we will use later in our proofs.

The following lemma which we prove in \cref{sec:missing-proofs:distance-I}, measures the distance between two uniform stings conditioned one a random index $i$ either being fixed to $0$ or to $1$.

\def\distanceILemma{
    Let $R \la \zo^n$. For any (randomized) function $f:\{0,1\}^n\rightarrow \{0,1\}$ and $\alpha > 0$, it holds that
    \begin{align}\label{eq:f-alpha}
        \ppr{i \la [n]}{\size{\:\ex{f(R) \mid R_i = 0}-\ex{f(R) \mid R_i = 1}\:}\geq \alpha} \leq \frac{2}{n \alpha^2},
    \end{align}
    where the expectations are taken over $R$ and the randomness of $f$.
}

\begin{lemma}\label{lem:distance-I}
    \distanceILemma
\end{lemma}
}

The following two propositions state that given the output of a differentially private function, it is not possible to predict well even a random index (even if all other indexes are leaked). The first proposition handles the information-theoretic case and the second handles the computation case. Both propositions are proven in \cref{sec:missing-proofs:hard-to-guess}. 

\def\propHardToGuessInf{
    Let $f\colon \oo^n \rightarrow \cY$ be an $(\eps,\delta)$-\DP function, let $g \colon [n] \times \oo^{n-1} \times \cY \rightarrow \set{-1,1,\bot}$ be a (randomized) function, and let $X = (X_1,\ldots,X_n) \la \oo^n$. Then the following holds for every $i \in [n]$ where $X_i^* = g(i,X_{-i},f(X_1,\ldots,X_n))$:
    \begin{align*}
        \pr{X_i^* = X_i} \leq e^{\eps}\cdot \pr{X_i^* = -X_i} + \delta.
    \end{align*}
}

\begin{proposition}\label{prop:hard-to-guess-inf}
    \propHardToGuessInf
\end{proposition}


\def\propHardToGuessComp{
    Let $f = \set{f_{\pk} \colon \oo^{n(\pk)} \rightarrow \zo^{m(\pk)}}_{\pk \in \bbN}$ be an $(\eps,\delta)$-\CDP function ensemble, and let $\set{g_{\pk}}_{\pk \in \bbN}$ be a poly-size circuit family. Then, for large enough $\pk$ and $X = (X_1,\ldots,X_{n(\pk)}) \la \oo^{n(\pk)}$, the following holds for every $i \in [n(\pk)]$ where $X_i^* = g_{\pk}(i,X_{-i},f_{\pk}(X_1,\ldots,X_n))$:
    \begin{align*}
        \pr{X_i^* = X_i} \leq e^{\eps(\pk)}\cdot \pr{X_i^* = -X_i} + \delta(\pk).
    \end{align*}
}

\begin{proposition}\label{prop:hard-to-guess-comp}
    \propHardToGuessComp
\end{proposition}





\remove{
\Enote{Chao's old statement:}
\begin{lemma}\label{lem:distance-I-old}
        Let $R \la \zo^n$. 
	For any function $f:\{0,1\}^n\rightarrow \{0,1\}$ and $\alpha<0.01$, it holds that
	$$
	\Pr_{i\la[n]}\left[\: \size{\:\mathbb{E}[f(R) \mid R_i = 0]-\mathbb{E}[f(R) \mid R_i = 1]\:}\geq \alpha\right]\leq \frac{2+2\log(\frac{1}{\alpha})}{n\alpha^2}.
	$$
\end{lemma}
\begin{proof}
	Define $S_1=\{r \in \zo^n \colon f(r)=1\}$. Then for any $i\in[n]$, we have
	$$
	\begin{array}{rl}
		\size{\mathbb{E}[f(R) \mid R_i = 0]-\mathbb{E}[f(R) \mid R_i = 1]}
		&=\size{\Pr[R\in S_1|R_i=0]-\Pr[R\in S_1|R_i=1]}\\
		&=\size{\frac{\Pr[R_i=0|R\in S_1]\cdot\Pr[R\in S_1]}{\Pr[R_i=0]}-\frac{\Pr[R_i=1|R\in S_1]\cdot\Pr[R\in S_1]}{\Pr[R_i=1]}}\\
		&=\frac{2\size{S_1}}{2^n}\size{\Pr[R_i=0|R\in S_1]-\Pr[R_i=1|R\in S_1]}
	\end{array}
	$$
	When $|S_1|\leq \alpha\cdot 2^{n-1}$, we have $\size{\mathbb{E}[f(R) \mid R_i = 0]-\mathbb{E}[f(R) \mid R_i = 1]}\leq\frac{2\size{S_1}}{2^n}\leq \alpha$ for any $i\in[n]$. Hence, in the following, we assume $|S_1|> \alpha\cdot 2^{n-1}$.

	%Define $I_{bad}=\{i|\size{\Pr[R_i=0|R\in S_1]-\Pr[R_i=1|R\in S_1]}>2\alpha\}$ and $k=\size{I_{bad}}$, then for any $i\notin I_{bad}$, we have 
    %$$
    %\begin{array}{rl}
    %    2\alpha&\geq \size{\Pr[R_i=0|R\in S_1]-\Pr[R_i=1|R\in S_1]}\\
    %    &=\size{\frac{\Pr[R\in S_1|R_i=0]\cdot\Pr[R_i=0]}{\Pr[R\in S_1]}-\frac{\Pr[R\in S_1|R_i=1]\cdot\Pr[R_i=1]}{\Pr[R\in S_1]}}\\
    %    &=\size{\Pr[R\in S_1|R_i=0]-\Pr[R\in S_1|R_i=1]}\cdot\frac{1}{2\Pr[R\in S_1]}\\
    %    &\geq \size{\mathbb{E}[f(R) \mid R_i = 0]-\mathbb{E}[f(R) \mid R_i = 1]}\cdot \frac{1}{2},
    %\end{array}
    %$$ 
    %where the last inequality is because $\Pr[R\in S_1]\leq 1$. So that $\size{\mathbb{E}}[f(R) \mid R_i = 0]-\mathbb{E}[f(R) \mid R_i = 1]\leq %4\alpha$.
    Define $I_{bad}=\{i \colon \size{\Pr[R_i=0|R\in S_1]-\Pr[R_i=1|R\in S_1]} \geq 2\alpha\}$ and $k=\size{I_{bad}}$, and denote $I_{bad}=\{i_1,\dots,i_k\}$. Define $(X_{i_1}, \ldots X_{i_k}) = (R_{i_1},\dots,R_{i_k})\mid_{R \in S_1}$. 
    Consider the min-entropy
	$$
	\begin{array}{rl}
		H_{min}(X_{i_1},\dots,X_{i_k})&\leq H(X_{i_1},\dots,X_{i_k})\\
		&\leq \sum_{j=1}^k H(X_{i_j})\\
		&\leq k\cdot \left(-(\frac{1}{2}+2\alpha)\cdot\log(\frac{1}{2}+2\alpha)-(\frac{1}{2}-2\alpha)\cdot\log(\frac{1}{2}-2\alpha)\right)\\
            &=k\cdot \left(-(\frac{1}{2}+2\alpha)\cdot(\log(1+4\alpha)-1)-(\frac{1}{2}-2\alpha)\cdot(\log(1-4\alpha)-1)\right)\\
            &=k\cdot \left(1-(\frac{1}{2}+2\alpha)\cdot\log(1+4\alpha)-(\frac{1}{2}-2\alpha)\cdot\log(1-4\alpha)\right),
		
	\end{array}
	$$
	where $H_{min}(Y)$ is the minimum entropy of $Y$ and $H(Y)$ is the Shannon entropy of $Y$.\Enote{add to preliminaries.}
        The third inequality holds since by the definition of $I_{bad}$, for every $j \in [k]$ it holds that $\size{\pr{X_{i_j} = 1}-\pr{X_{i_j} = 0}} > 2\alpha$, and therefore $H(X_{i_j}) \leq H(1/2 + 2\alpha)$\Enote{define}.
	
	Therefore, there exists $b_1,\dots,b_k\in\{0,1\}$, such that 
	
	\begin{align}\label{eq:min-entropy-result}
		\Pr\left[(R_{i_1},\ldots,R_{i_k}) = (b_1,\ldots,b_k) \mid R\in S_1\right]
		&= \pr{(X_{i_1},\ldots,X_{i_k}) = (b_1,\ldots,b_k)}\\
		&= 2^{-H_{min}(X_{i_1},\dots,X_{i_k})}\nonumber\\
		&\geq 2^{k\cdot \left(-1+(\frac{1}{2}+2\alpha)\cdot\log(1+4\alpha)+(\frac{1}{2}-2\alpha)\cdot\log(1-4\alpha)\right)}.\nonumber
	\end{align}
	
	Let $S_{bad}=\{r \in \zo^n  \colon \set{(r_{i_1},\ldots,r_{i_k}) = (b_1,\ldots,b_k)} \land \set{r\in S_1}\}$.
	It holds that
	\begin{align*}
		|S_{bad}|
		&= \size{S_1} \cdot \Pr\left[(R_{i_1},\ldots,R_{i_k}) = (b_1,\ldots,b_k) \mid R\in S_1\right]\\
		&\geq \alpha\cdot 2^{n-1}\cdot2^{k\cdot \left(-1+(\frac{1}{2}+2\alpha)\cdot\log(1+4\alpha)+(\frac{1}{2}-2\alpha)\cdot\log(1-4\alpha)\right)},
	\end{align*} 
	where the inequality holds by \cref{eq:min-entropy-result} and since $\size{S_1} \geq \alpha\cdot 2^{n-1}$.
	Notice that any string in $S_{bad}$ depends on at most $n-k$ bits. It implies that $|S_{bad}|\leq 2^{n-k}$. Therefore, we have
	$$
	\begin{array}{rl}
		&2^{n-k}\geq \alpha\cdot 2^{n-1}\cdot2^{k\cdot \left(-1+(\frac{1}{2}+2\alpha)\cdot\log(1+4\alpha)+(\frac{1}{2}-2\alpha)\cdot\log(1-4\alpha)\right)} \\
		\Rightarrow& n-k \geq \log \alpha+n-1+k\cdot \left(-1+(\frac{1}{2}+2\alpha)\cdot\log(1+4\alpha)+(\frac{1}{2}-2\alpha)\cdot\log(1-4\alpha)\right)\\
		\Rightarrow& 1-\log \alpha \geq k\cdot((\frac{1}{2}+2\alpha)\cdot\log(1+4\alpha)+(\frac{1}{2}-2\alpha)\cdot\log(1-4\alpha))\\
		\Rightarrow& 1-\log \alpha \geq k\cdot(4\alpha\cdot\log(1+4\alpha)+(\frac{1}{2}-2\alpha)\cdot\log(1-16\alpha^2))\\
        \Rightarrow& 1-\log\alpha \geq k\cdot(15.9\alpha^2-8\alpha^2+32\alpha^3)=k\cdot(7.9\alpha^2+32\alpha^3)>0.5k\alpha^2\\
		\Rightarrow& k\leq \frac{2-2\log \alpha}{\alpha^2} = \frac{2+2\log (1/\alpha)}{\alpha^2},
	\end{array}
	$$
	Where the third transition holds since 
	\begin{align*}
		\lefteqn{(\frac{1}{2}+2\alpha)\cdot\log(1+4\alpha)+(\frac{1}{2}-2\alpha)\cdot\log(1-4\alpha)}\\
		&= 4\alpha\cdot\log(1+4\alpha) + (\frac{1}{2}-2\alpha)\paren{\log(1+4\alpha)+\log(1-4\alpha)}\\
		&= 4\alpha\cdot\log(1+4\alpha)+(\frac{1}{2}-2\alpha)\cdot\log(1-16\alpha^2),
	\end{align*}
	and the forth transition holds since $4\alpha\cdot\log(1+4\alpha)+(\frac{1}{2}-2\alpha)\cdot\log(1-16\alpha^2) > 15.9\alpha^2-8\alpha^2+32\alpha^3$ for $\alpha < 0.01$.
	Thus, we conclude that 
	$$
	\Pr_{i\la[n]}\left[\size{\mathbb{E}[f(R) \mid R_i=0]-\mathbb{E}[f(R) \mid R_i = 1]}\geq \alpha\right]\leq \frac{k}{n}\leq \frac{2+2\log (1/\alpha)}{n\alpha^2}.
	$$
\end{proof}
}


\subsection{Channels and Two-Party Protocols}\label{sec:protocol}

\paragraph{Channels.}A channel is simply a distribution of a pair of tuples defined as follows. 
\begin{definition}[Channels]\label{def:channel} A {\sf channel} $C_{(X,U)(Y,V)}$ of size $\isize$ over alphabet $\Sigma$ is a probability distribution over $(\Sigma^\isize \times\zo^\ast) \times(\Sigma^\isize \times\zo^\ast)$. The ensemble $C_{(X,U)(Y,V)}= \set{C_{(X_\pk,U_\pk)(Y_\pk,V_\pk)}}_{\pk\in \N}$ is an $\isize$-size channel ensemble, if for every $\pk\in \N$, $C_{(X_\pk,U_\pk)(Y_\pk,V_\pk)}$ is an $\isize(\pk)$-size channel. %We denote a channel of size one by a \emph{single-bit} channel. 
We refer to $X$ and $Y$ as the {\sf local outputs}, and to $U$ and $V$ as the {\sf views}.	
\end{definition}

We view a  channel as the experiment in which there are two parties $\Ac$ and $\Bc$.  Party $\Ac$ receives ``output'' $X$ and ``view'' $U$, and party $\Bc$ receives ``output'' $Y$ and ``view'' $V$. Unless stated otherwise, the channels we consider are over the alphabet $\Sigma = \oo$. We naturally identify channels with the distribution that characterizes their output.








\subsubsection{Two-Party Protocols}

A two-party protocol $\Pi=(\Ac,\Bc)$ is \ppt if the running time of both parties is polynomial in their input length. We let $\Pi(x,y)(z)$ or $(\Ac(x),\Bc(y))(z)$ denote a random execution of $\Pi$ on a common input $z$, and private inputs $x,y$.%We assume \wlg that a protocol has a common output (part of its transcript).\Jnote{This is not really the case we consider in this paper..}

\begin{definition}[Oracle-aided protocols]\label{def:ChannelAidedProtocol}
	In a two-party protocol $\Pi$ with oracle access to a {\sf protocol} $\Psi$, denoted $\Pi^\Psi$, the parties make use of the \textit{next-message function} of $\Psi$.\footnote{The function that on a partial view of one of the parties, returns its next message.} In a two-party protocol $\Pi$ with oracle access to a {\sf channel} $C_{Z W}$, denoted $\Pi^C$, the parties can jointly invoke $C$ for several times. In each call, an independent pair $(z,w)$ is sampled according to $C_{Z W}$, one party gets $z$, the other gets $w$.
\end{definition}


\begin{definition}[The channel of a protocol]\label{def:ChannlOfProtocol}
	For a no-input two-party protocol $\Pi= (\Ac,\Bc)$, we associate the channel $C_\Pi$, defined by $\C_\Pi= C_{(X, U),(Y, V)}$, where $X$ and $Y$ are the local outputs of $\Ac$ and $\Bc$ (respectively) and
	$U$ and $V$ are the local views of $\Ac$ and $\Bc$ (respectively).
    
	For a two-party protocol $\Pi$ that gets a security parameter $1^\pk$ as its (only, common) input, we associate the channel ensemble $ \set{C_{\Pi(1^\pk)}}_{\pk\in \N}$. 
\end{definition}

\begin{definition}[$(\alpha,\gamma)$-Accurate channel]\label{def:accurate-func}
	A channel $C = C_{(X, U),(Y, V)}$ is {\sf $(\alpha,\gamma)$-accurate for the function $f$}, if $\ppr{C}{\size{\out(V)-f(X,Y)}\leq \alpha}\ge \gamma$, where $\out(V)$ is the designated output.
    A channel ensemble $C_{(X, U),(Y, V)}= \set{C_{(X_\pk, U_\pk),(Y_\pk, V_\pk)}}_{\pk\in \N}$ is  $(\alpha,\gamma)$-accurate for  $f$ if $C_{(X_\pk, U_\pk),(Y_\pk, V_\pk)}$ is $(\alpha(\pk),\gamma(\pk))$-accurate for $f$, for every $\pk \in \N$.
\end{definition}

\subsubsection{Differentially Private Channels}\label{sec:DPChannel}
Differentially private channels are naturally defined as follows:
\begin{definition}[Differentially private channels]\label{def:DPChannel}
	An $n$-size channel $C = C_{(X, U),(Y, V)}$ with $X, Y$ over $\oo^n$ 
	is {\sf$(\eps,\delta)$-differentially private} (denoted $(\eps,\delta)$-$\DP$) if for every $x \in \Supp(X)$ there exists an $n$-size $(\eps,\delta)$-$\DP$ mechanisms $\Mc_x$ such that $(X,Y,U) \equiv (X,Y,\Mc_X(Y))$, and for every $y \in \Supp(Y)$ there exists an $n$-size $(\eps,\delta)$-$\DP$ mechanisms $\Mc_y'$ such that $(X,Y,V) \equiv (X,Y,\Mc_Y'(X))$. In addition, we say that the channel is \emph{uniform} if $X$ and $Y$ are independent random variables uniformly distributed in $\oo^n$. 
\end{definition}

\begin{definition}[Computational differentially private channels]\label{def:CDPChannel}
	An $n$-size channel ensemble $C = \set{C_{(X_\pk, U_\pk),(Y_\pk, V_\pk)}}_{\pk\in\N}$ with $X_\pk, Y_\pk$ over $\oo^n$ 
	is {\sf$(\eps,\delta)$-computationally differentially private} (denoted $(\eps,\delta)$-$\CDP$) if for every ensemble $\set{x_\pk \in \Supp(X_\pk)}_{\pk\in\N}$ there exists an $n$-size $(\eps,\delta)$-\CDP mechanisms ensemble $\set{\Mc_{x_\pk}}_{\pk\in\N}$ such that $(X_\pk,Y_\pk,U_\pk) \equiv (X_\pk,Y_\pk,\Mc_{X_\pk}(Y_\pk))$, for every $\pk\in\N$, and for every ensemble $\set{y_\pk \in \Supp(Y_\pk)}_{\pk\in\N}$ there exists an $n$-size $(\eps,\delta)$-$\CDP$ mechanisms ensemble $\set{\Mc'_{y_\pk}}_{\pk\in\N}$ such that $(X_\pk,Y_\pk,V_\pk) \equiv (X_\pk,Y_\pk,\Mc_{Y_\pk}'(X_\pk))$ for every $\pk\in \N$. In addition, we say that the channel is \emph{uniform} if $X_\pk$ and $Y_\pk$ are independent random variables uniformly distributed in $\{\pm 1\}^n$ for all $\pk\in\N$.
\end{definition}




% \begin{lemma}~\label{lem:dp-sv-source}
% 	Let $P$ be an $\varepsilon$-DP randomized protocol. Let $X$ and $Y$ be independent random variables uniformly distributed in $\{\pm 1\}^n$ and let random variable $\Pi(X,Y)$ denote the transcript of running $P(X,y)$. Then for every $\pi\in Supp(\Pi)$, the random variables corresponding to the inputs conditioned on transcript $\pi$, $X_\pi$ and $Y_\pi$, are independent $e^{-\varepsilon}$-strong SV source.
% \end{lemma}





\subsubsection{Weak Erasure Channel (\WEC)}

\begin{definition}[\WEC]\label{def:WEC}
	A channel $((O_A,V_A), (O_B,V_B))$ with $O_A \in \set{0,1}$ and $O_B \in \set{0,1,\bot}$ is a {\sf weak erasure channel}, denoted $(\alpha,p,q)$-$\WEC$, if:
	\begin{itemize}
		%\item $O_A\in \set{-1,1}$ and $O_B\in \set{-1,1,\bot}$.
		\item Random erasure: $\pr{O_B = \perp} = 1/2$.
		
		\item Agreement: $\pr{O_A\ne O_B\mid O_B\ne \bot}\le \alpha$.
		
		\item Secrecy:
		
		\begin{enumerate}
			\item For every algorithm $\Dc$ it holds that\label{WEC:item:A}
			\begin{align*}
				%\size{\pr{\Ac(O_A,V_A) = 1 \mid O_B \neq \perp} - \pr{\Ac(O_A,V_A) = 1 \mid O_B = \perp}} \le p
				\size{\pr{\Dc(V_A) = 1 \mid O_B \neq \perp} - \pr{\Dc(V_A) = 1 \mid O_B = \perp}} \le p
			\end{align*}
			(Alice doesn't know if $O_B = \perp$.)
			
			\item For every algorithm $\Dc$ it holds that\label{WEC:item:B}
			\begin{align*}
				\pr{\Dc(V_B) = O_A \mid O_B=\bot} \leq \frac{1+q}{2}.
			\end{align*}
			(i.e., if $O_B=\bot$, Bob don't know what is the value of $O_A$).
			
			%\item $SD((O_A U|O_B=\bot),(O_A U|O_B\ne \bot))\le p$ (The sender don't know if $O_B=\bot$).
			
			%\item $SD(V O_A|O_B=\bot,V(-O_A)|O_B=\bot)\le q$ (If $O_B=\bot$, Bob don't know what the value of $O_A$).
		\end{enumerate}
	\end{itemize}
   We say that a channel ensemble $C=\set{C_\pk}_{\pk\in N}$ is a {\sf computational weak erasure channel}, denoted $(\alpha,p,q)$-\CompWEC, if for every \ppt algorithm $\Dc$ and every sufficiently large $\pk\in\N$, $C_\pk$ satisfies the properties stated in the items above, where the secrecy property holds with respect to a \ppt algorithm $\Dc$. A protocol $\Lambda$ is said to be $(\alpha,p,q)$-$\CompWEC$, if the ensemble induces by the protocol (that is, $C=\set{C_{\Lambda(\pk)}}_{\pk\in\N}$) is $(\alpha,p,q)$-$\CompWEC$.  
\end{definition}



\subsubsection{Approximate Weak Erasure Channel (\AWEC)}\label{sec:AWEC}

\begin{definition}[\AWEC]\label{def:AWEC}
	A channel $C = ((O_A,V_A), (O_B,V_B))$ over $([-n,n] \times \zo^*) \times (([-n,n] \cup \bot)  \times \zo^*)$ is an {\sf approximate weak erasure channel}, denoted $(\ell,\alpha,p,q)$-\AWEC if:
	\begin{itemize}
		
		\item Random erasure: $\pr{O_B = \perp} = 1/2$.
		
		\item Accuracy: $\pr{\size{O_A - O_B} > \ell \mid O_B \ne \bot}\le \alpha$.
		
		\item Secrecy:
		
		\begin{enumerate}
			\item For every algorithm $\Dc$ it holds that\label{AWEC:item:A}
			\begin{align*}
				%\size{\pr{\Ac(O_A,V_A) = 1 \mid O_B \neq \perp} - \pr{\Ac(O_A,V_A) = 1 \mid O_B = \perp}} \le p
				\size{\pr{\Dc(V_A) = 1 \mid O_B \neq \perp} - \pr{\Dc(V_A) = 1 \mid O_B = \perp}} \le p
			\end{align*}
			(Alice doesn't know if $O_B=\bot$).
			
			\item For every algorithm $\Dc$ it holds that\label{AWEC:item:B}
			\begin{align*}
				\pr{\size{\Dc(V_B) - O_A} \leq 1000 \ell \mid O_B=\bot} \leq q.
			\end{align*}
			(i.e., if $O_B=\bot$, Bob can't estimate the value of $O_A$ with error $\leq 1000 \ell$).
		\end{enumerate}
	\end{itemize}
     We say that a channel ensemble $C=\set{C_\pk}_{\pk\in N}$ is a {\sf computational approximate weak erasure channel}, denoted $(\ell,\alpha,p,q)$-\CompAWEC, if for every \ppt algorithm $\Dc$ and every sufficiently large $\pk\in\N$, $C_\pk$ satisfies the properties stated in the items above. A protocol $\Gamma$ is said to be $(\ell,\alpha,p,q)$-$\CompAWEC$, if the ensemble induced by the protocol (that is, $C=\set{C_{\Gamma(\pk)}}_{\pk\in\N}$) is $(\ell,\alpha,p,q)$-$\CompAWEC$.  
\end{definition}

We will make use of the following lemma, which shows that for some choices of the parameters, \AWEC implies \WEC. The lemma is proven in \cref{sec:AWEC-to-WEC}.

\begin{lemma}\label{lemma:AWEC-to-WEC}
	For every $\ell> 0$, there exists a \ppt protocol $\Lambda = (\Pc_1,\Pc_2)$ such that given an oracle access to an $(\ell,\alpha,p,q)$-\AWEC $C$, the channel $\tilde{C}$ induced by $\Lambda^C$ is $(\alpha'=\alpha+0.001,\: p' = p ,\:  q' = 1/2 + 2(q+0.01))$-\WEC.
	Furthermore, the proof is constructive in a black-box manner:
	\begin{enumerate}
		\item There exists an oracle-aided \ppt algorithm $\Ec_1$ such that for every channel $C = ((\OA,\VA), (\OB,\VB))$ and algorithm $\Dc$ violating the \WEC secrecy property~\ref{WEC:item:A} of $\tilde{C}$, algorithm $\Ec_1^{\Dc}$ violates the \AWEC secrecy property~\ref{AWEC:item:A} of $C$.
		
		\item There exists an oracle-aided \ppt algorithm $\Ec_2$ such that for every channel $C = ((\OA,\VA), (\OB,\VB))$ and algorithm $\Dc$ violating the \WEC secrecy property~\ref{WEC:item:B} of $\tilde{C}$, algorithm $\Ec_2^{\Dc}$ violates the \AWEC secrecy property~\ref{AWEC:item:B} of $C$.
	\end{enumerate}
\end{lemma}

Since \cref{lemma:AWEC-to-WEC} is constructive, the following is an immediate corollary.
\begin{corollary}\label{cor:CompAWEC to CompWEC}
There exists an oracle aided \ppt protocol $\Lambda$, such that given a protocol $\Gamma$ that induces $(\ell,\alpha,p,q)$-\CompAWEC, it holds that $\Lambda^\Gamma$ is $(\alpha'=\alpha+0.001,\: p' = p ,\:  q' = 1/2 + 2(q+0.01))$-\CompWEC.  
\end{corollary}
\begin{proof}[Proof of \ref{cor:CompAWEC to CompWEC}]
Let $\Lambda$ be the \ppt algorithm guaranteed  by Lemma \ref{lemma:AWEC-to-WEC}. Given an $(\ell,\alpha,p,q)$-\CompAWEC protocol $\Gamma$, we define $\Lambda(\pk)={\Lambda^{\Gamma(\pk)}(\pk)}$. Assume towards a contradiction that $\Lambda$ is not a $(\alpha',p',q')$-\CompWEC. It follows that there exists a \ppt $\Dc$ that for infinity many $\pk\in\N$ contradicts one of the \WEC secrecy properties of channel ensemble $\set{C_{\Lambda(\pk)}}_{\pk\in\N}$. Fix $\pk\in\N$ for which this holds. By Lemma \ref{lemma:AWEC-to-WEC}, there exists a \ppt $\Ec^\Dc$ that for every such $\pk$  contradicts one of the secrecy properties of the channel $C_{\Gamma(\pk)}$. This implies that for infinity many $\pk\in\N$, $\Ec^\Dc$  contradict the secrecy of the channel ensemble $\set{C_{\Gamma(\pk)}}_{\pk\in\N}$, which is a contradiction since this would means that $\Gamma$ is not a $(\ell,\alpha,p,q)$-\CompAWEC.       
\end{proof}



\subsection{Oblivious Transfer (\OT)}

\paragraph{Secure Computation.}
We use the standard notion of securely computing a functionality, \cf  \cite{Goldreich04}.
\begin{definition}[Secure computation]\label{def:SFE}
	A two-party protocol {\sf securely computes a functionality $f$}, if it does so according to the real/ideal paradigm.   We add the term perfectly/statistically/computationally/non-uniform computationally, if the simulator's output is  perfect/statistical/computationally indistinguishable/  non-uniformly indistinguishable from  the real distribution.  The protocol have the above notions of security {\sf against semi-honest  adversaries}, if its security only  guaranteed to holds against an adversary that follows the prescribed protocol.   Finally, for the case of perfectly secure computation, we naturally apply the above notion also to the non-asymptotic case: the protocol with no security parameter perfectly  compute a functionality $f$.
	
	A two-party protocol {\sf securely computes a functionality ensemble $f$ with oracle to a channel $C$}, if it does so according to the above definition when the parties have access to a trusted party computing $C$. All the above adjectives naturally extend to this setting.
\end{definition}

\paragraph{Oblivious Transfer.}
The (one-out-of-two) oblivious transfer functionality is defined as follows.
\begin{definition}[oblivious transfer functionality $f_{\OT}$]\label{def:OTfunc}
	The oblivious transfer functionality over $\zo \times (\zs)^2$ is defined by  $f_{\OT} (i,(\sigma_0,\sigma_1)) = (\perp,\sigma_i)$.
\end{definition}
A protocol is $\ast$ secure OT,   for \\$\ast\in \set{\text{semi-honest statistically/computationally/computationally non-uniform}}$, if it  compute the $f_{\OT}$  functionality with $\ast$ security.





% \begin{definition}[Computational oblivious transfer, semi-honest model]
% A protocol $\Pi=(\Ac,\Bc)$ is a semi-honest 1-out-of-2 computational oblivious transfer (comp-OT) protocol if the following holds. Given a common input $1^{\pk}$, the parties $\Ac$ and $\Bc$ run the protocol $\Pi(1^\pk)$ (in an honest manner) and    
% $\Ac$ outputs $X=(m_1,m_2)\in \zo\times\zo$ and has a view $U$ and $\Bc$ outputs $Y=(i,\hat{m})\in\zo\times\zo$ and has a view $V$, and the following properties are satisfied:
% \begin{enumerate}
%     \item \textbf{Correctness:} 
%     $\pr{\hat{m}\neq m_i}<\negl(\pk).$ 
    
%     \item \textbf{A's Privacy:} For every \ppt $\Dc$ and every sufficiently large $\pk$:
%     $\pr{\Dc(V)=m_{i-1}}<(1+\negl(\pk))/2$
    
%     \item \textbf{B's Privacy:} For every \ppt $\Dc$ and every sufficiently large $\pk$:
%     $\pr{\Dc(U)=i}<(1+\negl(\pk))/2$  
% \end{enumerate}
% \end{definition}

We make use of the following useful results by Wullschleger on oblivious transfer amplification from weak channels.
\begin{theorem}[\cite{Wullschleger09}, from \WEC to statistically secure \OT]\label{thm:WEC TO OT IT}
    There exists an oracle aided protocol $\Pi$ such that the following holds: Given a $(\alpha,p,q)$-\WEC $C$, if $44(\alpha+p)\le 1-q$ then $\Pi^{C}(1^\pk)$ is a semi-honest statistically secure \OT.
\end{theorem}

The following computational version of \cref{thm:WEC TO OT IT} is implicit in \cite{Wullschleger09} and is based on the computational proof explicitly stated in \cite{Wul07} (see Section 6 in \cite{Wullschleger09} for discussion).   

\begin{theorem}[\cite{Wullschleger09,   Wul07}, from \CompWEC to computinally secure \OT]\label{thm:WEC TO OT Comp}
    There exists an oracle aided protocol $\Pi$ such that the following holds: Given a $(\alpha,p,q)$-\CompWEC protocol $\Lambda$, if $44(\alpha+p)\le 1-q$ then $\Pi^{\Lambda}$ is a semi-honest computational secure \OT.
\end{theorem}



% \begin{definition}[Computational 1-out-of-2 Oblivious Transfer, semi-honest model]
% A protocol $\Pi=(\Ac,\Bc)$ is a semi-honest 1-out-of-2 $(\eps,\alpha,\beta)$-oblivious transfer (OT) protocol if the following holds. 

% The parties $\Ac$ and $\Bc$ run the protocol (in an honest manner) and    
% $\Ac$ outputs $X=(m_1,m_2)\in \zo\times\zo$ and has a view $U$ and $\Bc$ outputs $Y=(i,\hat{m})\in\zo\times\zo$ and has a view $V$, and following properties are satisfied:
% \begin{enumerate}
%     \item \textbf{Correctness:} 
%     $\pr{\hat{m}\neq m_i}<\eps.$ 
    
%     \item \textbf{A's Privacy:} For every adversary $\Dc$:
%     $\pr{\Dc(V)=m_{i-1}}<(1+\alpha)/2$
    
%     \item \textbf{B's Privacy:} For every adversary $\Dc$: $\pr{\Dc(U)=i}<(1+\beta)/2$  
% \end{enumerate}
% \end{definition}

\section{Adaptive tomography lower bound}
\subsection{Lower bound construction}
\begin{definition}
 \label{def:perturbation}
     Let $\dims^2/2\le\ell\le\dims^2-1$ and $\hbasis=(V_1, \ldots, V_{\dims^2}=\frac{\eye_\dims}{\sqrt{\dims}})$ be an orthonormal basis of $\Herm{\dims}$, and $\cd$ be a universal constant. Let  $\ptb=(\ptb_1, \ldots, \ptb_\ell)$ be uniformly drawn from $\{-1, 1\}^\ell$,
     \begin{equation}
         \Delta_{\ptb} = \frac{\cd\eps}{\sqrt{\dims}}\cdot\frac{1}{\sqrt{\ell}}\sum_{i=1}^\ell \ptb_iV_i, \quad \barDelta_{\ptb}= \Delta_{\ptb}\min\left\{1, \frac{1}{2\dims \opnorm{\Delta_{\ptb}}}\right\},
         \label{equ:delta_z}
     \end{equation}
     Finally we set $\sigma_{\ptb}=\qmm + \barDelta_{\ptb}$ whose distribution we denote as $\ptbDistr(\hbasis)$.
 \end{definition}

The construction adds independent binary perturbations to $\qmm$ along $\ell$ orthogonal trace-0 directions. With appropriate constant $\cd$, $\ptbDistr(\hbasis)$ has an exponentially small probability mass outside the set $\mathcal{P}_\eps\eqdef\{\rho: \tracenorm{\rho-\qmm}>\eps\}$.
\begin{theorem}[{\cite[Corollary 4.4]{liu2024role}}]
\label{prop:perturbation-trace-distance}
    Let $\cd= 10\sqrt{2}$, $\ell\ge \dims^2/2$, $\eps<1/200$. Then for $\sigma\sim \ptbDistr(\hbasis)$,  $\|\sigma-\qmm\|_1\ge \eps$ with probability at least $1-2\exp(-\dims)$. 
\end{theorem}

This is the result of a random matrix concentration.
\begin{restatable}[{\cite[Theorem 4.2]{liu2024role}}]{theorem}{randmatopnorm}
\label{thm:rand-mat-opnorm-concentration}
    Let $V_1, \ldots, V_{\dims^2}\in\C^{\dims\times \dims}$ be an orthonormal basis of $\C^{\dims\times \dims}$ and $\ptb_1, \ldots, \ptb_{\dims^2}\in\{-1, 1\}$ be independent symmetric Bernoulli random variables. Let $W=\sum_{i=1}^{\ell}\ptb_iV_i$ where $\ell\le \dims^2$. For all $\alpha>0$, there exists $\cop_\alpha$, {which is increasing in $\alpha$} such that
    \[
    \probaOf{\opnorm{W}>\cop_\alpha\sqrt{\dims}}\le 2\exp\{-\alpha\dims\}.
    \]
\end{restatable}

Let $z\sim\{-1,1\}^{\ell}$ and $\sigma_z\sim \ptbDistr(\hbasis)$ be defined in \cref{def:perturbation}. Use the shorthand $\p_z^{\out_i|\out^{i-1}}=\p_{\sigma_z}^{\out_i|\out^{i-1}}$. 
We define the following mixtures,
\begin{equation}
    \p_{+i}^{\out^\ns}\eqdef \frac{1}{2^{\ell-1}}\sum_{\ptb:\ptb_i=+1}\p_z^{\out^\ns},\quad  \p_{-i}^{\out^\ns}\eqdef \frac{1}{2^{\ell-1}}\sum_{\ptb:\ptb_i=-1}\p_z^{\out^\ns}.
\end{equation}
Which are the distributions of outcomes $\out^\ns$ when we fix the $i$th coordinate to be $+1$ and $-1$ respectively. Then we can define,
\begin{equation}
\label{equ:out-distr}
\q^{\out^\ns}\eqdef\frac{1}{2^\ell}\sum_{z\in\{-1,1\}^\ell}\p_z^{\out^\ns}=\frac{1}{2}(\p_{+i}^{\out^\ns}+\p_{-i}^{\out^\ns}).  
\end{equation}

This is exactly the distribution of $\out^\ns$ when $\sigma_z\sim \ptbDistr(\hbasis)$ and outcomes $\out^\ns$ are obtained by measuring $\sigma_z^{\otimes\ns}$ with the adaptive scheme $\POVM^\ns$.




\subsection{Mutual information upper bound via MIC}
The following theorem bounds the mutual information in terms of the measurement information channel.
\begin{theorem}
\label{thm:avg-MI-upper}
    Let $\sigma_\ptb\sim\ptbDistr(\hbasis)$ where $\ptb\sim\{-1,1\}^{\ell}$, $\out^\ns$ be the outcomes after applying $\POVM^\ns$. Then for $\dims\ge 1024$ and all $t\in[\ns]$,
    \begin{align}
         \frac{1}{\ell}\sum_{i=1}^{\ell}\mutualinfo{\ptb_i}{\out^t}&\le \frac{8 tc^2 \eps^2}{\ell^2}  \sup_{\POVM\in {\povmset}}{\sum_{i=1}^\ell \vadj{V_i} \Choi_{\POVM} \vvec{V_i}} +16\exp\{-\alpha\dims\}tc^2\eps^2 \label{equ:avg-MI-partial}\\
         &\le \frac{16tc^2\eps^2}{\ell^2}\tracenorm{\povmset}.\label{equ:avg-MI-tracenorm}
    \end{align}
\end{theorem}

\begin{proof}
    We start with the fundamental fact that mutual information is the conditional KL-divergence between the conditional distribution given the marginal $x^t$: $\p^{\out^t}_{z_i}$ for $1 \leq i \leq n$ and the marginal distribution $\q^{\out^t}$,
\begin{align*}
    I(z_i;x^t) &= \kldiv{\p^{\out^t}_{z_i}}{\q^{\out^t} \mid z_i} = \expectDistrOf{z_i}{\kldiv{\p^{\out^t}_{z_i}}{\q^{\out^t}}} \\ &= \frac{1}{2} \kldiv{\p_{+i}^{\out^t}}{\q^{\out^t}} + \frac{1}{2} \kldiv{\p_{-i}^{\out^t}}{\q^{\out^t}} \\ 
    &= \frac{1}{2} \kldiv{\p_{+i}^{\out^t}}{\frac{\p_{+i}^{\out^t} + \p_{-i}^{\out^t}}{2}} + \frac{1}{2} \kldiv{\p_{-i}^{\out^t}}{\frac{\p_{+i}^{\out^t} + \p_{-i}^{\out^t}}{2}}.
\end{align*}
Thus, by convexity, 
\begin{align}
    I(z_i;x^t) &\leq \frac{1}{4} \left[\kldiv{\p^{\out^t}_{+i}}{\p^{\out^t}_{+i}}+\kldiv{\p^{\out^t}_{-i}}{\p^{\out^t}_{+i}} \right] = \frac{1}{2} \kldivsym{\p^{\out^t}_{+i}}{\p^{\out^t}_{-i}} \label{pf:MI-bound}.
\end{align}
Where the last inequality comes from the convexity of KL-divergence with respect to its second argument. Given this symmetric KL-divergence between the mixture distribution conditioned on the i-th perturbation, we can further narrow the correlation between the measurement outcomes and the perturbation with the change in measurement outcome distribution when $z_i$ is flipped. We apply chain rule on the symmetric KL-divergence to allow us to isolate the per measurement round divergence,
\begin{align}
    \kldivsym{\p^{\out^t}_{+i}}{\p^{\out^t}_{-i}} = \sum_{j=1}^{t} \expectDistrOf{\q^{x^\ns}}{\kldivsym{\p^{\out_{j}|\out^{j-1}}_{+i}}{\p^{\out_{j}|\out^{j-1}}_{-i}}}.
    \label{eq:per-round-divergence}
\end{align}
We bound the symmetric KL by the chi-squared divergence,
\begin{align*}
    \kldivsym{\p^{\out_{j}|\out^{j-1}}_{+i}}{\p^{\out_{j}|\out^{j-1}}_{-i}} &\le \chisquare{\p^{\out_{j}|\out^{j-1}}_{+i}}{\p^{\out_{j}|\out^{j-1}}_{-i}} \\
    &\leq \frac{1}{2^{l-1}} \sum_{z \in \{+1,-1\}^\ell} \; \chisquare{\p^{\out_{j}|\out^{j-1}}_{z}}{\p^{\out_{j}|\out^{j-1}}_{z^{\oplus i}}}  \\
    &= \frac{1}{2^{\ell-1}} \sum_{z \in \{+1,-1\}^\ell} \; \expectDistrOf{X \sim \p^{\out_{j}|\out^{j-1}}_{z^{\oplus i}}}{\delta_j(X)^2}.
\end{align*}
 Where the last inequality is from the joint convexity of f-divergences. $\delta_t(X)$ follows the definition,
\begin{align*}
    \delta_j(x) \eqdef \frac{\p^{\out_{j}|\out^{j-1}}_{z}(x)-\p^{\out_{j}|\out^{j-1}}_{z^{\oplus i}}(x)}{\p^{\out_{j}|\out^{j-1}}_{z^{\oplus i}}(x)}.
\end{align*}
Furthermore, $\delta_j$ term can be bounded by extracting the MIC channel,
\begin{align*}
    \delta_j(x) &= \frac{\Tr[M_x^j (\qmm + \barDelta_{\ptb})] - \Tr[M_x^j (\qmm + \barDelta_{\ptb^{\oplus i}})]}{\Tr[M_x^j (\qmm + \barDelta_{\ptb^{\oplus i}})]} \\
    &= \frac{\Tr[M_x^j (\barDelta_{\ptb} - \barDelta_{\ptb ^{\oplus i}})]}{\Tr[M_x^j (\qmm + \barDelta_{\ptb^{\oplus i}})]}.
\end{align*}
Therefore, we plug $\delta_j(X)$ into the expectation and noting that $\p^{\out_{j}|\out^{j-1}}_{z^{\oplus i}}(x) = \Tr[M_x^j (\qmm + \barDelta_{\ptb^{\oplus i}})]$,
\begin{align*}
    \expectDistrOf{X \sim \p^{\out_{j}|\out^{j-1}}_{z^{\oplus i}}}{\delta_j(X)^2} &= \sum_{x \in X} \frac{\Tr[M_x^j (\barDelta_{\ptb} - \barDelta_{\ptb ^{\oplus i}})]^2}{\Tr[M_x^j (\qmm + \barDelta_{\ptb^{\oplus i}})]} \\
    &= \sum_{x \in X} \frac{\Tr[(\barDelta_{\ptb} - \barDelta_{\ptb ^{\oplus i}}) M_x^j] \Tr[(\barDelta_{\ptb} - \barDelta_{\ptb ^{\oplus i}}) M_x]^{\dagger}}{\Tr[M_x^j (\qmm + \barDelta_{\ptb^{\oplus i}})]} \\
    &= \sum_{x \in X} \frac{\Tr[(\barDelta_{\ptb} - \barDelta_{\ptb ^{\oplus i}}) M_x^j] \Tr[(\barDelta_{\ptb} - \barDelta_{\ptb ^{\oplus i}}) M_x^j]^{\dagger}}{\Tr[M_x^j (\qmm + \barDelta_{\ptb^{\oplus i}})]} \\
    &= \sum_{\out \in \mathcal{X}} \frac{\vvdotprod{(\barDelta_{\ptb} - \barDelta_{\ptb ^{\oplus i}})}{M_x^j}\vvdotprod{M_x^j}{(\barDelta_{\ptb} - \barDelta_{\ptb ^{\oplus i}})}}{\Tr[M_x^j(\qmm + \barDelta_{\ptb^{\oplus i}})]}.
\end{align*}
Note that $\qmm + \barDelta_{\ptb^{\oplus i}} \succcurlyeq \frac{1}{2} \qmm \implies \Tr[M_x^j(\qmm + \barDelta_{\ptb^{\oplus i}})] \geq \Tr[M_x^j(\frac{1}{2} \qmm)] = \frac{1}{2 \dims}\Tr[M_x^j]$. This statement comes from the fact that $\opnorm{\barDelta_{\ptb^{\oplus i}}} \leq \frac{1}{2 \dims}$~\eqref{equ:delta_z},
\begin{align*}
   \expectDistrOf{X \sim \p^{\out_{j}|\out^{j-1}}_{z^{\oplus i}}}{\delta_j(X)^2} &\le  2 \dims \sum_{\out \in \mathcal{X}} \frac{\vvdotprod{(\barDelta_{\ptb} - \barDelta_{\ptb ^{\oplus i}})}{M_x^j}\vvdotprod{M_x^j}{(\barDelta_{\ptb} - \barDelta_{\ptb ^{\oplus i}})}}{\Tr[M_x^j]} \\
   &=  2 \dims \vadj{(\barDelta_{\ptb} - \barDelta_{\ptb ^{\oplus i}})} \sum_{\out \in \mathcal{X}} \frac{\vvec{M_x^j}\vadj{M_x^j}}{\Tr[M_x^j]} \vvec{(\barDelta_{\ptb} - \barDelta_{\ptb ^{\oplus i}})}.
\end{align*}
We can then apply this bound to the per-round symmmetric KL-divergence,
\begin{align*}
   \kldivsym{\p^{\out_{j}|\out^{j-1}}_{+i}}{\p^{\out_{j}|\out^{j-1}}_{-i}}&\le \frac{1}{2^{\ell-1}} \sum_{z \in \{+1,-1\}^l} \; 2 \dims \vadj{(\barDelta_{\ptb} - \barDelta_{\ptb ^{\oplus i}})} \Sigma_{\out \in \mathcal{X}} \frac{\vvec{M_x^j}\vadj{M_x^j}}{\Tr[M_x^j]} \vvec{(\barDelta_{\ptb} - \barDelta_{\ptb ^{\oplus i}})} \\
   &= 4 \dims \expectDistrOf{z\sim\{-1, 1\}^{\ell}}{\vadj{(\barDelta_{z} - \barDelta_{z ^{\oplus i}})} \Choi_{\POVM_j} \vvec{(\barDelta_{\ptb} - \barDelta_{z ^{\oplus i}})}},
\end{align*}
where $z$ is drawn uniformly from $\{-1,1\}^{\ell}$.
Another key to this bound is that we have a concentration on the operator norm of the perturbation matrix such that the operator norm lies in the boundary (within some constant)  with exponentially decreasing probability, see \cref{thm:rand-mat-opnorm-concentration}. 
Intuitively, this means that it is rare that all of the $z_i$ components are selected in a way where eigenvectors of the $z_i V_i$ components are aligned, thus resulting in an equal contribution to the total perturbation from each $z_i V_i$ component. 
As a result, this concentration perspective allows us to see that flipping a single $z_i V_i$ entry will dictate a perturbation outcome with high probability.
For convenience, we define the concentration set for the perturbation parameters,
\begin{align}
    \mathcal{G} := \{z \in \{1,1\}^{\ell} \; | \; \opnorm{W_z} \leq \kappa_\alpha \sqrt{d}\} \label{eq:concentrated-set},   
\end{align}
where $\alpha$ is a positive constant and $\kappa_\alpha$ is a positive constant non-decreasing in $\alpha$. By \cref{thm:rand-mat-opnorm-concentration},
\[
\probaOf{z \in \mathcal{G}} \geq 1 - 2\exp\{- \alpha d\}.
\]
For more details on the constants involved, see Lemma 21 of \cite{liu2024role}. 
We then condition between the possible cases of perturbations with law of iterative expectation,

\begin{align}
    \kldivsym{\p^{\out_{j}|\out^{j-1}}_{+i}}{\p^{\out_{j}|\out^{j-1}}_{-i}} &\le 4 \dims \expect{\expectDistrOf{z}{\vadj{(\barDelta_{\ptb} - \barDelta_{\ptb ^{\oplus i}})} \Choi_{\POVM_j} \vvec{(\barDelta_{\ptb} - \barDelta_{\ptb ^{\oplus i}})} \;| \indic{z \in \mathcal{G}}}}\label{pf:KL-div-concentration}.
\end{align}
  Now, it suffices to bound the peturbation for when $z \in G$ and $z \notin G$. When $z \in G$, the following bound holds for $\eps \leq \frac{1}{4(\kappa_\alpha+1)}$,
\begin{align}
\opnorm{\Delta_z} &= \frac{c\eps}{\sqrt{\dims \ell}} \opnorm{W_z} \leq  \frac{\kappa_\alpha c\eps}{\sqrt{\ell}} \leq \frac{2 \kappa_\alpha c\eps}{\dims} \leq \frac{1}{2 \dims} .\label{eq:op-norm-bound}
\end{align}
In addition, the following holds when the i-th bit is flipped,
\begin{align*}
        \opnorm{W_{z ^{\oplus i}}} &= \opnorm{W_z - 2 z_i V_i} \leq \opnorm{W_z} + \opnorm{-2 z_i V_i} \\
    &\le \kappa_\alpha \sqrt{\dims} + 2 \leq (\kappa_\alpha + 1) \sqrt{d} \\
    \implies \opnorm{\Delta_{z^{\oplus i}}} &= \frac{c\eps}{\sqrt{\dims \ell}} \opnorm{W_{z^{\oplus i}}} \leq  \frac{(\kappa_\alpha + 1) c\eps}{\sqrt{\ell}} \leq \frac{2(\kappa_\alpha + 1) c\eps}{\dims} \leq \frac{1}{2\dims}.
\end{align*}
The second inequality follows because $ \opnorm{V_i}^2 = \|V_i\|_{S_\infty}^2 \leq\|V_i\|_{S_2}^2 = \hdotprod{V_i}{V_i} = 1$. For $z \in \mathcal{G}$, we have that $\opnorm{\Delta_{z}}, \opnorm{\Delta_{z^{\oplus i}}} \leq \frac{1}{2 \dims}$. This results in $\barDelta_{z ^{\oplus i}} = \Delta_{z ^{\oplus i}}, \;\barDelta_{z} = \Delta_{z}$, by definition of the normalization factor in \cref{equ:delta_z}. Thus,
\begin{align*}
 \vadj{(\barDelta_{z} - \barDelta_{z ^{\oplus i}})} \Choi_{\POVM_j} \vvec{(\barDelta_{z} - \barDelta_{z ^{\oplus i}})} &=  \vadj{(\Delta_{z} - \Delta_{z ^{\oplus i}})}
 \Choi_{\POVM_j} \vvec{(\Delta_{z} - \Delta_{z ^{\oplus i}})} \\
 &=\vadj{\frac{c\eps}{\sqrt{\dims \ell}} 2 z_i V_i}
 \Choi_{\POVM_j}
 \vvec{\frac{c\eps}{\sqrt{\dims \ell}} 2 z_i V_i} = \frac{4 c^2 \eps^2 z_{i}^2}{\dims \ell}  = \frac{4 c^2 \eps^2}{\dims \ell} \vadj{V_i} \Choi_{\POVM_j} \vvec{V_i}.
\end{align*}
We will later see that this will result in the trace decomposition of $\Choi_{\POVM_j}$ under the vectorized version of the orthornormal Hilbert basis $\hbasis$. Now, we will apply a more crude bound for the low-concentration set $z \notin \mathcal{G}$. We start by bounding the Hilbert-Schmidt norm of the perturbation matrix for every $z \in \{-1,1\}^\ell$,
\begin{align*}
\hsnorm{\barDelta_{z}} &= \sqrt{\vvdotprod{\barDelta_{z}}{\barDelta_{z}}} \\
&= \sqrt{\frac{c^2\eps^2}{\dims \ell}\vvdotprod{\sum_{i=1}^\ell \min\left\{1, \frac{1}{2\dims \opnorm{\Delta_{\ptb}}}\right\} z_i V_i}{\sum_{i=1}^\ell \min\left\{1, \frac{1}{2\dims \opnorm{\Delta_{\ptb}}}\right\} z_i V_i}}\\
&= \sqrt{\frac{c^2 \eps^2}{\dims \ell} \sum_{i \neq j} \min\left\{1, \frac{1}{2\dims \opnorm{\Delta_{\ptb}}}\right\}^2 z_i z_j \vvdotprod{V_i}{V_j} + \sum_{i}^\ell \min\left\{1, \frac{1}{2\dims \opnorm{\Delta_{\ptb}}}\right\}^2 z_i^2 \vvdotprod{V_i}{V_i}} \\
&= \sqrt{\frac{c\eps^2}{\dims \ell} \sum_{i}^\ell \min\left\{1, \frac{1}{2\dims \opnorm{\Delta_{\ptb}}}\right\}^2} \leq  \sqrt{\frac{c\eps^2}{\dims \ell} \sum_{i}^\ell 1} = \frac{c \eps}{\sqrt{\dims}}.
\end{align*} 
Where last line holds from the orthonormality of the perturbation basis. Now, we can use triangle inequality of the Hilbert-Schmidt norm to get the bound on the Hilbert-Schmidt norm of the difference between the perturbation matrices.
\begin{align*}
    \vadj{(\barDelta_{z} - \barDelta_{z ^{\oplus i}})} \Choi_{\POVM_j} \vvec{(\barDelta_{z} - \barDelta_{z ^{\oplus i}})} 
    &\le \opnorm{\Choi_{\POVM_j}}\hsnorm{\barDelta_{z} - \barDelta_{z ^{\oplus i}}}^2  \\
    &\leq 2(\hsnorm{\barDelta_{z}}^2 + \hsnorm{\barDelta_{z ^{\oplus i}}}^2) \\
    &\leq \frac{4c^2 \eps^2}{\dims}.
\end{align*}
The first step is due to the definition of operator norm. The second step is because $\opnorm{\Choi_{\POVM_j}} \le 1$, triangle inequality, and $(a+b)^2\le 2(a^2+b^2)$. We can further bound the symmetric KL-divergence in \cref{pf:KL-div-concentration},
 \begin{align*}
     \kldivsym{\p^{\out_{j}|\out^{j-1}}_{+i}}{\p^{\out_{j}|\out^{j-1}}_{-i}} &\le 4 \dims \left[\probaOf{z \in \mathcal{G}} \frac{4 c^2 \eps^2}{\dims\ell} \vadj{V_i} \Choi_{\POVM_j} \vvec{V_i} + (1 - \probaOf{z \in \mathcal{G}})   \frac{4 c^2 \eps^2}{\dims}\right] \\
     &= \probaOf{z \in \mathcal{G}} \frac{16 c^2 \eps^2}{\ell} \vadj{V_i} \Choi_{\POVM_j} \vvec{V_i} + (1-\probaOf{z \in \mathcal{G}}) 16 c^2 \eps^2.
 \end{align*} 

Thus combining with \eqref{pf:MI-bound} \eqref{eq:per-round-divergence},
\begin{align*}
    \frac{1}{\ell}\sum_{i=1}^{\ell}\mutualinfo{\ptb_i}{\out^t} &\le \frac{1}{2\ell}\sum_{i=1}^{\ell}\sum_{j=1}^{t} \expectDistrOf{\q^{\out^\ns}}{\kldivsym{\p^{\out_{j}|\out^{j-1}}_{+i}}{\p^{\out_{j}|\out^{j-1}}_{-i}}} \\
    &\leq \probaOf{z \in \mathcal{G}} \frac{8 c^2 \eps^2}{\ell^2} \sum_{j=1}^{t} \sum_{i=1}^{\ell} \expectDistrOf{\q^{\out^\ns}}{ \vadj{V_i} \Choi_{\POVM_j} \vvec{V_i}}
    + (1 - \probaOf{z \in \mathcal{G}}) \sum_{j=1}^{t} \sum_{i=1}^{\ell}  \frac{8c^2\eps^2}{\ell} \\
    &\le  \frac{8 tc^2 \eps^2}{\ell^2}  \expectDistrOf{\q^{\out^\ns}}{\frac{1}{t}\sum_{j=1}^{t}\sum_{i=1}^\ell \vadj{V_i} {\Choi}_{\POVM_j} \vvec{V_i}} +16\exp\{-\alpha\dims\}tc^2\eps^2\\
    &\le \frac{8 tc^2 \eps^2}{\ell^2}  \sup_{\POVM\in\povmset}\sum_{i=1}^\ell \vadj{V_i} {\Choi}_{\POVM} \vvec{V_i} +16\exp\{-\alpha\dims\}tc^2\eps^2,
\end{align*}
The second term in the final step is due to \cref{thm:rand-mat-opnorm-concentration}. This proves \eqref{equ:avg-MI-partial} in \cref{thm:avg-MI-upper}.

We continue to derive the remaining expression \eqref{equ:avg-MI-tracenorm}. We use the fact that for any matrix $A\in\C^{\dims\times\dims}$ and an orthonormal basis $\qbit{u_1}, \ldots, \qbit{u_\dims}$,
\[
\Tr[A]=\sum_{i=1}^{\dims}\matdotprod{u_i}{A}{u_i}.
\]
Combining with the fact that ${\Choi}_{\POVM}$ is p.s.d., we have
\[
\sum_{i=1}^{\ell} \vadj{V_i} {\Choi}_{\POVM} \vvec{V_i}\le \sum_{i=1}^{\dims^2}\vadj{V_i}\Choi_{\POVM} \vvec{V_i}=\Tr[\Choi_{\POVM}]=\tracenorm{\Choi_{\POVM}}.
\]
Therefore, continuing from \eqref{equ:avg-MI-partial},

\begin{align*}
     \frac{1}{\ell}\sum_{i=1}^{\ell}\mutualinfo{\ptb_i}{\out^t} 
    &\le \frac{8 t c^2 \eps^2}{\ell^2} \tracenorm{\povmset} + 16\exp\{-\alpha\dims\}tc^2\eps^2\\
    & \le \frac{16 t c^2 \eps^2}{\ell^2} \tracenorm{\povmset}.
\end{align*}
The first step is from the definition of $\tracenorm{\povmset}$ in \eqref{equ:max-povm-norm}. The second step holds as $\tracenorm{\povmset}\ge 1$ and $\exp\{-\alpha d\} \leq \frac{1}{d^4}$ when $d \geq 1024$. 
\end{proof}


\subsection{Mutual information lower bound}
We state some useful bounds on mutual information.
\begin{lemma}[{\cite[Lemma 10]{ACLST22iiuic}}]
\label{lem:MI-lower}
    Let $Z\in\{-1, 1\}^\ab$ be drawn uniformly and $Z-Y-\hat{Z}$ be a Markov chain where $\hat{Z}$ is an estimate of $Z$. Let $h(t)\eqdef -t\log t-(1-t)\log(1-t)$, then for each $i\in[\ab]$,
    \[
    \mutualinfo{Z_i}{Y}\ge 1-h(\probaOf{Z_i\ne \hat{Z}_i}).
    \]
\end{lemma}

The following lemma is an Assouad-type lower bound on the average mutual information. 
\begin{lemma}
\label{lem:avg-MI-lower}
    Let $\sigma_\ptb\sim\ptbDistr(\hbasis)$ where $\ptb\sim\{-1,1\}^{\ell}$, $\out^\ns$ be the outcomes after applying $\POVM^\ns$ to $\sigma_\ptb^{\otimes\ns}$, and $\qest$ be an estimator using $\out^\ns$ that achieves an accuracy of $\eps$. Then,
    \begin{equation}
        \frac{1}{\ell}\sum_{i=1}^{\ell}\mutualinfo{Z_i}{Y}\ge\frac{1}{100}.
    \end{equation}
\end{lemma}


Combining \cref{lem:avg-MI-lower} and \cref{thm:avg-MI-upper} proves the interactive lower bound for tomography.
\begin{proof}
   The idea behind this bound is that any good estimation $\qest$ of the parameterized state $\sigma_\ptb$ is close in the sense that the closest parameterized $\sigma_{\zest}$ to $\qest$ should also be sufficiently close. Then, we can relate the distance $\tracenorm{\sigma_{\ptb} - \sigma_{\zest}}$ to the hamming distance in $\sum_{i=1}^\ell \indic{z_i \neq \zest_i}$. Once this relation is established, then a optimal tomography algorithm should also have low probability of error in estimating $z$ with $\zest$. Thus, leading to lower bound of mutual information with the application of \cref{lem:MI-lower}. We begin by first bounding the error between the "parameterized version" of the estimator and $\sigma_{\hat{\ptb}}$,
   \begin{align*}
        \zest &:= \argmin_{\ptb \in \{-1,1\}^\ell } \tracenorm{\sigma_{\ptb} - \qest}\\
        \tracenorm{\sigma_{\zest} - \sigma_{\ptb}} &\leq \tracenorm{\sigma_{\ptb} - \qest} + \tracenorm{\qest-\sigma_{\zest}} \leq 2 \tracenorm{\sigma_{\ptb} - \qest},
   \end{align*}
   where the last line holds since $\tracenorm{\qest-\sigma_{\zest}} \leq \tracenorm{\qest-\sigma_{\ptb}}$ by definition of $\hat{\ptb}$. Notice $ \tracenorm{\qest-\sigma_{\ptb}} \leq \eps \implies \tracenorm{\sigma_{\zest} - \sigma_{\ptb}} \leq 2\eps $. Thus, 
   $$\Pr[\tracenorm{\sigma_{\zest} - \sigma_{\ptb}} \leq 2 \eps] \ge \Pr[\tracenorm{\sigma_{\zest} - \sigma_{\ptb}} \leq \eps] \geq 0.99.$$
   Now, we will introduce a lemma that will allow us to construct a informaton-theoretic packing around this estimator. This is done by relating the trace distance and the hamming distance between Z parameters. We present the formal version of~\cref{lemma:hamm-separation-informal}

   \begin{lemma}[Trace distance Hamming separation] \label{lemma:hamm-packing}
       Consider $z \in \mathcal{G}$, where $\mathcal{G}$ is defined from \cref{eq:concentrated-set}. For any  $\hat{z} \in \left\{-1,1\right\}^{\ell}$,
       \begin{equation}
           \tracenorm{\sigma_\ptb - \sigma_{\zest}} \geq \frac{c \eps}{2\kappa_\alpha \ell} \ham{\ptb}{\zest}.
       \end{equation}
   \end{lemma}

   \begin{proof}
       
   Let $C_{z} := \min\left\{1, \frac{1}{2\dims \opnorm{\Delta_{z}}}\right\}$ and define the matrices,
   \[\Delta_{w} := \frac{c \eps}{\sqrt{d\ell}}  \sum_{i=1}^\ell \indic{z_i \neq \hat{z}_i} z_i V_i, \;\Delta_{c} := \frac{c \eps}{\sqrt{d\ell}} \sum_{i=1}^\ell \indic{z_i = \hat{z}_i} z_i V_i.
   \]
   Notice the trace norm of distance between perturbation matrices has the following form,
   \begin{align*}   
   \tracenorm{\sigma_{\zest} - \sigma_{\ptb}} & = \tracenorm{\barDelta_{\hat{\ptb}} - \barDelta_{\ptb}} \\
   &=\tracenorm{C_{\hat{z}} \Delta_{\hat{\ptb}} - C_{z} \Delta_{\ptb}} \\
   &= \frac{c \eps}{\sqrt{d\ell}} \tracenorm{(-C_z-C_{\zest}) \sum_{i=1}^\ell \indic{z_i \neq \zest_i} z_i V_i + (C_{\zest} - C_z) \sum_{i=1}^\ell \indic{z_i = \zest_i} z_i V_i} \\
   &= \tracenorm{(C_z+C_{\zest}) \Delta_w + (C_z-C_{\zest}) \Delta_c)}.
   \end{align*}
    Now, we will take advantage of the duality between the trace and operator norm (\cref{lemma:trace-norm-dual}) to correlate the distance between perturbations to the hamming distance between $z$ and $\zest$. Let $W_z = \sum_{i=1}^\ell z_i V_i$. For $z$ such that $\opnorm{W_z} \leq \kappa_\alpha \sqrt{\dims}$, we have $C_z=1$, from \cref{eq:op-norm-bound}.
   \begin{align*}
    \tracenorm{\sigma_{\zest} - \sigma_{\ptb}} &=
     \tracenorm{((1+C_{\zest}) \Delta_w + (1-C_{\zest}) \Delta_c} = \sup_{\opnorm{B} \leq 1} |\Tr[B^{\dagger} \left[(1+C_{\zest}) \Delta_w + (1-C_{\zest}) \Delta_c\right]]| \\
     &\geq \frac{1}{\kappa_\alpha \sqrt{\dims}} 
     |\Tr[W_z^{\dagger} \left[(1+C_{\zest}) \Delta_w + (1-C_{\zest}) \Delta_c\right]]| = \frac{c \eps}{\sqrt{\dims \ell}} \frac{1}{\kappa_\alpha \sqrt{\dims}} |(1+C_{\zest}) \delta_w + (1-C_{\zest}) \delta_c| \\
     &= \frac{c \eps}{\kappa_\alpha d \sqrt{\ell}} \left[(1+C_{\zest}) \delta_w + (1-C_{\zest}) \delta_c\right] \geq \frac{c \eps}{2 \kappa_\alpha \ell} \delta_w.
   \end{align*}
   Where $\delta_w = \ham{z}{\zest}$ and $\delta_c = \ell - \delta_w = \ell - \ham{z}{\zest}$.The second inequality uses: $B = \frac{W_z}{\kappa_\alpha \sqrt{\dims}}$. The reduction $\Tr[W_z^{\dagger} \Delta_w] = \delta_w$ and $\Tr[W_z^{\dagger} \Delta_c] = \delta_c$ comes directly from the orthonormality of the peturbation matrices $\{V_i\}_{i=1}^\ell$ under the inner product: $\hdotprod{A}{B} = \vvdotprod{A}{B} = \Tr[A^\dagger B]$. With the last line, we have shown the desired bound.  
   \end{proof}
    
   Since this relation to $\ham{\cdot}{\cdot}$ only occurs for a concentrated set $\mathcal{G}$, we can show that the expected hamming distance is "approximately trace distance" for sufficiently large $\dims > \frac{\ln{5}}{\alpha}$. $\sigma_{\hat{z}}$ also has to be close to $\sigma_z$ with high probability to be a sufficient estimator of $\sigma_z$, inducing a upper bound on the error probability of estimating $Z$,
   \begin{align}
       \frac{1}{\ell} \expectDistrOf{}{\delta_w} &= \frac{1}{\ell} \expectDistrOf{}{\delta_w \mid \tracenorm{\sigma_z - \sigma_{\zest}} \leq 2 \eps} \Pr[\tracenorm{\sigma_z - \sigma_{\zest}} \leq 2 \eps] \nonumber\\
       &+ \frac{1}{\ell} \expectDistrOf{}{\delta_w | \tracenorm{\sigma_z - \sigma_{\zest}} > 2 \eps} \Pr[\tracenorm{\sigma_z - \sigma_{\zest}} > 2 \eps] \nonumber\\
       &\leq \frac{1}{\ell} \expectDistrOf{}{\delta_w \mid \tracenorm{\sigma_z - \sigma_{\zest}} \leq 2 \eps} + 0.01.  \label{eq:cond-expect-tom-lower}
   \end{align}
   It is enough to upper bound the remaining expectation term by a constant. We will case on whether $z$'s lead to a approximate hamming relationship with trace distance. When $z \in \mathcal{G}$, we apply \cref{lemma:hamm-packing}
   \begin{align*}
       \frac{c \eps}{2 \kappa_\alpha \ell} \delta_w \leq \tracenorm{\sigma_z - \sigma_{\zest}} \leq 2 \eps 
       \implies  \frac{1}{\ell} \delta_w \leq \frac{4 \kappa_\alpha}{c}.
   \end{align*}
   The conditional expectation will now be bounded by a small constant for $c \geq 10 \kappa_\alpha$,
   \begin{align*}
       \frac{1}{\ell} \expectDistrOf{}{\delta_w \mid \tracenorm{\sigma_\ptb - \sigma_{\zest}} \leq 2 \eps} &\leq \Pr[z \in \mathcal{G}]\frac{1}{\ell} \expectDistrOf{}{\delta_w \mid \tracenorm{\sigma_\ptb - \sigma_{\zest}} \leq 2 \eps \land z \in \mathcal{G}} \\
       &+\Pr[z \notin \mathcal{G}] \frac{1}{\ell} \expectDistrOf{}{\delta_w \mid \tracenorm{\sigma_z - \sigma_{\zest}} \leq 2 \eps \land z \notin \mathcal{G}} \\
       &\leq \frac{4 \kappa_\alpha}{c} +  2\exp\{-\alpha d\} \cdot 1 \leq \frac{2}{5} + \frac{2}{5} = 0.40.
   \end{align*}
   Substituting this result into \cref{eq:cond-expect-tom-lower}, we have $\frac{1}{\ell} \sum_{i=1}^\ell \Pr[Z_i \neq \hat{Z_i}] = \frac{1}{\ell} \expectDistrOf{}{\delta_w} \leq 0.41$. We can then apply \cref{lem:MI-lower} to obtain the mutual information bound,
   \begin{align*}
       \frac{1}{\ell} \sum_{i=1}^\ell \mutualinfo{Z_i}{Y} \geq 1 - h\left(\frac{1}{\ell} \sum_{i=1}^\ell \Pr[Z_i \neq \hat{Z_i}]\right) \geq 1 - h\left(0.41\right) \geq \frac{1}{100}.
   \end{align*}
   The first inequality is due to the concavity of the binary entropy function $h$.
\end{proof}
\section{Lower bound for tomography with Pauli measurements}
The key to proving a tight lower bound for Pauli measurements is to design a measurement-dependent hard instance. Recall that any quantum state $\rho$ is a linear combination of Pauli observables,
\[
\rho=\frac{\eye_\dims}{\dims}+\sum_{P\in\pauliObsSet}\frac{\Tr[\rho P]}{\dims}P.
\]

Further recall the observation \cref{sec:techniques} that Pauli measurements \eqref{equ:pauli-measurement} are better at learning information about directions 
$P\in \pauliObsSet$ with a small weight and less powerful $P$ with a larger weight. 
% Consider $P=\pauliX^{\otimes \nqubits}$ and its corresponding Pauli basis measurement $\POVM_P$. The measurement simultaneously provides information about $\nqubits$ weight-1 Pauli observables $Q=\sigma_1\otimes\cdots\otimes \sigma_\nqubits$ such that $\sigma_i=\pauliX$ for some $i$ and $\sigma_j=\eye_2$ everywhere else. However, it only provides information about 1 Pauli observable with weight $N$, which is exactly $\sigma_X^{\otimes\nqubits}$.
As such, we set the basis $V_1, \ldots, V_{\dims^2-1}$ in the lower bound construction (\cref{def:perturbation}) to be the (normalized) Pauli observables, sorted in increasing order of their weights, $w(V_1)\le w(V_2)\le \ldots \le w(V_{\dims^2-1})$. Applying \eqref{equ:avg-MI-partial} in \cref{thm:avg-MI-upper} and \cref{lem:avg-MI-lower},
\begin{equation}
    \frac{1}{100}\le \frac{1}{\ell}\sum_{i=1}^{\ell}\mutualinfo{\ptb_i}{\out^\ns}\le \frac{8 \ns c^2 \eps^2}{\ell^2} \sup_{\POVM\in\povmset}\sum_{i=1}^{\ell} \vadj{V_i}\Choi_{\POVM} \vvec{V_i} +16\exp\{-\alpha\dims\}\ns c^2\eps^2.
    \label{equ:pauli-lower-inequ}
\end{equation}


We need to choose an appropriate $\ell$ and to upper bound the average mutual information. We propose to select all Pauli observables with weight at least $N-w$. Then,
\begin{equation}
    \ell = g(w)\eqdef \sum_{m=0}^w{\nqubits \choose N-m}3^{N-m}.
    \label{equ:pauli-weight-number}
\end{equation}
This is because for Pauli observables with weight $N-m$, there are $N-m$ positions we can place the Pauli operators, and for each position, there are three choices $\pauliX, \pauliY, \pauliZ$.



According to \cref{prop:perturbation-trace-distance}, we must choose $\ell\ge \dims^2/2$ to ensure that the perturbations are $\eps$ far from $\qmm$ with high probability. In other words, $g(w)/\dims^2\ge 1/2$.
\[
\frac{g(w)}{\dims^2}=\sum_{m=0}^w{\nqubits \choose N-m}\frac{3^{N-m}}{4^\nqubits}=\sum_{m=0}^{w}{\nqubits \choose m}\Paren{\frac34}^{\nqubits-m}\Paren{\frac14}^m =\probaOf{\binomial{\nqubits}{1/4}\le w}.
\]
We have the following fact about the median of binomial distributions,
\begin{fact}[\cite{kaas1980mean}]
    The median of a binomial distribution $\binomial{N}{p}$ must lie in $[\lfloor Np\rfloor, \lceil Np\rceil]$.
\end{fact}
Thus, choosing $w=\lceil N/4\rceil$ guarantees that $g(w)/\dims^2\ge 1/2$. 

Next, we compute the inner product $\vadj{V_i} \Choi_{\POVM} \vvec{V_i}$. We first need to analyze the measurement information channel of Pauli measurements.

\begin{lemma}
\label{lem:pauli-mic-eigen}
    For $P=\sigma_1\otimes\cdots\otimes \sigma_{\nqubits}\in \Sigma^{\otimes \nqubits}$, let $\Luders_P$ be the measurement information channel of the Pauli measurement $\POVM_P$. Then for all Pauli observable $Q=\sigma_1'\otimes\cdots\otimes \sigma_{\nqubits}'\in(\Sigma\cup \eye_2)^{\otimes \nqubits}$, $Q$ is an eigenvector of $\Luders_P$ and
    \[
\Luders_P(Q)=Q\indic{\forall j\in[\nqubits], \sigma_j'\in\{\sigma_j,\eye_2\}}.
    \]
    In other words, the eigenvalue of ${Q}$ is 1 when the non-identity components of $Q$ match $P$, and 0 otherwise. 
\end{lemma}
\begin{proof}
 Let $\Luders_P$ be the measurement information channel of a Pauli measurement $\POVM_{P}$. From \cref{def:mic} and \cref{equ:pauli-measurement},
\begin{align*}
    \Luders_P(\cdot)&=\sum_{x\in\{-1,1\}^{\nqubits}}\frac{M_x^{P}}{\Tr[M_x^P]}\Tr[(\cdot)M_x^P]\\
    &=\sum_{x\in\{-1,1\}^{\nqubits}}M_x^{P}\Tr[(\cdot)M_x^P].
\end{align*}
The second step is because Pauli measurement is a basis measurement. Thus each $M_x^P=\qproj{u_x^P}$ where $\{\qbit{u_x^P}\}_{x\in\{-1,1\}^{\nqubits}}$ is an orthonormal basis, and $\Tr[M_x^P]=1$.

Let $Q=\sigma_1'\otimes\cdots\otimes\sigma_{\nqubits}'$. We want to argue that $Q$ is an eigenvector of $\Luders_P$.  
\begin{align*}
\Tr[M_x^PQ]&=\Tr\left[\bigotimes_{j=1}^{\nqubits}\frac{\eye_2+x_j\sigma_j}{2}\bigotimes_{j=1}^{\nqubits}\sigma_j'\right]\\
    &=\Tr\left[\bigotimes_{j=1}^{\nqubits}\frac{\sigma_j'+x_j\sigma_j\sigma_j'}{2}\right]\\
    &=\prod_{j=1}^{\nqubits}\frac{\Tr[\sigma_j']+x_j\Tr[\sigma_j\sigma_j']}{2}\\
    &=\prod_{j=1}^{N}(\indic{\sigma_j'=\eye_2}+x_j\indic{\sigma_{j}'=\sigma_j})
\end{align*}
The final step is due to \eqref{equ:pauli-property}.
If for some $j\in[N]$, $\sigma_j'\ne \eye_2$ and $\sigma_j'\ne\sigma_j$, then
\[
\Tr[M_x^PQ]=0\implies\Luders_P(Q)=0.
\]
In this case $Q$ is an eigenvector of $\Luders_P$ with eigenvalue of 0. If otherwise,
\begin{align*}
    \Luders_P(Q)&=\sum_{x\in\{-1,1\}^{\nqubits}}M_x^P\prod_{j=1}^{N}(\indic{\sigma_j'=\eye_2}+x_j\indic{\sigma_{j}'=\sigma_j})\\
    &=\sum_{x\in\{-1, 1\}^{\nqubits}}\bigotimes_{j=1}^{\nqubits}\frac{\eye_2+x_j\sigma_j}{2}(\indic{\sigma_j'=\eye_2}+x_j\indic{\sigma_{j}'=\sigma_j})\\
    &=\bigotimes_{j=1}^{\nqubits}\sum_{x_j\in\{-1,1\}}\frac{\eye_2+x_j\sigma_j}{2}(\indic{\sigma_j'=\eye_2}+x_j\indic{\sigma_{j}'=\sigma_j})\\
    &=\bigotimes_{j=1}^{\nqubits}\Paren{\indic{\sigma_j'=\eye_2}\sum_{x_j\in\{-1,1\}}\frac{\eye_2+x_j\sigma_j}{2}+\indic{\sigma_j'=\sigma_j}\sum_{x_j\in\{-1,1\}}\frac{x_j\eye_2+\sigma_j}{2}} \\
    &=\bigotimes_{j=1}^{\nqubits}(\eye_2\indic{\sigma_j'=\eye_2}+\sigma_j\indic{\sigma_{j}'=\sigma_j})\\
    &=\bigotimes_{j=1}^{\nqubits}\sigma_j'=Q.\qedhere
\end{align*}
\end{proof}

Let $P,Q$ be defined in \cref{lem:pauli-mic-eigen} and $\Choi_P$ be the matrix form of $\Luders_P$ given a Pauli measurement $\POVM_P$. An immediate corollary is that
\begin{equation}
    \frac{1}{\dims}\vadj{Q} {\Choi}_P \vvec{Q}=\indic{\forall j\in[\nqubits], \sigma_j'\in\{\sigma_j,\eye_2\}}.
    \label{equ:pauli-mic-sum}
\end{equation}

When $V_1, \ldots, V_{\dims^2-1}$ are the normalized Pauli observables sorted in increasing order of their weights, setting $\ell=g(w)$ (the number of Pauli observables with weight at least $N-w$), we have
\begin{align*}
    \sum_{i=1}^{\ell}\vadj{V_i} {\Choi}_P \vvec{V_i}=\sum_{m=0}^{w}{\nqubits\choose m}.
\end{align*}
This is because $\vadj{V_i} {\Choi}_P \vvec{V_i}=1$ 
only when $V_i$ has non-identity components that match the ones in $P$ and 0 otherwise. 
There are only ${\nqubits\choose N-m}={\nqubits\choose m}$ of them among all $V_i$'s with weight $\nqubits-m$.

The following result gives an upper bound on the sum of binomial coefficients,
\begin{lemma}[{\cite[Lemma 16.19]{downey2012parameterized}}]
\label{lem:sum-binomial-coef}
    Let $n\ge 1$ and $0\le q\le 1/2$, then
    \[
    \sum_{i=0}^{\lfloor nq\rfloor}{n\choose i}\le 2^{n h(q)},
    \]
    where $h(q)=-q\log q -(1-q)\log(1-q)$ is the binary entropy function.
\end{lemma}

Combining with \eqref{equ:pauli-lower-inequ}\eqref{equ:pauli-weight-number}\eqref{equ:pauli-mic-sum}, setting $w=\lceil N/4\rceil$,
\begin{align*}
    \frac{1}{100}&\le \frac{8 \ns c^2 \eps^2}{\ell^2} \sup_{P\in\Sigma^{\otimes\nqubits}}\sum_{i=1}^{\ell}\vadj{V_i} {\Choi}_P \vvec{V_i} +16\exp\{-\alpha\dims\}\ns c^2\eps^2\\
    &=8nc^2\eps^2\Paren{\frac{\sum_{m=0}^{w}{\nqubits\choose m}}{g(w)^2}+2\exp\{-\alpha\dims\}}\\
    &\le 8nc^2\eps^2\Paren{\frac{2\cdot 2^{\nqubits h(1/4)}}{\dims^4/4}+2\exp\{-\alpha\dims\}}\\
    &\le 16\ns \cd^2\eps^2\Paren{4\cdot 2^{(h(1/4)-4)\nqubits}+\exp\{-\alpha2^{\nqubits}\}}.
\end{align*}
When $N\ge 10$, the second term is negligible. Rearranging the terms, we must have
\[
\ns = \bigOmega{\frac{2^{(4-h(1/4))\nqubits}}{\eps^2}}.
\]
Finally, noting that $2^{4-h(1/4)}\ge 9.118$ completes the proof.

\section{Upper bound for Pauli measurements}
\label{sec:pauli-upper}
This section starts with an observation about Pauli measurements, which is common knowledge for quantum information experimentalists. Then, we employ this observation to improve previous results about quantum state tomography using Pauli measurements.

\subsection{ An Observation about Pauli Measurements}
When we measure an element of the Pauli group, for instance, $\sigma_X\otimes \sigma_Y$, on a two-qubit state $\rho$, the outcome is a sample from a $4$-dimensional probability distribution, says $(p_{00},p_{01},p_{10},p_{11})$, such that
\begin{align*}
\tr(\rho(\sigma_X\otimes \sigma_Y))=p_{00}-p_{01}-p_{10}+p_{11}.
\end{align*}

One can easily observe that
\begin{align*}
\Tr[\rho(\sigma_X\otimes \sigma_I)]=p_{00}+p_{01}-p_{10}-p_{11},\\
\Tr[\rho(\sigma_I\otimes \sigma_Y)]=p_{00}-p_{01}+p_{10}-p_{11},\\
\Tr[\rho(\sigma_I\otimes \sigma_I)]=p_{00}+p_{01}+p_{10}+p_{11}.
\end{align*}


In other words, measuring $XY$, we obtained a sample of $\sigma_X\sigma_I$, a sample of $\sigma_I\sigma_Y$, and a sample of $\sigma_I\sigma_I$. 

For a general $n$-qubit system, we have the following observation.
\begin{observation}
For any $P=P_1\otimes P_2\otimes\cdots\otimes P_n\in\{\sigma_X,\sigma_Y,\sigma_Z\}^{\otimes N}$, the measurement result of performing measurement $P_i$ on the $i$-th qubit is an $N$-bit string $s$. One can interpret the measurement result of performing $Q_i\in\{\sigma_I,\sigma_X,\sigma_Y,\sigma_Z\}$ on the $i$-th qubit if $Q_i=P_i$ or $Q_i=\sigma_I$. We call those $Q=Q_1\otimes Q_2\otimes\cdots\otimes Q_N$'s correspond to $P$.
\end{observation}



\subsection{Algorithm and error analysis} 
Our measurement scheme is as follows: For any $\eps>0$, fix an integer $m$.
\begin{enumerate}
    \item  For any $P\in\{\sigma_X,\sigma_Y,\sigma_Z\}^{\otimes N}$, one performs $m$ times $P$ on $\rho$, and records the $m$ samples of the $2^N$ dimensional outcome distribution.


According to the key observation, this measurement scheme provides $m\cdot 3^{N-w}$ samples of the expectation $\tr(\rho P)$, say, $\frac{\mu_P}{m\cdot 3^{N-w}}$, for each Pauli operator $P\in \{\sigma_I,\sigma_X,\sigma_Y,\sigma_Z\}^{\otimes N}$ with weight $w$, where $-m\cdot 3^{N-w}\leq\mu_P\leq m\cdot 3^{N-w}$.

\item Output 
\begin{align*}
\sigma=\sum_P \frac{\mu_P}{m\cdot 3^{N-w}\cdot 2^N} P.
\end{align*}
\end{enumerate}


Using this scheme, we obtained $m\cdot 3^N$ independent samples,
\begin{align*}
X_1,X_2,\cdots, X_{m\cdot 3^N}.
\end{align*}
Each  $X_i$ is an $N$-bit string recording outcomes on all qubits (using bit 0 to denote the +1 eigenvalue and bit 1 to denote the -1 eigenvalue of the measured Pauli operator).   
Given that each operator is measured $m$ times, specifically, we assign that $X_1,X_2,\cdots,X_{m}$ correspond to the measurement $\sigma_X^{\otimes N}$, $X_{m+1},X_{m+2}$, and $\cdots,X_{2m}$ corresponds to the measurement $\sigma_X^{\otimes N-1}\otimes \sigma_Y$, \dots, and until $\sigma_Z^{\otimes N}$.

We observe that for any $P$ of weight $w$,
$\mu_P=\sum_{j=0}^{m\cdot 3^{N-w}-1} Z_j$,
where $Z_j$ are independent samples from the distribution $Z$
\begin{align*}
\mathrm{Pr}(Z=1)=\frac{1+\tr (\rho P)}{2}, \;
\mathrm{Pr}(Z=-1)=\frac{1-\tr (\rho P)}{2}.
\end{align*}
We have
\begin{align*}
\expectDistrOf{}{Z}&=\tr (\rho P), \expectDistrOf{}{Z^2}=1, \\
\expectDistrOf{}{\mu_P}&=m\cdot 3^{N-w}\cdot\tr (\rho P), \\
\expectDistrOf{}{\mu_P^2}&=\expectDistrOf{}{\mu_P}^2+\Var[\mu_P] \\
&=\expectDistrOf{}{\mu_P}^2+m\cdot 3^{N-w}\Var[Z] \\
&=m^2\cdot 9^{N-w}\cdot \tr^2 (\rho P)+m\cdot 3^{N-w}(1-\tr^2 (\rho P)).
\end{align*}
% Furthermore, $Z_j$ can be obtained from samples
% \begin{align*}
% X_1,X_2,\cdots, X_{m\cdot 3^N}.
% \end{align*}
% Therefore,
% \begin{align*}
% \sigma=\sum_P \frac{\mu_P}{m\cdot 3^{N-w_P}\cdot 2^N} P.
% \end{align*}
% is defined according to the samples
% \begin{align*}
% X_1,X_2,\cdots, X_{m\cdot 3^N}.
% \end{align*}
Thus, we can verify that
\begin{align*}
\expectDistrOf{}{\sigma}=\rho,
\end{align*}
where the expectation is taken over the probabilistic distribution according to the measurements.

For convenience, we define the function $f:X_1\times X_2\times \cdots\times X_{m\cdot 3^N}\mapsto \mathbb{R}$
\begin{align*}
f(\sigma)=\hsnorm{\rho-\sigma}=\sqrt{{\rm Tr}[(\rho-\sigma)^\dagger (\rho-\sigma)]}.
\end{align*}
Note that we can write the unknown state $\rho$ as
\begin{align*}
\rho=\sum_P \frac{\alpha_P}{2^N} P.
\end{align*}
According to Cauchy-Schwarz and Jensen's inequality, we have
\begin{align*}
&\expectDistrOf{}{f(\sigma)} \leq\sqrt{\expectDistrOf{}{f(\sigma)^2}} = \sqrt{\expectDistrOf{}{\tr\rho^2-2\tr \rho\sigma+\tr\sigma^2}}\\
=& \sqrt{\expectDistrOf{}{\tr\sigma^2-\tr\rho^2}}
= \sqrt{\frac{1}{2^N}\sum_P \expectDistrOf{}{\frac{\mu_P^2}{m^2\cdot 9^{N-w_P}}-\alpha_P^2}} \\
=& \sqrt{\frac{1}{2^N}\sum_P \left(\frac{m^2\cdot 9^{N-w_P}\cdot \alpha_P^2+m\cdot 3^{N-w_P}(1-\alpha_P^2)}{m^2\cdot 9^{N-w_P}}-\alpha_P^2\right)} \\
=&\sqrt{\frac{1}{m\cdot 2^N}\cdot{\sum_P \frac{1-\alpha_P^2}{3^{N-w_P}}}} < \sqrt{\frac{1}{m\cdot 2^N}\cdot{\sum_P \frac{1}{3^{N-w_P}}}}\\
=& \sqrt{\frac{1}{m\cdot 2^N}\cdot{\sum_{w_P=0}^N \frac{1}{3^{N-w_P}}{{N}\choose{w_P}}3^{w_P}}} = \sqrt{\frac{1}{m\cdot 6^N}\cdot (1+9)^N} \\
=& \sqrt{\frac{5^{\nqubits}}{m\cdot 3^N}}.
\end{align*}

For any sample $X_i$ corresponding to $P\in\{\sigma_X,\sigma_Y,\sigma_Z\}^{\otimes N}$, if only $X_i$ is changed, $\mu_Q$ would be changed only for those $Q\in\{\sigma_I,\sigma_X,\sigma_Y,\sigma_Z\}^{\otimes N}$ where
$Q$ is obtained by replacing some $\{\sigma_X,\sigma_Y,\sigma_Z\}$'s of $P$ by $\sigma_I$.
Moreover, the resultant value of $\mu_Q$ would change by two at most.
According to the triangle inequality, $f$ would change at most 
\begin{align*}
\hsnorm{\sum_Q  \frac{\Delta\mu_Q}{m\cdot 3^{N-w_Q}\cdot 2^N} Q}&=\sqrt{\sum_Q \frac{\Delta\mu_Q^2}{m^2\cdot 9^{N-w_Q}\cdot 2^N}}\\
&\leq \sqrt{\sum_Q \frac{2^2}{m^2\cdot 9^{N-w_Q}\cdot 2^N}} \\
&= \sqrt{\sum_{w_Q=0}^N \frac{2^2}{m^2\cdot 9^{N-w_Q}\cdot 2^N}{{N}\choose{w_Q}}} = \frac{2\cdot \sqrt{5}^N}{m\cdot 3^N},
\end{align*}
where $Q$ ranges over all Paulis which correspond to $P$'s, and $\Delta\mu_Q$ denotes the difference of $\mu_Q$ when $X_i$ is changed.

We use McDiarmid's inequality to bound the probability of success.
\begin{lemma}\label{mc}
Consider independent random variables ${\displaystyle X_{1},X_{2},\dots X_{n}}$ on probability space $ {\displaystyle (\Omega ,{\mathcal {F}},{\text{P}})}$ where ${\displaystyle X_{i}\in {\mathcal {X}}_{i}}$ for all ${\displaystyle i}$ and a mapping ${\displaystyle f:{\mathcal {X}}_{1}\times {\mathcal {X}}_{2}\times \cdots \times {\mathcal {X}}_{n}\rightarrow \mathbb {R} }$. Assume there exist constant $ {\displaystyle c_{1},c_{2},\dots ,c_{n}} $ such that for all $ {\displaystyle i}$,
%\begin{widetext} 
\begin{align}{\displaystyle {\underset {x_{1},\cdots ,x_{i-1},x_{i},x_{i}',x_{i+1},\cdots ,x_{n}}{\sup }}|f(x_{1},\dots ,x_{i-1},x_{i},x_{i+1},\cdots ,x_{n})-f(x_{1},\dots ,x_{i-1},x_{i}',x_{i+1},\cdots ,x_{n})|\leq c_{i}.} 
\end{align}
%\end{widetext}
In other words, changing the value of the ${\displaystyle i}$-th coordinate ${\displaystyle x_{i}}$ changes the value of ${\displaystyle f}$ by at most ${\displaystyle c_{i}}$. Then, for any ${\displaystyle \epsilon >0}$,
%\begin{widetext} 
\begin{align} 
{\displaystyle {\mathrm{Pr}}(f(X_{1},X_{2},\cdots ,X_{n})-\expectDistrOf{}{f(X_{1},X_{2},\cdots ,X_{n})}\geq \epsilon )\leq \exp \left(-{\frac {2\epsilon ^{2}}{\sum _{i=1}^{n}c_{i}^{2}}}\right)} .
\end{align}
%\end{widetext}
\end{lemma}
We only consider $\delta<1/3$, then $\log(1/\delta)>1$.
For any $\eps'>0$, by choosing $m=(3+2\sqrt{2})\cdot\frac{5^N\log\frac{1}{\delta}}{3^N\cdot \eps'^2}$, we have $\expectDistrOf{}{f(\sigma)}< (\sqrt{2}-1){\eps'}$.
Therefore,
\begin{align*}
\mathrm{Pr}(f(\sigma) > \eps') &< \mathrm{Pr}(f(\sigma)-\expectDistrOf{}{f(\sigma)}>(2-\sqrt{2}){\eps'}) \\
&<\exp(-\frac{(12-8\sqrt{2})\cdot \eps'^2}{4\cdot \frac{5^N}{m^2\cdot 9^N}\cdot m\cdot 3^N}) \\
=& \exp(-\frac{m\cdot(3-2\sqrt{2})\cdot 3^N\cdot\eps'^2}{5^N}) <\delta, 
\end{align*}
where the inequality is by \cref{mc}.

For a general quantum state and $\eps>0$, we let $\eps'=\frac{\eps}{\sqrt{2^N}}$, and know that
$||\rho-\sigma||_1>\eps$ implies $\hsnorm{\rho-\sigma}>\eps'$. Therefore,
\begin{align*}
\mathrm{Pr}(||\rho-\sigma||_1>\eps) &\leq \mathrm{Pr}(\hsnorm{\rho-\sigma}>\eps')=\mathrm{Pr}(f>\eps').
\end{align*}

The total number of used copies is 
\begin{align*}
\ns = m\cdot 3^N=(3+2\sqrt{2})\cdot\frac{10^N\log\frac{1}{\delta}}{\eps^2}.
\end{align*}



\newcommand{\MUB}{\POVM_{MUB}}
\section{Upper bound for tomography with finite outcomes}
\label{sec:finite-upper}
We will show the tightness of the adaptive tomography bounds for k-outcome POVMs by modifying the Projected Least Squares Method (PLS)~\cite{guctua2020fast} to work with k-outcome POVMs. We present these adjustments for the case when $k = d$ and $k < d$. As a result, we will have the first upper and lower bounds for adaptive tomography for k-outcome measurements, where the upper bound is achieved with non-adaptive algorithms. The key component is reducing the $\MUB$ to a k outcome measurement,
$$\MUB \eqdef \left\{\frac{1}{d+1} \ket{\psi_x^k} \bra{\psi_x^k}\right\}_{k \in [d+1], x \in [d]},$$
where each fixed $k$ corresponds to each one of the Maximally mutually unbiased bases. The reduction will follow similarly to~\cite{liu2024restricted}.
\subsection{Algorithm for $k=d$} \label{sub-keqd}
Measuring with $\MUB$ acts as a uniform sampling $i \sim Unif([d+1])$ to select one of the MUB and measuring with the POVM described by $\left\{\ket{\psi_x^i} \bra{\psi_x^i}\right\}_{x \in [d]}$. So, we can split each of the MUB bases across the multiple copies and uniformly sample amongst them to replicate the outcome distribution of measuring with $\MUB$. 

\begin{algorithm}
\caption{Finite Outcome Tomography for $k = d$}\label{alg:tom-keqd}
\hspace*{0.1cm} \textbf{Input:} $n$ copies of state $\rho$ \\
\hspace*{0.1cm} \textbf{Output:} Estimate $\hat{\rho} \in \mathcal{C}^{d \times d}$
\begin{algorithmic}
\State Divide $\MUB$ into $d+1$ groups of d-outcome measurements $\mathcal{M}_j := \left\{\ket{\psi_x^j} \bra{\psi_x^j}\right\}_{x \in [d]}$.
\State Divide $n$ copies into $d+1$ equally sized groups, each group has $n_0 = n/(d+1)$ copies.
\For {$j = 1, ..., d+1$}
\State For group $j$, apply $\mathcal{M}_j$. Let the outcomes be $x_1^{(j)}, ..., x_{n_0}^{(j)}$.
\EndFor
\State Generate n/2 i.i.d samples from $Unif([d+1])$ and  let $m_j$ be the number of times $j$ appears.
\State Let $x = (x_1, ..., x_{d+1})$ where $x_j = (x_1^{(j)}, ..., x_{\min\{n_0, m_j\}}^{(j)}$)
\State From $x$, obtain empirical frequencies $F = (f_1, ..., f_{d (d+1)})$ by obtaining group specific frequencies of each $x_i$ and concatenating the frequency vectors together.
\State \Return $\hat{\rho} = PLS(F)$
\end{algorithmic}
\end{algorithm}

For the analysis, we will use the multiplicative Chernoff Bound for sums of i.i.d random variables.
\begin{lemma}[Multiplicative Chernoff Bound]
    \label{mult_chernoff}
   Let $X_1, ..., X_n$ be i.i.d with $\expectDistrOf{}{X_1} = \mu$. Then, 
   $$\Pr\left[\sum_{i}^n X_i \geq n(1+\alpha)\mu\right] \leq \exp{\left\{-\frac{n \alpha^2 \mu}{2 + \alpha}\right\}}\;, \alpha > 0$$ 
   $$\Pr\left[\sum_{i}^n X_i \geq n(1-\alpha)\mu\right] \leq \exp{\left\{-\frac{n \alpha^2 \mu}{2}\right\}}\;, \alpha \in (0,1)$$ 
\end{lemma}
\begin{theorem} \label{thm_keqd}
    For $k=d$, Algorithm \cref{alg:tom-keqd} will give estimate $\hat{\rho}$ such that $\Pr[\tracenorm{\hat{\rho} - \rho} \leq \eps] \geq \frac{2}{3}$ with $n = O\left( \frac{d^3 \log d}{\eps^2}\right)$
\end{theorem}
\begin{proof}
Notice that each sample made will follow the outcome distribution of applying $\MUB$ to a single copy of $\rho$. 
Given $n$ copies, it will be shown that $\frac{n}{2}$ such samples will be made with sufficiently high probability. This is when $m_j \leq n_j$ for all $j \in [d+1]$. 
Using~\cref{mult_chernoff}, on the $m_j \sim Bin(\frac{n}{2}, \frac{1}{d+1})$, which is sum of $Y_1,...,Y_{\frac{n}{2}} \sim Bern(\frac{1}{d+1})$, $\mu = \expectDistrOf{}{Y_1} = \frac{1}{d+1}$,
\begin{align*}
    \Pr\left[m_j > n_j\right] = \Pr\left[\sum_{i=1}^\frac{n}{2} Y_i > 2 n \mu \right] \leq \exp{\left\{-\frac{n}{6(d+1)}\right\}}.
\end{align*}
Furthermore, by union bound,
\begin{align*}
    \Pr\left[\exists_j m_j > n_j\right] \leq \sum_{j=1}^{d+1} \Pr\left[m_j > n_j\right] \leq (d+1) \exp{\left\{-\frac{n}{6(d+1)}\right\}}.
\end{align*}
From previous work \cite{guctua2020fast}, we have the following guarantee on the estimation error using the PLS method using the outcome of $\MUB$ measurements,
\begin{align*}
    \Pr\left[\tracenorm{\hat{\rho}_n - \rho} \geq \eps \right] \leq d \exp{\left\{-\frac{n \eps^2}{86 d^3}\right\}}.
\end{align*}
With the algorithm, we can bound the probability of the estimate not being optimal,
\begin{align*}
\Pr\left[\tracenorm{\hat{\rho} - \rho} \geq \eps \right] &\leq \Pr\left[\exists_j m_j > n_j \lor \tracenorm{\hat{\rho}_{n/2} - \rho} \geq \eps\right] \\
&\leq (d+1) \exp{\left\{-\frac{n}{6(d+1)}\right\}} + d \exp{\left\{-\frac{n \eps^2}{172 d^3}\right\}}.
\end{align*}
With $d \geq 16$ and $n = \frac{172d^3 \ln{200d}}{\eps^2}  = O\left(\frac{d^3 \log{d}}{e^2} \right)$, we will have $\Pr\left[\tracenorm{\hat{\rho} - \rho} \le \eps \right] \geq \frac{99}{100}$
\end{proof}
\subsection{Algorithm for $k < d$}
For $\ab<\dims$, it is helpful to think of the problem as follows: there are $\ns$ players, each of whom holds a copy of $\rho$, but can only send $\log\ab$ classical bits to a central server that collects the messages and learn about the state. 

The idea is then to simulate each $\dims$-outcome POVM using only $\log\ab$ bits for each player. 
Using results from \cite{ACT:19:IT2}, the number of players (or copies) required to simulate the original $\dims$-outcome POVM is roughly $O(\dims/\ab)$, and thus we have a $O(\dims/\ab)$ factor blow up in the sample complexity compared to $\dims$-outcome measurements.

\begin{definition} [$\eta$-Simulation]
\label{def-simulate}
We are given $n$ players each with i.i.d sample from an unknown distribution $\p \in \Delta_d$. Each player can only send $w$ bits to the server. The server can then perform a $\eta$-simulation where $\hat{X} = [d] \cup \{\perp\}$.
\begin{equation}
    \Pr[\hat{X} = x \mid \hat{X} \neq \perp] = p_x, \; \Pr[\hat{X} = \perp] \leq \eta
\end{equation}

It can be shown that there exists an algorithm that can perform a $\eta$ simulation with $O(\dims/\ab)$ players,

\end{definition}
\begin{theorem}[\cite{ACT:19:IT2}, Theorem IV.5]
\label{thm:simulation}
   For every $\eta \in (0,1)$, there exists an algorithm that $\eta$ simulates $p \in \Delta_d$ using 
   \begin{equation}
       M = 40 \left\lceil \log{\frac{1}{\eta}} \right\rceil \left\lceil \frac{d}{2^w - 1} \right\rceil
   \end{equation}
   players from the setting in \cref{def-simulate}. The algorithm only requires private randomness for each player.
\end{theorem}
Therefore, for each MUB measurement $\POVM$ we assign $M=O(\dims/\ab)$ players. 
Each player applies $\POVM$ to $\rho$ and compresses the outcome to $\log\ab$ bits using the simulation algorithm in \cref{thm:simulation}. 
This process is a valid $\ab$-outcome POVM. 
The server then can use the $M=O(\dims/\ab)$ messages to simulate the outcome of $\POVM$ applied to $\rho$.
From \cref{thm_keqd}, we need $\tildeO{\dims^3/\eps^2}$ simulated samples, and thus the total number of copies required to simulate those samples is $M=O(\dims/\ab)$ times more. The detailed proof is given in \cref{thm:k-outcome-tomography}.
% ect $O(\frac{d}{k})$ multiplicative factor on the copy complexity from \cref{sub-keqd}, as each simulation of a per copy measurement will require $O(\frac{d}{k})$ samples. 

% \begin{algorithm}
% \caption{Finite Outcome Tomography for k < d}\label{alg:tom-klessd}
% \hspace*{0.1cm} \textbf{Input:} $n$ copies of state $\rho$ \\
% \hspace*{0.1cm} \textbf{Output:} Estimate $\hat{\rho} \in \mathcal{C}^{d \times d}$
% \begin{algorithmic}
% \State Divide $\MUB$ into $d+1$ groups of d-outcome measurements $\mathcal{M}_j := \left\{\ket{\psi_x^j} \bra{\psi_x^j}\right\}_{x \in [d]}$.
% \State Divide $n$ copies into $d+1$ equally sized groups, each group has $n_0 = n/(d+1)$ copies.
% \For {$j = 1, ..., d+1$}
% \State For group $j$, apply $\mathcal{M}_j$. Let the outcomes be $y_1^{(j)}, ..., y_{n_0}^{(j)}$.
% \State Perform distributed simulation: Splitting up into $\frac{n_0}{M}$ groups, each group producing a single sample. 
% \State Let the non-aborted samples be $x_1^{(j)}, ..., x_{n_j}^{(j)}$
% \EndFor
% \State Generate n/2 i.i.d samples from $Unif([d+1])$ and  let $m_j$ be the number of times $j$ appears.
% \State Let $x = (x_1, ..., x_{d+1})$ where $x_j = (x_1^{(j)}, ..., x_{\min\{n_j, m_j\}}^{(j)}$)
% \State From $x$, obtain empirical frequencies $F = (f_1, ..., f_{d (d+1)})$ by computing group specific frequencies of each $x_i$ and concatenating the frequency vectors together.
% \State \Return $\hat{\rho} = PLS(F)$
% \end{algorithmic}
% \end{algorithm}

\begin{theorem}
\label{thm:k-outcome-tomography}
For $\ab < \dims$, Algorithm 1 with distributed simulation will give estimate $\hat{\rho}$ such that $Pr[\tracenorm{\rho-\hat{\rho}} \le \eps ]\ge 0.99$ with  $\ns = \bigO{\frac{\dims^4\log\dims}{\ab\eps^2}}$.
\end{theorem}
\begin{proof}
   The proof will follow the same steps as \cref{thm_keqd}, but also considering $n_j \sim Bin(1-\eta,n_0/M)$ taking the role of $n_0$. Since $n_j$ and $m_j$ are both Binomial random variables, it is enough to say that $m_j$ and $n_j$ are on the opposite sides of a mean threshold with exponentially decreasing probability. Denote $\hat{n}_0 = \frac{n_0}{M}$ and $\hat{n} = \frac{n}{M}$,
   \begin{align*}
       \Pr\left[m_j \leq n_j\right] &\geq \Pr\left[m_j \leq \frac{3}{4} \hat{n}_0 \land n_j \geq \frac{3}{4} \hat{n}_0\right] \\
       \Pr\left[m_j > n_j\right] &< \Pr\left[m_j > \frac{3}{4} \hat{n}_0 \lor n_j < \frac{3}{4} \hat{n}_0\right].
   \end{align*}
   We will now bound each of the above union events with exponentially decreasing probability. We will apply \cref{mult_chernoff} on $m_j \sim Bin(\hat{n}/2, \frac{1}{d+1}), \expectDistrOf{}{m_j} = \hat{n}_0/2$ with $\alpha = 1/2$,
   \begin{align*}
       \Pr\left[m_j > \frac{3}{4} \hat{n}_0\right] \leq \exp{\left\{- \frac{\hat{n}}{10(d+1)}\right\}}.
   \end{align*}
   Now, we will apply \cref{mult_chernoff} once more for $n_j \sim Bin(1-\eta,\hat{n}_0), \expectDistrOf{}{n_j} = (1-\eta) \hat{n}_0$ and $\alpha = \frac{1/4 + \eta}{\eta} \geq \frac{1}{4}$,
   \begin{align*}
          \Pr\left[n_j > \frac{3}{4} \hat{n}_0\right] \leq \exp{\left\{- \frac{\hat{n}}{32(d+1)}\right\}}.
   \end{align*}
   Thus, 
   \begin{align*}
       \Pr\left[m_j > n_j \right] &\leq \Pr\left[m_j > \frac{3}{4} \hat{n}_0 \right] + \Pr\left[n_j < \frac{3}{4} \hat{n}_0\right] \\
       &\leq 2 \exp{\left\{- \frac{\hat{n}}{32(d+1)}\right\}}.
   \end{align*}
   By union bound,
   \begin{align*}
       \Pr\left[\exists_j m_j > n_j \right] \leq (d+1) \exp{\left\{- \frac{\hat{n}}{32(d+1)}\right\}}.
   \end{align*}
   Now we will repeat the argument from \cref{sub-keqd} for plugging in the samples into the PLS estimator,
   \begin{align*}
       \Pr\left[\tracenorm{\hat{\rho} - \rho} > \eps \right] &\leq \Pr\left[\exists_j m_j > n_j \cup \tracenorm{\hat{\rho}_{\hat{n}/2} - \rho} > \eps\right] \\
       &\leq (d+1) \exp{\left\{-\frac{\hat{n}}{32(d+1)}\right\}} + d \exp{\left\{-\frac{\hat{n} \eps^2}{172 d^3}\right\}}.
   \end{align*}
With $d \geq 16$ and $\hat{n} = \frac{172d^3 \ln{200d}}{\eps^2}$, we will have $\Pr\left[\tracenorm{\hat{\rho} - \rho} \le \eps \right] \geq \frac{99}{100}$. With $w = \log k$ and $\eta = 0.01$, we have that $\hat{n} = \Theta(\frac{k}{d}) \cdot n$, so $n = O(\frac{d^4 \log{d}}{k \eps^2} )$.
\end{proof}
Thus, the upper bound can be compactly written as $O\left(\frac{\dims^4 \log{d}}{\eps^2\min\{\ab, \dims\}}\right)$, combining the $k=d$ and $k < d$ cases. With \cref{cor:finite-lower}, we have proven \cref{thm:nearly-tight-finite-out}.

\begin{remark}
    We note that running the distributed simulation with $\log\ab$ bits requires first obtaining the $\dims$ outcomes for each qubit. Thus, the algorithm is more relevant in the distributed setting as described in this section. Nevertheless, the compression step for each copy defines a valid $\ab$-outcome measurement and thus proves that our lower bound in~\cref{cor:finite-lower} is tight.
\end{remark}


\section*{Acknowledgements}

JA and YL were partially supported by NSF award 1846300 (CAREER), NSF CCF-1815893. AD was supported by a Cornell University Graduate Fellowship. YL was also supported by a Rice University Chairman postdoctoral fellowship. NY was supported by DARPA  SciFy Award 102828.

\bibliography{refs}
\bibliographystyle{alpha}



\end{document}
