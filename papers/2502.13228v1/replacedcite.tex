\section{Related Work}
\paragraph{Statistical Prediction Analysis.} Statistical prediction analysis____ deals with the use of statistical inference to reason about the likely outcomes of future prediction tasks given past ones. Within statistical prediction analysis, the area of distribution-free prediction assumes that the parameters or the form of the distributions involved cannot be identified. This idea can be traced back to ____, who constructed a method to form distribution-free tolerance regions. ____ generalized distribution-free tolerance regions and introduced the concept of statistically equivalent blocks, which are analogous to the intervals between consecutive order statistics of the losses. Much of the relevant theory is summarized by____, and the Dirichlet distribution of quantile spacing is discussed by____. We build upon these works by connecting them to Bayesian quadrature and applying them in the more modern context of distribution-free uncertainty quantification.

\paragraph{Bayesian Quadrature.} The use of Bayesian probability to represent the outcome of a arbitrary computation is termed \emph{probabilistic numerics}____. Since our approach is fundamentally based on integration, we focus primarily on the relationship with the more narrow approach of Bayesian quadrature, which employs Bayes rule to estimate the value of an integral. A lucid overview of this approach is discussed under the term \emph{Bayesian numerical analysis} by ____, who traces it back to the late nineteenth century ____. The use of Gaussian processes in performing Bayesian quadrature is discussed in detail by ____. Our approach is formulated similarly but differs in two main ways: (a) we use a conservative bound instead of an explicit prior, and (b) we have input noise induced by the random quantile spacings.

\paragraph{Distribution-Free Uncertainty Quantification.}
Relevant background on distribution-free uncertainty quantification techniques is discussed in Section~\ref{sec:background_conformal}. A recent and comprehensive introduction to conformal prediction and related techniques may be found in____. Some recent works, like ours, also make use of quantile functions____ but remain grounded in frequentist probability.  Separately, Bayesian approaches to predictive uncertainty are popular____ but make extensive assumptions about the form of the underlying predictive model. To our knowledge, we are the first to apply statistical prediction analysis and Bayesian quadrature in order to analyze the performance of black-box predictive models in a distribution-free way.