\documentclass{article}

\usepackage{microtype}
\usepackage{graphicx}
\usepackage{subfigure}
\usepackage{booktabs} %

\usepackage{hyperref}


\newcommand{\theHalgorithm}{\arabic{algorithm}}


\usepackage[accepted]{icml2025}

\usepackage{amsmath}
\usepackage{amssymb}
\usepackage{mathtools}
\usepackage{amsthm}

\usepackage[capitalize,noabbrev]{cleveref}

\usepackage{amsmath}
\usepackage{thmtools}
\usepackage{bbm}
\usepackage{epigraph}
\usepackage{threeparttable}
\usepackage[group-separator={,}]{siunitx}

% This must be in the first 5 lines to tell arXiv to use pdfLaTeX, which is strongly recommended.
\pdfoutput=1
% In particular, the hyperref package requires pdfLaTeX in order to break URLs across lines.

\documentclass[11pt]{article}

% Change "review" to "final" to generate the final (sometimes called camera-ready) version.
% Change to "preprint" to generate a non-anonymous version with page numbers.
\usepackage[preprint]{acl}
\usepackage{booktabs}
\usepackage{amsfonts}
\usepackage{amsmath}
\usepackage{multirow}
\usepackage{amsthm}
\usepackage{algorithm}
\usepackage{algorithmic}
\newtheorem{theorem}{Theorem}[section]
\newtheorem{assumption}{Assumption}[section]
\newtheorem{definition}{Definition}[section]
\newtheorem{proposition}{Proposition}[section]
\newtheorem{corollary}{Corollary}[theorem]
\newtheorem{lemma}[theorem]{Lemma}
\newtheorem*{remark}{Remark}
% Standard package includes
\usepackage{times}
\usepackage{latexsym}

% For proper rendering and hyphenation of words containing Latin characters (including in bib files)
\usepackage[T1]{fontenc}
% For Vietnamese characters
% \usepackage[T5]{fontenc}
% See https://www.latex-project.org/help/documentation/encguide.pdf for other character sets

% This assumes your files are encoded as UTF8
\usepackage[utf8]{inputenc}

% This is not strictly necessary, and may be commented out,
% but it will improve the layout of the manuscript,
% and will typically save some space.
\usepackage{microtype}

% This is also not strictly necessary, and may be commented out.
% However, it will improve the aesthetics of text in
% the typewriter font.
\usepackage{inconsolata}

%Including images in your LaTeX document requires adding
%additional package(s)
\usepackage{graphicx}

% If the title and author information does not fit in the area allocated, uncomment the following
%
%\setlength\titlebox{<dim>}
%
% and set <dim> to something 5cm or larger.

\title{A statistically consistent measure of Semantic Variability using Language Models}

% Author information can be set in various styles:
% For several authors from the same institution:
% \author{Author 1 \and ... \and Author n \\
%         Address line \\ ... \\ Address line}
% if the names do not fit well on one line use
%         Author 1 \\ {\bf Author 2} \\ ... \\ {\bf Author n} \\
% For authors from different institutions:
% \author{Author 1 \\ Address line \\  ... \\ Address line
%         \And  ... \And
%         Author n \\ Address line \\ ... \\ Address line}
% To start a separate ``row'' of authors use \AND, as in
% \author{Author 1 \\ Address line \\  ... \\ Address line
%         \AND
%         Author 2 \\ Address line \\ ... \\ Address line \And
%         Author 3 \\ Address line \\ ... \\ Address line}

\author{Yi Liu \\
  Seattle, Washington, USA \\
  %\texttt{liuyi3@microsoft.com} 
}

%\author{
%  \textbf{First Author\textsuperscript{1}},
%  \textbf{Second Author\textsuperscript{1,2}},
%  \textbf{Third T. Author\textsuperscript{1}},
%  \textbf{Fourth Author\textsuperscript{1}},
%\\
%  \textbf{Fifth Author\textsuperscript{1,2}},
%  \textbf{Sixth Author\textsuperscript{1}},
%  \textbf{Seventh Author\textsuperscript{1}},
%  \textbf{Eighth Author \textsuperscript{1,2,3,4}},
%\\
%  \textbf{Ninth Author\textsuperscript{1}},
%  \textbf{Tenth Author\textsuperscript{1}},
%  \textbf{Eleventh E. Author\textsuperscript{1,2,3,4,5}},
%  \textbf{Twelfth Author\textsuperscript{1}},
%\\
%  \textbf{Thirteenth Author\textsuperscript{3}},
%  \textbf{Fourteenth F. Author\textsuperscript{2,4}},
%  \textbf{Fifteenth Author\textsuperscript{1}},
%  \textbf{Sixteenth Author\textsuperscript{1}},
%\\
%  \textbf{Seventeenth S. Author\textsuperscript{4,5}},
%  \textbf{Eighteenth Author\textsuperscript{3,4}},
%  \textbf{Nineteenth N. Author\textsuperscript{2,5}},
%  \textbf{Twentieth Author\textsuperscript{1}}
%\\
%\\
%  \textsuperscript{1}Affiliation 1,
%  \textsuperscript{2}Affiliation 2,
%  \textsuperscript{3}Affiliation 3,
%  \textsuperscript{4}Affiliation 4,
%  \textsuperscript{5}Affiliation 5
%\\
%  \small{
%    \textbf{Correspondence:} \href{mailto:email@domain}{email@domain}
%  }
%}

\begin{document}
\maketitle
\begin{abstract}
To address the challenge of variability in the output generated by language models, we introduce a measure of semantic variability that remains statistically consistent under mild assumptions. This measure, termed semantic spectral entropy, is an easily implementable algorithm that requires only standard, pre-trained language models. Our approach imposes minimal restrictions on the choice of language models, and through rigorous simulation studies, we demonstrate that this method can produce an accurate and reliable metric despite the inherent randomness in language model outputs.
\end{abstract}

\section{Introduction}

\label{introduction}
{\color{white}..} The birth of Large Language Models (LLM) has given rise to the possibility of a wide range of industry applications \cite{touvron2023llama,chowdhery2023palm}. One of the key applications of generative models that has garnered significant interest is the development of specialized chatbots with domain-specific expertise such as legal and healthcare \cite{Lexis,mesko2023top}. These applications illustrate how generative models can improve decision-making and improve the efficiency of professional services in specialized fields.

This new LLM capability is made possible by the strong understanding of generative capabilities of the models \cite{liu2023mmc,long2023large} and the advent of Retrieval-Augmented Generation (RAG) \cite{lewis2020retrieval, gao2023retrieval}. In an RAG system, the user interacts by submitting queries, which trigger a search for relevant documents within a pre-established database. These pertinent documents are retrieved based on the query and serve as a context for the LLM to generate an appropriate response. Since the implementation of RAG does not require a custom-trained LLM, it offers a cost-effective solution. The resulting chatbot can perform tasks traditionally handled by domain experts, improving operational efficiency and driving cost reductions.

However, a critical challenge impeding the widespread deployment of generative models in industry is the inherent variability present in these models \cite{amodei2016concrete,hendrycks2021unsolved}.  Although parameters such as temperature, top-k, top-p, and repetition penalty are known to significantly influence model performance \cite{wang2020contextual,wang2023cost, song2024good}, even when these parameters are tuned to achieve deterministic output (e.g. setting temperature to 0 or top-p to 1), differences in the generated results can still occur in multiple runs. This persistent variability poses a significant barrier to the reliable and consistent application of generative models in practical settings.

Atil et al. (2024) conducted a series of experiments involving six deterministically configured large language models (LLMs), with temperature set to 0 and top p set to 1, across eight common tasks and five identical trials per task. The study aimed to assess the repeatability of model outputs by examining whether the generated strings were consistent between runs. The authors found that none of the LLMs demonstrated consistent performance in terms of generating identical outputs on all tasks \cite{atil2024llm}. 
%For complex tasks, such as college-level mathematics, the models often produced lexically different outputs for each run, leading to zero consistency in terms of exact string matching. 
However, the authors noted that when accounting for syntactical variations, the observed differences were relatively minor as many of the generated strings were semantically equivalent. 


The variability in output has been attributed to the use of GPUs in large language model (LLM) inference processes, where premature rounding during computations can lead to discrepancies \cite{nvidia2024,atil2024llm}. Given this, it is reasonable to conclude that complete elimination of variability is unfeasible in any empirical setting. Consequently, we must acknowledge that the output of LLMs is inherently uncertain. In light of this, it becomes essential, similar to practices in statistics, to assess and quantify the level of uncertainty in the text generated by LLMs for any given scenario. 

Most prior studies on uncertainty in foundation models for natural language processing (NLP) have focused primarily on the calibration of classifiers and text regressors \cite{jiang2021can, desai2020calibration, glushkova2021uncertainty}. Other research has addressed uncertainty by prompting models to evaluate their own outputs or fine-tuning generative models to predict their own uncertainty \cite{linteaching, kadavath2022language}. However, these approaches require additional training and supervision, making them difficult to reproduce, costly to implement, and sensitive to distributional shifts. 

 Our work follows from a line of work inline with the concept of semantic entropy proposed in \cite{kuhn2023semantic, nikitin2024kernel,duan-etal-2024-shifting,lin2023generating}. \cite{kuhn2023semantic} explore the entropy of the generated text by assigning semantic equivalence to the pairs of text and subsequently estimating the entropy. Similarly, \cite{nikitin2024kernel} and \cite{lin2023generating} utilize graphical spectral analysis to enhance empirical results. However, a notable limitation in the entropy estimators proposed by \cite{kuhn2023semantic} and \cite{nikitin2024kernel} is their reliance on token likelihoods when assessing semantic equivalence, which may not always be accessible. Furthermore, \cite{kuhn2023semantic} acknowledge that the clustering process employed in their framework is susceptible to the order of comparisons, introducing variability into the results. 

Moreover, previous work focuses on the empirical performance of the estimator. As such, while these methods have demonstrated favorable empirical outcomes, to the best of our knowledge, no authors have established using a theoretical analysis that their entropy estimators converge to a true entropy value as the sample size increases under an underlying generative model. Exploring the theoretical properties allows us to have a clear understanding of how the number of clusters and size of data would affect the estimator. 

Our approach seeks to address these limitations by developing a robust theoretical analysis of the clustering procedure, ensuring convergence properties, and mitigating the variability inherent in prior methodologies. We propose a theoretically analyzable metric for quantifying the variation within a collection of texts, which we refer to as semantic spectral entropy. This measure addresses the observation that many generated strings, while lexically and syntactically distinct, may convey equivalent semantic content. To identify these semantic equivalences, we advocate the use of off-the-shelf generative language models (LMs). Moreover, we acknowledge that the LM used to evaluate semantic similarity is itself a stochastic generator. In response, we employ the well-established technique of spectral clustering, which is provably consistent under minimal assumptions on the generator, thereby ensuring the robustness and reliability of the proposed metric. Specifically, we demonstrate that the measure is statistically consistent under a weak assumption on the LM. To the best of our knowledge, this is the first semantic variability measure with proven convergence properties. As an empirical evaluation studies, we also propose a simple method for constructing clusters of different lexically and syntactically distinct but semantically equivalent text using compound  propositions from \cite{wittgenstein2023tractatus}.

\section{Semantic spectral entropy}
\label{methodology}
\subsection{Semantic entropy}

{\color{white}..} We begin with a collection of textual pieces \( n \), denoted \( \mathcal{T} = (t_1, \cdots, t_n) \). Unlike that in \cite{kuhn2023semantic}, our assumption is that we have access only to $\mathcal{T}$. In fact, we do not require the existence of a generative model and is interested only in variability of the semantics in the text. To evaluate the semantic variability of these texts in the context of a specific use case, we propose a theoretically proven measure of semantic entropy which we named semantic spectral entropy. 

A key reason for opting against the use of variance as a measure of variability is that computing variance requires the definition of a mean, which is challenging to establish for semantic distributions. Although it is possible to define an arbitrary reference point, such as a standard answer in a chatbot that answers questions, evaluating the variability with respect to such a reference introduces bias. 

In contrast, entropy is a well-established measure of variation, particularly for multinomial distributions. For a distribution \( \mathcal{P}(t) \) over a set of semantic clusters \( \{C_1, \cdots, C_k\} \), the entropy \( \mathcal{E} \) is defined as:
\begin{equation}
\label{equ:entropy}
\mathcal{E}(t) = - \sum_{i} p(t \in C_i)\log p(t \in C_i).
\end{equation}
This formulation captures the uncertainty or disorder associated with assigning a given text \( t \) to one of the clusters. Consequently, it provides a quantitative measure of semantic variability that avoids the biases introduced by arbitrary reference points.

To estimate the entropy for a given data set \( t_1, \cdots, t_n \), we first calculate the number of occurrences of each text \( t_i \) in each group \( C_j \). This is achieved by computing:
\[
n_j = \sum_{i=1}^n \mathbb{I}(t_i \in C_j),
\]
where \( \mathbb{I}(t_i \in C_j) \) is an indicator function that equals 1 if \( t_i \) belongs to the cluster \( C_j \), and 0 otherwise. 

Next, the true probability \( p(t \in C_j) \) is approximated using the empirical distribution:
\[
\bar{p}(t \in C_j) = \frac{n_j}{n},
\]
which represents the fraction of texts assigned to cluster \( C_j \). Using this empirical distribution, the empirical entropy is defined as:
\[
\bar{\mathcal{E}}(\mathcal{T}) = - \sum_{j} \bar{p}(t \in C_j) \log \bar{p}(t \in C_j).
\]
This measure provides a practical estimation of semantic entropy based on observed data.


One critical step in this process is clustering the texts $t_i$ into disjoint groups. To do so, it is sufficient to define a relationship between $t_i \sim t_j$, such that they satisfy the properties of equivalence relation. Specifically, one needs to demonstrate 
\begin{enumerate}
    \item Reflexivity: For every $t_i$, we have $t_i \sim t_i$, meaning that any text is equivalent to itself.
    \item Symmetry: If $t_i \sim t_j$, then $t_j \sim t_i$, meaning that equivalence is bidirectional.
    \item Transitivity: If $t_i \sim t_j$ and $t_j \sim t_k$, then $t_i \sim t_k$, which means that equivalence is transitive.
\end{enumerate}

It turns out the existence of an equivalence equation is both a necessary and sufficient condition for a definition of a breakdown of $\mathcal{T}$ into disjoint clusters \cite{liebeck2018concise}. In light of this, defining $\sim$ should be based on the linguist properties of entropy measurement. 

 Direct string comparison, defined as \( t_i \sim t_j \) if and only if \( t_i \) and \( t_j \) share identical characters, reflects lexicon equality and constitutes an equivalence relation. However, this criterion is overly restrictive. In a question-and-response context, a more appropriate equivalence relation might be defined as \( t_i \sim t_j \) if and only if \( t_i \) and \( t_j \) yield identical scores when evaluated by a language model (LM) prompt. This criterion, however, requires an answer statement as a point of reference. We are more interested in a stand-alone metric that can capture the semantic equivalence. For example, consider the sentences \( t_1 = \text{"Water is vital to human survival"} \) and \( t_2 = \text{"Humans must have water to survive"}\). Despite differences in language, both sentences convey the same underlying meaning.

To address such challenges, \cite{kuhn2023semantic,nikitin2024kernel} propose an equivalence relation wherein \( t_i \sim t_j \) if and only if \( t_i \) is true if and only if \( t_j \) is true. This formulation ensures that two texts, \( t_i \) and \( t_j \), belong to the same equivalence class if they are logically equivalent. This broader definition allows for greater flexibility and applicability in assessing semantic equivalence beyond superficial lexical similarity.\cite{copi2016introduction}. We will present their argument as a proposition where we will put the verification in the appendix
\begin{proposition}
    \label{prop:equ}
    The relation $t_i \sim t_j$ if "$t_i$ is true if and only if $t_j$ is true" is an equiva
    
    lence relation.
\end{proposition}

% one considers segmenting the set into equivalence classes based on the following equivalence relation:  


%We first establish that this relation $\sim$ indeed defines well-defined, disjoint subsets of $t_i, i \in \{1,\dots n\}$. 



% Now, we can conclude that $\sim$ is indeed an equivalence relation, and thus, the set of texts $t_1\dots t_n$ can be partitioned into disjoint equivalence classes, where each class represents a distinct semantic group. 
%\begin{remark}
    
%\end{remark}
In light of the fact that equivalence relations can be defined arbitrarily based on the needs of the user. We propose that the determination of equivalence relations, denoted as $\sim$, is performed through a LM that generates responses independently of the specific generation of terms $t_1, \dots, t_n$. However, we do not assume that we have access to probability distribution of the tokens as proposed by \cite{kuhn2023semantic,nikitin2024kernel} which is not always available. Rather, we just require a generator LM which can generate a determination of this relationship. Therefore, this LM can be general generative language model with a crafted prompt which we will use in our simulation studies. The error in this LM will be removed in the spectral clustering algorithm at the later stage. By leveraging this LM, we can define a function $e:{\mathcal{T}, \mathcal{T}}\rightarrow {0,1}$, which is formally expressed as follows: \begin{equation} e(t_i, t_j) = \begin{cases} 1 & \text{if } t_i \sim t_j, \\ 0& \text{otherwise.} \end{cases} \end{equation}

However, since the function relies on an LM, $e(t_i, t_j)$ can be viewed as a Bernoulli random variable, whose value is dependent on the terms $t_i$ and $t_j$.\nocite{kuhn2023semantic} did not address this issue but instead offers adopting a very powerful entailment identification model which the authors trust to identify the equivalence relation perfectly. In contrast, we suggest modeling the outputs of the LM as a random graph with an underlying distribution. In this framework, $t_i$ and $t_j$ represent nodes, while $e(t_i, t_j)$ are random variables that indicate the presence of an edge between the two nodes. Specifically, when $t_i \sim t_j$, the edge existence is governed by the following probability distribution: \begin{equation} \label{eqn:equation_p} e(t_i, t_j) = \begin{cases} 1 & \text{with probability } p, \\ 0 & \text{with probability } 1-p. \end{cases} \end{equation} Conversely, when $t_i \not\sim t_j$, the edge existence follows a different probability distribution: \begin{equation} \label{eqn:equation_q} e(t_i, t_j) = \begin{cases} 1 & \text{with probability } q, \\ 0 & \text{with probability } 1-q. \end{cases} \end{equation}

To mitigate the inherent randomness introduced by the LLM, we propose leveraging spectral clustering to identify clusters of semantically similar texts.
\subsection{Spectral clustering}

{\color{white}..} To compute semantic entropy, it is crucial to identify the clusters of nodes and count the number of nodes within each cluster. Identifying these clusters in a random graph is analogous to detecting clusters in a stochastic block model \cite{holland1983stochastic}. We propose employing the spectral clustering algorithm, with the number of clusters $K$ specified in advance, as an effective approach for this task.

Spectral Clustering is a well-established algorithm for graph clustering, supported by strong theoretical foundations and efficient implementations \cite{shi2000normalized, lei2015consistency, su2019strong, scikit-learn}.  To compute semantic entropy, we aim to cluster a random graph with adjacency matrix $E$ where $E_{ij} = e(t_i, t_j)$, representing the pairwise similarity between text elements $t_i$ and $t_j$. 

We begin by computing the Laplacian matrix $L =  D-E$ where $D$ is the degree matrix.  This is followed by the decomposition of the eigenvalue of $L$. Next, we construct the matrix formed by the first $K$ eigenvectors of $L$ denoted $\hat{U} \in \mathbb{R}^{n\times K}$. This matrix serves as input to an appropriate $(1+\epsilon)-$ k-means clustering algorithm \cite{kumar2004simple,choo2020k}.

The output of this procedure is $K$ distinct clusters $C_1,\cdots C_K$. For each text element $t_i$, we assign a corresponding vector $g_{i}$ where 
$$ g_{ij} = \begin{cases} 1 \text{ if } t_i \in C_j\\
    0 \text{ otherwise }
\end{cases}$$
This binary indicator vector $g_i$ encodes the cluster membership for each text element $t_i$

Finally, we compute the estimated entropy based on the number of texts within each cluster. The entropy $\hat{\mathcal{E}}$ can be approximated using the following formula:
\begin{equation}
\hat{\mathcal{E}}(\mathcal{T}) = - \sum_{j=1}^k\hat{p}( C_j) \log(\hat{p}( C_j)),
\end{equation}
where $\hat{p}(C_j) = \frac{1}{n}\sum_{i=1}^n g_{ij}$.This expression represents the empirical entropy based on the distribution of texts among the $K$ clusters, providing a measure of the uncertainty or diversity within the semantic structure of the data.
\subsection{Full algorithm and implementation}
{\color{white}..} We merge the process of finding sermantic entropy with spectral clustering to present the full algorithm as Algorithm \ref{algo:1}: Sermantic Spectral Entropy. 
\begin{algorithm}
\begin{algorithmic}
    \STATE Begin with $\mathcal{T} = \{t_1, \cdots t_n\}$
    \FOR{$i, j \in \{1,\cdots n\} \times \{1, \cdots n\}, i\neq j$}
    \STATE Use LLM to compute $E_{i,j} = e(t_i, t_j)$. 
    \ENDFOR
    \STATE Find the Laplacian of $E$, $L = D -E$
    \STATE Compute the first $K$ eigenvectors $u_1,\dots,u_k$ of $L$ and the top $K$ eigenvalues $\lambda_1,\cdots \lambda_k$.
    \STATE Let $\hat{U} \in \mathbb{R}^{n\times k}$ be the matrix containing the vectors $u_1,\dots,u_k$ as columns.
    \STATE Use $(1+\epsilon)$ K-means clustering algorithm to cluster the rows of $U$
    \STATE Let $g_{ij}$ be an $(1+\epsilon)-$approximate solution to a $K-$means clustering algorithm
    \STATE Compute $\hat{\mathcal{E}}(\mathcal{T})$ using $ g_{ij}$
\end{algorithmic}
\caption{\label{algo:1} Sermantic Spectral Entropy }
\end{algorithm}

This polynomial-time algorithm is characterized by the largest computational cost associated with the determination of $E_{ij}$. However, computing $E_{ij}$ is embarrassingly parallel, meaning that it can be efficiently distributed across multiple processing units. Furthermore, there are well-established implementation, such as Microsoft Azure's Prompt-Flow \cite{esposito2024programming} and LangChain \cite{mavroudis2024langchain} that facilitate the implementation of parallel workflows, making it feasible to deploy such parallelized tasks with relative ease.
\subsection{Finding K}

{\color{white}..} A notable limitation of this analysis is the unavailability of $K$ in the direct computation of semantic spectral entropy. However, the determination of $K$ for stochastic block model has been well studied \cite{lei2016goodness,wang2017likelihood,chen2018network}. We will describe the cross-validation approach \cite{chen2018network} in detail. The principle behind cross-validation involves predicting the probabilities associated with inter-group connections ($p$) and intra-group connections ($q$). If the estimated value of $K$ is too small, it fails to accurately recover the true underlying probabilities; conversely, if $K$ is too large, it leads to overfitting to noisy data. This approach has the potential to recover the true cluster size under relatively mild conditions.

\section{Theoretical Results}
\label{theory}

{\color{white}..} Our theoretical analysis involves a proof that the estimator is strongly consistent, i.e. the estimator converges to true value almost surely, and an analysis of its rate with respect to the number of cluster $K$. 

We divide our analysis into two subsections. The first subsection examines a fixed set of $\mathcal{T} = {t_1, \dots, t_n}$, which is assumed to exhibit some inherent clusters $C_1, \dots, C_K$. Under the assumption of perfect knowledge of these clusters, the empirical entropy $\bar{\mathcal{E}}$ can be determined. The primary focus in this subsection is on the performance of spectral clustering algorithms. The second subsection explores a scenario in which there exists an underlying generative mechanism that allows for the infinite generation of $t_i$. In this case, we permit $K$ to increase with $n$, though at a significantly slower rate. This scenario is particularly relevant for evaluating the performance of RAG in the context of continuous generation of results in response to a given query.

\subsection{Performance of spectral clustering algorithms}
{\color{white}..}  We model the LM determination of $e(t_i,t_j)$ as a random variable, as described in Equations \ref{eqn:equation_p} and \ref{eqn:equation_q}. In the theoretical analysis presented here, we assume that the number of clusters, $K$, is known and fixed. To derive various results, we first establish the relationship between the difference $|\bar{\mathcal{E}}(\mathcal{T}) - \hat{\mathcal{E}}(\mathcal{T})|$ and the miscluster error, denoted $M_\text{error}$.
\begin{lemma}
 \label{lemma:error}
Suppose that there exists $0<c_2<1$ such that $2Kn_{\min}/n \geq c_2$, 
\begin{equation}
     |\hat{\mathcal{E}}(\mathcal{T}) - \bar{\mathcal{E}}(\mathcal{T})|\leq h\left(\frac{2K}{c_2}\right) \left|\frac{1}{n} (M_\text{error})\right| 
\end{equation}
where $h(x) = \left(x+\log\left(x\right)\right)$.
\end{lemma}
The proof is presented in the Appendix section \ref{Appendix:proofoflemma:error}. 
We begin by presenting the result of strong consistency for the spectral clustering algorithm.  

\begin{theorem}
\label{the:strongConsistensy}
Under regularity conditions, the estimated entropy empirical entropy $\hat{\mathcal{E}}(\mathcal{T})$ is strongly consistent with the empirical entropy, i.e. 
\begin{equation}
    |\bar{\mathcal{E}}(\mathcal{T}) - \hat{\mathcal{E}}(\mathcal{T}) | \rightarrow 0 \text{ almost surely }
\end{equation}
\end{theorem}
The proof is provided in the Appendix section \ref{appendix:sec:the:strongconsistency}. This establishes strong consistency result that we aim to present. At the same time, we also want to show the finite sample properties of the estimator $\hat{\mathcal{E}}(\mathcal{T})$.

\begin{theorem}
    \label{the:finite_sample}
    If there exists $0<c_2\leq1$ and $\lambda > 0$ such that $2Kn_{\min}/n \geq c_2$, and $p = \alpha_n = \alpha_n(q + \lambda) $, where $\alpha_n \geq \log(n)$ then with probability at least $1-\frac{1}{n}$

\begin{equation}
|\bar{\mathcal{E}}(\mathcal{T}) - \hat{\mathcal{E}}(\mathcal{T}) |  \leq h\left(\frac{2K}{c_2}\right) \frac{n_{\max }}{4c_2^2n_{\min }^{2} \alpha_{n}K^2} 
\end{equation}
where $h(x) = \left(x+\log\left(x\right)\right)$, $n_{\max} = \max_j\{n_j : j = 1,\dots K\}$, and $n_{\min} = \min_j\{n_j : j = 1,\dots K\}$.
\end{theorem}
The full proof is provided in the appendix section \ref{appendix:proofofthe:finite_sample}. A brief outline of the proof is as follows: we begin by using the results from \cite{lei2015consistency}, which establish the rate of convergence for the stochastic block model. Next, we relate the errors of the spectral clustering algorithm to the errors in the empirical entropy, using the lemma \ref{lemma:error} to establish this connection.

\begin{remark}
    This result is particularly relevant for computing semantic entropy, as the output generated by LMs is produced with a probability that is independent of $n$. As a result, we have $\alpha_n = O(1)$. Assuming balanced community sizes, the convergence rate is therefore $O(\frac{1}{n})$. This is formally stated in the following corollary:
\end{remark}


\begin{corollary} \label{corollary:rate} If there exists a constant $0 < c_2 \leq 1$ such that $2Kn_{\min}/n \geq c_2$ and $\alpha_n = alpha >0$, then there exists a constant $\alpha$ such that with probability at least $1 - \frac{1}{n}$, \begin{equation} |\bar{\mathcal{E}}(\mathcal{T}) - \hat{\mathcal{E}}(\mathcal{T})| \leq h\left(\frac{2K}{c_2}\right) \frac{1}{c_2^4 \alpha n}. \end{equation} \end{corollary}

The proof of this result is provided in the Appendix section \ref{appendix:proofofcorollary:rate}. 
\begin{remark}
    In particular, we observe that the convergence rate is $O\left(\frac{1}{n}\right)$. This means that the error associated with spectral clustering is small, and our estimated entropy converges to the empirically entropy quickly.
\end{remark}


\subsection{Performance under a generative model}

{\color{white}..} In practical terms, we assume the presence of a generator, specifically an RAG, that produces identically distributed independent random variables $t_i$' that collectively form semantic clusters $C_1 \dots C_K$. In essence, we have $t_i \sim G$ such that $t_i \in C_j$ with probability $p(C_j)$. In this model, there is a true value of entropy $\mathcal{E}(\mathcal{T})$ given in Equation \ref{equ:entropy}, and we want to find the convergence rate of our method. 
\begin{theorem}
    \label{the:final} If there exists a constant $\alpha $ such that $p = \alpha  = \alpha(q + \lambda) $, then with probability at least $1-\frac{3}{n}$,
    \begin{equation}
    \label{eqn:final_the}
       \begin{array}{cc}
         |\mathcal{E} - \hat{\mathcal{E}}|&  \leq h\left(\frac{1}{p_{\min}}\right)K\sqrt{\frac{1}{2n}\log\left(2Kn\right)}\\
         & +h\left(\frac{1}{m(n)p_{\min}}\right)\frac{1}{16K^4m(n)^4p_{\min}^4n}
    \end{array} 
    \end{equation}

where $m(n) = \left(1- \sqrt{2\log(nK)/np_{\min}}\right)$ and $p_{\min} = \min\{p(C_1)\dots p(C_K)\}$.
\end{theorem}
Most of the material used for this proof is presented in Corollary \ref{corollary:rate}. 
\begin{proof}
Consider the following equality
$$|\mathcal{E} - \hat{\mathcal{E}}| \leq |\mathcal{E} -\bar{\mathcal{E}} +\bar{\mathcal{E}}-  \hat{\mathcal{E}}| \leq |\mathcal{E} -\bar{\mathcal{E}}| + | \bar{\mathcal{E}}-  \hat{\mathcal{E}}|,$$  
 We know that there are three sufficient conditions for Equation \ref{eqn:final_the}. These are
\begin{enumerate}
    \item[C1:]$|\mathcal{E} -\bar{\mathcal{E}}| \leq  h\left(\frac{1}{p_{\min}}\right)K\sqrt{\frac{1}{2n}\log\left(2Kn\right)},$
    \item[C2:]$ \exists c_2 \text{ such that } 0 < c_2 \leq 1$ and $2Kn_{\min}/n \geq c_2,$ 
    \item[C3:] $ |\bar{\mathcal{E}}(\mathcal{T}) - \hat{\mathcal{E}}(\mathcal{T})| \leq h\left(\frac{2K}{c_2}\right) \frac{1}{c_2^4 n}.$
\end{enumerate}
Then, using union bound
\begin{align*}
    \mathbb{P}(\text{Not (\ref{eqn:final_the})}) &\leq \mathbb{P}( \text{Not C1 or Not C2 or Not C3})\\
    &\leq \mathbb{P}( \text{Not C1}) + \mathbb{P}( \text{Not C2})+ \mathbb{P}( \text{Not C3}).
\end{align*}
In Lemma \ref{lemma:final1} and \ref{lemma:Final2} of the appendix, we show that $|\mathcal{E} -\bar{\mathcal{E}}| \geq  h\left(\frac{1}{p_{\min}}\right)K\sqrt{\frac{1}{2n}\log\left(2Kn\right)}$ with probability at most $\frac{1}{n}$.

In Lemma \ref{Lemma:Final3} of the Appendix, we show that setting $c_2 = 2K\left(1-\sqrt{\frac{2\log(nK)}{np_{\min}}}\right)p_{\min}$, we have $2Kn_{\min}/n< c_2$ with probability at most $\frac{1}{n}$.

Finally, the corollary \ref{corollary:rate} tells us that $ |\bar{\mathcal{E}}(\mathcal{T}) - \hat{\mathcal{E}}(\mathcal{T})| > h\left(\frac{2K}{c_2}\right) \frac{1}{c_2^4 n}$ occurs with probability at most $\frac{1}{n}$.
\end{proof}
\begin{remark}
One observation is that the empirical entropy converges to true entropy at a rate slower than that of estimated entropy to the empirical entropy. This is natural since each $t_i$ has the opportunity to make a $n-1$ connection with other $t_j$s, resulting in $n(n-1)/2$ independent observations, whereas each generator generates only $n$ independent observations.
\end{remark}
\subsection{Discussion on $K$}
{\color{white}..} An intriguing question to consider is the rate at which \( K \), the number of clusters, can grow with \( n \), the number of texts, as it is natural to expect \( K \) to increase with \( n \). Focusing solely on the spectral clustering algorithm, the error is characterized as \( O((K + \log(K))/n) \). Thus, under the condition \( K = o(n^{1-\delta}) \) for some \( \delta > 0 \), we have \( |\bar{\mathcal{E}}(\mathcal{T}) - \hat{\mathcal{E}}(\mathcal{T})| \to 0 \) in probability. In contrast, when considering a scenario involving a generative model, a stricter condition is required. Specifically, \( K \) must satisfy \( K = o(n^{1/2 - \delta}) \), with \( \delta > 0 \), to ensure \( |\mathcal{E}(\mathcal{T}) - \hat{\mathcal{E}}(\mathcal{T})| \to 0 \) in probability.

\section{Simulation and data studies}
\label{simulation}

{\color{white} .. }As this paper focuses more on the theoretical analysis of semantic spectral entropy with respect to variable $n$ and $K$, we decide against using the evaluation method proposed in \cite{kuhn2023semantic,duan-etal-2024-shifting, lin2023generating} in favor of constructing a simulation where we know the true entropy $\bar{\mathcal{E}}$. This allows us to better analyze how $|\bar{\mathcal{E}} -\hat{\mathcal{E}}|$ changes with choice of generator $e$, $K$ and $n_{\min}$.

To construct a non-trivial simulation for this use case, we evaluate the performance of our algorithms within the context of an unordered set of elementary proposition statements that has no logical interconnections. This approach draws upon the philosophical framework defined by \citep{wittgenstein2023tractatus} in Tractatus Logico-Philosophicus, where each elementary proposition represents a singular atomic fact. Within this framework, texts containing an identical set of elementary propositions are deemed semantically equivalent. The primary advantage of this experimental design lies in its efficiency, as it facilitates the generation of thousands of samples with minimal generator propositions, all while maintaining knowledge of the ground truth.

For example, we can consider a list of things that a hypothetical individual "John" likes to do in his free time: 
\begin{itemize}
    \item Running/Jogging 
    \item Drone Flying/ Pilot Aerial drones
    \item jazzercise / aerobics
    \item ...
\end{itemize}

To generate a cluster of text from this set of hobbies, we begin by randomly selecting \( M \) items from a total of \( N \) items in the list to formulate the compound proportion. This selection process yields \( \binom{N}{M} \) potential subset of hobbies and we know that two subsets of hobbies are the same as long as their elements are the same. Next, to create individual text samples \( t_i \) within the group, we randomly permute the order of the \( M \) selected elements in the subset. This permutation process generates \( M! \) unique samples for each combination of hobbies. Finally, the hobbies are then placed in its permuted order in a sentence like that below. 
\begin{quote}
"In his free time, John likes hobby $1$, hobby $2$, hobby $3$, ..., and hobby $M$ as his hobbies."
\end{quote}
In order to prevent models to rely on sentence structure, a few of these sentences are being designed. 

%We replicate this simulation set-up in different 2 settings. The  setting is the 10 common hobbies that this hypothetical individual likes to do in his free time. %The second set-up is 10 events that happened on the date December 3 in history which we collect from Wikipedia \cite{wiki}. %The last setting  %need to think about how to build these algorithm
%\cite{atil2024llm}

We utilize Microsoft Phi-3.5 \cite{abdin2024phi}, OpenAI GPT3.5-turbo \cite{hurst2024gpt}, A21-Jamba 1.5 Mini \cite{lieber2021jurassic}, Cohere-command-r-08-2024 \cite{Ustun2024AyaMA},Ministral-3B \cite{jiang2023mistral} and the Llama 3.2 70B model \cite{dubey2024llama} as \( e \). These models are lightweight, off-the-shelf language models that are cost-effective to deploy and exhibit efficiency in generating outputs, thereby off-setting the computational cost of determining sermantic relationships. The exact prompt used to generate the verdict is specified in Appendix \ref{appendix_sec:prompt_engineering}.

\begin{table*}[ht]
\centering
\begin{tabular}{l|rrr|rrr|rrr|}
\toprule
ratio & \multicolumn{3}{r|}{0.2,0.3,0.5} & \multicolumn{3}{r|}{0.3,0.3,0.4} & \multicolumn{3}{r|}{0.5,0.5}\\
datasize & 30 & 50 & 70 & 30 & 50 & 70 & 30 & 50 & 70\\
\midrule
LLAMA & 0.36 & 0.49 & 0.44 & 0.34 & 0.43 & 0.46 & 0.30 & 0.27 & 0.26 \\x
MINISTRAL & 0.22 & 0.27 & 0.13 & 0.25 & 0.23 & 0.21 & 0.14 & 0.22 & 0.21 \\
COHERE & 0.04 & 0.02 & 0.06 & 0.02 & 0.03 & 0.00 & 0.00 & 0.00 & 0.00 \\
A21 & 0.05 & 0.00 & 0.00 & 0.00 & 0.01 & 0.00 & 0.00 & 0.00 & 0.00 \\
PHI & 0.08 & 0.07 & 0.07 & 0.03 & 0.03 & 0.00 & 0.00 & 0.00 & 0.00 \\
GPT & 0.06 & 0.02 & 0.00 & 0.01 & 0.00 & 0.00 & 0.00 & 0.00 & 0.00 \\
\bottomrule
\end{tabular}
\caption{\label{tab:basic_simu} Average $|\bar{\mathcal{E}}- \hat{\mathcal{E}}|$ over simulation 10 iterations. We have three different ratio value run over three different data sizes. For $e$, we use Microsoft Phi-3.5 \cite{abdin2024phi}, OpenAI GPT3.5-turbo \cite{hurst2024gpt}, A21-Jamba 1.5 Mini \cite{lieber2021jurassic}, Cohere-command-r-08-2024 \cite{Ustun2024AyaMA}, Ministral-3B \cite{jiang2023mistral} and the Llama 3.2 70B model \cite{dubey2024llama}. }
\end{table*}

\begin{figure}
    \centering
    \includegraphics[width=1\linewidth]{LambdaExp1.pdf}
    \caption{\label{fig:dotdata} A scatter plot of $p-q$ against $|\bar{\mathcal{E}}- \hat{\mathcal{E}}|$. The different colors represents different language models used as $e$: A21 in blue, Phi in Orange, GPT in Green, Cohere in Red, Llama is Purple and Ministral in Brown. We notice that there is clear phrase change point where for $p-q <0.4$, we have that $|\bar{\mathcal{E}}- \hat{\mathcal{E}}|$ is very high most of the time, for $p-q >0.4$, $|\bar{\mathcal{E}}- \hat{\mathcal{E}}|$ is small with occasional jumps that the theory predicts.}
    
\end{figure}

\begin{table}[]
    \centering
\begin{tabular}{lrrr}
\toprule
 $e$& $p-q$ & $p$ & $q$ \\
\midrule
LLAMA & 0.17 & 0.17 & 0.00 \\
MINISTRAL & 0.22 & 0.99 & 0.77 \\
COHERE & 0.55 & 0.61 & 0.05 \\
A21 & 0.81 & 0.96 & 0.15 \\
PHI & 0.67 & 0.67 & 0.01 \\
GPT & 0.80 & 0.87 & 0.07 \\
\bottomrule
\end{tabular}
\caption{\label{tab:p-q} $p$, $q$ and $p-q$.  For $e$, we use Microsoft Phi-3.5 \cite{abdin2024phi}, OpenAI GPT3.5-turbo \cite{hurst2024gpt}, A21-Jamba 1.5 Mini \cite{lieber2021jurassic}, Cohere-command-r-08-2024 \cite{Ustun2024AyaMA}, Ministral-3B \cite{jiang2023mistral} and the Llama 3.2 70B model \cite{dubey2024llama}.}
\end{table}

We complete simulation studies for a ratio of (0.2,0.3,0.5), (0.3, 0.3,0.4), and (0.5,0.5) and a sample size of 30, 50, 70. The average $|\bar{\mathcal{E}}- \hat{\mathcal{E}}|$ over 10 iterations using different models as $e$ is recorded in table \ref{tab:basic_simu}. The performance of algorithm using Cohere, A21, Phi, and GPT is strong while the performance of the algorithm with Minstral and Llama is weak. We primary attribute this to the inability of Llama and Minstral to make correct statements. $p-q$ is small for Llama and Minstral and large for Cohere, A21, Phi, and GPT (shown in Table \ref{tab:p-q}). In fact, when we plot $p-q$ against $|\bar{\mathcal{E}}- \hat{\mathcal{E}}|$ in Figure \ref{fig:dotdata}, we notice that there is phrase change at value $p-q = 0.4$. $p-q < 0.4$ $|\bar{\mathcal{E}}- \hat{\mathcal{E}}|$ is high but  $p-q > 0.4$ implies that $|\bar{\mathcal{E}}- \hat{\mathcal{E}}|$ is generally small. This phrase change is not predicted in the theory and suggests that more work is needed. 
\section{Discussion}
\label{conclusion}
{\color{white} .. }Many natural language processing tasks exhibit a fundamental invariance: sequences of distinct tokens can convey identical meanings. This paper introduces a theoretically grounded metric for quantifying semantic variation, referred to as semantic spectral clustering. This approach reframes the challenge of measuring semantic variation as a prompt-engineering problem, which can be applied to any large language model (LLM), as demonstrated through our simulation analysis. In addition, unsupervised uncertainty can offer a solution to the issue identified in prior research, where supervised uncertainty measures face challenges in handling distributional shifts.

While we define two texts as having equivalent meaning if and only if they mutually imply one another, alternative definitions may be appropriate for specific use cases. For example, legal documents could be clustered based on the adoption of similar legal strategies, with documents grouped together if they demonstrate comparable approaches. In such scenarios, the entropy of the legal documents could also be computed to quantify their informational diversity. We have demonstrated that, provided there exists a function $e$ capable of performing the evaluation with weak accuracy, this estimator remains consistent. Given the reasoning capabilities of large language models (LLMs), we foresee numerous possibilities for extending this method to a wide range of applications.

In addition to the methodology presented, we present a theoretical analysis of the proposed algorithms by proving a theorem concerning the contraction rates of the entropy estimator and its strong consistency. Although the algorithm utilizes generative models, which are typically treated as black-boxes, we simplify the analysis by considering the outputs of these models as random variables. We demonstrate that only a few conditions on the generative are sufficient for our spectral clustering algorithm to achieve strong consistency. Our approach allows for many statistical methodologies to be applied in conjunctions with generative models to analyze text at a level previously not achievable by humans. 



\section{Limitation}
{\color{white} .. }We acknowledge that, while this research offers a theoretically consistent measurement of variation, it does not account for situations where two pieces of text may partially agree. For instance, two texts may contain points of agreement as well as points of disagreement. This is particularly common when different authors cite the same sources but reach contradictory conclusions.
%\section{Acknowledgments}



% Bibliography entries for the entire Anthology, followed by custom entries
%\bibliography{anthology,custom}
% Custom bibliography entries only
\bibliography{custom}
\onecolumn
\appendix

\section{Theoretical Result}
\subsection{Proof of proposition  \ref{prop:equ}}
\begin{proof} 
    To prove that the relation $t_i \sim t_j$ if $t_i$ is true if and only if $t_j$ is true is an equivalence relation, we need to meet 3 key criteria, namely symmetry, reflexivity, and Transitivity. 

    First, symmetry 
    $t_i \sim t_j$ implies that $t_j$ is true $\Leftrightarrow$ $t_j$ is true, but this also means $t_j$ is true $\Leftrightarrow$ $t_i$ is true. Then we have $t_j \sim t_i$. 

    Second, reflexivity, 
    $t_i \sim t_j$ implies $t_j$ is true $\Leftrightarrow$ $t_j$ is true. But this means that $t_j$ is true  $\Leftrightarrow$ $t_i$ is true. Then we have $t_i \sim t_j$. 
    
    Third, transitivity,
    If $t_i \sim t_j$ and $t_j \sim t_k$, Then if $t_i$ is true $\Rightarrow$ $t_j$ is true $\Rightarrow$ $t_k$ is true, which means $t_i$ is true $\Rightarrow$ $t_k$ is true. On the other hand, using the same argument, $t_k$ is true $\Rightarrow$ $t_j$ is true $\Rightarrow$ $t_i$ is true. This means the $t_k$ is true $\Rightarrow$ $t_i$ is true. Therefore $t_i \sim t_k$. 

    The three points is sufficient to demonstrate that $\sim$ is a equivalence relation. 
\end{proof}
\subsection{Proof of Theorem \ref{the:strongConsistensy}}
\label{appendix:sec:the:strongconsistency}
To prove Theorem \ref{the:strongConsistensy}, we adopt notations from \cite{su2019strong}.
Consider the adjacency matrix $E$ which is determined by a Language model. 

Let $d_i = \sum_{j=1}^n E_{ij}$ denote the degree of node $i$,  $D = \text{diag}(d_1,\cdots, d_n)$, and $L = D^{-1/2}ED^{-1/2}$ be the graph Laplacian. We also define $n_k$ be the number of text in each cluster. We denote a block probability matrix $B = B_{k_1k_2}$ where $k_1,k_2 \in\{1,\cdots K\}$ be the clusters index.  i.e. 
$$ B_{k_1 k_2} = \begin{cases}
    p \quad \text{if $k_1 = k_2$}\\
    1-q \quad \text{otherwise.}
\end{cases}$$

Let $\mathbb{E}(E) = P$ i.e. the probability of edge between $i$ and $j$ is given by $P_{ij} = B_{k_1k_2}$ if text $i$ is in $C_{k_1}$ and $j$ is in $C_{k_2}$.
Denote $Z = \{Z_{ik}\}$ be a $n\times K$  binary matrix providing the cluster membership of text $t$, i.e., $Z_{ik} = 1$ if text $i$ is in $C_k$ and $Z_{ik} = 0$ otherwise. The population version of the Laplacian is given by $\mathcal{L} = \mathcal{D}^{-1/2}P\mathcal{D}^{-1/2}$ where  $\mathcal{D} = \text{diag}(d_1 \cdots d_n)$ where $d_i =\sum_{j=1}^{n}P_{ij} = p + (n-1)(q)$.

Let $\pi_{kn} = n_k/n, W_k = \sum_{l=1}^KB_{kl}\pi_{ln}$, $\mathcal{D}_B = \text{diag}(W_1,\cdots W_K)$, and $B_0=\mathcal{D}_B^{-1/2}B\mathcal{D}_B^{-1/2}  $
%C^star = 3528C_1 c_1^{-1/2}
\begin{assumption}[Assumption 1 in \cite{su2019strong}]
\label{assumption:eigenvalues}
$P$ is rank $k$ and spectral decomposition $\Pi_{n}^{1/2}P\Pi_{n}^{1/2}$ is $S_n \Omega_n S_n^T$ in which $S_n$ is a $K \times K$ matrix such that $S_n^T S_n = I_{K\times K}$  and $\Omega_n = \text{diag}(\omega_1 \cdots \omega_{K_n})$ such that $|\omega_1|\geq |\omega_2|\geq\cdots \geq|\omega_{K_n}|$
\end{assumption}
Assumption \ref{assumption:eigenvalues} implies that the spectral decomposition $$\mathcal{L} = U_n \Sigma_n U_n^T = U_{1n}\Sigma_{1n}U_{1n}^T$$

where \(\Sigma_{n}=\operatorname{diag}\left(\sigma_{1 n}, \ldots, \sigma_{K n}, 0, \ldots, 0\right)\) is a \(n \times n\) matrix that contains the eigenvalues of \(\mathcal{L}\) such that \(\left|\sigma_{1 n}\right| \geq\left|\sigma_{2 n}\right| \geq \cdots \geq\left|\sigma_{K n}\right|>0, \Sigma_{1 n}=\operatorname{diag}\left(\sigma_{1 n}, \ldots, \sigma_{K n}\right)\), the columns of \(U_{n}\) contain the 
 eigenvectors of \(\mathcal{L}\) associated with the eigenvalues in \(\Sigma_{n}, U_{n}=\left(U_{1 n}, U_{2 n}\right)\), and \(U_{n}^{T} U_{n}=I_{n}\) \cite{su2019strong}.
\begin{assumption}[Assumption 2 in \cite{su2019strong}]
\label{assumption:limits_nk}
    There exists constant $C_1 >0$ and $c_2>0$ such that
    $$C_1 \geq \lim\sup_n\sup_k n_k K/n \geq \lim \inf_n \inf_k n_k K/n \geq c_2  $$
\end{assumption}

\begin{assumption}[Assumption 3 in \cite{su2019strong}]
\label{assumption:bound_eigenvalues}
    Let $\mu_n = \min_i d_i$ and $\rho_n = \max(\sup_{k_1k_2}[B_0]_{k_1k_2},1)$. Then $n$ sufficiently large, 
    $$ 
\frac{K \rho_{n} \log ^{1 / 2}(n)}{\mu_{n}^{1 / 2} \sigma_{K n}^{2}}\left(1+\rho_{n}+\left(\frac{1}{K}+\frac{\log (5)}{\log (n)}\right)^{1 / 2} \rho_{n}^{1 / 2}\right) \leq 10^{-8} C_{1}^{-1} c_{2}^{1 / 2} .
$$
    
\end{assumption}
Let 
$$ 
\hat{O}_{n}=\bar{U} \bar{V}^{T}
$$
where \(\bar{U} \bar{\Sigma} \bar{V}^{T}\) is the singular value decomposition of \(\hat{U}_{1 n}^{T} U_{1 n}\). we also denote \(\hat{u}_{1 i}^{T}\) and \(u_{1 i}^{T}\) as the \(i\)-th rows of \(\hat{U}_{1 n}\) and \(U_{1 n}\), respectively.

Now we present the notation of the K-means algorithm. With a little abuse of notation, let \(\hat{\beta}_{\text {in }} \in \mathbb{R}^{K}\) be a generic estimator of \(\beta_{g_{i}^{0} n} \in \mathbb{R}^{K}\) for \(i=1, \ldots, n\). To recover the community membership structure (i.e., to estimate \(g_{i}^{0}\) ), it is natural to apply the  K-means clustering algorithm to \(\left\{\widehat{\beta}_{\text {in }}\right\}\). Specifically, let \(\mathcal{A}=\left\{\alpha_{1}, \ldots, \alpha_{K}\right\}\) be a set of \(K\) arbitrary  \(K \times 1\) vectors: \(\alpha_{1}, \ldots, \alpha_{K}\). Define
\[
\widehat{Q}_{n}(\mathcal{A})=\frac{1}{n} \sum_{i=1}^{n} \min _{1 \leq l \leq K}\left\|\hat{\beta}_{i n}-\alpha_{l}\right\|^{2}
\]

and \(\widehat{\mathcal{A}}_{n}=\left\{\widehat{\alpha}_{1}, \ldots, \widehat{\alpha}_{K}\right\}\), where \(\widehat{\mathcal{A}}_{n}=\arg \min _{\mathcal{A}} \widehat{Q}_{n}(\mathcal{A})\). Then we compute the estimated cluster  identity as
\[
\hat{g}_{i}=\underset{1 \leq l \leq K}{\arg \min }\left\|\hat{\beta}_{\text {in }}-\widehat{\alpha}_{l}\right\|,
\]

where if there are multiple \(l\) 's that achieve the minimum, \(\hat{g}_{i}\) takes value of the smallest one. We then state the key assumption that relates to K-means clustering algorithm. 

\begin{assumption}[Assumption 7 in \cite{su2019strong}]
\label{assumption:K-means}
     Suppose for \(n\) sufficiently large,
     \[
15 C^{*} \frac{K \rho_{n} \log ^{1 / 2}(n)}{\mu_{n}^{1 / 2} \sigma_{K n}^{2}}\left(1+\rho_{n}+\left(\frac{1}{K}+\frac{\log (5)}{\log (n)}\right)^{1 / 2} \rho_{n}^{1 / 2}\right) \leq c_{2} C_{1}^{-1 / 2} \sqrt{2}
\]
Where \(C^{*} = 3528C_1 c_2^{-1/2} \)
\end{assumption}

\begin{theorem}(Collorary 2.2)
\label{theorem:no_error}
    Corollary 2.2. Suppose that Assumptions \ref{assumption:eigenvalues},  \ref{assumption:limits_nk}, \ref{assumption:bound_eigenvalues}, and \ref{assumption:K-means} hold and the \(K\)-means algorithm is applied  to \(\hat{\beta}_{i n}=(n / K)^{1 / 2} \hat{u}_{1 i}\) and \(\beta_{g_{i}^{0} n}=(n / K)^{1 / 2} \hat{O}_{n} u_{1 i}\) Then, 
    \[
\sup _{1 \leq i \leq n} \mathbf{1}\left\{\tilde{g}_{i} \neq g_{i}^{0}\right\}=0 \quad \text { a.s. }
\]
\end{theorem}

We now have define the error of mis-classification. 

% Since there is no true $j$, we have to take all permutation of $j$ which we denote as $\sigma(j)$. 
\begin{definition}
Denote $M_\text{error} = \sum_{j} \sum_{i}\mathbb{I}(g_{ij}  \neq g^{\text{True}}_{ij})$ as the mis-classification error.
\end{definition}

\begin{lemma}
\label{lemma:error_connections}
    If $\sup_{i,j} \mathbb{I}(g_{ij}  \neq g^{\text{True}}_{ij}) = 0 \quad \text{a.s.}$, then $M_\text{error} = 0 \quad \text{a.s.}$
\end{lemma}
\begin{proof}
Notice $\mathbb{I}(g_{ij}  \neq g^{\text{True}}_{ij})$ can only takes up value $1$ or $0$. Therefore $\sum_{j} \sum_{i}\mathbb{I}(g_{ij}\neq g^{\text{True}}_{ij}) \neq 0 \Leftrightarrow \exists i, j  \text{ s.t }\mathbb{I}(g_{ij}\neq g^{\text{True}}_{ij}) \neq 0 \Leftrightarrow  \sup_{i,j}\mathbb{I}(g_{ij}\neq g^{\text{True}}_{ij}) \neq 0$  
    \begin{align*}
        \mathbb{P}(M_\text{error} \neq 0 \text{ i.o. }) &= \mathbb{P}\left( \sum_{j} \sum_{i}\mathbb{I}(g_{ij}\neq g^{\text{True}}_{ij}) \neq 0 \text{ i.o.}\right)\\
        &= \mathbb{P}\left( \exists i, j  \text{ s.t }\mathbb{I}(g_{ij}\neq g^{\text{True}}_{ij}) \neq 0 \text{ i.o. } \right)\\
        &= \mathbb{P}\left( \sup_{i,j}\mathbb{I}(g_{ij}\neq g^{\text{True}}_{ij}) \neq 0 \text{ i.o }\right)\\
        &= 0 \quad \text{ since $\sup_{i,j} \mathbb{I}(g_{ij}  \neq g^{\text{True}}_{ij}) = 0$ \text{ a.s.}}
    \end{align*}
    Here we use the classical notation i.o. as happens infinitely often. 
\end{proof}
\begin{lemma}
    \label{lemma:misclassification}
    $\sum_j \left|\sum_{i=1}^n g_{ij}-n_j\right| \leq M_\text{error}$
\end{lemma}
\begin{proof}

\begin{align*}
\sum_j \left|\sum_{i=1}^n g_{ij}-n_j\right|
&= \sum_j \Biggl| \sum_i \mathbb{I}(g_{ij} = 1, g^{\text{True}}_{ij} = 0 ) + \mathbb{I}(g_{ij} = 1, g^{\text{True}}_{ij} = 1 ) + \mathbb{I}(g_{ij} = 0, g^{\text{True}}_{ij} = 1 ) \\
&- \mathbb{I}(g_{ij} = 0, g^{\text{True}}_{ij} = 1 )- n_j \Biggr|\\
& = \sum_j \Biggl| \sum_i \mathbb{I}(g_{ij} = 1, g^{\text{True}}_{ij} = 0 ) - \mathbb{I}(g_{ij} = 0, g^{\text{True}}_{ij} = 1 ) \\
& + \sum_i \mathbb{I}(g_{ij} = 0, g^{\text{True}}_{ij} = 1 )+ \mathbb{I}(g_{ij} = 0, g^{\text{True}}_{ij} = 1 ) - n_j \Biggr|\\
& = \sum_j \Biggl| \sum_i \mathbb{I}(g_{ij} = 1, g^{\text{True}}_{ij} = 0 ) - \mathbb{I}(g_{ij} = 0, g^{\text{True}}_{ij} = 1 ) + n_j - n_j \Biggr|\\
&= \sum_j \Biggl| \sum_i \mathbb{I}(g_{ij} = 1, g^{\text{True}}_{ij} = 0 ) -\mathbb{I}(g_{ij} = 0, g^{\text{True}}_{ij} = 1 ) \Biggr|\\
&\leq \sum_j\sum_i \mathbb{I}(g_{ij} = 1, g^{\text{True}}_{ij} = 0 ) + \mathbb{I}(g_{ij} = 0, g^{\text{True}}_{ij} = 1 )\\
&=   \sum_{j} \sum_{i}\mathbb{I}(g_{ij}  \neq g^{\text{True}}_{ij}) \\
&= M_\text{error}
\end{align*}
\end{proof}
\newpage
\subsubsection{Proof of lemma \ref{lemma:error}}
\label{Appendix:proofoflemma:error}
Now we prove lemma \ref{lemma:error}.
\begin{proof}
Recall that
\begin{itemize}
    \item $\hat{p}(C_j) = \frac{1}{n}\sum_{i=1}^n g_{ij}$ and $\hat{\mathcal{E}}(\mathcal{T}) = - \sum_{j=1}^K\hat{p}( C_j) \log(\hat{p}( C_j))$
    \item $\bar{p}(C_j) = \frac{n_j}{n}$ and $\bar{\mathcal{E}}(\mathcal{T}) = - \sum_{j=1}^K\bar{p}( C_j) \log(\bar{p}( C_j))$
\end{itemize}
\begin{align*}
    |\hat{\mathcal{E}}(\mathcal{T}) - \bar{\mathcal{E}}(\mathcal{T})| &= \left| \sum_{j=1}^k\hat{p}( C_j) \log(\hat{p}( C_j)) -  \bar{p}( C_j) \log(\bar{p}( C_j)) \right|\\
    &=  \left|\sum_{j=1}^K\hat{p}( C_j)\log(\hat{p}( C_j)) -  \hat{p}( C_j)\log(\bar{p}( C_j)) + \hat{p}( C_j)\log(\bar{p}( C_j)) -  \bar{p}( C_j) \log(\bar{p}( C_j)) \right|\\
    &= \left|\sum_{j=1}^K \hat{p}( C_j)\log\left(\frac{\hat{p}( C_j)}{\bar{p}( C_j)}\right) - \left(\hat{p}( C_j) -\bar{p}( C_j)\right)\log(\bar{p}( C_j)) \right|\\
    &=  \left|\sum_{j=1}^K \hat{p}( C_j)\log\left(\frac{\hat{p}( C_j)}{p( C_j)}\right) - \left(\hat{p}( C_j) -\bar{p}( C_j)\right)\log(\bar{p}( C_j)) \right|\\
    &\leq \left|\sum_{j=1}^K \hat{p}( C_j)\log\left(\frac{\hat{p}( C_j)}{\bar{p}( C_j)}\right)\right| + \left|\sum_{j=1}^K \left(\hat{p}( C_j) -\bar{p}( C_j)\right)\log(\bar{p}( C_j)) \right|\\
    &\leq  \left|\sum_{j=1}^K \left(\frac{\hat{p}( C_j)-\bar{p}( C_j)}{\bar{p}( C_j)}\right)\right| + \left|\sum_{j=1}^K \left(\hat{p}( C_j) -\bar{p}( C_j)\right)\log(\bar{p}( C_j)) \right|\\
    &= \left|\sum_{j=1}^K \left(\frac{\frac{1}{n}\sum_{i=1}^n g_{ij}-\bar{p}( C_j)}{\bar{p}( C_j)}\right)\right| + \left|\sum_{j=1}^K \left(\frac{1}{n}\sum_{i=1}^n g_{ij} -\bar{p}( C_j)\right)\log(\bar{p}( C_j)) \right|\\
    &= \left|\sum_{j=1}^K \left(\frac{\frac{1}{n}\left(\sum_{i=1}^n g_{ij}-n_j\right)}{\bar{p}( C_j)}\right)\right| + \left|\sum_{j=1}^K \left(\frac{1}{n}\sum_{i=1}^n g_{ij} -\bar{p}( C_j)\right)\log(\bar{p}( C_j)) \right|\\
    &\leq \sum_{j=1}^K \left|\frac{\frac{1}{n}\left(\sum_{i=1}^n g_{ij}-n_j\right)}{\bar{p}( C_j)}\right| + \sum_{j=1}^K \left|\frac{1}{n}\sum_{i=1}^n (g_{ij} - n_j)\right|\left|\log(\bar{p}( C_j)) \right|\\
    &\leq \left|\frac{\frac{2K}{n}(M_\text{error})}{c_2}\right| + \log\left(\frac{2K}{c_2}\right) \left|\frac{1}{n} (M_\text{error})\right|\\
    &= h\left(\frac{2K}{c_2}\right) \left|\frac{1}{n} (M_\text{error})\right| 
\end{align*}

where $h(x) = \left(x+\log\left(x\right)\right)$. 
\end{proof}

We prove Theorem \ref{the:strongConsistensy}. To do so, we first restate Theorem \ref{the:strongConsistensy} with all the conditions required to get to the outcome.
\begin{theorem}[Theorem \ref{the:strongConsistensy} with all conditions stated]
    Assume that Assumptions \ref{assumption:eigenvalues},  \ref{assumption:limits_nk}, \ref{assumption:bound_eigenvalues}, and \ref{assumption:K-means} hold and the \(K\)-means algorithm is applied  to \(\hat{\beta}_{i n}=(n / K)^{1 / 2} \hat{u}_{1 i}\) and \(\beta_{g_{i}^{0} n}=(n / K)^{1 / 2} \hat{O}_{n} u_{1 i}\) Then 
    \[ |\bar{\mathcal{E}}(\mathcal{T}) - \hat{\mathcal{E}}(\mathcal{T}) | \rightarrow 0 \text{ almost surely }\]
\end{theorem}

\begin{proof}

Using Theorem \ref{theorem:no_error}, we know that under Assumptions \ref{assumption:eigenvalues},  \ref{assumption:limits_nk}, \ref{assumption:bound_eigenvalues}, and \ref{assumption:K-means}, we have that 
 \[
\sup _{1 \leq i \leq n} \mathbf{1}\left\{\tilde{g}_{i} \neq g_{i}^{0}\right\}=0 \quad \text { a.s. }
\]
Using Lemma \ref{lemma:error_connections}, we know that 

$$M_\text{error} = 0 \quad \text{a.s.}$$
Using results from Lemma \ref{lemma:error}, we know that $M_\text{error} \rightarrow 0 \quad a.s. \Rightarrow \hat{\mathcal{E}}(\mathcal{T}) \rightarrow \bar{\mathcal{E}}(\mathcal{T}) \quad a.s. $. 
\end{proof}
Now we try to prove Theorem \ref{the:finite_sample}. To do so, we state corollary 3.2 in \cite{lei2015consistency}.
\subsection{Proof of Theorem \ref{the:finite_sample}}
\label{appendix:proofofthe:finite_sample}
\begin{theorem}[Corollary 3.2 in \cite{lei2015consistency}]
\label{the:finite_sample_core}
 Let $E$ be an adjacency matrix from the $\operatorname{SBM}(Z, B)$, where $B=\alpha_{n} B_{0}$ for some $\alpha_{n} \geq \log n / n$ and with $B_{0}$ having minimum absolute eigenvalue $\geq \lambda>0$ and $\max _{k \ell} B_{0}(k, \ell)=1$. Let $g_{ij}$ be the output of spectral clustering using $(1+\varepsilon)$-approximate $k$-means. Then  there exists an absolute constant $c$ such that if 

\begin{equation*}
(2+\varepsilon) \frac{K n}{n_{\min }^{2} \lambda^{2} \alpha_{n}}<c
\end{equation*}
then with probability at least $1-n^{-1}$,
$$
\frac{1}{n}M_{\text{error}}\leq c^{-1}(2+\varepsilon) \frac{K n_{\max }}{n_{\min }^{2} \lambda^{2} \alpha_{n}}
$$
\end{theorem}


\begin{proof}
We now prove Theorem  \ref{the:finite_sample}.

    Under the model we have, we know that minimum eigenvalue of $B$ is $\lambda$. Use theorem \ref{the:finite_sample_core} to replace $h\left(\frac{2K}{c_2}\right) \left|\frac{1}{n} (M_\text{error})\right|$ with $ h\left(\frac{2K}{c_2}\right) c^{-1}(2+\varepsilon) \frac{K n_{\max }}{n_{\min }^{2} \lambda^{2} \alpha_{n}} $ in lemma \ref{lemma:error}.

We now have to show the existence of $c$ in Theorem \ref{the:finite_sample_core}.

\begin{align*}
    &\quad 2Kn_{\min}/n \geq c_2 \\
    &\Rightarrow 1/n_{\min}^2 \leq 4K^2/n^2c_2^2\\
    &\Rightarrow (2+\epsilon)\frac{Kn}{n_{\min}^2\lambda^2 \alpha_n} \leq (2+\epsilon)\frac{4K^3 }{n\lambda^2 \alpha_nc_2^2}\leq (2+\epsilon)\frac{4K^3}{\lambda^2c_2^2}\\
    &\text{Let $c = (2+\epsilon)\frac{4K^3}{\lambda^2}c_2^2$}
\end{align*}
substitute $c$ to $ h\left(\frac{2K}{c_2}\right) c^{-1}(2+\varepsilon) \frac{K n_{\max }}{n_{\min }^{2} \lambda^{2} \alpha_{n}} $, we have that 
\begin{equation*}
|\bar{\mathcal{E}}(\mathcal{T}) - \hat{\mathcal{E}}(\mathcal{T}) |  \leq h\left(\frac{2K}{c_2}\right) \frac{n_{\max }}{4c_2^2n_{\min }^{2} \alpha_{n}K^2}
\end{equation*}

\end{proof}
\subsubsection{Proof of Corollary \ref{corollary:rate}}
\label{appendix:proofofcorollary:rate}
\begin{proof}
Now we prove Corollary \ref{corollary:rate}. Note that $n \geq n_{\max} \geq n_{\min} \geq nc_2/2K$.

\begin{equation*}
|\bar{\mathcal{E}}(\mathcal{T}) - \hat{\mathcal{E}}(\mathcal{T}) |  \leq h\left(\frac{2K}{c_2}\right) \frac{n_{\max }}{4c_2^2n_{\min }^{2} \alpha_{n}K^2}\leq h\left(\frac{2K}{c_2}\right) \frac{1}{c_2^4 \alpha n}
\end{equation*}
\end{proof}

\newpage
\begin{lemma}
\label{lemma:final1}
    $$|\mathcal{E} - \hat{\mathcal{E}}| \leq \sum_{j=1}^K \left( \left| \frac{p(C_j) - \bar{p}(C_j)}{p(C_j)}\right| + \log\left(\frac{1}{p(C_j)}\right)\left| p(C_j) - \bar{p}(C_j)\right|\right) + h\left(\frac{2K}{c_2}\right) \left|\frac{1}{n} (M_\text{error})\right| $$
\end{lemma}
\begin{proof}
    First, we have that 
    $$|\mathcal{E} - \hat{\mathcal{E}}| \leq |\mathcal{E} -\bar{\mathcal{E}} +\bar{\mathcal{E}}-  \hat{\mathcal{E}}| \leq |\mathcal{E} -\bar{\mathcal{E}}| + | \bar{\mathcal{E}}-  \hat{\mathcal{E}}| \leq |\mathcal{E} -\bar{\mathcal{E}}| + h\left(\frac{2K}{c_2}\right) \left|\frac{1}{n} (M_\text{error})\right| $$
    Next, 
    \begin{align*}
        |\mathcal{E} -\bar{\mathcal{E}}| &\leq  \left| \sum_{j=1}^kp( C_j) \log(p( C_j)) -  \bar{p}( C_j) \log(\bar{p}( C_j)) \right|\\  
        &\leq \left| \sum_{j=1}^kp( C_j) \log(p( C_j)) -  \bar{p}( C_j) \log(p( C_j)) + \bar{p}( C_j) \log(p( C_j)) - \bar{p}( C_j) \log(\bar{p}( C_j)) \right|\\
        &\leq \sum_{j=1}^k \left|p( C_j) \log(p( C_j)) -  \bar{p}( C_j) \log(p( C_j)) \right| + \left|\bar{p}( C_j) \log(p( C_j)) - \bar{p}( C_j) \log(\bar{p}( C_j)) \right| \\
        &\leq \sum_{j=1}^k \left|p( C_j)  -  \bar{p}( C_j)\right| \log\left(\frac{1}{p( C_j)}\right)  + \left| \frac{p( C_j) -  \bar{p}( C_j)}{p( C_j)} \right|
    \end{align*}
\end{proof}

\begin{lemma}
\label{lemma:Final2}
With probability at least $1-\frac{1}{n}$,
$$\sum_{j=1}^k \left|p( C_j)  -  \bar{p}( C_j)\right| \leq K\sqrt{\frac{1}{2n}\log(2Kn)} $$
\end{lemma}
\begin{proof}

       $$ \left|p( C_j)  -  \bar{p}( C_j)\right| = \frac{1}{n}\left|np( C_j)  -  n_j\right|$$
Now use Hoeffding bound, we notice that for any $j$
$$\mathbb{P}(|n_j - np(C_j)| \geq \delta) \leq 2\exp\left(-\frac{2\delta^2}{n}\right) $$
Using union bound 
$$\mathbb{P}(\exists j \text{ such that }|n_j - np(C_j)| \geq \delta) \leq \sum_{j=1}^K\mathbb{P}(|n_j - np(C_j)| \geq \delta) \leq 2K\exp\left(-\frac{2\delta^2}{n}\right) $$

$\exists j \text{ such that }|n_j - np(C_j)| \geq \delta \Leftarrow\max |n_j - np(C_j)| \geq \delta \Leftarrow \sum_{j=1}^K |n_j - np(C_j)| \geq K\delta.$ 

Now, let $ 2K\exp\left(-\frac{2\delta^2}{n}\right) = \frac{1}{n}$, we have that $\delta = \sqrt{\frac{n}{2}\log(2Kn)}$

This gives us that with probability at least $1-\frac{1}{n}$,

$$ \sum_{j=1}^k \left|p( C_j)  -  \bar{p}( C_j)\right| \leq K\sqrt{\frac{1}{2n}\log(2Kn)} $$ 
\end{proof}
\begin{lemma}
\label{Lemma:Final3}
With probability at least $1-\frac{1}{n}$
$$n_{\min} \geq \frac{nc_2}{2K}$$
where $c_2 = 2K\left(1-\sqrt{\frac{2\log(nK)}{np_{\min}}}\right)p_{\min}$ and $p_{\min} = \min \{p(C_1) \dots p(C_K) \}$
\end{lemma}
\begin{proof}
    Using the Chernoff inequality, we have $$\mathbb{P}\left(n_j \leq (1-\delta)np(C_j)\right) \leq \exp\left(\frac{-np(C_j)}{2}\right)$$
Using the union bound
$$\mathbb{P}(n_{\min} \leq nc_2/2K) \leq \mathbb{P}\left(\exists j \text{ such that }n_j \leq (1-\delta)np(C_j)\right) \leq K\exp\left(\frac{-np_{\min}}{2}\right) $$
Let $K\exp\left(\frac{-np_{\min}}{2}\right) = \frac{1}{n}$, we get $\delta = \sqrt{\frac{2\log(nK)}{np_{\min}}}.$
Finally, we have $c_2 = 2K\left(1-\sqrt{\frac{2\log(nK)}{np_{\min}}}\right)p_{\min}$ 

\end{proof}
\newpage
\section{Simulations}
\subsection{Hobby Examples}
We can consider a list of things that a hypothetical individual "John" likes to do in his free time: 
\begin{itemize}
    \item running / jogging 
    \item Drone flying / pilot Aerial drones
    \item jazzercise / aerobics
    \item making pottery / making ceramics
    \item water gardening / aquatic gardening
    \item caving / spelunking / potholing
    \item cycling / bicycling / biking
    \item reading
    \item writing journals / journal writings/ journaling
    \item sculling / rowing
\end{itemize}
\iffalse
\subsection{Historical Examples}
On the day December 3,
\begin{itemize}
    \item 915 – Pope John X crowns Berengar I of Italy as Holy Roman Emperor
    \item 1775 – American Revolutionary War: USS Alfred becomes the first vessel to fly the Grand Union Flag; the flag is hoisted by John Paul Jones.
    \item 1800 – War of the Second Coalition: Battle of Hohenlinden: French General Jean Victor Marie Moreau decisively defeats the Archduke John of Austria near Munich. Coupled with First Consul Napoleon Bonaparte's earlier victory at Marengo, this will force the Austrians to sign an armistice and end the war.
    \item 1818 – Illinois becomes the 21st U.S. state.
    \item 1834 – The Zollverein (German Customs Union) begins the first regular census in Germany.
    \item 1898 – The Duquesne Country and Athletic Club defeats an all-star collection of early football players 16–0, in what is considered to be the first all-star game for professional American football.
    \item 1920 – Following more than a month of Turkish–Armenian War, the Turkish-dictated Treaty of Alexandropol is concluded.
    \item 1929 – President Herbert Hoover delivers his first State of the Union message to Congress. It is presented in the form of a written message rather than a speech
    \item 1959 – The current flag of Singapore is adopted, six months after Singapore became self-governing within the British Empire.
    \item 1979 – In Cincinnati, 11 fans are suffocated in a crush for seats on the concourse outside Riverfront Coliseum before a Who concert.
    \item 1979 – Iranian Revolution: Ayatollah Ruhollah Khomeini becomes the first Supreme Leader of Iran.
\end{itemize}
\fi
\newpage
\section{Prompt}
\label{appendix_sec:prompt_engineering}
This is the prompt we inserted for "Phi-3-mini-4k-instruct", "AI21-Jamba-1.5-Mini", "Cohere-command-r-08-2024".


\begin{verbatim}
'''
    You are a expert in logical deduction and you are given 2 piece of texts: TEXT A and TEXT B. 
    You are to identify if TEXT A implies TEXT B and TEXT B implies TEXT A at the same time. 
    
    TEXT A: 
    {text_A}
    
    TEXT B:
    {text_B}
    
    ## OUTPUT
    You are to return TRUE if TEXT A implies TEXT B and TEXT B implies TEXT A at the same time. 
    otherwise, you are to return FALSE 
'''
\end{verbatim}

This is the prompt we inserted for "Ministral-3B","Llama-3.3-70B-Instruct", "gpt-35-turbo"

\begin{verbatim}
''' 
    You are a expert in logical deduction and you are given 2 piece of texts: TEXT A and TEXT B. 
    You are to identify if TEXT A implies TEXT B and TEXT B implies TEXT A at the same time. 
    
    TEXT A: 
    {text_A}
    
    TEXT B:
    {text_B}
    
    ## OUTPUT
    You are to return TRUE if TEXT A implies TEXT B and TEXT B implies TEXT A at the same time. 
    otherwise, you are to return FALSE 
    
    ##FORMAT:
    START with either TRUE or FALSE, then detail your reasoning
'''
\end{verbatim}
\end{document}

\theoremstyle{plain}
\newtheorem{theorem}{Theorem}[section]
\newtheorem{proposition}[theorem]{Proposition}
\newtheorem{lemma}[theorem]{Lemma}
\newtheorem{corollary}[theorem]{Corollary}
\theoremstyle{definition}
\newtheorem{definition}[theorem]{Definition}
\newtheorem{assumption}[theorem]{Assumption}
\theoremstyle{remark}
\newtheorem{remark}[theorem]{Remark}

\usepackage[textsize=tiny]{todonotes}



\icmltitlerunning{Conformal Prediction as Bayesian Quadrature}

\begin{document}

\twocolumn[
\icmltitle{Conformal Prediction as Bayesian Quadrature}




\begin{icmlauthorlist}
\icmlauthor{Jake C. Snell}{cs}
\icmlauthor{Thomas L. Griffiths}{cs,psych}
\end{icmlauthorlist}

\icmlaffiliation{psych}{Department of Psychology, Princeton University}
\icmlaffiliation{cs}{Department of Computer Science, Princeton University}

\icmlcorrespondingauthor{Jake C. Snell}{jsnell@princeton.edu}

\icmlkeywords{Conformal Prediction, Bayesian Quadrature, Uncertainty Quantification}

\vskip 0.3in
]



\printAffiliationsAndNotice{}  %

\begin{abstract}
As machine learning-based prediction systems are increasingly used in
high-stakes situations, it is important to understand how such predictive models
will perform upon deployment. Distribution-free uncertainty quantification
techniques such as conformal prediction provide guarantees about the loss
black-box models will incur even when the details of the models are hidden.
However, such methods are based on frequentist probability, which unduly limits
their applicability.  We revisit the central aspects of conformal prediction
from a Bayesian perspective and thereby illuminate the shortcomings of
frequentist guarantees. We propose a practical alternative based on Bayesian
quadrature that provides interpretable guarantees and offers a richer
representation of the likely range of losses to be observed at test time.
\end{abstract}

\section{Introduction}

Machine learning systems based on deep learning are increasingly used in high-stakes settings, such as medical diagnosis or financial applications. These settings impose unique constraints on the performance of these systems: we want them to produce good outcomes in the aggregate, but also do so fairly and with a guarantee of a low probability of harm. However, predictive models based on deep learning can be difficult to interpret, and commercial models increasingly tend to offer little information about the techniques used in training. This creates a new challenge: How can we flexibly and reliably quantify the
suitability of a model for deployment without making too many assumptions about how the model was trained or in which settings it will be used?

Recent research on quantifying uncertainty has employed methods based on conformal prediction~\citep{vovk2005algorithmic}, which aim to provide guarantees for model performance in a distribution-free way. However, these techniques are based on ideas from frequentist statistics, making it difficult to %
incorporate prior knowledge that might be available about specific models. For example, in a particular setting we might have access to some information about the distribution of the data that is likely to be encountered, and can construct tighter guarantees on the performance of models by making use of this information. Moreover, they focus on controlling the expected loss averaged over many unobserved datasets rather than focusing on the actual set of observations.

In this paper, we show how methods for guaranteeing model performance can be understood and extended by viewing them from a Bayesian perspective. We develop a framework in which we explicitly model uncertainty in the quantile values associated with particular observations, providing a nonparametric tool for characterizing possible distributions where the model might be deployed that is appropriately constrained by observed data. This framework allows us to draw upon methods from the fields of statistical prediction analysis \citep{aitchison1975statistical} and probabilistic numerics \citep{cockayne2019bayesian,hennig2022probabilistic} to develop  guarantees that are interpretable and make adaptive use of available information.

We show that two popular uncertainty quantification methods, split conformal prediction \citep{vovk2005algorithmic, papadopoulos2002inductive} and conformal risk control \citep{angelopoulos2024conformal}, can both be recovered as special cases of our framework. Our approach gives a more complete characterization of the performance of these approaches, as we are able to determine the full distribution of possible outcomes rather than a single point estimate. Since our approach is grounded in Bayesian probability, we can easily incorporate knowledge relevant to evaluating the performance of these models when it is present, such as monotonicity or distributional assumptions, while defaulting to existing methods when absent. Our results show that Bayesian probability, while it is often discarded due to the apparent need to specify prior distributions, is actually well-suited for distribution-free uncertainty quantification.






\section{Background}

Conformal prediction methods apply a wrapper on top of black-box predictive models to be able to subject them to statistical analysis. In order to generate meaningful predictions about future performance, it is assumed that we have access to a small calibration dataset that is representative of the deployment conditions. %
Performance on this dataset then provides the foundation for generating predictions about future performance. We begin by reviewing existing current distribution-free uncertainty quantification techniques and Bayesian quadrature methods.

\subsection{Distribution-free Uncertainty Quantification Techniques}\label{sec:background_conformal}

Uncertainty quantification techniques provide guarantees on the future performance of a black-box predictive model mapping inputs $X$ to outputs $Y$ based on a calibration set consisting of $X_1, \ldots, X_n$ and $Y_1, \ldots, Y_n$. Different approaches do so in different ways.
For more information on these techniques, refer to \citet{shafer2008tutorial} or \citet{angelopoulos2023conformal}.
\paragraph{Split Conformal Prediction} The goal of Split Conformal Prediction~\citep{vovk2005algorithmic,papadopoulos2002inductive} is to generate a prediction set or interval that contains the ground-truth output with high probability. This is often expressed in terms of the coverage level $1 - \alpha$. It relies on a score function $s(x, y)$ which measures the disagreement between a predictor's output and the ground truth.

The conformal guarantee is
\begin{equation}
    \Pr\left(Y_{n+1} \notin \mathcal{C}(X_{n+1}) \right) \le \alpha \label{eq:conformal_guarantee},
\end{equation}
where
\begin{equation}
    \mathcal{C}(X_{n+1}) = \left\{ y : s(X_{n+1}, y) \le \hat{q} \right\}\label{eq:conformal_prediction_set}
\end{equation}
and
    $\hat{q}$ is the $\frac{\lceil (n+1)(1-\alpha) \rceil}{n}$ quantile of  $s_1 = s(X_1, Y_1), \ldots, s_n = s(X_n, Y_n)$.
Here, $\mathcal{C}(X_{{n+1}})$ is a prediction set or interval which aims to include the ground-truth output.
\paragraph{Conformal Risk Control} In Conformal Risk Control~\citep{angelopoulos2024conformal}, the goal is to generalize conformal prediction to more general loss functions that are monotonic functions of a single parameter $\lambda$. Conformal Risk Control (CRC) proceeds by viewing the coverage guarantee~\eqref{eq:conformal_guarantee} as the expected value of a 0-1 loss. It is assumed that the maximum possible value of the loss is $B$ and that the problem is ``achievable'' by design in that there exists some setting $\lambda_\text{max}$ that satisfies the conformal guarantee. Additionally, each loss function $L_i(\lambda)$ is assumed to be a monotonic non-increasing function of $\lambda$. The guarantee offered by Conformal Risk Control is of the  form
\begin{equation}
    E \left( \ell(\mathcal{C}_{\hat{\lambda}}(X_{n+1}), Y_{n+1}) \right) \le \alpha,\label{eq:crc_guarantee}
\end{equation}
where
\begin{equation}
    \hat{\lambda} = \inf \left\{ \lambda : \frac{n}{n+1} \hat{R}_n(\lambda) + \frac{B}{n+1} \le \alpha \right\} \label{eq:crc_estimator}
\end{equation}
and $\hat{R}_n(\lambda) = \frac{1}{n} \sum_{i=1}^n L_i(\lambda)$ is the empirical risk.

\subsection{Bayesian Quadrature}\label{sec:bayesquad}

Bayesian quadrature~\citep{diaconis1988bayesian,ohagan1991bayes} is a general technique for evaluating integrals that allows for uncertainty in the integrand. It estimates the value of an integral $\int_{a}^{b} f(x) \, dx$ by the following four steps: (1) place a prior $p(f)$ on functions, (2) evaluate $f$ at $x_{1}, x_{2}, \ldots, x_{n}$, (3) compute a posterior given the observed values of $f$ by Bayes' rule, and (4) estimate $\int_{a}^{b} f(x) \, dx$. Suppose that $f(x_{i}) = y_{i}$ for $i = 1, 2, \ldots, n$. The posterior over $f$ is %
\begin{equation}
  p(f \mid x_{1:n}, y_{1:n}) \propto p(f) \prod_{i=1}^{n} \delta(y_{i} - f(x_{i})),\label{eq:bayesquad_posterior}
\end{equation}
where $\delta(\cdot)$ is the Dirac delta function. The posterior mean then provides an estimate for the integral:
\begin{align}
  \int_{a}^{b} f(x) \, dx &\approx \int_{a}^{b} f_{n}(x) \,dx, \text{ where } \\
  f_{n}(t) &= E(f(t) \mid x_{1:n}, y_{1:n} ).
\end{align}
It has been demonstrated that many classical quadrature procedures such as the trapezoid rule can be recovered by placing a Gaussian process prior on functions~\citep{karvonen2017classical}.


\subsection{Summary and Prospectus}%

Bayesian quadrature provides an illustration of how a primarily numerical method can be connected to Bayesian inference, and in doing so potentially admit additional information about the underlying function that can be incorporated via a prior distribution. In next section, we will see how a similar approach can be applied to conformal prediction, identifying a Bayesian framework that reproduces existing distribution-free uncertainty quantification techniques. The challenge in doing so is that we want guarantees of the style obtained from Bayesian models, but we want to make the approach as general as possible in its assumptions about the underlying distribution. We solve this problem via an approach inspired by probabilistic numerics to construct a nonparametric characterization of the underlying distribution based on the calibration set.

\section{Decision-theoretic Formulation}\label{sec:decisiontheory}
In this section we show how split conformal prediction and conformal risk control can be formulated as instances of a general decision problem.

Let $z = (z_{1}, \ldots, z_{n})$ be a set of calibration data where each observation $z_{i} = (x_{i}, y_{i})$ consists of an input and a ground truth label. Let $\theta$ denote the true state of nature that defines a shared density $f(z_{i} \mid \theta)$ for the data.\footnote{In the interest of notational convenience, we assume densities and integrals over $z_{i}$ but these may be replaced by probability mass functions and summations as appropriate.} A new test point $z_{\text{new}}$ is assumed to have the same distribution. Let $\lambda$ be a control parameter (e.g.\ threshold) that must be chosen based on the calibration data. We assume the presence of a loss function $L(\theta, \lambda)$ which quantifies the loss incurred by selecting $\lambda$ when the true state of nature is $\theta$.

The decision-theoretic goal is to choose a decision rule $\lambda(z)$ that controls the \emph{risk}:
\begin{equation}
  R(\theta, \lambda) = \int L(\theta, \lambda(z)) f(z \mid \theta) \, dz.\label{eq:risk_definition}
\end{equation}
It is often desirable to choose $\lambda$ so that it is robust to any possible state of nature $\theta$. The \emph{maximum risk} is defined as
\begin{equation}
  \bar{R}(\lambda) = \sup_{\theta} R(\theta, \lambda).\label{eq:maximum_risk}
\end{equation}
In distribution-free uncertainty quantification applications, it is often trivial to achieve arbitrarily low risk (for example by forming prediction sets covering the entire output space). We thus want to find decision rules whose risk is upper bounded by a constant $\alpha$:
\begin{equation}
    \bar{R}(\lambda) \le \alpha,\label{eq:alpha_acceptable_risk}
\end{equation}
and use another criterion (such as expected prediction set size) to select among these.  We call a rule that satisfies~\eqref{eq:alpha_acceptable_risk} an \emph{$\alpha$-acceptable decision rule}.

\subsection{Recovering Split Conformal Prediction}
We now show how split conformal prediction is a special case of this decision-theoretic problem. Let $L_{\text{scp}}(\theta, \lambda)$ be the \emph{miscoverage loss}:
\begin{align}
  L_{\text{scp}}(\theta, \lambda) &= \Pr \{ s(z_{\text{new}}) > \lambda \}\label{eq:miscoverage_loss}\\
  &= 1 - \Pr \{ s(z_{\text{new}}) \le \lambda \} \nonumber \\
  &= 1 - \int \mathbbm{1} \{ s(z_{\text{new}}) \le \lambda \} f(z_{\text{new}} \mid \theta) \, d z_{\text{new}} \nonumber,
\end{align}
where $s$ is an arbitrary nonconformity function.
\begin{restatable}[]{proposition}{scprule}\label{thm:scprule}
  Define $s_{i} \triangleq s(z_{i})$ for $i = 1, \ldots, n$ and let $s_{(1)} \le s_{(2)} \le \ldots \le s_{(n)}$ be the corresponding order statistics. Let $\lambda_{\mathrm{scp}}$ be the following decision rule:
  \begin{equation}
    \lambda_{\mathrm{scp}} = \begin{cases}s_{(\lceil (n + 1) (1 - \alpha) \rceil)}, & \mathrm{if}\, \lceil (n + 1)(1 - \alpha)\rceil \le n \\ \infty, &\mathrm{otherwise}. \end{cases}\label{eq:scprule_formula}
  \end{equation}
  Then $\lambda_{\mathrm{scp}}$ is an $\alpha$-acceptable decision rule for the miscoverage loss $L_{\mathrm{scp}}$ defined in~\eqref{eq:miscoverage_loss}.
\end{restatable}
\begin{proof}
  Proofs for all theoretical results may be found in Appendix~\ref{sec:appendix_proofs}.
\end{proof}
Therefore the prediction set can be constructed as in~\eqref{eq:conformal_prediction_set}:
\begin{equation}
  \mathcal{C}_{\text{scp}}(x_{\text{new}}) = \{ y \in \mathcal{Y} : s(x_{\text{new}}, y) \le \lambda_{\text{scp}}\},
\end{equation}
and by \Cref{thm:scprule}, $\mathcal{C}_{\text{scp}}$ satisfies the conformal guarantee from~\eqref{eq:conformal_guarantee}.

\subsection{Recovering Conformal Risk Control}
Conformal risk control generalizes split conformal prediction by considering losses that are monotonic non-increasing functions of a single parameter $\lambda$.
\begin{equation}
  L_{\text{crc}}(\theta, \lambda) = \int \ell(z_{\text{new}}, \lambda) f(z_{\text{new}} \mid \theta)\, d z_{\text{new}},\label{eq:crc_loss}
\end{equation}
where $\ell(z_{\text{new}}, \lambda)$ is an individual loss function that is monotonically non-increasing in $\lambda$.
\begin{restatable}[]{proposition}{crcrule}\label{thm:crcrule}
  Let $\lambda_{\mathrm{crc}}$ be the following decision rule:
  \begin{equation}
    \lambda_{\mathrm{crc}} = \inf \left\{ \lambda : \frac{1}{n+1} \left(  \sum_{i=1}^{n} \ell(z_{i}, \lambda) + B \right) \le \alpha \right\}.\label{eq:crcrule_formula}
  \end{equation}
  Then $\lambda_{\mathrm{crc}}$ is an $\alpha$-acceptable decision rule for $L_{\mathrm{crc}}$ defined in~\eqref{eq:crc_loss}.
\end{restatable}
Note in particular that when $\ell(z, \lambda)$ can be expressed in the form  $\ell(\mathcal{C}_{\lambda}(x_{n + 1}), y_{n+1})$, this recovers the conformal risk control guarantee from~\eqref{eq:crc_guarantee}.

\section{Our Approach}\label{sec:our_method}

We introduce our approach by reinterpeting conformal prediction as a frequentist special case of a more general Bayesian procedure. In order to do so, we borrow ideas from both Bayesian quadrature~\citep{diaconis1988bayesian,ohagan1991bayes} and distribution-free tolerance regions~\citep{guttman1970statistical}. Bayesian quadrature (Section~\ref{sec:bayesquad}) solves a numerical integration problem by placing a prior on functions and using Bayesian inference to compute a distribution over the value of the integral. Distribution-free tolerance regions provide a distribution over quantile spacings that holds regardless of the original underlying distribution. Putting these ideas together allows us to extend conformal prediction by producing bounds on expected loss tailored to the actual losses observed in the calibration set.

The remainder of this section is structured as follows. In Section~\ref{sec:bayes_risk}, we discuss the relationship between risk control and Bayes risk.  In Section~\ref{sec:methods_goal}, we describe a general approach for using Bayesian quadrature to bound the posterior risk. In Section~\ref{sec:consquantile}, we make the quadrature ``distribution-free'' by removing the dependence on a prior over functions. In Section~\ref{sec:methods_quantiles} we handle uncertainty in the evaluation locations of the function by applying results that characterize the spacing between consecutive quantiles. In Section~\ref{sec:dfub_construction}, we show how to use these results to produce an upper bound on the expected loss. Finally, in Section~\ref{sec:our_recovery}, we show how previous conformal prediction techniques can be viewed as a special case of our procedure that only considers the expectation of the posterior loss.


\subsection{Bayes Risk}\label{sec:bayes_risk}
The risk $R(\theta, \lambda)$ measures the expected loss for one who already knows the true state of nature $\theta$ but not the particular data observed. However, in practical applications the situation is reversed: we \emph{do} know the observed data but there is uncertainty about the state of nature. Therefore, we want a decision rule that protects against high loss for a range of possible $\theta$. This idea is expressed as the \emph{integrated risk}:
\begin{equation}
  r(\pi, \lambda) = \int R(\theta, \lambda) \pi(\theta) \, d\theta,\label{eq:bayes_risk}
\end{equation}
where the prior $\pi(\theta) \ge 0$ measures the relative importance of the different possible states of nature. It is well-known that the minimizer of the integrated risk is the so-called \emph{Bayes decision rule}:
\begin{align}
  \lambda^{\pi} &\triangleq \argmin_{\lambda} r(\lambda \mid z),
\end{align}
where $r(\lambda \mid z)$ is the \emph{posterior risk}
\begin{equation}
  r(\lambda \mid z) = E(L_\lambda \mid z) = \int L(\theta, \lambda(z)) \pi(\theta \mid z) \,d\theta,\label{eq:posterior_risk}
\end{equation}
and $\pi(\theta \mid z) \propto \pi(\theta) f(z \mid \theta)$. Interestingly, the worst-case integrated risk of a decision rule is identical to its maximum risk~\eqref{eq:maximum_risk}
\begin{equation}
  \bar{r}(\lambda) \triangleq \sup_{\pi} r(\pi, \lambda) = \sup_{\theta} R(\theta, \lambda) = \bar{R}(\lambda).\label{eq:worst_integrated_risk}
\end{equation}
We can therefore focus on bounding the worst-case integrated risk $\bar{r}(\lambda)$, since this will also bound the maximum risk $\bar{R}(\lambda)$.

\subsection{Reformulation as Bayesian Quadrature}\label{sec:methods_goal}
We now turn our attention to finding $\lambda$ minimizing the posterior risk~\eqref{eq:posterior_risk}. Consider risks that can be expressed as the expectation over individual losses:
\begin{equation}
    L(\theta, \lambda) = \int \ell( z_{\text{new}}, \lambda) f(z_{\text{new}} \mid \theta) \, dz_{\text{new}}.
\end{equation}
It is well-known that the expectation of a random variable is equal to the definite integral of its quantile function over its domain~\citep[p.~116]{shorack2000probability}. Consider the distribution function of individual losses induced by $\lambda$ for a particular value of $\theta$:
\begin{equation}
    F(\ell) \triangleq \Pr \{ \ell(z_\text{new}, \lambda) \le \ell \mid \theta \}
\end{equation}
The corresponding quantile function is:
\begin{equation}
    K(t) \equiv F^{-1}(t) = \inf \{ \ell: F(\ell) \ge t \},
\end{equation}
and the expected loss given $K$ is simply $\int_0^1 K(t) \,dt$.

Instead of performing posterior inference over $\theta$, we propose to take an approach inspired by Bayesian quadrature that places a corresponding prior over $K$. Please refer to Figure~\ref{fig:prob_numerics} for a schematic overview of Bayesian quadrature in this setting.
\begin{figure*}[t]
    \centering
    \includegraphics[width=\linewidth]{figures/bernstein4_v3.png}
    \vspace*{-6mm}
    \caption{Bayesian quadrature produces a posterior distribution over the expected loss when applied to the quantile function of the loss distribution. Left: The expected value $L$ of a random variable with distribution function $F(\ell)$ may be computed as the definite integral of the corresponding quantile function $K(t) \equiv F^{-1}(t)$. Middle: When there is uncertainty about $K$, Bayesian quadrature places a prior on $K$ and uses Bayesian inference to obtain a posterior distribution over functions. Right: The posterior over $K$ induces a corresponding posterior over the expected value.}\label{fig:prob_numerics}
\end{figure*}
The posterior risk given the observed individual losses $\ell_i \triangleq \ell(z_i, \lambda)$ for $i = 1, \ldots, n$ becomes:
\begin{equation}
    E(L \mid \ell_{1:n}) = \int J[K]  p(K \mid \ell_{1:n}) \,dK,
\end{equation}
where $J[K] \triangleq \int_0^1 K(t)\,dt$ and we have suppressed the dependence on $\lambda$ for notational convenience. The posterior over quantile functions can be expressed as:
\begin{align}
\hspace{-5mm}    &p(K \mid \ell_{1:n}) = \int p(K \mid t_{1:n}, \ell_{1:n}) p(t_{1:n} \mid \ell_{1:n}) \, dt_{1:n} \\
    &p(K \mid t_{1:n}, \ell_{1:n}) \propto \pi(K) \prod_{i=1}^n \delta(\ell_i - K(t_i)).
\end{align}

This resembles the Bayesian quadrature problem from Section~\ref{sec:bayesquad}, except the evaluation sites $t_{1}, \ldots, t_{n}$ are unknown.  Fortunately, the distribution of $t_{1}, \ldots, t_{n}$ is independent of the true distribution of the losses, as we shall now show.

\subsection{Elimination of the Prior Distribution}\label{sec:consquantile}
In order to address the dependence of the posterior risk on the prior $\pi(K)$, we derive an upper bound on the posterior expected loss. The bound takes the form of a weighted sum of the observed losses, where the weights are determined by the spacing between consecutive quantiles.
\begin{restatable}[]{theorem}{consquantile}\label{thm:consquantile}
  Let $t_{{(0)}} = 0$, $t_{{(n+1)}} = 1$, and $\ell_{{(n+1)}} = B$. Then
  \begin{equation}
    \sup_\pi E(L \mid t_{1:n}, \ell_{1:n}) \le \sum_{i=1}^{n+1} u_{i}\ell_{{(i)}},
  \end{equation}
  where $u_{i} = t_{(i)} - t_{(i-1)}$.
\end{restatable}
\Cref{thm:consquantile} is based on the definite integral of the ``worst-case'' quantile function that is consistent with the observations. This strategy eliminates the need to specify a prior or evaluate an integral over functions $K$. We now turn our attention to handling the uncertainty over the quantiles $t_{1:n}$.

\subsection{Random Quantile Spacings}\label{sec:methods_quantiles}

We now appeal to a result about distribution-free tolerance regions that characterizes the distribution of spacings between consecutive ordered quantiles. Knowledge of this distribution will allow us to handle the input noise in the quadrature problem.

\begin{restatable}[Distribution of Quantile Spacings~\protect{\citep[p.~140]{aitchison1975statistical}}]{lemma}{dirspacings}\label{thm:dirspacings}
Suppose that $\ell_1, \ldots, \ell_n$ are drawn i.i.d.\ with continuous\footnote{The correspondence to a Dirichlet distribution holds exactly for continuous distributions. Weighted sums of Dirichlet random variates stochastically dominate weighted sums of discrete quantile spacings, and thus due to space constraints we only consider continuous distributions here.} distribution function $F$. Let $t_i = F(\ell_{i})$ and $u_i = t_{(i)} - t_{(i-1)}$, where by convention $t_{(0)} = 0$ and $t_{(n+1)} = 1$. Then $(u_1, u_2, \ldots, u_{n+1}) \cong \Dir(1, \ldots, 1)$.
\end{restatable}
We are now ready to present our algorithm for bounding the expected loss $E(L \mid \ell_{1:n})$.
\subsection{Bound on Maximum Posterior Risk}\label{sec:dfub_construction}
Putting together \Cref{thm:dirspacings} and \Cref{thm:consquantile} allows us to bound the maximum posterior risk.
\begin{restatable}[]{theorem}{stochasticub}\label{thm:stochasticub}
Define $\ell_{(i)}$ to be the order statistics of $\ell_1, \ldots, \ell_n$ for $i = 1, \ldots, n$ and $\ell_{(n+1)} \triangleq B$. Let $L^{+}$ be the random variable defined as follows:
\begin{equation}
U_{1}, \ldots, U_{n+1} \sim \Dir(1, \ldots, 1),\,  L^{+} = \sum_{i=1}^{n+1} U_{i} \ell_{{(i)}}.\label{eq:l_plus_definition}
\end{equation}
Then for any $b \in \halfclosed{-\infty}{B}$,
\begin{equation}
\inf_\pi \Pr(L \le b \mid \ell_{1:n}) \ge \Pr(L^+ \le b).\label{eq:stochasticub_statement}
\end{equation}
\end{restatable}
\Cref{thm:stochasticub} states that $L^+$ stochastically dominates the posterior risk, which allows us to directly form upper confidence bounds as follows.
\begin{restatable}[]{corollary}{ucbdf}\label{thm:ucbdf}
  For any desired confidence level $\beta \in (0, 1)$, define
  \begin{equation}
    b^{*}_\beta = \inf_{b} \{ b : \Pr(L^{+} \le b \mid \ell_{1:n} ) \ge \beta \}.\label{eq:ucbdf}
  \end{equation}
  Then $\inf_\pi \Pr(L \le b \mid \ell_{1:n}) \ge \beta$ for any $b \ge b^{*}_{\beta}$.
\end{restatable}
The critical value $b^{*}_\beta$ can be calculated by applying techniques for bounding linear combinations of Dirichlet random variables~\citep[p.~63]{ng2011dirichlet}. Alternatively, straightforward Monte Carlo simulation of $L^+$ is often sufficient, and is the approach we take in our experiments. An illustration is shown in Figure~\ref{fig:our_approach_sketch}.
\begin{figure*}[t]
    \centering
    \includegraphics[width=\linewidth]{figures/quadrature_simulation.png}
    \vspace*{-6mm}
    \caption{Our Bayesian approach to conformal prediction accounts for the variability in quantile levels better than previous approaches. Left: Conformal Risk Control~\citep{angelopoulos2024conformal} considers only the expectation over the unobserved quantile values $t_1, \ldots, t_n$. This can underestimate the true expected loss (shown here: estimated expected loss $0.45$ vs.\ true expected loss $0.50$). Right: Our approach makes use of the fact that the quantile spacings are drawn from a Dirichlet distribution. By considering the full distribution over quantiles, we gain a more complete view of the expected loss. Shown here is one sample drawn from this distribution, which estimates the expected loss as $0.58$.}\label{fig:our_approach_sketch}
\end{figure*}

\subsection{Recovering Conformal Methods}\label{sec:our_recovery}
This perspective puts the previous distribution-free uncertainty techniques in a new light. Taking the expected value of $L^+$, we find
\begin{equation}
    E(L^+) = \sum_{i=1}^{n+1} E(U_i) \ell_{(i)} = \frac{1}{n + 1} \left( \sum_{i=1}^n \ell_i + B \right).
\end{equation}
The Conformal Risk Control decision rule \eqref{eq:crcrule_formula} then is simply the infimum over $\lambda$ for which $E(L^+) \le \alpha$.

For, split conformal prediction, the individual loss is defined as $\ell_i = 1 - \mathbbm{1} \{ s_i \le \lambda \}$. Therefore, suppose that $\lambda = s_{(k)}$. The expected value of $L^+$ then becomes:
\begin{align}
    E(L^+) &= \frac{1}{n+1} \left(n + 1 - \sum_{i=1}^n \mathbbm{1}\{ s_i \le s_{(k)}\} \right) \\
    &= 1 - \frac{k}{n + 1}
\end{align}
Therefore, $E(L^+) \le \alpha$ is satisfied whenever $k \ge (n + 1)(1 - \alpha)$, and in particular by $k^* = \lceil (n + 1) (1 - \alpha) \rceil$. This recovers \eqref{eq:scprule_formula} when $\lceil (n + 1)( 1- \alpha) \rceil \le n$.

Putting these results together, we have recovered standard conformal prediction techniques but have the additional flexibility of considering the distribution of $L^+$ rather than the expected value alone. Our experiments explore the value of this approach.


\section{Experiments}

The primary goal of our experiments is to demonstrate the utility of producing a posterior distribution over the expected loss. We conduct experiments on both syntheticldata and calibration data collected from MS-COCO~\citep{lin2014microsoft}. For each data setting, we randomly generate $M = \num{10000}$ data splits. Each method is used to select $\lambda$ with the goal of controlling the risk such that $R(\theta, \lambda) \le \alpha$ for unknown $\theta$. We compare algorithms on the basis of both the relative frequency of incurring risk greater than $\alpha$ and the prediction set size of the chosen $\lambda$. The ideal algorithm would select $\lambda$ such that the relative frequency of exceeding the target risk is at most a target failure rate of $1 - \beta = 0.05$ while minimizing prediction set size.

As demonstrated in Section~\ref{sec:our_recovery}, our method recovers conformal risk control by taking the expected value of $L^+$. Therefore, in order to demonstrate the effect of targeting a conditional guarantee (as opposed to a marginal one as in conformal risk control), we use our Bayesian quadrature-based method to compute the decision rule based on the one-sided highest posterior density (HPD) interval:
\begin{equation}
    \lambda^{\beta}_{\text{hpd}} \triangleq \inf_\lambda \{ \lambda : \Pr(L^+ \le \alpha \mid \ell_{1:n}) \ge \beta \},
\end{equation}
by finding the corresponding critical values $b_\beta^*$ according to~\eqref{eq:ucbdf} via Monte Carlo simulation of Dirichlet random variates with $\num{1000}$ samples. 
 
\subsection{Synthetic Data}

\paragraph{Setup.} In our synthetic experiments, we sample from a known loss distribution so that we can directly compute the frequency of excessively large risk. Here the loss distribution is chosen to be a scaled binomial distribution, normalized to have a maximum loss of $B = 1$ and probability of failure set to $1 - \lambda$. This was simulated by computing
\begin{equation}
    \ell(z_i, \lambda) = \frac{1}{K} \sum_{k=1}^{K} \mathbbm{1}\{V_{ik} > \lambda \},\label{eq:experimental_loss}
\end{equation}
where $V_{ik} \sim \Uniform(0, 1)$ for $i = 1, \ldots, n$ and $k = 1, \ldots, K$. This loss is therefore monotonically non-increasing in $\lambda$ and achieves zero loss at $\lambda_\text{max} = 1$. We set $n$ = 10, $K = 4$, and $\alpha = 0.4$.

\paragraph{Results.} Since the expectation of the loss \eqref{eq:experimental_loss} is $1 - \lambda$, any trial for which $\lambda < 0.6$ constitutes a risk exceeding the $\alpha$ threshold. The relative frequency of trials exceeding this risk threshold are tabulated in Table~\ref{tab:risk_count}.
\begin{table}[tb]
    \centering
    \begin{threeparttable}
        \caption{Relative frequency of trials (out of 10,000) for which the resulting decision rule $\lambda$ exceeded the target risk threshold $\alpha$. }
        \begin{tabular}{lccc}
        \toprule
        Decision Rule &  Relative Freq. & 95\% CI \\
        \midrule
        CRC & 20.92\% & [20.13\%, 21.73\%] \\
        Ours ($\beta = 0.95$) & 0.11\% & [0.05\%, 0.20\%] \\
        \bottomrule
        \end{tabular}
        \begin{tablenotes}
        \item Note: Error bars are computed as 95\% Clopper-Pearson confidence intervals for binomial proportions.
        \end{tablenotes}
    \end{threeparttable}
    \label{tab:risk_count}
\end{table}
A histogram of the chosen $\lambda$ for each of the methods across all 10,000 trials is shown in Figure~\ref{fig:binomial_comparison}. For conformal risk control, the mean risk across all trials was $0.3363 \pm 0.0007$ and for our approach $\lambda_{\text{hpd}}^{0.95}$ the mean risk was $0.1758 \pm 0.0006$. In order to visualize the distribution of $L^+$, we plot a histogram of $L^+$ according to~\eqref{eq:l_plus_definition} estimated with 100,000 Dirichlet samples for three settings of $\lambda \in \{0.7, 0.8, 0.9\}$. The results are shown in  Figure~\ref{fig:posterior}.

\begin{figure*}[t]
  \centering
  \includegraphics[width=\linewidth]{figures/lambda_hat_composite.png}
  \vspace*{-6mm}
  \caption{Comparison of risk incurred by each procedure across multiple trials. Left: Histogram of the decision rule $\lambda_\text{crc}$ chosen by Conformal Risk Control across $M$ = 10,000 randomly sampled calibration sets. The region where per-trial risk exceeds $\alpha$ is highlighted in red.  Right: Histogram of the $\lambda_\text{hpd}^{0.95}$ chosen according to our 95\% Bayesian posterior interval.}\label{fig:binomial_comparison}
\end{figure*}

\begin{figure}[tb]
  \centering
  \includegraphics[width=\linewidth]{figures/bayesian_posterior.png}
  \vspace*{-6mm}
  \caption{Probability density for $L^+$ with $\lambda \in \{0.7, 0.8, 0.9\}$ estimated using 100,000 Dirichlet samples. }\label{fig:posterior}
\end{figure}

\subsection{False Negative Rate on MS-COCO}

We also compare methods on controlling the false negative rate of multilabel classification on the MS-COCO dataset~\citep{lin2014microsoft}. The experimental setup mirrors that used by~\citet[Section~5.1]{angelopoulos2023conformal}. Each random split contains $\num{1000}$ calibration examples and $\num{3952}$ test examples. We also include Risk-controlling Prediction Sets (RCPS)~\citep{bates2021distributionfree} with Hoeffding upper confidence bound as an additional baseline. The results of this experiment are summarized in Table~\ref{tab:risk_count_mscoco}.

\begin{table}[tb]
    \centering
    \begin{threeparttable}
        \caption{Results on MS-COCO comparing relative frequency of trials for which the resulting decision rule $\lambda$ exceeded the target risk threshold $\alpha$ and average prediction set size.}
        \begin{tabular}{lcc}
        \toprule
        Method &  Relative Freq. & Pred. Set Size\\
        \midrule
        CRC & 43.76\% & 2.93 \\
        RCPS & 0.0\% & 3.57 \\
        Ours ($\beta = 0.95$) & 5.11\% & 3.04 \\
        \bottomrule
        \end{tabular}
    \end{threeparttable}
    \label{tab:risk_count_mscoco}
\end{table}

\section{Discussion}\label{sec:discussion}
\epigraph{An astronaut starting a five-year flight to Mars would not be particularly comforted to be told, ``We are 95 percent confident that the average life of an imaginary population of space vehicles like yours, is at least ten years''. He would much rather hear, ``There is 95 percent probability that \emph{this} vehicle will operate without breakdown for ten years''.}{E.T. Jaynes, Confidence Intervals vs Bayesian Intervals \citep{jaynes1976confidence}}

Our results in Table~\ref{tab:risk_count} demonstrate that even though the Conformal Risk Control marginal guarantee holds, a significant number of individual trials (20.92\%) may incur risk exceeding the target threshold. In contrast, by using the more conservative HPD criterion, very few of the trials (0.11\%) exceeded the target risk. This points to the qualitative difference in a marginal guarantee, which averages over an ``imaginary population'' of unobserved data sets vs.\ a conditional guarantee which focuses on knowledge about the state of nature conditioned on the calibration data actually observed. These results are confirmed in Table~\ref{tab:risk_count_mscoco} on MS-COCO, which show that the marginal guarantees of Conformal Risk Control lead to an even greater percentage of trials exceeding the risk threshold. On the other hand, RCPS is able to control the risk but this comes at the cost of larger prediction sets. Our approach successfully balances these two concerns. It is also clear that the distribution of the expected loss upper bound $L^+$ in Figure~\ref{fig:posterior} provides a more complete view of the range of possible losses and its dependence on $\lambda$, a perspective that is not offered by previous methods.


The limitations of our method lie primarily in the two main assumptions it makes. First, it assumes that the data at deployment time are independent and identically distributed to the calibration data. Second, it assumes an upper bound $B$ on the losses. If either of these assumptions do not hold, then the guarantees produced by our method are no longer valid. Additionally, the bounds produced by our method are conservative in the sense that they hold for any choice of prior for the loss distribution (provided that the prior is consistent with the calibration data). Therefore, if the two aforementioned assumptions do hold, the actual loss values may be significantly less than indicated by our method.


Overall, our approach demonstrates how conformal prediction techniques can be recovered and extended using Bayesian probability, all without having to specify a prior distribution. This Bayesian formulation is highly flexible due to its nonparametric nature, yet is amenable to incorporating specific information about the distribution of losses likely to be encountered. In practical applications, maximizing the risk with respect to all possible priors may be too conservative, and thus future work may explore the effect of specific priors on the risk estimate.


\section{Related Work}

\paragraph{Statistical Prediction Analysis.} Statistical prediction analysis~\citep{aitchison1975statistical} deals with the use of statistical inference to reason about the likely outcomes of future prediction tasks given past ones. Within statistical prediction analysis, the area of distribution-free prediction assumes that the parameters or the form of the distributions involved cannot be identified. This idea can be traced back to \citet{wilks1941determination}, who constructed a method to form distribution-free tolerance regions. \citet{tukey1947nonparametric,tukey1948nonparametric} generalized distribution-free tolerance regions and introduced the concept of statistically equivalent blocks, which are analogous to the intervals between consecutive order statistics of the losses. Much of the relevant theory is summarized by~\citet{guttman1970statistical}, and the Dirichlet distribution of quantile spacing is discussed by~\citet{aitchison1975statistical}. We build upon these works by connecting them to Bayesian quadrature and applying them in the more modern context of distribution-free uncertainty quantification.

\paragraph{Bayesian Quadrature.} The use of Bayesian probability to represent the outcome of a arbitrary computation is termed \emph{probabilistic numerics}~\citep{cockayne2019bayesian,hennig2022probabilistic}. Since our approach is fundamentally based on integration, we focus primarily on the relationship with the more narrow approach of Bayesian quadrature, which employs Bayes rule to estimate the value of an integral. A lucid overview of this approach is discussed under the term \emph{Bayesian numerical analysis} by \citet{diaconis1988bayesian}, who traces it back to the late nineteenth century \citep{poincare1896calcul}. The use of Gaussian processes in performing Bayesian quadrature is discussed in detail by \citet{ohagan1991bayes}. Our approach is formulated similarly but differs in two main ways: (a) we use a conservative bound instead of an explicit prior, and (b) we have input noise induced by the random quantile spacings.

\paragraph{Distribution-Free Uncertainty Quantification.}
Relevant background on distribution-free uncertainty quantification techniques is discussed in Section~\ref{sec:background_conformal}. A recent and comprehensive introduction to conformal prediction and related techniques may be found in~\citep{angelopoulos2023conformal}. Some recent works, like ours, also make use of quantile functions~\citep{snell2023quantile,farzaneh2024quantile} but remain grounded in frequentist probability.  Separately, Bayesian approaches to predictive uncertainty are popular~\citep{hobbhahn2022fast} but make extensive assumptions about the form of the underlying predictive model. To our knowledge, we are the first to apply statistical prediction analysis and Bayesian quadrature in order to analyze the performance of black-box predictive models in a distribution-free way.


\section{Conclusion}

Safely deploying black-box predictive models, such as those based on deep neural networks, requires developing methods that provide guarantees of their performance.  Existing techniques for solving this problem are based on frequentist statistics, and  %
are thus difficult to extend to incorporate knowledge about the situation in which models may be deployed. In this work we provided a Bayesian alternative to distribution-free uncertainty quantification, showing that two popular existing methods are special cases of this approach. Our results show that Bayesian probability can be used to extend uncertainty quantification techniques, making their underlying assumptions more explicit, allowing incorporation of additional knowledge, and providing a more intuitive foundation for constructing performance guarantees that avoids overly-optimistic guarantees that can be produced by existing methods.






\section*{Acknowledgements}

This work was supported by grant N00014-23-1-2510 from the Office of Naval Research.



\bibliography{arxiv}
\bibliographystyle{icml2025}


\newpage
\appendix
\onecolumn

\section{Theoretical Preliminaries}\label{sec:theoretical_preliminaries}

\subsection{Review of Problem Setup}
We first review some relevant aspects of our problem setup.
\paragraph{Loss Function.} We assume an upper bound on the losses: $\ell_i \in \halfclosed{-\infty}{B}$ for $i = 1, \ldots, n$. We assume the same upper bound for $\ell_\text{new}$.

\paragraph{Bayesian Quadrature of Quantile Functions.} Recall Bayes rule for quantile functions:
\begin{equation}
    p(K \mid t_{1:n}, \ell_{1:n}) \propto \pi(K) \prod_{i=1}^n \delta(\ell_i - K(t_i)),\label{eq:posterior_review}
\end{equation}
where $\delta$ is the Dirac delta function. The prior $\pi(K)$ is assumed to be sufficiently expressive to have nonzero measure for the set $\mathcal{K}_n$ of quantile functions such that $K(t_i) = \ell_i$ for $i = 1, \ldots, n$ and $K \in \mathcal{K}_n$. This is necessary to prevent the posterior distribution in~\eqref{eq:posterior_review} from becoming degenerate.

\subsection{Background}
We begin by recalling some basic properties of distribution functions and quantile functions.

\begin{restatable}[Properties of Distribution Functions~\protect{\citep[p.~4]{shao2003mathematical}}]{proposition}{dfproperties}\label{thm:dfproperties}
Let $F(x) = \Pr(X \le x)$ be a distribution function. Then $F(-\infty) = \lim_{x \rightarrow -\infty} F(x) = 0$, $F(\infty) = \lim_{x \rightarrow \infty} F(x) = 1$, $F$ is nondecreasing (i.e., $F(x) \le F(y)$ if $x \le y$), and $F$ is right continuous (i.e., $\lim_{y \rightarrow x, y>x} F(y) = F(x)$).
\end{restatable}

Let $F$ be a distribution function and $K(t) \equiv F^{-1}(t) = \inf \{x : F(x) \ge t \} $ be the corresponding quantile function.

\begin{restatable}[Quantile Functions are Nondecreasing]{proposition}{qfmonotonic}\label{thm:qfmonotonic}
If $t \le u$, then $K(t) \le K(u)$.
\end{restatable}
\begin{proof}
Since $u \ge t$, it follows that $\{ x : F(x) \ge u \} \subseteq \{x : F(x) \ge t \}$. Taking the infimum of both sides yields
\begin{equation}
    \inf \{ x: F(x) \ge u \} \ge \inf \{ x : F(x) \ge t \} \Rightarrow  K(u) \ge K(t).
\end{equation}
\end{proof}

We also will make use of the probability integral transformation, which we state here for convenience.
\begin{restatable}[Probability Integral Transformation~\protect{\citep[p.~5]{shorack2009empirical}}]{proposition}{pit}\label{thm:pit}
    If $X$ has distribution function $F$, then
    \begin{equation}
        \Pr( F(X) \le t) \le t \qquad \text{ for all } 0 \le t \le 1,
    \end{equation}
    with equality failing if and only if $t$ is not in the closure of the range of $F$. Thus if $F$ is continuous, then $T = F(X)$ is $\Uniform(0, 1)$.
\end{restatable}


\section{Proof of Results from the Main Paper}\label{sec:appendix_proofs}
\subsection{Proof of Proposition~\ref{thm:scprule}}
Recall that $L_{\text{scp}}(\theta, \lambda)$ is the \emph{miscoverage loss}:
\begin{align}
  L_{\text{scp}}(\theta, \lambda) &= \Pr \{ s(z_{\text{new}}) > \lambda \}\label{eq:appendix_miscoverage_loss}\\
  &= 1 - \Pr \{ s(z_{\text{new}}) \le \lambda \} \nonumber \\
  &= 1 - \int \mathbbm{1} \{ s(z_{\text{new}}) \le \lambda \} f(z_{\text{new}} \mid \theta) \, d z_{\text{new}} \nonumber,
\end{align}
where $s$ is an arbitrary nonconformity function.
\scprule*
\begin{proof}
By \citet[Section 2]{lei2018distributionfree},
\begin{equation}
    \Pr(s_{\text{new}} \le \hat{q}_{1-\alpha}) \ge 1 - \alpha,
\end{equation}
where
\begin{equation}
    \hat{q}_{1-\alpha} = \begin{cases}
        s_{(\lceil (n+1)(1-\alpha) \rceil} & \text{if }\lceil (n+1)(1-\alpha) \rceil \le n \\
        \infty, & \text{otherwise}.
    \end{cases}
\end{equation}
\end{proof}
But $L_{\text{scp}}(\theta, \lambda) = 1 - \Pr(s_{\text{new}} \le \lambda \mid \theta )$, so for $\lambda = \hat{q}_{1-\alpha}$, $R(\theta, \lambda_\text{scp}) \le \alpha$. This statement not depend on $\theta$, and so $\bar{R}(\lambda_\text{scp}) \le \alpha$.

\subsection{Proof of Proposition~\ref{thm:crcrule}}
Recall that the $L_{\text{crc}}$ is defined as:
\begin{equation}
  L_{\text{crc}}(\theta, \lambda) = \int \ell(z_{\text{new}}, \lambda) f(z_{\text{new}} \mid \theta)\, d z_{\text{new}},
\end{equation}
where $\ell(z_{\text{new}}, \lambda)$ is an individual loss function that is monotonically non-increasing in $\lambda$.
\crcrule*
\begin{proof}
Let $L_1, \ldots, L_n, L_{n+1}$ be an exchangeable collection of non-increasing random functions $L_i : \Lambda \rightarrow \halfclosed{-infty}{B}$. By~\citet[Theorem 1]{angelopoulos2024conformal},
\begin{equation}
    \mathbb{E}[L_{n+1}(\hat{\lambda})] \le \alpha,\label{eq:supplemental_crc_guarantee}
\end{equation}
where
\begin{equation}
    \hat{\lambda} = \inf \left\{\lambda : \frac{n}{n+1} \hat{R}_n(\lambda) + \frac{B}{n+1} \le \alpha \right\}\label{eq:supplemental_crc_lambda_hat}
\end{equation}
and $\hat{R}_n(\lambda) = (L_1(\lambda) + \ldots + L_n(\lambda)) / n$.

Interpreting these results using the notation from Section~3 of the main paper, we identify:
\begin{itemize}
    \item $L_i(\lambda) = \ell(z_i, \lambda)$ for $i=1, \ldots, n$ and $L_{n+1}(\lambda) = \ell(z_\text{new}, \lambda)$,
    \item $\lambda_\text{crc}$ is identical to $\hat{\lambda}$ from~\eqref{eq:supplemental_crc_lambda_hat}, and
    \item \eqref{eq:supplemental_crc_guarantee} states that $R(\theta, \lambda_\text{crc}) \le \alpha$ for any $\theta$.
\end{itemize}
Therefore, $\bar{R}(\lambda_\text{crc}) = \sup_\theta R(\theta, \lambda_\text{crc}) \le \alpha$.
\end{proof}

\subsection{Proof of Theorem~\ref{thm:consquantile}}
In order to prove Theorem~\ref{thm:consquantile}, we will need to make use of two auxiliary propositions (\Cref{thm:aux1a} and \Cref{thm:aux1}). We state and prove these first, and then proceed to prove Theorem~\ref{thm:consquantile}.

\begin{restatable}[]{proposition}{aux1a}\label{thm:aux1a}
Consider the following variational maximization problem:
\begin{equation}
    I[f] = \int_a^b f(x) \,dx
\end{equation}
subject to $f(a) = f_a$, $f(b) = f_b$, and $f_a \le f(x) \le f_b$ for all $x \in [a, b]$, where $f_a \le f_b$.  Then $I[f]$ is maximized by
\begin{equation}
    f^*(x) = \begin{cases}
        f_a & \text{if } x = a, \\
        f_b & \text{otherwise},
    \end{cases}\label{eq:generic_variational_solution}
\end{equation}
and $I[f^*] = (b - a) f_b$.
\end{restatable}
\begin{proof}
    We apply Euler's method~\citep[Section 2.2]{kot2014first}, which approximates the variational problem as an $m$-dimensional problem and takes the limit as $m \rightarrow \infty$. Let the interval $[a, b]$ be divided into $m + 1$ subintervals of equal width $\displaystyle \Delta x = \frac{b - a}{m + 1}$. The objective functional can then be approximated as
    \begin{equation}
        I(f_1, \ldots, f_m) \equiv \sum_{j=0}^m f_j \Delta x,
    \end{equation}
    where $f_0 = f_a$ and $f_{m + 1} = f_b$ due to the boundary conditions. In order to handle the $f_a \le f(x) \le f_b$ constraint, we first impose $f(x) \le f_b$ and check if the solution also satisfies $f(x) \ge f_a$. To that end, we substitute $f_j = f_b - \xi_j^2$:
    \begin{equation}
        I(\xi_1, \ldots, \xi_m) = \sum_{j=0}^m (f_b - \xi_j^2) \Delta x.
    \end{equation}
    We then take partial derivatives with respect to $\xi_k$:
    \begin{equation}
        \frac{\partial I}{\partial \xi_k} = -2 \xi_k \Delta x \Rightarrow \frac{1}{\Delta x} \frac{\partial I}{\partial \xi_k} = -2 \xi_k.
    \end{equation}
    Taking the limit as $m \rightarrow \infty$ and $\Delta x \rightarrow 0$, the variational derivative becomes:
    \begin{equation}
        \frac{\delta I}{\delta \xi} = -2\xi.
    \end{equation}
    Setting $\displaystyle \frac{\delta I}{\delta \xi} = 0$ yields $\xi(x) = 0$, which recovers $f(x) = f_b$, except at $x = a$, where $f(a) = f_a$ by the boundary conditions. This recovers $f^*(x)$ from~\eqref{eq:generic_variational_solution}, which indeed satisfies $f(x) \ge f_a$. For $f^*$, it is evident that the value of the value of the functional is $I[f^*] = (b - a) f_b$.
\end{proof}

\begin{restatable}[]{proposition}{aux1}\label{thm:aux1}
Let $\mathcal{K}_n$ be the set of quantile functions for which $K(t_i) = \ell_i$ for $i = 1, \ldots, n$. Then
\begin{equation}
    \sup_{K \in \mathcal{K}_n} J[K] = \sum_{i=1}^{n+1} (t_{(i)} - t_{(i-1)}) \ell_{(i)},
\end{equation}
where $t_{(0)} = 0$, $t_{(n+1)} = 1$, $\ell_{(n+1)} = B$, and $J[K] \triangleq \int_0^1 K(t) \, dt$.

\end{restatable}
\begin{proof}
By \Cref{thm:qfmonotonic}, quantile functions preserve orderings and therefore $K(t_{(i)}) = \ell_{(i)}$. We divide $J[K]$ into intervals with endpoints $(0, t_{(1)}), (t_{(1)}, t_{(2)}), \ldots, (t_{(n)}, 1)$:
\begin{align}
\sup_{K \in \mathcal{K}_n} J[K] &= \sup_{K \in \mathcal{K}_n} \int_0^1 K(t) \, dt \\
&= \sup_{K \in \mathcal{K}_n} \sum_{i=1}^{n+1} \int_{t_{(i-1)}}^{t_{(i)}}K(t) \, dt \\
&\le \sum_{i=1}^{n+1} \sup_{K \in \mathcal{K}_n} \int_{t_{(i-1)}}^{t_{(i)}}K(t) \, dt
\end{align}
By \Cref{thm:qfmonotonic}, $K(t_{(i-1)}) \le K(t) \le K(t_{(i)})$ for any $t \in [t_{(i-1)}, t_{(i)}]$. We view each term as a variational subproblem where $\displaystyle J_i[K_i] \triangleq \int_{t_{(i-1)}}^{t_{(i)}} K_i(t) \, dt$ with boundary conditions $K_i(t_{(i-1)}) = \ell_{(i-1)}$ and $K_i(t_{(i)}) = \ell_{(i)}$. We therefore appeal to \Cref{thm:aux1a} to conclude that
\begin{equation}
    K_i^*(t) =  \begin{cases}
        \ell_{(i-1)} & \text{if } t = t_{(i-1)}, \\
        \ell_{(i)} & \text{otherwise},
    \end{cases}
\end{equation}
and $J[K_i^*] = (t_{(i)} - t_{(i-1)}) \ell_{(i)}$. We therefore have
\begin{equation}
    \sup_{K \in \mathcal{K}_n} J[K] \le \sum_{i=1}^{n+1} (t_{(i)} - t_{(i-1)}) \ell_{(i)}.
\end{equation}
By composing $K_i^*$ from each subinterval, it is straightforward to see that the bound is tight for
\begin{equation}
    K^*_{t_{1:n}, \ell_{1:n}}(t) = \begin{cases}
        \ell_{(1)} & \text{ if } t \le t_{(1)} \\
        \ell_{(2)} & \text{ if } t_{(1)} < t \le t_{(2)} \\
        \ldots \\
        \ell_{(n)} & \text{ if } t_{(n-1)} < t \le t_{(n)} \\
        B & \text{ if } t > t_{(n)}.
    \end{cases}\label{eq:worst_case_quantile_function}
\end{equation}
$K^*_{t_{1:n}, \ell_{1:n}}$ is therefore the ``worst-case'' quantile function that is consistent with the observations, and $J[K^*_{t_{1:n}, \ell_{1:n}}] = \sum_{i=1}^{n+1} (t_{(i)} - t_{(i-1)}) \ell_{(i)}$.
\end{proof}

We are now ready to prove Theorem~\ref{thm:consquantile}.

\consquantile*
\begin{proof}
    Let $J[K] = \int_0^1 K(t) \, dt$. The conditional expected loss can be expressed as:
    \begin{align}
        E(L \mid t_{1:n}, \ell_{1:n}) &= \int J[K] p(K \mid t_{1:n}, \ell_{1:n}) \, dK \\
        &\le \sup_{K \in \mathcal{K}_n} J[K],
    \end{align}
    where $\mathcal{K}_n$ is the set of quantile functions for which $K(t_i) = \ell_i$ for $i = 1, \ldots, n$. By~\Cref{thm:aux1}, it follows that
    \begin{equation}
        E(L \mid t_{1:n}, \ell_{1:n}) \le \sum_{i=1}^{n+1} (t_{(i)} - t_{(i-1)}) \ell_{(i)} = \sum_{i=1}^{n+1} u_i \ell_{(i)}
    \end{equation}
\end{proof}

\subsection{Proof of Lemma~\ref{thm:dirspacings}}
\dirspacings*
\begin{proof}
By the probability integral transformation~(\Cref{thm:pit}), $T_i$ is $\Uniform(0, 1)$ for $i = 1, \ldots, n$.  Since the transformation from $(t_1, \ldots, t_n) \rightarrow (t_{(1)}, \ldots, t_{(n)})$ is a sorting operation where $n!$ permutations map to the same vector of order statistics, the probability density for  $t_{(1)}, \ldots, t_{(n)}$ is therefore
\begin{equation}
    f_{t_{(1:n)}}(t_{(1)}, \ldots, t_{(n)}) = n!, \qquad 0 \le t_{(1)} \le t_{(2)} \le \ldots \le t_{(n)} \le 1.
\end{equation}
If $u_{1:n} = G(t_{(1:n)})$ where $G$ is differentiable and invertible, then by change of variables the density for $u_{1:n}$ can be expressed as
\begin{equation}
    f_{u_{1:n}}(u_{1:n}) = f_{t_{(1:n)}}(G^{-1}(u_{1:n})) \left| \det \left( \frac{\partial}{\partial u_{1:n}} G^{-1}(u_{1:n}) \right) \right|.
\end{equation}
Observe that the inverse transformation $t_{(1:n)} = G^{-1}(u_{1:n})$ can be expressed as
\begin{equation}
    \begin{bmatrix}
    t_{(1)} \\ t_{(2)} \\ t_{(3)} \\ \vdots \\ t_{(n-1)} \\ t_{(n)}
    \end{bmatrix}
    =
    \begin{bmatrix}
    1 & 0 & 0 & \ldots & 0 & 0 \\
    1 & 1 & 0 & \ldots & 0 & 0 \\
    1 & 1 & 1 & \ldots & 0 & 0 \\
    \vdots & \vdots & \vdots & \ddots & \vdots & \vdots \\
    1 & 1 & 1 & \ldots & 1 & 0 \\
    1 & 1 & 1 & \ldots & 1 & 1 \\
    \end{bmatrix}
    \begin{bmatrix}
    u_{1} \\ u_{2} \\ u_{3} \\ \vdots \\ u_{n-1} \\ u_{n}
    \end{bmatrix}.
\end{equation}
Hence the absolute Jacobian of inverse transformation $t_{(1:n)} = G^{-1}(u_{1:n})$ is 1. The density of $u_{1:n}$ is therefore
\begin{equation}
    f_{u_{1:n}}(u_{1:n}) = f_{t_{(1:n)}}(G^{-1}(u_{1:n})) = n!, \quad \text{ where } u_i \ge 0 \text{ for } i = 1, \ldots, n \text{ and }\sum_{i=1}^n u_i \le 1.\label{eq:spacing_density}
\end{equation}
Recall that the Dirichlet density with parameter $\alpha_1, \ldots, \alpha_{n+1}$ is:
\begin{equation}
    \Dir(u_{1:n+1} \mid \alpha_{1:n+1}) = \frac{ \Gamma( \sum_{i=1}^{n+1} \alpha_i)}{\Gamma (\alpha_1) \ldots \Gamma(\alpha_{n+1})} \prod_{i=1}^{n+1} u_i^{\alpha_i - 1}, \quad \text{ where } u_i \ge 0 \text{ and } \sum_{i=1}^{n+1} u_i = 1.
\end{equation}
In particular, if $\alpha_1 = \alpha_2 = \ldots = \alpha_{n+1} = 1$,
\begin{equation}
    \Dir(u_{1:n+1} \mid 1, \ldots, 1) = \Gamma(n+1) = n!,
\end{equation}
which is identical to~\eqref{eq:spacing_density} with $u_{n+1} = 1 - u_1 - \ldots - u_n$. Therefore, $(u_1, u_2, \ldots, u_{n+1}) \cong \Dir(1, \ldots, 1)$.

\end{proof}

\subsection{Proof of Theorem~\ref{thm:stochasticub}}

\stochasticub*
\begin{proof}
\begin{align}
    \inf_\pi \Pr(L \le b \mid \ell_{1:n}) &= \inf_\pi \int \mathbbm{1} \left\{ J[K] \le b \right\} p(K \mid \ell_{1:n}) \, dK \\
    &= \inf_\pi \int \mathbbm{1} \left\{ J[K] \le b \right\} \left( \int p(K \mid t_{1:n}, \ell_{1:n}) p(t_{1:n} \mid \ell_{1:n}) \, dt_{1:n} \right)  \, dK \\
    &= \inf_\pi \int \left(\int \mathbbm{1} \left\{ J[K] \le b \right\} p(K \mid t_{1:n}, \ell_{1:n}) \, dK \right) p(t_{1:n} \mid \ell_{1:n}) \, dt_{1:n} \\
    &\ge  \int \left(\inf_\pi \int \mathbbm{1}\left\{ J[K] \le b \right\} p(K \mid t_{1:n}, \ell_{1:n}) \, dK \right) p(t_{1:n} \mid \ell_{1:n}) \, dt_{1:n} \\
    &\ge  \int \left(\inf_{K \in \mathcal{K}_n} \mathbbm{1}\left\{ J[K] \le b \right\} \right) p(t_{1:n} \mid \ell_{1:n}) \, dt_{1:n} \\
    &\ge  \int \mathbbm{1}\left\{ J[K^*_{t_{1:n}, \ell_{1:n}}] \le b \right\} p(t_{1:n} \mid \ell_{1:n}) \, dt_{1:n} \\
    &= \int \mathbbm{1}\left\{ \sum_{i=1}^{n+1} (t_{(i)} - t_{(i-1)}) \ell_{(i)} \le b \right\} p(t_{1:n} \mid \ell_{1:n}) \, dt_{1:n} \\
    &= \int \mathbbm{1}\left\{ \sum_{i=1}^{n+1} u_i \ell_{(i)} \le b \right\} p(u_{1:n+1} \mid \ell_{1:n}) \, du_{1:n+1} \\
    &= \Pr( L^+ \le b)
\end{align}
\end{proof}

\subsection{Proof of Corollary~\ref{thm:ucbdf}}

\ucbdf*
\begin{proof}
For any $b \ge b_\beta^*$, $\Pr(L^+ \le b \mid \ell_{1:n}) \ge \beta$. Substitution into \eqref{eq:stochasticub_statement} provides the desired result.
\end{proof}

\end{document}
