% \appendix
\myifconf{}{\section{Appendix}
\label{section:appendix}

% \begin{definition}(robust stability~\cite{San2021})
% \label{def:robust-stability}
% { Given a hybrid closed-loop system $\mc{H}$, a nonempty closed set $\mc{A}\subset \ms{M}$ and an open set $\mc{U}\subset \ms{M}$ such that $\mc{A} \subset \mc{U}$, the set $\mc{A}$ is said to be robustly stable for $\mc{H}$ on $\mc{U}$ if for every proper indicator function $\varpi$ of $\mc{A}$ on $\mc{U}$, every function $\beta \in \mc{KL}$ such that 
% \[
% \varpi(x(t,j)) \leq \beta(\varpi(x(0,0)), t+j)\quad \forall (t,j)\in\dom x 
% \]
% for the solutions to $\mc{H}$ from $\mc{U}$, and every continuous function 
% $\rho^{*}:\ms{M} \to \Real_{\geq 0}$ that is positive on $\mc{U} \setminus \mc{A}$, the following holds: for each compact set $K \subset \mc{U}$ and each $\epsilon >0$, there exists $\delta^{*} >0$ such that for each solution {$x_{\rho}$} the perturbed system $\mc{H}_{\rho}$ with $\rho = \delta^{*}\rho^{*}$, starting from $x_{\rho}(0,0) \in K$ satisfies 
% \[
% \varpi(x_{\rho}(t,j)) \leq \beta(\varpi(x_{\rho}(0,0)), t+j)+ \epsilon \quad   \forall (t,j)\in\dom x_{\rho}.
% \]}
% \end{definition}

{\myblue Consider a time-invariant, finite-dimensional, deterministic control-affine system with $m$ inputs, $u:=[u_1 \cdots u_m]^{\top}\in \mathbb{R}^m$ and $p$ outputs, along with smooth maps $f:\mathbb{R}^n\rightarrow \mathbb{R}^n$, $g_{i}:\mathbb{R}^n\rightarrow \mathbb{R}^n$ and $h:\mathbb{R}^n\rightarrow \mathbb{R}^p$.
%
\begin{eqnarray}\label{eq:most_general_sys_mathprelim}
%\begin{split}
 \dot{x}=f(x)+ \sum^{m}_{i=1}g_{i}(x)u_{i} \eqdef f(x)+ g(x)u,
 % \end{split}
\end{eqnarray}
and consider a function,
\begin{eqnarray}\label{eq:most_general_output_appendex}
%\begin{split}
  y=h(x)=\left[
  \begin{array}{c}
      h_1 (x)\\
      \vdots\\
      h_p (x)\\
  \end{array} \right],\forall y \in \mathbb{R}^p,
 % \end{split}
\end{eqnarray}
which is the output of the system. 
% The relative degree is the key concept in solving the feedback linearization problem.
% \begin{definition}(Relative degree)
% Consider system~\eqref{eq:most_general_sys_mathprelim} with $u \in \mathbb{R}$ and with output function~\eqref{eq:most_general_output_appendex} with $m=p=1$ i.e., $y=h(x)$, $y \in \mathbb{R}$. The system has a relative degree of $r$ at a point $x_0$ if
% \begin{enumerate}
%   \item $L_g L_f ^k h(x)=0$, for all $x \in $ a neighborhood of $x_0$ and for all $k<r-1$,
%   \item $L_g L_f ^{r-1} h(x_0)\neq 0$.
% \end{enumerate}
% \end{definition}
% The relative degree of a single input single output (SISO) system in the number of times we need to differentiate the output before the control input appears. 
The vector relative degree can be defined for the multiple-input multiple-output (MIMO) systems.

\begin{definition}(Vector relative degree \cite{Sastry})
\label{def:vector_relative_deg}
Consider system \eqref{eq:most_general_sys_mathprelim} with $m=p$. We define an $m \times m$ matrix

\begin{equation*}
\label{eq:matix_vector_relative_degree}
 A(x):=
  \left[
  \begin{array}{ccc}
      L_{g_1}L_{f}^{r_1 -1}h_1(x)& \cdots & L_{g_m}L_{f}^{r_1 -1}h_1(x)\\
      L_{g_1}L_{f}^{r_2 -1}h_2(x)& \cdots & L_{g_m}L_{f}^{r_2 -1}h_2(x)\\
      \vdots & \ddots & \vdots \\
      L_{g_1}L_{f}^{r_m -1}h_m(x)& \cdots & L_{g_m}L_{f}^{r_m -1}h_m(x)\\
  \end{array} \right].
\end{equation*}
%
The system has a vector relative degree of $\{r_1, \dots r_m \}$ at a point $x_0$ if
\begin{enumerate}
  \item $L_{g_j} L_f ^k h_i(x)=0,$ for all $1\leq j \leq m $ for all $ k<r_i-1$ for all $1\leq i \leq m$ and for all $x$ in a neighborhood of $x_0$, and
  \item  the matrix $A(x)$ is nonsingular at $x=x_0$.
\end{enumerate}
\end{definition}}

\begin{proposition}[Proposition 8.1, \cite{bhat2005geometric}]\label{prop:linearfinitetime}
Let $k_1,..., k_{n} > 0$ be such that the polynomial $s^n + k_ns^{n-1} + \cdots + k_{2}s + k_{1}$ is Hurwitz, and consider the system
\begin{equation}\label{eq:linearsystem}
    \begin{aligned}
    \dot{x}_{1} = x_{2},\\
    \vdots\\
    \dot{x}_{n - 1} = \dot{x}_{n},\\
    \dot{x}_{n} = u.
\end{aligned}
\end{equation}
There exists $\epsilon\in (0, 1)$ such that, for every $\alpha\in (1 - \epsilon, 1)$, the origin is a globally finite-time-stable equilibrium for the system in (\ref{eq:linearsystem}) under the feedback
$$
u = -k_1 \sign x_{1} |x_{1}|^{\alpha} - \cdots - k_{n}\sign x_{n} |x_{n}|^{\alpha_{n}},
$$
where $\alpha_{1},..., \alpha_{n} $ satisfy
\begin{equation}\label{eq:xicontrol}
    \alpha_{i - 1} = \frac{\alpha_{i}\alpha_{i + 1}}{2\alpha_{i + 1}- \alpha_{i}}, i = 2, ..., n,
\end{equation}
with $\alpha_{n + 1} = 1$ and $\alpha_{n} = \alpha$.
\end{proposition}
\begin{theorem}[Theorem 2 in \cite{XiaBelCas2022}]
    Given the reference profile $\eta_{2}^{\text{ref}}\in \reals$ and $\eta_{3}^{\text{ref}}\in \reals$ and the initial $\eta_{0}\in \reals^{6}$, if  If Problem 1 is initially feasible and the CBF constraint
in (24) corresponding to (22) does not conflict with both the control
bounds (2) and (18) at the same time, any controller $u\in$
guarantees the feasibility of problem (9), subject to (10)–(12).
\end{theorem}}