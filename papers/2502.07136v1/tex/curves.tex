\subsection{Class of Curves}
\label{sec:curves}
Given a smooth curve $\mc{C}$ in $\Real^2$ without self intersections, the curve $\mc{C}$ has a regular parametric representation, namely,
\begin{equation}
\label{eq:general_path}
%\begin{aligned}
\sigma : \dom \sigma\rightarrow\mathbb{R}^{2}, \qquad
\lambda\mapsto{\left(\sigma_{1}(\lambda), \sigma_{2}(\lambda)\right)},
%\end{aligned}
\end{equation}
where $\sigma\subset \Real$ is at least twice continuously differentiable, i.e., $C^2$, and $\mc{C}= \image{(\sigma)}$.
Since $\sigma$ is regular, without losing generality, we assume it is
unit-speed parameterized, i.e., $\norm{\sigma^\prime} \equiv 1$, where $\sigma^\prime$ is the derivative of $\sigma$ with respect to the parameter $\lambda$. Consequently, the curve $\sigma$ is parameterized by its arc length; for details, see~\cite{Pres2010,AkhNieWas2015}.  For a unit-speed curve $\sigma$ with parameter $\lambda$, its curvature $K(\lambda)$ at the point $\sigma(\lambda)$ is defined to be  $\norm{\sigma^{\prime\prime}(\lambda)}$,  where $\sigma^{\prime\prime}$ is the second derivative of $\sigma$ with respect to the parameter $\lambda$.
\begin{assumption}[Implicit representation]
  The curve $\mathcal{C}\subset \Real^2$ has implicit representation $
  \gamma = \set{y \in W : s(y) = 0}$, 
  % {\myblue (AA: We used the small Greek letter $\gamma$ because later we ``lift'' this set from the output space to state space and then call it capital gamma $\Gamma$)} 
  where $s: \dom s \to \Real$ is a smooth function such that the {Jacobian of $s$ evaluated \mynn{at} each point on the path is not zero, i.e., $\D_y s \neq 0$
  for each $y \in \mathcal{C}$} and $\dom s \mynn{\subset}
  \Real^2$ is a set {consisting of an open neighborhood of the curve $\mathcal{C}$}.  
%   Moreover, there
%     exist two class-$\mathcal{K}_\infty$ functions $\alpha, \beta :
%     [0, \infty) \rightarrow [0, \infty)$ such that
% \begin{equation}
%   \left( \forall y \in W \right) \; \alpha{\left(
%       \|y\|_{\mathcal{C}} \right)} \leq \|s(y)\| \leq
%   \beta{\left(\|y\|_{\mathcal{C}} \right)}.
%  \label{eq:classK}
% \end{equation}
\label{ass:implicit}
\end{assumption}

Assumption~\ref{ass:implicit} assumes that the entire path is represented as
the zero-level set of the function $s$, at least locally. A simple example of such a curve is a unit circle with a parametric representation of $\lambda\mapsto(\cos \lambda,\sin\lambda)$ and an implicit representation of $s(y) = y_1^2 + y_2^2 - 1 = 0$,  {and $\dom s = \Real^2$. On the other hand, for $y = (y_1,y_2)\in\Real^2$, an $n$-th order polynomial in variable $y_1$ can be expressed as $y_2 = \sum_{i=0}^{n} a_iy_1^i$, where  scalars $a_i\in\Real$ and $y_1^i$ is the $i$-th power of $y_1$ for $i \in \{0, 1,..., n\}$. Moreover, the polynomial can be expressed implicitly as $s(y) = y_2 - \sum_{i=0}^{n} a_iy_1^i$, and in the form of a parametric curve as $\lambda\mapsto(\lambda,\sum_{i=0}^{n} a_i\lambda^i)$}. 

% {\myred We need to define curvature of the parametric curve $\sigma$.}
