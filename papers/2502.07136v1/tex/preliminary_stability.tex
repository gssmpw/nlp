\ifbool{conf}{Consider the following continuous-time system
\begin{equation}\label{eq:differentialequation}
    \dot{x} = f(x)
\end{equation}
with state $x\in \mathbb{R}^{n}$. The solution to (\ref{eq:differentialequation}) starting from $x_{0}\in\mathbb{R}^{n}$ is defined as a locally absolutely continuous function $\phi: \dom\phi\to\mathbb{R}^{n}$ satisfying $\phi(0) = x_{0}$ and $\dot{\phi}(t) = f(\phi(t))$ for each  $t\in\dom\phi\subset\mathbb{R}_{\geq 0}$. We assume uniqueness of maximal solutions to (\ref{eq:differentialequation}).
Following~\cite{bhat2000finite}, finite-time stability can be defined as follows.
\begin{definition}[Finite-time stability]
Given (\ref{eq:differentialequation})
with state $x\in \mathbb{R}^{n}$, a nonempty closed set $\mathcal{K}\subset \mathbb{R}^{n}$ is finite-time stable for (\ref{eq:differentialequation}) if
\begin{enumerate}
    \item $\mathcal{K}$ is Lyapunov stable for (\ref{eq:differentialequation});
    \item there exists an open neighborhood $\mathcal{N}$ of $\mathcal{K}$ that is positively invariant  for (\ref{eq:differentialequation}) and a positive definite function $T: \mathcal{N}\to \mathbb{R}_{\geq 0}$, called the settling-time function, such that\pn{,} for each initial state $x_{0}\in \mathcal{N}$, each maximal solution $\phi$ to (\ref{eq:differentialequation}) from $x_{0}$ satisfies
    $|\phi(T(x_{0}))|_{\mathcal{K}} = 0$ and, \mynne{when} $x_{0}\in \mathcal{N}\backslash\mathcal{K}$,
    $|\phi(t)|_{\mathcal{K}} > 0 \quad \mynne{\forall t\in [0, T(x_{0}))}.$
\end{enumerate}
\end{definition}}{
Consider the following continuous-time system
\begin{equation}\label{eq:differentialequation}
    \dot{x} = f(x)
\end{equation}
with state $x\in \mathbb{R}^{n}$. The solution to (\ref{eq:differentialequation}) starting from $x_{0}\in\mathbb{R}^{n}$ is defined as a locally absolutely continuous function $\phi: \dom\phi\to\mathbb{R}^{n}$ satisfying $\phi(0) = x_{0}$ and $\dot{\phi}(t) = f(\phi(t))$ for each  $t\in\dom\phi\subset\mathbb{R}_{\geq 0}$. A solution $\phi$ to (\ref{eq:differentialequation}) is said to be \emph{maximal} if there does not exist another solution $\phi_{1}$ such that $\dom \phi$ is a proper subset of $\dom \phi_{1}$ and $\phi(t) = \phi_{1}(t)$ for all $t\in \dom \phi$. We assume  uniqueness of maximal solutions to (\ref{eq:differentialequation}).
\begin{assumption}\label{assu:unimax}
    For any $x_{0}\in\Real^{n}$, there exists an unique maximal solution to (\ref{eq:differentialequation}) starting from $x_0$.
\end{assumption}
Next, following \cite{bhat2005geometric}, we introduce positive invariance (also known as forward invariance), Lyapunov stability, and finite-time stability for (\ref{eq:differentialequation}).
%, for any initial state $x_{0}\in \mathbb{R}^{n}$, the system has a unique solution $\phi$ from $x_{0}$ defined on $[0, \infty)$.
\begin{definition}[Positive invariance]
Given (\ref{eq:differentialequation})
with state $x\in \mathbb{R}^{n}$ \mynne{satisfying Assumption~\ref{assu:unimax}}, a set $\mathcal{K}\subset\mathbb{R}^{n}$ is positively invariant for (\ref{eq:differentialequation}) if, for any initial state $x_{0}\in\mathcal{K}$, each solution to (\ref{eq:differentialequation}) starting from $x_{0}$, denoted $\phi$, satisfies $\phi(t)\in\mathcal{K}$ for each $t\in\dom \phi$.
\end{definition}
\begin{definition}[Lyapunov stability]
Given (\ref{eq:differentialequation}) with state $x\in \mathbb{R}^{n}$ \mynne{satisfying Assumption~\ref{assu:unimax}}, a nonempty closed set $\mathcal{A}\subset \mathbb{R}^{n}$ is Lyapunov stable for (\ref{eq:differentialequation}) if, for every open neighborhood $\mathcal{N}_{\epsilon}$ of $\mathcal{A}$, there exists an open neighborhood $\mathcal{N}_{\delta}$ of $\mathcal{A}$ such that, for any initial state $x_{0}\in\mathcal{N}_{\delta}$, each solution to (\ref{eq:differentialequation}) starting from $x_{0}$, denoted $\phi$, satisfies $\phi(t)\in\mathcal{N}_{\epsilon}$ for each $t\in\dom \phi$.
\end{definition}

Having defined positive invariance and Lyapunov stability, we proceed to define finite-time stability.
\begin{definition}[Finite-time stability]
Given (\ref{eq:differentialequation})
with state $x\in \mathbb{R}^{n}$ \mynnd{satisfying Assumption~\ref{assu:unimax}}, a nonempty closed set $\mathcal{K}\subset \mathbb{R}^{n}$ is finite-time stable for (\ref{eq:differentialequation}) if
\begin{enumerate}
    \item $\mathcal{K}$ is Lyapunov stable for (\ref{eq:differentialequation});
    \item there exists an open neighborhood $\mathcal{N}$ of $\mathcal{K}$ that is positively invariant  for (\ref{eq:differentialequation}) and a positive definite function $T: \mathcal{N}\to \mathbb{R}_{\geq 0}$, called the settling-time function, such that\pn{,} for each initial state $x_{0}\in \mathcal{N}$, each maximal solution $\phi$ to (\ref{eq:differentialequation}) from $x_{0}$ satisfies
    $|\phi(T(x_{0}))|_{\mathcal{K}} = 0$ and, \mynne{when} $x_{0}\in \mathcal{N}\backslash\mathcal{K}$,
    $|\phi(t)|_{\mathcal{K}} > 0 \quad \mynne{\forall t\in [0, T(x_{0}))}.$
\end{enumerate}
\end{definition}
}

