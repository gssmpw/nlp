% \section{Hybrid System Modeling of the Vehicle System}\label{sec:hybridmodel}
% % The cyber-physical model has four elements: the continuous system, the analog to digital converter, the discrete controller, and the digital to analog converter. 

% % \begin{figure}[ht]
% %     \centering
% %     \includegraphics[width = \columnwidth]{Figures/SystemDiagram.pdf}
% %     \caption{Cyber Physical System Diagram}
% %     \label{fig:CyberPhysical}
% % \end{figure}
% % Figure \ref{fig:CyberPhysical} depicts an overview of the cyber-physical system. 
% % The discrete controller contains the finite state machine (FSM) that switches between the two states $Q = \{0,1\}$ with $q \in Q$. 
% In this section, the vehicle system under the proposed feedback control is modeled using hybrid system model in (\ref{model:generalhybridsystem}). The state $\agx$ of the vehicle system is defined as \pn{[NW: $\agx$ is duplicated defined.]}
% \pn{\begin{equation}\label{eq:2}
%     \agx := (x, v, q) = (x_{1}, x_{2}, x_{3}, x_{4}, v, q)\in \mathbb{R}^{5}\times Q
% \end{equation}}
% where $x_{1}$ and $x_{2}$ denote the position of the vehicle in $\mathbb{R}^{2}$, $x_{3}$ is the orientation of the vehicle in the $XY$ plane, $x_{4}$ is the steering angle of the vehicle, $v$ is an additional state for the velocity, $q$ is an additional state that represents which controller is being used, and $Q:= \{0, 1\}$.
% The control input $u\in \mathbb{R}^{2}$ of the vehicle system is assumed to be constrained by $u \in [\delta_{min}, \delta_{max}] \times [v_{min}, v_{max}]$, where $\delta_{min}$ and $\delta_{max}$ are the minimum and maximum turning angles of the vehicle, and $v_{min}$ and $v_{max}$ are the minimum and maximum speeds of the vehicle. The \pn{control input} functions, which are detailed in Section \ref{sec:control}, are defined as \pn{$\kappa_q: \mathbb{R}^4 \to \mathbb{R}^{2}$, for both $q\in Q$.} 
% %The transition function of the FSM is defined by Equation \eqref{eq:5}.

% The continuous physical system can be defined as an extension of Equation \eqref{eq:car_robot} as
% \pn{\begin{equation} \label{eq:3}
% \dot{\agx} = 
%     \begin{bmatrix} v \cos(x_{3}) \\
%                     v \sin(x_{3}) \\
%                     \frac{v \tan(x_{4})}{l} \\
%                     \begin{bmatrix}
%                         -\delta\\
%                         -v
%                     \end{bmatrix} + \kappa_q(x)\\ 
%                     0
%     \end{bmatrix} =: f(\agx)
% \end{equation}}
% where $l$ is the length of the vehicle. Note that the flow is allowed when 1) $q = 0$ and $(x_{1}, x_{2})\in \mathcal{N}(\mathcal{C})$, or 2) $q = 1$ and $(x_{1}, x_{2})\notin \mathcal{N}(\mathcal{C})$. Hence, 
% \pn{\begin{equation}
% \label{eq:flowset}
% \begin{aligned}
%     &C := \{\agx \in \mathbb{R}^{5}\times Q: q = 0, (x_{1}, x_{2})\in \mathcal{N}(\mathcal{C})\}\\
%     & \cup \{\agx \in \mathbb{R}^{5}\times Q: q = 1, (x_{1}, x_{2})\notin \mathcal{N}(\mathcal{C})\}
% \end{aligned}
% \end{equation}}

% The discrete evolution of the controller is described by the equation
% \pn{\begin{equation} \label{eq:4}
% \agx^+ = \begin{bmatrix} x_{1}^{+} \\
%                     x_{2}^{+} \\
%                     x_{3}^{+} \\
%                     x_{4}^{+} \\
%                     v^{+} \\ 
%                     q^{+}
%     \end{bmatrix}  =
%     \begin{bmatrix} x_{1} \\
%                     x_{2} \\
%                     x_{3} \\
%                     x_{4} \\
%                     v \\ 
%                     1- q
%     \end{bmatrix} =: g(\agx)
% \end{equation}}
% % where
% % \begin{equation} \label{eq:5}
% % N(z) = 
% % \begin{cases} 
% %       1  & z \in \mathcal{N}\\
% %       0  & z \notin \mathcal{N}
% %    \end{cases}
% % \end{equation}
% % and $\mathcal{N}$ is the set that describes the neighborhood to our path. 
% Notice that there is no change in vehicle state during jumps; the only change made is to the additional state $q$ to govern which control strategy is being used. The jumps are allowed when 1) $q = 1$ and $\pn{(x_{1}, x_{2})\in \mathcal{N}(\mathcal{C})}$ \pn{[NW: is this inclusion correct?]}, or 2) $q = 0$ and $(x_{1}, x_{2})\notin \mathcal{N}(\mathcal{C})$. Therefore, 
% \pn{\begin{equation}
% \label{eq:jumpset}
% \begin{aligned}
%     &D := \{\agx \in \mathbb{R}^{5}\times Q: q = 1, (x_{1}, x_{2})\in \mathcal{N}(\mathcal{C})\}\\
%     & \cup \{\agx \in \mathbb{R}^{5}\times Q: q = 0, (x_{1}, x_{2})\notin \mathcal{N}(\mathcal{C})\}
% \end{aligned}
% \end{equation}}

% In conclusion, the hybrid model of the vehicle system is given by  (\ref{model:generalhybridsystem}) where the flow map $f$ is given in (\ref{eq:3}), the flow set $C$ is given in (\ref{eq:flowset}), the jump map $g$ is given in (\ref{eq:jumpset}), and the jump set $D$ is given in (\ref{eq:4}).
% % \subsection{Hybrid Equations}
% % Hybrid systems combine elements of continuous and discrete systems through flows and jumps\cite{hybridsystems}. The system designed in this paper will flow according to the physics and control law being applied and jump when the vehicle moves in or out of the set $\mathcal{N}$. The conditions for flowing and jumping can be described with the following functions: $F$ the flow map, $C$ the flow set, $G$ the jump map, and $D$ the jump set. Our flow map $F$ can be described by Equation \eqref{eq:3}, and the jump map $G$ by Equation \eqref{eq:4}. The flow set $C$ is the set in which the system evolves according to $F$. Here we can impose constraints on the system, namely the steering angle and velocity limitations. The flow set is given by the equation 
% % \begin{equation} \label{eq:6}
% % \begin{aligned}
% %     z \in C :=  \mathbb{R}^3  \times [\delta_{min}, \delta_{max}]  \times  [v_{min}, v_{min}] \times Q.
% % \end{aligned}
% % \end{equation}
% % This set allows the system to flow anywhere on the plane, as long as the steering angle is $\delta_{min}\leq \delta \leq \delta_{max}$, and the velocity is $v_{min} \leq v \leq v_{max}$. 

% % The jump set $D$ is defined as the set in which jumps are allowed to occur. The system should jump whenever the vehicle leaves or enters the set $\mathcal{N}$ to update the control strategy. The jump set is given by the equation 
% % \begin{equation} \label{eq:7}
% % \begin{aligned}
% %     z & \in D := 
% %     \begin{cases}
% %         \mathcal{N} & q = 1 \\
% %         \mathcal{N}' & q = 0
% %     \end{cases}.
% % \end{aligned}
% % \end{equation}