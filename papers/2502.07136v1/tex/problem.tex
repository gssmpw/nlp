%% Mathematical model
\section{Problem Statement}\label{section:problem}
\label{sec:mathematical_model}
Consider the kinematic car-like robot model in~\cite{AkhNieWas2015} 
\begin{equation}
\begin{aligned}
\label{eq:car_robot}
    \dot{x}_{1} &= v\cos x_{3}, \quad\quad \dot{x}_{2} = v\sin x_{3}, \\ \dot{x}_{3} &= \frac{1}{\ell}\tan x_{4}, \quad\quad \dot{x}_{4} = \omega,
\end{aligned}
\end{equation}
% \begin{aligned}
%   \dot{x}_{1} &= v\cos x_{3}, \dot{x}_{2} = v\sin x_{3}, \\ \dot{x}_{3} &= \frac{1}{\ell}\tan x_{4}, \dot{x}_{4} = \omega,
% \end{aligned}
% \begin{equation}
% \label{eq:car_robot}
%  \dot{x}=
%   \left[
%   \begin{array}{cc}
%       \cos{x_{3}} & 0\\
%       \sin{x_{3}} & 0\\
%       \frac{1}{\ell}\tan x_{4} & 0\\
%       0 & 1\\
%   \end{array} \right]\left[
%   \begin{array}{c}
%       v\\
%       \omega\\
%   \end{array}\right],
% \end{equation}
where $(x_1,x_2)\in\Real^2$ is the position in the two-dimensional plane, $x_3\in\Real$ is the orientation, the constant $\ell\in\mathbb R_{>0}$ is the length of the car-like robot, and $x_4\in\Real$ is the steering angle. For a constant upper bound $x^{{\max}}_4 \in (0,\pi/2)$, the steering angle has the following limits:
\begin{equation}
 |x_4| \leq x^{{\max}}_4.
\label{eq:constraint}
\end{equation}
The input $(v,\omega) \in \Real^{2}$ is the translational speed and angular velocity, respectively. The state $x = (x_{1}, x_{2}, x_{3}, x_{4})$ is assumed to be measurable, but we define the output of~\eqref{eq:car_robot} as the position of the car-like robot in the plane, given by
%
%
\begin{equation}
  \tilde{y} =\tilde{h}(x):={\left(x_1, x_2\right)}.
\label{eq:output}
\end{equation}
This output is used as feedback to the path-invariant controller.

\begin{assumption}
  Given the steering angle constraint~\eqref{eq:constraint}, the
  curvature of the curve $\sigma$ in~\eqref{eq:general_path}, denoted $K(\lambda)$, 
  satisfies
  $
  K^{\max}\eqdef \sup_{\lambda \in \nw{\dom \sigma}} K(\lambda) < \frac{1}{\ell}\tan{(x^{\max}_4)}
  $
% $
%  K(\lambda)
%  < \frac{1}{\ell}\tan{(x^{\text{max}}_4)},
%  \label{eq:curvature}
% $
for all $\lambda\in\dom \sigma$.
%Moreover, for a constant $\vrm > 0$, $x_{5}$ is such that $\vrm + \pn{x_5} \neq 0$ \pn{[NW: What is $x_{5}$?]}.
\label{ass:SteeringAngle}
\end{assumption}

Assumption~\ref{ass:SteeringAngle} ensures that the path is feasible,
in light of the steering angle constraint, for the car-like robot.

\begin{problem}\label{problem:globalinvariance}
Given a car-like robot modeled as in~\eqref{eq:car_robot} with constraint in (\ref{eq:constraint}), and a curve $\mc{C}$ satisfying Assumptions~\ref{ass:implicit} and~\ref{ass:SteeringAngle}, design a controller $\kappa:\reals^{4}\to\reals^{2}$ such that, by applying $(v, w) = \kappa(x)$ to (\ref{eq:car_robot}) and with nonzero initial speed, the following holds: i) the output of (\ref{eq:car_robot}) converges to the curve $\mc{C}$ from any point in $\reals^{4}$, ii) the constraint in ~\eqref{eq:constraint} is satisfied,
% $$
% \mathcal{B} = \{x_{0}\in\reals^{4}: (\hat{x}(t|x_{0}), \kappa(\hat{x}(t|x_{0}))\notin \mathcal{S}\quad\forall t\in\mathbb{R}_{\geq 0}\}
% $$ 
% where $\hat{x}(t|x_{0})$ denotes the predicted state $x$ of the system (\ref{eq:car_robot}) controlled by $(v, w) = \kappa(x)$, denoted $P$, at time $t$ when starting from the initial state $x_{0}$, and $\mathcal{S}\subset\mathbb{R}^{6}$ is the set of singular states and inputs for $P$.
and iii) the system tracks the curve with a given non-zero speed, in the sense that the \pn{associated} set $\gamma$ is invariant. 
% Given a car-like robot modeled as in~\eqref{eq:car_robot}, and a curve $\mc{C}$ satisfying Assumptions~\ref{ass:implicit} and~\ref{ass:SteeringAngle}, design a controller such that output of the system converges to the curve $\mc{C}$ with a basin of attraction equal to $\Real^2$. Moreover, the system tracks the curve with a given non-zero speed, in the sense that the set $\gamma$ is invariant. 
\end{problem}
\ifbool{conf}{To \pn{solve} Problem \ref{problem:globalinvariance}, the hybrid control framework shown in Figure \ref{fig:controldiagram} \mynne{is employed}, comprising a global tracking controller that, powered by motion planning technology, drives the car-like robot into the neighborhood of the path, in conjunction with a local controller that guarantees path invariance of this neighborhood. A hysteresis-based switching scheme, following the uniting control framework introduced in \cite{San2021}, combines the two controllers.}{
 To \pn{solve} Problem \ref{problem:globalinvariance}, the hybrid control framework shown in Figure \ref{fig:controldiagram} \mynne{is employed}, comprising a global tracking controller that, powered by motion planning technology, drives the car-like robot into the neighborhood of the path, in conjunction with a local controller that guarantees path invariance of this neighborhood. Initially, the motion planner is tasked to generate a motion plan steering the robot to the neighbourhood of the desired path, where the locally path-invariant controller is effective. The global tracking controller is responsible for tracking the aforementioned motion plan, effectively guiding the robot into the designated neighborhood. Upon reaching this neighborhood, the hybrid control logic selects the locally path-invariant controller, which subsequently ensures path invariance. Conversely, if the robot exits some larger neighborhood of the path, designed to prevent chattering, the global controller takes over. This logic introduces \mynne{hysteresis} to prevent chattering when switching between the controllers, thereby ensuring robust switching.}
% The switching scheme, following the uniting control framework introduced in \cite{San2021}, is designed based on the \pn{distance between the position of the car-like robot and the path.}}
\begin{figure*}[htbp]
    \centering
    \incfig[1.73]{framework_globalpathinvariant}
    \caption{The hybrid control diagram of the globally path-invariant control framework.}
    \label{fig:controldiagram}
    \vspace{-0.6cm}
\end{figure*}

\section{Local Path-invariant Control Design}\label{section:localcontrol}
\label{sec:path-invariant-control-design}
In this section, we design a local controller to achieve the path invariance property. The \mynn{design} of \mynn{the} local controller is based on the transverse feedback linearization technique~\cite{NieMag04,AkhNieWas2015}. 
A necessary condition for feedback linearization is that the system possesses a well-defined vector relative degree~\cite{Isi95}. The next result shows that there does not exist any function in the output space of~\eqref{eq:car_robot} for which the system has a well-defined vector relative degree\myifconf{}{; \mynnd{see Definition \ref{def:vector_relative_deg}}}. 
\begin{lemma}
\label{lemm:no-vector-relative-degree}
    Given the system dynamics defined in~\eqref{eq:car_robot} and \mynnd{any} two smooth functions $(x_1,x_2)\mapsto A(x_1,x_2)\in\reals$ and $(x_1,x_2)\mapsto B(x_1,x_2)\in\reals$, the system~\eqref{eq:car_robot} does not have a well-defined vector relative degree. 
\end{lemma}
\myifconf{
\begin{proof}
    See \cite{wang2025hybrid}.
\end{proof}
}{\begin{proof}
    We can write the system~\eqref{eq:car_robot} as $\dot x = f(x) + g_1(x)v + g_2(x)\omega$, with $f(x) = [0,0,1/\ell\tan x_4,0]^\top$, $g_1(x) = [\cos x_3,\sin x_3, 0, 0]^\top$ and $g_2(x) = [0,0,0,1]^\top.$ {\akh For any smooth function $A$ and $B$, direct calculations result in $L_{g_{2}}A = 0$ and $L_{g_{2}}B = 0$, and the decoupling matrix given by
        $\begin{bmatrix}
            L_{g_{1}}A & L_{g_{1}}B\\
            L_{g_{2}}A & L_{g_{2}}B
        \end{bmatrix}$
        is always singular for any $A$ and $B$.} This implies that the system~\eqref{eq:car_robot} does not have a well-defined vector relative degree for any $A$ and $B$.
\end{proof}}

As a consequence of Lemma~\ref{lemm:no-vector-relative-degree}, it is not even possible to convert the given system~\mynne{\eqref{eq:car_robot}} into a \textit{partially linear} system~\mynne{~\cite[(4.26)]{Isi95}} using a \mynnd{pre-}feedback and coordinate transformation because the system does not possess a well-defined vector relative degree~\cite{Isi95}. 
To achieve a well-defined vector relative degree, similar to~\cite{AkhNieWas2015,AkhNie2011}, we perform dynamic extension by treating the control input $v\in\Real$ as a state and ``extend'' the system by adding two auxiliary states $x_5$ and $x_6$. 
% Let $x_5\in \mathbb{R}$ and $x_6\in \mathbb{R}$, be the two auxiliary states, and 
For a constant $\vrm \neq 0$, let  $v = \vrm + x_5$
% , where 
% $\zeta_1$ is
% the first state of our dynamic controller and 
 % is constant, $x_5 \in \reals\backslash \{-\vrm\}$, 
and $x_6 \in\Real$. \mynn{The dynamics of $x_{5}$ and $x_{6}$ are defined} as $\dot x_5 = x_6$ and $\dot{x}_6 = u_1$, where
$u_1$ is a new auxiliary input. In simple words, we have delayed the control input $v$ \mynn{via a double} integrator. Next, we relabel the input $\omega$ as $u_2$ and denote the state of the extended system as $\agx\eqdef (x_1, x_{2}, \cdots,x_6)$. \mynn{T}he extended system is defined as follows:
%
%
% In this section, we design a controller using transverse feedback linearization to achieve local path invariance ~\cite{NieMag04} for car-like robot. To achieve a well-defined vector relative degree for any curve in $\Real^2$, we perform dynamic extension~\cite{AkhNieWas2015} by adding two extra states of controller, \ak{$\zeta_1\in \mathbb{R}$ and $\zeta_2\in \mathbb{R}$, that capture input information}. Let $v = \vrm + \zeta_1$, where 
% % $\zeta_1$ is
% % the first state of our dynamic controller and 
% $\vrm \neq 0$ is
% constant. Furthermore, we define $\dot{\zeta_1} = \zeta_2$ and $\dot{\zeta}_2 = u_1$, where
% $u_1$ is a new auxiliary input. To simplify notation, henceforth, we do not distinguish between
% physical states of the system $(x_1,x_2, x_3, x_4)$ and states of the
% controller $(\zeta_1, \zeta_2)$. Let $x_5 \eqdef \zeta_1\in \reals\backslash \{-\vrm\}$, $x_6
%  \eqdef \zeta_2$ for some $\vrm > 0$. We denote the state of the extended system as $\agx\eqdef \col(x_1,\cdots,x_6)$. Therefore, the extended system is defined as follows:
%$ \dot{x}= f(x) + g_1(x)u_1 + g_2(x)u_2 $ where
\begin{equation}
\begin{aligned}
\label{eq:dynamic_car_robot}
 &\dot{\agx} = F_{P}(\agx, u)\eqdef f(\agx) + g_1(\agx)u_1 + g_2(\agx)u_2\\
  &  \mynn{:=}\left[
  \begin{array}{c}
      (\vrm + x_{5})\cos x_{3}\\
      (\vrm + x_{5})\sin x_{3}\\
      \frac{(\vrm + x_{5})}{\ell}\tan x_{4} \\
      0\\
      x_{6}\\
      0\\
  \end{array}\right]+ \left[
  \begin{array}{cc}
      0 \\
      0 \\
      0 \\
      0 \\
      0 \\
      1 \\
  \end{array}\right]u_1 +\left[
  \begin{array}{cc}
      0\\
      0\\
      0\\
      1\\
      0\\
      0\\
  \end{array}\right]u_2.
\end{aligned}
\end{equation}
%
The output of the extended model of the car-like robot is defined as
$
    {y} = h(\agx) \mynnd{:} = (x_1,x_2).
$
We lift the path $\gamma$ to the extended state space and construct the following set:
$$
\Gamma \eqdef \left(s \circ h\right)^{-1}(0) = \set{ \agx \in
  \mathbb{R}^6: s(h(\agx)) = 0},
\label{eq:lift}
$$
\mynn{with} the map $s$ \mynn{given as} in Assumption~\ref{ass:implicit}.
It should be noted that steering the output of the system to the curve is equivalent to steering the state of~\eqref{eq:dynamic_car_robot} to $\Gamma$. 
\mynn{For some positive real number $\by \in \left[0, \frac{\ell}{\tan{(x^{{\max}}_4)}}\right)$, where $\frac{\ell}{\tan{(x^{{\max}}_4)}}$ represents the minimal radius of the vehicle,} we construct a neighborhood of the path $\gamma$ \mynn{as follows}:
\begin{equation}
    \label{eq:nbh_set}
    {\mc{N}_{\gamma}^{\by} \eqdef \set{ y \in \Real^2 : \norm{y}_{\gamma} < \by}}.
\end{equation}
We lift this neighborhood \mynn{to the space of $\overline{x}$ and define}
\begin{equation}
    \label{eq:nbh_lift_set}
    {\mc{N}_{\Gamma}^{\by}} \eqdef \set{ \agx\in \Real^6 : \norm{\agx}_{\Gamma} < \by}.
\end{equation}

To solve Problem~\ref{problem:globalinvariance}, we need to satisfy two requirements: i) render the path invariant and \pn{(locally)} attractive, and ii) move along the curve \mynn{satisfying a given velocity profile}. 
%It should be noted that fulfilling the first requirement alone is trivial as one can turn off all the control inputs once the robot reaches the curve, and make the system stay on the path forever. 
However, fulfilling the path invariance requirement with motion along the curve with non-zero speed is challenging, as it is not easy to guarantee that once the robot reaches the path, it will never leave the path. In other words, we require a control law that renders a subset of $\mc{N}_\Gamma$
% \begin{equation}
%     \label{eq:nbh_lift_set}
%     {\mc{N}_{\Gamma}} \eqdef \set{ \agx\in \mc{N}^{\uparrow} : \vrm + x_5 \geq \delta}
% \end{equation}
invariant and \mynn{finite-time} attractive along the path, namely, satisfying $s(x) = 0$, where the condition $\vrm + x_5 = 0$ does not occur. To \mynne{satisfy requirements} i and ii, we exploit both the parametric and zero-level set representation of the path to construct two functions in the output space and invoke the local transverse feedback linearization design procedure~\cite{NieFulMag10,AkhNieWas2015}. This \mynn{requires} the formulation of a virtual output function.
%%%%%%%%%%%%%%%%%%%%%%%%%%%%%%%%%%%%%%%%%%%%%%%
% We treat the path following problem as a set stabilization problem and
% we follow the general approach of~\cite{NieFulMag10, HlaNieWan13}. In
% order to satisfy $\textbf{PF1}$ and $\textbf{PF2}$ we first stabilize
% the path following manifold $\Gamma^\star$. { Once the
%   path manifold has been stabilized we use the remaining freedom in the
%   control law to impose desired dynamics on the path and satisfy
%   $\textbf{PF3}$.}

\subsection{Virtual Output Function}
We construct a virtual output function using the definition of the path. \mynnd{Recalling the definition of} the \nbhd $\mc{N}_\gamma^{\by}$ in~\eqref{eq:nbh_set}, \mynn{there exists a small enough $\by$} such that if $y \in \mc{N}_{\gamma}^{\by}$ then there exists a unique $y^\star \in \gamma$ such
that $\norm{y}_\mc{\pn{\gamma}} = \norm{y - y^\star}$. This allows us to define the
function
\begin{equation}
\varpi : \; \mc{N}_{\gamma}^{\by} \rightarrow \nw{\dom \sigma},\; y \mapsto \varpi(y) :=\arg
\inf_{\lambda \in \nw{\dom \sigma}}\norm{y - \sigma(\lambda)}.
\label{eq:proj}
\end{equation}
This function is at least three times continuously differentiable. Next, we define \ak{the} virtual output function
%
%
\begin{equation}\label{eq:virtual}
\begin{aligned}
\hat{y} = \left[\begin{array}{c}\pi(\agx)\\\alpha(\agx) \end{array}\right] \eqdef
\left[\begin{array}{c}\varpi \circ h (\agx)\\s \circ h(\agx)\end{array}\right].
\end{aligned}
\end{equation}
% \begin{lemma}
% There does not exists any curve $\mc{C} \in \Real^2$ such that the corresponding path following output~\eqref{eq:virtual} has a well-defined vector relative degree for system~\eqref{eq:car_robot}. 
% \end{lemma}
% To overcome this problem let $v = \vrm + \zeta_1$, where $\zeta_1$ is
% the first state of our dynamic controller and $\vrm \neq 0$ is
% constant. We take the simplest possible structure for the control
% law~\eqref{eq:control} and let $\dot{\zeta_1} = \zeta_2$. In order to
% finish defining the control law we let $\dot{\zeta}_2 = u_1$ where
% $u_1$ is a new, auxiliary input. To simplify notation, henceforth we do not distinguish between
% physical states of the system $(x_1,x_2, x_3, x_4)$ and states of the
% controller $(\zeta_1, \zeta_2)$. Let $x_5 \eqdef \zeta_1$, $x_6
%  \eqdef \zeta_2$. Therefore the system we study has the form 
% %$ \dot{x}= f(x) + g_1(x)u_1 + g_2(x)u_2 $ where
% \begin{equation}
% \begin{aligned}
% \label{eq:dynamic_car_robot}
%  \dot{x}&= f(x) + g_1(x)u_1 + g_2(x)u_2\\&= \left[
%   \begin{array}{c}
%       (\vrm + x_{5})\cos x_{3}\\
%       (\vrm + x_{5})\sin x_{3}\\
%       \frac{(\vrm + x_{5})}{\ell}\tan x_{4}\\
%       0\\
%       x_{6}\\
%       0\\
%   \end{array}\right]+ \left[
%   \begin{array}{cc}
%       0 \\
%       0 \\
%       0 \\
%       0 \\
%       0 \\
%       1 \\
%   \end{array}\right]u_1 +\left[
%   \begin{array}{cc}
%       0\\
%       0\\
%       0\\
%       1\\
%       0\\
%       0\\
%   \end{array}\right]u_2
% \end{aligned}
% \end{equation}
The following results guarantee that the extended system in (\ref{eq:dynamic_car_robot}) with the output~\eqref{eq:virtual} has a well-defined vector relative degree (see \myifconf{\cite{Sastry}}{Definition~\ref{def:vector_relative_deg} in Appendix}) on $\mc{N}_{\Gamma}^{\by}$. It will be shown in the subsequent results that a well-defined vector-relative degree is key to guaranteeing the invertibility of a matrix that is required for the control law. \mynnd{Below, $\mc{N}^\star \eqdef \mc{N}_\Gamma^{\by} \setminus \set{\agx\in \Real^6 : \vrm + x_5  = 0}$.}
\begin{proposition}
\label{prop:relative_degree}
  Suppose Assumptions~\ref{ass:implicit} and~\ref{ass:SteeringAngle} hold. Then the extended model of the car-like robot~\eqref{eq:dynamic_car_robot} with output~\eqref{eq:virtual}
yields a well-defined vector relative degree of $\{3,3\}$ at each point {in} \akh{$\mc{N}^\star$}.
\end{proposition}
\myifconf{\begin{proof}
    See \cite{wang2025hybrid}.
\end{proof}}{\begin{proof}
The proof is similar to Lemma~{{III.1}} in~\cite{AkhNieWas2015}. 
  Let $x \in \mc{N}^\star$ be arbitrary. 
  % By definition of $\Gamma$ the output $h(x^\star)$ is on the path $\gamma$. Let $\lambda^\star \in \mathbb{D}$ be such that $h(x^\star) = \sigma(\lambda^\star)$.  
  By the definition of vector relative degree
  we must show that i) $ L_{g_1}L^i_f\pi(x) = L_{g_2}L^i_f\pi(x) =
  L_{g_1}L^i_f\alpha(x) = L_{g_2}L^i_f\alpha(x) = 0$ for $i \in \{0,
  1\}$ in a \nbhd of $x^\star$ and ii) the decoupling matrix
\begin{equation}
\label{eq:decoupling_matrix_general}
 D(x)=
  \left[\begin{array}{c c}
      L_{g_{1}}L_{f}^{2}\pi(x) & L_{g_{2}}L_{f}^{2}\pi(x)\\
      L_{g_{1}}L_{f}^{2}\alpha(x) & L_{g_{2}}L_{f}^{2}\alpha(x)\\
  \end{array}\right]
\end{equation}
is non-singular at $x\in\mc{N}^\star$. Since
\[
\DER{\pi(x)}{x_i} = \DER{\alpha(x)}{x_i} \equiv 0
\]
for {\akh each} $i \in \{3, 4, 5, 6\}$, direct calculations give
$L_{g_j}L^i_f\pi(x) = L_{g_j}L^i_f\alpha(x) = 0$ for $i \in \{0,1\}$,
  $j \in \{1, 2\}$. This satisfies the first condition of the vector relative degree.
{First, we write expressions for each entry of the decoupling matrix when $\alpha(x) = x_1^2 + x_2^2 - 1$ and $\pi(x) = \tan^{-1}(x_2/x_1)$. The closed-form expressions 
\begin{align*}
    L_{g_{1}}L_{f}^{2}\alpha(x) = 2 x_1 \cos x_3 + 2 x_2 \sin x_3,
\end{align*}
\begin{align*}
    L_{g_{2}}L_{f}^{2}\alpha(x) =\frac{2 (\vrm + x_5)^2 (\tan x_4^2 + 1) (x_2 \cos x_3 - x_1 \sin x_3)}{\ell},
\end{align*}
\begin{align*}
    L_{g_{1}}L_{f}^{2}\pi(x)=-\frac{x_2 \cos x_3 - x_1 \sin x_3}{x_1^2 + x_2^2},
\end{align*}
\begin{align*}
    L_{g_{2}}L_{f}^{2}\pi(x)=\frac{(\vrm + x_5)^2 (\tan x_4^2 + 1) (x_1 \cos x_3 + x_2 \sin x_3)}{\ell (x_1^2 + x_2^2)}.
\end{align*}
  For arbitrary functions, similar expressions can be computed from Maple. Next, we show that the decoupling
  matrix~\eqref{eq:decoupling_matrix_general} is non-singular at $x \in
  \mc{N}^\star$, we first find that
\begin{equation*}
\begin{aligned}
\displaystyle \det{(D(x))}=
\frac{(\vrm+x_{5})^2}{\ell\cos^{2}x_{4}}&\left(\partial_{x_{1}}\pi(x) \partial_{x_{2}}\alpha(x)\right.\\
 &\left.- \partial_{x_{2}}\pi(x) \partial_{x_{1}}\alpha(x) \right).
\end{aligned}
\end{equation*}
The only way for this determinant to vanish is if either (i) $\vrm =
-x_{5}$ or (ii)
$\left(\partial_{x_{1}}\pi(x) \partial_{x_{2}}\alpha(x) -
  \partial_{x_{2}}\pi(x)\partial_{x_{1}}\alpha(x)\right)$. Condition
(i) does not occur for $x \in \mc{N}^\star$. We now
argue that condition (ii) never occurs on the path because the term $\left(\partial_{x_{1}}\pi(x) \partial_{x_{2}}\alpha(x) -
  \partial_{x_{2}}\pi(x)\partial_{x_{1}}\alpha(x)\right)$ is the cross product of vectors
$\col(\partial_{x_{1}}\alpha,\partial_{x_{2}}\alpha)$ and
$\col(\partial_{x_{1}}\pi,\partial_{x_{2}}\pi)$. The cross product is zero if and only if these two vectors are linearly dependent. The construction of the maps $\alpha$ and $\pi$ in~\eqref{eq:virtual} along with Assumption~\ref{ass:implicit} guarantees that $\left(\partial_{x_{1}}\pi(x) \partial_{x_{2}}\alpha(x) -
  \partial_{x_{2}}\pi(x)\partial_{x_{1}}\alpha(x)\right) \neq 0$. This satisfies the second condition of the vector relative degree and the system has a well-defined vector relative degree of $\{3,3\}$ at each point \ak{in} {$\mc{N}^\star$}. 

% Returning to the expression for $\det{(D(x))}$, we have that
% \[
% \begin{aligned}
%   \sigma^\prime_{1}(\lambda^\star)\partial_{x_{2}}\alpha -
%   \sigma^\prime_{2}(\lambda^\star)\partial_{x_{1}}\alpha &=
%   \inner{R_{\frac{\pi}{2}}\D s^\top_{h(x^\star)}}{\sigma^\prime(\lambda^\star)}\\
%   &= k(\sigma(\lambda^\star))\inner{\sigma^\prime(\lambda^\star)}{\sigma^\prime(\lambda^\star)}\\
%   &= k(\sigma(\lambda^\star))\|\sigma^\prime(\lambda^\star)\|^2\\
%   &= k(\sigma(\lambda^\star)).
% \end{aligned}
% \]

}
\end{proof}}

The next result is a direct consequence of Proposition~\ref{prop:relative_degree}.
\begin{corollary}\label{cor:diffeo}
 The map $\mathscr{T}: \mc{N}^\star \rightarrow V \subset \Real^6 $ is defined as
  \begin{equation}
  \label{eq:diffeo}
        \mathscr{T}(\mynne{x}) := (L^{i-1}_f\pi(\mynne{x}),L^{i-1}_f\alpha(\mynne{x}))
  \end{equation}
%  
%   $T : \mc{N}_\Gamma{\myblue \setminus \set{\agx\in \Real^6 : \vrm + x_5 \neq 0} } \rightarrow
%   V \subset \Real^6$, where 
% \begin{equation}
% \label{eq:diffeo}
% T(x^\star) = (\eta_i,\xi_i) \eqdef (L^{i-1}_f\pi(x^\star),L^{i-1}_f\alpha(x^\star))
% \end{equation}
% \begin{equation}
% T(x^\star) = \left[\begin{array}{c}\eta_1\\\eta_2\\\eta_3\\ \xi_1\\\xi_2\\\xi_3\end{array}\right]
%  \eqdef
% \left[\begin{array}{c}\pi(x^\star)\\L_f\pi(x^\star)\\L^2_f\pi(x^\star)\\ \alpha(x^\star)\\L_f\alpha(x^\star)\\L^2_f\alpha(x^\star)\end{array}\right]
% \label{eq:diffeo}
% \end{equation} 
for $i\in\set{1,2,3}$ and, for any $\mynne{x} \in \mc{N}^\star $,
is a diffeomorphism onto its image $V\subset \Real^6$.
\end{corollary}
\myifconf{\begin{proof}
    See \cite{wang2025hybrid}.
\end{proof}}{\begin{proof}
% Let $\mc{N}^\star \eqdef \mc{N}_\Gamma^{\by} \setminus \set{\agx\in \Real^6 : \vrm + x_5  = 0}$. 
For each $x\in \mc{N}^\star$, the explicit expressions of $\pi$ and $\alpha$ and their Lie derivatives define transformation 
\begin{equation}
\mathscr{T}(x) = \left[\begin{array}{c}\eta_1\\\eta_2\\\eta_3\\ \xi_1\\\xi_2\\\xi_3\end{array}\right]
\eqdef 
\left[\begin{array}{c}\pi(x)\\L_f\pi(x)\\L^2_f\pi(x)\\ \alpha(x)\\L_f\alpha(x)\\L^2_f\alpha(x)\end{array}\right]
\label{eq:diffeo}
\end{equation}
in closed from. In order to show that~\eqref{eq:diffeo} is a diffeomorphism in a
  \nbhd of each point $x \in \mc{N}^\star  $ we appeal to the generalized inverse
  function theorem~\cite[pg. 56]{GuiPol74}. We must show that 1) for
  all $x \in \mc{N}^\star$, the Jacobian of $\mathscr{T}$ is an isomorphism, and 2)
  $\left.\mathscr{T}\right|_{\mc{N}^\star } : \mc{N}^\star  \to
  \mathscr{T}(\mc{N}^\star)$ is a diffeomorphism. An immediate consequence of
  Proposition~\ref{prop:relative_degree} and~\cite[Lemma 5.2.1]{Isi95} is
  that the first condition holds. To show that the second condition
  holds we explicitly construct the inverse of $\mathscr{T}$ restricted to
  $\mc{N}^\star$. On $\mc{N}^\star$, $\xi_1(x) = \xi_2(x) =
  \xi_3(x) = 0$ and simple calculations show that the inverse of $\mathscr{T}$
  restricted to $\mc{N}^\star$ is\footnote{The inverse is obtained
    under the assumption that the curve is arc-length parameterized.}
\[
\left[\begin{array}{c}x_1\\x_2\\x_3\\ x_4\\x_5\\x_6\end{array}\right]
= \left.\mathscr{T}\right|^{-1}_{\mc{N}^\star}(\eta, 0) =
\left[\begin{array}{c}\sigma_1(\eta_1)\\\sigma_2(\eta_1)\\\varphi(\eta_1)\\
    \arctan{\left(\ell K(\eta_1)\right)}\\\eta_2 - \vrm \\\eta_3\end{array}\right]
\]
where $\varphi : \dom\varphi \to \mod{\Real}{2\pi}$ is the map that
associates to each $\eta_1 \in \dom\varphi$ the angle of the tangent
vector $\sigma^\prime(\eta_1)$ to path $\gamma$ at $\sigma(\eta_1)$ and
$K$ is the signed curvature. The inverse
is clearly smooth which shows that $\left.\mathscr{T}\right|_{\mc{N}^\star}$
is a diffeomorphism onto its image.
\end{proof}}
Corollary~\ref{cor:diffeo} allows \mynne{to express} the system dynamics in $\xi$ and $\eta$ coordinates everywhere on $\mc{N}^\star$ and is given by
\begin{equation}
\label{eq:dynamic_quadrotor_new_coordinates}
\begin{aligned}
      \dot{\eta}_1 &= \eta_2\quad\dot{\eta}_2 = \eta_3\\
      \dot{\eta}_3 &= L_{f}^{3}\pi(x) +  \left(L_{g_1}L^2_{f}\pi(x)\right) u_1 +  \left(L_{g_2}L^2_{f}\pi(x)\right) u_2\\
      \dot{\xi}_1 &= \xi_2\quad\dot{\xi}_2 = \xi_3\\
      \dot{\xi}_3 &= L_{f}^{3}\alpha(x) +  \left(L_{g_1}L^2_{f}\alpha(x)\right) u_1 +  \left(L_{g_2}L^2_{f}\alpha(x)\right) u_2\\
\end{aligned}
\end{equation}

Next, we define a feedback $\kappa_{\mathrm{fb}}:\mc{N}^\star \to \Real^2 $, which is well-defined for all $x \in \mc{N}^\star$  by Proposition~\ref{prop:relative_degree}, as follows:
\begin{eqnarray}
\label{eq:regular_feedback_transformation}
 \left[\!\!\!
  \begin{array}{c}
      u_{1}\\
      u_{2}\\
  \end{array} \!\!\!\right]= \kappa_{\mathrm{fb}}(x):=D^{-1}(x)\left( \left[\!\!\!
  \begin{array}{c}
      -L_{f}^{3}\ak{\pi(x)}\\
      -L_{f}^{3}\ak{\alpha(x)}
  \end{array} \!\!\!\right]+
  \left[\!\!\!
  \begin{array}{c}
      v^{\parallel}\\
      v^{\pitchfork}\\
  \end{array} \!\!\!\right]
  \right),
\end{eqnarray}
where $(v^{\parallel},v^{\pitchfork}) \in \Real^{2}$ are
auxiliary control inputs and 
$
\label{eq:decoupling_matrix_general1}
 D(x)=
  \left[\begin{array}{c c}
      L_{g_{1}}L_{f}^{2}\pi(x) & L_{g_{2}}L_{f}^{2}\pi(x)\\
      L_{g_{1}}L_{f}^{2}\alpha(x) & L_{g_{2}}L_{f}^{2}\alpha(x)\\
  \end{array}\right].
$
On the set $\mc{N}^\star$, the coordinate transformation~\eqref{eq:diffeo} and the feedback~\eqref{eq:regular_feedback_transformation} converts the system into the linear system
\begin{subequations}
\label{eq:LTI_representation}
\begin{align}
    &\dot{\eta}_{1} =\eta_{2},\ \dot{\eta}_{2} =\eta_{3},\ \dot{\eta}_{3} =v^{\parallel}\label{eq:LTI_representationeta}\\
    &\dot{\xi}_{1} =\xi_{2},\  \dot{\xi}_{2} =\xi_{3}, \ \dot{\xi}_{3} =  v^{\pitchfork}\label{eq:LTI_representationxi}.
\end{align}
\end{subequations}
{
\begin{remark}
    It should be noted that the coordinate transformation ${\mathscr{T}}$ given in~\eqref{eq:diffeo} and the feedback law given in~\eqref{eq:regular_feedback_transformation} \nw{are} valid in the neighborhood of the path $\mc{N}_{\gamma}^{\by}$ if $\vrm + x_5 \mynnd{\neq} 0$. \akh{Moreover, $\eta_1$ is the path parameter, and on the path $\eta_2$ is the velocity of the robot.}
\end{remark}

% the condition $\vrm + x_5 \neq 0$ never occurs for the resulting closed-loop system.
}
\subsection{{Local Path-Invariant Controller with \nw{CBF-based} Singularity Filter}}
Next, we design a local path-invariant controller with a singularity filter that prevents $\vrm + x_5 = 0$, if { the system is initialized such that $\vrm + x_5 \geq \delta$ for some positive $\delta$.} We highlight that the linear system~\eqref{eq:LTI_representation} is geometrically equivalent to~\eqref{eq:dynamic_car_robot} at each point on the set $\mc{N}^\star$. Moreover, it consists of two chains of decoupled integrators and is valid everywhere in $\mc{N}^\star$. Therefore, we design a controller for~\eqref{eq:LTI_representation}.
% as it is linear
{We stabilize the origin of the $\xi$-subsystem by designing the \nw{local} controller $\kappa_\xi: \mc{N}_\Gamma^{\by} \to \Real$, which is given as
\begin{equation}
\label{eq:v_trans}
   v^\pitchfork = \kappa_{\xi}(\xi) =  -\sum_{i=1}^{3}k_{i} \sign(\xi_{i})|\xi_i|^{\beta_i},
\end{equation}
where for $i \in \{1, 2, 3\}$, $\beta_i>0$ is given as $\beta_1 = \frac{\beta}{2-\beta}$, $\beta_2 = \beta$ and $\beta_3 = 1$, for $\beta\in (1-\varepsilon, 1)$ where $0<\varepsilon<1$, and $k_i>0$ are such that the polynomial \myifconf{$
    \tilde s^3 + k_3\tilde s^2 + k_2\tilde s + k_1
$}{\begin{equation}\label{eq:hurwitz}
    \tilde s^3 + k_3\tilde s^2 + k_2\tilde s + k_1
\end{equation}} is Hurwitz (see \cite[Proposition 8.1]{bhat2005geometric}). 
% Before designing the controller for the $\eta$-subsystem, 
The transformed state $\eta_2$ is related to the speed of the robot, which is established in the following result.
\begin{proposition}
\label{prop:nonzero-speed}
{\akh For each $\bar{x}\in\mc{N}_\Gamma^{\by}$ satisfying $\inner{(\frac{\partial }{\partial x_1}\pi(\bar{x}),\frac{\partial }{\partial x_2}\pi(\bar{x}))}{\left(\cos \bar{x}^{0}_3,\sin \bar{x}^{0}_3\right)} >0$, where $\bar{x}^{0}_{i}$ denotes the $i$-th component of $\bar{x}^{0}$ for $i\in\{1, \ldots,6\}$, then $\eta_2 = 0$ if and only if $\mynne{\vrm + x_5} = 0$. Moreover, $\eta_2$ and $\mynne{\vrm + x_5}$ \mynne{have} the same sign.}
  % $\eta_2$ implies a nonzero robot's velocity, i.e., $\eta_2(t) \neq 0 \implies v + x_5 \neq 0$.} 
% For each trajectory solution $t\mapsto \agx := (x_{1}, x_{2}, x_{3}, x_{4}, x_{5}, x_{6})$ to (\ref{eq:dynamic_car_robot}) and its pointwise transformation $t\mapsto (\eta, \xi) := (\eta_{1}, \eta_{2}, \eta_{3}, \xi_{1}, \xi_{2}, \xi_{3})$ by $\mathscr{T}$ \mynnd{in (\ref{eq:diffeo})}, for each $t\in \dom(\eta, \xi) = \dom \agx$ such that  $\eta_2(t) = 0$, if and only if $v + x_5(t) \neq 0$.
  % $\eta_2$ implies a nonzero robot's velocity, i.e., $\eta_2(t) \neq 0 \implies v + x_5 \neq 0$.   
\end{proposition}
\myifconf{\begin{proof}
    See \cite{wang2025hybrid}.
\end{proof}}{\begin{proof}
    First, we argue that $v+x_5$ represents the magnitude of the velocity {\akh $(\dot x_1,\dot x_2)$} of the robot. From~\eqref{eq:dynamic_car_robot}, the $x_1$ and $x_2$ component of the velocity is given by $\dot x_1 = (v+x_5)\cos x_3$ and $\dot x_2 = (v+x_5)\sin x_3$. {\akh The} magnitude of the velocity is by $\sqrt{\dot x_1^2 + \dot x_2^2} = \sqrt{ (v+x_5)^2(\cos^2x_3 + \sin^2x_3) } = {\akh\norm{v+ x_5}}$. 
    
    {\akh Recall from~\eqref{eq:diffeo} that $\eta_1 = \pi(x)$. Moreover, it follows from~\eqref{eq:proj} and~\eqref{eq:virtual} that $\pi$ is only a function of $x_1$ and $x_2$. The derivative of $\eta_1$ is given by 
    \begin{align}
        \dot \eta_1 &= \eta_2 = \dot \pi(x)\\
             \eta_2 &= \partial_{x_1}\pi(x) \dot x_1 + \partial_{x_2}\pi(x) \dot x_2\\
                    & = \inner{(\partial_{x_1}\pi(x),\partial_{x_2}\pi(x))}{(\dot x_1,\dot x_2)}\\
                     & = (\vrm + x_5)\inner{(\partial_{x_1}\pi(x),\partial_{x_2}\pi(x))}{\left(\cos x_3,\sin x_3\right)}.
    \end{align}
    It follows from~\eqref{eq:proj} and~\eqref{eq:virtual} that $(\partial_{x_1}\pi(x),\partial_{x_2}\pi(x))$ is the velocity of the curve $\sigma$. Since $\sigma$ is regular, this implies that  $(\partial_{x_1}\pi(x),\partial_{x_2}\pi(x))$ is a non-zero vector. For any value of $x_3$, the vector $(\cos x_3,\sin x_3)$ is non-zero. Finally, by assumption $\inner{(\partial_{x_1}\pi(x),\partial_{x_2}\pi(x))}{\left(\cos x_3,\sin x_3\right)} >0$, which implies that $\eta_2 = 0$ if and only if $(\vrm + x_5) = 0$. Moreover, $\eta_2$ and $\mynne{\vrm + x_5}$ have the same sign. 
    }
    

    % Next, from~\eqref{eq:proj} it should be noted that the path parameter is $\lambda$, and by~\eqref{eq:virtual} and~\eqref{eq:diffeo}, the path parameter is equal to $\eta_1$. Moreover, $\eta_1$ represents the position of the robot on the path. Suppose the robot is moving along the path with some nonzero speed, i.e., $\dot\eta_1\neq 0$. By definition, $\eta_2 = \dot\eta_1$. The projection of $\eta_2$ on to the $x_1$ and $x_2$ axis is given by $\pr_{x_1}\eta_2$ and $\pr_{x_1}\eta_2$. Since, $\eta_2$ is nonzero, $\sqrt{ (\pr_{x_1}\eta_2)^2 + (\pr_{x_2}\eta_2)^2 } \neq 0$. Moreover, by the definition of $\eta_2$, it follows that $\pr_{x_1}\eta_2 = \dot x_1$ and $\pr_{x_2}\eta_2 = \dot x_2$. Hence, $\eta_2\neq 0 \implies \sqrt{ (\pr_{x_1}\eta_2)^2 + (\pr_{x_2}\eta_2)^2 } \neq 0 \implies v + x_5 \neq 0$.  
\end{proof}}
\begin{remark}
The condition on $\mc{N}_\Gamma^{\by}$, namely,
$\inner{(\frac{\partial }{\partial x_1}\pi(\bar{x}),\frac{\partial }{\partial x_2}\pi(\bar{x}))}{\left(\cos \bar{x}^{0}_3,\sin \bar{x}^{0}_3\right)} >0$, implies that if the robot is sufficiently close to the desired curve, then angle between the tangent vector of the curve representing the direction of curve and the velocity of the robot is less than $90^\circ$.
    % The transformation $\mathscr{T}$ in (\ref{eq:diffeo}) is not dependent on the input, as it only requires the state and the Lie derivative of the functions $\pi$ and $\alpha$ in (\ref{eq:virtual}) along the vector field $f$ in (\ref{eq:dynamic_car_robot}). Therefore, Proposition \ref{prop:nonzero-speed} holds regardless of input values.
\end{remark}
% \nw{Note that} the transformed state $\eta_1$ represents the position of the robot along the path, and $\eta_2$ represents the speed of the robot along the path and is equal to $\vrm + x_5$.

One can express the $\eta$ subsystem in the control affine form as $\dot \eta = \tilde{f}(\eta) + \tilde{g}(\eta)v^\parallel$, where $\tilde{f}\nw{(\eta)} = (\eta_2,\eta_3,0)$ and $\tilde{g}\nw{(\eta)} = (0,0,1)$. \nw{To} follow the path with a given reference velocity ${\eta}^{\mathrm{ref}}_{2}\in\reals$ and its derivative $\eta^{\mathrm{ref}}_{3}\in\reals$, we construct the Lyapunov function $V: \reals\times\reals\times \reals\times\reals\to \reals_{\geq 0}$ as follows:
$
V(\eta_{2}, \eta_{3}, {\eta}^{\mathrm{ref}}_{2}, \eta^{\mathrm{ref}}_{3}) = 1/2(\eta_2 - {\eta}^{\mathrm{ref}}_{2})^2 + 1/2(\eta_3 - { \eta}^{\mathrm{ref}}_{3})^2
$ and apply any control input from the set-valued map \nw{$K_{\mathrm{clf}}: \reals\times \reals\times\reals\times\reals\rightrightarrows \reals$} at the current \mynnd{$\eta_{2}, \eta_{3}$} and reference ${\eta}^{\mathrm{ref}}_{2}$ and $\eta^{\mathrm{ref}}_{3}$:
\myifconf{$
    K_{\mathrm{clf}}(\mynnd{\eta_{2}, \eta_{3}}, {\eta}^{\mathrm{ref}}_{2}, \eta^{\mathrm{ref}}_{3}) \eqdef\{v^{\parallel}\in\Real: L_{\tilde f}V(\mynnd{\eta_{2}, \eta_{3}}, {\eta}^{\mathrm{ref}}_{2}, \eta^{\mathrm{ref}}_{3})
    + L_{\tilde g}V(\mynnd{\eta_{2}, \eta_{3}}, {\eta}^{\mathrm{ref}}_{2}, \eta^{\mathrm{ref}}_{3})v^{\parallel}
    \leq {\akh -}\beta(V(\mynnd{\eta_{2}, \eta_{3}}, {\eta}^{\mathrm{ref}}_{2}, \eta^{\mathrm{ref}}_{3})) \}
$}{\begin{equation}
\begin{aligned}
    \label{eq:CLF}
    &K_{\mathrm{clf}}(\mynnd{\eta_{2}, \eta_{3}}, {\eta}^{\mathrm{ref}}_{2}, \eta^{\mathrm{ref}}_{3}) \eqdef\{v^{\parallel}\in\Real: L_{\tilde f}V(\mynnd{\eta_{2}, \eta_{3}}, {\eta}^{\mathrm{ref}}_{2}, \eta^{\mathrm{ref}}_{3})\\
    &+ L_{\tilde g}V(\mynnd{\eta_{2}, \eta_{3}}, {\eta}^{\mathrm{ref}}_{2}, \eta^{\mathrm{ref}}_{3})v^{\parallel}
    \leq {\akh -}\beta(V(\mynnd{\eta_{2}, \eta_{3}}, {\eta}^{\mathrm{ref}}_{2}, \eta^{\mathrm{ref}}_{3})) \}
\end{aligned}
\end{equation}}
to track the desired speed profile ${\eta}^{\mathrm{ref}}_{2}$ and $\eta^{\mathrm{ref}}_{3}$, where $\beta$ is a class $\mc{K}$ function. Next, we \nw{construct} a barrier function 
$
b: \Real^3 \to \Real
$ as 
$
b(\eta) \eqdef \delta - \eta_2
$ to guarantee that $\vrm + x_5 > \delta$, for some positive $\delta$.
% In other words, on the set $\mc{N}_\Gamma$, $\eta_2 > \delta$. 
\nw{The set \begin{equation}
\label{eq:set-S1}
 S_1 \eqdef \set{ \eta \in \Real^3 : b(\eta) \leq 0}   
\end{equation} cannot be forward invariant by selecting proper input $v^\parallel$ because the system has a relative degree two for the barrier function $b$ and the rate of change of $b$ along the vector fields $\tilde g$ is zero. Similar to~\cite{XiaBelCas2022}, let $\psi_0(\eta) \eqdef b(\eta)$ and $\psi_1(\eta) \eqdef \dot\psi_0{\eta} + \beta(\psi_0(\eta))$, where $\beta$ is a class $\mc{K}$ function. Next, using $\psi_1(\eta)$
we construct a second set 
\begin{equation}
\label{eq:set-S2}
S_2 \eqdef \set{ \eta \in \Real^3 : \psi_1(\eta) \leq 0}.   
% S_2 \eqdef \set{\eta \in \Real^3 : b(\eta) + L_{\tilde f}b(\eta) \leq 0 }.
\end{equation}
It should be noted that by Corollary~\ref{cor:diffeo}, the transformation $\mathscr{T}$ is a local diffeomorphism. Therefore the sets $S_1$ and $S_2$ can be expressed in $x$-coordinates by applying the inverse transformation, i.e., $\mathscr{T}^{-1}$.
%
By selecting a control input from the set-valued map $K_{\mathrm{cbf}}: \reals^{3}\to \reals$  at the current $\eta$ as follows:
\myifconf{$
        K_{\mathrm{cbf}}(\eta) \eqdef \{ v^{\parallel}\in\Real : L_{\tilde f}^2b(\eta) + L_{\tilde f}b(\eta) +L_{\tilde g}L_{\tilde f}b(\eta)v^{\parallel}
    \leq -\beta_k( \psi_1(\eta) )  \},
$}{\begin{equation}\label{eq:CBF}
\begin{aligned}
        K_{\mathrm{cbf}}(\eta) &\eqdef \{ v^{\parallel}\in\Real : L_{\tilde f}^2b(\eta) + L_{\tilde f}b(\eta) +L_{\tilde g}L_{\tilde f}b(\eta)v^{\parallel}  \\ 
    &\leq -\beta_k( \psi_1(\eta) )  \},
\end{aligned}
\end{equation}}
it will be shown in the following result that the set $S_1 \cap S_2$ is forward invariant.} Finally, to track the desired speed profile $({\eta}^{\mathrm{ref}}_{2}, \eta^{\mathrm{ref}}_{3})$ and guarantee that the system trajectories never reach the singularity point, i.e., $\vrm + x_5  = 0$, the control input is selected 
% from the set $K_{\mathrm{clf}}  \cap K_{\mathrm{cbf}}$. In other words, the controller $\kappa_\eta: \mc{N}_{\Gamma} \subset \Real^3 \to \Real$ such that 
\nw{as $\kappa_\eta(\eta, {\eta}^{\mathrm{ref}}_{2}, \eta^{\mathrm{ref}}_{3}) \in K_{\mathrm{clf}}(\eta, {\eta}^{\mathrm{ref}}_{2}, \eta^{\mathrm{ref}}_{3})\cap K_{\mathrm{cbf}}\nw{(\eta)}$}. In summary, for a given reference velocity profile ${\eta}^{\mathrm{ref}}_{2}$ and $ \eta^{\mathrm{ref}}_{3}$ that the robot is required to follow, we design the following controller:
$
    \kappa_0 : \mc{N}^{\by}_\Gamma\times\reals\times\reals \to \Real^{2}
$
such that, for {\akh each} $(\xi,\eta)\in \mc{N}_\Gamma^{\by}$,
\begin{equation}
\label{eq:kappa_0}
\begin{aligned}
     (\xi,\eta, {\eta}^{\mathrm{ref}}_{2}, \eta^{\mathrm{ref}}_{3}) &\mapsto \kappa_0(\xi,\eta, {\eta}^{\mathrm{ref}}_{2}, \eta^{\mathrm{ref}}_{3})\\
     &\eqdef  (\kappa_\xi(\xi), \kappa_\eta(\eta, {\eta}^{\mathrm{ref}}_{2}, \eta^{\mathrm{ref}}_{3})).
     % \left[ \begin{array}{c}
         % v^\pitchfork \\
         % v^\parallel 
    % \end{array}\right] \\
    % &= \left[ \begin{array}{c}
          % -\sum_{i=1}^{3}k_{i} \sign(\xi_{i})|\xi_i|^{\beta_i} \\
          % k_{4}(\eta_{2}-{\eta}^{\mathrm{ref}})+
      % k_{5}(\eta_{3}- \dot{\eta}^{\mathrm{ref}}) 
    % \end{array}\right],    
\end{aligned}
\end{equation}
% where for $i \in \{1, 2, 3\}$, $\beta_i>0$ is given as $\beta_1 = \frac{\beta}{2-\beta}$, $\beta_2 = \beta$ and $\beta_3 = 1$, for $\beta\in (1-\varepsilon, 1)$ where $0<\varepsilon<1$, and $k_i>0$ are such that $\tilde s^3 + k_3\tilde s^2 + k_2\tilde s + k_1$ is Hurwitz (see \cite[Proposition 8.1]{bhat2005geometric}). Moreover, we select $k_{4} > 0$ and $k_5 >0$ using pole placement, and the {\myblue parameter ${\dot\eta}^{\mathrm{ref}}$ is the derivative of the reference velocity along the curve}. 
% {\myblue
% \begin{definition}[Feasibility Condition]
%    A control policy $\kappa_\eta : \mc{N}_\Gamma^{\by}\times\reals\times\reals \to \Real$ for the closed-loop system $\dot \eta = \tilde{f}(\eta) + \tilde{g}(\eta)\kappa_\eta(\eta, {\eta}^{\mathrm{ref}}_{2}, \eta^{\mathrm{ref}}_{3})$ is feasible if it satisfies both the higher-order control barrier function constraint~\eqref{eq:CBF} and the control Lyapunov function constraint~\eqref{eq:CLF} \nw{for each point in $\mc{N}_\Gamma^{\by}\times\reals\times\reals$}. In other words, \nw{for each $\eta\in\mc{N}_\Gamma^{\by}$, ${\eta}^{\mathrm{ref}}_{2}\in\reals$ and  ${\eta}^{\mathrm{ref}}_{3}\in\reals$,} 
%  $K_{\mathrm{clf}}\nw{(\eta, {\eta}^{\mathrm{ref}}_{2}, {\eta}^{\mathrm{ref}}_{3})} \cap K_{\mathrm{cbf}}\nw{(\eta)} \neq \emptyset$.
% \end{definition}
% }


Then, we are ready to present the local path invariance result.
% We want to highlight that $\eta_1$ and $\eta_2$ are the position and velocity of the robot along the given path, respectively, as defined in~\eqref{eq:diffeo}. In this setting, the velocity of the robot is the control variable.
% 
% \begin{assumption}
% \label{ass:feasible_control}
%     \nw{For any $(\xi,\eta)\in \mc{N}_\Gamma$}, ${\eta}^{\mathrm{ref}}_{2}\in\reals$ and $\eta^{\mathrm{ref}}_{3}\in\reals$, the set of feasible control inputs in nonempty, i.e., $K_{\mathrm{clf}}\nw{(\eta, {\eta}^{\mathrm{ref}}_{2}, {\eta}^{\mathrm{ref}}_{3})} \cap K_{\mathrm{cbf}}\nw{(\eta)} \neq \emptyset$.
% \end{assumption}
  \begin{lemma}
  \label{lemma:invariance}
  \mynne{For each initial state $\bar{x}^{0} \in \mc{N}_\Gamma^{\by}$ satisfying i) the heading condition: $\inner{(\frac{\partial }{\partial x_1}\pi(\bar{x}),\frac{\partial }{\partial x_2}\pi(\bar{x}))|_{\bar{x}=\bar{x}^{0}}}{\left(\cos \bar{x}^{0}_3,\sin \bar{x}^{0}_3\right)} >0$; ii) and the velocity condition: $\vrm + \bar{x}^{0}_5 > \delta$ for some arbitrarily small positive $\delta\in\reals_{>0}$, where $\bar{x}^{0}_{i}$ denotes the $i$-th component of $\bar{x}^{0}$ for $i\in\{1, \ldots,6\}$,}
  {\akh then
  % for each initial conditions $\agx^{0}\in \mc{N}_\Gamma^{\by}$$
  % \setminus \{\agx\in\Real^6 : \vrm + x_5 > \delta\}
  the closed-loop system obtained by applying the controllers $\kappa_0$ given in~\eqref{eq:kappa_0}, to~\eqref{eq:LTI_representation}, renders the set
  $
  \Gamma^\star \eqdef \set{\agx\in\Real^6 : \alpha(\agx) = \dot\alpha (\agx)= \ddot\alpha(\agx) = 0}
  $
  finite-time stable with basin of attraction $\mc{N}^{\star}$  in~\eqref{eq:nbh_lift_set} and forward invariant. Moreover, the trajectories of the closed-loop system remain safe in the sense that the set $S_1 \cap S_2$ defined by~\eqref{eq:set-S1} and~\eqref{eq:set-S2} is forward invariant. Furthermore, \nw{for each $\eta\in\mc{N}_\Gamma^{\by}$, ${\eta}^{\mathrm{ref}}_{2}\in\reals$ and  ${\eta}^{\mathrm{ref}}_{3}\in\reals$,} 
 $K_{\mathrm{clf}}\nw{(\eta, {\eta}^{\mathrm{ref}}_{2}, {\eta}^{\mathrm{ref}}_{3})} \cap K_{\mathrm{cbf}}\nw{(\eta)} \neq \emptyset$.} 
  \end{lemma}
\myifconf{\begin{proof}
    See \cite{wang2025hybrid}.
\end{proof}}{\begin{proof}
  We first utilize Proposition 8.1 from \cite{bhat2005geometric} (refer to Proposition \ref{prop:linearfinitetime} in the Appendix) to demonstrate that zero is a globally finite-time-stable equilibrium for the subsystem (\ref{eq:LTI_representationxi}) under the feedback control in (\ref{eq:v_trans}), namely, the feedback controller in~\eqref{eq:kappa_0} steers all the $\xi$-states to converge to zero in finite time. Note that  for $i \in \{1, 2, 3\}$, $k_{i}$ is designed such that (\ref{eq:hurwitz}) satisfies the Hurwitz condition, and the feedback control in (\ref{eq:v_trans}) conforms to the structure of (\ref{eq:xicontrol}) with $n = 3$. Therefore, all the conditions in \cite[Proposition 8.1]{bhat2005geometric} are met, thereby establishing the finite-time stability of the origin for (\ref{eq:LTI_representationxi}).

  
  % Since~\eqref{eq:LTI_representation} is a linear system, by~\cite[Proposition 8.1]{bhat2005geometric} the controller~\eqref{eq:kappa_0} forces all the $\xi$-states to converge to zero in finite time. 
  From the definition of the coordinate transformation $\mathscr{T}$ given in~\eqref{eq:diffeo} $\xi_1$ is the zero-th Lie derivative of $\alpha(\agx)$, i.e., $\xi_1 = L_f^0\alpha(\agx) = \alpha(\agx)$. The first and the second Lie derivatives of $\alpha(\agx)$ defines $\xi_2$ and $\xi_3$ states, respectively, i.e., $\xi_2 = L_f\alpha(\agx) = \dot\alpha(\agx)$ and $\xi_3 = L_f^2\alpha(\agx) = \ddot\alpha(\agx)$.
  %
  % Moreover, as defined in~\eqref{eq:diffeo}, $\xi_1 = \alpha(\agx)$, $\xi_2 = \dot \alpha(\agx)$, and $\xi_3 = \ddot\alpha(\agx)$.
  Therefore by~\cite[Proposition 8.1]{bhat2005geometric}, the set $\Gamma^\star = \set{(\xi,\eta)\in\Real^6: \alpha(\agx) = \dot\alpha(\agx) = \ddot\alpha(\agx)}$ is finite-time stable and invariant for the closed-loop system~\eqref{eq:LTI_representation}. It follows from Corollary~\ref{cor:diffeo} that the basin of attraction of $\kappa_0$ is $\mc{N}_\Gamma^{\by}$. 
  To prove that the trajectories of the closed-loop system never get arbitrarily close to the point $\vrm + x_5 = 0$ if the initial condition belongs to the set $\mc{N}_\Gamma^{\by} \setminus \{\agx\in\Real^6 : \vrm + x_5 > \delta\}$. We note that by Proposition~\ref{prop:nonzero-speed}, it is sufficient to prove that $\eta_2 > 0$. One can readily verify that by~\cite[Definition 4]{XiaBelCas2022}, the candidate barrier function $b = \delta - \eta_2$ is a high-order control barrier function. Moreover, by~\cite[Theorem 2]{XiaBelCas2022} there exists a feasible control input in the set $K_{\mathrm{clf}}\nw{(\eta, {\eta}^{\mathrm{ref}}_{2}, {\eta}^{\mathrm{ref}}_{3})} \cap K_{\mathrm{cbf}}\nw{(\eta)} $ such that the set $S_1 \cap S_2$ defined by~\eqref{eq:set-S1} and~\eqref{eq:set-S2} is forward invariant.
  \end{proof}}
  \begin{remark}
     The invariance guarantee provided by Lemma~\ref{lemma:invariance} is twofold. First, it guarantees that the given path is locally attractive and forward-invariant, which means that the system converges to the path and then never leaves the path. Since by assumption, the path does not have obstacles, and by selecting a ``tight" obstacle-free neighborhood, one can guarantee safety. Second, it certifies in the neighborhood of the path the singularity condition ($\vrm + x_5 = 0$) will never occur by establishing the forward invariance of the set $S_1 \cap S_2$. 
  \end{remark}

}