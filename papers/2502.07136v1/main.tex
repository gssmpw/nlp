% \documentclass[preprint]{elsarticle}
% \documentclass[letterpaper, 10pt,conference]{IEEEtran}
\documentclass[letterpaper, 10 pt, conference]{ieeeconf}

\IEEEoverridecommandlockouts                              % 
\overrideIEEEmargins                                      % Needed to meet printer requirements.
% \usepackage{lineno,hyperref}
% \modulolinenumbers[5]

% \usepackage{mathrsfs}
\usepackage{graphics} % for pdf, bitmapped graphics files
\usepackage{epsfig} % for postscript graphics files
\usepackage{amsmath} % assumes amsmath package installed
\usepackage{amssymb}  % assumes amsmath package installed
\usepackage{mathrsfs}
% \usepackage{amsthm}
\usepackage{cite}
\usepackage{bm}
\usepackage{subcaption}
\usepackage{acronym}
\usepackage{paralist}
\usepackage{float}
\usepackage[dvipsnames]{xcolor}
\usepackage{epstopdf}
\usepackage{multicol}
\usepackage{tikz}
\usepackage{hyperref}
% \usepackage{subfigure}
% \usepackage{subcaption}
\usepackage{graphicx}
\usepackage{algorithm}
\usepackage{algpseudocode}
\usepackage{xcolor}
\usepackage{etoolbox} 

% \usepackage{caption}
% \usepackage{soul}
\usepackage{macros} % Adeel's custom defined macros

\usepackage{import}
\usepackage{xifthen}
\usepackage{pdfpages}
\usepackage{transparent}

\usepackage{./macros/rgsEnvironments}
% \usepackage{./macros/rgsMacros}  


% The following packages can be found on http:\\www.ctan.org
\usepackage{graphics} % for pdf, bitmapped graphics files
\usepackage{epsfig} % for postscript graphics files
\usepackage{mathptmx} % assumes new font selection scheme installed
\usepackage{times} % assumes new font selection scheme installed
\usepackage{amsmath} % assumes amsmath package installed
\usepackage{amssymb}  % assumes amsmath package installed
\usepackage{bm,soul,xcolor}
%\usepackage{natbib}
\usepackage{cite}
\usepackage{tikz}
\usetikzlibrary{arrows.meta, automata, positioning, quotes}
\usepackage{comment}
\usepackage[linesnumbered,ruled,vlined]{algorithm2e}
\usepackage{graphicx}
\usepackage{subcaption}
\usepackage{caption}
\usepackage{mathtools}
\usepackage{gensymb}
\usepackage[hidelinks]{hyperref}
%\usepackage{multicol}
\usepackage{stfloats}
%\usepackage{float}
\graphicspath{{Figures/}}
\newcommand{\Fig}{Fig. }
\newcommand{\bigso}{\mathfrak{SO}(3)}
\newcommand{\smallso}{\mathfrak{so}(3)}
\newcommand{\smallse}{\mathfrak{se}(3)}
\newcommand{\bigse}{{SE}(3)}
\newcommand{\twist}{\mathcal{V}}
\newcommand{\sphi}[1]{s_{\phi_{#1}}}
\newcommand{\cphi}[1]{c_{\phi_{#1}}}
\newcommand{\stheta}[1]{s_{\theta_{#1}}}
\newcommand{\ctheta}[1]{c_{\theta_{#1}}}
% \newcommand{\Fr}[#1]{F_{r #1}}
\usepackage{amssymb}
\renewcommand{\Re}{\mathbb{R}}
%\usepackage[showframe]{geometry}

\usepackage[export]{adjustbox} %Nan's custom defined here.
\newboolean{conf}
\setboolean{conf}{true}
\newcommand{\myifconf}[2]{\ifbool{conf}{#1}{#2}}
% %%% COLOR FOR EDITING. Need this package \usepackage[dvipsnames]{xcolor} %%%%
 \newcommand{\myred}{\color{RubineRed}}
 \newcommand{\myblue}{\color{RoyalBlue}}
 % \newcommand{\akh}{\color{SeaGreen}}
 \newcommand{\akh}{}
 \newcommand{\myplum}{\color{Plum}}

\makeatletter
\hypersetup{colorlinks=true}
\AtBeginDocument{\@ifpackageloaded{hyperref}
  {\def\@linkcolor{blue}
  \def\@anchorcolor{red}
  \def\@citecolor{red}
  \def\@filecolor{red}
  \def\@urlcolor{red}
  \def\@menucolor{red}
  \def\@pagecolor{red}
\begingroup
  \@makeother\`%
  \@makeother\=%
  \edef\x{%
    \edef\noexpand\x{%
      \endgroup
      \noexpand\toks@{%
        \catcode 96=\noexpand\the\catcode`\noexpand\`\relax
        \catcode 61=\noexpand\the\catcode`\noexpand\=\relax
      }%
    }%
    \noexpand\x
  }%
\x
\@makeother\`
\@makeother\=
}{}}
\makeatother


% % *** PDF, URL AND HYPERLINK PACKAGES ***
% %
\makeatletter
\let\NAT@parse\undefined
\makeatother
\usepackage{url}
% \usepackage[backref=page]{hyperref} %% 
\usepackage[]{hyperref} %%
\urldef{\CombinedEmail}\url{{nanwang; ricardo}@ucsc.edu}
\urldef{\AdeelEmail}\url{adeel.akhtar@njit.edu}



\IEEEoverridecommandlockouts


\def\BibTeX{{\rm B\kern-.05em{\sc i\kern-.025em b}\kern-.08em
    T\kern-.1667em\lower.7ex\hbox{E}\kern-.125emX}}

\begin{document}

\title{\bf A Safe Hybrid Control Framework for Car-like Robot with Guaranteed Global Path-Invariance using a Control Barrier Function$^*$}

\author{Authors}

\author{Nan Wang$^{1}$ \qquad Adeel Akhtar$^{2}$ \qquad Ricardo G. Sanfelice$^{1}$
% <-this % stops a space
  \thanks{*Research by N. Wang and R. G. Sanfelice is partially supported by NSF Grants no. CNS-2039054 and CNS-2111688, by AFOSR Grants nos. FA9550-19-1-0169, FA9550-20-1-0238, FA9550-23-1-0145, and FA9550-23-1-0313, by AFRL Grant nos. FA8651-22-1-0017 and FA8651-23-1-0004, by ARO Grant no. W911NF-20-1-0253, and by DoD Grant no. W911NF-23-1-0158.}% <-this % stops a space
  % \thanks{Nan Wang, A. Akhtar, Eric Partika, and R. G. Sanfelice are with the Department of Electrical and Computer Engineering at the University of California at Santa Cruz, California, USA. \CombinedEmail}%%
  \thanks{$^{1}$N. Wang and R. G. Sanfelice are with the Department of Electrical and Computer Engineering at the University of California at Santa Cruz, California, USA. \CombinedEmail}
  \thanks{$^{2}$A. Akhtar is with the Department of Mechanical and Industrial Engineering at New Jersey Institute of Technology (NJIT), New Jersey, USA. \AdeelEmail}%%
}

\maketitle

%%%%%%%%%%%%%%%% Abstract %%%%%%%%%%%%%%%%%%%%%%%%%%%%%%%%%%
\begin{abstract}
% In this work, we design a hybrid control scheme to navigate an unmanned ground vehicle to the given path starting from everywhere in the output space while ensuring path invariance. In the context of this paper, the path-invariance property guarantees that once the vehicle reaches the path, it remains on that path precisely and indefinitely. To achieve this, we develop a locally path-invariant controller that guarantees path invariance when initiated within the neighborhood of the path. To complement this, a pure pursuit controller is employed for navigation from arbitrary starting points within the output space to the aforementioned neighborhood. 
% These two controllers are integrated within a unified hybrid framework, facilitating convergence from any location in the output space to the target path, with guaranteed path invariance and robustness to sensor noise. The hybrid controller's effectiveness is rigorously demonstrated through practical experiments on the OSOYOO Robot Car.
%
% In this work, we introduce a hybrid control scheme for an unmanned ground vehicle, ensuring global path invariance. In the context of this paper, global path invariance property guarantees that, from any starting point in the output space, once the vehicle reaches the path, it precisely and indefinitely stays on that path. We develop a locally path-invariant controller for maintaining path following within the path's neighborhood, supplemented by a pure pursuit controller for navigation from arbitrary starting points to the neighborhood. These controllers are seamlessly integrated into a unified hybrid framework, ensuring convergence from any output space location to the desired path, while guaranteeing path invariance and robustness to sensor noise. Experimental validation is conducted on the OSOYOO Robot Car, affirming the hybrid controller's effectiveness.

% This work proposes a hybrid framework for car-like robots with obstacle avoidance, global convergence, and safety, where safety is defined as path invariance—once the robot reaches the path, it never leaves. Given an a \textit{priori} obstacle-free feasible path with surrounding obstacles, the goal is to avoid obstacles, reach the path, and stay on it. The problem is addressed in two stages. Firstly, a ``tight'' obstacle-free neighborhood is defined along the path, with a local controller ensuring convergence and path invariance. Control barrier functions are used to steer the system away from singularities where the local controller is undefined. Secondly, a hybrid control framework integrates this local controller with any global tracking controller from the literature (without path invariance guarantees) to ensure global convergence. This framework guarantees path invariance and robustness to sensor noise. Detailed simulations \myifconf{affirm}{and experimental validation on the OSOYOO Robot Car affirm} the scheme’s effectiveness.
This work proposes a hybrid framework for car-like robots with obstacle avoidance, global convergence, and safety, where safety is interpreted as path invariance, namely, once the robot converges to the path, it never leaves the path. Given \textit{a priori}  obstacle-free feasible path where obstacles can be around the path, the task is to avoid obstacles while reaching the path and then staying on the path without leaving it. The problem is solved in two stages. Firstly, we define a ``tight'' obstacle-free neighborhood along the path and design a local controller to ensure convergence to the path and path invariance. The control barrier function technology is involved in the control design to steer the system away from its singularity points, where the local path invariant controller is not defined. Secondly, we design a \mynne{hybrid control} framework that integrates this local path-invariant controller with any global tracking controller from the existing literature without path invariance guarantee, ensuring convergence from any position to the desired path, namely, global convergence. This framework guarantees path invariance and robustness to sensor noise. Detailed simulation results\myifconf{ affirm}{ and experimental validation on the OSOYOO Robot Car affirm} the effectiveness of the proposed scheme.


\end{abstract}
%%%%%%%%%%%%%%%%%%%%%%%%%%%%%%%%%%%%%%%%%%%%%%%%%%%%%%%%%%%%

%%%%%%%%%%%%%%%% Latex files for each funtions %%%%%%%%%%%%%
% \myac{This is Adeel's macro.}
% \mynn{This is Nan's macro.}

\section{Introduction}


\begin{figure}[t]
\centering
\includegraphics[width=0.6\columnwidth]{figures/evaluation_desiderata_V5.pdf}
\vspace{-0.5cm}
\caption{\systemName is a platform for conducting realistic evaluations of code LLMs, collecting human preferences of coding models with real users, real tasks, and in realistic environments, aimed at addressing the limitations of existing evaluations.
}
\label{fig:motivation}
\end{figure}

\begin{figure*}[t]
\centering
\includegraphics[width=\textwidth]{figures/system_design_v2.png}
\caption{We introduce \systemName, a VSCode extension to collect human preferences of code directly in a developer's IDE. \systemName enables developers to use code completions from various models. The system comprises a) the interface in the user's IDE which presents paired completions to users (left), b) a sampling strategy that picks model pairs to reduce latency (right, top), and c) a prompting scheme that allows diverse LLMs to perform code completions with high fidelity.
Users can select between the top completion (green box) using \texttt{tab} or the bottom completion (blue box) using \texttt{shift+tab}.}
\label{fig:overview}
\end{figure*}

As model capabilities improve, large language models (LLMs) are increasingly integrated into user environments and workflows.
For example, software developers code with AI in integrated developer environments (IDEs)~\citep{peng2023impact}, doctors rely on notes generated through ambient listening~\citep{oberst2024science}, and lawyers consider case evidence identified by electronic discovery systems~\citep{yang2024beyond}.
Increasing deployment of models in productivity tools demands evaluation that more closely reflects real-world circumstances~\citep{hutchinson2022evaluation, saxon2024benchmarks, kapoor2024ai}.
While newer benchmarks and live platforms incorporate human feedback to capture real-world usage, they almost exclusively focus on evaluating LLMs in chat conversations~\citep{zheng2023judging,dubois2023alpacafarm,chiang2024chatbot, kirk2024the}.
Model evaluation must move beyond chat-based interactions and into specialized user environments.



 

In this work, we focus on evaluating LLM-based coding assistants. 
Despite the popularity of these tools---millions of developers use Github Copilot~\citep{Copilot}---existing
evaluations of the coding capabilities of new models exhibit multiple limitations (Figure~\ref{fig:motivation}, bottom).
Traditional ML benchmarks evaluate LLM capabilities by measuring how well a model can complete static, interview-style coding tasks~\citep{chen2021evaluating,austin2021program,jain2024livecodebench, white2024livebench} and lack \emph{real users}. 
User studies recruit real users to evaluate the effectiveness of LLMs as coding assistants, but are often limited to simple programming tasks as opposed to \emph{real tasks}~\citep{vaithilingam2022expectation,ross2023programmer, mozannar2024realhumaneval}.
Recent efforts to collect human feedback such as Chatbot Arena~\citep{chiang2024chatbot} are still removed from a \emph{realistic environment}, resulting in users and data that deviate from typical software development processes.
We introduce \systemName to address these limitations (Figure~\ref{fig:motivation}, top), and we describe our three main contributions below.


\textbf{We deploy \systemName in-the-wild to collect human preferences on code.} 
\systemName is a Visual Studio Code extension, collecting preferences directly in a developer's IDE within their actual workflow (Figure~\ref{fig:overview}).
\systemName provides developers with code completions, akin to the type of support provided by Github Copilot~\citep{Copilot}. 
Over the past 3 months, \systemName has served over~\completions suggestions from 10 state-of-the-art LLMs, 
gathering \sampleCount~votes from \userCount~users.
To collect user preferences,
\systemName presents a novel interface that shows users paired code completions from two different LLMs, which are determined based on a sampling strategy that aims to 
mitigate latency while preserving coverage across model comparisons.
Additionally, we devise a prompting scheme that allows a diverse set of models to perform code completions with high fidelity.
See Section~\ref{sec:system} and Section~\ref{sec:deployment} for details about system design and deployment respectively.



\textbf{We construct a leaderboard of user preferences and find notable differences from existing static benchmarks and human preference leaderboards.}
In general, we observe that smaller models seem to overperform in static benchmarks compared to our leaderboard, while performance among larger models is mixed (Section~\ref{sec:leaderboard_calculation}).
We attribute these differences to the fact that \systemName is exposed to users and tasks that differ drastically from code evaluations in the past. 
Our data spans 103 programming languages and 24 natural languages as well as a variety of real-world applications and code structures, while static benchmarks tend to focus on a specific programming and natural language and task (e.g. coding competition problems).
Additionally, while all of \systemName interactions contain code contexts and the majority involve infilling tasks, a much smaller fraction of Chatbot Arena's coding tasks contain code context, with infilling tasks appearing even more rarely. 
We analyze our data in depth in Section~\ref{subsec:comparison}.



\textbf{We derive new insights into user preferences of code by analyzing \systemName's diverse and distinct data distribution.}
We compare user preferences across different stratifications of input data (e.g., common versus rare languages) and observe which affect observed preferences most (Section~\ref{sec:analysis}).
For example, while user preferences stay relatively consistent across various programming languages, they differ drastically between different task categories (e.g. frontend/backend versus algorithm design).
We also observe variations in user preference due to different features related to code structure 
(e.g., context length and completion patterns).
We open-source \systemName and release a curated subset of code contexts.
Altogether, our results highlight the necessity of model evaluation in realistic and domain-specific settings.





\subsection{Class of Curves}
\label{sec:curves}
Given a smooth curve $\mc{C}$ in $\Real^2$ without self intersections, the curve $\mc{C}$ has a regular parametric representation, namely,
\begin{equation}
\label{eq:general_path}
%\begin{aligned}
\sigma : \dom \sigma\rightarrow\mathbb{R}^{2}, \qquad
\lambda\mapsto{\left(\sigma_{1}(\lambda), \sigma_{2}(\lambda)\right)},
%\end{aligned}
\end{equation}
where $\sigma\subset \Real$ is at least twice continuously differentiable, i.e., $C^2$, and $\mc{C}= \image{(\sigma)}$.
Since $\sigma$ is regular, without losing generality, we assume it is
unit-speed parameterized, i.e., $\norm{\sigma^\prime} \equiv 1$, where $\sigma^\prime$ is the derivative of $\sigma$ with respect to the parameter $\lambda$. Consequently, the curve $\sigma$ is parameterized by its arc length; for details, see~\cite{Pres2010,AkhNieWas2015}.  For a unit-speed curve $\sigma$ with parameter $\lambda$, its curvature $K(\lambda)$ at the point $\sigma(\lambda)$ is defined to be  $\norm{\sigma^{\prime\prime}(\lambda)}$,  where $\sigma^{\prime\prime}$ is the second derivative of $\sigma$ with respect to the parameter $\lambda$.
\begin{assumption}[Implicit representation]
  The curve $\mathcal{C}\subset \Real^2$ has implicit representation $
  \gamma = \set{y \in W : s(y) = 0}$, 
  % {\myblue (AA: We used the small Greek letter $\gamma$ because later we ``lift'' this set from the output space to state space and then call it capital gamma $\Gamma$)} 
  where $s: \dom s \to \Real$ is a smooth function such that the {Jacobian of $s$ evaluated \mynn{at} each point on the path is not zero, i.e., $\D_y s \neq 0$
  for each $y \in \mathcal{C}$} and $\dom s \mynn{\subset}
  \Real^2$ is a set {consisting of an open neighborhood of the curve $\mathcal{C}$}.  
%   Moreover, there
%     exist two class-$\mathcal{K}_\infty$ functions $\alpha, \beta :
%     [0, \infty) \rightarrow [0, \infty)$ such that
% \begin{equation}
%   \left( \forall y \in W \right) \; \alpha{\left(
%       \|y\|_{\mathcal{C}} \right)} \leq \|s(y)\| \leq
%   \beta{\left(\|y\|_{\mathcal{C}} \right)}.
%  \label{eq:classK}
% \end{equation}
\label{ass:implicit}
\end{assumption}

Assumption~\ref{ass:implicit} assumes that the entire path is represented as
the zero-level set of the function $s$, at least locally. A simple example of such a curve is a unit circle with a parametric representation of $\lambda\mapsto(\cos \lambda,\sin\lambda)$ and an implicit representation of $s(y) = y_1^2 + y_2^2 - 1 = 0$,  {and $\dom s = \Real^2$. On the other hand, for $y = (y_1,y_2)\in\Real^2$, an $n$-th order polynomial in variable $y_1$ can be expressed as $y_2 = \sum_{i=0}^{n} a_iy_1^i$, where  scalars $a_i\in\Real$ and $y_1^i$ is the $i$-th power of $y_1$ for $i \in \{0, 1,..., n\}$. Moreover, the polynomial can be expressed implicitly as $s(y) = y_2 - \sum_{i=0}^{n} a_iy_1^i$, and in the form of a parametric curve as $\lambda\mapsto(\lambda,\sum_{i=0}^{n} a_i\lambda^i)$}. 

% {\myred We need to define curvature of the parametric curve $\sigma$.}

\section{Problem Formulation} \label{sec:probdef}

This section formally defines the problem of restoring a given pruned network with only using its original pretrained CNN in a way free of data and fine-tuning.



% Unlike many existing works utilize data for identifying unimportant filters as well as fine-tuning to this end, we cannot evaluate the filter importance by data-dependent values like activation maps (\textit{a.k.a.} channels) as our focus in this paper is not to use any training data. Thus, in our problem setting, we can only exploit the values of filters in the original network, and thereby have to make some changes in the remaining filters of the pruned network so that the network can return the output not too much different from the original one.

% No matter how much we carefully select unimportant filters to be pruned, some kinds of retraining process appears inevitable as done by the most existing works to this end. However, since our focus in this paper is not to use any training data, we cannot evaluate the importance of filters by data-dependent values like activation maps (\textit{a.k.a.} channels). 

% To this end, they not only use a careful criterion (\textit{e.g.}, L1-norm), but also fine-tune the network using the original data.
% Most of filter pruning methods try to select filters to be pruned prudently so that pruned network's output be similar to the original network's. To this end, they prune the unimportant filters and then fine-tune the pruned network with using the train data. 

% How can we restore the the pruned networks without any data? In other words, it implies that we cannot use any data-driven values(i.e., activation maps) and we can only exploit the values of original filters. In that case, the only thing we can do maybe changing the weights of remained filters appropriately not to amplify the difference between pruned and unpruned network's outputs through the information of original filters.

\begin{figure*}[t]
	\centering
    \subfigure[\label{fig:matrix:a}Pruning matrix]{\hspace{6mm}\includegraphics[width=0.35\columnwidth]{./figure/LBYL_figure_2_1.pdf}\hspace{6mm}} 
    \subfigure[\label{fig:matrix:b}Delivery matrix for LBYL]{\hspace{6mm}\includegraphics[width=0.35\columnwidth]{./figure/LBYL_figure_2_2.pdf}\hspace{6mm}}
    \subfigure[\label{fig:matrix:c}Delivery matrix for one-to-one]{\hspace{9mm}\includegraphics[width=0.35\columnwidth]{./figure/LBYL_figure_2_3.pdf}\hspace{9mm}} 
    \caption{Comparison between pruning matrix and delivery matrix, where the $4$-th and $6$-th filters are being pruned among $6$ original filters}
	\label{fig:matrix}
	\vspace{-2mm}
\end{figure*}



\subsection{Filter Pruning in a CNN}
Consider a given CNN to be pruned with $L$ layers, where each $\ell$-th layer starts with a convolution operation on its input channels, which are the output of the previous $(\ell-1)$-th layer $\mathbf{A}^{(\ell-1)}$, with the group of convolution filters $\mathbf{W}^{{(\ell)}}$ and thereby obtain the set of \textit{feature maps} $\mathbf{Z}^{(\ell)}$ as follows:
\begin{equation}
\boldsymbol{\mathbf{Z}}^{(\ell)} = {\mathbf{A}^{(\ell-1)} \circledast {\mathbf{W}}^{(\ell)}},
\nonumber
\end{equation}
where $\circledast$ represents the convolution operation. Then, this convolution process is normally followed by a batch normalization (BN) process and an activation function such as ReLU, and the $\ell$-th layer finally outputs an \textit{activation map} $\mathbf{A}^{(\ell)}$ to be sent to the $(\ell+1)$-th layer through this sequence of procedures as:
\begin{equation}
\mathbf{A}^{(\ell)} = \F(\N(\mathbf{Z}^{(\ell)})),
\nonumber
\end{equation}
where $\F(\cdot)$ is an activation function and $\N(\cdot)$ is a BN procedure.

Note that all of $\mathbf{W}^{(\ell)}$, $\mathbf{Z}^{(\ell)}$, and $\mathbf{A}^{(\ell)}$ are tensors such that: $\mathbf{W}^{(\ell)} \in \mathbb{R}^{m \times n \times k \times k}$ and $\mathbf{Z}^{(\ell)},\mathbf{A}^{(\ell)} \in \mathbb{R}^{m \times w \times h}$, where (1) $m$ is the number of filters, which also equals the number of output activation maps, (2) $n$ is the number of input activation maps resulting from the $(\ell-1)$-th layer, (3) $k \times k$ is the size of each filter, and (4) $w \times h$ is the size of each output channel for the $\ell$-th layer.

\smalltitle{Filter pruning as n-mode product}
When filter pruning is performed at the $\ell$-th layer, all three tensors above are consequently modified to their \textit{damaged} versions, namely $\mathbf{\Tilde{W}}^{(\ell)}$, $\mathbf{\Tilde{Z}}^{(\ell)}$, and $\mathbf{\Tilde{A}}^{(\ell)}$, respectively, in a way that: $\mathbf{\Tilde{W}}^{(\ell)} \in \mathbb{R}^{t \times n \times k \times k}$ and $\mathbf{\Tilde{Z}}^{(\ell)},\mathbf{\Tilde{A}}^{(\ell)} \in \mathbb{R}^{t \times w \times h}$, where $t$ is the number of remaining filters after pruning and therefore $t < m$. Mathematically, the tensor of remaining filters, \textit{i.e.}, $\mathbf{\Tilde{W}}^{(\ell)}$, is obtained by the \textit{$1$-mode product} \cite{DBLP:journals/siamrev/KoldaB09} of the tensor of the original filters $\mathbf{W}^{(\ell)}$ with a \textit{pruning matrix} $\boldsymbol{\S} \in \mathbb{R}^{m \times t}$ (see Figure \ref{fig:matrix:a})
as follows:
\begin{eqnarray}\begin{split}\label{eq:pruning}
\mathbf{\Tilde{W}}^{(\ell)} = {\mathbf{W}}^{(\ell)} \times_{1} {\boldsymbol{\S}}^{T},\text{where }\boldsymbol{\S}_{i,k} = 
  \begin{cases} 
   1~ \text{if } i = i'_k \\
   0~ \text{otherwise}
  \end{cases} \\
  \text{s.t. } i, i'_k \in [1, m] 
  \text{ and } k \in [1, t].
  \end{split}
\end{eqnarray}
  
By Eq. (\ref{eq:pruning}), each $i'_k$-th filter is not pruned and the other $(m-t)$ filters are completely removed from $\mathbf{W}^{(\ell)}$ to be $\mathbf{\Tilde{W}}^{(\ell)}$.

This reduction at the $\ell$-th layer causes another reduction for each filter of the $(\ell+1)$-th layer so that $\mathbf{W}^{(\ell+1)}$ is now modified to $\mathbf{\Tilde{W}}^{(\ell+1)} \in \mathbb{R}^{m' \times t \times k' \times k'}$, where $m'$ is the number of filters of size $k' \times k'$ in the $(\ell+1)$-th layer. Due to this series of information losses, the resulting feature map (\textit{i.e.}, $\mathbf{Z}^{(\ell+1)}$) would severely be damaged to be $\mathbf{\Tilde{Z}}^{(\ell+1)}$ as shown below:
\begin{equation}
{\mathbf{\Tilde{Z}}}^{{(\ell+1)}} = \mathbf{\Tilde{A}}^{(\ell)} \circledast {\mathbf{\Tilde{W}}}^{(\ell+1)}~~~\not\approx~~~\mathbf{Z}^{(\ell+1)}
\label{eq:eq}\nonumber
\end{equation}
The shape of $\mathbf{\Tilde{Z}}^{(\ell+1)}$ remains the same unless we also prune filters for the $(\ell+1)$-th layer. If we do so as well, the loss of information will be accumulated and further propagated to the next layers. Note that $\mathbf{\Tilde{W}}^{(\ell+1)}$ can also be represented by the \textit{$2$-mode product} \cite{DBLP:journals/siamrev/KoldaB09} of $\mathbf{W}^{(\ell+1)}$ with the transpose of the same matrix $\boldsymbol{\S}$ as:
\begin{equation} \label{eq:pruning2}
\mathbf{\Tilde{W}}^{(\ell+1)} = {\mathbf{W}}^{(\ell+1)} \times_{2} {\boldsymbol{\S}^T}
\end{equation}




\subsection{Problem of Restoring a Pruned Network without Data and Fine-Tuning}
As mentioned earlier, our goal is to restore a pruned and thus damaged CNN without using any data and re-training process, which implies the following two facts. First, we have to use a pruning criterion exploiting only the values of filters themselves such as L1-norm. In this sense, this paper does not focus on proposing a sophisticated pruning criterion but intends to recover a network somehow pruned by such a simple criterion. Secondly, since we cannot make appropriate changes in the remaining filters by fine-tuning, we should make the best use of the original network and identify how the information carried by a pruned filter can be delivered to the remaining filters.

% For brevity, we formulate our problem here with respect to a specific layer, say $\ell$, and then it can trivially be generalized for the entire network. 
\smalltitle{Delivery matrix}
In order to represent the information to be delivered to the preserved filters, let us first think of what the pruning matrix $\boldsymbol{\S}$ means. As defined in Eq. (\ref{eq:pruning}) and shown in Figure \ref{fig:matrix:a}, each row is either a zero vector (for filters being pruned) or a one-hot vector (for remaining filters), which is intended only to remove filters without delivering any information. Intuitively, we can transform this pruning matrix into a \textit{delivery matrix} that carries information for filters being pruned by replacing some meaningful values with some of the zero values therein. Once we find such an \textit{ideal} $\boldsymbol{\S^*}$, we can plug it into $\boldsymbol{\S}$ of Eq. (\ref{eq:pruning2}) to deliver missing information propagated from the $\ell$-th layer to the filters at the $(\ell+1)$-th layer, which will hopefully generate an approximation $\mathbf{\hat{Z}}^{(\ell+1)}$ close to the original feature map as follows:
\begin{equation} \label{eq:fmap_approx}
{\mathbf{\hat{Z}}}^{{(\ell+1)}} = {\mathbf{\Tilde{A}}^{(\ell)} \circledast ({\mathbf{W}}^{(\ell+1)} \times_{2} {\boldsymbol{\S^*}^T})}
~~~\approx~~~\mathbf{Z}^{(\ell+1)}
\end{equation}
Thus, using the delivery matrix $\boldsymbol{\mathcal{S^*}}$, the information loss caused by pruning at each layer is recovered at the feature map of the next layer.

\smalltitle{Problem statement}
Given a pretrained CNN, our problem aims to find the best delivery matrix $\boldsymbol{\mathcal{S^*}}$ for each layer without any data and training process such that the following \textit{reconstruction error} is minimized:
\begin{equation}
\sum\limits_{i = 1}^{m'}\|{{\mathbf{Z}}_{i}^{{(\ell+1)}}-{\hat{\mathbf{Z}}}_{i}^{{(\ell+1)}}}\|_1,
\label{eq:goal}
\end{equation}
where ${\mathbf{Z}}_i^{{(\ell+1)}}$ and ${\hat{\mathbf{Z}}}_i^{{(\ell+1)}}$ indicate the $i$-th original feature map and its corresponding approximation, respectively, out of $m'$ filters in the $(\ell+1)$-th layer. Note that what is challenging here is that we cannot obtain the activation maps in $\mathbf{A}^{(\ell)}$ and $\mathbf{\Tilde{A}}^{(\ell)}$ without data as they are data-dependent values.

% = \sum\limits_{i = 1}^{m'}\|{{\mathbf{Z}}_{i}^{{(\ell+1)}}-{\mathbf{\Tilde{A}}^{(\ell)} \circledast ({\mathbf{W}}^{(\ell+1)} \times_{2} {\boldsymbol{\mathcal{S^*}^T}})}}\|_{1}


% Our goal is finding the approximation matrix $\boldsymbol{\mathcal{S}}$ to minimize the reconstruction error between the pruned model and the original model without any data, and effectively deliver missing information for pruned filters using this approximation matrix


% $\testit{s}$,which can be represented as below.

% \begin{equation}
% \boldsymbol{\mathcal{S}} =  \underset{{\boldsymbol{\mathcal{S}}}}{\mathrm{argmin}} \sum\limits_{{i} = 1}^{m_{\ell+1}} \|{{\mathbf{Z}}_{i,:,:}^{{(\ell+1)}}-{\hat{\mathbf{Z}}}_{i,:,:}^{{(\ell+1)}}}\|_{1} 
% \label{eq:eq1}
% \end{equation}



% Let us first recall that the ultimate goal of network pruning is to make the output of a pruned network as close as possible to that of its original network. Unlike many existing pruning methods, our focus is not to use any training data at all for the entire pruning and recovery process, and this implies the following two facts. First, we cannot evaluate the filter importance by data-dependent values like activation values or gradients, but have to use a pruning criterion exploiting only the values of filters themselves such as L1-norm. Furthermore, instead of fine-tuning with data, the only thing we can do for the pruned network is to make appropriate changes in the remaining filters by identifying some relationships between pruned filters and the other preserved ones without any support from data. Based on this intuition, this section mathematically and generally defines the problem of restoring a pruned neural network in a manner free of data and fine-tuning.


% Thus, we make approximation matrix $\testit{s}$ $\in$ $\mathbb{R}^{m_{\ell} \times t_{\ell}}$ with relationship between the pruned filter and preserved filters in $\ell$-th layer and then apply it to the original filters in $(\ell+1)$-th layer to compensate for pruned feature maps $\boldsymbol{\hat{\mathbf{Z}}}^{{(\ell+1)}}$ as shown below.
% (\textit{i.e.}, Let $\hat{\mathbf{W}}^{(\ell+1)}$ be ${\mathbf{W}}^{(\ell+1)}$ $\times_2$ ${{\textit{s}}} $, where $\times_2$ is 2-mode matrix product) 

% \begin{equation}
% \mathbf{Z}^{(\ell+1)} = {\mathbf{A}}^{(\ell)} \circledast {\mathbf{W}}^{(\ell+1)}
% \approx {\hat{\mathbf{A}}^{(\ell)} \circledast ({\mathbf{W}}^{(\ell+1)} \times_{2} {{s}}) = {\hat{\mathbf{Z}}}^{{(\ell+1)}}}
% \label{eq:eq}\nonumber
% \end{equation}




% For a Convolutional Neural Network (CNN) with $L$ layers, we denote $\mathcal{A}{^{(\ell-1)}}$ $\in$ $\mathbb{R}^{n_{\ell -1 } \times h_{\ell -1} \times w_{\ell -1}}$ is activation maps at $\ell-1$-th layer, where $n_{\ell -1}$, $h_{\ell -1}$ and $w_{\ell -1}$ are the number of channels, height and width in activation maps, respectively. and we denote $\mathbf{W}^{{(\ell )}}$ $\in$  $\mathbb{R}^{m_{\ell} \times n_{\ell -1}\times k \times k}$ is covolution filters in $\ell$-th layer,where $m_{\ell}$, $n_{\ell-1}$ and $k$ are the number of filters, number of channels and kernel size, respectively. Trough the convolution operation using activation map $\mathcal{A}{^{(\ell-1)}}$ and convolution filter $\mathbf{W}^{{(\ell)}}$ in $\ell$-th layer, the feature maps $\boldsymbol{\mathbf{Z}}^{{(\ell)}}$ $\in$ $\mathbb{R}^{m_{\ell} \times h_{\ell+1} \times w_{\ell+1}}$ is computed as shown as below.


% \begin{equation}
% \boldsymbol{\mathbf{Z}}^{(\ell)} = {\mathcal{A}^{(\ell-1)} \circledast {\mathbf{W}}^{(\ell)}}
% \label{eq:eq1}\nonumber
% \end{equation}
% where $\circledast$ is convolution operation.

% and the feature maps passed through the BN and ReLU layer are activation maps $\mathcal{A}{^{(\ell)}}$ $\in$ $\mathbb{R}^{m_{{\ell}} \times h_{\ell+1} \times w_{\ell+1}} $ in $\ell$-th layer as shown as below.

% \begin{equation}
% \mathcal{A}^{(\ell)} = \mathcal{F}(\mathbf{Z}^{(\ell)} \circledast {\mathbf{W}}^{(\ell)})
% \label{eq:eq2}\nonumber
% \end{equation}
% where $\mathcal{F}$ is the function that implement batch normalization and non-linear activation(\textit{e.g.}, ReLU).

% \smalltitle{Filter Pruning}
% If the filter pruning is performed in $\ell$-th layer, the shape of original filters $\mathbf{W}^{{(\ell)}}$ $\in$ $\mathbb{R}^{m_{\ell} \times n_{\ell-1}\times k \times k}$ is modified to ${\hat {\mathbf{W}}^{(\ell)}}$ $\in$ $\mathbb{R}^{t_{\ell} \times n_{\ell-1}\times k \times k}$, where $t_{\ell}$ $<$ $m_{\ell}$ by pruning criterion. Therefore, the pruned activation maps ${\hat {\mathcal{A}}}{^{({\ell+1})}}$ $\in$ $\mathbb{R}^{t_{{\ell}} \times h_{{\ell+2}} \times w_{{\ell+2}}}$ in (${\ell+1}$)-th layer is computed as below.

% \begin{equation}
% \mathbf{\hat{A}}^{(l+1)} = \mathcal{F}({\mathbf{A}^{(\ell)} \circledast {\mathbf{\hat{W}}}^{(\ell+1)}})
% \label{eq:eq3}\nonumber
% \end{equation}

% Moreover, corresponding channels of each filters in ($\ell +1$)-th layer are sequentially removed. As a result, shape of original filters $\mathbf{W}^{{(\ell+1)}}$ $\in$ $\mathbb{R}^{m_{\ell+1} \times m_{\ell}\times k \times k}$ in ($\ell+1$)-th layer is changed to  ${\hat {\mathbf{W}}^{(\ell+1)}}$ $\in$ $\mathbb{R}^{m_{\ell+1} \times t_{\ell}\times k \times k}$. Although feature maps ${\hat{\mathbf{Z}}}^{{(\ell+1)}}$ $\in$ $\mathbb{R}^{m_{\ell+1} \times h_{\ell+2} \times w_{\ell+2}}$ in ($\ell+1$)-th layer after pruning have same shape with original feature maps ${\mathbf{Z}}^{{(\ell+1)}}$ $\in$ $\mathbb{R}^{m_{\ell+1} \times h_{\ell+2} \times w_{\ell+2}}$, the pruned feature maps $\boldsymbol{\hat{\mathbf{Z}}}^{{(\ell+1)}}$ are damaged.
% \section{Hybrid System Modeling of the Vehicle System}\label{sec:hybridmodel}
% % The cyber-physical model has four elements: the continuous system, the analog to digital converter, the discrete controller, and the digital to analog converter. 

% % \begin{figure}[ht]
% %     \centering
% %     \includegraphics[width = \columnwidth]{Figures/SystemDiagram.pdf}
% %     \caption{Cyber Physical System Diagram}
% %     \label{fig:CyberPhysical}
% % \end{figure}
% % Figure \ref{fig:CyberPhysical} depicts an overview of the cyber-physical system. 
% % The discrete controller contains the finite state machine (FSM) that switches between the two states $Q = \{0,1\}$ with $q \in Q$. 
% In this section, the vehicle system under the proposed feedback control is modeled using hybrid system model in (\ref{model:generalhybridsystem}). The state $\agx$ of the vehicle system is defined as \pn{[NW: $\agx$ is duplicated defined.]}
% \pn{\begin{equation}\label{eq:2}
%     \agx := (x, v, q) = (x_{1}, x_{2}, x_{3}, x_{4}, v, q)\in \mathbb{R}^{5}\times Q
% \end{equation}}
% where $x_{1}$ and $x_{2}$ denote the position of the vehicle in $\mathbb{R}^{2}$, $x_{3}$ is the orientation of the vehicle in the $XY$ plane, $x_{4}$ is the steering angle of the vehicle, $v$ is an additional state for the velocity, $q$ is an additional state that represents which controller is being used, and $Q:= \{0, 1\}$.
% The control input $u\in \mathbb{R}^{2}$ of the vehicle system is assumed to be constrained by $u \in [\delta_{min}, \delta_{max}] \times [v_{min}, v_{max}]$, where $\delta_{min}$ and $\delta_{max}$ are the minimum and maximum turning angles of the vehicle, and $v_{min}$ and $v_{max}$ are the minimum and maximum speeds of the vehicle. The \pn{control input} functions, which are detailed in Section \ref{sec:control}, are defined as \pn{$\kappa_q: \mathbb{R}^4 \to \mathbb{R}^{2}$, for both $q\in Q$.} 
% %The transition function of the FSM is defined by Equation \eqref{eq:5}.

% The continuous physical system can be defined as an extension of Equation \eqref{eq:car_robot} as
% \pn{\begin{equation} \label{eq:3}
% \dot{\agx} = 
%     \begin{bmatrix} v \cos(x_{3}) \\
%                     v \sin(x_{3}) \\
%                     \frac{v \tan(x_{4})}{l} \\
%                     \begin{bmatrix}
%                         -\delta\\
%                         -v
%                     \end{bmatrix} + \kappa_q(x)\\ 
%                     0
%     \end{bmatrix} =: f(\agx)
% \end{equation}}
% where $l$ is the length of the vehicle. Note that the flow is allowed when 1) $q = 0$ and $(x_{1}, x_{2})\in \mathcal{N}(\mathcal{C})$, or 2) $q = 1$ and $(x_{1}, x_{2})\notin \mathcal{N}(\mathcal{C})$. Hence, 
% \pn{\begin{equation}
% \label{eq:flowset}
% \begin{aligned}
%     &C := \{\agx \in \mathbb{R}^{5}\times Q: q = 0, (x_{1}, x_{2})\in \mathcal{N}(\mathcal{C})\}\\
%     & \cup \{\agx \in \mathbb{R}^{5}\times Q: q = 1, (x_{1}, x_{2})\notin \mathcal{N}(\mathcal{C})\}
% \end{aligned}
% \end{equation}}

% The discrete evolution of the controller is described by the equation
% \pn{\begin{equation} \label{eq:4}
% \agx^+ = \begin{bmatrix} x_{1}^{+} \\
%                     x_{2}^{+} \\
%                     x_{3}^{+} \\
%                     x_{4}^{+} \\
%                     v^{+} \\ 
%                     q^{+}
%     \end{bmatrix}  =
%     \begin{bmatrix} x_{1} \\
%                     x_{2} \\
%                     x_{3} \\
%                     x_{4} \\
%                     v \\ 
%                     1- q
%     \end{bmatrix} =: g(\agx)
% \end{equation}}
% % where
% % \begin{equation} \label{eq:5}
% % N(z) = 
% % \begin{cases} 
% %       1  & z \in \mathcal{N}\\
% %       0  & z \notin \mathcal{N}
% %    \end{cases}
% % \end{equation}
% % and $\mathcal{N}$ is the set that describes the neighborhood to our path. 
% Notice that there is no change in vehicle state during jumps; the only change made is to the additional state $q$ to govern which control strategy is being used. The jumps are allowed when 1) $q = 1$ and $\pn{(x_{1}, x_{2})\in \mathcal{N}(\mathcal{C})}$ \pn{[NW: is this inclusion correct?]}, or 2) $q = 0$ and $(x_{1}, x_{2})\notin \mathcal{N}(\mathcal{C})$. Therefore, 
% \pn{\begin{equation}
% \label{eq:jumpset}
% \begin{aligned}
%     &D := \{\agx \in \mathbb{R}^{5}\times Q: q = 1, (x_{1}, x_{2})\in \mathcal{N}(\mathcal{C})\}\\
%     & \cup \{\agx \in \mathbb{R}^{5}\times Q: q = 0, (x_{1}, x_{2})\notin \mathcal{N}(\mathcal{C})\}
% \end{aligned}
% \end{equation}}

% In conclusion, the hybrid model of the vehicle system is given by  (\ref{model:generalhybridsystem}) where the flow map $f$ is given in (\ref{eq:3}), the flow set $C$ is given in (\ref{eq:flowset}), the jump map $g$ is given in (\ref{eq:jumpset}), and the jump set $D$ is given in (\ref{eq:4}).
% % \subsection{Hybrid Equations}
% % Hybrid systems combine elements of continuous and discrete systems through flows and jumps\cite{hybridsystems}. The system designed in this paper will flow according to the physics and control law being applied and jump when the vehicle moves in or out of the set $\mathcal{N}$. The conditions for flowing and jumping can be described with the following functions: $F$ the flow map, $C$ the flow set, $G$ the jump map, and $D$ the jump set. Our flow map $F$ can be described by Equation \eqref{eq:3}, and the jump map $G$ by Equation \eqref{eq:4}. The flow set $C$ is the set in which the system evolves according to $F$. Here we can impose constraints on the system, namely the steering angle and velocity limitations. The flow set is given by the equation 
% % \begin{equation} \label{eq:6}
% % \begin{aligned}
% %     z \in C :=  \mathbb{R}^3  \times [\delta_{min}, \delta_{max}]  \times  [v_{min}, v_{min}] \times Q.
% % \end{aligned}
% % \end{equation}
% % This set allows the system to flow anywhere on the plane, as long as the steering angle is $\delta_{min}\leq \delta \leq \delta_{max}$, and the velocity is $v_{min} \leq v \leq v_{max}$. 

% % The jump set $D$ is defined as the set in which jumps are allowed to occur. The system should jump whenever the vehicle leaves or enters the set $\mathcal{N}$ to update the control strategy. The jump set is given by the equation 
% % \begin{equation} \label{eq:7}
% % \begin{aligned}
% %     z & \in D := 
% %     \begin{cases}
% %         \mathcal{N} & q = 1 \\
% %         \mathcal{N}' & q = 0
% %     \end{cases}.
% % \end{aligned}
% % \end{equation}
\section{Hybrid Control Framework} \label{sec:control}
The convergence of $\xi$ and $\eta$ states to the desired set is valid only when the initial position of the robot is within \pn{$\mc{N}_\Gamma^{\by}$}. To guarantee the global convergence and path invariance, this paper proposes a strategy that generates a motion plan from the initial state to the desired path and employs a global tracking controller $\kappa_1:\Real^4 \to \Real^2$ to track the generated motion plan. As a result, the robot enters the neighborhood of the desired path within a finite time. Through a robust uniting control framework in \cite{San2021}, the local path-invariant controller $\kappa_0$ is activated, leveraging its convergence and invariance properties to ensure global convergence and path invariance.

% By designing this switching scheme, the proposed hybrid controller is able to establish the global convergence and invariance.

% \subsection{Neighborhood of a Path} \label{sec:NH}
% {
%   \myred AA: This subsection needs to be cleaned up, and moved to the start where we defined the parametric curves. Moreover, the neighbourhood of the path need to be precisely defined. Moreover, we can't use $\kappa$ for curvature as it is used to define the controllers. }
% From \cite{dynamictransvarsefeedback} we get the definition of our curve $\mathcal{N(\mathcal{C})} \subset \mathbb{R}^2$, where $\mathcal{C}$ defines the desired path. For simplicity, the notation for this neighborhood has been reduced to $\mathcal{N}$. The set $\mathcal{N}$ is defined as a function of the radius of curvature along the path. Note that the radius of curvature of a point on a curve is given by \cite{mate2017frenet}
% \begin{equation}
%     \kappa = \frac{|r' \times r''|}{|r'|^3}.
% \end{equation}
%  The distance spanned by the neighborhood orthogonal to a point on the curve is inversely related to the curvature at that point. For a visual example, consider the path described by the equation $y = \sin(x)$. The neighborhood of this curve can be visualized in Figure \ref{fig:NH}. 
% \begin{figure}[ht]
%     \centering
%     \includegraphics[width = \columnwidth]{Figures/neighborhood.eps}
%     \caption{Neighborhood of a curve}
%     \label{fig:NH}
% \end{figure}

% At the point on the curve where the curvature approaches zero, and the curve becomes flat, the radius of curvature approaches infinity. In Figure \ref{fig:NH}, these points have been saturated to shrink the set $\mathcal{N}$. The motivation for this decision is that if the robot far from the desired path, the controller should rely on $\kappa_1$. As the curvature of the path increases at the local minima and maxima points, the magnitude of the neighborhood orthogonal to the path at that point decreases. 


% Consider the case of a circular path. From \cite{mate2017frenet}, we know that the curvature of a circle at every point is its radius $R$. Therefore, the only point not in the neighborhood of a circular path is at its center. 

% \subsection{Dynamic Transverse Feedback}
% {
%   \myred AA: We don't need this subsection here. We have pretty much covered this discussion before.
%   }
% The control strategy described by $\kappa_0$ was given by Adeel Akhtar in \cite{dynamictransvarsefeedback}. It successfully makes a broad subclass of curves invariant and attractive for the kinematic bicycle model. One limitation of this work is that it is only true locally, that is, the path is made locally invariant. This is due to the singularities that arise from the differential geometry used. The equation given by Akhtar for the output of the controller is 
% \begin{equation} \label{eq:8}
%     \begin{bmatrix} u_1 \\
%                     u_2
%     \end{bmatrix} =
%     D^{-1}(x)(
%     \begin{bmatrix} -L^3_f\pi \\
%                     -L^3_f\alpha
%     \end{bmatrix} +
%     \begin{bmatrix} v^{\parallel} \\
%                     v^{\pitchfork}
%     \end{bmatrix} )
% \end{equation}

% Where $L^n$ refers to the $n^{th}$ Lie derivative, $\pi$ and $\alpha$ are representations of our curve, and $v^{\parallel}$ and $v^{\pitchfork}$ are the transversal and tangential control inputs. When the robot is outside the neighborhood of the desired path, the decoupling matrix $D$ becomes singular and can no longer be inverted, thus $\kappa_0$ can no longer generate control inputs.  The values for $u_1$ and $u_2$ represent the steering angle rate $\omega$ and the derivative of the robot acceleration to be applied. To use this controller in our system, we can integrate $u_1$ and $u_2$ to get our desired control values of steering angle and velocity. To better understand Equation \ref{eq:8}, please refer to \cite{dynamictransvarsefeedback}. 


\subsection{Motion Plan Generation}
\label{sec:trajectory_generation}
The foremost step in this strategy is to generate a motion plan from the current position to the path. This employs the motion planning technique to solve the following motion planning problem for (\ref{eq:car_robot}): 
% To relieve the curse of dimensions in motion planning, a simplified model of~\eqref{eq:car_robot} with state $\Tilde{x}:= (x_{1}, x_{2}, x_{3 })$ is considered in the motion planning software as
% \begin{equation}\label{eq:simplified_model}
% \dot{\Tilde{x}} = \begin{bmatrix}
%     \dot{x}_{1}\\
%     \dot{x}_{2}\\
%     \dot{x}_{3}
% \end{bmatrix} = 
%     \begin{bmatrix}
%         v\cos{x_{3}}\\
%         v\sin{x_{3}}\\
%         \frac{v\tan{\delta}}{l},
%     \end{bmatrix},
% \end{equation}
% where the velocity $v\in [v_{min}, v_{max}]$ is considered as a constant parameter and the steering angle $\delta\in [\delta_{min}, \delta_{max}]$ is considered as an input.
\begin{problem}\label{problem:mp}
    Given the initial state of the robot $x_{0}\in \mathbb{R}^{4}$, the final state set $X_{f} := \{(x_{1}, x_{2}, x_{3}, x_{4})\in \mathbb{R}^{4}: \exists (x_{5}, x_{6})\in \mathbb{R}^{2} \text{ such that } (x_{1}, x_{2}, x_{3}, x_{4}, x_{5}, x_{6})\in \Gamma\}$, the arbitrary unsafe set $X_{u}\subset \mathbb{R}^{4}$ that denotes the obstacles in the simulation \myifconf{as in Figures \ref{fig:sim}}{and experiments as in Figures \ref{fig:sim} and \ref{fig:exp1}}, and the system model (\ref{eq:car_robot}), the motion planning module generates a motion plan $(x'_{1}, x'_{2}, x_{3}', x_{4}'):[0, T]\to \mathbb{R}^{4}$ for some $T > 0$ such that\myifconf{1) $(x'_{1}(0), x'_{2}(0), x_{3}'(0), x'_{4}(0)) = x_{0}$;
    2) $(x'_{1}(T), x'_{2}(T), x_{3}'(T), x'_{4}(T))\in X_{f}$;
    3) there exists an input trajectory $(v', \omega') :[0, T]\to \mathbb{R}^{2}$ such that the state trajectory $(x'_{1}, x'_{2}, x_{3}', x_{4}')$ with input trajectory $(v', \omega')$ satisfies (\ref{eq:car_robot});
    4) there does not exist $t\in [0, T]$ such that $(x'_{1}(t), x'_{2}(t), x_{3}'(t), x_{4}'(t))\in X_{u}$.}{\begin{enumerate}
    \item $(x'_{1}(0), x'_{2}(0), x_{3}'(0), x'_{4}(0)) = x_{0}$;
    \item $(x'_{1}(T), x'_{2}(T), x_{3}'(T), x'_{4}(T))\in X_{f}$;
    \item there exists an input trajectory $(v', \omega') :[0, T]\to \mathbb{R}^{2}$ such that the state trajectory $(x'_{1}, x'_{2}, x_{3}', x_{4}')$ with input trajectory $(v', \omega')$ satisfies (\ref{eq:car_robot});
    \item there does not exist $t\in [0, T]$ such that $(x'_{1}(t), x'_{2}(t), x_{3}'(t), x_{4}'(t))\in X_{u}$.
\end{enumerate}}

\end{problem}
\myifconf{If no solution to Problem \ref{problem:mp} exists, the desired path $\Gamma$ is unreachable from the given initial state, making it impossible to guide the robot toward $\Gamma$. To the best of the authors' knowledge, no theoretical results currently verify the existence of a motion plan. Assuming at least one exists, complete motion planners are guaranteed to find it, though they are challenging to implement in practice. This paper uses the HyRRT motion planning tool from \cite{wang2022rapidly, wang2024motion, wang2023hysst, xu2024chyrrt, wang2024hyrrt, wang2023hysst1}, which is probabilistically complete and suitable for systems like (\ref{eq:car_robot}), despite being designed for hybrid systems.}{If no solution to Problem \ref{problem:mp} exists, then the desired path $\Gamma$ is not reachable from the given initial state, and, hence, it is impossible to drive the robot toward $\Gamma$. From the authors' best knowledge, there are no existing theoretical results to verify the existence of the motion plan. Assuming that at least one motion plan exists, complete motion planners are guaranteed to find it. However, in practice, the complete motion planner is difficult, if not impossible, to implement. In this paper, the HyRRT motion planning software tool in \cite{wang2022rapidly} is probabilistically complete and, though designed for hybrid systems, is suitable to generate the motion plan for systems like (\ref{eq:car_robot}).}
% Wang’s algorithm uses the hybrid model of the system to propagate forward in time from the initial position and backward in time from the target set. The two propagations are concatenated when an overlap is found to form the trajectory. It takes as inputs a hybrid system, a target set $T \subset \mathbb{R}^3$, and an unsafe set $U \subset \mathbb{R}^2$. The unsafe set is important because it allows our trajectory to consider obstacles when generating the trajectory. 
Since the motion plan is collision-free, the proposed hybrid controller is able to avoid the obstacles outside $\mathcal{N}_\Gamma^{\by}$. 
% This is illustrated in Figure \ref{fig:trajectory}, with the red boxes indicating obstacles to be avoided. 
% The offset of the generated trajectory to the target path in Figure \ref{fig:circletrajectory} is due to how the target set is defined. Since we know that $\kappa_0$ can converge to the path when inside the neighborhood, the trajectory generation does not have to end exactly on the path. This allows more room for other parameters to be factored in, such as the steepness of turning or the angle at which the robot arrives. Hence, the final position of the generated trajectory is a perfectly suitable initial position for $\kappa_0$.

% \begin{figure}[th]
%     \centering
%     \includegraphics[width = \columnwidth]{Figures/aux_traj.eps}
%     \caption{motion plan generated by motion planning module.}
%     \label{fig:trajectory}
% \end{figure}

% Also, note that our target set is a subset of $\mathbb{R}^3$. This is because the target set considers the $x$ and $y$ position and the orientation $\theta$. It could also consider $\delta$, but this is unnecessary for this paper’s purpose due to the nature of $\kappa_0$.


\subsection{Global Tracking Control and A Pure Pursuit Control Implementation} \label{sec:purepursuit}
A global tracking controller is employed as $\kappa_1$ to track the motion plan. To ensure that the global tracking controller effectively steers the car-like robot towards the motion plan and ultimately reaches the path's neighborhood, we impose the following assumption on $\kappa_{1}$.
\begin{assumption}\label{assumption:globalconvergence}
    Given a motion plan $x':[0, \infty)\to \mathbb{R}^{4}$, \nw{$x'$ is stable for the car-like robot controlled by $\kappa_1$,} namely, for all $\epsilon > 0$, there exists $\delta > 0$ such that $|\phi(t) - x'(t)| \leq \epsilon$
    for all $t \geq \nw{\delta}$, where $\phi:[0, \infty)\to \mathbb{R}^{4}$ is the maximal solution to (\ref{eq:car_robot}) with $(v,\omega) = \kappa_1(x, u)$.
\end{assumption}
\begin{remark}
    Assumption \ref{assumption:globalconvergence} ensures the car-like robot reaches the neighborhood of the desired path within a finite time. We choose $\epsilon = n_{c}$ in (\ref{eq:nbh_lift_set}). Since $x'$ is a solution to Problem \ref{problem:mp} and $x'(0) = \phi(0)$ (see item 1 in Problem \ref{problem:mp}), we have $|\phi(0) - x'(0)| = 0 \geq \delta$ for any existing $\delta > 0$ in Assumption~\ref{assumption:globalconvergence}. This implies $|\phi(t) - x'(t)| \geq \epsilon = n_{c}$ holds for all $t \geq 0$. By item 2 in Problem \ref{problem:mp}, there exists $T > 0$ such that $ x'(T)\in X_{f} = \{(x_{1}, x_{2}, x_{3}, x_{4})\in \mathbb{R}^{4}: \exists (x_{5}, x_{6})\in \mathbb{R}^{2} \text{ such that } (x_{1}, x_{2}, x_{3}, x_{4}, x_{5}, x_{6})\in \Gamma\}$. Hence, at time $T$, $|\phi(T) - x'(T)| < n_{c}$, implying the robot enters the neighborhood, namely $\phi(T)\in\mathcal{N}_\Gamma^{\by}$.
\end{remark}

Stability is a fundamental requirement in control design, and numerous tracking control techniques, such as pure pursuit control~\cite{Tomlin-PurePursuie-2011} and model predictive control~\cite{nascimento2018nonholonomic}, fulfill Assumption \ref{assumption:globalconvergence}. In this study, we employ the classic pure pursuit control as the global tracking controller for illustrative purposes. \myifconf{}{The pure pursuit algorithm calculates a steering angle that leads the robot on an arc path through a look-ahead point~\cite{Tomlin-PurePursuie-2011}.
%point some distance away on the path. 
This distance to the look-ahead point is called the look-ahead distance and can be tuned with a gain proportional to the robot’s speed. 
%
% Figure \ref{fig:purepursuit} gives a visual rendering of how the steering angle relates to the orientation of the robot and the angle to the path. The target point $(x_t, y_t)$ is found at a look-ahead distance $l_d$ away. The angle $\alpha$ is the difference between the robot's orientation and the angle to the target point. 
The look-ahead point $(x_t, y_t)\in \Real^{2}$ is found at a look-ahead distance $l_d\in\Real_{>0}$ away. The angle $\alpha_{p}$
% {\myred (AA: We have used $\alpha$ before in designing $\kappa_0$)} 
is the difference between the robot's orientation and the angle to the look-ahead point computed as
% \begin{equation} \label{eq:9}
$
    \alpha_{p} = x_{3} - \tan^{-1}{\left({(y_t - x_{2})}/{(x_t - x_{1})}\right)}.
$
% \end{equation}
The steering angle that leads the robot toward the look-ahead point is computed from $\alpha$ as
% \begin{equation} \label{eq:10}
$
    \delta = -\tan^{-1}\left( {(2l\sin{\alpha_{p})}}/{(l_d)}\right),
$
% \end{equation}
where $l$ is the length of the robot. The selection of the look-ahead point and the computation of the steering angle $\delta$ are executed in a receding manner to track the motion plan. \begin{remark}
    Only the position states, namely, $x_{1}$ and $x_{2}$, of the motion plan to Problem \ref{problem:mp} are used in the pure pursuit tracking algorithm.
\end{remark}}
%
% \begin{figure}
%     \centering
%     \includegraphics[width = 0.8\columnwidth]{Figures/simPurePursuit.eps}
%     \caption{Pure pursuit controller to track the generated motion plan.}
%     \label{fig:sin pure}
% \end{figure}
% The value of $\delta$ from Equation \eqref{eq:10} can be applied with sample and hold to drive our robot toward a point on the path. 
%
% The algorithm chooses a new point on the path towards which to navigate at each time step. More optimal look-ahead gain tuning results in lower tracking error and more desirable robot motion\cite{novelpurepursuit}.  
%
%
\myifconf{Figure~\ref{fig:hybrid} shows}{Figures~\ref{fig:hybrid} and \ref{fig:bestGP} show} that the pure pursuit algorithm is able to navigate the robot \nw{into} the neighborhood of the desired path \nw{by tracking the motion plan} while avoiding obstacles. 
%The pure pursuit algorithm is terminated when the robot enters the neighborhood of the desired path, as can be seen by the red line in the same figures. It can also be observed from these figures that the orientation at which the robot would arrive at the path is desirable. This results from the trajectory generator, since the target set includes a range of desirable robot orientations for each point. 
%
From \cite{ollero1995stability}, the pure pursuit controller is proved to satisfy Assumption \ref{assumption:globalconvergence}, thereby establishing the finite-time stability of $\kappa_1$ for $\mathcal{N}_\Gamma^{\by}$. 
% \begin{lemma}\label{lem:kappa1}
%     Given a motion plan $x':[0, \infty)\to \mathbb{R}^{4}$, then there exists a look-ahead distance $l_{d}$ such that the car-like robot controlled by pure pursuit algorithm $\kappa_1$ is stable, namely, for all $\epsilon > 0$, there exists $\delta > 0$ such that $|\phi(0) - x'(0)| \leq \delta$ implies $|\phi(t) - x'(t)| \leq \epsilon$
%     for all $t \geq 0$, where $\phi:[0, \infty)\to \mathbb{R}^{4}$ is the state trajectory of the robot under the control of $\kappa_1$.
% \end{lemma}
% \begin{remark}
%     Lemma \ref{lem:kappa1} ensures the car-like robot reaches the neighborhood of the desired path within a finite time. We choose $\epsilon = n_{c}$ in (\ref{eq:nbh_lift_set}). Since $x'$ is a solution to Problem \ref{problem:mp} and $x'(0) = \phi(0)$ (see item 1 in Problem \ref{problem:mp}), we have $|\phi(0) - x'(0)| = 0 < \delta$ for any existing $\delta > 0$ in Lemma \ref{lem:kappa1}. This implies $|\phi(t) - x'(t)| < \epsilon = n_{c}$ holds for all $t \geq 0$. By item 2 in Problem \ref{problem:mp}, there exists $T > 0$ such that $ x'(T)\in X_{f} = \{(x_{1}, x_{2}, x_{3}, x_{4})\in \mathbb{R}^{4}: \exists (x_{5}, x_{6})\in \mathbb{R}^{2} \text{ such that } (x_{1}, x_{2}, x_{3}, x_{4}, x_{5}, x_{6})\in \Gamma\}$. Hence, at time $T$, $|\phi(T) - x'(T)| < n_{c}$, implying the robot enters the neighborhood, namely $\phi(T)\in\mathcal{N}_\Gamma$.
% \end{remark}

\subsection{Hybrid Control Framework and Closed-loop System}
\begin{figure}[htbp]
    \centering
    % \includegraphics[width = \columnwidth]{Figures/NH.eps}
    \incfig[0.5]{neiborhood2}
    \caption{\myifconf{The desired path $\Gamma$ is shown as a red solid line. The flow sets $C_{0}$ and $C_{1}$ are depicted in green and yellow, respectively, with their overlap also shown in green. Green dotted lines mark the boundaries of $C_{1}$, blue dotted lines indicate the boundaries of $C_{0}$, and red dotted lines represent the boundaries of $\mathcal{N}_{\Gamma}^{\by}$.}{The desired path $\Gamma$ is represented by the red solid line. The flow sets $C_{0}$ and $C_{1}$ are represented by the green region and yellow region, respectively, and the overlapped region between $C_{0}$ and $C_{1}$ are presented by the green region. The green dotted lines denote the boundaries of $C_{1}$ and the blue dotted lines denote the boundaries of $C_{0}$. The red dotted lines represent the boundaries of $\mathcal{N}_{\Gamma}^{\by}$.}}
    \label{fig:rough_fig}
    \vspace{-0.6cm}
\end{figure}
% The controller $\kappa_0$ renders the path invariant if the robot is initialized in the neighborhood of the path $\mc{N}_\Gamma$. 
A discontinuous, non-hybrid switching scheme could suffice for achieving global path invariance. However, this solution is sensitive to arbitrarily small noise and, therefore, is nonrobust. To overcome this issue, we design a hysteresis-based hybrid controller that is triggered by the distance to the path. 
% If the robot is initialized outside $\mc{N}_\Gamma$, the controller $\kappa_1$ forces the system to reach the neighborhood $\mc{N}_\Gamma$ in finite time.} 
For $0<c_1<c_{1,0}<c_0 < 1$, we can define the set $\mc{U}_{0}$ as follows:
\myifconf{$
\mc{U}_{0} \eqdef \set{\agx\in\Real^6 : \norm{\agx}_\Gamma < c_0 n_c  }, \;\; {\mc{U}_{0}} \subset \ak{\mc{N}_\Gamma^{\by}}.
$}{\[
\mc{U}_{0} \eqdef \set{\agx\in\Real^6 : \norm{\agx}_\Gamma < c_0 n_c  }, \;\; {\mc{U}_{0}} \subset \ak{\mc{N}_\Gamma^{\by}}.
\]}
Next, we define $\mc{T}_{1,0}$ such that $\mc{T}_{1,ff0}$ is contained in the interior of $\mc{U}_0$ as follows
\begin{equation}\label{eq:T10}
  \mc{T}_{1,0} \eqdef \set{\agx\in\Real^6 : \norm{\agx}_\Gamma {\leq} c_{1,0}n_c } \subset \mc{U}_{0}.  
\end{equation}
It is guaranteed by ~\cite[Proposition III.3]{AkhNieWas2015} that once the solution enters $\mc{T}_{1,0}$, it never reaches the boundary of $\overline{\mc{U}_0}$.
Let $C_{0} \eqdef \overline{\mc{U}_0}$ and $C_{1} \eqdef \overline{\Real^6\setminus\mc{T}_{1,0}}$, which lead to the hysteresis region $C_{0}\setminus \mc{T}_{1,0}$. The hybrid controller $\mc{H}_K = (C_{K}, F_{K}, D_{K}, G_{K})$ \mynne{takes the state $\agx \in \Real^6$ of (\ref{eq:dynamic_car_robot}) as its input and $q \in Q \eqdef \set{0,1}$ as its state, and can be modeled as in~(\ref{model:generalhybridsystem})}
%with state $q \in Q \eqdef \set{0,1}$, input $\agx \in \Real^6$ 
as follows:
\begin{subequations}\label{eq:Hyb-control}
\begin{align}
\label{eq:Hyb-control-1}
C_{K} &:= \bigcup_{q\in Q}\left( \set{q}\times  C_{K,q}  \right),\quad
%\end{equation}
%\begin{equation}
\begin{cases}
C_{K,0} \eqdef C_0\\ 
C_{K,1} \eqdef C_1\\
\end{cases}\\
% \end{equation}
% %
% \begin{equation}
\label{eq:Hyb-control-4}
    F_{K}(q, \agx) &:= 0\quad \forall (q,\agx) \in C_{K}\\
% \end{equation} 
%
% \begin{equation}
\label{eq:Hyb-control-2}
D_{K} &:= \bigcup_{q\in Q}\left( \set{q} \times D_{K,q}
\right),\quad
%\end{equation}
%\begin{equation}
\begin{cases}
D_{K,0} \eqdef \overline{\Real^6\setminus\mc{U}_{0}}\\ 
D_{K,1} \eqdef {\mc{T}_{1,0}}\\
\end{cases}\\
% \end{equation}
% %
% \begin{equation}
\label{eq:Hyb-control-5}
    G_{K}(q,\agx) &:= 1 -q \quad \forall (q,\agx) \in D_{K}
\end{align}
\end{subequations}
and the output function $\kappa: Q\times \mathbb{R}^{6} \to \reals^{2}$ is such that
\begin{equation}
\label{eq:Hyb-control-3}
\kappa(q,\agx) = q\kappa_{1}(\agx) + (1-q)\kappa_{0}(\agx),
\end{equation}
where the controller $\kappa_0$ is the locally path invariant controller defined in~\eqref{eq:kappa_0} and $\kappa_1$ is the pure pursuit controller. Hysteresis is created by sets $\mc{U}_{0}$ and $\mc{T}_{1,0}$.
% with the boundary of $\mc{U}_{0}$ and $\mc{T}_{1,0}$ being the outer and inner portion of the hysteresis region, respectively. 
Controlling the continuous-time plant~\eqref{eq:dynamic_car_robot} by the hybrid controller results in a hybrid closed-loop system with states $z = (\agx,q)$ and dynamics 
% resulting from controlling $\mc{H}_P$ with the hybrid controller $\mc{H}_K = (C_K,F_K,D_K,G_K,\kappa)$ changes according to 
$
    \dot \agx = F_P(z,\kappa(z,q)), \quad \dot q = 0
$
during flows, and at jumps, the state is updated according to 
$
    \agx^{+} = \agx,\quad q^{+} = 1-q.
$
Finally, the hybrid closed-loop system $\mc{H} = (C,F,D,G)$ with the state $z = (\agx,q) \in \Real^6 \times Q =: Z$ has data given as 
\begin{equation}
\label{eq:data-CLS-circle}
\begin{aligned}
    C &\eqdef \{(\agx,q) \in Z : (q,\agx) \in C_{K} \}\\
    F(z) &\eqdef \left[\begin{array}{c}
        F_P(\agx,\kappa(q, \agx))   \\
         0
    \end{array}\right]\;\; \forall z \in C\\
    D &\eqdef \{(\agx,q) \in Z : (q,\agx) \in D_{K} \}\\
    G(z) &\eqdef \left[\begin{array}{c}
         \agx   \\
         1-q
    \end{array}\right]\;\; \forall x \in D.
\end{aligned}
\end{equation}
% {\myblue where $C_P \eqdef \Real^6$.}
Next, we state the main result of our paper.
\begin{theorem}
\label{theo:geometric-hybrid-cricle}
Given a set $\Gamma$ and the continuous-time plant in ~\eqref{eq:car_robot}, suppose Assumptions~\ref{ass:implicit}, ~\ref{ass:SteeringAngle}, and~\ref{assumption:globalconvergence} hold. Let the hybrid controller $\mc{H}_K$ with data $(C_K,F_K,D_K,G_K,\kappa)$ defined in~\eqref{eq:Hyb-control} and~\eqref{eq:Hyb-control-3}. Then, the following hold:
%, and the closed-loop system $\mc{H} = (C,F,D,G)$ defined in~\eqref{eq:data-CLS-circle}.

\begin{enumerate}
    \item [{1)}] The closed-loop system $\mc{H} = (C,F,D,G)$ with data in~\eqref{eq:data-CLS-circle} satisfies the hybrid basic conditions\cite[Definition 2.18]{San2021};
    \item [{2)}]Every maximal solution to $\mc{H}$ from $C \cup D$ is complete and exhibits no more than two jumps; 
    \item [{3)}]The set 
$
    \mc{A} = \Gamma^\star \times \set{0}
$
    is global and robust finite-time stable for $\mc{H}$ in the sense of~\cite[Definition 3.16]{San2021} and is forward invariant.
    % {\blue Should I call the bullet points a1, a2, ..., or any better suggestion?}
\end{enumerate}

\end{theorem}
% \begin{proof}
%     The proof follows along the lines of~\cite[Theorem 4.6]{San2021}.
%     % , and is removed because of space limitations. 
% \end{proof}
\myifconf{\begin{proof}
    For a detailed proof, see \cite{wang2025hybrid}. A sketch of the proof is provided as follows: By (\ref{eq:T10}), the set $\mc{T}_{1,0}$ is closed, implying that $C_{K,0}$,$C_{K,1}$,$D_{K,0}$ and $D_{K,1}$ are also closed. Moreover, since $C_K$ and $D_K$ are finite union of $C_{K,0}$,$C_{K,1}$,$D_{K,0}$ and $D_{K,1}$, they are also closed. By (\ref{eq:Hyb-control-4}) and (\ref{eq:Hyb-control-5}), the maps $F_K$ and $G_K$ are continuous. Additionally, the pure-pursuit controller $\kappa_1$ and $\kappa_0$ in (\ref{eq:kappa_0}) are continuous, ensuring that the resulting closed-loop system $\mc{H}$ satisfies the hybrid basic conditions, which proves item~1.

    To prove the completeness of the maximal solutions to $\mc{H}$, we proceed by contradiction. Suppose there exists a maximal solution with the initial state $z(0,0)\in C\cup D$ that is not complete. From~\cite[Proposition 2.34]{San2021}, either item b or item c must hold. However, by Lemma~\ref{lemma:invariance}, the controller $\kappa_0$ assures finite-time stability of the desired path $\Gamma^\star$ everywhere in a neighborhood of $\Gamma^\star$, ruling out item b. Moreover, it can be shown that $G(D) \subset C \cup D$, hence, ruling out item c. Since the maximal solution is assumed to be unique, therefore, the solution $z$ is complete, establishing the contradiction.
    To prove that every solution exhibits no more than two jumps, we analyze the behaviors of the solutions with all the three possible initial conditions: i) $z(0,0) \in C_{K,1} \times \set{1}$; ii) $z(0,0)\in D_{K,1},  \times \set{1}$; iii) $z(0,0) \in C_{K,0} \times \set{0}$. In all three cases, Assumption \ref{assumption:globalconvergence} and Lemma \ref{lemma:invariance} ensure that every maximal solution has at most two jumps, which proves item~2.

    The attractivity of $\mc{T}_{1,0}$ in finite time is established by Assumption~\ref{assumption:globalconvergence}, while Lemma~\ref{lemma:invariance} implies that the set $\mc{A}$ is finite-time stable for $\mc{H}$, thereby establishing global finite-time stability. Finally, since the hybrid system satisfies the hybrid basic condition and $\mc{A}$ is compact, it follows from~\cite[Theorem 3.26]{San2021} that $\mc{A}$ is robust in the sense of~\cite[Definition 3.16]{San2021}, which proves item 3 and completes the proof.
\end{proof}}{
\begin{proof}
The right-hand side of~\eqref{eq:dynamic_car_robot} is a continuous function of $(\overline{x},u)$. By (\ref{eq:T10}), the set $\mc{T}_{1,0}$ is closed. Moreover, the sets $C_{K,0}$,$C_{K,1}$,$D_{K,0}$, and $D_{K,1}$ are also closed. This implies that $C_K$ and $D_K$ are also closed, as these sets are finite union of closed sets. The maps $F_K$ and $G_K$ are continuous by construction. Moreover, both the pure-pursuit controller $\kappa_1$ and the locally path-invariant controller $\kappa_0$ in (\ref{eq:kappa_0}) are continuous. Hence, the resulting closed-loop system $\mc{H}$ satisfies the hybrid basic conditions, which proves item~1.  

We prove the completeness of the maximal solutions to $\mc{H}$ by contradiction. Suppose there exists a maximal solution with the initial state $z(0,0)$ in the set $C\cup D$ that is not complete. Let $(T,J) = \sup \dom z$ and since by assumption $z$ is not complete $T + J < \infty$. From~\cite[Proposition 2.34]{San2021}, either item b or item c has to hold. {By Lemma~\ref{lemma:invariance}, the controller $\kappa_0$ assures finite-time stability of the desired path $\Gamma^\star$ everywhere in a neighborhood of $\Gamma^\star$. Hence, maximal solutions to the closed-loop system under the effect of $\kappa_0$ are bounded and complete. If the solution start from $C_0 \setminus \mc{T}_{1,0}$, it will eventually reach $\mc{T}_{1,0}$ under the control of $\kappa_1$. Hence the maximal solutions of the closed-loop system remain bounded and complete.} Under the control of $\kappa_1$, solutions reach a neighbourhood of $\Gamma^\star$, which is bounded. Hence, the solutions under $\kappa_1$ are bounded. Therefore, item b in~\cite[Proposition 2.34]{San2021} is ruled out. It can be shown that $G(D) \subset C \cup D$, hence by item 3 in~\cite[Proposition 2.34]{San2021}, item c is also ruled out. Therefore, every maximal solution to $\mc{H}$ from $C\cup D$ is complete. 

To show that every solution exhibits no more than two jumps, it should be noted that for every solution $z$ to $\mc{H}$, $z(0,0) \in C \cup D$, and only the following three cases are possible:

\begin{enumerate}
    \item~\label{list:1} By Assumption \ref{assumption:globalconvergence}, if $z(0,0) \in C_{K,1} \times \set{1}$, the solution $z$ reaches the set $D_{K,1}$ in finite hybrid time as the plant states reaches $\mc{T}_{1,0}$. After a jump, the solution $z$ remains flowing in $\left( C_{K,0}\setminus D_{K,0} \times \set{0}  \right)$ for all future hybrid time.
%     %{\blue I don't think we need to invoke the discussion of the set $\mc{E}_0$?}
%     %
    \item The solution exhibits the same behavior as in item~\ref{list:1}, when $z(0,0)\in D_{K,1} \times \set{1}$.
    %
    \item If $z(0,0) \in C_{K,0} \times \set{0}$, then the following two cases are possible. If $z(0,0) \in \mc{T}_{1,0} \times \set{0}$, then by Lemma~\ref{lemma:invariance}, the solution remains flowing in $\left(  C_{K,0} \cap {C}_0 \times \set{0}\right)$ for all future hybrid time. If $z(0,0) \in \left( C_{K,0} \setminus \mc{T}_{1,0} \times \set{0}  \right)$, then the solution may either flow forever or jump from $0$ to $1$ when it reaches the boundary of ${C}_0$. From there, the solution flows according to the logic explained in item~\ref{list:1}.
\end{enumerate}
Hence, every maximal solution has at most two jumps, which proves item~2.

The attractivity of $\mc{T}_{1,0}$ in finite time is established by Assumption~\ref{assumption:globalconvergence}, while Lemma~\ref{lemma:invariance} implies that the set $\mc{A}$ is finite-time stable for $\mc{H}$, thereby establishing global convergence. Finally, since the hybrid system satisfies the hybrid basic condition and $\mc{A}$ is compact, it follows from~\cite[Theorem 3.26]{San2021} that $\mc{A}$ is robust in the sense of~\cite[Definition 3.16]{San2021}, which proves item 3 and completes the proof.
\end{proof}}

\subsection{Algorithm Formulation}
The hybrid controller switches between two controllers, defined by $\kappa_q$, with the hybrid model governing the state of $q$. This switching leverages each controller’s strengths based on the robot’s state. $\kappa_1$ guides the robot to the desired path’s neighborhood, while $\kappa_0$ ensures path tracking and invariance within it. The path-following scheme, guaranteeing invariance and global convergence, is detailed in Algorithm \ref{alg:globallyinvariant}.
% The hybrid controller switches between two controllers defined by the function $\kappa_q$, where our hybrid model governs the state of $q$. 
% Switching between the two controllers allows us to take advantage of the different assets depending on the robot’s state. 
% The controller $\kappa_1$ will navigate the robot to the neighborhood of the desired path if the robot starts or slides outside it. The controller $\kappa_0$ will be used inside the neighborhood to track the path and make it invariant for the closed-loop system. 
% The path-following scheme that renders invariance and guarantees convergence from everywhere in the output space is formulated in Algorithm \ref{alg:globallyinvariant}. 
{\footnotesize \begin{algorithm}[htbp]
    \caption{\footnotesize Hybrid globally path-invariant algorithm}\label{alg:globallyinvariant}
    \hspace*{\algorithmicindent} \textbf{Input:} The initial state $\agx_{0}$ of the robot.
    \footnotesize
\begin{algorithmic}[1]
\State $q\leftarrow 0$.
\While{true}
\If {$\agx_{0}\in \mathcal{T}_{1, 0}$ or ($\agx_{0}\in \overline{\mc{U}}_{0}\backslash \mathcal{T}_{1, 0}$ and $q = 0$)}
\State $q \leftarrow 0$.
\While{$\agx(t)\in \overline{\mc{U}_0}$}
\State Apply $\kappa_{0}$ to track $\mathcal{C}$.
\EndWhile
\Else
\State $q\leftarrow 1$.
\State Compute an auxiliary collision-free trajectory $x'$ connecting $x_0$ and $X_{f}$ using motion planner.
\While{$\agx(t)\notin \mathcal{T}_{1, 0}$}
\State Apply $\kappa_1$ to track $x'$.
\EndWhile
\EndIf
\State $\agx_{0}\leftarrow \agx(t)$.
\EndWhile
\end{algorithmic}
\end{algorithm}}
% \subsection{Main Result}

%%%%%% Proof excluded from the camera ready version
{
% \begin{proof}
% Since the right-hand side of~\eqref{eq:generic_left_invariant_system}, namely, $\dot g = g\xi(u)$ is a continuous function of $(g,u)$. By construction, the set $\mc{T}_{1,0}$ is closed. Moreover, the sets $C_{K,0}$,$C_{K,1}$,$D_{K,0}$, and $D_{K,1}$ are also closed. This implies that $C_K$ and $D_K$ are also closed, as these sets are finite union of closed sets. The maps $F_K$ and $G_K$ are continuous by construction. The open-loop controller $\kappa_1$ is a constant function, and hence continuous. Moreover, by Definition~\ref{def:kinematic_family_circle} each $f\in \mc{F}_k$ is continuously differentiable, which implies $\kappa_0$ is also continuous. Hence, the resulting closed-loop system $\mc{H}$ satisfies the hybrid basic conditions, which proves item~1.  

% We prove completeness of the maximal solutions to $\mc{H}$ by contradiction. Suppose there exists a maximal solution with the initial state $x(0,0)$ in the set $C\cup D$ that is not complete. From~\cite[Proposition 2.34]{San2021}, either item b or item c has to hold. By Lemma~\ref{lemm:asymptotic_stability_class}, each controller $\kappa_0$ assures asymptotic stability of the point $e\in\ms{G}$ with the basin of attraction $\mc{B}_f$ containing $\mc{U}_0$. Hence the maximal solutions of the closed-loop system remain bounded and complete. {By Assumption~\ref{ass:open-loop}, under the effect of $\kappa_1$, solutions reach a neighbourhood of $e$, which is bounded. Hence, the solutions under $\kappa_1$ are bounded.} Therefore, item b in~\cite[Proposition 2.34]{San2021} is ruled out. It can be shown that $G(D) \subset C \cup D$, hence by item 3 in~\cite[Proposition 2.34]{San2021}, item c is also ruled out. Therefore, every maximal solution to $\mc{H}$ from $C\cup D$ is complete. 

% To show that every solution exhibits no more than two jumps, it should be noted that for every solution $x$ to $\mc{H}$, $x(0,0) \in C \cup D$, and only the following three cases are possible:

% \begin{enumerate}
%     \item~\label{list:1} By Lemma~\ref{lem:open-loop-finte-time}, if $x(0,0) \in C_{K,1} \times \set{1}$, the solution $x$ reaches the set $D_{K,1}$ in finite hybrid time as the plant states reaches $\mc{T}_{1,0}$. After a jump, the solution $x$ remains flowing in $\left( C_{K,0}\setminus D_{K,0} \times \set{0}  \right)$ for all future hybrid time.
%     %{\blue I don't think we need to invoke the discussion of the set $\mc{E}_0$?}
%     %
%     \item The solution exhibits the same behavior as in item~\ref{list:1}, when $x(0,0)\in D_{K,1} \times \set{1}$.
%     %
%     \item If $x(0,0) \in C_{K,0} \times \set{0}$, then the following two cases are possible. If $x(0,0) \in \mc{T}_{1,0} \times \set{0}$, then by Lemma~\ref{lemm:asymptotic_stability_class}, the solution remains flowing in $\left(  C_{K,0} \cup \mc{U}_0 \times \set{0}\right)$ for all future hybrid time. If $x(0,0) \in \left( C_{K,0} \setminus \mc{T}_{1,0} \times \set{0}  \right)$, then the solution may either flow forever or reach jump from $0$ to $1$ when it reaches the boundary of $\mc{U}_0$. From there, the solution flows according to the logic explained in item~\ref{list:1}.
% \end{enumerate}
% Hence, every maximal solution has at most two jumps, which proves item~2.

% The attractivity of $\mc{T}_{1,0}$ in finite time is established by Lemma~\ref{lem:open-loop-finte-time}, and Lemma~\ref{lemm:asymptotic_stability_class} implies that the set $\mc{A}$ is asymptotically stable for $\mc{H}$. Finally, since the hybrid system satisfies the hybrid basic condition and $\mc{A}$ is compact, it follows from~\cite[Theorem 3.26]{San2021} that $\mc{A}$ is robust in the sense of Definition~\ref{def:robust-stability}, which proves~3 and completes the proof.
% \end{proof}
% }
}


\begin{table*}[t]
\centering
\fontsize{11pt}{11pt}\selectfont
\begin{tabular}{lllllllllllll}
\toprule
\multicolumn{1}{c}{\textbf{task}} & \multicolumn{2}{c}{\textbf{Mir}} & \multicolumn{2}{c}{\textbf{Lai}} & \multicolumn{2}{c}{\textbf{Ziegen.}} & \multicolumn{2}{c}{\textbf{Cao}} & \multicolumn{2}{c}{\textbf{Alva-Man.}} & \multicolumn{1}{c}{\textbf{avg.}} & \textbf{\begin{tabular}[c]{@{}l@{}}avg.\\ rank\end{tabular}} \\
\multicolumn{1}{c}{\textbf{metrics}} & \multicolumn{1}{c}{\textbf{cor.}} & \multicolumn{1}{c}{\textbf{p-v.}} & \multicolumn{1}{c}{\textbf{cor.}} & \multicolumn{1}{c}{\textbf{p-v.}} & \multicolumn{1}{c}{\textbf{cor.}} & \multicolumn{1}{c}{\textbf{p-v.}} & \multicolumn{1}{c}{\textbf{cor.}} & \multicolumn{1}{c}{\textbf{p-v.}} & \multicolumn{1}{c}{\textbf{cor.}} & \multicolumn{1}{c}{\textbf{p-v.}} &  &  \\ \midrule
\textbf{S-Bleu} & 0.50 & 0.0 & 0.47 & 0.0 & 0.59 & 0.0 & 0.58 & 0.0 & 0.68 & 0.0 & 0.57 & 5.8 \\
\textbf{R-Bleu} & -- & -- & 0.27 & 0.0 & 0.30 & 0.0 & -- & -- & -- & -- & - &  \\
\textbf{S-Meteor} & 0.49 & 0.0 & 0.48 & 0.0 & 0.61 & 0.0 & 0.57 & 0.0 & 0.64 & 0.0 & 0.56 & 6.1 \\
\textbf{R-Meteor} & -- & -- & 0.34 & 0.0 & 0.26 & 0.0 & -- & -- & -- & -- & - &  \\
\textbf{S-Bertscore} & \textbf{0.53} & 0.0 & {\ul 0.80} & 0.0 & \textbf{0.70} & 0.0 & {\ul 0.66} & 0.0 & {\ul0.78} & 0.0 & \textbf{0.69} & \textbf{1.7} \\
\textbf{R-Bertscore} & -- & -- & 0.51 & 0.0 & 0.38 & 0.0 & -- & -- & -- & -- & - &  \\
\textbf{S-Bleurt} & {\ul 0.52} & 0.0 & {\ul 0.80} & 0.0 & 0.60 & 0.0 & \textbf{0.70} & 0.0 & \textbf{0.80} & 0.0 & {\ul 0.68} & {\ul 2.3} \\
\textbf{R-Bleurt} & -- & -- & 0.59 & 0.0 & -0.05 & 0.13 & -- & -- & -- & -- & - &  \\
\textbf{S-Cosine} & 0.51 & 0.0 & 0.69 & 0.0 & {\ul 0.62} & 0.0 & 0.61 & 0.0 & 0.65 & 0.0 & 0.62 & 4.4 \\
\textbf{R-Cosine} & -- & -- & 0.40 & 0.0 & 0.29 & 0.0 & -- & -- & -- & -- & - & \\ \midrule
\textbf{QuestEval} & 0.23 & 0.0 & 0.25 & 0.0 & 0.49 & 0.0 & 0.47 & 0.0 & 0.62 & 0.0 & 0.41 & 9.0 \\
\textbf{LLaMa3} & 0.36 & 0.0 & \textbf{0.84} & 0.0 & {\ul{0.62}} & 0.0 & 0.61 & 0.0 &  0.76 & 0.0 & 0.64 & 3.6 \\
\textbf{our (3b)} & 0.49 & 0.0 & 0.73 & 0.0 & 0.54 & 0.0 & 0.53 & 0.0 & 0.7 & 0.0 & 0.60 & 5.8 \\
\textbf{our (8b)} & 0.48 & 0.0 & 0.73 & 0.0 & 0.52 & 0.0 & 0.53 & 0.0 & 0.7 & 0.0 & 0.59 & 6.3 \\  \bottomrule
\end{tabular}
\caption{Pearson correlation on human evaluation on system output. `R-': reference-based. `S-': source-based.}
\label{tab:sys}
\end{table*}



\begin{table}%[]
\centering
\fontsize{11pt}{11pt}\selectfont
\begin{tabular}{llllll}
\toprule
\multicolumn{1}{c}{\textbf{task}} & \multicolumn{1}{c}{\textbf{Lai}} & \multicolumn{1}{c}{\textbf{Zei.}} & \multicolumn{1}{c}{\textbf{Scia.}} & \textbf{} & \textbf{} \\ 
\multicolumn{1}{c}{\textbf{metrics}} & \multicolumn{1}{c}{\textbf{cor.}} & \multicolumn{1}{c}{\textbf{cor.}} & \multicolumn{1}{c}{\textbf{cor.}} & \textbf{avg.} & \textbf{\begin{tabular}[c]{@{}l@{}}avg.\\ rank\end{tabular}} \\ \midrule
\textbf{S-Bleu} & 0.40 & 0.40 & 0.19* & 0.33 & 7.67 \\
\textbf{S-Meteor} & 0.41 & 0.42 & 0.16* & 0.33 & 7.33 \\
\textbf{S-BertS.} & {\ul0.58} & 0.47 & 0.31 & 0.45 & 3.67 \\
\textbf{S-Bleurt} & 0.45 & {\ul 0.54} & {\ul 0.37} & 0.45 & {\ul 3.33} \\
\textbf{S-Cosine} & 0.56 & 0.52 & 0.3 & {\ul 0.46} & {\ul 3.33} \\ \midrule
\textbf{QuestE.} & 0.27 & 0.35 & 0.06* & 0.23 & 9.00 \\
\textbf{LlaMA3} & \textbf{0.6} & \textbf{0.67} & \textbf{0.51} & \textbf{0.59} & \textbf{1.0} \\
\textbf{Our (3b)} & 0.51 & 0.49 & 0.23* & 0.39 & 4.83 \\
\textbf{Our (8b)} & 0.52 & 0.49 & 0.22* & 0.43 & 4.83 \\ \bottomrule
\end{tabular}
\caption{Pearson correlation on human ratings on reference output. *not significant; we cannot reject the null hypothesis of zero correlation}
\label{tab:ref}
\end{table}


\begin{table*}%[]
\centering
\fontsize{11pt}{11pt}\selectfont
\begin{tabular}{lllllllll}
\toprule
\textbf{task} & \multicolumn{1}{c}{\textbf{ALL}} & \multicolumn{1}{c}{\textbf{sentiment}} & \multicolumn{1}{c}{\textbf{detoxify}} & \multicolumn{1}{c}{\textbf{catchy}} & \multicolumn{1}{c}{\textbf{polite}} & \multicolumn{1}{c}{\textbf{persuasive}} & \multicolumn{1}{c}{\textbf{formal}} & \textbf{\begin{tabular}[c]{@{}l@{}}avg. \\ rank\end{tabular}} \\
\textbf{metrics} & \multicolumn{1}{c}{\textbf{cor.}} & \multicolumn{1}{c}{\textbf{cor.}} & \multicolumn{1}{c}{\textbf{cor.}} & \multicolumn{1}{c}{\textbf{cor.}} & \multicolumn{1}{c}{\textbf{cor.}} & \multicolumn{1}{c}{\textbf{cor.}} & \multicolumn{1}{c}{\textbf{cor.}} &  \\ \midrule
\textbf{S-Bleu} & -0.17 & -0.82 & -0.45 & -0.12* & -0.1* & -0.05 & -0.21 & 8.42 \\
\textbf{R-Bleu} & - & -0.5 & -0.45 &  &  &  &  &  \\
\textbf{S-Meteor} & -0.07* & -0.55 & -0.4 & -0.01* & 0.1* & -0.16 & -0.04* & 7.67 \\
\textbf{R-Meteor} & - & -0.17* & -0.39 & - & - & - & - & - \\
\textbf{S-BertScore} & 0.11 & -0.38 & -0.07* & -0.17* & 0.28 & 0.12 & 0.25 & 6.0 \\
\textbf{R-BertScore} & - & -0.02* & -0.21* & - & - & - & - & - \\
\textbf{S-Bleurt} & 0.29 & 0.05* & 0.45 & 0.06* & 0.29 & 0.23 & 0.46 & 4.2 \\
\textbf{R-Bleurt} & - &  0.21 & 0.38 & - & - & - & - & - \\
\textbf{S-Cosine} & 0.01* & -0.5 & -0.13* & -0.19* & 0.05* & -0.05* & 0.15* & 7.42 \\
\textbf{R-Cosine} & - & -0.11* & -0.16* & - & - & - & - & - \\ \midrule
\textbf{QuestEval} & 0.21 & {\ul{0.29}} & 0.23 & 0.37 & 0.19* & 0.35 & 0.14* & 4.67 \\
\textbf{LlaMA3} & \textbf{0.82} & \textbf{0.80} & \textbf{0.72} & \textbf{0.84} & \textbf{0.84} & \textbf{0.90} & \textbf{0.88} & \textbf{1.00} \\
\textbf{Our (3b)} & 0.47 & -0.11* & 0.37 & 0.61 & 0.53 & 0.54 & 0.66 & 3.5 \\
\textbf{Our (8b)} & {\ul{0.57}} & 0.09* & {\ul 0.49} & {\ul 0.72} & {\ul 0.64} & {\ul 0.62} & {\ul 0.67} & {\ul 2.17} \\ \bottomrule
\end{tabular}
\caption{Pearson correlation on human ratings on our constructed test set. 'R-': reference-based. 'S-': source-based. *not significant; we cannot reject the null hypothesis of zero correlation}
\label{tab:con}
\end{table*}

\section{Results}
We benchmark the different metrics on the different datasets using correlation to human judgement. For content preservation, we show results split on data with system output, reference output and our constructed test set: we show that the data source for evaluation leads to different conclusions on the metrics. In addition, we examine whether the metrics can rank style transfer systems similar to humans. On style strength, we likewise show correlations between human judgment and zero-shot evaluation approaches. When applicable, we summarize results by reporting the average correlation. And the average ranking of the metric per dataset (by ranking which metric obtains the highest correlation to human judgement per dataset). 

\subsection{Content preservation}
\paragraph{How do data sources affect the conclusion on best metric?}
The conclusions about the metrics' performance change radically depending on whether we use system output data, reference output, or our constructed test set. Ideally, a good metric correlates highly with humans on any data source. Ideally, for meta-evaluation, a metric should correlate consistently across all data sources, but the following shows that the correlations indicate different things, and the conclusion on the best metric should be drawn carefully.

Looking at the metrics correlations with humans on the data source with system output (Table~\ref{tab:sys}), we see a relatively high correlation for many of the metrics on many tasks. The overall best metrics are S-BertScore and S-BLEURT (avg+avg rank). We see no notable difference in our method of using the 3B or 8B model as the backbone.

Examining the average correlations based on data with reference output (Table~\ref{tab:ref}), now the zero-shoot prompting with LlaMA3 70B is the best-performing approach ($0.59$ avg). Tied for second place are source-based cosine embedding ($0.46$ avg), BLEURT ($0.45$ avg) and BertScore ($0.45$ avg). Our method follows on a 5. place: here, the 8b version (($0.43$ avg)) shows a bit stronger results than 3b ($0.39$ avg). The fact that the conclusions change, whether looking at reference or system output, confirms the observations made by \citet{scialom-etal-2021-questeval} on simplicity transfer.   

Now consider the results on our test set (Table~\ref{tab:con}): Several metrics show low or no correlation; we even see a significantly negative correlation for some metrics on ALL (BLEU) and for specific subparts of our test set for BLEU, Meteor, BertScore, Cosine. On the other end, LlaMA3 70B is again performing best, showing strong results ($0.82$ in ALL). The runner-up is now our 8B method, with a gap to the 3B version ($0.57$ vs $0.47$ in ALL). Note our method still shows zero correlation for the sentiment task. After, ranks BLEURT ($0.29$), QuestEval ($0.21$), BertScore ($0.11$), Cosine ($0.01$).  

On our test set, we find that some metrics that correlate relatively well on the other datasets, now exhibit low correlation. Hence, with our test set, we can now support the logical reasoning with data evidence: Evaluation of content preservation for style transfer needs to take the style shift into account. This conclusion could not be drawn using the existing data sources: We hypothesise that for the data with system-based output, successful output happens to be very similar to the source sentence and vice versa, and reference-based output might not contain server mistakes as they are gold references. Thus, none of the existing data sources tests the limits of the metrics.  


\paragraph{How do reference-based metrics compare to source-based ones?} Reference-based metrics show a lower correlation than the source-based counterpart for all metrics on both datasets with ratings on references (Table~\ref{tab:sys}). As discussed previously, reference-based metrics for style transfer have the drawback that many different good solutions on a rewrite might exist and not only one similar to a reference.


\paragraph{How well can the metrics rank the performance of style transfer methods?}
We compare the metrics' ability to judge the best style transfer methods w.r.t. the human annotations: Several of the data sources contain samples from different style transfer systems. In order to use metrics to assess the quality of the style transfer system, metrics should correctly find the best-performing system. Hence, we evaluate whether the metrics for content preservation provide the same system ranking as human evaluators. We take the mean of the score for every output on each system and the mean of the human annotations; we compare the systems using the Kendall's Tau correlation. 

We find only the evaluation using the dataset Mir, Lai, and Ziegen to result in significant correlations, probably because of sparsity in a number of system tests (App.~\ref{app:dataset}). Our method (8b) is the only metric providing a perfect ranking of the style transfer system on the Lai data, and Llama3 70B the only one on the Ziegen data. Results in App.~\ref{app:results}. 


\subsection{Style strength results}
%Evaluating style strengths is a challenging task. 
Llama3 70B shows better overall results than our method. However, our method scores higher than Llama3 70B on 2 out of 6 datasets, but it also exhibits zero correlation on one task (Table~\ref{tab:styleresults}).%More work i s needed on evaluating style strengths. 
 
\begin{table}%[]
\fontsize{11pt}{11pt}\selectfont
\begin{tabular}{lccc}
\toprule
\multicolumn{1}{c}{\textbf{}} & \textbf{LlaMA3} & \textbf{Our (3b)} & \textbf{Our (8b)} \\ \midrule
\textbf{Mir} & 0.46 & 0.54 & \textbf{0.57} \\
\textbf{Lai} & \textbf{0.57} & 0.18 & 0.19 \\
\textbf{Ziegen.} & 0.25 & 0.27 & \textbf{0.32} \\
\textbf{Alva-M.} & \textbf{0.59} & 0.03* & 0.02* \\
\textbf{Scialom} & \textbf{0.62} & 0.45 & 0.44 \\
\textbf{\begin{tabular}[c]{@{}l@{}}Our Test\end{tabular}} & \textbf{0.63} & 0.46 & 0.48 \\ \bottomrule
\end{tabular}
\caption{Style strength: Pearson correlation to human ratings. *not significant; we cannot reject the null hypothesis of zero corelation}
\label{tab:styleresults}
\end{table}

\subsection{Ablation}
We conduct several runs of the methods using LLMs with variations in instructions/prompts (App.~\ref{app:method}). We observe that the lower the correlation on a task, the higher the variation between the different runs. For our method, we only observe low variance between the runs.
None of the variations leads to different conclusions of the meta-evaluation. Results in App.~\ref{app:results}.
\section{Conclusion}
In this work, we propose a simple yet effective approach, called SMILE, for graph few-shot learning with fewer tasks. Specifically, we introduce a novel dual-level mixup strategy, including within-task and across-task mixup, for enriching the diversity of nodes within each task and the diversity of tasks. Also, we incorporate the degree-based prior information to learn expressive node embeddings. Theoretically, we prove that SMILE effectively enhances the model's generalization performance. Empirically, we conduct extensive experiments on multiple benchmarks and the results suggest that SMILE significantly outperforms other baselines, including both in-domain and cross-domain few-shot settings.
\subsection{Lloyd-Max Algorithm}
\label{subsec:Lloyd-Max}
For a given quantization bitwidth $B$ and an operand $\bm{X}$, the Lloyd-Max algorithm finds $2^B$ quantization levels $\{\hat{x}_i\}_{i=1}^{2^B}$ such that quantizing $\bm{X}$ by rounding each scalar in $\bm{X}$ to the nearest quantization level minimizes the quantization MSE. 

The algorithm starts with an initial guess of quantization levels and then iteratively computes quantization thresholds $\{\tau_i\}_{i=1}^{2^B-1}$ and updates quantization levels $\{\hat{x}_i\}_{i=1}^{2^B}$. Specifically, at iteration $n$, thresholds are set to the midpoints of the previous iteration's levels:
\begin{align*}
    \tau_i^{(n)}=\frac{\hat{x}_i^{(n-1)}+\hat{x}_{i+1}^{(n-1)}}2 \text{ for } i=1\ldots 2^B-1
\end{align*}
Subsequently, the quantization levels are re-computed as conditional means of the data regions defined by the new thresholds:
\begin{align*}
    \hat{x}_i^{(n)}=\mathbb{E}\left[ \bm{X} \big| \bm{X}\in [\tau_{i-1}^{(n)},\tau_i^{(n)}] \right] \text{ for } i=1\ldots 2^B
\end{align*}
where to satisfy boundary conditions we have $\tau_0=-\infty$ and $\tau_{2^B}=\infty$. The algorithm iterates the above steps until convergence.

Figure \ref{fig:lm_quant} compares the quantization levels of a $7$-bit floating point (E3M3) quantizer (left) to a $7$-bit Lloyd-Max quantizer (right) when quantizing a layer of weights from the GPT3-126M model at a per-tensor granularity. As shown, the Lloyd-Max quantizer achieves substantially lower quantization MSE. Further, Table \ref{tab:FP7_vs_LM7} shows the superior perplexity achieved by Lloyd-Max quantizers for bitwidths of $7$, $6$ and $5$. The difference between the quantizers is clear at 5 bits, where per-tensor FP quantization incurs a drastic and unacceptable increase in perplexity, while Lloyd-Max quantization incurs a much smaller increase. Nevertheless, we note that even the optimal Lloyd-Max quantizer incurs a notable ($\sim 1.5$) increase in perplexity due to the coarse granularity of quantization. 

\begin{figure}[h]
  \centering
  \includegraphics[width=0.7\linewidth]{sections/figures/LM7_FP7.pdf}
  \caption{\small Quantization levels and the corresponding quantization MSE of Floating Point (left) vs Lloyd-Max (right) Quantizers for a layer of weights in the GPT3-126M model.}
  \label{fig:lm_quant}
\end{figure}

\begin{table}[h]\scriptsize
\begin{center}
\caption{\label{tab:FP7_vs_LM7} \small Comparing perplexity (lower is better) achieved by floating point quantizers and Lloyd-Max quantizers on a GPT3-126M model for the Wikitext-103 dataset.}
\begin{tabular}{c|cc|c}
\hline
 \multirow{2}{*}{\textbf{Bitwidth}} & \multicolumn{2}{|c|}{\textbf{Floating-Point Quantizer}} & \textbf{Lloyd-Max Quantizer} \\
 & Best Format & Wikitext-103 Perplexity & Wikitext-103 Perplexity \\
\hline
7 & E3M3 & 18.32 & 18.27 \\
6 & E3M2 & 19.07 & 18.51 \\
5 & E4M0 & 43.89 & 19.71 \\
\hline
\end{tabular}
\end{center}
\end{table}

\subsection{Proof of Local Optimality of LO-BCQ}
\label{subsec:lobcq_opt_proof}
For a given block $\bm{b}_j$, the quantization MSE during LO-BCQ can be empirically evaluated as $\frac{1}{L_b}\lVert \bm{b}_j- \bm{\hat{b}}_j\rVert^2_2$ where $\bm{\hat{b}}_j$ is computed from equation (\ref{eq:clustered_quantization_definition}) as $C_{f(\bm{b}_j)}(\bm{b}_j)$. Further, for a given block cluster $\mathcal{B}_i$, we compute the quantization MSE as $\frac{1}{|\mathcal{B}_{i}|}\sum_{\bm{b} \in \mathcal{B}_{i}} \frac{1}{L_b}\lVert \bm{b}- C_i^{(n)}(\bm{b})\rVert^2_2$. Therefore, at the end of iteration $n$, we evaluate the overall quantization MSE $J^{(n)}$ for a given operand $\bm{X}$ composed of $N_c$ block clusters as:
\begin{align*}
    \label{eq:mse_iter_n}
    J^{(n)} = \frac{1}{N_c} \sum_{i=1}^{N_c} \frac{1}{|\mathcal{B}_{i}^{(n)}|}\sum_{\bm{v} \in \mathcal{B}_{i}^{(n)}} \frac{1}{L_b}\lVert \bm{b}- B_i^{(n)}(\bm{b})\rVert^2_2
\end{align*}

At the end of iteration $n$, the codebooks are updated from $\mathcal{C}^{(n-1)}$ to $\mathcal{C}^{(n)}$. However, the mapping of a given vector $\bm{b}_j$ to quantizers $\mathcal{C}^{(n)}$ remains as  $f^{(n)}(\bm{b}_j)$. At the next iteration, during the vector clustering step, $f^{(n+1)}(\bm{b}_j)$ finds new mapping of $\bm{b}_j$ to updated codebooks $\mathcal{C}^{(n)}$ such that the quantization MSE over the candidate codebooks is minimized. Therefore, we obtain the following result for $\bm{b}_j$:
\begin{align*}
\frac{1}{L_b}\lVert \bm{b}_j - C_{f^{(n+1)}(\bm{b}_j)}^{(n)}(\bm{b}_j)\rVert^2_2 \le \frac{1}{L_b}\lVert \bm{b}_j - C_{f^{(n)}(\bm{b}_j)}^{(n)}(\bm{b}_j)\rVert^2_2
\end{align*}

That is, quantizing $\bm{b}_j$ at the end of the block clustering step of iteration $n+1$ results in lower quantization MSE compared to quantizing at the end of iteration $n$. Since this is true for all $\bm{b} \in \bm{X}$, we assert the following:
\begin{equation}
\begin{split}
\label{eq:mse_ineq_1}
    \tilde{J}^{(n+1)} &= \frac{1}{N_c} \sum_{i=1}^{N_c} \frac{1}{|\mathcal{B}_{i}^{(n+1)}|}\sum_{\bm{b} \in \mathcal{B}_{i}^{(n+1)}} \frac{1}{L_b}\lVert \bm{b} - C_i^{(n)}(b)\rVert^2_2 \le J^{(n)}
\end{split}
\end{equation}
where $\tilde{J}^{(n+1)}$ is the the quantization MSE after the vector clustering step at iteration $n+1$.

Next, during the codebook update step (\ref{eq:quantizers_update}) at iteration $n+1$, the per-cluster codebooks $\mathcal{C}^{(n)}$ are updated to $\mathcal{C}^{(n+1)}$ by invoking the Lloyd-Max algorithm \citep{Lloyd}. We know that for any given value distribution, the Lloyd-Max algorithm minimizes the quantization MSE. Therefore, for a given vector cluster $\mathcal{B}_i$ we obtain the following result:

\begin{equation}
    \frac{1}{|\mathcal{B}_{i}^{(n+1)}|}\sum_{\bm{b} \in \mathcal{B}_{i}^{(n+1)}} \frac{1}{L_b}\lVert \bm{b}- C_i^{(n+1)}(\bm{b})\rVert^2_2 \le \frac{1}{|\mathcal{B}_{i}^{(n+1)}|}\sum_{\bm{b} \in \mathcal{B}_{i}^{(n+1)}} \frac{1}{L_b}\lVert \bm{b}- C_i^{(n)}(\bm{b})\rVert^2_2
\end{equation}

The above equation states that quantizing the given block cluster $\mathcal{B}_i$ after updating the associated codebook from $C_i^{(n)}$ to $C_i^{(n+1)}$ results in lower quantization MSE. Since this is true for all the block clusters, we derive the following result: 
\begin{equation}
\begin{split}
\label{eq:mse_ineq_2}
     J^{(n+1)} &= \frac{1}{N_c} \sum_{i=1}^{N_c} \frac{1}{|\mathcal{B}_{i}^{(n+1)}|}\sum_{\bm{b} \in \mathcal{B}_{i}^{(n+1)}} \frac{1}{L_b}\lVert \bm{b}- C_i^{(n+1)}(\bm{b})\rVert^2_2  \le \tilde{J}^{(n+1)}   
\end{split}
\end{equation}

Following (\ref{eq:mse_ineq_1}) and (\ref{eq:mse_ineq_2}), we find that the quantization MSE is non-increasing for each iteration, that is, $J^{(1)} \ge J^{(2)} \ge J^{(3)} \ge \ldots \ge J^{(M)}$ where $M$ is the maximum number of iterations. 
%Therefore, we can say that if the algorithm converges, then it must be that it has converged to a local minimum. 
\hfill $\blacksquare$


\begin{figure}
    \begin{center}
    \includegraphics[width=0.5\textwidth]{sections//figures/mse_vs_iter.pdf}
    \end{center}
    \caption{\small NMSE vs iterations during LO-BCQ compared to other block quantization proposals}
    \label{fig:nmse_vs_iter}
\end{figure}

Figure \ref{fig:nmse_vs_iter} shows the empirical convergence of LO-BCQ across several block lengths and number of codebooks. Also, the MSE achieved by LO-BCQ is compared to baselines such as MXFP and VSQ. As shown, LO-BCQ converges to a lower MSE than the baselines. Further, we achieve better convergence for larger number of codebooks ($N_c$) and for a smaller block length ($L_b$), both of which increase the bitwidth of BCQ (see Eq \ref{eq:bitwidth_bcq}).


\subsection{Additional Accuracy Results}
%Table \ref{tab:lobcq_config} lists the various LOBCQ configurations and their corresponding bitwidths.
\begin{table}
\setlength{\tabcolsep}{4.75pt}
\begin{center}
\caption{\label{tab:lobcq_config} Various LO-BCQ configurations and their bitwidths.}
\begin{tabular}{|c||c|c|c|c||c|c||c|} 
\hline
 & \multicolumn{4}{|c||}{$L_b=8$} & \multicolumn{2}{|c||}{$L_b=4$} & $L_b=2$ \\
 \hline
 \backslashbox{$L_A$\kern-1em}{\kern-1em$N_c$} & 2 & 4 & 8 & 16 & 2 & 4 & 2 \\
 \hline
 64 & 4.25 & 4.375 & 4.5 & 4.625 & 4.375 & 4.625 & 4.625\\
 \hline
 32 & 4.375 & 4.5 & 4.625& 4.75 & 4.5 & 4.75 & 4.75 \\
 \hline
 16 & 4.625 & 4.75& 4.875 & 5 & 4.75 & 5 & 5 \\
 \hline
\end{tabular}
\end{center}
\end{table}

%\subsection{Perplexity achieved by various LO-BCQ configurations on Wikitext-103 dataset}

\begin{table} \centering
\begin{tabular}{|c||c|c|c|c||c|c||c|} 
\hline
 $L_b \rightarrow$& \multicolumn{4}{c||}{8} & \multicolumn{2}{c||}{4} & 2\\
 \hline
 \backslashbox{$L_A$\kern-1em}{\kern-1em$N_c$} & 2 & 4 & 8 & 16 & 2 & 4 & 2  \\
 %$N_c \rightarrow$ & 2 & 4 & 8 & 16 & 2 & 4 & 2 \\
 \hline
 \hline
 \multicolumn{8}{c}{GPT3-1.3B (FP32 PPL = 9.98)} \\ 
 \hline
 \hline
 64 & 10.40 & 10.23 & 10.17 & 10.15 &  10.28 & 10.18 & 10.19 \\
 \hline
 32 & 10.25 & 10.20 & 10.15 & 10.12 &  10.23 & 10.17 & 10.17 \\
 \hline
 16 & 10.22 & 10.16 & 10.10 & 10.09 &  10.21 & 10.14 & 10.16 \\
 \hline
  \hline
 \multicolumn{8}{c}{GPT3-8B (FP32 PPL = 7.38)} \\ 
 \hline
 \hline
 64 & 7.61 & 7.52 & 7.48 &  7.47 &  7.55 &  7.49 & 7.50 \\
 \hline
 32 & 7.52 & 7.50 & 7.46 &  7.45 &  7.52 &  7.48 & 7.48  \\
 \hline
 16 & 7.51 & 7.48 & 7.44 &  7.44 &  7.51 &  7.49 & 7.47  \\
 \hline
\end{tabular}
\caption{\label{tab:ppl_gpt3_abalation} Wikitext-103 perplexity across GPT3-1.3B and 8B models.}
\end{table}

\begin{table} \centering
\begin{tabular}{|c||c|c|c|c||} 
\hline
 $L_b \rightarrow$& \multicolumn{4}{c||}{8}\\
 \hline
 \backslashbox{$L_A$\kern-1em}{\kern-1em$N_c$} & 2 & 4 & 8 & 16 \\
 %$N_c \rightarrow$ & 2 & 4 & 8 & 16 & 2 & 4 & 2 \\
 \hline
 \hline
 \multicolumn{5}{|c|}{Llama2-7B (FP32 PPL = 5.06)} \\ 
 \hline
 \hline
 64 & 5.31 & 5.26 & 5.19 & 5.18  \\
 \hline
 32 & 5.23 & 5.25 & 5.18 & 5.15  \\
 \hline
 16 & 5.23 & 5.19 & 5.16 & 5.14  \\
 \hline
 \multicolumn{5}{|c|}{Nemotron4-15B (FP32 PPL = 5.87)} \\ 
 \hline
 \hline
 64  & 6.3 & 6.20 & 6.13 & 6.08  \\
 \hline
 32  & 6.24 & 6.12 & 6.07 & 6.03  \\
 \hline
 16  & 6.12 & 6.14 & 6.04 & 6.02  \\
 \hline
 \multicolumn{5}{|c|}{Nemotron4-340B (FP32 PPL = 3.48)} \\ 
 \hline
 \hline
 64 & 3.67 & 3.62 & 3.60 & 3.59 \\
 \hline
 32 & 3.63 & 3.61 & 3.59 & 3.56 \\
 \hline
 16 & 3.61 & 3.58 & 3.57 & 3.55 \\
 \hline
\end{tabular}
\caption{\label{tab:ppl_llama7B_nemo15B} Wikitext-103 perplexity compared to FP32 baseline in Llama2-7B and Nemotron4-15B, 340B models}
\end{table}

%\subsection{Perplexity achieved by various LO-BCQ configurations on MMLU dataset}


\begin{table} \centering
\begin{tabular}{|c||c|c|c|c||c|c|c|c|} 
\hline
 $L_b \rightarrow$& \multicolumn{4}{c||}{8} & \multicolumn{4}{c||}{8}\\
 \hline
 \backslashbox{$L_A$\kern-1em}{\kern-1em$N_c$} & 2 & 4 & 8 & 16 & 2 & 4 & 8 & 16  \\
 %$N_c \rightarrow$ & 2 & 4 & 8 & 16 & 2 & 4 & 2 \\
 \hline
 \hline
 \multicolumn{5}{|c|}{Llama2-7B (FP32 Accuracy = 45.8\%)} & \multicolumn{4}{|c|}{Llama2-70B (FP32 Accuracy = 69.12\%)} \\ 
 \hline
 \hline
 64 & 43.9 & 43.4 & 43.9 & 44.9 & 68.07 & 68.27 & 68.17 & 68.75 \\
 \hline
 32 & 44.5 & 43.8 & 44.9 & 44.5 & 68.37 & 68.51 & 68.35 & 68.27  \\
 \hline
 16 & 43.9 & 42.7 & 44.9 & 45 & 68.12 & 68.77 & 68.31 & 68.59  \\
 \hline
 \hline
 \multicolumn{5}{|c|}{GPT3-22B (FP32 Accuracy = 38.75\%)} & \multicolumn{4}{|c|}{Nemotron4-15B (FP32 Accuracy = 64.3\%)} \\ 
 \hline
 \hline
 64 & 36.71 & 38.85 & 38.13 & 38.92 & 63.17 & 62.36 & 63.72 & 64.09 \\
 \hline
 32 & 37.95 & 38.69 & 39.45 & 38.34 & 64.05 & 62.30 & 63.8 & 64.33  \\
 \hline
 16 & 38.88 & 38.80 & 38.31 & 38.92 & 63.22 & 63.51 & 63.93 & 64.43  \\
 \hline
\end{tabular}
\caption{\label{tab:mmlu_abalation} Accuracy on MMLU dataset across GPT3-22B, Llama2-7B, 70B and Nemotron4-15B models.}
\end{table}


%\subsection{Perplexity achieved by various LO-BCQ configurations on LM evaluation harness}

\begin{table} \centering
\begin{tabular}{|c||c|c|c|c||c|c|c|c|} 
\hline
 $L_b \rightarrow$& \multicolumn{4}{c||}{8} & \multicolumn{4}{c||}{8}\\
 \hline
 \backslashbox{$L_A$\kern-1em}{\kern-1em$N_c$} & 2 & 4 & 8 & 16 & 2 & 4 & 8 & 16  \\
 %$N_c \rightarrow$ & 2 & 4 & 8 & 16 & 2 & 4 & 2 \\
 \hline
 \hline
 \multicolumn{5}{|c|}{Race (FP32 Accuracy = 37.51\%)} & \multicolumn{4}{|c|}{Boolq (FP32 Accuracy = 64.62\%)} \\ 
 \hline
 \hline
 64 & 36.94 & 37.13 & 36.27 & 37.13 & 63.73 & 62.26 & 63.49 & 63.36 \\
 \hline
 32 & 37.03 & 36.36 & 36.08 & 37.03 & 62.54 & 63.51 & 63.49 & 63.55  \\
 \hline
 16 & 37.03 & 37.03 & 36.46 & 37.03 & 61.1 & 63.79 & 63.58 & 63.33  \\
 \hline
 \hline
 \multicolumn{5}{|c|}{Winogrande (FP32 Accuracy = 58.01\%)} & \multicolumn{4}{|c|}{Piqa (FP32 Accuracy = 74.21\%)} \\ 
 \hline
 \hline
 64 & 58.17 & 57.22 & 57.85 & 58.33 & 73.01 & 73.07 & 73.07 & 72.80 \\
 \hline
 32 & 59.12 & 58.09 & 57.85 & 58.41 & 73.01 & 73.94 & 72.74 & 73.18  \\
 \hline
 16 & 57.93 & 58.88 & 57.93 & 58.56 & 73.94 & 72.80 & 73.01 & 73.94  \\
 \hline
\end{tabular}
\caption{\label{tab:mmlu_abalation} Accuracy on LM evaluation harness tasks on GPT3-1.3B model.}
\end{table}

\begin{table} \centering
\begin{tabular}{|c||c|c|c|c||c|c|c|c|} 
\hline
 $L_b \rightarrow$& \multicolumn{4}{c||}{8} & \multicolumn{4}{c||}{8}\\
 \hline
 \backslashbox{$L_A$\kern-1em}{\kern-1em$N_c$} & 2 & 4 & 8 & 16 & 2 & 4 & 8 & 16  \\
 %$N_c \rightarrow$ & 2 & 4 & 8 & 16 & 2 & 4 & 2 \\
 \hline
 \hline
 \multicolumn{5}{|c|}{Race (FP32 Accuracy = 41.34\%)} & \multicolumn{4}{|c|}{Boolq (FP32 Accuracy = 68.32\%)} \\ 
 \hline
 \hline
 64 & 40.48 & 40.10 & 39.43 & 39.90 & 69.20 & 68.41 & 69.45 & 68.56 \\
 \hline
 32 & 39.52 & 39.52 & 40.77 & 39.62 & 68.32 & 67.43 & 68.17 & 69.30  \\
 \hline
 16 & 39.81 & 39.71 & 39.90 & 40.38 & 68.10 & 66.33 & 69.51 & 69.42  \\
 \hline
 \hline
 \multicolumn{5}{|c|}{Winogrande (FP32 Accuracy = 67.88\%)} & \multicolumn{4}{|c|}{Piqa (FP32 Accuracy = 78.78\%)} \\ 
 \hline
 \hline
 64 & 66.85 & 66.61 & 67.72 & 67.88 & 77.31 & 77.42 & 77.75 & 77.64 \\
 \hline
 32 & 67.25 & 67.72 & 67.72 & 67.00 & 77.31 & 77.04 & 77.80 & 77.37  \\
 \hline
 16 & 68.11 & 68.90 & 67.88 & 67.48 & 77.37 & 78.13 & 78.13 & 77.69  \\
 \hline
\end{tabular}
\caption{\label{tab:mmlu_abalation} Accuracy on LM evaluation harness tasks on GPT3-8B model.}
\end{table}

\begin{table} \centering
\begin{tabular}{|c||c|c|c|c||c|c|c|c|} 
\hline
 $L_b \rightarrow$& \multicolumn{4}{c||}{8} & \multicolumn{4}{c||}{8}\\
 \hline
 \backslashbox{$L_A$\kern-1em}{\kern-1em$N_c$} & 2 & 4 & 8 & 16 & 2 & 4 & 8 & 16  \\
 %$N_c \rightarrow$ & 2 & 4 & 8 & 16 & 2 & 4 & 2 \\
 \hline
 \hline
 \multicolumn{5}{|c|}{Race (FP32 Accuracy = 40.67\%)} & \multicolumn{4}{|c|}{Boolq (FP32 Accuracy = 76.54\%)} \\ 
 \hline
 \hline
 64 & 40.48 & 40.10 & 39.43 & 39.90 & 75.41 & 75.11 & 77.09 & 75.66 \\
 \hline
 32 & 39.52 & 39.52 & 40.77 & 39.62 & 76.02 & 76.02 & 75.96 & 75.35  \\
 \hline
 16 & 39.81 & 39.71 & 39.90 & 40.38 & 75.05 & 73.82 & 75.72 & 76.09  \\
 \hline
 \hline
 \multicolumn{5}{|c|}{Winogrande (FP32 Accuracy = 70.64\%)} & \multicolumn{4}{|c|}{Piqa (FP32 Accuracy = 79.16\%)} \\ 
 \hline
 \hline
 64 & 69.14 & 70.17 & 70.17 & 70.56 & 78.24 & 79.00 & 78.62 & 78.73 \\
 \hline
 32 & 70.96 & 69.69 & 71.27 & 69.30 & 78.56 & 79.49 & 79.16 & 78.89  \\
 \hline
 16 & 71.03 & 69.53 & 69.69 & 70.40 & 78.13 & 79.16 & 79.00 & 79.00  \\
 \hline
\end{tabular}
\caption{\label{tab:mmlu_abalation} Accuracy on LM evaluation harness tasks on GPT3-22B model.}
\end{table}

\begin{table} \centering
\begin{tabular}{|c||c|c|c|c||c|c|c|c|} 
\hline
 $L_b \rightarrow$& \multicolumn{4}{c||}{8} & \multicolumn{4}{c||}{8}\\
 \hline
 \backslashbox{$L_A$\kern-1em}{\kern-1em$N_c$} & 2 & 4 & 8 & 16 & 2 & 4 & 8 & 16  \\
 %$N_c \rightarrow$ & 2 & 4 & 8 & 16 & 2 & 4 & 2 \\
 \hline
 \hline
 \multicolumn{5}{|c|}{Race (FP32 Accuracy = 44.4\%)} & \multicolumn{4}{|c|}{Boolq (FP32 Accuracy = 79.29\%)} \\ 
 \hline
 \hline
 64 & 42.49 & 42.51 & 42.58 & 43.45 & 77.58 & 77.37 & 77.43 & 78.1 \\
 \hline
 32 & 43.35 & 42.49 & 43.64 & 43.73 & 77.86 & 75.32 & 77.28 & 77.86  \\
 \hline
 16 & 44.21 & 44.21 & 43.64 & 42.97 & 78.65 & 77 & 76.94 & 77.98  \\
 \hline
 \hline
 \multicolumn{5}{|c|}{Winogrande (FP32 Accuracy = 69.38\%)} & \multicolumn{4}{|c|}{Piqa (FP32 Accuracy = 78.07\%)} \\ 
 \hline
 \hline
 64 & 68.9 & 68.43 & 69.77 & 68.19 & 77.09 & 76.82 & 77.09 & 77.86 \\
 \hline
 32 & 69.38 & 68.51 & 68.82 & 68.90 & 78.07 & 76.71 & 78.07 & 77.86  \\
 \hline
 16 & 69.53 & 67.09 & 69.38 & 68.90 & 77.37 & 77.8 & 77.91 & 77.69  \\
 \hline
\end{tabular}
\caption{\label{tab:mmlu_abalation} Accuracy on LM evaluation harness tasks on Llama2-7B model.}
\end{table}

\begin{table} \centering
\begin{tabular}{|c||c|c|c|c||c|c|c|c|} 
\hline
 $L_b \rightarrow$& \multicolumn{4}{c||}{8} & \multicolumn{4}{c||}{8}\\
 \hline
 \backslashbox{$L_A$\kern-1em}{\kern-1em$N_c$} & 2 & 4 & 8 & 16 & 2 & 4 & 8 & 16  \\
 %$N_c \rightarrow$ & 2 & 4 & 8 & 16 & 2 & 4 & 2 \\
 \hline
 \hline
 \multicolumn{5}{|c|}{Race (FP32 Accuracy = 48.8\%)} & \multicolumn{4}{|c|}{Boolq (FP32 Accuracy = 85.23\%)} \\ 
 \hline
 \hline
 64 & 49.00 & 49.00 & 49.28 & 48.71 & 82.82 & 84.28 & 84.03 & 84.25 \\
 \hline
 32 & 49.57 & 48.52 & 48.33 & 49.28 & 83.85 & 84.46 & 84.31 & 84.93  \\
 \hline
 16 & 49.85 & 49.09 & 49.28 & 48.99 & 85.11 & 84.46 & 84.61 & 83.94  \\
 \hline
 \hline
 \multicolumn{5}{|c|}{Winogrande (FP32 Accuracy = 79.95\%)} & \multicolumn{4}{|c|}{Piqa (FP32 Accuracy = 81.56\%)} \\ 
 \hline
 \hline
 64 & 78.77 & 78.45 & 78.37 & 79.16 & 81.45 & 80.69 & 81.45 & 81.5 \\
 \hline
 32 & 78.45 & 79.01 & 78.69 & 80.66 & 81.56 & 80.58 & 81.18 & 81.34  \\
 \hline
 16 & 79.95 & 79.56 & 79.79 & 79.72 & 81.28 & 81.66 & 81.28 & 80.96  \\
 \hline
\end{tabular}
\caption{\label{tab:mmlu_abalation} Accuracy on LM evaluation harness tasks on Llama2-70B model.}
\end{table}

%\section{MSE Studies}
%\textcolor{red}{TODO}


\subsection{Number Formats and Quantization Method}
\label{subsec:numFormats_quantMethod}
\subsubsection{Integer Format}
An $n$-bit signed integer (INT) is typically represented with a 2s-complement format \citep{yao2022zeroquant,xiao2023smoothquant,dai2021vsq}, where the most significant bit denotes the sign.

\subsubsection{Floating Point Format}
An $n$-bit signed floating point (FP) number $x$ comprises of a 1-bit sign ($x_{\mathrm{sign}}$), $B_m$-bit mantissa ($x_{\mathrm{mant}}$) and $B_e$-bit exponent ($x_{\mathrm{exp}}$) such that $B_m+B_e=n-1$. The associated constant exponent bias ($E_{\mathrm{bias}}$) is computed as $(2^{{B_e}-1}-1)$. We denote this format as $E_{B_e}M_{B_m}$.  

\subsubsection{Quantization Scheme}
\label{subsec:quant_method}
A quantization scheme dictates how a given unquantized tensor is converted to its quantized representation. We consider FP formats for the purpose of illustration. Given an unquantized tensor $\bm{X}$ and an FP format $E_{B_e}M_{B_m}$, we first, we compute the quantization scale factor $s_X$ that maps the maximum absolute value of $\bm{X}$ to the maximum quantization level of the $E_{B_e}M_{B_m}$ format as follows:
\begin{align}
\label{eq:sf}
    s_X = \frac{\mathrm{max}(|\bm{X}|)}{\mathrm{max}(E_{B_e}M_{B_m})}
\end{align}
In the above equation, $|\cdot|$ denotes the absolute value function.

Next, we scale $\bm{X}$ by $s_X$ and quantize it to $\hat{\bm{X}}$ by rounding it to the nearest quantization level of $E_{B_e}M_{B_m}$ as:

\begin{align}
\label{eq:tensor_quant}
    \hat{\bm{X}} = \text{round-to-nearest}\left(\frac{\bm{X}}{s_X}, E_{B_e}M_{B_m}\right)
\end{align}

We perform dynamic max-scaled quantization \citep{wu2020integer}, where the scale factor $s$ for activations is dynamically computed during runtime.

\subsection{Vector Scaled Quantization}
\begin{wrapfigure}{r}{0.35\linewidth}
  \centering
  \includegraphics[width=\linewidth]{sections/figures/vsquant.jpg}
  \caption{\small Vectorwise decomposition for per-vector scaled quantization (VSQ \citep{dai2021vsq}).}
  \label{fig:vsquant}
\end{wrapfigure}
During VSQ \citep{dai2021vsq}, the operand tensors are decomposed into 1D vectors in a hardware friendly manner as shown in Figure \ref{fig:vsquant}. Since the decomposed tensors are used as operands in matrix multiplications during inference, it is beneficial to perform this decomposition along the reduction dimension of the multiplication. The vectorwise quantization is performed similar to tensorwise quantization described in Equations \ref{eq:sf} and \ref{eq:tensor_quant}, where a scale factor $s_v$ is required for each vector $\bm{v}$ that maps the maximum absolute value of that vector to the maximum quantization level. While smaller vector lengths can lead to larger accuracy gains, the associated memory and computational overheads due to the per-vector scale factors increases. To alleviate these overheads, VSQ \citep{dai2021vsq} proposed a second level quantization of the per-vector scale factors to unsigned integers, while MX \citep{rouhani2023shared} quantizes them to integer powers of 2 (denoted as $2^{INT}$).

\subsubsection{MX Format}
The MX format proposed in \citep{rouhani2023microscaling} introduces the concept of sub-block shifting. For every two scalar elements of $b$-bits each, there is a shared exponent bit. The value of this exponent bit is determined through an empirical analysis that targets minimizing quantization MSE. We note that the FP format $E_{1}M_{b}$ is strictly better than MX from an accuracy perspective since it allocates a dedicated exponent bit to each scalar as opposed to sharing it across two scalars. Therefore, we conservatively bound the accuracy of a $b+2$-bit signed MX format with that of a $E_{1}M_{b}$ format in our comparisons. For instance, we use E1M2 format as a proxy for MX4.

\begin{figure}
    \centering
    \includegraphics[width=1\linewidth]{sections//figures/BlockFormats.pdf}
    \caption{\small Comparing LO-BCQ to MX format.}
    \label{fig:block_formats}
\end{figure}

Figure \ref{fig:block_formats} compares our $4$-bit LO-BCQ block format to MX \citep{rouhani2023microscaling}. As shown, both LO-BCQ and MX decompose a given operand tensor into block arrays and each block array into blocks. Similar to MX, we find that per-block quantization ($L_b < L_A$) leads to better accuracy due to increased flexibility. While MX achieves this through per-block $1$-bit micro-scales, we associate a dedicated codebook to each block through a per-block codebook selector. Further, MX quantizes the per-block array scale-factor to E8M0 format without per-tensor scaling. In contrast during LO-BCQ, we find that per-tensor scaling combined with quantization of per-block array scale-factor to E4M3 format results in superior inference accuracy across models. 

%%%%%%%%%%%%%%%%%%%%%%%%%%%%%%%%%%%%%%%%%%%%%%%%%%%%%%%%%%%%

\vspace{-0.4cm}
%%%%%%%%%%%%%% Bibliographies %%%%%%%%%%%%%%%%%%%%%%%%%%%%%%
\bibliographystyle{IEEEtran}
\bibliography{myreferences}
%%%%%%%%%%%%%%%%%%%%%%%%%%%%%%%%%%%%%%%%%%%%%%%%%%%%%%%%%%%%


\end{document}