In this study, we address the problem of influencing an election under the FJ opinion dynamics model, a vital concern for developing network defense strategies. We propose two innovative approaches inspired by real-world scenarios: maximizing the median opinion or altering it to sway the election, even marginally. We demonstrate that achieving our computational objectives are NP-hard and difficult to approximate. We develop three novel algorithms leveraging both continuous and discrete optimization techniques, applied across a variety of real-world and synthetic networks. Additionally, we establish that exact solutions for flipping the median are possible on rooted directed trees, which mimic hierarchical structures. Our research opens up several intriguing questions for future exploration. Is it possible to enhance the scalability of our methods using the advancements in sublinear algorithms for opinion dynamics~\cite{neumann2024sublinear}? Can we develop efficient algorithms that are parameterized by treewidth or other graph metrics?

% and continuous attacks on the
% susceptibility of persuasion. Targeted
% attacks employ stooges in the network,
% and our results show that these can be
% effective, but only if a central actor
% with positive opinion exists in the
% network. However, such targeted attacks
% are difficult to carry out and
% identifying the right stooges is
% computationally challenging. A continuous
% attacks corresponds to changing the
% susceptibility of persuasion of a larger
% group of actors. Carrying out such an
% attack is simpler and shows effectiveness
% especially for highly polarizing topics
% which lead to a skewed distribution of
% opinions, such as the real-world networks
% we use in our experiments.
 