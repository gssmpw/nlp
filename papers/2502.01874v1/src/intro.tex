Social media platforms such as Facebook, Twitter, Instagram, and Reddit have created positive social outcomes on major issues such as voter registration and mobilization within authoritarian regimes, but there is increasing concern that they are a major contributor to the rise of political dysfunction seen in the USA and some Western democracies since the early 2010s~\cite{haidt2019dark,mcnamee2020zucked}.   America has a widening political divide, and it stands out above other nations, according to a recent study by Stanford economists \cite{boxell2020cross}.  In 1994,  49\% of Americans held mixed political views with the rest of the mass divided between liberal and conservative \cite{pew2}, while the number of moderates  decreased to 39\% by 2014 \cite{bbc}. This political division is more pronounced than ever; Republicans and Democrats are arguably further apart ideologically than ever  before \cite{nytimes,bbc2,pew1}. For instance, an increasing fraction of Democrats and Republicans view the presidential candidate of the other party ``very unfavorable''. According to Pew Research in 1994 the fraction of  Republicans  that deemed  Democrats unfavorable was 17\% while in the recent presidential elections this fraction was 58\%. Abramowitz  and Webster show that partisan identities have become increasingly associated with divisive issues  in American society, including racial, cultural and ideological issues. This is reflected clearly into the elections' outcomes; the Democratic share of the House vote and the Democratic share of the presidential vote has increased from 0.54 in 1972 to 0.97 by 2018~\cite{abramowitz2015all}. 
 The same polarizing phenomenon is observed on a variety of topics, not just politics, including vaccination and the security measures against COVID-19~\cite{klein2020we,smith2019mapping}.  

Due to the planetary scale of social media, despite the fact that U.S. law bans foreign nationals from making certain expenditures or financial disbursements for the purpose of influencing federal elections, an adversary  has an unprecedented opportunity to shape public opinion using computational propaganda tools {\it remotely}. Few years ago,  the U.S. Department of Justice indicted numerous Russian agents that were  affiliated with the St. Petersburg-based Internet Research Agency (IRA), an organization that allegedly engaged in political and electoral interference operations in the United States which included the purchase of American computer server space, the creation of hundreds of fictitious online personas, and the use of stolen identities of persons from the United States.  According to the indictment issued by Robert Mueller against the Internet Research
Agency, the core strategy was to increase disaffection, distrust, and polarization in American politics~\cite{usjustice}. This can be achieved by using malicious accounts that attack moderate politicians, spread  misinformation in order to bolster controversial candidates, and increase societal polarization~\cite{allcott2017social,rolling}. 

It is not only foreign actors, or malicious entities that manipulate opinion dynamics through  social media. Most notably, Cambridge Analytica used personalized social influence techniques that  exploited personality traits of voters to sway voters away from Hillary Clinton towards Donald Trump~\cite{nix}.  However, there exist plenty of less known ``Cambridge Analytica'' cases. For example, Rally Forge is a private organization for sustainable natural resource conservation. It was proved that it orchestrated astroturfing campaigns, in which a combination of real and inauthentic accounts posted comments and replies to relevant conversational threads to create a misconception that public opinion lied on one side of a particular topic~\cite{astroturf}. In addition to such private organizations whose self-interest lies in manipulating opinion dynamics,    corporations design recommendation systems to maximize user engagement for revenue; however, this can unintentionally lead to increasing societal polarization and radicalization ~\cite{ledwich2019algorithmic}. 



In this paper we focus on a problem that is rooted in timely real-world scenarios~\cite{nix,haidt2019dark,mcnamee2020zucked,abramowitz2015all,usjustice}. Specifically, this issue revolves around the unstudied yet crucial concept of the median in equilibrium vectors. To illustrate the importance of this idea, imagine the following scenario: an individual, who has vested interests in political party $Y$, is aware through reliable surveys that parties $X$ and $Y$ are likely to get 50.1\% and 49.9\% of the vote, respectively. Could this individual employ social media strategies to alter public opinions and consequently invert these polling numbers in favor of party $Y$?  Alternatively, can an individual boost the median even if the majority is already voting for their preferred party?

More broadly, what tools are available for a potential manipulator to influence election results? We operate under the assumption that such an individual has comprehensive knowledge of both the network structure and the opinion dynamics described by the Friedkin-Johnsen (FJ) dynamics, c.f.~\cite{friedkin1997social} and   Equation~\eqref{eq:fj}. One objective of our study is to determine whether small interventions have the power to tip an election or, more universally, to shift the median  viewpoint in a given contentious subject towards a particular side.   We focus on interventions similar to those used by Solomon Asch, where certain participants are turned into {\it stooges}. Asch's pioneering experiments, central to social psychology and the study of opinion dynamics, explored how much social pressure from a majority could compel an individual to conform~\cite{asch1955opinions}. Essentially, our objective is to explore the challenge of influencing election results within the Friedkin-Johnsen (FJ) model of opinion formation by strategically selecting a limited number of stooges, similar to Asch's renowned conformity experiments. In the FJ model, this type of intervention is effectively represented by altering the resistance parameter of the stooges~\cite{abebe2020opinion,ristache2024wiser}. For further details, see Section~\ref{sec:proposed}. We state this informally as the next problem. 


\begin{tcolorbox}
\begin{problem}[\textsc{Informal Statement - Flipping the Median under the FJ Dynamics}]
\label{prob:informal}
Given a network of agents $G$ whose opinions form according to the FJ model, and an integer $k$, choose a set of $k$ stooges from the node set so that the median of the opinions at equilibrium changes significantly.
\end{problem}
\end{tcolorbox}

%\fabian{Should we (informally) clarify that selecting a stooge means that we change its resistance? And motivate this a bit like in our previous paper}

\noindent
We make the following contributions
related to this problem.

$\bullet$
We introduce a novel formulation for altering the median of opinions at equilibrium under the FJ dynamics, subject to Asch-like interventions. On one hand, the median presents substantial challenges because it is a non-smooth function of the equilibrium. On the other hand, it effectively models the contemporary issue of influencing elections through adversarial interventions on social media.    
To the best of our knowledge, Problem~\ref{prob:informal}  has not been studied before in any prior works.

$\bullet$
We prove that maximizing the median and any quantile under the FJ model is NP-hard and inapproximable to any constant factor.   

$\bullet$
We develop a continuous optimization framework that  utilizes two different approaches. The first uses Huber's M-estimators to approximate the median effectively. We design a robust gradient descent heuristic and provide a way to choose a good value of Huber's function parameter $c$. This allows us to run our algorithm only once instead of trying it on a large number of possible values.     
We develop an alternative objective via the sigmoid threshold influence function. We also demonstrate that Problem~\ref{prob:elections} can be solved exactly in polynomial time for the
special case of directed trees.

$\bullet$
We demonstrate that the straightforward discrete greedy algorithm is computationally impractical for even small-scale networks, resulting in a complexity of $\Omega(n^4)$, where $n$ is the total number of actors. Consequently, we develop a lazy version of the greedy approach that in practice enables us to apply it to larger networks.
    
$\bullet$
We conduct extensive experiments on real-world datasets and discover that our strategies effectively identify a small number of pivotal agents capable of shifting the median. This insight is vital for understanding how to protect the network against adversarial disturbances in the FJ model. 


