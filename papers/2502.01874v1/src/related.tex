

\spara{Opinion Dynamics and the Friedkin-Johnsen Model.} The exchange of opinions   among people is a core social activity that influences almost every social, political, and economic endeavor.  Opinion dynamics  models  have been used in various disciplines to model social learning~\cite{acemoglu2011opinion,acemouglu2013opinion,proskurnikov2017tutorial,proskurnikov2018tutorial,lorenz2007continuous,grabisch2020survey}.   In Friedkin \cite{friedkin2015problem}, it is highlighted that the opinions of individuals represent their cognitive perceptions of certain entities, such as specific issues, events, or other people. This can be seen in manifested attitudes as mentioned by Abelson~\cite{abelson1964mathematical}, Hunter \cite{hunter1984mathematical} or in the personal certainties of belief as noted by Halpern~\cite{halpern1991relationship}. From a mathematical perspective, opinions can be understood as scalar or vector values related to a set of agents.  
There exist several types of opinion dynamics models, see the tutorial by Proskurnikov and Tempo~\cite{proskurnikov2017tutorial,proskurnikov2018tutorial}.
We focus on discrete-time models where the opinions get updated in rounds.
There exist two types of discrete-time models,  models where the opinions are discrete  ~\cite{kempe2003maximizing,richardson2002mining,yildiz2013binary,berenbrink2016bounds} and continuous~\cite{french1956formal,degroot1974reaching,castellano2009statistical,hegselmann2002opinion,friedkin1997social,tsitsiklis1986distributed,lorenz2007continuous}.
The French-DeGroot model describes how individuals reach a consensus through stochastic interactions~\cite{french1956formal,degroot1974reaching}.  Let $G(V,E,w)$ be a weighted graph where the weighted adjacency matrix $W$ is row-stochastic. Each node $u \in V$ has an initial opinion $x_u(0) \in [0,1]$ (or, without loss of generality, in  $[-1,1]$) and let $x(k)=(x_u(k))_{u \in V}$ be the vector of opinions in round $k$. The French-DeGroot model updates the opinion vector according to $x(k+1)=W x(k), k=0,1,\ldots$. The convergence criteria are well understood~\cite{proskurnikov2017tutorial}.


Friedkin and Johnsen extended the French-DeGroot model to incorporate individuals' intrinsic beliefs and prejudices~\cite{friedkin1990social}. 
Each node $u\in [n]$ corresponds to a person who has
an \emph{innate opinion} $s_u$ and an \emph{expressed opinion}.
For each node~$u$, the innate opinion $s_u\in[0,1]$ is fixed over time and kept
private; the expressed opinion $x_u(t) \in[0,1]$ is publicly known 
and it changes over time $t\in\mathbb{N}$ due to peer pressure. 
Initially, $x_u(0)=s_u$ for all users $u\in V$. 
At each time
$t>0$, all users $u\in V$ update their expressed opinion $x_u(t)$ as the
weighted average of their innate opinion and the expressed opinions of their
neighbors, as 
follows:
\begin{align}
\label{eq:update-opinions}
	x_u(t) \!
	=\! \dfrac{s_u + \sum_{v\in N(u)} w_{uv} x_v(t)}{1 + \sum_{v\in N(u)} w_{uv}} \! =\! \dfrac{s_u + \sum_{v\in N(u)} w_{uv} x_v(t)}{1 + \deg(u)}
\end{align}
% 
We refer to this model as  the classic FJ model~\cite{friedkin1990social}. In the limit $t\to \infty$, the expressed opinions under mild conditions that are well understood reach an equilibrium $x^\star=(I+L)^{-1}s$.
In this work, we concentrate on the Friedkin-Johnsen model along with an enhanced variant that  delves into the practical consideration  that agents  exhibit varying degrees of susceptibility to persuasion~\cite{cialdini2001science}:
%
%
\begin{equation}
\label{eq:dynamics}
\boxed{\textstyle x_u(t+1) = \alpha_u s_u + \frac{1-\alpha_u}{\deg(u)} \sum_{v \in N(u)}  w_{uv} x_v(t) \ \text{~for all~~}u \in V.}
\end{equation}
% 
%
We refer to this variant as the generalized FJ model~\cite{abebe2020opinion,ghaderi2013opinion,bindel2015bad}, as it is clear that Equation~\eqref{eq:update-opinions} is a special case of Equation~\eqref{eq:dynamics}.  Let us represent the row-stochastic normalized adjacency matrix as \( W \) and the diagonal susceptibility matrix as \( A = {\sf Diag}(\alpha_1, \ldots, \alpha_n) \). With these definitions, we can rewrite Equation~\eqref{eq:dynamics} in  matrix form:
%
\begin{equation}
\label{eq:fj}
x(t+1) = A s + (I-A) W x(t) 
\end{equation}
%
%
By setting $x(t+1)$ equal to $x(t)$, it becomes evident that the generalized FJ equilibrium~\cite{abebe2020opinion,ghaderi2013opinion,bindel2015bad} vector is
%
%
\begin{equation}
\label{eq:fj2}
\boxed{x^\star = (I - (I-A) W)^{-1}As} 
\end{equation}
%
%
The criteria for convergence have been thoroughly discussed in references such as~\cite{ghaderi2014opinion,proskurnikov2017tutorial}. The FJ model     strikes an excellent balance between being both mathematically rigorous and manageable as a model, while also capturing the realistic dynamics of opinion evolution. Friedkin and Bullo~\cite{friedkin2017truth}  emphasized that the FJ model stands alone as the sole framework that has been subjected to a consistent series of human-subject experiments, enabling the assessment of its predictive capabilities concerning changes in opinions~\cite{friedkin2011social,friedkin2016theory}. It is essential to highlight that $x^\star=x^\star(\mathbf{\alpha},W,s)$ depends on the resistance values $\mathbf{\alpha}$, the topology of the graph $W$, and the innate opinions $\mathbf{s}$.   Our paper examines interventions at the resistance parameter level, \(\mathbf{\alpha}\), which represent an Asch-type intervention involving the recruitment of stooges.


\spara{Optimizing objectives within the Friedkin-Johnsen model} While influence maximization has a well-established history in discrete models, starting with the seminal work of Kempe, Kleinberg, and Tardos~\cite{kempe2003maximizing}, its application in continuous opinion dynamics models has been largely neglected, with most focus on the development of the models themselves. The research initiated by Gionis, Terzi, and Tsaparas~\cite{gionis2013opinion} marked a significant shift by addressing the optimization of the aggregate sum of opinions at equilibrium through the selection of $k$ nodes and consistently fixing their expressed opinion to 1. This approach utilizes the standard Friedkin-Johnsen (FJ) model~\cite{friedkin1990social}, incorporating the concept of stubbornness by keeping the opinions of the selected $k$ nodes constant. 
Musco, Musco, and Tsourakakis~\cite{musco18} further expanded on this by optimizing an objective that balances disagreement and polarization at equilibrium.  Since then, various other formulations and algorithmic solutions have been proposed~\cite{biondi2023dynamics,zhu2022nearly,sun2023opinion,tang2021susceptible,zhu2021minimizing}.  For instance, Gaitonde, Kleinberg, and Tardos~\cite{gaitonde2020adversarial}, along with Chen and Rácz et al.~\cite{chen22,DBLP:journals/corr/abs-2206-08996}, investigated budgeted adversarial interventions on inherent opinions, revealing significant links between spectral graph theory and opinion dynamics. The work of Abebe et al.~\cite{abebe2020opinion}, which is closely related to our research, pioneered adversarial interventions at the susceptibility level using the generalized FJ model. They addressed an unbudgeted optimization problem that is neither convex nor concave, and developed a local search algorithm uncovering a remarkable structure in the local optima of the objective functions. Furthermore, Chan and Shan~\cite{chan2021hardness} demonstrated the NP-hardness of this problem when subject to budget constraints. Closest to our work lies the recent work by Ristache, Spaeh, and Tsourakakis~\cite{ristache2024wiser} that explores the impact of interventions targeting susceptibility on the wisdom of crowds, specifically analyzing how these affect mean squared error (MSE) and polarization, and demonstrating their significant interrelation. It should be noted that the median is a non-smooth statistic that has not been explored in this research area before, despite its significant implications in the context of elections.