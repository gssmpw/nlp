\section{Related Work}
\label{sec:related_work}

The notion of reversibility in the context of the voter model has previously been used by Hassin \& Peleg~\cite{hassin2001distributed} to provide winning probabilities on a class of dynamic networks called `stabilising dynamic graphs'. 
In these networks, until a given round, edges may disconnect and nodes attempting to copy a disconnected neighbour keep their own colour, similarly to gnostic nodes choosing agnostic ones in our protocol. 
The authors showed that their previous results hold for networks with reversible Markov chains, i.e. the total influence of the nodes of a given colour remains a martingale in the new process.

Several works studied agents that can be `undecided' as an intermediate state, with nodes transitioning to this state when they select a differently coloured neighbour (\cite{angluin,perron, clementi_et_al:LIPIcs.MFCS.2018.28, petra1}). For a complete graph with binary opinions, the synchronous variant of this protocol has been shown to converge to the most common (plurality) colour in $O(\log n)$ rounds with high probability, assuming there is an initial difference of $\Omega(\sqrt{n \log n})$ in the numbers of agents with each colour~\cite{clementi2018tight}. Similar results have been obtained for the asynchronous protocol in the context of chemical reaction networks~\cite{condon2020approximate}. Additionally, for the consensus problem with $k>2$ opinions, Becchetti et al. defined a `monochromatic distance' function which measures the distance between any colour configuration and consensus, and used this to bound the convergence time of the synchronous process by $O(k \log n)$~\cite{becchetti2015plurality}. 

On the other hand, Demers et al. proposed rumour-spreading protocols to aid the maintenance of distributed databases; these include push, pull, and push-pull transmissions~\cite{demers1987epidemic}. For the synchronous push model, it is known that the number of rounds required to broadcast the rumour to all nodes is at most $O(n \log n)$, which is tight for the star graph~\cite{feige1990randomized}. This process has also been analysed for several other topologies, including complete graphs, hypercubes, bounded-degree graphs, and random graphs~\cite{feige1990randomized}.

Our model also has similarities to the biased voter model proposed in \cite{george}, where one colour (corresponding to the agnostic state) has a bias of $0$. However, their results do not apply in our setting since they assume that one colour has a strictly higher preference than all other colours. See also \cite{Lanchier_Neuhauser_2007} for the biased voter model with $2$ opinions in the continuous-time model. Our work is also related to~\cite{zehmakan2024viral}. In it, the authors study the evolution of a process with agnostic nodes, where the key difference is that gnostic nodes can never change colour. 

Previous works on opinion diffusion have also studied related concepts to agnostic nodes, such as stubbornness. Those are mostly in the context of the majority model~\cite{auletta2017information,out2021majority} and the related Friedkin-Johnson model~\cite{xu2022effects, shirzadi2024stubborn}. The main difference between these models and ours is that the process dynamics is deterministic in the majority and Friedkin-Johnson models, whereas in the voter model, the process dynamics are randomised.