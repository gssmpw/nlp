\section{Limitations and Future Work}\label{sec:future}
\noindent\textbf{Scope of our evaluation: }
Performing a thorough exploration and evaluation of all the defenses in Table~\ref{tab:defenses}, along with additional ones, goes beyond the scope of a single paper. Performing a thorough exploration of one defense from each dimension, type, and attribute is also beyond the scope of one single paper as it requires running the full set of experiments for every defense and then comparing them.

\noindent\textbf{Expandability of our systemization: }
The defenses chosen for evaluation serve as representatives of the broader defense literature; however, it is acknowledged that alternative evaluations may differ based on different sets of defenses. Also, a newly designed defense based on a novel technique might not fit exactly along the attributes in our systemization. Therefore, our systemization remains adaptable (as detailed in \S\ref{sec:introduction}), allowing for the incorporation of additional type, attributes, and even dimensions in the future.

\noindent\textbf{Incorporating other modalities:}
As highlighted in \S\ref{sec:introduction}, our aim is to expand the scope of our evaluation across diverse dimensions, encompassing data distributions, data modalities, and the nature of the ML task. While we have addressed image classification and NLP, other modalities, such as multimodal time-series tasks~\cite{zhao2022multimodal}, and the emerging paradigm of \emph{vision-language models}~\cite{radford2021learning}, remain unexplored. The evaluation of these modalities, along with more complex vision-language models, under traditional threat models could prove intriguing. Such exploration might lead to the development of new attacks and defenses, potentially uncovering novel pitfalls in the process.
