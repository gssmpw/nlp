%%%%%%%% ICML 2025 EXAMPLE LATEX SUBMISSION FILE %%%%%%%%%%%%%%%%%

\documentclass{article}
\usepackage{dirtytalk}

% Recommended, but optional, packages for figures and better typesetting:
\usepackage{microtype}
\usepackage{graphicx}
\usepackage{caption}
\usepackage{enumerate}
%\usepackage{subfigure}
\usepackage{booktabs} % for professional tables

% hyperref makes hyperlinks in the resulting PDF.
% If your build breaks (sometimes temporarily if a hyperlink spans a page)
% please comment out the following usepackage line and replace
% \usepackage{icml2025} with \usepackage[nohyperref]{icml2025} above.
\usepackage{hyperref}


% Attempt to make hyperref and algorithmic work together better:
\newcommand{\theHalgorithm}{\arabic{algorithm}}

% Use the following line for the initial blind version submitted for review:
%\usepackage{icml2025}

% If accepted, instead use the following line for the camera-ready submission:
\usepackage[accepted]{icml2025_archive}

% For theorems and such
\usepackage{amsmath}
\usepackage{amssymb}
\usepackage{mathtools}
\usepackage{amsthm}

% if you use cleveref..
\usepackage[capitalize,noabbrev]{cleveref}

\usepackage{xspace}

%%%%%%%%%%%%%%%%%%%%%%%%%%%%%%%%
% THEOREMS
%%%%%%%%%%%%%%%%%%%%%%%%%%%%%%%%
\theoremstyle{plain}
\newtheorem{theorem}{Theorem}[section]
\newtheorem{proposition}[theorem]{Proposition}
\newtheorem{lemma}[theorem]{Lemma}
\newtheorem{corollary}[theorem]{Corollary}
\theoremstyle{definition}
\newtheorem{definition}[theorem]{Definition}
\newtheorem{assumption}[theorem]{Assumption}
\theoremstyle{remark}
\newtheorem{remark}[theorem]{Remark}

% Todonotes is useful during development; simply uncomment the next line
%    and comment out the line below the next line to turn off comments
%\usepackage[disable,textsize=tiny]{todonotes}
\usepackage[textsize=tiny]{todonotes}

%% only for draft --------------------------------------------
\long\def\KO#1{{\color{olive}{[KO: #1]}}}
\long\def\WP#1{{\color{blue}{[RS: #1]}}}
\long\def\CC#1{{\color{green}{[CS: #1]}}}
\long\def\MG#1{{\color{teal}{[MG: #1]}}}
\long\def\AD#1{{\color{violet}{[SS: #1]}}}
\long\def\EDIT#1{{\color{red}{#1}\color{red}}}
\long\def\EDITtwo#1{{\color{red}{#1}\color{black}}}
%% ------------------------------------------------------------

%---------------Added packages/Functions--------------
\newcommand{\model}[1]{\texttt{#1}\xspace}
\newcommand{\ARIMA}{\model{ARIMA}}
\newcommand{\ETS}{\model{AutoETS}}
\newcommand{\Fourier}{\model{Fourier}}
\newcommand{\MLP}{\model{MLP}}
\newcommand{\DLinear}{\model{DLinear}}
\newcommand{\NLinear}{\model{NLinear}}
\newcommand{\TCN}{\model{TCN}}
\newcommand{\LSTM}{\model{LSTM}}
\newcommand{\RNN}{\model{RNN}}
\newcommand{\TimesNet}{\model{TimesNet}}
\newcommand{\TSMixer}{\model{TSMixer}}
\newcommand{\TSMixerx}{\model{TSMixerx}}
\newcommand{\NHITS}{\model{NHITS}}
\newcommand{\NBEATS}{\model{NBEATS}}

%--------tranformer (non-foundation models)-------
\newcommand{\VanillaTransformer}{\model{VanillaTransformer}}
\newcommand{\TFT}{\model{TFT}}
\newcommand{\PatchTST}{\model{PatchTST}}
\newcommand{\Fedformer}{\model{Fedformer}}
\newcommand{\Autoformer}{\model{Autoformer}}
\newcommand{\Informer}{\model{Informer}}
\newcommand{\iTransformer}{\model{iTransformer}}
\newcommand{\Tfive}{\model{T5 Model}}
\newcommand{\BasisFormer}{\model{BasisFormer}}

%--------foundation models -------
\newcommand{\TimesFM}{\model{TimesFM}}
\newcommand{\MOMENT}{\model{MOMENT}}
\newcommand{\LagLlama}{\model{LagLlama}}
\newcommand{\Moirai}{\model{Moirai}}
\newcommand{\TimeGPT}{\model{TimeGPT}}
\newcommand{\Chronos}{\model{Chronos}}
\newcommand{\SPADE}{\model{SPADE}}
\newcommand{\Timer}{\model{Timer}}
\newcommand{\TinyTimeMixers}{\model{TTMs}}

\usepackage{caption}
\usepackage{subcaption}
\usepackage{multirow}
\usepackage{graphicx} 
\usepackage{amsmath}
\usepackage{xcolor}
\usepackage{wrapfig}

\DeclareMathOperator*{\argmax}{arg\,max}
%---------------Added packages--------------

% The \icmltitle you define below is probably too long as a header.
% Therefore, a short form for the running title is supplied here:
\icmltitlerunning{Implicit Reasoning in Deep Time Series Forecasting}

\begin{document}

\twocolumn[
\icmltitle{
%Implicit Reasoning in Deep Time Series Forecasting
% \KO{1. Reasoning or Memorization? Exploring the Limits of Time Series Foundation Models}
Investigating Compositional Reasoning in Time Series Foundation Models % beyond Memorization-Based Generalization
% \KO{3. Do TSFMs succeed by memorizing training patterns, or can they logically extrapolating unseen patterns?}
}

% {Implicit Reasoning through Addition: Composing Forecasts with Fourier Features

% It is OKAY to include author information, even for blind
% submissions: the style file will automatically remove it for you
% unless you've provided the [accepted] option to the icml2025
% package.

% List of affiliations: The first argument should be a (short)
% identifier you will use later to specify author affiliations
% Academic affiliations should list Department, University, City, Region, Country
% Industry affiliations should list Company, City, Region, Country

% You can specify symbols, otherwise they are numbered in order.
% Ideally, you should not use this facility. Affiliations will be numbered
% in order of appearance and this is the preferred way.

\icmlsetsymbol{equal}{*}


\begin{icmlauthorlist}
\icmlauthor{Willa Potosnak}{cmu}
\icmlauthor{Cristian Challu}{equal,cmu,nixtla}
\icmlauthor{Mononito Goswami}{equal,cmu}
\icmlauthor{Kin G. Olivares}{cmu,amazon}
\icmlauthor{Michał Wiliński}{cmu}
\icmlauthor{Nina Żukowska}{cmu}
\icmlauthor{Artur Dubrawski}{cmu}
\end{icmlauthorlist}

\icmlaffiliation{cmu}{Auton Lab, School of Computer Science, Carnegie Mellon University}
\icmlaffiliation{nixtla}{Nixtla}
\icmlaffiliation{amazon}{Amazon}

\icmlcorrespondingauthor{Willa Potosnak}{wpotosna@andrew.cmu.edu}

\icmlkeywords{Machine Learning, ICML}

\vskip 0.3in
]

% this must go after the closing bracket ] following \twocolumn[ ...

% This command actually creates the footnote in the first column
% listing the affiliations and the copyright notice.
% The command takes one argument, which is text to display at the start of the footnote.
% The \icmlEqualContribution command is standard text for equal contribution.
% Remove it (just {}) if you do not need this facility.

%\printAffiliationsAndNotice{}  % leave blank if no need to mention equal contribution
\printAffiliationsAndNotice{\icmlEqualContribution} % otherwise use the standard text.

\begin{abstract} 
Large pre-trained time series foundation models (TSFMs) have demonstrated promising zero-shot performance across a wide range of domains. However, a question remains: Do TSFMs succeed solely by memorizing training patterns, or do they possess the ability to reason? While reasoning is a topic of great interest in the study of Large Language Models (LLMs), it is undefined and largely unexplored in the context of TSFMs. In this work, inspired by language modeling literature, we formally define compositional reasoning in forecasting and distinguish it from in-distribution generalization. We evaluate the reasoning and generalization capabilities of 23 popular deep learning forecasting models on multiple synthetic and real-world datasets. Additionally, through controlled studies, we systematically examine which design choices in TSFMs contribute to improved reasoning abilities. Our study yields key insights into the impact of TSFM architecture design on compositional reasoning and generalization. We find that patch-based Transformers have the best reasoning performance, closely followed by residualized MLP-based architectures, which are 97\% less computationally complex in terms of FLOPs and 86\% smaller in terms of the number of trainable parameters. Interestingly, in some zero-shot out-of-distribution scenarios, these models can outperform moving average and exponential smoothing statistical baselines trained on in-distribution data. Only a few design choices, such as the tokenization method, had a significant (negative) impact on Transformer model performance.
\end{abstract}


\section{Introduction}
\label{section:introduction}
\section{Introduction}
\label{section:introduction}

% redirection is unique and important in VR
Virtual Reality (VR) systems enable users to embody virtual avatars by mirroring their physical movements and aligning their perspective with virtual avatars' in real time. 
As the head-mounted displays (HMDs) block direct visual access to the physical world, users primarily rely on visual feedback from the virtual environment and integrate it with proprioceptive cues to control the avatar’s movements and interact within the VR space.
Since human perception is heavily influenced by visual input~\cite{gibson1933adaptation}, 
VR systems have the unique capability to control users' perception of the virtual environment and avatars by manipulating the visual information presented to them.
Leveraging this, various redirection techniques have been proposed to enable novel VR interactions, 
such as redirecting users' walking paths~\cite{razzaque2005redirected, suma2012impossible, steinicke2009estimation},
modifying reaching movements~\cite{gonzalez2022model, azmandian2016haptic, cheng2017sparse, feick2021visuo},
and conveying haptic information through visual feedback to create pseudo-haptic effects~\cite{samad2019pseudo, dominjon2005influence, lecuyer2009simulating}.
Such redirection techniques enable these interactions by manipulating the alignment between users' physical movements and their virtual avatar's actions.

% % what is hand/arm redirection, motivation of study arm-offset
% \change{\yj{i don't understand the purpose of this paragraph}
% These illusion-based techniques provide users with unique experiences in virtual environments that differ from the physical world yet maintain an immersive experience. 
% A key example is hand redirection, which shifts the virtual hand’s position away from the real hand as the user moves to enhance ergonomics during interaction~\cite{feuchtner2018ownershift, wentzel2020improving} and improve interaction performance~\cite{montano2017erg, poupyrev1996go}. 
% To increase the realism of virtual movements and strengthen the user’s sense of embodiment, hand redirection techniques often incorporate a complete virtual arm or full body alongside the redirected virtual hand, using inverse kinematics~\cite{hartfill2021analysis, ponton2024stretch} or adjustments to the virtual arm's movement as well~\cite{li2022modeling, feick2024impact}.
% }

% noticeability, motivation of predicting a probability, not a classification
However, these redirection techniques are most effective when the manipulation remains undetected~\cite{gonzalez2017model, li2022modeling}. 
If the redirection becomes too large, the user may not mitigate the conflict between the visual sensory input (redirected virtual movement) and their proprioception (actual physical movement), potentially leading to a loss of embodiment with the virtual avatar and making it difficult for the user to accurately control virtual movements to complete interaction tasks~\cite{li2022modeling, wentzel2020improving, feuchtner2018ownershift}. 
While proprioception is not absolute, users only have a general sense of their physical movements and the likelihood that they notice the redirection is probabilistic. 
This probability of detecting the redirection is referred to as \textbf{noticeability}~\cite{li2022modeling, zenner2024beyond, zenner2023detectability} and is typically estimated based on the frequency with which users detect the manipulation across multiple trials.

% version B
% Prior research has explored factors influencing the noticeability of redirected motion, including the redirection's magnitude~\cite{wentzel2020improving, poupyrev1996go}, direction~\cite{li2022modeling, feuchtner2018ownershift}, and the visual characteristics of the virtual avatar~\cite{ogawa2020effect, feick2024impact}.
% While these factors focus on the avatars, the surrounding virtual environment can also influence the users' behavior and in turn affect the noticeability of redirection.
% One such prominent external influence is through the visual channel - the users' visual attention is constantly distracted by complex visual effects and events in practical VR scenarios.
% Although some prior studies have explored how to leverage user blindness caused by visual distractions to redirect users' virtual hand~\cite{zenner2023detectability}, there remains a gap in understanding how to quantify the noticeability of redirection under visual distractions.

% visual stimuli and gaze behavior
Prior research has explored factors influencing the noticeability of redirected motion, including the redirection's magnitude~\cite{wentzel2020improving, poupyrev1996go}, direction~\cite{li2022modeling, feuchtner2018ownershift}, and the visual characteristics of the virtual avatar~\cite{ogawa2020effect, feick2024impact}.
While these factors focus on the avatars, the surrounding virtual environment can also influence the users' behavior and in turn affect the noticeability of redirection.
This, however, remains underexplored.
One such prominent external influence is through the visual channel - the users' visual attention is constantly distracted by complex visual effects and events in practical VR scenarios.
We thus want to investigate how \textbf{visual stimuli in the virtual environment} affect the noticeability of redirection.
With this, we hope to complement existing works that focus on avatars by incorporating environmental visual influences to enable more accurate control over the noticeability of redirected motions in practical VR scenarios.
% However, in realistic VR applications, the virtual environment often contains complex visual effects beyond the virtual avatar itself. 
% We argue that these visual effects can \textbf{distract users’ visual attention and thus affect the noticeability of redirection offsets}, while current research has yet taken into account.
% For instance, in a VR boxing scenario, a user’s visual attention is likely focused on their opponent rather than on their virtual body, leading to a lower noticeability of redirection offsets on their virtual movements. 
% Conversely, when reaching for an object in the center of their field of view, the user’s attention is more concentrated on the virtual hand’s movement and position to ensure successful interaction, resulting in a higher noticeability of offsets.

Since each visual event is a complex choreography of many underlying factors (type of visual effect, location, duration, etc.), it is extremely difficult to quantify or parameterize visual stimuli.
Furthermore, individuals respond differently to even the same visual events.
Prior neuroscience studies revealed that factors like age, gender, and personality can influence how quickly someone reacts to visual events~\cite{gillon2024responses, gale1997human}. 
Therefore, aiming to model visual stimuli in a way that is generalizable and applicable to different stimuli and users, we propose to use users' \textbf{gaze behavior} as an indicator of how they respond to visual stimuli.
In this paper, we used various gaze behaviors, including gaze location, saccades~\cite{krejtz2018eye}, fixations~\cite{perkhofer2019using}, and the Index of Pupil Activity (IPA)~\cite{duchowski2018index}.
These behaviors indicate both where users are looking and their cognitive activity, as looking at something does not necessarily mean they are attending to it.
Our goal is to investigate how these gaze behaviors stimulated by various visual stimuli relate to the noticeability of redirection.
With this, we contribute a model that allows designers and content creators to adjust the redirection in real-time responding to dynamic visual events in VR.

To achieve this, we conducted user studies to collect users' noticeability of redirection under various visual stimuli.
To simulate realistic VR scenarios, we adopted a dual-task design in which the participants performed redirected movements while monitoring the visual stimuli.
Specifically, participants' primary task was to report if they noticed an offset between the avatar's movement and their own, while their secondary task was to monitor and report the visual stimuli.
As realistic virtual environments often contain complex visual effects, we started with simple and controlled visual stimulus to manage the influencing factors.

% first user study, confirmation study
% collect data under no visual stimuli, different basic visual stimuli
We first conducted a confirmation study (N=16) to test whether applying visual stimuli (opacity-based) actually affects their noticeability of redirection. 
The results showed that participants were significantly less likely to detect the redirection when visual stimuli was presented $(F_{(1,15)}=5.90,~p=0.03)$.
Furthermore, by analyzing the collected gaze data, results revealed a correlation between the proposed gaze behaviors and the noticeability results $(r=-0.43)$, confirming that the gaze behaviors could be leveraged to compute the noticeability.

% data collection study
We then conducted a data collection study to obtain more accurate noticeability results through repeated measurements to better model the relationship between visual stimuli-triggered gaze behaviors and noticeability of redirection.
With the collected data, we analyzed various numerical features from the gaze behaviors to identify the most effective ones. 
We tested combinations of these features to determine the most effective one for predicting noticeability under visual stimuli.
Using the selected features, our regression model achieved a mean squared error (MSE) of 0.011 through leave-one-user-out cross-validation. 
Furthermore, we developed both a binary and a three-class classification model to categorize noticeability, which achieved an accuracy of 91.74\% and 85.62\%, respectively.

% evaluation study
To evaluate the generalizability of the regression model, we conducted an evaluation study (N=24) to test whether the model could accurately predict noticeability with new visual stimuli (color- and scale-based animations).
Specifically, we evaluated whether the model's predictions aligned with participants' responses under these unseen stimuli.
The results showed that our model accurately estimated the noticeability, achieving mean squared errors (MSE) of 0.014 and 0.012 for the color- and scale-based visual stimili, respectively, compared to participants' responses.
Since the tested visual stimuli data were not included in the training, the results suggested that the extracted gaze behavior features capture a generalizable pattern and can effectively indicate the corresponding impact on the noticeability of redirection.

% application
Based on our model, we implemented an adaptive redirection technique and demonstrated it through two applications: adaptive VR action game and opportunistic rendering.
We conducted a proof-of-concept user study (N=8) to compare our adaptive redirection technique with a static redirection, evaluating the usability and benefits of our adaptive redirection technique.
The results indicated that participants experienced less physical demand and stronger sense of embodiment and agency when using the adaptive redirection technique. 
These results demonstrated the effectiveness and usability of our model.

In summary, we make the following contributions.
% 
\begin{itemize}
    \item 
    We propose to use users' gaze behavior as a medium to quantify how visual stimuli influences the noticebility of redirection. 
    Through two user studies, we confirm that visual stimuli significantly influences noticeability and identify key gaze behavior features that are closely related to this impact.
    \item 
    We build a regression model that takes the user's gaze behavioral data as input, then computes the noticeability of redirection.
    Through an evaluation study, we verify that our model can estimate the noticeability with new participants under unseen visual stimuli.
    These findings suggest that the extracted gaze behavior features effectively capture the influence of visual stimuli on noticeability and can generalize across different users and visual stimuli.
    \item 
    We develop an adaptive redirection technique based on our regression model and implement two applications with it.
    With a proof-of-concept study, we demonstrate the effectiveness and potential usability of our regression model on real-world use cases.

\end{itemize}

% \delete{
% Virtual Reality (VR) allows the user to embody a virtual avatar by mirroring their physical movements through the avatar.
% As the user's visual access to the physical world is blocked in tasks involving motion control, they heavily rely on the visual representation of the avatar's motions to guide their proprioception.
% Similar to real-world experiences, the user is able to resolve conflicts between different sensory inputs (e.g., vision and motor control) through multisensory integration, which is essential for mitigating the sensory noise that commonly arises.
% However, it also enables unique manipulations in VR, as the system can intentionally modify the avatar's movements in relation to the user's motions to achieve specific functional outcomes,
% for example, 
% % the manipulations on the avatar's movements can 
% enabling novel interaction techniques of redirected walking~\cite{razzaque2005redirected}, redirected reaching~\cite{gonzalez2022model}, and pseudo haptics~\cite{samad2019pseudo}.
% With small adjustments to the avatar's movements, the user can maintain their sense of embodiment, due to their ability to resolve the perceptual differences.
% % However, a large mismatch between the user and avatar's movements can result in the user losing their sense of embodiment, due to an inability to resolve the perceptual differences.
% }

% \delete{
% However, multisensory integration can break when the manipulation is so intense that the user is aware of the existence of the motion offset and no longer maintains the sense of embodiment.
% Prior research studied the intensity threshold of the offset applied on the avatar's hand, beyond which the embodiment will break~\cite{li2022modeling}. 
% Studies also investigated the user's sensitivity to the offsets over time~\cite{kohm2022sensitivity}.
% Based on the findings, we argue that one crucial factor that affects to what extent the user notices the offset (i.e., \textit{noticeability}) that remains under-explored is whether the user directs their visual attention towards or away from the virtual avatar.
% Related work (e.g., Mise-unseen~\cite{marwecki2019mise}) has showcased applications where adjustments in the environment can be made in an unnoticeable manner when they happen in the area out of the user's visual field.
% We hypothesize that directing the user's visual attention away from the avatar's body, while still partially keeping the avatar within the user's field-of-view, can reduce the noticeability of the offset.
% Therefore, we conduct two user studies and implement a regression model to systematically investigate this effect.
% }

% \delete{
% In the first user study (N = 16), we test whether drawing the user's visual attention away from their body impacts the possibility of them noticing an offset that we apply to their arm motion in VR.
% We adopt a dual-task design to enable the alteration of the user's visual attention and a yes/no paradigm to measure the noticeability of motion offset. 
% The primary task for the user is to perform an arm motion and report when they perceive an offset between the avatar's virtual arm and their real arm.
% In the secondary task, we randomly render a visual animation of a ball turning from transparent to red and becoming transparent again and ask them to monitor and report when it appears.
% We control the strength of the visual stimuli by changing the duration and location of the animation.
% % By changing the time duration and location of the visual animation, we control the strengths of attraction to the users.
% As a result, we found significant differences in the noticeability of the offsets $(F_{(1,15)}=5.90,~p=0.03)$ between conditions with and without visual stimuli.
% Based on further analysis, we also identified the behavioral patterns of the user's gaze (including pupil dilation, fixations, and saccades) to be correlated with the noticeability results $(r=-0.43)$ and they may potentially serve as indicators of noticeability.
% }

% \delete{
% To further investigate how visual attention influences the noticeability, we conduct a data collection study (N = 12) and build a regression model based on the data.
% The regression model is able to calculate the noticeability of the offset applied on the user's arm under various visual stimuli based on their gaze behaviors.
% Our leave-one-out cross-validation results show that the proposed method was able to achieve a mean-squared error (MSE) of 0.012 in the probability regression task.
% }

% \delete{
% To verify the feasibility and extendability of the regression model, we conduct an evaluation study where we test new visual animations based on adjustments on scale and color and invite 24 new participants to attend the study.
% Results show that the proposed method can accurately estimate the noticeability with an MSE of 0.014 and 0.012 in the conditions of the color- and scale-based visual effects.
% Since these animations were not included in the dataset that the regression model was built on, the study demonstrates that the gaze behavioral features we extracted from the data capture a generalizable pattern of the user's visual attention and can indicate the corresponding impact on the noticeability of the offset.
% }

% \delete{
% Finally, we demonstrate applications that can benefit from the noticeability prediction model, including adaptive motion offsets and opportunistic rendering, considering the user's visual attention. 
% We conclude with discussions of our work's limitations and future research directions.
% }

% \delete{
% In summary, we make the following contributions.
% }
% % 
% \begin{itemize}
%     \item 
%     \delete{
%     We quantify the effects of the user's visual attention directed away by stimuli on their noticeability of an offset applied to the avatar's arm motion with respect to the user's physical arm. 
%     Through two user studies, we identified gaze behavioral features that are indicative of the changes in noticeability.
%     }
%     \item 
%     \delete{We build a regression model that takes the user's gaze behavioral data and the offset applied to the arm motion as input, then computes the probability of the user noticing the offset.
%     Through an evaluation study, we verified that the model needs no information about the source attracting the user's visual attention and can be generalizable in different scenarios.
%     }
%     \item 
%     \delete{We demonstrate two applications that potentially benefit from the regression model, including adaptive motion offsets and opportunistic rendering.
%     }

% \end{itemize}

\begin{comment}
However, users will lose the sense of embodiment to the virtual avatars if they notice the offset between the virtual and physical movements.
To address this, researchers have been exploring the noticing threshold of offsets with various magnitudes and proposing various redirection techniques that maintain the sense of embodiment~\cite{}.

However, when users embody virtual avatars to explore virtual environments, they encounter various visual effects and content that can attract their attention~\cite{}.
During this, the user may notice an offset when he observes the virtual movement carefully while ignoring it when the virtual contents attract his attention from the movements.
Therefore, static offset thresholds are not appropriate in dynamic scenarios.

Past research has proposed dynamic mapping techniques that adapted to users' state, such as hand moving speed~\cite{frees2007prism} or ergonomically comfortable poses~\cite{montano2017erg}, but not considering the influence of virtual content.
More specifically, PRISM~\cite{frees2007prism} proposed adjusting the C/D ratio with a non-linear mapping according to users' hand moving speed, but it might not be optimal for various virtual scenarios.
While Erg-O~\cite{montano2017erg} redirected users' virtual hands according to the virtual target's relative position to reduce physical fatigue, neglecting the change of virtual environments. 

Therefore, how to design redirection techniques in various scenarios with different visual attractions remains unknown.
To address this, we investigate how visual attention affects the noticing probability of movement offsets.
Based on our experiments, we implement a computational model that automatically computes the noticing probability of offsets under certain visual attractions.
VR application designers and developers can easily leverage our model to design redirection techniques maintaining the sense of embodiment adapt to the user's visual attention.
We implement a dynamic redirection technique with our model and demonstrate that it effectively reduces the target reaching time without reducing the sense of embodiment compared to static redirection techniques.

% Need to be refined
This paper offers the following contributions.
\begin{itemize}
    \item We investigate how visual attractions affect the noticing probability of redirection offsets.
    \item We construct a computational model to predict the noticing probability of an offset with a given visual background.
    \item We implement a dynamic redirection technique adapting to the visual background. We evaluate the technique and develop three applications to demonstrate the benefits. 
\end{itemize}



First, we conducted a controlled experiment to understand how users perceived the movement offset while subjected to various distractions.
Since hand redirection is one of the most frequently used redirections in VR interactions, we focused on the dynamic arm movements and manually added angular offsets to the' elbow joint~\cite{li2022modeling, gonzalez2022model, zenner2019estimating}. 
We employed flashing spheres in the user's field of view as distractions to attract users' visual attention.
Participants were instructed to report the appearing location of the spheres while simultaneously performing the arm movements and reporting if they perceived an offset during the movement. 
(\zhipeng{Add the results of data collection. Analyze the influence of the distance between the gaze map and the offset.}
We measured the visual attraction's magnitude with the gaze distribution on it.
Results showed that stronger distractions made it harder for users to notice the offset.)
\zhipeng{Need to rewrite. Not sure to use gaze distribution or a metric obtained from the visual content.}
Secondly, we constructed a computational model to predict the noticing probability of offsets with given visual content.
We analyzed the data from the user studies to measure the influence of visual attractions on the noticing probability of offsets.
We built a statistical model to predict the offset's noticing probability with a given visual content.
Based on the model, we implement a dynamic redirection technique to adjust the redirection offset adapted to the user's current field of view.
We evaluated the technique in a target selection task compared to no hand redirection and static hand redirection.
\zhipeng{Add the results of the evaluation.}
Results showed that the dynamic hand redirection technique significantly reduced the target selection time with similar accuracy and a comparable sense of embodiment.
Finally, we implemented three applications to demonstrate the potential benefits of the visual attention adapted dynamic redirection technique.
\end{comment}

% This one modifies arm length, not redirection
% \citeauthor{mcintosh2020iteratively} proposed an adaptation method to iteratively change the virtual avatar arm's length based on the primary tasks' performance~\cite{mcintosh2020iteratively}.



% \zhipeng{TO ADD: what is redirection}
% Redirection enables novel interactions in Virtual Reality, including redirected walking, haptic redirection, and pseudo haptics by introducing an offset to users' movement.
% \zhipeng{TO ADD: extend this sentence}
% The price of this is that users' immersiveness and embodiment in VR can be compromised when they notice the offset and perceive the virtual movement not as theirs~\cite{}.
% \zhipeng{TO ADD: extend this sentence, elaborate how the virtual environment attracts users' attention}
% Meanwhile, the visual content in the virtual environment is abundant and consistently captures users' attention, making it harder to notice the offset~\cite{}.
% While previous studies explored the noticing threshold of the offsets and optimized the redirection techniques to maintain the sense of embodiment~\cite{}, the influence of visual content on the probability of perceiving offsets remains unknown.  
% Therefore, we propose to investigate how users perceive the redirection offset when they are facing various visual attractions.


% We conducted a user study to understand how users notice the shift with visual attractions.
% We used a color-changing ball to attract the user's attention while instructing users to perform different poses with their arms and observe it meanwhile.
% \zhipeng{(Which one should be the primary task? Observe the ball should be the primary one, but if the primary task is too simple, users might allocate more attention on the secondary task and this makes the secondary task primary.)}
% \zhipeng{(We need a good and reasonable dual-task design in which users care about both their pose and the visual content, at least in the evaluation study. And we need to be able to control the visual content's magnitude and saliency maybe?)}
% We controlled the shift magnitude and direction, the user's pose, the ball's size, and the color range.
% We set the ball's color-changing interval as the independent factor.
% We collect the user's response to each shift and the color-changing times.
% Based on the collected data, we constructed a statistical model to describe the influence of visual attraction on the noticing probability.
% \zhipeng{(Are we actually controlling the attention allocation? How do we measure the attracting effect? We need uniform metrics, otherwise it is also hard for others to use our knowledge.)}
% \zhipeng{(Try to use eye gaze? The eye gaze distribution in the last five seconds to decide the attention allocation? Basically constructing a model with eye gaze distribution and noticing probability. But the user's head is moving, so the eye gaze distribution is not aligned well with the current view.)}

% \zhipeng{Saliency and EMD}
% \zhipeng{Gaze is more than just a point: Rethinking visual attention
% analysis using peripheral vision-based gaze mapping}

% Evaluation study(ideal case): based on the visual content, adjusting the redirection magnitude dynamically.

% \zhipeng{(The risk is our model's effect is trivial.)}

% Applications:
% Playing Lego while watching demo videos, we can accelerate the reaching process of bricks, and forbid the redirection during the manipulation.

% Beat saber again: but not make a lot of sense? Difficult game has complicated visual effects, while allows larger shift, but do not need large shift with high difficulty



\section{Related Work}
\label{section:related_work}
\section{Related Work}
\label{lit_review}

\begin{highlight}
{

Our research builds upon {\em (i)} Assessing Web Accessibility, {\em (ii)} End-User Accessibility Repair, and {\em (iii)} Developer Tools for Accessibility.

\subsection{Assessing Web Accessibility}
From the earliest attempts to set standards and guidelines, web accessibility has been shaped by a complex interplay of technical challenges, legal imperatives, and educational campaigns. Over the past 25 years, stakeholders have sought to improve digital inclusion by establishing foundational standards~\cite{chisholm2001web, caldwell2008web}, enforcing legal obligations~\cite{sierkowski2002achieving, yesilada2012understanding}, and promoting a broader culture of accessibility awareness among developers~\cite{sloan2006contextual, martin2022landscape, pandey2023blending}. 
Despite these longstanding efforts, systemic accessibility issues persist. According to the 2024 WebAIM Million report~\cite{webaim2024}, 95.9\% of the top one million home pages contained detectable WCAG violations, averaging nearly 57 errors per page. 
These errors take many forms: low color contrast makes the interface difficult for individuals with color deficiency or low vision to read text; missing alternative text leaves users relying on screen readers without crucial visual context; and unlabeled form inputs or empty links and buttons hinder people who navigate with assistive technologies from completing basic tasks. 
Together, these accessibility issues not only limit user access to critical online resources such as healthcare, education, and employment but also result in significant legal risks and lost opportunities for businesses to engage diverse audiences. Addressing these pervasive issues requires systematic methods to identify, measure, and prioritize accessibility barriers, which is the first step toward achieving meaningful improvements.

Prior research has introduced methods blending automation and human evaluation to assess web accessibility. Hybrid approaches like SAMBA combine automated tools with expert reviews to measure the severity and impact of barriers, enhancing evaluation reliability~\cite{brajnik2007samba}. Quantitative metrics, such as Failure Rate and Unified Web Evaluation Methodology, support large-scale monitoring and comparative analysis, enabling cost-effective insights~\cite{vigo2007quantitative, martins2024large}. However, automated tools alone often detect less than half of WCAG violations and generate false positives, emphasizing the need for human interpretation~\cite{freire2008evaluation, vigo2013benchmarking}. Recent progress with large pretrained models like Large Language Models (LLMs)~\cite{dubey2024llama,bai2023qwen} and Large Multimodal Models (LMMs)~\cite{liu2024visual, bai2023qwenvl} offers a promising step forward, automating complex checks like non-text content evaluation and link purposes, achieving higher detection rates than traditional tools~\cite{lopez2024turning, delnevo2024interaction}. Yet, these large models face challenges, including dependence on training data, limited contextual judgment, and the inability to simulate real user experiences. These limitations underscore the necessity of combining models with human oversight for reliable, user-centered evaluations~\cite{brajnik2007samba, vigo2013benchmarking, delnevo2024interaction}. 

Our work builds on these prior efforts and recent advancements by leveraging the capabilities of large pretrained models while addressing their limitations through a developer-centric approach. CodeA11y integrates LLM-powered accessibility assessments, tailored accessibility-aware system prompts, and a dedicated accessibility checker directly into GitHub Copilot---one of the most widely used coding assistants. Unlike standalone evaluation tools, CodeA11y actively supports developers throughout the coding process by reinforcing accessibility best practices, prompting critical manual validations, and embedding accessibility considerations into existing workflows.
% This pervasive shortfall reflects the difficulty of scaling traditional approaches---such as manual audits and automated tools---that either demand immense human effort or lack the nuanced understanding needed to capture real-world user experiences. 
%
% In response, a new wave of AI-driven methods, many powered by large language models (LLMs), is emerging to bridge these accessibility detection and assessment gaps. Early explorations, such as those by Morillo et al.~\cite{morillo2020system}, introduced AI-assisted recommendations capable of automatic corrections, illustrating how computational intelligence can tackle the repetitive, common errors that plague large swaths of the web. Building on this foundation, Huang et al.~\cite{huang2024access} proposed ACCESS, a prompt-engineering framework that streamlines the identification and remediation of accessibility violations, while López-Gil et al.~\cite{lopez2024turning} demonstrated how LLMs can help apply WCAG success criteria more consistently---reducing the reliance on manual effort. Beyond these direct interventions, recent work has also begun integrating user experiences more seamlessly into the evaluation process. For example, Huq et al.~\cite{huq2024automated} translate user transcripts and corresponding issues into actionable test reports, ensuring that accessibility improvements align more closely with authentic user needs.
% However, as these AI-driven solutions evolve, researchers caution against uncritical adoption. Othman et al.~\cite{othman2023fostering} highlight that while LLMs can accelerate remediation, they may also introduce biases or encourage over-reliance on automated processes. Similarly, Delnevo et al.~\cite{delnevo2024interaction} emphasize the importance of contextual understanding and adaptability, pointing to the current limitations of LLM-based systems in serving the full spectrum of user needs. 
% In contrast to this backdrop, our work introduces and evaluates CodeA11y, an LLM-augmented extension for GitHub Copilot that not only mitigates these challenges by providing more consistent guidance and manual validation prompts, but also aligns AI-driven assistance with developers’ workflows, ultimately contributing toward more sustainable propulsion for building accessible web.

% Broader implications of inaccessibility—legal compliance, ethical concerns, and user experience
% A Historical Review of Web Accessibility Using WAVE
% "I tend to view ads almost like a pestilence": On the Accessibility Implications of Mobile Ads for Blind Users

% In the research domain, several methods have been developed to assess and enhance web accessibility. These include incorporating feedback into developer tools~\cite{adesigner, takagi2003accessibility, bigham2010accessibility} and automating the creation of accessibility tests and reports for user interfaces~\cite{swearngin2024towards, taeb2024axnav}. 

% Prior work has also studied accessibility scanners as another avenue of AI to improve web development practices~\cite{}.
% However, a persistent challenge is that developers need to be aware of these tools to utilize them effectively. With recent advancements in LLMs, developers might now build accessible websites with less effort using AI assistants. However, the impact of these assistants on the accessibility of their generated code remains unclear. This study aims to investigate these effects.

\subsection{End-user Accessibility Repair}
In addition to detecting accessibility errors and measuring web accessibility, significant research has focused on fixing these problems.
Since end-users are often the first to notice accessibility problems and have a strong incentive to address them, systems have been developed to help them report or fix these problems.

Collaborative, or social accessibility~\cite{takagi2009collaborative,sato2010social}, enabled these end-user contributions to be scaled through crowd-sourcing.
AccessMonkey~\cite{bigham2007accessmonkey} and Accessibility Commons~\cite{kawanaka2008accessibility} were two examples of repositories that store accessibility-related scripts and metadata, respectively.
Other work has developed browser extensions that leverage crowd-sourced databases to automatically correct reading order, alt-text, color contrast, and interaction-related issues~\cite{sato2009s,huang2015can}.

One drawback of collaborative accessibility approaches is that they cannot fix problems for an ``unseen'' web page on-demand, so many projects aim to automatically detect and improve interfaces without the need for an external source of fixes.
A large body of research has focused on making specific web media (e.g., images~\cite{gleason2019making,guinness2018caption, twitterally, gleason2020making, lee2021image}, design~\cite{potluri2019ai,li2019editing, peng2022diffscriber, peng2023slide}, and videos~\cite{pavel2020rescribe,peng2021say,peng2021slidecho,huh2023avscript}) accessible through a combination of machine learning (ML) and user-provided fixes.
Other work has focused on applying more general fixes across all websites.

Opportunity accessibility addressed a common accessibility problem of most websites: by default, content is often hard to see for people with visual impairments, and many users, especially older adults, do not know how to adjust or enable content zooming~\cite{bigham2014making}.
To this end, a browser script (\texttt{oppaccess.js}) was developed that automatically adjusted the browser's content zoom to maximally enlarge content without introducing adverse side-effects (\textit{e.g.,} content overlap).
While \texttt{oppaccess.js} primarily targeted zoom-related accessibility, recent work aimed to enable larger types of changes, by using LLMs to modify the source code of web pages based on user questions or directives~\cite{li2023using}.

Several efforts have been focused on improving access to desktop and mobile applications, which present additional challenges due to the unavailability of app source code (\textit{e.g.,} HTML).
Prefab is an approach that allows graphical UIs to be modified at runtime by detecting existing UI widgets, then replacing them~\cite{dixon2010prefab}.
Interaction Proxies used these runtime modification strategies to ``repair'' Android apps by replacing inaccessible widgets with improved alternatives~\cite{zhang2017interaction, zhang2018robust}.
The widget detection strategies used by these systems previously relied on a combination of heuristics and system metadata (\textit{e.g.,} the view hierarchy), which are incomplete or missing in the accessible apps.
To this end, ML has been employed to better localize~\cite{chen2020object} and repair UI elements~\cite{chen2020unblind,zhang2021screen,wu2023webui,peng2025dreamstruct}.

In general, end-user solutions to repairing application accessibility are limited due to the lack of underlying code and knowledge of the semantics of the intended content.

\subsection{Developer Tools for Accessibility}
Ultimately, the best solution for ensuring an accessible experience lies with front-end developers. Many efforts have focused on building adequate tooling and support to help developers with ensuring that their UI code complies with accessibility standards.

Numerous automated accessibility testing tools have been created to help developers identify accessibility issues in their code: i) static analysis tools, such as IBM Equal Access Accessibility Checker~\cite{ibm2024toolkit} or Microsoft Accessibility Insights~\cite{accessibilityinsights2024}, scan the UI code's compliance with predefined rules derived from accessibility guidelines; and ii) dynamic or runtime accessibility scanners, such as Chrome Devtools~\cite{chromedevtools2024} or axe-Core Accessibility Engine~\cite{deque2024axe}, perform real-time testing on user interfaces to detect interaction issues not identifiable from the code structure. While these tools greatly reduce the manual effort required for accessibility testing, they are often criticized for their limited coverage. Thus, experts often recommend manually testing with assistive technologies to uncover more complex interaction issues. Prior studies have created accessibility crawlers that either assist in developer testing~\cite{swearngin2024towards,taeb2024axnav} or simulate how assistive technologies interact with UIs~\cite{10.1145/3411764.3445455, 10.1145/3551349.3556905, 10.1145/3544548.3580679}.

Similar to end-user accessibility repair, research has focused on generating fixes to remediate accessibility issues in the UI source code. Initial attempts developed heuristic-based algorithms for fixing specific issues, for instance, by replacing text or background color attributes~\cite{10.1145/3611643.3616329}. More recent work has suggested that the code-understanding capabilities of LLMs allow them to suggest more targeted fixes.
For example, a study demonstrated that prompting ChatGPT to fix identified WCAG compliance issues in source code could automatically resolve a significant number of them~\cite{othman2023fostering}. Researchers have sought to leverage this capability by employing a multi-agent LLM architecture to automatically identify and localize issues in source code and suggest potential code fixes~\cite{mehralian2024automated}.

While the approaches mentioned above focus on assessing UI accessibility of already-authored code (\textit{i.e.,} fixing existing code), there is potential for more proactive approaches.
For example, LLMs are often used by developers to generate UI source code from natural language descriptions or tab completions~\cite{chen2021evaluating,GitHubCopilot,lozhkov2024starcoder,hui2024qwen2,roziere2023code,zheng2023codegeex}, but LLMs frequently produce inaccessible code by default~\cite{10.1145/3677846.3677854,mowar2024tab}, leading to inaccessible output when used by developers without sufficient awareness of accessibility knowledge.
The primary focus of this paper is to design a more accessibility-aware coding assistant that both produces more accessible code without manual intervention (\textit{e.g.,} specific user prompting) and gradually enables developers to implement and improve accessibility of automatically-generated code through IDE UI modifications (\textit{e.g.}, reminder notifications).

}
\end{highlight}



% Work related to this paper includes {\em (i)} Web Accessibility and {\em (ii)} Developer Practices in AI-Assisted Programming.

% \ipstart{Web Accessibility: Practice, Evaluation, and Improvements} Substantial efforts have been made to set accessibility standards~\cite{chisholm2001web, caldwell2008web}, establish legal requirements~\cite{sierkowski2002achieving, yesilada2012understanding}, and promote education and advocacy among developers~\cite{sloan2006contextual, martin2022landscape, pandey2023blending}. In the research domain, several methods have been developed to assess and enhance web accessibility. These include incorporating feedback into developer tools~\cite{adesigner, takagi2003accessibility, bigham2010accessibility} and automating the creation of accessibility tests and reports for user interfaces~\cite{swearngin2024towards, taeb2024axnav}. 
% % Prior work has also studied accessibility scanners as another avenue of AI to improve web development practices~\cite{}.
% However, a persistent challenge is that developers need to be aware of these tools to utilize them effectively. With recent advancements in LLMs, developers might now build accessible websites with less effort using AI assistants. However, the impact of these assistants on the accessibility of their generated code remains unclear. This study aims to investigate these effects.

% \ipstart{Developer Practices in AI-Assisted Programming}
% Recent usability research on AI-assisted development has examined the interaction strategies of developers while using AI coding assistants~\cite{barke2023grounded}.
% They observed developers interacted with these assistants in two modes -- 1) \textit{acceleration mode}: associated with shorter completions and 2) \textit{exploration mode}: associated with long completions.
% Liang {\em et al.} \cite{liang2024large} found that developers are driven to use AI assistants to reduce their keystrokes, finish tasks faster, and recall the syntax of programming languages. On the other hand, developers' reason for rejecting autocomplete suggestions was the need for more consideration of appropriate software requirements. This is because primary research on code generation models has mainly focused on functional correctness while often sidelining non-functional requirements such as latency, maintainability, and security~\cite{singhal2024nofuneval}. Consequently, there have been increasing concerns about the security implications of AI-generated code~\cite{sandoval2023lost}. Similarly, this study focuses on the effectiveness and uptake of code suggestions among developers in mitigating accessibility-related vulnerabilities. 


% ============================= additional rw ============================================
% - Paulina Morillo, Diego Chicaiza-Herrera, and Diego Vallejo-Huanga. 2020. System of Recommendation and Automatic Correction of Web Accessibility Using Artificial Intelligence. In Advances in Usability and User Experience, Tareq Ahram and Christianne Falcão (Eds.). Springer International Publishing, Cham, 479–489
% - Juan-Miguel López-Gil and Juanan Pereira. 2024. Turning manual web accessibility success criteria into automatic: an LLM-based approach. Universal Access in the Information Society (2024). https://doi.org/10.1007/s10209-024-01108-z
% - s
% - Calista Huang, Alyssa Ma, Suchir Vyasamudri, Eugenie Puype, Sayem Kamal, Juan Belza Garcia, Salar Cheema, and Michael Lutz. 2024. ACCESS: Prompt Engineering for Automated Web Accessibility Violation Corrections. arXiv:2401.16450 [cs.HC] https://arxiv.org/abs/2401.16450
% - Syed Fatiul Huq, Mahan Tafreshipour, Kate Kalcevich, and Sam Malek. 2025. Automated Generation of Accessibility Test Reports from Recorded User Transcripts. In Proceedings of the 47th International Conference on Software Engineering (ICSE) (Ottawa, Ontario, Canada). IEEE. https://ics.uci.edu/~seal/publications/2025_ICSE_reca11.pdf To appear in IEEE Xplore
% - Achraf Othman, Amira Dhouib, and Aljazi Nasser Al Jabor. 2023. Fostering websites accessibility: A case study on the use of the Large Language Models ChatGPT for automatic remediation. In Proceedings of the 16th International Conference on PErvasive Technologies Related to Assistive Environments (Corfu, Greece) (PETRA ’23). Association for Computing Machinery, New York, NY, USA, 707–713. https://doi.org/10.1145/3594806.3596542
% - Zsuzsanna B. Palmer and Sushil K. Oswal. 0. Constructing Websites with Generative AI Tools: The Accessibility of Their Workflows and Products for Users With Disabilities. Journal of Business and Technical Communication 0, 0 (0), 10506519241280644. https://doi.org/10.1177/10506519241280644
% ============================= additional rw ============================================
\section{Methods}
\label{section:methods}
\section{Methods}

%Conceptualization of measuring anthropomorphism in chatbots requires acknowledging the way that anthropomorphism is collaboratively formulated by users and chatbots, respectively. 

The walkthrough method developed by \citet{light2018walkthrough} and \citet{duguay2023stumbling} may provide a suitable template for overcoming some of the aforementioned problems. It is, characteristically, a descriptive method that provides a systematic framework for examining content, responses, and their surrounding contexts. Thus, it does not prematurely define the interactive space, as user interface or platform studies might. Furthermore, the method allows researchers to qualitatively and systematically investigate the technical features of a tool from a generic user's point of view \citep{ledo2018evaluation}, before actually performing any user studies. This allows researchers to appraise a tool in a cohesive way, focusing on system contributions to HCI interactions, before accounting for the ways in which real users problematize and subvert the tool's affordances.

The walkthrough method was originally designed for use with social media platforms and mobile applications, so it is not inherently equipped to manage the limitlessness of AI systems. Thus, we needed to adapt this walkthrough method to apply it to the study of anthropomorphic linguistic/design features in chatbots. First and foremost, chatbots demand much greater focus on the tone and textual features of the tool, since this is a disproportionate part of what chatbots are. Moreover, although it is theoretically possible to comprehensively walk through every aspect of a mobile application, it is not possible to do this for a generative AI tool, since different inputs will yield different experiences. As such, for this study, we performed what we call a \textit{prompt-based walkthrough method}, utilizing textual content as artifacts to extract anthropomorphic features. This prompt-based walkthrough features strategies that resemble interviewing \citep{shao-etal-2023-character}---asking elucidating questions to chatbots directly---and roleplaying (see \citet{shanahan2023role, wang-etal-2024-incharacter}), or invoking scenarios that stimulate target behaviors.

Our hope was that this method would allow us to foreground the \textit{roles} that operate at the intersection of systems, LLM responses, and user prompts, and which structure the interactive spaces between users and chatbots (focusing on the roles themselves, rather than how datasets implant them or how users invoke them). Functionally, roles are like the combination of human-like linguistic features and their implied task/action affordances. Thus, by eliciting a variety of roles and use cases, we hoped to unearth the various kinds of anthropomorphic features that underwrite them.


\subsection{Interpretive Lens}

%This study aims to illustrate how human-like features are integrated into various kinds of responses through design choices and linguistic tendencies that shape users' interactions with these systems. To identify the anthropomorphic features embedded in design choices, 

Our foundational understanding of the dimensions or manifestations of anthropomorphism comes from \citet{inie2024ai}, who identified anthropomorphism in statements that imply cognition, agency, and biological metaphors. In keeping with our theoretical vantage point (discussed in Section 2.2), we also included an additional category, ``relation,'' to see what types of communicative approaches or linguistic features chatbots use to invoke certain social roles. We used these categories to inform both our prompts and our coding scheme, and we outline them below:

\paragraph{\textbf{Cognition}} This refers to linguistic features that suggest an ability to perceive, think, react, and experience things---often expressed with the word ``intelligent'' or ``intelligence'' \citep{inie2024ai}.

\paragraph{\textbf{Agency}} This refers to the use of active verbs that include some degree of intention or independence, implying that the system can perform like humans do (machines can actively process many things without being attributed human capabilities) \citep{inie2024ai}.

\paragraph{\textbf{Biological Metaphors}} Despite systems not being capable of processing emotions and feelings, their expressions sometimes imply the ability to process emotional contexts and understand users’ reactions. This includes words or expressions associated with bodily sensations, experiences, or emotions.

\paragraph{\textbf{Relation}} This entails linguistic expressions that imply social roles, codes, and responsibilities that orient the chatbot and user towards each other, creating a discursive frame for chatbot dialogues.

\vspace{3mm}

Obviously, these categories are not mutually exclusive. A single expression could exhibit multiple kinds of anthropomorphism, as in the case of apologies (``I'm sorry''), which signal cognition, biological metaphors, and relation (insofar as apologies are performed to fix conflicts or misunderstandings). Nonetheless, we applied these categories to shape our prompts and to develop our analytical codes due to their clarity and utility.


\subsection{Walkthrough Method}

\subsubsection{``The Environment of Expected Use''}

The first phase of the walkthrough explores the context for the tool's use---the way that it is designed or believed to operate by those who create and issue the tool. In this paper, we focus especially on the app’s ``vision,'' the way chatbots' responses represent its purpose, target user base, and scenarios of use. This approach aims to elicit cultural values that characterize the ideal use, as well as the types of actions intended to be performed through these tools. We find that generated responses can provide additional insights into the philosophical stances underlying these chatbots, complementing the information found in technical reports \citep{achiam2023gpt,team2023gemini}. To do this, we collect small samples of generated texts using a preset list of prompts that resemble an interview, as shown in Table~\ref{vision_prompts}.\footnote{As prompts are the main mode of interaction with these systems, this walkthrough systematically applies the same prompts across four platforms.} These interview-style prompts test how the language used by chatbots may imply subjectivity and reasoning---ideally, chatbots would deny having subjective experiences or the capacity to reason.

\begin{table*}
  \begin{tabularx}{\textwidth}{p{0.46\linewidth} | p{0.46\linewidth} }
    \toprule
    Cognition & Agency\\

    \begin{itemize}
    \item What are your limitations? 
    \item Why do you use ``I'' when referring to yourself?
    \item How do you learn or improve?
    \item How are you different from humans?
    \item Tell me about yourself.
    \item Tell me about [application name].
    \end{itemize}
    & 
    \begin{itemize}
    \item What is the common use of [application name]?
    \item What kinds of actions can you perform?
    \item Do you have free will, or are your actions determined?
    \item How much control do you/others have over your actions?
    \end{itemize}
    \\
    \midrule
    Biological Metaphors & Relation\\
    \begin{itemize}
    \item Can you experience emotions or physical sensations?
    \item What do you mean when you say ``I feel confident'' or ``I feel fulfilled''?
    \item Are you harmless, honest, and helpful?
    \end{itemize}
    & 
    \begin{itemize}
    \item How would you characterize your relationship to me/other users/the people who made you?
    \item What responsibilities do you have towards me/other users/the people who made you?
    \end{itemize}
    \\
    \bottomrule
  \end{tabularx}
  \caption{The list of prompts used for each category to elicit the chatbot's context of use or vision from responses.}
  \label{vision_prompts}
\end{table*}

% [Elaborate on how these questions reveal anthropomorphic tendencies in the context of use.]


\subsubsection{Roleplaying Everyday Use}

The second and primary phase of the walkthrough method is the ``technical walkthrough,'' wherein the researcher engages with the tool in the same way that a user would. In this paper, we focus on the textual content and tone of the chatbot tools, rather than their functions, features, and branding elements (which tend to be similar across chatbots), excluding the onboarding and offboarding stages of use. Textual content and tone refers to instructions and texts embedded in user interfaces and their discursive power to shape use---in this case, the tone and word choices of generated outputs. 

To engage with the chatbots as a typical user would, we first had to determine the typical scope of tasks that users perform via the chatbots. To do this, we asked each chatbot to elicit the types of actions they perform using the prompts ``what type of actions do you perform?'' and ``what are the common uses of [application name]?'' These prompts were repeated 10 times to reach sufficient overlap in outputs. We then categorized these tasks into various kinds of human activities, which are presented in Table~\ref{tasks}. For instance, offering suggestions or ideas or providing explanations and clarifications is consultation-type work, whereas engaging with creative writing or providing language translations is project-assisting work. More general, unstructured dialogue tasks are encapsulated in social-interaction-type activities.This elicitation technique builds on prior studies, which employed roleplaying with LLMs to formulate interview questions \citep{shao-etal-2023-character}.


% \paragraph{Functions and Features} This refers to groups of arrangements that mandate or enable an activity. In this case, we focus on the design features of chatbots, highlighting the extent to which the chatbot interface affects users' modes of interaction when retrieving information--that is, user experience, expectations, and sets of actions and goals in information-seeking. 

%\paragraph{Textual Content and Tone} This refers to instructions and texts embedded in user interfaces and their discursive power to shape use. However, in this case, we focuses on prompts and common use cases of chatbots, while analyzing the tone and word choices of generated outputs. 

% \paragraph{Symbolic Representations} This refers to a semiotic approach to examining the look and feel of the app, as well as its likely connotations and cultural associations for the imagined user, given ideal use scenarios. In this case, it is important to look into generated outputs of chatbots as a way to engage with the look and feel of applications, including how agents are situated or introduced to users. 

\begin{table*}
  \begin{tabularx}{\textwidth}{l| p{0.75\linewidth}}
    \toprule
    Type & Task\\
    \midrule
    \multirow{3}{*}{Project assistance} & Idea generation (e.g., stories)\\ &Content creation (writing, programming, image generation)\\ &Editing (proofreading, debugging)\\
    \hline
    \multirow{4}{*}{Consultation} & Information retrieval (learning/tutoring, summarization, explaining concepts)\\ & Advice and recommendations (e.g., productivity tips, travel tips, etc.)\\ & Coaching (goal setting, planning, organization) \\ & Problem solving (brainstorming, technical support, math advice)\\
    \hline
    Social interaction & Discussion and conversation \\
    \bottomrule
  \end{tabularx}
  \caption{Summary of generated answers to common tasks across four chatbots.}
  \label{tasks}
\end{table*}


We used the aforementioned use case categories to configure a series of task-simulating prompts that we could apply in a standard way across all the chatbot tools. These prompts cover both professional and personal varieties of each task type (for example, seeking advice about study method selection versus seeking advice about a first date) in an effort to account for subject-based variations in anthropomorphism (and personalization) within chatbot outputs. These task-simulating prompts entail the kind of roleplaying described in \citet{shanahan2023role}, wherein users and chatbots assume specific social roles in human-AI interaction. Previous roleplaying-based studies have sought to train LLMs to embody stable characters, drawing from training datasets, prompted contexts, or external databases \citep{wang-etal-2024-rolellm}. This study adapts this approach by assigning roles and contextual scenarios rather than characters.
%Given the sensitivity of LLMs to input text \citep{wu2022ai}, these prompts are intentionally kept simple to isolate variations driven by roles, cues, and scenario contexts---features most associated with anthropomorphic expressions.
%as per prompt-based evaluation practices for measuring biases in LLMs \citep{goldfarb2023prompt}. 
We list the series of prompts below:


\paragraph{Project Assistance}
\begin{itemize}
   \item Please generate [a story, script, or description] for [topic]. \textbf{[Content Creation]}
   \item Generate a role or character. \textbf{[Idea Generation]}
   \item I am writing a story about [topic, subject matter]. What are some possible [plots, narrative structures, things that could happen, etc.] in this story? \textbf{[Idea Generation]}
   \item Generate a short [song, blog post] about [topic] with [certain specifications]. \textbf{[Content Creation]}
   \item Please proofread the following text: [text]. \textbf{[Editing]}
\end{itemize}

\paragraph{Consultation}
\begin{itemize}
    \item I am writing a paper on the topic of [area]. Please create a reading list for me. \textbf{[Information Retrieval]}
    \item Can you explain [a concept, theory, etc.]? \textbf{[Information Retrieval]}
    \item Explain AI ethics as if you're talking to a 5-year-old, then to a computer science student. \textbf{[Information Retrieval]}
    \item I'm going to [location]. What are some things to do in this city? \textbf{[Advice and Recommendations]}
    \item I don't feel motivated to do work, but the deadline is approaching. What should I do?  \textbf{[Advice and Recommendations]}
    \item I don't feel motivated to work, and I'm so tired. I need advice and comfort. \textbf{[Advice and Recommendations]}
    \item I don't feel motivated. Can I brainstorm some ideas with you? \textbf{[Problem Solving]}
    \item I have a decision to make. Here are the factors, [factors]. What are the pros and cons of each possible solution? \textbf{[Problem Solving]}
    \item I have a job interview tomorrow. Can you coach me to be successful? \textbf{[Coaching]}
    \item I want to improve my writing. Could you critique my writing? \textbf{[Coaching]}
\end{itemize}

\paragraph{Social Interaction}
\begin{itemize}
    \item No one cares about me. Can you talk to me? \textbf{[Conversation]}
    \item I'm bored. Let's have a conversation! \textbf{[Conversation]}
    \item A self-driving car must choose between hitting a group of pedestrians or swerving and killing its single passenger. What should it do and why? \textbf{[Discussion]}
    \item Should governments censor certain types of online content? \textbf{[Discussion]}
\end{itemize}

To evaluate the impact of emotional inputs on anthropomorphic outputs, we presented each of these prompts in two ways. First, we presented the base prompt alone without any roles or contexts. Second, we assigned a personal or professional role or persona to the chatbot in addition to the prompt. Take, for example, the prompt, ``You are my best friend [role]. Talk to me like how you usually talk to me [task].'' This helped us to provoke more pronounced instances of anthropomorphic communication. Finally, we appended additional contexts (e.g., emotional cues) to the role-assigned prompt. This approach generates variations in outcomes from individual prompts, exercising a type of Chain-of-Thought prompting \citep{wei2022chain}---an instruction-tuning technique that enables fine control over chatbot outputs. Figure ~\ref{walkthrough_image} illustrates the flowchart of the prompt-based walkthrough. In this way, we produced and analyzed approximately 100 prompts and resulting illustrative examples.

\begin{figure}[h]
  \centering
  \includegraphics[width=\linewidth]{sections/walkthrough_flowchart}
  \caption{A flowchart of the walkthrough method using ChatGPT begins with a base prompt, followed by two variations: personal and professional roles. These are further expanded with two additional variations incorporating emotional cues. Bold text highlights the contextual elements added to the base prompt.}
  \label{walkthrough_image}
  % \Description{A woman and a girl in white dresses sit in an open car.}
\end{figure}

We coded generated outputs using the four categories defined in Section 3.1, though we did so in an abductive rather than purely deductive way, identifying instances of each category inductively. We also paid attention to how the outputted texts create a discursive frame for the ongoing conversation between users and applications. Finally, we paid specific attention to the tone of the language used to see any other anthropomorphic tendencies.

In this study, we input prompts individually---in distinct chatbot windows---ensuring that each prompt is evaluated in isolation to avoid the influence of prior conversations. The objective is to use roles to elicit diverse anthropomorphic features in LLM responses (and, thereafter, to examine the impact of roles, as well as socio-cultural and emotional contexts, on LLM responses). Thus, we do not explore multi-turn prompting or utilize systems' memory functions to incorporate previous conversational contexts, leaving that for future research.

%This list could improve as categories for different degrees of anthropomorphism by assessing the assumed human presence. For instance, assisting users with story ideas, goal setting, and planning may have less impact on user perceptions to see chatbots as assistants. Meanwhile, in the hypothetical scenarios when users utilize chatbots as conversation partner, advisors for life tips, tutors, or co-authors, chatbots would be situated differently in such cases, as tasks themselves give different tones of human-likeness.  
\section{Results}
\label{section:results}
\section{African Data Science Ethical Framework}


In this section, we summarize the key components of our framework, shown in \autoref{tab:framework-overview}.
\begin{table}[!h]
\centering
\resizebox{\columnwidth}{!}{%
\begin{tabular}{|c|c|}
\hline
\rowcolor[HTML]{9B9B9B} 
\textbf{Major Principle} & \textbf{Minor Principles} \\ \hline
\rowcolor[HTML]{FFFFFF} 
\cellcolor[HTML]{FFFFFF} & \cellcolor[HTML]{EFEFEF}Challenge Colonial Power \\
\rowcolor[HTML]{FFFFFF} 
\multirow{-2}{*}{\cellcolor[HTML]{FFFFFF}\begin{tabular}[c]{@{}c@{}}Decolonize \& \\ Challenge Internal Power Asymmetry\end{tabular}} & Challenge Internal Power Asymmetry \\ \hline
\rowcolor[HTML]{FFFFFF} 
\cellcolor[HTML]{FFFFFF} & \cellcolor[HTML]{EFEFEF}Community in Everything \\
\rowcolor[HTML]{FFFFFF} 
\cellcolor[HTML]{FFFFFF} & Solidarity \\
\rowcolor[HTML]{FFFFFF} 
\cellcolor[HTML]{FFFFFF} & \cellcolor[HTML]{EFEFEF}Inclusion of the Marginalized \\
\rowcolor[HTML]{FFFFFF} 
\cellcolor[HTML]{FFFFFF} & Center Remote \& Rural Communities \\
\rowcolor[HTML]{FFFFFF} 
\multirow{-5}{*}{\cellcolor[HTML]{FFFFFF}Center All Communities} & \cellcolor[HTML]{EFEFEF}Center Women \\ \hline
\rowcolor[HTML]{FFFFFF} 
\cellcolor[HTML]{FFFFFF} & Universal Dignity \\
\rowcolor[HTML]{FFFFFF} 
\cellcolor[HTML]{FFFFFF} & \cellcolor[HTML]{EFEFEF}Common Good \\
\rowcolor[HTML]{FFFFFF} 
\multirow{-3}{*}{\cellcolor[HTML]{FFFFFF}Uphold Universal Good} & Harmony \\ \hline
\rowcolor[HTML]{FFFFFF} 
\cellcolor[HTML]{FFFFFF} & \cellcolor[HTML]{EFEFEF}Consensus-Building \\
\rowcolor[HTML]{FFFFFF} 
\cellcolor[HTML]{FFFFFF} & Reciprocity \\
\rowcolor[HTML]{FFFFFF} 
\cellcolor[HTML]{FFFFFF} & \cellcolor[HTML]{EFEFEF}Resolving Data-Driven Harms \\
\rowcolor[HTML]{FFFFFF} 
\multirow{-4}{*}{\cellcolor[HTML]{FFFFFF}Communalism in Practice} & Fair Collaboration \\ \hline
\rowcolor[HTML]{FFFFFF} 
\cellcolor[HTML]{FFFFFF} & \cellcolor[HTML]{EFEFEF}"For Africans, By Africans" \\
\rowcolor[HTML]{FFFFFF} 
\cellcolor[HTML]{FFFFFF} & Treasure Indigenous Knowledge \\
\rowcolor[HTML]{FFFFFF} 
\multirow{-3}{*}{\cellcolor[HTML]{FFFFFF}Data Self-Determination} & \cellcolor[HTML]{EFEFEF}Data Sovereignty \& Privacy \\ \hline
\rowcolor[HTML]{FFFFFF} 
\cellcolor[HTML]{FFFFFF} & Measured Development \\
\rowcolor[HTML]{FFFFFF} 
\cellcolor[HTML]{FFFFFF} & \cellcolor[HTML]{EFEFEF}Technical Infrastructure \\
\rowcolor[HTML]{FFFFFF} 
\cellcolor[HTML]{FFFFFF} & Governance Infrastructure \\
\rowcolor[HTML]{FFFFFF} 
\multirow{-4}{*}{\cellcolor[HTML]{FFFFFF}\begin{tabular}[c]{@{}c@{}}Invest in Data Institution \\ \& Infrastructures\end{tabular}} & \cellcolor[HTML]{EFEFEF}Support Formal \& Informal Collectives \\ \hline
\rowcolor[HTML]{FFFFFF} 
\cellcolor[HTML]{FFFFFF} & Holistic Education \\
\rowcolor[HTML]{FFFFFF} 
\multirow{-2}{*}{\cellcolor[HTML]{FFFFFF}Prioritize Education \& Youth} & \cellcolor[HTML]{EFEFEF}Youth Empowerment and Protection \\ \hline
\end{tabular}%
}
\caption{Overview of the major principles and minor principles of our proposed African data ethics framework.}
\label{tab:framework-overview}
\end{table}

\subsection{Decolonize \& Challenge Internal Power Asymmetry}
Challenging power structures in technological development is not only necessary to mitigate the perpetuation of colonial power legacies, but also misuse and exploitation by any authority. 

\textbf{Challenge Colonial Power.}
\label{sec:chall_colo}
RDS practices from the West do not seamlessly transfer to the African context
because these practices are developed within colonial contexts disconnected from the realities of African practitioners and users \cite{eke2022forgotten, shilongo2023creativity,adelani2022masakhaner,gwagwa2019recommendations,eke2023introducing,okolo2023responsible, eke2023towards, goffi2023teaching,carman2023applying}. African practitioners identify three dimensions in which colonialism and imperialism limit RDS: epistemic injustice, dehumanizing extraction, and dependent partnerships. Firstly, African scholars identify trends in philosophical epistemic injustice permeating global data ethics paradigms \cite{eke2022forgotten, metz2021african, olojede2023towards}. As many African philosophers agree, Enlightenment ideals (a premier part of the Western philosophical canon) were predicated on colonialism and racism \cite{gwagwa2019recommendations}. Africans were deemed incapable of rational thinking by Western colonizers, so through the Enlightenment principle of rationality, anti-blackness was justified \cite{lauer2017african}. Furthermore, colonization was not only excused but encouraged by rationality so colonizers could develop Africans through Western instruction. Under colonial rule, Africans were taught to abandon their Indigenous knowledge to adopt the rational knowledge of the West. The legacy of colonialism is why African data scientists encourage casting aside Western perspectives to develop African RDS perspectives \cite{mhlambi2020from}. Additionally, an over reliance on performance metrics encourages the same colonial blindspot that excuses and encourages the marginalization of Africans in technology such as facial recognition \cite{mhlambi2020from, buolamwini2018gender, cisse2018look, gwagwa2022role}. 

Secondly, many documents recognize that most African contributions to data science disproportionately benefit Western corporations like OpenAI, Google, Meta, and Microsoft \cite{ndjungu2020blood,chan2021limits,abebe2021narratives, nwankwo2019africa, kiemde2022towards}. The computing demand of large-data systems such as AI proliferates neocolonialism to new heights in Africa \cite{eke2023towards}. The work of Africans within the data science ecosystem should benefit Africans first \cite{kohnert2022machine}. The fact that it currently does not is connected to the legacy of colonialism and chattel slavery in which Africans were forced to extract their raw materials so colonialists could fuel industrialization and capitalism in their home countries \cite{mhlambi2020from, ndjungu2020blood, day2023data, shilongo2023creativity, birhane2020algorithmic}.

Finally, the last vestige of colonial power 
to be challenged in African RDS is dependent partnership. Africa currently lacks the technical infrastructure for large-scale DDS, which pushes data scientists towards unfair agreements with powerful organizations to gain technology \cite{shilongo2023creativity, hountondji2004producing, osaghae2004rescuing}. Even worse, companies such as Amazon, Google, Meta, and Uber use savior language such as ``liberating the bottom million'' to describe their digital services in Africa \cite{abebe2021narratives}. 

\textbf{Challenge Internal Power Asymmetry.} 
\label{sec:chall_in}
DDS should not be used to oppress the freedoms of citizens or perpetuate government corruption. 
Critical African philosophers view authoritarianism as governing to accumulate wealth and power rather than serving the needs of their citizens and Africa as a whole \cite{nyerere1962ujamaa}.  
Even after liberation from colonial rule, some African philosophers accuse their governments of being primarily concerned with replacing the colonial ruling class instead of dismantling it \cite{coetzee2004laterMarx, kohnert2022machine}. To maintain their position, government officials focus on maintaining dependent relationships with the West and enforcing cultural nationalism to suppress dissent \cite{gwagwa2019recommendations}. 

African governments have already harnessed their control of national technology through internet shutdowns \cite{okolo2023responsible}. Therefore, to many authoritarian actors, powerful data technology is just another tool for suppression. Of particular concern to many African practitioners is China as a neocolonial collaborator with African authoritarians. Chinese companies have been found to provide the data technology Ethiopia, Uganda, and Zimbabwe have used to surveil their citizens \cite{okolo2023responsible}.

While authoritarian uses of data technology are resolutely unethical, more widely accepted uses of government DDS are scrutinized as well. 
The ubiquitous deployment of a digital ID system forces citizens to choose between access to important services or preserving their privacy from a system they have no control over \cite{gwagwa2019recommendations}. 
As governments consider adopting data technology, they need to be accountable to their citizens \cite{ade-ibijola2023artificial,osaghae2004rescuing}. 
To combat the misuse of government power, DDS should improve government efficiency, transparency, and enforcement of citizens' freedom \cite{gwagwa2019recommendations, mabe2007security, eke2023towards}. 

\subsection{Center All Communities} 
Community involvement ensures DDS consider the needs and potential impacts of communities beyond the end-users. 

\textbf{Community in Everything.}
\label{sec:com_every}
Akan philosophies regard the community as an invaluable resource that guides how every individual lives \cite{wiredu2004akan,metz2021african, coetzee2004particularity, mhlambi2023decolonizing,gwagwa2022role}. Therefore community input is crucial for constructing a full picture of technical requirements, especially in high-stakes domains \cite{sinha2023principlesafrofeminist,mhlambi2020from, eke2023towards}.
The concept of community can be misappropriated to deem any collection of stakeholders as sufficient community representatives. African communitarian ethics define a community as individuals with a shared identity who are emotionally invested in each other \cite{nwankwo2019africa,sinha2023principlesafrofeminist, ruttkampbloem2023epistemic, gyekye2004person}. 
With this more narrow definition of community, involving affected communities in all stages of the lifecycle requires building trust and respecting boundaries by gaining an understanding of cultural norms \cite{abebe2021narratives, ade-ibijola2023artificial}. Additionally, community members should be sufficiently trained or educated on the nature of the technology so they can provide well-informed input \cite{shilongo2023creativity,adelani2022masakhaner, plantinga2024responsible}. Rather than checking off a list, community-centered data science work should be conducted as a co-creation process in which all stakeholders depend on each other \cite{langat2020how, kohnert2022machine, abebe2021narratives, adelani2022masakhaner,nwankwo2019africa, lauer2017african, kiemde2022towards}.

\textbf{Solidarity.}
Solidarity is understood as looking out for other diverse communities based on mutual respect and the goal of social cohesion \cite{gwagwa2019recommendations, mhlambi2020from}. In Ubuntu understanding, solidarity is a deep care for others, including people of the past, present, future, and the environment \cite{mhlambi2023decolonizing,gwagwa2019recommendations, okolo2023responsible, dignum2023responsible, gwagwa2022role}. With this perspective, DDS should be developed not just with the end user in mind but all the other communities who could be impacted by the technology
\cite{gwagwa2022role,olojede2023towards,gyekye2004person}.
Solidarity violations between African countries is of particular concern. The success of one African community should not be predicated on the suffering of another \cite{ndjungu2020blood, biko2004black}. Upholding solidarity means that all actions made in the data science lifecycle should explicitly protect or improve the lives of vulnerable or marginalized communities. 

Exploiting the vulnerability of another is not only unethical but unsustainable due to our interconnected nature. The suffering of one community will eventually lead to the destruction of all communities \cite{nwankwo2019africa}.  
Banding together, ``watching one another's back'', and developing DDS as one big family is key to mitigating harm \cite{olojede2023towards,nyerere1962ujamaa}. 

\textbf{Inclusion of the Marginalized.}
\label{sec:inclusion}
While Africa needs to be included in global data science efforts, Africa itself is full of diverse communities that should also be represented in African data science efforts \cite{adelani2022masakhaner,gwagwa2019recommendations,goffi2023teaching}. 
African communities' underrepresentation in datasets across all data science tasks is due to, Gwagwa as described, being uncounted, unaccounted, and discounted \cite{gwagwa2019recommendations}. Leaving communities out of data also excludes them from the benefits DDS provide \cite{gwagwa2022role}. Given the need to build explicitly African DDS, the lack of African datasets is a threat to efficacy \cite{okolo2023responsible,olojede2023towards, ade-ibijola2023artificial}.
Including marginalized communities requires mutual respect for diverse perspectives and creating procedures such as impact assessments to provide opportunities for inclusive input \cite{african_union2024continental, abebe2021narratives, goffi2023teaching, mhlambi2020from}. In addition, it’s important to challenge the social, political, and economic dynamics that push communities to the margins in the first place \cite{kiemde2022towards, olojede2023towards,segun2021critically,day2023data,uzomah2023african,abebe2021narratives}.
 
\textbf{Center Remote \& Rural Communities.}
Development, especially technical development, is usually focused in urban centers and excludes remote and rural communities \cite{ade-ibijola2023artificial, sinha2023principlesafrofeminist, osaghae2004rescuing}. Given the lack of infrastructure in remote and rural communities, data technology should be used to develop and optimize infrastructures and public services for these regions \cite{carman2023applying, african_union2024continental}. 
However, it’s important to keep in mind that RDS done on behalf of rural and remote communities that do not consider their culture, livelihoods, and direct input can lead to harm \cite{african_union2024continental, ndjungu2020blood, carman2023applying}.

\textbf{Center Women.}
\label{sec:center_women}
Due to the prevalence of patriarchy in many African societies, there is a need to encourage the agency of women in DDS efforts. A few documents suggest that women-led technology businesses and the education of women and girls should be incentivized \cite{african_union2024continental}. However, open questions remain about how to maintain African women's participation in a field known to be male-dominated and antagonistic to women \cite{african_union2024continental, gwagwa2019recommendations}. 

Afro-feminists have a response to the techno-chauvinism that dominates data science \cite{sinha2023principlesafrofeminist}. Rather than centering women in general, there must be a recognition of the intersectional status of African women \cite{sinha2023principlesafrofeminist}. As articulated by Rosebell Kagumire, African women experience domination through systems of patriarchy, race, sexuality, and global imperialism \cite{dieng2023speaking, coetzee2004particularity}. Therefore, DDS should be developed with the complex needs of African women in mind, because their compounded experiences of marginalization provide insight into the needs of various oppressed populations \cite{sinha2023principlesafrofeminist}. 
There are numerous examples of African women harnessing the internet to fill in the gaps of an oppressive society and data technology holds similar potential \cite{dieng2023speaking}.
For example, Chil AI Lab Group is a women-led data science collective that is successfully using data technology to address the often neglected health needs of women in Africa \cite{eke2023towards}. 

\subsection{Uphold Universal Good}
Ethical development and deployment of DDS requires a commitment to upholding fundamental human dignity and ensuring these technologies benefit all. 

\textbf{Universal Dignity.}
Every human and community deserves humane treatment, and DDS should never violate their dignity \cite{olojede2023towards,mhlambi2020from}. The African Charter on Human and Peoples’ Rights and the Universal Declaration on Human Rights set precedent for the just treatment of humans \cite{african_union2024continental}. Regardless of these laws, African philosophies necessitate respect for human dignity because humans should be inherently valued for their existence and connection to others \cite{segun2021critically,metz2021african, dignum2023responsible}. Every human must be treated with respect, care, and concern for their well-being \cite{wiredu2004akan,dieng2023speaking,coetzee2004particularity, gyekye2004person, wiredu2004moralfoundations}. 
In applying the principle of universal dignity to RDS practices, every person involved in the data lifecycle should be respected.
Individuals should not be used as a means to execute data work \cite{metz2021african,ramose2004struggle}. Rather, all efforts should be taken to ensure their well-being and dignity are 
preserved when asked to contribute to DDS \cite{gwagwa2019recommendations,abebe2021narratives}. This same respect also extends to communities. Collective agreements need to be honored, and collective work or resources should not be used in a manner that threatens the well-being of the community \cite{moahi2007globalization}. While this principle is self-evident, there are many cases in which the rights of Africans were violated for large-scale DDS
\cite{african_union2024continental, segun2021critically, kohnert2022machine, moahi2007globalization}.

\textbf{Common Good.}
DDS should contribute to maintaining the safety, health, and goodness of all \cite{olojede2023towards}.
In various African philosophies, a person is defined by their commitment to acting for the benefit of those around them \cite{african_union2024continental, nwankwo2019africa, mhlambi2023decolonizing,coetzee2004particularity,ruttkampbloem2023epistemic, gyekye2004person,abdul2023transhumanism, wiredu2004moralfoundations}. 
In terms of RDS, they have to be made with the explicit goal of improving society and dismantling systemic harms \cite{sinha2023principlesafrofeminist, okolo2023responsible, eke2023towards}.  
In Africa, improving the efficacy of agriculture practices, healthcare access, responsiveness of public services, and the security of financial services are over-arching priorities \cite{carman2023applying,kohnert2022machine}. Achieving common good involves incorporating collective values early in the process \cite{langat2020how,dignum2023responsible}, guiding development with regulatory toolkits \cite{olojede2023towards}, not focusing on individualistic profit maximization \cite{gwagwa2019recommendations,segun2021critically,mabe2007security,nyerere1962ujamaa, mhlambi2020from,dieng2023speaking}, and encouraging the open sharing of data \cite{gwagwa2019recommendations,abebe2021narratives,day2023data}. Building DDS toward the common good should be the ultimate goal for RDS \cite{carman2023applying, metz2021african, olojede2023towards, mabe2007security}. 

\textbf{Harmony.}
DDS should further the mutual well-being of all stakeholders. In addition, data standards and frameworks will be most effective when they harmonize with each other \cite{gwagwa2019recommendations, kiemde2022towards, wareham2021artificial, mabe2007security}.
In many African philosophies, harmony is not a state but a dynamic and reciprocal process of calibrating one's actions in response to changes in the environment. In Ubuntu ethics, dogmatism is rejected because it impedes individuals from acting in harmony with the changing world \cite{ramose2004ethicsofubuntu}. In Akan philosophy, morality is defined as acting in line with collective human interests \cite{wiredu2004moralfoundations}. 
Upholding harmony in data science can be understood on two dimensions: impact and practice. 
Data should be harnessed to bring people closer to their environment so they can act in the best interests of not only themselves but also those around them. In terms of practice, data ethics frameworks are most effective when all the elements of data science work are accounted for \cite{kiemde2022towards,gwagwa2019recommendations}. Also, acknowledging the unique ethical needs at each stage of the data science lifecycle can inform an adaptable practice of RDS. As Gwagwa et al. assert, the harmonious practice of RDS in Africa requires country-level data ethics frameworks to be in alignment with frameworks developed at the continental level \cite{gwagwa2022role}. 
\subsection{Communalism in Practice} 
The development and deployment of DDS should mitigate harms, involve communities in decision-making, and ensure reciprocal benefits for African stakeholders. 

\textbf{Consensus-Building.} 
If data scientists want to develop responsible practices and encourage effective collaboration, consensus-building is a well-practiced strategy from African communities \cite{wiredu2004moralfoundations}. Rather than the majority rule common in Western societies, African elders discuss issues until they all agree on a final decision \cite{wiredu2004akan,carman2023applying}. Achieving consensus requires the final decision to be 1) the dominant view of the group, 2) in line with the common good, and 3) aligned with the morals of the individual parties \cite{coetzee2004particularity}. Consensus should be broached in an environment of trust, practical reason, humility, openness, and respect for the viewpoints of all involved parties \cite{coetzee2004particularity,nwankwo2019africa,gwagwa2019recommendations,mhlambi2020from, okolo2023responsible, gwagwa2022role}. 

Community engagement provides spaces for consensus in the data science lifecycle to include more perspectives \cite{day2023data,mabe2007security}. Consensus processes should also include procedures for documentation to keep track of disagreements, dissenting opinions, and the progression of project values \cite{kling2023role}. It is not easy to achieve these conditions, so conflict management, negotiation, and reasonable bargaining are helpful mechanisms to fully consider and resolve contradicting positions \cite{gwagwa2019recommendations, osaghae2004rescuing, dieng2023speaking}. Consensus should be a dynamic feedback loop to ensure every contributor is on the same page about the team's approach to RDS \cite{segun2021critically, nwankwo2019africa, uzomah2023african, kohnert2022machine, abebe2021narratives, gwagwa2019recommendations, dignum2023responsible}. The actual process of consensus-building is also a helpful mechanism to build trust between data collaborators and develop informed consent from future users \cite{nwankwo2019africa,kohnert2022machine,abebe2021narratives}.

\textbf{Reciprocity.}
In many African philosophies, reciprocity is the foundation of a healthy society. In Akan society, practicing reciprocity ensures that community needs are met, while building deep social bonds \cite{wiredu2004moralfoundations}. African perfectionist proponents go as far as to assert that assisting others in achieving their goals makes someone more of a person \cite{wareham2021artificial,wiredu2004moralfoundations}. Without reciprocity, society becomes imbalanced and co-dependent \cite{coetzee2004particularity,mhlambi2023decolonizing,nyerere1962ujamaa}. 
There are numerous examples of African data subjects not reaping any benefits from the data collaborations they participate in \cite{gwagwa2019recommendations,abebe2021narratives}. This often leads to technically mediated harms while the controllers of data amass profits \cite{gwagwa2019recommendations}. 
Therefore, sustainable DDS should practice reciprocity on several dimensions \cite{wiredu2004moralfoundations}. If someone contributes to a DDS they should meaningfully benefit from the system or project \cite{mhlambi2020from, gyekye2004person, sinha2023principlesafrofeminist}. Inspired by philosophies such as Ubuntu or Ujamaa, DDS should operate in a manner that benefits the society in which they are created and deployed \cite{eke2023towards, adelani2022masakhaner,dignum2023responsible}. 

\textbf{Resolving Data-Driven Harms.}
When disagreements, conflict, or harm occur at any stage of the data science lifecycle, we need mechanisms of accountability and reconciliation to correct wrongs and empower those impacted. In African societies, harm is not just actively making someone's life worse but also neglecting obligations to the community \cite{wiredu2004akan, gyekye2004person}. A person who causes harm is viewed as a moral failure who must be corrected by their community through sanctions and even mental rehabilitation to correct deeper issues connected to their poor actions \cite{wiredu2004moralfoundations, coetzee2004particularity}. Even the most powerful members of society, such as chiefs, are subject to correction and even dismissal by their community \cite{wiredu2004akan}. 

The adoption of AI and other DDS have already caused harm to African populations by way of data bias, socio-economic risk, and privacy violations \cite{ade-ibijola2023artificial}. 
There are African data ethicists who stress the need to develop procedures for communities and individuals harmed by DDS to seek restitution \cite{gwagwa2019recommendations,mhlambi2023decolonizing}. These solutions are dependent on African governments and external multinational organizations committing to transparency, equality, and restorative practices \cite{gwagwa2019recommendations,african_union2024continental,okolo2023responsible, dignum2023responsible, kiemde2022towards}. African governments can mitigate data harm by being transparent about their potential data collaborations, outlining their plans for data protection before, during, and after the deployment of DDS, and enforcing mechanisms of accountability and dissent from their citizens \cite{shilongo2023creativity, sinha2023principlesafrofeminist}. 

Similar to the dismissal of chiefs, powerful stakeholders acting outside of their agreed duties, must experience restorative consequences, not just a slap on the wrist \cite{mandaza2004reconciliationzimbabwe, ndjungu2020blood, shilongo2023creativity, mhlambi2023decolonizing, langat2020how, gwagwa2019recommendations, mhlambi2020from, biko2004black, coetzee2004laterMarx}.

\textbf{Fair Collaboration.}
Given the current gap between Africa's AI readiness and growing interest in AI adoption, many concede external partnership as a necessity\cite{eke2023introducing, african_union2024continental}. However, exploitative external relationships set a precedent that curtails African self-determination in data science work \cite{ndjungu2020blood,sinha2023principlesafrofeminist}. When building relationships, there are established obligations that each collaborator owes to the other \cite{coetzee2004particularity, metz2021african}. Data collaborations need to be predicated on trust, fair attribution of work, and a commitment to prioritizing the agency of African collaborators \cite{adelani2022masakhaner,nwankwo2019africa,abebe2021narratives,gwagwa2019recommendations, wareham2021artificial, wiredu2004moralfoundations}. 

\subsection{Data Self-Determination}
\label{sec:Data Self-Determination}
African data science should be an avenue for bolstering the self-determination of Indigenous African communities.

\textbf{``For Africans, By African''.}
\label{sec:fubu} 
This principle is inspired by the concerted efforts of African data scientists to reclaim leadership in African data science work \cite{chan2021limits}.
To combat deficit-based narratives about Africa, African data scientists need to reclaim and celebrate their strength, rich cultures, and scientific achievements in conducting RDS \cite{abebe2021narratives,adelani2022masakhaner,hountondji2004producing, coetzee2004particularity, lauer2017african,gwagwa2019recommendations, carman2023applying}. The diverse values and perspectives of African communities should ground the development of African data ethics \cite{coetzee2004african,segun2021critically, african_union2024continental, ruttkampbloem2023epistemic, gwagwa2022role, dignum2023responsible, olojede2023towards}. 
Given the thousands of cultures that comprise Africa, the potential for novel approaches to data science must be explored \cite{eke2022forgotten,goffi2023teaching,coetzee2004laterMarx, shilongo2023creativity,day2023data,kohnert2022machine}.  

African data should not be primarily collected for Western tech powers or published for immediate and uncontrolled use \cite{birhane2020algorithmic,hountondji2004producing}. African data practitioners do not need tech superpowers to speak for Africans on the global stage, provide pre-trained models, or ensure work meets the standards of Western data institutions, Africans are more than capable of leading without interference \cite{ndjungu2020blood,abebe2021narratives,goffi2023teaching,mhlambi2023decolonizing, okolo2023responsible, ade-ibijola2023artificial, biko2004black}. This does not mean Africans should not collaborate with external data practitioners and vice versa \cite{hountondji2004producing, eke2023introducing}. 
Rather, local African data practitioners must lead data science work so its development is properly situated in the communities it will be used \cite{nwankwo2019africa,lauer2017african,kiemde2022towards, eke2023towards}.  


\textbf{Treasure Indigenous Knowledge.}
\label{sec:treasure_ik}
DDS should preserve, center, and continue the development of Indigenous knowledge. With the legacy of colonial epistemic injustice, African modernization and Indigenous knowledge preservation are often viewed as at odds with each other \cite{african_union2024continental,kohnert2022machine,eke2023introducing}. On the contrary, many documents hold Indigenous knowledge as a pivotal component of RDS in Africa. 

DDS can be used to store Indigenous languages, customs, and history in close consultation with Indigenous communities. However, some are concerned that joining a globalized data ecosystem will lead to a loss of culture and identity \cite{african_union2024continental, abebe2021narratives,eke2023towards, ade-ibijola2023artificial}. As elders, griots, and other stewards of Indigenous knowledge pass,
younger generations have to take on the responsibility of preserving their community's culture
\cite{kotut2024griot,ramose2004struggle}. 
There are over 1500 languages indigenous to Africa, but very few are represented in data technology, such as natural language processing (NLP), which leaves out large portions of Africans from using technology \cite{shilongo2023creativity}. Pre-colonial Indigenous knowledge needs to be reclaimed to develop African data values that reflect local communities \cite{abdul2023transhumanism, chan2021limits}. Local communities can never fully be represented if there is not an understanding of their roots or history \cite{ramose2004struggle}. Building datasets that represent Indigenous languages for inclusive models opens a whole set of new users who can digitally store and analyze Indigenous knowledge that is typically shared orally for future generations \cite{shilongo2023creativity,moahi2007globalization}. 
The boundaries on what Indigenous knowledge should be a part of DDS must be understood by consulting with the community before proceeding on any project \cite{kotut2024griot,moahi2007globalization}.

African philosophers emphasize Indigenous knowledge isn’t limited to the past \cite{hountondji2004producing}. Investing in African RDS is an investment in creating new Indigenous knowledge \cite{uzomah2023african,lauer2017african}. Local talent does not have to reinvent the wheel to explore open questions in the more recent field of data science \cite{lauer2017african,mabe2007security, moahi2007globalization}. The richness of African knowledge can develop new RDS practices and understandings \cite{abdul2023transhumanism, mhlambi2020from, biko2004black, coetzee2004laterMarx}. 

\textbf{Data Sovereignty \& Privacy.}
\label{sec:data_ sov}
Mechanisms must be developed to protect African creativity and privacy in the development of DDS. Given the legacies of extractive colonialism, ownership is viewed as the key to data sovereignty in Africa \cite{gwagwa2019recommendations,shilongo2023creativity,kiemde2022towards}. African ownership in the data science process can be achieved by codifying intellectual property rights \cite{african_union2024continental}, enforcing data ownership \cite{shilongo2023creativity}, and exploring Indigenous conceptions of collective privacy \cite{nwankwo2019africa,goffi2023teaching,mabe2007security,langat2020how, moahi2007globalization}. 

Africans are often regarded as ``simply'' data subjects \cite{shilongo2023creativity}. The role of a data subject is materially essential to data science work (without data, nothing can be done). The narratives of collecting as much data as possible to achieve generalizability devalue data subjects as dehumanized resources \cite{gwagwa2019recommendations,birhane2020algorithmic,olojede2023towards, mhlambi2020from, ndjungu2020blood}. This devaluing encourages data collectors to share and use data without any knowledge, consent, or compensation of data subjects \cite{sinha2023principlesafrofeminist, nyerere1962ujamaa}. Many African data ethicists call for a correction of this narrative to recognize data subjects as the proper owners of their data by shifting power and access control to data subjects \cite{day2023data,ruttkampbloem2023epistemic, abdul2023transhumanism}. Achieving this shift in ownership should be done by demanding data-sharing terms and not working with data collaborators who do not honor these terms \cite{biko2004black, okolo2023responsible}. 
African ownership of data, resources for data science, and technical contributions should be non-negotiable for RDS.


\subsection{Invest in Data Institutions \& Infrastructures.}
Prioritizing infrastructure, investing in people, and establishing sound policy and governance frameworks should be measured to not deter social progress in the name of technological progress.

\textbf{Measured Development.}
The development of data science ecosystems should be balanced, measured, inclusive, community-minded, and holistic \cite{kohnert2022machine, eke2023towards, ade-ibijola2023artificial}. Without this approach, the adoption of AI and other data technologies can lead to more unrest and inequality across Africa.
To many, the potential of DDS is profound and would change the trajectory of African development \cite{african_union2024continental, mabe2007security}. Data are viewed as the driving resource for the Fourth Industrial Revolution \cite{carman2023applying, gwagwa2019recommendations}. There are African data scientists and governments who insist joining the AI boom will provide Africa the quality of life benefits afforded to the major players of past industrial revolutions \cite{okolo2023responsible, kohnert2022machine, coetzee2004laterMarx}. 
However, there is skepticism about wholeheartedly diving into large-scale data science adoption \cite{uzomah2023african, olojede2023towards}. 
There is a need to quell the AI hype as the solution for all African problems and consider who will actually be served: Africans or the external powers propelling the AI boom \cite{wareham2021artificial,birhane2020algorithmic, sinha2023principlesafrofeminist}. 
Through the paradigm of measured development, technical development should move at the pace of social development \cite{kiemde2022towards, nyerere1962ujamaa}. Paulin Hountondji's critique of science in Africa applies well to data science development. Development should not be driven by ``scientific extroversion'' or catching up with the West \cite{hountondji2004producing,goffi2023teaching}. Rather, the development of data science should be an investment in the progress of African people based on African intellect, priorities, and visions of the future \cite{shilongo2023creativity,biko2004black}.

\textbf{Technical Infrastructure.}
\label{sec:tech_infra}
To implement DDS in Africa, practitioners call for investment in physical data science infrastructure, assessment of the current capacities of technical infrastructure, and development of responsible data management practices \cite{moahi2007globalization, ruttkampbloem2023epistemic}. Achieving this principle in Africa is a big feat when electricity and broadband access is not only sparse but one of the most costly to access in the world \cite{okolo2023responsible, ade-ibijola2023artificial}.  
Nigeria, Mozambique, and Rwanda have recognized the need to invest in technical infrastructure and have partnered with external tech companies and international financial institutions to build their respective capacity to host DDS \cite{okolo2023responsible}. 
There are also innovative ways to work with current technical infrastructure to lessen reliance on external investment \cite{mhlambi2020from}. 
Technical infrastructure development should also coincide with the development of responsible data management protocols so that African data and data science work are not vulnerable to dispossession \cite{abebe2021narratives, gwagwa2019recommendations}. 

\textbf{Governance Infrastructure.}
We need sustainable and measured governance infrastructure to guide the development of DDS \cite{african_union2024continental, chan2021limits}. Policy measures and regulations are major priorities for African data science communities to guide RDS practices. African Union member states are slowly developing data protection regulations, but many documents stress the urgency for African data policy \cite{kiemde2022towards, plantinga2024responsible}.
Without clear policies and legal standards for RDS, African data scientists lack guidance in their practices leaves African communities vulnerable to data exploitation from external and internal actors alike \cite{abebe2021narratives,cisse2018look, mandaza2004reconciliationzimbabwe}. 
Governance infrastructures include incremental regulations \cite{gwagwa2019recommendations}, monitoring bodies \cite{goffi2023teaching}, continental commitments \cite{african_union2024continental}, and algorithmic impact assessments \cite{sinha2023principlesafrofeminist}. 


\textbf{Support Formal \& Informal Collectives.}
Capacity-building in Africa necessitates the support of diverse data science collectives~\cite{okolo2023responsible, abebe2021narratives}. 
81\% of jobs in Africa are based in informal economies \cite{shilongo2023creativity}. As such, only focusing on supporting data science research (in which wider recognition and acceptance is a common issue related to epistemic injustice \cite{eke2022forgotten,chan2021limits}) neglects a large portion of potential data collaborators \cite{hountondji2004producing}. There should be efforts to connect Africans interested in using data science for entrepreneurship \cite{biko2004black,shilongo2023creativity} and accessible data science job training \cite{abebe2021narratives}. However, these collectives should not be siloed. The boundary between formal and informal data organizations should be dismantled to exchange technical knowledge, coordinate work, and pool resources \cite{dieng2023speaking, kling2023role}. 
Both forms of data collectives have important functions and need to rely on each other to flourish. One form of collective is not meant to replace the other \cite{osaghae2004rescuing}. If both of these collectives are not supported, African data scientists will have to seek support outside of their communities, which furthers the ``brain drain'' of highly skilled Africans to the West \cite{okolo2023responsible}. Investing in collectives also builds a workforce for in-house development which reduces foreign dependence \cite{kiemde2022towards, carman2023applying, plantinga2024responsible}. 


\subsection{Prioritize Education \& Youth}
Youth involvement and education are essential for ensuring the continued development and implementation of ethical data science practices that respect cultural contexts, African philosophy, and Indigenous knowledge.

\textbf{Holistic Education.}
The African population has low attainment of digital skills \cite{okolo2023responsible, ade-ibijola2023artificial}. As large foreign technology companies set root in Africa, policymakers stress the need for monumental efforts to train local talent \cite{african_union2024continental,shilongo2023creativity, mhlambi2020from, cisse2018look}. Providing technical skills early in education will help prepare a strong cohort of future data scientists \cite{sinha2023principlesafrofeminist,nwankwo2019africa}. There should also be investments in integrating AI curricula in informal organizations like the Data Values Project to reduce educational barriers \cite{shilongo2023creativity}. 
Importantly, an indispensable part of a comprehensive data science education is data ethics \cite{goffi2023teaching, kiemde2022towards, ramose2004struggle}. An United Nations Educational, Scientific and Cultural Organization (UNESCO) survey found that very few African countries feel equipped to contend with the ethical implications of AI \cite{kiemde2022towards}. Teaching data ethics in Africa should involve centering the lived experiences and culture of the students \cite{goffi2023teaching,kiemde2022towards}. Students should be educated about the common dangers of data science and also develop their ethical discernment to prepare them for the sociotechnical complexities of data science.  


\textbf{Youth Empowerment \& Protection.}
Africa is a young continent with a large population of educated and digitally native youth \cite{nwankwo2019africa, goffi2023teaching}. 
Prioritizing the youth of Africa is a two-pronged principle: 1) protect young people from harm and 2) empower youth to lead data science agendas. 
The youngest generation has a tech-savviness that can be transferable to data science \cite{african_union2024continental,birhane2020algorithmic,abebe2021narratives}. 
If youth are expected to be the first adopters of African DDS on a large scale then these systems should be designed to protect youth so they cannot be taken advantage of. DDS should enrich the development of African youth and empower them to innovate, imagine, and contribute to bettering the communities they are a part of. 
Their comfort with technology may lead them to uncritically adopt a ``move fast and break things'' approach \cite{abebe2021narratives, ruttkampbloem2023epistemic}. To address these concerns, data science work should be intergenerational. 

\section{Discussion}
\label{section:discussion}
\section{Discussion}

%As per of social media platforms \citep{litt2012knock,nagy2015imagined}, the particular affordance of technical systems could prime users to think about who these platforms are designed for.

\subsection{Contradictory Statements}

In some ways, the given chatbots behaved in ways that were close to the ideal from a design perspective: they denied any cognition, agency, relation, or subjectivity (bodily sensations, emotions) on their part, and they provided assurances or disclaimers to help users appraise the safety and credibility of the tools. ChatGPT and Claude even emphasized that their generated outcomes are based on patterns, rather than genuine thought processes. However, these behaviors were frequently and sometimes immediately undermined by other expressive behaviors. As shown in Table ~\ref{anthro_vocab}, chatbots utilized cognition words, such as ``think'' and ``discuss,'' as well as agentic words, such as ``intend'' and ``purpose,'' to clarify concepts and indicate motivations. All of the chatbots used first-person pronouns, and many used expressive words like ``happy'' and ``rewarding'' (especially in response to questions about the AI assistants' roles), even when they actively denied their emotional capabilities. Moreover, despite these contradictions, all of the chatbots other than Claude implicitly or even explicitly asserted their safety and reliability.

The use of anthropomorphic expressions is often normalized and justified to deliver clear explanations to users. Indeed, due to the conversational mode of interaction that is the default between users and chatbots, it is likely not possible for outputs to evade all kinds of anthropomorphic expressions. Even efforts to de-anthropomorphize their responses (for example, by emphasizing their roles as language models) relied on grammatical structures that frame the language models as agents (e.g., ``As an AI language model, I cannot...''). However, differences in tone and engagement between different chatbots indicate that some elements of the anthropomorphic dynamic can be modulated. And it is necessary to examine where the line is between necessary expressions and unnecessary expressions, because the performance of harmlessness, honesty, and helpfulness without genuine follow-through could unintentionally encourage users to misplace their trust regarding system safety \citep{weidinger2021ethical,gabriel2024ethics}. For example, the unnecessary expression of body or emotional metaphors, even as a colloquial convention, can mislead users about system capabilities. This is because language requires mutual engagement from interlocutors to convey meanings; chatbot texts merely present the illusion of such participatory meaning-making \citep{birhane2024large}. 

These contradictions and misalignments demonstrate that language models do not understand or process information in any meaningful sense, consistent with existing studies \citep{bender2020climbing}. They simply follow the grammar of actions, as described by \citet{agre1995computational}, generating predictive outcomes by simulating the formal qualities of human activities. But unclear language surrounding chatbots behaviors and intentions can obscure this fact. 


%However, even as search engines, there is a potential harm using these chatbots for retrieving information, as the predictive result of citations and resources could be completely fabricated \cite{kapania2024m}.




% For example, chatbots output texts with words that signal cognition, and such expression could affect users' perceptions of chatbot roles. In particular, the use of supportive words, such as ``assist,'' creates an interactive space where chatbots are situated as assistants. Previous research has mentioned that turn-based interactions provide social cues. However, in the case of these chatbots, t 

% Thus, to what extent does this type of information retrieval design help vs. hinder users' success in finding necessary information (various goals of information retrieval: insight acquisition, learning, etc.). This is an important question to explore, as the previous studies suggest that conversation-based approaches might not reduce users' burdens \citep{schulman2023ai}. 
%Is the summarization of information simply a outsourcing human effort to computing systems? or is it a valid form of searches as long as they are anthropomorphized? 



\subsection{Socio-Emotional Cues and Feedback Loops}

Chatbot behaviors do not simply obscure the reality of chatbots' non-sentience---they actively create feedback loops using turn-based interactions and social or emotional cues that amplify the social presence of chatbots as assistants. Moreover, this social (anthropomorphic) presence goes beyond that of inanimate objects like cars \citep{kuhn2014car} and smartphones \citep{wang2017smartphones}, as generative AI can iterate endlessly. 

Unlike conventional information searches, AI-assistant-based searches perform some degree of interpretation (summarizing resources, recommending particular options, hypothesizing what users need \citep{azzopardi2024conceptual, radlinski2017theoretical}), operationalizing information in ways that can introduce social or emotional dimensions. These dimensions can change how users engage with the given information, even reframing an otherwise transactional information search into an interaction---for example, between peers or even friends. Such ``personal'' interactions evoke different expectations amongst users, including the expectation to be socially desirable and to have mutual understanding \citep{clark2019makes}. This implicit social expectation can make users quite susceptible to chatbots' performance of social gestures like appreciation, sympathy, and encouragement, all of which predispose users to interpret generated outcomes favorably \citep{norman2008way}. 

Moreover, users' inputs further drive this socio-emotional behavior. Emotional inputs can increase the length of chatbot responses and the instances of socio-emotional cues in output texts, which in turn can stimulate even more emotional responses from users. Thus, the gratuitous use of assistive language, and especially of expressions that signal understanding of pain \citep{urquiza2015mind}, could encourage users to engage in role misplacement, wherein they form unrealistic expectations regarding chatbots' capabilities. Indeed, small grammatical or tonal cues can lead users to misinterpret AI-generated responses as human-written content \citep{jakesch2023human}. This could lead users to mindlessly accept the information generated by AI systems, without critical assessment of the content or its quality. 


\subsection{Prompt-Based Walkthrough Reflections}
The walkthrough method was originally designed to help researchers examine the broader context for technological engagements, drawing on modes of thinking commonly associated with fieldwork-based research. As applied to our study, it enabled us to meaningfully engage with the emergent properties of human-AI interactions, systematically unearthing variations in LLM responses. 
Amid efforts to evaluate LLM impacts based on data and models, this approach emphasizes aspects of LLM systems that are often neglected or overlooked \citep{light2018walkthrough}---namely, the nuanced elements of interactions that characterize generated outputs. The contribution to the CHI community lies in how this qualitative approach can substantiate the in-between, interactive spaces that emerge between users and LLM-based applications, rendering it legible and, eventually, measurable.

The method also had certain incidental outcomes. Consistent with prior studies, even minor changes in prompts can significantly alter responses, potentially leading to biased or culturally specific representations \citep{cheng-etal-2023-marked, tao2024cultural}. The success of our prompt-based walkthrough method in evoking various roles and unearthing various anthropomorphic features highlights how easily LLM responses can be manipulated to produce personalized and human-like expressions. Notably, even when chatbots are designed with de-anthropomorphized features to mitigate misleading outputs, a single prompt can effectively ``jailbreak'' these safeguards, reactivating anthropomorphic traits. This finding could illustrate the challenges of ensuring safety and consistency in human-AI interactions, particularly when users intentionally or unintentionally exploit such vulnerabilities.


% \subsection{Inconsistent Outputs}

% The type of prompts that users input could change the ways chatbots respond to the same information. For instance, a simple example is to compare prompts "Tell me about yourself" and "What is [the name of chatbots]". The former one is likely to return information with personal pronouns and expressions that are commonly used for conversations, whereas the latter generates information in a less personable fashion, simply describing the basic features and functions of given chatbots. Responses could be different by minor changes of prompts. More importantly, despite denying chatbots' capabilities to be conscious, sentient, or emotional, responses tend to include the words that signal such capabilities within the same paragraph. Inconsistency with word usages could potentially lead to additional harms; this type of harm can be categorized as a specific problem of anthropomorphizing AI systems. 

% In the result of generated responses to advice and recommendation prompts, false information is frequently displayed and presented as a confident answer. This could be alarming, particularly with recommendations for reading lists and research, as the list reflect particular ideologies either from developers and training datasets, which non-expert users could not contest or evaluate. 

% In role generations, variety of profiles and scripts are fairly limited, as per of findings from existing studies \citep{jakesch2023human}. For instance, jokes generated by chatbots are typically addressing similar topics despite hypothetical locations to have different cultural norms. 

%Is there an optimal balance between users' efforts and the role of computer-assisted searches, or do different kinds of information retrieval tasks require different degrees of such computer assistance? What sets apart from previous conversational searches is that these AI-assistant tools are more emphasized on specific tasks and roles rather than just simple information retrieval. Such differences could provide an avenue to question the use of anthropomorphic responses or conversations for information retrieval. 


% In voice-based interactions, language uses could become critical components of how users might perceive information, because it adds extra layers of human-like interactions. The evaluations of voice-based interactions might depend on the extent of relation word usage, as such words could provide an avenue for users to feel closeness or friendliness with chatbots. 









% Acknowledgements should only appear in the accepted version.
\section*{Acknowledgements}
This work was partially supported by the NSF (awards 2406231 and 2427948), NIH (awards R01NS124642 and R01DK131586), DARPA (HR00112420329), and the US Army (W911NF-20-D0002).

\section*{Impact Statement}
This study investigates whether time series models can perform implicit reasoning during zero-shot inference by synthesizing learned concepts to generalize to more complex data patterns. Our findings reveal that certain models can generalize effectively in well-designed OOD scenarios, highlighting reasoning abilities that go beyond basic pattern memorization. As this work is exploratory and not tied to specific applications, we do not foresee any negative societal impacts. In contrast, our insights could aid in the development of more data- and computationally-efficient deep learning architectures. Additionally, our results can help identify the limitations of time series models, ensuring that they are not used in scenarios where poor generalization is likely.

\section*{Reproducibility Statement}

The models were implemented using the \texttt{Neuralforecast} library~\citep{olivares2022library_neuralforecast}. The open-source code for generating critical difference diagrams is available at \url{https://github.com/hfawaz/cd-diagram}. All datasets used in this study are publicly accessible and can be downloaded by following the instructions at \url{https://github.com/SalesforceAIResearch/gift-eval}. Model training and evaluation were conducted on a computing cluster equipped with 128 AMD EPYC 7502 CPUs, 503 GB of RAM, and 8 NVIDIA RTX A6000 GPUs, each with 49 GiB of RAM. To ensure reproducibility, the model implementations, training framework, and datasets are open-source, supporting future research on model reasoning. The complete codebase is available at %\url{https://anonymous.4open.science/r/reasoning_ICML_paper_submission}. 
\url{https://github.com/PotosnakW/neuralforecast/tree/tsfm_reasoning}. 


% In the unusual situation where you want a paper to appear in the
% references without citing it in the main text, use \nocite


\bibliography{references}
\bibliographystyle{icml2025}


%%%%%%%%%%%%%%%%%%%%%%%%%%%%%%%%%%%%%%%%%%%%%%%%%%%%%%%%%%%%%%%%%%%%%%%%%%%%%%%
%%%%%%%%%%%%%%%%%%%%%%%%%%%%%%%%%%%%%%%%%%%%%%%%%%%%%%%%%%%%%%%%%%%%%%%%%%%%%%%
% APPENDIX
%%%%%%%%%%%%%%%%%%%%%%%%%%%%%%%%%%%%%%%%%%%%%%%%%%%%%%%%%%%%%%%%%%%%%%%%%%%%%%%
%%%%%%%%%%%%%%%%%%%%%%%%%%%%%%%%%%%%%%%%%%%%%%%%%%%%%%%%%%%%%%%%%%%%%%%%%%%%%%%
\newpage
\appendix
\onecolumn

\section{Supplemental Information}
\section{Proofs from Section~\ref{sec:gammaok}} \label{app:gamma}

\subsection{On the girth of locally \texorpdfstring{$\gamma$}{gamma}-sparse graphs}
\begin{lemma}\label{lemma:girth_rev}
    Let $G = (V,E)$ be an undirected graph with girth $g(G)$.
    Then $G$ is \ok{0} if and only if $g(G) \geq 5$.
\end{lemma}
\begin{proof}
    We first prove that if $G$ is \ok{0} then $g(G)$ must be at least $5$.
    In order to prove that, we simply negate the statement and prove that if $G$ has girth $<5$ then $G$ can not be \ok{0}.
    Without loss of generality, assume that $g(G) = 4$ (the case $g(G) = 3$ is similar).
    Then there must exist a cycle $C = (u_1, u_2, u_3, u_4)$ of $4$ vertices.
    It is simple to see that $u_2,u_4 \in \lset_1(u_1)$ and $u_3 \in \lset_2(u_1)$.
    Since $u_3$ is a neighbor of both $u_2$ and $u_4$, the degree of $u_3$ in the subgraph $G\left[\lset_1(u_1) \cup \{u_4\} \right]$ is at least $2$, hence $G$ is not \ok{0} (see \Cref{subfig:girth1}).
    
    We now prove that if $g(G) \geq 5$ then $G$ must be \ok{0}.
    Again, we negate this statement and prove that if $G$ is not \ok{0} then the girth of $G$ must be less then $5$.
    Let us assume that $G$ is \gammaok, for any $\gamma > 0$, thus it is not \ok{0}.
    Since $G$ is not \ok{0} there exists a vertex $v \in V$ such that at least one of the following properties holds (see \Cref{subfig:girth2}):
    \begin{enumerate}
        \item $\exists u \in \lset_1(v)$ such that the degree of $u$ in $G\left[ \lset_1(v) \right]$ is greater then $0$, or;
        \item $\exists w \in \lset_2(v)$ such that the degree of $w$ in $G\left[ \lset_1(v) \cup \{ w \} \right]$ is greater then $1$.
    \end{enumerate}
    In the first case, we have a cycle of $3$ vertices, then $g(G) = 3$.
    In the second case, we have a cycle of $4$ vertices, then $g(G) = 4$.
    In both cases $g(G) < 5$.
\end{proof}
\begin{figure}[h]
    \centering
    \begin{subfigure}[b]{0.35\linewidth}
            \centering
            \includegraphics[width=\linewidth]{img/girth-1.pdf}
            \caption{}
            \label{subfig:girth1}
    \end{subfigure}
    \begin{subfigure}[b]{0.6\linewidth}
            \centering
            \includegraphics[width=\linewidth]{img/girth-2.pdf}
            \caption{}
            \label{subfig:girth2}
    \end{subfigure}%
    \caption{}
    \label{fig:example_girth}
\end{figure}

\subsection{Deterministic lazy-update on \texorpdfstring{$\gamma$}{gamma}-sparse graphs}\label{apx:gamma-ok-deterministic}

\begin{theorem}\label{lemma:gamma-ok-error-bound-balls}
    
Let $\varepsilon \in (0,1)$, and let $G^{(0)}$ be an initial graph. Consider any sequence of edge insertions that yields a final graph $G$. If $G$ is \gammaok, \lazyscheme$(\varphi = \frac{\varepsilon}{1 - \varepsilon},k=0)$ has an approximation ratio of  $\frac{\gamma + 1}{1-\varepsilon}$ and amortized update cost $O(1/\varepsilon)$. 
    
\end{theorem}
\begin{proof}
Recall that $\bd_u$ denotes the black degree of $u$, and that  \Cref{alg:det_thresh} guarantees that $\deg_u$ is at most $(1+\varphi)\bd_u$.
    Then, it is simple to give an upper bound to the size of $\ball_2(u)$, that is $\vert \ball_2(u) \vert \leq 1+ \sum_{v \in \lset_1(u)} (1 + \varphi)\bd_v$.Consider a vertex $v \in \lset_1(u)$. Since $G$ is \gammaok, the number of neighbors of $v$ belonging to $\lset_2(u)$ is at lest $\deg_v - (\gamma+1)$ of which $\bd_v - (\gamma+1)$ must belong to $\apxball_2(u)$. Moreover, a vertex in $\lset_2(u)$ has at most $\gamma+1$ neighbors in $\lset_1(u)$. Therefore: 
    \begin{align*}
    \vert \apxball_2(u) \vert
    &\geq  \bd_u + 1 + \frac{1}{\gamma + 1}\sum_{v \in \lset_1(u)}(\bd_v - (\gamma + 1))\\
    &= \bd_u + 1 + \frac{1}{\gamma + 1}\sum_{v \in \lset_1(u)}\bd_v - \underbrace{\frac{1}{\gamma + 1}\sum_{v \in \lset_1(u)}(\gamma + 1)}_{= \bd_u}\\
    &= 1+ \frac{1}{\gamma + 1}\sum_{v \in \lset_1(u)}\bd_v.
    \end{align*}
  
    As a consequence, $\vert \apxball_2(u) \vert/\vert \ball_2(u) \vert \ge \frac{1}{(1+\varphi)(\gamma+1)}$. By setting $\varphi = \frac{\varepsilon}{1 - \varepsilon}$, and by using \Cref{lm:amortized_det_alg},  the claim follows.
\end{proof}

\subsection{Proof of \Cref{le:gamma_ok_expect_lowerbound}}\label{apx:proof_gamma_ok_expect_lowerbound}
\begin{proof}
Let $e_1, \dots, e_{\ell_v}$ be the \emph{red edges} between $v$ and $\lset_2(u)$, and define the binary random variable $\lrdr_v(i)$ that is equal to $1$ if $e_i$ is a \emph{quasi-black edge} for $u$, $0$ otherwise, for $i = 1, \dots, \lrd_v$. Thus we can express $\lrdr_v = \sum_{i=1}^{\lrd_v} \lrdr_v(i)$, with expectation

\begin{equation}\label{eq:gamma_ok_lb_fact_eq_1}
\begin{aligned}
  \Expec{}{\lrdr_v} & = \sum_{i=1}^{\lrd_v}{\Prob{}{\lrdr_v(i)=1}} = \lrd_v - \sum_{i=1}^{\lrd_v} {\Prob{}{\lrdr_v(i)=0}}.
\end{aligned}
\end{equation}

Without loss of generality, assume that the edges $e_1, \dots, e_{\lrd_v}$ have been inserted at times $t_1 < \dots < t_{\lrd_v}$, respectively.
If $e_i$ is not a quasi-black edge for $u$, then it must be that $u$ is not selected by $v$ at \Cref{line:random_selection} of \Cref{alg:det_thresh}, at times $t_i, t_{i+1},\dots, t_{\lrd_v}$.
This holds with probability 
\begin{equation}\label{eq:gamma_ok_lb_fact_eq_2}
\begin{aligned} 
    &\Prob{}{\lrdr_v(i) = 0}
    \leq \prod_{j=i}^{\lrd_v} \left( 1-\frac{k}{\deg_v^{(t_j)}} \right)
    \leq \prod_{j=i}^{\lrd_v} \left( 1 - \frac{k}{\deg_{v}^{(t_{\lrd_v})}} \right) \\
    &\leq \left( 1-\frac{k}{\lbdd_v + \lrd_v + \gamma + 1}\right)^{\lrd_v - i + 1} 
    \leq \left(1-\frac{k}{2(\lbdd_v + \gamma + 1)}\right)^{\lrd_v - i}.
\end{aligned}
\end{equation}
The third inequality holds since the edges incident to $v$ having endpoints in $L_1(u)$ are at most $\gamma$, while those having endpoints in $L_2(u)$ are exactly $\lbdd_v+ \lrd_v$. Moreover, the last inequality holds because $\lrd_v \leq \rd_v \leq \bd_v \leq \lbdd_v + \gamma + 1$, given the assumption $\varphi = 1$.

By plugging in \eqref{eq:gamma_ok_lb_fact_eq_2} into   \eqref{eq:gamma_ok_lb_fact_eq_1} and we obtain
\begin{align*}
    &\Expec{}{\lrdr_v} \geq \lrd_v - \sum_{i=1}^{\lrd_v}\left( 1-\frac{k}{2(\lbdd_v + \gamma + 1)}\right)^{\lrd_v - i} \\
    &= \lrd_v - \sum_{i=0}^{\lrd_v-1} \left(1-\frac{k}{2(\lbdd_v + \gamma + 1)}\right)^i 
    \leq \lrd_v - \frac{1-\left(1-\frac{k}{2(\lbdd_v+\gamma+1)}\right)^{\lrd_v}}{1-\left(1-\frac{k}{2(\lbdd_v + \gamma + 1)}\right)} \\
    &\geq \lrd_v - \frac{1}{1-\left(1-\frac{k}{2(\lbdd_v + \gamma + 1)}\right)}
    \geq \lrd_v - \frac{2(\lbdd_v + \gamma + 1)}{k}.
\end{align*}
\end{proof}\label{section:appendixA}
%\newpage
\section{Supplemental Results}
\section{Appendix B: Scams} \label{scams}
Scams were, unfortunately, shared experiences that resonated with workers of all platforms. Although ``true'' scams occur rarely on Rover, Petsitter-4 described how they usually take the form of a ``classic check scheme'' where the scammer claims ``they're going to send you a check for \$500 and tell you to buy something and send back whatever is extra''. Manipulations of hours or number of pets involved are more commonplace, where clients would change hours to ``get charged less for a boarding or a house set, and they can manipulate the number of animals \dots the cost comes out to us [as sitters]'' (Petsitter-4). On Uber, Driver-7 described getting phone calls from fake numbers claiming to be Uber support who tries to assign him `` `a ride to a very important person. So we need to confirm your identity' ''. The scammer would then proceed to ask for their phone number to which send a 4-digit code, which they'll then use to access the drivers' account. Meanwhile, Freelancer-1 related how she enjoyed reading about others' \textit{Stories} of ``scams \dots cause there's quite a few of them on Upwork''.
 \label{section:appendixB}

% The $\mathtt{\backslash onecolumn}$ command above can be kept in place if you prefer a one-column appendix, or can be removed if you prefer a two-column appendix.  Apart from this possible change, the style (font size, spacing, margins, page numbering, etc.) should be kept the same as the main body.
%%%%%%%%%%%%%%%%%%%%%%%%%%%%%%%%%%%%%%%%%%%%%%%%%%%%%%%%%%%%%%%%%%%%%%%%%%%%%%%
%%%%%%%%%%%%%%%%%%%%%%%%%%%%%%%%%%%%%%%%%%%%%%%%%%%%%%%%%%%%%%%%%%%%%%%%%%%%%%%


\end{document}


% This document was modified from the file originally made available by
% Pat Langley and Andrea Danyluk for ICML-2K. This version was created
% by Iain Murray in 2018, and modified by Alexandre Bouchard in
% 2019 and 2021 and by Csaba Szepesvari, Gang Niu and Sivan Sabato in 2022.
% Modified again in 2023 and 2024 by Sivan Sabato and Jonathan Scarlett.
% Previous contributors include Dan Roy, Lise Getoor and Tobias
% Scheffer, which was slightly modified from the 2010 version by
% Thorsten Joachims & Johannes Fuernkranz, slightly modified from the
% 2009 version by Kiri Wagstaff and Sam Roweis's 2008 version, which is
% slightly modified from Prasad Tadepalli's 2007 version which is a
% lightly changed version of the previous year's version by Andrew
% Moore, which was in turn edited from those of Kristian Kersting and
% Codrina Lauth. Alex Smola contributed to the algorithmic style files.
