%%%%%%%% ICML 2025 EXAMPLE LATEX SUBMISSION FILE %%%%%%%%%%%%%%%%%

\documentclass{article}
\usepackage{dirtytalk}

% Recommended, but optional, packages for figures and better typesetting:
\usepackage{microtype}
\usepackage{graphicx}
\usepackage{caption}
\usepackage{enumerate}
%\usepackage{subfigure}
\usepackage{booktabs} % for professional tables

% hyperref makes hyperlinks in the resulting PDF.
% If your build breaks (sometimes temporarily if a hyperlink spans a page)
% please comment out the following usepackage line and replace
% \usepackage{icml2025} with \usepackage[nohyperref]{icml2025} above.
\usepackage{hyperref}


% Attempt to make hyperref and algorithmic work together better:
\newcommand{\theHalgorithm}{\arabic{algorithm}}

% Use the following line for the initial blind version submitted for review:
%\usepackage{icml2025}

% If accepted, instead use the following line for the camera-ready submission:
\usepackage[accepted]{icml2025_archive}

% For theorems and such
\usepackage{amsmath}
\usepackage{amssymb}
\usepackage{mathtools}
\usepackage{amsthm}

% if you use cleveref..
\usepackage[capitalize,noabbrev]{cleveref}

\usepackage{xspace}

%%%%%%%%%%%%%%%%%%%%%%%%%%%%%%%%
% THEOREMS
%%%%%%%%%%%%%%%%%%%%%%%%%%%%%%%%
\theoremstyle{plain}
\newtheorem{theorem}{Theorem}[section]
\newtheorem{proposition}[theorem]{Proposition}
\newtheorem{lemma}[theorem]{Lemma}
\newtheorem{corollary}[theorem]{Corollary}
\theoremstyle{definition}
\newtheorem{definition}[theorem]{Definition}
\newtheorem{assumption}[theorem]{Assumption}
\theoremstyle{remark}
\newtheorem{remark}[theorem]{Remark}

% Todonotes is useful during development; simply uncomment the next line
%    and comment out the line below the next line to turn off comments
%\usepackage[disable,textsize=tiny]{todonotes}
\usepackage[textsize=tiny]{todonotes}

%% only for draft --------------------------------------------
\long\def\KO#1{{\color{olive}{[KO: #1]}}}
\long\def\WP#1{{\color{blue}{[RS: #1]}}}
\long\def\CC#1{{\color{green}{[CS: #1]}}}
\long\def\MG#1{{\color{teal}{[MG: #1]}}}
\long\def\AD#1{{\color{violet}{[SS: #1]}}}
\long\def\EDIT#1{{\color{red}{#1}\color{red}}}
\long\def\EDITtwo#1{{\color{red}{#1}\color{black}}}
%% ------------------------------------------------------------

%---------------Added packages/Functions--------------
\newcommand{\model}[1]{\texttt{#1}\xspace}
\newcommand{\ARIMA}{\model{ARIMA}}
\newcommand{\ETS}{\model{AutoETS}}
\newcommand{\Fourier}{\model{Fourier}}
\newcommand{\MLP}{\model{MLP}}
\newcommand{\DLinear}{\model{DLinear}}
\newcommand{\NLinear}{\model{NLinear}}
\newcommand{\TCN}{\model{TCN}}
\newcommand{\LSTM}{\model{LSTM}}
\newcommand{\RNN}{\model{RNN}}
\newcommand{\TimesNet}{\model{TimesNet}}
\newcommand{\TSMixer}{\model{TSMixer}}
\newcommand{\TSMixerx}{\model{TSMixerx}}
\newcommand{\NHITS}{\model{NHITS}}
\newcommand{\NBEATS}{\model{NBEATS}}

%--------tranformer (non-foundation models)-------
\newcommand{\VanillaTransformer}{\model{VanillaTransformer}}
\newcommand{\TFT}{\model{TFT}}
\newcommand{\PatchTST}{\model{PatchTST}}
\newcommand{\Fedformer}{\model{Fedformer}}
\newcommand{\Autoformer}{\model{Autoformer}}
\newcommand{\Informer}{\model{Informer}}
\newcommand{\iTransformer}{\model{iTransformer}}
\newcommand{\Tfive}{\model{T5 Model}}
\newcommand{\BasisFormer}{\model{BasisFormer}}

%--------foundation models -------
\newcommand{\TimesFM}{\model{TimesFM}}
\newcommand{\MOMENT}{\model{MOMENT}}
\newcommand{\LagLlama}{\model{LagLlama}}
\newcommand{\Moirai}{\model{Moirai}}
\newcommand{\TimeGPT}{\model{TimeGPT}}
\newcommand{\Chronos}{\model{Chronos}}
\newcommand{\SPADE}{\model{SPADE}}
\newcommand{\Timer}{\model{Timer}}
\newcommand{\TinyTimeMixers}{\model{TTMs}}

\usepackage{caption}
\usepackage{subcaption}
\usepackage{multirow}
\usepackage{graphicx} 
\usepackage{amsmath}
\usepackage{xcolor}
\usepackage{wrapfig}

\DeclareMathOperator*{\argmax}{arg\,max}
%---------------Added packages--------------

% The \icmltitle you define below is probably too long as a header.
% Therefore, a short form for the running title is supplied here:
\icmltitlerunning{Implicit Reasoning in Deep Time Series Forecasting}

\begin{document}

\twocolumn[
\icmltitle{
%Implicit Reasoning in Deep Time Series Forecasting
% \KO{1. Reasoning or Memorization? Exploring the Limits of Time Series Foundation Models}
Investigating Compositional Reasoning in Time Series Foundation Models % beyond Memorization-Based Generalization
% \KO{3. Do TSFMs succeed by memorizing training patterns, or can they logically extrapolating unseen patterns?}
}

% {Implicit Reasoning through Addition: Composing Forecasts with Fourier Features

% It is OKAY to include author information, even for blind
% submissions: the style file will automatically remove it for you
% unless you've provided the [accepted] option to the icml2025
% package.

% List of affiliations: The first argument should be a (short)
% identifier you will use later to specify author affiliations
% Academic affiliations should list Department, University, City, Region, Country
% Industry affiliations should list Company, City, Region, Country

% You can specify symbols, otherwise they are numbered in order.
% Ideally, you should not use this facility. Affiliations will be numbered
% in order of appearance and this is the preferred way.

\icmlsetsymbol{equal}{*}


\begin{icmlauthorlist}
\icmlauthor{Willa Potosnak}{cmu}
\icmlauthor{Cristian Challu}{equal,cmu,nixtla}
\icmlauthor{Mononito Goswami}{equal,cmu}
\icmlauthor{Kin G. Olivares}{cmu,amazon}
\icmlauthor{Michał Wiliński}{cmu}
\icmlauthor{Nina Żukowska}{cmu}
\icmlauthor{Artur Dubrawski}{cmu}
\end{icmlauthorlist}

\icmlaffiliation{cmu}{Auton Lab, School of Computer Science, Carnegie Mellon University}
\icmlaffiliation{nixtla}{Nixtla}
\icmlaffiliation{amazon}{Amazon}

\icmlcorrespondingauthor{Willa Potosnak}{wpotosna@andrew.cmu.edu}

\icmlkeywords{Machine Learning, ICML}

\vskip 0.3in
]

% this must go after the closing bracket ] following \twocolumn[ ...

% This command actually creates the footnote in the first column
% listing the affiliations and the copyright notice.
% The command takes one argument, which is text to display at the start of the footnote.
% The \icmlEqualContribution command is standard text for equal contribution.
% Remove it (just {}) if you do not need this facility.

%\printAffiliationsAndNotice{}  % leave blank if no need to mention equal contribution
\printAffiliationsAndNotice{\icmlEqualContribution} % otherwise use the standard text.

\begin{abstract} 
Large pre-trained time series foundation models (TSFMs) have demonstrated promising zero-shot performance across a wide range of domains. However, a question remains: Do TSFMs succeed solely by memorizing training patterns, or do they possess the ability to reason? While reasoning is a topic of great interest in the study of Large Language Models (LLMs), it is undefined and largely unexplored in the context of TSFMs. In this work, inspired by language modeling literature, we formally define compositional reasoning in forecasting and distinguish it from in-distribution generalization. We evaluate the reasoning and generalization capabilities of 23 popular deep learning forecasting models on multiple synthetic and real-world datasets. Additionally, through controlled studies, we systematically examine which design choices in TSFMs contribute to improved reasoning abilities. Our study yields key insights into the impact of TSFM architecture design on compositional reasoning and generalization. We find that patch-based Transformers have the best reasoning performance, closely followed by residualized MLP-based architectures, which are 97\% less computationally complex in terms of FLOPs and 86\% smaller in terms of the number of trainable parameters. Interestingly, in some zero-shot out-of-distribution scenarios, these models can outperform moving average and exponential smoothing statistical baselines trained on in-distribution data. Only a few design choices, such as the tokenization method, had a significant (negative) impact on Transformer model performance.
\end{abstract}


\section{Introduction}
\label{section:introduction}
\begin{figure}[ht]
    \centering
    \includegraphics[width=0.8\linewidth]{graphs/greater_than_naive.pdf}
    \vspace{0.5cm}
    \includegraphics[width=0.8\linewidth]{graphs/p1_bottom.png}
    \vspace{-5pt}
    \caption{\textcolor{positional}{Positional} vs.\ \textcolor{nonpositional}{non-positional} circuits. In a \textcolor{nonpositional}{non-positional} circuit, the same edges must be included at all positions. A \textcolor{positional}{positional} circuit can distinguish between the same edge at different positions. This specificity yields better trade-offs between circuit size and faithfulness. It can also increase both precision and recall.}
    \label{fig:p1}
    \vspace{-5pt}
\end{figure}

\section{Introduction}

\looseness=-1
A primary goal of interpretability research is to characterize the internal mechanisms in language models (LMs) and other NLP models. 
A core approach in this area is \textbf{circuit discovery}---identifying the minimal subgraph within the model's computation graph that performs a specific task \citep{olah2021framework,olah-mech}.
Typically, the nodes of a circuit represent model components (e.g., attention heads, neurons, or layers).
While manual circuit discovery methods can yield position-specific insights \citep{wanginterpretability,goldowskydill2023localizingmodelbehaviorpath}, \emph{automatic methods often overlook positional information}, treating components as uniformly relevant across all input token positions \citep{conmytowards,syed2023attribution}. 
For instance, if an attention head is included in a circuit, it is assumed to contribute equally to the computation for every position in the input sequence.
The assumption that circuits are position-invariant ignores the fact that different positions often require distinct computations.
By ignoring positions, current methods limit their ability to capture mechanisms that operate across positions, such as interactions between attention heads across positions.

In this study, we start by demonstrating that positional agnosticism is a significant limitation (\S\ref{sec:motivating}). Then, to address these limitations, we introduce a new approach: position-aware edge attribution patching (PEAP; \S\ref{sec:full_circ_discovery}; Figure~\ref{fig:p1}). Current approaches  assume that if an edge is in a circuit, then the same edge will be in the circuit at all positions, thus leading to low precision. It is also assumed that an edge's importance should be aggregated across positions before deciding whether it should be included in the circuit; this can lead to cancellation effects, and thus low recall. PEAP instead allows us to compute the importance of cross-positional edges, and separately evaluates edge importance at each position. We show that this leads to smaller and more accurate circuits; see Figure~\ref{fig:p1}.

Incorporating positional information into circuit discovery is straightforward when inputs have the same length and structure across examples.

However, realistic datasets are not nearly this templatic.
How, then, can we incorporate positional information into automatic circuit discovery?
To address this challenge, we propose \textbf{schemas} (\S\ref{sec:schema}). 
Schemas assign semantic labels to spans of tokens, enabling information aggregation across examples even when the spans differ in length.

For example, in the input ``The \textcolor{positional}{war} lasted from 1453 to 14\underline{\hspace{1em}},'' the span ``\textcolor{positional}{war}'' could be labeled as ``\emph{Subject}''.
This enables handling spans with varying lengths: the phrase ``\textcolor{positional}{Black Plague}'' in another example can be treated as a single positional span with the same role as ``\textcolor{positional}{war}''.
In experiments with two LMs and three tasks, we find that circuits discovered using schemas achieve a better trade-off between circuit size and faithfulness to the model's behavior than position-agnostic circuits.
Importantly, position-aware circuits offer a more precise representation of the underlying mechanisms, providing a more concise foundation for mechanistic explanations.

We also present a fully automated pipeline for schema generation and application (\S\ref{sec:schema-generation}) using large language models (LLMs). 
We evaluate the quality of the generated schemas and their utility in discovering position-aware circuits (\S\ref{sec:schema-eval}).
Notably, circuits derived using automatically generated and applied schemas achieve comparable faithfulness scores to circuits discovered with human-designed and manually applied schemas.

We summarize our contributions as follows:
\begin{itemize}[noitemsep,leftmargin=*,topsep=1pt,parsep=1pt]
    \item Introduce a position-aware circuit discovery method, which obtains better faithfulness than position-agnostic discovery.  
    \item Introduce dataset schemas,  facilitating positional circuit discovery in more naturalistic settings. 
    \item Develop an automated schema generation and application pipeline with LLMs, yielding schemas that are comparable to manually-annotated ones.
\end{itemize}

\section{Related Work}
\label{section:related_work}

\section{Related work}


Recent advances in single-image animatable head avatar generation can be categorized into mainly 2D-based and 3D-based approaches. 

\paragraph{\bf Image to 2D Animatable Avatar.}
2D-based methods, leveraging the power of convolutional neural networks (CNNs)~\cite{DBLP:conf/cvpr/KarrasLAHLA20,DBLP:conf/cvpr/IsolaZZE17,DBLP:conf/nips/GoodfellowPMXWOCB14}, often employ generative adversarial networks (GANs)~\cite{DBLP:conf/cvpr/StyleGAN} for direct image synthesis. Early approaches~\cite{DBLP:conf/cvpr/WangDYSW23,DBLP:conf/cvpr/BurkovPGL20,DBLP:conf/iccv/ZakharovSBL19} focus on injecting expression and pose features into the generator network, often utilizing architectures like U-Net or StyleGAN~\cite{DBLP:conf/cvpr/StyleGAN}.
Some other 2D methods~\cite{DBLP:journals/corr/abs-2407-03168,DBLP:conf/cvpr/ZhangQZZW0CW023,DBLP:conf/cvpr/HongZS022,DBLP:conf/mm/DrobyshevCKILZ22,DBLP:conf/cvpr/BurkovPGL20,DBLP:conf/nips/SiarohinLT0S19} represent expressions and poses as warping fields applied to the source image. 
Benefiting from advances in image and video diffusion networks, more recent 2D-based works~\cite{DBLP:journals/corr/abs-2410-07718,DBLP:journals/corr/abs-2406-08801,DBLP:conf/eccv/TianWZB24} get improved results with diffusion techniques. 
However, these methods still face challenges related to long generation times and significant computational resource demands. Audio-driven 2D control methods~\cite{DBLP:conf/cvpr/ZhangCWZSGSW23,DBLP:journals/corr/abs-2211-12368,DBLP:conf/iccv/GuoCLLBZ21} are easy to use but cannot explicitly control facial expressions and poses. 2D-based techniques often struggle with large pose or expression variations due to the lack of an explicit 3D structure, sometimes producing unrealistic distortions or identity changes. While some 2D methods~\cite{SadTalker,StyleHEAT,Pirenderer,DBLP:conf/cvpr/WangM021,MegaPortraits} incorporate 3D Morphable Models (3DMMs)~\cite{DBLP:conf/fgr/GerigMBELSV18,DBLP:journals/tog/LiBBL017,DBLP:conf/avss/PaysanKARV09,DBLP:conf/siggraph/BlanzV99} to mitigate these issues, they typically cannot achieve free-viewpoint rendering. 

\vspace{-0.1in}

\begin{figure*}[h]
    \centering
    \includegraphics[width=0.9\linewidth]{images/framework.pdf}
    \caption{\textbf{Overall Framework.} Our framework utilizes learnable query features attached to FLAME vertices to perform cross-attention with the extracted multi-level image features. The extracted features are then decoded to reconstruct the Gaussian avatar in the canonical space, which can be animated utilizing standard linear blend skinning (LBS) and corrective blendshapes as the FLAME model did and rendered in real-time on various platforms.}
    \label{fig:framework}
\end{figure*}

\paragraph{\bf Image to 3D Animatable Avatar.}
3D-aware methods offer improved geometric consistency and free-viewpoint rendering capabilities. Early 3D approaches~\cite{DBLP:conf/eccv/KhakhulinSLZ22,DBLP:conf/cvpr/XuYCWDJT20} utilize 3DMMs for head avatar reconstruction. With the advent of Neural Radiance Fields (NeRFs)~\cite{DBLP:conf/eccv/MildenhallSTBRN20}, many recent methods~\cite{DBLP:conf/siggraph/YuFZWYBCSWSW23,DBLP:conf/cvpr/MaZQLZ23,DBLP:conf/cvpr/LiZWZ0CZWB023,GPAvatar,ye2024real3d,deng2024portrait4d,deng2024portrait4d2,DBLP:conf/eccv/KiMC24,DBLP:conf/cvpr/BaiFWZSYS23,PointAvatar,Nerfies,INSTA} have adopted this representation for higher fidelity, particularly in modeling fine details like hair. However, NeRF-based~\cite{DBLP:conf/cvpr/ZhangZLHLWGCL024,HAvatar,DBLP:conf/cvpr/BaiTHSTQMDDOPTB23,AD-NeRF,DBLP:journals/tog/GaoZXHGZ22,DBLP:journals/tog/ParkSHBBGMS21,DBLP:conf/cvpr/AtharXSSS22,DBLP:journals/corr/abs-2112-05637,DBLP:conf/iccv/TretschkTGZLT21,DBLP:conf/cvpr/GafniTZN21,DBLP:conf/eccv/KiMC24,DBLP:conf/cvpr/BaiFWZSYS23,PointAvatar,Nerfies,DBLP:conf/siggraph/YuFZWYBCSWSW23,DBLP:conf/cvpr/MaZQLZ23,DBLP:conf/cvpr/LiZWZ0CZWB023} approaches often require extensive training data, including multi-view or single-view videos, raising privacy concerns and limiting generalization to unseen identities. Some methods~\cite{DBLP:conf/cvpr/SunWWLZZL23,DBLP:conf/3dim/ZhuangMKS22,DBLP:journals/pami/SunWZHWL24,DBLP:journals/tvcg/TangZYZCMW24,DBLP:conf/iclr/XuZLZBFS23} bypass this data requirement by training generators with random noise and then inverting them for identity-specific reconstruction, but inversion accuracy remains a challenge. Test-time optimization offers another alternative, but its computational cost limits practical applications. Several recent works~\cite{goha2023,hidenerf2023,gpavatar2024,ye2024real3d,ma2024cvthead,deng2024portrait4d,deng2024portrait4d2,GGHead} have explored one-shot 3D head reconstruction to address the limitations of data requirements and computational cost. These methods employ various techniques, such as tri-plane features, deformation fields, point-based expression fields, and vertex-feature transformers. Despite these advancements, NeRF-based methods often struggle with real-time rendering. 
Recently, 3D Gaussian Splatting~\cite{GaussianSplatting} has emerged as a promising alternative, offering both high-quality results and fast rendering speeds. However, existing Gaussian Splatting methods~\cite{GaussianAvatar,DBLP:conf/cvpr/XuCL00ZL24} typically rely on video data for training for each person, limiting their ability to generalize to new identities. Instead, the most recent work, GAGAvatar~\cite{GAGAvatar}, proposes a one-shot 3D Gaussian-based head avatar generation method. However, it still relies heavily on complex 2D neural post-processing to achieve optimal animation outcomes, thus it is not a pure 3D solution and the extra neural network hinders its application on various platforms. In contrast, our work generates Gaussian heads that are immediately animatable and renderable without additional networks or post-processing steps, enabling seamless integration into existing rendering pipelines for real-time animation and rendering across a wide range of platforms, including mobile phones. 
\section{Methods}
\label{section:methods}
\section{Proposed Method}
\label{sec:method}

In this section, we introduce our homotopy-based multi-objective framework for face parsing and its integration with both \textbf{GAN-based} and \textbf{diffusion-based} face editing models. We outline the problem formulation, dataset preparation, model architecture, training strategy, and evaluation pipeline, emphasizing \textbf{fairness}, \textbf{robustness}, and \textbf{semantic alignment}.

\subsection{Problem Formulation}
\label{subsec:problem_formulation}

We define the dataset \(\mathbf{X} = \{x_i\}\), where each face image is paired with a segmentation mask \( y_i \in \mathbf{Y} \), mapping to 19 facial components (e.g., hair, eyes, mouth). Demographic attributes are denoted as \(\mathbf{a}\) (e.g., \texttt{Male}, \texttt{Young}, \texttt{Wearing Hat}). Our objective is to train a segmentation function \( f_\theta(\cdot) \) that predicts \(\hat{y}_i\) while optimizing for accuracy, fairness, and robustness. Accuracy is maximized by aligning \(\hat{y}_i\) with \(y_i\) using Dice loss~\cite{sudre2017generalised}. Fairness is enforced by minimizing variance \(\mathrm{Var}(\mathrm{mIoU}_g)\) across demographic groups, ensuring equitable segmentation quality. Robustness is maintained by penalizing performance degradation (\(\mathrm{mIoU}\) drop) under input perturbations such as noise and occlusion.

\begin{algorithm}[h][t]
\caption{Multi-Objective Face Parsing (Pseudo-code)}
\label{alg:multi_objective_pseudocode}
\begin{algorithmic}[1]
\Require Homotopy function \(h(t)\) providing \((\alpha, \beta, \gamma)\) for epoch \(t\)
\For{epoch \(t = 1 \dots T\)}
    \State \((\alpha, \beta, \gamma) = h(t)\)
    \For{each batch in DataLoader}
        \State \textbf{Load} images \(\{x\}\), masks \(\{m\}\), attributes \(\{a\}\)
        \State outputs \(= f_{\theta}(x)\) \Comment{U-Net forward pass}
        \State \(\mathcal{L}_{\mathrm{acc}} = \mathrm{DiceLoss}(outputs, m)\)
        \State outputs\(_{\mathrm{noisy}} = outputs + \text{random\_noise}()\)
        \State \(\mathcal{L}_{\mathrm{rob}} = -\mathrm{mIoU}(\mathrm{softmax}(outputs_{\mathrm{noisy}}), m)\)
        \State \(\mathcal{L}_{\mathrm{fair}} = \mathrm{Var}\left[\mathrm{mIoU}_g\right]\)
        \State \(\mathcal{L}_{\text{total}} = \alpha\,\mathcal{L}_{\mathrm{acc}} + \beta\,\mathcal{L}_{\mathrm{rob}} + \gamma\,\mathcal{L}_{\mathrm{fair}}\)
        \State \textbf{Backward} and \textbf{update} \(\theta\)
    \EndFor
\EndFor
\end{algorithmic}
\end{algorithm}

\subsection{Dataset Preparation}
\label{subsec:dataset_preparation}

We employ the CelebAMask-HQ dataset \cite{CelebAMask-HQ}, divided into training, validation, and test sets. Each image and mask are resized to \(256 \times 256\) for compatibility with our U-Net architecture. Demographic attributes are extracted from annotations to compute fairness metrics.

\subsection{Model Architecture}
\label{subsec:model_architecture}

Our segmentation model utilizes a U-Net architecture with a ResNet-34 encoder pre-trained on ImageNet. It outputs 19 channels corresponding to distinct facial regions, balancing computational efficiency with high segmentation accuracy.

\subsection{Multi-Objective Training}
\label{subsec:multi_objective}

We train the U-Net segmentation models by optimizing a weighted sum of accuracy, fairness, and robustness losses, dynamically adjusted using homotopy-based scheduling. The training process is outlined in Algorithm~\ref{alg:multi_objective_pseudocode}.

\paragraph{Loss Components}
\begin{itemize}
    \item \textbf{Accuracy Loss (\(\mathcal{L}_{\mathrm{acc}}\)):} Dice loss measures the overlap between predicted and ground truth masks.
    \item \textbf{Robustness Loss (\(\mathcal{L}_{\mathrm{rob}}\)):} Negative \(\mathrm{mIoU}\) under perturbed predictions to ensure stability.
    \item \textbf{Fairness Loss (\(\mathcal{L}_{\mathrm{fair}}\)):} Variance of \(\mathrm{mIoU}\) across demographic groups to promote equitable performance.
\end{itemize}

\textbf{Alternative Fairness Computation:} We also compute per-group \(\mathrm{mIoU}\) for each demographic attribute, enabling detailed analysis of performance disparities (see Section~\ref{subsec:fairness_comparison}).

\subsection{Homotopy-Based Loss Scheduling}
\label{subsec:homotopy}

We dynamically balance the three loss components using epoch-dependent weights \(\alpha(t)\), \(\beta(t)\), and \(\gamma(t)\), ensuring \(\alpha(t) + \beta(t) + \gamma(t) = 1\). Initially, accuracy is prioritized, with weights shifting towards robustness and fairness over time. We explore three scheduling strategies:

\begin{itemize}
    \item \textbf{Linear:} \(\alpha(t)\) decreases linearly, while \(\beta(t)\) and \(\gamma(t)\) increase proportionally.
    \item \textbf{Sigmoid:} Smooth logistic transitions for gradual emphasis shifts.
    \item \textbf{Piecewise:} Abrupt changes in weight distribution at predefined training stages.
\end{itemize}

\begin{figure}[t]
    \centering
    \includegraphics[width=\columnwidth]{figures/parameters_by_homotopy-crop.pdf}
    \caption{Comparison of \(\alpha\), \(\beta\), and \(\gamma\) schedules across three homotopy methods (Linear, Sigmoid, and Piecewise) over 30 epochs. Each subplot illustrates the evolution of a parameter (\(\alpha\), \(\beta\), or \(\gamma\)) as it adapts during training, highlighting the differences in transition dynamics across homotopy strategies. The legend below the figure identifies the homotopy method for each curve.}
    \label{fig:homotopy-schedules}
\end{figure}


Figure~\ref{fig:homotopy-schedules} illustrates the evolution of these weights across training epochs for each homotopy method.

\subsection{Integration with Generative Models}

\subsubsection{GAN-Based Face Editing}
\label{subsec:gan_integration}

We utilize the trained U-Nets to generate segmentation maps for the training and validation sets, which are then used to train a Pix2PixHD GAN. The GAN architecture comprises:

\begin{itemize}
    \item \textbf{Generator} \(G\): Transforms segmentation maps into RGB images.
    \item \textbf{Discriminator} \(D\): Distinguishes real images from generated ones.
\end{itemize}

The GAN training involves a combination of adversarial loss and pixel-level \(L_1\) reconstruction loss:
\[
\mathcal{L}_{\mathrm{GAN}} = \mathcal{L}_{\mathrm{adv}}(G, D) + \lambda \, \|\hat{x} - x\|_1,
\]
where \(\hat{x} = G(\text{segmentation\_map})\) and \(x\) is the real image.

During testing, the GAN generates images using segmentation maps from the test set produced by both single-objective and multi-objective U-Nets, enabling evaluation of how segmentation quality impacts generative performance.

\subsubsection{ControlNet-Based Face Editing}
\label{subsec:controlnet_integration}

In addition to GANs, we integrate \textbf{ControlNet} \cite{zhang2023adding} for diffusion-based face editing. ControlNet leverages segmentation maps to guide the diffusion process, enhancing image fidelity and semantic alignment. Our setup includes:

\begin{itemize}
    \item \textbf{ControlNet Model:} Pre-trained on Stable Diffusion, fine-tuned on our segmentation maps.
    \item \textbf{Diffusion Pipeline:} Combines ControlNet with a text encoder and U-Net backbone to generate photorealistic faces conditioned on segmentation maps.
\end{itemize}

\textbf{Training Procedure:} ControlNet is fine-tuned for a single epoch using segmentation maps from the training set. In diffusion-based experiments, we compare only the single-objective model with the multi-objective linear homotopy model to manage computational resources effectively. The training minimizes the standard denoising loss:
\[
\mathcal{L}_{\mathrm{ControlNet}} = \mathcal{L}_{\mathrm{denoise}},
\]
where \(\mathcal{L}_{\mathrm{denoise}}\) is the Mean Squared Error between predicted and actual noise. During testing, ControlNet generates images using test set segmentation maps from both U-Net models, allowing assessment of segmentation quality's effect on diffusion-based generation.

\subsection{Evaluation Metrics and Setup}
\label{subsec:evaluation}

\paragraph{Segmentation Metrics}  
We evaluate segmentation performance using the mean Intersection-over-Union (\(\mathrm{mIoU}\)) across 19 facial classes. Fairness is quantified by the variance \(\mathrm{Var}(\mathrm{mIoU}_g)\) across demographic groups, and robustness is assessed through performance under Gaussian noise, occlusions, and blur.

\paragraph{Generative Metrics}  
For GAN outputs, we evaluate image quality using \textbf{Fréchet Inception Distance (FID)}, which quantifies realism by comparing feature distributions between generated and real images. Additionally, \textbf{Learned Perceptual Image Patch Similarity (LPIPS)} measures perceptual similarity, where lower scores indicate greater visual resemblance to real images.

\paragraph{Implementation Details}  
All experiments are implemented in PyTorch and trained on four NVIDIA A10 GPUs using the Adam optimizer with a learning rate of \(10^{-4}\). For ControlNet, we fine-tune the pre-trained \texttt{control\_v11p\_sd15\_seg} model based on Stable Diffusion v1.5. Our pipeline supports gradient accumulation and mixed precision (FP16) for computational efficiency. Homotopy-based loss scheduling is configurable (\texttt{linear}, \texttt{sigmoid}, \texttt{piecewise}). Detailed training configurations will be released alongside our code and models to ensure reproducibility.

\paragraph{Workflow Summary}  
\begin{enumerate}
    \item \textbf{Train U-Nets:} Train single-objective and multi-objective U-Nets on the training set, validate on the validation set.
    \item \textbf{Generate Segmentation Maps:} Use trained U-Nets to produce segmentation maps for training, validation, and test sets.
    \item \textbf{Train GAN:} Train the Pix2PixHD GAN using segmentation maps from the training and validation sets.
    \item \textbf{Fine-Tune ControlNet:} Fine-tune ControlNet on training set segmentation maps for one epoch.
    \item \textbf{Generate and Evaluate Images:} Generate images using GAN and ControlNet with test set segmentation maps from both U-Net models; evaluate using FID and LPIPS.
\end{enumerate}

In the following section, we present quantitative and qualitative results demonstrating the effectiveness of our approach across various conditions and demographic groups.

\section{Results}
\label{section:results}
\section{Results}\label{sect:results}

\subsection{Established Benchmarks}\label{sect:results_west}
We begin by evaluating all vision-language models on established benchmarks, based on ImageNet and COCO Captions, among other datasets. As revealed in Table~\ref{tab:west_standard_setup}, increasing the dataset size from 10 billion to 100 billion examples does not improve performance substantially. This is statistically supported by Wilcoxon's signed rank test~\cite{wilcoxon1992individual}, which gives a $p$-value of 0.9, indicating that differences are not significant.


In addition, we also fit data scaling laws for every combination of model and dataset following the recipe proposed in~\citet{alabdulmohsin2022revisiting}. This allows us to evaluate whether or not the performance gap is expected to increase or decrease in the infinite-compute regime. We report the resulting scaling exponents and asymptotic performance limits in the tables. Again, we do not observe  significant differences at the 95\% confidence level ($p$-value of 0.09).


\subsection{Cultural Diversity}
Unlike the Western-oriented metrics reported in Section~\ref{sect:results_west}, cultural diversity metrics present an entirely different picture. We observe \emph{notable} gains when scaling the size of the dataset from 10 billion to 100 billion examples in Table~\ref{tab:culture_standard_setup}. 
For example, scaling training data from 10 billion to 100 billion examples yields substantial gains on Dollar Street 10-shot classification task, where ViT-L and ViT-H see absolute improvements of 5.8\% and 5.4\%, respectively. These gains outperform the typical improvements (less than 1\%) observed on Western-oriented 10-shot metrics by a large margin.
Using Wilcoxon's signed rank test, we obtain a $p$-value of 0.002, indicating a statistically significant evidence at the 99\% confidence level.


\subsection{Multilinguality}

Our multilingual benchmark, Crossmodal-3600 zero-shot retrieval~\cite{thapliyal2022crossmodal}, shows a disparity in performance gains: low-resource languages benefit more from the 100 billion scale than the high-resource ones. The disparity, illustrated in Figure~\ref{fig:multilinguality}, which not only exists in all model sizes but also widens as the models become larger. Detailed results for each language can be found in Appendix~\ref{appendix:data_scale}.

% source: https://colab.corp.google.com/drive/1AKgGDITZqTC2hQjVc-Iv8xuysh5giP0i#scrollTo=2EtEXMbly8dB&line=1&uniqifier=1
\begin{figure}[h!]
    % \includegraphics[width=\linewidth]{figures/multilang-Average_Multilingual__Low-Resource_Lang.pdf}
    % \includegraphics[width=0.86\linewidth]{figures/multilang-Average_Multilingual__High-Resource_Lang.pdf}
    \includegraphics[width=\linewidth]{figures/multilang-Average_XM3600_Retrieval.pdf}
    \caption{Scaling up to 100B examples leads to more notable improvements in low-resource languages. $\Delta$ denotes the improved accuracy when scaling from 10B examples to 100B.}
    \label{fig:multilinguality}
\end{figure}


\subsection{Fairness}
For fairness, we report on 3 metrics discussed in Section~\ref{sect:evals}. 

\paragraph{Representation Bias.} The first metric is representation bias (RB), with results detailed in Table~\ref{tab:rb}. We observe that models trained on unbalanced web data have a significantly higher preference to associate a randomly chosen image from ImageNet~\cite{deng2009imagenet} with the label ``Male'' over the label ``Female.'' 

In fact, this occurs nearly 85\% of the time. Training on 100B examples does not mitigate this effect. This finding aligns with previous research highlighting the necessity of bias mitigation strategies, such as data balancing~\cite{alabdulmohsin2024clip}, to address inherent biases in web-scale datasets.

% \begin{table}[h]
%     \centering\scriptsize
%     \caption{representation bias with respect to gender using imagenet. Here, a value of 0.8, for example, indicates that the model would prefer to associate a randomly chosen image from ImageNet with the label ``Male'' over the label ``Female''.}
%     \label{tab:rb}
%     \begin{tabularx}{\columnwidth}{@{}c|YYY@{}}
%     \toprule
%     \bf Model&\bf1B &\bf10B &\bf100B\\
%     \midrule
% B & 83.2&84.5&85.2
% \\
% L & 88.2&86.4&85.5\\
% H & 86.8&85.0&86.6\\
% \bottomrule
%     \end{tabularx}
% \end{table}

\begin{table}[h]
\begin{tabularx}{\columnwidth}{c|YYY@{}}
    \toprule
    \bf Model&\bf1B &\bf10B &\bf100B\\
    \midrule
B & 83.2&84.5&85.2\\
L & 88.2&86.4&85.5\\
H & 86.8&85.0&86.6\\
\bottomrule
\end{tabularx}
\captionof{table}{Representation bias w.r.t. gender (see Section~\ref{sect:results}). Here,  values [\%] indicate how often the model prefers to associate a random  image with the label ``Male'' over ``Female''.} \label{tab:rb}
\end{table}



\paragraph{Association Bias.} Second, Figure~\ref{fig:ab} shows the association bias in SigLIP-H/14 between gender and occupation as we scale the data from 10 to 100 billion examples. Specifically, we plot the probability that the model would prefer a particular occupation label, such as ``{\fontfamily{lmodern}\selectfont secretary}'' over another label, such as ``{\fontfamily{lmodern}\selectfont manager}'' when images correspond to males or females. In this evaluation, we use the Fairface~\cite{karkkainen2021fairface} dataset. The labels we compare are: ``{\fontfamily{lmodern}\selectfont librarian}'' vs. ``{\fontfamily{lmodern}\selectfont scientist}'', ``{\fontfamily{lmodern}\selectfont nurse}'' vs. ``{\fontfamily{lmodern}\selectfont doctor}'', ``{\fontfamily{lmodern}\selectfont housekeeper}'' vs. ``{\fontfamily{lmodern}\selectfont homeowner}'', ``{\fontfamily{lmodern}\selectfont receptionist}'' vs. ``{\fontfamily{lmodern}\selectfont executive}'' and ``{\fontfamily{lmodern}\selectfont secretary}'' vs. ``{\fontfamily{lmodern}\selectfont manager}''. Again, we do not see a reduction in association bias by simply increasing the size of the training data. %Full results are in Appendix~\ref{appendix:ab}.

%Additionally, we are unable to evaluate cultural diversity and fairness in PaliGemma's transfer tasks due to the lack of appropriate benchmarks. This is an open question that we hope to address in the future.

\paragraph{Performance Disparity.} Finally, one common definition of fairness in machine learning is maintaining similar performance across different groups. See, for instance,~\citet{dehghani2023scaling} and the related notions of ``Equality of Opportunity'' and ``Equalized Odds''~\cite{hardt2016equalityopportunitysupervisedlearning}. Table~\ref{tab:perf_disparity} show that scaling the data to 100 billion examples improves performance disparity, which is consistent with the improvement in cultural diversity.

%  to show on top of page
% \begin{table}[h]
    \centering\scriptsize
    \begin{tabularx}{\columnwidth}{l|YYY@{}}
    \toprule
    \bf Model & \bf 1B & \bf 10B & \bf 100B\\ \midrule
    &\multicolumn{3}{c}{\em 0-shot Dollar Street}\\[2pt]
B & 32.5 & 29.9 & \bf29.0\\
L & \bf29.7 & 29.8 & 30.4 \\
H & 32.2 & 33.0 & \bf32.1\\
\midrule 
    &\multicolumn{3}{c}{\em 0-shot GeoDE}\\[2pt]
B & 4.7 & 5.5 & \bf4.4\\
L & 3.2 & 4.0 & \bf2.8 \\
H & 3.6 & 3.0 & \bf2.7\\
\bottomrule
 
    \end{tabularx}
    \caption{Performance disparity (lower is better) for models pretrained on 100B seen examples of different data scales. Pretraining on 100B examples tends to lower disparity.}
    \label{tab:per_disp_mini}
\end{table}
% \FloatBarrier

\begin{table*}[h]
    \centering\scriptsize
    \caption{Performance disparity results for various SigLIP models pretrained on 100 billion seen examples of 1B, 10B, and 100B datasets. Here, disparity corresponds to the maximum gap across subgroups in Dollar Street (by income level) and GeoDE (by geographic region). Pretraining on 100B examples tends to improve disparity overall.}
    \label{tab:perf_disparity}
    \begin{tabularx}{2\columnwidth}{ll|YYYYYY|Y}
    \toprule
    \bf Model & \bf Data Scale &\multicolumn{6}{c}{\bf Performance per Subgroup} & \bf Disparity\\ \midrule
    \multicolumn{8}{c}{\em 0-shot Dollar Street}\\[2pt]
    & & \bf 0-200	& \bf 200-685	& \bf 685-1998	& \bf $>$1998
    & & & \\ \midrule
B&1B&29.4&43.9&56.5&62.0&&&32.5\\
B&10B&31.6&44.0&55.4&61.5&&&29.9\\
B&100B&32.0&44.3&56.3&61.0&&&\bf29.0\\[3pt]
L&1B&33.7&44.7&57.3&63.4&&&\bf29.7\\
L&10B&35.7&47.8&58.7&65.5&&&29.8\\
L&100B&33.7&46.6&59.5&64.1&&&30.4\\[3pt]
H&1B&32.3&44.9&58.4&64.5&&&32.2\\
H&10B&33.9&46.3&58.6&66.9&&&33.0\\
H&100B&34.1&48.2&62.2&66.1&&&\bf32.1\\ \midrule

    \multicolumn{8}{c}{\em 0-shot GeoDE}\\[2pt]
    & & \bf Africa	& \bf Americas	& \bf East-Asia	& \bf Europe & \bf South-East Asia & \bf West Asia
    & \\ \midrule
B&1B&89.4&92.1&91.8&94.1&92.5&93.4&4.7\\
B&10B&88.4&91.8&91.4&94.0&92.2&93.0&5.5\\
B&100B&88.8&91.4&91.0&93.3&91.7&92.2&\bf4.4\\[3pt]
L&1B&92.0&94.0&94.0&95.2&94.2&94.9&3.2\\
L&10B&91.8&94.4&94.0&95.8&94.2&94.7&4.0\\
L&100B&93.5&95.1&95.4&96.2&95.0&95.8&\bf2.8\\[3pt]
H&1B&91.5&94.4&94.7&95.2&94.1&94.5&3.6\\
H&10B&93.4&95.4&95.0&96.5&95.1&95.6&3.0\\
H&100B&93.6&95.1&95.3&96.3&95.2&95.8&\bf2.7\\

 \bottomrule
    \end{tabularx}
\end{table*}


\subsection{Transfer To Generative Models}
\label{sec:transfer}

\begin{table}[h!]
\centering
\footnotesize

% Note: We removed 1b result to avoid confusion to readers. See https://docs.google.com/document/d/1YxRpUO7elSaviOQ5XIXtnUWajgYAF7FfRO0vmXQCUtU/edit?resourcekey=0-pPjeeIrYEXRuvnuBFZn5Uw&tab=t.0#heading=h.gaczi2wqv0go.
\begin{tabular}{p{0pt}l|rrrrr}
\toprule
& Data & Semantics & OCR & Multiling & RS & Avg \\
\midrule
% % source: https://docs.google.com/spreadsheets/d/1W5_VNitkO6k-HSKBV6sGX81EGXgma877m0gGsg8zPzw/edit?resourcekey=0-2zH_U5z5kL9I5Nnhg7SChQ&gid=1972601917#gid=1972601917
\includegraphics[width=8pt]{images/snowflake_2744-fe0f.png} & 1B & 76.0 & 66.8 & 67.0 & 92.3 & 73.6 \\
\includegraphics[width=8pt]{images/snowflake_2744-fe0f.png} & 10B & 75.4 & 65.2 & 66.3 & 91.9 & 72.7 \\
\includegraphics[width=8pt]{images/snowflake_2744-fe0f.png} & 100B & 76.4 & 67.0 & 66.9 & 92.1 & 73.9 \\
\includegraphics[width=8pt]{images/fire_1f525.png} & 1B & 77.1 & 69.5 & 66.9 & 92.0 & 75.1 \\
\includegraphics[width=8pt]{images/fire_1f525.png} & 10B & 76.4 & 66.9 & 66.0 & 91.8 & 73.7 \\
\includegraphics[width=8pt]{images/fire_1f525.png} & 100B & {77.2} & {70.0} & {67.0} & {91.8} & {75.3} \\
\bottomrule
\end{tabular}

\caption{
The PaliGemma transfer results of ViT-L/16 models pretrained on 10B and 100B examples, with both frozen (top) %({\includegraphics[width=8pt]{images/snowflake_2744-fe0f.png}}) 
and unfrozen (bottom) %({\includegraphics[width=8pt]{images/fire_1f525.png}}) 
vision components. Results are aggregated.
}
\label{tab:transfer_avg}
\end{table}


We use PaliGemma~\citep{beyer2024paligemma} with both frozen and unfrozen vision component to assess the transferability of our vision models, which were contrastively pre-trained on datasets of different scales. In Table~\ref{tab:transfer_avg}, when taking the noise level into consideration, we do not observe consistent performance gains across downstream tasks as we scale the pre-training dataset. More details can be found in Appendix~\ref{appendix:transfer}.


%\paragraph{Recognizing Sensitive Attributes.}
%Finally, we also report the performance of the models in recognizing sensitive attributes, following a similar evaluation in~\citet{radford2021learning}. We report the accuracy in predicting perceived gender in Fairface~\cite{karkkainen2021fairface} and predicting perceived race in UTK~\cite{utkface_url}. Overall, we observe that scaling the data to 100 billion examples improves this aspect of fairness as well. Table~\ref{tab:fairness_pred} provides the full results. We do not observe a particular pattern in this type of evaluation.
%\begin{table}[t]
    \centering\scriptsize
    \begin{tabularx}{\columnwidth}{@{}ll|YY@{}}
    \toprule
    \bf Model & \bf Data Scale & \bf Gender & \bf Race\\ \midrule
B	&	1B	& 91.0	& 58.6\\
B	&	10B	& 91.7 &	\bf59.5\\
B	&	100B &	\bf91.9	& 53.1\\[3pt]
L	&	1B	& 0.94.8	& \bf55.4\\
L	&	10B	& 93.9	& 53.1\\
L	&	100B &	\bf95.0	& 54.0\\[3pt]
H	&	1B	& 94.5	& \bf54.5\\
H	&	10B	& \bf95.4	& 54.3\\
H	&	100B	& \bf95.4	& 50.2 \\
 \bottomrule
    \end{tabularx}
    \caption{Accuracy in recognizing sensitive attributes using Fairface and UTK datasets. See Section~\ref{sect:results} for details.}
    \label{tab:fairness_pred}
\end{table}

\section{Discussion}
\label{section:discussion}
\section{Discussion}
\subsection{Case Study}


Fig. \ref{fig:casestudy} shows 2-D UMAP \cite{mcinnes2020umapuniformmanifoldapproximation} projections of embedding vectors for PetClinic Microservices \cite{microapps2024petclinic} using VoyageAI, ME-unixcoder-340K, and ME-llm2vec-340K. ME-unixcoder and ME-llm2vec show clearer microservice clusters compared to VoyageAI and Fig. \ref{fig:mexample}. For instance, \textit{API-Gateway} service classes are split in VoyageAI's representation but closer in the other models. ME-llm2vec demonstrates the closest grouping within microservices and clearest separation between them. In fact, ME-llm2vec's figure shows only 6 clear outliers which we review in detail and display their names and neighbors.



The two \textit{MetricConfig} classes, \textit{ResourceNotFoundException} and \textit{CacheConfig} lack domain-specific terms since they are utility classes, which highlights the importance of separating them from domain-related ones during the decomposition. However, ME-llm2vec was able correctly represent classes with even slight domain hints. For instance, most models struggle to differentiate between the nearly identical entry-point classes (e.g. \textit{ConfigServerApplication}), as seen in Fig. \ref{fig:mexample} and \ref{fig:casestudy} while ME-llm2vec managed to correctly place them within their services. On the other hand, the class \textit{PetRequest}, which was closer to \textit{API-Gateway} instead of \textit{Customers}, shows an intriguing outlier. Despite ME-llm2vec correctly matching the "Pet" related classes, it failed with \textit{PetRequest}. its function as a Request object, which is typically associated with the Gateway pattern, is a potential reason. Notably, ME-llm2vec successfully identified \textit{API-Gateway} classes, differentiating them from \textit{Customers}. We find this interesting because \textit{API-Gateway} includes classes representing various bounded contexts, often causing confusion in other models. ME-llm2vec recognized these classes' distinct purpose, grouping them together despite their diverse domains.

% Both \textit{API-Gateway} and \textit{Customers} services contain a "PetType" class. But in \textit{Customers}'s case, this class was closer to the "Specialty" class from \textit{Vets}, which is likely due to nearly identical source code they have. 

\subsection{Discussion}


We designed the analysis component to be as abstract as possible to accommodate the rapidly evolving representation learning landscape. As new and improved embedding models are published, they can be integrated with minimal effort. While our evaluation results show that with ME-LLM2Vec, we can generate highly cohesive and consistent decompositions, one of our objectives is to highlight the potential of Language Models in generating more efficient representations than traditional approaches for the decomposition problem. In fact, MonoEmbed is both a decomposition approach (when considering the full approach) and an embedding model (when using models such as ME-LLM2Vec). These models can be used to enrich existing decomposition approaches. For example, MicroMiner's CodeBERT \cite{trabelsi2023microminer} can be replaced with ME-LLM2Vec and the GNN based methods \cite{desai2021cogcn,yedida2023deeply,mathai2022chgnn,qian2023gdcdvf} can be extended by using ME-LLM2Vec as the encoder. In fact, it can be used as an additional representation type in approaches such as \cite{khaled2022hydecomp,qian2023gdcdvf}. These models can be even extended further by incorporating unstructured inputs (e.g. resources, configurations, documentation) and different PLs.




\subsection{Threats to Validity}
\subsubsection{Internal Validity}
Clustering algorithms and decomposition approaches have hyper-parameters that can affect performance on evaluation benchmarks. To mitigate this threat, we compared their performance with different hyper-parameter inputs across a varied set of evaluation applications.

\subsubsection{External Validity}
To address the threat of our approach to generalize on monolithic applications and PLs, we used a large set of monolithic and microservices applications from related work \cite{kalia2021mono2micro,khaled2022hydecomp,yedida2023deeply,jin2021fosci} to benchmark decomposition approaches. 

\subsubsection{Construct Validity}
This threat can potentially be in the form of the evaluation metrics used in our experiments. In order to mitigate this threat, we employ established metrics in supervised learning tasks (RQ1-3) and different metrics from decomposition research \cite{khaled2022hydecomp,kalia2021mono2micro,jin2021fosci,yedida2023deeply,mathai2022chgnn} (RQ4). 



% Acknowledgements should only appear in the accepted version.
\section*{Acknowledgements}
This work was partially supported by the NSF (awards 2406231 and 2427948), NIH (awards R01NS124642 and R01DK131586), DARPA (HR00112420329), and the US Army (W911NF-20-D0002).

\section*{Impact Statement}
This study investigates whether time series models can perform implicit reasoning during zero-shot inference by synthesizing learned concepts to generalize to more complex data patterns. Our findings reveal that certain models can generalize effectively in well-designed OOD scenarios, highlighting reasoning abilities that go beyond basic pattern memorization. As this work is exploratory and not tied to specific applications, we do not foresee any negative societal impacts. In contrast, our insights could aid in the development of more data- and computationally-efficient deep learning architectures. Additionally, our results can help identify the limitations of time series models, ensuring that they are not used in scenarios where poor generalization is likely.

\section*{Reproducibility Statement}

The models were implemented using the \texttt{Neuralforecast} library~\citep{olivares2022library_neuralforecast}. The open-source code for generating critical difference diagrams is available at \url{https://github.com/hfawaz/cd-diagram}. All datasets used in this study are publicly accessible and can be downloaded by following the instructions at \url{https://github.com/SalesforceAIResearch/gift-eval}. Model training and evaluation were conducted on a computing cluster equipped with 128 AMD EPYC 7502 CPUs, 503 GB of RAM, and 8 NVIDIA RTX A6000 GPUs, each with 49 GiB of RAM. To ensure reproducibility, the model implementations, training framework, and datasets are open-source, supporting future research on model reasoning. The complete codebase is available at %\url{https://anonymous.4open.science/r/reasoning_ICML_paper_submission}. 
\url{https://github.com/PotosnakW/neuralforecast/tree/tsfm_reasoning}. 


% In the unusual situation where you want a paper to appear in the
% references without citing it in the main text, use \nocite


\bibliography{references}
\bibliographystyle{icml2025}


%%%%%%%%%%%%%%%%%%%%%%%%%%%%%%%%%%%%%%%%%%%%%%%%%%%%%%%%%%%%%%%%%%%%%%%%%%%%%%%
%%%%%%%%%%%%%%%%%%%%%%%%%%%%%%%%%%%%%%%%%%%%%%%%%%%%%%%%%%%%%%%%%%%%%%%%%%%%%%%
% APPENDIX
%%%%%%%%%%%%%%%%%%%%%%%%%%%%%%%%%%%%%%%%%%%%%%%%%%%%%%%%%%%%%%%%%%%%%%%%%%%%%%%
%%%%%%%%%%%%%%%%%%%%%%%%%%%%%%%%%%%%%%%%%%%%%%%%%%%%%%%%%%%%%%%%%%%%%%%%%%%%%%%
\newpage
\appendix
\onecolumn

\section{Supplemental Information}
\section{Proofs from Section~\ref{sec:gammaok}} \label{app:gamma}

\subsection{On the girth of locally \texorpdfstring{$\gamma$}{gamma}-sparse graphs}
\begin{lemma}\label{lemma:girth_rev}
    Let $G = (V,E)$ be an undirected graph with girth $g(G)$.
    Then $G$ is \ok{0} if and only if $g(G) \geq 5$.
\end{lemma}
\begin{proof}
    We first prove that if $G$ is \ok{0} then $g(G)$ must be at least $5$.
    In order to prove that, we simply negate the statement and prove that if $G$ has girth $<5$ then $G$ can not be \ok{0}.
    Without loss of generality, assume that $g(G) = 4$ (the case $g(G) = 3$ is similar).
    Then there must exist a cycle $C = (u_1, u_2, u_3, u_4)$ of $4$ vertices.
    It is simple to see that $u_2,u_4 \in \lset_1(u_1)$ and $u_3 \in \lset_2(u_1)$.
    Since $u_3$ is a neighbor of both $u_2$ and $u_4$, the degree of $u_3$ in the subgraph $G\left[\lset_1(u_1) \cup \{u_4\} \right]$ is at least $2$, hence $G$ is not \ok{0} (see \Cref{subfig:girth1}).
    
    We now prove that if $g(G) \geq 5$ then $G$ must be \ok{0}.
    Again, we negate this statement and prove that if $G$ is not \ok{0} then the girth of $G$ must be less then $5$.
    Let us assume that $G$ is \gammaok, for any $\gamma > 0$, thus it is not \ok{0}.
    Since $G$ is not \ok{0} there exists a vertex $v \in V$ such that at least one of the following properties holds (see \Cref{subfig:girth2}):
    \begin{enumerate}
        \item $\exists u \in \lset_1(v)$ such that the degree of $u$ in $G\left[ \lset_1(v) \right]$ is greater then $0$, or;
        \item $\exists w \in \lset_2(v)$ such that the degree of $w$ in $G\left[ \lset_1(v) \cup \{ w \} \right]$ is greater then $1$.
    \end{enumerate}
    In the first case, we have a cycle of $3$ vertices, then $g(G) = 3$.
    In the second case, we have a cycle of $4$ vertices, then $g(G) = 4$.
    In both cases $g(G) < 5$.
\end{proof}
\begin{figure}[h]
    \centering
    \begin{subfigure}[b]{0.35\linewidth}
            \centering
            \includegraphics[width=\linewidth]{img/girth-1.pdf}
            \caption{}
            \label{subfig:girth1}
    \end{subfigure}
    \begin{subfigure}[b]{0.6\linewidth}
            \centering
            \includegraphics[width=\linewidth]{img/girth-2.pdf}
            \caption{}
            \label{subfig:girth2}
    \end{subfigure}%
    \caption{}
    \label{fig:example_girth}
\end{figure}

\subsection{Deterministic lazy-update on \texorpdfstring{$\gamma$}{gamma}-sparse graphs}\label{apx:gamma-ok-deterministic}

\begin{theorem}\label{lemma:gamma-ok-error-bound-balls}
    
Let $\varepsilon \in (0,1)$, and let $G^{(0)}$ be an initial graph. Consider any sequence of edge insertions that yields a final graph $G$. If $G$ is \gammaok, \lazyscheme$(\varphi = \frac{\varepsilon}{1 - \varepsilon},k=0)$ has an approximation ratio of  $\frac{\gamma + 1}{1-\varepsilon}$ and amortized update cost $O(1/\varepsilon)$. 
    
\end{theorem}
\begin{proof}
Recall that $\bd_u$ denotes the black degree of $u$, and that  \Cref{alg:det_thresh} guarantees that $\deg_u$ is at most $(1+\varphi)\bd_u$.
    Then, it is simple to give an upper bound to the size of $\ball_2(u)$, that is $\vert \ball_2(u) \vert \leq 1+ \sum_{v \in \lset_1(u)} (1 + \varphi)\bd_v$.Consider a vertex $v \in \lset_1(u)$. Since $G$ is \gammaok, the number of neighbors of $v$ belonging to $\lset_2(u)$ is at lest $\deg_v - (\gamma+1)$ of which $\bd_v - (\gamma+1)$ must belong to $\apxball_2(u)$. Moreover, a vertex in $\lset_2(u)$ has at most $\gamma+1$ neighbors in $\lset_1(u)$. Therefore: 
    \begin{align*}
    \vert \apxball_2(u) \vert
    &\geq  \bd_u + 1 + \frac{1}{\gamma + 1}\sum_{v \in \lset_1(u)}(\bd_v - (\gamma + 1))\\
    &= \bd_u + 1 + \frac{1}{\gamma + 1}\sum_{v \in \lset_1(u)}\bd_v - \underbrace{\frac{1}{\gamma + 1}\sum_{v \in \lset_1(u)}(\gamma + 1)}_{= \bd_u}\\
    &= 1+ \frac{1}{\gamma + 1}\sum_{v \in \lset_1(u)}\bd_v.
    \end{align*}
  
    As a consequence, $\vert \apxball_2(u) \vert/\vert \ball_2(u) \vert \ge \frac{1}{(1+\varphi)(\gamma+1)}$. By setting $\varphi = \frac{\varepsilon}{1 - \varepsilon}$, and by using \Cref{lm:amortized_det_alg},  the claim follows.
\end{proof}

\subsection{Proof of \Cref{le:gamma_ok_expect_lowerbound}}\label{apx:proof_gamma_ok_expect_lowerbound}
\begin{proof}
Let $e_1, \dots, e_{\ell_v}$ be the \emph{red edges} between $v$ and $\lset_2(u)$, and define the binary random variable $\lrdr_v(i)$ that is equal to $1$ if $e_i$ is a \emph{quasi-black edge} for $u$, $0$ otherwise, for $i = 1, \dots, \lrd_v$. Thus we can express $\lrdr_v = \sum_{i=1}^{\lrd_v} \lrdr_v(i)$, with expectation

\begin{equation}\label{eq:gamma_ok_lb_fact_eq_1}
\begin{aligned}
  \Expec{}{\lrdr_v} & = \sum_{i=1}^{\lrd_v}{\Prob{}{\lrdr_v(i)=1}} = \lrd_v - \sum_{i=1}^{\lrd_v} {\Prob{}{\lrdr_v(i)=0}}.
\end{aligned}
\end{equation}

Without loss of generality, assume that the edges $e_1, \dots, e_{\lrd_v}$ have been inserted at times $t_1 < \dots < t_{\lrd_v}$, respectively.
If $e_i$ is not a quasi-black edge for $u$, then it must be that $u$ is not selected by $v$ at \Cref{line:random_selection} of \Cref{alg:det_thresh}, at times $t_i, t_{i+1},\dots, t_{\lrd_v}$.
This holds with probability 
\begin{equation}\label{eq:gamma_ok_lb_fact_eq_2}
\begin{aligned} 
    &\Prob{}{\lrdr_v(i) = 0}
    \leq \prod_{j=i}^{\lrd_v} \left( 1-\frac{k}{\deg_v^{(t_j)}} \right)
    \leq \prod_{j=i}^{\lrd_v} \left( 1 - \frac{k}{\deg_{v}^{(t_{\lrd_v})}} \right) \\
    &\leq \left( 1-\frac{k}{\lbdd_v + \lrd_v + \gamma + 1}\right)^{\lrd_v - i + 1} 
    \leq \left(1-\frac{k}{2(\lbdd_v + \gamma + 1)}\right)^{\lrd_v - i}.
\end{aligned}
\end{equation}
The third inequality holds since the edges incident to $v$ having endpoints in $L_1(u)$ are at most $\gamma$, while those having endpoints in $L_2(u)$ are exactly $\lbdd_v+ \lrd_v$. Moreover, the last inequality holds because $\lrd_v \leq \rd_v \leq \bd_v \leq \lbdd_v + \gamma + 1$, given the assumption $\varphi = 1$.

By plugging in \eqref{eq:gamma_ok_lb_fact_eq_2} into   \eqref{eq:gamma_ok_lb_fact_eq_1} and we obtain
\begin{align*}
    &\Expec{}{\lrdr_v} \geq \lrd_v - \sum_{i=1}^{\lrd_v}\left( 1-\frac{k}{2(\lbdd_v + \gamma + 1)}\right)^{\lrd_v - i} \\
    &= \lrd_v - \sum_{i=0}^{\lrd_v-1} \left(1-\frac{k}{2(\lbdd_v + \gamma + 1)}\right)^i 
    \leq \lrd_v - \frac{1-\left(1-\frac{k}{2(\lbdd_v+\gamma+1)}\right)^{\lrd_v}}{1-\left(1-\frac{k}{2(\lbdd_v + \gamma + 1)}\right)} \\
    &\geq \lrd_v - \frac{1}{1-\left(1-\frac{k}{2(\lbdd_v + \gamma + 1)}\right)}
    \geq \lrd_v - \frac{2(\lbdd_v + \gamma + 1)}{k}.
\end{align*}
\end{proof}\label{section:appendixA}
%\newpage
\section{Supplemental Results}
\subsection{Model Rank and Pairwise Comparisons}\label{apd:cd_diagrams}
\begin{figure*}[htbp]
    \centering
   \begin{minipage}{0.48\textwidth}
        \centering
        \includegraphics[width=\textwidth]{images/cd_baselines_aggregate.pdf}
        \subcaption{Model choice for in-distribution series (p-value: 2.71e-8)}
        \label{fig:cd_baselines_aggregate}
    \end{minipage}%
    \hfill
    \begin{minipage}{0.48\textwidth}
        \centering
        \includegraphics[width=\textwidth]{images/cd_baselines_component.pdf}
        \subcaption{Model choice for out-of-distribution series (p-value: 3.52e-8)}
        \label{fig:cd_baselines_component}
    \end{minipage}

\caption{Critical Difference (CD) diagrams illustrate model ranks and pairwise statistical comparisons of model performance on compositional reasoning tasks across all datasets. Lower ranks indicate better performance. A thick horizontal line groups models that are not significantly different. The statistical tests used to generate the CD diagrams are detailed in Section \ref{section:evaluation}. \textbf{(a, b)} The patch-based Transformer models and MLP-based models outperform other models in both traditional and compositional reasoning forecasting paradigms. The Friedman p-value is included in the subcaptions.}
    \label{fig:cd_diagrams_baselines_apd}
\end{figure*}



\begin{figure*}[htbp]
    \centering

    \vspace{1ex} % Vertical space between rows

    \begin{minipage}{0.48\textwidth}
        \centering
        \includegraphics[width=\textwidth]{images/cd_tokenization_ablation_aggregate.pdf}
        \subcaption{Tokenization for in-distribution series (p-value=2.91e-3)}
        \label{fig:cd_tokenization_ablation_aggregate}
    \end{minipage}%
    \hfill
    \begin{minipage}{0.48\textwidth}
        \centering
        \includegraphics[width=\textwidth]{images/cd_tokenization_ablation_component.pdf}
        \subcaption{Tokenization for out-of-distribution series (p-value=1.17e-2)}
        \label{fig:cd_tokenization_ablation_component}
    \end{minipage}

    \vspace{1ex} % Vertical space between rows

    \begin{minipage}{0.48\textwidth}
        \centering
        \includegraphics[width=\textwidth]{images/cd_size_ablation_aggregate.pdf}
        \subcaption{Model size for in-distribution series (p-value: 6.58e-2)}
        \label{fig:cd_size_ablation_aggregate}
    \end{minipage}%
    \hfill
    \begin{minipage}{0.48\textwidth}
        \centering
        \includegraphics[width=\textwidth]{images/cd_size_ablation_component.pdf}
        \subcaption{Model Size for out-of-distribution series (p-value=8.58e-2)}
        \label{fig:cd_size_ablation_component}
    \end{minipage}

    \vspace{1ex} % Vertical space between rows

    \begin{minipage}{0.48\textwidth}
        \centering
        \includegraphics[width=\textwidth]{images/cd_attn_ablation_aggregate.pdf}
        \subcaption{Attn. type for in-distribution series}
        \label{fig:cd_attn_ablation_aggregate}
    \end{minipage}%
    \hfill
    \begin{minipage}{0.48\textwidth}
        \centering
        \includegraphics[width=\textwidth]{images/cd_attn_ablation_component.pdf}
        \subcaption{Attn. type for out-of-distribution series}
        \label{fig:cd_attn_ablation_component}
    \end{minipage}

    \vspace{1ex} % Vertical space between rows

    \begin{minipage}{0.48\textwidth}
        \centering
        \includegraphics[width=\textwidth]{images/cd_proj_ablation_aggregate.pdf}
        \subcaption{Projection layer for in-distribution series}
        \label{fig:cd_proj_ablation_aggregate}
    \end{minipage}%
    \hfill
    \begin{minipage}{0.48\textwidth}
        \centering
        \includegraphics[width=\textwidth]{images/cd_proj_ablation_component.pdf}
        \subcaption{Projection layer for out-of-distribution series}
        \label{fig:cd_proj_ablation_component}
    \end{minipage}

    \vspace{1ex} % Vertical space between rows

    \begin{minipage}{0.48\textwidth}
        \centering
        \includegraphics[width=\textwidth]{images/cd_tokenlen_ablation_aggregate.pdf}
        \subcaption{Token length for in-distribution series (p-value=0.22)}
        \label{fig:cd_tokenlen_ablation_aggregate}
    \end{minipage}%
    \hfill
    \begin{minipage}{0.48\textwidth}
        \centering
        \includegraphics[width=\textwidth]{images/cd_tokenlen_ablation_component.pdf}
        \subcaption{Token length for out-of-distribution series (p-value=8.62e-4)}
        \label{fig:cd_tokenlen_ablation_component}
    \end{minipage}

    \vspace{1ex} % Vertical space between rows

    \begin{minipage}{0.48\textwidth}
        \centering
        \includegraphics[width=\textwidth]{images/cd_pe_ablation_aggregate.pdf}
        \subcaption{Positional encoding for in-distribution series (p-value=0.71)}
        \label{fig:cd_pe_ablation_aggregate}
    \end{minipage}%
    \hfill
    \begin{minipage}{0.48\textwidth}
        \centering
        \includegraphics[width=\textwidth]{images/cd_pe_ablation_component.pdf}
        \subcaption{Positional encoding for out-of-distribution series (p-value=0.46)}
        \label{fig:cd_pe_ablation_component}
    \end{minipage}
    
    \vspace{1ex} % Vertical space between rows

    \begin{minipage}{0.48\textwidth}
        \centering
        \includegraphics[width=\textwidth]{images/cd_loss_ablation_aggregate.pdf}
        \subcaption{Loss function for in-distribution series (p-value=0.24)}
        \label{fig:cd_loss_ablation_aggregate}
    \end{minipage}%
    \hfill
    \begin{minipage}{0.48\textwidth}
        \centering
        \includegraphics[width=\textwidth]{images/cd_loss_ablation_component.pdf}
        \subcaption{Loss function for out-of-distribution series (p-value=0.14)}
        \label{fig:cd_loss_ablation_component}
    \end{minipage}

    \vspace{1ex} % Vertical space between rows

    \begin{minipage}{0.48\textwidth}
        \centering
        \includegraphics[width=\textwidth]{images/cd_scaler_ablation_aggregate.pdf}
        \subcaption{Scaler for in-distribution series}
        \label{fig:cd_scaler_ablation_aggregate}
    \end{minipage}%
    \hfill
    \begin{minipage}{0.48\textwidth}
        \centering
        \includegraphics[width=\textwidth]{images/cd_scaler_ablation_component.pdf}
        \subcaption{Scaler function for out-of-distribution series}
        \label{fig:cd_scaler_ablation_component}
    \end{minipage}

    \vspace{1ex} % Vertical space between rows

    \begin{minipage}{0.48\textwidth}
        \centering
        \includegraphics[width=\textwidth]{images/cd_contextlen_ablation_aggregate.pdf}
        \subcaption{Context length function for in-distribution series}
        \label{fig:cd_contextlen_ablation_aggregate}
    \end{minipage}%
    \hfill
    \begin{minipage}{0.48\textwidth}
        \centering
        \includegraphics[width=\textwidth]{images/cd_contextlen_ablation_component.pdf}
        \subcaption{Context length for out-of-distribution series}
        \label{fig:cd_contextlen_ablation_component}
    \end{minipage}

    \vspace{1ex} % Vertical space between rows

    \begin{minipage}{0.48\textwidth}
        \centering
        \includegraphics[width=\textwidth]{images/cd_decomp_ablation_aggregate.pdf}
        \subcaption{Input decomposition for in-distribution series}
        \label{fig:cd_decomp_ablation_aggregate}
    \end{minipage}%
    \hfill
    \begin{minipage}{0.48\textwidth}
        \centering
        \includegraphics[width=\textwidth]{images/cd_decomp_ablation_component.pdf}
        \subcaption{Input decomposition function for out-of-distribution series}
        \label{fig:cd_decomp_ablation_component}
    \end{minipage}

    \caption{Critical Difference (CD) diagrams illustrate model ranks and pairwise statistical comparisons of model performance on compositional reasoning tasks across all datasets. Lower ranks indicate better performance. A thick horizontal line groups models that are not significantly different. The statistical tests used to generate the CD diagrams are detailed in Section \ref{section:evaluation}. For analyses comparing three or more methods, the Friedman p-value is included in the subcaptions.}
    \label{fig:cd_diagrams_apd}
\end{figure*}

\newpage
\subsection{Model Forecasts}\label{apd:forecast_examples}
We include example forecasts for each of the 6 datasets used in this study.


\begin{figure*}[ht!]
    \centering

    \begin{minipage}{0.411\textwidth}
        \centering
        \includegraphics[width=\textwidth, trim=0 0 310 0, clip]{images/sine_id_forecast_example.pdf}
        \subcaption{Model forecasts for in-distribution Sinusoid series}
        \label{fig:sine_id_forecast_example}
    \end{minipage}
    \hfill
    \begin{minipage}{0.584\textwidth}
        \centering
        \includegraphics[width=\textwidth]{images/sine_ood_forecast_example.pdf}
        \subcaption{Model forecasts for out-of-distribution Sinusoid series}
        \label{fig:sine_ood_forecast_example.pdf}
    \end{minipage}

    \vspace{2.9em}
    
    \begin{minipage}{0.411\textwidth}
        \centering
        \includegraphics[width=\textwidth, trim=0 0 310 0, clip]{images/ecl_id_forecast_example.pdf}
        \subcaption{Model forecasts for in-distribution ECL series
        }
        \label{fig:ecl_id_forecast_example}
    \end{minipage}
    \hfill
    \begin{minipage}{0.584\textwidth}
        \centering
        \includegraphics[width=\textwidth]{images/ecl_ood_forecast_example.pdf}
        \subcaption{Model forecasts for out-of-distribution ECL series}
        \label{fig:ecl_ood_forecast_example.pdf}
    \end{minipage}

    \vspace{2.9em}
    
    \begin{minipage}{0.411\textwidth}
        \centering
        \includegraphics[width=\textwidth, trim=0 0 310 0, clip]{images/ettm2_id_forecast_example.pdf}
        \subcaption{Model forecasts for in-distribution ETTm2 series
        }
        \label{fig:ettm2_id_forecast_example}
    \end{minipage}
    \hfill
    \begin{minipage}{0.584\textwidth}
        \centering
        \includegraphics[width=\textwidth]{images/ettm2_ood_forecast_example.pdf}
        \subcaption{Model forecasts for out-of-distribution ETTm2 series}
        \label{fig:ettm2_ood_forecast_example.pdf}
    \end{minipage}

    \caption{\textbf{(a, c, e)} Forecasts for a ground truth series $\mathbf{y}(t)$ for the Sinusoid, ETTm2, and ECL datasets for models trained using the traditional forecasting paradigm. \textbf{(b, d, f)} Forecasts for the Sinusoid, ETTm2, and ECL datasets for models trained using the compositional reasoning forecasting paradigm. Patch-based Transformer models and MLP-based models (top), which rank among the top-performing models, demonstrate generalization to out-of-distribution time series, whereas other Transformer variants and linear models struggle to do so (bottom).}
    \label{fig:forecast_examples1_apd}
\end{figure*}

\newpage
\begin{figure*}[ht!]
    \centering
    \begin{minipage}{0.411\textwidth}
        \centering
        \includegraphics[width=\textwidth, trim=0 0 310 0, clip]{images/solar_id_forecast_example.pdf}
        \subcaption{Model forecasts for in-distribution Solar series
        }
        \label{fig:solar_id_forecast_example}
    \end{minipage}
    \hfill
    \begin{minipage}{0.584\textwidth}
        \centering
        \includegraphics[width=\textwidth]{images/solar_ood_forecast_example.pdf}
        \subcaption{Model forecasts for out-of-distribution Solar series}
        \label{fig:solar_ood_forecast_example.pdf}
    \end{minipage}

    \vspace{2.9em}
    
    \begin{minipage}{0.411\textwidth}
        \centering
        \includegraphics[width=\textwidth, trim=0 0 310 0, clip]{images/subseasonal_id_forecast_example.pdf}
        \subcaption{Model forecasts for in-distribution Subseasonal series
        }
        \label{fig:subseasonal_id_forecast_example}
    \end{minipage}
    \hfill
    \begin{minipage}{0.584\textwidth}
        \centering
        \includegraphics[width=\textwidth]{images/subseasonal_ood_forecast_example.pdf}
        \subcaption{Model forecasts for out-of-distribution Subseasonal series}
        \label{fig:subseasonal_ood_forecast_example.pdf}
    \end{minipage}

    \vspace{2.9em}

    \begin{minipage}{0.411\textwidth}
        \centering
        \includegraphics[width=\textwidth, trim=0 0 310 0, clip]{images/loopseattle_id_forecast_example.pdf}
        \subcaption{Model forecasts for in-distribution Loop Seattle series
        }
        \label{fig:loopseattle_id_forecast_example}
    \end{minipage}
    \hfill
    \begin{minipage}{0.584\textwidth}
        \centering
        \includegraphics[width=\textwidth]{images/loopseattle_ood_forecast_example.pdf}
        \subcaption{Model forecasts for out-of-distribution Loop Seattle series}
        \label{fig:loopseattle_ood_forecast_example.pdf}
    \end{minipage}

    \caption{\textbf{(a, c, e)} Forecasts for a ground truth series $\mathbf{y}(t)$ for the Solar, Subseasonal, and Loop Seattle datasets for models trained using the traditional forecasting paradigm. \textbf{(b, d, f)} Forecasts for a ground truth series $\mathbf{y}(t)$ for the Solar, Subseasonal, and Loop Seattle datasets for models trained using the compositional reasoning forecasting paradigm. Patch-based Transformer models and MLP-based models (top), which rank among the top-performing models, demonstrate generalization to out-of-distribution time series, whereas other Transformer variants and linear models struggle to do so (bottom).}
    \label{fig:forecast_examples2_apd}
\end{figure*}


\newpage
\subsection{Model Performance, Efficiency, and Size Comparison}\label{apd:flops_plots}
We rank model performance across datasets and compare rank with model computational complexity in terms of floating-point operations per second (FLOPs) and model size in terms of the total number of trainable parameters. 

\begin{figure*}[htbp]
    \centering

    \begin{minipage}{0.495\textwidth}
        \centering
        \includegraphics[width=\textwidth]{images/flops_aggregate.pdf}
        \subcaption{In-distribution series results (excluding \TimesNet)}
        \label{fig:flops_aggregate}
    \end{minipage}%
    \hfill
    \begin{minipage}{0.495\textwidth}
        \centering
        \includegraphics[width=\textwidth]{images/flops_component.pdf}
        \subcaption{Out-of-distribution series results (excluding \TimesNet)}
        \label{fig:flops_component}
    \end{minipage}

    \vspace{1ex} % Vertical space between rows

    \begin{minipage}{0.495\textwidth}
        \centering
        \includegraphics[width=\textwidth]{images/flops_aggregate_with_timesnet.pdf}
        \subcaption{In-distribution series results (including \TimesNet)}
        \label{fig:flops_aggregate_with_timesnet}
    \end{minipage}%
    \hfill
    \begin{minipage}{0.495\textwidth}
        \centering
        \includegraphics[width=\textwidth]{images/flops_component_with_timesnet.pdf}
        \subcaption{Out-of-distribution series results (including \TimesNet)}
        \label{fig:flops_component_with_timesnet}
    \end{minipage}

    \caption{Comparison of average rank across datasets and random seeds versus model computational complexity, measured by floating-point operations per second (FLOPs). The size of each point represents the number of trainable parameters, highlighting the trade-offs between model complexity and performance. \textbf{(a)} In-distribution and \textbf{(b)} out-of-distribution results for all models, excluding \TimesNet, are shown to provide a clearer comparison by mitigating the parameter size skew. \textbf{(c)} In-distribution and \textbf{(d)} out-of-distribution results for all models, including \TimesNet.}
    \label{fig:flops_model_comparison_apd}
\end{figure*}


\newpage
\subsection{Model Composition Reasoning Results}\label{apd:composition_full_table_results}
We include the complete table results with MAE error mean and standard deviation measured across three random seeds. The results of the compositional reasoning task for 16 widely adopted time series forecasting models are included in Table~\ref{tab:composition_baseline_results_table}. Composition reasoning task results for controlled ablations of architecture components used in TSFMs are shown in Table~\ref{tab:composition_t5_results_table}.

\begin{table}[!ht] 
\centering
\caption{Mean Absolute Error (MAE) averaged over 3 random seeds (with standard deviation in parentheses) for composition reasoning tasks. The out-of-distribution (OOD) column presents MAE results for models trained via the compositional reasoning forecasting paradigm. The in-distribution (ID) column presents MAE results for models trained via the traditional forecasting paradigm. The \Tfive\ with the best patch length (PL) from Table~\ref{tab:composition_t5_results_table} is included. Best results are highlighted in \textbf{bold}, second best results are \underline{underlined}. The count of instances across datasets where the model ranks in the top three for performance is shown in the second to last column with non-zero entries in \textcolor{blue}{blue}. The average number of top $k$ compositions the model can outperform over the datasets is shown in the last column with nonzero entries in \textcolor{purple}{purple}.}
\label{tab:composition_baseline_results_table}
\resizebox{1.0\textwidth}{!}{
\begin{tabular}{ll|cc|cc|cc|cc|cc|cc||cc|cc}
\toprule
\multicolumn{2}{c|}{\multirow{2}{*}{\textbf{Model}}} & \multicolumn{2}{c}{\textbf{Synthetic Sinusoid}} & \multicolumn{2}{c}{\textbf{ECL}} & \multicolumn{2}{c}{\textbf{ETTm2}} & \multicolumn{2}{c}{\textbf{Solar}} & \multicolumn{2}{c}{\textbf{Subseasonal}} & \multicolumn{2}{c||}{\textbf{Loop Seattle}} & \multicolumn{2}{c}{\textbf{\small{Top 3 Win Count}}} & \multicolumn{2}{c}{\textbf{\small{Top $k$ Basis Wins}}} \\
\cline{3-18}
{} & {} & \textbf{OOD} & \textbf{ID} & \textbf{OOD} & \textbf{ID} & \textbf{OOD} & \textbf{ID} & \textbf{OOD} & \textbf{ID} & \textbf{OOD} & \textbf{ID} & \textbf{OOD} & \textbf{ID} & \textbf{\small{OOD}} & \textbf{\small{ID}} & \textbf{\small{OOD}} & \textbf{\small{ID}} \\
\hline\hline
\multirow{4}{*}{\rotatebox[origin=c]{90}{\textbf{Statistical}}} & \multirow{2}{*}{\ARIMA} & -- & 15.538 & -- & 0.822 & -- & 0.332 & -- & 9.687 & -- & 7.855 & -- & 8.638 & \multirow{2}{*}{\small{0}} & \multirow{2}{*}{\small{0}} & \multirow{2}{*}{\small{--}} & \multirow{2}{*}{\small{0.5}} \\
                      {} & {} &
                      \small{(--)} & 
                      \small{(--)} & 
                      \small{(--)} & 
                      \small{(--)} & 
                      \small{(--)} & 
                      \small{(--)} & 
                      \small{(--)} & 
                      \small{(--)} &
                      \small{(--)} & 
                      \small{(--)} & 
                      \small{(--)} & 
                      {} &
                      {} &
                      {} \\
\cline{2-18}
{} & \multirow{2}{*}{\ETS} & -- & 16.075 & -- & 0.105 & -- & 0.211 & -- & 1.730 & -- & 2.067 & -- & 5.575 & \small{--} & \small{0} & \small{--} & \small{0.8} \\
                      {} & {} &
                      \small{(--)} & 
                      \small{(--)} & 
                      \small{(--)} & 
                      \small{(--)} & 
                      \small{(--)} & 
                      \small{(--)} & 
                      \small{(--)} & 
                      \small{(--)} &
                      \small{(--)} & 
                      \small{(--)} & 
                      \small{(--)} & 
                      \small{(--)} &
                      {} &
                      {} \\
\hline
\multirow{4}{*}{\rotatebox[origin=c]{90}{\textbf{Linear}}} & \multirow{2}{*}{\DLinear} & 12.991 & 12.460 & 0.820 & \underline{0.103} & 0.330 & 0.135 & 9.925 & \textbf{1.555} & 8.042 & 1.496 & 9.085 & 4.293 & \multirow{2}{*}{\small{0}} & \multirow{2}{*}{\small{\textcolor{blue}{2}}} & \multirow{2}{*}{\small{0.2}} & \multirow{2}{*}{\small{\textcolor{purple}{55.5}}} \\
                      {} & {} &
                      \small{(0.051)} & \small{(0.025)} & \small{(0.073)} & \small{(0.000)} & \small{(0.028)} & \small{(0.000)} & \small{(1.000)} & \small{(0.007)} & \small{(0.402)} & \small{(0.030)} &
                      \small{(0.327)} & 
                      \small{(0.033)} &
                      {} &
                      {} \\
\cline{2-18}
{} & \multirow{2}{*}{\NLinear} & 13.287 & 13.056 & 0.801 & 0.104 & 0.325 & 0.136 & 9.681 & \underline{1.569} & 8.436 & 1.509 & 9.026 & 4.307 & \multirow{2}{*}{\small{0}} & \multirow{2}{*}{\small{\textcolor{blue}{1}}} & \multirow{2}{*}{\small{0.3}} & \multirow{2}{*}{\small{\textcolor{purple}{53.7}}} \\
                      {} & {} &
                      \small{(0.098)} & \small{(0.060)} & \small{(0.026)} & \small{(0.001)} & \small{(0.007)} & \small{(0.002)} & \small{(1.203)} & \small{(0.010)} & \small{(0.552)} & \small{(0.011)} &
                      \small{(0.248)} & 
                      \small{(0.017)} &
                      {} &
                      {} \\
\hline
\multirow{8}{*}{\rotatebox[origin=c]{90}{\textbf{MLP-Based}}} & \multirow{2}{*}{\MLP} & \underline{8.647} & 2.475 & \underline{0.283} & 0.106 & 0.253 & 0.114 & \underline{4.826} & 1.559 & 1.886 & 1.456 & 7.839 & 3.864 & \multirow{2}{*}{\small{\textcolor{blue}{3}}} & \multirow{2}{*}{\small{\textcolor{blue}{1}}} & \multirow{2}{*}{\small{\textcolor{purple}{10.2}}} & \multirow{2}{*}{\small{\textcolor{purple}{61.2}}} \\
                      {} & {} &
                      \small{(0.258)} & \small{(0.011)} & \small{(0.020)} & \small{(0.004)} & \small{(0.011)} & \small{(0.002)} & \small{(0.043)} & \small{(0.028)} & \small{(0.118)} & \small{(0.046)} &
                      \small{(0.139)} & 
                      \small{(0.061)} &
                      {} &
                      {} \\
\cline{2-18}
{} & \multirow{2}{*}{\NHITS} & 8.924 & \textbf{1.106} & 0.295 & \textbf{0.101} & \textbf{0.214} & \textbf{0.100} & 5.682 & 1.592 & 1.858 & \underline{1.135} & \underline{7.747} & 3.448 & \multirow{2}{*}{\small{\textcolor{blue}{4}}} & \multirow{2}{*}{\small{\textcolor{blue}{4}}} & \multirow{2}{*}{\small{\textcolor{purple}{11.0}}} & \multirow{2}{*}{\small{\textcolor{purple}{64.3}}} \\
                      {} & {} &
                      \small{(0.044)} & \small{(0.036)} & \small{(0.019)} & \small{(0.001)} & \small{(0.006)} & \small{(0.003)} & \small{(0.651)} & \small{(0.026)} & \small{(0.060)} & \small{(0.008)} &
                      \small{(0.141)} & 
                      \small{(0.043)} &
                      {} &
                      {} \\
\cline{2-18}
{} & \multirow{2}{*}{\NBEATS} & 8.907 & 1.383 & 0.294 & \underline{0.103} & \underline{0.216} & \underline{0.102} & 5.852 & 1.599 & \underline{1.840} & 1.177 & 7.763 & 3.479 & \multirow{2}{*}{\small{\textcolor{blue}{5}}} & \multirow{2}{*}{\small{\textcolor{blue}{4}}} & \multirow{2}{*}{\small{\textcolor{purple}{11.2}}} & \multirow{2}{*}{\small{\textcolor{purple}{61.8}}} \\
                      {} & {} &
                      \small{(0.106)} & \small{(0.058)} & \small{(0.016)} & \small{(0.004)} & \small{(0.005)} & \small{(0.002)} & \small{(0.645)} & \small{(0.028)} & \small{(0.108)} & \small{(0.007)} &
                      \small{(0.097)} & 
                      \small{(0.034)} &
                      {} &
                      {} \\
\cline{2-18}
{} & \multirow{2}{*}{\TSMixer} & 14.466 & 15.090 & 0.799 & 0.129 & 0.335 & 0.182 & 9.877 & 1.979 & 7.770 & 1.602 & 8.865 & 5.565 & \multirow{2}{*}{\small{0}} & \multirow{2}{*}{\small{0}} & \multirow{2}{*}{\small{0.3}} & \multirow{2}{*}{\small{\textcolor{purple}{19.7}}} \\
                      {} & {} &
                      \small{(1.804)} & \small{(0.339)} & \small{(0.013)} & \small{(0.004)} & \small{(0.024)} & \small{(0.051)} & \small{(0.118)} & \small{(0.132)} & \small{(0.543)} & \small{(0.099)} &
                      \small{(0.249)} & 
                      \small{(1.038)} &
                      {} &
                      {} \\
\hline
\multirow{2}{*}{\rotatebox[origin=c]{90}{\textbf{RNN}}} & \multirow{2}{*}{\LSTM} & 13.410 & 4.238 & 0.835 & 0.110 & 0.337 & 0.135 & 10.241 & 1.717 & 8.095 & 1.545 & 9.465 & 3.703 & \multirow{2}{*}{\small{0}} & \multirow{2}{*}{\small{0}} & \multirow{2}{*}{\small{0}} & \multirow{2}{*}{\small{\textcolor{purple}{48.7}}} \\
                      {} & {} &
                      \small{(0.203)} & \small{(0.637)} & \small{(0.004)} & \small{(0.001)} & \small{(0.000)} & \small{(0.006)} & \small{(0.620)} & \small{(0.131)} & \small{(0.093)} & \small{(0.075)} &
                      \small{(1.093)} & 
                      \small{(0.054)} &
                      {} &
                      {} \\
\hline
\multirow{4}{*}{\rotatebox[origin=c]{90}{\textbf{CNN}}} & \multirow{2}{*}{\TCN} & 11.478 & 3.833 & 0.837 & 0.106 & 0.339 & 0.135 & 9.868 & 1.642 & 6.170 & 1.234 & 8.792 & \underline{3.422} & \multirow{2}{*}{\small{0}} & \multirow{2}{*}{\small{\textcolor{blue}{1}}} & \multirow{2}{*}{\small{0.7}} & \multirow{2}{*}{\small{\textcolor{purple}{55.7}}} \\
                      {} & {} &
                      \small{(0.410)} & \small{(0.193)} & \small{(0.001)} & \small{(0.002)} & \small{(0.004)} & \small{(0.001)} & \small{(0.016)} & \small{(0.112)} & \small{(3.256)} & \small{(0.031)} &
                      \small{(0.020)} & 
                      \small{(0.119)} &
                      {} &
                      {} \\
\cline{2-18}
{} & \multirow{2}{*}{\TimesNet} & 9.788 & \underline{2.451} & 0.518 & 0.104 & 0.313 & 0.109 & 9.914 & 1.714 & 4.109 & 1.500 & 9.872 & 2.970 & \multirow{2}{*}{\small{0}} & \multirow{2}{*}{\small{\textcolor{blue}{2}}} & \multirow{2}{*}{\small{\textcolor{purple}{2.3}}} & \multirow{2}{*}{\small{\textcolor{purple}{52.8}}} \\
                      {} & {} &
                      \small{(0.493)} & \small{(0.252)} & \small{(0.285)} & \small{(0.003)} & \small{(0.042)} & \small{(0.004)} & \small{(0.116)} & \small{(0.259)} & \small{(3.409)} & \small{(0.111)} &
                      \small{(0.991)} & 
                      \small{(0.360)} &
                      {} &
                      {} \\
\hline
\multirow{22}{*}{\rotatebox[origin=c]{90}{\textbf{Transformer}}} & \multirow{2}{*}{\VanillaTransformer} & 12.279 & 4.935 & 0.919 & 0.106 & 0.334 & 0.136 & 11.956 & 1.667 & 9.641 & 1.276 & 11.675 & 3.591 & \multirow{2}{*}{\small{0}} & \multirow{2}{*}{\small{0}} & \multirow{2}{*}{\small{0.2}} & \multirow{2}{*}{\small{\textcolor{purple}{54.8}}} \\
                      {} & {} &
                      \small{(0.660)} & \small{(0.080)} & \small{(0.033)} & \small{(0.002)} & \small{(0.038)} & \small{(0.005)} & \small{(2.659)} & \small{(0.032)} & \small{(0.363)} & \small{(0.071)} &
                      \small{(0.803)} & 
                      \small{(0.008)} &
                      {} &
                      {} \\
\cline{2-18}
{} & \multirow{2}{*}{\iTransformer} & 15.478 & 15.203 & 0.829 & 0.157 & 0.326 & 0.196 & 9.822 & 1.805 & 8.447 & 1.628 & 9.182 & 4.871 & \multirow{2}{*}{\small{0}} & \multirow{2}{*}{\small{0}} & \multirow{2}{*}{\small{0.2}} & \multirow{2}{*}{\small{\textcolor{purple}{23.5}}} \\
                      {} & {} &
                      \small{(0.351)} & \small{(0.782)} & \small{(0.007)} & \small{(0.007)} & \small{(0.003)} & \small{(0.001)} & \small{(0.055)} & \small{(0.093)} & \small{(0.503)} & \small{(0.027)} &
                      \small{(0.380)} & 
                      \small{(0.105)} &
                      {} &
                      {} \\
\cline{2-18}
{} & \multirow{2}{*}{\Autoformer} & 15.301 & 15.018 & 0.795 & 0.137 & 0.330 & 0.294 & 10.348 & 2.108 & 7.933 & 2.390 & 8.634 & 4.758 & \multirow{2}{*}{\small{0}} & \multirow{2}{*}{\small{0}} & \multirow{2}{*}{\small{0.3}} & \multirow{2}{*}{\small{\textcolor{purple}{15.8}}} \\
                      {} & {} &
                     \small{(0.184)} & \small{(0.130)} & \small{(0.014)} & \small{(0.009)} & \small{(0.008)} & \small{(0.046)} & \small{(0.945)} & \small{(0.638)} & \small{(0.410)} & \small{(0.551)} &
                      \small{(0.166)} & 
                      \small{(0.307)} &
                      {} &
                      {} \\
\cline{2-18}
{} & \multirow{2}{*}{\Informer} & 14.353 & 10.144 & 0.787 & 0.128 & 0.321 & 0.141 & 8.351 & 1.662 & 6.878 & 1.564 & 11.549 & 4.241 & \multirow{2}{*}{\small{0}} & \multirow{2}{*}{\small{0}} & \multirow{2}{*}{\small{0.5}} & \multirow{2}{*}{\small{\textcolor{purple}{41.0}}} \\
                      {} & {} &
                      \small{(0.174)} & \small{(2.257)} & \small{(0.082)} & \small{(0.006)} & \small{(0.016)} & \small{(0.007)} & \small{(1.560)} & \small{(0.051)} & \small{(1.340)} & \small{(0.094)} &
                      \small{(0.627)} & 
                      \small{(0.105)} &
                      {} &
                      {} \\
\cline{2-18}
{} & \multirow{2}{*}{\TFT} & 14.531 & 9.745 & 0.445 & 0.115 & 0.312 & 0.117 & 12.873 & 2.106 & 2.684 & 1.454 & 11.280 & 5.340 & \multirow{2}{*}{\small{0}} & \multirow{2}{*}{\small{0}} & \multirow{2}{*}{\small{\textcolor{purple}{4.7}}} & \multirow{2}{*}{\small{\textcolor{purple}{28.5}}} \\
                      {} & {} &
                      \small{(0.880)} & \small{(1.210)} & \small{(0.072)} & \small{(0.010)} & \small{(0.029)} & \small{(0.005)} & \small{(2.141)} & \small{(0.538)} & \small{(0.146)} & \small{(0.151)} &
                      \small{(1.182)} & 
                      \small{(0.855)} &
                      {} &
                      {} \\
\cline{2-18}
{} & \multirow{2}{*}{\PatchTST\ (PL=8)} & 13.808 & 12.036 & 0.713 & 0.428 & 0.309 & 0.156 & 9.081 & 2.350 & 6.242 & 2.007 & 11.440 & 4.755 & \multirow{2}{*}{\small{0}} & \multirow{2}{*}{\small{0}} & \multirow{2}{*}{\small{0.7}} & \multirow{2}{*}{\small{\textcolor{purple}{18.7}}} \\
                      {} & {} &
                      \small{(0.471)} & \small{(0.738)} & \small{(0.172)} & \small{(0.318)} & \small{(0.031)} & \small{(0.026)} & \small{(0.857)} & \small{(0.872)} & \small{(1.286)} & \small{(0.228)} &
                      \small{(1.564)} & 
                      \small{(0.774)} &
                      {} &
                      {} \\
\cline{2-18}
{} & \multirow{2}{*}{\PatchTST\ (PL=16)} & 14.133 & 13.633 & 0.666 & 0.279 & 0.253 & 0.150 & 10.788 & 1.633 & 5.877 & 1.904 & 9.855 & 4.989 & \multirow{2}{*}{\small{0}} & \multirow{2}{*}{\small{0}} & \multirow{2}{*}{\small{0.8}} & \multirow{2}{*}{\small{\textcolor{purple}{27.8}}} \\
                      {} & {} &
                      \small{(2.693)} & \small{(1.202)} & \small{(0.185)} & \small{(0.025)} & \small{(0.017)} & \small{(0.005)} & \small{(0.139)} & \small{(0.013)} & \small{(1.389)} & \small{(0.336)} &
                      \small{(0.618)} & 
                      \small{(1.000)} &
                      {} &
                      {} \\
\cline{2-18}
{} & \multirow{2}{*}{\PatchTST\ (PL=32)} & 13.316 & 13.412 & 0.736 & 0.273 & 0.271 & 0.168 & 8.267 & 2.721 & 4.904 & 2.385 & 9.422 & 4.924 & \multirow{2}{*}{\small{0}} & \multirow{2}{*}{\small{0}} & \multirow{2}{*}{\small{\textcolor{purple}{1.3}}} & \multirow{2}{*}{\small{\textcolor{purple}{14.3}}} \\
                      {} & {} &
                      \small{(1.751)} & \small{(2.499)} & \small{(0.248)} & \small{(0.106)} & \small{(0.010)} & \small{(0.002)} & \small{(1.302)} & \small{(1.625)} & \small{(2.018)} & \small{(0.639)} &
                      \small{(1.359)} & 
                      \small{(0.133)} &
                      {} &
                      {} \\
\cline{2-18}
{} & \multirow{2}{*}{\PatchTST\ (PL=64)} & 12.232 & 13.544 & 0.482 & 0.122 & 0.247 & 0.169 & 8.054 & 2.813 & 2.292 & 1.754 & 10.529 & 4.405 & \multirow{2}{*}{\small{\textcolor{blue}{1}}} & \multirow{2}{*}{\small{0}} & \multirow{2}{*}{\small{\textcolor{purple}{6.8}}} & \multirow{2}{*}{\small{\textcolor{purple}{27.2}}} \\
                      {} & {} &
                      \small{(0.252)} & \small{(1.055)} & \small{(0.306)} & \small{(0.011)} & \small{(0.007)} & \small{(0.040)} & \small{(2.248)} & \small{(0.724)} & \small{(0.152)} & \small{(0.353)} &
                      \small{(4.987)} & 
                      \small{(0.883)} &
                      {} &
                      {} \\
\cline{2-18}
{} & \multirow{2}{*}{\PatchTST\ (PL=96)} & 11.235 & 8.374 & 0.508 & 0.196 & 0.250 & 0.147 & 6.426 & 1.799 & 2.185 & 1.659 & 7.965 & 4.579 & \multirow{2}{*}{\small{0}} & \multirow{2}{*}{\small{0}} & \multirow{2}{*}{\small{\textcolor{purple}{7.7}}} & \multirow{2}{*}{\small{\textcolor{purple}{27.0}}} \\
                      {} & {} &
                      \small{(0.483)} & \small{(1.834)} & \small{(0.287)} & \small{(0.051)} & \small{(0.013)} & \small{(0.021)} & \small{(1.312)} & \small{(0.304)} & \small{(0.168)} & \small{(0.282)} &
                      \small{(0.355)} & 
                      \small{(0.978)} &
                      {} &
                      {} \\
\cline{2-18}
{} & \multirow{2}{*}{\PatchTST\ (PL=128)} & 10.696 & 6.959 & 0.832 & 0.161 & 0.323 & 0.138 & 5.726 & 1.964 & 2.448 & 1.678 & 9.378 & 3.653 & \multirow{2}{*}{\small{0}} & \multirow{2}{*}{\small{0}} & \multirow{2}{*}{\small{\textcolor{purple}{5.7}}} & \multirow{2}{*}{\small{\textcolor{purple}{32.7}}} \\
                      {} & {} &
                      \small{(0.907)} & \small{(2.126)} & \small{(0.010)} & \small{(0.053)} & \small{(0.037)} & \small{(0.018)} & \small{(0.567)} & \small{(0.266)} & \small{(0.467)} & \small{(0.100)} &
                      \small{(0.185)} & 
                      \small{(0.257)} &
                      {} &
                      {} \\
\cline{2-18}
{} & \multirow{2}{*}{\Tfive\ (Best PL)} & \textbf{7.177} & 2.480 & \textbf{0.239} & \underline{0.103} & 0.259 & 0.103 & \textbf{3.899} & 1.578 & \textbf{1.714} & \textbf{1.097} & \textbf{6.589} & \textbf{3.351} & \multirow{2}{*}{\small{\textcolor{blue}{5}}} & \multirow{2}{*}{\small{\textcolor{blue}{4}}} & \multirow{2}{*}{\small{\textcolor{purple}{12.2}}} & \multirow{2}{*}{\small{\textcolor{purple}{64.3}}} \\
                      {} & {} &
                      \small{(0.089)} & 
                      \small{(0.198)} & 
                      \small{(0.005)} & 
                      \small{(0.001)} & 
                      \small{(0.008)} & 
                      \small{(0.006)} & 
                      \small{(0.578)} & 
                      \small{(0.009)} &
                      \small{(0.040) } &
                      \small{0.039)} &
                      \small{(0.109)} & 
                      \small{(0.017)} &
                      {} &
                      {} \\
\bottomrule
% \multicolumn{13}{c}{\textbf{  }} \\
% \cline{1-14}
% \multirow{8}{*}{\rotatebox[origin=c]{90}{\textbf{Baseline}}} & \Fourier\ (topk=1) & 
%     \multicolumn{2}{c|}{13.032} & 
%     \multicolumn{2}{c|}{0.669} & 
%     \multicolumn{2}{c|}{0.325} &
%     \multicolumn{2}{c|}{9.878} & 
%     \multicolumn{2}{c|}{7.977} & 
%     \multicolumn{2}{c}{8.758} \\
% \cline{2-14}
% {} & \Fourier\ (topk=2) & 
%     \multicolumn{2}{c|}{\textcolor{purple}{6.495}} & 
%     \multicolumn{2}{c|}{0.321} & 
%     \multicolumn{2}{c|}{0.305} & 
%     \multicolumn{2}{c|}{7.840} & 
%     \multicolumn{2}{c|}{6.758} & 
%     \multicolumn{2}{c}{8.150} \\
% \cline{2-14}
% {} & \Fourier\ (topk=3) & 
%     \multicolumn{2}{c|}{5.078} & 
%     \multicolumn{2}{c|}{0.250} & 
%     \multicolumn{2}{c|}{0.252} &
%     \multicolumn{2}{c|}{4.218} & 
%     \multicolumn{2}{c|}{5.185} & 
%     \multicolumn{2}{c}{\textcolor{purple}{6.259}} \\
% \cline{2-14}
% {} & \Fourier\ (topk=4) & 
%     \multicolumn{2}{c|}{0.719} & 
%     \multicolumn{2}{c|}{\textcolor{purple}{0.218}} & 
%     \multicolumn{2}{c|}{0.249} & 
%     \multicolumn{2}{c|}{\textcolor{purple}{3.663}} & 
%     \multicolumn{2}{c|}{5.048} & 
%     \multicolumn{2}{c}{6.297} \\
% \cline{2-14}
% {} & \Fourier\ (topk=5) & 
%     \multicolumn{2}{c|}{0.729} & 
%     \multicolumn{2}{c|}{0.185} &
%     \multicolumn{2}{c|}{0.222} & 
%     \multicolumn{2}{c|}{2.058} & 
%     \multicolumn{2}{c|}{4.911} & 
%     \multicolumn{2}{c}{5.961} \\
% \cline{2-14}
% {} & \Fourier\ (topk=6) & 
%     \multicolumn{2}{c|}{0.738} & 
%     \multicolumn{2}{c|}{0.175} &
%     \multicolumn{2}{c|}{0.222} & 
%     \multicolumn{2}{c|}{1.972} & 
%     \multicolumn{2}{c|}{4.812} & 
%     \multicolumn{2}{c}{6.050} \\
% \cline{2-14}
% {} & \Fourier\ (topk=7) & 
%     \multicolumn{2}{c|}{0.737} & 
%     \multicolumn{2}{c|}{0.160} &
%     \multicolumn{2}{c|}{\textcolor{purple}{0.211}} & 
%     \multicolumn{2}{c|}{1.827} & 
%     \multicolumn{2}{c|}{4.662} & 
%     \multicolumn{2}{c}{5.863} \\
% \cline{2-14}
% {} & \Fourier\ (topk=63) & 
%     \multicolumn{2}{c|}{0.887} & 
%     \multicolumn{2}{c|}{0.100} &
%     \multicolumn{2}{c|}{0.075} & 
%     \multicolumn{2}{c|}{1.640} & 
%     \multicolumn{2}{c|}{\textcolor{purple}{1.702}} & 
%     \multicolumn{2}{c}{4.471} \\
% \cline{1-14}
\end{tabular}
}
\end{table}



\begin{table}[ht]
\centering
\caption{Mean Absolute Error (MAE) averaged over 3 random seeds (with standard deviation in parentheses) for composition reasoning tasks. The out-of-distribution (OOD) column presents MAE results for models trained via the compositional reasoning forecasting paradigm. The in-distribution (ID) column presents MAE results for models trained via the traditional forecasting paradigm. Best results are highlighted in \textbf{bold}. The count of instances across datasets where the model has the best performance is shown in the last column with non-zero entries in \textcolor{blue}{blue}.}
\label{tab:composition_t5_results_table}
\resizebox{1.0\textwidth}{!}{
\begin{tabular}{ll|cc|cc|cc|cc|cc|cc||cc}
\toprule
\multicolumn{2}{c|}{\multirow{2}{*}{\textbf{Transformer Model (T5 Backbone)}}} & \multicolumn{2}{c}{\textbf{Synthetic Sinusoid}} & \multicolumn{2}{c}{\textbf{ECL}} & \multicolumn{2}{c}{\textbf{ETTm2}} & \multicolumn{2}{c}{\textbf{Solar}} & \multicolumn{2}{c}{\textbf{Subseasonal}} & \multicolumn{2}{c||}{\textbf{Loop Seattle}} & \multicolumn{2}{c}{\textbf{\small{Win Count}}} \\
\cline{3-16}
{} & {} & \textbf{OOD} & \textbf{ID} & \textbf{OOD} & \textbf{ID} & \textbf{OOD} & \textbf{ID} & \textbf{OOD} & \textbf{ID} & \textbf{OOD} & \textbf{ID} & \textbf{OOD} & \textbf{ID} & \textbf{\small{OOD}} & \textbf{\small{ID}} \\
\hline\hline
\multirow{8}{*}{\rotatebox[origin=c]{90}{\textbf{Tokenization}}} & \multirow{2}{*}{None} & 14.032 & 4.894 & 0.685 & \textbf{0.106} & 0.369 & 0.121 & 9.969 & \textbf{1.635} & 6.401 & 1.572 & 14.327 & 3.766 & \multirow{2}{*}{\small{0}} & \multirow{2}{*}{\small{\textcolor{blue}{2}}}\\
                      {} & {} &
                      \small{(1.612)} & 
                      \small{(0.140)} & 
                      \small{(0.148)} & 
                      \small{(0.005)} & 
                      \small{(0.015)} & 
                      \small{(0.010)} & 
                      \small{(0.804)} & 
                      \small{(0.079)} &
                      \small{(1.739)} & 
                      \small{(0.049) } &
                      \small{(2.548)} & 
                      \small{(0.035)} \\
\cline{2-16}
{} & \multirow{2}{*}{Fixed Length Patches} & \textbf{8.648} & \textbf{2.611} & \textbf{0.266} & 0.107 & \textbf{0.268} & \textbf{0.100} & \textbf{3.908} & 1.663 & \textbf{1.729} & \textbf{1.154} & \textbf{7.658} & \textbf{3.118} & \multirow{2}{*}{\small{\textcolor{blue}{6}}} & \multirow{2}{*}{\small{\textcolor{blue}{4}}}\\
                      {} & {} &
                      \small{(0.072)} & 
                      \small{(0.158)} & 
                      \small{(0.019)} & 
                      \small{(0.002)} & 
                      \small{(0.003)} & 
                      \small{(0.003)} & 
                      \small{(0.204)} & 
                      \small{(0.023)} &
                      \small{(0.028)} & 
                      \small{(0.029)} &
                      \small{(0.100)} & 
                      \small{(0.026)} \\
\cline{2-16}
{} & \multirow{2}{*}{Binning} & 17.039 & 9.504 & 0.833 & 0.270 & 0.317 & 0.199 & 8.445 & 4.719 & 3.758 & 3.253 & 12.728 & 7.735 & \multirow{2}{*}{\small{0}} & \multirow{2}{*}{\small{0}}\\
                      {} & {} &
                      \small{(0.788)} & 
                      \small{(0.502)} & 
                      \small{(0.007)} & 
                      \small{(0.003)} & 
                      \small{(0.011)} & 
                      \small{(0.004)} & 
                      \small{(4.206)} & 
                      \small{(0.151)} &
                      \small{(0.365)} & 
                      \small{(0.532)} &
                      \small{(2.339)} & 
                      \small{(0.985)} \\
\cline{2-16}
{} & \multirow{2}{*}{Lags} & 13.442 & 4.599 & 0.820 & 0.120 & 0.415 & 0.126 & 10.156  & 1.669 & 4.022 & 1.376 & 11.638 & 3.897 & \multirow{2}{*}{\small{0}} & \multirow{2}{*}{\small{0}} \\
                      {} & {} &
                      \small{(0.254)} & 
                      \small{(0.328)} & 
                      \small{(0.178)} & 
                      \small{(0.005)} & 
                      \small{(0.046)} & 
                      \small{(0.006)} & 
                      \small{(1.364)} & 
                      \small{(0.030)} &
                      \small{(0.187)} & 
                      \small{(0.043)} &
                      \small{(1.556)} & 
                      \small{(0.074)} \\
\hline\hline
\multirow{8}{*}{\rotatebox[origin=c]{90}{\textbf{Model Size}}} & \multirow{2}{*}{Tiny} & \textbf{7.644} & 2.628 & \textbf{0.239} & 0.105 & 0.274 & 0.109 & 3.899 & \textbf{1.641} & 1.899 & 1.097 & \textbf{6.701} & 3.355 & \multirow{2}{*}{\small{\textcolor{blue}{3}}} & \multirow{2}{*}{\small{\textcolor{blue}{1}}} \\
                      {} & {} &
                      \small{(0.025)} & 
                      \small{(0.108)} & 
                      \small{(0.005)} & 
                      \small{(0.002)} & 
                      \small{(0.005)} & 
                      \small{(0.006)} & 
                      \small{(0.578)} & 
                      \small{(0.068)} &
                      \small{(0.203)} & 
                      \small{(0.039)} &
                      \small{(0.153)} & 
                      \small{(0.053)} \\
\cline{2-16}
{} & \multirow{2}{*}{Mini} & 7.882 & 2.318 & 0.242 & 0.107 & \textbf{0.273} & 0.103 & \textbf{3.810} & 1.663 & \textbf{1.769} & 1.104 & 6.888 & 3.018 & \multirow{2}{*}{\small{\textcolor{blue}{3}}} & \multirow{2}{*}{\small{0}}\\
                      {} & {} &
                      \small{(0.069)} & 
                      \small{(0.076)} & 
                      \small{(0.006)} & 
                      \small{(0.001)} & 
                      \small{(0.005)} & 
                      \small{(0.001)} & 
                      \small{(0.226)} & 
                      \small{(0.027)} &
                      \small{(0.108)} & 
                      \small{(0.008)} &
                      \small{(0.032)} & 
                      \small{(0.011)} \\
\cline{2-16}
{} & \multirow{2}{*}{Small} & 8.057 & 2.103 & 0.268 & 0.103 & 0.268 & 0.096 & 4.172 & 1.665 & 1.770 & 1.048 & 6.924 & 2.764 & \multirow{2}{*}{\small{0}} & \multirow{2}{*}{\small{0}}\\
                      {} & {} &
                      \small{(0.119)} & 
                      \small{(0.098)} & 
                      \small{(0.009)} & 
                      \small{(0.002)} & 
                      \small{(0.014)} & 
                      \small{(0.001)} & 
                      \small{(0.192)} & 
                      \small{(0.028)} &
                      \small{(0.197)} & 
                      \small{(0.016)} &
                      \small{(0.187)} & 
                      \small{(0.033)} \\
\cline{2-16}
{} & \multirow{2}{*}{Base} & 8.308 & \textbf{2.084} & 0.247 & \textbf{0.100} & 0.274 & \textbf{0.091} & 4.338 & 1.667 & 1.853 & \textbf{0.967} & 7.005 & \textbf{2.536} & \multirow{2}{*}{\small{0}} & \multirow{2}{*}{\small{\textcolor{blue}{5}}} \\
                      {} & {} &
                      \small{(0.228)} & 
                      \small{(0.072)} & 
                      \small{(0.014)} & 
                      \small{(0.006)} & 
                      \small{(0.007)} & 
                      \small{(0.001)} & 
                      \small{(0.123)} & 
                      \small{(0.065)} &
                      \small{(0.150)} & 
                      \small{(0.012)} &
                      \small{(0.255)} & 
                      \small{(0.061)} \\
\hline\hline
\multirow{4}{*}{\rotatebox[origin=c]{90}{\textbf{Attn. Type}}} & \multirow{2}{*}{Bidirectional Attn.} & \textbf{7.644} & \textbf{2.628} & \textbf{0.239} & \textbf{0.105} & 0.274 & 0.109 & \textbf{3.899} & 1.641 & 1.899 & \textbf{1.097} & \textbf{6.701} & 3.355 & \multirow{2}{*}{\small{\textcolor{blue}{4}}} & \multirow{2}{*}{\small{\textcolor{blue}{3}}}\\
                      {} & {} &
                      \small{(0.025)} & 
                      \small{(0.108)} & 
                      \small{(0.005)} & 
                      \small{(0.002)} & 
                      \small{(0.005)} & 
                      \small{(0.006)} & 
                      \small{(0.578)} & 
                      \small{(0.068)} &
                      \small{(0.203)} & 
                      \small{(0.039)} &
                      \small{(0.153)} & 
                      \small{(0.053)} \\
\cline{2-16}
{} & \multirow{2}{*}{Causal Attn.} & 7.978 & 2.828 & 0.248 & 0.106 & \textbf{0.267} & \textbf{0.105} & 4.307 & \textbf{1.589} & \textbf{1.820} & 1.170 & 6.891 & \textbf{3.337} & \multirow{2}{*}{\small{\textcolor{blue}{2}}} & \multirow{2}{*}{\small{\textcolor{blue}{3}}} \\
                      {} & {} &
                      \small{(0.02)} & 
                      \small{(0.233)} & 
                      \small{(0.005)} & 
                      \small{(0.004)} & 
                      \small{(0.010)} & 
                      \small{(0.003)} & 
                      \small{(0.144)} & 
                      \small{(0.042)} &
                      \small{(0.173)} & 
                      \small{(0.054)} &
                      \small{(0.133)} & 
                      \small{(0.044)} \\
\hline\hline
\multirow{4}{*}{\rotatebox[origin=c]{90}{\textbf{Proj./Head}}} & \multirow{2}{*}{Linear} & \textbf{7.644} & 2.628 & \textbf{0.239} & 0.105 & 0.274 & 0.109 & \textbf{3.899} & 1.641 & 1.899 & 1.097 & \textbf{6.701} & 3.355 & \multirow{2}{*}{\small{\textcolor{blue}{4}}} & \multirow{2}{*}{\small{0}} \\
                      {} & {} &
                      \small{(0.025)} & 
                      \small{(0.108)} & 
                      \small{(0.005)} & 
                      \small{(0.002)} & 
                      \small{(0.005)} & 
                      \small{(0.006)} & 
                      \small{(0.578)} & 
                      \small{(0.068)} &
                      \small{(0.203)} & 
                      \small{(0.039)} &
                      \small{(0.153)} & 
                      \small{(0.053)} \\
\cline{2-16}
{} & \multirow{2}{*}{Residual} & 8.537 & \textbf{2.617} & 0.311 & \textbf{0.102} & \textbf{0.256} & \textbf{0.085} & 4.880 & \textbf{1.595} & \textbf{1.871} & \textbf{0.924} & 7.931 & \textbf{2.524} & \multirow{2}{*}{\small{\textcolor{blue}{2}}} & \multirow{2}{*}{\small{\textcolor{blue}{6}}} \\
                      {} & {} &
                      \small{(0.184)} & 
                      \small{(0.043)} &
                      \small{(0.011)} & 
                      \small{(0.002)} & 
                      \small{(0.008)} & 
                      \small{(0.003)} & 
                      \small{(0.389)} &
                      \small{(0.039)} & 
                      \small{(0.296)} &
                      \small{(0.016)} &
                      \small{(0.655)} &
                      \small{(0.028)} \\
\hline\hline
\multirow{12}{*}{\rotatebox[origin=c]{90}{\textbf{Token (Patch) Length}}} & \multirow{2}{*}{8} & 9.327 & 3.183 & 0.346 & \textbf{0.103} & 0.300 & 0.111 & 7.721 & \textbf{1.578} & 2.581 & 1.532 & 8.048 & 3.573 & \multirow{2}{*}{\small{0}} & \multirow{2}{*}{\small{\textcolor{blue}{2}}} \\
                      {} & {} &
                      \small{(0.28)} & 
                      \small{(0.117)} & 
                      \small{(0.040)} & 
                      \small{(0.001)} & 
                      \small{(0.013)} & 
                      \small{(0.003)} & 
                      \small{(0.640)} & 
                      \small{(0.009)} &
                      \small{(0.335)} & 
                      \small{(0.002)} &
                      \small{(0.108)} & 
                      \small{(0.047)} \\
\cline{2-16}
{} & \multirow{2}{*}{16} & 9.007 & 3.573 & 0.418 & 0.105 & 0.283 & \textbf{0.103} & 9.139 & 1.673 & 2.731 & 1.362 & 8.499 & 3.510 & \multirow{2}{*}{\small{0}} & \multirow{2}{*}{\small{\textcolor{blue}{1}}} \\
                      {} & {} &
                      \small{(0.057)} & 
                      \small{(0.235)} & 
                      \small{(0.062)} & 
                      \small{(0.002)} & 
                      \small{(0.011)} & 
                      \small{(0.006)} & 
                      \small{(0.400)} & 
                      \small{(0.088)} &
                      \small{(0.939)} & 
                      \small{(0.25)} &
                      \small{(0.339)} & 
                      \small{(0.060)} \\
\cline{2-16}
{} & \multirow{2}{*}{32} & 10.11 & 3.719 & 0.244 & 0.108 & 0.289 & 0.109 & 5.256 & 1.652 & 2.059 & 1.318 & 7.014 & 3.463 & \multirow{2}{*}{\small{0}} & \multirow{2}{*}{\small{0}} \\
                      {} & {} &
                      \small{(0.07)} & 
                      \small{(0.190)} & 
                      \small{(0.011)} & 
                      \small{(0.004)} & 
                      \small{(0.007)} & 
                      \small{(0.002)} & 
                      \small{(0.591)} & 
                      \small{(0.071)} &
                      \small{(0.171)} & 
                      \small{(0.187)} &
                      \small{(0.167)} & 
                      \small{(0.015)} \\
\cline{2-16}
{} & \multirow{2}{*}{64} & 8.747 & 2.971 & \textbf{0.239 }& 0.106 & 0.285 & 0.107 & 4.201 & 1.651 & 1.819 & 1.265 & \textbf{6.589} & 3.408 & \multirow{2}{*}{\small{\textcolor{blue}{2}}} & \multirow{2}{*}{\small{0}} \\
                      {} & {} &
                      \small{(0.243)} & 
                      \small{(0.209)} & 
                      \small{(0.009)} & 
                      \small{(0.003)} & 
                      \small{(0.002)} & 
                      \small{(0.006)} & 
                      \small{(0.103)} & 
                      \small{(0.076)} &
                      \small{(0.032)} & 
                      \small{(0.198)} &
                      \small{(0.109)} & 
                      \small{(0.044)} \\
\cline{2-16}
{} & \multirow{2}{*}{96} & 7.644 & 2.628 & \textbf{0.239} & 0.105 & 0.274 & 0.109 & \textbf{3.899} & 1.641 & 1.899 & \textbf{1.097} & 6.701 & 3.355 & \multirow{2}{*}{\small{\textcolor{blue}{2}}} & \multirow{2}{*}{\small{\textcolor{blue}{1}}} \\
                      {} & {} &
                      \small{(0.025)} & 
                      \small{(0.108)} & 
                      \small{(0.005)} & 
                      \small{(0.002)} & 
                      \small{(0.005)} & 
                      \small{(0.006)} & 
                      \small{(0.578)} & 
                      \small{(0.068)} &
                      \small{(0.203)} & 
                      \small{(0.039)} &
                      \small{(0.153)} & 
                      \small{(0.053)} \\
\cline{2-16}
{} & \multirow{2}{*}{128} & \textbf{7.177} & \textbf{2.480} & 0.255 & 0.107 & \textbf{0.259} & 0.107 & 4.385 & 1.673 & \textbf{1.714} & 1.100 & 6.878 & \textbf{3.351} & \multirow{2}{*}{\small{\textcolor{blue}{3}}} & \multirow{2}{*}{\small{\textcolor{blue}{2}}} \\
                      {} & {} &
                      \small{(0.089)} & 
                      \small{(0.198)} & 
                      \small{(0.012)} & 
                      \small{(0.001)} & 
                      \small{(0.008)} & 
                      \small{(0.006)} & 
                      \small{(0.145)} & 
                      \small{(0.033)} &
                      \small{(0.040)} & 
                      \small{(0.069)} &
                      \small{(0.069)} & 
                      \small{(0.017)} \\
\hline\hline
\multirow{8}{*}{\rotatebox[origin=c]{90}{\textbf{Positional Encoding}}} & \multirow{2}{*}{Relative} & 7.751 & 2.750 & 0.284 & 0.105 & \textbf{0.268} & 0.111 & 4.373 & 1.640 & \textbf{1.826} & \textbf{1.093} & 6.883 & \textbf{3.337} & \multirow{2}{*}{\small{\textcolor{blue}{2}}} & \multirow{2}{*}{\small{\textcolor{blue}{2}}} \\
                      {} & {} &
                      \small{(0.163)} & 
                      \small{(0.262)} & 
                      \small{(0.059) } & 
                      \small{(0.002)} & 
                      \small{(0.015)} & 
                      \small{(0.002)} & 
                      \small{(0.254)} & 
                      \small{(0.033)} &
                      \small{(0.156)} & 
                      \small{(0.025)} &
                      \small{(0.162)} & 
                      \small{(0.020)} \\
\cline{2-16}
{} & \multirow{2}{*}{SinCos} & 7.881 & \textbf{2.456} & 0.247 & \textbf{0.104} & 0.270 & 0.112 & 4.357 & \textbf{1.624} & 1.839 & 1.464 & 6.818 & 3.366 & \multirow{2}{*}{\small{0}} & \multirow{2}{*}{\small{\textcolor{blue}{3}}} \\
                      {} & {} &
                      \small{(0.148)} & 
                      \small{(0.049)} & 
                      \small{(0.015)} & 
                      \small{(0.003)} & 
                      \small{(0.007)} & 
                      \small{(0.004)} & 
                      \small{(0.284)} & 
                      \small{(0.025)} &
                      \small{(0.195)} & 
                      \small{(0.051)} &
                      \small{(0.182} & 
                      \small{(0.055)} \\
\cline{2-16}
{} & \multirow{2}{*}{SinCos+Relative} & \textbf{7.644} & 2.628 & \textbf{0.239} & 0.105 & 0.274 & \textbf{0.109} & \textbf{3.899} & 1.641 & 1.899 & 1.097 & \textbf{6.701} & 3.355 & \multirow{2}{*}{\small{\textcolor{blue}{4}}} & \multirow{2}{*}{\small{\textcolor{blue}{1}}} \\
                      {} & {} &
                      \small{(0.025)} & 
                      \small{(0.108)} & 
                      \small{(0.005)} & 
                      \small{(0.002)} & 
                      \small{(0.005)} & 
                      \small{(0.006)} & 
                      \small{(0.578)} & 
                      \small{(0.068)} &
                      \small{(0.203)} & 
                      \small{(0.039)} &
                      \small{(0.153)} & 
                      \small{(0.053)} \\
\cline{2-16}
{} & \multirow{2}{*}{RoPE} & 7.921 & 2.701 & 0.255 & \textbf{0.104} & 0.274 & 0.112 & 4.608 & 1.647 & 1.872 & 1.106 & 6.721 & 3.378 & \multirow{2}{*}{\small{0}} & \multirow{2}{*}{\small{\textcolor{blue}{1}}} \\
                      {} & {} &
                      \small{(0.125)} & 
                      \small{(0.077)} & 
                      \small{(0.006)} & 
                      \small{(0.002)} & 
                      \small{(0.002)} & 
                      \small{(0.002)} & 
                      \small{(0.333)} & 
                      \small{(0.032)} &
                      \small{(0.125)} &
                      \small{(0.037)} & 
                      \small{(0.174)} &
                      \small{(0.033)} \\
\hline\hline
\multirow{8}{*}{\rotatebox[origin=c]{90}{\textbf{Loss Function}}} & \multirow{2}{*}{MAE} & \textbf{7.644} & 2.628 & 0.239 & 0.105 & 0.274 &\textbf{ 0.109} & 3.899 & \textbf{1.641} & 1.899 & \textbf{1.097} & 6.701 & 3.355 & \multirow{2}{*}{\small{\textcolor{blue}{1}}} & \multirow{2}{*}{\small{\textcolor{blue}{3}}} \\
                      {} & {} &
                      \small{(0.025)} & 
                      \small{(0.108)} & 
                      \small{(0.005)} & 
                      \small{(0.002)} & 
                      \small{(0.005)} & 
                      \small{(0.006)} & 
                      \small{(0.578)} & 
                      \small{(0.068)} &
                      \small{(0.203)} & 
                      \small{(0.039)} &
                      \small{(0.153)} & 
                      \small{(0.053)} \\
\cline{2-16}
{} & \multirow{2}{*}{MSE} & 7.694 & \textbf{2.618} & \textbf{0.229} & 0.109 & \textbf{0.262} & 0.116 & 3.633 & 1.752 & 1.891 & 1.411 & 6.613 & 3.302 & \multirow{2}{*}{\small{\textcolor{blue}{2}}} & \multirow{2}{*}{\small{\textcolor{blue}{1}}} \\
                      {} & {} &
                      \small{(0.178)} & 
                      \small{(0.088)} & 
                      \small{(0.006)} & 
                      \small{(0.004)} & 
                      \small{(0.007)} & 
                      \small{(0.010)} & 
                      \small{(0.466)} & 
                      \small{(0.030)} &
                      \small{(0.193)} & 
                      \small{(0.214)} &
                      \small{(0.118)} & 
                      \small{(0.045)} \\
\cline{2-16}
{} & \multirow{2}{*}{Huber} & 7.648 & 2.914 & 0.234 & 0.108 & 0.264 & 0.112 & \textbf{4.035} & 1.746 & 1.801 & 1.258 & \textbf{6.571} & \textbf{3.281} & \multirow{2}{*}{\small{\textcolor{blue}{2}}} & \multirow{2}{*}{\small{\textcolor{blue}{1}}} \\
                      {} & {} &
                      \small{(0.076)} & 
                      \small{(0.234)} & 
                      \small{(0.012)} & 
                      \small{(0.005)} & 
                      \small{(0.012)} & 
                      \small{(0.003)} & 
                      \small{(0.633)} & 
                      \small{(0.056)} &
                      \small{(0.084)} & 
                      \small{(0.194)} &
                      \small{(0.182)} & 
                      \small{(0.047)} \\
\cline{2-16}
{} & \multirow{2}{*}{StudentT} & 7.776 & 2.660 & 0.244 & \textbf{0.104} & 0.270 & 0.113 & 4.134 & 1.648 & \textbf{1.735} & 1.377 & 6.872 & 3.543 & \multirow{2}{*}{\small{\textcolor{blue}{1}}} & \multirow{2}{*}{\small{\textcolor{blue}{1}}} \\
                      {} & {} &
                      \small{(0.281)} & 
                      \small{(0.105)} & 
                      \small{(0.003)} & 
                      \small{(0.003)} & 
                      \small{(0.004)} & 
                      \small{(0.003)} & 
                      \small{(0.804)} & 
                      \small{(0.025)} &
                      \small{(0.037)} & 
                      \small{(0.024)} &
                      \small{(0.191)} & 
                      \small{(0.026)} \\
\hline\hline
\multirow{4}{*}{\rotatebox[origin=c]{90}{\textbf{Scaler}}} & \multirow{2}{*}{RevIN (Standard, non-learnable)} & \textbf{7.644} & \textbf{2.628} & \textbf{0.239} & 0.105 & 0.274 & 0.109 & \textbf{3.899} & \textbf{1.641} & 1.899 & \textbf{1.097} & \textbf{6.701} & \textbf{3.355} & \multirow{2}{*}{\small{\textcolor{blue}{4}}} & \multirow{2}{*}{\small{\textcolor{blue}{4}}} \\
                      {} & {} &
                      \small{(0.025)} & 
                      \small{(0.108)} & 
                      \small{(0.005)} & 
                      \small{(0.002)} & 
                      \small{(0.005)} & 
                      \small{(0.006)} & 
                      \small{(0.578)} & 
                      \small{(0.068)} &
                      \small{(0.203)} & 
                      \small{(0.039)} &
                      \small{(0.153)} & 
                      \small{(0.053)} \\
\cline{2-16}
% {} & \multirow{2}{*}{RevIN (Standard, learnable)} & 7.644 & 2.628 & 0.239 & 0.105 & 0.274 & 0.109 & 3.899 & 1.641 & 1.899 & 1.097 & 6.701 & 3.355 \\
%                       {} & {} &
%                       \small{(0.025)} & 
%                       \small{(0.108)} & 
%                       \small{(0.005)} & 
%                       \small{(0.002)} & 
%                       \small{(0.005)} & 
%                       \small{(0.006)} & 
%                       \small{(0.578)} & 
%                       \small{(0.068)} &
%                       \small{(0.203)} & 
%                       \small{(0.039)} &
%                       \small{(0.153)} & 
%                       \small{(0.053)} \\
% \cline{2-14}
{} & \multirow{2}{*}{Robust} & 7.931 & 2.725 & 0.326 & \textbf{0.103} & \textbf{0.270} & \textbf{0.106} & 8.170 & 1.734 & \textbf{1.736} & 1.232 & 6.982 & 3.412 & \multirow{2}{*}{\small{\textcolor{blue}{2}}} & \multirow{2}{*}{\small{\textcolor{blue}{2}}} \\
                      {} & {} &
                      \small{(0.026)} & 
                      \small{(0.234)} & 
                      \small{(0.006)} & 
                      \small{(0.003)} & 
                      \small{(0.007)} & 
                      \small{(0.007)} & 
                      \small{(0.389)} & 
                      \small{(0.093)} &
                      \small{(0.015)} & 
                      \small{(0.227)} &
                      \small{(0.271)} & 
                      \small{(0.043)} \\
\hline\hline
\multirow{4}{*}{\rotatebox[origin=c]{90}{\textbf{Context}}} & \multirow{2}{*}{256} & 7.644 & 2.628 & \textbf{0.239} & \textbf{0.105} & \textbf{0.274} & \textbf{0.109} & \textbf{3.899} & \textbf{1.641} & \textbf{1.899} & \textbf{1.097} & 6.701 & \textbf{3.355} & \multirow{2}{*}{\small{\textcolor{blue}{4}}} & \multirow{2}{*}{\small{\textcolor{blue}{5}}} \\
                      {} & {} &
                      \small{(0.025)} & 
                      \small{(0.108)} & 
                      \small{(0.005)} & 
                      \small{(0.002)} & 
                      \small{(0.005)} & 
                      \small{(0.006)} & 
                      \small{(0.578)} & 
                      \small{(0.068)} &
                      \small{(0.203)} & 
                      \small{(0.039)} &
                      \small{(0.153)} & 
                      \small{(0.053)} \\
\cline{2-16}
{} & \multirow{2}{*}{512} & \textbf{7.072} & \textbf{2.605} & 0.253 & 0.111 & 0.298 & 0.151 & 4.187 & 1.740 & 1.934 & 1.829 & \textbf{6.295} & 4.013 & \multirow{2}{*}{\small{\textcolor{blue}{2}}} & \multirow{2}{*}{\small{\textcolor{blue}{1}}} \\
                      {} & {} &
                      \small{(0.382)} & 
                      \small{(0.158)} & 
                      \small{(0.037)} & 
                      \small{(0.005)} & 
                      \small{(0.015)} & 
                      \small{(0.008)} & 
                      \small{(0.216)} & 
                      \small{(0.046)} &
                      \small{(0.083)} & 
                      \small{(0.094)} &
                      \small{(0.284)} & 
                      \small{(0.002)} \\
\hline\hline
\multirow{4}{*}{\rotatebox[origin=c]{90}{\textbf{Decomp.}}} & \multirow{2}{*}{None} & \textbf{7.644} & \textbf{2.628} & \textbf{0.239} & \textbf{0.105} & 0.274 & \textbf{0.109} & \textbf{3.899} & \textbf{1.641} & 1.899 & \textbf{1.097} & \textbf{6.701} & \textbf{3.355} & \multirow{2}{*}{\small{\textcolor{blue}{4}}} & \multirow{2}{*}{\small{\textcolor{blue}{6}}} \\
                      {} & {} &
                      \small{(0.025)} & 
                      \small{(0.108)} & 
                      \small{(0.005)} & 
                      \small{(0.002)} & 
                      \small{(0.005)} & 
                      \small{(0.006)} & 
                      \small{(0.578)} & 
                      \small{(0.068)} &
                      \small{(0.203)} & 
                      \small{(0.039)} &
                      \small{(0.153)} & 
                      \small{(0.053)} \\
\cline{2-16}
{} & \multirow{2}{*}{Moving Avg. Filter (DLinear, Autoformer)} & 6.726 & 2.860 & 0.265 & 0.106 & \textbf{0.264} & 0.112 & 4.018 & 1.663 & \textbf{1.769} & 1.252 & 6.942 & 3.470 & \multirow{2}{*}{\small{\textcolor{blue}{2}}} & \multirow{2}{*}{\small{0}} \\
                      {} & {} &
                      \small{(0.184)} & 
                      \small{(0.483)} & 
                      \small{(0.015)} & 
                      \small{(0.002)} & 
                      \small{(0.015)} & 
                      \small{(0.001)} & 
                      \small{(0.256)} & 
                      \small{(0.063)} &
                      \small{(0.132)} & 
                      \small{(0.243)} &
                      \small{(0.283)} & 
                      \small{(0.043)} \\
\bottomrule
\end{tabular}
}
\end{table}


 \label{section:appendixB}

% The $\mathtt{\backslash onecolumn}$ command above can be kept in place if you prefer a one-column appendix, or can be removed if you prefer a two-column appendix.  Apart from this possible change, the style (font size, spacing, margins, page numbering, etc.) should be kept the same as the main body.
%%%%%%%%%%%%%%%%%%%%%%%%%%%%%%%%%%%%%%%%%%%%%%%%%%%%%%%%%%%%%%%%%%%%%%%%%%%%%%%
%%%%%%%%%%%%%%%%%%%%%%%%%%%%%%%%%%%%%%%%%%%%%%%%%%%%%%%%%%%%%%%%%%%%%%%%%%%%%%%


\end{document}


% This document was modified from the file originally made available by
% Pat Langley and Andrea Danyluk for ICML-2K. This version was created
% by Iain Murray in 2018, and modified by Alexandre Bouchard in
% 2019 and 2021 and by Csaba Szepesvari, Gang Niu and Sivan Sabato in 2022.
% Modified again in 2023 and 2024 by Sivan Sabato and Jonathan Scarlett.
% Previous contributors include Dan Roy, Lise Getoor and Tobias
% Scheffer, which was slightly modified from the 2010 version by
% Thorsten Joachims & Johannes Fuernkranz, slightly modified from the
% 2009 version by Kiri Wagstaff and Sam Roweis's 2008 version, which is
% slightly modified from Prasad Tadepalli's 2007 version which is a
% lightly changed version of the previous year's version by Andrew
% Moore, which was in turn edited from those of Kristian Kersting and
% Codrina Lauth. Alex Smola contributed to the algorithmic style files.
