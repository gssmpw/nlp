\subsection{Model Rank and Pairwise Comparisons}\label{apd:cd_diagrams}
\begin{figure*}[htbp]
    \centering
   \begin{minipage}{0.48\textwidth}
        \centering
        \includegraphics[width=\textwidth]{images/cd_baselines_aggregate.pdf}
        \subcaption{Model choice for in-distribution series (p-value: 2.71e-8)}
        \label{fig:cd_baselines_aggregate}
    \end{minipage}%
    \hfill
    \begin{minipage}{0.48\textwidth}
        \centering
        \includegraphics[width=\textwidth]{images/cd_baselines_component.pdf}
        \subcaption{Model choice for out-of-distribution series (p-value: 3.52e-8)}
        \label{fig:cd_baselines_component}
    \end{minipage}

\caption{Critical Difference (CD) diagrams illustrate model ranks and pairwise statistical comparisons of model performance on compositional reasoning tasks across all datasets. Lower ranks indicate better performance. A thick horizontal line groups models that are not significantly different. The statistical tests used to generate the CD diagrams are detailed in Section \ref{section:evaluation}. \textbf{(a, b)} The patch-based Transformer models and MLP-based models outperform other models in both traditional and compositional reasoning forecasting paradigms. The Friedman p-value is included in the subcaptions.}
    \label{fig:cd_diagrams_baselines_apd}
\end{figure*}



\begin{figure*}[htbp]
    \centering

    \vspace{1ex} % Vertical space between rows

    \begin{minipage}{0.48\textwidth}
        \centering
        \includegraphics[width=\textwidth]{images/cd_tokenization_ablation_aggregate.pdf}
        \subcaption{Tokenization for in-distribution series (p-value=2.91e-3)}
        \label{fig:cd_tokenization_ablation_aggregate}
    \end{minipage}%
    \hfill
    \begin{minipage}{0.48\textwidth}
        \centering
        \includegraphics[width=\textwidth]{images/cd_tokenization_ablation_component.pdf}
        \subcaption{Tokenization for out-of-distribution series (p-value=1.17e-2)}
        \label{fig:cd_tokenization_ablation_component}
    \end{minipage}

    \vspace{1ex} % Vertical space between rows

    \begin{minipage}{0.48\textwidth}
        \centering
        \includegraphics[width=\textwidth]{images/cd_size_ablation_aggregate.pdf}
        \subcaption{Model size for in-distribution series (p-value: 6.58e-2)}
        \label{fig:cd_size_ablation_aggregate}
    \end{minipage}%
    \hfill
    \begin{minipage}{0.48\textwidth}
        \centering
        \includegraphics[width=\textwidth]{images/cd_size_ablation_component.pdf}
        \subcaption{Model Size for out-of-distribution series (p-value=8.58e-2)}
        \label{fig:cd_size_ablation_component}
    \end{minipage}

    \vspace{1ex} % Vertical space between rows

    \begin{minipage}{0.48\textwidth}
        \centering
        \includegraphics[width=\textwidth]{images/cd_attn_ablation_aggregate.pdf}
        \subcaption{Attn. type for in-distribution series}
        \label{fig:cd_attn_ablation_aggregate}
    \end{minipage}%
    \hfill
    \begin{minipage}{0.48\textwidth}
        \centering
        \includegraphics[width=\textwidth]{images/cd_attn_ablation_component.pdf}
        \subcaption{Attn. type for out-of-distribution series}
        \label{fig:cd_attn_ablation_component}
    \end{minipage}

    \vspace{1ex} % Vertical space between rows

    \begin{minipage}{0.48\textwidth}
        \centering
        \includegraphics[width=\textwidth]{images/cd_proj_ablation_aggregate.pdf}
        \subcaption{Projection layer for in-distribution series}
        \label{fig:cd_proj_ablation_aggregate}
    \end{minipage}%
    \hfill
    \begin{minipage}{0.48\textwidth}
        \centering
        \includegraphics[width=\textwidth]{images/cd_proj_ablation_component.pdf}
        \subcaption{Projection layer for out-of-distribution series}
        \label{fig:cd_proj_ablation_component}
    \end{minipage}

    \vspace{1ex} % Vertical space between rows

    \begin{minipage}{0.48\textwidth}
        \centering
        \includegraphics[width=\textwidth]{images/cd_tokenlen_ablation_aggregate.pdf}
        \subcaption{Token length for in-distribution series (p-value=0.22)}
        \label{fig:cd_tokenlen_ablation_aggregate}
    \end{minipage}%
    \hfill
    \begin{minipage}{0.48\textwidth}
        \centering
        \includegraphics[width=\textwidth]{images/cd_tokenlen_ablation_component.pdf}
        \subcaption{Token length for out-of-distribution series (p-value=8.62e-4)}
        \label{fig:cd_tokenlen_ablation_component}
    \end{minipage}

    \vspace{1ex} % Vertical space between rows

    \begin{minipage}{0.48\textwidth}
        \centering
        \includegraphics[width=\textwidth]{images/cd_pe_ablation_aggregate.pdf}
        \subcaption{Positional encoding for in-distribution series (p-value=0.71)}
        \label{fig:cd_pe_ablation_aggregate}
    \end{minipage}%
    \hfill
    \begin{minipage}{0.48\textwidth}
        \centering
        \includegraphics[width=\textwidth]{images/cd_pe_ablation_component.pdf}
        \subcaption{Positional encoding for out-of-distribution series (p-value=0.46)}
        \label{fig:cd_pe_ablation_component}
    \end{minipage}
    
    \vspace{1ex} % Vertical space between rows

    \begin{minipage}{0.48\textwidth}
        \centering
        \includegraphics[width=\textwidth]{images/cd_loss_ablation_aggregate.pdf}
        \subcaption{Loss function for in-distribution series (p-value=0.24)}
        \label{fig:cd_loss_ablation_aggregate}
    \end{minipage}%
    \hfill
    \begin{minipage}{0.48\textwidth}
        \centering
        \includegraphics[width=\textwidth]{images/cd_loss_ablation_component.pdf}
        \subcaption{Loss function for out-of-distribution series (p-value=0.14)}
        \label{fig:cd_loss_ablation_component}
    \end{minipage}

    \vspace{1ex} % Vertical space between rows

    \begin{minipage}{0.48\textwidth}
        \centering
        \includegraphics[width=\textwidth]{images/cd_scaler_ablation_aggregate.pdf}
        \subcaption{Scaler for in-distribution series}
        \label{fig:cd_scaler_ablation_aggregate}
    \end{minipage}%
    \hfill
    \begin{minipage}{0.48\textwidth}
        \centering
        \includegraphics[width=\textwidth]{images/cd_scaler_ablation_component.pdf}
        \subcaption{Scaler function for out-of-distribution series}
        \label{fig:cd_scaler_ablation_component}
    \end{minipage}

    \vspace{1ex} % Vertical space between rows

    \begin{minipage}{0.48\textwidth}
        \centering
        \includegraphics[width=\textwidth]{images/cd_contextlen_ablation_aggregate.pdf}
        \subcaption{Context length function for in-distribution series}
        \label{fig:cd_contextlen_ablation_aggregate}
    \end{minipage}%
    \hfill
    \begin{minipage}{0.48\textwidth}
        \centering
        \includegraphics[width=\textwidth]{images/cd_contextlen_ablation_component.pdf}
        \subcaption{Context length for out-of-distribution series}
        \label{fig:cd_contextlen_ablation_component}
    \end{minipage}

    \vspace{1ex} % Vertical space between rows

    \begin{minipage}{0.48\textwidth}
        \centering
        \includegraphics[width=\textwidth]{images/cd_decomp_ablation_aggregate.pdf}
        \subcaption{Input decomposition for in-distribution series}
        \label{fig:cd_decomp_ablation_aggregate}
    \end{minipage}%
    \hfill
    \begin{minipage}{0.48\textwidth}
        \centering
        \includegraphics[width=\textwidth]{images/cd_decomp_ablation_component.pdf}
        \subcaption{Input decomposition function for out-of-distribution series}
        \label{fig:cd_decomp_ablation_component}
    \end{minipage}

    \caption{Critical Difference (CD) diagrams illustrate model ranks and pairwise statistical comparisons of model performance on compositional reasoning tasks across all datasets. Lower ranks indicate better performance. A thick horizontal line groups models that are not significantly different. The statistical tests used to generate the CD diagrams are detailed in Section \ref{section:evaluation}. For analyses comparing three or more methods, the Friedman p-value is included in the subcaptions.}
    \label{fig:cd_diagrams_apd}
\end{figure*}

\newpage
\subsection{Model Forecasts}\label{apd:forecast_examples}
We include example forecasts for each of the 6 datasets used in this study.


\begin{figure*}[ht!]
    \centering

    \begin{minipage}{0.411\textwidth}
        \centering
        \includegraphics[width=\textwidth, trim=0 0 310 0, clip]{images/sine_id_forecast_example.pdf}
        \subcaption{Model forecasts for in-distribution Sinusoid series}
        \label{fig:sine_id_forecast_example}
    \end{minipage}
    \hfill
    \begin{minipage}{0.584\textwidth}
        \centering
        \includegraphics[width=\textwidth]{images/sine_ood_forecast_example.pdf}
        \subcaption{Model forecasts for out-of-distribution Sinusoid series}
        \label{fig:sine_ood_forecast_example.pdf}
    \end{minipage}

    \vspace{2.9em}
    
    \begin{minipage}{0.411\textwidth}
        \centering
        \includegraphics[width=\textwidth, trim=0 0 310 0, clip]{images/ecl_id_forecast_example.pdf}
        \subcaption{Model forecasts for in-distribution ECL series
        }
        \label{fig:ecl_id_forecast_example}
    \end{minipage}
    \hfill
    \begin{minipage}{0.584\textwidth}
        \centering
        \includegraphics[width=\textwidth]{images/ecl_ood_forecast_example.pdf}
        \subcaption{Model forecasts for out-of-distribution ECL series}
        \label{fig:ecl_ood_forecast_example.pdf}
    \end{minipage}

    \vspace{2.9em}
    
    \begin{minipage}{0.411\textwidth}
        \centering
        \includegraphics[width=\textwidth, trim=0 0 310 0, clip]{images/ettm2_id_forecast_example.pdf}
        \subcaption{Model forecasts for in-distribution ETTm2 series
        }
        \label{fig:ettm2_id_forecast_example}
    \end{minipage}
    \hfill
    \begin{minipage}{0.584\textwidth}
        \centering
        \includegraphics[width=\textwidth]{images/ettm2_ood_forecast_example.pdf}
        \subcaption{Model forecasts for out-of-distribution ETTm2 series}
        \label{fig:ettm2_ood_forecast_example.pdf}
    \end{minipage}

    \caption{\textbf{(a, c, e)} Forecasts for a ground truth series $\mathbf{y}(t)$ for the Sinusoid, ETTm2, and ECL datasets for models trained using the traditional forecasting paradigm. \textbf{(b, d, f)} Forecasts for the Sinusoid, ETTm2, and ECL datasets for models trained using the compositional reasoning forecasting paradigm. Patch-based Transformer models and MLP-based models (top), which rank among the top-performing models, demonstrate generalization to out-of-distribution time series, whereas other Transformer variants and linear models struggle to do so (bottom).}
    \label{fig:forecast_examples1_apd}
\end{figure*}

\newpage
\begin{figure*}[ht!]
    \centering
    \begin{minipage}{0.411\textwidth}
        \centering
        \includegraphics[width=\textwidth, trim=0 0 310 0, clip]{images/solar_id_forecast_example.pdf}
        \subcaption{Model forecasts for in-distribution Solar series
        }
        \label{fig:solar_id_forecast_example}
    \end{minipage}
    \hfill
    \begin{minipage}{0.584\textwidth}
        \centering
        \includegraphics[width=\textwidth]{images/solar_ood_forecast_example.pdf}
        \subcaption{Model forecasts for out-of-distribution Solar series}
        \label{fig:solar_ood_forecast_example.pdf}
    \end{minipage}

    \vspace{2.9em}
    
    \begin{minipage}{0.411\textwidth}
        \centering
        \includegraphics[width=\textwidth, trim=0 0 310 0, clip]{images/subseasonal_id_forecast_example.pdf}
        \subcaption{Model forecasts for in-distribution Subseasonal series
        }
        \label{fig:subseasonal_id_forecast_example}
    \end{minipage}
    \hfill
    \begin{minipage}{0.584\textwidth}
        \centering
        \includegraphics[width=\textwidth]{images/subseasonal_ood_forecast_example.pdf}
        \subcaption{Model forecasts for out-of-distribution Subseasonal series}
        \label{fig:subseasonal_ood_forecast_example.pdf}
    \end{minipage}

    \vspace{2.9em}

    \begin{minipage}{0.411\textwidth}
        \centering
        \includegraphics[width=\textwidth, trim=0 0 310 0, clip]{images/loopseattle_id_forecast_example.pdf}
        \subcaption{Model forecasts for in-distribution Loop Seattle series
        }
        \label{fig:loopseattle_id_forecast_example}
    \end{minipage}
    \hfill
    \begin{minipage}{0.584\textwidth}
        \centering
        \includegraphics[width=\textwidth]{images/loopseattle_ood_forecast_example.pdf}
        \subcaption{Model forecasts for out-of-distribution Loop Seattle series}
        \label{fig:loopseattle_ood_forecast_example.pdf}
    \end{minipage}

    \caption{\textbf{(a, c, e)} Forecasts for a ground truth series $\mathbf{y}(t)$ for the Solar, Subseasonal, and Loop Seattle datasets for models trained using the traditional forecasting paradigm. \textbf{(b, d, f)} Forecasts for a ground truth series $\mathbf{y}(t)$ for the Solar, Subseasonal, and Loop Seattle datasets for models trained using the compositional reasoning forecasting paradigm. Patch-based Transformer models and MLP-based models (top), which rank among the top-performing models, demonstrate generalization to out-of-distribution time series, whereas other Transformer variants and linear models struggle to do so (bottom).}
    \label{fig:forecast_examples2_apd}
\end{figure*}


\newpage
\subsection{Model Performance, Efficiency, and Size Comparison}\label{apd:flops_plots}
We rank model performance across datasets and compare rank with model computational complexity in terms of floating-point operations per second (FLOPs) and model size in terms of the total number of trainable parameters. 

\begin{figure*}[htbp]
    \centering

    \begin{minipage}{0.495\textwidth}
        \centering
        \includegraphics[width=\textwidth]{images/flops_aggregate.pdf}
        \subcaption{In-distribution series results (excluding \TimesNet)}
        \label{fig:flops_aggregate}
    \end{minipage}%
    \hfill
    \begin{minipage}{0.495\textwidth}
        \centering
        \includegraphics[width=\textwidth]{images/flops_component.pdf}
        \subcaption{Out-of-distribution series results (excluding \TimesNet)}
        \label{fig:flops_component}
    \end{minipage}

    \vspace{1ex} % Vertical space between rows

    \begin{minipage}{0.495\textwidth}
        \centering
        \includegraphics[width=\textwidth]{images/flops_aggregate_with_timesnet.pdf}
        \subcaption{In-distribution series results (including \TimesNet)}
        \label{fig:flops_aggregate_with_timesnet}
    \end{minipage}%
    \hfill
    \begin{minipage}{0.495\textwidth}
        \centering
        \includegraphics[width=\textwidth]{images/flops_component_with_timesnet.pdf}
        \subcaption{Out-of-distribution series results (including \TimesNet)}
        \label{fig:flops_component_with_timesnet}
    \end{minipage}

    \caption{Comparison of average rank across datasets and random seeds versus model computational complexity, measured by floating-point operations per second (FLOPs). The size of each point represents the number of trainable parameters, highlighting the trade-offs between model complexity and performance. \textbf{(a)} In-distribution and \textbf{(b)} out-of-distribution results for all models, excluding \TimesNet, are shown to provide a clearer comparison by mitigating the parameter size skew. \textbf{(c)} In-distribution and \textbf{(d)} out-of-distribution results for all models, including \TimesNet.}
    \label{fig:flops_model_comparison_apd}
\end{figure*}


\newpage
\subsection{Model Composition Reasoning Results}\label{apd:composition_full_table_results}
We include the complete table results with MAE error mean and standard deviation measured across three random seeds. The results of the compositional reasoning task for 16 widely adopted time series forecasting models are included in Table~\ref{tab:composition_baseline_results_table}. Composition reasoning task results for controlled ablations of architecture components used in TSFMs are shown in Table~\ref{tab:composition_t5_results_table}.

\begin{table}[!ht] 
\centering
\caption{Mean Absolute Error (MAE) averaged over 3 random seeds (with standard deviation in parentheses) for composition reasoning tasks. The out-of-distribution (OOD) column presents MAE results for models trained via the compositional reasoning forecasting paradigm. The in-distribution (ID) column presents MAE results for models trained via the traditional forecasting paradigm. The \Tfive\ with the best patch length (PL) from Table~\ref{tab:composition_t5_results_table} is included. Best results are highlighted in \textbf{bold}, second best results are \underline{underlined}. The count of instances across datasets where the model ranks in the top three for performance is shown in the second to last column with non-zero entries in \textcolor{blue}{blue}. The average number of top $k$ compositions the model can outperform over the datasets is shown in the last column with nonzero entries in \textcolor{purple}{purple}.}
\label{tab:composition_baseline_results_table}
\resizebox{1.0\textwidth}{!}{
\begin{tabular}{ll|cc|cc|cc|cc|cc|cc||cc|cc}
\toprule
\multicolumn{2}{c|}{\multirow{2}{*}{\textbf{Model}}} & \multicolumn{2}{c}{\textbf{Synthetic Sinusoid}} & \multicolumn{2}{c}{\textbf{ECL}} & \multicolumn{2}{c}{\textbf{ETTm2}} & \multicolumn{2}{c}{\textbf{Solar}} & \multicolumn{2}{c}{\textbf{Subseasonal}} & \multicolumn{2}{c||}{\textbf{Loop Seattle}} & \multicolumn{2}{c}{\textbf{\small{Top 3 Win Count}}} & \multicolumn{2}{c}{\textbf{\small{Top $k$ Basis Wins}}} \\
\cline{3-18}
{} & {} & \textbf{OOD} & \textbf{ID} & \textbf{OOD} & \textbf{ID} & \textbf{OOD} & \textbf{ID} & \textbf{OOD} & \textbf{ID} & \textbf{OOD} & \textbf{ID} & \textbf{OOD} & \textbf{ID} & \textbf{\small{OOD}} & \textbf{\small{ID}} & \textbf{\small{OOD}} & \textbf{\small{ID}} \\
\hline\hline
\multirow{4}{*}{\rotatebox[origin=c]{90}{\textbf{Statistical}}} & \multirow{2}{*}{\ARIMA} & -- & 15.538 & -- & 0.822 & -- & 0.332 & -- & 9.687 & -- & 7.855 & -- & 8.638 & \multirow{2}{*}{\small{0}} & \multirow{2}{*}{\small{0}} & \multirow{2}{*}{\small{--}} & \multirow{2}{*}{\small{0.5}} \\
                      {} & {} &
                      \small{(--)} & 
                      \small{(--)} & 
                      \small{(--)} & 
                      \small{(--)} & 
                      \small{(--)} & 
                      \small{(--)} & 
                      \small{(--)} & 
                      \small{(--)} &
                      \small{(--)} & 
                      \small{(--)} & 
                      \small{(--)} & 
                      {} &
                      {} &
                      {} \\
\cline{2-18}
{} & \multirow{2}{*}{\ETS} & -- & 16.075 & -- & 0.105 & -- & 0.211 & -- & 1.730 & -- & 2.067 & -- & 5.575 & \small{--} & \small{0} & \small{--} & \small{0.8} \\
                      {} & {} &
                      \small{(--)} & 
                      \small{(--)} & 
                      \small{(--)} & 
                      \small{(--)} & 
                      \small{(--)} & 
                      \small{(--)} & 
                      \small{(--)} & 
                      \small{(--)} &
                      \small{(--)} & 
                      \small{(--)} & 
                      \small{(--)} & 
                      \small{(--)} &
                      {} &
                      {} \\
\hline
\multirow{4}{*}{\rotatebox[origin=c]{90}{\textbf{Linear}}} & \multirow{2}{*}{\DLinear} & 12.991 & 12.460 & 0.820 & \underline{0.103} & 0.330 & 0.135 & 9.925 & \textbf{1.555} & 8.042 & 1.496 & 9.085 & 4.293 & \multirow{2}{*}{\small{0}} & \multirow{2}{*}{\small{\textcolor{blue}{2}}} & \multirow{2}{*}{\small{0.2}} & \multirow{2}{*}{\small{\textcolor{purple}{55.5}}} \\
                      {} & {} &
                      \small{(0.051)} & \small{(0.025)} & \small{(0.073)} & \small{(0.000)} & \small{(0.028)} & \small{(0.000)} & \small{(1.000)} & \small{(0.007)} & \small{(0.402)} & \small{(0.030)} &
                      \small{(0.327)} & 
                      \small{(0.033)} &
                      {} &
                      {} \\
\cline{2-18}
{} & \multirow{2}{*}{\NLinear} & 13.287 & 13.056 & 0.801 & 0.104 & 0.325 & 0.136 & 9.681 & \underline{1.569} & 8.436 & 1.509 & 9.026 & 4.307 & \multirow{2}{*}{\small{0}} & \multirow{2}{*}{\small{\textcolor{blue}{1}}} & \multirow{2}{*}{\small{0.3}} & \multirow{2}{*}{\small{\textcolor{purple}{53.7}}} \\
                      {} & {} &
                      \small{(0.098)} & \small{(0.060)} & \small{(0.026)} & \small{(0.001)} & \small{(0.007)} & \small{(0.002)} & \small{(1.203)} & \small{(0.010)} & \small{(0.552)} & \small{(0.011)} &
                      \small{(0.248)} & 
                      \small{(0.017)} &
                      {} &
                      {} \\
\hline
\multirow{8}{*}{\rotatebox[origin=c]{90}{\textbf{MLP-Based}}} & \multirow{2}{*}{\MLP} & \underline{8.647} & 2.475 & \underline{0.283} & 0.106 & 0.253 & 0.114 & \underline{4.826} & 1.559 & 1.886 & 1.456 & 7.839 & 3.864 & \multirow{2}{*}{\small{\textcolor{blue}{3}}} & \multirow{2}{*}{\small{\textcolor{blue}{1}}} & \multirow{2}{*}{\small{\textcolor{purple}{10.2}}} & \multirow{2}{*}{\small{\textcolor{purple}{61.2}}} \\
                      {} & {} &
                      \small{(0.258)} & \small{(0.011)} & \small{(0.020)} & \small{(0.004)} & \small{(0.011)} & \small{(0.002)} & \small{(0.043)} & \small{(0.028)} & \small{(0.118)} & \small{(0.046)} &
                      \small{(0.139)} & 
                      \small{(0.061)} &
                      {} &
                      {} \\
\cline{2-18}
{} & \multirow{2}{*}{\NHITS} & 8.924 & \textbf{1.106} & 0.295 & \textbf{0.101} & \textbf{0.214} & \textbf{0.100} & 5.682 & 1.592 & 1.858 & \underline{1.135} & \underline{7.747} & 3.448 & \multirow{2}{*}{\small{\textcolor{blue}{4}}} & \multirow{2}{*}{\small{\textcolor{blue}{4}}} & \multirow{2}{*}{\small{\textcolor{purple}{11.0}}} & \multirow{2}{*}{\small{\textcolor{purple}{64.3}}} \\
                      {} & {} &
                      \small{(0.044)} & \small{(0.036)} & \small{(0.019)} & \small{(0.001)} & \small{(0.006)} & \small{(0.003)} & \small{(0.651)} & \small{(0.026)} & \small{(0.060)} & \small{(0.008)} &
                      \small{(0.141)} & 
                      \small{(0.043)} &
                      {} &
                      {} \\
\cline{2-18}
{} & \multirow{2}{*}{\NBEATS} & 8.907 & 1.383 & 0.294 & \underline{0.103} & \underline{0.216} & \underline{0.102} & 5.852 & 1.599 & \underline{1.840} & 1.177 & 7.763 & 3.479 & \multirow{2}{*}{\small{\textcolor{blue}{5}}} & \multirow{2}{*}{\small{\textcolor{blue}{4}}} & \multirow{2}{*}{\small{\textcolor{purple}{11.2}}} & \multirow{2}{*}{\small{\textcolor{purple}{61.8}}} \\
                      {} & {} &
                      \small{(0.106)} & \small{(0.058)} & \small{(0.016)} & \small{(0.004)} & \small{(0.005)} & \small{(0.002)} & \small{(0.645)} & \small{(0.028)} & \small{(0.108)} & \small{(0.007)} &
                      \small{(0.097)} & 
                      \small{(0.034)} &
                      {} &
                      {} \\
\cline{2-18}
{} & \multirow{2}{*}{\TSMixer} & 14.466 & 15.090 & 0.799 & 0.129 & 0.335 & 0.182 & 9.877 & 1.979 & 7.770 & 1.602 & 8.865 & 5.565 & \multirow{2}{*}{\small{0}} & \multirow{2}{*}{\small{0}} & \multirow{2}{*}{\small{0.3}} & \multirow{2}{*}{\small{\textcolor{purple}{19.7}}} \\
                      {} & {} &
                      \small{(1.804)} & \small{(0.339)} & \small{(0.013)} & \small{(0.004)} & \small{(0.024)} & \small{(0.051)} & \small{(0.118)} & \small{(0.132)} & \small{(0.543)} & \small{(0.099)} &
                      \small{(0.249)} & 
                      \small{(1.038)} &
                      {} &
                      {} \\
\hline
\multirow{2}{*}{\rotatebox[origin=c]{90}{\textbf{RNN}}} & \multirow{2}{*}{\LSTM} & 13.410 & 4.238 & 0.835 & 0.110 & 0.337 & 0.135 & 10.241 & 1.717 & 8.095 & 1.545 & 9.465 & 3.703 & \multirow{2}{*}{\small{0}} & \multirow{2}{*}{\small{0}} & \multirow{2}{*}{\small{0}} & \multirow{2}{*}{\small{\textcolor{purple}{48.7}}} \\
                      {} & {} &
                      \small{(0.203)} & \small{(0.637)} & \small{(0.004)} & \small{(0.001)} & \small{(0.000)} & \small{(0.006)} & \small{(0.620)} & \small{(0.131)} & \small{(0.093)} & \small{(0.075)} &
                      \small{(1.093)} & 
                      \small{(0.054)} &
                      {} &
                      {} \\
\hline
\multirow{4}{*}{\rotatebox[origin=c]{90}{\textbf{CNN}}} & \multirow{2}{*}{\TCN} & 11.478 & 3.833 & 0.837 & 0.106 & 0.339 & 0.135 & 9.868 & 1.642 & 6.170 & 1.234 & 8.792 & \underline{3.422} & \multirow{2}{*}{\small{0}} & \multirow{2}{*}{\small{\textcolor{blue}{1}}} & \multirow{2}{*}{\small{0.7}} & \multirow{2}{*}{\small{\textcolor{purple}{55.7}}} \\
                      {} & {} &
                      \small{(0.410)} & \small{(0.193)} & \small{(0.001)} & \small{(0.002)} & \small{(0.004)} & \small{(0.001)} & \small{(0.016)} & \small{(0.112)} & \small{(3.256)} & \small{(0.031)} &
                      \small{(0.020)} & 
                      \small{(0.119)} &
                      {} &
                      {} \\
\cline{2-18}
{} & \multirow{2}{*}{\TimesNet} & 9.788 & \underline{2.451} & 0.518 & 0.104 & 0.313 & 0.109 & 9.914 & 1.714 & 4.109 & 1.500 & 9.872 & 2.970 & \multirow{2}{*}{\small{0}} & \multirow{2}{*}{\small{\textcolor{blue}{2}}} & \multirow{2}{*}{\small{\textcolor{purple}{2.3}}} & \multirow{2}{*}{\small{\textcolor{purple}{52.8}}} \\
                      {} & {} &
                      \small{(0.493)} & \small{(0.252)} & \small{(0.285)} & \small{(0.003)} & \small{(0.042)} & \small{(0.004)} & \small{(0.116)} & \small{(0.259)} & \small{(3.409)} & \small{(0.111)} &
                      \small{(0.991)} & 
                      \small{(0.360)} &
                      {} &
                      {} \\
\hline
\multirow{22}{*}{\rotatebox[origin=c]{90}{\textbf{Transformer}}} & \multirow{2}{*}{\VanillaTransformer} & 12.279 & 4.935 & 0.919 & 0.106 & 0.334 & 0.136 & 11.956 & 1.667 & 9.641 & 1.276 & 11.675 & 3.591 & \multirow{2}{*}{\small{0}} & \multirow{2}{*}{\small{0}} & \multirow{2}{*}{\small{0.2}} & \multirow{2}{*}{\small{\textcolor{purple}{54.8}}} \\
                      {} & {} &
                      \small{(0.660)} & \small{(0.080)} & \small{(0.033)} & \small{(0.002)} & \small{(0.038)} & \small{(0.005)} & \small{(2.659)} & \small{(0.032)} & \small{(0.363)} & \small{(0.071)} &
                      \small{(0.803)} & 
                      \small{(0.008)} &
                      {} &
                      {} \\
\cline{2-18}
{} & \multirow{2}{*}{\iTransformer} & 15.478 & 15.203 & 0.829 & 0.157 & 0.326 & 0.196 & 9.822 & 1.805 & 8.447 & 1.628 & 9.182 & 4.871 & \multirow{2}{*}{\small{0}} & \multirow{2}{*}{\small{0}} & \multirow{2}{*}{\small{0.2}} & \multirow{2}{*}{\small{\textcolor{purple}{23.5}}} \\
                      {} & {} &
                      \small{(0.351)} & \small{(0.782)} & \small{(0.007)} & \small{(0.007)} & \small{(0.003)} & \small{(0.001)} & \small{(0.055)} & \small{(0.093)} & \small{(0.503)} & \small{(0.027)} &
                      \small{(0.380)} & 
                      \small{(0.105)} &
                      {} &
                      {} \\
\cline{2-18}
{} & \multirow{2}{*}{\Autoformer} & 15.301 & 15.018 & 0.795 & 0.137 & 0.330 & 0.294 & 10.348 & 2.108 & 7.933 & 2.390 & 8.634 & 4.758 & \multirow{2}{*}{\small{0}} & \multirow{2}{*}{\small{0}} & \multirow{2}{*}{\small{0.3}} & \multirow{2}{*}{\small{\textcolor{purple}{15.8}}} \\
                      {} & {} &
                     \small{(0.184)} & \small{(0.130)} & \small{(0.014)} & \small{(0.009)} & \small{(0.008)} & \small{(0.046)} & \small{(0.945)} & \small{(0.638)} & \small{(0.410)} & \small{(0.551)} &
                      \small{(0.166)} & 
                      \small{(0.307)} &
                      {} &
                      {} \\
\cline{2-18}
{} & \multirow{2}{*}{\Informer} & 14.353 & 10.144 & 0.787 & 0.128 & 0.321 & 0.141 & 8.351 & 1.662 & 6.878 & 1.564 & 11.549 & 4.241 & \multirow{2}{*}{\small{0}} & \multirow{2}{*}{\small{0}} & \multirow{2}{*}{\small{0.5}} & \multirow{2}{*}{\small{\textcolor{purple}{41.0}}} \\
                      {} & {} &
                      \small{(0.174)} & \small{(2.257)} & \small{(0.082)} & \small{(0.006)} & \small{(0.016)} & \small{(0.007)} & \small{(1.560)} & \small{(0.051)} & \small{(1.340)} & \small{(0.094)} &
                      \small{(0.627)} & 
                      \small{(0.105)} &
                      {} &
                      {} \\
\cline{2-18}
{} & \multirow{2}{*}{\TFT} & 14.531 & 9.745 & 0.445 & 0.115 & 0.312 & 0.117 & 12.873 & 2.106 & 2.684 & 1.454 & 11.280 & 5.340 & \multirow{2}{*}{\small{0}} & \multirow{2}{*}{\small{0}} & \multirow{2}{*}{\small{\textcolor{purple}{4.7}}} & \multirow{2}{*}{\small{\textcolor{purple}{28.5}}} \\
                      {} & {} &
                      \small{(0.880)} & \small{(1.210)} & \small{(0.072)} & \small{(0.010)} & \small{(0.029)} & \small{(0.005)} & \small{(2.141)} & \small{(0.538)} & \small{(0.146)} & \small{(0.151)} &
                      \small{(1.182)} & 
                      \small{(0.855)} &
                      {} &
                      {} \\
\cline{2-18}
{} & \multirow{2}{*}{\PatchTST\ (PL=8)} & 13.808 & 12.036 & 0.713 & 0.428 & 0.309 & 0.156 & 9.081 & 2.350 & 6.242 & 2.007 & 11.440 & 4.755 & \multirow{2}{*}{\small{0}} & \multirow{2}{*}{\small{0}} & \multirow{2}{*}{\small{0.7}} & \multirow{2}{*}{\small{\textcolor{purple}{18.7}}} \\
                      {} & {} &
                      \small{(0.471)} & \small{(0.738)} & \small{(0.172)} & \small{(0.318)} & \small{(0.031)} & \small{(0.026)} & \small{(0.857)} & \small{(0.872)} & \small{(1.286)} & \small{(0.228)} &
                      \small{(1.564)} & 
                      \small{(0.774)} &
                      {} &
                      {} \\
\cline{2-18}
{} & \multirow{2}{*}{\PatchTST\ (PL=16)} & 14.133 & 13.633 & 0.666 & 0.279 & 0.253 & 0.150 & 10.788 & 1.633 & 5.877 & 1.904 & 9.855 & 4.989 & \multirow{2}{*}{\small{0}} & \multirow{2}{*}{\small{0}} & \multirow{2}{*}{\small{0.8}} & \multirow{2}{*}{\small{\textcolor{purple}{27.8}}} \\
                      {} & {} &
                      \small{(2.693)} & \small{(1.202)} & \small{(0.185)} & \small{(0.025)} & \small{(0.017)} & \small{(0.005)} & \small{(0.139)} & \small{(0.013)} & \small{(1.389)} & \small{(0.336)} &
                      \small{(0.618)} & 
                      \small{(1.000)} &
                      {} &
                      {} \\
\cline{2-18}
{} & \multirow{2}{*}{\PatchTST\ (PL=32)} & 13.316 & 13.412 & 0.736 & 0.273 & 0.271 & 0.168 & 8.267 & 2.721 & 4.904 & 2.385 & 9.422 & 4.924 & \multirow{2}{*}{\small{0}} & \multirow{2}{*}{\small{0}} & \multirow{2}{*}{\small{\textcolor{purple}{1.3}}} & \multirow{2}{*}{\small{\textcolor{purple}{14.3}}} \\
                      {} & {} &
                      \small{(1.751)} & \small{(2.499)} & \small{(0.248)} & \small{(0.106)} & \small{(0.010)} & \small{(0.002)} & \small{(1.302)} & \small{(1.625)} & \small{(2.018)} & \small{(0.639)} &
                      \small{(1.359)} & 
                      \small{(0.133)} &
                      {} &
                      {} \\
\cline{2-18}
{} & \multirow{2}{*}{\PatchTST\ (PL=64)} & 12.232 & 13.544 & 0.482 & 0.122 & 0.247 & 0.169 & 8.054 & 2.813 & 2.292 & 1.754 & 10.529 & 4.405 & \multirow{2}{*}{\small{\textcolor{blue}{1}}} & \multirow{2}{*}{\small{0}} & \multirow{2}{*}{\small{\textcolor{purple}{6.8}}} & \multirow{2}{*}{\small{\textcolor{purple}{27.2}}} \\
                      {} & {} &
                      \small{(0.252)} & \small{(1.055)} & \small{(0.306)} & \small{(0.011)} & \small{(0.007)} & \small{(0.040)} & \small{(2.248)} & \small{(0.724)} & \small{(0.152)} & \small{(0.353)} &
                      \small{(4.987)} & 
                      \small{(0.883)} &
                      {} &
                      {} \\
\cline{2-18}
{} & \multirow{2}{*}{\PatchTST\ (PL=96)} & 11.235 & 8.374 & 0.508 & 0.196 & 0.250 & 0.147 & 6.426 & 1.799 & 2.185 & 1.659 & 7.965 & 4.579 & \multirow{2}{*}{\small{0}} & \multirow{2}{*}{\small{0}} & \multirow{2}{*}{\small{\textcolor{purple}{7.7}}} & \multirow{2}{*}{\small{\textcolor{purple}{27.0}}} \\
                      {} & {} &
                      \small{(0.483)} & \small{(1.834)} & \small{(0.287)} & \small{(0.051)} & \small{(0.013)} & \small{(0.021)} & \small{(1.312)} & \small{(0.304)} & \small{(0.168)} & \small{(0.282)} &
                      \small{(0.355)} & 
                      \small{(0.978)} &
                      {} &
                      {} \\
\cline{2-18}
{} & \multirow{2}{*}{\PatchTST\ (PL=128)} & 10.696 & 6.959 & 0.832 & 0.161 & 0.323 & 0.138 & 5.726 & 1.964 & 2.448 & 1.678 & 9.378 & 3.653 & \multirow{2}{*}{\small{0}} & \multirow{2}{*}{\small{0}} & \multirow{2}{*}{\small{\textcolor{purple}{5.7}}} & \multirow{2}{*}{\small{\textcolor{purple}{32.7}}} \\
                      {} & {} &
                      \small{(0.907)} & \small{(2.126)} & \small{(0.010)} & \small{(0.053)} & \small{(0.037)} & \small{(0.018)} & \small{(0.567)} & \small{(0.266)} & \small{(0.467)} & \small{(0.100)} &
                      \small{(0.185)} & 
                      \small{(0.257)} &
                      {} &
                      {} \\
\cline{2-18}
{} & \multirow{2}{*}{\Tfive\ (Best PL)} & \textbf{7.177} & 2.480 & \textbf{0.239} & \underline{0.103} & 0.259 & 0.103 & \textbf{3.899} & 1.578 & \textbf{1.714} & \textbf{1.097} & \textbf{6.589} & \textbf{3.351} & \multirow{2}{*}{\small{\textcolor{blue}{5}}} & \multirow{2}{*}{\small{\textcolor{blue}{4}}} & \multirow{2}{*}{\small{\textcolor{purple}{12.2}}} & \multirow{2}{*}{\small{\textcolor{purple}{64.3}}} \\
                      {} & {} &
                      \small{(0.089)} & 
                      \small{(0.198)} & 
                      \small{(0.005)} & 
                      \small{(0.001)} & 
                      \small{(0.008)} & 
                      \small{(0.006)} & 
                      \small{(0.578)} & 
                      \small{(0.009)} &
                      \small{(0.040) } &
                      \small{0.039)} &
                      \small{(0.109)} & 
                      \small{(0.017)} &
                      {} &
                      {} \\
\bottomrule
% \multicolumn{13}{c}{\textbf{  }} \\
% \cline{1-14}
% \multirow{8}{*}{\rotatebox[origin=c]{90}{\textbf{Baseline}}} & \Fourier\ (topk=1) & 
%     \multicolumn{2}{c|}{13.032} & 
%     \multicolumn{2}{c|}{0.669} & 
%     \multicolumn{2}{c|}{0.325} &
%     \multicolumn{2}{c|}{9.878} & 
%     \multicolumn{2}{c|}{7.977} & 
%     \multicolumn{2}{c}{8.758} \\
% \cline{2-14}
% {} & \Fourier\ (topk=2) & 
%     \multicolumn{2}{c|}{\textcolor{purple}{6.495}} & 
%     \multicolumn{2}{c|}{0.321} & 
%     \multicolumn{2}{c|}{0.305} & 
%     \multicolumn{2}{c|}{7.840} & 
%     \multicolumn{2}{c|}{6.758} & 
%     \multicolumn{2}{c}{8.150} \\
% \cline{2-14}
% {} & \Fourier\ (topk=3) & 
%     \multicolumn{2}{c|}{5.078} & 
%     \multicolumn{2}{c|}{0.250} & 
%     \multicolumn{2}{c|}{0.252} &
%     \multicolumn{2}{c|}{4.218} & 
%     \multicolumn{2}{c|}{5.185} & 
%     \multicolumn{2}{c}{\textcolor{purple}{6.259}} \\
% \cline{2-14}
% {} & \Fourier\ (topk=4) & 
%     \multicolumn{2}{c|}{0.719} & 
%     \multicolumn{2}{c|}{\textcolor{purple}{0.218}} & 
%     \multicolumn{2}{c|}{0.249} & 
%     \multicolumn{2}{c|}{\textcolor{purple}{3.663}} & 
%     \multicolumn{2}{c|}{5.048} & 
%     \multicolumn{2}{c}{6.297} \\
% \cline{2-14}
% {} & \Fourier\ (topk=5) & 
%     \multicolumn{2}{c|}{0.729} & 
%     \multicolumn{2}{c|}{0.185} &
%     \multicolumn{2}{c|}{0.222} & 
%     \multicolumn{2}{c|}{2.058} & 
%     \multicolumn{2}{c|}{4.911} & 
%     \multicolumn{2}{c}{5.961} \\
% \cline{2-14}
% {} & \Fourier\ (topk=6) & 
%     \multicolumn{2}{c|}{0.738} & 
%     \multicolumn{2}{c|}{0.175} &
%     \multicolumn{2}{c|}{0.222} & 
%     \multicolumn{2}{c|}{1.972} & 
%     \multicolumn{2}{c|}{4.812} & 
%     \multicolumn{2}{c}{6.050} \\
% \cline{2-14}
% {} & \Fourier\ (topk=7) & 
%     \multicolumn{2}{c|}{0.737} & 
%     \multicolumn{2}{c|}{0.160} &
%     \multicolumn{2}{c|}{\textcolor{purple}{0.211}} & 
%     \multicolumn{2}{c|}{1.827} & 
%     \multicolumn{2}{c|}{4.662} & 
%     \multicolumn{2}{c}{5.863} \\
% \cline{2-14}
% {} & \Fourier\ (topk=63) & 
%     \multicolumn{2}{c|}{0.887} & 
%     \multicolumn{2}{c|}{0.100} &
%     \multicolumn{2}{c|}{0.075} & 
%     \multicolumn{2}{c|}{1.640} & 
%     \multicolumn{2}{c|}{\textcolor{purple}{1.702}} & 
%     \multicolumn{2}{c}{4.471} \\
% \cline{1-14}
\end{tabular}
}
\end{table}



\begin{table}[ht]
\centering
\caption{Mean Absolute Error (MAE) averaged over 3 random seeds (with standard deviation in parentheses) for composition reasoning tasks. The out-of-distribution (OOD) column presents MAE results for models trained via the compositional reasoning forecasting paradigm. The in-distribution (ID) column presents MAE results for models trained via the traditional forecasting paradigm. Best results are highlighted in \textbf{bold}. The count of instances across datasets where the model has the best performance is shown in the last column with non-zero entries in \textcolor{blue}{blue}.}
\label{tab:composition_t5_results_table}
\resizebox{1.0\textwidth}{!}{
\begin{tabular}{ll|cc|cc|cc|cc|cc|cc||cc}
\toprule
\multicolumn{2}{c|}{\multirow{2}{*}{\textbf{Transformer Model (T5 Backbone)}}} & \multicolumn{2}{c}{\textbf{Synthetic Sinusoid}} & \multicolumn{2}{c}{\textbf{ECL}} & \multicolumn{2}{c}{\textbf{ETTm2}} & \multicolumn{2}{c}{\textbf{Solar}} & \multicolumn{2}{c}{\textbf{Subseasonal}} & \multicolumn{2}{c||}{\textbf{Loop Seattle}} & \multicolumn{2}{c}{\textbf{\small{Win Count}}} \\
\cline{3-16}
{} & {} & \textbf{OOD} & \textbf{ID} & \textbf{OOD} & \textbf{ID} & \textbf{OOD} & \textbf{ID} & \textbf{OOD} & \textbf{ID} & \textbf{OOD} & \textbf{ID} & \textbf{OOD} & \textbf{ID} & \textbf{\small{OOD}} & \textbf{\small{ID}} \\
\hline\hline
\multirow{8}{*}{\rotatebox[origin=c]{90}{\textbf{Tokenization}}} & \multirow{2}{*}{None} & 14.032 & 4.894 & 0.685 & \textbf{0.106} & 0.369 & 0.121 & 9.969 & \textbf{1.635} & 6.401 & 1.572 & 14.327 & 3.766 & \multirow{2}{*}{\small{0}} & \multirow{2}{*}{\small{\textcolor{blue}{2}}}\\
                      {} & {} &
                      \small{(1.612)} & 
                      \small{(0.140)} & 
                      \small{(0.148)} & 
                      \small{(0.005)} & 
                      \small{(0.015)} & 
                      \small{(0.010)} & 
                      \small{(0.804)} & 
                      \small{(0.079)} &
                      \small{(1.739)} & 
                      \small{(0.049) } &
                      \small{(2.548)} & 
                      \small{(0.035)} \\
\cline{2-16}
{} & \multirow{2}{*}{Fixed Length Patches} & \textbf{8.648} & \textbf{2.611} & \textbf{0.266} & 0.107 & \textbf{0.268} & \textbf{0.100} & \textbf{3.908} & 1.663 & \textbf{1.729} & \textbf{1.154} & \textbf{7.658} & \textbf{3.118} & \multirow{2}{*}{\small{\textcolor{blue}{6}}} & \multirow{2}{*}{\small{\textcolor{blue}{4}}}\\
                      {} & {} &
                      \small{(0.072)} & 
                      \small{(0.158)} & 
                      \small{(0.019)} & 
                      \small{(0.002)} & 
                      \small{(0.003)} & 
                      \small{(0.003)} & 
                      \small{(0.204)} & 
                      \small{(0.023)} &
                      \small{(0.028)} & 
                      \small{(0.029)} &
                      \small{(0.100)} & 
                      \small{(0.026)} \\
\cline{2-16}
{} & \multirow{2}{*}{Binning} & 17.039 & 9.504 & 0.833 & 0.270 & 0.317 & 0.199 & 8.445 & 4.719 & 3.758 & 3.253 & 12.728 & 7.735 & \multirow{2}{*}{\small{0}} & \multirow{2}{*}{\small{0}}\\
                      {} & {} &
                      \small{(0.788)} & 
                      \small{(0.502)} & 
                      \small{(0.007)} & 
                      \small{(0.003)} & 
                      \small{(0.011)} & 
                      \small{(0.004)} & 
                      \small{(4.206)} & 
                      \small{(0.151)} &
                      \small{(0.365)} & 
                      \small{(0.532)} &
                      \small{(2.339)} & 
                      \small{(0.985)} \\
\cline{2-16}
{} & \multirow{2}{*}{Lags} & 13.442 & 4.599 & 0.820 & 0.120 & 0.415 & 0.126 & 10.156  & 1.669 & 4.022 & 1.376 & 11.638 & 3.897 & \multirow{2}{*}{\small{0}} & \multirow{2}{*}{\small{0}} \\
                      {} & {} &
                      \small{(0.254)} & 
                      \small{(0.328)} & 
                      \small{(0.178)} & 
                      \small{(0.005)} & 
                      \small{(0.046)} & 
                      \small{(0.006)} & 
                      \small{(1.364)} & 
                      \small{(0.030)} &
                      \small{(0.187)} & 
                      \small{(0.043)} &
                      \small{(1.556)} & 
                      \small{(0.074)} \\
\hline\hline
\multirow{8}{*}{\rotatebox[origin=c]{90}{\textbf{Model Size}}} & \multirow{2}{*}{Tiny} & \textbf{7.644} & 2.628 & \textbf{0.239} & 0.105 & 0.274 & 0.109 & 3.899 & \textbf{1.641} & 1.899 & 1.097 & \textbf{6.701} & 3.355 & \multirow{2}{*}{\small{\textcolor{blue}{3}}} & \multirow{2}{*}{\small{\textcolor{blue}{1}}} \\
                      {} & {} &
                      \small{(0.025)} & 
                      \small{(0.108)} & 
                      \small{(0.005)} & 
                      \small{(0.002)} & 
                      \small{(0.005)} & 
                      \small{(0.006)} & 
                      \small{(0.578)} & 
                      \small{(0.068)} &
                      \small{(0.203)} & 
                      \small{(0.039)} &
                      \small{(0.153)} & 
                      \small{(0.053)} \\
\cline{2-16}
{} & \multirow{2}{*}{Mini} & 7.882 & 2.318 & 0.242 & 0.107 & \textbf{0.273} & 0.103 & \textbf{3.810} & 1.663 & \textbf{1.769} & 1.104 & 6.888 & 3.018 & \multirow{2}{*}{\small{\textcolor{blue}{3}}} & \multirow{2}{*}{\small{0}}\\
                      {} & {} &
                      \small{(0.069)} & 
                      \small{(0.076)} & 
                      \small{(0.006)} & 
                      \small{(0.001)} & 
                      \small{(0.005)} & 
                      \small{(0.001)} & 
                      \small{(0.226)} & 
                      \small{(0.027)} &
                      \small{(0.108)} & 
                      \small{(0.008)} &
                      \small{(0.032)} & 
                      \small{(0.011)} \\
\cline{2-16}
{} & \multirow{2}{*}{Small} & 8.057 & 2.103 & 0.268 & 0.103 & 0.268 & 0.096 & 4.172 & 1.665 & 1.770 & 1.048 & 6.924 & 2.764 & \multirow{2}{*}{\small{0}} & \multirow{2}{*}{\small{0}}\\
                      {} & {} &
                      \small{(0.119)} & 
                      \small{(0.098)} & 
                      \small{(0.009)} & 
                      \small{(0.002)} & 
                      \small{(0.014)} & 
                      \small{(0.001)} & 
                      \small{(0.192)} & 
                      \small{(0.028)} &
                      \small{(0.197)} & 
                      \small{(0.016)} &
                      \small{(0.187)} & 
                      \small{(0.033)} \\
\cline{2-16}
{} & \multirow{2}{*}{Base} & 8.308 & \textbf{2.084} & 0.247 & \textbf{0.100} & 0.274 & \textbf{0.091} & 4.338 & 1.667 & 1.853 & \textbf{0.967} & 7.005 & \textbf{2.536} & \multirow{2}{*}{\small{0}} & \multirow{2}{*}{\small{\textcolor{blue}{5}}} \\
                      {} & {} &
                      \small{(0.228)} & 
                      \small{(0.072)} & 
                      \small{(0.014)} & 
                      \small{(0.006)} & 
                      \small{(0.007)} & 
                      \small{(0.001)} & 
                      \small{(0.123)} & 
                      \small{(0.065)} &
                      \small{(0.150)} & 
                      \small{(0.012)} &
                      \small{(0.255)} & 
                      \small{(0.061)} \\
\hline\hline
\multirow{4}{*}{\rotatebox[origin=c]{90}{\textbf{Attn. Type}}} & \multirow{2}{*}{Bidirectional Attn.} & \textbf{7.644} & \textbf{2.628} & \textbf{0.239} & \textbf{0.105} & 0.274 & 0.109 & \textbf{3.899} & 1.641 & 1.899 & \textbf{1.097} & \textbf{6.701} & 3.355 & \multirow{2}{*}{\small{\textcolor{blue}{4}}} & \multirow{2}{*}{\small{\textcolor{blue}{3}}}\\
                      {} & {} &
                      \small{(0.025)} & 
                      \small{(0.108)} & 
                      \small{(0.005)} & 
                      \small{(0.002)} & 
                      \small{(0.005)} & 
                      \small{(0.006)} & 
                      \small{(0.578)} & 
                      \small{(0.068)} &
                      \small{(0.203)} & 
                      \small{(0.039)} &
                      \small{(0.153)} & 
                      \small{(0.053)} \\
\cline{2-16}
{} & \multirow{2}{*}{Causal Attn.} & 7.978 & 2.828 & 0.248 & 0.106 & \textbf{0.267} & \textbf{0.105} & 4.307 & \textbf{1.589} & \textbf{1.820} & 1.170 & 6.891 & \textbf{3.337} & \multirow{2}{*}{\small{\textcolor{blue}{2}}} & \multirow{2}{*}{\small{\textcolor{blue}{3}}} \\
                      {} & {} &
                      \small{(0.02)} & 
                      \small{(0.233)} & 
                      \small{(0.005)} & 
                      \small{(0.004)} & 
                      \small{(0.010)} & 
                      \small{(0.003)} & 
                      \small{(0.144)} & 
                      \small{(0.042)} &
                      \small{(0.173)} & 
                      \small{(0.054)} &
                      \small{(0.133)} & 
                      \small{(0.044)} \\
\hline\hline
\multirow{4}{*}{\rotatebox[origin=c]{90}{\textbf{Proj./Head}}} & \multirow{2}{*}{Linear} & \textbf{7.644} & 2.628 & \textbf{0.239} & 0.105 & 0.274 & 0.109 & \textbf{3.899} & 1.641 & 1.899 & 1.097 & \textbf{6.701} & 3.355 & \multirow{2}{*}{\small{\textcolor{blue}{4}}} & \multirow{2}{*}{\small{0}} \\
                      {} & {} &
                      \small{(0.025)} & 
                      \small{(0.108)} & 
                      \small{(0.005)} & 
                      \small{(0.002)} & 
                      \small{(0.005)} & 
                      \small{(0.006)} & 
                      \small{(0.578)} & 
                      \small{(0.068)} &
                      \small{(0.203)} & 
                      \small{(0.039)} &
                      \small{(0.153)} & 
                      \small{(0.053)} \\
\cline{2-16}
{} & \multirow{2}{*}{Residual} & 8.537 & \textbf{2.617} & 0.311 & \textbf{0.102} & \textbf{0.256} & \textbf{0.085} & 4.880 & \textbf{1.595} & \textbf{1.871} & \textbf{0.924} & 7.931 & \textbf{2.524} & \multirow{2}{*}{\small{\textcolor{blue}{2}}} & \multirow{2}{*}{\small{\textcolor{blue}{6}}} \\
                      {} & {} &
                      \small{(0.184)} & 
                      \small{(0.043)} &
                      \small{(0.011)} & 
                      \small{(0.002)} & 
                      \small{(0.008)} & 
                      \small{(0.003)} & 
                      \small{(0.389)} &
                      \small{(0.039)} & 
                      \small{(0.296)} &
                      \small{(0.016)} &
                      \small{(0.655)} &
                      \small{(0.028)} \\
\hline\hline
\multirow{12}{*}{\rotatebox[origin=c]{90}{\textbf{Token (Patch) Length}}} & \multirow{2}{*}{8} & 9.327 & 3.183 & 0.346 & \textbf{0.103} & 0.300 & 0.111 & 7.721 & \textbf{1.578} & 2.581 & 1.532 & 8.048 & 3.573 & \multirow{2}{*}{\small{0}} & \multirow{2}{*}{\small{\textcolor{blue}{2}}} \\
                      {} & {} &
                      \small{(0.28)} & 
                      \small{(0.117)} & 
                      \small{(0.040)} & 
                      \small{(0.001)} & 
                      \small{(0.013)} & 
                      \small{(0.003)} & 
                      \small{(0.640)} & 
                      \small{(0.009)} &
                      \small{(0.335)} & 
                      \small{(0.002)} &
                      \small{(0.108)} & 
                      \small{(0.047)} \\
\cline{2-16}
{} & \multirow{2}{*}{16} & 9.007 & 3.573 & 0.418 & 0.105 & 0.283 & \textbf{0.103} & 9.139 & 1.673 & 2.731 & 1.362 & 8.499 & 3.510 & \multirow{2}{*}{\small{0}} & \multirow{2}{*}{\small{\textcolor{blue}{1}}} \\
                      {} & {} &
                      \small{(0.057)} & 
                      \small{(0.235)} & 
                      \small{(0.062)} & 
                      \small{(0.002)} & 
                      \small{(0.011)} & 
                      \small{(0.006)} & 
                      \small{(0.400)} & 
                      \small{(0.088)} &
                      \small{(0.939)} & 
                      \small{(0.25)} &
                      \small{(0.339)} & 
                      \small{(0.060)} \\
\cline{2-16}
{} & \multirow{2}{*}{32} & 10.11 & 3.719 & 0.244 & 0.108 & 0.289 & 0.109 & 5.256 & 1.652 & 2.059 & 1.318 & 7.014 & 3.463 & \multirow{2}{*}{\small{0}} & \multirow{2}{*}{\small{0}} \\
                      {} & {} &
                      \small{(0.07)} & 
                      \small{(0.190)} & 
                      \small{(0.011)} & 
                      \small{(0.004)} & 
                      \small{(0.007)} & 
                      \small{(0.002)} & 
                      \small{(0.591)} & 
                      \small{(0.071)} &
                      \small{(0.171)} & 
                      \small{(0.187)} &
                      \small{(0.167)} & 
                      \small{(0.015)} \\
\cline{2-16}
{} & \multirow{2}{*}{64} & 8.747 & 2.971 & \textbf{0.239 }& 0.106 & 0.285 & 0.107 & 4.201 & 1.651 & 1.819 & 1.265 & \textbf{6.589} & 3.408 & \multirow{2}{*}{\small{\textcolor{blue}{2}}} & \multirow{2}{*}{\small{0}} \\
                      {} & {} &
                      \small{(0.243)} & 
                      \small{(0.209)} & 
                      \small{(0.009)} & 
                      \small{(0.003)} & 
                      \small{(0.002)} & 
                      \small{(0.006)} & 
                      \small{(0.103)} & 
                      \small{(0.076)} &
                      \small{(0.032)} & 
                      \small{(0.198)} &
                      \small{(0.109)} & 
                      \small{(0.044)} \\
\cline{2-16}
{} & \multirow{2}{*}{96} & 7.644 & 2.628 & \textbf{0.239} & 0.105 & 0.274 & 0.109 & \textbf{3.899} & 1.641 & 1.899 & \textbf{1.097} & 6.701 & 3.355 & \multirow{2}{*}{\small{\textcolor{blue}{2}}} & \multirow{2}{*}{\small{\textcolor{blue}{1}}} \\
                      {} & {} &
                      \small{(0.025)} & 
                      \small{(0.108)} & 
                      \small{(0.005)} & 
                      \small{(0.002)} & 
                      \small{(0.005)} & 
                      \small{(0.006)} & 
                      \small{(0.578)} & 
                      \small{(0.068)} &
                      \small{(0.203)} & 
                      \small{(0.039)} &
                      \small{(0.153)} & 
                      \small{(0.053)} \\
\cline{2-16}
{} & \multirow{2}{*}{128} & \textbf{7.177} & \textbf{2.480} & 0.255 & 0.107 & \textbf{0.259} & 0.107 & 4.385 & 1.673 & \textbf{1.714} & 1.100 & 6.878 & \textbf{3.351} & \multirow{2}{*}{\small{\textcolor{blue}{3}}} & \multirow{2}{*}{\small{\textcolor{blue}{2}}} \\
                      {} & {} &
                      \small{(0.089)} & 
                      \small{(0.198)} & 
                      \small{(0.012)} & 
                      \small{(0.001)} & 
                      \small{(0.008)} & 
                      \small{(0.006)} & 
                      \small{(0.145)} & 
                      \small{(0.033)} &
                      \small{(0.040)} & 
                      \small{(0.069)} &
                      \small{(0.069)} & 
                      \small{(0.017)} \\
\hline\hline
\multirow{8}{*}{\rotatebox[origin=c]{90}{\textbf{Positional Encoding}}} & \multirow{2}{*}{Relative} & 7.751 & 2.750 & 0.284 & 0.105 & \textbf{0.268} & 0.111 & 4.373 & 1.640 & \textbf{1.826} & \textbf{1.093} & 6.883 & \textbf{3.337} & \multirow{2}{*}{\small{\textcolor{blue}{2}}} & \multirow{2}{*}{\small{\textcolor{blue}{2}}} \\
                      {} & {} &
                      \small{(0.163)} & 
                      \small{(0.262)} & 
                      \small{(0.059) } & 
                      \small{(0.002)} & 
                      \small{(0.015)} & 
                      \small{(0.002)} & 
                      \small{(0.254)} & 
                      \small{(0.033)} &
                      \small{(0.156)} & 
                      \small{(0.025)} &
                      \small{(0.162)} & 
                      \small{(0.020)} \\
\cline{2-16}
{} & \multirow{2}{*}{SinCos} & 7.881 & \textbf{2.456} & 0.247 & \textbf{0.104} & 0.270 & 0.112 & 4.357 & \textbf{1.624} & 1.839 & 1.464 & 6.818 & 3.366 & \multirow{2}{*}{\small{0}} & \multirow{2}{*}{\small{\textcolor{blue}{3}}} \\
                      {} & {} &
                      \small{(0.148)} & 
                      \small{(0.049)} & 
                      \small{(0.015)} & 
                      \small{(0.003)} & 
                      \small{(0.007)} & 
                      \small{(0.004)} & 
                      \small{(0.284)} & 
                      \small{(0.025)} &
                      \small{(0.195)} & 
                      \small{(0.051)} &
                      \small{(0.182} & 
                      \small{(0.055)} \\
\cline{2-16}
{} & \multirow{2}{*}{SinCos+Relative} & \textbf{7.644} & 2.628 & \textbf{0.239} & 0.105 & 0.274 & \textbf{0.109} & \textbf{3.899} & 1.641 & 1.899 & 1.097 & \textbf{6.701} & 3.355 & \multirow{2}{*}{\small{\textcolor{blue}{4}}} & \multirow{2}{*}{\small{\textcolor{blue}{1}}} \\
                      {} & {} &
                      \small{(0.025)} & 
                      \small{(0.108)} & 
                      \small{(0.005)} & 
                      \small{(0.002)} & 
                      \small{(0.005)} & 
                      \small{(0.006)} & 
                      \small{(0.578)} & 
                      \small{(0.068)} &
                      \small{(0.203)} & 
                      \small{(0.039)} &
                      \small{(0.153)} & 
                      \small{(0.053)} \\
\cline{2-16}
{} & \multirow{2}{*}{RoPE} & 7.921 & 2.701 & 0.255 & \textbf{0.104} & 0.274 & 0.112 & 4.608 & 1.647 & 1.872 & 1.106 & 6.721 & 3.378 & \multirow{2}{*}{\small{0}} & \multirow{2}{*}{\small{\textcolor{blue}{1}}} \\
                      {} & {} &
                      \small{(0.125)} & 
                      \small{(0.077)} & 
                      \small{(0.006)} & 
                      \small{(0.002)} & 
                      \small{(0.002)} & 
                      \small{(0.002)} & 
                      \small{(0.333)} & 
                      \small{(0.032)} &
                      \small{(0.125)} &
                      \small{(0.037)} & 
                      \small{(0.174)} &
                      \small{(0.033)} \\
\hline\hline
\multirow{8}{*}{\rotatebox[origin=c]{90}{\textbf{Loss Function}}} & \multirow{2}{*}{MAE} & \textbf{7.644} & 2.628 & 0.239 & 0.105 & 0.274 &\textbf{ 0.109} & 3.899 & \textbf{1.641} & 1.899 & \textbf{1.097} & 6.701 & 3.355 & \multirow{2}{*}{\small{\textcolor{blue}{1}}} & \multirow{2}{*}{\small{\textcolor{blue}{3}}} \\
                      {} & {} &
                      \small{(0.025)} & 
                      \small{(0.108)} & 
                      \small{(0.005)} & 
                      \small{(0.002)} & 
                      \small{(0.005)} & 
                      \small{(0.006)} & 
                      \small{(0.578)} & 
                      \small{(0.068)} &
                      \small{(0.203)} & 
                      \small{(0.039)} &
                      \small{(0.153)} & 
                      \small{(0.053)} \\
\cline{2-16}
{} & \multirow{2}{*}{MSE} & 7.694 & \textbf{2.618} & \textbf{0.229} & 0.109 & \textbf{0.262} & 0.116 & 3.633 & 1.752 & 1.891 & 1.411 & 6.613 & 3.302 & \multirow{2}{*}{\small{\textcolor{blue}{2}}} & \multirow{2}{*}{\small{\textcolor{blue}{1}}} \\
                      {} & {} &
                      \small{(0.178)} & 
                      \small{(0.088)} & 
                      \small{(0.006)} & 
                      \small{(0.004)} & 
                      \small{(0.007)} & 
                      \small{(0.010)} & 
                      \small{(0.466)} & 
                      \small{(0.030)} &
                      \small{(0.193)} & 
                      \small{(0.214)} &
                      \small{(0.118)} & 
                      \small{(0.045)} \\
\cline{2-16}
{} & \multirow{2}{*}{Huber} & 7.648 & 2.914 & 0.234 & 0.108 & 0.264 & 0.112 & \textbf{4.035} & 1.746 & 1.801 & 1.258 & \textbf{6.571} & \textbf{3.281} & \multirow{2}{*}{\small{\textcolor{blue}{2}}} & \multirow{2}{*}{\small{\textcolor{blue}{1}}} \\
                      {} & {} &
                      \small{(0.076)} & 
                      \small{(0.234)} & 
                      \small{(0.012)} & 
                      \small{(0.005)} & 
                      \small{(0.012)} & 
                      \small{(0.003)} & 
                      \small{(0.633)} & 
                      \small{(0.056)} &
                      \small{(0.084)} & 
                      \small{(0.194)} &
                      \small{(0.182)} & 
                      \small{(0.047)} \\
\cline{2-16}
{} & \multirow{2}{*}{StudentT} & 7.776 & 2.660 & 0.244 & \textbf{0.104} & 0.270 & 0.113 & 4.134 & 1.648 & \textbf{1.735} & 1.377 & 6.872 & 3.543 & \multirow{2}{*}{\small{\textcolor{blue}{1}}} & \multirow{2}{*}{\small{\textcolor{blue}{1}}} \\
                      {} & {} &
                      \small{(0.281)} & 
                      \small{(0.105)} & 
                      \small{(0.003)} & 
                      \small{(0.003)} & 
                      \small{(0.004)} & 
                      \small{(0.003)} & 
                      \small{(0.804)} & 
                      \small{(0.025)} &
                      \small{(0.037)} & 
                      \small{(0.024)} &
                      \small{(0.191)} & 
                      \small{(0.026)} \\
\hline\hline
\multirow{4}{*}{\rotatebox[origin=c]{90}{\textbf{Scaler}}} & \multirow{2}{*}{RevIN (Standard, non-learnable)} & \textbf{7.644} & \textbf{2.628} & \textbf{0.239} & 0.105 & 0.274 & 0.109 & \textbf{3.899} & \textbf{1.641} & 1.899 & \textbf{1.097} & \textbf{6.701} & \textbf{3.355} & \multirow{2}{*}{\small{\textcolor{blue}{4}}} & \multirow{2}{*}{\small{\textcolor{blue}{4}}} \\
                      {} & {} &
                      \small{(0.025)} & 
                      \small{(0.108)} & 
                      \small{(0.005)} & 
                      \small{(0.002)} & 
                      \small{(0.005)} & 
                      \small{(0.006)} & 
                      \small{(0.578)} & 
                      \small{(0.068)} &
                      \small{(0.203)} & 
                      \small{(0.039)} &
                      \small{(0.153)} & 
                      \small{(0.053)} \\
\cline{2-16}
% {} & \multirow{2}{*}{RevIN (Standard, learnable)} & 7.644 & 2.628 & 0.239 & 0.105 & 0.274 & 0.109 & 3.899 & 1.641 & 1.899 & 1.097 & 6.701 & 3.355 \\
%                       {} & {} &
%                       \small{(0.025)} & 
%                       \small{(0.108)} & 
%                       \small{(0.005)} & 
%                       \small{(0.002)} & 
%                       \small{(0.005)} & 
%                       \small{(0.006)} & 
%                       \small{(0.578)} & 
%                       \small{(0.068)} &
%                       \small{(0.203)} & 
%                       \small{(0.039)} &
%                       \small{(0.153)} & 
%                       \small{(0.053)} \\
% \cline{2-14}
{} & \multirow{2}{*}{Robust} & 7.931 & 2.725 & 0.326 & \textbf{0.103} & \textbf{0.270} & \textbf{0.106} & 8.170 & 1.734 & \textbf{1.736} & 1.232 & 6.982 & 3.412 & \multirow{2}{*}{\small{\textcolor{blue}{2}}} & \multirow{2}{*}{\small{\textcolor{blue}{2}}} \\
                      {} & {} &
                      \small{(0.026)} & 
                      \small{(0.234)} & 
                      \small{(0.006)} & 
                      \small{(0.003)} & 
                      \small{(0.007)} & 
                      \small{(0.007)} & 
                      \small{(0.389)} & 
                      \small{(0.093)} &
                      \small{(0.015)} & 
                      \small{(0.227)} &
                      \small{(0.271)} & 
                      \small{(0.043)} \\
\hline\hline
\multirow{4}{*}{\rotatebox[origin=c]{90}{\textbf{Context}}} & \multirow{2}{*}{256} & 7.644 & 2.628 & \textbf{0.239} & \textbf{0.105} & \textbf{0.274} & \textbf{0.109} & \textbf{3.899} & \textbf{1.641} & \textbf{1.899} & \textbf{1.097} & 6.701 & \textbf{3.355} & \multirow{2}{*}{\small{\textcolor{blue}{4}}} & \multirow{2}{*}{\small{\textcolor{blue}{5}}} \\
                      {} & {} &
                      \small{(0.025)} & 
                      \small{(0.108)} & 
                      \small{(0.005)} & 
                      \small{(0.002)} & 
                      \small{(0.005)} & 
                      \small{(0.006)} & 
                      \small{(0.578)} & 
                      \small{(0.068)} &
                      \small{(0.203)} & 
                      \small{(0.039)} &
                      \small{(0.153)} & 
                      \small{(0.053)} \\
\cline{2-16}
{} & \multirow{2}{*}{512} & \textbf{7.072} & \textbf{2.605} & 0.253 & 0.111 & 0.298 & 0.151 & 4.187 & 1.740 & 1.934 & 1.829 & \textbf{6.295} & 4.013 & \multirow{2}{*}{\small{\textcolor{blue}{2}}} & \multirow{2}{*}{\small{\textcolor{blue}{1}}} \\
                      {} & {} &
                      \small{(0.382)} & 
                      \small{(0.158)} & 
                      \small{(0.037)} & 
                      \small{(0.005)} & 
                      \small{(0.015)} & 
                      \small{(0.008)} & 
                      \small{(0.216)} & 
                      \small{(0.046)} &
                      \small{(0.083)} & 
                      \small{(0.094)} &
                      \small{(0.284)} & 
                      \small{(0.002)} \\
\hline\hline
\multirow{4}{*}{\rotatebox[origin=c]{90}{\textbf{Decomp.}}} & \multirow{2}{*}{None} & \textbf{7.644} & \textbf{2.628} & \textbf{0.239} & \textbf{0.105} & 0.274 & \textbf{0.109} & \textbf{3.899} & \textbf{1.641} & 1.899 & \textbf{1.097} & \textbf{6.701} & \textbf{3.355} & \multirow{2}{*}{\small{\textcolor{blue}{4}}} & \multirow{2}{*}{\small{\textcolor{blue}{6}}} \\
                      {} & {} &
                      \small{(0.025)} & 
                      \small{(0.108)} & 
                      \small{(0.005)} & 
                      \small{(0.002)} & 
                      \small{(0.005)} & 
                      \small{(0.006)} & 
                      \small{(0.578)} & 
                      \small{(0.068)} &
                      \small{(0.203)} & 
                      \small{(0.039)} &
                      \small{(0.153)} & 
                      \small{(0.053)} \\
\cline{2-16}
{} & \multirow{2}{*}{Moving Avg. Filter (DLinear, Autoformer)} & 6.726 & 2.860 & 0.265 & 0.106 & \textbf{0.264} & 0.112 & 4.018 & 1.663 & \textbf{1.769} & 1.252 & 6.942 & 3.470 & \multirow{2}{*}{\small{\textcolor{blue}{2}}} & \multirow{2}{*}{\small{0}} \\
                      {} & {} &
                      \small{(0.184)} & 
                      \small{(0.483)} & 
                      \small{(0.015)} & 
                      \small{(0.002)} & 
                      \small{(0.015)} & 
                      \small{(0.001)} & 
                      \small{(0.256)} & 
                      \small{(0.063)} &
                      \small{(0.132)} & 
                      \small{(0.243)} &
                      \small{(0.283)} & 
                      \small{(0.043)} \\
\bottomrule
\end{tabular}
}
\end{table}


