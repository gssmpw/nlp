\begin{figure}[htbp]
    \centering
    \begin{subfigure}[b]{0.495\textwidth}
        \centering
        \includegraphics[width=\textwidth,trim=9 0 0 0,clip]{images/cd_baselines_component.pdf}
        \caption{Model Choice}
        \label{fig:cd_baselines_component_main}
    \end{subfigure}

    \vspace{1em}

    \begin{subfigure}[b]{0.495\textwidth}
        \centering
        \includegraphics[width=\textwidth,trim=9 0 0 0,clip]{images/cd_tokenization_ablation_component.pdf}
        \caption{\Tfive\ Tokenization}
        \label{fig:cd_tokenization_ablation_component_main}
    \end{subfigure}

    \vspace{1em}

    \begin{subfigure}[b]{0.495\textwidth}
        \centering
        \includegraphics[width=\textwidth,trim=40 0 0 0,clip]{images/cd_size_ablation_component.pdf}
        \caption{\Tfive\ size}
        \label{fig:cd_size_ablation_component_main}
    \end{subfigure}

    \vspace{1em}

    \begin{subfigure}[b]{0.495\textwidth}
        \centering
        \includegraphics[width=\textwidth,trim=9 0 0 0,clip]{images/cd_pe_ablation_component.pdf}
        \caption{\Tfive\ positional encoding}
        \label{fig:cd_pe_ablation_component_main}
    \end{subfigure}
    
    \caption{Critical difference diagrams showing performance ranks for composition reasoning tasks across datasets for various \textbf{(a)} models and \textbf{(b, c, e)} \Tfive\ model design decisions. Lower ranks indicate better performance. A thick horizontal line groups a set of models that are not significantly different using the Wilcoxon signed-rank test with Holm correction.}
    \label{fig:main_comparison}
\end{figure}
