%%%%%%%%%%%%%%%%%%%%%%% file template.tex %%%%%%%%%%%%%%%%%%%%%%%%%
%
% This is a general template file for the LaTeX package SVJour3
% for Springer journals.          Springer Heidelberg 2010/09/16
%
% Copy it to a new file with a new name and use it as the basis
% for your article. Delete % signs as needed.
%
% This template includes a few options for different layouts and
% content for various journals. Please consult a previous issue of
% your journal as needed.
%
%%%%%%%%%%%%%%%%%%%%%%%%%%%%%%%%%%%%%%%%%%%%%%%%%%%%%%%%%%%%%%%%%%%
%
% First comes an example EPS file -- just ignore it and
% proceed on the \documentclass line
% your LaTeX will extract the file if required
\begin{filecontents*}{example.eps}
%!PS-Adobe-3.0 EPSF-3.0
%%BoundingBox: 19 19 221 221
%%CreationDate: Mon Sep 29 1997
%%Creator: programmed by hand (JK)
%%EndComments
gsave
newpath
  20 20 moveto
  20 220 lineto
  220 220 lineto
  220 20 lineto
closepath
2 setlinewidth
gsave
  .4 setgray fill
grestore
stroke
grestore
\end{filecontents*}
%
\RequirePackage{fix-cm}
%
%\documentclass{svjour3}                     % onecolumn (standard format)
%\documentclass[smallcondensed]{svjour3}     % onecolumn (ditto)
\documentclass[smallextended]{svjour3}       % onecolumn (second format)
%\documentclass[twocolumn]{svjour3}          % twocolumn
%
\smartqed  % flush right qed marks, e.g. at end of proof
%
\usepackage{graphicx}
\usepackage{subcaption}
\usepackage{url}
\usepackage{booktabs}
%
% \usepackage{mathptmx}      % use Times fonts if available on your TeX system
%
% insert here the call for the packages your document requires
%\usepackage{latexsym}
% etc.
%
% please place your own definitions here and don't use \def but
% \newcommand{}{}
%
% Insert the name of "your journal" with
% \journalname{myjournal}
%
\usepackage{lipsum, color}
\usepackage{comment}
\usepackage{all_math_commands}
% \usepackage{mycmdsSciComp}
\usepackage{soul}

\def\BW#1{\textcolor{red}{#1}}
\def\TT#1{\textcolor{cyan}{#1}}

\usepackage{amsmath,amssymb,amsfonts}
\newtheorem{assumption}[theorem]{Assumption}% 

\DeclareMathOperator{\mean}{mean}
\DeclareMathOperator{\var}{var}
\DeclareMathOperator{\unweave}{unweave}
\DeclareMathOperator{\ZipChain}{ZipChain}
\DeclareMathOperator{\ConvOneD}{Conv1d}
\DeclareMathOperator{\EqConv}{EqConv}
\DeclareMathOperator*{\Aggregate}{Agg}

% Time-Delayed Aggregation GNN
\newcommand{\Centralized}{Centralized}
\newcommand{\TimeStep}{{\Delta t}}
\newcommand{\ChannelsIn}{{C_{\mathrm{in}}}}
\newcommand{\ChannelsOut}{{C_{\mathrm{out}}}}
\newcommand{\FeaturesIn}{{F_{\mathrm{in}}}}
\newcommand{\FeaturesOut}{{F_{\mathrm{out}}}}
\newcommand{\RowsIn}{{R_{\mathrm{in}}}}
\newcommand{\RowsOut}{{R_{\mathrm{out}}}}
\newcommand{\ColsIn}{{K_{\mathrm{in}}}}
\newcommand{\ColsOut}{{K_{\mathrm{out}}}}
\newcommand{\TimeInitial}{{t_{\mathrm{init}}}}


\newcommand{\RadiusTannerEtAlMinU}{{R_U}}
\newcommand{\RadiusMin}{{R_{\min}}}

\DeclareMathOperator{\Unif}{Unif}

% prevent one footnote from spanning multiple pages
% source: https://tex.stackexchange.com/questions/148411/url-in-footnote-spans-two-pages
\interfootnotelinepenalty=10000
% \input{math_commands_SciComp}

\begin{document}
% \include{commands/Conv}
% \include{commands/CuckerSmale}
% \include{commands/DAgger}
% \include{commands/EqConv}
% \include{commands/Expert}
% \include{commands/Flock}
% \section{Analysis}
\label{sec:Analysis}
% In this section, we will discuss the generalization of this phenomenon. 
% Considering content limitation, We put the respective confusion matrixes in Appendix \ref{app:generalization}.
In this section, we perform analytical experiments on paragraph-level paraphrase datasets to generalize our findings to longer texts.
We first demonstrate the extension of our findings to other task formats (Section~\ref{sec:beyond paraphrasing}). 
Then we go through a set of methods to try to escape from the attractor cycles in the remaining subsections.

\subsection{Beyond Paraphrase Generation}
\label{sec:beyond paraphrasing}

% \textbf{Effect of text length.} 
% A longer context can naturally be conveyed in many different ways. However, this does not hold true for LLMs.
% We extend our sentence-level experiments to the paraghraph-level.
% We utilize GPT-4o-mini to successively rephrase the text 15 times in our paragraph datasets and measure the periodicity and the differences between paraphrases.

% \textbf{Task Extension.} 

% According to our analysis in Section~\ref{sec:Convergence}, LLMs are expected to exhibit periodicity in invertible tasks. We further propose 4 tasks with the invertible characteristic, including polishing (\textbf{Pol.}), informal-to-formal style transfer (\textbf{I/F.}), clarify (\textbf{Clar.}), and translation (\textbf{Trans.}) using our paragraph dataset.
% The details of these tasks can be accessed in Appendix \ref{App:tasks}.
Our earlier results indicate that successive paraphrasing leads LLMs to settle into periodic attractors—specifically, 2-period limit cycles. According to the systems-theoretic perspective, such cycles should arise whenever the transformation is invertible, enabling a bidirectional mapping that makes prior states easily reproducible. To test this, we examine four additional invertible tasks at the paragraph level: polishing (Pol.), clarification (Clar.), informal-to-formal style transfer (I/F.), and forward/backward translation (Trans.). These tasks are defined in Appendix~\ref{App:tasks}.

\begin{figure}
\centering
\includegraphics[width=0.45\textwidth]{article/figures/app_task_extv2.pdf}
\caption{The difference confusion matrix for four tasks beyond paraphrasing. 
Note that in translations, the difference between texts in two different languages is set to one.}
\label{figs:task_extensions}
\end{figure}


% We plot the difference between \(T_i\) and \(T_{i-2}\) as the number of steps increases, and compare the periodicity of four tasks alongside paraphrasing. 
% As illustrated in Figures \ref{figs:Other_tasks_div_trend}, even at the paragraph level, LLMs tend to become trapped in specific stable states, exhibiting high 2-periodicity degrees. 
Figure~\ref{figs:task_extensions} shows that even for these varied tasks, LLMs repeatedly converge to stable states, exhibiting pronounced 2-periodicity. 
Table~\ref{table:task_extension} shows the degree of 2-periodicity across these tasks, with values ranging from 0.65 to 0.87. 
This finding reinforces the idea that invertibility fosters the emergence of limit cycles, as the model iterates the transformation and settles into an attractor. 
While paraphrasing is our primary lens, these findings confirm that stable attractor cycles are a broader characteristic of LLM behavior in iterative, invertible mappings.
% Confusion matrices of these tasks are presented in Appendix~\ref{App:tasks}. 



\begin{table}[!h]
    \centering
    \small
    \begin{tabular}{cccccc}
        \toprule
        \textbf{Tasks}& \textbf{Para.} & \textbf{Clar.} &\textbf{Pol.} & \textbf{I/F.} & \textbf{Trans.} \\
        \midrule
        \textbf{$\tau$} &0.80 & 0.83 &  0.86 & 0.65 & 0.87 \\
        \bottomrule
    \end{tabular}
    \caption{Impact of perturbations on periodicity compared to the original during paraphrasing.}
    \label{table:task_extension}
\end{table}

%\begin{figure}[!h]
%    \centering
%    \includegraphics[width=1\linewidth]{article/figures/TaskExtentionv2.pdf}
%    \caption{The periodicity of paraphrasing and other tasks on a paragraph-level English dataset using GPT-4o-mini, where a lower value indicates higher periodicity.} 
%    \label{figs:Other_tasks_Periodicity}
%\end{figure}




%
%Similar to paraphrasing, we iteratively perform these tasks on a paragraph-level dataset.
%We plot the difference between \(T_i\) and \(T_{i-2}\), and compare the 2-periodicity of four tasks %alongside paraphrasing.
%Figure \ref{figs:Other_tasks_div_trend} illustrates the decreasing trend of the four tasks, similar to that of paraphrasing, while Figure \ref{figs:Other_tasks_Periodicity} demonstrates their comparable periodicity to paraphrasing.





\subsection{Alternating Models and Prompts}
% We extend our experimental settings by introducing model and prompt variations in the process of successive paraphrasing.

% \textbf{Paraphrasing with prompt variation.}
% We design four distinct prompts that are used to rephrase text while preserving its original meaning, which can be accessed in (Appendix \ref{App:prompt_var}).
% GPT-4o-mini is then employed for successive paraphrasing, with a prompt randomly selected from the set for each iteration. 
% Although the prompt is changed during successive paraphrasing, it maintains a 2-periodicity, experiencing minimal degradation, as shown in Figure \ref{figs:model_prompt_var}.


% \textbf{Paraphrasing with model variation.}
% We selected a model set consisting of GPT-4o-mini, GPT-4o, Llama3-8B, and Qwen2.5-7B. 
% We randomly chose a different LLM from the set for each iteration. 
% Although each LLM exhibits its own paraphrasing style, contributing to the diversity of paraphrases, the 2-periodicity still persists.
% In other aspects, we calculated the perplexity of \(T_i\) conditioned on \(T_{i-1}\) using Llama3-8B, where both \(T_i\) and \(T_{i-1}\) were generated by different LLMs.
% As shown in Figure \ref{figs:PPL_conditioned_Llama3}, the perplexity of \(T_i\) decreases during successive paraphrasing, despite the fact that it was not generated by the LLM performing the calculation. 
% Both the existing convergence and 2-periodicity indicate the presence of statistically optimal patterns that exist across all LLMs.

One intuitive approach to escape an attractor is to introduce perturbations in the transformation itself. We attempt this by varying both models and prompts during successive paraphrasing. 
For \textbf{prompt variation}, we design four different paraphrasing prompts (refer to Appendix~\ref{App:prompt_var}) and randomly select one at each iteration. Despite regularly switching prompts, the 2-period cycle persists, as shown in Figure~\ref{figs:model_prompt_var}.

\begin{figure}[!hb]
    \centering
    \begin{minipage}{0.45\textwidth}
        \centering
        \includegraphics[width=\linewidth]{article/figures/app_model_prompt_var.pdf}
        \caption{The difference confusion matrices for model variation and prompt variation.}
        \label{figs:model_prompt_var}
    \end{minipage}
    \hfill
    \begin{minipage}{0.45\textwidth}
          \centering
    \includegraphics[width=1\linewidth]{article/figures/model_var_conditioned_llama3-7b.pdf}
    \caption{The perplexity of \(T_i\) conditioned on \(T_{i-1}\) calculated by Llama3-8B. Both \(T_i\) and \(T_{i-1}\) are generated by other LLMs. }    
    \label{figs:PPL_conditioned_Llama3}
    \end{minipage}
    \label{fig:overall}
\end{figure}

Similarly, we introduce \textbf{model variation} by alternating among GPT-4o-mini, GPT-4o, Llama3-8B, and Qwen2.5-7B during successive paraphrasing. 
Although each model brings its own stylistic biases, the fundamental attractor cycle remains intact. 
Interestingly, perplexity computed by a single model (e.g., Llama3-8B) on paraphrases generated by other models still decreases over iterations in Figure~\ref{figs:PPL_conditioned_Llama3}. 
This suggests that the attractor states are not confined to a single model’s parameter space.
Instead, they reflect a more general statistical optimum that multiple LLMs gravitate toward.



From a systems perspective, this findings suggest that randomizing the transformation function $P$ does not inherently break the attractor. 
The system remains in a basin of attraction shared across these varied modeling conditions, implying that the stable cycle is a robust property of the iterative transformation rather than a quirk of any particular prompt or model.


\subsection{Increasing Generation Randomness}
\label{sec:temperature}
% When LLMs become entrenched in what appears to be an optimal state, increasing the temperature has little effect in enabling them to escape this condition.
% We repeated our paraphrasing experiments using GPT-4o-mini with temperature settings of  0.6, 0.9, 1.2 and 1.5.

% As shown in Figure \ref{figs:Temperature}, while increasing the temperature from 0.6 to 1.5, The difference between paraphrases increases too.
% Nevertheless, the 2-periodicity phenomenon still remains. 
% Further elevation of the temperature leads to the generation of nonsensical text.
% In other words, LLMs cannot explore more valid expressions during paraphrasing.


\begin{figure}[!h]
    \centering
    \includegraphics[width=\linewidth]{article/figures/Temperature.pdf}
    \caption{
    The difference between \(T_{15}\) and \(T_i\) generated by GPT-4o-mini. 
    By increasing the temperature, randomness is amplified, causing the differences to grow as well.
    }    
    \label{figs:Temperature}
\end{figure}
Another strategy is introducing more stochasticity in the generation process by increasing the generation temperature. 
Higher temperatures expand the immediate token selection space, potentially allowing trajectories to wander away from the attractor. However, as shown in Figure~\ref{figs:Temperature}, while higher temperatures do increase the difference between successive paraphrases, the system still exhibits a 2-period cycle. 
Further increases in temperature lead only to nonsensical outputs.
This outcome aligns with dynamical systems theory: a small increase in stochasticity may create local perturbations, but if the basin of attraction is strong, the system remains near the limit cycle. 
Excessive stochastic forcing can push the system out of meaningful regions of state space entirely, leading to “chaotic” or nonsensical behavior, rather than discovering a new stable attractor with richer linguistic diversity.



\subsection{Experiments with Complex Prompts}
Previous experiments were conducted using a simple paraphrase prompt, leading to existing limitations. To solve this, we experimented with a more complicated prompt, and the results indicated similar periodicity patterns. This prompt forces LLMs to enhance grammatical and syntactical variety. 
We used this prompt to instruct GPT-4o-mini to successively paraphrase the paragraph-level test set for 15 rounds. The empirical evaluation of periodicity (2-periodicity score) and convergence (PPL) of the successive paraphrasing with the complex prompt is listed below. Both the difference confusion matrix and the prompt are shown in  Appendix~\ref{app:complex_prompt}.

\begin{table}[h]
\centering
\small
\begin{tabular}{ccc}
\toprule
\textbf{Model} & \textbf{Periodicity} & \textbf{Convergence} \\ \midrule
Original & 0.80 & 1.19 \\
Complex  & 0.67 & 1.33 \\ \bottomrule
\end{tabular}
\caption{Periodicity and Convergence Table}
\end{table}

Although the sophisticated prompt alleviated the periodicity and convergence in some degree, the pattern of 2-period cycle remained strong. For context, a periodicity score of 0.67 implies an average edit distance of 0.33 between paraphrases two steps apart, whereas direct paraphrase exhibits an edit distance of 0.68. 


\subsection{Incoporating Local Perturbations}


% To account for perturbations that occur in real-world usage, we simulate human interference during successive paraphrasing.
We introduce local perturbations to mitigate the attractor cycle pattern. 
At the end of each iteration, we edit 5\% of the text by introducing perturbations using three methods: synonym replacement (S.R.), word swapping (W.S.), and random insertion or deletion (I./D.).
% According to Figure \ref{figs:3-periodicity}, synonym replacement results in minimal periodicity degradation, followed by random insertion or deletion. In contrast, word swapping leads to significant degradation.
As shown in Table~\ref{table:huamn_interven}, 
among these interventions, synonym replacement barely affects periodicity, suggesting that minor lexical changes do not move the system out of the attractor’s basin. 
It indicates that except during the first paraphrasing, LLMs primarily perform synonym replacements for words or phrases, as shown in Figure \ref{figs:intro}.
Word swapping, however, causes more significant disruption, lowering periodicity more effectively. 
From a dynamical standpoint, large structural perturbations are needed to shift the system’s state out of a stable cycle. 
Local lexical tweaks do not suffice because the attractor’s pull is strong and preserved at a deeper structural level.
\begin{table}[!h]
    \centering
    \small
    \begin{tabular}{cccc}
        \toprule
            \textbf{w/o Perturb.} & \textbf{S.R.} & \textbf{W.S.} & \textbf{I./D.} \\
        \midrule
             0.77 & 0.73 &  0.62 & 0.66 \\
        % \midrule
        % Difference & - & \textbf{-0.04} & -0.15 & -0.11 \\
        \bottomrule
    \end{tabular}
    \caption{Impact of different types of perturbations on 2-periodicity degrees $\tau$, compared to the original text during paraphrasing.}
    \label{table:huamn_interven}
\end{table}

% The minimal impact of synonym replacement aligns with our observations, as shown in Figure \ref{figs:intro}.
% These paraphrases, which involve synonym replacement, demonstrate differences similar to those observed without any perturbation at odd steps.
% It indicates that LLMs primarily perform synonym replacements for words or phrases, except during the first paraphrasing, which significantly alters the text's structure and language use.
% Although random insertion and deletion may lead to information loss, it does not significantly alter the text structure compared to word swapping, which causes substantial changes.



\subsection{Paraphrasing with History Paraphrases}


% We also examine the impact of providing historical context during paraphrasing. 
We consider a scenario where the transformation $\hat{P}$ depends on both $T_i$ and $T_{i-1}$.
This added historical context can alter the equilibrium states. 
In a scenario where we paraphrase \(T_{i}\) based on the reference \(T_{i-1}\), it is essential that \(T_{i+1}\) differs from both \(T_{i}\) and \(T_{i-1}\). This function can be expressed as: $T_{i+1} = \hat{P}(T_{i}, T_{i-1})$.
In this context, \(P_{i-1}\) emerges as a strong candidate for paraphrasing \(P(T_{i+1}, T_{i})\), as it aligns with the distribution of LLMs while maintaining difference from \(\hat{P}(T_{i+1}, T_{i})\), satisfying the task requirement.
As a result, this more complex cycle still represents a stable attractor, albeit of higher order, as shown in Figure~\ref{figs:3-periodicity}.
% Consequently, the periodicity in this scenario will be three.
% We also examine the impact of providing historical context during paraphrasing.
% In a scenario where we paraphrase \(T_{i}\) based on the reference \(T_{i-1}\), it is essential that \(T_{i+1}\) differs from both \(T_{i}\) and \(T_{i-1}\).
% This added historical context can alter the equilibrium states. In fact, incorporating the immediate history leads the system to settle into a 3-period cycle rather than a 2-period cycle, as shown in Figure~\ref{figs:3-periodicity}. 
% This more complex cycle still represents a stable attractor, albeit of higher order.
% It suggests that even with more elaborate conditioning, the model is not freed from limit cycles; instead, it can become locked into more intricate periodic behaviors.



\label{sec:history}
\begin{figure}[!h]
    \centering
    \subfigure{\includegraphics[width=0.6\linewidth]{article/figures/3-periodicity.pdf}} 
    %\includegraphics[width=\linewidth]{article/figures/perturbation_periodicity.pdf}
    \caption{When adding historical paraphrases, LLMs exhibit 3-periodicity in the paraphrasing task.}    
    \label{figs:3-periodicity}
\end{figure}

% To verify this, we utilize \(\hat{P}\) during successive paraphrasing. As shown in Figure \ref{figs:3-periodicity}, it indeed performs 3-periodicity, highlighting the limited expression capabilities of LLMs again.

\subsection{Sample Selection Strategies}

\label{sec:sampling_strat}
%\begin{figure}[h]
%    \centering
%    \includegraphics[width=1\linewidth]{article/figures/sim.pdf}
%    \caption{The similarity between paraphrases and the original texts increases during the paraphrasing process.} 
%    \label{figs:sim}
%\end{figure}



% Finally, we investigate methods to steer the system away from stable attractors at the least cost of generation quality. 
% Increasing temperature alone did not help, but controlling perplexity and selecting among multiple sampled paraphrases at each iteration shows promise. 
% By choosing a paraphrase with a higher perplexity, we partially weaken the attractor’s grip, reducing periodicity. 
% Rather than always taking the one with the highest or lowest perplexity. 


% However, as shown in Figures 8 and 14, this may come at the cost of semantic fidelity.

We investigate methods to steer the system away from stable attractors at the least cost of generation quality. 
Given the correlation between periodicity and perplexity, it is intuitive to mitigate this issue by increasing perplexity while maintaining generation quality.
To achieve this, we can randomly sample multiple paraphrases at each iteration and select the one based on perplexity.
We design three types of strategies: selecting the paraphrase with the maximum or minimum perplexity or randomly choosing one at each iteration.
Figures \ref{figs:strategy} illustrate that selecting a higher perplexity can reduce periodicity.
However, such diversity comes at the cost of semantic equivalence (Appendix~\ref{app:sample_selection}).
Considering both periodicity and meaning preservation, we recommend the random strategy, which effectively reduces periodicity while incurring minimal information loss compared to selecting the option with the lowest perplexity.


\begin{figure}[h]
    \centering
    \includegraphics[width=1\linewidth]{article/figures/strategy.pdf}
    \caption{The periodicity of three strategies using different LLMs. } 
    \label{figs:strategy}
\end{figure}

\subsection{The Benefit of Mitigating Small-Size Cycles}
As a data augmentation method, paraphrase diversity should impact downstream tasks.
To figure out this, we conducted an experiment on domain classification using successive paraphrasing for data augmentation. We selected a commonly used dataset, AG News~\cite{zhang2016characterlevelconvolutionalnetworkstext} as the testbed, and trained BERT-based models using different data. For data augmentation, we conducted 5 rounds of successive paraphrasing under two different settings—min-strategy (Min.~Strat.) and max-strategy (Max.~Strat.), as detailed in Section~\ref{sec:sampling_strat}. Below are our results:

\begin{table}[h]
\centering
\small
\begin{tabular}{cccc}
\toprule
\textbf{Metric} & \textbf{w/o Aug.} & \textbf{Min. Strat.} & \textbf{ Max. Strat.} \\ \midrule
    Accuracy↑                       & 83.10        & 83.80        & 84.41        \\ 
2-periodicity↓    &       -     & 3.38          & 4.15          \\ \bottomrule
\end{tabular}
\caption{Performance on AG News across different data augmentation strategies.}
\end{table}

The table shows that our max-strategy, which more effectively mitigates the 2-periodicity cycle, yields more diverse paraphrases for data augmentation and therefore achieves higher classification accuracy (84.4\%) compared to both no augmentation and the min-strategy. We will add these findings to our revision to illustrate the beneficial impact of mitigating small-size cycles on downstream tasks.
% \include{commands/LossFunction}
% \include{commands/Pupil}
% \include{commands/TannerEtAl}
% \include{commands/Util}
\newcommand{\gExpertName}[1][\FlockAgentIndex]{{\ensuremath{\va_{#1}}}}
\newcommand{\gExpertInput}[1][]{{\FlockPosMatrix#1, \FlockVelMatrix#1}}
\newcommand{\gGeneralizationGap}{\cR_{\GeneralizationSampleName,\LossFunctionName}}
\newcommand{\gConvInput}[1][\PupilLayerIndex]{{\PupilHistoryMatrix_{\FlockAgentIndex,#1}}}
\newcommand{\gConvInputAsMLP}[1][\PupilLayerIndex]{{\PupilHistoryMatrix_{\FlockAgentIndex,#1}^{\mathrm{MLP}}}}
\newcommand{\gConvInputWithBias}[1][\PupilLayerIndex]{{\widetilde{\gConvInput[#1]}}}
\newcommand{\gDAggerDatasetDatum}[1][]{{\gExpertInput[#1], \set{\PupilHistoryMatrix_\FlockAgentIndex#1}_{\FlockAgentIndex = 1}^\FlockAgentCount}}
\newcommand{\gConvChannelsIn}[1][\PupilLayerIndex]{{C_{#1}}}
\newcommand{\gConvFeaturesIn}[1][\PupilLayerIndex]{{F_{#1}}}

\title{
Learning Decentralized Swarms Using\\Rotation Equivariant Graph Neural Networks
}
% \subtitle{Do you have a subtitle?\\ If so, write it here}

%\titlerunning{Short form of title}        % if too long for running head

\author{
  Taos Transue
  \and
  Bao Wang
}

%\authorrunning{Short form of author list} % if too long for running head

\institute{
    Taos Transue \and Bao Wang
    \at
    Department of Mathematics, University of Utah, 155 South 1400 East, Salt Lake City, Utah 84112, USA.
    \and
    Bao Wang
    \at
    Scientific Computing and Imaging Institute, University of Utah, 72 S Central Campus Drive, Salt Lake City, Utah 84112, USA.\\
    Corresponding author \email{taos.j.transue@gmail.com} \\
    \email{wangbaonj@gmail.com}
}


\date{Received: date / Accepted: date}
% The correct dates will be entered by the editor


\maketitle

\begin{abstract}
The orchestration of agents to optimize a collective objective without centralized control is challenging yet crucial for applications such as controlling autonomous fleets, and surveillance and reconnaissance using sensor networks. Decentralized controller design has been inspired by self-organization found in nature, with a prominent source of inspiration being flocking; however, decentralized controllers struggle to maintain flock cohesion. The graph neural network (GNN) architecture has emerged as an indispensable machine learning tool for developing decentralized controllers capable of maintaining flock cohesion, but they fail to exploit the symmetries present in flocking dynamics, hindering their generalizability. We enforce rotation equivariance and translation invariance symmetries in decentralized flocking GNN controllers and achieve comparable flocking control with 70\% less training data and 75\% fewer trainable weights than existing GNN controllers without these symmetries enforced. We also show that our symmetry-aware controller generalizes better than existing GNN controllers. Code and animations are available at \url{github.com/Utah-Math-Data-Science/Equivariant-Decentralized-Controllers}. 
\keywords{machine learning \and graph neural networks \and flocking \and equivariance \and computational learning theory}
% \PACS{PACS code1 \and PACS code2 \and more}
\subclass{MSC 68T05 \and MSC 68Q32 \and MSC 68T42}
\end{abstract}



\section{Introduction}
Orchestrating a system of agents to optimize a collective objective is crucial in autonomous vehicle control \cite{perea2009extension}, surveillance and reconnaissance with sensor networks \cite{akyildiz2002survey}, and beyond \cite{hua2024towards}. Designing controllers for these systems is often inspired by self-organization in nature. A prominent source of inspiration is flocking: myxobacteria travel in clusters to increase the concentration of an enzyme they use to stun their prey, making predation more effective for each individual \cite{camazineSelforganizationBiologicalSystems2003}; pelicans follow a leader pelican in a vee formation, taking advantage of special aerodynamics to reduce the energy expenditure of flight \cite{weimerskirchEnergySavingFlight2001}; and schools of fish coordinate their movement in several formations, minimizing each individual's risk of predation \cite{parrishSelfOrganizedFishSchools2002}. 





When designing a controller to optimize a collective objective, there are three pertinent questions: (1) What objective should an agent optimize individually? (2) What information about other agents or the environment does an agent need to act? (3) How do the actions of each agent culminate in optimizing the collective objective?
% edit(Taos): On using past tense
% This paragraph introduces Reynolds' initial description of flocking -- each agent optimizes Alignment, Cohesion, and Separation.
% The definitions Reynolds gave of these three objectives were qualitative, and his description of the flock's behavior was also qualitative.
% After Reynolds, the definitions of these objectives and of flocking behavior were quantified by asymptotic flocking and separation.
% The use of past tense conveys that researchers considered qualitative definitions of flocking in the past, but consider quantitative definitions in the present.
% While the qualitative definitions are useful for intuition, quantitative definitions should be used to evaluate the flock-like motion of a controller.
%
%
% Reference on verb tense in scientific manuscripts: https://www.unlv.edu/sites/default/files/page_files/27/GradCollege-VerbTenseScientificManuscripts.pdf
% In this reference, they state ``when referring to a previous study with results that are still relevant, use the present perfect tense [(e.g., `have shown`)]''.
% I am trying to convey that the quantitative definitions supersede the qualitative definitions when evaluating a flocking controller, because quantitative definitions allow us to answer question (3).
% The qualitative definitions are not relevant for answering the question (3), and so I use simple past tense (e.g., `was shown`).
%
% Nature papers with a mixed of past and present tense:
% - https://www.nature.com/articles/s41586-025-08598-8: ``We designed[...]``
% - https://www.nature.com/articles/s41586-025-08600-3: ``we carried out[...]''
% In Reynolds paper, ``Traditional hand-drawn cel animation was produced with a medium that was completely inert.''
The first two questions were studied for flocking controllers in \cite{reynoldsFlocksHerdsSchools1987}, where Reynolds sought to reproduce bird flocking and created one of the earliest simulations of flock-like motion. His simulation had each agent (so-called ``bird-oid'' or ``boid'') balance the following three objectives:
\begin{itemize}
    \item \textbf{Alignment:} Each agent should steer toward its neighbors' average direction.
    \item \textbf{Cohesion:} Each agent should move toward the average position of its neighbors.
    \item \textbf{Separation:} Each agent should move away from its neighbors if they are too close.
\end{itemize}
To optimize these objectives, each agent used the position and velocity of all other agents. With qualitative answers to questions (1) and (2), Reynolds programmed agents that ``participate in an acceptable approximation of flock-like motion.''
% edit(Taos): The text in the quote cannot be changed, because then it is not a quote. Also, "flock-like motion" is what needs to quantified to answer question (3).
Since flock-like motion was not yet quantified, question (3) remained unanswered.
% edit(Taos): Keep past tense, not present tense, because question (3) was quantified before we wrote this paper.
Later, researchers quantified flock-like motion with
% edit(Taos): Incorrect usage of c.f. This Latin abbreviation means "compare," but we are not comparing anything here. We are defining terms for the first time.
% edit(Taos): See https://writingcenter.unc.edu/tips-and-tools/latin-terms-and-abbreviations/ for more information.
\textit{asymptotic flocking} and \textit{separation} (Definitions~\ref{def:asymptotic-flocking} and \ref{def:agent-separation}).
Now, question (3) is answered by arguing that a given controller induces asymptotic flocking.


Notable flocking controllers include those designed in \cite{tannerStableFlockingMobile2003a,cuckerEmergentBehaviorFlocks2007}.
%, and Cucker and Smale \cite{cuckerEmergentBehaviorFlocks2007}.
%proposed by Tanner et al. \cite{tannerStableFlockingMobile2003a}, and Cucker and Smale \cite{cuckerEmergentBehaviorFlocks2007}. 
In \cite{tannerStableFlockingMobile2003a}, Tanner et al. set the agents' accelerations equal to a term that encourages alignment plus the negative gradient of a potential function for cohesion and separation. If the graph representing what agents can exchange information is fully connected, asymptotic flocking and separation are guaranteed. 
%In fact, they are guaranteed when the communication graph is only connected.
%; this relaxation yields a decentralized controller, which we will discuss in detail later. After Tanner et al., 
In \cite{cuckerEmergentBehaviorFlocks2007}, Cucker and Smale define the agents' accelerations in a way that aligns their velocities while preserving the momentum of the flock. They achieve this with an ``influence function,'' which gauges an agent's influence on another as a function of their distance. The benefit of momentum conservation is that the limiting velocity of the flock can be computed from the flock's initial conditions. Cucker and Smale's controller guarantees asymptotic flocking, and guarantees separation with singular influence functions \cite{cuckerAvoidingCollisionsFlocks2010,minakowskiSingularCuckerSmaleDynamics2019}.
%asymptotic flocking is guaranteed, and separation is guaranteed with singular influence functions \cite{cuckerAvoidingCollisionsFlocks2010,minakowskiSingularCuckerSmaleDynamics2019}.
%
%
Further developments in flocking controllers can be viewed as adding constraints when optimizing a collective flocking objective. 
%For example, 
%unlike Tanner et al. \cite{tannerStableFlockingMobile2003a}, the Cucker-Smale controller \cite{cuckerEmergentBehaviorFlocks2007} can optimize the collective objective when a desired limiting velocity is specified by the initial velocities of the agents. 
Developments that add constraints to Tanner et al.'s controller include leader following -- requiring the flock to follow a leader agent \cite{guLeaderFollowerFlocking2009}, and segregation -- flocking while having agents cluster into preassigned groups \cite{santosSegregationMultipleHeterogeneous2014}. Developments that add constraints to the Cucker-Smale controller include encouraging the agents to be a fixed distance $R>0$ from each other using a PID controller \cite{parkCuckerSmaleFlockingInterParticle2010}, and pattern formation via control terms added to the agents' acceleration \cite{ohFlockingBehaviorStochastic2021,choiCollisionlessSingularCuckerSmale2019,choiControlledPatternFormation2022}.



All these controllers require agents to exchange information with all others agents, making them {\bf centralized controllers}. 
%-- the communication graph must be fully connected. Such controllers are centralized. 
As the number of agents 
%that a flocking controller is required to orchestrate 
grows, an increasingly prominent alternative is {\bf decentralized controllers} where agent communication is represented by a sparse graph.
%-- agents communicate via a sparse graph. 
%that is, flocking controllers must be capable of operating on sparse communication graphs, a.k.a. overlay networks.
%Decentralization may be enforced as centralization requires prohibitive communication costs. 
In a flock of $\FlockAgentCount$ agents, a centralized controller requires each agent to 
%exchange information 
communicate with %$M - 1$ 
$\FlockAgentCount-1$ agents, resulting in quadratic ($\cO\pn{\FlockAgentCount^2}$) communication overhead growth. 
%so the communication overhead grows quadratically -- \BW{$\cO\pn{N^2}$}. 
%$\cO\pn{M^2}$. 
%A more subtle benefit of decentralization is that 
Moreover, a centralized controller imposes a bound on $\FlockAgentCount$ when separation is required, and each agent has a maximum communication distance. Maximum communication distance constraints arise in wireless communications where the power required to send information over a distance $r$ is proportional to $r^p$ for some $p \in \hlpn{2, 4}$ \cite{akyildiz2002survey}. A maximum communication distance $\FlockAgentCommRadius$ and a minimum separation of $\FlockAgentMinDist$ bound $\FlockAgentCount$ because only a finite number of disjoint balls with diameter $2\FlockAgentMinDist$ can be packed into a ball whose diameter is the maximum diameter of the flock ($2\FlockAgentCommRadius$).
{\bf Decentralization is crucial for building %practical and 
scalable flocking controllers}, but compared to centralization, decentralization requires additional assumptions to guarantee asymptotic flocking and may not be compatible with other guarantees (e.g., separation).



Examples of decentralized controllers include those proposed by Tanner et al. \cite{tannerStableFlockingMobile2003a,tannerStableFlockingMobile2003}.
The controller proposed in \cite{tannerStableFlockingMobile2003a} uses a fixed connected communication graph to guarantee asymptotic flocking,\footnote{In practice, the communication graph can be constucted using the idea of virtual nodes; see e.g., \cite{9850408,doi:10.1137/21M1465081}.} but it does not guarantee separation because nonadjacent agents cannot communicate. Though flock cohesion is guaranteed, adjacent agents may need to communicate over arbitrarily large distances while the controller is active. Instead of a fixed agent neighborhood, the neighborhood of an agent in \cite{tannerStableFlockingMobile2003} changes over time by only containing other agents currently within the agent's communication radius $\FlockAgentCommRadius$. In this case, the authors guarantee asymptotic flocking and separation assuming that the communication graph is connected for all time; \figref{fig:tanner:decentralized:dynamic:failure} shows that this assumption is invalid for some initial conditions. In \cite{guLeaderFollowerFlocking2009}, the authors propose a decentralized flocking controller with time-dependent agent neighbors and a leader following constraint, but they only test the controller with a fully connected communication graph.
Without a fixed communication graph, decentralized flocking controllers struggle to maintain communication graph connectivity and flock cohesion -- a requirement for effectively optimizing the collective objective.




As suggested by question (2), a plausible approach to improve decentralized controllers is providing more information to each agent.
All the controllers we described, centralized or decentralized, only use information from the current time. %, and past information is discarded.
Tolstaya et al. \cite{tolstayaLearningDecentralizedControllers2017b} apply machine learning (ML) to leverage current and past information.
\figref{fig:tanner:decentralized:dynamic:failure} shows that their ML controller can achieve asymptotic flocking for flock initial conditions where the controller in \cite{tannerStableFlockingMobile2003} fails (see \figref{fig:tanner:decentralized:dynamic:failure}).
The ML controller works by having each agent retain a summary of the information it receives during the current time step and, during the next time step, send that summary to its neighbors.
In addition, each agent retains a summary of the summaries it receives to send to its neighbors later, and this recursion continues up to $\PupilHistoryCount - 1$ time steps in the past. This process gives each agent access to information $\PupilHistoryCount$ hops away in the communication graph.
%(although only the information from the first hop is up-to-date). 
In training, the ML controller learns to utilize patterns in how the information exchanged in the network changes over time.
%
%
Though Tolstaya et al. \cite{tolstayaLearningDecentralizedControllers2017b} have found a way to provide additional information to each agent, their ML controller processes the information suboptimally. The flocking controllers in \cite{tannerStableFlockingMobile2003a,cuckerEmergentBehaviorFlocks2007} exhibit crucial symmetries, e.g., rotation equivariance, that have been ignored. 
%in the ML controller design.
Symmetry-aware ML models have demonstrated remarkable advantages in physical congruence and data efficiency 
%, and generalizability
%, and robustness 
\cite{cohen2017steerable,fuchs2020se,wang2024rethinking,baker2024an}. 
We aim to bridge this gap in this paper. %We address this in our contributions.
%\BW{[ROTATION IS EASY TO HANDLE]}

\section{Our contributions}
% edit(Taos): Springer says the introduction should not include subheadings.

We improve decentralized ML flocking controllers by enforcing rotation equivariance into time-delayed aggregation GNNs (\PupilTDAGNN{}s) \cite{tolstayaLearningDecentralizedControllers2017b}, restoring the symmetries 
%the symmetries exhibited by 
of non-ML flocking controllers. %In this paper, 
%A summary of our key contributions is the following:
We summarize our key contributions as follows:
\begin{itemize}
\item We present a simple yet efficient rotation equivariant convolutional neural network (CNN) and integrate it into \PupilTDAGNN{} for learning decentralized flocking.

\item We justify the theoretical advantages of rotation equivariance in learning decentralized flocking by demonstrating better generalization compared to non-equivariant controllers.

\item We demonstrate the advantages of our new ML controller in decentralized flocking, leader following, and obstacle avoidance. 
%Our rotation equivariant ML controller performs comparably compared to non-equivariant controllers with 70\% less training data and 75\% fewer trainable weights.
\end{itemize}

\subsection{Organization}
% edit(Taos): Springer says the introduction should not include subheadings.

We organize this paper as follows. In \secref{sec:bg}, we review some related works on flocking controllers. In \secref{sec:model-proposed}, we present our rotation equivariant GNN for decentralized flocking. In \secref{sec:main:theory} and \secref{sec:experiments}, we analyze the generalization advantages of the rotation equivariant GNN for learning decentralized flocking and verify its effectiveness. Technical proofs and additional details are provided in the appendix. 

\section{Background}\label{sec:bg}
\subsection{Flock representation and definitions}
\label{sec:flocking-control-description}
Let $\FlockPosMatrix=[\FlockAgentPos_1,\;\ldots,\;\FlockAgentPos_\FlockAgentCount]\in\bR^{2\times \FlockAgentCount}$ be the position vectors of \FlockAgentCount{} agents, $\FlockVelMatrix$ be their velocities, and $\FlockAgentRelativePos = \FlockAgentPos_\FlockAgentIndex - \FlockAgentPos_\FlockAgentIndexTwo$ be the relative position. 
We formulate the alignment and cohesion rules for asymptotic flocking as follows:

\begin{definition}[Asymptotic flocking \cite{choiEmergentDynamicsCuckerSmale2016}]
\label{def:asymptotic-flocking}
A flock of \FlockAgentCount{} agents flocks asymptotically if and only if the following two conditions are satisfied: (1)  {\bf Alignment:} $\lim\limits_{t \to \infty} \max\limits_{\FlockAgentIndex,\FlockAgentIndexTwo} \norm{\FlockAgentRelativeVel\pn{t}} = 0$, and (2) {\bf Cohesion:} $\sup\limits_{0 \leq t < \infty,\FlockAgentIndex,\FlockAgentIndexTwo} \norm{\FlockAgentRelativePos\pn{t}} < \infty$.
\iffalse
\begin{enumerate}
\item {\bf Alignment:} $\lim\limits_{t \to \infty} \max\limits_{\FlockAgentIndex,\FlockAgentIndexTwo} \norm{\FlockAgentRelativeVel\pn{t}} = 0$,

\item {\bf Cohesion:} $\sup\limits_{0 \leq t < \infty,\FlockAgentIndex,\FlockAgentIndexTwo} \norm{\FlockAgentRelativePos\pn{t}} < \infty$.
\end{enumerate}
\fi
\end{definition}

Asymptotic flocking is usually the primary objective of flocking controllers.
Controllers can also make guarantees concerning how their objectives are met, such as agent separation, which is defined as follows:

\begin{definition}[Agent $\FlockAgentMinDist$-separation]\label{def:agent-separation}
Let $\FlockAgentMinDist \geq 0$. Agents are separated if for all time $t$, $\norm{\FlockAgentRelativePos\pn{t}} \geq \FlockAgentMinDist$.
\end{definition}

Other objectives or guarantees beyond those mentioned include segregation \cite{santosSegregationMultipleHeterogeneous2014}, where agents cluster into preassigned groups, and pattern formation \cite{ohFlockingBehaviorStochastic2021,choiCollisionlessSingularCuckerSmale2019,choiControlledPatternFormation2022}.


The acceleration of the $\FlockAgentIndex$-th agent is $\FlockAgentAccel_\FlockAgentIndex = \gExpertName\pn{\gExpertInput[\pn{t}]}$ where $\gExpertName$ is the flocking controller.
The design of $\gExpertName$ ensures the flocking controller's objectives are met while maintaining its guarantees.
The two main classes of flocking controllers -- centralized and decentralized -- are separated by whether they use the entire flock state $\pn{\gExpertInput}$ or only a subset of the columns of $\FlockPosMatrix$ and $\FlockVelMatrix$ based on what agents can communicate.



\subsection{Centralized flocking}\label{sec:bg:centralized:math}
We recap on the centralized flocking controller from \cite{tannerStableFlockingMobile2003a}, which defines an agent's acceleration as
\begin{equation}\label{eq:tanner:centralized}
\begin{aligned}
\FlockAgentAccel_\FlockAgentIndex = \gExpertName\pn{\gExpertInput} := -\sum_{\FlockAgentIndexTwo = 1}^\FlockAgentCount \FlockAgentRelativeVel - \sum_{\FlockAgentIndexTwo = 1}^\FlockAgentCount \nabla \TannerEtAlPotential\pn{\norm{\FlockAgentRelativePos}},
\end{aligned}
\end{equation}
where $\TannerEtAlPotential:\pn{0, \infty} \to \bR$ is a potential function such that $\lim_{r \to \infty} \TannerEtAlPotential\pn{r} = \infty$ and $r_* > 0$ is $\TannerEtAlPotential$'s unique minimizer representing the desired distance between two agents.
The first summation aligns the agents' velocities, and the second summation uses $\TannerEtAlPotential$ to control flock cohesion and agent separation.
The potential function for $\TannerEtAlPotential$ used in \cite{tannerStableFlockingMobile2003} is
%\begin{align*}
$\TannerEtAlPotential\pn{r} = 1/r^2 + \ln\pn{r^2}$,
%\end{align*}
which encourages agents to be a distance $r_* = 1$  from each other for cohesion and is singular at $r = 0$ for separation. The function $\TannerEtAlPotential$ can be changed to induce a variety of flock behaviors. For example, organizing leader-following \cite{guLeaderFollowerFlocking2009} and segregating the agents into clusters \cite{santosSegregationMultipleHeterogeneous2014}.



\subsection{Decentralized flocking}\label{sec:bg:decentralized:math}
We briefly review the decentralized flocking controllers 
%relevant to this work 
from \cite{tannerStableFlockingMobile2003a,tannerStableFlockingMobile2003}.
%, including two decentralized versions of the centralized flocking controller in \eqref{eq:tanner:centralized}. 
The first decentralized controller \cite{tannerStableFlockingMobile2003a} operates on a fixed sparse communication graph and defines an agent's acceleration as
$$
\begin{aligned}
\FlockAgentAccel_\FlockAgentIndex = \gExpertName\pn{\gExpertInput} := -\sum_{\FlockAgentIndexTwo \in \FlockAgentNeighborhood_\FlockAgentIndex} \FlockAgentRelativeVel - \sum_{\FlockAgentIndexTwo \in \FlockAgentNeighborhood_\FlockAgentIndex} \nabla \TannerEtAlPotential\pn{\norm{\FlockAgentRelativePos}},
\end{aligned}
$$
where $\FlockAgentNeighborhood_\FlockAgentIndex$ is the neighborhood of agent $\FlockAgentIndex$. This controller guarantees asymptotic flocking but does not guarantee separation because nonadjacent agents cannot communicate to avoid collision. Another drawback of this controller is that any adjacent agents must be able to communicate regardless of their spatial distance.
% edit(Taos): Use ``distance'' because ``proximity'' implies closeness, and we are discussing agents that are far apart from each other.





%Instead of a fixed neighborhood, 
The second decentralized controller \cite{tannerStableFlockingMobile2003} defines the neighborhood of an agent to only include agents within %its communication 
radius $\FlockAgentCommRadius$, i.e., $\FlockAgentNeighborhood_\FlockAgentIndex\pn{t} = \set{\FlockAgentIndexTwo : \FlockAgentIndexTwo \neq \FlockAgentIndex,\ \norm{\FlockAgentRelativePos} \leq \FlockAgentCommRadius}$. The agent's acceleration is defined as
$$
\begin{aligned}
\FlockAgentAccel_\FlockAgentIndex = \gExpertName\pn{\gExpertInput} := -\sum_{\FlockAgentIndexTwo \in \FlockAgentNeighborhood_\FlockAgentIndex\pn{t}} \FlockAgentRelativeVel - \sum_{\FlockAgentIndexTwo \in \FlockAgentNeighborhood_\FlockAgentIndex\pn{t}} \nabla \TannerEtAlPotential\pn{\norm{\FlockAgentRelativePos}}.
\end{aligned}
$$
%Unlike the first decentralized controller, 
This controller guarantees asymptotic flocking and separation.
However, it assumes the communication graph is connected for all time, but \Figref{fig:tanner:decentralized:dynamic:failure}(a,b,c) show that this assumption may be invalid.
%\triFigref{fig:tanner:decentralized:dynamic:failure:time0}{fig:tanner:decentralized:dynamic:failure:time27}{fig:tanner:decentralized:dynamic:failure:time53} shows an initial condition of a flock where this assumption does not hold.


\begin{figure}[!ht]
\centering
\begin{subfigure}[t]{.22\textwidth}
\centering
\includegraphics[width=\textwidth]{figsMMDecTannerDynamicNeighborFailuretime0.pdf}
\caption{\footnotesize $t_0$}
\label{fig:tanner:decentralized:dynamic:failure:time0}
\end{subfigure}
\hspace{1em}
\begin{subfigure}[t]{.22\textwidth}
\centering
\includegraphics[width=\textwidth]{figsMMDecTannerDynamicNeighborFailuretime27.pdf}
\caption{
% edit(Taos): 27 is not the time, it is the time step index.
\footnotesize $t_{27}$
}
\label{fig:tanner:decentralized:dynamic:failure:time27}
\end{subfigure}
\hspace{1em}
\begin{subfigure}[t]{.22\textwidth}
\centering
\includegraphics[width=\textwidth]{figsMMDecTannerDynamicNeighborFailuretime53.pdf}
\caption{\footnotesize $t_{53}$}
\label{fig:tanner:decentralized:dynamic:failure:time53}
\end{subfigure} \\
%\vspace{1em}
\begin{subfigure}[t]{.22\textwidth}
\centering
\includegraphics[width=\textwidth]{figsMMDecTolstayaDynamicNeighborhoodSuccesstime0.pdf}
\caption{\footnotesize $t_0$}\label{fig:tolstaya:dynamic:success:time0}
\end{subfigure}
\hspace{1em}
\begin{subfigure}[t]{.22\textwidth}
\centering
\includegraphics[width=\textwidth]{figsMMDecTolstayaDynamicNeighborhoodSuccesstime27.pdf}
\caption{\footnotesize $t_{27}$}
\label{fig:tolstaya:dynamic:success:time27}
\end{subfigure}
\hspace{1em}
\begin{subfigure}[t]{.22\textwidth}
\centering
\includegraphics[width=\textwidth]{figsMMDecTolstayaDynamicNeighborhoodSuccesstime53.pdf}
\caption{\footnotesize $t_{53}$}
\label{fig:tolstaya:dynamic:success:time53}
\end{subfigure}%\vspace{-0.3cm}
\caption{%\small 
Snapshots at times $t_\DAggerSimulationTimeStepIndex$ of a flock of agents (orange dots) with velocities (blue arrows) and the flock's communication graph (light blue edges). When controlled by the time-dependent neighborhood decentralized flocking controller from Tanner et al. \cite{tannerStableFlockingMobile2003} (top row), the communication graph loses connectivity from time $t_{27}$ onward. In contrast, the ML-based decentralized flocking controller from Tolstaya et al. \cite{tolstayaLearningDecentralizedControllers2017b} (bottom row), not trained on these agent initial conditions, successfully maintains communication graph connectivity and achieves asymptotic flocking.
%\BW{a. At time step $t_0$, agents' initial positions and velocities form a connected communication graph. b. At time step $t_{27}$, the communication graph is disconnected. c. At time step $t_{53}$, a connected component of the communication graph becomes disconnected. d. At time step $t_0$, agents' initial positions and velocities form a connected communication graph. e. At time step $t_{27}$, the communication graph is still connected. f. At the time step $t_{53}$, the communication graph is still connected, and the agents' velocities are aligning to the upper left.}
}%\vspace{-0.3cm}
\label{fig:tanner:decentralized:dynamic:failure}
\end{figure}


\subsection{ML-based decentralized flocking}\label{sec:bg:decentralized:ml}
\subsubsection{Time-delayed aggregation graph neural networks}
We review \PupilTDAGNN{} from \cite{tolstayaLearningDecentralizedControllers2017b}. 
%This is the ML controller that we modify to enforce rotational equivariance. 
Let $\FlockAgentNeighborhood_\FlockAgentIndex\pn{t} = \set{\FlockAgentIndexTwo: \FlockAgentIndexTwo \neq \FlockAgentIndex,\ \norm{\FlockAgentRelativePos} \leq \FlockAgentCommRadius}$ be the neighborhood of agent $\FlockAgentIndex$ at time $t$.
Define a function $\PupilFeatureFunc: \bR^2 \times \bR^2 \to \bR^{\gConvChannelsIn[1]}$ and the recurrence relation at time $t_\DAggerSimulationTimeStepIndex$ as
\begin{equation}%\small
\begin{aligned}
 \PupilHistoryVector_\FlockAgentIndex^1\pn{t_\DAggerSimulationTimeStepIndex} & = \sum_{\FlockAgentIndexTwo \in \FlockAgentNeighborhood_\FlockAgentIndex\pn{t_\DAggerSimulationTimeStepIndex}} \psi\pn{\FlockAgentRelativePos\pn{t_\DAggerSimulationTimeStepIndex}, \FlockAgentRelativeVel\pn{t_\DAggerSimulationTimeStepIndex}}, 
\end{aligned}
\label{eq:tolstaya:node-features:one-hop}
\end{equation}
\begin{equation}%\small
\begin{aligned}
\PupilHistoryVector_\FlockAgentIndex^\PupilHistoryIndex\pn{t_\DAggerSimulationTimeStepIndex} & = \frac{1}{\abs{\FlockAgentNeighborhood_\FlockAgentIndex\pn{t_\DAggerSimulationTimeStepIndex}}}\sum_{\FlockAgentIndexTwo \in \FlockAgentNeighborhood_\FlockAgentIndex\pn{t_\DAggerSimulationTimeStepIndex}} \PupilHistoryVector_\FlockAgentIndexTwo^{\PupilHistoryIndex - 1}\pn{t_{\DAggerSimulationTimeStepIndex - 1}}.
\end{aligned}
\label{eq:tolstaya:node-features:k-hop}
\end{equation}
\iffalse
\begin{align}\small
    \PupilHistoryVector_\FlockAgentIndex^1\pn{t_\DAggerSimulationTimeStepIndex} & = \sum_{\FlockAgentIndexTwo \in \FlockAgentNeighborhood_\FlockAgentIndex\pn{t_\DAggerSimulationTimeStepIndex}} \psi\pn{\FlockAgentRelativePos\pn{t_\DAggerSimulationTimeStepIndex}, \FlockAgentRelativeVel\pn{t_\DAggerSimulationTimeStepIndex}}, \label{eq:tolstaya:node-features:one-hop} \\
    \PupilHistoryVector_\FlockAgentIndex^\PupilHistoryIndex\pn{t_\DAggerSimulationTimeStepIndex} & = \frac{1}{\abs{\FlockAgentNeighborhood_\FlockAgentIndex\pn{t_\DAggerSimulationTimeStepIndex}}}\sum_{\FlockAgentIndexTwo \in \FlockAgentNeighborhood_\FlockAgentIndex\pn{t_\DAggerSimulationTimeStepIndex}} \PupilHistoryVector_\FlockAgentIndexTwo^{\PupilHistoryIndex - 1}\pn{t_{\DAggerSimulationTimeStepIndex - 1}}. \label{eq:tolstaya:node-features:k-hop}
\end{align}\fi
With $\PupilFeatureFunc\pn{\FlockAgentRelativePos\pn{t_\DAggerSimulationTimeStepIndex}, \FlockAgentRelativeVel\pn{t_\DAggerSimulationTimeStepIndex}}$ representing the message from agent $\FlockAgentIndexTwo$ to agent $\FlockAgentIndex$ at time $t_\DAggerSimulationTimeStepIndex$, $\PupilHistoryVector_\FlockAgentIndex^1\pn{t_\DAggerSimulationTimeStepIndex}$ summarizes the messages within a one-hop neighborhood of agent $\FlockAgentIndex$, and $\PupilHistoryVector_\FlockAgentIndex^2\pn{t_\DAggerSimulationTimeStepIndex}$ summarizes the one-hop summaries from the previous time step within a one-hop neighborhood. Let $\PupilHistoryMatrix_\FlockAgentIndex\pn{t_\DAggerSimulationTimeStepIndex} = [ \PupilHistoryVector_\FlockAgentIndex^1\pn{t_\DAggerSimulationTimeStepIndex},\ldots,\PupilHistoryVector_\FlockAgentIndex^\PupilHistoryCount\pn{t_\DAggerSimulationTimeStepIndex}]$.
%\begin{bmatrix}
%    \PupilHistoryVector_\FlockAgentIndex^1\pn{t_\DAggerSimulationTimeStepIndex} & \PupilHistoryVector_\FlockAgentIndex^2\pn{t_\DAggerSimulationTimeStepIndex} & \dots & \PupilHistoryVector_\FlockAgentIndex^\PupilHistoryCount\pn{t_\DAggerSimulationTimeStepIndex}
%\end{bmatrix}$. 
Finally, \PupilTDAGNN{} outputs the vector $\PupilName\pn{\PupilHistoryMatrix_\FlockAgentIndex\pn{t_\DAggerSimulationTimeStepIndex}}$ for agent $\FlockAgentIndex$'s acceleration,
where $\PupilName:\bR^{\gConvChannelsIn[1] \times \PupilHistoryCount} \to \bR^{\gConvChannelsIn[\PupilLayerCount + 1]}$ is a 1D CNN with element-wise $\tanh$ activation after each convolutional layer except the last layer $\PupilLayerCount$.
For agents in $\bR^2$, $\gConvChannelsIn[\PupilLayerCount + 1] = 2$.
$\PupilHistoryCount$ is the depth that the recurrence is computed to, and the larger $\PupilHistoryCount$ is, the more message summaries each agent needs to store; i.e., each agent $\FlockAgentIndex$ stores $\PupilHistoryMatrix_{\FlockAgentIndex} \pn{t_{\DAggerSimulationTimeStepIndex - 1}}$ to be passed on at time $t_{\DAggerSimulationTimeStepIndex}$.




\PupilTDAGNN{} is trained using imitation learning (IL) reviewed in \secref{sec:bg:imitation-learning}.
In our context of flocking control, the expert controller for IL is the centralized controller in \eqref{eq:tanner:centralized} with the potential function
\begin{equation}%\small 
\label{eq:tanner:centralized:tolstaya}
\begin{aligned}
    U\pn{r} = \begin{cases}
        1/r^2 + \ln\pn{r^2} & r \leq \FlockAgentCommRadius, \\
        U\pn{\FlockAgentCommRadius} & \text{else}.
    \end{cases}
\end{aligned}
\end{equation}
The output of the expert controller $\gExpertName$ is clamped so that $\norm{\gExpertName\pn{\gExpertInput}}_\infty \leq \ExpertAccelerationClampConstant$.
%Next, the authors of 
In \cite{tolstayaLearningDecentralizedControllers2017b}, $\PupilFeatureFunc: \bR^2 \times \bR^2 \to \bR^6$ computes the terms that enter linearly into the centralized controller's acceleration:
\begin{equation}%\small
\label{eq:tolstaya:node-features}
\begin{aligned}
    \PupilFeatureFunc\pn{\FlockAgentRelativePos, \FlockAgentRelativeVel} = \spn{
        \FlockAgentRelativeVel[_{\FlockAgentIndex\FlockAgentIndexTwo}^\top] ,\; \norm{\FlockAgentRelativePos}^{-4}\FlockAgentRelativePos[_{\FlockAgentIndex\FlockAgentIndexTwo}^\top] ,\; \norm{\FlockAgentRelativePos}^{-2}\FlockAgentRelativePos[_{\FlockAgentIndex\FlockAgentIndexTwo}^\top]
    }^\top.
\end{aligned}
\end{equation}
\PupilTDAGNN{} trained using this $\TannerEtAlPotential$ and $\PupilFeatureFunc$ learns flocking control.
Using the $\TannerEtAlPotential$ from \cite{santosSegregationMultipleHeterogeneous2014} and the $\PupilFeatureFunc$ from \cite{omotuyiLearningDecentralizedControllers2022}, \PupilTDAGNN{} learns flocking control with segregation.




\subsubsection{Imitaiton learning}\label{sec:bg:imitation-learning}
Flocking controllers are evaluated by whether they achieve their collective objective %(i.e., asymptotic flocking) 
and by what intermediate steps they take to do so. %depending on their guarantees. 
For example, controllers that guarantee separation are devalued if they take intermediate steps that cause agent collisions. While non-ML controllers can be designed with these evaluation criteria in mind, ML controllers must be trained to comply with the constraints imposed by these criteria. IL is a framework that trains ML controllers to produce the same output as a compliant ``expert'' controller. Let $\gExpertName$ be the expert controller in \eqref{eq:tanner:centralized:tolstaya} and \PupilName{} be \PupilTDAGNN{} -- a pupil controller.
The IL loss function used in \cite{tolstayaLearningDecentralizedControllers2017b} is
\begin{equation}%\small 
\label{eq:imitation-learning:loss-function}
\begin{aligned}
    & \LossFunctionName\pn{\gDAggerDatasetDatum} = \frac{1}{\FlockAgentCount}\sum_{\FlockAgentIndex = 1}^\FlockAgentCount \ell\pn{\gExpertName\pn{\gExpertInput}, \PupilName\pn{\PupilHistoryMatrix_\FlockAgentIndex}},
\end{aligned}
\end{equation}
where $\ell$ is another loss function (e.g., the squared error loss), and $\PupilHistoryMatrix_\FlockAgentIndex$ is the input to \PupilTDAGNN{} for agent $\FlockAgentIndex$. 
%given $\FlockPosMatrix$ and $\FlockVelMatrix$. 
Ideally, we would compute $\LossFunctionName$ for every tuple $\pn{\gDAggerDatasetDatum}$ %and compute an overall loss, 
but this is intractable. Instead, we sample a training set from the space of these tuples using Dataset Aggregation (DAgger) reinforcement learning \cite{rossReductionImitationLearning2011}. %Tolstaya et al. 
DAgger is employed in \cite{tolstayaLearningDecentralizedControllers2017b} to train \PupilTDAGNN{} as follows: %\BW{\bf [Need some details of DAgger]}. 
Let $E$ and $T$ be the number of training epochs and time steps in a flocking simulation, respectively.
At the beginning of each epoch $\DAggerEpochIndex \in \set{0, \dots, \DAggerEpochCount - 1}$, an initial flock state $\pn{\FlockPosMatrix\pn{t_0}, \FlockVelMatrix\pn{t_0}}$ is chosen from a dataset of initial conditions. Next, flocking is simulated for $T$ time steps. For each $\DAggerSimulationTimeStepIndex \in \set{0,\dots,\DAggerSimulationTimeStepsCount-1}$, $\gExpertName\pn{\gExpertInput[\pn{t_\DAggerSimulationTimeStepIndex}]}$ and $\PupilHistoryMatrix_\FlockAgentIndex\pn{t_\DAggerSimulationTimeStepIndex}$ are computed. With probability $\DAggerActionPropability_\DAggerEpochIndex$, the acceleration $\gExpertName\pn{\gExpertInput[\pn{t_\DAggerSimulationTimeStepIndex}]}$ is used to update the flock state, computing $\FlockPosMatrix\pn{t_{\DAggerSimulationTimeStepIndex + 1}}$ and $\FlockVelMatrix\pn{t_{\DAggerSimulationTimeStepIndex + 1}}$; otherwise, the acceleration $\PupilName\pn{\PupilHistoryMatrix_\FlockAgentIndex\pn{t_\DAggerSimulationTimeStepIndex}}$ is used.
Finally, the tuple $\pn{\gDAggerDatasetDatum[\pn{t_\DAggerSimulationTimeStepIndex}]}$ is added to the training set. Once $T$ time steps of flocking are complete, a batch $\GeneralizationSampleName$ of tuples is uniformly randomly sampled from the training set.
The average value of $\LossFunctionName$ over the batch is computed and \PupilTDAGNN{}'s weights are updated.
We can update the weights more than once per epoch by sampling more batches.
%Extra details of the training procedure include that $\DAggerActionPropability_\DAggerEpochIndex$ may vary over epochs and is usually non-increasing, and the training set will delete its oldest elements first once it grows to some maximum size.


\subsection{Rotation equivariance and translation invariance of flocking controllers
%systems
}
\label{sec:flocking-symmetry}
Symmetry is a fundamental inductive bias for designing reliable and efficient neural networks \cite{pmlr-v80-kondor18a,pmlr-v162-lawrence22a,van2020mdp,yarotsky2022universal,wang2024rethinking}.
The symmetry of a %neural network 
function $f: X \to Y$ is often described by equivariance: $f$ is equivariant if $f(T_g(x)) = T'_g(f(x))$, where $T_g, T'_g$ are transformations representing the group element $g$ on $X$ and $Y$, respectively.
If $T_g'$ is the identity, $f$ is invariant.
A crucial property of GNNs is that their node feature aggregations are permutation invariant \cite{garg2020generalization}, and GNNs have also been extended to be roto-translation equivariant \cite{satorras2021n,brandstetter2022geometric}.
% An agent balancing the three objectives of Reynolds' flocking simulation \cite{reynoldsFlocksHerdsSchools1987} observes its neighboring agents' positions and velocities relative to its own, so it is natural for a flocking controller to work with the relative vectors $\FlockAgentRelativePos$ and $\FlockAgentRelativeVel$.
For flocking, prominent controllers $\gExpertName$ (e.g., \cite{tannerStableFlockingMobile2003a,cuckerEmergentBehaviorFlocks2007}) have the following symmetries:
% \BW{\bf [We should show translation invariance and rotation and reflection equivariance. Centralized model.]}
\begin{itemize}
\item \textbf{Translation invariance:} For any $\vt_1, \vt_2 \in \bR^2$,  
%$$\small
%\begin{aligned}
%\gExpertName\pn{\FlockPosMatrix + \vt_1\vone_\FlockAgentCount^\top, \FlockVelMatrix + \vt_2\vone_\FlockAgentCount^\top} = \gExpertName\pn{\gExpertInput}
%\end{aligned}
%$$
considering the controller in \eqref{eq:tanner:centralized}, we have 
$$%\small
\begin{aligned}
&\ \ \ \ \ \gExpertName\pn{\FlockPosMatrix + \vt_1\vone_\FlockAgentCount^\top, \FlockVelMatrix + \vt_2\vone_\FlockAgentCount^\top}\\ & = -\sum_{\FlockAgentIndexTwo = 1}^{\FlockAgentCount} \pn{\FlockAgentVel_\FlockAgentIndex + \vt_2} - \pn{\FlockAgentVel_\FlockAgentIndexTwo + \vt_2} - \sum_{\FlockAgentIndexTwo = 1}^{\FlockAgentCount} \nabla\TannerEtAlPotential\pn{\norm{\pn{\FlockAgentPos_\FlockAgentIndex + \vt_1} - \pn{\FlockAgentPos_\FlockAgentIndexTwo + \vt_1}}} \\
& = -\sum_{\FlockAgentIndexTwo = 1}^{\FlockAgentCount} \FlockAgentRelativeVel - \sum_{\FlockAgentIndexTwo = 1}^{\FlockAgentCount} \nabla\TannerEtAlPotential\pn{\norm{\FlockAgentRelativePos}} \\
& = \gExpertName\pn{\gExpertInput}
\end{aligned}
$$

\item \textbf{Rotation and reflection equivariance:} For any orthogonal matrix $\UtilOrthogonalMatrix \in \OrthogonalGroup{2}$, we have
%$$\small
%\begin{aligned}
%\gExpertName\pn{\UtilOrthogonalMatrix\FlockPosMatrix, \UtilOrthogonalMatrix\FlockVelMatrix} = \UtilOrthogonalMatrix\gExpertName\pn{\gExpertInput}
%\end{aligned}
%$$
%Considering the controller in \eqref{eq:tanner:centralized},
$$%\small
\begin{aligned}
\gExpertName\pn{\UtilOrthogonalMatrix\FlockPosMatrix, \UtilOrthogonalMatrix\FlockVelMatrix} & = -\sum_{\FlockAgentIndexTwo = 1}^{\FlockAgentCount} \UtilOrthogonalMatrix\FlockAgentVel_\FlockAgentIndex - \UtilOrthogonalMatrix\FlockAgentVel_\FlockAgentIndexTwo - \sum_{\FlockAgentIndexTwo = 1}^{\FlockAgentCount} \nabla\TannerEtAlPotential\pn{\norm{\UtilOrthogonalMatrix\FlockAgentPos_\FlockAgentIndex - \UtilOrthogonalMatrix\FlockAgentPos_\FlockAgentIndexTwo}} \\
%& = -\sum_{\FlockAgentIndexTwo = 1}^{\FlockAgentCount} \UtilOrthogonalMatrix\FlockAgentRelativeVel - \sum_{\FlockAgentIndexTwo = 1}^{\FlockAgentCount} \nabla\TannerEtAlPotential\pn{\norm{\UtilOrthogonalMatrix\FlockAgentRelativePos}} \\
& = -\sum_{\FlockAgentIndexTwo = 1}^{\FlockAgentCount} \UtilOrthogonalMatrix\FlockAgentRelativeVel - \sum_{\FlockAgentIndexTwo = 1}^{\FlockAgentCount} \TannerEtAlPotential'\pn{\norm{\UtilOrthogonalMatrix\FlockAgentRelativePos}}\frac{\UtilOrthogonalMatrix\FlockAgentRelativePos}{\norm{\UtilOrthogonalMatrix\FlockAgentRelativePos}} \\
%& = \UtilOrthogonalMatrix\pn*{-\sum_{\FlockAgentIndexTwo = 1}^{\FlockAgentCount} \FlockAgentRelativeVel - \sum_{\FlockAgentIndexTwo = 1}^{\FlockAgentCount} \TannerEtAlPotential'\pn{\norm{\FlockAgentRelativePos}}\frac{\FlockAgentRelativePos}{\norm{\FlockAgentRelativePos}}} \\
& = \UtilOrthogonalMatrix\pn*{-\sum_{\FlockAgentIndexTwo = 1}^{\FlockAgentCount} \FlockAgentRelativeVel - \sum_{\FlockAgentIndexTwo = 1}^{\FlockAgentCount} \nabla\TannerEtAlPotential\pn{\norm{\FlockAgentRelativePos}}} \\
& = \UtilOrthogonalMatrix\gExpertName\pn{\gExpertInput}.
\end{aligned}
$$
\end{itemize}
Existing ML models, like TDAGNN, do not satisfy these symmetries.
%, and we aim to bridge this gap in this paper.
%we propose that ML flocking controllers should also satisfy these two conditions. 





\section{Equivariant controllers for learning decentralized flocking}\label{sec:model-proposed}
In this section, we present a rotation equivariant ML controller for decentralized flocking. Our approach replaces the CNN in \PupilTDAGNN{} with an \OrthogonalGroup{2} equivariant CNN, ensuring rotation and reflection equivariance of the resulting ML controller.
%We describe the rotation equivariant convolutional layers and activations used by the CNN.
%In addition, we propose two further improvements to \PupilTDAGNN{}. %unrelated to equivariance.

%\TT{(I added a couple sentences to clarify that we need the activations to be equivariant too, so we can delete the following sentence.)}
%\textcolor{blue}{In addition, we propose two rotation equivariant activation functions and two improvements to \PupilTDAGNN{} unrelated to equivariance.} \BW{\bf [Do we need both activation and CNN to ensure equivariance?]}


\subsection{%Enforcing rotational equivariance
Rotation equivariant convolution layers
}
\label{sec:methods:equiv-model}
\PupilTDAGNN{} is translation invariant, but not rotation equivariant because its CNN component \PupilName{} is not. %since its convolutional layers and activation functions are not. 
There has been significant effort put toward developing roto-translation equivariant CNNs (e.g., \SpecialOrthogonalGroup{2}-steerable CNNs \cite{weiler2019general}); however, these cannot be directly integrated into \PupilTDAGNN{}. As such, we aim to replace \PupilName{} with an \OrthogonalGroup{2} equivariant CNN \PupilEquivariantName{} equipped with rotation equivariant convolutional layers and activations. To ease our presentation, we make the following two assumptions:
%we assume the inputs of $\phi$ always have the same size. 


%Instead, our equivariant ML controller replaces \PupilName{} with an \OrthogonalGroup{2} equivariant CNN \PupilEquivariantName{}.
%Notice that, the 1D CNN \PupilName{} is not rotation equivariant because its convolutional layers and activation functions are not. Before demonstrating why, we first simplify our analysis by assuming the following: \BW{shedding lightg on designing equivariant models}

\begin{assumption}\label{assumption:methods:no-padding}
The 1D convolutional layers 
$$\set{\ConvOneD_\PupilLayerIndex: \bR^{\gConvChannelsIn \times \gConvFeaturesIn} \to \bR^{\gConvChannelsIn[\PupilLayerIndex+1] \times \gConvFeaturesIn[\PupilLayerIndex+1]}}_{\PupilLayerIndex = 1}^{\PupilLayerCount}
$$ 
of \PupilName{} have no padding.
\end{assumption}

\begin{assumption}\label{assumption:methods:same-size-input}
The input of \PupilName{} always has the same size.
\end{assumption}


We represent the input $\gConvInput$ of $\ConvOneD_\PupilLayerIndex$ as a block matrix composing vectors in $\bR^2$ that are \OrthogonalGroup{2} equivariant with respect to the agents' positions and velocities:
\begin{align*}
\gConvInput = \spn{
\EqConvChannel_{\ConvChannelInIndex,\ConvFeatureIndex}
}_{\ConvChannelInIndex=1,\ConvFeatureIndex=1}^{\pn{\gConvChannelsIn/2},\gConvFeaturesIn} \subset \bR^{\gConvChannelsIn \times \gConvFeaturesIn}
\end{align*}
%Note that 
If $\PupilLayerIndex = 1$ and we %are using 
use $\PupilFeatureFunc$ from \eqref{eq:tolstaya:node-features}, then $\gConvInput = \PupilHistoryMatrix_\FlockAgentIndex$ from section~\ref{sec:bg:decentralized:ml}, 
$\gConvChannelsIn = 6$, and $\gConvFeaturesIn = \PupilHistoryCount$.
Moreover, the last convolutional layer %(i.e., layer $\PupilLayerCount$) 
in $\PupilName$ outputs acceleration so $\gConvChannelsIn[\PupilLayerCount+1] = 2$ and $\gConvFeaturesIn[\PupilLayerCount+1] = 1$.
%
%
By Assumption~\ref{assumption:methods:no-padding}, $\ConvOneD_\PupilLayerIndex$ can be represented by a collection of Toeplitz matrices $\PupilWeightMatrix: \set{1,\dots,\gConvChannelsIn} \times \set{1,\dots,\gConvChannelsIn[\PupilLayerIndex+1]} \to \bR^{\gConvFeaturesIn \times \gConvFeaturesIn[\PupilLayerIndex+1]}$ and bias terms $\ConvBias: \set{1,\dots,\gConvChannelsIn[\PupilLayerIndex+1]} \to \bR$.
% edit(Taos): the bias terms do depend on the out channel index, so we should write them as this function. Writing them as just  $b$ incorrectly suggests $b$ is a fixed constant. See https://pytorch.org/docs/stable/generated/torch.nn.Conv1d.html
For $\ConvChannelOutIndex \in \set{1,\dots,\gConvChannelsIn[\PupilLayerIndex+1]}$, the output channel $\ConvOneD_\PupilLayerIndex\pn{\gConvInput}[\ConvChannelOutIndex]$ can be represented as
$$%\small
\begin{aligned}
    &\ \ \ConvOneD_\PupilLayerIndex\pn{\gConvInput}[\ConvChannelOutIndex]\\ 
    & = \ConvBias\pn{\ConvChannelOutIndex}\vone_{\gConvFeaturesIn[\PupilLayerIndex+1]}^\top + \sum_{\ConvChannelInIndex = 1}^{\gConvChannelsIn} \spn{
        \pn{\gConvInput}_{\ConvChannelInIndex,1}
        ,\; \dots
        ,\; \pn{\gConvInput}_{\ConvChannelInIndex,\gConvFeaturesIn}
    }\PupilWeightMatrix\pn{\ConvChannelInIndex,\ConvChannelOutIndex} \\
    & = \ConvBias\pn{\ConvChannelOutIndex}\vone_{\gConvFeaturesIn[\PupilLayerIndex+1]}^\top + \sum_{\ConvChannelInIndex = 1}^{\gConvChannelsIn/2} \sum_{c = 1}^2 \spn{
        \pn{\EqConvChannel_{\ConvChannelInIndex, 1}}_c
        ,\; \dots
        ,\; \pn{\EqConvChannel_{\ConvChannelInIndex, \gConvFeaturesIn}}_c
    }\PupilWeightMatrix\pn{2\pn{\ConvChannelInIndex - 1} + c, \ConvChannelOutIndex} \\
    & = \ConvBias\pn{\ConvChannelOutIndex}\vone_{\gConvFeaturesIn[\PupilLayerIndex+1]}^\top + \sum_{\ConvChannelInIndex = 1}^{\gConvChannelsIn/2} \vone_2^\top\begin{bmatrix}
        \spn{
            \pn{\EqConvChannel_{\ConvChannelInIndex, 1}}_1
            ,\; \dots
            ,\; \pn{\EqConvChannel_{\ConvChannelInIndex, \gConvFeaturesIn}}_1
        }\PupilWeightMatrix\pn{2\pn{\ConvChannelInIndex - 1} + 1, \ConvChannelOutIndex} \\[.1em]
        \spn{
            \pn{\EqConvChannel_{\ConvChannelInIndex, 1}}_2
            ,\; \dots
            ,\; \pn{\EqConvChannel_{\ConvChannelInIndex, \gConvFeaturesIn}}_2
        }\PupilWeightMatrix\pn{2\pn{\ConvChannelInIndex - 1} + 2, \ConvChannelOutIndex}
    \end{bmatrix}.
\end{aligned}
\iffalse
\begin{aligned}
    \ConvOneD_\PupilLayerIndex\pn{\gConvInput}[\ConvChannelOutIndex] & = \ConvBias\pn{\ConvChannelOutIndex}\vone_{\gConvFeaturesIn[\PupilLayerIndex+1]}^\top + \sum_{\ConvChannelInIndex = 1}^{\gConvChannelsIn} \begin{bmatrix}
        \pn{\gConvInput}_{\ConvChannelInIndex,1}
        & \pn{\gConvInput}_{\ConvChannelInIndex,2}
        & \dots
        & \pn{\gConvInput}_{\ConvChannelInIndex,\gConvFeaturesIn}
    \end{bmatrix}\PupilWeightMatrix\pn{\ConvChannelInIndex,\ConvChannelOutIndex} \\
    & = \ConvBias\pn{\ConvChannelOutIndex}\vone_{\gConvFeaturesIn[\PupilLayerIndex+1]}^\top + \sum_{\ConvChannelInIndex = 1}^{\gConvChannelsIn/2} \sum_{c = 1}^2 \begin{bmatrix}
        \pn{\EqConvChannel_{\ConvChannelInIndex, 1}}_c
        & \pn{\EqConvChannel_{\ConvChannelInIndex, 2}}_c
        & \dots
        & \pn{\EqConvChannel_{\ConvChannelInIndex, \gConvFeaturesIn}}_c
    \end{bmatrix}\PupilWeightMatrix\pn{2\pn{\ConvChannelInIndex - 1} + c, \ConvChannelOutIndex} \\
    & = \ConvBias\pn{\ConvChannelOutIndex}\vone_{\gConvFeaturesIn[\PupilLayerIndex+1]}^\top + \sum_{\ConvChannelInIndex = 1}^{\gConvChannelsIn/2} \vone_2^\top\begin{bmatrix}
        \begin{bmatrix}
            \pn{\EqConvChannel_{\ConvChannelInIndex, 1}}_1
            & \pn{\EqConvChannel_{\ConvChannelInIndex, 2}}_1
            & \dots
            & \pn{\EqConvChannel_{\ConvChannelInIndex, \gConvFeaturesIn}}_1
        \end{bmatrix}\PupilWeightMatrix\pn{2\pn{\ConvChannelInIndex - 1} + 1, \ConvChannelOutIndex} \\[.1em]
        \begin{bmatrix}
            \pn{\EqConvChannel_{\ConvChannelInIndex, 1}}_2
            & \pn{\EqConvChannel_{\ConvChannelInIndex, 2}}_2
            & \dots
            & \pn{\EqConvChannel_{\ConvChannelInIndex, \gConvFeaturesIn}}_2
        \end{bmatrix}\PupilWeightMatrix\pn{2\pn{\ConvChannelInIndex - 1} + 2, \ConvChannelOutIndex}
    \end{bmatrix}.
\end{aligned}
\fi
$$
Applying an orthogonal matrix $\UtilOrthogonalMatrix \in \OrthogonalGroup{2}$ to the agents' position and velocity, we have
$$%\small
\begin{aligned}
    &\ConvOneD_\PupilLayerIndex\pn*{\spn{
        \UtilOrthogonalMatrix\EqConvChannel_{\ConvChannelInIndex,\ConvFeatureIndex}
    }_{\ConvChannelInIndex=1,\ConvFeatureIndex=1}^{\pn{\gConvChannelsIn/2},\gConvFeaturesIn}}[\ConvChannelOutIndex]\\
     =\; & \ConvBias\pn{\ConvChannelOutIndex}\vone_{\gConvFeaturesIn[\PupilLayerIndex+1]}^\top 
     + \sum_{\ConvChannelInIndex = 1}^{\gConvChannelsIn/2} \vone_2^\top\begin{bmatrix}
        \spn{
            \pn{\UtilOrthogonalMatrix\EqConvChannel_{\ConvChannelInIndex, 1}}_1
            ,\; \dots
            ,\; \pn{\UtilOrthogonalMatrix\EqConvChannel_{\ConvChannelInIndex, \gConvFeaturesIn}}_1
        }\PupilWeightMatrix\pn{2\pn{\ConvChannelInIndex - 1} + 1, \ConvChannelOutIndex} \\[.1em]
        \spn{
            \pn{\UtilOrthogonalMatrix\EqConvChannel_{\ConvChannelInIndex, 1}}_2
            ,\; \dots
            ,\; \pn{\UtilOrthogonalMatrix\EqConvChannel_{\ConvChannelInIndex, \gConvFeaturesIn}}_2
        }\PupilWeightMatrix\pn{2\pn{\ConvChannelInIndex - 1} + 2, \ConvChannelOutIndex}
    \end{bmatrix} \\
    &
    \neq \UtilOrthogonalMatrix\ConvOneD_\PupilLayerIndex\pn*{\spn{
        \EqConvChannel_{\ConvChannelInIndex,\ConvFeatureIndex}
    }_{\ConvChannelInIndex=1,\ConvFeatureIndex=1}^{\pn{\gConvChannelsIn/2},\gConvFeaturesIn}}.
\end{aligned}
\iffalse
\begin{aligned}
    &\ConvOneD_\PupilLayerIndex\pn*{\begin{bmatrix}
        \UtilOrthogonalMatrix\EqConvChannel_{\ConvChannelInIndex,\ConvFeatureIndex}
    \end{bmatrix}_{\ConvChannelInIndex=1,\ConvFeatureIndex=1}^{\pn{\gConvChannelsIn/2},\gConvFeaturesIn}}[\ConvChannelOutIndex]\\
     = & \ConvBias\pn{\ConvChannelOutIndex}\vone_{\gConvFeaturesIn[\PupilLayerIndex+1]}^\top 
     \quad + \sum_{\ConvChannelInIndex = 1}^{\gConvChannelsIn/2} \vone_2^\top\begin{bmatrix}
        \begin{bmatrix}
            \pn{\UtilOrthogonalMatrix\EqConvChannel_{\ConvChannelInIndex, 1}}_1
            & \pn{\UtilOrthogonalMatrix\EqConvChannel_{\ConvChannelInIndex, 2}}_1
            & \dots
            & \pn{\UtilOrthogonalMatrix\EqConvChannel_{\ConvChannelInIndex, \gConvFeaturesIn}}_1
        \end{bmatrix}\PupilWeightMatrix\pn{2\pn{\ConvChannelInIndex - 1} + 1, \ConvChannelOutIndex} \\[.1em]
        \begin{bmatrix}
            \pn{\UtilOrthogonalMatrix\EqConvChannel_{\ConvChannelInIndex, 1}}_2
            & \pn{\UtilOrthogonalMatrix\EqConvChannel_{\ConvChannelInIndex, 2}}_2
            & \dots
            & \pn{\UtilOrthogonalMatrix\EqConvChannel_{\ConvChannelInIndex, \gConvFeaturesIn}}_2
        \end{bmatrix}\PupilWeightMatrix\pn{2\pn{\ConvChannelInIndex - 1} + 2, \ConvChannelOutIndex}
    \end{bmatrix} \\
    \neq & \UtilOrthogonalMatrix\ConvOneD_\PupilLayerIndex\pn*{\begin{bmatrix}
        \EqConvChannel_{\ConvChannelInIndex,\ConvFeatureIndex}
    \end{bmatrix}_{\ConvChannelInIndex=1,\ConvFeatureIndex=1}^{\pn{\gConvChannelsIn/2},\gConvFeaturesIn}}.
\end{aligned}
\fi
$$
From the above analysis, we see that {\bf $\ConvOneD_\PupilLayerIndex$ is not \OrthogonalGroup{2} equivariant} for three reasons: (1) the bias term, (2) vector element indexing is not \OrthogonalGroup{2} equivariant, and (3) $\UtilOrthogonalMatrix$ does not commute with $\vone_2^\top$. 
%-- the 2D vector with both entries being 1. 
We need to address these issues to develop an \OrthogonalGroup{2} equivariant convolution. For the first issue, we eliminate the bias term.
For the second issue, we eliminate the vector element indexing through weight sharing.
By setting $\PupilWeightMatrix\pn{2\pn{\ConvChannelInIndex - 1} + 1, \ConvChannelOutIndex} = \PupilWeightMatrix\pn{2\pn{\ConvChannelInIndex - 1} + 2, \ConvChannelOutIndex}$ for all $\ConvChannelInIndex$, then for an output channel $\ConvChannelOutIndex$, we have
$$%\small
\begin{aligned}
    &\ \  \sum_{\ConvChannelInIndex = 1}^{\gConvChannelsIn/2} \vone_2^\top\begin{bmatrix}
        \spn{
            \pn{\EqConvChannel_{\ConvChannelInIndex, 1}}_1
            ,\; \dots
            ,\; \pn{\EqConvChannel_{\ConvChannelInIndex, \gConvFeaturesIn}}_1
        }\PupilWeightMatrix\pn{2\pn{\ConvChannelInIndex - 1} + 1, \ConvChannelOutIndex} \\[.1em]
        \spn{
            \pn{\EqConvChannel_{\ConvChannelInIndex, 1}}_2
            ,\; \dots
            ,\; \pn{\EqConvChannel_{\ConvChannelInIndex, \gConvFeaturesIn}}_2
        }\PupilWeightMatrix\pn{2\pn{\ConvChannelInIndex - 1} + 2, \ConvChannelOutIndex}
    \end{bmatrix} \\
 &    =  \sum_{\ConvChannelInIndex = 1}^{\gConvChannelsIn/2} \vone_2^\top\spn{
        \EqConvChannel_{\ConvChannelInIndex, 1}
        ,\; \dots
        ,\; \EqConvChannel_{\ConvChannelInIndex, \gConvFeaturesIn}
    }\PupilWeightMatrix\pn{2\pn{\ConvChannelInIndex - 1} + 1, \ConvChannelOutIndex}.
\end{aligned}
\iffalse
\begin{aligned}
    & \sum_{\ConvChannelInIndex = 1}^{\gConvChannelsIn/2} \vone_2^\top\begin{bmatrix}
        \begin{bmatrix}
            \pn{\EqConvChannel_{\ConvChannelInIndex, 1}}_1
            & \pn{\EqConvChannel_{\ConvChannelInIndex, 2}}_1
            & \dots
            & \pn{\EqConvChannel_{\ConvChannelInIndex, \gConvFeaturesIn}}_1
        \end{bmatrix}\PupilWeightMatrix\pn{2\pn{\ConvChannelInIndex - 1} + 1, \ConvChannelOutIndex} \\[.1em]
        \begin{bmatrix}
            \pn{\EqConvChannel_{\ConvChannelInIndex, 1}}_2
            & \pn{\EqConvChannel_{\ConvChannelInIndex, 2}}_2
            & \dots
            & \pn{\EqConvChannel_{\ConvChannelInIndex, \gConvFeaturesIn}}_2
        \end{bmatrix}\PupilWeightMatrix\pn{2\pn{\ConvChannelInIndex - 1} + 2, \ConvChannelOutIndex}
    \end{bmatrix} \\
     = & \sum_{\ConvChannelInIndex = 1}^{\gConvChannelsIn/2} \vone_2^\top\begin{bmatrix}
        \EqConvChannel_{\ConvChannelInIndex, 1}
        & \EqConvChannel_{\ConvChannelInIndex, 2}
        & \dots
        & \EqConvChannel_{\ConvChannelInIndex, \gConvFeaturesIn}
    \end{bmatrix}\PupilWeightMatrix\pn{2\pn{\ConvChannelInIndex - 1} + 1, \ConvChannelOutIndex}
\end{aligned}
\fi
$$
For the third issue, we drop $\vone_2^\top$, which changes the number of rows per output channel from one to two. To retain the original number of rows in the output of $\ConvOneD$, we halve the number of output channels but set each output channel to have two rows (or ``subchannels'').
With these changes, we define a new \OrthogonalGroup{2} equivariant convolution $\EqConv_\PupilLayerIndex$: For output channels $\ConvChannelOutIndex \in \set{1,\dots,\gConvChannelsIn[\PupilLayerIndex+1]/2}$,
$$%\small
\begin{aligned}
    \EqConv_\PupilLayerIndex\pn{\gConvInput}[\ConvChannelOutIndex] & = \sum_{\ConvChannelInIndex = 1}^{\gConvChannelsIn/2} \spn{
        \EqConvChannel_{\ConvChannelInIndex, 1}
        ,\; \dots
        ,\; \EqConvChannel_{\ConvChannelInIndex, \gConvFeaturesIn}
    }\PupilWeightMatrix\pn{2\pn{\ConvChannelInIndex - 1} + 1, \ConvChannelOutIndex}.
\end{aligned}
$$
$\EqConv_\PupilLayerIndex$ is \OrthogonalGroup{2} equivariant since
$$%\small
\begin{aligned}
  &\ \   \EqConv_\PupilLayerIndex\pn*{\spn{
        \UtilOrthogonalMatrix\EqConvChannel_{\ConvChannelInIndex,\ConvFeatureIndex}
    }_{\ConvChannelInIndex=1,\ConvFeatureIndex=1}^{\pn{\gConvChannelsIn/2},\gConvFeaturesIn}}[\ConvChannelOutIndex] \\
    & = \sum_{\ConvChannelInIndex = 1}^{\gConvChannelsIn/2} \spn{
        \UtilOrthogonalMatrix\EqConvChannel_{\ConvChannelInIndex, 1}
        ,\; \dots
        ,\; \UtilOrthogonalMatrix\EqConvChannel_{\ConvChannelInIndex, \gConvFeaturesIn}
    }\PupilWeightMatrix\pn{2\pn{\ConvChannelInIndex - 1} + 1, \ConvChannelOutIndex} \\
    %& = \sum_{\ConvChannelInIndex = 1}^{\gConvChannelsIn/2} \UtilOrthogonalMatrix\begin{bmatrix}
     %   \EqConvChannel_{\ConvChannelInIndex, 1}
     %   & \dots
     %   & \EqConvChannel_{\ConvChannelInIndex, \gConvFeaturesIn}
    %\end{bmatrix}\PupilWeightMatrix\pn{2\pn{\ConvChannelInIndex - 1} + 1, \ConvChannelOutIndex} \\
    & = \UtilOrthogonalMatrix\EqConv_\PupilLayerIndex\pn*{\spn{
        \EqConvChannel_{\ConvChannelInIndex,\ConvFeatureIndex}
    }_{\ConvChannelInIndex=1,\ConvFeatureIndex=1}^{\pn{\gConvChannelsIn/2},\gConvFeaturesIn}}[\ConvChannelOutIndex].
\end{aligned}
$$
% edit(Taos): Clarifying the kernel size and stride is valuable because it helps the reader check their understanding.
$\EqConv_\PupilLayerIndex$ can be implemented using a standard 2D convolutional layer whose kernels have one row and a vertical stride of one.
%(e.g., PyTorch's Conv2D). 
%$\EqConv_\PupilLayerIndex$ is a 2D convolutional layer whose kernels have one row and a vertical stride of one.
%
%With the \OrthogonalGroup{2} equivariant convolution $\EqConv_\PupilLayerIndex$, 


\subsection{%Proposed 
Rotation equivariant activations}
\label{sec:methods:equivariant-activations}
%\BW{\bf [How about those in SEGNN paper?]}
To build an \OrthogonalGroup{2} equivariant CNN, we need \OrthogonalGroup{2} equivariant activations.
Notice that the output of $\EqConv_\PupilLayerIndex$ is also a block matrix composing 2D vectors that are \OrthogonalGroup{2} equivariant:
%with respect to the agents' positions and velocities:
$$%\small
\begin{aligned}
    \gConvInput[\PupilLayerIndex+1] = \EqConv_\PupilLayerIndex\pn{\gConvInput} = \spn{
        \EqConvChannel_{\ConvChannelOutIndex,\ConvFeatureIndex}
    }_{\ConvChannelOutIndex=1,\ConvFeatureIndex=1}^{\pn{\gConvChannelsIn[\PupilLayerIndex+1]/2},\gConvFeaturesIn[\PupilLayerIndex+1]} \subset \bR^{\gConvChannelsIn[\PupilLayerIndex+1] \times \gConvFeaturesIn[\PupilLayerIndex+1]}.
\end{aligned}
$$
We construct an \OrthogonalGroup{2} equivariant activation for $\gConvInput[\PupilLayerIndex+1]$ by applying an \OrthogonalGroup{2} equivariant activation to each $\EqConvChannel_{\ConvChannelOutIndex, \ConvFeatureIndex}$ separately. 
By Proposition~1 of \cite{maWhySelfattentionNatural2022}, any O(2) equivariant function  can be expressed as $\PupilActivationEquivariantName\pn{\FlockAgentPos} = \FlockAgentPos\PupilActivationInvariantName\pn{\norm{\FlockAgentPos}}$ for some function $\PupilActivationInvariantName: \bR \to \bR$.
Therefore, the output channel $\ConvChannelOutIndex \in \set{1,\dots,\gConvChannelsIn[\PupilLayerIndex+1]/2}$ with an equivariant activation applied is given by
$$%\small
\begin{aligned}
    \PupilActivationEquivariantName\pn{\gConvInput[\PupilLayerIndex+1]} [\ConvChannelOutIndex] & = \spn{
        \PupilActivationEquivariantName\pn{\EqConvChannel_{\ConvChannelOutIndex, 1}}
        ,\; \dots
        ,\; \PupilActivationEquivariantName\pn{\EqConvChannel_{\ConvChannelOutIndex, \gConvFeaturesIn[\PupilLayerIndex+1]}}
    }\\
    &= \gConvInput[\PupilLayerIndex+1] [\ConvChannelOutIndex] \odot \vone_2 \spn{
        \PupilActivationInvariantName\pn{\norm{\EqConvChannel_{\ConvChannelOutIndex, 1}}}
        ,\; \dots
        ,\; \PupilActivationInvariantName\pn{\norm{\EqConvChannel_{\ConvChannelOutIndex, \gConvFeaturesIn[\PupilLayerIndex+1]}}}
    },
\end{aligned}
$$
where $\odot$ denotes the Hadamard product. The \OrthogonalGroup{2} equivariant convolution layer and \OrthogonalGroup{2} equivariant activations are combined to create an \OrthogonalGroup{2} equivariant convolution neural network \PupilEquivariantName{}.


We construct the following tailored activations for our application:
$$%\small
\begin{aligned}
\PupilETDAGNNActivationScaleLogInvariant\pn{x} = \begin{cases}
        1 & x = 0 \\
        \frac{\ln\pn{1 + x}}{x} & x \neq 0
    \end{cases}, \quad \text{and } \PupilETDAGNNActivationScaleTanhInvariant\pn{x} = \tanh\pn{x},
\end{aligned}
$$
$$%\small
\begin{aligned}%\label{eq:equivariant-activations}\small
    \, & \PupilETDAGNNActivationScaleLog\pn{\gConvInput{}} [\ConvChannelOutIndex] = \gConvInput{} [\ConvChannelOutIndex] \odot \vone_2 \spn{
        \PupilETDAGNNActivationScaleLogInvariant\pn{\norm{\EqConvChannel_{\ConvChannelOutIndex, 1}}}
        ,\; \dots
        ,\; \PupilETDAGNNActivationScaleLogInvariant\pn{\norm{\EqConvChannel_{\ConvChannelOutIndex, \gConvFeaturesIn[\PupilLayerIndex+1]}}}
    }, \\
    \, & \PupilETDAGNNActivationScaleTanh\pn{\gConvInput{}} [\ConvChannelOutIndex] = \gConvInput{} [\ConvChannelOutIndex] \odot \vone_2 \spn{
        \PupilETDAGNNActivationScaleTanhInvariant\pn{\norm{\EqConvChannel_{\ConvChannelOutIndex, 1}}}
        ,\; \dots
        ,\; \PupilETDAGNNActivationScaleTanhInvariant\pn{\norm{\EqConvChannel_{\ConvChannelOutIndex, \gConvFeaturesIn[\PupilLayerIndex+1]}}}
    }.
\end{aligned}
$$
The activation \PupilETDAGNNActivationScaleLog{} provides an important normalization effect similar to $\tanh$ in the non-equivariant ML controller \PupilTDAGNN{}.
Large convolutional layer inputs can occur because the input features in \eqref{eq:tolstaya:node-features} are unbounded as $\norm{\FlockAgentRelativePos} \to 0$, potentially causing instabilities during training. Applying $\tanh$ elementwise bounds the features passed inside the CNN to $\pn{-1, 1}$. However, in the non-equivariant controllers, $\tanh$ is not applied before the first convolutional layer because it reduces all large-magnitude features to approximately the same magnitude (i.e., it is saturated); doing so would remove important information about the proximity of neighboring agents. Unfortunately, the first convolutional layer is still subject to large inputs. In contrast, for \PupilEquivariantName{} we can reduce the size of the inputs to the first convolutional layer and avoid saturation by using $\PupilETDAGNNActivationScaleLog$.
%
%
Next, the activation \PupilETDAGNNActivationScaleTanh{} is simply a nonlinear scaling of the feature vectors. %, where smaller feature vectors are made even smaller.




\subsection{%Improvements to 
Further improving \PupilTDAGNN{}}\label{sec:methods:improving-tdagnn}
We present two strategies to improve both equivariant and non-equivariant ML controllers further. 
%The following are two improvements to \PupilTDAGNN{} that are unrelated to equivariance. In the results section, we evaluate the benefits of these improvements.


{\bf Activate once:} The $\tanh$ activation of the CNN \PupilName{}  normalizes the output of each convolutional layer %to the range 
into $\pn{-1, 1}$.
Since the activation is applied after all but the last convolutional layer, the last convolutional layer must have large weights to ensure the controller can output a large acceleration vector. A large acceleration is needed whenever the controller needs to react quickly to maintain a connected communication graph or avoid collision. However, large weights concentrated in the last convolutional layer may cause the controller to output large accelerations too often, leading to over-corrections of the agents' velocities. When training \PupilTDAGNN{}, the CNN \PupilName{} must balance the need for large weights in the last convolutional layer with their risk. This challenge is lessened by applying the activation only after the first convolutional layer, allowing the controller to learn large weights across its convolutional layers.


{\bf Mean one-hop aggregation:} %As seen in \eqref{eq:tolstaya:node-features:one-hop}, 
In \eqref{eq:tolstaya:node-features:one-hop}, \PupilTDAGNN{} uses a sum aggregation of the messages in the one-hop neighborhood of agent $\FlockAgentIndex$. Sum aggregation can cause over-corrections when agent $\FlockAgentIndex$'s neighbors send similar messages. Consider the scenario where all neighbors of agent $\FlockAgentIndex$ have the same velocity $\vv_{\FlockAgentNeighborhood_\FlockAgentIndex}$ and agent $\FlockAgentIndex$ has velocity $\FlockAgentVel_\FlockAgentIndex \neq \vv_{\FlockAgentNeighborhood_\FlockAgentIndex}$. Each neighbor sends a message to agent $\FlockAgentIndex$ requesting that it corrects its velocity by $-\pn{\FlockAgentVel_\FlockAgentIndex - \vv_{\FlockAgentNeighborhood_\FlockAgentIndex}}$. With sum aggregation, agent $\FlockAgentIndex$ overcorrects its velocity by $-\abs{\FlockAgentNeighborhood_\FlockAgentIndex}\pn{\FlockAgentVel_\FlockAgentIndex - \vv_{\FlockAgentNeighborhood_\FlockAgentIndex}}$, but with mean aggregation, agent $\FlockAgentIndex$ corrects its velocity by $-\pn{\FlockAgentVel_\FlockAgentIndex - \vv_{\FlockAgentNeighborhood_\FlockAgentIndex}}$, exactly what was requested by all its neighbors.



\section{Generalization analysis %of \PupilTDAGNN{}
}\label{sec:main:theory}
%\BW{\bf [Name of the model -- equivariance?]}
%\TT{(The analysis applies to all the ML controllers discussed here, equivariant or not. We could probably generalize the analysis to time-delayed aggregation GNNs applied to any graph-level task (not just flocking))}
In this section, we 
%prove a bound for the generalization gap 
analyze the generalizability
of \PupilTDAGNN{}-related models when trained for flocking following the generalization analysis %for EGNNs given 
framework in \cite{pmlr-v235-karczewski24a}. 
%The bound is dependent on the space of training data, the probability distribution of the data, and the loss function used to train \PupilTDAGNN{}. We specify all these in the following subsections and then state the bound.


\subsection{Generalization gap}
%We recap on some crucial concepts on geneeralization analysis.
%Generalizability characterizes the out-of-sample accuracy of an ML model, and the 
%The generalizability can be measured using the generalization gap:

%Two important qualities of ML models are their ability to fit training data (i.e., their expressiveness), and their ability to transfer their accuracy on training data to new data (i.e., their generalizability). An ML model's generalizability can be measured using the generalization gap:

\begin{definition}[Generalization gap]\label{def:generalization-gap}
Let $f: \GeneralizationDomain \to \GeneralizationRange$ and define the loss function $\LossFunctionName: \GeneralizationRange \times \GeneralizationRange \to \hlpn{0, \infty}$. Let $\GeneralizationSampleName = \set{\pn{x_\GeneralizationSampleIndex, y_\GeneralizationSampleIndex}}_{\GeneralizationSampleIndex = 1}^\GeneralizationSampleSize$ be a set of i.i.d. samples from a probability distribution $\GeneralizationDatasetDistribution$ over $\GeneralizationDomain \times \GeneralizationRange$. The \textit{generalization gap} $\gGeneralizationGap$ of $f$ is the difference between the expected risk and empirical risk, i.e.,
$$%\small
\begin{aligned}
\gGeneralizationGap\pn{f} = \underbrace{\bE_{\pn{x, y} \sim \GeneralizationDatasetDistribution}\spn{\LossFunctionName\pn{f\pn{x}, y}}}_{\mathrm{expected\ risk}} - \underbrace{\frac{1}{\GeneralizationSampleSize}\sum_{\GeneralizationSampleIndex = 1}^\GeneralizationSampleSize \LossFunctionName\pn{f\pn{x_\GeneralizationSampleIndex}, y_\GeneralizationSampleIndex}}_{\mathrm{empirical\ risk}}.
\end{aligned}
$$
\end{definition}


%Bounding the generalization gap provides assurance for the accuracy of a model when it is tested on new data. 
The generalization gap can be bounded by the expressiveness of the class of ML models. 
%One notion of expressiveness, 
Empirical Rademacher complexity (ERC) 
%-- an expressiveness notion -- 
measures how well the family of functions can fit random noise.

\begin{definition}[Empirical Rademacher complexity]\label{def:empirical-rademacher-complexity}
Let $\GeneralizationFunctionsLearners = \set{f: \GeneralizationDomain \to \bR}$ be a family of bounded functions and $\GeneralizationSampleName = \set{x_\GeneralizationSampleIndex}_{\GeneralizationSampleIndex = 1}^\GeneralizationSampleSize \subset \GeneralizationDomain$. The \textit{empirical Rademacher complexity} (ERC) of $\GeneralizationFunctionsLearners$ is
$$%\small
\begin{aligned}
\GeneralizationEmpiricalRademacherComplexity\pn{\GeneralizationFunctionsLearners} = \bE_\GeneralizationRademacherRandomVector\spn*{\sup_{f \in \GeneralizationFunctionsLearners} \frac{1}{\GeneralizationSampleSize}\sum_{\GeneralizationSampleIndex = 1}^\GeneralizationSampleSize \pn{\GeneralizationRademacherRandomVector}_\GeneralizationSampleIndex f\pn{x_\GeneralizationSampleIndex}},
\end{aligned}
$$
where the entries of $\vsigma \in \set{-1, 1}^\GeneralizationSampleSize$ are distributed such that $P\pn{\pn{\vsigma}_\GeneralizationSampleIndex = -1} = 1/2$ and $P\pn{\pn{\vsigma}_\GeneralizationSampleIndex = 1} = 1/2$.
\end{definition}

Using the ERC, a probabilistic bound for the generalization gap can be derived.

\begin{theorem}[ERC bounds generalization gap \cite{mohriFoundationsMachineLearning2012}]
\label{theorem:erc-bounds-generalization-gap}
Define the loss funciton $\LossFunctionName: \GeneralizationRange \times \GeneralizationRange \to \spn{0, 1}$, let $\GeneralizationFunctionsLearners = \set{f: \GeneralizationDomain \to \GeneralizationRange}$, and let $\GeneralizationDatasetDistribution$ be a probability distribution over $\GeneralizationDomain \times \GeneralizationRange$. Let $\GeneralizationSampleName = \set{\pn{x_\GeneralizationSampleIndex, y_\GeneralizationSampleIndex}}_{\GeneralizationSampleIndex = 1}^\GeneralizationSampleSize$ be a set of i.i.d. samples from $\GeneralizationDatasetDistribution$. For any $\GeneralizationGapBoundFailProbability > 0$, for all $f \in \GeneralizationFunctionsLearners$, the following bound holds with probability $1 - \GeneralizationGapBoundFailProbability$:
$$%\small
\begin{aligned}
\gGeneralizationGap\pn{f} \leq 2\GeneralizationEmpiricalRademacherComplexity\pn{\GeneralizationFunctionsDatumToLoss} + 3\sqrt{\frac{\ln\pn*{\frac{2}{\GeneralizationGapBoundFailProbability}}}{2\GeneralizationSampleSize}}, %\quad 
\end{aligned}
$$
where
$\GeneralizationFunctionsDatumToLoss = \set{\pn{x, y} \mapsto \LossFunctionName\pn{f\pn{x}, y}: f \in \GeneralizationFunctionsLearners}$.
\end{theorem}

Therefore, we only need to bound ERC. 
%With this result, to obtain a bound on the generalization gap only a bound on the ERC is needed.
The covering number is a celebrated tool to bound ERC. 
%Before quoting a bound on the ERC, we introduce the concept of a covering number, which gauges how easily a set can be approximated by another set.


{
\newcommand{\zSetOfInterest}{{\cZ}}
\newcommand{\zSetOfInterestItem}{{z}}
\newcommand{\zCoveringSet}{{\zSetOfInterest'}}
\newcommand{\zCoveringSetItem}{{\zSetOfInterestItem'}}
\newcommand{\zCoveringDistance}{{r}}
\begin{definition}[Covering number]
    \label{def:covering-number}
    The covering number $\GeneralizationCoveringNumber\pn{\zSetOfInterest, \zCoveringDistance, \UtilNormWithPlaceholder}$ of a set $\zSetOfInterest$ with respect to some norm $\UtilNormWithPlaceholder$ is the minimum cardinality of a set $\zCoveringSet$ such that, for any element of $\zSetOfInterest$, there is an element in $\zCoveringSet$ that is within a distance $\zCoveringDistance$ of it, i.e., $\GeneralizationCoveringNumber\pn{\zSetOfInterest, \zCoveringDistance, \UtilNormWithPlaceholder} = \min\set{\abs{\zCoveringSet}: \forall \zSetOfInterestItem \in \zSetOfInterest,\ \exists \zCoveringSetItem \in \zCoveringSet,\ \norm{\zSetOfInterestItem - \zCoveringSetItem} \leq \zCoveringDistance}$. 
    %\begin{align*}
    %    \GeneralizationCoveringNumber\pn{\zSetOfInterest, \zCoveringDistance, \UtilNormWithPlaceholder} = \min\set{\abs{\zCoveringSet}: \forall \zSetOfInterestItem \in \zSetOfInterest,\ \exists \zCoveringSetItem \in \zCoveringSet,\ \norm{\zSetOfInterestItem - \zCoveringSetItem} \leq \zCoveringDistance}
    %\end{align*}
\end{definition}
}

Now, we state an established bound of the ERC in terms of the covering number.

\begin{lemma}[Bounding ERC \cite{bartlettSpectrallynormalizedMarginBounds2017}]\label{lemma:bounding-erc}
Let $\GeneralizationFunctionsLearners = \set{f: \GeneralizationDomain \to \spn{-\GeneralizationBartlettBound,\GeneralizationBartlettBound}}$, and assume that there exists a function $\GeneralizationFunctionDatumToLossZero \in \GeneralizationFunctionsLearners$ such that $\GeneralizationFunctionDatumToLossZero\pn{x} = 0$ for all $x \in \GeneralizationDomain$. With $\norm{f}_\infty = \sup_{x \in \GeneralizationDomain} \abs{f\pn{x}}$, for any $\GeneralizationSampleName = \set{x_i}_{i = 1}^\GeneralizationSampleSize \subset \GeneralizationDomain$,
$$%\small
\begin{aligned}
\GeneralizationEmpiricalRademacherComplexity\pn{\GeneralizationFunctionsLearners} \leq \inf_{\alpha > 0} \pn*{\frac{4\alpha}{\sqrt{\GeneralizationSampleSize}} + \frac{12}{\GeneralizationSampleSize}\int_{\alpha}^{2\GeneralizationBartlettBound\sqrt{\GeneralizationSampleSize}} \sqrt{\ln\pn{\GeneralizationCoveringNumber\pn{\GeneralizationFunctionsLearners, r, \UtilNormWithPlaceholder_\infty}}} \wrt{r}}
\end{aligned}.
$$
\end{lemma}

%By combining Theorem~\ref{theorem:erc-bounds-generalization-gap} and Lemma~\ref{lemma:bounding-erc}, the generalization gap of ML models in $\GeneralizationFunctionsLearners$ have a probabilistic bound with respect to the dataset they are trained on and the loss function $\LossFunctionName$. 
%Following the work of \cite{pmlr-v235-karczewski24a} to prove a generalization gap bound for EGNNs \cite{satorras2021n}, in this work we prove a generalization gap bound for time-delayed aggregation graph neural networks trained for flocking control.




\subsection{Behavior cloning with fast-forwarding}
The space of training data for \PupilTDAGNN{} -- using IL with DAgger -- is the set of tuples $\pn{\gDAggerDatasetDatum}$, and the probability distribution on that space is induced by how DAgger generates these tuples. However, Definition~\ref{def:generalization-gap} requires that the %sampled training data be 
training samples are
independent, but the tuples generated by DAgger are not for two reasons.
% edit(Taos): There are exactly two reasons, so I do not say "two main reasons."
%
%
First, each tuple is computed using the previously generated tuple. At the beginning of each epoch, flocking is simulated for $\DAggerSimulationTimeStepsCount$ time steps where $\pn{\gExpertInput[\pn{t_{\DAggerSimulationTimeStepIndex + 1}}]}$ is computed by applying an acceleration to $\pn{\gExpertInput[\pn{t_\DAggerSimulationTimeStepIndex}]}$. In addition, $\PupilHistoryMatrix_\FlockAgentIndex\pn{t_\DAggerSimulationTimeStepIndex}$ is computed using $\set{\pn{\gExpertInput[\pn{t_{\DAggerSimulationTimeStepIndex - \PupilHistoryIndex + 1}}]}}_{\PupilHistoryIndex = 1}^\PupilHistoryCount$. The second reason is that the ML controller is trained on previously generated tuples, and it can influence what tuples are generated next by applying its acceleration.

The second reason cannot be addressed without fundamentally changing DAgger, and therefore, the probability distribution it induces on the space of training data. A primary motivation for the DAgger algorithm is allowing the ML controller to influence what training examples (e.g., tuples) are generated, and since the ML controller's weights depend on previously generated training examples, the training examples generated next cannot be independent. Therefore, we cannot analyze the generalization gap of ML controllers trained using IL with DAgger. To move forward with our analysis, we substitute DAgger with a variation of behavior cloning we call {\bf fast-forward behavior cloning} (\GeneralizationFastForwardBC{}).



\GeneralizationFastForwardBC{} addresses the second reason by only using the expert controller's acceleration during the flocking simulations run before each epoch.
The first reason is addressed by ensuring every tuple saved is derived from a newly sampled flock initial condition using fast-forwarding.
Instead of sampling an initial condition once at the beginning of each simulation, we sample an initial condition at each time step of the simulation. At time $t_\DAggerSimulationTimeStepIndex$, we sample $\pn{\gExpertInput[\pn{0}]}$, and then use the expert controller to advance it to $\pn{\gExpertInput[\pn{t_\DAggerSimulationTimeStepIndex}]}$ by sequentially applying the accelerations $\set{\gExpertName\pn{\gExpertInput[\pn{t_{\DAggerSimulationTimeStepIndex'}}]}}_{\DAggerSimulationTimeStepIndex' = 0}^{\DAggerSimulationTimeStepIndex - 1}$. Once the last acceleration is applied, we have also computed $\set{\PupilHistoryMatrix_\FlockAgentIndex\pn{t_\DAggerSimulationTimeStepIndex}}_{\FlockAgentIndex = 1}^\FlockAgentCount$. Finally, we save the tuple $\pn{\gDAggerDatasetDatum[\pn{t_\DAggerSimulationTimeStepIndex}]}$ to the training set.
%Note that fast-forwarding causes the flocking simulations take $\cO\pn{\DAggerSimulationTimeStepsCount^2}$ instead of $\cO\pn{\DAggerSimulationTimeStepsCount}$.



\GeneralizationFastForwardBC{} is independent of the ML controller's weights, so the dataset of tuples may be computed before training by running $\DAggerEpochCount$ flocking simulations. Note that the dataset is dependent on whether the ML controller uses sum aggregation or mean aggregation (see \secref{sec:methods:improving-tdagnn}) because the tuples contain $\PupilHistoryMatrix_\FlockAgentIndex$. The datasets are split into training and test sets, and each ML controller is trained on its respective training set. We train the ML controller on the entire training set each epoch so that the training set size does not vary as the ML controller trains, allowing us to compare generalization gaps at different epochs.



\subsection{\PupilTDAGNN{} reinterpretations and loss function}
The bound proved in \cite{pmlr-v235-karczewski24a} is for EGNN \cite{satorras2021n} 
%-- a Euclidean equivariant GNN model -- 
applied to graph-level tasks (e.g., graph classification). For these tasks, all \textit{invariant} node features after the last EGNN layer are aggregated, %(e.g., averaging), 
and the aggregated feature is fed to a final scoring model (e.g., an MLP) to produce a label for the graph. In flocking, each agent attempts to compute an acceleration for itself to match that of an expert controller, %would assign it, 
so the ML controller is performing a node-level task. 
%To follow the analysis given in \cite{pmlr-v235-karczewski24a}, 
We need to convert the node-level task into a graph-level task to adapt the analysis in \cite{pmlr-v235-karczewski24a}. 
%As shown in \eqref{eq:results:loss-function},
% Taos: squared error averaged over agents, not mean squared error averaged over agents.
We normally train ML controllers using squared error (SE) between their acceleration and the expert's acceleration averaged over all agents, resulting in a mean SE (MSE) loss. To convert to a graph-level task, we modify the output of each agent to be the %squared error 
SE instead of its acceleration, then define the scoring model $\PupilGraphScorer$ for the flock as
\begin{equation}%\small 
\label{eq:generalization-gap:scoring-model}
\begin{aligned}
    \PupilGraphScorer\pn{\mathrm{MSE}} = \PupilGraphScorerBias^2 + \mathrm{MSE} 
    %\text{where } & \mathrm{MSE} = \frac{1}{\FlockAgentCount} \sum_{\FlockAgentIndex = 1}^\FlockAgentCount \mathrm{SE}_\FlockAgentIndex \ \ 
    %\text{and}\ \   \mathrm{SE}_\FlockAgentIndex =  \norm{\gExpertName\pn{\gExpertInput} - \PupilName\pn{\PupilHistoryMatrix_\FlockAgentIndex}}^2
\end{aligned},
\end{equation}
where $\mathrm{MSE} = \frac{1}{\FlockAgentCount} \sum_{\FlockAgentIndex = 1}^\FlockAgentCount \mathrm{SE}_\FlockAgentIndex$, $\mathrm{SE}_\FlockAgentIndex =  \norm{\gExpertName\pn{\gExpertInput} - \PupilName\pn{\PupilHistoryMatrix_\FlockAgentIndex}}^2$,
and $\PupilGraphScorerBias$ is a trainable parameter in the scoring model. Note that $\mathrm{SE}_\FlockAgentIndex$ is \OrthogonalGroup{2} invariant, so MSE is an aggregation of invariant node features. The scoring model now computes the original loss given by \eqref{eq:results:loss-function}, so we can simply choose the identity function $\LossFunctionName\pn{y} = y$ as the loss function. However, to comply with the assumptions of Theorem~\ref{theorem:erc-bounds-generalization-gap}, we instead choose the loss function $\LossFunctionName\pn{y} = \min\set{1, y/\LossFunctionMSENormalizationConstant}$ for some $\LossFunctionMSENormalizationConstant > 0$.



\subsection{Bounding the generalization gap}
%The authors of 
Paper \cite{pmlr-v235-karczewski24a} uses MLPs to update the node features of EGNN, so our adaptation of their proof considers the MLP representations of $\PupilName$ given by \Lemmaref{lemma:results:conv1d-as-linear} or \ref{lemma:results:eqconv-as-linear}. The authors also assume that the input to EGNN is bounded, so we also assume a bound on the input of the MLP representation in Assumption~\ref{assumption:results:dagger-dataset-non-equivariant-bounded}.

\begin{definition}[Input of \PupilTDAGNN{} as an MLP]\label{def:results:input-for-mlp-non-equivariant}
The first convolutional layer $\ConvOneD_1: \bR^{\gConvChannelsIn[1] \times \gConvFeaturesIn[1]} \to \bR^{\gConvChannelsIn[2] \times \gConvFeaturesIn[2]}$ of $\PupilName$ has $\gConvChannelsIn[1]$ input channels that are row vectors of the form $\gConvInput[1] [\ConvChannelInIndex] = \spn{
\pn{\PupilHistoryMatrix_\FlockAgentIndex}_{\ConvChannelInIndex,1},\; \dots,\; \pn{\PupilHistoryMatrix_\FlockAgentIndex}_{\ConvChannelInIndex,\gConvFeaturesIn[1]}
}$. 
    % Let $\PupilAsMLPName$ be the MLP representation of $\PupilName$ given by Lemma~\ref{lemma:results:conv1d-as-linear}.
    The input $\gConvInputAsMLP[1]$ of $\PupilName$ as an MLP is the row-wise concatenation of the input channels 
    and $\vone_{\gConvFeaturesIn[2]}^\top$ to account for the bias term of $\ConvOneD_1$: $\gConvInputAsMLP[1]=\spn{\gConvInput[1] [1],\; 
            \dots,\; 
            \gConvInput[1] [\gConvChannelsIn],\; 
            \vone_{\gConvFeaturesIn[2]}^\top}$.
\end{definition}


\begin{definition}[Input of \OrthogonalGroup{2} equivariant \PupilTDAGNN{} as an MLP]\label{def:results:input-for-mlp-equivariant}
$\EqConv_1: \bR^{\gConvChannelsIn[1] \times \gConvFeaturesIn[1]} \to \bR^{\gConvChannelsIn[2] \times \gConvFeaturesIn[2]}$ of $\PupilEquivariantName$ has $\gConvChannelsIn[1]/2$ input channels that are two-row matrices of the form $\gConvInput[1] [\ConvChannelInIndex] = \spn{
            \EqConvChannel_{\ConvChannelInIndex,1}
            ,\; \dots
            ,\; \EqConvChannel_{\ConvChannelInIndex,\gConvFeaturesIn[1]}
}$.
% Let $\PupilAsMLPName$ be the MLP representation of $\PupilEquivariantName$ given by Lemma \ref{lemma:results:eqconv-as-linear}.
The input $\gConvInputAsMLP[1]$ of $\PupilName$ as an MLP is the row-wise concatenation of the input channels:
$
\gConvInputAsMLP[1]  = \spn{
    \gConvInput[1] [1]
    ,\; \dots
    ,\; \gConvInput[1] [\gConvChannelsIn[1]/2]
}
$.
\end{definition}

\begin{assumption}[\GeneralizationFastForwardBC{} datasets are bounded] 
\label{assumption:results:dagger-dataset-non-equivariant-bounded}
There exists $\DAggerDatumBound \geq 1$ such that for all tuples $\pn{\gDAggerDatasetDatum}$ in the \GeneralizationFastForwardBC{} dataset,
$$%\small
\begin{aligned}
        \max_\FlockAgentIndex\set{
            \norm{\gExpertName\pn{\gExpertInput}},
            \norm{
                \gConvInputAsMLP[1]
            }_F
        } \leq \DAggerDatumBound,
\end{aligned}
$$
where \gExpertName{} is the expert controller and $\gConvInputAsMLP[1]$ is defined in either Definition~\ref{def:results:input-for-mlp-non-equivariant} or Definition~\ref{def:results:input-for-mlp-equivariant}.
\end{assumption}


Now we state the generalization gap bound for the ML controllers with proof in the appendix.

\begin{proposition}[Generalization bound of TDAGNN]
Let $P$ be the probability distribution over tuples $\pn{\gDAggerDatasetDatum}$ induced by \GeneralizationFastForwardBC{}. Let 
%$\LossFunctionName: \hlpn{0, \infty} \to \spn{0, 1}$ defined as 
$\cL\pn{y} = \min\set{1, y/\LossFunctionMSENormalizationConstant}$ for $\LossFunctionMSENormalizationConstant > 0$ be the loss function. Let $\set{\PupilWeightMatrix_\PupilLayerIndex}_{\PupilLayerIndex = 1}^{\PupilLayerCount}$ be the weights of the MLP representation $\PupilAsMLPName$ of $\PupilName$ given by \Lemmaref{lemma:results:conv1d-as-linear} or \ref{lemma:results:eqconv-as-linear}, and let $\PupilGraphScorerBias$ of $\PupilGraphScorer$ in \eqref{eq:generalization-gap:scoring-model} be such that $\PupilGraphScorerBias \in \spn{0, \sqrt{\LossFunctionMSENormalizationConstant}}$. For any $\delta > 0$, with probability at least $1 - \delta$ over choosing a batch $\GeneralizationSampleName$ of $\GeneralizationSampleSize$ tuples sampled from $P$, the following bound holds:
\begin{equation}\label{eq:generalization-gap:bound}%\small
\begin{aligned}
    &\ \gGeneralizationGap\pn{\PupilGraphScorer} \leq \frac{8}{\GeneralizationSampleSize} +\\
    &\ 
    \frac{48\PupilWeightMatrixLargestDim}{\sqrt{\GeneralizationSampleSize}}\sqrt{\pn{3\PupilLayerCount + 1}\ln\pn{10\PupilLayerCount\DAggerDatumBound\PupilActivationLipshitzConstant^{\PupilLayerCount}\sqrt{\PupilWeightMatrixLargestDim\GeneralizationSampleSize\LossFunctionMSENormalizationConstant}} + \pn{2\PupilLayerCount + 3}\sum_{\PupilLayerIndex = 1}^{\PupilLayerCount} \ln\pn{\max\set{1,\norm{\PupilWeightMatrix_\PupilLayerIndex}_F}}} + \\
    &\ 3\sqrt{\frac{\ln\pn{\frac{2}{\delta}}}{2\GeneralizationSampleSize}}.
\end{aligned}
\end{equation}
\end{proposition}



\section{Experiments}\label{sec:experiments}
{\bf Controllers and hyperparameters:} We compare four ML controllers and the expert controller used to train them. The expert controller is \ExpertTanner{} from \eqref{eq:tanner:centralized} with the potential function from \eqref{eq:tanner:centralized:tolstaya}. In \cite{tolstayaLearningDecentralizedControllers2017b}, \ExpertTanner{}'s acceleration $\gExpertName\pn{\gExpertInput}$ is clamped so that $\norm{\gExpertName\pn{\gExpertInput}}_\infty \leq \ExpertAccelerationClampConstant$, but this breaks \ExpertTanner{}'s \OrthogonalGroup{2} equivariance. To retain \OrthogonalGroup{2} equivariance, we enforce $\norm{\gExpertName\pn{\gExpertInput}}_2 \leq \ExpertAccelerationClampConstant$.

\begin{table}[!ht]
    \centering\small
    \begin{tabular}{lrrr}
        \toprule
         \textbf{Controller} & \textbf{%One-hop 
         Aggregation} & \textbf{Activations (after conv. $\PupilLayerIndex$)} & \textbf{Conv. layers} \\
         \midrule
         \PupilTDAGNN{} & Sum & $\PupilActivationName_\PupilLayerIndex = \tanh$, $\PupilLayerIndex \in \set{1,\dots,\PupilLayerCount - 1}$ & $\ConvOneD$ \\
         \midrule
         \PupilTDAGNNTF{} & Sum & $\PupilActivationName_1 = \tanh$ & $\ConvOneD$ \\
         \midrule
         \PupilTDAGNNTFMu{} & Mean & $\PupilActivationName_1 = \tanh$ & $\ConvOneD$ \\
         \midrule
         \PupilETDAGNN{} & Mean & $\PupilActivationEquivariantName_0 = \PupilETDAGNNActivationScaleLog$, $\PupilActivationEquivariantName_1 = \PupilETDAGNNActivationScaleTanh$ & $\EqConv$ \\
         \bottomrule
    \end{tabular}%\vspace{-0.3cm}
\caption{%\small 
Summary of the architectural differences between the compared ML controllers for flocking.
The subscripts $\PupilLayerIndex$ of the activations in the Activations column indicate the convolutional layer that the activation is applied after. If the subscript $\PupilLayerIndex$ is 0, then the activation is applied before the first convolutional layer. If a subscript value is not listed, then no (or the identity) activation is applied after the corresponding convolutional layer.}
    \label{tab:results:models}
\end{table}


The first controller is \PupilTDAGNN{} from \cite{tolstayaLearningDecentralizedControllers2017b} and serves as a baseline. The second is \PupilTDAGNN{} with the  ``Activate once'' improvement described in \secref{sec:methods:improving-tdagnn} and is denoted by \PupilTDAGNNTF{} -- ``TF'' stands for ``tanh first.'' The third is \PupilTDAGNN{} with both the improvements described in \secref{sec:methods:improving-tdagnn} and is denoted by \PupilTDAGNNTFMu{}. The last is \PupilTDAGNNTFMu{} with the CNN \PupilEquivariantName{} described in \secref{sec:methods:equiv-model} and the equivariant activations proposed in \secref{sec:methods:equivariant-activations}, and it is denoted by \PupilETDAGNN{}. The architectural differences, including what activations are used after each convolutional layer, are summarized in Table~\ref{tab:results:models}.
%
%
%
All ML controllers use $\PupilLayerCount = 3$ convolutional layers mapping $\bR^{\gConvChannelsIn \times \gConvFeaturesIn}$ to $\bR^{\gConvChannelsIn[\PupilLayerIndex+1] \times \gConvFeaturesIn[\PupilLayerIndex+1]}$ for $\PupilLayerIndex \in \set{1,\dots,\PupilLayerCount}$. %respectively. 
Explictly writing their domains and ranges, $\pn{\gConvChannelsIn[1], \gConvFeaturesIn[1]} = \pn{6, \PupilHistoryCount}$, $\pn{\gConvChannelsIn[2], \gConvFeaturesIn[2]} = \pn{32, 1}$, $\pn{\gConvChannelsIn[3], \gConvFeaturesIn[3]} = \pn{32, 1}$, and $\pn{\gConvChannelsIn[4], \gConvFeaturesIn[4]} = \pn{2, 1}$. 
%From that, we see that the first convolutional layer has kernels in $\bR^{1 \times \PupilHistoryCount}$ with stride $\PupilHistoryCount$, and the following convolutional layers have kernels in $\bR$ with stride 1.
With these hyperparameters, Table \ref{tab:results:num-model-weights} shows each ML controller's trainable parameter count.


\begin{table}[!ht]
    \centering\small
    \begin{tabular}{lrrrr}
        \toprule
        %& \multicolumn{4}{c}{\textbf{Model}} \\
        %\cmidrule{2-5}
        & \textbf{\textit{\PupilTDAGNN{}}} & \textbf{\textit{\PupilTDAGNNTF{}}} & \textbf{\textit{\PupilTDAGNNTFMu{}}} & \textbf{\textit{\PupilETDAGNN{}}} \\
        \midrule
        \textbf{\#Weights} & 1,730 & 1,730 & 1,730 & 416 \\
        \bottomrule
    \end{tabular}%\vspace{-0.3cm}
\caption{%\small 
Number of trainable weights of the ML controllers for decentralized flocking.}
\label{tab:results:num-model-weights}
\end{table}


\noindent{\bf Training:} All ML controllers are trained for flocking control using IL with DAgger for $\DAggerEpochCount = 400$ epochs and then tested in flocking, leader following, and obstacle avoidance scenarios. 
%The ML controllers are trained using imitation learning with DAgger for $\DAggerEpochCount = 400$ epochs.
We initialize $\DAggerActionPropability_0 = \DAggerActionPropabilityInit=0.993$, and set $\DAggerActionPropability_\DAggerEpochIndex = \max\set{\DAggerActionPropability_{\DAggerEpochIndex - 1}\DAggerActionPropabilityInit, 0.5}$.
Tuples are generated and added to the training set by running flocking simulations for $\DAggerSimulationTimeStepsCount = 2/\TimeStep$ time steps with $\TimeStep = 10^{-2}$. The initial condition for each simulation is sampled from a dataset of initial conditions. The training set of tuples is capped at 10,000 examples, and the oldest examples are discarded first after reaching the cap. After the flocking simulation of each epoch, we sample 200 batches of 20 tuples with replacement from the training set. For each batch, we compute the loss averaged over the batch and update the ML controller's weights.
The loss function for one tuple is
\begin{equation}%\small
\label{eq:results:loss-function} 
\begin{aligned}
\LossFunctionName\pn{\gDAggerDatasetDatum} = \frac{1}{\FlockAgentCount} \sum_{\FlockAgentIndex = 1}^\FlockAgentCount \norm{\gExpertName\pn{\gExpertInput} - \PupilName\pn{\PupilHistoryMatrix_\FlockAgentIndex}}^2
\end{aligned}.
\end{equation}
The trainable parameters are initialized using Xavier uniform initialization with gain 1, and optimized using the Adam optimizer with learning rate $5 \times 10^{-5}$, $\beta_1 = 0.9$, and $\beta_2 = 0.999$.


%\paragraph{Flock initial conditions dataset:} 
\noindent{\bf Initial conditions:}
The dataset of flock initial conditions is randomly generated following the procedure described in \cite{tolstayaLearningDecentralizedControllers2017b}. We refer to this dataset as the \textit{RandomDisk} dataset. Each initial condition composes $\FlockAgentCount$ agents whose positions are distributed in a 2D disk of radius $\sqrt{\FlockAgentCount}$. Having radius $\sqrt{\FlockAgentCount}$ implies that the ratio of the number of agents to the disk's area is the constant $\pi$. The agents are placed in the disk uniformly randomly such that three conditions are met: the agents are not too close (for $\FlockAgentIndexTwo \in \FlockAgentNeighborhood_\FlockAgentIndex\pn{0}$,  $\FlockAgentMinDist \leq \norm{\FlockAgentRelativePos} \leq \FlockAgentCommRadius$); 
%where $\RadiusMin \in \pn{0, \FlockAgentCommRadius}$); 
the agents have enough neighbors ($\abs{\FlockAgentNeighborhood_\FlockAgentIndex\pn{0}} \geq \FlockCommunicationGraphMinDegree \geq 0$); and, the flock's communication graph is connected. The agents' velocities are initialized to $\FlockVelMatrix\pn{0} = \FlockAgentVelocityInitial + \FlockAgentVelocityBias\vone_\FlockAgentCount^\top$ where the entries of $\FlockAgentVelocityInitial$ and $\FlockAgentVelocityBias$ are uniformly randomly sampled from $\spn{-\FlockAgentVelocityMaxMaxNorm, \FlockAgentVelocityMaxMaxNorm}$ for $\FlockAgentVelocityMaxMaxNorm \in \hlpn{0, \infty}$. \Figref{fig:dataset:randomdisk} shows example initial conditions of this dataset. Following \cite{tolstayaLearningDecentralizedControllers2017b}, %in our experiments 
we choose the RandomDisk dataset parameters as $\FlockAgentCount = 100$, $\FlockAgentMinDist = 0.1$, $\FlockAgentCommRadius = 1$, $\FlockCommunicationGraphMinDegree = 2$, and $\FlockAgentVelocityMaxMaxNorm = 3$.



\begin{figure}[!ht]
    \centering
    \begin{subfigure}[t]{.29\textwidth}
        \centering
        \includegraphics[width=\textwidth]{figsDatasetsRDDdata_0.pdf}
        \label{fig:dataset:randomdisk:0}
    \end{subfigure}
    \hspace{1em}
    \begin{subfigure}[t]{.29\textwidth}
        \centering
        \includegraphics[width=\textwidth]{figsDatasetsRDDdata_1.pdf}
        \label{fig:dataset:randomdisk:1}
    \end{subfigure}
    \hspace{1em}
    \begin{subfigure}[t]{.29\textwidth}
        \centering
        \includegraphics[width=\textwidth]{figsDatasetsRDDdata_3.pdf}
        \label{fig:dataset:randomdisk:2}
    \end{subfigure}%\vspace{-0.45cm}
\caption{%\small 
Examples from the RandomDisk dataset of flock initial conditions. There are $\FlockAgentCount = 100$ agents (orange dots) with at least $\deg_{\min} = 2$ neighbors (indicated by light blue edges connecting them). The distance between an agent and its neighbors is between $\FlockAgentMinDist = 0.1$ and $\FlockAgentCommRadius = 1$. The agents' velocities (dark blue arrows) have magnitudes no larger than $2\FlockAgentVelocityMaxMaxNorm = 6$. 
%This dataset provides flock initial conditions for training the decentralized machine learning flocking controllers using imitation learning.
}
\label{fig:dataset:randomdisk}
\end{figure}


%\paragraph{Flocking validation metrics:} 
\noindent{\bf Metrics:} We quantify the performance of the ML controllers using the two metrics:
%The performance of ML controllers controlling the flock for $\DAggerSimulationTimeStepsCount$ time steps is quantified by two validation metrics:
\begin{itemize}
\item \textbf{Velocity variance:} The variance of velocities is $\var\pn{\FlockAgentVel} = \frac{1}{\FlockAgentCount}\sum_{\FlockAgentIndex = 1}^\FlockAgentCount \norm{\FlockAgentVel_\FlockAgentIndex - \mean\pn{\FlockAgentVel}}^2$ where $\mean\pn{\FlockAgentVel} := \frac{1}{\FlockAgentCount}\sum_{\FlockAgentIndex = 1}^\FlockAgentCount \FlockAgentVel_\FlockAgentIndex$. 
%        \begin{align*}
%            \var\pn{\FlockAgentVel} = \frac{1}{\FlockAgentCount}\sum_{\FlockAgentIndex = 1}^\FlockAgentCount \norm{\FlockAgentVel_\FlockAgentIndex - \mean\pn{\FlockAgentVel}}^2 \quad \text{where } \mean\pn{\FlockAgentVel} := \frac{1}{\FlockAgentCount}\sum_{\FlockAgentIndex = 1}^\FlockAgentCount \FlockAgentVel_\FlockAgentIndex,
%        \end{align*}
Lower variance implies the flock is closer to a limiting velocity of asymptotic flocking.

\item \textbf{Mean acceleration norm:} The mean acceleration norm of the agents is $\frac{1}{\FlockAgentCount} \sum_{\FlockAgentIndex = 1}^\FlockAgentCount \norm{\FlockAgentAccel_\FlockAgentIndex\pn{t_\DAggerSimulationTimeStepIndex}}$. 
        %\begin{align*}
        %    \frac{1}{\FlockAgentCount} \sum_{\FlockAgentIndex = 1}^\FlockAgentCount \norm{\FlockAgentAccel_\FlockAgentIndex\pn{t_\DAggerSimulationTimeStepIndex}}.
        %\end{align*}
A lower mean acceleration norm means the controller uses a smaller control input to achieve its objective. This metric measures the controller's efficiency because less acceleration implies less energy expenditure for the flock.
\end{itemize}


\subsection{Flocking}\label{sec:results:flocking}
%{\bf Setup:} 
\newcommand{\RiemannSumVelVar}{IVV}
\newcommand{\RiemannSumMeanAccelNorm}{IMAN}

In this task, we observe significant performance gaps
between \ExpertTanner{}, the ML controllers using sum aggregation (\PupilTDAGNN{} and \PupilTDAGNNTFMu{}), and the ML controllers using mean aggregation (\PupilTDAGNNTFMu{} and \PupilETDAGNN{}).
When presenting these results, we will compare these groups, and then compare the controllers within these groups when necessary. Keep in mind that \PupilETDAGNN{} has 75\% fewer trainable weights than other controllers (see Table~\ref{tab:results:num-model-weights}).


%Starting with the metrics of ML controllers for training, 
For training, \figref{fig:results:flocking:randomdisk:by-epoch} shows the median Integral of the Velocity Variance (\RiemannSumVelVar{}) % by Riemann sum, Taos meant that the integral is approximated with the midpoint rule, so better to just say interal.
and the median Integral of the Mean Acceleration Norm (\RiemannSumMeanAccelNorm{}) on validation set of the RandomDisk dataset with $\FlockAgentCount = 100$ and $\TimeStep = 10^{-2}$. The mean-aggregation ML controllers achieve a lower median \RiemannSumVelVar{} and \RiemannSumMeanAccelNorm{} by epoch 80 than the sum-aggregation ML controllers do by epoch 400. Furthermore, at epoch 400, the \RiemannSumVelVar{} and \RiemannSumMeanAccelNorm{} IQRs 
% edit(Taos): What does not overlap if not the IQRs?
of the best sum-aggregation ML controllers and worst mean-aggregation ML controllers do not overlap.

\begin{figure}[!ht]
    \centering
    \begin{subfigure}[T]{\textwidth}
        \centering
        \includegraphics[width=0.8\textwidth]{figsERRDDTS98yi04w0.legend.pdf}
    \end{subfigure} \\
    \begin{subfigure}[T]{.49\textwidth}
        \centering
        \includegraphics[width=.7\textwidth]{figsERRDDTS98yi04w0.0.pdf}
    \end{subfigure}
    \begin{subfigure}[T]{.49\textwidth}
        \centering
        \includegraphics[width=.7\textwidth]{figsERRDDTS98yi04w0.1.pdf}
    \end{subfigure}
    %\vspace{-0.35cm}
\caption{%\small 
Performance of ML controllers in flocking as they train. They are evaluated on the RandomDisk validation set with $100$ agents. Each simulation is run for $\DAggerSimulationTimeStepsCount = 2/\TimeStep$ time steps where $\TimeStep = 10^{-2}$.
The curves show the median values of the respective metrics' integrals and the colored areas show their corresponding interquartile ranges.}
\label{fig:results:flocking:randomdisk:by-epoch}%\vspace{-0.15cm}
\end{figure}


%Looking at the sum-aggregation ML controllers only, 
% edit(Taos): wording the reference to these ML controllers this way ("For ML controllers with the sum-aggregation function,") is fine, but then we should consistently reference them that way. To avoid changing it everywhere, I change it back to the following:
For the sum-aggregation ML controllers,
%ML controllers, 
\PupilTDAGNNTF{} has a lower median \RiemannSumVelVar{} than \PupilTDAGNN{} for all epochs. 
\PupilTDAGNNTF{} has a higher \RiemannSumMeanAccelNorm{} than \PupilTDAGNN{} at epoch 40, and is over double compared to \PupilTDAGNN{} from epoch 80 to epoch 160. It takes until epoch 320 for it to become lower than \PupilTDAGNN{}.
%
%
For the mean-aggregation ML controllers, at epoch 40, \PupilTDAGNNTFMu{} has a median \RiemannSumVelVar{} less than half that of \PupilETDAGNN{}, but \PupilETDAGNN{} matches \PupilTDAGNNTFMu{} by epoch 120.
By epoch 400, the \RiemannSumVelVar{} of \PupilTDAGNNTFMu{} is only larger than \PupilETDAGNN{} by a few hundredths.
At epoch 40, \PupilTDAGNNTFMu{} also has a lower \RiemannSumMeanAccelNorm{} than \PupilETDAGNN{}, but by epoch 80 and for the rest of the epochs, the \RiemannSumVelVar{} of \PupilTDAGNNTFMu{} and \PupilETDAGNN{} remain within that gap, both increasing at the same rate.



After training, we test each ML controller's ability to achieve asymptotic flocking with separation on 50 RandomDisk initial conditions not used for training or hyperparameter tuning.
We use $\FlockAgentCount \in \set{50, 100, 200, 400}$ and $\TimeStep = 10^{-3}$.
\Figref{fig:results:flocking:randomdisk:vel-var} shows the median velocity variance over time.
For all $\FlockAgentCount$, the mean-aggregation controllers reduce the velocity variance from about 4 to below 0.2 faster than the sum-aggregation controllers.
When $\FlockAgentCount \leq 100$, the mean-aggregation ML controllers also reach a lower velocity variance at the last time step, but when $\FlockAgentCount \geq 200$, the sum-aggregation ML controllers reach a lower velocity variance.
Surprisingly, when $\FlockAgentCount = 50$, the velocity variances of \PupilTDAGNNTF{} and the mean-aggregation ML controllers are below that of \ExpertTanner{} for the majority of the simulation.
The exception is from times $t_\DAggerSimulationTimeStepIndex\TimeStep \in \spn{0.1, 0.3}$ where \PupilTDAGNNTF{} has a larger velocity variance than \ExpertTanner{}.
When $\FlockAgentCount = 100$, \PupilETDAGNN{} reaches a lower velocity variance than \ExpertTanner{} at the end of the simulation.
%
%
\Figref{fig:results:flocking:randomdisk:accel} shows the median mean acceleration norm over time.
The sum-aggregation ML controllers' median mean acceleration norm at the last time step matches or is a few hundredths smaller than the mean-aggregation ML controllers.

\begin{figure}[!ht]
    \centering
    \begin{subfigure}[T]{\textwidth}
        \centering
        \includegraphics[width=.95\textwidth]{figsERRDDBTVelocityVarc2b008pp.legend.pdf}
    \end{subfigure} \\
    \begin{subfigure}[T]{.23\textwidth}
        \centering
        \hspace{-2.8em}
        \includegraphics[width=1.21\textwidth]{figsERRDDBTVelocityVarc2b008pp__0.001__50.pdf}
    \end{subfigure}
    \begin{subfigure}[T]{.23\textwidth}
        \centering
        \includegraphics[width=\textwidth]{figsERRDDBTVelocityVarc2b008pp__0.001__100.pdf}
    \end{subfigure}
    \begin{subfigure}[T]{.23\textwidth}
        \centering
        \includegraphics[width=\textwidth]{figsERRDDBTVelocityVarc2b008pp__0.001__200.pdf}
    \end{subfigure}
    \begin{subfigure}[T]{.23\textwidth}
        \centering
        \includegraphics[width=\textwidth]{figsERRDDBTVelocityVarc2b008pp__0.001__400.pdf}
    \end{subfigure}
    %\vspace{-0.3cm}
\caption{%\small 
Velocity variance of the controllers in flocking over the simulation time with time step size $\TimeStep = 10^{-3}$.
They are evaluated on the RandomDisk test set with the number of agents $\FlockAgentCount \in \set{50, 100, 200, 400}$. The lines show the median metric values and the colored areas show the corresponding interquartile ranges.}
    \label{fig:results:flocking:randomdisk:vel-var}\vspace{-0.15cm}
\end{figure}



\begin{figure}[!ht]
    \centering
    \begin{subfigure}[T]{\textwidth}
        \centering
        \includegraphics[width=.95\textwidth]{figsERRDDBTMeanAccelNormmmgrlo2f.legend.pdf}
    \end{subfigure} \\
    \begin{subfigure}[T]{.23\textwidth}
        \centering
        \hspace{-2.8em}
        \includegraphics[width=1.21\textwidth]{figsERRDDBTMeanAccelNormmmgrlo2f__0.001__50.pdf}
    \end{subfigure}
    \begin{subfigure}[T]{.23\textwidth}
        \centering
        \includegraphics[width=\textwidth]{figsERRDDBTMeanAccelNormmmgrlo2f__0.001__100.pdf}
    \end{subfigure}
    \begin{subfigure}[T]{.23\textwidth}
        \centering
        \includegraphics[width=\textwidth]{figsERRDDBTMeanAccelNormmmgrlo2f__0.001__200.pdf}
    \end{subfigure}
    \begin{subfigure}[T]{.23\textwidth}
        \centering
        \includegraphics[width=\textwidth]{figsERRDDBTMeanAccelNormmmgrlo2f__0.001__400.pdf}
    \end{subfigure}
    %\vspace{-0.3cm}
\caption{%\small 
Mean acceleration norm of the controllers in flocking over the simulation time with time step size $\TimeStep$. They are evaluated on the RandomDisk test set with the number of agents $\FlockAgentCount \in \set{50, 100, 200, 400}$. The lines show the median metric values and the colored areas show the corresponding interquartile ranges.}
\label{fig:results:flocking:randomdisk:accel}\vspace{-0.15cm}
\end{figure}


%In summary of these results, 
In summary, there is no best-performing ML controller for all tested flock sizes. For all tests, the mean-aggregation ML controllers reduce the velocity variance of the flock faster than the sum-aggregation ones when the variance is large. 
%(e.g., near the beginning of the simulation).
The ML controller that reaches the smallest velocity variance by the last time step depends on $\FlockAgentCount$. When $\FlockAgentCount \leq 100$, \PupilETDAGNN{} reaches the lowest %velocity 
variance at the last time step %compared to the 
than other ML controllers.
For larger $\FlockAgentCount$, \PupilTDAGNNTF{} reaches the smallest velocity variance. The ML controllers have approximately the same mean acceleration norm.
%, particularly for larger flock sizes.
% Based on the performance of the ML controllers for the flocking task, if a fine time step size (e.g., $\TimeStep = 10^{-3}$) is tractable, we recommend using \PupilETDAGNN{} when $\FlockAgentCount \leq 200$ and \PupilTDAGNNTF{} otherwise.
% If a coarser time step size is required (e.g., $\TimeStep \sim 10^{-2}$), we recommend \PupilETDAGNN{}.
Animations of flocking are available on GitHub.\footnote{Flocking animations: \url{github.com/Utah-Math-Data-Science/Equivariant-Decentralized-Controllers/tree/main/misc/animations/flocking}}


\subsection{Leader following}\label{sec:results:leader-following}

{
\newcommand{\zAgentLeaderVelocity}{{\vv_{\mathrm{ldr}}}}

%\paragraph{\bf Setup} 
%Leader following is a technique for controlling the trajectory of the flock.
In leader following, the leader agents are selected from the flock and instructed to move along some predefined path (e.g., a line), and the %remaining
other agents are followers. 
%and are expected to follow the leaders. 
%If the followers successfully follow the leaders, then the flock's trajectory is controlled by the leader's trajectory.
%
%
%We test the ability of the ML controllers trained 
We test the ability of ML controllers, already trained for flocking, in conducting leader following using 50 RandomDisk initial conditions. 
%not used for training or hyperparameter tuning. 
Two agents from each initial condition are randomly selected to be leaders of the flock and the leaders' velocities are set equal. To prevent the followers from changing the leaders' trajectories, the leaders ignore all messages from the followers, i.e., the leaders only have directed edges from them to other agents. %(followers and other leaders). 
Consequently, the leaders do not pass on summaries of the messages they receive to their neighbors.
In addition, the ML controllers only control the followers, and since the ML controllers are trained to maintain communication graph connectivity, the controllers are compelled to have the followers match the velocity of the leaders.
%
The leader following simulations are run for $\DAggerSimulationTimeStepsCount = 3/\TimeStep$ time steps. In addition to the flocking validation metrics, leader following adds this validation metric:
\begin{itemize}
\item \textbf{Mean leader velocity distance (MLVD):} Let $\zAgentLeaderVelocity$ be the velocity of the leaders in the flock. %which will be 
%i.e., the limiting velocity of the flock. 
We measure how close the flock is to the limiting velocity with $\frac{1}{\FlockAgentCount}\sum_{\FlockAgentIndex = 1}^\FlockAgentCount \norm{\FlockAgentVel_\FlockAgentIndex\pn{t_\DAggerSimulationTimeStepIndex} - \zAgentLeaderVelocity}$. 
\end{itemize}

\Figref{fig:results:leader-following:randomdisk:by-time} shows the median MLVD and median mean acceleration norm over time. By the last time step, \PupilTDAGNNTFMu{} has the lowest median MLVD. 
%with \PupilTDAGNNTF{}'s median a few hundredths larger.
The medians of \PupilETDAGNN{} and \PupilTDAGNN{} are nearly double and triple that of \PupilTDAGNNTFMu{}.
%
%
The ML controllers have approximately the same mean acceleration norm.
%The ML controllers' mean acceleration norms at the last time step are within a few hundredths of each other.
%
%
Based on the performance of the ML controllers for the leader following task, we recommend \PupilTDAGNNTFMu{} for the best performance; however, \PupilETDAGNN{} provides comparable performance with 75\% fewer trainable parameters.
%
%
Animations of leader following are available on GitHub.\footnote{Leader following animations: \url{github.com/Utah-Math-Data-Science/Equivariant-Decentralized-Controllers/tree/main/misc/animations/leader_following}}

\begin{figure}[!ht]
    \centering
    \begin{subfigure}[T]{\textwidth}
        \centering
        \includegraphics[width=0.95\textwidth]{figsERLFBTg23zc0a7.legend.pdf}
    \end{subfigure} \\
    \begin{subfigure}[T]{.49\textwidth}
        \centering
        \includegraphics[width=.9\textwidth]{figsERLFBTg23zc0a7__0.001__MeanLeaderVelDist.pdf}
    \end{subfigure}
    \begin{subfigure}[T]{.49\textwidth}
        \centering
        \includegraphics[width=.9\textwidth]{figsERLFBTg23zc0a7__0.001__MeanAccelNorm.pdf}
    \end{subfigure}
    %\vspace{-0.3cm}
\caption{%\small 
Performance of controllers in leader following at each time step of the simulation. They are evaluated on the RandomDisk test set with $\FlockAgentCount = 100$ agents where two agents are leaders. 
%The lines show the median metric values and the colored areas show the corresponding interquartile ranges.
}
\label{fig:results:leader-following:randomdisk:by-time}\vspace{-0.15cm}
\end{figure}
}



\subsection{Obstacle avoidance}
When a flock is intercepting an obstacle, the ML controllers should have the flock circumnavigate it. For centralized flocking controllers, the primary requirement is that the agents do not collide with the obstacle. The centralized flocking controllers discussed can guarantee agent separation, which is readily extendable to obstacle collision avoidance if each agent can compute its distance from the obstacle. Decentralized flocking controllers, however, face the significant challenge of managing communication graph connectivity. An obstacle's diameter can be larger than the communication radius $\FlockAgentCommRadius$ of the agents, preventing agents on opposite sides of the obstacle from communicating. Decentralized flocking controllers have two options for successful circumnavigation -- ensure that all agents move around the obstacle in the same direction, or devise a scheme that will guarantee communication graph connectivity is restored if groups of agents go around in different directions. We present a technique for helping decentralized flocking controllers conduct obstacle avoidance inspired by Fig.~2 of \cite{reynoldsFlocksHerdsSchools1987}. The technique is tested using disk-shaped obstacles with diameters up to nearly half of the flock's diameter. 
%Future work includes extending this technique to larger obstacle sizes and different obstacle shapes.


%\paragraph{Setup}

{
\newcommand{\zFlockAgentExtremeIndex}{{\FlockAgentIndex_*}}
\newcommand{\zFlockAgentExtremeIndexTwo}{{\FlockAgentIndexTwo_*}}
\newcommand{\zObstacleSideCount}{{s}}

%First, we describe the obstacle avoidance dataset, and then our obstacle avoidance technique.
%We describe the obstacle avoidance dataset and our obstacle avoidance technique. 
The obstacle avoidance dataset is built upon the RandomDisk dataset. For each initial condition, we construct a disk-shaped obstacle as follows. First, we place the center of a regular polygon with $s$ sides of length $\FlockAgentMinDist$ relative to the flock such that two conditions are met:
\begin{enumerate}
\item \textbf{Polygon is in the middle:} Let $\FlockAgentPos_\zFlockAgentExtremeIndex$ and $\FlockAgentPos_\zFlockAgentExtremeIndexTwo$ be agents whose distance is the diameter of the flock: $\norm{\vr_{\zFlockAgentExtremeIndex\zFlockAgentExtremeIndexTwo}} = \max_{\FlockAgentIndex,\FlockAgentIndexTwo} \norm{\FlockAgentRelativePos}$. The polygon's center is on the line passing through $\mean\set{\FlockAgentPos_\zFlockAgentExtremeIndex, \FlockAgentPos_\zFlockAgentExtremeIndexTwo}$ that is perpendicular to the line passing through $\FlockAgentPos_\zFlockAgentExtremeIndex$ and $\FlockAgentPos_\zFlockAgentExtremeIndexTwo$.

\item \textbf{Polygon is in front:} At time zero, the minimum distance between the polygon center and the flock agents is greater than the polygon's circumradius plus $\FlockAgentCommRadius$.
\end{enumerate}

The obstacle is the circumscribed circle of the polygon.
Next, the agents need a mechanism to compute their position relative to the obstacle. In our simulation, we utilize the existing communication mechanism
by placing additional \textit{obstacle agents} on the vertices of the polygon. The obstacle agents only send their position to the flock agents. Moreover, the obstacle agents do not receive any messages from the flock agents; that is, the obstacle agents only have directed edges from them to the flock agents in the communication graph.


The final step is to randomly select two agents in the flock to be leaders (see \secref{sec:results:leader-following}). Without leaders, the flock will not attempt to circumnavigate the obstacle; instead, the flock will turn around or simply halt in front of the obstacle. Leaders force the flock to continue past the obstacle. Leaders do not receive any messages from the obstacle agents. The leaders' velocity is fixed, so we select leaders that will always be at least $\RadiusMin$ away from the obstacle boundary. We set the initial velocity of leaders and followers to the unit vector pointing from $\mean\set{\FlockAgentPos_\zFlockAgentExtremeIndex, \FlockAgentPos_\zFlockAgentExtremeIndexTwo}$ to the center of the obstacle. Examples are shown in \Figref{fig:dataset:obstacle-avoidance:randomdisk}.
}

\begin{figure}[!ht]
    \centering
    \begin{subfigure}[T]{.3\textwidth}
        \centering
        \includegraphics[width=\textwidth]{figsDatasetsOAobs12.pdf}
        \label{fig:dataset:obstacle-avoidance:randomdisk:0}
    \end{subfigure}
    \hspace{1em}
    \begin{subfigure}[T]{.3\textwidth}
        \centering
        \includegraphics[width=\textwidth]{figsDatasetsOAobs48.pdf}
        \label{fig:dataset:obstacle-avoidance:randomdisk:1}
    \end{subfigure}
    \hspace{1em}
    \begin{subfigure}[T]{.3\textwidth}
        \centering
        \includegraphics[width=\textwidth]{figsDatasetsOAobs96.pdf}
        \label{fig:dataset:obstacle-avoidance:randomdisk:2}
    \end{subfigure}%\vspace{-0.5cm}
\caption{%\small 
Examples from the Obstacle Avoidance RandomDisk dataset of initial conditions of agents with $\FlockCommunicationGraphMinDegree = 2$, $\FlockAgentMinDist = 0.1$, $\FlockAgentCommRadius = 1$, initial velocity (dark blue arrows) with magnitude 1, $98$ agent followers (yellow), two leaders (green), and obstacles (purple) of perimeter 12, 48, and 96 (all multiplied by $\FlockAgentMinDist$).}
    \label{fig:dataset:obstacle-avoidance:randomdisk}\vspace{-0.15cm}
\end{figure}

Instead of terminating the obstacle avoidance simulations after some fixed number of time steps $\DAggerSimulationTimeStepsCount$, the simulations are run until any of the following three termination conditions are met:
\begin{enumerate}
\item \textbf{Disconnected communication graph:}
First, check if the minimum distance between the flock 
and the obstacle is greater than $\FlockAgentCommRadius$, implying that the flock is unaware of the obstacle. If so, terminate the simulation if the communication graph containing leaders and followers is disconnected. Otherwise, terminate the simulation if the graph containing leaders, followers, and obstacle agents is disconnected.


\item \textbf{Collision:} Terminate the simulation if any flock agents are closer than $\RadiusMin$ to each other or any obstacle agents, or the distance of any flock agent to the center of the obstacle is less than or equal to the radius of the obstacle. 
%The second condition is often redundant, but due to time discretization error in the simulation, agents may sometimes ``jump'' inside the obstacle.
    {
    \newcommand{\zTimeStepsAfterObstaclePassed}{{\DAggerSimulationTimeStepsCount_{\mathrm{passed}}}}
\item \textbf{Obstacle passed:} Terminate the simulation if the minimum distance between the flock 
%agents and the 
and obstacle agents has been less than 
%distance 
$\FlockAgentCommRadius$ for at least one %time 
step, and that distance has been larger than $\FlockAgentCommRadius$ for the past $\zTimeStepsAfterObstaclePassed \geq 1$ time steps. This termination condition indicates successful obstacle avoidance.
    }
\end{enumerate}
The obstacle avoidance 
validation metric is the fraction of terminations not due to passing the obstacle.
%simulations terminated for a reason that is not obstacle passed. 
%Obstacle Passed, 
%the fraction terminated due to Disconnected Communication Graph, and the fraction terminated due to Collision.



Now, we describe our obstacle avoidance technique.
%which leverages the position of a follower relative to the obstacle agents.
At a high level, we have the followers orbit the obstacle, and then we use the leaders to draw the followers away from the obstacle and terminate the followers' orbits. The technique influences the acceleration of a follower $\FlockAgentIndex$ by formulating the relative velocity $-\FlockAgentRelativeVel = \FlockAgentRelativeVel[_{\FlockAgentIndexTwo\FlockAgentIndex}]$ between it and an obstacle agent $\FlockAgentIndexTwo$. Notice that we wrote the relative velocity vector so that it is rooted at the velocity of the follower $\FlockAgentIndex$.
This lets us work from the reference frame of the follower.
% We will describe how the orbit about the obstacle is initiated and then how the leaders draw the followers away from the obstacle.


{
\newcommand{\zRotationMatrix}{{\UtilRotationMatrix}}
\newcommand{\zAngle}{{\theta}}
\newcommand{\zLinearDiscriminantName}{{\gamma}}
\newcommand{\zFlockAgentVelocity}{\vv}
\newcommand{\zVelocityNeighborAverage}{{\overline{\zFlockAgentVelocity}}}
\newcommand{\zRelativeVelocityMagnitude}{{\alpha_1}}
\newcommand{\zRelativeVelocityGravityBreak}{{\alpha_2}}
\newcommand{\zRelativeVelocityObstacleDodgeAngle}{{\alpha_\zAngle}}

The technique defines the interaction between a follower agent and a single obstacle agent.
As we will later show empirically, all these pairwise interactions culminate in the flock's obstacle-avoidance capability.
To implement our technique, we define a parametrized linear discriminant to make two classifications about a follower: (1) whether the follower is moving towards or away from an obstacle agent, 
and (2) whether the obstacle agent is on the left or right side of the follower's heading. The parameterized linear discriminant is
\begin{align}\label{eq:results:obstacle-avoidance-linear-discriminant}
\zLinearDiscriminantName\pn{\FlockAgentRelativePos[], \zFlockAgentVelocity, \zAngle} = \frac{\FlockAgentRelativePos[]^\top}{\norm{\FlockAgentRelativePos[]}}\zRotationMatrix\pn{\zAngle}\frac{\zFlockAgentVelocity}{\norm{\zFlockAgentVelocity}},
\end{align}
for %$\FlockAgentRelativePos[] \neq \vzero \neq \zFlockAgentVelocity$ 
$\FlockAgentRelativePos[],\zFlockAgentVelocity \neq \vzero$ and 
%counter-clockwise 
rotation matrix $\UtilRotationMatrix\pn{\cdot}$. 
%$\UtilRotationMatrix\pn{\cdot} \in \SpecialOrthogonalGroup{2}$. 
Fixing $\FlockAgentVel_\FlockAgentIndex$, the linear discriminant $\FlockAgentRelativePos[_{\FlockAgentIndexTwo\FlockAgentIndex}] \mapsto \zLinearDiscriminantName\pn{\FlockAgentRelativePos[_{\FlockAgentIndexTwo\FlockAgentIndex}], \FlockAgentVel_\FlockAgentIndex, 0}$ classifies whether the follower is moving toward (positive value) or away (negative value) from the obstacle agent. Moreover, the linear discriminant $\FlockAgentRelativePos[_{\FlockAgentIndexTwo\FlockAgentIndex}] \mapsto \zLinearDiscriminantName\pn{\FlockAgentRelativePos[_{\FlockAgentIndexTwo\FlockAgentIndex}], \FlockAgentVel_\FlockAgentIndex, \frac{\pi}{2}}$ classifies whether the obstacle agent is to the left (positive value) or right (negative value) of the follower's heading.
%
%
These linear discriminants are assembled to determine how the follower should accelerate when receiving the position of an obstacle agent. When the follower accelerates, $\FlockAgentVel_\FlockAgentIndex$ changes, and so do the outputs of the linear discriminants. To reduce the linear discriminants sensitivity to $\FlockAgentVel_\FlockAgentIndex$, we replace $\FlockAgentVel_\FlockAgentIndex$ with the mean velocity of the follower and its neighbors: $\zVelocityNeighborAverage_\FlockAgentIndex\pn{t} = \mean\set{\FlockAgentVel_j\pn{t}: j \in \FlockAgentNeighborhood_\FlockAgentIndex\pn{t} \cup \set{i}}$.
% Now, as $\FlockAgentVel_\FlockAgentIndex$ changes, the outputs of the linear discriminates do not change too quickly.


Finally, our technique formulates the relative velocity as
\begin{align}
    \label{eq:results:obstacle-avoidance:relative-velocity}
    -\FlockAgentRelativeVel\pn{t} & = \begin{cases}
        \zRelativeVelocityMagnitude\pn{\norm{\FlockAgentRelativePos[_{\FlockAgentIndexTwo\FlockAgentIndex}]}, \norm{\zVelocityNeighborAverage_\FlockAgentIndex}}\frac{\FlockAgentRelativePos[_{\FlockAgentIndexTwo\FlockAgentIndex}]}{\norm{\FlockAgentRelativePos[_{\FlockAgentIndexTwo\FlockAgentIndex}]}} 
        %& 
        \ \text{if } -\zRelativeVelocityGravityBreak \leq \zLinearDiscriminantName\pn{\FlockAgentRelativePos[_{\FlockAgentIndexTwo\FlockAgentIndex}],\zVelocityNeighborAverage_\FlockAgentIndex, 0} \leq 0, \\
    \zRelativeVelocityMagnitude\pn{\norm{\FlockAgentRelativePos[_{\FlockAgentIndexTwo\FlockAgentIndex}]},\norm{\zVelocityNeighborAverage_\FlockAgentIndex}}\pn{-\mathrm{sgn}\spn{\zLinearDiscriminantName\pn{\FlockAgentRelativePos[_{\FlockAgentIndexTwo\FlockAgentIndex}], \zVelocityNeighborAverage_\FlockAgentIndex, \frac{\pi}{2}}}\zRotationMatrix\pn{\alpha_\zAngle}}\frac{\FlockAgentRelativePos[_{\FlockAgentIndexTwo\FlockAgentIndex}]}{\norm{\FlockAgentRelativePos[_{\FlockAgentIndexTwo\FlockAgentIndex}]}} & \text{else},
    \end{cases}
\end{align}
where $\zRelativeVelocityMagnitude:\mathbb{R}_+^2\to\mathbb{R}_+$ is a rescaling function, 
%relative velocity, 
$\zRelativeVelocityGravityBreak \in \spn{0, 1}$, and $\zRelativeVelocityObstacleDodgeAngle \in \hrpn{0, \frac{\pi}{2}}$. A follower may have multiple obstacle agent 
% edit(Taos): "obstacle agent neighbors" is clearer because a follower cannot be neighbors with the obstacle (a disk, not an agent).
neighbors, but we limit the follower to only process the relative velocity computed from the position of the closest obstacle agent.
% edit(Taos): saying obstacle agent here is important because the followers can only compute their distance to obstacle agents, not the obstacle (the disk).



We explain the relative velocity formulation considering a follower $\FlockAgentIndex$, its flock agent neighbors, and the obstacle agent $\FlockAgentIndexTwo$ that it is closest to.
From the definition, when $\zLinearDiscriminantName\pn{\FlockAgentRelativePos[_{\FlockAgentIndexTwo\FlockAgentIndex}], \zVelocityNeighborAverage_\FlockAgentIndex, 0} > 0$, the follower is roughly heading toward the obstacle agent, 
so the relative velocity accelerates the follower to the left or right of the obstacle agent (depending on the sign of $\zLinearDiscriminantName\pn{\FlockAgentRelativePos[_{\FlockAgentIndexTwo\FlockAgentIndex}], \zVelocityNeighborAverage_\FlockAgentIndex, \frac{\pi}{2}}$).
When moving left or right, eventually $\zLinearDiscriminantName\pn{\FlockAgentRelativePos[_{\FlockAgentIndexTwo\FlockAgentIndex}], \zVelocityNeighborAverage_\FlockAgentIndex, 0} \in \spn{-\zRelativeVelocityGravityBreak, 0}$, meaning the follower either is heading tangent to a disk of radius $\norm{\FlockAgentRelativePos[_{\FlockAgentIndexTwo\FlockAgentIndex}]}$ centered at the obstacle agent or is heading away from the obstacle agent. 
In this case, the relative velocity accelerates the follower toward the obstacle agent, initiating an orbit about the obstacle agent. The orbit helps the follower reunite with the followers that moved around the obstacle agent the opposite way. When $\zRelativeVelocityGravityBreak = 1$, the relative velocity accelerates the follower toward the obstacle agent even when its flock agent neighbors are moving away from the obstacle agent. The follower can get stuck in the ``gravity'' of the obstacle agent, so setting $\zRelativeVelocityGravityBreak \in \hlpn{0, 1}$ can help the follower terminate its orbit. When $\zLinearDiscriminantName\pn{\FlockAgentRelativePos[_{\FlockAgentIndexTwo\FlockAgentIndex}], \zVelocityNeighborAverage_\FlockAgentIndex, 0} < -\zRelativeVelocityGravityBreak$ and $\zRelativeVelocityObstacleDodgeAngle = \pi/2$, the relative velocity accelerates the followers in the direction tangent to a disk of radius $\norm{\FlockAgentRelativePos[_{\FlockAgentIndexTwo\FlockAgentIndex}]}$ centered at the obstacle agent.

The followers are drawn away from the obstacle agents by the leaders because the leaders will always eventually head away from all of the obstacle agents. The followers try to align their velocity with the leaders' velocity, and if they can, $\zLinearDiscriminantName\pn{\FlockAgentRelativePos[_{\FlockAgentIndexTwo\FlockAgentIndex}], \zVelocityNeighborAverage_\FlockAgentIndex, 0}$ will become negative. How closely the followers need to align their velocity with the leaders in order to move away from the obstacle agents depends on $\zRelativeVelocityGravityBreak$.
%
In our experiment, we choose $\zRelativeVelocityGravityBreak = 0.5$, $\zRelativeVelocityObstacleDodgeAngle = \pi/2$, and $\zRelativeVelocityMagnitude$ as 
%\begin{align*}
$\zRelativeVelocityMagnitude\pn{r, v} = e^{-r} + e^{-v}$.
%\end{align*}
%Future work includes finding a way to learn these. 
We offer an intuition for our choice of $\zRelativeVelocityMagnitude$. The $e^{-\norm{\FlockAgentRelativePos[_{\FlockAgentIndexTwo\FlockAgentIndex}]}}$ term amplifies the acceleration %signal 
supplied by the relative velocity when the follower and obstacle agent 
are closer. The $e^{-\norm{\zVelocityNeighborAverage_\FlockAgentIndex}}$ term strengthens the acceleration if the follower slows down to avoid colliding with the obstacle. This helps the follower maintain the magnitude of its velocity prior to detecting the obstacle, allowing the flock %as a whole 
to circumnavigate the obstacle faster. 
%and not fall behind the leaders. 
Attenuating the signal when the follower's velocity is small would allow it to stall in front of the obstacle and fall behind the leaders.


%\TT{(I later suggest that this being \SpecialOrthogonalGroup{2} equivariant is why \PupilETDAGNN{} does better than the other ML controllers)}
\iffalse
A notable property of this equation is that it is \SpecialOrthogonalGroup{2} equivariant.
It is also translation invariant with respect to position, but not velocity.

\begin{proposition}
\Eqref{eq:results:obstacle-avoidance:relative-velocity} is equivariant with respect to \SpecialOrthogonalGroup{2}.
\end{proposition}
\fi

}

%\paragraph{Results}
\Figref{fig:results:obstacle-avoidance:randomdisk:by-perimeter:no-transfer-learning-algorithm} shows the failure rate of ML controllers on the test simulations when they do \textbf{\textit{not}} use our technique. Their failure rate is near 100\% with over 95\% of failures due to disconnected communication graphs. Using our obstacle avoidance technique makes obstacle avoidance possible for ML controllers, as shown in \figref{fig:results:obstacle-avoidance:randomdisk:by-perimeter}.
When the obstacle perimeter is $24\FlockAgentMinDist$ or smaller, the failure fraction decreases by at least %80 percentage points 
80\% 
for all ML controllers. It decreases by %55 percentage points 
55\%
for perimeter $48\FlockAgentMinDist$ and at least %10 percentage points 
10\%
for perimeter $96\FlockAgentMinDist$.

\begin{figure}[!ht]
    \centering
    \begin{subfigure}[T]{\textwidth}
        \centering
        \includegraphics[width=0.95\textwidth]{figsEROANaive9eyiyg2j.legend.pdf}
    \end{subfigure} \\
    \begin{subfigure}[T]{.3\textwidth}
        \centering
        \hspace{-2.3em}
        \includegraphics[width=1.125\textwidth]{figsEROANaive9eyiyg2j__0.001__ObstaclePassed.pdf}
        \caption{}
    \end{subfigure}
    \begin{subfigure}[T]{.3\textwidth}
        \centering
        \includegraphics[width=\textwidth]{figsEROANaive9eyiyg2j__0.001__GraphDisconnected.pdf}
        \caption{}
    \end{subfigure}
    \begin{subfigure}[T]{.3\textwidth}
        \centering
        \includegraphics[width=\textwidth]{figsEROANaive9eyiyg2j__0.001__Collision.pdf}
        \caption{}
    \end{subfigure}
    %\vspace{-0.3cm}
\caption{%\small 
Performance of ML controllers in obstacle avoidance varying the perimeter of the regular polygon inscribed in the obstacle when our obstacle avoidance technique is \textbf{\textit{not}} used. The obstacle is a disk, and the regular polygon has $s$ sides of length $\FlockAgentMinDist$ whose vertices are on the boundary of the disk. They are evaluated on the RandomDisk test set with $\FlockAgentCount = 100$ agents where two agents are leaders.
(a) Fraction of simulations that failed.
(b) Fraction of failures due to a disconnected communication graph.
(c) Fraction of failures due to a collision.
}
\label{fig:results:obstacle-avoidance:randomdisk:by-perimeter:no-transfer-learning-algorithm}%\vspace{-0.15cm}
\end{figure}

\begin{figure}[!ht]
    \centering
    \begin{subfigure}[T]{\textwidth}
        \centering
        \includegraphics[width=0.95\textwidth]{figsEROAWithTechniqueoqo2x03v.legend.pdf}
    \end{subfigure} \\
    \begin{subfigure}[T]{.3\textwidth}
        \centering
        \hspace{-2.3em}
        \includegraphics[width=1.125\textwidth]{figsEROAWithTechniqueoqo2x03v__0.001__ObstaclePassed.pdf}
        \caption{}
    \end{subfigure}
    \begin{subfigure}[T]{.3\textwidth}
        \centering
        \includegraphics[width=\textwidth]{figsEROAWithTechniqueoqo2x03v__0.001__GraphDisconnected.pdf}
        \caption{}
    \end{subfigure}
    \begin{subfigure}[T]{.3\textwidth}
        \centering
        \includegraphics[width=\textwidth]{figsEROAWithTechniqueoqo2x03v__0.001__Collision.pdf}
        \caption{}
    \end{subfigure}
    %\vspace{-0.3cm}
\caption{%\small
Performance of ML controllers in obstacle avoidance using our obstacle avoidance technique varying the perimeter of the regular polygon inscribed in the obstacle. The obstacle is a disk, and the regular polygon has $s$ sides of length $\FlockAgentMinDist$ whose vertices are on the boundary of the disk. They are evaluated on the RandomDisk test set with $\FlockAgentCount = 100$ agents where two agents are leaders.
(a) Fraction of simulations that failed.
(b) Fraction of failures due to a disconnected communication graph.
(c) Fraction of failures due to a collision.
}
\label{fig:results:obstacle-avoidance:randomdisk:by-perimeter}%\vspace{-0.15cm}
\end{figure}


Communication graph disconnection is still the dominant cause of failure. Failures due to collisions with the obstacle only occur when the obstacle perimeter is $24\FlockAgentMinDist$, representing less than 5\% of failures. \PupilETDAGNN{} has a failure rate at least 10\% lower than other controllers when the obstacle perimeter is $48\FlockAgentMinDist$, and 20\% lower for perimeter $96\FlockAgentMinDist$. For perimeter $48\FlockAgentMinDist$ and $96\FlockAgentMinDist$, the other ML controllers' failure rates are within 10\% of each other.
%
%
For obstacles a perimeter $6\FlockAgentMinDist$ or smaller, all ML controllers have a failure rate of less than 10\%.
For larger obstacles, we recommend \PupilETDAGNN{} since its failure fraction is up to 20\% smaller than other ML controllers. Rotation equivariance provides a significant performance advantage in this obstacle avoidance 
experiment.
%
%
Animations of obstacle avoidance are available on GitHub.\footnote{Obstacle avoidance animations: \url{github.com/Utah-Math-Data-Science/Equivariant-Decentralized-Controllers/tree/main/misc/animations/obstacle_avoidance}}



{
\newcommand{\zGeneralizationTestSampleName}{\GeneralizationSampleName^\mathrm{test}}
\newcommand{\zGeneralizationTestSampleSize}{\GeneralizationSampleSize^\mathrm{test}}
\newcommand{\zEmpiricalGeneralizationGap}{\hat{\cR}_{\zGeneralizationTestSampleName,\GeneralizationSampleName,\LossFunctionName}}
\subsection{Generalization gap}
\label{sec:results:generalization-gap}
We %empirically 
verify the generalization bound in \eqref{eq:generalization-gap:bound} for each ML controller with respect to its \GeneralizationFastForwardBC{} dataset. Each dataset has 80,400 tuples with 30,150 for training and 50,250 for testing.
We train the ML controllers on their training sets with $\LossFunctionMSENormalizationConstant = 2$, and evaluate them on their test sets. The constant $\LossFunctionMSENormalizationConstant$ helps avoid the loss gradients being %zeroed 
zero due to the $\min\set{1, \cdot}$ function.
%
%
We compute the generalization bound in \eqref{eq:generalization-gap:bound} with $\GeneralizationGapBoundFailProbability = 10^{-3}$ and the empirical bound.
The empirical generalization bound is the difference of the empirical risk over the test set minus the empirical risk over the training set
$$%\small
\begin{aligned}
\zEmpiricalGeneralizationGap\pn{f} = \frac{1}{\zGeneralizationTestSampleSize} \sum_{\GeneralizationSampleIndex = 1}^{\zGeneralizationTestSampleSize} \LossFunctionName\pn{f\pn{x_\GeneralizationSampleIndex^\mathrm{test}}, y^\mathrm{test}} - \frac{1}{\GeneralizationSampleSize} \sum_{\GeneralizationSampleIndex = 1}^{\GeneralizationSampleSize} \LossFunctionName\pn{f\pn{x_\GeneralizationSampleIndex}, y}.
\end{aligned}
$$
% We plot the training and test losses, and the empirical generalization gap 
%with the bound as the ML models train 
% in Fig.~\ref{fig:results:generalization-gap:all}. 
%\Figref{fig:results:generalization-gap:losses} and \figref{fig:results:generalization-gap:bound} 
\Figref{fig:results:generalization-gap:losses} and \ref{fig:results:generalization-gap:bound} show that the empirical generalization gap is near zero for all controllers. %as they train. 
%Note that, As required by this analysis, 
Here, the training and the test sets are sampled from the same probability distribution induced by \GeneralizationFastForwardBC{}. 
%The loss values, which are always in $\spn{0, 1}$, are about 0.1 or smaller.



\begin{figure}[!ht]
    \centering
    \begin{subfigure}[T]{\textwidth}
        \centering
        \includegraphics[width=.5\textwidth]{figsERGGLosses3xh5mbgs.legend.pdf}
    \end{subfigure} \\
    \begin{subfigure}[T]{.23\textwidth}
        \centering
        \hspace{-2.5em}
        \includegraphics[width=1.187\textwidth]{figsERGGLosses3xh5mbgs__TDAGNN.pdf}
    \end{subfigure}
    \begin{subfigure}[T]{.23\textwidth}
        \centering
        \includegraphics[width=\textwidth]{figsERGGLosses3xh5mbgs__TDAGNNTF.pdf}
    \end{subfigure}
    \begin{subfigure}[T]{.23\textwidth}
        \centering
        \includegraphics[width=\textwidth]{figsERGGLosses3xh5mbgs__TDAGNNTFMu.pdf}
    \end{subfigure}
    \begin{subfigure}[T]{.23\textwidth}
        \centering
        \includegraphics[width=\textwidth]{figsERGGLosses3xh5mbgs__ETDAGNN.pdf}
    \end{subfigure}
    %\vspace{-.3cm}
    \caption{%\small
        The losses on the behavior cloning training and test sets as the flocking ML controllers train.
        %The empirical generalization gap is near zero for all ML controllers and epochs.
    }
    \label{fig:results:generalization-gap:losses}%\vspace{-0.25cm}
\end{figure}

\begin{figure}[!ht]
    \centering
    \begin{subfigure}[T]{\textwidth}
        \centering
        \includegraphics[width=.95\textwidth]{figsERGGGapmshujupc.legend.pdf}
    \end{subfigure} \\
    \begin{subfigure}[T]{.23\textwidth}
        \centering
        \hspace{-2.5em}
        \includegraphics[width=1.187\textwidth]{figsERGGGapmshujupc__TDAGNN.pdf}
    \end{subfigure}
    \begin{subfigure}[T]{.23\textwidth}
        \centering
        \includegraphics[width=\textwidth]{figsERGGGapmshujupc__TDAGNNTF.pdf}
    \end{subfigure}
    \begin{subfigure}[T]{.23\textwidth}
        \centering
        \includegraphics[width=\textwidth]{figsERGGGapmshujupc__TDAGNNTFMu.pdf}
    \end{subfigure}
    \begin{subfigure}[T]{.23\textwidth}
        \centering
        \includegraphics[width=\textwidth]{figsERGGGapmshujupc__ETDAGNN.pdf}
    \end{subfigure}
    %\vspace{-.3cm}
    \caption{%\small
        The generalization bound and empirical generalization gap over epochs.
        The generalization bound 
        improves with the ML controllers from left to right, and adding equivariance lowers the generalization gap the most.
        %Empirically, all the ML controllers generalize to the test set of the behavior cloning datasets.
    }
    \label{fig:results:generalization-gap:bound}%\vspace{-0.2cm}
\end{figure}


The generalization bound reduces as we add the improvements (described in \secref{sec:methods:improving-tdagnn}) to \PupilTDAGNN{}. 
%\PupilTDAGNN{}'s bound is about 5 points larger than \PupilTDAGNNTF{}'s bound, and \PupilTDAGNNTF{}'s bound is about 10 points larger than \PupilTDAGNNTFMu{}'s bound. 
%Notice that 
Enforcing equivariance significantly reduces the generalization gap, as \PupilETDAGNN{}'s bound is about half that of \PupilTDAGNNTFMu{}.
%The largest reduction occurs by adding equivariance, where \PupilETDAGNN{}'s bound is about half that of \PupilTDAGNNTFMu{}'s bound. 
The term in the generalization bound that varies between ML controllers is $48\PupilWeightMatrixLargestDim/\sqrt{\GeneralizationSampleSize}$ multiplied by the square root term $\sqrt{\cdot}$. %that it is the coefficient of.
\Figref{fig:results:generalization-gap:variables} gives insight into how these terms influence the generalization bound.
The plot of $48\PupilWeightMatrixLargestDim/\sqrt{\GeneralizationSampleSize}$ explains why the generalization bound of \PupilETDAGNN{} is about half of the non-equivariant controllers' -- \PupilETDAGNN{} has $\PupilWeightMatrixLargestDim = 16$ whereas the non-equivariant ML controllers have $\PupilWeightMatrixLargestDim = 33$. Next, $\DAggerDatumBound$ is an order of magnitude smaller for mean-aggregation controllers compared to sum-aggregation; however, it only introduces a 10-point difference between the bounds of \PupilTDAGNNTF{} and \PupilTDAGNNTFMu{} seen in the square root term $\sqrt{\cdot}$. The summation depending on the Frobenius norm of the weight matrices is remarkably similar for the non-equivariant controllers and \PupilETDAGNN{}. \PupilETDAGNN{} has about 75\% fewer weights (see Table~\ref{tab:results:num-model-weights}), so its weights tend to be larger than that of the non-equivariant controllers. Though these bounds are much larger than the empirical generalization gap, \figref{fig:results:generalization-gap:corr} shows that they have a high correlation ($\rho \geq 0.95$) with the empirical generalization gap.

\iffalse
\begin{figure}
    \centering
    \begin{subfigure}[T]{.49\textwidth}
        \centering
        \includegraphics[width=\textwidth]{figsERGGmn9yayn1.constants.pdf} % F
        \caption{\small
            The \GeneralizationFastForwardBC{} data bound (left) and the coefficient to the large square root term in the generalization gap bound (right).
            These terms are constant with respect to epochs.
            The data bound of \PupilTDAGNNTFMu{} and \PupilETDAGNN{} (mean-aggregation ML controllers) is an order of magnitude smaller compared to \PupilTDAGNN{} and \PupilTDAGNNTF{} (sum-aggregation ML controllers).
            The coefficient term of \PupilETDAGNN{} is less than half that of the other ML controllers.
        }
        \label{fig:results:generalization-gap:variables:constants}
    \end{subfigure}
    \hspace{.1em}
    \begin{subfigure}[T]{.49\textwidth}
        \centering
        \includegraphics[width=\textwidth]{figsERGGmn9yayn1.weight_vars.pdf} % F
        \caption{\small
            Terms in the generalization gap bound that depend on the Frobenius norm of the ML controllers' weight matrices when represented as MLPs -- the summation (left) and the large square root term (right) containing that summation.
        }
        \label{fig:results:generalization-gap:variables:weight_vars}
    \end{subfigure}\vspace{-0.3cm}
    \caption{\small
        Select terms of the generalization gap bound for each flocking ML controller as it trains on its \GeneralizationFastForwardBC{} dataset.
    }
    \label{fig:results:generalization-gap:variables}
\end{figure}
\fi

\begin{figure}[!ht]
    \centering
    \begin{subfigure}[T]{\textwidth}
        \centering
        \includegraphics[width=.95\textwidth]{figsERGGWeightVarsu6vyp7q7.legend.pdf}
    \end{subfigure} \\
    \begin{subfigure}[T]{.23\textwidth}
        \centering
        \hspace{-2em}
        \includegraphics[width=1.15\textwidth]{figsERGGConstants3xf6am2n__DAggerDatumBound.pdf}
        \caption{}
    \end{subfigure}
    \begin{subfigure}[T]{.23\textwidth}
        \centering
        \includegraphics[width=\textwidth]{figsERGGConstants3xf6am2n__BoundSqrtPrefix.pdf}
        \caption{}
    \end{subfigure}
    \begin{subfigure}[T]{.23\textwidth}
        \centering
        \includegraphics[width=.87\textwidth]{figsERGGWeightVarsu6vyp7q7__GGBigTerm.pdf}
        \caption{}
    \end{subfigure}
    \begin{subfigure}[T]{.23\textwidth}
        \centering
        \includegraphics[width=.87\textwidth]{figsERGGWeightVarsu6vyp7q7__BoundSqrtTerm.pdf}
        \caption{}
    \end{subfigure}
    %\vspace{-.3cm}
    \caption{%\small
    Select terms of the generalization bound for each ML controller as it trains on its \GeneralizationFastForwardBC{} dataset.
    (a) The \GeneralizationFastForwardBC{} data bound.
    (b) The coefficient to the large square root term in the generalization bound.
    These terms are constant with respect to epochs.
    The data bound of \PupilTDAGNNTFMu{} and \PupilETDAGNN{} (mean-aggregation ML controllers) is an order of magnitude smaller compared to \PupilTDAGNN{} and \PupilTDAGNNTF{} (sum-aggregation ML controllers).
    The coefficient term of \PupilETDAGNN{} is less than half that of the other ML controllers.
    The terms in the generalization gap bound that depend on the Frobenius norm of the ML controllers' weight matrices when represented as MLPs are (c) the summation and (d) the large square root term containing that summation.
    }
    \label{fig:results:generalization-gap:variables}%\vspace{-0.25cm}
\end{figure}

\begin{figure}[!ht]
    \centering
    \begin{subfigure}[T]{\textwidth}
        \centering
        \includegraphics[width=.95\textwidth]{figsERGGCorrgs33rn57.legend.pdf}
    \end{subfigure} \\
    \begin{subfigure}[T]{.23\textwidth}
        \centering
        \hspace{-2.5em}
        \includegraphics[width=1.187\textwidth]{figsERGGCorrgs33rn57__TDAGNN.pdf}
    \end{subfigure}
    \begin{subfigure}[T]{.23\textwidth}
        \centering
        \includegraphics[width=\textwidth]{figsERGGCorrgs33rn57__TDAGNNTF.pdf}
    \end{subfigure}
    \begin{subfigure}[T]{.23\textwidth}
        \centering
        \includegraphics[width=\textwidth]{figsERGGCorrgs33rn57__TDAGNNTFMu.pdf}
    \end{subfigure}
    \begin{subfigure}[T]{.23\textwidth}
        \centering
        \includegraphics[width=\textwidth]{figsERGGCorrgs33rn57__ETDAGNN.pdf}
    \end{subfigure}
    %\vspace{-.3cm}
    \caption{%\small
        The correlation of generalization %gap 
        bound and empirical generalization gap.
        We plot each of these sequences over epochs with their mean subtracted and divided by their standard deviation.
        %The generalization gap bound and empirical generalization gap have a high correlation.
    }
    \label{fig:results:generalization-gap:corr}%\vspace{-0.15cm}
\end{figure}

% \begin{figure}[!ht]\vspace{-0.2cm}
% \centering
% \begin{tabular}{cc}
% \includegraphics[width=0.38\linewidth]{figsERGGmn9yayn1.constants.pdf}&
% \includegraphics[width=0.38\linewidth]{figsERGGmn9yayn1.weight_vars.pdf}\\[-3pt]
% (a) & (b) \\
% \end{tabular}\vspace{-0.35cm}
% \caption{\small
% Select terms of the generalization bound for each ML controller trains on its \GeneralizationFastForwardBC{} dataset. (a) The \GeneralizationFastForwardBC{} data bound (left) and the coefficient to the large square root term in the generalization bound (right).
%             These terms are constant with respect to epochs.
%             The data bound of \PupilTDAGNNTFMu{} and \PupilETDAGNN{} (mean-aggregation ML controllers) is an order of magnitude smaller compared to \PupilTDAGNN{} and \PupilTDAGNNTF{} (sum-aggregation ML controllers).
%             The coefficient term of \PupilETDAGNN{} is less than half that of the other ML controllers.
% (b) Terms in the generalization gap bound that depend on the Frobenius norm of the ML controllers' weight matrices when represented as MLPs -- the summation (left) and the large square root term (right) containing that summation.
% }
% \label{fig:results:generalization-gap:variables}%\vspace{-0.3cm}
% \end{figure}


}


\section{Conclusion}
%A prominent source of inspiration found in nature for designing decentralized controllers is flocking. Three key decisions when designing a flocking controller are choosing the objective each agent individually optimizes, deciding what information each agent has available about the flock or the environment to make its decision, and how all agents' decisions culminate to optimize a collective objective. After Reynolds' simulation of flock-like motion, centralized flocking controllers found viable choices for these three key decisions, guaranteeing asymptotic flocking and separation. In contrast, decentralized flocking controllers have struggled to maintain flock cohesion. Graph neural networks have emerged as decentralized flocking controllers capable of maintaining flock cohesion by utilizing both current and past information exchanged in the decentralized agent network; however, their current design omits the rotational equivariance prior present in centralized flocking controllers. 
In response to the challenges in building decentralized flocking controllers, we presented an enhanced ML-based decentralized flocking controller that leverages a rotation equivariant and translation invariant GNN.
We demonstrated the advantages of the proposed decentralized controller over existing non-equivariant ML-based controllers and other flocking controllers using three representative case studies -- flocking, leader following, and obstacle avoidance.
We also analyzed the generalization gap of the proposed decentralized flocking controller.
Our numerical results show that the proposed decentralized controller achieves comparable performance compared to non-equivariant ML-based controllers with 70\% less training data, 75\% fewer trainable weights and a 50\% smaller generalization bound.

% \backmatter

% \bmhead{Acknowledgements}

\section*{Declarations}

% \subsection*{CRediT authorship contribution statement}
% \textbf{Taos Transue:}
% Data curation,
% Formal analysis,
% Investigation,
% Methodology,
% Project administration,
% Software,
% Validation,
% Visualization,
% Writing -- original draft,
% Writing -- review and editing.
% \textbf{Bao Wang:}
% Conceptualization,
% %Funding acquisition,
% Methodology,
% Project administration,
% %Resources,
% Writing -- review and editing.

\subsection*{Data availability}
Code and animations are available at \url{github.com/Utah-Math-Data-Science/Equivariant-Decentralized-Controllers}.

\subsection*{Funding}
This material is based on research sponsored by National Science Foundation (NSF) grants DMS-2152762, DMS-2208361, DMS-2219956, and DMS-2436344, and Department of Energy grants DE-SC0023490, DE-SC0025589, and DE-SC0025801.
Taos Transue received partial financial support from the NSF under Award 2136198.

\subsection*{Competing interets}
All authors certify that they have no affiliations with or involvement in any organization or entity with any financial interest or non-financial interest in the subject matter or materials discussed in this manuscript.

\begin{appendix}

\section{Proofs}

\subsection{Preliminaries}

\begin{lemma}[Frobenius norm upper bounds spectral norm]
    \label{lemma:results:frobenius-geq-spectral}
    Let $\mA \in \bR^{m \times n}$, and then $\norm{\mA}_2 \leq \norm{\mA}_F$.
\end{lemma}
\begin{proof}
    Let $\mA = \mU\mSigma\mV^\top$ be the singular value decomposition of $\mA$ where $\mU \in \bR^{m \times m}$ and $\mV \in \bR^{n \times n}$ are orthogonal matrices.
    Let $\sigma_i$ and $\sigma_{\max}$ denote the $i$-th and the largest singular value, respectively.
    Then,
    $$%\small
        \norm{\mA}_2 = \norm{\mU\mSigma\mV^\top}_2 = \norm{\mSigma}_2 = \sigma_{\max} \leq \sum_{i = 1}^{\min\set{m,n}} \sigma_i = \norm{\mSigma}_F = \norm{\mU\mSigma\mV^\top}_F = \norm{\mA}_F
    $$
    This bound is sharp since, for $c \in \bR^{1 \times 1}$, $\norm{c}_2 = \abs{c} = \norm{c}_F$.
\end{proof}

\subsection{Generalization gap}
\begin{lemma}
%[\PupilTDAGNN{}, \PupilTDAGNNTF{}, and \PupilTDAGNNTFMu{} are bounded with respect to their input]
\label{lemma:results:tdagnn-bounded}
Let \PupilName{} be either \PupilTDAGNN{}, \PupilTDAGNNTF{}, or \PupilTDAGNNTFMu{}. Then,
$$%\small
    \begin{aligned}
        \norm{\PupilName\pn{\gConvInput[1]}} =
        \norm{\PupilAsMLPName{}\pn{\gConvInputAsMLP[1]}} \leq
        \pn{\prod_{\PupilLayerIndex = 1}^{\PupilLayerCount} \PupilActivationLipshitzConstant \norm{\PupilWeightMatrix_\PupilLayerIndex}_F}\norm{\gConvInputAsMLP[1]}_F,
    \end{aligned}
$$
where \PupilAsMLPName{} is the MLP representation of \PupilName{}, $\gConvInputAsMLP[1]$ is from Definition~\ref{def:results:input-for-mlp-non-equivariant}, $\PupilActivationLipshitzConstant \geq 1$ bounds the largest Lipshitz constant of the activations used by \PupilName{}, and $\set{\PupilWeightMatrix_\PupilLayerIndex}_{\PupilLayerIndex = 1}^{\PupilLayerCount}$ are the weight matrices of \PupilAsMLPName{}.
\end{lemma}

\begin{proof}
    The activations of \PupilName{} are either $\PupilActivationName_\PupilLayerIndex(x) = \tanh(x)$ or $\PupilActivationName_\PupilLayerIndex\pn{x} = x$ for $\PupilLayerIndex \in \set{1,\dots,\PupilLayerCount}$.
    Both have Lipshitz constant $\PupilActivationLipshitzConstant = 1$ and satisfy $\PupilActivationName\pn{0} = 0$.
    By \lemmaref{lemma:results:conv1d-as-linear}, $\PupilName$ can be expressed as an MLP \PupilAsMLPName{}.
    Adapting Lemma~B.1 of \cite{pmlr-v235-karczewski24a},
$$%\small
    \begin{aligned}
        \norm{\PupilAsMLPName_{\PupilLayerCount}\pn{\gConvInputAsMLP[1]}}_2 & =
        \norm{\PupilActivationName_{\PupilLayerCount}\pn{\PupilAsMLPName_{\PupilLayerCount - 1}\pn{\gConvInputAsMLP[1]}\PupilWeightMatrix_{\PupilLayerCount}}}_2 =
        \norm{\PupilActivationName_{\PupilLayerCount}\pn{\PupilAsMLPName_{\PupilLayerCount - 1}\pn{\gConvInputAsMLP[1]}\PupilWeightMatrix_{\PupilLayerCount}} - \PupilActivationName_{\PupilLayerCount}\pn{\mzero}}_2
        \\ & \leq
        \PupilActivationLipshitzConstant\norm{\PupilAsMLPName_{\PupilLayerCount - 1}\pn{\gConvInputAsMLP[1]}\PupilWeightMatrix_{\PupilLayerCount}}_2 \leq
        \PupilActivationLipshitzConstant\norm{\PupilWeightMatrix_{\PupilLayerCount}}_2\norm{\PupilAsMLPName_{\PupilLayerCount - 1}\pn{\gConvInputAsMLP[1]}}_2
        \\ & \leq
        \dots \leq
        \pn{\prod_{\PupilLayerIndex = 1}^{\PupilLayerCount} \PupilActivationLipshitzConstant\norm{\PupilWeightMatrix_\PupilLayerIndex}_2}\norm{\gConvInputAsMLP[1]}_2
        \\ & =
        \pn{\prod_{\PupilLayerIndex = 1}^{\PupilLayerCount} \PupilActivationLipshitzConstant\norm{\PupilWeightMatrix_\PupilLayerIndex}_2}\norm{\gConvInputAsMLP[1]}_F \quad \text{by Definition~\ref{def:results:input-for-mlp-non-equivariant}}
        \\ & \leq
        \pn{\prod_{\PupilLayerIndex = 1}^{\PupilLayerCount} \PupilActivationLipshitzConstant\norm{\PupilWeightMatrix_\PupilLayerIndex}_F}\norm{\gConvInputAsMLP[1]}_F \quad \text{by \Lemmaref{lemma:results:frobenius-geq-spectral}}
    \end{aligned}
$$
\end{proof}

\begin{lemma}%[\PupilETDAGNN{} is bounded with respect to its input]
    \label{lemma:results:etdagnn-bounded}
    Let \PupilName{} be an \PupilETDAGNN{}.
    Then,
    $$%\small
    \begin{aligned}
        \norm{\PupilName\pn{\gConvInput[1]}} =
        \norm{\PupilAsMLPName{}\pn{\gConvInputAsMLP[1]}} \leq
        \pn{\prod_{\PupilLayerIndex = 1}^{\PupilLayerCount} \PupilActivationLipshitzConstant \norm{\PupilWeightMatrix_\PupilLayerIndex}_F}\norm{\gConvInputAsMLP[1]}_F,
    \end{aligned}
    $$
    where \PupilAsMLPName{} is the MLP representation of \PupilETDAGNN{}, $\gConvInputAsMLP[1]$ is defined in Definition~\ref{def:results:input-for-mlp-equivariant}, $\PupilActivationLipshitzConstant \geq 1$ bounds the largest Lipshitz constant of the activations in \PupilETDAGNN{}, and $\set{\PupilWeightMatrix_\PupilLayerIndex}_{\PupilLayerIndex = 1}^{\PupilLayerCount}$ are the weight matrices of \PupilAsMLPName{}.
\end{lemma}

\begin{proof}
    \PupilName{} uses the \OrthogonalGroup{n} equivariant activations $\bm\PupilActivationName_\PupilLayerIndex \in \set{\PupilETDAGNNActivationScaleLog, \PupilETDAGNNActivationScaleTanh, \vx \mapsto \vx}$ for $\PupilLayerIndex \in \set{1,\dots,\PupilLayerCount}$.
    Let \PupilActivationLipshitzConstant{} be the maximum of their Lipshitz constants.
    By \Lemmaref{lemma:results:equivariant-block-matrix-func-lipshitz-multiple-output-channels}, \ref{lemma:results:etdagnn-sigma0-lipshitz}, and \ref{lemma:results:etdagnn-sigma1-lipshitz}, $\bm\PupilActivationName_\PupilLayerIndex$ is an \OrthogonalGroup{2} equivariant activation with global Lipshitz \PupilActivationLipshitzConstant{} in the Frobenius norm, and $\bm\PupilActivationName_\PupilLayerIndex\pn{\mzero} = \mzero$ by \Lemmaref{lemma:results:equivariant-zero-at-zero}.
    Next, by \Lemmaref{lemma:results:eqconv-as-linear}, \PupilName{} can be expressed as an MLP \PupilAsMLPName{}.
    Adapting Lemma~B.1 of \cite{pmlr-v235-karczewski24a},
$$%\small
    \begin{aligned}
        \norm{\PupilAsMLPName_{\PupilLayerCount}\pn{\gConvInputAsMLP[1]}}_2 & =
        \norm{\PupilActivationName_{\PupilLayerCount}\pn{\PupilAsMLPName_{\PupilLayerCount - 1}\pn{\gConvInputAsMLP[1]}\PupilWeightMatrix_{\PupilLayerCount}}}_2 =
        \norm{\PupilActivationName_{\PupilLayerCount}\pn{\PupilAsMLPName_{\PupilLayerCount - 1}\pn{\gConvInputAsMLP[1]}\PupilWeightMatrix_{\PupilLayerCount}} - \PupilActivationName_{\PupilLayerCount}\pn{\mzero}}_2
        \\ & \leq
        \PupilActivationLipshitzConstant\norm{\PupilAsMLPName_{\PupilLayerCount - 1}\pn{\gConvInputAsMLP[1]}\PupilWeightMatrix_{\PupilLayerCount}}_2 \leq
        \PupilActivationLipshitzConstant\norm{\PupilWeightMatrix_{\PupilLayerCount}}_2\norm{\PupilAsMLPName_{\PupilLayerCount - 1}\pn{\gConvInputAsMLP[1]}}_2
        \\ & \leq
        \dots \leq
        \pn{\prod_{\PupilLayerIndex = 1}^{\PupilLayerCount} \PupilActivationLipshitzConstant\norm{\PupilWeightMatrix_\PupilLayerIndex}_2}\norm{\gConvInputAsMLP[1]}_2
        \\ & \leq
        \pn{\prod_{\PupilLayerIndex = 1}^{\PupilLayerCount} \PupilActivationLipshitzConstant\norm{\PupilWeightMatrix_\PupilLayerIndex}_F}\norm{\gConvInputAsMLP[1]}_F \quad \text{by \Lemmaref{lemma:results:frobenius-geq-spectral}}
    \end{aligned}
$$
\end{proof}

\begin{lemma}%[The scoring model is bounded with bounded input]
    \label{lemma:results:scoring-model-bounded-with-bounded-input}
    Let the scoring model $\PupilGraphScorer$ be as given in \eqref{eq:generalization-gap:scoring-model} equivalently written to take an \GeneralizationFastForwardBC{} tuple as input.
    Using assumption~\ref{assumption:results:dagger-dataset-non-equivariant-bounded},
$$%\small
    \begin{aligned}
        \PupilGraphScorer\pn{\gDAggerDatasetDatum} \leq \PupilGraphScorerBias^2 + \pn{1 + \pn{\prod_{\PupilLayerIndex = 1}^{\PupilLayerCount} \PupilActivationLipshitzConstant\norm{\PupilWeightMatrix_\PupilLayerIndex}_F}^2}\DAggerDatumBound^2.
    \end{aligned}
$$
\end{lemma}
\begin{proof}
    Using Assumption~\ref{assumption:results:dagger-dataset-non-equivariant-bounded}, and \Lemmaref{lemma:results:tdagnn-bounded} and \ref{lemma:results:etdagnn-bounded},
$$%\small
    \begin{aligned}
        \PupilGraphScorer\pn{\gDAggerDatasetDatum} & \leq \PupilGraphScorerBias^2 + \frac{1}{\FlockAgentCount}\sum_{\FlockAgentIndex = 1}^\FlockAgentCount \norm{\gExpertName\pn{\gExpertInput}}^2 + \norm{\PupilName\pn{\gConvInput[1]}}^2 \\
        & = \PupilGraphScorerBias^2 + \frac{1}{\FlockAgentCount}\sum_{\FlockAgentIndex = 1}^\FlockAgentCount \norm{\gExpertName\pn{\gExpertInput}}^2 + \norm{\PupilAsMLPName\pn{\gConvInputAsMLP[1]}}^2 \\
        & \leq \PupilGraphScorerBias^2 + \frac{1}{\FlockAgentCount}\sum_{\FlockAgentIndex = 1}^\FlockAgentCount \norm{\gExpertName\pn{\gExpertInput}}_F^2 + \pn{\prod_{\PupilLayerIndex = 1}^{\PupilLayerCount} \PupilActivationLipshitzConstant\norm{\PupilWeightMatrix_\PupilLayerIndex}_F}^2\norm{\gConvInputAsMLP[1]}_F^2 \\
        & \leq \PupilGraphScorerBias^2 + \frac{1}{\FlockAgentCount}\sum_{\FlockAgentIndex = 1}^\FlockAgentCount \beta^2 + \pn{\prod_{\PupilLayerIndex = 1}^{\PupilLayerCount} \PupilActivationLipshitzConstant\norm{\PupilWeightMatrix_\PupilLayerIndex}_F}^2\DAggerDatumBound^2 \\
        & \leq \PupilGraphScorerBias^2 + \pn{1 + \pn{\prod_{\PupilLayerIndex = 1}^{\PupilLayerCount} \PupilActivationLipshitzConstant\norm{\PupilWeightMatrix_\PupilLayerIndex}_F}^2}\DAggerDatumBound^2.
    \end{aligned}
$$
\end{proof}

The ML controllers are parameterized by weight matrices, which have a bounded covering number (Lemma~8 of \cite{chenGeneralizationBoundsFamily2019}).
Using Lemma~G.1 of \cite{pmlr-v235-karczewski24a}, if we can show the ML controllers are Lipshitz with respect to the weight matrices, then we can bound the covering number of the ML controllers.
Next, we show that the ML controllers and their scoring function are Lipshitz. %continuous with respect to their weights.

\begin{lemma}[Lipshitz continuity of %time-delayed aggregation GNN 
TDAGNN] %with respect to weights]
    \label{lemma:results:tdagnn-lipshitz-wrt-weights}
    Let \PupilName{} be either \PupilTDAGNN{}, \PupilTDAGNNTF{}, \PupilTDAGNNTFMu{}, or \PupilETDAGNN{} with an MLP representation $\PupilAsMLPName$.
    Take two functions $\PupilAsMLPName\pn{\gConvInputAsMLP[1]; \PupilWeightMatrixCollection}$ and $\PupilName\pn{\gConvInputAsMLP[1]; \tilde{\PupilWeightMatrixCollection}}$ with \PupilLayerCount{} layers where \PupilWeightMatrixCollection{} and $\tilde{\PupilWeightMatrixCollection}$ are collections of weight matrices. 
    %parametrizing $\PupilAsMLPName$.
    Let $\PupilWeightMatrixBound_\PupilLayerIndex \geq 1$ s.t. $\max\set{\norm{\PupilWeightMatrix_\PupilLayerIndex}_F, \norm{\tilde{\PupilWeightMatrix}_\PupilLayerIndex}_F} \leq \PupilWeightMatrixBound_\PupilLayerIndex$.
    Then,
$$%\small
    \begin{aligned}
        \norm{\PupilAsMLPName\pn{\gConvInputAsMLP[1]; \PupilWeightMatrixCollection} - \PupilAsMLPName\pn{\gConvInputAsMLP[1]; \tilde{\PupilWeightMatrixCollection}}} \leq \norm{\gConvInputAsMLP[1]}_F\pn{\prod_{\PupilLayerIndex = 1}^{\PupilLayerCount} \PupilActivationLipshitzConstant\PupilWeightMatrixBound_\PupilLayerIndex}\sum_{\PupilLayerIndex = 1}^{\PupilLayerCount} \norm{\PupilWeightMatrix_\PupilLayerIndex - \tilde{\PupilWeightMatrix}_\PupilLayerIndex}_F,
    \end{aligned}
$$
    where $\PupilActivationLipshitzConstant \geq 1$ bounds the largest Lipshitz constant of the activations used by \PupilAsMLPName{}.
\end{lemma}

\begin{proof}
    The proof follows the proof of Lemma B.2 of \cite{pmlr-v235-karczewski24a} where $\UtilNormWithPlaceholder = \UtilNormWithPlaceholder_F$.
\end{proof}


\begin{lemma}%[Scoring function is locally Lipshitz with respect to weights]
    \label{lemma:results:scoring-function-locally-lipshitz}
    Let \PupilAsMLPName{} be the MLP representation of \PupilName{} and $\PupilGraphScorer$ be as defined in \eqref{eq:generalization-gap:scoring-model}.
    Consider two parameterizations of \PupilAsMLPName{} and $\PupilGraphScorer$: $\set{\PupilGraphScorerBias,\PupilWeightMatrixCollection},\set{\tilde\PupilWeightMatrixCollection,\tilde\PupilGraphScorerBias}$.
    Let $\PupilWeightMatrixBound_\PupilLayerIndex \geq 1$ and $\PupilWeightMatrixBound_\PupilGraphScorer \geq 1$ where $\max\set{\norm{\PupilWeightMatrix_\PupilLayerIndex}_F, \norm{\tilde{\PupilWeightMatrix}_\PupilLayerIndex}_F} \leq \PupilWeightMatrixBound_\PupilLayerIndex$ and $\max\set{\abs{\PupilGraphScorerBias}, \abs{\tilde{\PupilGraphScorerBias}}} \leq \PupilWeightMatrixBound_\PupilGraphScorer$.
    By Assumption~\ref{assumption:results:dagger-dataset-non-equivariant-bounded},
$$%\small
    \begin{aligned}
        &\ \ \abs{\PupilGraphScorer\pn{\gDAggerDatasetDatum; \PupilWeightMatrixCollection, \PupilGraphScorerBias} - \PupilGraphScorer\pn{\gDAggerDatasetDatum; \tilde{\PupilWeightMatrixCollection}, \tilde\PupilGraphScorerBias}}\\
        & \leq 2\PupilWeightMatrixBound_\PupilGraphScorer\abs{\PupilGraphScorerBias - \tilde\PupilGraphScorerBias} + 2\pn{1 +  \PupilLipshitzConstantName}\DAggerDatumBound^2\PupilLipshitzConstantName\sum_{\PupilLayerIndex = 1}^{\PupilLayerCount} \norm{\PupilWeightMatrix_\PupilLayerIndex - \tilde{\PupilWeightMatrix}_\PupilLayerIndex}_F.
    \end{aligned}
$$
    where $\PupilLipshitzConstantName = \PupilLipshitzConstantDefn$.
\end{lemma}
\begin{proof}
$$%\small
    \begin{aligned}
        &\ \ \abs{\PupilGraphScorer\pn{\cdot; \PupilWeightMatrixCollection} - \PupilGraphScorer\pn{\cdot; \tilde{\PupilWeightMatrixCollection}}}\\ & = \abs{\PupilGraphScorerBias^2 + \frac{1}{\FlockAgentCount}\sum_{\FlockAgentIndex = 1}^\FlockAgentCount \norm{\gExpertName\pn{\gExpertInput} - \PupilAsMLPName\pn{\gConvInputAsMLP[1]; \PupilWeightMatrixCollection}}^2 - \PupilGraphScorerBias^2 - \frac{1}{\FlockAgentCount}\sum_{\FlockAgentIndex = 1}^\FlockAgentCount\norm{\gExpertName\pn{\gExpertInput} - \PupilAsMLPName\pn{\gConvInputAsMLP[1]; \tilde{\PupilWeightMatrixCollection}}}^2} \\
        & \leq \abs{\PupilGraphScorerBias^2 - \tilde{\PupilGraphScorerBias}^2} + \frac{1}{\FlockAgentCount}\sum_{\FlockAgentIndex = 1}^\FlockAgentCount \abs{\norm{\gExpertName\pn{\gExpertInput} - \PupilAsMLPName\pn{\gConvInputAsMLP[1]; \PupilWeightMatrixCollection}}^2 - \norm{\gExpertName\pn{\gExpertInput} - \PupilAsMLPName\pn{\gConvInputAsMLP[1]; \tilde{\PupilWeightMatrixCollection}}}^2}.
    \end{aligned}
$$
    Using the fact $\abs{\norm{\vx - \va}^2 - \norm{\vx - \vb}^2} \leq \pn{2\norm{\vx} + \norm{\va} + \norm{\vb}}\norm{\va - \vb}$, $\norm{\gExpertName\pn{\gExpertInput}} \leq \DAggerDatumBound$ by Assumption~\ref{assumption:results:dagger-dataset-non-equivariant-bounded}, and $\norm{\PupilAsMLPName\pn{\gConvInputAsMLP[1]}} \leq \pn{\prod_{\PupilLayerIndex = 1}^{\PupilLayerCount} \PupilActivationLipshitzConstant\norm{\PupilWeightMatrix_\PupilLayerIndex}_F}\norm{\gConvInputAsMLP[1]}_F \leq \PupilLipshitzConstantName\norm{\gConvInputAsMLP[1]}_F$ by \Lemmaref{lemma:results:tdagnn-bounded} or \ref{lemma:results:etdagnn-bounded},
$$%\small
    \begin{aligned}
        \abs{\PupilGraphScorer\pn{\cdot; \PupilWeightMatrixCollection} - \PupilGraphScorer\pn{\cdot; \tilde{\PupilWeightMatrixCollection}}} & \leq 2\PupilWeightMatrixBound_\PupilGraphScorer\abs{\PupilGraphScorerBias - \tilde{\PupilGraphScorerBias}} + \frac{2}{\FlockAgentCount} \sum_{\FlockAgentIndex = 1}^\FlockAgentCount \pn{\DAggerDatumBound + \PupilLipshitzConstantName\norm{\gConvInputAsMLP[1]}_F}\norm{\PupilAsMLPName\pn{\gConvInputAsMLP[1]; \PupilWeightMatrixCollection} - \PupilAsMLPName\pn{\gConvInputAsMLP[1]; \tilde{\PupilWeightMatrixCollection}}}.
    \end{aligned}
$$
    By Lemma~\ref{lemma:results:tdagnn-lipshitz-wrt-weights},
$$%\small
    \begin{aligned}
       &\ \  \abs{\PupilGraphScorer\pn{\cdot; \PupilWeightMatrixCollection} - \PupilGraphScorer\pn{\cdot; \tilde{\PupilWeightMatrixCollection}}}\\ & \leq 2\PupilWeightMatrixBound_\PupilGraphScorer\abs{\PupilGraphScorerBias - \tilde{\PupilGraphScorerBias}} + \frac{2}{\FlockAgentCount} \sum_{\FlockAgentIndex = 1}^\FlockAgentCount \pn{\DAggerDatumBound + \PupilLipshitzConstantName\norm{\gConvInputAsMLP[1]}_F}\norm{\gConvInputAsMLP[1]}_F\PupilLipshitzConstantName\sum_{\PupilLayerIndex = 1}^{\PupilLayerCount}\norm{\PupilWeightMatrix_\PupilLayerIndex - \tilde\PupilWeightMatrix_\PupilLayerIndex}_F.
    \end{aligned}
$$
    Using that $\norm{\gConvInputAsMLP[1]}_F \leq \DAggerDatumBound$ by Assumption~\ref{assumption:results:dagger-dataset-non-equivariant-bounded},
$$%\small
    \begin{aligned}
        \abs{\PupilGraphScorer\pn{\cdot; \PupilWeightMatrixCollection} - \PupilGraphScorer\pn{\cdot; \tilde{\PupilWeightMatrixCollection}}} & \leq 2\PupilWeightMatrixBound_\PupilGraphScorer\abs{\PupilGraphScorerBias - \tilde{\PupilGraphScorerBias}} + \frac{2}{\FlockAgentCount} \sum_{\FlockAgentIndex = 1}^\FlockAgentCount \pn{\DAggerDatumBound + \PupilLipshitzConstantName\DAggerDatumBound}\DAggerDatumBound\PupilLipshitzConstantName\sum_{\PupilLayerIndex = 1}^{\PupilLayerCount}\norm{\PupilWeightMatrix_\PupilLayerIndex - \tilde\PupilWeightMatrix_\PupilLayerIndex}_F \\
        & = 2\PupilWeightMatrixBound_\PupilGraphScorer\abs{\PupilGraphScorerBias - \tilde{\PupilGraphScorerBias}} + 2\pn{1 + \PupilLipshitzConstantName}\DAggerDatumBound^2\PupilLipshitzConstantName\sum_{\PupilLayerIndex = 1}^{\PupilLayerCount}\norm{\PupilWeightMatrix_\PupilLayerIndex - \tilde\PupilWeightMatrix_\PupilLayerIndex}_F.
    \end{aligned}
$$
\end{proof}

With the above supporting lemmas, we are ready to prove the following result:
%a proposition based on proposition~4.1 of \cite{pmlr-v235-karczewski24a}.

\begin{proposition}[Generalization bound of TDAGNN]
Let $P$ be the probability distribution over tuples $\pn{\gDAggerDatasetDatum}$ induced by \GeneralizationFastForwardBC{}. 
%Let $\LossFunctionName: \hlpn{0, \infty} \to \spn{0, 1}$ defined as 
Let $\cL\pn{y} = \min\set{1, y/\LossFunctionMSENormalizationConstant}$ for $\LossFunctionMSENormalizationConstant > 0$ be the loss function.
Let $\set{\PupilWeightMatrix_\PupilLayerIndex}_{\PupilLayerIndex = 1}^{\PupilLayerCount}$ be the weights of the MLP representation $\PupilAsMLPName$ of $\PupilName$ given by \Lemmaref{lemma:results:conv1d-as-linear} or \ref{lemma:results:eqconv-as-linear}, and let $\PupilGraphScorerBias$ of $\PupilGraphScorer$ in \eqref{eq:generalization-gap:scoring-model} such that $\PupilGraphScorerBias \in \spn{0, \sqrt{\LossFunctionMSENormalizationConstant}}$.
For any $\delta > 0$, with probability at least $1 - \delta$ over choosing a batch $\GeneralizationSampleName$ of $\GeneralizationSampleSize$ tuples sampled from $P$, the following bound holds:
$$%\small
\begin{aligned}
   &\ \  \gGeneralizationGap\pn{\PupilGraphScorer} \leq \frac{8}{\GeneralizationSampleSize} \\
   & + \frac{48\PupilWeightMatrixLargestDim}{\sqrt{\GeneralizationSampleSize}}\sqrt{\pn{3\PupilLayerCount + 1}\ln\pn{10\PupilLayerCount\DAggerDatumBound\PupilActivationLipshitzConstant^{\PupilLayerCount}\sqrt{\PupilWeightMatrixLargestDim\GeneralizationSampleSize\LossFunctionMSENormalizationConstant}} + \pn{2\PupilLayerCount + 3}\sum_{\PupilLayerIndex = 1}^{\PupilLayerCount} \ln\pn{\max\set{1,\norm{\PupilWeightMatrix_\PupilLayerIndex}_F}}} + 3\sqrt{\frac{\ln\pn{\frac{2}{\delta}}}{2\GeneralizationSampleSize}}.
\end{aligned}
$$
\end{proposition}


\begin{proof}
    %Define the set of approximators as
    Let $\GeneralizationFunctionsLearners = \set{\PupilGraphScorer\pn{\cdot; \PupilWeightMatrixCollection, \PupilGraphScorerBias}: \PupilWeightMatrixCollection = \set{\PupilWeightMatrix_\PupilLayerIndex}_{\PupilLayerIndex = 1}^{\PupilLayerCount}, \norm{\PupilWeightMatrix_\PupilLayerIndex}_F \leq \PupilWeightMatrixBound_\PupilLayerIndex, \abs{\PupilGraphScorerBias} \leq \PupilWeightMatrixBound_\PupilGraphScorer}$, where $\PupilWeightMatrixBound_\PupilGraphScorer = \sqrt{\LossFunctionMSENormalizationConstant}$.
    Define the set of datum-to-loss functions $\GeneralizationFunctionsDatumToLoss = \set{\pn{\gDAggerDatasetDatum} \mapsto \LossFunctionName\pn{\PupilGraphScorer\pn{\gDAggerDatasetDatum}}: \PupilGraphScorer \in \GeneralizationFunctionsLearners}$.
    %To start, 
    We follow the steps from proof of Proposition 4.1 from \cite{pmlr-v235-karczewski24a}. 
    %up to equation~(31).
    Applying these steps requires that we find an $\GeneralizationFunctionDatumToLossHatZero \in \GeneralizationFunctionsDatumToLossHat$ where
$$%\small
    \begin{aligned}
        \GeneralizationFunctionsDatumToLossHat = \set{\pn{\gDAggerDatasetDatum} \mapsto 1 - \LossFunctionName\pn{\PupilGraphScorer\pn{\gDAggerDatasetDatum}}: \PupilGraphScorer \in \GeneralizationFunctionsLearners},
    \end{aligned}
$$
    and $\GeneralizationFunctionDatumToLossHatZero\pn{\gDAggerDatasetDatum} = 0$ for all $\pn{\gDAggerDatasetDatum}$ satisfying Assumption~\ref{assumption:results:dagger-dataset-non-equivariant-bounded}.
    By definition of $\LossFunctionName$, we can construct $\GeneralizationFunctionDatumToLossHatZero$ by finding $\PupilGraphScorer$ s.t. $\PupilGraphScorer\pn{\gDAggerDatasetDatum} \geq \LossFunctionMSENormalizationConstant$ for all $\pn{\gDAggerDatasetDatum}$.
    This can be achieved by setting the kernel weights and bias weights (if present) of $\PupilName$ to zero, and setting $\PupilGraphScorerBias$ of $\PupilGraphScorer$ in \eqref{eq:generalization-gap:scoring-model} to $\PupilGraphScorerBias = \sqrt{\LossFunctionMSENormalizationConstant}$. Therefore, we choose $\GeneralizationFunctionDatumToLossHatZero\pn{\cdot} = 1 - \LossFunctionName\pn{\PupilGraphScorer\pn{\cdot; \set{\mzero}_{\PupilLayerIndex = 1}^{\PupilLayerCount}, \sqrt{\LossFunctionMSENormalizationConstant}}}$. With $\GeneralizationFunctionDatumToLossHatZero$ found, we follow 
    %the remaining steps up to equation~(31) in 
    \cite{pmlr-v235-karczewski24a} to obtain
$$%\small
    \begin{aligned}
        \GeneralizationEmpiricalRademacherComplexity\pn{\GeneralizationFunctionsDatumToLoss} \leq \frac{4}{\GeneralizationSampleSize} + \frac{24}{\GeneralizationSampleSize}\sqrt{\ln\GeneralizationCoveringNumber\pn{\GeneralizationFunctionsLearners, \frac{1}{2\sqrt{\GeneralizationSampleSize}}, \UtilNormWithPlaceholder_F}}.
    \end{aligned}
$$
    Next, we adapt the steps in \cite{pmlr-v235-karczewski24a} to bound $\ln\GeneralizationCoveringNumber\pn{\GeneralizationFunctionsLearners, \frac{1}{2\sqrt{\GeneralizationSampleSize}}, \UtilNormWithPlaceholder_\infty}$.
    Using \Lemmaref{lemma:results:scoring-function-locally-lipshitz}, we bound the supremum distance between functions in $\GeneralizationFunctionsLearners$ by a function of the distance between their weight matrices:
$$%\small
    \begin{aligned}
        & \ \ \norm{\PupilGraphScorer\pn{\cdot; \PupilWeightMatrixCollection, \PupilGraphScorerBias} - \PupilGraphScorer\pn{\cdot; \tilde{\PupilWeightMatrixCollection}, \tilde\PupilGraphScorerBias}}_\infty\\ & = \sup_{\pn{\gDAggerDatasetDatum} \in \GeneralizationDatasetDistribution} \abs{\PupilGraphScorer\pn{\gDAggerDatasetDatum;\PupilWeightMatrixCollection} - \PupilGraphScorer\pn{\gDAggerDatasetDatum;\tilde{\PupilWeightMatrixCollection}}} \\
        & \leq 2\PupilWeightMatrixBound_\PupilGraphScorer\abs{\PupilGraphScorerBias - \tilde{\PupilGraphScorerBias}} + 2\pn{1 + \PupilLipshitzConstantName}\DAggerDatumBound^2\PupilLipshitzConstantName\sum_{\PupilLayerIndex = 1}^{\PupilLayerCount} \norm{\PupilWeightMatrix_\PupilLayerIndex - \tilde{\PupilWeightMatrix}_\PupilLayerIndex}_F
        \\
        & \leq \underbrace{2\pn{\PupilWeightMatrixBound_\PupilGraphScorer + \pn{1 + \PupilLipshitzConstantName}\DAggerDatumBound^2\PupilLipshitzConstantName}}_{:= \PupilGraphScorerLipshitzConstantName}\pn{\abs{\PupilGraphScorerBias - \tilde{\PupilGraphScorerBias}} +\sum_{\PupilLayerIndex = 1}^{\PupilLayerCount} \norm{\PupilWeightMatrix_\PupilLayerIndex - \tilde{\PupilWeightMatrix}_\PupilLayerIndex}_F}.
    \end{aligned}
$$
    Using that bound, Lemma~G.1 from \cite{pmlr-v235-karczewski24a} bounds the covering number of $\GeneralizationFunctionsLearners$ by the covering number of the weight matrices.
$$%\small
    \begin{aligned}
        \ln\GeneralizationCoveringNumber\pn{\GeneralizationFunctionsLearners, r, \UtilNormWithPlaceholder_\infty} \leq \ln\GeneralizationCoveringNumber\pn{\PupilGraphScorerBias \in \spn{0, \sqrt{\LossFunctionMSENormalizationConstant}}, \frac{r}{\PupilLayerCount\PupilGraphScorerLipshitzConstantName}, \abs{\;\cdot\;}} + \sum_{\PupilLayerIndex = 1}^{\PupilLayerCount} \ln\GeneralizationCoveringNumber\pn{\PupilWeightMatrixCollection_\PupilLayerIndex, \frac{r}{\PupilLayerCount\PupilGraphScorerLipshitzConstantName}, \UtilNormWithPlaceholder_F},
    \end{aligned}
$$
    where $\PupilWeightMatrixCollection_\PupilLayerIndex$ is the set of possible matrices for $\PupilWeightMatrix_\PupilLayerIndex$. Using Lemma~3.2 from \cite{chenGeneralizationBoundsFamily2019},
    %Next, \cite{pmlr-v235-karczewski24a} cites lemma~3.2 from \cite{chenGeneralizationBoundsFamily2019} to bound the covering number of $\PupilWeightMatrixCollection_\PupilLayerIndex$.
$$%\small
    \begin{aligned}
        \ln\GeneralizationCoveringNumber\pn{\PupilWeightMatrixCollection_\PupilLayerIndex, \frac{r}{\PupilLayerCount\PupilGraphScorerLipshitzConstantName}, \UtilNormWithPlaceholder_F} \leq \PupilWeightMatrixLargestDim^2\ln\pn{1 + 2\frac{\PupilLayerCount\PupilGraphScorerLipshitzConstantName\PupilWeightMatrixBound_\PupilLayerIndex\sqrt{\PupilWeightMatrixLargestDim}}{r}},
    \end{aligned}
$$
    where $\PupilWeightMatrixLargestDim = \max_\PupilLayerIndex \dim\pn{\PupilWeightMatrix_\PupilLayerIndex}$.
    Choosing $r = \frac{1}{2\sqrt{\GeneralizationSampleSize}}$,
$$%\small
    \begin{aligned}
        \ln\GeneralizationCoveringNumber\pn{\GeneralizationFunctionsLearners, \frac{1}{2\sqrt{\GeneralizationSampleSize}}, \UtilNormWithPlaceholder_\infty} & \leq
        \ln\pn{1 + 4\PupilLayerCount\PupilGraphScorerLipshitzConstantName\PupilWeightMatrixBound_\PupilGraphScorer\sqrt{\GeneralizationSampleSize}} +
        \PupilWeightMatrixLargestDim^2\sum_{\PupilLayerIndex = 1}^{\PupilLayerCount} \ln\pn{1 + 4\PupilLayerCount\PupilGraphScorerLipshitzConstantName\PupilWeightMatrixBound_\PupilLayerIndex\sqrt{\PupilWeightMatrixLargestDim\GeneralizationSampleSize}}
        \\ & \leq
        \ln\pn{5\PupilLayerCount\PupilGraphScorerLipshitzConstantName\PupilWeightMatrixBound_\PupilGraphScorer\sqrt{\GeneralizationSampleSize}} +
        \PupilWeightMatrixLargestDim^2\sum_{\PupilLayerIndex = 1}^{\PupilLayerCount} \ln\pn{5\PupilLayerCount\PupilGraphScorerLipshitzConstantName\PupilWeightMatrixBound_\PupilLayerIndex\sqrt{\PupilWeightMatrixLargestDim\GeneralizationSampleSize}}
        \\ & \leq
        \PupilWeightMatrixLargestDim^2\ln\pn{5\PupilLayerCount\PupilGraphScorerLipshitzConstantName\PupilWeightMatrixBound_\PupilGraphScorer\sqrt{\GeneralizationSampleSize}} +
        \PupilWeightMatrixLargestDim^2\sum_{\PupilLayerIndex = 1}^{\PupilLayerCount} \ln\pn{5\PupilLayerCount\PupilGraphScorerLipshitzConstantName\PupilWeightMatrixBound_\PupilLayerIndex\sqrt{\PupilWeightMatrixLargestDim\GeneralizationSampleSize}}.
    \end{aligned}
$$
    Plugging into the bound for empirical Rademacher complexity,
$$%\small
    \begin{aligned}
        \GeneralizationEmpiricalRademacherComplexity\pn{\GeneralizationFunctionsDatumToLoss} \leq \frac{4}{\GeneralizationSampleSize} + \frac{24\PupilWeightMatrixLargestDim}{\sqrt{\GeneralizationSampleSize}}\sqrt{\underbrace{\ln\pn{5\PupilLayerCount\PupilGraphScorerLipshitzConstantName\PupilWeightMatrixBound_\PupilGraphScorer\sqrt{\GeneralizationSampleSize}} + \sum_{\PupilLayerIndex = 1}^{\PupilLayerCount} \ln\pn{5\PupilLayerCount\PupilGraphScorerLipshitzConstantName\PupilWeightMatrixBound_\PupilLayerIndex\sqrt{\PupilWeightMatrixLargestDim\GeneralizationSampleSize}}}_{:= \Sigma}}.
    \end{aligned}
$$
    Next, we simplify the bound.
    Using logarithm identities,
$$%\small
    \begin{aligned}
        \Sigma & =
        \frac{\PupilLayerCount}{2}\ln\pn{\PupilWeightMatrixLargestDim} +
        \pn{\PupilLayerCount + 1}\spn{\ln\pn{5\PupilLayerCount\sqrt{\GeneralizationSampleSize}} + \ln\pn{\PupilGraphScorerLipshitzConstantName}} + \ln\pn{\PupilWeightMatrixBound_\PupilGraphScorer} + \sum_{\PupilLayerIndex = 1}^{\PupilLayerCount} \ln\pn{\PupilWeightMatrixBound_\PupilLayerIndex}.
    \end{aligned}
$$
    Finding an upper bound for $\ln\pn{\PupilGraphScorerLipshitzConstantName}$,
$$%\small
    \begin{aligned}
        \ln\pn{\PupilGraphScorerLipshitzConstantName} & = \ln\pn{2} + \ln\pn{\PupilWeightMatrixBound_\PupilGraphScorer + \pn{1 + \PupilLipshitzConstantName}\DAggerDatumBound^2\PupilLipshitzConstantName} \\
        & \leq \ln\pn{2} + \ln\pn{\PupilWeightMatrixBound_\PupilGraphScorer + \PupilWeightMatrixBound_\PupilGraphScorer\pn{1 + \PupilLipshitzConstantName}\DAggerDatumBound^2\PupilLipshitzConstantName} \\
        %& = \ln\pn{2} + \ln\pn{\PupilWeightMatrixBound_\PupilGraphScorer} + \ln\pn{1 + \pn{1 + \PupilLipshitzConstantName}\DAggerDatumBound^2\PupilLipshitzConstantName} \\
        & \leq \ln\pn{2} + \ln\pn{\PupilWeightMatrixBound_\PupilGraphScorer} + \ln\pn{2\pn{1 + \PupilLipshitzConstantName}\DAggerDatumBound^2\PupilLipshitzConstantName} \\
        & = 2\ln\pn{2} + \ln\pn{\PupilWeightMatrixBound_\PupilGraphScorer} + \ln\pn{1 + \PupilLipshitzConstantName} + 2\ln\pn{\DAggerDatumBound} + \ln\pn{\PupilLipshitzConstantName} \\
        & \leq 2\ln\pn{2} + \ln\pn{\PupilWeightMatrixBound_\PupilGraphScorer} + \ln\pn{2\PupilLipshitzConstantName} + 2\ln\pn{\DAggerDatumBound} + \ln\pn{\PupilLipshitzConstantName} \\
        & = 3\ln\pn{2} + \ln\pn{\PupilWeightMatrixBound_\PupilGraphScorer} + 2\ln\pn{\PupilLipshitzConstantName} + 2\ln\pn{\DAggerDatumBound} \\
        & = 3\ln\pn{2} + \ln\pn{\PupilWeightMatrixBound_\PupilGraphScorer} + 2\ln\pn{\PupilLipshitzConstantDefn} + 2\ln\pn{\DAggerDatumBound} \\
        & = 3\ln\pn{2} + \ln\pn{\PupilWeightMatrixBound_\PupilGraphScorer} + 2\ln\pn{\DAggerDatumBound} + 2\ln\pn{\PupilActivationLipshitzConstant^{\PupilLayerCount}} + 2\sum_{\PupilLayerIndex = 1}^{\PupilLayerCount} \ln\pn{\PupilWeightMatrixBound_\PupilLayerIndex}.
    \end{aligned}
$$
    Combining these expressions, we find an upper bound for $\Sigma$:
$$%\small
    \begin{aligned}
        \Sigma
        % & = \frac{\PupilLayerCount}{2}\ln\pn{\PupilWeightMatrixLargestDim} + \pn{\PupilLayerCount + 1}\spn{\ln\pn{5\PupilLayerCount\sqrt{\GeneralizationSampleSize}} + \ln\pn{\PupilGraphScorerLipshitzConstantName}} + \ln\pn{\PupilWeightMatrixBound_\PupilGraphScorer} + \sum_{\PupilLayerIndex = 1}^{\PupilLayerCount} \ln\pn{\PupilWeightMatrixBound_\PupilLayerIndex} \\
        % edit(Taos): including this next step makes the orperations clear. We can skip repeating what $\Sigma$ equals instead.
        & \leq \frac{\PupilLayerCount}{2}\ln\pn{\PupilWeightMatrixLargestDim} + \pn{\PupilLayerCount + 1}\spn{\ln\pn{5\PupilLayerCount\sqrt{\GeneralizationSampleSize}} + 3\ln\pn{2} + 2\ln\pn{\DAggerDatumBound} + 2\ln\pn{\PupilActivationLipshitzConstant^{\PupilLayerCount}}} + \\ 
&\qquad\qquad        \pn{\PupilLayerCount + 2}\ln\pn{\PupilWeightMatrixBound_\PupilGraphScorer} + \pn{2\PupilLayerCount + 3}\sum_{\PupilLayerIndex = 1}^{\PupilLayerCount} \ln\pn{\PupilWeightMatrixBound_\PupilLayerIndex} \\
        & \leq \frac{\PupilLayerCount}{2}\ln\pn{\PupilWeightMatrixLargestDim} + 3\pn{\PupilLayerCount + 1}\spn{\ln\pn{5\PupilLayerCount\sqrt{\GeneralizationSampleSize}} + \ln\pn{2} + \ln\pn{\DAggerDatumBound} + \ln\pn{\PupilActivationLipshitzConstant^{\PupilLayerCount}}} + \\ &\qquad\qquad \pn{\PupilLayerCount + 2}\ln\pn{\PupilWeightMatrixBound_\PupilGraphScorer} + \pn{2\PupilLayerCount + 3}\sum_{\PupilLayerIndex = 1}^{\PupilLayerCount} \ln\pn{\PupilWeightMatrixBound_\PupilLayerIndex} \\
        %& = \frac{\PupilLayerCount}{2}\ln\pn{\PupilWeightMatrixLargestDim} + 3\pn{\PupilLayerCount + 1}\ln\pn{10\PupilLayerCount\DAggerDatumBound\PupilActivationLipshitzConstant^{\PupilLayerCount}\sqrt{\GeneralizationSampleSize}} + \pn{\PupilLayerCount + 2}\ln\pn{\PupilWeightMatrixBound_\PupilGraphScorer} + \pn{2\PupilLayerCount + 3}\sum_{\PupilLayerIndex = 1}^{\PupilLayerCount} \ln\pn{\PupilWeightMatrixBound_\PupilLayerIndex} \\
        % & \leq 3\pn{\PupilLayerCount + 1}\ln\pn{10\PupilLayerCount\DAggerDatumBound\PupilActivationLipshitzConstant^{\PupilLayerCount}\sqrt{\PupilWeightMatrixLargestDim\GeneralizationSampleSize}} + \pn{\PupilLayerCount + 2}\ln\pn{\PupilWeightMatrixBound_\PupilGraphScorer} + \pn{2\PupilLayerCount + 3}\sum_{\PupilLayerIndex = 1}^{\PupilLayerCount} \ln\pn{\PupilWeightMatrixBound_\PupilLayerIndex} \\
        & \leq 3\pn{\PupilLayerCount + 1}\ln\pn{10\PupilLayerCount\DAggerDatumBound\PupilActivationLipshitzConstant^{\PupilLayerCount}\PupilWeightMatrixBound_\PupilGraphScorer\sqrt{\PupilWeightMatrixLargestDim\GeneralizationSampleSize}} + \pn{2\PupilLayerCount + 3}\sum_{\PupilLayerIndex = 1}^{\PupilLayerCount} \ln\pn{\PupilWeightMatrixBound_\PupilLayerIndex}.
    \end{aligned}
$$
    Using that $\PupilWeightMatrixBound_\PupilGraphScorer = \sqrt{\LossFunctionMSENormalizationConstant}$, $\PupilWeightMatrixBound_\PupilLayerIndex \leq \max\set{1, \norm{\PupilWeightMatrix_\PupilLayerIndex}_F}$, and Theorem~\ref{theorem:erc-bounds-generalization-gap}, we attain the bound for $\gGeneralizationGap\pn{\PupilGraphScorer}$.
\end{proof}





\subsubsection{Convolutional layers as linear layers}

{
\newcommand{\zWeightBias}{{w_{\mathrm{bias}}}}
\begin{lemma}[$\ConvOneD_\PupilLayerIndex$ as a linear layer]
    \label{lemma:results:conv1d-as-linear}
    Assume that all input channels to $\ConvOneD_\PupilLayerIndex: \bR^{\gConvChannelsIn \times \gConvFeaturesIn} \to \bR^{\gConvChannelsIn[\PupilLayerIndex+1] \times \gConvFeaturesIn[\PupilLayerIndex+1]}$ are of length $\gConvFeaturesIn$, then
    $\ConvOneD_\PupilLayerIndex$ is expressible as a matrix multiplication with a fixed-dimension weight matrix.
\end{lemma}
\begin{proof}
    $\ConvOneD_\PupilLayerIndex: \bR^{\gConvChannelsIn \times \gConvFeaturesIn} \to \bR^{\gConvChannelsIn[\PupilLayerIndex+1] \times \gConvFeaturesIn[\PupilLayerIndex+1]}$ has $\gConvChannelsIn$ input channels that are row vectors of the form
$$%\small
\begin{aligned}
    \gConvInput{} [\ConvChannelInIndex] & = \spn{
        \pn{\gConvInput}_{\ConvChannelInIndex,1}
        ,\; \dots
        ,\; \pn{\gConvInput}_{\ConvChannelInIndex,\gConvFeaturesIn}
    }.
\end{aligned}
$$
    Arranging each input channel row vector end to end and appending $\vone_{\gConvFeaturesIn[\PupilLayerIndex+1]}^\top$ to account for the bias term, an output channel $\ConvChannelOutIndex \in \set{1,\dots,\gConvChannelsIn[\PupilLayerIndex+1]}$ is the row vector
$$%\small
    \begin{aligned}
        \gConvInput[\PupilLayerIndex+1][\ConvChannelOutIndex] = \ConvOneD_\PupilLayerIndex\pn{\gConvInput}[\ConvChannelOutIndex] & = \spn{
            \gConvInput{} [1]
            ,\; \dots
            ,\; \gConvInput{} [\gConvChannelsIn]
            ,\; \vone_{\gConvFeaturesIn[\PupilLayerIndex+1]}^\top
        }\begin{bmatrix}
            \PupilWeightMatrix\pn{1, \ConvChannelOutIndex} \\
            %\PupilWeightMatrix\pn{2, \ConvChannelOutIndex} \\
            \vdots \\
            \PupilWeightMatrix\pn{\gConvChannelsIn, \ConvChannelOutIndex} \\
            \zWeightBias\pn{\ConvChannelOutIndex}
        \end{bmatrix}.
    \end{aligned}
$$
    If $\PupilLayerIndex < \PupilLayerCount$, then all of the output channels may be computed with one matrix multiplication by
$$%\small
    \begin{aligned}
        & \spn{
            \gConvInput{} [1]
            ,\; \dots
            ,\; \gConvInput{} [\gConvChannelsIn]
            ,\; \vone_{\gConvFeaturesIn[\PupilLayerIndex+1]}^\top
        }\begin{bmatrix}
            \PupilWeightMatrix\pn{1, 1} & \dots & \PupilWeightMatrix\pn{1, \gConvChannelsIn[\PupilLayerIndex+1]} & \mzero \\
            %\PupilWeightMatrix\pn{2, 1} & \PupilWeightMatrix\pn{1, 2} & \dots & \PupilWeightMatrix\pn{1, \gConvChannelsIn[\PupilLayerIndex+1]} & \mzero \\
            \vdots & \ddots & \vdots \\
            \PupilWeightMatrix\pn{\gConvChannelsIn, 1} & \dots & \PupilWeightMatrix\pn{\gConvChannelsIn, \gConvChannelsIn[\PupilLayerIndex+1]} & \mzero \\
            \zWeightBias\pn{1}\mI & \dots & \zWeightBias\pn{\gConvChannelsIn[\PupilLayerIndex+1]}\mI & \mM
        \end{bmatrix}\\
       &\qquad = \spn{
            \ConvOneD_\PupilLayerIndex\pn{\gConvInput}[1]
            ,\; \dots
            ,\; \ConvOneD_\PupilLayerIndex\pn{\gConvInput}[\gConvChannelsIn[\PupilLayerIndex+1]]
            ,\; \vone_{\gConvFeaturesIn[\PupilLayerIndex+2]}^\top
        },
    \end{aligned}
$$
    where $\mM \in \bR^{\gConvFeaturesIn[\PupilLayerIndex+1] \times \gConvFeaturesIn[\PupilLayerIndex+2]}$ maps $\vone_{\gConvFeaturesIn[\PupilLayerIndex+1]}^\top$ to $\vone_{\gConvFeaturesIn[\PupilLayerIndex+2]}^\top$.
    If $\gConvFeaturesIn[\PupilLayerIndex + 1] = \gConvFeaturesIn[\PupilLayerIndex + 2]$, then $\mM = \mI_{\gConvFeaturesIn[\PupilLayerIndex+1] \times \gConvFeaturesIn[\PupilLayerIndex+1]}$.
    If $\gConvFeaturesIn[\PupilLayerIndex + 1] < \gConvFeaturesIn[\PupilLayerIndex + 2]$, then
$$%\small
    \begin{aligned}
        \mM = \begin{bmatrix}
            \mI_{\pn{\gConvFeaturesIn[\PupilLayerIndex+1] - 1} \times \pn{\gConvFeaturesIn[\PupilLayerIndex+1] - 1}} &
            \vzero_{\gConvFeaturesIn[\PupilLayerIndex+1]} &
            \mzero_{\pn{\gConvFeaturesIn[\PupilLayerIndex+1] - 1} \times \pn{\gConvFeaturesIn[\PupilLayerIndex+2] - \gConvFeaturesIn[\PupilLayerIndex+1]}}
            \\
            \vzero_{\gConvFeaturesIn[\PupilLayerIndex+1] - 1}^\top &
            1 &
            \vone_{\gConvFeaturesIn[\PupilLayerIndex+2] - \gConvFeaturesIn[\PupilLayerIndex+1]}^\top
        \end{bmatrix}
    \end{aligned}.
$$
    If $\gConvFeaturesIn[\PupilLayerIndex + 1] > \gConvFeaturesIn[\PupilLayerIndex + 2]$, then
$$%\small
    \begin{aligned}
        \mM = \begin{bmatrix}
            \mI_{\pn{\gConvFeaturesIn[\PupilLayerIndex+1] - \gConvFeaturesIn[\PupilLayerIndex+2]} \times \gConvFeaturesIn[\PupilLayerIndex+2]}
            \\
            \mzero_{\gConvFeaturesIn[\PupilLayerIndex+2] \times \gConvFeaturesIn[\PupilLayerIndex+2]}
        \end{bmatrix}
    \end{aligned}.
$$
    If $\PupilLayerIndex = \PupilLayerCount$, then all of the output channels may be computed with one matrix multiplication by
$$%\small
    \begin{aligned}
        & \spn{
            \gConvInput{} [1]
            ,\; \dots
            ,\; \gConvInput{} [\gConvChannelsIn]
            ,\; \vone_{\gConvFeaturesIn[\PupilLayerIndex+1]}^\top
        }\begin{bmatrix}
            \PupilWeightMatrix\pn{1, 1} & \dots & \PupilWeightMatrix\pn{1, \gConvChannelsIn[\PupilLayerIndex+1]} \\
            %\PupilWeightMatrix\pn{2, 1} & \PupilWeightMatrix\pn{1, 2} & \dots & \PupilWeightMatrix\pn{1, \gConvChannelsIn[\PupilLayerIndex+1]} \\
            \vdots & \ddots & \vdots \\
            \PupilWeightMatrix\pn{\gConvChannelsIn, 1} & \dots & \PupilWeightMatrix\pn{\gConvChannelsIn, \gConvChannelsIn[\PupilLayerIndex+1]} \\
            \zWeightBias\pn{1}\mI & \dots & \zWeightBias\pn{\gConvChannelsIn[\PupilLayerIndex+1]}\mI
        \end{bmatrix}\\
        &\qquad = \spn{
            \ConvOneD_\PupilLayerIndex\pn{\gConvInput}[1]
            ,\; \dots
            ,\; \ConvOneD_\PupilLayerIndex\pn{\gConvInput}[\gConvChannelsIn[\PupilLayerIndex+1]]
        }
    \end{aligned}.
$$
\end{proof}
}

\begin{lemma}[$\EqConv_\PupilLayerIndex$ as a linear layer]
    \label{lemma:results:eqconv-as-linear}
    Assume that all input channels to $\EqConv_\PupilLayerIndex: \bR^{\gConvChannelsIn \times \gConvFeaturesIn} \to \bR^{\gConvChannelsIn[\PupilLayerIndex+1] \times \gConvFeaturesIn[\PupilLayerIndex+1]}$ are of length $\gConvFeaturesIn$, then $\EqConv_\PupilLayerIndex$ is expressible as a matrix multiplication with a fixed-dimension weight matrix.
\end{lemma}
\begin{proof}
    $\EqConv_\PupilLayerIndex: \bR^{\gConvChannelsIn \times \gConvFeaturesIn} \to \bR^{\gConvChannelsIn[\PupilLayerIndex+1] \times \gConvFeaturesIn[\PupilLayerIndex+1]}$ has $\gConvChannelsIn/2$ input channels that are two-row matrices of the form
$$%\small
    \begin{aligned}
        \gConvInput{} [\ConvChannelInIndex] & = \spn{
            \EqConvChannel_{\ConvChannelInIndex,1}
            ,\; \dots
            ,\; \EqConvChannel_{\ConvChannelInIndex,\gConvFeaturesIn}
        }
    \end{aligned}.
$$
    Arranging each input channel matrix end to end, an output channel $\ConvChannelOutIndex \in \set{1,\dots,\gConvChannelsIn[\PupilLayerIndex+1]/2}$ is 
    %the two-row matrix
$$%\small
    \begin{aligned}
        \gConvInput[\PupilLayerIndex+1][\ConvChannelOutIndex] = \EqConv_\PupilLayerIndex\pn{\gConvInput}[\ConvChannelOutIndex] & = \spn{
            \gConvInput{} [1]
            ,\; \dots
            ,\; \gConvInput{} [\gConvChannelsIn/2]
        }\begin{bmatrix}
            \PupilWeightMatrix\pn{1, \ConvChannelOutIndex} \\
            \vdots \\
            \PupilWeightMatrix\pn{\gConvChannelsIn/2, \ConvChannelOutIndex} \\
        \end{bmatrix}
    \end{aligned}.
$$
    All of the output channels may be computed with one matrix multiplication by
$$%\small
    \begin{aligned}
        & \spn{
            \gConvInput{} [1]
            ,\; \dots
            ,\; \gConvInput{} [\gConvChannelsIn/2]
        }\begin{bmatrix}
            \PupilWeightMatrix\pn{1, 1}
            & \dots
            & \PupilWeightMatrix\pn{1, \gConvChannelsIn[\PupilLayerIndex+1]/2} \\
            \vdots & \ddots & \vdots \\
            \PupilWeightMatrix\pn{\gConvChannelsIn/2, 1}
            & \dots
            & \PupilWeightMatrix\pn{\gConvChannelsIn/2, \gConvChannelsIn[\PupilLayerIndex+1]/2} \\
        \end{bmatrix}\\
        &\qquad = \spn{
            \EqConv_\PupilLayerIndex\pn{\gConvInput}[1]
            ,\; \dots
            ,\; \EqConv_\PupilLayerIndex\pn{\gConvInput}[\gConvChannelsIn[\PupilLayerIndex+1]/2]
        }.
    \end{aligned}
$$
\end{proof}


\subsubsection{Equivariant functions}

{
\newcommand{\zEquivFunc}{\vf}
\newcommand{\zEquivMatrixFunc}{\mF}
\newcommand{\zInput}{x}
\newcommand{\zvInput}{\bm{\zInput}}
\newcommand{\zmInput}{\bm{G}}
\begin{lemma}
    \label{lemma:results:equivariant-zero-at-zero}
    \OrthogonalGroup{n} equivariant functions $\zEquivMatrixFunc$ satisfy $\zEquivMatrixFunc\pn{\mzero} = \mzero$.
\end{lemma}
\begin{proof}
    By Proposition~1 in \cite{maWhySelfattentionNatural2022}, there exists a function $\PupilActivationInvariantName$ such that
    \begin{align*}
        \zEquivMatrixFunc\pn{\zmInput} = \zmInput \PupilActivationInvariantName\pn{\zmInput^\top\zmInput},
    \end{align*}
    where the output of $\PupilActivationInvariantName$ has the appropriate shape.
    Then, $\zEquivMatrixFunc\pn{\mzero} = \mzero \PupilActivationInvariantName\pn{\mzero^\top\mzero} = \mzero$.
\end{proof}

\subsubsection{Lipshitz \OrthogonalGroup{n} equivariant functions}

We derive a sufficient condition for an \OrthogonalGroup{n} equivariant function to be Lipshitz.
\begin{lemma}
\label{lemma:results:equivariant-func-lipshitz}
    Let $\zEquivFunc: \bR^n \to \bR^n$ be an \OrthogonalGroup{n} equivariant function.
    By Proposition~1 in \cite{maWhySelfattentionNatural2022}, $\zEquivFunc\pn{\zvInput} = \zvInput \PupilActivationInvariantName\pn{\norm{\zvInput}}$ where $\PupilActivationInvariantName$ is scalar-valued.
    If \PupilActivationInvariantName{} is continuously differentiable, $\lim_{\zInput \to \infty} \PupilActivationInvariantName\pn{\zInput} < \infty$, and $\lim_{\zInput \to \infty} \zInput\PupilActivationInvariantName'\pn{\zInput} < \infty$, then $\zEquivFunc\pn{\zvInput}$ is Lipshitz.
    Moreover, $\zEquivFunc\pn{\zvInput}$ has Lipshitz constant $L = \max_{\zInput \geq 0} \PupilActivationInvariantName\pn{\zInput} + \zInput\PupilActivationInvariantName'\pn{\zInput}$.
\end{lemma}
\begin{proof}
    We start by bounding the derivative of $\zEquivFunc\pn{\zvInput}$ in the spectral norm:
$$%\small
    \begin{aligned}
        \norm*{\ddfrac{\zvInput}\zEquivFunc\pn{\zvInput}}_2 & = \norm*{\PupilActivationInvariantName\pn{\norm{\zvInput}}\mI + \zvInput\ddfrac{\zvInput}\PupilActivationInvariantName\pn{\norm{\zvInput}}}_2 \\
        & = \norm*{\PupilActivationInvariantName\pn{\norm{\zvInput}}\mI + \frac{\PupilActivationInvariantName'\pn{\norm{\zvInput}}}{\norm{\zvInput}}\zvInput\zvInput^\top}_2 \\
        & \leq \norm*{\PupilActivationInvariantName\pn{\norm{\zvInput}}\mI}_2 + \norm*{\frac{\PupilActivationInvariantName'\pn{\norm{\zvInput}}}{\norm{\zvInput}}\zvInput\zvInput^\top}_2 \\
        %& = \PupilActivationInvariantName\pn{\norm{\zvInput}} + \frac{\PupilActivationInvariantName'\pn{\norm{\zvInput}}}{\norm{\zvInput}}\norm*{\zvInput\zvInput^\top}_2 \\
        & \leq \PupilActivationInvariantName\pn{\norm{\zvInput}} + \frac{\PupilActivationInvariantName'\pn{\norm{\zvInput}}}{\norm{\zvInput}}\norm{\zvInput}^2 \\
        & = \PupilActivationInvariantName\pn{\norm{\zvInput}} + \norm{\zvInput}\PupilActivationInvariantName'\pn{\norm{\zvInput}}
    \end{aligned}.
$$
    By the assumptions on \PupilActivationInvariantName{},
$$%\small
    \begin{aligned}
        \norm*{\ddfrac{\zvInput}\zEquivFunc\pn{\zvInput}}_2 & \leq \PupilActivationInvariantName\pn{\norm{\zvInput}} + \norm{\zvInput}\PupilActivationInvariantName'\pn{\norm{\zvInput}} < \infty
    \end{aligned}.
$$
    Since \PupilActivationInvariantName{} is continuously differentiable, $L = \sup_{\zInput \geq 0} \PupilActivationInvariantName\pn{\zInput} + \zInput\PupilActivationInvariantName'\pn{\zInput} < \infty$.
    Finally, $\zEquivFunc\pn{\zvInput}$ is Lipshitz with constant $L$ since, for $\va, \PupilFeatureBlockVector \in \bR^n$,
$$%\small
    \begin{aligned}
        \norm{\zEquivFunc\pn{\va} - \zEquivFunc\pn{\PupilFeatureBlockVector}} & \leq \norm{\va - \PupilFeatureBlockVector}\sup_{\zvInput}\norm*{\frac{\mathrm{d}\zEquivFunc}{\mathrm{d}\zvInput}\pn{\zvInput}}_2 \leq L\norm{\va - \PupilFeatureBlockVector}.
    \end{aligned}
$$
\end{proof}

Next, we show that the block-wise application of a Lipshitz \OrthogonalGroup{n} equivariant function is Lipshitz.
\begin{lemma}
    \label{lemma:results:equivariant-block-matrix-func-lipshitz}
    Let $\zEquivMatrixFunc: \bR^{n \times \gConvFeaturesIn[\PupilLayerIndex + 1]} \to \bR^{n \times \gConvFeaturesIn[\PupilLayerIndex + 1]}$ be given by
$$%\small
    \begin{aligned}
        \zEquivMatrixFunc\pn{\zmInput} = \zmInput \odot \vone_n\spn{
            \PupilActivationInvariantName\pn{\norm{\EqConvChannel_{1}}}
            ,\; \dots
            ,\; \PupilActivationInvariantName\pn{\norm{\EqConvChannel_{\gConvFeaturesIn[\PupilLayerIndex+1]}}}
        },
    \end{aligned}
$$
    where $\zEquivFunc\pn{\zvInput} = \zvInput\PupilActivationInvariantName\pn{\norm{\zvInput}}$ is \OrthogonalGroup{2} equivariant and Lipshitz with constant $L$.
    $\zEquivMatrixFunc\pn{\zmInput}$ is Lipshitz with constant $L$:
    $$%\small
    \begin{aligned}
        \norm*{\zEquivMatrixFunc\pn{\mA} - \zEquivMatrixFunc\pn{\mB}}_F \leq L\norm*{\mA - \mB}_F.
    \end{aligned}
    $$
\end{lemma}
\begin{proof}
    This is proven using the definition of the Frobenius norm:
$$%\small
    \begin{aligned}
        \norm*{\zEquivMatrixFunc\pn{\mA} - \zEquivMatrixFunc\pn{\mB}}_F^2 & = \sum_{\ConvFeatureIndex = 1}^{\gConvFeaturesIn[\PupilLayerIndex+1]} \norm*{\zEquivFunc\pn{\va_\ConvFeatureIndex} - \zEquivFunc\pn{\vb_\ConvFeatureIndex}}^2 \leq L^2 \sum_{\ConvFeatureIndex = 1}^{\gConvFeaturesIn[\PupilLayerIndex+1]} \norm*{\va_\ConvFeatureIndex - \vb_\ConvFeatureIndex}^2 \leq L^2\norm*{\mA - \mB}_F^2,
    \end{aligned}
$$
    Taking the square root of both sizes gives the result.
    % and therefore,
    % $$\small
    % \begin{aligned}
    %     \norm*{\zEquivMatrixFunc\pn{\mA} - \zEquivMatrixFunc\pn{\mB}}_F \leq L\norm*{\mA - \mB}_F
    % \end{aligned}.
    % $$
\end{proof}

\begin{lemma}
    \label{lemma:results:equivariant-block-matrix-func-lipshitz-multiple-output-channels}
    Let $\zEquivMatrixFunc: \bR^{\pn{n\gConvChannelsIn[\PupilLayerIndex+1]/n} \times \gConvFeaturesIn[\PupilLayerIndex+1]}$ such that $\gConvChannelsIn[\PupilLayerIndex+1]/n \in \bN$.
    For $\ConvChannelOutIndex \in \set{1,\dots,\gConvChannelsIn[\PupilLayerIndex+1]/n}$ and $\zmInput$ where
    $
    \begin{aligned}
        \zmInput[\ConvChannelOutIndex] = \spn{
            \EqConvChannel_{\ConvChannelOutIndex,1}
            ,\; \dots
            ,\; \EqConvChannel_{\ConvChannelOutIndex,\gConvFeaturesIn[\PupilLayerIndex+1]}
        } \in \bR^{n \times \gConvFeaturesIn[\PupilLayerIndex+1]}
    \end{aligned}
    $, suppose
$$
    \begin{aligned}
        \zEquivMatrixFunc\pn{\zmInput}[\ConvChannelOutIndex] = \zmInput[\ConvChannelOutIndex] \odot \vone_n\spn{
            \PupilActivationInvariantName\pn{\norm{\EqConvChannel_{\ConvChannelOutIndex,1}}}
            ,\; \dots
            ,\; \PupilActivationInvariantName\pn{\norm{\EqConvChannel_{\ConvChannelOutIndex,\gConvFeaturesIn[\PupilLayerIndex+1]}}}
        }
    \end{aligned}.
$$
    If $\zEquivFunc: \bR^n \to \bR^n$ is as defined in \Lemmaref{lemma:results:equivariant-block-matrix-func-lipshitz} with Lipshitz constant $L$, then
    $$%\small
    \begin{aligned}
        \norm*{\zEquivMatrixFunc\pn{\mA} - \zEquivMatrixFunc\pn{\mB}}_F \leq L\norm*{\mA - \mB}_F.
    \end{aligned}
    $$
\end{lemma}
\begin{proof}
    By \Lemmaref{lemma:results:equivariant-block-matrix-func-lipshitz} and the definition of the Frobenius norm,
    $$%\small
    \begin{aligned}
        \norm{\zEquivMatrixFunc\pn{\mA} - \zEquivMatrixFunc\pn{\mB}}_F^2 & = \sum_{\ConvChannelOutIndex = 1}^{\gConvChannelsIn[\PupilLayerIndex+1]/n} \norm{\zEquivMatrixFunc\pn{\mA}[\ConvChannelOutIndex] - \zEquivMatrixFunc\pn{\mB}[\ConvChannelOutIndex]}_F^2\\
        &\ \leq L^2\sum_{\ConvChannelOutIndex = 1}^{\gConvChannelsIn[\PupilLayerIndex+1]/n} \norm{\mA[\ConvChannelOutIndex] - \mB[\ConvChannelOutIndex]}_F^2 = L^2\norm{\mA - \mB}_F^2.
    \end{aligned}
    $$
    Taking the square root of both sizes gives the result.
    % implying $\norm{\zEquivMatrixFunc\pn{\mA} - \zEquivMatrixFunc\pn{\mB}}_F \leq L\norm{\mA - \mB}_F$.
\end{proof}

Using Lemmas~\ref{lemma:results:equivariant-func-lipshitz}, \ref{lemma:results:equivariant-block-matrix-func-lipshitz}, and \ref{lemma:results:equivariant-block-matrix-func-lipshitz-multiple-output-channels}, we show that the activations \PupilETDAGNNActivationScaleLog{} and \PupilETDAGNNActivationScaleTanh{} in \PupilEquivariantName{} are Lipshitz.
\begin{lemma}
    \label{lemma:results:etdagnn-sigma0-lipshitz}
    \OrthogonalGroup{n} equivariant activation \PupilETDAGNNActivationScaleLog{} is Lipshitz.
\end{lemma}
\begin{proof}
    Define
    $$
        \PupilETDAGNNActivationScaleLogInvariant\pn{\zInput} = \begin{cases}
            1 & \zInput = 0 \\
            \frac{\ln\pn{1 + \zInput}}{\zInput} & \zInput \neq 0
        \end{cases}
    $$
    with derivative
    $$
        \PupilETDAGNNActivationScaleLogInvariant'\pn{\zInput} = \begin{cases}
            0 & \zInput = 0 \\
            \frac{1}{\zInput\pn{1 + \zInput}} - \frac{\ln\pn{1 + \zInput}}{\zInput^2} & \zInput \neq 0
        \end{cases}.
    $$
    In order to apply \Lemmaref{lemma:results:equivariant-func-lipshitz}, we compute the following limits:
    \begin{align*}
        \lim_{\zInput \to \infty} \PupilETDAGNNActivationScaleLogInvariant\pn{\zInput} & = \lim_{\zInput \to \infty} \frac{\ln\pn{1 + \zInput}}{\zInput} \substack{\text{L'H\^opital}\\=} \lim_{\zInput \to \infty} \frac{1}{1 + \zInput} = 0 < \infty, \\
        \lim_{\zInput \to \infty} \zInput\PupilETDAGNNActivationScaleLogInvariant'\pn{\zInput} & = \lim_{\zInput \to \infty} \frac{1}{1 + \zInput} - \frac{\ln\pn{1 + \zInput}}{\zInput}\\
        &= 0 - \lim_{\zInput \to \infty} \frac{\ln\pn{1 + \zInput}}{\zInput} \substack{\text{L'H\^opital}\\=} 0 - \lim_{\zInput \to \infty} \frac{1}{1 + \zInput} = 0 - 0 = 0.
    \end{align*}
    By \Lemmaref{lemma:results:equivariant-func-lipshitz}, $\zEquivFunc\pn{\zvInput} = \zvInput\PupilETDAGNNActivationScaleLogInvariant\pn{\norm{\zvInput}}$ is Lipshitz with constant $L = \max_{\zInput \geq 0} \PupilETDAGNNActivationScaleLogInvariant\pn{\zInput} + \zInput\PupilETDAGNNActivationScaleLogInvariant\pn{\zInput}$.
    Applying Lemmas~\ref{lemma:results:equivariant-block-matrix-func-lipshitz} and \ref{lemma:results:equivariant-block-matrix-func-lipshitz-multiple-output-channels}, $\PupilETDAGNNActivationScaleLog\pn{\zmInput}$ is Lipshitz with constant $L$.
\end{proof}

\begin{lemma}
    \label{lemma:results:etdagnn-sigma1-lipshitz}
    \OrthogonalGroup{n} equivariant activation \PupilETDAGNNActivationScaleTanh{} is Lipshitz.
\end{lemma}
\begin{proof}
    Define $\PupilETDAGNNActivationScaleTanhInvariant\pn{\zInput} = \tanh\pn{\zInput}$, and then $\PupilETDAGNNActivationScaleTanhInvariant'\pn{\zInput} = 1 - \tanh^2\pn{\zInput}$.
    In order to apply \Lemmaref{lemma:results:equivariant-func-lipshitz}, we compute the following limits:
    $$%\small
    \begin{aligned}
        \lim_{\zInput \to \infty} \PupilETDAGNNActivationScaleTanhInvariant\pn{\zInput} & = \lim_{\zInput \to \infty} \tanh\pn{\zInput} = 1 < \infty, \\
        \lim_{\zInput \to \infty} \zInput\PupilETDAGNNActivationScaleTanhInvariant'\pn{\zInput} & = \lim_{\zInput \to \infty} \zInput\pn{1 - \tanh^2\pn{\zInput}} \substack{\text{ L'H\^opital}\\=} \lim_{\zInput \to \infty} \frac{1}{\frac{2\tanh\pn{\zInput}}{1 - \tanh^2\pn{\zInput}}} = \lim_{\zInput \to \infty} \frac{1 - \tanh^2\pn{\zInput}}{2\tanh\pn{\zInput}} = 0.
    \end{aligned}
    $$
    By \Lemmaref{lemma:results:equivariant-func-lipshitz}, $\zEquivFunc\pn{\zvInput} = \zvInput\PupilETDAGNNActivationScaleTanhInvariant\pn{\norm{\zvInput}}$ is Lipshitz with constant $L = \max_{\zInput \geq 0} \PupilETDAGNNActivationScaleTanhInvariant\pn{\zInput} + \zInput\PupilETDAGNNActivationScaleTanhInvariant\pn{\zInput}$.
    Applying Lemmas~\ref{lemma:results:equivariant-block-matrix-func-lipshitz} and \ref{lemma:results:equivariant-block-matrix-func-lipshitz-multiple-output-channels}, $\PupilETDAGNNActivationScaleTanh\pn{\zmInput}$ is Lipshitz with constant $L$.
\end{proof}
}

\subsection{Obstacle avoidance}

{
\newcommand{\zRotationMatrix}{{\UtilRotationMatrix}}
\newcommand{\zAngle}{{\theta}}
\newcommand{\zLinearDiscriminantName}{{\gamma}}
\newcommand{\zFlockAgentVelocity}{\vv}
\newcommand{\zVelocityNeighborAverage}{{\overline{\zFlockAgentVelocity}}}
\newcommand{\zRelativeVelocityMagnitude}{{\alpha_1}}
\newcommand{\zRelativeVelocityGravityBreak}{{\alpha_2}}
\newcommand{\zRelativeVelocityObstacleDodgeAngle}{{\alpha_\zAngle}}

\begin{lemma}
    $\zLinearDiscriminantName\pn{\FlockAgentRelativePos[], \zFlockAgentVelocity, \zAngle}$ from \eqref{eq:results:obstacle-avoidance-linear-discriminant} is invariant with respect to \SpecialOrthogonalGroup{2}.
\end{lemma}
\begin{proof}
    Let $\zRotationMatrix \in $ \SpecialOrthogonalGroup{2}.
    $$%\small
    \begin{aligned}
        \zLinearDiscriminantName\pn{\zRotationMatrix\FlockAgentRelativePos[], \zRotationMatrix\zFlockAgentVelocity, \zAngle} & = \frac{\pn{\zRotationMatrix\FlockAgentRelativePos[]}^\top}{\norm{\zRotationMatrix\FlockAgentRelativePos[]}}\zRotationMatrix\pn{\zAngle}\frac{\zRotationMatrix\zFlockAgentVelocity}{\norm{\zRotationMatrix\zFlockAgentVelocity}} = \frac{\FlockAgentRelativePos[]^\top\zRotationMatrix^\top\zRotationMatrix\pn{\zAngle}\zRotationMatrix\zFlockAgentVelocity}{\norm{\FlockAgentRelativePos[]}\norm{\zFlockAgentVelocity}} \\
        & = \frac{\FlockAgentRelativePos[]^\top\zRotationMatrix^\top\zRotationMatrix\zRotationMatrix\pn{\zAngle}\zFlockAgentVelocity}{\norm{\FlockAgentRelativePos[]}\norm{\zFlockAgentVelocity}} \quad \text{since matrices of \SpecialOrthogonalGroup{2} commute} \\
        & = \frac{\FlockAgentRelativePos[]^\top\mI\zRotationMatrix\pn{\zAngle}\zFlockAgentVelocity}{\norm{\FlockAgentRelativePos[]}\norm{\zFlockAgentVelocity}} = \zLinearDiscriminantName\pn{\FlockAgentRelativePos[], \zFlockAgentVelocity, \zAngle}.
    \end{aligned}
    $$
    Therefore, $\zLinearDiscriminantName$ is invariant with respect to \SpecialOrthogonalGroup{2}.
\end{proof}

\begin{proposition}
    The relative velocity in \eqref{eq:results:obstacle-avoidance:relative-velocity} is equivariant with respect to \SpecialOrthogonalGroup{2}.
\end{proposition}
\begin{proof}
    Let $\zRotationMatrix \in$ \SpecialOrthogonalGroup{2}.
    First, $\zVelocityNeighborAverage_\FlockAgentIndex$ is equivariant with respect to \SpecialOrthogonalGroup{2} since
    \begin{align*}
        \mean\set{\zRotationMatrix\FlockAgentVel_j\pn{t} : j \in \FlockAgentNeighborhood_\FlockAgentIndex\pn{t}} = \zRotationMatrix\mean\set{\FlockAgentVel_j\pn{t} : j \in \FlockAgentNeighborhood_\FlockAgentIndex\pn{t}} = \zRotationMatrix\zVelocityNeighborAverage_\FlockAgentIndex\pn{t}.
    \end{align*}
    We check each case of the relative velocity formula.
    When $-\zRelativeVelocityGravityBreak \leq \zLinearDiscriminantName\pn{\FlockAgentRelativePos[_{\FlockAgentIndexTwo\FlockAgentIndex}], \zVelocityNeighborAverage_\FlockAgentIndex, 0} \leq 0$,
    $$%\small
    \begin{aligned}
        \zRelativeVelocityMagnitude\pn{\norm{\zRotationMatrix\FlockAgentRelativePos[_{\FlockAgentIndexTwo\FlockAgentIndex}]}, \norm{\zRotationMatrix\zVelocityNeighborAverage_\FlockAgentIndex}}\frac{\zRotationMatrix\FlockAgentRelativePos[_{\FlockAgentIndexTwo\FlockAgentIndex}]}{\norm{\zRotationMatrix\FlockAgentRelativePos[_{\FlockAgentIndexTwo\FlockAgentIndex}]}} & = \zRelativeVelocityMagnitude\pn{\norm{\FlockAgentRelativePos[_{\FlockAgentIndexTwo\FlockAgentIndex}]}, \norm{\zVelocityNeighborAverage_\FlockAgentIndex}}\frac{\zRotationMatrix\FlockAgentRelativePos[_{\FlockAgentIndexTwo\FlockAgentIndex}]}{\norm{\FlockAgentRelativePos[_{\FlockAgentIndexTwo\FlockAgentIndex}]}} \\
        & = \zRotationMatrix\zRelativeVelocityMagnitude\pn{\norm{\FlockAgentRelativePos[_{\FlockAgentIndexTwo\FlockAgentIndex}]}, \norm{\zVelocityNeighborAverage_\FlockAgentIndex}}\frac{\FlockAgentRelativePos[_{\FlockAgentIndexTwo\FlockAgentIndex}]}{\norm{\FlockAgentRelativePos[_{\FlockAgentIndexTwo\FlockAgentIndex}]}} = \zRotationMatrix\pn{-\FlockAgentRelativeVel\pn{t}}.
    \end{aligned}
    $$
    For the other case,
    $$%\small
    \begin{aligned}
        & \zRelativeVelocityMagnitude\pn{\norm{\zRotationMatrix\FlockAgentRelativePos[_{\FlockAgentIndexTwo\FlockAgentIndex}]},\norm{\zRotationMatrix\zVelocityNeighborAverage_\FlockAgentIndex}}\pn{-\mathrm{sgn}\spn{\zLinearDiscriminantName\pn{\zRotationMatrix\FlockAgentRelativePos[_{\FlockAgentIndexTwo\FlockAgentIndex}], \zRotationMatrix\zVelocityNeighborAverage_\FlockAgentIndex, \frac{\pi}{2}}}\zRotationMatrix\pn{\zRelativeVelocityObstacleDodgeAngle}}\frac{\zRotationMatrix\FlockAgentRelativePos[_{\FlockAgentIndexTwo\FlockAgentIndex}]}{\norm{\zRotationMatrix\FlockAgentRelativePos[_{\FlockAgentIndexTwo\FlockAgentIndex}]}} \\
        = & \zRelativeVelocityMagnitude\pn{\norm{\FlockAgentRelativePos[_{\FlockAgentIndexTwo\FlockAgentIndex}]},\norm{\zVelocityNeighborAverage_\FlockAgentIndex}}\pn{-\mathrm{sgn}\spn{\zLinearDiscriminantName\pn{\FlockAgentRelativePos[_{\FlockAgentIndexTwo\FlockAgentIndex}], \zVelocityNeighborAverage_\FlockAgentIndex, \frac{\pi}{2}}}\zRotationMatrix\pn{\zRelativeVelocityObstacleDodgeAngle}}\frac{\zRotationMatrix\FlockAgentRelativePos[_{\FlockAgentIndexTwo\FlockAgentIndex}]}{\norm{\FlockAgentRelativePos[_{\FlockAgentIndexTwo\FlockAgentIndex}]}} \\
        = & \zRotationMatrix\zRelativeVelocityMagnitude\pn{\norm{\FlockAgentRelativePos[_{\FlockAgentIndexTwo\FlockAgentIndex}]},\norm{\zVelocityNeighborAverage_\FlockAgentIndex}}\pn{-\mathrm{sgn}\spn{\zLinearDiscriminantName\pn{\FlockAgentRelativePos[_{\FlockAgentIndexTwo\FlockAgentIndex}], \zVelocityNeighborAverage_\FlockAgentIndex, \frac{\pi}{2}}}\zRotationMatrix\pn{\zRelativeVelocityObstacleDodgeAngle}}\frac{\FlockAgentRelativePos[_{\FlockAgentIndexTwo\FlockAgentIndex}]}{\norm{\FlockAgentRelativePos[_{\FlockAgentIndexTwo\FlockAgentIndex}]}} \quad \text{since matrices of \SpecialOrthogonalGroup{2} commute} \\
        = & \zRotationMatrix\pn{-\FlockAgentRelativeVel\pn{t}}.
    \end{aligned}
    $$
    Therefore, the relative velocity is \SpecialOrthogonalGroup{2} equivariant.
\end{proof}
}

\end{appendix}

% BibTeX users please use one of
%\bibliographystyle{spbasic}      % basic style, author-year citations
\bibliographystyle{spmpsci}      % mathematics and physical sciences
%\bibliographystyle{spphys}       % APS-like style for physics
\bibliography{references}   % name your BibTeX data base

\end{document}
% end of file template.tex

