Our study proposes a hierarchical tool stacking method, named ChemHTS, to optimize the tool invocation process of LLMs in chemistry-related tasks. ChemHTS iteratively refines and optimizes the tool stacking trajectory, ultimately identifying the optimal sequence of tool calls to enhance computational performance (Fig.\ref{fig:010method}).
\subsection{Step 1: Tool Self-stacking Warmup}

%将工具分为两类,计算和检索。
%由于化学任务研究往往是理论计算与实验知识的结合,通过计算预测未知性质或过程,以及通过检索获取已有知识,因此本文将工具集$\mathcal{L}$分为compute-based tools $\mathcal{L}_c$ and retriever-based tools $\mathcal{L}_r$两类别的工具。
%为了通过测试单个工具在多次调用后的表现,确定其是否适合进行堆叠,我们首先进行单个工具的自我叠加预热阶段。
To evaluate the performance of a single tool after multiple invocations and determine its suitability for stacking, we first conduct a self-stacking warm-up phase for the individual tool.
Since research in chemical tasks often integrates theoretical computations with experimental knowledge, leveraging computation to predict unknown properties or processes and retrieval to access existing knowledge, we categorize the chemical LLM toolset $\mathcal{L}$ into two types: compute-based tools $\mathcal{L}_c$ and retriever-based tools $\mathcal{L}_r$.
For computational tools, it may be necessary to test whether repeated invocations lead to improved computational accuracy or stability. In contrast, for retriever-based tools, it may be important to evaluate whether multiple retrievals can enhance knowledge coverage.

For a given chemical task $\mathcal{T}$ and its corresponding tool library $\mathcal{L}$, the performance of an individual tool ($ToolA_i$) and its self-stacking ($A_i \to A_i$) is evaluated. If repeated invocations of the same tool type result in performance degradation, further stacking of that tool is discontinued. From each category, the top-k most effective tool combinations are selected for the next stage of stacking, as outlined in Algorithm~\ref{alg:HTS}.. 
These refined tool-stacking pathways are encapsulated as new tools, reducing the subsequent search space, improving computational efficiency, and providing stronger tool combination strategies for the next phase.
\subsection{Step 2: Multi-Layer Optimization}

\section{Local samplers for spin systems}\label{sec:local}

%In this section, we will construct a new marginal sampling oracle for permissive $q$-spin systems, and use it to construct our local sampler and prove Theorems \ref{theorem:local-sampler} and \ref{theorem:sublinear-sampler}.  Our construction is inspired by the following simple rejection sampling procedure of sampling from $\mu^{\sigma}_{v}$ when given the neighborhood configuration $\sigma\in [q]^{N(v)}$ for some $v\in V$:
In this section, we construct a new marginal sampling oracle for permissive $q$-spin systems. We will use this oracle to build our local sampler and prove Theorems \ref{theorem:local-sampler} and \ref{theorem:sublinear-sampler}. Our construction is inspired by a simple rejection sampling procedure for sampling from $\mu^{\sigma}_v$, given the neighborhood configuration $\sigma \in [q]^{N(v)}$ for some $v \in V$. 
This rejection sampling procedure is as follows:
\begin{itemize}
    \item Repeat the following steps:\label{item:reject-sampling}
    \begin{enumerate}
        \item propose a value $c\in [q]$ distributed according to $\lambda_v$;\label{item:propose}
        \item with probability $\prod\limits_{e=(u,v)\in E}A_e(\sigma(u),c)$, accept the proposal and return $c$ as the final value, otherwise, reject the proposal and go to Step \ref{item:propose}. 
    \end{enumerate}
\end{itemize}


%Note that the well-definedness of the above procedure is by \Cref{cond:main} that all $\lambda_v$ and $A_e$ are normalized. For any $q$-spin system $\+S=(G=(V,E),(\bm{\lambda},\bm{A})))$, given a vertex $v \in V$, we can then define a marginal sampling oracle at $v$ based on the rejection sampling procedure above.

Note that the well-definedness of the above procedure follows from \Cref{cond:main}, which ensures that all $\lambda_v$ and $A_e$ are normalized. Given a $q$-spin system $\+S = (G = (V, E), \bm{\lambda}, \bm{A})$, for any vertex $v \in V$, we can then define a marginal sampling oracle at $v$ based on the rejection sampling procedure.

\begin{algorithm}[H]
\caption{a marginal sampling oracle for $q$-spin systems} \label{Alg:coupler-spin-system}
\SetKwInput{KwData}{Oracle access}
\KwIn{A $q$-spin system $\+S=(G=(V,E),\bm{\lambda},\bm{A})$, a vertex $v\in V$.}
\KwOut{A value $X \in [q]$.}
\KwData{$\+O(u)$ for each $u\in N(v)$.}
Sample an infinite long sequence of i.i.d.~tuples $\{(c_i,(r_{i,u})_{u\in N(v)})\}_{1\leq i<\infty}$ where
%\begin{itemize}
    each $c_i\in [q]$ is distributed as $\lambda_v$ and each $r_{i,u}$ is chosen uniformly from $[0,1]$\;
%\end{itemize}  
\label{line:interpret}
$i^* \gets \min \{i \mid \forall e=(u,v)\in E, r_{i,u}<A_e(\+O(u), c_i)\}$\;
\label{line:coupler-neighbors}
\Return $c_{i^*}$\;
%\label{line:aaa}
\end{algorithm}

%We remark that in \Cref{line:coupler-neighbors} of \Cref{Alg:coupler-spin-system}, such an $i^*$ always exists as $A_{e}$ is normalized to have its maximum entry equal to $1$, according to \Cref{cond:main}. We first prove the following lemma.
We remark that in \Cref{line:coupler-neighbors} of \Cref{Alg:coupler-spin-system}, such an $i^*$ always exists because $A_e$ is normalized to have its maximum entry equal to $1$, as stated in \Cref{cond:main}. We now present the following lemma.

\begin{lemma}\label{lemma:coupler-correctness}
Suppose that the input $q$-spins system $\+S$ satisfies \Cref{cond:main}. Then, \Cref{Alg:coupler-spin-system} implements a marginal sampling oracle at $v$.
%\todo{correctness of the coupler}
\end{lemma}

\begin{proof}
Given a $q$-spin system $\+S=(G=(V,E),\bm{\lambda},\bm{A})$, in \Cref{Alg:coupler-spin-system}, for each $i\geq 1$ and each $u\in N(v)$, recall that each $c_i$ is chosen distributed as $p\in \Delta_q$ where $p(x) \propto \lambda(x)$ and each $r_{i,u}$ is independently chosen uniformly from $[0,1]$. Let $\+{D}_i$ be the event that 
\[
\+D_i: \forall e=(u,v)\in E, r_{i,u}<A_e(\+O(u), c_i),
\]
then for any $x\in [q]$, note that $\+O(u_j) = \sigma(u_j)$ under assumption, we have
%$$\Pr{\forall e=(u,v)\in E, r_i(u)<\bm{A_e}(\sigma_u, c_i) }=\frac{\sum_{c\in [q]}\lambda(c)\prod_{(u,v)\in E}A(\sigma_u,c)}{\sum_{c\in [q]}\lambda(c)}$$ 
\begin{align*}
    \Pr{c_i=x \mid \+{D}_i} 
    &= \frac{\Pr{c_i=x \wedge \+{D}_i}}{\Pr{\+{D}_i}} \\
    &= \frac{\lambda_v(x)\prod\limits_{e=(u,v)\in E}A_e(\sigma(u),x)}{\sum\limits_{c\in [q]}\left(\lambda_v(c)\prod\limits_{e=(u,v)\in E}A_e(\sigma(u),c)\right)} = \mu_v^\sigma(x).
\end{align*}
Let $i^*$ be the smallest index chosen in \Cref{line:coupler-neighbors} of \Cref{Alg:coupler-spin-system}, i.e., 
$i^* = \min \{i \mid \+{D}_i\}$.
$f(v, \sigma )$ follows the distribution of $c_{i^*}$ conditioning on $\+{D}_{i^*}$, concluding the proof of the lemma.
\end{proof}

The marginal sampling oracle in \Cref{Alg:coupler-spin-system} as originally designed would require a significant number of oracle calls in \Cref{line:coupler-neighbors}, potentially violating the efficiency condition outlined in \Cref{condition:fast-termination}. 
%However, the marginal sampling oracle in \Cref{Alg:coupler-spin-system} would require too many oracle calls in \Cref{line:coupler-neighbors}, contradicting the efficiency condition (\Cref{condition:fast-termination}). 
The key optimization is to invoke the oracle $\+O(u)$ only when necessary for each neighbor $u \in N(v)$, rather than for every iteration in \Cref{Alg:coupler-spin-system}.
%The key idea in optimizing the above process is straightforward: invoke $\+O(u)$ only when necessary for each $u\in N(v)$. Recall the definition $\lb=C(\Delta,\delta)\defeq 1-\frac{1-\delta}{2\Delta}$ from \Cref{cond:main}. 
Formally, assuming \Cref{cond:main} holds, in \Cref{line:coupler-neighbors} of \Cref{Alg:coupler-spin-system}, if $r_{i,u} < \lb$, where $\lb = C(\Delta, \delta) \defeq 1 - \frac{1-\delta}{2\Delta}$, then the inequality $r_{i,u}<A_e(\+O(u),c_i)$ will hold true regardless of the value of $\+O(u)$. This is because the term $\lb$ is chosen such that $r_{i,u}$ is sufficiently small to ensure success in the comparison without needing the actual value of $\+O(u)$. Consequently, it becomes unnecessary to call $\+O(u)$ when $r_{i,u} < \lb$.
With this idea of optimization, we propose the following implementation of $\eval^{\+O}(v)$, presented in \Cref{Alg:eval}, which builds upon the above idea to efficiently sample without violating the fast termination condition.
%Formally, assuming \Cref{cond:main} holds, in \Cref{line:coupler-neighbors} of \Cref{Alg:coupler-spin-system}, if $r_{i,u} < \lb$, then the inequality $r_{i,u}<A_e(\+O(u),c_i)$ holds true regardless of the value of $\+O(u)$. Consequently, it is unnecessary to call $\+O(u)$ in this case. We then propose $\eval^{\+O}(v)$ building on the optimization above.


%if we directly call $\resolve(\pred_t(u);M,R)$ for each $u\in N(v)$ to construct the configuration $\sigma \in [q]^{N(v)}$, \emph{\coupler} satisfies \Cref{condition:local-correctness}. 

%if for each $u\in N(v)$, $\sigma_u$ is determined as the color of $u$ at time $t$ by recursively calling $\resolve(\pred_t(u);M,R)$. If we directly implement \Cref{Alg:resolve} by using \Cref{Alg:coupler-spin-system} as the $\sample$ process in \Cref{Line:resolve-sample}, according to \Cref{theorem:lsample-correctness}, the algorithm produce perfect sample if it satisfies \Cref{condition:immediate-termination}. However, the above approach always leads to infinite recursion. 

%Consider optimizing the above process, instead of recursively calling $\resolve(\pred_t(u);M,R)$ for all of $v$'s neighbors, we only call it when 'necessary'. Formally, in \Cref{line:coupler-neighbors} of \Cref{Alg:coupler-spin-system}, if $r_i(u) < \min\{A_e(x,c_i)\mid x\in [q]\}$, then the expression $r_i(u)<A_e(\sigma_u,c_i)$ holds true regardless of the value of $\sigma_u$. In this case, it is not necessary to call $\resolve(\pred_t(u);M,R)$ recursively. Based on the above optimization, we give our main algorithm as follow.

\begin{algorithm}[H]
\caption{$\eval^{\+O}(v)$} \label{Alg:eval}
\SetKwInput{KwData}{Global variables}
\KwIn{A $q$-spin systems $\+S=(G=(V,E),\bm{\lambda},\bm{A})$, a vertex $v\in V$.}
\KwOut{A value $c \in [q]$.}
%\KwData{a randomness map $R: \mathbb{Z} \to [0,1] \cup \{\perp\}$, a color map $M: \mathbb{Z}\to [q]\cup\{\perp\}$;}
%\lIf{$M(t) \neq \perp$}{\Return $M(t)$}
%\lIf{$R(t) = \perp$}{$R(t) \gets \randseed \in [0,1]_{\text{u.a.r}}$}
%$M(t)\gets \sample(t)$\;
 Sample an infinite long sequence of i.i.d.~tuples $\{(c_i,(r_{i,u})_{u\in N(v)})\}_{1\leq i<\infty}$, where
%\begin{itemize}
    each $c_i\in [q]$ is distributed as $\lambda_v$ and each $r_{i,u}$ is chosen uniformly from $[0,1]$\;\label{Line:eval-sample}
\For{$i=1,2,...$\label{Line:reject}}{
${flag} \gets 1$\;
%\For{$j=1$ to $|N(v)|$\label{Line:eval-for}}{
\For{$e=(u,v)\in E$\label{Line:eval-for}}{
\If{$r_{i,u}\geq \lb$\label{Line:eval-cond}}{
\lIf{$r_{i,u}\geq A_{e}(\+O(u),c_i)$\label{Line:eval-if}}{${flag} \gets 0$}
}
}
\lIf{${flag}=1$\label{Line:eval-return}}{\Return $c_i$}
}
\end{algorithm}


\begin{remark}[principle of deferred decision]\label{remark:lazy-samples}
    In \Cref{Line:eval-sample} of \Cref{Alg:eval}, we are required to sample an infinitely long sequence $\{(c_i,\{r_{i,u}\}_{u\in N(v)})\}_{1\leq i<\infty}$. 
    Obviously, it is not feasible to directly sample an infinite number of random variables for implementation.  
    Instead, we adopt the principle of deferred decision: each $c_i$ and $r_{i,u}$ is generated only when they are accessed in  \Cref{Line:eval-for,Line:eval-if} of \Cref{Alg:eval}. 
\end{remark}

Next, we show that the marginal sampling oracle $\eval^{\+O}(v)$ in \Cref{Alg:eval} satisfies the conditions for both correctness and efficiency. 

\begin{lemma}\label{lemma:eval-correctness}
Suppose that the input $q$-spins system $\+S$ satisfies \Cref{cond:main}. Then, the marginal sampling oracle $\eval^{\+O}(v)$ implemented as in \Cref{Alg:eval} satisfies both \Cref{condition:local-correctness} and \Cref{condition:immediate-termination}.
\end{lemma}

\begin{proof}

Condition \ref{condition:local-correctness} can be verified directly by \Cref{lemma:coupler-correctness} and comparing Algorithms \ref{Alg:coupler-spin-system} and \ref{Alg:eval}.

For Condition \ref{condition:immediate-termination}, recall that $\+E_v$ is the event that $\eval^{\+O}(v)$ in \Cref{Alg:eval} terminates without any calls to $\+O$. Note that $\+E_v$ occurs if and only if \Cref{Alg:eval} terminates within $1$ round of the loop at \Cref{Line:reject} and $r_{1,u}<\lb$ holds for each $u\in N(v)$. Since $c_1$ and each $r_{1,u}$ are independent, assuming the $q$-spin system $\+S$ satisfies \Cref{cond:main} with constant $\delta>0$, we immediately have
\begin{align*}
\Pr{\+E_t} \geq C^{\Delta}>0. 
\end{align*}

It verifies that $\eval^{\+O}(v)$ satisfies Condition \ref{condition:immediate-termination} assuming Condition \ref{cond:main} holds.
\end{proof}

\begin{lemma}\label{lemma:eval-efficiency}
Suppose that the input $q$-spins system $\+S$ satisfies \Cref{cond:main}. Then, the marginal sampling oracle $\eval^{\+O}(v)$ implemented as in \Cref{Alg:eval} satisfies \Cref{condition:fast-termination}.
\end{lemma}

\begin{proof}
Fix some $v\in V$. Also fix some $\sigma\in [q]^{N(v)}$ and assume that $\+O(u)$ returns $\sigma(u)$ within $\eval^{\+O}(v)$. In one round of the outer for loop at \Cref{Line:reject}, it is direct to see the expected number of calls to $\+O(u)$ for each $u\in N(v)$ is
\begin{align}
 \sum\limits_{c\in[q]}\left(\lambda_v(c) \sum_{e=(u,v)\in E} (1-\lb)\right).\label{eq:expect-one-round}
\end{align}


 Let $\+I^{\sigma}_v$ be the number of executions of the outer for loop at \Cref{Line:reject} in $\eval^{\+O}(v)$. Note that $\+I^{\sigma}_v$ corresponds exactly to the number of executions of \Cref{item:propose} in the rejection sampling. Therefore,
\begin{align}
\E{\+{I}^{\sigma}_v} = \frac{1}{\sum\limits_{c\in[q]}\left(\lambda_v(c)\prod\limits_{e=(u,v)\in E}A_e(\sigma(u),c)\right)}.
\label{eq:expect-number-of-round}
\end{align}

Combining \eqref{eq:expect-one-round}, \eqref{eq:expect-number-of-round} and the linearity of expectation, assuming the $q$-spins system $\+S$ satisfies \Cref{cond:main} with constant $\delta>0$,
\begin{align*}
\E{\+{T}^{\sigma}_v} =& \E{\+{I}^{\sigma}_v}\cdot \sum\limits_{c\in[q]}\left(\lambda_v(c) \sum\limits_{e=(u,v)\in E} (1-\lb)\right) \\
 \text{(by \Cref{cond:main})}\quad \leq &\frac{\sum\limits_{c\in [q]}\left(\lambda_v(c)\cdot (1-\lb)\cdot |N(v)| \right)}{\sum\limits_{c\in[q]}\left(\lambda_v(c)\cdot C^{\Delta}\right)} \\
 \text{(by $\lb=1-\frac{\delta}{2\Delta}$)}\quad        \leq &\frac{\sum\limits_{c\in [q]}(\lambda_v(c)(1-\delta)/2)}{\sum\limits_{c\in [q]}(\lambda_v(c)(1+\delta)/2)} \\
         < &1.
\end{align*}

It proves that $\eval^{\+O}(v)$ satisfies Condition \ref{condition:fast-termination} assuming Condition \ref{cond:main} holds.
\end{proof}

We are now ready to prove \Cref{theorem:local-sampler}.

\begin{proof}[Proof of \Cref{theorem:local-sampler}]
    We use \Cref{Alg:lsample} as our local sampler, where the subroutine $\eval^{\+O}(v)$ is implemented as \Cref{Alg:eval} with \Cref{remark:lazy-samples}. Here, the guarantee of correctness is by \Cref{lemma:lsample-correctness} and \Cref{lemma:eval-correctness}.
    For efficiency, by Lemmas \ref{lemma:lsample-efficiency} and \ref{lemma:eval-efficiency}, we have that the expected number of $\resolve$  calls is $O(|\Lambda|)$. Also, note that each outer loop either terminates directly or results in at least one call to $\resolve$. Hence, the overall running time is bounded by $\Delta \log q$ times the total number of $\resolve$  calls, which is $O(|\Lambda|\Delta\log q)$ in expectation.
\end{proof}


 \subsection{Faster local sampler with improved marginal sampling oracle}
We now present an optimized implementation of the marginal sampling oracle $\eval^{\+O}(v)$ by leveraging the additional assumption on preprocessing in \Cref{assumption:additional-access-model}.
This results in a more efficient local sampler, proving \Cref{theorem:sublinear-sampler}.

In \Cref{Alg:eval},  Lines \ref{Line:eval-for}-\ref{Line:eval-if}  identify all neighbors $u\in N(v)$ for which $r_{i,u}\geq \lb$, and then check whether $r_{i,u}$ additionally satisfies $r_{i,u}\geq A_{(u,v)}(\+O(u),c_i)$. The running time in \Cref{theorem:local-sampler} arises because, while the number of recursive calls to $\resolve$ is bounded by \Cref{lemma:lsample-correctness}, such $u$s are typically sparse among the $|N(v)|$ neighbors. As a result, an extra $O(\Delta)$ factor is incurred in these cases. 
To mitigate this overhead, we propose using binary search to efficiently locate all such $u$s. The improved implementation is detailed in \Cref{Alg:eval-efficient}.


    
      

\begin{algorithm}
\caption{$\eval^{\+O}(v)$ (more efficient implementation)} \label{Alg:eval-efficient}
\SetKwInput{KwData}{Global variables}
\KwIn{A $q$-spin systems $\+S=(G=(V,E),\bm{\lambda},\bm{A})$, a variable $v\in V$.}
\KwOut{A value $c \in [q]$.}
%\KwData{a randomness map $R: \mathbb{Z} \to [0,1] \cup \{\perp\}$, a color map $M: \mathbb{Z}\to [q]\cup\{\perp\}$;}
%\lIf{$M(t) \neq \perp$}{\Return $M(t)$}
%\lIf{$R(t) = \perp$}{$R(t) \gets \randseed \in [0,1]_{\text{u.a.r}}$}
%$M(t)\gets \sample(t)$\;
Assume an arbitrary ordering $N(v)=\{u_1,u_2,\dots,u_{|N(v)|}\}$ as in \Cref{assumption:additional-access-model}\label{Line:eval-efficient-assumption}\;
\For{$i=1,2,...$}{
Draw $c\sim \lambda_v$\label{Line:eval-efficient-draw-c}\;
$flag \gets 1,counter\gets 1$\;\label{Line:eval-efficient-initialize}
\While{$counter\leq |N(v)|$\label{Line:eval-efficient-while}}{
$l\gets counter,r\gets |N(v)|$\label{Line:eval-efficient-initialize-bs}\;
Draw $\text{ind}_{{l,r}}\sim \textrm{Bern}(1-C^{r-l+1})$\label{Line:eval-efficient-sample-initial}\;
\lIf{$\text{ind}_{{l,r}}=0$\label{Line:eval-efficient-break-cond}}{ \textbf{break}}
\While{$l<r$\label{Line:eval-efficient-while-2}}{
    $mid\gets \left\lfloor\frac{l+r}{2}\right\rfloor$\label{Line:eval-efficient-set-mid}\;
    Draw $\text{ind}_{{l,mid}}\sim \textrm{Bern}\left(\frac{1-\lb^{mid-l+1}} {1-\lb^{r-l+1}}\right)$\label{Line:eval-efficient-sample-mid}\;
    \eIf{$\text{ind}_{{l,mid}}=1$}{
        $r\gets mid$;
    }{$l\gets mid+1$\;\label{Line:eval-efficient-l-assign}}
}
Draw $r\in [\lb,1]$ u.a.r. \label{Line:eval-efficient-draw-r}\;
\lIf{$r\geq A_{(v,u_l)}(\+O(u_l),c)$\label{Line:eval-efficient-update-flag}}{$flag \gets 0$}
$counter\gets l+1$\label{Line:eval-efficient-update-counter}\;
}
\lIf{$flag=1$\label{Line:eval-efficient-return}}{\Return $c_i$}
}
\end{algorithm}

We briefly explain \Cref{Alg:eval-efficient}. First, we order all neighbors of $v$ as assumed in \Cref{assumption:access-model}. In each outer loop iteration, we sample  $c_i$ according to the correct distribution (\Cref{Line:eval-efficient-draw-c}) and initialize a counter at $1$ (\Cref{Line:eval-efficient-initialize}). We then use binary search to find the first position $j\in [counter,N(v)]$ where the condition in \Cref{Line:eval-cond} of \Cref{Alg:eval} holds for $u_j$, or determine no such $j$ exists (Lines \ref{Line:eval-efficient-while-2}-\ref{Line:eval-efficient-l-assign}).  If a valid $j$ is found, we execute \Cref{Line:eval-if} of \Cref{Alg:eval} and update $counter\gets j+1$ (Lines \ref{Line:eval-efficient-draw-r}-\ref{Line:eval-efficient-update-counter}). Otherwise, we proceed directly to \Cref{Line:eval-return} of \Cref{Alg:eval} (\Cref{Line:eval-efficient-return}).

\begin{comment}
To actually perform the binary search,  
\begin{itemize}
    \item we initialize the interval $[l,r]=[counter,N(v)]$ (\Cref{Line:eval-efficient-initialize-bs}) and sample $\textrm{ind}_{l,r}\sim \textrm{Bern}1-\lb^{r-l+1}$ as the indicator variable of whether there exists some $j\in [l,r]$ such that $r_{i,u_j}\geq \lb$ (\Cref{Line:eval-efficient-sample-initial}).  Note that this $1-\lb^{r-l+1}$ is exactly the probability at least one such $j$ exists in $[l,r]$, assuming  $\{(c_i,(r_{i,u})_{u\in N(v)})\}_{1\leq i<\infty}$ is an infinite long i.i.d sequence where
%\begin{itemize}
    each $c_i\in [q]$ is distributed as $\lambda_v$ and each $r_{i,u}$ is chosen uniformly from $[0,1]$. 
    \item At each step of the binary search, we maintain the invariant that if our current interval is $[l,r]$, then there exists at least one $j\in [l,r]$  such that $r_{i,u_j}\geq \lb$. We then choose $mid=\left\lfloor\frac{l+r}{2}\right\rfloor$ (\Cref{Line:eval-efficient-set-mid}) and try to narrow down the search interval to either $[l,mid]$ or $[mid+1]$. To do this, we check if at least one such $j\in [l,mid]$ exists through sampling an indicator variable $\textrm{ind}_{l,mid}\sim\textrm{Bern}\left(\frac{\lb^{mid-l+1}}{\lb^{r-l+1}}\right)$ (\Cref{Line:eval-efficient-sample-mid}). Note that this indicator variable is drawn according to the probability that there exists at least one $j\in [l,mid]$  such that $r_{i,u_j}\geq \lb$ conditioning on such  there exists at least one such $j\in [l,r]$, assuming  $\{(c_i,(r_{i,u})_{u\in N(v)})\}_{1\leq i<\infty}$ is an infinite long i.i.d sequence where
%\begin{itemize}
    each $c_i\in [q]$ is distributed as $\lambda_v$ and each $r_{u,j}$ is chosen uniformly from $[0,1]$. 
\end{itemize}
\end{comment}

We now present the following technical lemma, which establishes a perfect coupling between randomness of \Cref{Alg:eval-efficient} and \Cref{Alg:eval}. Consequently, by Lemmas \ref{lemma:eval-correctness} and \ref{lemma:eval-efficiency}, \Cref{Alg:eval-efficient} satisfies Conditions \ref{condition:local-correctness}, \ref{condition:immediate-termination}, and \ref{condition:fast-termination}, assuming \Cref{cond:main} holds. %The proof of \Cref{lemma:eval-efficient-coupling} is deferred to \Cref{sec:appendix-coupling}.

\begin{lemma}\label{lemma:eval-efficient-coupling}
For each $v\in V$, assuming that $\+O(u)$ returns a consistent value for each $u\in N(v)$ in $\eval^{\+O}(v)$ 
in both \Cref{Alg:eval} and \Cref{Alg:eval-efficient}, 
the randomness in these algorithms can be coupled such that:
\begin{enumerate}
    \item \Cref{Alg:eval} and \Cref{Alg:eval-efficient} always return the same value.
    \item For each $u\in N(v)$, the number of calls to $\+O(u)$ within \Cref{Alg:eval} and \Cref{Alg:eval-efficient} is identical.
\end{enumerate}
\end{lemma}


%\section{More efficient implementation of the local sampler}\label{sec:appendix-coupling}
%In this section, we will finish the technical details of the more efficient implementation of the local sampler (\Cref{Alg:eval-efficient}). Specifically, we will prove \Cref{lemma:eval-efficient-coupling}.

\begin{proof}%[Proof of \Cref{lemma:eval-efficient-coupling}]
We couple the randomness in \Cref{Alg:eval-efficient} with the infinite long sequence of tuples 
$$\{(c_i,(r_{i,u})_{u\in N(v)})\}_{1\leq i<\infty}$$ 
in \Cref{Alg:eval} by the following process. 

\begin{definition}[coupling of randomness between \Cref{Alg:eval} and \Cref{Alg:eval-efficient}]\label{definition:process-couple}
Suppose we run \Cref{Alg:eval} and order $N(v)=\{u_1,u_2,\dots,u_{|N(v)|}\}$ as in \Cref{Line:eval-efficient-assumption} of \Cref{Alg:eval-efficient}. Then, for each $i\geq 1$, during the $i$-th iteration of the outer loop in \Cref{Alg:eval-efficient},
the randomness is coupled as follows:
\begin{enumerate}
        \item  Each time \Cref{Line:eval-efficient-draw-c} of \Cref{Alg:eval-efficient} is executed, set the drawn value of $c$  to be the same as $c_i$ drawn in \Cref{Alg:eval}. \label{item:eval-efficient-coupling-1}
        \item Each time \Cref{Line:eval-efficient-sample-initial} of \Cref{Alg:eval-efficient} is executed, set $\textrm{ind}_{l,r}=1$  if there exists at least one $x\in [counter,|N(v)|]$ such that $r_{i,u_x}\geq \lb$, and set $\textrm{ind}_{l,r}=0$ otherwise. Here, $counter$  refers to the value of the corresponding variable in \Cref{Alg:eval-efficient} at that time, and $r_{i,u_x}$ is the value drawn in \Cref{Alg:eval}.  \label{item:eval-efficient-coupling-2}
        \item Each time \Cref{Line:eval-efficient-sample-mid} of \Cref{Alg:eval-efficient} is executed, set $\textrm{ind}_{l,mid}=1$ if there exists at least one $x\in [l,mid]$ such that $r_{i,u_x}\geq \lb$, and set $\textrm{ind}_{l,mid}=0$ otherwise. Here, $counter$ refers to the value of the corresponding variable in \Cref{Alg:eval-efficient} at that time, and $r_{i,u_x}$ is the value drawn in \Cref{Alg:eval}.  \label{item:eval-efficient-coupling-3}
        \item  Each time \Cref{Line:eval-efficient-draw-r} of \Cref{Alg:eval-efficient} is executed, set the drawn value of $r$ to be $r_{i,u_l}$. Here, $l$ is the current value of the corresponding variable in \Cref{Alg:eval-efficient}, and $r_{i,u_l}$ is the corresponding value drawn in \Cref{Alg:eval}. \label{item:eval-efficient-coupling-4}
\end{enumerate}
\end{definition}

     We first show the lemma assuming that \Cref{definition:process-couple} indeed generates a coupling between the randomness of \Cref{Alg:eval} and \Cref{Alg:eval-efficient}.
     
     Note that for each $i\geq 1$ such that the $i$-th outer for loop is actually executed in \Cref{Alg:eval},
     \begin{itemize}
         \item in \Cref{Alg:eval}, we make recursive calls to $\resolve(\pred_t(u))$ for those $u$ satisfying $r_{i,u}\geq \lb$;
         \item in \Cref{Alg:eval-efficient}, we make recursive calls to $\resolve(\pred_t(u_{\ell}))$ for those $u_l$ after \Cref{Line:eval-efficient-draw-r} is executed;
     \end{itemize}
      Note that by Items \ref{item:eval-efficient-coupling-2} and \ref{item:eval-efficient-coupling-3} of \Cref{definition:process-couple} and Lines \ref{Line:eval-efficient-initialize}-\ref{Line:eval-efficient-l-assign} of \Cref{Alg:eval-efficient}, all $l$s in \Cref{Line:eval-efficient-l-assign} are the positions in \Cref{Alg:eval} such that $r_{i,u_l}\geq \lb$.  So, for these iterations, the recursive calls to $\resolve(\pred_t(u))$ in \Cref{Alg:eval} and \Cref{Alg:eval-efficient} are the same.

      When \Cref{Alg:eval} stops at some iteration $i$, it should happen that the condition at \Cref{Line:eval-if} of \Cref{Alg:eval} is never satisfied, that is, $r_{i,u}<A_e(x,c_i)$ for each $u\in N(v)$, hence \Cref{Alg:eval-efficient} should also stop at the same iteration $i$ by \Cref{item:eval-efficient-coupling-4} of \Cref{definition:process-couple} and \Cref{Line:eval-efficient-update-flag} of \Cref{Alg:eval-efficient}. Also, the returned values at this iteration are the same by \Cref{item:eval-efficient-coupling-1} of \Cref{definition:process-couple} and \Cref{Line:eval-efficient-return} of \Cref{Alg:eval-efficient}. This already proves the lemma.

      It remains to prove that \Cref{definition:process-couple} is indeed a coupling. It is sufficient to show that in the above process, conditioning on the previous randomness in \Cref{Alg:eval-efficient},
      \begin{enumerate}[label=(\alph*)]
          \item each $c$ drawn in \Cref{Line:eval-efficient-draw-c} is distributed as $\lambda_v$; \label{item:coupling-1}
          \item each $\textrm{ind}_{l,r}$ drawn in \Cref{Line:eval-efficient-sample-initial} is  distributed as $\textrm{Bern}\left(\frac{\lb^{mid-l+1}} {\lb^{r-l+1}}\right)$;  \label{item:coupling-2}
          \item each $\textrm{ind}_{l,mid}$ drawn in \Cref{Line:eval-efficient-sample-mid} is  distributed as $\textrm{Bern}\left(\frac{\lb^{mid-l+1}} {\lb^{r-l+1}}\right)$; \label{item:coupling-3}
          \item each $r$ drawn in \Cref{Line:eval-efficient-draw-r} is distributed uniformly in $[b_{v,u_l,c},1]$.\label{item:coupling-4}
      \end{enumerate} 
      Here, \Cref{item:coupling-1} is by \Cref{item:eval-efficient-coupling-1} of \Cref{definition:process-couple}. \Cref{item:coupling-2} is by \Cref{item:eval-efficient-coupling-2} of \Cref{definition:process-couple} that all the randomness $r_{i,u_x}$ for $x\in [counter,|N(v)|]$ is independent of the previous samples in \Cref{Alg:eval-efficient}. \Cref{item:coupling-3} is by Items \ref{item:eval-efficient-coupling-2} and \ref{item:eval-efficient-coupling-3} of \Cref{definition:process-couple} and each time when we draw $\textrm{ind}_{l,mid}$, the event 
      \[
      \exists x\in [l,r],\quad \text{s.t. } r_{i,u}\geq \lb,
      \]
      holds conditioning on the previous samples. Finally, \Cref{item:coupling-4} is by \Cref{item:eval-efficient-coupling-4} and the event $r_{i,u_l}\geq \lb$ holds conditioning on the previous samples. This finishes the proof of the lemma.
\end{proof}


We are now ready to prove \Cref{theorem:sublinear-sampler}.

\begin{proof}[Proof of \Cref{theorem:sublinear-sampler}]
    We use \Cref{Alg:lsample} as our local sampler, where the subroutine $\eval^{\+O}(v)$ is implemented as \Cref{Alg:eval-efficient}. Here, the guarantee of correctness is by \Cref{lemma:lsample-correctness}, \Cref{lemma:eval-correctness}, and \Cref{lemma:eval-efficient-coupling}.
    For efficiency, by Lemmas \ref{lemma:lsample-efficiency}, \ref{lemma:eval-efficiency},  and \ref{lemma:eval-efficient-coupling} we have the expected number of total $\resolve$ calls is $O(|\Lambda|)$. Note that now each outer loop of \Cref{Alg:eval-efficient} only takes $O(\log \Delta\log q)$ time, as we only need $O(\log q)$ time to sample a $q$-valued variable in \Cref{Line:eval-efficient-draw-c} of \Cref{Alg:eval-efficient} and the inner loop in Lines \ref{Line:eval-efficient-while}-\ref{Line:eval-efficient-update-counter} of \Cref{Alg:eval-efficient} runs in $O(\log \Delta)$ time due to the binary search. Note that here each binary search done results in exactly one call of $\resolve$, and the overall running time is bounded by $O(\log \Delta\log q)$ times the total number of $\resolve$  calls, which is $O(|\Lambda|\log \Delta\log q)$ in expectation.
\end{proof}


%由于遍历 Top-k 最优工具组合,并结合所有工具进行进一步堆叠。计算不同堆叠方式的任务性能。若堆叠导致性能下降或达到设定的最大堆叠深度 n,则停止堆叠。选择最终性能最优的工具堆叠路径,作为该任务的最优执行轨迹(T*,Optimal Stacking Pathway)。

%By iterating through the Top-k most optimal tool combinations and further stacking them with all available tools, the task performance of different stacking configurations is evaluated. If stacking results in performance degradation or the predefined maximum stacking depth n is reached, the stacking process is terminated. The tool stacking pathway with the highest overall performance is selected as the task's optimal execution trajectory.

To explore the optimal invocation pathways across tool combinations and progressively encapsulate the best tools, ultimately determining the globally optimal tool invocation strategy, we divide the tool-stacking process into multiple layers and perform stacking optimization step by step.
In the first layer, the Top-1 computational tool selected in Stage 1 is combined with the Top-k retrieval tools, and the Top-1 retrieval tool is combined with the Top-k computational tools. Each agent is restricted to selecting the best tool from a pool of $m$ candidate tools in each step. If the optimal performance in this layer is worse than that of Stage 1, the stacking process is terminated; otherwise, the iteration continues.
In subsequent layers, all tools are re-ranked, and the top $k$ optimal pathways in each category are selected and encapsulated as new tools. This process continues until either the maximum number of layers $n$ is reached or no further optimization space remains.
Finally, the optimal tool invocation pathway is selected as the general tool invocation strategy for the given task.



