\subsection{Single Tool Invocation}
An LLM that depends on invoking a single tool to complete a task operates with fixed inputs and outputs, limiting its ability to dynamically adapt its workflow. This is exemplified in Algorithm.~\ref{tool stacking} with maxIter set to 1.


\subsection{Tool Stacking}
%工具堆叠(Tool Stacking)可以理解为在任务执行过程中,依次调用多个工具,并将前一个工具的输出作为下一个工具的输入,以逐步优化任务的执行效果。这种方法类似于多步推理(Multi-step Reasoning),在 LLM 任务执行过程中起到了类似流水线(Pipeline)的作用。
Tool stacking refers to the ability of LLMs to invoke multiple tools iteratively and integrate different tools at various levels, thereby optimizing task execution. This can be understood as sequentially invoking multiple tools during task processing as shown in Algorithm.~\ref{tool stacking}, where the output of one tool serves as the input for the next. This iterative process incrementally enhances task performance. 
\begin{algorithm}
\caption{Tool Stacking Execution}
\begin{algorithmic}[1]
\REQUIRE Input $I$, Tool Set $\{T_1, T_2, \dots, T_n\}$, Maximum Iterations $maxIter$
\ENSURE Final Output $O_{final}$
\STATE $O \gets I$
\STATE $iter \gets 0$
\WHILE{$iter < maxIter$}
    \STATE Model selects $T_r$ from $\{T_1, T_2, \dots, T_n\}$
    \STATE $O \gets T_r(O)$
    \STATE $iter \gets iter + 1$
\ENDWHILE
\STATE $O_{final} \gets O$
\RETURN $O_{final}$
\end{algorithmic}
\label{tool stacking}
\end{algorithm}


