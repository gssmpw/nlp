\setlength\tabcolsep{1pt}
%
\begin{table*}[!htb]
    \centering
    \tiny
    \definecolor{lightgray}{gray}{0.9} % 定义浅灰色
    \resizebox{1\textwidth}{!}{
    \begin{tabular}[l]{|>{\raggedright\arraybackslash}m{1.5cm}|>{\raggedright\arraybackslash}m{4cm}|>{\raggedright\arraybackslash}m{10cm}|}
    \hline
    \textbf{Stacking Pattern} & \textbf{Input} & \textbf{Output} \\ 
    \hline
    \textbf{Modify} 
    & The molecule is a peptide anion that is the conjugate base of gamma-Glu-Abu, obtained by removal of protons from the two carboxy groups as well as protonation of the amino group; major species at pH 7.3. It is a conjugate base of a gamma-Glu-Abu. Please try to infer the SMILES of this molecule.
    \newline
    \newline
    \newline
    \textcolor{red}{Gold Answer: CCC(NC(=O)CCC([NH3+])C(=O)[O-])C(=O)[O-]
}
    & To find the SMILES representation of the peptide anion that is the conjugate base of gamma-Glu-Abu, I first need to understand the structure of gamma-Glu-Abu. Gamma-Glu-Abu stands for gamma-glutamyl-aminobutyric acid. I will start by finding the SMILES of gamma-Glu-Abu.
    \newline
    \textbf{Action:} Name2SMILES  
    \newline
    \textbf{Input:} gamma-Glu-Abu
    \newline
    \textbf{Output:} CCC(NC(=O)CCC(N)C(=O)O)C(=O)O  \textcolor{red}{\ding{55}}
    \newline
    \textbf{Observation:} The next step is to modify this SMILES to represent the conjugate base of gamma-Glu-Abu at pH 7.3, where the two carboxyl groups are deprotonated and the amino group is protonated. This involves changing the carboxyl groups from "-C(=O)O" to "-C(=O)[O-]" and the amino group from "-N" to "-[NH3+]".
    \newline
    \textbf{Action:} ChemDFM  
    \newline
    \textbf{Input:} Modify the "CCC(NC(=O)CCC([NH3+])C(=O)[O-])C(=O)[O-]
"  from "-C(=O)O" to "-C(=O)[O-]" and the amino group from "-N" to "-[NH3+].
    \newline
    \textbf{Output:} CCC(NC(=O)CCC([NH3+])C(=O)[O-])C(=O)[O-] 
    \newline
    \textbf{Thought:} I now know the final answer
    \newline
    \textbf{Final Answer: CCC(NC(=O)CCC([NH3+])C(=O)[O-])C(=O)[O-]  \textcolor{red}{\ding{52}}}\\
    \hline
    \textbf{Judge}&The molecule is a cycloalkane composed of three carbon atoms to form a ring. It has a role as an inhalation anaesthetic. It is a cycloalkane and a member of cyclopropanes.Please try to infer the SMILES of this molecule.
    \newline
    \newline
    \newline
    \textcolor{red}{Gold Answer: CC(=O)N[C@@H]1[C@@H](O[C@@H]2O[C
    @@H](C)[C@@H](O)[C@@H](O)[C@@H]2
    O)[C@H](O[C@@H]2O[C@H](CO)[C@H](O
    )[C@H](O)[C@H]2O[C@@H]2O[C@@H](C)
    [C@@H](O)[C@@H](O)[C@@H]2O)[C@@
    H](CO)O[C@H]1O
}&The molecule described is a complex carbohydrate structure involving alpha-L-Fucp, beta-D-Galp, and D-GlcNAc units. The specific linkages and configurations are crucial for determining the correct structure. Given the complexity of the molecule, the best approach is to use the Molecule\_Design tools, which are designed to handle detailed questions about molecular structures. I will start with Molecule\_Design\_1 to see if it can provide a correct SMILES representation based on the detailed description provided.
\newline
\textbf{Action:} Molecule\_Design\_1
\newline
\textbf{Action Input:} The molecule is an alpha-L-Fucp-(1->2)-beta-D-Galp-(1->3)-[alpha-L-Fucp-(1->4)]-D-GlcNAc where the glucosamine at the reducing end has beta-configuration at its anomeric centre. It has a role as an epitope. Please provide the SMILES of this molecule.
\newline
\textbf{Observation:}CC(=O)N[C@@H]1[C@@H](O[C@@H]2O[C@@H](C)[C@@H](O)[C@@H](O)[C@@H]2O)[C@H]
(O[C@@H]2O[C@H](CO)[C@H](O)[C@H](O)[C@H]2O[C@@H]2O[C@@H](C)[C@@H](O)[C@@H](O)[C@@
H]2O)[C@@H](CO)O[C@H]1O
\newline
\textbf{Thought:}The SMILES provided by Molecule\_Design\_1 seems to represent a complex carbohydrate structure, but to ensure accuracy, I will use Molecule\_Design\_2 for a second opinion. This will help verify the correctness of the SMILES or provide an alternative if there are discrepancies.
\newline
\textbf{Action:} Molecule\_Design\_2
\newline
\textbf{Action Input:} The molecule is an alpha-L-Fucp-(1->2)-beta-D-Galp-(1->3)-[alpha-L-Fucp-(1->4)]-D-GlcNAc where the glucosamine at the reducing end has beta-configuration at its anomeric centre. It has a role as an epitope. Please provide the SMILES of this molecule.
\newline
\textbf{Observation:}Nc1ccn([C@@H]2O[C@H](COP(=O)([O-])[O-])[C@@H](O)[C@H]2O)c(=O)n1
\newline
\textbf{Thought:}The SMILES provided by Molecule\_Design\_2 is significantly different from the one provided by Molecule\_Design\_1, indicating a discrepancy. The structure from Molecule\_Design\_2 seems less likely to represent the complex carbohydrate structure described in the question. The SMILES from Molecule\_Design\_1 appears more plausible as it includes multiple sugar units linked together, which aligns better with the description of the molecule involving alpha-L-Fucp, beta-D-Galp, and D-GlcNAc. Therefore, I will consider the SMILES from Molecule\_Design\_1 as the more likely correct representation.
\newline
\textbf{Thought:} I now know the final answer
\newline
\textbf{Final Answer: CC(=O)N[C@@H]1[C@@H](O[C@@H]2O[C@@H](C)[C@@H](O)[C@@H](O)[C@@H]2O)
[C@H](O[C@@H]2O[C@H](CO)[C@H](O)[C@H](O)[C@H]2O[C@@H]2O[C@@H](C)[C@@H](O)[C@@H]
(O)[C@@H]2O)[C@@H](CO)O[C@H]1O\textcolor{red}{\ding{52}}}\\ \hline
\textbf{Reserve}&The molecule is a linear amino tetrasaccharide composed of N-acetyl-beta-D-galactosamine, alpha-D-galactose, beta-D-galactose and N-acetyl-beta-D-galactosamine units joined by sequential (1->3)-, (1->4)- and (1->4)-linkages. It has a role as an epitope.Please try to give SMILES of this molecule.
    \newline
    \newline
    \newline
    \textcolor{red}{Gold Answer: CC(=O)N[C@@H]1[C@@H](O)[C@H](O[C@
    @H]2O[C@H](CO)[C@H](O[C@H]3O[C@H]
    (CO)[C@H](O)[C@H](O[C@@H]4O[C@H](C
    O)[C@H](O)[C@H](O)[C@H]4NC(C)=O)[C@
    H]3O)[C@H](O)[C@H]2O)[C@@H](CO)O[C
    @H]1O
}&The SMILES representation for the molecule described is not readily available due to the complexity of the glycosidic linkages and the specific stereochemistry involved. Further detailed chemical synthesis knowledge or specialized software would be required to accurately construct the full SMILES notation of this tetrasaccharide.
\newline
\textbf{Thought:}The initial attempt using the Molecule\_Design\_2 tool did not yield a direct SMILES result, likely due to the complexity of the molecule described. As an alternative approach, I can attempt to construct the molecule step-by-step using the Name2SMILES tool for each individual sugar unit and then logically deduce the possible structure of the entire molecule. However, constructing the full SMILES for the entire molecule including specific linkages manually might not be feasible without specialized chemical software. I will first retrieve the SMILES for each sugar unit.
\newline
\textbf{Action:} Name2SMILES
\newline
\textbf{Action Input:} N-acetyl-beta-D-galactosamine
\newline
\textbf{Observation:} CC(=O)N[C@@H]1[C@@H](O)[C@@H](O)[C@@H](CO)O[C@H]1O
\newline
\textbf{Thought:}I have obtained the SMILES for N-acetyl-beta-D-galactosamine. Next, I will retrieve the SMILES for alpha-D-galactose.
\newline
\textbf{Action:} Name2SMILES
\newline
\textbf{Action Input:} alpha-D-galactose
\newline
\textbf{Observation:}OC[C@H]1O[C@H](O)[C@H](O)[C@@H](O)[C@H]1O
\newline
\textbf{Thought:}I have obtained the SMILES for alpha-D-galactose. Next, I will retrieve the SMILES for beta-D-galactose.
\newline
\textbf{Action:} ChemDFM
\newline
\textbf{Action Input:} What is the SMILES of beta-D-galactose
\newline
\textbf{Observation:} OC[C@H]1O[C@@H](O)[C@H](O)[C@@H](O)[C@H]1O
\newline
\textbf{Thought:}I have obtained the SMILES for beta-D-galactose. Since the molecule starts and ends with N-acetyl-beta-D-galactosamine, and I already have the SMILES for this unit, I do not need to retrieve it again. However, constructing the full SMILES string for the tetrasaccharide with specific glycosidic linkages (1->3, 1->4, 1->4) between these units is beyond the capability of the Name2SMILES tool and requires specialized chemical synthesis software or expertise in carbohydrate chemistry. Therefore, I will conclude my process.
\newline
\textbf{Thought: }I now know the final answer
\newline
\textbf{Final Answer: Unable to provide the exact SMILES string for the described tetrasaccharide due to the complexity of the glycosidic linkages and the limitations of the tools available.}\\ \hline
    \end{tabular}
    }
    \caption{Case analysis of some behavior patterns of agents during tools stacking.}
    \label{tab:modify}
\end{table*}


%该分子由三个氮原子组成的环状结构,它具有芳香性的特性和两个亚甲基。它是一种环烯烃,也是两种异构体之一。请预测它的 SMILES。
%