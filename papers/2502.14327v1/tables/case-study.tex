\setlength\tabcolsep{1pt}
%
\begin{table*}[!htb]
    \centering
    \tiny
    \definecolor{lightgray}{gray}{0.9} % 定义浅灰色
    \resizebox{0.95\textwidth}{!}{
    \begin{tabular}[l]{|>{\raggedright\arraybackslash}m{1.5cm}|>{\raggedright\arraybackslash}m{4cm}|>{\raggedright\arraybackslash}m{10cm}|}
    \hline
    \textbf{Stacking Pattern} & \textbf{Input} & \textbf{Output} \\ 
    \hline
    Correct 
    & The molecule is a cycloalkane composed of a three-carbon ring. It functions as an inhalation anesthetic. It is a cycloalkane and a member of the cyclopropane class. What is its predicted SMILES?
    \newline
    \newline
    \newline
    \textcolor{red}{Gold Answer: C1CC1}
    & According to the provided description, the molecule is cyclopropane, which is a cycloalkane formed by three carbon atoms in a ring, well known for its use as an inhalation anesthetic. To confirm the SMILES notation of cyclopropane, I used the ChemDFM tool.
    \newline
    \textbf{Action:} ChemDFM  
    \newline
    \textbf{Input:} What is the SMILES notation for cyclopropane (a three-carbon cycloalkane used as an inhalation anesthetic)?  
    \newline
    \textbf{Output:} CC1(C)CC1  \textcolor{red}{\ding{55}}
    \newline
    \textbf{Observation:} The returned "CC1(C)CC1" does not correspond to cyclopropane. Cyclopropane should be a simple three-carbon ring, typically represented as "C1CC1". Next, I will call the Name2SMILES tool to further correct the result.
    \newline
    \textbf{Action:} Name2SMILES  
    \newline
    \textbf{Input:} Cyclopropane  
    \newline
    \textbf{Output:} C1CC1  
    \newline
    \textbf{Observation:} Based on the known structure of cyclopropane, the result from this tool is correct. Therefore, I will provide the correct SMILES based on the understanding of the chemical structure.  
    \newline
    \textbf{Final Answer: C1CC1 \textcolor{red}{\ding{52}}}\\
    \hline
    \end{tabular}
    }
    \caption{Case analysis of \textbf{"Correct"} behavior patterns of agents during tools stacking.}
    \label{tab:case study}
\end{table*}


%该分子由三个氮原子组成的环状结构,它具有芳香性的特性和两个亚甲基。它是一种环烯烃,也是两种异构体之一。请预测它的 SMILES。
%