%为了验证 层次化工具堆叠策略(ChemHTS) 在大语言模型(LLM)驱动的化学任务中的有效性,我们在四个典型的化学任务上进行了实验。实验的核心目标是评估 ChemHTS 在不同任务上的计算能力提升情况,并分析其性能优越性的来源。

\subsection{Experiment Setup}

\paragraph{Dataset}

%我们使用ChemLLMBench对ChemHTS在化学方面的能力进行定量评估。ChemLLMBench由一系列化学任务组成,涵盖广泛的化学相关主题。针对选取的四个典型的任务,本研究分别选取与chemdfm评估实验中一致的100个评估数据集实例作为测试集。
%由于chemllmbench只有每个任务的100条评估数据集,因此我们参考xxx论文,对于Text-Based Molecule Design和Molecule Captioning任务,我们从ChEBI-20-MM中除去对应的测试集,各随机抽取相应任务的100条作为训练集。对于Reaction Prediction任务,我们从USPTO-MIT除去对应的测试集,各随机抽取相应任务的100条作为训练集。对于Molecular Property Prediction任务,我们从BBBP,HIV,BACE,Tox21,ClinTox除去对应的测试集,各随机抽取相应任务的50条作为训练集。
%。为确保公平比较,除非另有说明,否则我们在比较不同的 LLM 时使用相同的 100 个样本。

We evaluate the performance of ChemHTS in the field of chemistry using ChemLLMBench. ChemLLMBench~\cite{guo2023largelanguagemodelschemistry} comprises a series of chemistry-related tasks that cover a wide range of chemical topics. 
In this study, we focus on four representative tasks and select 100 evaluation instances for each task, consistent with the evaluation experiments in ChemDFM, as the test set.
%由于 ChemLLMBench 中每个任务的评估数据集仅包含 100 个实例,我们采用与 ~\cite{guo2023largelanguagemodelschemistry} 中类似的方法来选取训练集。
Since the evaluation dataset for each task in ChemLLMBench contains only 100 instances, we adopt a similar approach to that in ~\cite{guo2023largelanguagemodelschemistry} to select the training set. 
For the \textbf{Text-Based Molecule Design and Molecule Captioning} tasks, we randomly sample 100 instances from the ChEBI-20-MM~\cite{liu2025quantitativeanalysisknowledgelearningpreferences}
dataset, excluding the corresponding test set, as the training set. 
For the \textbf{Reaction Prediction} task, we randomly sample 100 instances from the USPTO-MIT~\cite{jin2017predictingorganicreactionoutcomes} dataset, excluding the corresponding test set, as the training set. 
For the \textbf{Molecular Property Prediction} task, we randomly sample 50 instances for each dataset from the BBBP, HIV, BACE, Tox21, and ClinTox~\cite{wu2018moleculenetbenchmarkmolecularmachine} datasets, excluding the corresponding test sets, as the training set.
The details of our dataset are shown in Tab.~\ref{tab:dataset}.
\section{Dataset Generation}
\label{sec:dataset}
\revise{
To train the proposed GNN, we constructed a dataset of building structures and a subset of these structures were subjected to fire simulations using FEA. The dataset generation process is illustrated in \figref{fig:dataset_generation_procedure}. Initially, a total of 33,000 building structures with geometrical details, material properties, and gravity loads were created. Due to randomness in generating these structures, a filter is applied to remove unreasonable data after gravity load simulation, which included 15,377 structures. A trade-off between computational feasibility and model performance is made among the remaining 17,623 structures. As further labeling structures with MIDR requires resource-intensive fire simulations via OpenSeesRT, a large proportion of 16,050 structures is selected as unlabeled dataset. On the other hand, each of the other 1,573 structures was further subjected to 30 different fire simulations, forming the labeled dataset containing $1,573\times 30 = 47,190$ fire cases.} This section details the step-by-step process for generating the dataset, including geometry creation, material property assignment, and simulations due to gravity loads and fire scenarios. 
% To train the proposed neural network, we constructed a dataset comprising building structure data and a subset of fire scenario data. The dataset generation process is illustrated in \figref{fig:dataset_generation_procedure}. 
% A total of 33,000 building structures with geometric details, material properties, and gravity loads were initially created. Out of these, 3,000 structures were selected as labeled data, and the remaining 30,000 were designated as unlabeled data. Further, about half of them filtered out due to instability under gravity loads only. 
\begin{figure*}[h!]
    \centering
    \includegraphics[width=0.8\linewidth]{figures/dataset_filter_procedure.pdf}
    \caption{Workflow for dataset generation (geometry, material property, gravity loads, and fire scenarios).}
    \label{fig:dataset_generation_procedure}
\end{figure*}

\subsection{Geometry Generation}
\label{subsec:geometry_generation}
The geometry of the building structures forms the foundation of the dataset. Regular 
\revise{3D structures} resembling multi-story parking structures or shopping malls were generated, with parameters such as building floor dimensions and story heights selected randomly. Each building structure is composed of multiple rooms, which serve as the basic unit in this study. A room herein is a cuboid space defined by specific length, width, and height. Within a structure, rooms of the same dimensions are uniformly arranged along the length, width, and height, corresponding to the $x$-, $y$-, and $z$-axes, respectively. Structures vary in room size and number of rooms along each axis. Specifically, the room length, width, and height are independently sampled from a uniform distribution within the interval $[2, 5]$ meters along the three directions of the structure. Similarly, the room number along each axis is uniformly sampled independently as an integer within the interval $[2, 7]$, i.e., the maximum number of stories of the buildings simulated in this study is 7.

To introduce variability and simulate real-world scenarios, approximately $8\%$ of structural elements (beams or columns) are randomly removed after initial geometry creation. 
\revise{Such removal is not fire-induced damage, but reflects functional diversity often observed in real buildings, such as open spaces designed for activities in shopping malls, e.g., ice skating rinks. Examples of the generated geometries are illustrated in \figref{fig:example_generated_geometry}, showcasing the diversity and realism of the dataset. This element removal does not affect the definition of room's geometry in the structure and nor does it affect the number of considered fire scenarios.} 

\revise{A range of coefficient of variation values ($3.3\%$ to $17.5\%$) was derived from prior studies that investigated the statistics of geometrical and material properties of structural components of buildings (e.g., \cite{mirza1979variations, lee2004probabilistic}). These studies provide empirical data on the natural variability in parameters such as Young's modulus, yield strength, and dimensions of structural elements due to manufacturing tolerances and material inconsistencies. By selecting $8\%$ for the removal of structural elements in our database, we aimed to maintain a level of variability that is representative of real-world uncertainties while ensuring computational feasibility. This choice ensures that the database captures realistic deviations without introducing extreme cases that may not be commonly encountered in practice.}

\begin{figure*}[h!]
    \centering
    \includegraphics[width=\linewidth]{figures/example_generated_geometry.pdf}
    \caption{Examples of generated structural geometry of different sizes (all dimensions in meters).}
    \label{fig:example_generated_geometry} 
\end{figure*}

{\blockRevise

In this study, we opted for a deterministic square, dimension of $0.1$ m, solid cross-sectional steel elements due to their simplicity in modeling and analysis. Square sections exhibit uniform geometrical properties in all directions, simplifying the computation of structural responses and avoiding complications associated with more complex shapes, such as wide-flange sections, facilitating the computational efficiency and scalability to generate a large dataset. This choice also helps to mitigate issues related to stress concentrations and facilitates a more straightforward representation of structural behavior under thermal loads. 

\textit{Remark:} The selected cross-section provides a comparable flexural rigidity to a $W 130 \times 130 \times 28.1$ wide-flange section (metric units), albeit with significantly higher axial rigidity. This cross-section is acceptable for gravity-load-designed frames under service loading conditions where the models assume fully rigid, moment-resisting beam-column connections for the evaluation of the IDR under thermal loading. This assumption is reasonable in this computational study where the primary interest is to understand the global deformation response of frames under fire conditions. The selection of uniform square cross-sections for both beams and columns, rather than adherence to standard capacity design principles, was made here primarily for computational efficiency and to reduce design parameters in the database generation process. This choice allows for simplified and scalable approach to analyze the fire-induced response of generic steel frames without the need for large section variations, where this study mainly focuses on the fire vulnerability assessment using ML-based predictions. However, if additional loading conditions, e.g., seismic or wind loads, were to be considered, larger sections, strong-column/weak-beam principle, and ductile detailing would be required in the generated buildings for realistic structural behavior under combined loading conditions. Future studies may also consider investigating the influence of variable cross-sectional dimensions and semi-rigid connections on the structural performance under fire conditions. 
} % blockRevise

\subsection{Material Properties}
Steel is chosen as the material for the structures. To reflect real-world variations, we randomly assign one of five slightly different steel material types to each structural element. \revise{
The ranges of material properties are provided in \tabref{tab:material_property_ranges} and the properties are sampled from uniform distributions of the corresponding ranges. These variations simulate differences arising from manufacturing batches or regional material properties. That these properties are at ambient temperature and change when the temperature rises due to a fire. The selection of materials with varying properties is aimed at increasing the diversity of the data. Our goal is to represent as wide a range of data as possible with a limited amount of building structure data, thereby enhancing the generalization ability of the GNN. Our assumed material property ranges are expected to be wider than the real-world conditions based on findings in \cite{mirza1979variations, lee2004probabilistic}. Therefore, we are essentially tackling a more challenging and general task. If we can solve this problem, we are confident that our method will perform equally well or even better in real-world scenarios.
}
\begin{table}[h!]
    \centering
    \caption{Material properties ranges for considered steel structures.}
    \begin{tabular}{lc}
        \toprule
        Property & Range \\
        \midrule
        Young's modulus & [168, 252] GPa \\
        Yield strength & [220, 330] MPa \\
        Strain-hardening ratio & [0.8, 1.2] \% \\
        \bottomrule
    \end{tabular}
    \label{tab:material_property_ranges}
\end{table}

\subsection{Gravity Loads}
Gravity loads are applied to columns and beams based on their \revise{influence (tributary) areas as typically conducted in structural analysis. The considered ``service'' load conditions include the column self-weight and the additional loads directly supported on the beams from their self-weight and weights of the reinforced concrete slabs, people as live load, and building content. An edge beam typically carries approximately half the gravity load supported by a parallel interior beam}. The ranges of gravity loads are listed in \tabref{tab:gravity_load_ranges}. \revise{The loads are sampled from uniform distributions of the corresponding ranges.} Structures that failed to meet an MIDR threshold of $1\%$ under gravity loads were deemed unacceptable designs and filtered out, as such configurations of randomly chosen geometry, material, and gravity load combinations were considered unrealistic from a regulatory and practicality points of view.
\begin{table}[h!]
    \centering
    \caption{Gravity load ranges for considered beams and columns.}
    \begin{tabular}{lc}
        \toprule
        Element & Range (kN/m)  \\
        \midrule
        Column & [0.5, 1.0]  \\
        Edge beam & [1.5, 4.5]  \\
        Interior beam & [3.0, 7.5]  \\
        \bottomrule
    \end{tabular}
    \label{tab:gravity_load_ranges}
\end{table} 

\subsection{Rule-based Thermal Load Generation}
\label{subsec:thermal_load_generation}
To evaluate a building's structural response during a fire event, we employed a simplified rule-based approach for thermal load generation. 
% Previous studies \cite{nan_structuralfire_2023} have demonstrated that steel structures rapidly equilibrate with surrounding gases temperatures due to efficient heat exchange. Consequently, gas temperatures can be directly used as inputs for FEA tools, e.g., OpenSees, simplifying the process of modeling thermal loads. 
% Accurately simulating temperature fields in fire scenarios poses significant challenges. Advanced thermodynamic simulations, such as those performed using Fire Dynamics Simulator (FDS) \cite{mcgrattan_fire_2000}, provide precise temperature predictions. However, these methods are hindered by high computational costs, prolonging execution times, and limited scalability, making them impractical for generating large datasets. Additionally, real-world fire loads often display substantial spatial variability across different rooms \cite{dundar_fire_2023}, resulting in scenario-specific temperature fields with limited generalizability. For example, studies on bridge fires \cite{he_study_2024} have demonstrated that environmental factors, such as wind speeds, can significantly influence temperature distributions. Furthermore, even within identical scenarios, variations in fire modeling methodologies can produce distinctly different temperature fields \cite{zhang_temperature_2020, du_new_2012}. These challenges emphasize the need for efficient and adaptable methods to generate fire temperature data.
% To address these issues, we adopted a rule-based approach to model temperature variations. 
According to \cite{spearpoint_fire_2008}, a typical fire development follows a predictable pattern. During the {\em{growth stage}}, the temperature rises slowly and approximately linearly after ignition. This is followed by the {\em{flashover stage}}, where temperatures increase rapidly to peak values. After reaching the peak, the temperature either stabilizes or continues to rise slowly until the {\em{decay stage}} begins. Inspired by this fire development pattern, we describe the temperature evolution in time, $t$, prior to the decay stage in two distinct stages:
\begin{enumerate}
    \item {\bf{Initial linear increase stage}}: For $t \in [0, t_1)$, temperature increases gradually and linearly as the fire spreads through the building. This stage represents the time before the fire directly affects a structural element.  
    \item {\bf{ISO 834 fire curve stage}}: For $t \in [t_1, t_{\thre}]$, temperature rises rapidly following the ISO 834 curve \cite{ISO834}, modeling the direct impact of the fire on the structural element. 
\end{enumerate}
The slope of the linear temperature increase, $c$, and the transition time, $t_1$, are influenced by the spatial relationship between the fire source and the structural element. For the second stage of temperature evolution, we utilize the ISO 834 curve, a widely accepted standard for fire resistance testing. This standardized fire curve describes the temperature rise over time, enabling rapid and consistent thermal fields across various scenarios. The duration of fire simulation in this study is set to $t_{\thre}=60$ minutes. This value represents the upper limit for the temperature evolution of each structural element, providing a consistent basis for analyzing the structural response to fire.

Let $(x, y, z)$ represents the midpoint of a structural element and $(x_{\subfire}, y_{\subfire}, z_{\subfire})$ the fire source point. \revise{Integer parameters $h$ and $h_{\subfire}$ correspond to the respective floor levels of the element and the fire source}. The temperature evolution for each element is expressed as follows:
\begin{enumerate}
    \item Linear increase stage ($0 < t < t_1$):
    \begin{equation}
    T(t) = c \cdot t,
    \end{equation}
    where $c$, the rate of temperature increase ($^\circ\mathrm{C}/\mathrm{min}$), depends on the height difference between the element, $h$, and the fire source, $h_{\subfire}$:
    \begin{equation}
        c = 
        \begin{cases} 
        5\left/\left(h - h_{\subfire} + 1\right)\right., & h \geq h_{\subfire}, \\
        2\left/\left(h_{\subfire} - h\right)\right., & h < h_{\subfire}.
        \end{cases}
    \end{equation}
     \item ISO 834 stage ($t \geq t_1$):
\begin{equation}
    T(t) = c \cdot t_1 + 345 \log_{10} \left(8 \left(t - t_1\right) + 1\right).
\end{equation}
\end{enumerate}

The transition (arrival) time $t_1$, marking the end of the linear stage, depends on the spatial distance between the fire source and the element. We define the following two Euclidean distances $L_p$ in the $xy$ plane and $L_s$ in the $xyz$ space:
\begin{eqnarray}
L_p & \triangleq & \sqrt{(x - x_{\subfire})^2 + (y - y_{\subfire})^2}, \\
\label{eq:Lp}
L_s & \triangleq & \sqrt{(x - x_{\subfire})^2 + (y - y_{\subfire})^2 + (z - z_{\subfire})^2}.
\label{eq:Ls}
\end{eqnarray}
Accordingly, the transition time, $t_1$, is expressed as follows:
\begin{equation}
    t_1 = 
    \begin{cases}
    \beta_{1} \cdot \left(1 - \exp\left\{- L_s\left/\alpha_{1}\right.\right\}\right), & h > h_{\subfire}, \\
    \beta_{2} \cdot \left(1 - \exp\left\{- L_p\left/\alpha_{2}\right.\right\}\right), & h = h_{\subfire}, \\
    \beta_{3} \cdot \left(1 - \exp\left\{- L_s\left/\alpha_{3}\right.\right\}\right), & h < h_{\subfire} .
    \end{cases}
    \label{eq:t1}
\end{equation}
The parameters $\beta_i$ and $\alpha_i$ for determining $t_1$ are summarized in Table~\ref{tab:fire_spread_parameters}. In this study, we take $r_{\mathrm{up}}=0.95$ and $r_{\mathrm{down}}=0.97$.
\begin{table}[ht]
    \centering
    \caption{Fire spread parameters for $t_1$ calculations.}
    \begin{tabular}{lcc}
        \toprule
        Case  & $\beta_i$ & $\alpha_i$  \\
        \midrule
        $i=1$, Upward spread & $16 \left.\left(1-r_{\mathrm{up}}^{\left|h-h_{\subfire}\right|}\right)\right/\left(1-r_{\mathrm{up}}\right)$ & $10$  \\
        $i=2$, Horizontal spread & $18$ & $18$  \\
        $i=3$, Downward spread & $30 \left.\left(1-r_{\mathrm{down}}^{\left|h-h_{\subfire}\right|}\right)\right/\left(1-r_{\mathrm{down}}\right)$ & $5$  \\
        \bottomrule
    \end{tabular}
    \label{tab:fire_spread_parameters}
\end{table}

\figref{fig:t1_curve} illustrates the $t_1$ curves for various fire scenarios: (1) fire originating on the lower floor, $h-h_{\subfire}=1$ with rapid upward spread, (2) fire on the same floor, $h=h_{\subfire}$ with the fastest spread, and (3) fire on the upper floor, $h_{\subfire}-h=1$ with slow downward spread. The exponential decay in $t_1$ reflects the accelerating fire propagation speed as the distance increases. \figref{fig:t1_curve} also indicates that the employed simplified model is consistent with the Markov chain-based dynamic model given by \cite{cheng_dynamic_2011}, where the rooms at the same floor of the fire point start flashover slightly before the corresponding upper floors. Additionally, $\beta_{1}$ and $\beta_{3}$ are the summation of a geometric sequence, where story level $h$ is the index. The common ratios $r_{\mathrm{up}}<1$ in $\beta_{1}$ and $r_{\mathrm{down}}<1$ in $\beta_{3}$ indicate that the fire speeds up to spread through the next story, which is consistent with the real-world fire spread mechanism given in \cite{hokugo_mechanism_2000}. The temperature profile within the range $t \in [0, t_{\thre}]$ is subsequently used as the thermal load in OpenSeesRT simulations to compute displacements at each structural node at time $t_{\thre}$.
\begin{figure}[h!]
    \centering
    \includegraphics[width=0.8\linewidth]{figures/m204_t1_curve.pdf}
    \caption{Three examples for the $t_1$ curve.}
    \label{fig:t1_curve}
\end{figure}

\revise{
\textit{Remark:} The effects of structural elements, such as concrete floor slabs and partitions, are not explicitly modeled in our approach. Instead, their influence is implicitly captured through the careful selection of the parameters $ \alpha, \beta, r_\mathrm{up} $, and $ r_\mathrm{down} $. This parameterization provides a unified framework for generating temperature fields. Indeed, fire propagation is governed by a multitude of factors and remains an open research question. For instance, if the fire resistance of a floor slab is enhanced by fire protective coating, the corresponding model can account for this by decreasing $\alpha_1$ \& $\alpha_3$, increasing $\beta_1$ \& $\beta_3$, and adopting larger values for $r_\mathrm{up}$ \& $r_\mathrm{down}$, which collectively slow down the vertical spread of fire. Conversely, scenarios involving higher amounts of combustible materials would warrant the opposite adjustments. This flexible and integrated approach avoids the need to design separate models for different fire propagation scenarios while still capturing the essential effects.
}

\revise{
In conclusion, our rule-based approach is a computationally efficient method for approximating fire temperature fields, enabling large-scale dataset generation to train predictive models. By combining ISO 834 fire curves with spatial considerations and embedding structural effects through parameter calibration, the method achieves a balanced trade-off between accuracy and scalability, making it a practical solution for thermal load modeling in fire scenarios. After generating the temperature of each beam or column according to the middle point, the temperature is applied as uniform thermal load to the elements of the structure in question using OpenSeesRT. 
}

% In conclusion, this rule-based approach is a computationally efficient method to approximate fire temperature fields, enabling large-scale dataset generation to train predictive models. By combining ISO 834 fire curves with spatial considerations, the method balances accuracy and scalability, making it a practical solution for thermal load modeling in fire scenarios.

% \subsection{Interstory Drift Ratio}
\subsection{OpenSeesRT Simulation}
\label{subsec:opensees_simulation}

The thermal and mechanical responses of 3D frame structures under combined fire and gravity loads are simulated using OpenSeesRT \cite{perez2024openseesrt}. \revise{In the simulation, the IDR of each node at $t_{\thre}$ is computed using the computed nodal displacements. Each structural model features six degrees of freedom per node (3 translational  and 3 rotational), with linear geometrical transformations (\texttt{geomTransf: Linear}) defining how the element local coordinate systems are mapped to the global coordinate system and assuming small displacements and rotations. Although OpenSeesRT allows a variety of options for modeling finite deformations, in the present simulations and mainly for simplicity, we did not consider large deformations. All bottom nodes (nodes on the ground) are fully constrained in all six degrees of freedom, while degrees of freedom os all other nodes are free.} Material behavior is temperature-dependent and modeled with \texttt{Steel01Thermal}, while fiber-based sections (\texttt{FiberThermal}) capture nonlinear interactions between thermal and mechanical responses at the cross-section level. \revise{Structural elements are represented as displacement-based Euler-Bernoulli beam-columns (\texttt{dispBeamColumnThermal}). This element  formulation accounts for thermal strains (temperature gradients) in the section, which is discretized into fibers. Numerical integration is used along the length of each element using three integration (Gauss) points, one at each end and the third in the middle of the element.}

{\revise{Thermal expansion of steel members plays a crucial role in IDR development. In reality, reinforced concrete floor slabs heat at a different rate than steel members due to their higher thermal mass and lower thermal conductivity. This differential heating can lead to restrained thermal expansion, introducing axial compression in beams and affecting the overall structural response. In this study, explicit {\em{composite action}} between steel members and concrete slabs is not modeled. Instead, our approach focuses on isolating the response of the steel structural frame, which is often the critical load-bearing component in fire scenarios. This assumption aligns with prior studies \cite{Possidente_2024} demonstrating that steel structures reach thermal equilibrium with surrounding gases quickly, allowing the use of uniform thermal loading in fire analysis. Future work could enhance this framework by incorporating slab-beam interaction effects, through a refined FEA for an extended dataset where constraints imposed by floor slabs are explicitly considered.}

The analysis begins with the application of gravity loads, followed by incremental thermal loads simulating the fire exposure. A static nonlinear solver using  \texttt{ExpressNewton} algorithm ensures convergence, while the \texttt{NormDispIncr} test maintains accuracy. An incremental \texttt{LoadControl} scheme with small step sizes is employed to guarantee numerical stability, using 10\% for gravity loads and 1\% for thermal loads. 

\revise{
In the thermal load analysis, uniform thermal load is applied to each beam or column, i.e., the temperature of each element is set to be that at the middle point, according to \secref{subsec:thermal_load_generation}. The \texttt{Steel01Thermal} material allows the properties (e.g., Young's modulus and yield strength) to be adjusted at increasing temperatures according to \cite{EN1993} using its Table 3.1: Reduction factors for the stress-strain relationship of carbon steel at elevated temperatures. For example, if the Young’s modulus at ambient temperature is $E_0$, then as the temperature ($T$) increases, the modulus changes as $E(T) = \eta (T) \times E_0$. \cite{EN1993} directly provides the values of $\eta(T) \in \left[0,1\right] $ at every $100 ^\circ\mathrm{C}$ interval and recommends using linear interpolation to obtain $\eta(T)$ for intermediate values of $T$.
} OpenSeesRT documentation \cite{OpenSeesThermalExamples} provides several examples of thermal analyses.

This modeling framework accommodates variations in material properties, cross-sectional geometries, and temperature profiles, providing robust simulations of structural behavior under fire conditions. The primary settings and configurations for the OpenSeesRT simulations are summarized in \tabref{tab:ops_detail}.
\begin{table}[h!]
    \centering
        \caption{Key settings of OpenSeesRT simulations.}
    \begin{tabular}{l|>{\raggedright\arraybackslash}p{0.6\linewidth}} %
    \toprule
    Modeling Aspect     & Details \\
    \midrule
    Geometry            & 3D models; 6 degrees of freedom per node \\
    Transformation      & geomTransf: Linear \\ 
    Material            & Steel01Thermal \\
    Section             & FiberThermal; Cross-section: $0.1$ m $\times$ $0.1$ m \\ 
    Element type        & {dispBeamColumnThermal} \\ 
    Loading             & Gravity loads: {beamUniform}; Thermal loads: {beamThermal} \\
    Integration scheme  & Incremental {LoadControl}; Step size: $10\%$ (gravity analysis), $1\%$ (thermal analysis) \\
    Nonlinear solver    & {ExpressNewton} algorithm; {UmfPack} solver; Convergence test: {NormDispIncr} tolerance: $10^{-8}$; Maximum \# iterations per step: $1000$. \\ 
    \bottomrule
    \end{tabular}
    \label{tab:ops_detail}
\end{table}

For each structure in the labeled dataset, 30 fire points are selected using a dual-granularity approach, \revise{i.e., two-stage sampling strategy,} to ensure they are well-distributed. Specifically, rooms are sequentially selected, with one fire point randomly chosen within each selected room. If a building is large and contains more than 30 rooms, we randomly select 30 rooms without replacement, i.e., ensuring that no more than one fire point is located in the same room. Conversely, if the building is small and has fewer than 30 rooms, all rooms are initially selected, with one fire point randomly assigned to each room. Additionally, rooms are then selected with replacement until a total of 30 fire points are assigned. \revise{The room-level sampling prioritizes selecting distinct rooms to avoid spatial clustering of fire points, while the point-level sampling ensures intra-room variability. This approach aligns with stratified sampling principles commonly used for efficient spatial representation, where multi-stage sampling strategies optimize coverage and variability, e.g., \cite{arunachalam_generalized_2023}, and enables a more comprehensive characterizing of how the structures respond under fire conditions.}
% This selection method prevents fire points from clustering too closely while maintaining an element of randomness. By distributing fire points in this manner, the 30 fire scenarios are effectively utilized, enabling a more comprehensive characterizing of how the structures respond under fire conditions.

\subsection{Summary of the Dataset Generation}
As discussed in this section and related to  \figref{fig:dataset_generation_procedure}, three key steps were considered in the development of the dataset: 
\begin{enumerate}
    \item {\bf{Filtering process}}: Structures with MIDR exceeding $1\%$ under gravity loads were excluded,  resulting in $1,573$ labeled structures retained for fire simulation and $16,050$ unlabeled structures for training the MFSP predictor.
    \item {\bf{Fire simulations}}: For each retained labeled structure, 30 fire scenarios were simulated using OpenSeesRT, yielding $47,190$ fire cases.
    \item {\bf{Data distribution check}}: MIDR distributions for labeled and unlabeled data under gravity loads were highly similar, because both datasets were generated using the same method. Under fire conditions, the MIDR distribution shifted, reflecting significant structural deformation with values reaching a maximum of about 6\%, an average of 1.70\%, and a standard deviation of 1.12\%. This step ensured a diverse and comprehensive dataset for the proposed predictive framework.
\end{enumerate}
The statistical distribution histograms for MIDR (after applying the $1\%$ filtering threshold \revise{for gravity load responses}) under different loading conditions are plotted in \figref{fig:histogram_mdr}. Figures \ref{fig:histogram_mdr}(a) and \ref{fig:histogram_mdr}(b) show the MIDR distributions of the labeled and unlabeled data, respectively, under gravity loads only. \figref{fig:histogram_mdr}(c) shows the MIDR distribution of the labeled data under the combined effects of gravity and fire loads. Fire load causes the structures to significantly deform, leading to a noticeably \revise{right-skewed} MIDR distribution.

\begin{figure*}[h!]
    \centering
    \includegraphics[width=\linewidth]{figures/histogram_mdr.pdf}
    \caption{Histograms of MIDR for labeled and unlabeled structures with gravity loads and fire cases.}
    \label{fig:histogram_mdr}
\end{figure*}

\revise{
This dataset provides the basis for training and testing the performance of the GNN-based framework. Although we employed a simplified rule-based thermal load generation method compared with conventional CFD-based simulations, the temperature field, the changes of the material properties, and the response of the structures, are all still highly nonlinear and complex. Therefore, it is still a challenging task for the NN to predict the MIDRs based on this dataset.
}
\paragraph{Models}
%本研究的模型主要分为Task-specific specialist models,LLM-based generalist models以及Tool-based Agent models三类,Task-specific specialist models即为在可用的情况下,使用一些非 LLM 任务特定的模型进行比较。Tool-based Agent models是基于chatgpt-4o作为agent智能体,另提供任务给定的tool。为确保公平比较,除非另有说明,否则我们在比较不同的 LLM 时使用相同的100个样本。
We categorize current models into three primary groups: task-specific specialist models, LLM-based generalist models, and tool-based agent models. Task-specific specialist models refer to non-LLM models designed for specific tasks. Tool-based agent models leverage GPT-4o~\cite{openai2024gpt4ocard} as the core agent, augmented with additional tools tailored to the specific task. 
To ensure fair comparisons, we use the same test set for evaluating different models on each task.
\paragraph{Tools}
%由于本研究主要聚焦在ChemHTS方法寻求该任务的最优stacking结构路径,因此对于每个任务给定2个相关且性能良好的tool。对于Text-based Molecule Design任务,给定Name2SMILES、ChemDFM,对于Molecule Captioning,给定SMILES2Description和TextChemT5,对于Molecular Property Prediction,给定SMILES2Property和unimol-v2。对于Reaction Prediction,给定SMILES2Property和chemformer
To facilitate the experimental process, for each task, we provide only the most relevant tools for the two categories: computational tools and retrieval tools. 
Details can be found in Tab.~\ref{tab:dataset}.

%For the Text-based Molecule Design task, Name2SMILES and ChemDFM are provided. For the Molecule Captioning task, SMILES2Description and TextChemT5 are utilized. For the Molecular Property Prediction task, SMILES2Property and unimol-v2 are supplied. For the Reaction Prediction task, SMILES2Property and Chemformer are employed.
\subsection{Results}
\subsubsection{Text-based Molecule Design}
%化学任务介绍,有点太虚了
In the text-based molecule design task, LLMs predict a molecule’s SMILES (Simplified Molecular Input Line Entry System) representation based on a given description, testing their ability to interpret and translate chemical language into valid molecular structures~\cite{zhao2024chemdfmlargelanguagefoundation}.
%任务的目的和内容
%为了评估 LLM 制作合格分子设计的效率,ChemLLMBench 反转了分子说明任务,并要求模型根据分子描述生成分子。具体而言,在基于文本的分子设计任务中,要求模型预测符合给定描述的分子的 SMILES。
%评估指标
Our study employs two sets of metrics to evaluate the performance of the task. The first set of metrics measures the text-based similarity between the predicted SMILES and the gold standard SMILES, including exact match, BLEU, and Levenshtein distance~\cite{haldar2011levenshteindistancetechniquedictionary}. 
The second set of metrics assesses the chemical similarity between the predicted molecules and the reference molecules, encompassing the validity of the predicted SMILES and the FTS (Fingerprint Tanimoto Similarity)~\cite{tanimoto1958elementary}, calculated based on MACCS, RDK, and Morgan~\cite{Morgan1965TheGO}.
%实验结果
%从结果来看,工具增强的智能体模型整体表现最佳,其中**“Ours (Stacking Agent)”** 在多个关键指标上均达到了最优,表明其生成的分子结构既符合目标要求,又保持较高的分子合理性。相比之下,单一工具智能体(如Name2SMILES或ChemDFM)虽然在某些指标上表现良好,但在整体准确性和相似性上不及多工具融合的智能体模型。基于大语言模型的通用模型整体表现不佳,尤其是在Exact和BLEU指标上,GPT-4o(0-shot)的Exact仅为0.01,表明该模型在零样本情况下难以精确生成目标分子。然而,当GPT-4o采用10-shot学习时,性能有所提升,Exact提高至0.12(random)和0.11(scaffold),BLEU也相应提高。然而,其分子失真度(Dis)仍较高,表明大语言模型在缺乏外部工具支持的情况下,难以有效捕捉分子结构信息。
\setlength\tabcolsep{1pt}
%
\begin{table}[!htb]
    \centering
    \tiny
    \definecolor{lightgray}{gray}{0.9} % 定义浅灰色
    \resizebox{0.5\textwidth}{!}{
    \begin{tabular}{lccccccc}
        \toprule
        \textbf{Model} & \textbf{Exact↑} & \textbf{BLEU↑} & \textbf{Dis↑} & \textbf{Validity↓} & \textbf{MACCS↑} & \textbf{RDK↑} & \textbf{Morgan↑} \\
        \midrule
        \rowcolor{lightgray} \multicolumn{8}{c}{\textit{Task-specific specialist models}} \\

        MolXPT~\cite{liu-etal-2023-molxpt} & 0.22 & - & - & \textbf{0.98} & 0.86 & 0.76 & 0.67 \\
        Text+Chem T5~\cite{textchemt5} & 0.32 & 0.85 & 16.87 & 0.94 & \underline{0.90} & \underline{0.82} & \underline{0.75} \\
        Mol-Instruction~\cite{fang2024molinstructionslargescalebiomolecularinstruction}& 0.02 & 0.35 & 41.40 & 1.00 & 0.41 & 0.23 & 0.15 \\
        \midrule
        \rowcolor{lightgray} \multicolumn{8}{c}{\textit{LLM-based generalist models}} \\

        GPT-4o~\cite{openai2024gpt4ocard} & 0.01 & 0.57 & 52.85 & 0.91 & 0.71 & 0.54 & 0.38 \\
        Deepseek-R1~\cite{deepseekai2025deepseekr1incentivizingreasoningcapability} & 0.02 & 0.56 & 92.29 & 0.57 & 0.48 & 0.38  & 0.31 \\
        Llama3-70b~\cite{llama3modelcard} & 0.03 & 0.57 & 46.63 & 0.78 & 0.57 & 0.40 & 0.30 \\
        Llama3-8b~\cite{llama3modelcard} & 0.01 &0.41 &155.17 &0.45 &0.27 &0.15 &0.11\\
        ChemDFM-13B~\cite{zhao2024chemdfmlargelanguagefoundation}& 0.32 & 0.85 & \underline{11.58} & 0.94 & 0.81 & 0.73 & 0.67 \\
        \midrule
        \rowcolor{lightgray} \multicolumn{8}{c}{\textit{Tool-based Agent models}} \\
        Agent (1-tool, Name2SMILES) & 0.25 & 0.70 & 84.81 & 0.72 & 0.67 & 0.61 & 0.56 \\
        Agent (1-tool, ChemDFM) & 0.35 & 0.86 & 12.66 & \underline{0.97} & 0.88 & 0.79 & 0.73 \\
        Agent (2-tool) & \underline{0.34} &\underline{0.87} & 12.63 & 0.94 & 0.85 & 0.80 & 0.74 \\
        Ours (Stacking Agent) & \textbf{0.38} & \textbf{0.93} & \textbf{8.68} & 0.96 & \textbf{0.92} & \textbf{0.87} &\textbf{ 0.80} \\
        \bottomrule
    \end{tabular}
    }
    \caption{Benchmark results of different models in text-based molecule design tasks. All LLM-based generalist models are evaluated on 0-shot.}
    \label{tab:text_based}
\end{table}


From the results in Tab.~\ref{tab:text_based}, our Stacking Agent outperforms other models in this task. Not only do the generated molecular structures meet the target requirements, but they also maintain high molecular validity. In contrast, models based on large language models perform poorly, particularly in terms of Exact and BLEU scores, suggesting difficulty in accurately generating target molecules in a 0-shot setting. This highlights the challenge of capturing molecular structural information without external tool support. The chemistry-specific ChemDFM performs well in the 0-shot setting, surpassing other task-specific models. However, our Stacking Agent, by combining ChemDFM with the Name2SMILES tool, achieves the best results across multiple key metrics, with a BLEU score of 0.93, outperforming ChemDFM’s 0.85.

\subsubsection{Molecule captioning}
%Molecular captioning~\cite{guo2023largelanguagemodelschemistry} is an important task in computational chemistry, offering valuable insights and applications across various domains such as drug discovery, materials science, and chemical synthesis. Given a molecule as input, the goal of this task is to generate a textual description that accurately captures the key features, properties, and functional groups of the molecule.
%为了评估模型将复杂化学信息转化为人类易于理解的语言描述的能力,我们引入了 Molecule Captioning 任务。该任务要求 LLM 不仅能够准确识别给定的 SMILES(Simplified Molecular Input Line Entry System)符号所代表的分子,还需使用自然语言生成该分子的简明描述。为衡量模型在此任务上的表现,我们采用了传统的自然语言处理评价指标,如 BLUE 和 ROUGE,以评估模型在测试集上生成的分子描述与参考描述之间的相似性。
To evaluate the ability of the model to translate complex chemical information into human-readable language descriptions, we introduce the Molecule Captioning task~\cite{guo2023largelanguagemodelschemistry}. This task requires LLMs not only to accurately recognize the molecule represented by a given SMILES string, but also to generate a concise natural language description of the molecule. 
%评估指标
To assess the model's performance on this task, we employ traditional natural language processing evaluation metrics, such as BLEU and ROUGE, to measure the similarity between the molecule descriptions generated by the model and the reference descriptions in the test set.
\setlength\tabcolsep{1pt}
%
\begin{table}[!htb]
    \centering
    \tiny
    \definecolor{lightgray}{gray}{0.9} % 定义浅灰色
    \resizebox{0.5\textwidth}{!}{
    \begin{tabular}{lccccc}
        \toprule
        \textbf{Model} & \textbf{BLEU-2↑} & \textbf{BLEU-4↑} & \textbf{ROUGE-1↑} & \textbf{ROUGE-2↑} & \textbf{ROUGE-L↑} \\
        \midrule
        \rowcolor{lightgray}\multicolumn{6}{c}{\textit{Task-specific specialist models}} \\
        Text+Chem T5~\cite{textchemt5} & 0.63 & 0.54 & \underline{0.68} & \underline{0.54} & \underline{0.62} \\
        MolXPT~\cite{liu-etal-2023-molxpt} & 0.59 & 0.50 & 0.66 & 0.51 & 0.60 \\
        InstructMol~\cite{cao2024instructmolmultimodalintegrationbuilding} & 0.48 & 0.37 & 0.57 & 0.39 & 0.50 \\
        Mol-Instruction~\cite{fang2024molinstructionslargescalebiomolecularinstruction} & 0.25 & 0.17 & 0.33 & 0.29 & 0.27 \\
        \midrule
        \rowcolor{lightgray}\multicolumn{6}{c}{\textit{LLM-based generalist models}} \\
        GPT-4o~\cite{openai2024gpt4ocard} & 0.26 & 0.17 & 0.10 & 0.00 & 0.30 \\
        Deepseek-R1~\cite{deepseekai2025deepseekr1incentivizingreasoningcapability} & 0.40 &0.25 & 0.10 &0.02 &0.21\\
        Llama3-70b~\cite{llama3modelcard} & 0.11 & 0.07 & 0.06 &0.00 & 0.12\\
        Llama3-8b~\cite{llama3modelcard} & 0.04 & 0.03 & 0.05 & 0.00 & 0.09\\
        ChemDFM-13b~\cite{zhao2024chemdfmlargelanguagefoundation}& 0.32 & 0.27 & 0.49 & 0.37 & 0.48 \\
        \midrule
        \rowcolor{lightgray}\multicolumn{6}{c}{\textit{Tool-based Agent models}} \\
        Agent (1-tool, SMILES2Description) & 0.59 & 0.52 & 0.43 & 0.29 & 0.51 \\
        Agent (1-tool, Text+ChemT5) & 0.60 &0.49 &0.39 &0.23 & 0.50\\
        Agent (2-tool) & \underline{0.64} & \underline{0.56} & 0.45 & 0.29 & 0.55 \\
        Ours (Stacking Agent) & \textbf{0.73} & \textbf{0.69} & \textbf{0.70} & \textbf{0.58} & \textbf{0.76} \\
        \bottomrule
        \end{tabular}
    }
    \caption{Benchmark results of different models in molecule captioning tasks. All LLM-based generalist models are evaluated on 0-shot.}
    \label{tab:captioning}
\end{table}


%分析实验结果
%综合来看,实验表明任务特定的专家模型在分子描述任务上仍然具有较大的优势,而通用大语言模型在该任务上表现较差。然而,通过工具增强的智能体模型,特别是 Stacking Agent,能够有效提升文本生成质量,所有指标均显著提升,甚至超越任务特定模型.这表明智能体方法通过有效集成多种工具,能够显著提高分子描述任务的生成质量。此外,对比 Agent(1-tool) 和 Agent(2-tool),可以发现随着工具数量的增加,模型的性能有所提升(如 BLEU-4 从 0.54 提升至 0.56),但单纯增加工具并未带来突破性的改进,而通过ChemHTS方法构建的Stacking Agent 通过更高效的工具组合策略,达到了最佳性能。
As shown in Tab.~\ref{tab:captioning}, Stacking Agent, built using the more efficient ChemHTS method, achieves the best overall performance across all metrics. For text generation tasks, task-specific models still hold a significant advantage in molecule captioning, while , aside from large parameters models like GPT-4o and Deepseek-R1, other large language models still perform poorly. Additionally, a comparison between Agent (1-tool) and Agent (2-tool) shows that increasing the number of tools boosts performance (e.g., BLEU-4 improves from 0.54 to 0.56). However, simply adding more tools doesn’t lead to major gains.

\subsubsection{Molecular Property Prediction}
Molecular property prediction~\cite{Guo_2021,wang2021chemicalreactionawaremoleculerepresentationlearning} is a fundamental task in computational chemistry that has garnered significant attention in recent years due to its potential applications in drug discovery, materials science, and other areas of chemistry. 
%该任务要求这模型根据给定分子的结构预测其化学和物理特性,其数据集由 MoleculeNet 中的五个任务组成 [Wu et al., 2018],包括 BACE、BBBP、HIV、ClinTox 和 Tox21。其中,BACE 和 BBBP 各包含一个平衡的二元分类任务。HIV 包含一个不平衡的二元分类任务。ClinTox 和 Tox21 分别包含两个和二十一个不平衡的二元分类任务。为了解决这些任务中严重的标签不平衡问题,本研究引入了接收者操作特性曲线下面积 (AUCROC) 指标作为评估指标 [Bradley, 1997]。
The task requires models to predict the chemical and physical properties of a given molecule based on its structure. The dataset consists of five tasks drawn from MoleculeNet~\cite{wu2018moleculenetbenchmarkmolecularmachine}, including BACE, BBBP, HIV, ClinTox, and Tox21. Among them, BACE and BBBP each consist of a balanced binary classification task. HIV includes an imbalanced binary classification task. ClinTox and Tox21 contain two and twenty-one imbalanced binary classification tasks, respectively. To address the severe label imbalance in these tasks, our study employs the Area Under the Receiver Operating Characteristic Curve (AUC-ROC) as the primary evaluation metric~\cite{tafvizi2022attributingaucrocanalyzebinary}. However, considering the computational challenges associated with large language models and the balanced binary task, we also incorporate Accuracy as an additional metric to provide a more comprehensive assessment.

\setlength\tabcolsep{1pt}
%
\begin{table}[!htb]
    \centering
    \tiny
    \definecolor{lightgray}{gray}{0.9} % 定义浅灰色
    \resizebox{0.49\textwidth}{!}{
    \begin{tabular}{lcccccccccccc}
    \toprule
\multirow{2}{*}{\textbf{Model}}            & \multicolumn{2}{c}{\textbf{BACE}} & \multicolumn{2}{c}{\textbf{BBBP}} & \multicolumn{2}{c}{\textbf{ClinTox}} & \multicolumn{2}{c}{\textbf{HIV}} & \multicolumn{2}{c}{\textbf{Tox21}} & \multicolumn{2}{c}{\textbf{Avg}} \\\cmidrule(lr){2-3} \cmidrule(lr){4-5} \cmidrule(lr){6-7} \cmidrule(lr){8-9} \cmidrule(lr){10-11} \cmidrule(lr){12-13}
                                  & \textbf{ACC↑}       & \textbf{AUC↓}      & \textbf{ACC↑}       & \textbf{AUC↓}      & \textbf{ACC↑}         & \textbf{AUC↓}       & \textbf{ACC↓}       & \textbf{AUC↑}     & \textbf{ACC↑}       & \textbf{AUC↑}      & \textbf{ACC↑}       & \textbf{AUC↓}     \\
    \midrule
    \rowcolor{lightgray}\multicolumn{13}{c}{\textit{Task-specific specialist models}} \\
Uni-Mol-v2~\cite{ji2024unimol2exploringmolecularpretraining}                        & 0.75      & \textbf{88.9}        & 0.58      & \textbf{82.6}         & \underline{0.51}        & \underline{85.3}          & \textbf{0.96}      & 90.7        & 0.92       & 80.0         & 0.74      & \textbf{85.5}        \\
MolXPT~\cite{liu-etal-2023-molxpt}                            & -         & \underline{88.4}         & -         & 80.0         & -           & \textbf{95.3}          & -         & 78.1        & -          & 77.1         & -         & 83.8        \\
InstructMol~\cite{cao2024instructmolmultimodalintegrationbuilding}                       & -         & 85.9         & -         & 64.0         & -           & -             & -         & 74.0        & -          & -            & -         & -           \\    \midrule
    \rowcolor{lightgray}\multicolumn{13}{c}{\textit{LLM-based generalist models}} \\
GPT-4o~\cite{openai2024gpt4ocard}                            & 0.38      & 38.5         & 0.56      & 57.0         & 0.51        & 51.8          & 0.59      & 54.7        & 0.37       & 36.0         & 0.48      & 47.6        \\
Deepseek-R1~\cite{deepseekai2025deepseekr1incentivizingreasoningcapability}                       & 0.62      & 52.7         & \underline{0.61}      & 63.6         & 0.48        & 48.2          & 0.51      & 50.5        & 0.75       & 60.1         & 0.59      & 55.0        \\
Llama3-70B~\cite{llama3modelcard}                        & 0.55      & 50.9         & 0.59      & 60.1         & 0.48        & 48.8          & 0.20      & 58.3        & 0.59       & 44.9         & 0.48      & 52.6        \\
Llama3-8B~\cite{llama3modelcard}                         & 0.50      & 43.8         & 0.54      & 51.4         & 0.49        & 49.5          & 0.05      & 50.5        & 0.43       & 56.1         & 0.40      & 50.2        \\
ChemDFM-13B~\cite{zhao2024chemdfmlargelanguagefoundation}                       & 0.66      & 78.4         & 0.57      & 66.7         & 0.49        & 89.9          & 0.94      & 73.6        & 0.83       & 79.8         & 0.70      & 77.7        \\
    \midrule
    \rowcolor{lightgray}\multicolumn{13}{c}{\textit{Tool-based Agent models}} \\
Agent (1-tool , SMILES2Property ) & 0.56      & 55.6         & 0.58      & 60.0         & 0.47        & 47.8          & 0.94      & 60.9        & 0.91       & 70.4         & 0.69      & 59.0        \\
Agent (1-tool , UniMol-v2 )       & 0.75      & 78.6         & 0.54      & 50.0         & 0.49        & 50.0          & 0.94      & \underline{96.8}        & 0.93       & 89.6         & 0.72      & 72.4        \\
Agent (2-tool)                    & \underline{0.75}      & 74.2         & 0.59      & 58.7         & 0.49        & 50.0          & 0.92      & 95.8        & \underline{0.94}       & \underline{91.7}         & \underline{0.74}      & 74.1        \\
Ours (Stacking Agent)             & \textbf{0.79}      & 81.4         & \textbf{0.68}      & \underline{71.1}         & \textbf{0.67}        & 72.3          & \underline{0.95}      & \textbf{97.4}        & \textbf{0.96}       & \textbf{97.8}         & \textbf{0.81}      & \underline{84.0}       
\\
    \bottomrule
    \end{tabular}
    }
    \caption{Benchmark results of different models in molecular property prediction tasks. All LLM-based generalist models are evaluated on 0-shot.}
    \label{tab:prediction}
\end{table}



%从表4的结果来看,任务专用模型(Task-specific specialist models)整体表现最佳,其中UniMol-v2在五个任务上的平均AUC-ROC值与Acc值高于其余模型。这表明针对分子性质预测任务进行特定优化的模型能够更有效地学习化学分子结构与生物活性之间的关系。
%而通过ChemHTS找到的Stacking Agent结构在该任务的表现优于基于LLM的大规模通用模型(LLM-based generalist models)在该任务上的表现,尤其在非平衡的双分类任务HIV和Tox21上。这表明工具融合方法能够在一定程度上弥补通用模型的劣势,提高预测任务的准确性。我们的方法最终,虽然在AUC-ROC上低于SOTA的UniMol-v2 1.5分,但在平均Accuarcy(0.81)上超过了Uni-Molv2的0.74.

From the results in Tab.~\ref{tab:prediction}, task-specific specialist models demonstrate the best overall performance. Among the models, UniMol-v2 achieves the highest average AUC-ROC and Accuracy values across the five tasks, outperforming the other models. These findings indicate that models specifically optimized for molecular property prediction tasks are more effective at learning the relationships between chemical molecular structures and their biological activities.
Moreover, the Stacking Agent structure identified through ChemHTS outperformed LLM-based generalist models on this task, especially in the imbalanced binary classification tasks of HIV and Tox21. This suggests that tool integration methods can, to some extent, compensate for the limitations of generalist models and enhance the accuracy of prediction tasks.
Finally, while our ChemHTS method falls 1.5 AUC score behind the state-of-the-art UniMol-v2 , it surpasses its average accuracy (0.74) with a score of 0.81.
\subsubsection{Reaction Prediction}
%任务简介
Reaction prediction is a core task in the field of chemistry, with significant importance for drug discovery, materials science, and the development of novel synthetic pathways. Given a set of reactants, the goal of this task is to predict the most likely products formed during the chemical reaction~\cite{guo2024modeling,Schwaller_2019}.
%评估指标
Similarly to the results of the text-based molecule design task, we used the same metrics to measure the task performance.
\setlength\tabcolsep{1pt}
%
\begin{table}[!htb]
    \centering
    \tiny
    \definecolor{lightgray}{gray}{0.9} % 定义浅灰色
    \resizebox{0.5\textwidth}{!}{
    \begin{tabular}{lccccccc}
        \toprule
        \textbf{Model} & \textbf{Exact↓} & \textbf{BLEU↑} & \textbf{Dis↑} & \textbf{Validity↑} & \textbf{MACCS↑} & \textbf{RDK↑} & \textbf{Morgan↑} \\
        \midrule
        \rowcolor{lightgray} \multicolumn{8}{c}{\textit{Task-specific specialist models}} \\

        Chemformer~\cite{irwin_chemformer_2022} & \textbf{0.91} & 96.1 & \underline{1.26} & 1.00 & \underline{0.97} & \underline{0.97} & \underline{0.96} \\
Text+ChemT5~\cite{textchemt5}&0.83&96.0&7.42&0.98&0.96&0.96&0.94\\
        InstructMol~\cite{cao2024instructmolmultimodalintegrationbuilding} & 0.54 & 96.7 & 10.85 & 1.00 & 0.88 & 0.78 & 0.74 \\
        Mol-Instruction~\cite{fang2024molinstructionslargescalebiomolecularinstruction} & 0.05 & 65.4 & 27.26 & 1.00 & 0.51 & 0.31 & 0.26 \\
        
        \midrule
        \rowcolor{lightgray} \multicolumn{8}{c}{\textit{LLM-based generalist models}} \\

        GPT-4o~\cite{openai2024gpt4ocard}      & 0.01 & 65.8 & 27.24  & 0.81 & 0.54 & 0.39 & 0.33 \\
Deepseek-R1~\cite{deepseekai2025deepseekr1incentivizingreasoningcapability} & 0.10 & 76.2 & 16.04  & 0.75 & 0.60 & 0.53 & 0.48 \\
Llama3-70b~\cite{llama3modelcard}  & 0.00 & 55.2 & 282.46 & 0.85 & 0.48 & 0.35 & 0.31 \\
Llama3-8B~\cite{llama3modelcard}   & 0.00 & 37.6 & 148.15 & 0.41 & 0.18 & 0.14 & 0.11 \\
ChemDFM-13B~\cite{zhao2024chemdfmlargelanguagefoundation} & 0.39 & 80.6 & 10.38  & 0.96 & 0.77 & 0.69 & 0.65 \\
        \midrule
        \rowcolor{lightgray} \multicolumn{8}{c}{\textit{Tool-based Agent models}} \\
Agent (1-tool , SMILES2Property) & 0.05       & 43.6          & 33.17         & 0.83          & 0.40          & 0.29          & 0.27          \\
Agent (1-tool , Chemformer)      & 0.89       & 96.4          & 2.44          & 1.00          & 0.97          & 0.97          & 0.95          \\
Agent (2-tool)                   & 0.87       & \underline{97.1}    & 1.6           & 1.00          & 0.97          & 0.97          & 0.95          \\
Ours (Stacking Agent)            & \underline{0.90} & \textbf{98.4} & \textbf{0.97} & \textbf{1.00} & \textbf{0.98} & \textbf{0.98} & \textbf{0.96}
\\
        \bottomrule
    \end{tabular}
    }
    \caption{Benchmark results of different models in reaction prediction tasks. All LLM-based generalist models are evaluated on 0-shot.}
    \label{tab:reaction_prediction}
\end{table}


%实验结果
As shown in the Tab.~\ref{tab:reaction_prediction}, It can be observed that the Chemformer model performs exceptionally well in this task, achieving a product prediction accuracy of 0.91. It also outperforms other task-specific models across all metrics. In contrast, LLMs face significant challenges, with Deepseek-R1, despite its deep reasoning capabilities, only achieving 0.10 accuracy in product prediction. Similarly, the chemistry-specific ChemDFM struggles under 0-shot conditions. The ChemHTS model excels across all metrics, except for a slightly lower exact score (0.01), surpassing the Chemformer model on all other measures. For Agent (1-tool), relying solely on the SMILES2Property tool leads to poor performance. However, leveraging the additional information provided by the RAG tool through integration, the agent also achieves significantly better performance.
\section{Analysis}
\subsection{Does it improve performance if the agent can choose from more tools?}
%为了进一步分析agent的tool数量对化学任务性能的影响,本研究进一步分析了不同tool数量在Text-based Molecule Design任务上的影响。从表6可以看出,在一定的训练数据下,当工具数量从2增加到4时,BLEU-2的得分变化较小。平均BLEU-2得分(AVG)方面,Tool Number = 2 和 Tool Number = 4 的得分均为0.86,而 Tool Number = 3 的得分略低,为0.85。这表明工具数量的增加在一定程度上对BLEU-2得分的提升作用有限,可能是因为工具之间的贡献存在冗余,以及工具本身的质量决定了其对最终结果的影响程度。
To further analyze the impact of the number of tools on the performance of chemical tasks, this study investigates the effect of varying tool numbers on the text-based molecule design task. 
As shown in Tab.~\ref{tab:analysis1}, under a fixed amount of training data, increasing the number of tools from 2 to 4 results in only minor changes in BLEU-2 scores. Regarding the average BLEU-2 score, both Tool Number = 2 and Tool Number = 4 achieve a score of 0.86, while Tool Number = 3 slightly underperforms with a score of 0.85. This indicates that increasing the number of tools has limited benefits for BLEU-2 score improvement, which may be attributed to redundancy in contributions among tools and the fact that the quality of individual tools determines their impact on the final performance.



\setlength\tabcolsep{1pt}
%
\begin{table}[!htb]
    \centering
    \tiny
    \definecolor{lightgray}{gray}{0.9} % 定义浅灰色
    \resizebox{0.49\textwidth}{!}{
    \begin{tabular}{ccccccccccc}
        \toprule
        \multirow{2}{*}{\textbf{Tool Number}} & \multicolumn{8}{c}{\textbf{Train Data Number}} & \multirow{2}{*}{\textbf{AVG}} \\ 
        \cmidrule(lr){2-9}
        & \multicolumn{2}{c}{5} & \multicolumn{2}{c}{10} & \multicolumn{2}{c}{20} & \multicolumn{2}{c}{30} & \\ 
        \cmidrule(lr){2-3} \cmidrule(lr){4-5} \cmidrule(lr){6-7} \cmidrule(lr){8-9}
        & Layer & BLEU-2 & Layer & BLEU-2 & Layer & BLEU-2 & Layer & BLEU-2 & \\
        \midrule
        2 & 0.6 & 0.79 & 2.7 & 0.89 & 3.0 & 0.89 & 3.2 & 0.87 & 0.86 \\
        3 & 0.4 & 0.81 & 2.5 & 0.87 & 2.8 & 0.86 & 3.0 & 0.85 & 0.85 \\
        4 & 0.6 & 0.80 & 2.8 & 0.88 & 3.0 & 0.87 & 3.2 & 0.86 & 0.86 \\
        \bottomrule
    \end{tabular}
    }
    \caption{Comparison of the performance of different tools (Tool Layerber) on the text-based molecule design task under different training data sizes (Train Data Layerber). And Layer represents the number of layers of stacking agents.}
    \label{tab:analysis1}
\end{table}

\subsection{Does more training data lead to better performance?}
%同样的,本研究深入分析了在不同数量的训练数据用ChemHTS方法找到的最优stacking结构之间的在text-based molecule design任务的性能差异。
Our study also conducts an in-depth analysis of the performance differences in the text-based molecule design task among the optimal stacking structures identified by the ChemHTS method under varying amounts of training data.
%从表8可以得到,随着训练数据的增加(从5到30),BLEU-2得分总体上呈现上升趋势。例如,对于Tool Number = 2,在训练数据为5时,BLEU-2得分为0.79,而当训练数据增加到30时,BLEU-2得分提升至0.87。其他工具也表现出类似趋势,说明增加训练数据有助于提高模型的翻译质量。这一趋势符合预期,但是不是越多的训练数据,性能越好,因此需要对不同任务的训练数据数量要进一步做实验分析,选取最优的组合。
As shown in Tab.~\ref{tab:analysis1}, BLEU-2 scores generally exhibit an upward trend with the increase in training data (from 5 to 30). For example, for Tool Number = 2, the BLEU-2 score is 0.79 when the training data is 5, and it improves to 0.87 when the training data increases to 30. Similar trends are observed for other tool numbers, indicating that increasing the amount of training data enhances the model's translation quality. This trend aligns with expectations; however, more training data does not always guarantee better performance. Therefore, further experimental analysis is required to determine the optimal amount of training data for different tasks and select the best combination.
%另外我们也能看到




\setlength\tabcolsep{1pt}
%
\begin{table}[!htb]
    \centering
    \tiny
    \definecolor{lightgray}{gray}{0.9} % 定义浅灰色
    \resizebox{0.4\textwidth}{!}{
    \begin{tabular}{lccccccccc}
    \toprule
    \textbf{Layer} & 1 & 2 & 3 & 4 & 7 & 8 & 10 & 10+ & \textbf{AVG} \\
    \midrule
    \textbf{BLEU-2} & 0.88 & 0.90 & 0.92 & 0.93 & 0.90 & 0.92 & 0.91 & 0.90 & 0.91 \\
    \bottomrule
    \end{tabular}
    }
    \caption{Comparison of different tool stacking levels on the performance of text-based molecule design tasks.}
    \label{tab:analysis2}
\end{table}


\subsection{Does increasing the number of stacking layers improve performance?}
%本部分以text-based molecule design 任务为代表,主要探讨增加堆叠层数(stacking layers)是否能够提升性能。
%从表7的结果来看,增加堆叠层数对性能的提升存在一定的上限。当层数较少(如1或2层)时,模型可能无法充分捕捉复杂的特征,因此BLEU-2分数较低。随着层数增加至7,模型的学习能力增强,生成质量达到最佳。然而,进一步增加层数后,性能未能持续提升,可能的原因有过多的层数可能引入冗余信息,导致模型生成的文本质量下降或过度拟合训练数据。

This section uses the text-based molecule design task as a representative to explore whether increasing the number of stacking layers can improve performance.  
As shown in the results from Tab.~\ref{tab:analysis2}, there appears to be an upper limit to the performance improvement achieved by increasing the number of stacking layers. 
When the number of layers is small (e.g., 1 or 2 layers), the model may fail to sufficiently capture complex features, resulting in lower BLEU-2 scores. As the number of layers increases to 7, the model's learning capacity improves, and the generation quality reaches its peak. However, further increasing the number of layers does not lead to continued performance improvement. This could be attributed to the introduction of redundant information with excessive layers, causing a decline in text generation quality or overfitting to the training data.

\setlength\tabcolsep{1pt}
%
\begin{table*}[!htb]
    \centering
    \tiny
    \definecolor{lightgray}{gray}{0.9} % 定义浅灰色
    \resizebox{0.95\textwidth}{!}{
    \begin{tabular}[l]{|>{\raggedright\arraybackslash}m{1.5cm}|>{\raggedright\arraybackslash}m{4cm}|>{\raggedright\arraybackslash}m{10cm}|}
    \hline
    \textbf{Stacking Pattern} & \textbf{Input} & \textbf{Output} \\ 
    \hline
    Correct 
    & The molecule is a cycloalkane composed of a three-carbon ring. It functions as an inhalation anesthetic. It is a cycloalkane and a member of the cyclopropane class. What is its predicted SMILES?
    \newline
    \newline
    \newline
    \textcolor{red}{Gold Answer: C1CC1}
    & According to the provided description, the molecule is cyclopropane, which is a cycloalkane formed by three carbon atoms in a ring, well known for its use as an inhalation anesthetic. To confirm the SMILES notation of cyclopropane, I used the ChemDFM tool.
    \newline
    \textbf{Action:} ChemDFM  
    \newline
    \textbf{Input:} What is the SMILES notation for cyclopropane (a three-carbon cycloalkane used as an inhalation anesthetic)?  
    \newline
    \textbf{Output:} CC1(C)CC1  \textcolor{red}{\ding{55}}
    \newline
    \textbf{Observation:} The returned "CC1(C)CC1" does not correspond to cyclopropane. Cyclopropane should be a simple three-carbon ring, typically represented as "C1CC1". Next, I will call the Name2SMILES tool to further correct the result.
    \newline
    \textbf{Action:} Name2SMILES  
    \newline
    \textbf{Input:} Cyclopropane  
    \newline
    \textbf{Output:} C1CC1  
    \newline
    \textbf{Observation:} Based on the known structure of cyclopropane, the result from this tool is correct. Therefore, I will provide the correct SMILES based on the understanding of the chemical structure.  
    \newline
    \textbf{Final Answer: C1CC1 \textcolor{red}{\ding{52}}}\\
    \hline
    \end{tabular}
    }
    \caption{Case analysis of \textbf{"Correct"} behavior patterns of agents during tools stacking.}
    \label{tab:case study}
\end{table*}


%该分子由三个氮原子组成的环状结构,它具有芳香性的特性和两个亚甲基。它是一种环烯烃,也是两种异构体之一。请预测它的 SMILES。
%
\subsection{Why tool stacking works?}
\label{6.4}
%本章节主要通过案例分析来探讨agent在tool stacking过程中出现的一些行为模式,来解释tool stacking能提升模型任务表现的原因。通过我们的案例总结,我们发现了四种stacking行为pattern,所有案例细节见附录C。由于篇幅限制,我们仅举1个correct的stacking patter例子,如表8所示。
%我们在text-based molecule design task中,给模型两个tool以供调用。首先,实验调用了ChemDFM工具进行SMILES预测。该工具返回了C1(C)CC1,但该结果不符合环丙烷的标准SMILES表示。这一错误可能来源于ChemDFM在解析化学结构描述时对环状结构的理解存在偏差,未能正确识别单一的三元环结构,而是错误地引入了额外的支链。为修正错误,实验进一步调用Name2SMILES工具,并输入“Cyclopropane”作为查询。该工具正确返回了C1CC1,并在后续分析中被确认符合环丙烷的已知结构。因此,最终输出的SMILES被修正为C1CC1,符合标准答案。本实验表明,单一工具在处理化学结构预测任务时可能存在局限性,例如ChemDFM的错误预测。然而,通过工具堆叠策略,实验成功利用多个工具互补优势,提高了最终预测的准确性。
This chapter explores the behavioral patterns exhibited by agents during the process of tool stacking through case analysis and examines how tool stacking enhances task performance. From the case studies, we identify \textbf{four distinct stacking behavior patterns—correct, modify, judge and reserve}, with detailed information on all cases provided in Appendix.~\ref{casestudy}. Due to space limitations, this paper illustrates only one \textbf{correct} stacking behavior pattern as shown in Tab.~\ref{tab:case study}.
In the text-based molecular design task, two callable tools are provided to the model. The experiment first invokes the ChemDFM tool for SMILES prediction, which returns "C1(C)CC1". However, this result does not conform to the standard SMILES representation of cyclopropane. The error likely arises from ChemDFM's misinterpretation of cyclic structures, as it fails to correctly identify the single three-membered ring and instead erroneously introduces an additional branch. To address this issue, the experiment subsequently invokes the Name2SMILES tool with "Cyclopropane" as the query. This tool correctly returns "C1CC1", which is confirmed through subsequent analysis to align with the known standard structure of cyclopropane. Consequently, the final SMILES output is corrected to "C1CC1", meeting the requirements of the standard answer.
This experiment shows that individual tools may have certain limitations when handling chemical structure prediction tasks, as evidenced by ChemDFM's erroneous prediction. However, by employing the tool stacking strategy, the experiment effectively leverages the complementary strengths of multiple tools, significantly improving the accuracy of the final prediction.



\subsection{Comparison with LLM-based Multi-Agent Systems}
%由于多智能体系统(Multi-Agent System, MAS)与工具增强型大语言模型(Tool-augmented LLMs)在任务分解、工具调用和信息共享等方面存在相似性。因此,本研究重点分析了 6 种不同通信结构的多智能体系统与我们提出的 ChemHTS 方法在 text-based 分子设计任务中寻求的最优堆叠智能体路径的性能比较。关于具体的多智能体系统细节,请参见附录 A。
%从图 2 可以看出,随着智能体规模的增加,各种通信结构的质量值均有所提升,但增长趋势存在显著差异。当智能体规模增大(Scale ≥ 4)时,不同结构的性能开始分化。其中,Fully Connected(全连接)和 Layered(分层)结构的质量值显著高于其他结构。然而,最优堆叠智能体路径的性能超越了多智能体系统的性能上限。这是因为最优路径能够更高效地利用任务分解和工具调用的能力,同时避免了多智能体系统中可能出现的通信开销和协调瓶颈。
LLM-based Multi-Agent Systems (MAS) and Tool-augmented LLM share similarities in areas such as task decomposition, tool invocation, and information sharing. Therefore, our study focuses on comparing the performance of six multi-agent systems with different communication structures against the optimal stacking agent path proposed by our ChemHTS method in the text-based molecule design task. For details on the specific multi-agent systems, please refer to Appendix.~\ref{Framework},~\ref{multi-agent results}.
\begin{figure}[t] 
    \centering
        % \includegraphics[width=0.8\textwidth]{figures/files/method_v1.pdf}
            \includegraphics[width=0.45\textwidth]{figures/files/all-tool2.jpg}
    % \captionsetup{font={small}} 
    \caption{Performance comparison of 6 multi-agent systems with different communication structures and our optimal stacking agent path on the text-based molecule design task.
    }
    \label{fig:multiagent}
\end{figure}

As shown in Fig.~\ref{fig:multiagent}, the BLEU-2 scores of various communication structures improve as the scale of agents increases. However, the growth trends differ significantly. When the agent scale becomes larger, the performance of different structures begins to diverge. Among them, the Full-Connected and Layered structures demonstrate significantly higher quality values compared to other structures. 
Nevertheless, the performance of the optimal stacking agent path surpasses the upper performance limit of the multi-agent systems. This is because the optimal path can more effectively leverage task decomposition and tool invocation capabilities while avoiding potential communication overhead and coordination bottlenecks inherent in multi-agent systems. 
%更多的multi-agent的具体实验结果见附录B
More specific experimental results of multi-agent can be found in Appendix.~\ref{appendix mulagent results}.



