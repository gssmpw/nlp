\begin{figure}[!ht]
    \centering
    \scalebox{.8}{
    \begin{tikzpicture}[node distance={20mm}, thick, main/.style = {draw, circle, minimum size=0.8cm}]
        \node[main] (e1S1) [] {$e_1S_1$};
        \node[main] (e1Si) [below = 2cm of e1S1] {$e_1S_i$};
        \node[main] (eiS1) [below = 2cm of e1Si] {$e_iS_1$};
        \node[main] (enSm) [below = 2cm of eiS1] {$e_nS_m$};
        \node[main] (x) [left = 2cm of eiS1] {$x$};
        \node[main] (S1) [below right = 1cm and 2cm of e1S1] {$S_1$};
        \node[main] (Si) [below = 2cm of S1] {$S_i$};
        \node[main] (Sm) [below = 2cm of Si] {$S_m$};
        \node (ints1) [draw=none, below = 0.5cm of S1] {\textbf{$\vdots$}};
        \node (intsi) [draw=none, below = 0.5cm of Si] {\textbf{$\vdots$}};
        \draw[dashed, red] (x) -- node[midway, above left] {1} (e1S1);
        \draw[dashed, red] (x) -- node[midway, below] {1} (eiS1);
        \draw[dashed, red] (x) -- node[midway, below left] {1} (enSm);
        \draw[] (e1S1) -- node[pos=0.5, above] {2} (S1);
        \draw[] (e1Si) -- node[pos=0.25, above] {2} (Si);
        \draw[] (eiS1) -- node[pos=0.75, above left] {2} (S1);
        \draw[] (enSm) -- node[pos=0.5, below] {2} (Sm);
        \draw[] (e1S1) -- node[pos=0.5, right] {1} (e1Si);
    \end{tikzpicture}
    }
    \caption{Example of a transformation of an HS instance into 2-TSA. Dashed red edges correspond to the augmentation set for the simple case.}
    \label{fig:2TSA2}
\end{figure}