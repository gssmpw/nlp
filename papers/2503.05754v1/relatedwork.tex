\section{Related Work Review}
\label{sec:lr}
The related work for this research can be divided into two major categories. The first category focuses on the passenger demand analysis, specifically with regard to local and transfer passengers. The second category relates to the time series clustering methods used in the model development herein.

    \subsection{Local and Transfer Passenger Related Research}\label{sec:lr_business}
    % Literature of the related analysis
    % part 1 - the demand of analysis and forcast is the biggest part 
    % part 2 - there are some examples of local and transfer passenger literaures
    % Part 3 - some exasmples of the study after COVID 
    
    % part 1
    Tourism demand modeling and forecasting have been a significant area of interest for both researchers and practitioners over the past five decades~\cite{gunter2021forecasting}. Classic time series models, interpretable regression models, and advanced machine learning techniques have been extensively studied to identify significant changes in passenger demand at various levels of granularity and to forecast future passenger demand. As interest in more detailed analyses of passenger demand grows, particularly the composition of local and transfer passengers, a substantial body of research has emerged on this topic. Maertens et al. \citep{DevelopmentoTransferPassenger} developed a model to analyze transfer passenger data and transfer share globally, albeit at an airport-level (and not at the O\&D level). Their research revealed that different types of airports exhibit varying levels of transfer share. For example, major hubs located in less tourism-oriented cities, such as Atlanta and Doha, achieve higher transfer shares compared to gateway hubs in high-profile destinations like Dubai and New York. This empirical evidence underscores that local and transfer shares are specific to individual airports, with substantial heterogeneity in characteristics. While much of the literature focused on airport-level analyses of passenger share, such as the work by  Choi et al. \citep{DetermingFactorsofAirPassengers}, this research significantly extends the analysis by examining local and transfer share at the O\&D level. By focusing on higher granularity information, this study provides deeper insights into the distribution of local and transfer passengers and facilitates the integration of these findings into broader airport-level analyses.
    
    % part 2
    
    Local and transfer passenger shares are also crucial metrics for network planning and airline competition. The airport choice for transfer passengers is influenced by factors such as airfares, minimum connecting time (MCT), service quality of flight connection, and travel time. Attracting transfer passengers, in particular, is a primary objective in many airports' marketing strategies, as these passengers are not tied to a specific airport's local catchment area, and can easily switch among alternative connections offered by various airlines and hub airports. Consequently, airports face greater competition in the transfer passenger market than in the local O\&D market. Therefore, the development of the transfer passenger market is critical for airports, as transfer traffic supports the hub-and-spoke network, enabling airlines to sustain direct flights on routes that would not be feasible solely based on the local O\&D demand. Redondi et al. \citep{De-hubbingofAirportsandTheir} developed a model to identify de-hubbing (the process where a hub-and-spoke structure airline stops utilizing a certain airport as a connecting hub) airports and analyzed their recovery, studying hub characteristics through quantitative conditions by looking at the number of offered connections. Similarly, Wei and Hansen \citep{AnAggregateDemandModel} developed a model to predict demand in hub-and-spoke networks, considering airline service variables such as service frequency, aircraft size, ticket price, flight distance, the number of spokes in the network, influence of local passengers, as well as social-economic and demographic conditions in the metropolitan areas of the spokes and the hubs. Their study also identified the impact of local passengers on the hub-and-spoke network, demonstrating that airlines can attract more connecting passengers by increasing service frequency. Further, Zheng and Wei \citep{zheng2019air, zheng2020air} conducted data-driven analyses on direct passenger distribution in U.S. domestic O\&D markets, highlighting that the direct share is O\&D specific, for different O\&D pairs the pattern of direct passengers varies significantly. Their research utilized time series and classic machine learning models in forecasting direct passenger distribution, revealing that airline competition is a critical factor influencing direct passenger share.
    
    % part 3
    The significant impact of the COVID-19 pandemic in 2020 prompted extensive research into passenger demand recovery and forecasting. Tirtha et al. \citep{AnAirortLevelFramework} developed a model to examine the impact of COVID-19 on airline demand, incorporating factors in socio-demographic (population, education, age distribution), socio-economic (income, unemployment rate, GDP), built environment variables (number of trade centers, tourist attractions), level of service factors (average airfare and distance) and historical demand as lag variables. Gao \citep{gao2022benchmarking} employed airport-level passenger data to categorize airports into distinct groups according to their air travel demand recovery patterns during the pandemic. Their research indicated that air travel demand at most airports in the U.S. reached the lowest point in April 2020 and has gradually recovered since. While airports in southern regions have seen demand recover to pre-COVID levels by late 2021, many airports in regions such as the Eastern, the Great Lakes, and the New England regions have struggled to regain these levels.


    \subsection{Time Series Clustering Methods in Data Analysis}\label{sec:lr_model}

    Time series clustering is a technique used to uncover patterns and structures within sequential data~\cite{liao2005clustering}. Given the absence of established classification criteria for O\&Ds based on local share, a predefined ground truth for such classification is not readily available. As a result, we employ the following time series clustering to classify local share data by analyzing patterns in the time series. This method can reveal trends and abnormalities across different O\&D groups, offering valuable insights into demand forecasting and network management strategies~\cite{zanin2013modelling}. 

    Hierarchical clustering is a widely used technique in clustering analysis and has been adapted for time series data through the use of Dynamic Time Warping(DynTW)~\footnote{DynTW refers to Dynamic Time Warping algorithm, a time series analysis method, and should not be confused with the Detroit Metro Airport (DTW) airport code} as a distance metric. DynTW allows for flexibility in aligning time series that may exhibit temporal misalignments~\cite{berndt1994using}. In aviation research, DynTW has been applied to cluster flight trajectories, helping to identify similar patterns and commonalities in aircraft movements~\cite{todoric2023comparison, Zhang2024}. Moreover, Wang et al. \citep{wang2023similarity} used DynTW to recognize pilots' operation process sequences during flight tasks to detect operational errors and improvements in overall safety measures.
    
    The $k$-shape clustering algorithm, proposed by Paparrizos and Gravano  \citep{paparrizos2015k}, is a generalization of the $k$-means algorithm that can handle time series data. Unlike traditional $k$-means clustering, the $k$-shape algorithm uses a normalized cross-correlation measure as the distance metric, which also addresses the issue of temporal misalignments found in time series data. For instance, Gao \citep{gao2022benchmarking} applied the $k$-shape algorithm to classify airports based on patterns of air travel demand recovery during the COVID-19 pandemic. The algorithm is effective in identifying clusters of airports that exhibited similar recovery trends, providing valuable insights into post-pandemic recovery strategies.

    Additionally, Self-Organizing Maps (SOMs) are a type of neural network designed to reduce the dimensionality of data while performing clustering based on the topological properties of the input space~\cite{kohonen1990self}. In SOMs, input data vectors are mapped to the closest neuron in a grid-like structure, and similar data points tend to be mapped to nearby neurons, creating a two-dimensional topological map that can reveal inherent patterns within time series data. Kumar \citep{kumar2014self} applied SOMs to airspace sectorization, demonstrating how this technique could improve air traffic management by creating balanced airspace sectors, thereby reducing collision risks and optimizing air traffic flow.

    Shape-based distance (SBD) is another measure used to calculate the similarity between two time series by employing a normalized version of cross-correlation that is invariant to scaling and translation. This indicates that it can effectively handle differences in magnitude or time shifts between series~\cite{paparrizos2017fast}. When combined with Affinity Propagation (AP), an algorithm that automatically determines the optimal number of clusters based on the input data, SBD proves to be effective in clustering time series without the need to specify the number of clusters beforehand~\cite{frey2007clustering}. For example, El-Samak and Ashour \citep{el2015optimization} applied AP clustering to optimize the traveling salesman problem, identifying cities that are efficiently accessed by travelers.

    Furthermore, DynTW Barycenter Averaging (DBA) combined with Gaussian Mixture Models (GMMs) provides an approach for soft clustering of time series data~\cite{petitjean2011global}. In soft clustering, GMMs assign probabilities to each data point's membership in a cluster, rather than a hard, definitive classification. This method is especially beneficial when the time series data exhibits overlap across clusters. 

    As clustering is an unsupervised learning method with no predefined ground truth, it is essential to evaluate the clustering performance using various metrics to ensure the quality and validity of the results. The following evaluation metrics are commonly employed in clustering evaluations:

    \begin{itemize}
        \item \textbf{Silhouette Score:} This metric measures how similar a data point is to its own cluster compared to other clusters~\cite{rousseeuw1987silhouettes}. It ranges from -1 to 1, with a higher score indicating better clustering performance. A higher score suggests that data points are well-matched to their clusters and poorly matched to neighboring clusters.
        \item \textbf{Davies-Bouldin Index:} This index quantifies the average similarity between each cluster and its most similar cluster, focusing on the worst-case pair of clusters~\cite{davies1979cluster}. A lower score is desirable, as it indicates more distinct and well-separated clusters.
        \item \textbf{Dunn Index:} This index evaluates the ratio of the minimum inter-cluster distance to the maximum intra-cluster distance~\cite{dunn1974well}. A higher score reflects better clustering, emphasizing compact clusters that are far apart from each other.
        \item \textbf{Calinski-Harabasz Index:} This index compares the overall inter-cluster variance to the intra-cluster variance~\cite{calinski1974dendrite}. A higher score indicates better clustering performance, as it suggests a higher degree of separation between clusters relative to the internal cohesion within clusters.
    \end{itemize}

    \subsection{Research Gaps}

    Existing research in air transportation passenger flow analysis has predominantly emphasized aggregate demand forecasts, airline competition, and macroeconomic factors, with limited differentiation among passenger categories. Moreover, the current air passenger demand studies often focus on broad metrics at the airport- or system-level, failing to capture the nuanced variations that occur at the O\&D pair level. In-depth research on local and transfer passengers on the O\&D level is becoming increasingly crucial for airline marketing and network planning, airport investment, and authority decision making. A comprehensive understanding of the distinctions between local and transfer passengers remains underdeveloped, leaving a critical gap in the ability to analyze passenger composition and its evolution over time. To address this need, this study examined local share on O\&D level, analyzing the factors driving changes in the local and transfer passenger demand change.