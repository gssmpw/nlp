\documentclass[11pt, sigconf]{acmart}


\AtBeginDocument{%
  \providecommand\BibTeX{{%
    \normalfont B\kern-0.5em{\scshape i\kern-0.25em b}\kern-0.8em\TeX}}}




\usepackage[ruled,vlined]{algorithm2e}
\usepackage[noend]{algpseudocode}


%
\setlength\unitlength{1mm}
\newcommand{\twodots}{\mathinner {\ldotp \ldotp}}
% bb font symbols
\newcommand{\Rho}{\mathrm{P}}
\newcommand{\Tau}{\mathrm{T}}

\newfont{\bbb}{msbm10 scaled 700}
\newcommand{\CCC}{\mbox{\bbb C}}

\newfont{\bb}{msbm10 scaled 1100}
\newcommand{\CC}{\mbox{\bb C}}
\newcommand{\PP}{\mbox{\bb P}}
\newcommand{\RR}{\mbox{\bb R}}
\newcommand{\QQ}{\mbox{\bb Q}}
\newcommand{\ZZ}{\mbox{\bb Z}}
\newcommand{\FF}{\mbox{\bb F}}
\newcommand{\GG}{\mbox{\bb G}}
\newcommand{\EE}{\mbox{\bb E}}
\newcommand{\NN}{\mbox{\bb N}}
\newcommand{\KK}{\mbox{\bb K}}
\newcommand{\HH}{\mbox{\bb H}}
\newcommand{\SSS}{\mbox{\bb S}}
\newcommand{\UU}{\mbox{\bb U}}
\newcommand{\VV}{\mbox{\bb V}}


\newcommand{\yy}{\mathbbm{y}}
\newcommand{\xx}{\mathbbm{x}}
\newcommand{\zz}{\mathbbm{z}}
\newcommand{\sss}{\mathbbm{s}}
\newcommand{\rr}{\mathbbm{r}}
\newcommand{\pp}{\mathbbm{p}}
\newcommand{\qq}{\mathbbm{q}}
\newcommand{\ww}{\mathbbm{w}}
\newcommand{\hh}{\mathbbm{h}}
\newcommand{\vvv}{\mathbbm{v}}

% Vectors

\newcommand{\av}{{\bf a}}
\newcommand{\bv}{{\bf b}}
\newcommand{\cv}{{\bf c}}
\newcommand{\dv}{{\bf d}}
\newcommand{\ev}{{\bf e}}
\newcommand{\fv}{{\bf f}}
\newcommand{\gv}{{\bf g}}
\newcommand{\hv}{{\bf h}}
\newcommand{\iv}{{\bf i}}
\newcommand{\jv}{{\bf j}}
\newcommand{\kv}{{\bf k}}
\newcommand{\lv}{{\bf l}}
\newcommand{\mv}{{\bf m}}
\newcommand{\nv}{{\bf n}}
\newcommand{\ov}{{\bf o}}
\newcommand{\pv}{{\bf p}}
\newcommand{\qv}{{\bf q}}
\newcommand{\rv}{{\bf r}}
\newcommand{\sv}{{\bf s}}
\newcommand{\tv}{{\bf t}}
\newcommand{\uv}{{\bf u}}
\newcommand{\wv}{{\bf w}}
\newcommand{\vv}{{\bf v}}
\newcommand{\xv}{{\bf x}}
\newcommand{\yv}{{\bf y}}
\newcommand{\zv}{{\bf z}}
\newcommand{\zerov}{{\bf 0}}
\newcommand{\onev}{{\bf 1}}

% Matrices

\newcommand{\Am}{{\bf A}}
\newcommand{\Bm}{{\bf B}}
\newcommand{\Cm}{{\bf C}}
\newcommand{\Dm}{{\bf D}}
\newcommand{\Em}{{\bf E}}
\newcommand{\Fm}{{\bf F}}
\newcommand{\Gm}{{\bf G}}
\newcommand{\Hm}{{\bf H}}
\newcommand{\Id}{{\bf I}}
\newcommand{\Jm}{{\bf J}}
\newcommand{\Km}{{\bf K}}
\newcommand{\Lm}{{\bf L}}
\newcommand{\Mm}{{\bf M}}
\newcommand{\Nm}{{\bf N}}
\newcommand{\Om}{{\bf O}}
\newcommand{\Pm}{{\bf P}}
\newcommand{\Qm}{{\bf Q}}
\newcommand{\Rm}{{\bf R}}
\newcommand{\Sm}{{\bf S}}
\newcommand{\Tm}{{\bf T}}
\newcommand{\Um}{{\bf U}}
\newcommand{\Wm}{{\bf W}}
\newcommand{\Vm}{{\bf V}}
\newcommand{\Xm}{{\bf X}}
\newcommand{\Ym}{{\bf Y}}
\newcommand{\Zm}{{\bf Z}}

% Calligraphic

\newcommand{\Ac}{{\cal A}}
\newcommand{\Bc}{{\cal B}}
\newcommand{\Cc}{{\cal C}}
\newcommand{\Dc}{{\cal D}}
\newcommand{\Ec}{{\cal E}}
\newcommand{\Fc}{{\cal F}}
\newcommand{\Gc}{{\cal G}}
\newcommand{\Hc}{{\cal H}}
\newcommand{\Ic}{{\cal I}}
\newcommand{\Jc}{{\cal J}}
\newcommand{\Kc}{{\cal K}}
\newcommand{\Lc}{{\cal L}}
\newcommand{\Mc}{{\cal M}}
\newcommand{\Nc}{{\cal N}}
\newcommand{\nc}{{\cal n}}
\newcommand{\Oc}{{\cal O}}
\newcommand{\Pc}{{\cal P}}
\newcommand{\Qc}{{\cal Q}}
\newcommand{\Rc}{{\cal R}}
\newcommand{\Sc}{{\cal S}}
\newcommand{\Tc}{{\cal T}}
\newcommand{\Uc}{{\cal U}}
\newcommand{\Wc}{{\cal W}}
\newcommand{\Vc}{{\cal V}}
\newcommand{\Xc}{{\cal X}}
\newcommand{\Yc}{{\cal Y}}
\newcommand{\Zc}{{\cal Z}}

% Bold greek letters

\newcommand{\alphav}{\hbox{\boldmath$\alpha$}}
\newcommand{\betav}{\hbox{\boldmath$\beta$}}
\newcommand{\gammav}{\hbox{\boldmath$\gamma$}}
\newcommand{\deltav}{\hbox{\boldmath$\delta$}}
\newcommand{\etav}{\hbox{\boldmath$\eta$}}
\newcommand{\lambdav}{\hbox{\boldmath$\lambda$}}
\newcommand{\epsilonv}{\hbox{\boldmath$\epsilon$}}
\newcommand{\nuv}{\hbox{\boldmath$\nu$}}
\newcommand{\muv}{\hbox{\boldmath$\mu$}}
\newcommand{\zetav}{\hbox{\boldmath$\zeta$}}
\newcommand{\phiv}{\hbox{\boldmath$\phi$}}
\newcommand{\psiv}{\hbox{\boldmath$\psi$}}
\newcommand{\thetav}{\hbox{\boldmath$\theta$}}
\newcommand{\tauv}{\hbox{\boldmath$\tau$}}
\newcommand{\omegav}{\hbox{\boldmath$\omega$}}
\newcommand{\xiv}{\hbox{\boldmath$\xi$}}
\newcommand{\sigmav}{\hbox{\boldmath$\sigma$}}
\newcommand{\piv}{\hbox{\boldmath$\pi$}}
\newcommand{\rhov}{\hbox{\boldmath$\rho$}}
\newcommand{\upsilonv}{\hbox{\boldmath$\upsilon$}}

\newcommand{\Gammam}{\hbox{\boldmath$\Gamma$}}
\newcommand{\Lambdam}{\hbox{\boldmath$\Lambda$}}
\newcommand{\Deltam}{\hbox{\boldmath$\Delta$}}
\newcommand{\Sigmam}{\hbox{\boldmath$\Sigma$}}
\newcommand{\Phim}{\hbox{\boldmath$\Phi$}}
\newcommand{\Pim}{\hbox{\boldmath$\Pi$}}
\newcommand{\Psim}{\hbox{\boldmath$\Psi$}}
\newcommand{\Thetam}{\hbox{\boldmath$\Theta$}}
\newcommand{\Omegam}{\hbox{\boldmath$\Omega$}}
\newcommand{\Xim}{\hbox{\boldmath$\Xi$}}


% Sans Serif small case

\newcommand{\Gsf}{{\sf G}}

\newcommand{\asf}{{\sf a}}
\newcommand{\bsf}{{\sf b}}
\newcommand{\csf}{{\sf c}}
\newcommand{\dsf}{{\sf d}}
\newcommand{\esf}{{\sf e}}
\newcommand{\fsf}{{\sf f}}
\newcommand{\gsf}{{\sf g}}
\newcommand{\hsf}{{\sf h}}
\newcommand{\isf}{{\sf i}}
\newcommand{\jsf}{{\sf j}}
\newcommand{\ksf}{{\sf k}}
\newcommand{\lsf}{{\sf l}}
\newcommand{\msf}{{\sf m}}
\newcommand{\nsf}{{\sf n}}
\newcommand{\osf}{{\sf o}}
\newcommand{\psf}{{\sf p}}
\newcommand{\qsf}{{\sf q}}
\newcommand{\rsf}{{\sf r}}
\newcommand{\ssf}{{\sf s}}
\newcommand{\tsf}{{\sf t}}
\newcommand{\usf}{{\sf u}}
\newcommand{\wsf}{{\sf w}}
\newcommand{\vsf}{{\sf v}}
\newcommand{\xsf}{{\sf x}}
\newcommand{\ysf}{{\sf y}}
\newcommand{\zsf}{{\sf z}}


% mixed symbols

\newcommand{\sinc}{{\hbox{sinc}}}
\newcommand{\diag}{{\hbox{diag}}}
\renewcommand{\det}{{\hbox{det}}}
\newcommand{\trace}{{\hbox{tr}}}
\newcommand{\sign}{{\hbox{sign}}}
\renewcommand{\arg}{{\hbox{arg}}}
\newcommand{\var}{{\hbox{var}}}
\newcommand{\cov}{{\hbox{cov}}}
\newcommand{\Ei}{{\rm E}_{\rm i}}
\renewcommand{\Re}{{\rm Re}}
\renewcommand{\Im}{{\rm Im}}
\newcommand{\eqdef}{\stackrel{\Delta}{=}}
\newcommand{\defines}{{\,\,\stackrel{\scriptscriptstyle \bigtriangleup}{=}\,\,}}
\newcommand{\<}{\left\langle}
\renewcommand{\>}{\right\rangle}
\newcommand{\herm}{{\sf H}}
\newcommand{\trasp}{{\sf T}}
\newcommand{\transp}{{\sf T}}
\renewcommand{\vec}{{\rm vec}}
\newcommand{\Psf}{{\sf P}}
\newcommand{\SINR}{{\sf SINR}}
\newcommand{\SNR}{{\sf SNR}}
\newcommand{\MMSE}{{\sf MMSE}}
\newcommand{\REF}{{\RED [REF]}}

% Markov chain
\usepackage{stmaryrd} % for \mkv 
\newcommand{\mkv}{-\!\!\!\!\minuso\!\!\!\!-}

% Colors

\newcommand{\RED}{\color[rgb]{1.00,0.10,0.10}}
\newcommand{\BLUE}{\color[rgb]{0,0,0.90}}
\newcommand{\GREEN}{\color[rgb]{0,0.80,0.20}}

%%%%%%%%%%%%%%%%%%%%%%%%%%%%%%%%%%%%%%%%%%
\usepackage{hyperref}
\hypersetup{
    bookmarks=true,         % show bookmarks bar?
    unicode=false,          % non-Latin characters in AcrobatÕs bookmarks
    pdftoolbar=true,        % show AcrobatÕs toolbar?
    pdfmenubar=true,        % show AcrobatÕs menu?
    pdffitwindow=false,     % window fit to page when opened
    pdfstartview={FitH},    % fits the width of the page to the window
%    pdftitle={My title},    % title
%    pdfauthor={Author},     % author
%    pdfsubject={Subject},   % subject of the document
%    pdfcreator={Creator},   % creator of the document
%    pdfproducer={Producer}, % producer of the document
%    pdfkeywords={keyword1} {key2} {key3}, % list of keywords
    pdfnewwindow=true,      % links in new window
    colorlinks=true,       % false: boxed links; true: colored links
    linkcolor=red,          % color of internal links (change box color with linkbordercolor)
    citecolor=green,        % color of links to bibliography
    filecolor=blue,      % color of file links
    urlcolor=blue           % color of external links
}
%%%%%%%%%%%%%%%%%%%%%%%%%%%%%%%%%%%%%%%%%%%





    

\newcommand{\sysname}{F\emph{in}P}

\setcopyright{acmlicensed}
\copyrightyear{2025}
\acmYear{2025}
\acmDOI{XXXXXXX.XXXXXXX}
%% These commands are for a PROCEEDINGS abstract or paper.
\acmConference[ACM'25]{ACM Conference}{2025}{ }

\makeatletter
\renewcommand\@formatdoi[1]{\ignorespaces}
\makeatother


\settopmatter{printfolios=true}


\begin{document}


\title{\sysname: Fairness-in-Privacy in Federated Learning by Addressing Disparities in Privacy Risk}




\author{Tianyu Zhao}
\email{tzhao15@uci.edu}
\affiliation{%
  \institution{University of California, Irvine}
  \city{}
  \state{}
  \country{}
}
\author{Mahmoud Srewa}
\email{msrewa@uci.edu}
\affiliation{%
  \institution{University of California, Irvine}
  \city{}
  \state{}
  \country{}
}

\author{Salma Elmalaki}
\email{salma.elmalaki@uci.edu}
\affiliation{%
  \institution{University of California, Irvine}
  \city{}
  \state{}
  \country{}
}

\renewcommand{\shortauthors}{Zhao et al.}








\begin{acronym}
    \acro{har}[HAR]{Human Activity Recognition}
    \acro{tcn}[TCN]{Temporal Convolutional Network }
    \acro{fl}[FL]{Federated Learning}
    \acro{aasd}[AASD]{Average Absolute SIA Difference From Mean}
    \acro{sia}[SIA]{Source Inference Attack}
    \acro{mia}[MIA]{Membership Inference Attack }
    \acro{mad}[MAD]{Mean Average Deviation}
    \acro{acc}[ACC]{Accuracy}
    \acro{rl}[RL]{Reinforcement learning}
    \acro{fedavg}[FedAvg]{Federated Averaging Algorithm}
    \acro{noniid}[non-IID]{Non-Independent and Identically Distributed}
    \acro{dp}[DF]{Differential Privacy}
    \acro{he}[HE]{Homomorphic Encryption}
    \acro{smpc}[SMPC]{Secure Multi-Party Computation}
    \acro{fe}[FE]{Functional Encryption}
\end{acronym}


\begin{abstract}
Ensuring fairness in machine learning, particularly in human-centric applications, extends beyond algorithmic bias to encompass fairness in privacy, specifically the equitable distribution of privacy risk. This is critical in federated learning (FL), where decentralized data necessitates balanced privacy preservation across clients. We introduce \sysname, a framework designed to achieve fairness in privacy by mitigating disproportionate exposure to source inference attacks (SIA). \sysname employs a dual approach: (1) server-side adaptive aggregation to address unfairness in client contributions in global model, and (2) client-side regularization to reduce client vulnerability. This comprehensive strategy targets both the symptoms and root causes of privacy unfairness. Evaluated on the Human Activity Recognition (HAR) and CIFAR-10 datasets, \sysname\ demonstrates a $\approx 20\%$ improvement in fairness in privacy on HAR with minimal impact on model utility, and effectively mitigates SIA risks on CIFAR-10, showcasing its ability to provide fairness in privacy in FL systems without compromising performance.



\end{abstract}

%%
%% The code below is generated by the tool at http://dl.acm.org/ccs.cfm.
%% Please copy and paste the code instead of the example below.
%%
% \begin{CCSXML}
% <ccs2012>
%    <concept>
%        <concept_id>10002978</concept_id>
%        <concept_desc>Security and privacy</concept_desc>
%        <concept_significance>500</concept_significance>
%        </concept>
%    <concept>
%        <concept_id>10010147.10010257.10010258.10010261</concept_id>
%        <concept_desc>Computing methodologies~Reinforcement learning</concept_desc>
%        <concept_significance>500</concept_significance>
%        </concept>
%    <concept>
%        <concept_id>10003120</concept_id>
%        <concept_desc>Human-centered computing</concept_desc>
%        <concept_significance>300</concept_significance>
%        </concept>
%  </ccs2012>
% \end{CCSXML}

% \ccsdesc[500]{Security and privacy}
% \ccsdesc[500]{Computing methodologies~Reinforcement learning}
% \ccsdesc[300]{Human-centered computing}

%%
%% Keywords. The author(s) should pick words that accurately describe
%% the work being presented. Separate the keywords with commas.
\keywords{fairness in privacy, federated learning, human-centered design}

\maketitle

\section{Introduction}
\label{section:introduction}

% redirection is unique and important in VR
Virtual Reality (VR) systems enable users to embody virtual avatars by mirroring their physical movements and aligning their perspective with virtual avatars' in real time. 
As the head-mounted displays (HMDs) block direct visual access to the physical world, users primarily rely on visual feedback from the virtual environment and integrate it with proprioceptive cues to control the avatar’s movements and interact within the VR space.
Since human perception is heavily influenced by visual input~\cite{gibson1933adaptation}, 
VR systems have the unique capability to control users' perception of the virtual environment and avatars by manipulating the visual information presented to them.
Leveraging this, various redirection techniques have been proposed to enable novel VR interactions, 
such as redirecting users' walking paths~\cite{razzaque2005redirected, suma2012impossible, steinicke2009estimation},
modifying reaching movements~\cite{gonzalez2022model, azmandian2016haptic, cheng2017sparse, feick2021visuo},
and conveying haptic information through visual feedback to create pseudo-haptic effects~\cite{samad2019pseudo, dominjon2005influence, lecuyer2009simulating}.
Such redirection techniques enable these interactions by manipulating the alignment between users' physical movements and their virtual avatar's actions.

% % what is hand/arm redirection, motivation of study arm-offset
% \change{\yj{i don't understand the purpose of this paragraph}
% These illusion-based techniques provide users with unique experiences in virtual environments that differ from the physical world yet maintain an immersive experience. 
% A key example is hand redirection, which shifts the virtual hand’s position away from the real hand as the user moves to enhance ergonomics during interaction~\cite{feuchtner2018ownershift, wentzel2020improving} and improve interaction performance~\cite{montano2017erg, poupyrev1996go}. 
% To increase the realism of virtual movements and strengthen the user’s sense of embodiment, hand redirection techniques often incorporate a complete virtual arm or full body alongside the redirected virtual hand, using inverse kinematics~\cite{hartfill2021analysis, ponton2024stretch} or adjustments to the virtual arm's movement as well~\cite{li2022modeling, feick2024impact}.
% }

% noticeability, motivation of predicting a probability, not a classification
However, these redirection techniques are most effective when the manipulation remains undetected~\cite{gonzalez2017model, li2022modeling}. 
If the redirection becomes too large, the user may not mitigate the conflict between the visual sensory input (redirected virtual movement) and their proprioception (actual physical movement), potentially leading to a loss of embodiment with the virtual avatar and making it difficult for the user to accurately control virtual movements to complete interaction tasks~\cite{li2022modeling, wentzel2020improving, feuchtner2018ownershift}. 
While proprioception is not absolute, users only have a general sense of their physical movements and the likelihood that they notice the redirection is probabilistic. 
This probability of detecting the redirection is referred to as \textbf{noticeability}~\cite{li2022modeling, zenner2024beyond, zenner2023detectability} and is typically estimated based on the frequency with which users detect the manipulation across multiple trials.

% version B
% Prior research has explored factors influencing the noticeability of redirected motion, including the redirection's magnitude~\cite{wentzel2020improving, poupyrev1996go}, direction~\cite{li2022modeling, feuchtner2018ownershift}, and the visual characteristics of the virtual avatar~\cite{ogawa2020effect, feick2024impact}.
% While these factors focus on the avatars, the surrounding virtual environment can also influence the users' behavior and in turn affect the noticeability of redirection.
% One such prominent external influence is through the visual channel - the users' visual attention is constantly distracted by complex visual effects and events in practical VR scenarios.
% Although some prior studies have explored how to leverage user blindness caused by visual distractions to redirect users' virtual hand~\cite{zenner2023detectability}, there remains a gap in understanding how to quantify the noticeability of redirection under visual distractions.

% visual stimuli and gaze behavior
Prior research has explored factors influencing the noticeability of redirected motion, including the redirection's magnitude~\cite{wentzel2020improving, poupyrev1996go}, direction~\cite{li2022modeling, feuchtner2018ownershift}, and the visual characteristics of the virtual avatar~\cite{ogawa2020effect, feick2024impact}.
While these factors focus on the avatars, the surrounding virtual environment can also influence the users' behavior and in turn affect the noticeability of redirection.
This, however, remains underexplored.
One such prominent external influence is through the visual channel - the users' visual attention is constantly distracted by complex visual effects and events in practical VR scenarios.
We thus want to investigate how \textbf{visual stimuli in the virtual environment} affect the noticeability of redirection.
With this, we hope to complement existing works that focus on avatars by incorporating environmental visual influences to enable more accurate control over the noticeability of redirected motions in practical VR scenarios.
% However, in realistic VR applications, the virtual environment often contains complex visual effects beyond the virtual avatar itself. 
% We argue that these visual effects can \textbf{distract users’ visual attention and thus affect the noticeability of redirection offsets}, while current research has yet taken into account.
% For instance, in a VR boxing scenario, a user’s visual attention is likely focused on their opponent rather than on their virtual body, leading to a lower noticeability of redirection offsets on their virtual movements. 
% Conversely, when reaching for an object in the center of their field of view, the user’s attention is more concentrated on the virtual hand’s movement and position to ensure successful interaction, resulting in a higher noticeability of offsets.

Since each visual event is a complex choreography of many underlying factors (type of visual effect, location, duration, etc.), it is extremely difficult to quantify or parameterize visual stimuli.
Furthermore, individuals respond differently to even the same visual events.
Prior neuroscience studies revealed that factors like age, gender, and personality can influence how quickly someone reacts to visual events~\cite{gillon2024responses, gale1997human}. 
Therefore, aiming to model visual stimuli in a way that is generalizable and applicable to different stimuli and users, we propose to use users' \textbf{gaze behavior} as an indicator of how they respond to visual stimuli.
In this paper, we used various gaze behaviors, including gaze location, saccades~\cite{krejtz2018eye}, fixations~\cite{perkhofer2019using}, and the Index of Pupil Activity (IPA)~\cite{duchowski2018index}.
These behaviors indicate both where users are looking and their cognitive activity, as looking at something does not necessarily mean they are attending to it.
Our goal is to investigate how these gaze behaviors stimulated by various visual stimuli relate to the noticeability of redirection.
With this, we contribute a model that allows designers and content creators to adjust the redirection in real-time responding to dynamic visual events in VR.

To achieve this, we conducted user studies to collect users' noticeability of redirection under various visual stimuli.
To simulate realistic VR scenarios, we adopted a dual-task design in which the participants performed redirected movements while monitoring the visual stimuli.
Specifically, participants' primary task was to report if they noticed an offset between the avatar's movement and their own, while their secondary task was to monitor and report the visual stimuli.
As realistic virtual environments often contain complex visual effects, we started with simple and controlled visual stimulus to manage the influencing factors.

% first user study, confirmation study
% collect data under no visual stimuli, different basic visual stimuli
We first conducted a confirmation study (N=16) to test whether applying visual stimuli (opacity-based) actually affects their noticeability of redirection. 
The results showed that participants were significantly less likely to detect the redirection when visual stimuli was presented $(F_{(1,15)}=5.90,~p=0.03)$.
Furthermore, by analyzing the collected gaze data, results revealed a correlation between the proposed gaze behaviors and the noticeability results $(r=-0.43)$, confirming that the gaze behaviors could be leveraged to compute the noticeability.

% data collection study
We then conducted a data collection study to obtain more accurate noticeability results through repeated measurements to better model the relationship between visual stimuli-triggered gaze behaviors and noticeability of redirection.
With the collected data, we analyzed various numerical features from the gaze behaviors to identify the most effective ones. 
We tested combinations of these features to determine the most effective one for predicting noticeability under visual stimuli.
Using the selected features, our regression model achieved a mean squared error (MSE) of 0.011 through leave-one-user-out cross-validation. 
Furthermore, we developed both a binary and a three-class classification model to categorize noticeability, which achieved an accuracy of 91.74\% and 85.62\%, respectively.

% evaluation study
To evaluate the generalizability of the regression model, we conducted an evaluation study (N=24) to test whether the model could accurately predict noticeability with new visual stimuli (color- and scale-based animations).
Specifically, we evaluated whether the model's predictions aligned with participants' responses under these unseen stimuli.
The results showed that our model accurately estimated the noticeability, achieving mean squared errors (MSE) of 0.014 and 0.012 for the color- and scale-based visual stimili, respectively, compared to participants' responses.
Since the tested visual stimuli data were not included in the training, the results suggested that the extracted gaze behavior features capture a generalizable pattern and can effectively indicate the corresponding impact on the noticeability of redirection.

% application
Based on our model, we implemented an adaptive redirection technique and demonstrated it through two applications: adaptive VR action game and opportunistic rendering.
We conducted a proof-of-concept user study (N=8) to compare our adaptive redirection technique with a static redirection, evaluating the usability and benefits of our adaptive redirection technique.
The results indicated that participants experienced less physical demand and stronger sense of embodiment and agency when using the adaptive redirection technique. 
These results demonstrated the effectiveness and usability of our model.

In summary, we make the following contributions.
% 
\begin{itemize}
    \item 
    We propose to use users' gaze behavior as a medium to quantify how visual stimuli influences the noticebility of redirection. 
    Through two user studies, we confirm that visual stimuli significantly influences noticeability and identify key gaze behavior features that are closely related to this impact.
    \item 
    We build a regression model that takes the user's gaze behavioral data as input, then computes the noticeability of redirection.
    Through an evaluation study, we verify that our model can estimate the noticeability with new participants under unseen visual stimuli.
    These findings suggest that the extracted gaze behavior features effectively capture the influence of visual stimuli on noticeability and can generalize across different users and visual stimuli.
    \item 
    We develop an adaptive redirection technique based on our regression model and implement two applications with it.
    With a proof-of-concept study, we demonstrate the effectiveness and potential usability of our regression model on real-world use cases.

\end{itemize}

% \delete{
% Virtual Reality (VR) allows the user to embody a virtual avatar by mirroring their physical movements through the avatar.
% As the user's visual access to the physical world is blocked in tasks involving motion control, they heavily rely on the visual representation of the avatar's motions to guide their proprioception.
% Similar to real-world experiences, the user is able to resolve conflicts between different sensory inputs (e.g., vision and motor control) through multisensory integration, which is essential for mitigating the sensory noise that commonly arises.
% However, it also enables unique manipulations in VR, as the system can intentionally modify the avatar's movements in relation to the user's motions to achieve specific functional outcomes,
% for example, 
% % the manipulations on the avatar's movements can 
% enabling novel interaction techniques of redirected walking~\cite{razzaque2005redirected}, redirected reaching~\cite{gonzalez2022model}, and pseudo haptics~\cite{samad2019pseudo}.
% With small adjustments to the avatar's movements, the user can maintain their sense of embodiment, due to their ability to resolve the perceptual differences.
% % However, a large mismatch between the user and avatar's movements can result in the user losing their sense of embodiment, due to an inability to resolve the perceptual differences.
% }

% \delete{
% However, multisensory integration can break when the manipulation is so intense that the user is aware of the existence of the motion offset and no longer maintains the sense of embodiment.
% Prior research studied the intensity threshold of the offset applied on the avatar's hand, beyond which the embodiment will break~\cite{li2022modeling}. 
% Studies also investigated the user's sensitivity to the offsets over time~\cite{kohm2022sensitivity}.
% Based on the findings, we argue that one crucial factor that affects to what extent the user notices the offset (i.e., \textit{noticeability}) that remains under-explored is whether the user directs their visual attention towards or away from the virtual avatar.
% Related work (e.g., Mise-unseen~\cite{marwecki2019mise}) has showcased applications where adjustments in the environment can be made in an unnoticeable manner when they happen in the area out of the user's visual field.
% We hypothesize that directing the user's visual attention away from the avatar's body, while still partially keeping the avatar within the user's field-of-view, can reduce the noticeability of the offset.
% Therefore, we conduct two user studies and implement a regression model to systematically investigate this effect.
% }

% \delete{
% In the first user study (N = 16), we test whether drawing the user's visual attention away from their body impacts the possibility of them noticing an offset that we apply to their arm motion in VR.
% We adopt a dual-task design to enable the alteration of the user's visual attention and a yes/no paradigm to measure the noticeability of motion offset. 
% The primary task for the user is to perform an arm motion and report when they perceive an offset between the avatar's virtual arm and their real arm.
% In the secondary task, we randomly render a visual animation of a ball turning from transparent to red and becoming transparent again and ask them to monitor and report when it appears.
% We control the strength of the visual stimuli by changing the duration and location of the animation.
% % By changing the time duration and location of the visual animation, we control the strengths of attraction to the users.
% As a result, we found significant differences in the noticeability of the offsets $(F_{(1,15)}=5.90,~p=0.03)$ between conditions with and without visual stimuli.
% Based on further analysis, we also identified the behavioral patterns of the user's gaze (including pupil dilation, fixations, and saccades) to be correlated with the noticeability results $(r=-0.43)$ and they may potentially serve as indicators of noticeability.
% }

% \delete{
% To further investigate how visual attention influences the noticeability, we conduct a data collection study (N = 12) and build a regression model based on the data.
% The regression model is able to calculate the noticeability of the offset applied on the user's arm under various visual stimuli based on their gaze behaviors.
% Our leave-one-out cross-validation results show that the proposed method was able to achieve a mean-squared error (MSE) of 0.012 in the probability regression task.
% }

% \delete{
% To verify the feasibility and extendability of the regression model, we conduct an evaluation study where we test new visual animations based on adjustments on scale and color and invite 24 new participants to attend the study.
% Results show that the proposed method can accurately estimate the noticeability with an MSE of 0.014 and 0.012 in the conditions of the color- and scale-based visual effects.
% Since these animations were not included in the dataset that the regression model was built on, the study demonstrates that the gaze behavioral features we extracted from the data capture a generalizable pattern of the user's visual attention and can indicate the corresponding impact on the noticeability of the offset.
% }

% \delete{
% Finally, we demonstrate applications that can benefit from the noticeability prediction model, including adaptive motion offsets and opportunistic rendering, considering the user's visual attention. 
% We conclude with discussions of our work's limitations and future research directions.
% }

% \delete{
% In summary, we make the following contributions.
% }
% % 
% \begin{itemize}
%     \item 
%     \delete{
%     We quantify the effects of the user's visual attention directed away by stimuli on their noticeability of an offset applied to the avatar's arm motion with respect to the user's physical arm. 
%     Through two user studies, we identified gaze behavioral features that are indicative of the changes in noticeability.
%     }
%     \item 
%     \delete{We build a regression model that takes the user's gaze behavioral data and the offset applied to the arm motion as input, then computes the probability of the user noticing the offset.
%     Through an evaluation study, we verified that the model needs no information about the source attracting the user's visual attention and can be generalizable in different scenarios.
%     }
%     \item 
%     \delete{We demonstrate two applications that potentially benefit from the regression model, including adaptive motion offsets and opportunistic rendering.
%     }

% \end{itemize}

\begin{comment}
However, users will lose the sense of embodiment to the virtual avatars if they notice the offset between the virtual and physical movements.
To address this, researchers have been exploring the noticing threshold of offsets with various magnitudes and proposing various redirection techniques that maintain the sense of embodiment~\cite{}.

However, when users embody virtual avatars to explore virtual environments, they encounter various visual effects and content that can attract their attention~\cite{}.
During this, the user may notice an offset when he observes the virtual movement carefully while ignoring it when the virtual contents attract his attention from the movements.
Therefore, static offset thresholds are not appropriate in dynamic scenarios.

Past research has proposed dynamic mapping techniques that adapted to users' state, such as hand moving speed~\cite{frees2007prism} or ergonomically comfortable poses~\cite{montano2017erg}, but not considering the influence of virtual content.
More specifically, PRISM~\cite{frees2007prism} proposed adjusting the C/D ratio with a non-linear mapping according to users' hand moving speed, but it might not be optimal for various virtual scenarios.
While Erg-O~\cite{montano2017erg} redirected users' virtual hands according to the virtual target's relative position to reduce physical fatigue, neglecting the change of virtual environments. 

Therefore, how to design redirection techniques in various scenarios with different visual attractions remains unknown.
To address this, we investigate how visual attention affects the noticing probability of movement offsets.
Based on our experiments, we implement a computational model that automatically computes the noticing probability of offsets under certain visual attractions.
VR application designers and developers can easily leverage our model to design redirection techniques maintaining the sense of embodiment adapt to the user's visual attention.
We implement a dynamic redirection technique with our model and demonstrate that it effectively reduces the target reaching time without reducing the sense of embodiment compared to static redirection techniques.

% Need to be refined
This paper offers the following contributions.
\begin{itemize}
    \item We investigate how visual attractions affect the noticing probability of redirection offsets.
    \item We construct a computational model to predict the noticing probability of an offset with a given visual background.
    \item We implement a dynamic redirection technique adapting to the visual background. We evaluate the technique and develop three applications to demonstrate the benefits. 
\end{itemize}



First, we conducted a controlled experiment to understand how users perceived the movement offset while subjected to various distractions.
Since hand redirection is one of the most frequently used redirections in VR interactions, we focused on the dynamic arm movements and manually added angular offsets to the' elbow joint~\cite{li2022modeling, gonzalez2022model, zenner2019estimating}. 
We employed flashing spheres in the user's field of view as distractions to attract users' visual attention.
Participants were instructed to report the appearing location of the spheres while simultaneously performing the arm movements and reporting if they perceived an offset during the movement. 
(\zhipeng{Add the results of data collection. Analyze the influence of the distance between the gaze map and the offset.}
We measured the visual attraction's magnitude with the gaze distribution on it.
Results showed that stronger distractions made it harder for users to notice the offset.)
\zhipeng{Need to rewrite. Not sure to use gaze distribution or a metric obtained from the visual content.}
Secondly, we constructed a computational model to predict the noticing probability of offsets with given visual content.
We analyzed the data from the user studies to measure the influence of visual attractions on the noticing probability of offsets.
We built a statistical model to predict the offset's noticing probability with a given visual content.
Based on the model, we implement a dynamic redirection technique to adjust the redirection offset adapted to the user's current field of view.
We evaluated the technique in a target selection task compared to no hand redirection and static hand redirection.
\zhipeng{Add the results of the evaluation.}
Results showed that the dynamic hand redirection technique significantly reduced the target selection time with similar accuracy and a comparable sense of embodiment.
Finally, we implemented three applications to demonstrate the potential benefits of the visual attention adapted dynamic redirection technique.
\end{comment}

% This one modifies arm length, not redirection
% \citeauthor{mcintosh2020iteratively} proposed an adaptation method to iteratively change the virtual avatar arm's length based on the primary tasks' performance~\cite{mcintosh2020iteratively}.



% \zhipeng{TO ADD: what is redirection}
% Redirection enables novel interactions in Virtual Reality, including redirected walking, haptic redirection, and pseudo haptics by introducing an offset to users' movement.
% \zhipeng{TO ADD: extend this sentence}
% The price of this is that users' immersiveness and embodiment in VR can be compromised when they notice the offset and perceive the virtual movement not as theirs~\cite{}.
% \zhipeng{TO ADD: extend this sentence, elaborate how the virtual environment attracts users' attention}
% Meanwhile, the visual content in the virtual environment is abundant and consistently captures users' attention, making it harder to notice the offset~\cite{}.
% While previous studies explored the noticing threshold of the offsets and optimized the redirection techniques to maintain the sense of embodiment~\cite{}, the influence of visual content on the probability of perceiving offsets remains unknown.  
% Therefore, we propose to investigate how users perceive the redirection offset when they are facing various visual attractions.


% We conducted a user study to understand how users notice the shift with visual attractions.
% We used a color-changing ball to attract the user's attention while instructing users to perform different poses with their arms and observe it meanwhile.
% \zhipeng{(Which one should be the primary task? Observe the ball should be the primary one, but if the primary task is too simple, users might allocate more attention on the secondary task and this makes the secondary task primary.)}
% \zhipeng{(We need a good and reasonable dual-task design in which users care about both their pose and the visual content, at least in the evaluation study. And we need to be able to control the visual content's magnitude and saliency maybe?)}
% We controlled the shift magnitude and direction, the user's pose, the ball's size, and the color range.
% We set the ball's color-changing interval as the independent factor.
% We collect the user's response to each shift and the color-changing times.
% Based on the collected data, we constructed a statistical model to describe the influence of visual attraction on the noticing probability.
% \zhipeng{(Are we actually controlling the attention allocation? How do we measure the attracting effect? We need uniform metrics, otherwise it is also hard for others to use our knowledge.)}
% \zhipeng{(Try to use eye gaze? The eye gaze distribution in the last five seconds to decide the attention allocation? Basically constructing a model with eye gaze distribution and noticing probability. But the user's head is moving, so the eye gaze distribution is not aligned well with the current view.)}

% \zhipeng{Saliency and EMD}
% \zhipeng{Gaze is more than just a point: Rethinking visual attention
% analysis using peripheral vision-based gaze mapping}

% Evaluation study(ideal case): based on the visual content, adjusting the redirection magnitude dynamically.

% \zhipeng{(The risk is our model's effect is trivial.)}

% Applications:
% Playing Lego while watching demo videos, we can accelerate the reaching process of bricks, and forbid the redirection during the manipulation.

% Beat saber again: but not make a lot of sense? Difficult game has complicated visual effects, while allows larger shift, but do not need large shift with high difficulty



\section{Background and Related Work}\label{sec:related}

\paragraph{\textbf{Privacy of Human-Centered Systems}}
Ensuring privacy in human-centric ML-based systems presents inherent conflicts among service utility, cost, and personal and institutional privacy~\cite{sztipanovits2019science}. Without appropriate incentives for societal information sharing, we may face decision-making policies that are either overly restrictive or that compromise private information, leading to adverse selection~\cite{jin2016enabling}. Such compromises can result in privacy violations, exacerbating societal concerns regarding the impact of emerging technology trends in human-centric systems~\cite{mulligan2016privacy,fox2021exploring,goldfarb2012shifts}. Consequently, several studies have aimed to establish privacy guarantees that allow auditing and quantifying compromises to make these systems more acceptable~\cite{jagielski2020auditing, raji2020saving}. ML models in decision-making systems have also been shown to leak significant amounts of private information that requires auditing platforms~\cite{hamon2022bridging}. Various studies focused on privacy-preserving machine learning techniques targeting decision-making systems~\cite{abadi2016deep, cummings2019compatibility, taherisadr2023adaparl, taherisadr2024hilt}. Recognizing that perfect privacy is often unattainable, this paper examines privacy from an equity perspective. We investigate how to ensure a fair distribution of harm when privacy leaks occur, addressing the technical challenges alongside the ethical imperatives of equitable privacy protection.


\paragraph{\textbf{\acf{fl}}}
\ac{fl} is an approach in machine learning that enables the collaborative training of models across multiple devices or institutions without requiring data to be centralized. This decentralized setup is particularly beneficial in fields where data-sharing restrictions are enforced by privacy regulations, such as healthcare and finance. \ac{fl} allows organizations to derive insights from data distributed across various locations while adhering to legal constraints, including the General Data Protection Regulation (GDPR) \cite{BG_Survey2,BG_Survey1}.

One of the most widely adopted methods in \ac{fl} is \ac{fedavg}, which operates through iterative rounds of communication between a central server and participating clients to collaboratively train a shared model. During each communication round, the server sends the current global model to each client, which uses their locally stored data to perform optimization steps. These optimized models are subsequently sent back to the server, where they are aggregated to update the global model. The process repeats until the model converges. Known for its simplicity and effectiveness, \ac{fedavg} serves as the primary technique for coordinating model updates across distributed clients in our work. Additionally, we specifically employ horizontal federated learning, where data is distributed across entities with similar feature spaces but distinct user groups \cite{BG_HorizontalFL}.

\paragraph{\textbf{Privacy Risks in \ac{fl}}}
Privacy risks are a critical concern in \ac{fl}, as collaborative training on decentralized data can inadvertently expose sensitive information. A primary threat is the \ac{mia}, where adversaries determine whether specific data records were part of the model's training set \cite{shokri2017membership,BG_MIA}. Researchers have since demonstrated \ac{mia}'s effectiveness across various machine learning models, including \ac{fl}, showing, for example, that adversaries can infer if a specific location profile contributed to an FL model \cite{BG_MIA_1,BG_MIA_2}. However, while \ac{mia} identifies training members, it does not reveal the client that contributed the data. \ac{sia}, introduced in \cite{BG_SIA_2}, extends \ac{mia} by identifying which client owns a training record, thus posing significant security risks by exposing client-specific information in \ac{fl} settings.

The \ac{noniid} nature of data in federated learning presents additional privacy challenges, as variations in data distributions across clients heighten the risk of privacy leakage. When data distributions differ widely among clients, individual model updates become more distinguishable, potentially allowing attackers to infer sensitive information \cite{BG_NON_IID}. This distinctiveness in updates can make federated models more susceptible to inference attacks, such as \ac{mia} and \ac{sia}, as malicious actors may exploit these distributional differences to trace updates back to specific clients. This vulnerability is especially relevant in our work, as we use the \ac{har} dataset, which is inherently \ac{noniid} across clients, thus posing an increased risk for privacy leakage.




\paragraph{\textbf{Fairness in \ac{fl}}}
Fairness in \ac{fl} is crucial due to the varied data distributions among clients, which can lead to biased outcomes favoring certain groups \cite{BG_Fairness_2}. Achieving fairness involves balancing the global model's benefits across clients despite the decentralized nature of the data. Approaches include group fairness, ensuring performance equity across client groups, and performance distribution fairness, which focuses on fair accuracy distribution~\cite{selialia2024mitigating}. Additional types are selection fairness (equitable client participation), contribution fairness (rewards based on contributions), and expectation fairness (aligning performance with client expectations) \cite{BG_Fairness}. Achieving fairness in \ac{fl} across these various dimensions remains challenging due to the inherent heterogeneity of client data and environments. In response to this heterogeneity, personalization has emerged as a strategy to tailor models to individual clients, enhancing local performance~\cite{BG_Personalization,BG_Personalization_2, BG_FairnessPrivacy}.   

When considering fairness in FL, it is crucial to address the interplay with privacy. Specifically, ensuring an equitable distribution of privacy risks across clients is paramount, preventing any group from being disproportionately vulnerable to privacy leakage, particularly under attacks such as source inference attacks (SIAs).



\section{Problem Statement}\label{sec:threatmodel}

 


Federated learning (FL) systems face significant privacy risks from malicious servers. Even an "honest-but-curious" server, while adhering to the FL protocol, can attempt to infer sensitive client information by analyzing aggregated model updates, potentially revealing private data points, patterns, or client identities. A key privacy threat is a two-stage attack: Membership Inference Attack (MIA) followed by Source Inference Attack (SIA).

\begin{itemize}[noitemsep, topsep=0pt]
    \item \textbf{MIA:} The server determines if a specific data point $x$ was used to train the global model $\theta_g$: MIA($\theta_g$, $x$) = P($x \in D_{\theta_g}$), where P($x \in D_{\theta_g}$) is the probability that $x$ belongs to the training data $D_{\theta_g}$.
    \item \textbf{SIA:} If the MIA suggests $x$ was part of the training data, the server identifies the contributing client $i$: SIA($\theta_i$, $x$) = P(Client$_i$ | $x$, $\theta_i$), where P(Client$_i$ | $x$, $\theta_i$) is the probability that client $i$ contributed $x$ to the model $\theta_i$.
\end{itemize}

As shown by Hongsheng et al. \cite{BG_SIA_2}, combining these attacks can severely compromise client privacy. Moreover, prior work has shown the inherent limitations of auditing MIA \cite{chang2024efficient}. 

Our work focuses on the disparity in privacy risk across clients, which we attribute to differences in local overfitting during training. This threat model underscores the need for equitable privacy mechanisms in FL. 

Given this threat model %in Figure \ref{fig:threatmodel}, 
where a compromised server enables SIA attacks, our objective is twofold:

\begin{enumerate}[noitemsep, topsep=0pt, start=1,label={(\bfseries O\arabic*):}]
    \item Addressing the symptoms: Develop an aggregation method on the server side to ensure fair privacy risk distribution among clients.
    \item Addressing the causes: Provide feedback to leaking clients, enabling them to adjust local updates to reduce overfitting and improve system fairness in privacy.
\end{enumerate}

Instead of eliminating SIA, we aim to mitigate its impact by equitably distributing the inherent privacy risk. We therefore assume a \textit{compromised} server and cooperative clients capable of tuning their local updates to enhance \sysname.


\section{Fairness-in-Privacy Framework in Federated Learning}\label{sec:finpmetric}

\begin{figure*}[!t]
\centering
\includegraphics[trim={0 9cm 3cm 0},clip,width=0.8\linewidth]{fig/framework.pdf}
\caption{Fairness in Privacy \sysname\ framework in federated learning. The framework addresses the causes and the symptoms to achieve \sysname.}
\label{fig:framework}
\end{figure*}


This section presents our framework, \sysname, designed to improve fairness in privacy within federated learning (FL), particularly in the context of Source Inference Attacks (SIAs). Our core principle is that privacy risks should be distributed equitably among all participating clients, preventing any single client from bearing a disproportionate burden. 
 

This disparity in privacy risks among clients can arise from various factors, including heterogeneous data distributions, varying computational resources, and differences in local training dynamics. Simply preventing average privacy leakage is insufficient; we must ensure that no individual client bears a disproportionate risk. This motivates our focus on fairness-in-privacy, which aims to equitably distribute privacy risks across all participating clients.

An overview of the \sysname\ framework is shown in~\autoref{fig:framework}. We argue that addressing fairness in privacy requires a two-pronged approach: handling it both at the server (during aggregation) and at the client (during local training). Server-side interventions, specifically adaptive aggregation, are crucial to mitigating the impact of existing disparities in privacy leakage. By carefully weighing client updates based on their estimated privacy risk, we can prevent highly vulnerable clients from unduly influencing the global model and further exacerbating the unfairness. However, server-side interventions alone are insufficient. They address the *symptoms* of unfairness but not the underlying *causes*.

The root cause of privacy disparity often lies in differences in local training dynamics, particularly local overfitting. When a client's model overfits its local data, it becomes more susceptible to privacy attacks, such as Source Inference Attacks (SIAs). Therefore, we also address fairness in privacy on the client side by introducing a collaborative overfitting reduction strategy. This strategy aims to proactively reduce the likelihood of local overfitting, thereby minimizing the initial disparity in privacy risks before aggregation. By ranking clients based on their estimated relative overfitting and incorporating this rank into a local regularization scheme, we encourage clients to learn more generalizable representations, reducing their vulnerability to the disparity in privacy leakage.

This two-pronged approach, combining adaptive aggregation at the server and collaborative overfitting reduction at the client, provides a comprehensive framework for achieving fairness in privacy in FL. By minimizing both the symptoms and the root causes of privacy disparity, our aim is to create a more equitable and robust FL system. This can be formalized in \autoref{eq:finp}. 

\begin{align}\label{eq:finp}
\text{F}in\text{P} &= \min (\text{Symptoms}, \text{Causes}) \nonumber \\ 
&= \text{F}in\text{P}_\text{server} + \text{F}in\text{P}_\text{client}
\end{align}


\subsection{Formalizing Symptoms of Fairness in Privacy on Server Side}

We formalize the fairness in privacy problem as follows: Given an FL system with $K$ clients and a global model $\theta_g$, our goal is to achieve fair privacy risk across all clients against successful SIAs.

We consider the privacy risk $p_k(\mathbf{w})$ for client $k$ to be influenced by the aggregation weights $\mathbf{w} = [w_1, w_2, ..., w_K]$, where $w_k$ represents the weight assigned for the client $k$, with the constraint $\sum_{k=1}^{K} a_k = 1$. This allows us to account for the varying client contributions to the global model.

We define Fairness in Privacy (F$in$P) as minimizing the variance in privacy risks across clients. Our objective is to find the optimal weights for aggregation $\mathbf{w}$ that minimize the difference between individual client privacy risks and the average privacy risk. This is expressed in \autoref{eq:finpserver} as:


\begin{align}\label{eq:finpserver}
    \text{F}in\text{P}_\text{server} = \min_{\mathbf{w}\in \mathcal{W}} \| \mathbf{p}(\mathbf{w}) - \frac{1}{K} \mathds{1}^T \mathbf{p}(\mathbf{w}) \otimes \mathds{1}  \| + \|\frac{1}{K} \mathds{1}^T \mathbf{p}(\mathbf{w})\|,
    %\vspace{-2mm}
\end{align}

Where:

\begin{itemize}
    \item $\mathbf{p}(\mathbf{w}) = [p_1(\mathbf{w}),\dots, p_K(\mathbf{w})]^T$ is the vector of privacy risks for all clients given the aggregation weights $\mathbf{w}$.
    \item $\mathds{1}$ is a vector of ones of length $K$.
    \item $\mathcal{W} = \{\mathbf{w} \in \mathbb{R}^K \mid \sum_{k=1}^{K} w_k = 1, w_k \geq 0 \ \forall k\}$ is the set of valid aggregation weights.
\end{itemize}

The term $\frac{1}{K} \mathds{1}^T \mathbf{p}(\mathbf{w})$ represents the average privacy risk. Equation \eqref{eq:finpserver} minimizes the Euclidean distance between individual privacy risks and this average, thus minimizing the disparity in privacy risks. Intuitively, we seek optimal aggregation weights to achieve a more equitable distribution of privacy risk, ensuring no client is disproportionately exposed.


We hypothesize that differences in local overfitting are a primary cause of unequal privacy leakage among FL clients. When a client's model overfits to its local data, it effectively memorizes sensitive information, making it more vulnerable to SIAs and leading to an unfair distribution of privacy risk.

We quantify the *symptoms* of overfitting in the server by measuring the discrepancy between each client's local model update and the global model using the Principle Component Analysis (PCA) distance \cite{durmus2021federated}. For client $k$, this distance, denoted as $p_k$, serves as a proxy for privacy risk; a larger $p_k$ signifies a symptom of greater overfitting and, thus, higher risk.

Our proposed adaptive aggregation method aims to balance client contributions based on these PCA distances. By minimizing the variance of $p_k$ using the \sysname$_{server}$ objective (\autoref{eq:finpserver}), we reduce the influence of clients exhibiting high overfitting (high $p_k$) and increase the influence of those with lower overfitting. This dynamic adjustment, performed in each FL round, promotes a more equitable distribution of privacy risk. However, adaptive aggregation alone is insufficient to eliminate overfitting entirely and achieve full fairness in privacy; collaborative client-side adjustments are also required, as will be explained in \autoref{sec:client_side_adjustments}.




\subsection{Formalizing Causes of Fairness in Privacy on Client Side}\label{sec:client_side_adjustments}

To further mitigate local overfitting (*causes*) and enhance fairness in privacy, we propose a collaborative client strategy. This leverages the principle that clients with higher overfitting benefit from more diverse data.

The top Hessian eigenvalue ($\lambda_{\text{max}}$) and Hessian trace ($H_{T}$) have been identified as important metrics for characterizing the loss landscape and generalization capabilities of neural networks~\cite{Jiang2020Fantastic}. Lower values of $\lambda_{\text{max}}$ and $H_{T}$ typically indicate improved robustness to weight perturbations, leading to smoother training and better convergence. This is especially critical in FL, where the non-IID nature of data across clients creates distributional shifts that can exacerbate training instability and introduce fairness concerns. These distributional shifts can disproportionately impact certain client groups, leading to biased model performance~\cite{mendieta2022local}. 


As we are interested in \sysname, we determine each client's relative overfitting by calculating the average pairwise difference across the top Hessian eigenvalue ($\lambda_{\text{max}}$) and Hessian trace ($H_{T}$):

\begin{align}
\begin{split}
\bar{\Delta}_k &= \frac{1}{K-1} \sum_{j=1, j\neq k}^{K} |\lambda_{\text{max}}^k - \lambda_{\text{max}}^j|, \\    \bar{H}_k &= \frac{1}{K-1} \sum_{j=1, j\neq k}^{K} |H_T^k - H_T^j|, \\  \rho_k &= \frac{\frac{\bar{\Delta}_k}{\max{\bar{\Delta} }} + \frac{\bar{H}_k}{\max{\bar{H} }} }{2},
\end{split} \label{eq:hessian}
\end{align}

where $\lambda_{\text{max}}^k$ and $\lambda_{\text{max}}^j$ are the top Hessian eigenvalue of the local models of clients $k$ and $j$, respectively.  Similarly, $H_T^k$, and $H_T^j$ are the Hessian trace of the local models of clients $k$ and $j$, respectively. We used the normalized average of both $\bar{\Delta}_k$ and $\bar{H_k}$ to quantify the client's \textit{overfitting relative rank} ($\rho_k$), to serve as a proxy for relative privacy leakage risk. Computing the Hessian eigenvalue and trace are done on the cloud server, and hence, there is no overhead of their computation on the client.


We incorporate this overfitting rank into the local training process using a regularization term based on the Lipschitz constant, approximated by the spectral norm of the Jacobian matrix ($||J_k||$)~\cite{liu2020simple}. In particular, a smaller Lipschitz constant implies smoother functions, less prone to overfitting, and better generalization. The modified local loss function for client $k$ is:

\begin{align}\label{eq:finpclient}
\mathcal{L}_k' &= \mathcal{L}_k + \beta \cdot \rho_k \cdot ||J_k||, \nonumber \\
\text{F}in\text{P}_\text{client} &= \min_{\theta_k} {\mathcal{L}_k'}
\end{align}

where:

\begin{itemize}
    \item $\mathcal{L}_k$ is the original local loss function.
    \item $\rho_k$ is an adaptive controlling regularization strength that depends on the overfitting rank.
    \item $\theta_k$ are the local parameters of the client model that minimize the total loss $\mathcal{L}'_k$
    \item  $\beta$ is the impact factor, which controls the impact of the Lipschitz constant based on the learning task.
\end{itemize}

This penalizes models with large Lipschitz constants, promoting generalization. The regularization strength is weighted by $\rho_k$ adaptively at each round, applying stronger regularization to clients with higher overfitting ranks. This collaborative approach, using $\rho_k$ to guide local training, preserves privacy while promoting equitable learning and reducing disparity in privacy risk. 

$\beta$ is a task-dependent parameter to balance the the loss $\mathcal{L}_k$ and Lipschitz loss. A larger $\beta$ greatly impacts fairness regularization but could make training unstable and fail to converge. $\beta$ is a trade-off parameter between fairness and accuracy while $\rho_k$ changes at every round to control regularization strength adaptively.











\section{Evaluation}\label{sec:eval}



\subsection{Federated Learning System Setup}

\paragraph{\textbf{Setup for Human Activity Recognition}} We utilized the UCI \ac{har} Dataset \cite{human_activity_recognition_using_smartphones_240}, a widely used dataset in activity recognition research, especially in FL~\cite{har2,har3}.
The dataset includes sensor data from 30 subjects (aged 19–48) performing six activities: walking, walking upstairs, walking downstairs, sitting, standing, and laying. The data was collected using a Samsung Galaxy S II smartphone worn on the waist, capturing readings from both the accelerometer and gyroscope sensors. Each subject in the dataset was treated as an individual client in the \ac{fl} setup, preserving the data's unique activity patterns and non-IID nature. We allocated 70\% of each client's data for training using 5-fold cross-validation and 30\% for testing, enabling evaluation of the model on independently collected test data. Data preprocessing involved applying noise filters to the raw signals and segmenting the data using a sliding window approach with a window length of 2.56 seconds and a 50\% overlap, resulting in 128 readings per window. We selected the HAR dataset for evaluation \sysname due to its inherited non-IID structure. 

We trained the model in a federated learning setting using the \ac{fedavg} aggregation method over 20 global communication rounds. Each client trained locally with a batch size of 64, 5 local epochs per round, a learning rate of 0.001 using Adam optimizer, and an impact factor $\beta$ of 1. These parameters ensured balanced model updates from each client while maintaining computational efficiency across the federated network. Each local model (one per subject) analyzes its time-series sensor data using \ac{tcn} model\cite{bai2018empirical}. 
The TCN model, designed for time-series data, uses causal convolutions to capture temporal dependencies while preserving sequence order. The architecture includes two convolutional layers, each followed by max-pooling and dropout, with a final fully connected layer for classifying the six activity classes.



\paragraph{\textbf{Setup for CIFAR-10}}
The CIFAR-10 dataset consists of 60000 32x32 color images in 10 classes, with 6000 images per class. There are 50000 training images and 10000 test images. We use the Dirichlet distribution $Dir(\alpha)$ to divide the CIFAR-10 dataset into $K$ unbalanced subsets similar to previous work in the literature ~\cite{mendieta2022local,BG_SIA_2}, with $\alpha=0.1$. \autoref{fig:CIFAR data} shows how the data are distributed among clients. We created 10 clients and employed ResNet56~\cite{he2016deep} as the model. Similar to the setup in HAR, we trained the model over 20 global communication rounds. Each client is trained locally with the same parameters in HAR and an impact factor $\beta$ of 0.05. A smaller $\beta$ is used here since CIFAR-10 is a more complicated task than HAR, and a smaller $\beta$ can make the model converge easier since the model is more sensitive to classification loss.



\paragraph{\textbf{SIA Attack}} We used the Source Inference Attacks (SIA) setup explained in ~\cite{hu2023source}, where we randomly sampled training data from each client dataset. We combined those samples in one dataset and used them as target records. This is a valid assumption, given an already successful Membership Inference Attacks (MIA) attack. SIA attacks in Federated Learning represent a privacy threat beyond Membership Inference Attacks (MIA). While MIAs determine whether a data instance was used for training, SIA aims to identify the specific client who owns that training record. In a practical scenario, an adversary, such as an honest-but-curious central server who knows the clients' identities and receives their model updates, could leverage this knowledge to trace training data back to its source, thus compromising client privacy. To launch SIA in FL setting,  clients send their updated local model parameters to the server. The server uses each client's model to calculate the prediction loss on the target record. The client with the smallest loss is identified as the most probable source of that target record. This approach exploits the differences in model performance on the target record to infer its origin.


\begin{figure*}
\centering
\includegraphics[width=0.8\linewidth]{fig/CIFAR/Fedalign/Data_profile.png}
\caption{CIFAR dataset profile for each client after Dirichlet sampling with $\alpha=0.1$}
\label{fig:CIFAR data}
\end{figure*}



\subsection{Metrics for Comparison}\label{sec:metrics}
Recent work in the literature suggests SIA vulnerability wherein an adversary can potentially identify the origin of a specific record can be achieved by analyzing the prediction loss of individual client models~\cite{hu2023source}. In particular, SIA exploits the observation that the client model %$C_{min}$ 
exhibiting the lowest prediction loss for a given record is most likely to be the source of that record. We assess our \sysname approach in achieving fairness in privacy using the following metrics:

 
\paragraph{\textbf{1- Reduction of SIA accuracy disparity among clients} } \sysname\  aims to reduce the SIA accuracy disparity among clients. A balanced SIA accuracy across clients indicates a more equitable distribution of privacy risk within the FL system. \sysname enhances the overall fairness in privacy by ensuring that no particular client is significantly more vulnerable to source inference attacks than others. 

To assess the fairness of risk of SIA accuracy across clients in our FL system, we employ the Coefficient of Variation (CoV). Recognizing that fairness is related to the variance of shared utility rather than strict equality \cite{jain1984quantitative}, we adapt the CoV to measure the dispersion of SIA accuracy among clients.

For K clients, we define the SIA accuracy for client $i$ as $\text{SIA}_i$. The mean SIA accuracy ($\mu$) is calculated as
$\mu = \frac{1}{K} \sum_{i=1}^{K} \text{SIA}_i$. The CoV of SIA accuracy \texttt{CoV(SIA)} is then computed as:

\begin{align}\label{eq:covsia}
\texttt{CoV(SIA)} = \frac{\sigma}{\mu} = \frac{\sqrt{\frac{1}{K} \sum_{i=1}^{K} (\text{SIA}_i - \mu)^2}}{\mu},
\end{align}

where $\sigma$ is the standard deviation of SIA accuracies. A lower CoV indicates a more equitable distribution of SIA accuracy across clients, suggesting greater fairness in privacy. To facilitate interpretation as a fairness percentage between $0$ and $1$ (where 1 represents perfect fairness), we use the following Fairness Index (\texttt{FI(SIA)}) transformation:

\begin{align}\label{eq:fisia}
\texttt{FI(SIA)} = \frac{1}{1 + \texttt{CoV(SIA)}} 
\end{align}

A FI value of $1$ indicates perfect fairness (all clients have the same SIA accuracy), while lower FI values indicate increasing disparities in SIA accuracy among clients.









\paragraph{\textbf{2- Reduction of SIA confidence disparity among clients}} Beside reduction of SIA accuracy disparity among clients, as the SIA approach relies on identifying the client model with the minimum prediction loss. When a significant discrepancy exists between the prediction losses of different client models, an attacker can make source inferences with higher confidence. \sysname\  aims to reduce inter-client loss differences so that it can diminish the effectiveness of SIA attacks by lowering the attacker's confidence in their inferences. We evaluate the SIA confidence disparity by using  the prediction loss \texttt{CoV(Loss)} and \texttt{FI(Loss)} similarly to \autoref{eq:covsia} and \autoref{eq:fisia}. 



\paragraph{\textbf{3- Success rate of SIA}}
\sysname\ aims to reduce disparities in both SIA success rate and prediction loss across clients. However, simply reducing disparity is insufficient; it is crucial to avoid achieving this by merely increasing the SIA success rate of less vulnerable clients to match that of the most vulnerable ones. Such an outcome would not represent a genuine improvement in privacy. Therefore, we evaluate the overall impact of \sysname\ on SIA vulnerability using two metrics: \texttt{Mean(SIA)} and \texttt{Max(SIA)}. In particular, \texttt{Mean(SIA)} represents the average SIA success rate across all clients and communication rounds, while \texttt{Max(SIA)} indicates the highest SIA success rate observed across all clients and rounds. Lower values for both metrics signify increased resilience against SIA attacks.

\paragraph{\textbf{4- Accuracy Metric}}

Accuracy is calculated over all test dataset points for all clients using the formula:

\begin{equation}
    \text{Accuracy} = \frac{\sum_{i=1}^{N} \mathbf{1}(\hat{y}_i = y_i)}{N}
\end{equation}

where:
\begin{itemize}
    \item $N$ is the total number of test samples across all clients,
    \item $\hat{y}_i$ is the predicted label for the $i$-th test sample,
    \item $y_i$ is the true label of the $i$-th test sample.
\end{itemize}

We evaluated \sysname\ through two distinct case studies, using the Human Activity Recognition (HAR) dataset (\autoref{sec:HAReval}) and the CIFAR-10 image classification dataset (\autoref{sec:CIFAReval}).  For HAR, we compared four approaches: (1) a Baseline Federated Learning (FL) implementation using FedAvg, adapted from \cite{hu2023source}; (2) \sysname$_\text{server}$, which applies adaptive aggregation at the server without client collaboration (\autoref{eq:finpserver}); (3) \sysname$_\text{client}$, which employs client-side collaboration to mitigate relative overfitting but omits adaptive server aggregation (Equation \ref{eq:finpclient}); and (4) the full \sysname\ approach, incorporating both \sysname$_\text{server}$ and \sysname$_\text{client}$ (\autoref{eq:finp}). 

In the CIFAR-10 evaluation, we compared three approaches: (1) the same Baseline FL using FedAvg from \cite{hu2023source}; (2) FedAlign \cite{mendieta2022local}, a state-of-the-art FL method designed to address data heterogeneity in CIFAR-10; and (3) the full \sysname\ approach, again incorporating both \sysname$_\text{server}$ and \sysname$_\text{client}$ (\autoref{eq:finp}).






\subsection{\sysname\ Performance on HAR}\label{sec:HAReval}

\begin{figure}[!t]
  \centering
  \begin{subfigure}[b]{0.99\columnwidth}
    \centering
    \includegraphics[width=0.8\textwidth]{fig/HAR/line_CoV_SIA.png}
    \caption{Coefficient of variation for SIA accuracy \texttt{CoV(SIA)}.}
    \label{fig:SIACV}
  \end{subfigure}
  \hfill
  \begin{subfigure}[b]{0.99\columnwidth}
    \centering
    \includegraphics[width=0.8\textwidth]{fig/HAR/line_FI_SIA.png}
    \caption{Fairness index of SIA accuracy \texttt{FI(SIA)}.}
    \label{fig:SIAFI}
  \end{subfigure}
  \caption{Disparity of SIA accuracy among clients using HAR dataset. }
  \label{fig:SIA}
\end{figure}



\begin{figure}[!t]
  \centering
  \begin{subfigure}[b]{0.99\columnwidth}
    \centering
    \includegraphics[width=0.8\textwidth]{fig/HAR/line_loss_CV.png}
    \caption{Coefficient of variation for the prediction loss \texttt{CoV(Loss)}.}
    \label{fig:lossCV}
  \end{subfigure}
  \hfill
  \begin{subfigure}[b]{0.99\columnwidth}
    \centering
    \includegraphics[width=0.8\textwidth]{fig/HAR/line_loss_FI.png}
    \caption{Fairness index of the prediction loss \texttt{FI(Loss)}.}
    \label{fig:lossFI}
  \end{subfigure}
  \caption{Disparity of prediction loss among clients using HAR dataset. }
  \label{fig:loss}
\end{figure}


\begin{figure}
  \centering
  \begin{subfigure}[b]{0.99\columnwidth}
    \centering
    \includegraphics[width=0.8\textwidth]{fig/HAR/line_average_train_accuracy.png}
    \caption{Global model training accuracy.}
    \label{fig:accCV}
  \end{subfigure}
  \hfill
  \begin{subfigure}[b]{0.99\columnwidth}
    \centering
    \includegraphics[width=0.8\textwidth]{fig/HAR/line_average_test_accuracy.png}
    \caption{Global model testing accuracy.}
    \label{fig:accFI}
  \end{subfigure}
  \caption{Global model classification accuracy using HAR dataset.}
  \label{fig:accu}
\end{figure}






\begin{figure}
  \centering
  \begin{subfigure}[b]{0.99\columnwidth}
    \centering
    \includegraphics[width=0.8\textwidth]{fig/HAR/line_CoV_PCA_d.png}
    \caption{Coefficient of Variation of the PCA distance to the global model (\texttt{PCA$_d$}). }
    \label{fig:har_pca_d_cov}
  \end{subfigure}
  \hfill
  \begin{subfigure}[b]{0.99\columnwidth}
    \centering
    \includegraphics[width=0.8\textwidth]{fig/HAR/line_FI_PCA_d.png}
    \caption{Fairness Index of PCA distance to the global model (\texttt{PCA$_d$}). }
    \label{fig:har_pca_d_fi}
  \end{subfigure}
  \caption{Disparity of PCA distance between the global model and the client models using HAR dataset. }
  \label{fig:pca}
\end{figure}









\subsubsection{\textbf{Impact on the disparity of SIA accuracy among clients}}
Our results demonstrate a significant improvement in fairness with minimal impact on overall performance. \autoref{fig:SIA} presents the Coefficient of Variation of SIA accuracy (\texttt{CoV(SIA)}) and Fairness Index of SIA accuracy (\texttt{FI(SIA)}), as defined in  \autoref{eq:covsia} and \autoref{eq:fisia}, respectively.  \sysname\ achieves a \texttt{CoV(SIA)} of $0.596$ and a \texttt{FI(SIA)} of $0.739$, compared to the Baseline's \texttt{CoV(SIA)} of $0.893$ and \texttt{FI(SIA)} of $0.617$. This represents a substantial reduction of $33.26\%$ in \texttt{CoV(SIA)} and a $19.77\%$ improvement in \texttt{FI(SIA)} compared to baseline, clearly indicating that \sysname\ significantly enhances the fairness of SIA accuracy.






\subsubsection{\textbf{Impact on the disparity of SIA confidence among clients}} 
Similarly, our results demonstrate improvement in fairness with respect to the SIA confidence in prediction among clients represented as \texttt{CoV(Loss)} and \texttt{FI(Loss)} as explained in \autoref{sec:metrics}. As shown in \autoref{fig:loss}, \sysname\ achieves a \texttt{CoV(Loss)} of $0.778$ and \texttt{FI(Loss)} of $62.8\%$. This represents a reduction of $10.95\%$ in \texttt{CoV(Loss)} and a $19.77\%$ improvement in \texttt{FI(Loss)} compared to the Baseline, clearly indicating that \sysname\ enhances the fairness of SIA confidence in prediction among clients.


\subsubsection{\textbf{Impact on SIA success rate}}
While \autoref{tbl:HAR_sia} shows a marginal increase of less than $1\%$ in \texttt{Mean(SIA)} success rate and less than $0.1\%$ in \texttt{Max(SIA)} success rate, these gains are secondary to the primary objective of fairness improvement.  The key achievement of \sysname\ is the demonstrably more equitable distribution of privacy protection.  \sysname\ achieves this by significantly improving the uniformity of SIA success rates and reducing SIA confidence across clients.

Furthermore, \sysname\ maintains competitive classification performance. As shown in \autoref{tbl:HAR_acc}, the global model's testing accuracy only decreases by a $1.02\%$.  This impact on accuracy is further supported by \autoref{fig:accu}, which demonstrates that \sysname\ converges at a comparable rate to the Baseline.  Therefore, \sysname\ effectively balances the critical need for fairness with the practical requirement of maintaining performance.

\subsubsection{Ablation study}




\begin{table}[!t]
\caption{SIA accuracy performance in HAR dataset.}
\label{tab:my-table}
\begin{tabular}{|c|l|l|}
\hline
\multicolumn{1}{|l|}{} & \texttt{Mean(SIA)}(\%) $\downarrow$ & \texttt{Max(SIA)}(\%) $\downarrow$   \\
\hline
Baseline~\cite{hu2023source} & \textbf{23.78} & 31.00   \\
\sysname$_{\text{server}}$ & 25.22 & 31.20 \\
\sysname$_{\text{client}}$ & 25.52 & \textbf{30.20}   \\
\sysname & 24.49 & 31.10 \\
\hline
\end{tabular}
 \label{tbl:HAR_sia}
\end{table}



\begin{table}[!t]
\caption{HAR experiment: global model classification accuracy.}
\label{tab:my-table}
\begin{tabular}{|c|l|l|}
\hline
\multicolumn{1}{|l|}{} &  Training (\%) & Testing (\%)   \\
\hline
Baseline~\cite{hu2023source} & \textbf{96.52} & \textbf{96.94}   \\
\sysname$_{\text{server}}$ & 95.79 & 95.77\\
\sysname$_{\text{client}}$ & 95.67 & 95.99   \\
\sysname & 95.05 & 95.92 \\
\hline
\end{tabular}
 \label{tbl:HAR_acc}
\end{table}








We conducted an ablation study to evaluate the individual contributions of \sysname's server-side and client-side components.  Isolating the server-side adaptive aggregation (\sysname$_\text{server}$) revealed a nuanced impact on fairness metrics.  While \sysname$_\text{server}$ reduced the variation in PCA distance (\texttt{PCA$_d$)} by $1.3\%$ (\autoref{fig:pca}), it also resulted in a slight shift in both \texttt{FI(SIA)} and \texttt{FI(Loss)} by $-0.2\%$ and $-1.3\%$, respectively (Figures \ref{fig:SIAFI} and \ref{fig:lossFI}). This suggests that server-side adaptation alone (\sysname$_\text{server}$) primarily influences the distribution of model distances and has a less direct impact on the fairness metrics themselves. This observation motivated the investigation of client-side factors, specifically the variation in overfitting among clients, to further enhance fairness.

Analysis of Hessian eigenvalues ($\lambda_{\text{max}}$) and trace ($H_{T}$) revealed a strong correlation (Spearman's rank correlation coefficient $\approx$ 1) between these two metrics, both indicative of how well a local model fits its local data (\autoref{fig:subHessian}).  Based on this correlation, these metrics were given equal weight in \autoref{eq:hessian}.  Focusing on mitigating client-side overfitting through \sysname$_\text{client}$ yielded significant improvements in fairness.  Figures \ref{fig:SIA} and \ref{fig:loss} demonstrate the substantial gains in both SIA accuracy and prediction loss fairness.  Specifically, \sysname$_\text{client}$ alone improved \texttt{FI(SIA)} by $16.37\%$ and \texttt{FI(Loss)} by $8.30\%$ (Figures \ref{fig:SIAFI} and \ref{fig:lossFI}).  Furthermore, combining \sysname$_\text{server}$ with \sysname$_\text{client}$ resulted in even greater fairness gains, with an additional $3.13\%$ improvement in \texttt{FI(SIA)} and $2.65\%$ in \texttt{FI(Loss)} compared to using \sysname$_\text{client}$ alone. This indicates that while \sysname$_\text{server}$'s primary effect is on model distance distribution, it contributes synergistically to the fairness improvements achieved by \sysname$_\text{client}$ when both are employed.


More results related to the adaptation of the aggregation weights $\mathcal{W}$ (\autoref{eq:finpserver}) and the regularization strength $\rho_k$ (\autoref{eq:finpclient}) are shown in \autoref{appendix:adaptation}. 

\begin{figure*}
\centering
\includegraphics[width=0.8\linewidth]{fig/HAR/scatter_eigenvalues_eigentrace.png}
\caption{Scatter figures for Hessian max eigenvalue ($\lambda_{\text{max}}$) and Hessian trace ($H_{T}$).  The figure shows the value of each clients Hessian max eigenvalue and trace in the Baseline method for HAR dataset from rounds 6 to 11. All the rounds are depicted in \autoref{appendix:Hessian}.}
\label{fig:subHessian}
\end{figure*}


\subsection{\sysname\ Performance in CIFAR-10 dataset} \label{sec:CIFAReval}
In the CIFAR-10 dataset, \sysname\ demonstrates a significant improvement in fairness in privacy, with competitive accuracy. \autoref{fig:CIFAR loss} shows the Fairness Index of prediction loss (\texttt{FI(Loss)}) for FedAvg, FedAlign, and \sysname\ are 68.8\%, 58.0\%, and 83.3\%, respectively.  \sysname\ achieves a substantial increase in \texttt{FI(Loss)} of 21.1\% compared to FedAvg and 43.6\% compared to FedAlign. Notably, despite employing a distillation technique, FedAlign failed to effectively mitigate SIA risks, exhibiting a higher \texttt{CoV(Loss)} of 0.862 compared to FedAvg's 0.674. This increased \texttt{CoV(Loss)} can empower attackers with greater confidence in predicting the source client, consequently leading to higher SIA success rates.

Although FedAlign and FedAvg exhibit similar Mean and Max SIA success rates (\autoref{tbl:CIFAR}), \sysname\ effectively mitigates these risks. As shown in \autoref{fig:sia overall} and \autoref{tbl:CIFAR}, \sysname\ reduces the \texttt{Mean(SIA)} success rate to 10.07\%, approaching the random-guess probability of 1/10 (10\%) for a 10-class classification task.  Specifically, \sysname\ reduces the \texttt{Mean(SIA)} success rate from 30.86\% to 10.07\% and the \texttt{Max(SIA)} success rate from 38.52\% to 10.67\%. As shown in \autoref{fig:CIFAR sia}, the \sysname\ demonstrates comparable \texttt{CoV(SIA)} and \texttt{FI(SIA)}, yet exhibits a substantial reduction in the average success rate of SIA as mentioned above.

Moreover, \sysname\ maintains and slightly improves classification accuracy.  Figure \ref{fig:accu cifar} shows that \sysname\ achieves a testing accuracy of 78.46\%, marginally higher than FedAvg's 77.62\%. This 0.84\% improvement is attributed to the global model's aggregation of generalized client models through Lipschitz regularization rather than models overfit to individual datasets.  In summary, \sysname\ effectively mitigates SIA privacy risks in FL training in CIFAR-10 by improving client generalization and reducing loss variation across client models, all while maintaining or slightly improving classification performance.








\begin{figure}[!t]
  \centering
  \begin{subfigure}[b]{0.99\columnwidth}
    \centering
    \includegraphics[width=0.8\textwidth]{fig/CIFAR/Fedalign/line_loss_CV.png}
    \caption{Coefficient of variation for the prediction loss \texttt{CoV(Loss)}.}
    \label{fig:CIFARCV}
  \end{subfigure}
  \hfill
  \begin{subfigure}[b]{0.99\columnwidth}
    \centering
    \includegraphics[width=0.8\textwidth]{fig/CIFAR/Fedalign/line_loss_FI.png}
    \caption{Fairness index of the prediction loss \texttt{FI(Loss)}.}
    \label{fig:CIFARFI}
  \end{subfigure}
  \caption{Disparity of prediction loss among clients using CIFAR-10 dataset.}
  \label{fig:CIFAR loss}
\end{figure}

\begin{table}[!t]
\caption{SIA accuracy performance in CIFAR-10 dataset with Resnet model.}
\label{tab:my-table}
\begin{tabular}{|c|l|l|}
\hline
\multicolumn{1}{|l|}{} & Mean SIA(\%) $\downarrow$ & Max SIA(\%) $\downarrow$   \\
\hline
Baseline~\cite{hu2023source} & 30.86 & 38.52   \\
FedAlign~\cite{mendieta2022local} & 30.72 & 38.46 \\
\sysname & \textbf{10.07} & \textbf{10.67} \\
\hline
\end{tabular}
\label{tbl:CIFAR}
\end{table} 



\begin{figure}[!t]
\centering
\includegraphics[width=0.8\columnwidth]{fig/CIFAR/Fedalign/line_SIA_Accuracy.png}
\caption{Average SIA accuracy across rounds in CIFAR-10 dataset. }
\label{fig:sia overall}
\end{figure}




\begin{figure}[!t]
  \centering
  \begin{subfigure}[b]{0.99\columnwidth}
    \centering
    \includegraphics[width=0.8\textwidth]{fig/CIFAR/Fedalign/line_CoV_SIA.png}
    \caption{Coefficient of variation for SIA accuracy \texttt{CoV(SIA)}.}
    \label{fig:CIFARCVsia}
  \end{subfigure}
  \hfill
  \begin{subfigure}[b]{0.99\columnwidth}
    \centering
    \includegraphics[width=0.8\textwidth]{fig/CIFAR/Fedalign/line_FI_SIA.png}
    \caption{Fairness index of SIA accuracy \texttt{FI(SIA)}.}
    \label{fig:CIFARFIsia}
  \end{subfigure}
  \caption{Disparity of SIA accuracy among clients using CIFAR-10 dataset.}
  \label{fig:CIFAR sia}
\end{figure}



\begin{figure}[!t]
  \centering
  \begin{subfigure}[b]{0.99\columnwidth}
    \centering
    \includegraphics[width=0.8\textwidth]{fig/CIFAR/Fedalign/line_average_train_accuracy.png}
    \caption{Global model training accuracy. }
  \end{subfigure}
  \hfill
  \begin{subfigure}[b]{0.99\columnwidth}
    \centering
    \includegraphics[width=0.8\textwidth]{fig/CIFAR/Fedalign/line_average_test_accuracy.png}
     \caption{Global model testing accuracy.}
  \end{subfigure}
  \caption{Global model classification accuracy using CIFAR-10.}
  \label{fig:accu cifar}
\end{figure}







\subsection{Summary of \sysname\ in HAR and CIFAR-10 results}
In summary, our evaluation across HAR and CIFAR-10 datasets demonstrates the effectiveness of \sysname\ in achieving fairness in the impact of source inference attacks (SIA) while maintaining or improving model performance.   Although \sysname successfully achieves a more equitable distribution of SIA risk among clients in the HAR dataset, the inherent variability of human activity data, with each client representing a single individual exclusively, limits the complete elimination of SIA vulnerability, as it is difficult for client models to be completely indistinguishable. Nevertheless, as detailed in \autoref{sec:HAReval}, \sysname\ significantly enhances fairness in privacy.  Conversely, on the CIFAR-10 dataset, \sysname\ effectively mitigates SIA risks as detailed in \autoref{sec:CIFAReval}. This result is attributed to the non-IID sampling of CIFAR-10 subsets across clients, where all data in each subset are part of the comprehensive CIFAR-10 dataset. This characteristic enables the regularization of client models to be indistinguishable and coupled with applying Lipschitz constant loss during local training and adaptive aggregation at the server. These mechanisms collectively promote client model generalization and reduce loss variation, thereby neutralizing the effectiveness of source inference attacks (SIA).











\section{Discussion}
\omniUIST is capable of tracking a passive tool with an accuracy of roughly 6.9 mm and, at the same time, deliver a maximum force of up to 2 N to the tool. This is enabled by our novel gradient-based approach in 3D position reconstruction that accounts for the force exerted by the electromagnet. 

Over extended periods of time, \omniUIST can comfortably produce a force of 0.615 N without the risk of overheating. In our applications, we show that \omniUIST has the potential for a wide range of usage scenarios, specifically to enrich AR and VR interactions.

\omniUIST is, however, not limited to spatial applications. We believe that \omniUIST can be a valuable addition to desktop interfaces, e.g., navigating through video editing tools or gaming. We plan to broaden \omniUIST's usage scenarios in the future.

The overall tracking performance of \omniUIST suffices for interactive applications such as the ones shown in this paper. The accuracy could be improved by adding more Hall sensors, or optimizing their placement further (e.g., placing them on the outer hull of the device).
Furthermore, a spherical tip on the passive tool that more closely resembles the dipole in our magnetic model could further improve \omniUIST's accuracy. We believe, however, that the design of \omniUIST represents a good balance of cost and complexity of manufacturing, and accuracy.

Our current implementation of \omniUIST and the accompanying tracking and actuation algorithms assumes the presence of a single passive tool. Our method, however, potentially generalizes to tracking multiple passive tools by accounting for the presence of multiple permanent magnets. This poses another interesting challenge: the magnets of multiple tools will interact with each other, i.e., attract and repel each other.The electromagnet will also jointly interact with those tools, leading to challenges in terms of computation and convergence. We believe that our gradient-based optimization can account for such interactions and plan to investigate this in the future.

In developing and testing our applications, we found that \omniUIST's current frame rate of 40 Hz suffices for many interactive scenarios. The frame rate is a trade-off between speed and accuracy. In our tests, decreasing the desired accuracy in our optimization doubled the frame rate, while resulting in errors in the 3D position estimation of more than 1 cm, however. Finding the sweet spot for this trade-off depends on the application. While our applications worked well with 40 Hz and the current accuracy, more intricate actions such as high-precision sculpting might benefit from higher frame rates \textit{and} precision.
Reducing the latency of several system components (e.g., sensor latency, convergence time) is another interesting direction of future research. 

Furthermore, the control strategy we used was fairly naïve, as it only takes the current tool position into account. A model predictive strategy could account for future states, user intent, and optimize to reduce heating. We will explore in the next chapter how model predictive approaches can be used for haptic systems.

Overall, the main benefits of \omniUIST lie in the high accuracy and large force it can produce. It does so without mechanically moving parts, which would be subject to wear.
Such wear is not the case for our device, because it is exclusively based on electromagnetic force. We believe that different form factors of \omniUIST (e.g., body-mounted, larger size) can present interesting directions of future research. \add{A body-mounted version could be interesting for VR applications in which the user moves in 3D space. The larger size could result in more discernible points.}

Additionally, the influence of strength on user perception and factors such as just-noticeable-difference will allow us to characterize the benefits and challenges of \omniUIST, and electromagnetic haptic devices in general.
We believe that \omniUIST opens interesting directions for future research in terms of novel devices, and magnetic actuation and tracking.
\section{Conclusion}
We introduced \Bench, the first ever IMTS forecasting benchmark.
\Bench's datasets are created with ODE models, that were defined in decades of research and published on
the Physiome Model Repository. Our experiments showed that LinODEnet and CRU are actually
better than previous evaluation on established datasets indicated. Nevertheless,
we also provided a few datasets, on which models are unable to outperform a
constant baseline model. We believe that our datasets, especially the very difficult ones,
can help to identify deficits of current architectures and support future research on
IMTS forecasting.



%%
%% The acknowledgments section is defined using the "acks" environment
%% (and NOT an unnumbered section). This ensures the proper
%% identification of the section in the article metadata, and the
%% consistent spelling of the heading.
\begin{acks}
This work is supported by the U.S. National Science Foundation (NSF) under grant number 2339266.
\end{acks}


%%
%% The next two lines define the bibliography style to be used, and
%% the bibliography file.
\balance
\bibliographystyle{ACM-Reference-Format}
\bibliography{facctref}


%%
%% If your work has an appendix, this is the place to put it.
\appendix
\renewcommand{\thefigure}{S\arabic{figure}}
\setcounter{figure}{0}  
\renewcommand{\thetable}{S\arabic{table}}
\setcounter{table}{0} 


\section{Proofs}
\subsection{Proof of Proposition~\ref{prop:cos_sim_grads}}
\label{prf:prop_grad_grows}
\cosgrads*
\begin{proof}
    We are taking the gradient of $\mathcal{L}^\mathcal{A}_i$ as a function of $z_i$. The principal idea is that the gradient has a term with direction $\hat{z}_j$ and a term with direction $-\hat{z}_i$. We then disassemble the vector with direction $\hat{z}_j$ into its component parallel to $z_i$ and its component orthogonal to $z_i$. In doing so, we find that the two terms with direction $z_i$ cancel, leaving only the one with direction orthogonal to $z_i$.
    
    Writing it out fully, we have $\mathcal{L}^\mathcal{A}_i = -z_i^\top z_j / (\|z_i\| \cdot \|z_j\|)$. Taking the gradient amounts to using the quotient rule, with $f = -z_i^\top z_j$ and $g = \|z_i\| \cdot \|z_j\| = \sqrt{z_i^\top z_i} \cdot \sqrt{z_j^\top z_j}$. Taking the derivative of each, we have
    \begin{align*}
        f' &= -\mathbf{z}_j \\
        g' &= \|z_j\| \frac{z_i}{\sqrt{z_i^\top z_i}} = \|z_j\| \frac{\mathbf{z}_i}{\|z_i\|} \\
        \implies \frac{f' g - g' f}{g^2} &= \frac{- \left(\mathbf{z}_j \cdot \|z_i\| \cdot \|z_j\| \right) + \left(\|z_j\| \frac{\mathbf{z}_i}{\|z_i\|} \cdot z_i^\top z_j \right)}{\|z_i\|^2 \cdot \|z_j\|^2} \\
        &= \frac{-\mathbf{z}_j}{\|z_i\| \cdot \|z_j\|} + \frac{\mathbf{z}_i z_i^\top z_j}{\|z_i\|^3 \|z_j\|},
    \end{align*}
    where we use boldface $\mathbf{z}$ to emphasize which direction each term acts along. We now substitute $\cos(\phi_{ij}) = z_i^\top z_j / (\|z_i\| \cdot \|z_j\|)$ in the second term to get
    \begin{equation}
        \label{eq:quotient_rule}
        \frac{f' g - g' f}{g^2} = \frac{-\hat{z}_j}{\|z_i\|} + \frac{\mathbf{z}_i \cos(\phi)}{\|z_i\|^2}
    \end{equation}

    It remains to separate the first term into its sine and cosine components and perform the resulting cancellations. To do this, we take the projection of $\hat{z}_j = \mathbf{z}_j / \|z_j\|$ onto $\mathbf{z}_i$ and onto the plane orthogonal to $\mathbf{z}_i$. The projection of $\hat{z}_j$ onto $\mathbf{z}_i$ is given by
    \[ \cos \phi_{ij} \frac{\mathbf{z}_i}{\|z_i\|} \]
    while the projection of $\mathbf{z}_j / \|z_j\|$ onto the plane orthogonal to $\mathbf{z}_i$ is
    \[ \left( \mathbf{I} - \frac{z_i z_i^\top}{\|z_i\|^2} \right) \frac{\mathbf{z}_j}{\|z_j\|}. \]
    It is easy to assert that these components sum to $\mathbf{z}_j/\|z_j\|$ by replacing the $\cos \phi_{ij}$ by $\frac{z_i^\top z_j}{\|z_i\|\cdot \|z_j\|}$.

    We plug these into Eq.~\ref{eq:quotient_rule} and cancel the first and third term to arrive at the desired value:
    \begin{align*}
        \frac{f' g - g' f}{g^2} = &-\frac{1}{\|z_i\|} \cos \phi \frac{\mathbf{z}_i}{\|z_i\|} \\
        &- \frac{1}{\|z_i\|} \cdot \left( \mathbf{I} - \frac{z_i z_i^\top}{\|z_i\|^2} \right) \frac{\mathbf{z}_j}{\|z_j\|} \\
        &+ \frac{\mathbf{z}_i \cos(\phi)}{\|z_i\|^2} \\
        = &\frac{-1}{\|z_i\|} \cdot \left( \mathbf{I} - \frac{z_i z_i^\top}{\|z_i\|^2} \right) \frac{\mathbf{z}_j}{\|z_j\|}.
    \end{align*}
\end{proof}

We visualize the loss landscape of the cosine similarity function in Figure \ref{fig:cos_sim_surface}. 

\begin{figure}
    \centering
    \begin{subfigure}{0.45\linewidth}
        \centering 
        \includegraphics[width=1\linewidth]{Images/cosine_similarity_surface_with_circles.pdf}
    \end{subfigure}%
    \begin{subfigure}{0.45\linewidth}
        \centering 
        \includegraphics[width=0.8\linewidth]{Images/cosine_similarity_2D_heatmap.pdf}
    \end{subfigure}
    \caption{Cosine similarity with respect to the direction indicated by the blue line. Three circles of radii 0.5, 1, and 2 are superimposed to show that, for higher norms, the cosine similarity is less steep. Left: 3D Surface plot, right: 2D topview plot.}
    \label{fig:cos_sim_surface}
\end{figure}


\subsection{InfoNCE Gradients}
\label{app:infonce_grads}
\infoncegrads*
\begin{proof}
    We are interested in the gradient of $\mathcal{L}_i^\mathcal{R}$ with respect to $z_i$. By the chain rule, we get
    \begin{align*}
        \nabla_i^\mathcal{R} &= -\frac{\sum_{k \not\sim i} \text{ExpSim}(z_i, z_k) \frac{\partial \frac{z_i^\top z_k}{\|z_i\| \cdot \|z_k\|}}{\partial z_i}}{\sum_{k \not\sim i} \text{ExpSim}(z_i, z_k)} \\
        &= -\frac{\sum_{k \not\sim i} \text{ExpSim}(z_i, z_k) \frac{\partial \frac{z_i^\top z_k}{\|z_i\| \cdot \|z_k\|}}{\partial z_i}}{S_i}
    \end{align*}
    It remains to substitute the result of Prop. \ref{prop:cos_sim_grads} for $\partial \frac{z_i^\top z_k}{\|z_i\| \cdot \|z_k\|} / \partial z_i$.

    We sum this this with the gradients of the attractive term to obtain the full InfoNCE gradient, completing the proof.
\end{proof}

We note that the repulsive force is weighted average over a set of unit vectors. Consequently, the repulsive gradient is smaller than the attractive one. Additionally, we point out that these gradients are symmetric: just like positive and negative samples $z_j$ and $z_k$ affect $z_i$, $z_i$ affects $z_j$ and $z_k$.

\subsection{Proof of Corollary~\ref{cor:embeddings_grow}}
\label{prf:cor_embeddings_grow}
\begin{proof}
    First, consider that we applied the cosine similarity's gradients from Proposition~\ref{prop:cos_sim_grads}. Since $z_i$ and $(z_j)_{\perp z_i}$ are orthogonal, $\|z_i'\|_2^2 = \|z_i\|^2 + \frac{\gamma^2}{\|z_i\|^2}\|(z_j)_{\perp z_i}\|^2$. The second term is positive if $\sin \phi_{ij} > 0$.

    The same exact argument holds for the InfoNCE gradients. The gradient is orthogonal to the embedding, so a step of gradient descent can only increase the embedding's magnitude.
\end{proof}

\subsection{Proof of Theorem~\ref{thm:convergence_rate}}
\label{prf:thm_convergence_rate}
We first restate the theorem:

Let $z_i$ and $z_j$ be positive embeddings with equal norm, i.e. $\|z_i\| = \|z_j\| = \rho$. Let $z_i'$ and $z_j'$ be the embeddings after 1 step of gradient descent with learning rate $\gamma$. Then the change in cosine similarity is bounded from above by:
\begin{equation*}
    \hat{z}_i'^\top \hat{z}_j' - \hat{z}_i^\top \hat{z}_j < \frac{\gamma \sin^2 \phi_{ij}}{\rho^2} \left[ 2 - \frac{\gamma \cos \phi}{\rho^2} \right].
\end{equation*}

\noindent We now proceed to the proof:
\begin{proof}
    Let $z_i$ and $z_j$ be two embeddings with equal norm\footnote{We assume the Euclidean distance for all calculations.}, i.e. $\|z_i\| = \|z_j\| = \rho$. We then perform a step of gradient descent to maximize $\hat{z}_i^\top \hat{z}_j$. That is, using the gradients in \ref{prop:cos_sim_grads} and learning rate $\gamma$, we obtain new embeddings $z_i' = z_i + \frac{\gamma}{\|z_i\|} (\hat{z}_j)_{\perp z_i}$ and $z_j' = z_j + \frac{\gamma}{\|z_j\|} (\hat{z}_i)_{\perp z_j}$. Going forward, we write $\delta_{ij} = (\hat{z}_j)_{\perp z_i}$ and $\delta_{ji} = (\hat{z}_i)_{\perp z_j}$, so $z_i' = z_i + \frac{\gamma}{\rho} \delta_{ij}$ and $z_j' = z_j + \frac{\gamma}{\rho} \delta_{ji}$. Notice that since $z_i$ and $\delta_{ij}$ are orthogonal, by the Pythagorean theorem we have $\|z_i'\|^2 = \|z_i\|^2 + \frac{\gamma^2}{\rho^2}\|\delta_{ij}\|^2 \geq \|z_i\|^2$. Lastly, we define $\rho' = \|z_i'\| = \|z_j'\|$.

    We are interested in analyzing $\hat{z}_i'^\top \hat{z}_j' - \hat{z}_i^\top \hat{z}_j$. To this end, we begin by re-framing $\hat{z}_i'^\top \hat{z}_j'$:
    \begin{align*}
        \hat{z}_i'^\top \hat{z}_j' &= \left(\frac{z_i + \frac{\gamma}{\rho} \delta_{ij}}{\rho'}\right)^\top \left(\frac{z_j + \frac{\gamma}{\rho} \delta_{ji}}{\rho'}\right) \\
        &= \frac{1}{\rho'^2}\left[ z_i^\top z_j + \gamma \frac{z_i^\top \delta_{ji}}{\rho'} + \gamma \frac{z_j^\top \delta_{ij}}{\rho'} + \gamma^2 \frac{\delta_{ij}^\top \delta_{ji}}{\rho'^2} \right].
    \end{align*}

    We now consider that, since $\delta_{ij}$ is the projection of $\hat{z}_j$ onto the subspace orthogonal to $z_i$, we have that the angle between $z_i$ and $\delta_{ji}$ is $\pi/2 - \phi_{ij}$. Plugging this in and simplifying, we obtain
    \begin{align*}
        z_i^\top \delta_{ji} &= \|z_i\| \cdot \|\delta_{ji}\| \cos (\pi/2 - \phi_{ij}) \\
        &= \|z_i\| \cdot \|\delta_{ji}\| \sin \phi_{ij} \\
        &= \rho \sin^2 \phi_{ij}.
    \end{align*}
    By symmetry, the same must hold for $z_j^\top \delta_{ij}$.
    
    Similarly, we notice that the angle $\psi_{ij}$ between $\delta_{ij}$ and $\delta_{ji}$ is $\psi_{ij} = \pi - \phi_{ij}$. The reason for this is that we must have a quadrilateral whose four internal angles must sum to $2\pi$, i.e. $\psi_{ij} + \phi_{ij} + 2 \frac{\pi}{2} = 2 \pi$. Thus, we obtain $\delta_{ij}^\top \delta_{ji} = \|\delta_{ij}\| \cdot \|\delta_{ji}\| \cos(\psi) = -\sin^2 \phi_{ij} \cos \phi_{ij}$.

    We plug these back into our equation for $\hat{z}_i'^\top \hat{z}_j'$ and simplify:
    \begin{align*}
        \hat{z}_i'^\top \hat{z}_j' &= \frac{1}{\rho'^2}\left[ z_i^\top z_j + \gamma \frac{z_i^\top \delta_{ji}}{\rho} + \gamma \frac{z_j^\top \delta_{ij}}{\rho} + \gamma^2 \frac{\delta_{ij}^\top \delta_{ji}}{\rho^2} \right] \\
        &= \frac{1}{\rho'^2}\left[ z_i^\top z_j + \gamma \frac{\rho \sin^2 \phi_{ij}}{\rho} + \gamma \frac{\rho \sin^2 \phi_{ij}}{\rho} - \gamma^2 \frac{\sin^2 \phi_{ij} \cos \phi_{ij}}{\rho^2} \right] \\
        &= \frac{1}{\rho'^2}\left[ z_i^\top z_j + 2 \gamma \sin^2 \phi_{ij} - \gamma^2 \frac{\sin^2 \phi_{ij} \cos \phi_{ij}}{\rho^2} \right].
    \end{align*}

    We now consider the original term in question:
    \begin{align*}
        \hat{z}_i'^\top \hat{z}_j' - \hat{z}_i^\top \hat{z}_j &= \frac{1}{\rho'^2}\left[ z_i^\top z_j + 2 \gamma \sin^2 \phi_{ij} - \gamma^2 \frac{\sin^2 \phi_{ij} \cos \phi_{ij}}{\rho^2} \right] - \frac{z_i^\top z_j}{\rho^2} \\
        &\leq \frac{1}{\rho^2}\left[ z_i^\top z_j + 2 \gamma \sin^2 \phi_{ij} - \gamma^2 \frac{\sin^2 \phi_{ij} \cos \phi_{ij}}{\rho^2} \right] - \frac{z_i^\top z_j}{\rho^2} \\
        &= \frac{1}{\rho^2}\left[ 2 \gamma \sin^2 \phi_{ij} - \gamma^2 \frac{\sin^2 \phi_{ij} \cos \phi_{ij}}{\rho^2} \right] \\
        &= \frac{\gamma \sin^2 \phi_{ij}}{\rho^2}\left[ 2 - \frac{\gamma \cos \phi_{ij}}{\rho^2} \right]\\
        &\leq \frac{2 \gamma \sin^2 \phi_{ij}}{\rho^2}
    \end{align*}
    
    This concludes the proof.
\end{proof}

\section{Simulations}
\label{app:simulations}

\subsection{Aparametric Simulations}

For the simulations in Section \ref{ssec:convergence_simulations}, we produced two datasets, $\mathbf{X}_1$ and $\mathbf{X}_2$, independently by randomly sampling points in $\mathbb{R}^20$ from a standard normal distribution and normalizing them to the hypersphere. The $i$-th point in dataset $\mathbf{X}_1$ is the positive counterpart for the $i$-th point in dataset $\mathbf{X}_2$. The first dataset is then set to be static while the second is modified in order to control for the embedding norms and angles between positive pairs.

We optimize the cosine similarity by performing standard gradient descent on the embeddings themselves with learning rate $10$. We consider a dataset ``converged'' when the average cosine similarity between positive pairs exceeds $0.999$.

\paragraph{Controlling for angles.} In order to control for the angle between positive pairs, we use an interpolation value $\alpha \in [-1, 1]$. Let $x_1$ be a static embedding in $\mathbf{X}_1$ and $x_2$ be the embedding in $\mathbf{X}_2$ whose angle we wish to control. In expectation, $\phi(x_1, x_2)$ will be $\pi / 2$. We therefore define the embedding $x_2$ whose angle has been controlled as 
\[ x_2' = x_2 \cdot (1 - |\alpha|) + x_1 \cdot \alpha. \]

In essence, when $\alpha=0$, $x_2' = x_2$. However, when $\alpha=1$ (resp. $\alpha=-1$), $x_2' = x_1$ (resp. $x_2' = -x_1$).

\paragraph{Controlling for embedding norms.} This setting is simpler than the angles between positive pairs. We simply scale $\mathbf{X}_2$ by the desired value.

\subsection{Parametric Simulations}
\label{app:parametric_sim}

We restate the entire implementation for the simulations in Section \ref{ssec:confidence_simulations} for completeness. We choose centers for 4 latent classes $\{c_1, c_2, c_3, c_4\}$ uniformly at random from $\mathbb{S}^{10}$ by randomly sampling vectors from a standard multivariate normal distribution and normalizing them to the hypersphere. We then obtain the latent samples $\tilde{z}$ around center $c_i$ via $z \sim \mathcal{N}(c_i, 0.1 \cdot \mathbf{I})$ and re-normalizing to the hypersphere. For each center, we produce 1K latent samples; these constitute our latent classes. We depict an example of 8 such latent classes (in 3 dimensions) in Figure \ref{fig:orig_latents}. We finally obtain the dataset by generating a random matrix in $\mathbb{R}^{11 \times 64}$ and applying it to the latent samples.

We train the InfoNCE loss via a 2-layer feedforward neural network with the ReLU activation function in the hidden layer. The network's output dimensionality is $\mathbb{R}^{11}$ so that, after normalization, it can reconstruct the original latent classes. We train the network using the supervised InfoNCE loss with a batch size of 128. Each data point's positive pair is simply another data point from the same latent class.

We visualize the learned (unnormalized) embedding space in Figure \ref{fig:learned_latents}.

\begin{figure}
    \centering
    \begin{subfigure}{0.4\linewidth}
    \includegraphics[width=\linewidth]{Images/orig_latents.png}
    \caption{}
    \label{fig:orig_latents}
    \end{subfigure}
    \quad\quad
    \begin{subfigure}{0.4\linewidth}
    \includegraphics[width=\linewidth]{Images/learned_latents.png}
    \caption{}
    \label{fig:learned_latents}
    \end{subfigure}
    \caption{\emph{Left}: A depiction of $8$ latent classes in $3$D obtained via the description in Section \ref{app:parametric_sim}. Dashed lines represent vectors from the origin to the mean of the distribution. \emph{Right}: A depiction of the learned latent space (unnormalized) using the supervised InfoNCE loss after 50 epochs of training.}
    
\end{figure}


\section{Further Discussion and Experiments}
\label{app:experiments}

\subsection{Experimental Setup}
\label{app:experiment_setup}
Unless otherwise stated, we use a ResNet-50 backbone \cite{resnet} and the default settings outlined in the SimCLR \cite{simclr} and SimSiam \cite{simsiam} papers. We use $1$e-$6$ as the default SimCLR weight decay and $5$e-$4$ as the default SimSiam one. When running on Cifar-10 and Cifar-100, we amend the backbone network's first layer as detailed in \citet{simclr}. We use embedding dimensionality $256$ in SimCLR and $2048$ in SimSiam. When reporting embedding norms, we use the projector's output in SimCLR and the predictor's output in SimSiam: these are the spaces where the loss function is applied and therefore where our theory holds.

Due to computational constraints, we run with batch-size 256 in SimCLR. Although each batch is still 256 samples in SimSiam, we simulate larger batch sizes using gradient accumulation. Thus, our default batch-size for SimSiam is 1024. 

\subsection{Opposite-Halves Effects}
\label{app:opposite_halves_effect}

We devote this section of the Appendix to studying the role of the angle between positive samples on the cosine similarity's convergence under gradient descent. Referring back to Figure~\ref{fig:convergence_sim}, we see that the effect is most impactful when the angle between positive embeddings is close to $\pi$, i.e. $\phi_{ij} > \pi - \varepsilon$ for $\varepsilon \rightarrow 0$. The following result shows that this is exceedingly unlikely for a single pair of embeddings in high-dimensional space:
\begin{proposition}
    \label{prop:unlikely_opp_halves}
    Let $x_i, x_j \sim \mathcal{N}(0, \mathbf{I})$ be $d$-dimensional, i.i.d. random variables and let $0 < \varepsilon < 1$. Then \vspace*{-0.1cm}
    \begin{equation}
    \label{eq:opp_halves_unlikely}
    \mathbb{P}\left[ \hat{x}_i^\top \hat{x}_j \geq 1 - \varepsilon \right] \leq \frac{1}{2d(1-\varepsilon)^2}.
    \end{equation}\vspace*{-0.3cm}
\end{proposition}
\begin{proof}
By \citet{distribution_of_cosine_sim}, the cosine similarity between two i.i.d. random variables drawn from $\mathcal{N}(0, \mathbf{I})$ has expected value $\mu = 0$ and variance $\sigma^2 = 1/d$, where $d$ is the dimensionality of the space. We therefore plug these into Chebyshev's inequality:
\begin{align*}
    &\text{Pr} \left[ \left|\frac{x_i^\top x_j}{\|x_i\|\cdot \|x_j\|} - \mu \right|\geq k \sigma \right] \leq \frac{1}{k^2} \\
    \rightarrow & \text{Pr} \left[ \left |\frac{x_i^\top x_j}{\|x_i\|\cdot \|x_j\|} \right |\geq \frac{k}{\sqrt{d}} \right] \leq \frac{1}{k^2}
\end{align*}

\noindent We now choose $k = \sqrt{d}(1 - \varepsilon)$, giving us
\[ \mathbb{P}\left[ \left |\frac{x_i^\top x_j}{\|x_i\| \cdot \|x_j\|}\right | \geq 1 - \varepsilon \right] \leq \frac{1}{d(1-\varepsilon)^2}.\]

It remains to remove the absolute values around the cosine similarity. Since the cosine similarity is symmetric around $0$, the likelihood that its absolute value exceeds $1 - \varepsilon$ is twice the likelihood that its value exceeds $1- \varepsilon$, concluding the proof.

We note that this is actually an extremely optimistic bound since we have not taken into account the fact that the maximum of the cosine similarity is 1.
\end{proof}

The above proposition represents the likelihood that \emph{one} pair of embeddings has large angle between them. It is therefore \emph{exponentially} unlikely for every pair of embeddings in a dataset to have angle close to $\pi$, as we would require Proposition \ref{prop:unlikely_opp_halves} to hold across every pair of embeddings. Thus, the opposite-halves effect is exceedingly unlikely to occur.

\begin{table}
    \centering
    \quad
    \parbox{.47\linewidth}{
        \begin{tabular}{lrcc}
        \toprule
        Model & Dataset \quad\quad & \makecell{Effect Rate\\Epoch 1} & \makecell{Effect Rate\\Epoch 16} \\
        \midrule
        \multirow{2}{*}{SimCLR} & Imagenet-100 & 2\% & 0\%  \\
        & Cifar-100 & 11\% & 1\% \\
        \cmidrule{1-4}
        \multirow{2}{*}{SimSiam} & Imagenet-100 & 26\% & 1\% \\
        & Cifar-100 & 21\% & 0\% \\
        \cmidrule{1-4}
        \multirow{2}{*}{BYOL} & Imagenet-100 & 28\% & 1\% \\
        & Cifar-100 & 20\% & 0\% \\
        \bottomrule
        \end{tabular}
        \captionof{table}{The rate at which embeddings are on opposite sides of the latent space (angle between a positive pair is greater than $\pi / 2$) for various datasets and SSL models.}
        \label{tbl:opposite_halves_effect}
    }
    \hfill
    \parbox{.38\linewidth}{
        \begin{tabular}{cc ccc}
        \toprule
        \multirow{2}{*}{Epoch} & & \multicolumn{3}{c}{Batch Size}\\
        & & 256 & 512 & 1024 \\
        \cmidrule{3-5}
        \multirow{2}{*}{100} & Default & 46.1 & 41.2 & 32.6 \\
        & Cut ($c=9$) & 43.1 & 46.5 & 44.3 \\
        \cmidrule{2-5}
        \multirow{2}{*}{500} & Default & 59.1 & 60.4 & 61.3\\
        & Cut ($c=9$) & 59.4 & 58.9 & 61.5 \\
        \bottomrule
        \end{tabular}
        \captionof{table}{$k$-nn accuracies for SimSiam trained with various batch sizes. We performed training for both the default and cut-initialized variants and reported $k$-nn accuracies at 100 and 500 epochs.}
        \label{tbl:cut_batch_size}
    }
\end{table}

In accordance with this, Table~\ref{tbl:opposite_halves_effect} shows that, after one epoch of training, embeddings have angle greater than $\pi/2$ at a rate of around $5\%$ and $25\%$ for SimCLR and SimSiam/BYOL, respectively. So even if the `strongest' variant of the opposite-halves effect is not occurring, a weaker one may still be. However, very early into training (epoch 16), every method has a rate of effectively 0 for the opposite-halves effect. Furthermore, the rates in Table~\ref{tbl:opposite_half_effect} measure how often $\phi_{ij} > \frac{\pi}{2}$. This is the absolute weakest version of the opposite-halves effect. Thus, while some weak variant of the opposite-halves effect may occur at the beginning of training, it does not have a strong impact on the convergence dynamics and, in either case, disappears quite quickly.

\subsection{Weight Decay}
\label{app:weight_decay}

We evaluate the effect of weight decay in the imbalanced setting in \ref{fig:weight_decay_imbalanced}, which is an analog of Figure \ref{fig:weight_decay_ablation} for the imbalanced Cifar-10 dataset detailed in Section \ref{sec:convergence}. We again see that using weight decay controls for the embedding norms and improves the convergence of both models. In correspondence with the other results on imbalanced training, we find that stronger control over the embedding norms leads to improved convergence: the high weight decay value does not perform as poorly on SimCLR as in Figure \ref{fig:weight_decay_ablation} and, on SimSiam, outperforms the other weight decay options.

\begin{figure}
    \centering
    \begin{tikzpicture}
    \node () at (0, 0) {\includegraphics[width=0.4\linewidth]{Images/wd_sweep_imbalanced.png}};

    \draw[ballblue, line width=0.07cm] (-4, -3.8) -- (-3.4, -3.8);
    \draw[azure, line width=0.07cm] (-1.3, -3.8) -- (-0.7, -3.8);
    \draw[darkblue, line width=0.07cm] (2, -3.8) -- (2.6, -3.8);

    \node () at (-1.15, -3.33) {\small \textcolor{darkgray}{Train Epoch}};
    \node () at (1.95, -3.33) {\small \textcolor{darkgray}{Train Epoch}};

    \node[inner sep=0pt] () at (-2.5, -3.82) {\textcolor{darkgray}{\scriptsize No weight decay}};
    \node[inner sep=0pt] () at (0.48, -3.82) {\textcolor{darkgray}{\scriptsize Standard weight decay}};
    \node[inner sep=0pt] () at (3.57, -3.82) {\textcolor{darkgray}{\scriptsize High weight decay}};
        
        
    \end{tikzpicture}
    \caption{An analog to Figure \ref{fig:weight_decay_ablation} performed on the exponentially imbalanced Cifar-10 dataset. Weight decays are [$0$, $1$e-$5$, $5$e-$2$] for SimCLR and [$0$, $5$e-$4$, $5$e-$2$] for SimSiam. We plot the effective learning rate in the bottom row, calculated in accordance with Section \ref{sec:convergence}.}
    \label{fig:weight_decay_imbalanced}
\end{figure}

\subsection{Cut-Initialization}
\label{app:cut_init}
We plot the effect of the cut constant on the embedding norms and accuracies over training in Figure~\ref{fig:cut_experiments}. To make the effect more apparent, we use weight-decay $\lambda=5e-4$ in all models. We see that dividing the network's weights by $c>1$ leads to immediate convergence improvements in all models. Furthermore, this effect degrades gracefully: as $c > 1$ becomes $c < 1$, the embeddings stay large for longer and, as a result, the convergence is slower. We also see that cut-initialization has a more pronounced effect in attraction-only models -- a trend that remains consistent throughout the experiments.

We also show the relationship between cut-initialization and the network's batch size on SimSiam in Table \ref{tbl:cut_batch_size}. Consistent with the literature, we see that training with large batches provides improvements to training accuracy. However, we note that larger batch sizes also significantly slow down convergence. However, cut-initialization seems to counteract this and accelerate convergence accordingly. Thus, training with cut-initialization and large batches seems to be the most effective method for SSL training (at least in the non-contrastive setting).

\begin{figure}[t!]
    \centering
    \includegraphics[width=0.95\textwidth]{Images/init_experiments.png}
    \caption{The effect of cut-initialization on Cifar10 SSL representations. $x$-axis and embedding norm's $y$-axis are log-scale. $\lambda=5$e$-4$ in all experiments.}
    \label{fig:cut_experiments}
\end{figure}

\section{More details on gradient scaling layer}
\label{app:grad_scaling}

An implementation of our GradScale layer can be found in Listing \ref{alg:grad_scaling_use}.
We note that this layer is purely a PyTorch optimization trick and does not amount to implicitly choosing a different loss function:

\begin{restatable}{proposition}{nopotential}
    \label{prop:no_potential}
    Let $t\in\mathbb{R}^n$ be a unit vector, $p: \mathbb{R}^n\backslash \{0\} \to [-1, 1], z\mapsto t^\top z/\|z\|$ the cosine similarity with respect to $t$, $\alpha \in \mathbb{R}$, and $\sigma: \mathbb{R}^n \to  \mathbb{R}, z\mapsto \|z\|^\alpha$. Then the vector field $\sigma\nabla p$ has a potential $q$, i.e., $\nabla q = \sigma \nabla p$, only for $\alpha=0$.
\end{restatable}

\begin{proof}
    Suppose $\sigma \nabla p$ has potential. Consider two paths with segments $s_1, s_2$ and $s_3, s_4$ going $t \to 2t \to -2t$ and $t \to -t \to -2t$, where the segments $s_1, s_4$ scaling $\pm t \to \pm2t$ are straight lines and the other segments $s_2, s_3$ follow great circles on $S^{n-1}$. By Proposition~\ref{prop:cos_sim_grads}, we know that $\nabla p(z)=0$ for $z\in \mathbb{R}_{\neq 0}\cdot t$. So $\sigma \nabla p$ is zero on $s_1$ and $s_4$. Moreover, we have
    \begin{align}
        \int_{s_2} \sigma \nabla p \,dz &= \int_{s_2} \|z\|^\alpha \nabla p \,dz
        = \int_{s_2} 2^\alpha \nabla p \,dz 
        = 2^\alpha \int_{s_2} \nabla p \,dz 
        = 2^\alpha \big(p(2t) - p(-2t)\big) = 2^{\alpha+1}
    \end{align}
    and similarly 
    \begin{align}
        \int_{s_3} \sigma \nabla p dz = 1^\alpha \cdot 2 = 2.
    \end{align}
    Since we assume the existence of a potential, we can use path independence to conclude 
    \begin{align}
        2^{\alpha+1} &= \int_{s_2} \sigma \nabla p \,dz 
        = \int_{s_1, s_2} \sigma \nabla p \,dz 
        = \int_{s_3, s_4} \sigma \nabla p \,dz 
        = \int_{s_3} \sigma \nabla p \,dz 
        = 2.
    \end{align}
    Thus, $\alpha=0$ and $\sigma$ does not perform any scaling.
\end{proof}




\begin{figure}
    \begin{lstlisting}[caption={PyTorch code for gradient scaling layer}, label={alg:grad_scaling}]
class scale_grad_by_norm(torch.autograd.Function):
    @staticmethod
    def forward(ctx, z, power=0):
        ctx.save_for_backward(z)
        ctx.power = power
        return z
    @staticmethod
    def backward(ctx, grad_output):
        z = ctx.saved_tensors[0]
        power = ctx.power
        norm = torch.linalg.vector_norm(z, dim=-1, keepdim=True)
        return grad_output * norm**power, None
\end{lstlisting}
\end{figure}

\begin{algorithm}[tb]
   \caption{Pytorch-like pseudo-code using the gradient scaling layer}
   \label{alg:grad_scaling_use}
\begin{algorithmic}
   \STATE {\bfseries Input:} Encoder network $model$, gradient scaling power $p$
   \STATE $z = model(batch)$
   \STATE $z = grad\_scaling\_layer.apply(z, p)$
   \STATE $sim = (\frac{z}{\|z\|})^T \frac{z}{\|z\|}$
   \STATE $loss = InfoNCE(sim)$
   \STATE $loss.backward()$
\end{algorithmic}
\end{algorithm}


\section{Additional figures}
We provide a bar plot analogous to Figure \ref{fig:in_out_violin} in Figure \ref{fig:in_out_distribution_norms}.

\begin{figure}
    \centering
    \begin{tikzpicture}   
        \node[inner sep=0pt] (image) at (0,0) {\includegraphics[width=\textwidth]{Images/Confidence/per_class_norms.pdf}};
    \end{tikzpicture}
    \caption{Bar plot which is analogous to Figure \ref{fig:in_out_violin} showing embedding magnitudes on each dataset split as a function of which dataset the model was trained on. All values are normalized by training set's mean embedding magnitude. Normalized means are represented by black bars. We use the same data augmentations for the train and test sets for consistency.}
    \label{fig:in_out_distribution_norms}
\end{figure}

We also show each Cifar-10 class's 10 highest and 10 lowest embedding-norm samples in Figure \ref{fig:cifar_norms}. These are obtained after training default SimCLR on Cifar-10 for 512 epochs. We see that the high-norm class representatives are prototypical examples of the class while the low-norm representatives are obscure and qualitatively difficult to identify. This property was originally shown by \citet{embed_norm_confidence_2}.

\begin{figure}
    \centering
    \includegraphics[width=0.48\linewidth]{Images/high_norm.png}
    \quad
    \includegraphics[width=0.48\linewidth]{Images/low_norm.png}
    \caption{\emph{Left}: highest-norm representatives (top 10) per class. \emph{Right}: lowest-norm representatives (bottom 10) per class. All from default SimCLR trained on Cifar-10.}
    \label{fig:cifar_norms}
\end{figure}



\end{document}
\endinput

