\documentclass[11pt, sigconf]{acmart}


\AtBeginDocument{%
  \providecommand\BibTeX{{%
    \normalfont B\kern-0.5em{\scshape i\kern-0.25em b}\kern-0.8em\TeX}}}




\usepackage[ruled,vlined]{algorithm2e}
\usepackage[noend]{algpseudocode}

\newcommand{\thought}[1]{{\color[rgb]{0.2,0.39,0.66}(#1)}}
\newcommand{\todo}[1]{{\color[rgb]{1.0,0.0,0.0}(#1)}}
\newcommand{\hsh}[1]{{\color{green!50!black} Henrik: #1}}
\newcommand{\st}[1]{{\color{red!50!black} Sebastian: #1}}

\newcommand{\ulm}[1]{_{\scaleto{\mathrm{#1}}{3pt}}}
\newcommand\at[2]{\left.#1\right|_{#2}}











\newtheorem{assumption}{Assumption}

\DeclareMathOperator*{\argmax}{arg\,max}
\DeclareMathOperator*{\argmin}{arg\,min}

\newcommand{\swname}[1]{\texttt{#1}}
\newcommand{\ie}{i\/.\/e\/.,\/~}
\newcommand{\eg}{e\/.\/g\/.,\/~}
\newcommand{\cf}{cf\/.\/~}

\newcommand{\fig}{Fig\/.\/~}
\newcommand{\defn}{Def\/.\/~}
\newcommand{\sect}{Sec\/.\/~}
\newcommand{\tabl}{Tab\/.\/~}
\newcommand{\algo}{Algorithm~}
\newcommand{\theo}{Theorem~}

\newcommand{\bnnl}{3 hidden layers}
\newcommand{\bnnn}{50 neurons}
\newcommand{\bnna}{tanh activations}

\newcommand{\capt}[1]{\mdseries{\emph{#1}}}

\newcommand{\videolink}{at \url{https://youtu.be/_d7AqTRjz6g}}
\newcommand{\codelink}{\url{https://github.com/wheelbot/mini-wheelbot}}

\newcommand{\fakepar}[1]{\vspace{0mm}\noindent\textbf{#1.}}

\newcommand{\needref}{\textcolor{red}{[REF]}}

\newcommand{\plotfontsize}{9pt}




    

\newcommand{\sysname}{F\emph{in}P}

\setcopyright{acmlicensed}
\copyrightyear{2025}
\acmYear{2025}
\acmDOI{XXXXXXX.XXXXXXX}
%% These commands are for a PROCEEDINGS abstract or paper.
\acmConference[ACM'25]{ACM Conference}{2025}{ }

\makeatletter
\renewcommand\@formatdoi[1]{\ignorespaces}
\makeatother


\settopmatter{printfolios=true}


\begin{document}


\title{\sysname: Fairness-in-Privacy in Federated Learning by Addressing Disparities in Privacy Risk}




\author{Tianyu Zhao}
\email{tzhao15@uci.edu}
\affiliation{%
  \institution{University of California, Irvine}
  \city{}
  \state{}
  \country{}
}
\author{Mahmoud Srewa}
\email{msrewa@uci.edu}
\affiliation{%
  \institution{University of California, Irvine}
  \city{}
  \state{}
  \country{}
}

\author{Salma Elmalaki}
\email{salma.elmalaki@uci.edu}
\affiliation{%
  \institution{University of California, Irvine}
  \city{}
  \state{}
  \country{}
}

\renewcommand{\shortauthors}{Zhao et al.}








\begin{acronym}
    \acro{har}[HAR]{Human Activity Recognition}
    \acro{tcn}[TCN]{Temporal Convolutional Network }
    \acro{fl}[FL]{Federated Learning}
    \acro{aasd}[AASD]{Average Absolute SIA Difference From Mean}
    \acro{sia}[SIA]{Source Inference Attack}
    \acro{mia}[MIA]{Membership Inference Attack }
    \acro{mad}[MAD]{Mean Average Deviation}
    \acro{acc}[ACC]{Accuracy}
    \acro{rl}[RL]{Reinforcement learning}
    \acro{fedavg}[FedAvg]{Federated Averaging Algorithm}
    \acro{noniid}[non-IID]{Non-Independent and Identically Distributed}
    \acro{dp}[DF]{Differential Privacy}
    \acro{he}[HE]{Homomorphic Encryption}
    \acro{smpc}[SMPC]{Secure Multi-Party Computation}
    \acro{fe}[FE]{Functional Encryption}
\end{acronym}


\begin{abstract}
Ensuring fairness in machine learning, particularly in human-centric applications, extends beyond algorithmic bias to encompass fairness in privacy, specifically the equitable distribution of privacy risk. This is critical in federated learning (FL), where decentralized data necessitates balanced privacy preservation across clients. We introduce \sysname, a framework designed to achieve fairness in privacy by mitigating disproportionate exposure to source inference attacks (SIA). \sysname employs a dual approach: (1) server-side adaptive aggregation to address unfairness in client contributions in global model, and (2) client-side regularization to reduce client vulnerability. This comprehensive strategy targets both the symptoms and root causes of privacy unfairness. Evaluated on the Human Activity Recognition (HAR) and CIFAR-10 datasets, \sysname\ demonstrates a $\approx 20\%$ improvement in fairness in privacy on HAR with minimal impact on model utility, and effectively mitigates SIA risks on CIFAR-10, showcasing its ability to provide fairness in privacy in FL systems without compromising performance.



\end{abstract}

%%
%% The code below is generated by the tool at http://dl.acm.org/ccs.cfm.
%% Please copy and paste the code instead of the example below.
%%
% \begin{CCSXML}
% <ccs2012>
%    <concept>
%        <concept_id>10002978</concept_id>
%        <concept_desc>Security and privacy</concept_desc>
%        <concept_significance>500</concept_significance>
%        </concept>
%    <concept>
%        <concept_id>10010147.10010257.10010258.10010261</concept_id>
%        <concept_desc>Computing methodologies~Reinforcement learning</concept_desc>
%        <concept_significance>500</concept_significance>
%        </concept>
%    <concept>
%        <concept_id>10003120</concept_id>
%        <concept_desc>Human-centered computing</concept_desc>
%        <concept_significance>300</concept_significance>
%        </concept>
%  </ccs2012>
% \end{CCSXML}

% \ccsdesc[500]{Security and privacy}
% \ccsdesc[500]{Computing methodologies~Reinforcement learning}
% \ccsdesc[300]{Human-centered computing}

%%
%% Keywords. The author(s) should pick words that accurately describe
%% the work being presented. Separate the keywords with commas.
\keywords{fairness in privacy, federated learning, human-centered design}

\maketitle

\begin{figure}[ht]
    \centering
    \includegraphics[width=0.8\linewidth]{graphs/greater_than_naive.pdf}
    \vspace{0.5cm}
    \includegraphics[width=0.8\linewidth]{graphs/p1_bottom.png}
    \vspace{-5pt}
    \caption{\textcolor{positional}{Positional} vs.\ \textcolor{nonpositional}{non-positional} circuits. In a \textcolor{nonpositional}{non-positional} circuit, the same edges must be included at all positions. A \textcolor{positional}{positional} circuit can distinguish between the same edge at different positions. This specificity yields better trade-offs between circuit size and faithfulness. It can also increase both precision and recall.}
    \label{fig:p1}
    \vspace{-5pt}
\end{figure}

\section{Introduction}

\looseness=-1
A primary goal of interpretability research is to characterize the internal mechanisms in language models (LMs) and other NLP models. 
A core approach in this area is \textbf{circuit discovery}---identifying the minimal subgraph within the model's computation graph that performs a specific task \citep{olah2021framework,olah-mech}.
Typically, the nodes of a circuit represent model components (e.g., attention heads, neurons, or layers).
While manual circuit discovery methods can yield position-specific insights \citep{wanginterpretability,goldowskydill2023localizingmodelbehaviorpath}, \emph{automatic methods often overlook positional information}, treating components as uniformly relevant across all input token positions \citep{conmytowards,syed2023attribution}. 
For instance, if an attention head is included in a circuit, it is assumed to contribute equally to the computation for every position in the input sequence.
The assumption that circuits are position-invariant ignores the fact that different positions often require distinct computations.
By ignoring positions, current methods limit their ability to capture mechanisms that operate across positions, such as interactions between attention heads across positions.

In this study, we start by demonstrating that positional agnosticism is a significant limitation (\S\ref{sec:motivating}). Then, to address these limitations, we introduce a new approach: position-aware edge attribution patching (PEAP; \S\ref{sec:full_circ_discovery}; Figure~\ref{fig:p1}). Current approaches  assume that if an edge is in a circuit, then the same edge will be in the circuit at all positions, thus leading to low precision. It is also assumed that an edge's importance should be aggregated across positions before deciding whether it should be included in the circuit; this can lead to cancellation effects, and thus low recall. PEAP instead allows us to compute the importance of cross-positional edges, and separately evaluates edge importance at each position. We show that this leads to smaller and more accurate circuits; see Figure~\ref{fig:p1}.

Incorporating positional information into circuit discovery is straightforward when inputs have the same length and structure across examples.

However, realistic datasets are not nearly this templatic.
How, then, can we incorporate positional information into automatic circuit discovery?
To address this challenge, we propose \textbf{schemas} (\S\ref{sec:schema}). 
Schemas assign semantic labels to spans of tokens, enabling information aggregation across examples even when the spans differ in length.

For example, in the input ``The \textcolor{positional}{war} lasted from 1453 to 14\underline{\hspace{1em}},'' the span ``\textcolor{positional}{war}'' could be labeled as ``\emph{Subject}''.
This enables handling spans with varying lengths: the phrase ``\textcolor{positional}{Black Plague}'' in another example can be treated as a single positional span with the same role as ``\textcolor{positional}{war}''.
In experiments with two LMs and three tasks, we find that circuits discovered using schemas achieve a better trade-off between circuit size and faithfulness to the model's behavior than position-agnostic circuits.
Importantly, position-aware circuits offer a more precise representation of the underlying mechanisms, providing a more concise foundation for mechanistic explanations.

We also present a fully automated pipeline for schema generation and application (\S\ref{sec:schema-generation}) using large language models (LLMs). 
We evaluate the quality of the generated schemas and their utility in discovering position-aware circuits (\S\ref{sec:schema-eval}).
Notably, circuits derived using automatically generated and applied schemas achieve comparable faithfulness scores to circuits discovered with human-designed and manually applied schemas.

We summarize our contributions as follows:
\begin{itemize}[noitemsep,leftmargin=*,topsep=1pt,parsep=1pt]
    \item Introduce a position-aware circuit discovery method, which obtains better faithfulness than position-agnostic discovery.  
    \item Introduce dataset schemas,  facilitating positional circuit discovery in more naturalistic settings. 
    \item Develop an automated schema generation and application pipeline with LLMs, yielding schemas that are comparable to manually-annotated ones.
\end{itemize}

\section{Related Work}

\subsection{Penetration Depth Computation}

The computation of penetration depth often utilizes the Minkowski sum, a well-regarded algorithm documented in Dobkin et al.'s work~\cite{dobkin1993computing}.
This method shows high efficacy for convex shapes, where the simplicity of the objects allows for accurate and computationally efficient penetration depth calculations~\cite{dobkin1993computing,varadhan2004accurate,hachenberger2009exact}.
However, applying this algorithm to concave shapes significantly increases computational complexity.  
As a result, research has focused on developing methods to approximate penetration depth more efficiently for these shapes~\cite{cameron1997enhancing,bergen1999fast,lien2010simple,je2012polydepth}.  

Beyond the Minkowski sum, other methods have been explored, including techniques such as utilizing distance fields or the Hausdorff distance for penetration depth calculations~\cite{fisher2001fast,sud2006fast,SIG09HIST}.

Tang et al.\cite{SIG09HIST} devised an efficient algorithm for calculating the Hausdorff distance between two objects within a given error bound.
They also demonstrated that the proposed algorithm can accelerate penetration depth computation by focusing on the Hausdorff distance in overlapping regions of objects.
Building upon Tang et al.'s method, Zheng et al.\cite{zheng2022economic} improved performance using a BVH-based framework with a four-point strategy.
This method has achieved a performance improvement of up to 20 times compared to Tang et al.'s technique~\cite{SIG09HIST}.
\revision{A common feature of these works, known as the culling-based method, is computing bounds for the Hausdorff distance and reducing the search space.}

\revision{Although culling-based methods have demonstrated significant performance gains, they face challenges in leveraging parallel hardware.  
Updating and sharing bounds require synchronization, which is not well-suited for massively parallel processing architectures such as GPUs.}

\revision{In this work, we propose a GPU-based penetration depth algorithm that specifically accelerates two key processes using RT core technology:  
(1) detecting the overlapping volume and (2) calculating the Hausdorff distance.  
To highlight the effectiveness of our approach, we also implemented a CPU-based penetration depth algorithm based on Tang et al.~\cite{SIG09HIST} and Zheng et al.~\cite{zheng2022economic} for performance comparison.}

%In this work, we propose a GPU-based penetration depth algorithm, specifically accelerating two key processes with RT core technology:  
%first, detecting the overlapping volume; and second, calculating the Hausdorff distance.  
%To highlight our method's effectiveness, we also implemented a CPU-based penetration depth algorithm based on Tang et al.~\cite{SIG09HIST} and Zheng et al.~\cite{zheng2022economic} for performance comparison.  

%utilize a Hausdorff distance-based method for penetration depth calculation, accelerating two key processes with RT core technology: 

%A notable development in this area is the work of Tang et al., who devised algorithms for the rapid calculation of the Hausdorff distance between two objects~\cite{SIG09HIST}.
%Their approach is geared towards efficient penetration depth calculation by focusing on the Hausdorff distance in overlapping object regions.


%One of the algorithms for calculating penetration depth is the Minkowski sum.\cite{dobkin1993computing} The Minkowski sum is useful to compute penetration depth between two convex objects because they have a simple shape so the Minkowski sum can calculate accurate penetration depth with low computational complexity~\cite{dobkin1993computing,varadhan2004accurate,hachenberger2009exact}.
%However, applying the Minkowski sum in cases involving concave objects is challenging due to higher computational complexity. As a result, prior research has focused on quickly computing an approximate penetration depth in these scenarios~\cite{cameron1997enhancing,bergen1999fast,lien2010simple,je2012polydepth}.

%Instead of the Minkowski sum method, there have also been attempts to calculate the penetration depth based on the distance field or the vertices that make up the objects~\cite{fisher2001fast,sud2006fast,SIG09HIST}. Tang et al.~\cite{SIG09HIST} proposed the algorithms that compute the Hausdorff distance between two objects quickly and showed that can be computed penetration depth to fast by calculating the Hausdorff distance for the overlapping area of two objects.

%In this paper, the proposed method is based on Tang's methods~\cite{SIG09HIST}, and then partially divided into steps detecting overlapping volume step and the Hausdorff distance step. These two steps accelerated with RT core.

\subsection{Ray-Tracing Core-Based Acceleration}

\revision{Recent advancements in GPU technology have led to the integration of dedicated ray-tracing cores (RT cores), enabling hardware-accelerated ray tracing.
These cores optimize intersection checks between rays and objects, allowing for efficient ray-bounding box and ray-triangle intersection tests.
To utilize RT cores, various frameworks such as DXR, OptiX~\cite{parker2010optix}, and Vulkan have been developed.
RT cores primarily accelerate ray intersection tasks by efficiently traversing acceleration hierarchies.}

%The Ray-Tracing Core (RT-core) is NVIDIA’s specialized hardware for accelerating ray tracing.
%Integrated into RTX GPUs like the GeForce RTX series

%\revision{Notably, OptiX~\cite{parker2010optix} is an NVIDIA-supported SDK.
%The ray-tracing core primarily facilitates two tasks: building an acceleration hierarchy and executing ray intersection tasks with traversal.}
%OptiX operates by launching a CUDA kernel and invoking a ray generation ($ray_{gen}$) shader.
%Each CUDA core thread makes requests to the ray-tracing core, which then executes appropriate shaders like intersection ($IS$), miss($miss_{hit}$), closest hit($closest$), and any hit($any_{hit}$).
%Consequently, OptiX enables access to the results of ray-primitive intersection tests.

While the core purpose of ray-tracing cores is to expedite ray tracing, recent studies have explored their application beyond this traditional scope~\cite{wald2019rtx,zhu2022rtnn,thoman2022multi,nagarajan2023rt,meneses2023accelerating,morrical2023attribute}.
Wald et al.~\cite{wald2019rtx} addressed the problem of locating points within tetrahedra using ray-tracing cores.
Zhu et al.~\cite{zhu2022rtnn} introduced a K-Nearest Neighbor (K-NN) algorithm utilizing ray-tracing cores, achieving performance improvements of 2.2 to 65.0 times compared to previous GPU-based nearest neighbor search algorithms.
Thoman et al.~\cite{thoman2022multi} employed RT cores for Room Impulse Response (RIR) simulation.
Nagarajan et al.~\cite{nagarajan2023rt} implemented RT core-based DBSCAN clustering, reporting up to 4 times higher performance enhancement.
Meneses et al.~\cite{meneses2023accelerating} proposed RT core-based Range Minimum Query (RMQ) algorithms, yielding performance up to 2.3 times faster than existing RMQ methods.

\revision{
For collision detection between objects, one of the fundamental proximity queries, researchers have explored ray-tracing approaches even before the introduction of RT-core technology.
Hermann et al.\cite{hermann2008ray} proposed ray-tracing-based collision detection methods for deformable bodies.
Youngjun et al.\cite{kim2010mesh} applied Hermann's idea to medical simulation.
Lehericey et al.\cite{lehericey2015gpu} introduced GPU ray-traced collision detection algorithms for cloth simulation.
Recently, these approaches have been extended to utilize RT cores, as demonstrated by Sui et al.\cite{sui2024hardware}, who proposed discrete and continuous collision detection algorithms using ray-tracing cores.
Unlike these works, which focus on determining when and where collisions occur, our work focuses on calculating penetration depth.
}

In line with these advancements, this study uniquely applies RT-core technology to compute penetration depth, diverging from traditional ray-tracing applications and thereby contributing a novel approach to this field.

%\subsection{Collision detection with Ray-tracing}

%\YW{There have been attempts to apply the ray tracing approaches for collision detection~\cite{hermann2008ray, kim2010mesh, lehericey2015gpu}. Hermann et al~\cite{hermann2008ray} proposed ray tracing collision detection methods for deformable bodies. Youngjun et al~\cite{kim2010mesh} apply Hermann's idea for Medical simulation. Lehericey et al~\cite{lehericey2015gpu} introduced GPU ray-traced collision detection algorithms for cloth simulation.
%However, these methods proposed deformable objects, not solid- or discrete- objects, and there is no report about the result using ray tracing core yet. Therefore, our research implements the penetration depth algorithm with ray tracing methods and reports the benefit of ray tracing core.}

%\YW{Sui et al~\cite{sui2024hardware} proposed the method for discrete and continuous collision detection with ray tracing core. They generate the ray candidate as much as the edge of the source mesh and investigate the intersections to solve discrete collision detection. And also, to solve continuous collision detection, they build sphere-swept volumes with OptiX B-Spline curves using continuous trajectory points that are pre-computed and trace the ray samely. However, their implementation only considers non-penetrating collision, and because of that reason, there need for other approaches to compute penetration cases.}

%\YW{To address this issue, our approacthe has propose the methods to find penetration surface with RT core (that called RT-PPE). Not only that, our methods report the penetration depth as computing the Hausdorff distance between the penetration surface.}

%Recently, modern GPU embedded ray tracing core for hardware accelerated ray tracing.

%To access the ray tracing core, we can use DXR, OptiX~\cite{parker2010optix}, and Vulkan.
%Above all, OptiX~\cite{parker2010optix} is NVIDIA NVIDIA-supported SDK. The ray tracing core actually works about two tasks. One is a built acceleration hierarchy, and another is ray intersection task with traversal. Therefore OptiX launches one CUDA kernel and called $ray\_gen$ shader. Each CUDA core thread requests to ray tracing core, and then ray tracing core executes a suitable shader such as $IS$, $miss\_hit$, $closest$, $any\_hit$ shader.
%Finally, we can access ray-primitive intersection test results using OptiX shader.

%While the ray tracing core is designed for accelerating ray tracing, recent research tried using the ray tracing core for other purposes~\cite{wald2019rtx,zhu2022rtnn,thoman2022multi,nagarajan2023rt,meneses2023accelerating,morrical2023attribute}.
%%Beyond ray tracing
%Wald et al~\cite{wald2019rtx} solved the point in location of tetrahedron problem using ray tracing cores.
%%RTNN
%Zhu et al~\cite{zhu2022rtnn} proposed K-NN(K-Nearest Neighbor) algorithms using ray tracing cores. They achieved a performance of 2.2-65.0 times faster than prior GPU-based nearest neighbor search algorithms.
%%RIR Simulation
%Thoman et al~\cite{thoman2022multi} utilized the RT core to RIR(Room impulse response) simulation,
%%RT-DBSCAN
%Nagarajan et al~\cite{nagarajan2023rt} implemented DBSCAN clustering with RT core and achieved performance up to 4x times.
%%RTX-RMQ
%Meneses et al~\cite{meneses2023accelerating} proposed RT core-based RMQ(Range minimum query) algorithms, and they got performance up to 2.3x than state-of-the-art RMQ algorithms.

%%
%Similar to prior research, this study is distinguished by utilizing RT-core for computing penetration depth, as opposed to conventional ray tracing problems.



\section{Problem Statement}\label{sec:threatmodel}

 


Federated learning (FL) systems face significant privacy risks from malicious servers. Even an "honest-but-curious" server, while adhering to the FL protocol, can attempt to infer sensitive client information by analyzing aggregated model updates, potentially revealing private data points, patterns, or client identities. A key privacy threat is a two-stage attack: Membership Inference Attack (MIA) followed by Source Inference Attack (SIA).

\begin{itemize}[noitemsep, topsep=0pt]
    \item \textbf{MIA:} The server determines if a specific data point $x$ was used to train the global model $\theta_g$: MIA($\theta_g$, $x$) = P($x \in D_{\theta_g}$), where P($x \in D_{\theta_g}$) is the probability that $x$ belongs to the training data $D_{\theta_g}$.
    \item \textbf{SIA:} If the MIA suggests $x$ was part of the training data, the server identifies the contributing client $i$: SIA($\theta_i$, $x$) = P(Client$_i$ | $x$, $\theta_i$), where P(Client$_i$ | $x$, $\theta_i$) is the probability that client $i$ contributed $x$ to the model $\theta_i$.
\end{itemize}

As shown by Hongsheng et al. \cite{BG_SIA_2}, combining these attacks can severely compromise client privacy. Moreover, prior work has shown the inherent limitations of auditing MIA \cite{chang2024efficient}. 

Our work focuses on the disparity in privacy risk across clients, which we attribute to differences in local overfitting during training. This threat model underscores the need for equitable privacy mechanisms in FL. 

Given this threat model %in Figure \ref{fig:threatmodel}, 
where a compromised server enables SIA attacks, our objective is twofold:

\begin{enumerate}[noitemsep, topsep=0pt, start=1,label={(\bfseries O\arabic*):}]
    \item Addressing the symptoms: Develop an aggregation method on the server side to ensure fair privacy risk distribution among clients.
    \item Addressing the causes: Provide feedback to leaking clients, enabling them to adjust local updates to reduce overfitting and improve system fairness in privacy.
\end{enumerate}

Instead of eliminating SIA, we aim to mitigate its impact by equitably distributing the inherent privacy risk. We therefore assume a \textit{compromised} server and cooperative clients capable of tuning their local updates to enhance \sysname.


\section{Fairness-in-Privacy Framework in Federated Learning}\label{sec:finpmetric}

\begin{figure*}[!t]
\centering
\includegraphics[trim={0 9cm 3cm 0},clip,width=0.8\linewidth]{fig/framework.pdf}
\caption{Fairness in Privacy \sysname\ framework in federated learning. The framework addresses the causes and the symptoms to achieve \sysname.}
\label{fig:framework}
\end{figure*}


This section presents our framework, \sysname, designed to improve fairness in privacy within federated learning (FL), particularly in the context of Source Inference Attacks (SIAs). Our core principle is that privacy risks should be distributed equitably among all participating clients, preventing any single client from bearing a disproportionate burden. 
 

This disparity in privacy risks among clients can arise from various factors, including heterogeneous data distributions, varying computational resources, and differences in local training dynamics. Simply preventing average privacy leakage is insufficient; we must ensure that no individual client bears a disproportionate risk. This motivates our focus on fairness-in-privacy, which aims to equitably distribute privacy risks across all participating clients.

An overview of the \sysname\ framework is shown in~\autoref{fig:framework}. We argue that addressing fairness in privacy requires a two-pronged approach: handling it both at the server (during aggregation) and at the client (during local training). Server-side interventions, specifically adaptive aggregation, are crucial to mitigating the impact of existing disparities in privacy leakage. By carefully weighing client updates based on their estimated privacy risk, we can prevent highly vulnerable clients from unduly influencing the global model and further exacerbating the unfairness. However, server-side interventions alone are insufficient. They address the *symptoms* of unfairness but not the underlying *causes*.

The root cause of privacy disparity often lies in differences in local training dynamics, particularly local overfitting. When a client's model overfits its local data, it becomes more susceptible to privacy attacks, such as Source Inference Attacks (SIAs). Therefore, we also address fairness in privacy on the client side by introducing a collaborative overfitting reduction strategy. This strategy aims to proactively reduce the likelihood of local overfitting, thereby minimizing the initial disparity in privacy risks before aggregation. By ranking clients based on their estimated relative overfitting and incorporating this rank into a local regularization scheme, we encourage clients to learn more generalizable representations, reducing their vulnerability to the disparity in privacy leakage.

This two-pronged approach, combining adaptive aggregation at the server and collaborative overfitting reduction at the client, provides a comprehensive framework for achieving fairness in privacy in FL. By minimizing both the symptoms and the root causes of privacy disparity, our aim is to create a more equitable and robust FL system. This can be formalized in \autoref{eq:finp}. 

\begin{align}\label{eq:finp}
\text{F}in\text{P} &= \min (\text{Symptoms}, \text{Causes}) \nonumber \\ 
&= \text{F}in\text{P}_\text{server} + \text{F}in\text{P}_\text{client}
\end{align}


\subsection{Formalizing Symptoms of Fairness in Privacy on Server Side}

We formalize the fairness in privacy problem as follows: Given an FL system with $K$ clients and a global model $\theta_g$, our goal is to achieve fair privacy risk across all clients against successful SIAs.

We consider the privacy risk $p_k(\mathbf{w})$ for client $k$ to be influenced by the aggregation weights $\mathbf{w} = [w_1, w_2, ..., w_K]$, where $w_k$ represents the weight assigned for the client $k$, with the constraint $\sum_{k=1}^{K} a_k = 1$. This allows us to account for the varying client contributions to the global model.

We define Fairness in Privacy (F$in$P) as minimizing the variance in privacy risks across clients. Our objective is to find the optimal weights for aggregation $\mathbf{w}$ that minimize the difference between individual client privacy risks and the average privacy risk. This is expressed in \autoref{eq:finpserver} as:


\begin{align}\label{eq:finpserver}
    \text{F}in\text{P}_\text{server} = \min_{\mathbf{w}\in \mathcal{W}} \| \mathbf{p}(\mathbf{w}) - \frac{1}{K} \mathds{1}^T \mathbf{p}(\mathbf{w}) \otimes \mathds{1}  \| + \|\frac{1}{K} \mathds{1}^T \mathbf{p}(\mathbf{w})\|,
    %\vspace{-2mm}
\end{align}

Where:

\begin{itemize}
    \item $\mathbf{p}(\mathbf{w}) = [p_1(\mathbf{w}),\dots, p_K(\mathbf{w})]^T$ is the vector of privacy risks for all clients given the aggregation weights $\mathbf{w}$.
    \item $\mathds{1}$ is a vector of ones of length $K$.
    \item $\mathcal{W} = \{\mathbf{w} \in \mathbb{R}^K \mid \sum_{k=1}^{K} w_k = 1, w_k \geq 0 \ \forall k\}$ is the set of valid aggregation weights.
\end{itemize}

The term $\frac{1}{K} \mathds{1}^T \mathbf{p}(\mathbf{w})$ represents the average privacy risk. Equation \eqref{eq:finpserver} minimizes the Euclidean distance between individual privacy risks and this average, thus minimizing the disparity in privacy risks. Intuitively, we seek optimal aggregation weights to achieve a more equitable distribution of privacy risk, ensuring no client is disproportionately exposed.


We hypothesize that differences in local overfitting are a primary cause of unequal privacy leakage among FL clients. When a client's model overfits to its local data, it effectively memorizes sensitive information, making it more vulnerable to SIAs and leading to an unfair distribution of privacy risk.

We quantify the *symptoms* of overfitting in the server by measuring the discrepancy between each client's local model update and the global model using the Principle Component Analysis (PCA) distance \cite{durmus2021federated}. For client $k$, this distance, denoted as $p_k$, serves as a proxy for privacy risk; a larger $p_k$ signifies a symptom of greater overfitting and, thus, higher risk.

Our proposed adaptive aggregation method aims to balance client contributions based on these PCA distances. By minimizing the variance of $p_k$ using the \sysname$_{server}$ objective (\autoref{eq:finpserver}), we reduce the influence of clients exhibiting high overfitting (high $p_k$) and increase the influence of those with lower overfitting. This dynamic adjustment, performed in each FL round, promotes a more equitable distribution of privacy risk. However, adaptive aggregation alone is insufficient to eliminate overfitting entirely and achieve full fairness in privacy; collaborative client-side adjustments are also required, as will be explained in \autoref{sec:client_side_adjustments}.




\subsection{Formalizing Causes of Fairness in Privacy on Client Side}\label{sec:client_side_adjustments}

To further mitigate local overfitting (*causes*) and enhance fairness in privacy, we propose a collaborative client strategy. This leverages the principle that clients with higher overfitting benefit from more diverse data.

The top Hessian eigenvalue ($\lambda_{\text{max}}$) and Hessian trace ($H_{T}$) have been identified as important metrics for characterizing the loss landscape and generalization capabilities of neural networks~\cite{Jiang2020Fantastic}. Lower values of $\lambda_{\text{max}}$ and $H_{T}$ typically indicate improved robustness to weight perturbations, leading to smoother training and better convergence. This is especially critical in FL, where the non-IID nature of data across clients creates distributional shifts that can exacerbate training instability and introduce fairness concerns. These distributional shifts can disproportionately impact certain client groups, leading to biased model performance~\cite{mendieta2022local}. 


As we are interested in \sysname, we determine each client's relative overfitting by calculating the average pairwise difference across the top Hessian eigenvalue ($\lambda_{\text{max}}$) and Hessian trace ($H_{T}$):

\begin{align}
\begin{split}
\bar{\Delta}_k &= \frac{1}{K-1} \sum_{j=1, j\neq k}^{K} |\lambda_{\text{max}}^k - \lambda_{\text{max}}^j|, \\    \bar{H}_k &= \frac{1}{K-1} \sum_{j=1, j\neq k}^{K} |H_T^k - H_T^j|, \\  \rho_k &= \frac{\frac{\bar{\Delta}_k}{\max{\bar{\Delta} }} + \frac{\bar{H}_k}{\max{\bar{H} }} }{2},
\end{split} \label{eq:hessian}
\end{align}

where $\lambda_{\text{max}}^k$ and $\lambda_{\text{max}}^j$ are the top Hessian eigenvalue of the local models of clients $k$ and $j$, respectively.  Similarly, $H_T^k$, and $H_T^j$ are the Hessian trace of the local models of clients $k$ and $j$, respectively. We used the normalized average of both $\bar{\Delta}_k$ and $\bar{H_k}$ to quantify the client's \textit{overfitting relative rank} ($\rho_k$), to serve as a proxy for relative privacy leakage risk. Computing the Hessian eigenvalue and trace are done on the cloud server, and hence, there is no overhead of their computation on the client.


We incorporate this overfitting rank into the local training process using a regularization term based on the Lipschitz constant, approximated by the spectral norm of the Jacobian matrix ($||J_k||$)~\cite{liu2020simple}. In particular, a smaller Lipschitz constant implies smoother functions, less prone to overfitting, and better generalization. The modified local loss function for client $k$ is:

\begin{align}\label{eq:finpclient}
\mathcal{L}_k' &= \mathcal{L}_k + \beta \cdot \rho_k \cdot ||J_k||, \nonumber \\
\text{F}in\text{P}_\text{client} &= \min_{\theta_k} {\mathcal{L}_k'}
\end{align}

where:

\begin{itemize}
    \item $\mathcal{L}_k$ is the original local loss function.
    \item $\rho_k$ is an adaptive controlling regularization strength that depends on the overfitting rank.
    \item $\theta_k$ are the local parameters of the client model that minimize the total loss $\mathcal{L}'_k$
    \item  $\beta$ is the impact factor, which controls the impact of the Lipschitz constant based on the learning task.
\end{itemize}

This penalizes models with large Lipschitz constants, promoting generalization. The regularization strength is weighted by $\rho_k$ adaptively at each round, applying stronger regularization to clients with higher overfitting ranks. This collaborative approach, using $\rho_k$ to guide local training, preserves privacy while promoting equitable learning and reducing disparity in privacy risk. 

$\beta$ is a task-dependent parameter to balance the the loss $\mathcal{L}_k$ and Lipschitz loss. A larger $\beta$ greatly impacts fairness regularization but could make training unstable and fail to converge. $\beta$ is a trade-off parameter between fairness and accuracy while $\rho_k$ changes at every round to control regularization strength adaptively.











\section{Evaluation}\label{sec:eval}



\subsection{Federated Learning System Setup}

\paragraph{\textbf{Setup for Human Activity Recognition}} We utilized the UCI \ac{har} Dataset \cite{human_activity_recognition_using_smartphones_240}, a widely used dataset in activity recognition research, especially in FL~\cite{har2,har3}.
The dataset includes sensor data from 30 subjects (aged 19–48) performing six activities: walking, walking upstairs, walking downstairs, sitting, standing, and laying. The data was collected using a Samsung Galaxy S II smartphone worn on the waist, capturing readings from both the accelerometer and gyroscope sensors. Each subject in the dataset was treated as an individual client in the \ac{fl} setup, preserving the data's unique activity patterns and non-IID nature. We allocated 70\% of each client's data for training using 5-fold cross-validation and 30\% for testing, enabling evaluation of the model on independently collected test data. Data preprocessing involved applying noise filters to the raw signals and segmenting the data using a sliding window approach with a window length of 2.56 seconds and a 50\% overlap, resulting in 128 readings per window. We selected the HAR dataset for evaluation \sysname due to its inherited non-IID structure. 

We trained the model in a federated learning setting using the \ac{fedavg} aggregation method over 20 global communication rounds. Each client trained locally with a batch size of 64, 5 local epochs per round, a learning rate of 0.001 using Adam optimizer, and an impact factor $\beta$ of 1. These parameters ensured balanced model updates from each client while maintaining computational efficiency across the federated network. Each local model (one per subject) analyzes its time-series sensor data using \ac{tcn} model\cite{bai2018empirical}. 
The TCN model, designed for time-series data, uses causal convolutions to capture temporal dependencies while preserving sequence order. The architecture includes two convolutional layers, each followed by max-pooling and dropout, with a final fully connected layer for classifying the six activity classes.



\paragraph{\textbf{Setup for CIFAR-10}}
The CIFAR-10 dataset consists of 60000 32x32 color images in 10 classes, with 6000 images per class. There are 50000 training images and 10000 test images. We use the Dirichlet distribution $Dir(\alpha)$ to divide the CIFAR-10 dataset into $K$ unbalanced subsets similar to previous work in the literature ~\cite{mendieta2022local,BG_SIA_2}, with $\alpha=0.1$. \autoref{fig:CIFAR data} shows how the data are distributed among clients. We created 10 clients and employed ResNet56~\cite{he2016deep} as the model. Similar to the setup in HAR, we trained the model over 20 global communication rounds. Each client is trained locally with the same parameters in HAR and an impact factor $\beta$ of 0.05. A smaller $\beta$ is used here since CIFAR-10 is a more complicated task than HAR, and a smaller $\beta$ can make the model converge easier since the model is more sensitive to classification loss.



\paragraph{\textbf{SIA Attack}} We used the Source Inference Attacks (SIA) setup explained in ~\cite{hu2023source}, where we randomly sampled training data from each client dataset. We combined those samples in one dataset and used them as target records. This is a valid assumption, given an already successful Membership Inference Attacks (MIA) attack. SIA attacks in Federated Learning represent a privacy threat beyond Membership Inference Attacks (MIA). While MIAs determine whether a data instance was used for training, SIA aims to identify the specific client who owns that training record. In a practical scenario, an adversary, such as an honest-but-curious central server who knows the clients' identities and receives their model updates, could leverage this knowledge to trace training data back to its source, thus compromising client privacy. To launch SIA in FL setting,  clients send their updated local model parameters to the server. The server uses each client's model to calculate the prediction loss on the target record. The client with the smallest loss is identified as the most probable source of that target record. This approach exploits the differences in model performance on the target record to infer its origin.


\begin{figure*}
\centering
\includegraphics[width=0.8\linewidth]{fig/CIFAR/Fedalign/Data_profile.png}
\caption{CIFAR dataset profile for each client after Dirichlet sampling with $\alpha=0.1$}
\label{fig:CIFAR data}
\end{figure*}



\subsection{Metrics for Comparison}\label{sec:metrics}
Recent work in the literature suggests SIA vulnerability wherein an adversary can potentially identify the origin of a specific record can be achieved by analyzing the prediction loss of individual client models~\cite{hu2023source}. In particular, SIA exploits the observation that the client model %$C_{min}$ 
exhibiting the lowest prediction loss for a given record is most likely to be the source of that record. We assess our \sysname approach in achieving fairness in privacy using the following metrics:

 
\paragraph{\textbf{1- Reduction of SIA accuracy disparity among clients} } \sysname\  aims to reduce the SIA accuracy disparity among clients. A balanced SIA accuracy across clients indicates a more equitable distribution of privacy risk within the FL system. \sysname enhances the overall fairness in privacy by ensuring that no particular client is significantly more vulnerable to source inference attacks than others. 

To assess the fairness of risk of SIA accuracy across clients in our FL system, we employ the Coefficient of Variation (CoV). Recognizing that fairness is related to the variance of shared utility rather than strict equality \cite{jain1984quantitative}, we adapt the CoV to measure the dispersion of SIA accuracy among clients.

For K clients, we define the SIA accuracy for client $i$ as $\text{SIA}_i$. The mean SIA accuracy ($\mu$) is calculated as
$\mu = \frac{1}{K} \sum_{i=1}^{K} \text{SIA}_i$. The CoV of SIA accuracy \texttt{CoV(SIA)} is then computed as:

\begin{align}\label{eq:covsia}
\texttt{CoV(SIA)} = \frac{\sigma}{\mu} = \frac{\sqrt{\frac{1}{K} \sum_{i=1}^{K} (\text{SIA}_i - \mu)^2}}{\mu},
\end{align}

where $\sigma$ is the standard deviation of SIA accuracies. A lower CoV indicates a more equitable distribution of SIA accuracy across clients, suggesting greater fairness in privacy. To facilitate interpretation as a fairness percentage between $0$ and $1$ (where 1 represents perfect fairness), we use the following Fairness Index (\texttt{FI(SIA)}) transformation:

\begin{align}\label{eq:fisia}
\texttt{FI(SIA)} = \frac{1}{1 + \texttt{CoV(SIA)}} 
\end{align}

A FI value of $1$ indicates perfect fairness (all clients have the same SIA accuracy), while lower FI values indicate increasing disparities in SIA accuracy among clients.









\paragraph{\textbf{2- Reduction of SIA confidence disparity among clients}} Beside reduction of SIA accuracy disparity among clients, as the SIA approach relies on identifying the client model with the minimum prediction loss. When a significant discrepancy exists between the prediction losses of different client models, an attacker can make source inferences with higher confidence. \sysname\  aims to reduce inter-client loss differences so that it can diminish the effectiveness of SIA attacks by lowering the attacker's confidence in their inferences. We evaluate the SIA confidence disparity by using  the prediction loss \texttt{CoV(Loss)} and \texttt{FI(Loss)} similarly to \autoref{eq:covsia} and \autoref{eq:fisia}. 



\paragraph{\textbf{3- Success rate of SIA}}
\sysname\ aims to reduce disparities in both SIA success rate and prediction loss across clients. However, simply reducing disparity is insufficient; it is crucial to avoid achieving this by merely increasing the SIA success rate of less vulnerable clients to match that of the most vulnerable ones. Such an outcome would not represent a genuine improvement in privacy. Therefore, we evaluate the overall impact of \sysname\ on SIA vulnerability using two metrics: \texttt{Mean(SIA)} and \texttt{Max(SIA)}. In particular, \texttt{Mean(SIA)} represents the average SIA success rate across all clients and communication rounds, while \texttt{Max(SIA)} indicates the highest SIA success rate observed across all clients and rounds. Lower values for both metrics signify increased resilience against SIA attacks.

\paragraph{\textbf{4- Accuracy Metric}}

Accuracy is calculated over all test dataset points for all clients using the formula:

\begin{equation}
    \text{Accuracy} = \frac{\sum_{i=1}^{N} \mathbf{1}(\hat{y}_i = y_i)}{N}
\end{equation}

where:
\begin{itemize}
    \item $N$ is the total number of test samples across all clients,
    \item $\hat{y}_i$ is the predicted label for the $i$-th test sample,
    \item $y_i$ is the true label of the $i$-th test sample.
\end{itemize}

We evaluated \sysname\ through two distinct case studies, using the Human Activity Recognition (HAR) dataset (\autoref{sec:HAReval}) and the CIFAR-10 image classification dataset (\autoref{sec:CIFAReval}).  For HAR, we compared four approaches: (1) a Baseline Federated Learning (FL) implementation using FedAvg, adapted from \cite{hu2023source}; (2) \sysname$_\text{server}$, which applies adaptive aggregation at the server without client collaboration (\autoref{eq:finpserver}); (3) \sysname$_\text{client}$, which employs client-side collaboration to mitigate relative overfitting but omits adaptive server aggregation (Equation \ref{eq:finpclient}); and (4) the full \sysname\ approach, incorporating both \sysname$_\text{server}$ and \sysname$_\text{client}$ (\autoref{eq:finp}). 

In the CIFAR-10 evaluation, we compared three approaches: (1) the same Baseline FL using FedAvg from \cite{hu2023source}; (2) FedAlign \cite{mendieta2022local}, a state-of-the-art FL method designed to address data heterogeneity in CIFAR-10; and (3) the full \sysname\ approach, again incorporating both \sysname$_\text{server}$ and \sysname$_\text{client}$ (\autoref{eq:finp}).






\subsection{\sysname\ Performance on HAR}\label{sec:HAReval}

\begin{figure}[!t]
  \centering
  \begin{subfigure}[b]{0.99\columnwidth}
    \centering
    \includegraphics[width=0.8\textwidth]{fig/HAR/line_CoV_SIA.png}
    \caption{Coefficient of variation for SIA accuracy \texttt{CoV(SIA)}.}
    \label{fig:SIACV}
  \end{subfigure}
  \hfill
  \begin{subfigure}[b]{0.99\columnwidth}
    \centering
    \includegraphics[width=0.8\textwidth]{fig/HAR/line_FI_SIA.png}
    \caption{Fairness index of SIA accuracy \texttt{FI(SIA)}.}
    \label{fig:SIAFI}
  \end{subfigure}
  \caption{Disparity of SIA accuracy among clients using HAR dataset. }
  \label{fig:SIA}
\end{figure}



\begin{figure}[!t]
  \centering
  \begin{subfigure}[b]{0.99\columnwidth}
    \centering
    \includegraphics[width=0.8\textwidth]{fig/HAR/line_loss_CV.png}
    \caption{Coefficient of variation for the prediction loss \texttt{CoV(Loss)}.}
    \label{fig:lossCV}
  \end{subfigure}
  \hfill
  \begin{subfigure}[b]{0.99\columnwidth}
    \centering
    \includegraphics[width=0.8\textwidth]{fig/HAR/line_loss_FI.png}
    \caption{Fairness index of the prediction loss \texttt{FI(Loss)}.}
    \label{fig:lossFI}
  \end{subfigure}
  \caption{Disparity of prediction loss among clients using HAR dataset. }
  \label{fig:loss}
\end{figure}


\begin{figure}
  \centering
  \begin{subfigure}[b]{0.99\columnwidth}
    \centering
    \includegraphics[width=0.8\textwidth]{fig/HAR/line_average_train_accuracy.png}
    \caption{Global model training accuracy.}
    \label{fig:accCV}
  \end{subfigure}
  \hfill
  \begin{subfigure}[b]{0.99\columnwidth}
    \centering
    \includegraphics[width=0.8\textwidth]{fig/HAR/line_average_test_accuracy.png}
    \caption{Global model testing accuracy.}
    \label{fig:accFI}
  \end{subfigure}
  \caption{Global model classification accuracy using HAR dataset.}
  \label{fig:accu}
\end{figure}






\begin{figure}
  \centering
  \begin{subfigure}[b]{0.99\columnwidth}
    \centering
    \includegraphics[width=0.8\textwidth]{fig/HAR/line_CoV_PCA_d.png}
    \caption{Coefficient of Variation of the PCA distance to the global model (\texttt{PCA$_d$}). }
    \label{fig:har_pca_d_cov}
  \end{subfigure}
  \hfill
  \begin{subfigure}[b]{0.99\columnwidth}
    \centering
    \includegraphics[width=0.8\textwidth]{fig/HAR/line_FI_PCA_d.png}
    \caption{Fairness Index of PCA distance to the global model (\texttt{PCA$_d$}). }
    \label{fig:har_pca_d_fi}
  \end{subfigure}
  \caption{Disparity of PCA distance between the global model and the client models using HAR dataset. }
  \label{fig:pca}
\end{figure}









\subsubsection{\textbf{Impact on the disparity of SIA accuracy among clients}}
Our results demonstrate a significant improvement in fairness with minimal impact on overall performance. \autoref{fig:SIA} presents the Coefficient of Variation of SIA accuracy (\texttt{CoV(SIA)}) and Fairness Index of SIA accuracy (\texttt{FI(SIA)}), as defined in  \autoref{eq:covsia} and \autoref{eq:fisia}, respectively.  \sysname\ achieves a \texttt{CoV(SIA)} of $0.596$ and a \texttt{FI(SIA)} of $0.739$, compared to the Baseline's \texttt{CoV(SIA)} of $0.893$ and \texttt{FI(SIA)} of $0.617$. This represents a substantial reduction of $33.26\%$ in \texttt{CoV(SIA)} and a $19.77\%$ improvement in \texttt{FI(SIA)} compared to baseline, clearly indicating that \sysname\ significantly enhances the fairness of SIA accuracy.






\subsubsection{\textbf{Impact on the disparity of SIA confidence among clients}} 
Similarly, our results demonstrate improvement in fairness with respect to the SIA confidence in prediction among clients represented as \texttt{CoV(Loss)} and \texttt{FI(Loss)} as explained in \autoref{sec:metrics}. As shown in \autoref{fig:loss}, \sysname\ achieves a \texttt{CoV(Loss)} of $0.778$ and \texttt{FI(Loss)} of $62.8\%$. This represents a reduction of $10.95\%$ in \texttt{CoV(Loss)} and a $19.77\%$ improvement in \texttt{FI(Loss)} compared to the Baseline, clearly indicating that \sysname\ enhances the fairness of SIA confidence in prediction among clients.


\subsubsection{\textbf{Impact on SIA success rate}}
While \autoref{tbl:HAR_sia} shows a marginal increase of less than $1\%$ in \texttt{Mean(SIA)} success rate and less than $0.1\%$ in \texttt{Max(SIA)} success rate, these gains are secondary to the primary objective of fairness improvement.  The key achievement of \sysname\ is the demonstrably more equitable distribution of privacy protection.  \sysname\ achieves this by significantly improving the uniformity of SIA success rates and reducing SIA confidence across clients.

Furthermore, \sysname\ maintains competitive classification performance. As shown in \autoref{tbl:HAR_acc}, the global model's testing accuracy only decreases by a $1.02\%$.  This impact on accuracy is further supported by \autoref{fig:accu}, which demonstrates that \sysname\ converges at a comparable rate to the Baseline.  Therefore, \sysname\ effectively balances the critical need for fairness with the practical requirement of maintaining performance.

\subsubsection{Ablation study}




\begin{table}[!t]
\caption{SIA accuracy performance in HAR dataset.}
\label{tab:my-table}
\begin{tabular}{|c|l|l|}
\hline
\multicolumn{1}{|l|}{} & \texttt{Mean(SIA)}(\%) $\downarrow$ & \texttt{Max(SIA)}(\%) $\downarrow$   \\
\hline
Baseline~\cite{hu2023source} & \textbf{23.78} & 31.00   \\
\sysname$_{\text{server}}$ & 25.22 & 31.20 \\
\sysname$_{\text{client}}$ & 25.52 & \textbf{30.20}   \\
\sysname & 24.49 & 31.10 \\
\hline
\end{tabular}
 \label{tbl:HAR_sia}
\end{table}



\begin{table}[!t]
\caption{HAR experiment: global model classification accuracy.}
\label{tab:my-table}
\begin{tabular}{|c|l|l|}
\hline
\multicolumn{1}{|l|}{} &  Training (\%) & Testing (\%)   \\
\hline
Baseline~\cite{hu2023source} & \textbf{96.52} & \textbf{96.94}   \\
\sysname$_{\text{server}}$ & 95.79 & 95.77\\
\sysname$_{\text{client}}$ & 95.67 & 95.99   \\
\sysname & 95.05 & 95.92 \\
\hline
\end{tabular}
 \label{tbl:HAR_acc}
\end{table}








We conducted an ablation study to evaluate the individual contributions of \sysname's server-side and client-side components.  Isolating the server-side adaptive aggregation (\sysname$_\text{server}$) revealed a nuanced impact on fairness metrics.  While \sysname$_\text{server}$ reduced the variation in PCA distance (\texttt{PCA$_d$)} by $1.3\%$ (\autoref{fig:pca}), it also resulted in a slight shift in both \texttt{FI(SIA)} and \texttt{FI(Loss)} by $-0.2\%$ and $-1.3\%$, respectively (Figures \ref{fig:SIAFI} and \ref{fig:lossFI}). This suggests that server-side adaptation alone (\sysname$_\text{server}$) primarily influences the distribution of model distances and has a less direct impact on the fairness metrics themselves. This observation motivated the investigation of client-side factors, specifically the variation in overfitting among clients, to further enhance fairness.

Analysis of Hessian eigenvalues ($\lambda_{\text{max}}$) and trace ($H_{T}$) revealed a strong correlation (Spearman's rank correlation coefficient $\approx$ 1) between these two metrics, both indicative of how well a local model fits its local data (\autoref{fig:subHessian}).  Based on this correlation, these metrics were given equal weight in \autoref{eq:hessian}.  Focusing on mitigating client-side overfitting through \sysname$_\text{client}$ yielded significant improvements in fairness.  Figures \ref{fig:SIA} and \ref{fig:loss} demonstrate the substantial gains in both SIA accuracy and prediction loss fairness.  Specifically, \sysname$_\text{client}$ alone improved \texttt{FI(SIA)} by $16.37\%$ and \texttt{FI(Loss)} by $8.30\%$ (Figures \ref{fig:SIAFI} and \ref{fig:lossFI}).  Furthermore, combining \sysname$_\text{server}$ with \sysname$_\text{client}$ resulted in even greater fairness gains, with an additional $3.13\%$ improvement in \texttt{FI(SIA)} and $2.65\%$ in \texttt{FI(Loss)} compared to using \sysname$_\text{client}$ alone. This indicates that while \sysname$_\text{server}$'s primary effect is on model distance distribution, it contributes synergistically to the fairness improvements achieved by \sysname$_\text{client}$ when both are employed.


More results related to the adaptation of the aggregation weights $\mathcal{W}$ (\autoref{eq:finpserver}) and the regularization strength $\rho_k$ (\autoref{eq:finpclient}) are shown in \autoref{appendix:adaptation}. 

\begin{figure*}
\centering
\includegraphics[width=0.8\linewidth]{fig/HAR/scatter_eigenvalues_eigentrace.png}
\caption{Scatter figures for Hessian max eigenvalue ($\lambda_{\text{max}}$) and Hessian trace ($H_{T}$).  The figure shows the value of each clients Hessian max eigenvalue and trace in the Baseline method for HAR dataset from rounds 6 to 11. All the rounds are depicted in \autoref{appendix:Hessian}.}
\label{fig:subHessian}
\end{figure*}


\subsection{\sysname\ Performance in CIFAR-10 dataset} \label{sec:CIFAReval}
In the CIFAR-10 dataset, \sysname\ demonstrates a significant improvement in fairness in privacy, with competitive accuracy. \autoref{fig:CIFAR loss} shows the Fairness Index of prediction loss (\texttt{FI(Loss)}) for FedAvg, FedAlign, and \sysname\ are 68.8\%, 58.0\%, and 83.3\%, respectively.  \sysname\ achieves a substantial increase in \texttt{FI(Loss)} of 21.1\% compared to FedAvg and 43.6\% compared to FedAlign. Notably, despite employing a distillation technique, FedAlign failed to effectively mitigate SIA risks, exhibiting a higher \texttt{CoV(Loss)} of 0.862 compared to FedAvg's 0.674. This increased \texttt{CoV(Loss)} can empower attackers with greater confidence in predicting the source client, consequently leading to higher SIA success rates.

Although FedAlign and FedAvg exhibit similar Mean and Max SIA success rates (\autoref{tbl:CIFAR}), \sysname\ effectively mitigates these risks. As shown in \autoref{fig:sia overall} and \autoref{tbl:CIFAR}, \sysname\ reduces the \texttt{Mean(SIA)} success rate to 10.07\%, approaching the random-guess probability of 1/10 (10\%) for a 10-class classification task.  Specifically, \sysname\ reduces the \texttt{Mean(SIA)} success rate from 30.86\% to 10.07\% and the \texttt{Max(SIA)} success rate from 38.52\% to 10.67\%. As shown in \autoref{fig:CIFAR sia}, the \sysname\ demonstrates comparable \texttt{CoV(SIA)} and \texttt{FI(SIA)}, yet exhibits a substantial reduction in the average success rate of SIA as mentioned above.

Moreover, \sysname\ maintains and slightly improves classification accuracy.  Figure \ref{fig:accu cifar} shows that \sysname\ achieves a testing accuracy of 78.46\%, marginally higher than FedAvg's 77.62\%. This 0.84\% improvement is attributed to the global model's aggregation of generalized client models through Lipschitz regularization rather than models overfit to individual datasets.  In summary, \sysname\ effectively mitigates SIA privacy risks in FL training in CIFAR-10 by improving client generalization and reducing loss variation across client models, all while maintaining or slightly improving classification performance.








\begin{figure}[!t]
  \centering
  \begin{subfigure}[b]{0.99\columnwidth}
    \centering
    \includegraphics[width=0.8\textwidth]{fig/CIFAR/Fedalign/line_loss_CV.png}
    \caption{Coefficient of variation for the prediction loss \texttt{CoV(Loss)}.}
    \label{fig:CIFARCV}
  \end{subfigure}
  \hfill
  \begin{subfigure}[b]{0.99\columnwidth}
    \centering
    \includegraphics[width=0.8\textwidth]{fig/CIFAR/Fedalign/line_loss_FI.png}
    \caption{Fairness index of the prediction loss \texttt{FI(Loss)}.}
    \label{fig:CIFARFI}
  \end{subfigure}
  \caption{Disparity of prediction loss among clients using CIFAR-10 dataset.}
  \label{fig:CIFAR loss}
\end{figure}

\begin{table}[!t]
\caption{SIA accuracy performance in CIFAR-10 dataset with Resnet model.}
\label{tab:my-table}
\begin{tabular}{|c|l|l|}
\hline
\multicolumn{1}{|l|}{} & Mean SIA(\%) $\downarrow$ & Max SIA(\%) $\downarrow$   \\
\hline
Baseline~\cite{hu2023source} & 30.86 & 38.52   \\
FedAlign~\cite{mendieta2022local} & 30.72 & 38.46 \\
\sysname & \textbf{10.07} & \textbf{10.67} \\
\hline
\end{tabular}
\label{tbl:CIFAR}
\end{table} 



\begin{figure}[!t]
\centering
\includegraphics[width=0.8\columnwidth]{fig/CIFAR/Fedalign/line_SIA_Accuracy.png}
\caption{Average SIA accuracy across rounds in CIFAR-10 dataset. }
\label{fig:sia overall}
\end{figure}




\begin{figure}[!t]
  \centering
  \begin{subfigure}[b]{0.99\columnwidth}
    \centering
    \includegraphics[width=0.8\textwidth]{fig/CIFAR/Fedalign/line_CoV_SIA.png}
    \caption{Coefficient of variation for SIA accuracy \texttt{CoV(SIA)}.}
    \label{fig:CIFARCVsia}
  \end{subfigure}
  \hfill
  \begin{subfigure}[b]{0.99\columnwidth}
    \centering
    \includegraphics[width=0.8\textwidth]{fig/CIFAR/Fedalign/line_FI_SIA.png}
    \caption{Fairness index of SIA accuracy \texttt{FI(SIA)}.}
    \label{fig:CIFARFIsia}
  \end{subfigure}
  \caption{Disparity of SIA accuracy among clients using CIFAR-10 dataset.}
  \label{fig:CIFAR sia}
\end{figure}



\begin{figure}[!t]
  \centering
  \begin{subfigure}[b]{0.99\columnwidth}
    \centering
    \includegraphics[width=0.8\textwidth]{fig/CIFAR/Fedalign/line_average_train_accuracy.png}
    \caption{Global model training accuracy. }
  \end{subfigure}
  \hfill
  \begin{subfigure}[b]{0.99\columnwidth}
    \centering
    \includegraphics[width=0.8\textwidth]{fig/CIFAR/Fedalign/line_average_test_accuracy.png}
     \caption{Global model testing accuracy.}
  \end{subfigure}
  \caption{Global model classification accuracy using CIFAR-10.}
  \label{fig:accu cifar}
\end{figure}







\subsection{Summary of \sysname\ in HAR and CIFAR-10 results}
In summary, our evaluation across HAR and CIFAR-10 datasets demonstrates the effectiveness of \sysname\ in achieving fairness in the impact of source inference attacks (SIA) while maintaining or improving model performance.   Although \sysname successfully achieves a more equitable distribution of SIA risk among clients in the HAR dataset, the inherent variability of human activity data, with each client representing a single individual exclusively, limits the complete elimination of SIA vulnerability, as it is difficult for client models to be completely indistinguishable. Nevertheless, as detailed in \autoref{sec:HAReval}, \sysname\ significantly enhances fairness in privacy.  Conversely, on the CIFAR-10 dataset, \sysname\ effectively mitigates SIA risks as detailed in \autoref{sec:CIFAReval}. This result is attributed to the non-IID sampling of CIFAR-10 subsets across clients, where all data in each subset are part of the comprehensive CIFAR-10 dataset. This characteristic enables the regularization of client models to be indistinguishable and coupled with applying Lipschitz constant loss during local training and adaptive aggregation at the server. These mechanisms collectively promote client model generalization and reduce loss variation, thereby neutralizing the effectiveness of source inference attacks (SIA).












\section{Discussion}


In this paper, we adopted a learner-centered design approach, beginning with a formative study to identify students' challenges with existing tools. Based on these insights, we developed DBox, a tool that scaffolds students in breaking problems into smaller parts and provides personalized, adaptive support. Our user study demonstrated that DBox improved learners' performance on similar algorithmic problems, increased perceived learning gains, and fostered greater cognitive engagement, achievement, and satisfaction. In this section, we discuss design implications and generalizability based on our key findings.


\ms{
\subsection{Chaining Learners' Thoughts with Visualized Structured UI Components}

Decomposition requires students to effectively organize their thoughts. While visual elements are known to promote structured thinking and support mental model construction \cite{mcdougall2001effects, liu2010mental}, our formative and user studies revealed shortcomings in existing tools like LeetCode and ChatGPT, which rely on textual representations without adequately supporting structured mental models. In contrast, DBox uses an interactive step tree to visually organize learners' thoughts. This feature was praised by 22 of 24 participants for enhancing algorithmic thinking, serving as a progress tracker, and providing value even without AI assistance.

DBox's interactive step tree and tree-based scaffolding demonstrate the broader potential of intelligent tutoring systems (ITS) to promote active learning and self-regulated problem-solving in fields requiring problem decomposition. Similar principles could benefit STEM education, such as physics or engineering, by externalizing abstract concepts and facilitating multi-step problem-solving. Additionally, progress-tracking visual components may inspire designs for professional training tools in areas like medical diagnostics or software engineering.

\subsection{Promoting Independent Thinking and Active Decomposition Learning}

\subsubsection{\textbf{Transforming Learners from Passive Readers to Active Thinkers}}

Many coding tools provide direct answers or solutions \cite{kazemitabaar2023novices, phung2023generating}, which, while efficient, often bypass opportunities to develop critical problem-solving skills. In contrast, DBox cultivates students' decomposition abilities through structured scaffolding, fostering critical thinking and self-regulated learning in line with learning by doing \cite{anzai1979theory} and constructivist principles \cite{tobias2009constructivist}.

To strengthen decomposition skills, DBox first encourages students to develop their own decomposition strategies by coding or building a step tree from scratch. While DBox can generate parts of a step tree from a student's existing code, these steps are derived from the learner's own reasoning, with DBox acting solely as a modality converter. Besides, DBox provides feedback on tree node statuses, identifying potential errors or missing steps without directly showing the correct answer, challenging students to critically evaluate and refine their decomposition plans.


DBox's scaffolded hint system further supports decomposition skill development by providing adaptive guidance tailored to the student’s progress without overwhelming them. All hints are based on the learner's current decomposition skeleton, with the most detailed hint—``reveal substep''—triggered only after repeated attempts and struggles. Notably, even the most detailed hints prompt only one substep, requiring students to complete the rest independently. As shown in Sec \ref{hintusage}, only 19\% of hints are this detailed, with students primarily relying on simpler, thought-provoking question hints. This scaffolded support system balances guidance and independent thinking, keeping students engaged during challenges without compromising their ability to independently decompose problems \cite{kinnunen2006students}.

Based on these findings, we recommend fostering active problem-solving by shifting students from passive content consumption to active solution creation. Designers could adopt layered scaffolding, starting with minimal guidance and increasing support as needed, to help students progressively master decomposition skills while maintaining confidence and avoiding frustration. Additionally, adaptive learning techniques, such as real-time feedback and progress tracking, can further tailor the support to individual decomposition barriers, encouraging deeper engagement with decomposition tasks. Moreover, designers could integrate metacognitive strategies, such as encouraging students to articulate or reflect on their decomposition approaches, to further enhance critical thinking and foster habits of independent thinking.




\subsubsection{\textbf{Choice of Scaffolding: Balancing Independent Problem-Solving and Efforts}}

Scaffolding involves providing tailored support to help learners accomplish tasks they cannot yet complete independently \cite{kim2011scaffolding, tobias2009constructivist}. Broadly, scaffolding strategies fall into two categories \cite{van2010scaffolding}: (1) gradually reducing assistance as learners gain proficiency, and (2) encouraging independent problem-solving while offering incremental support to address challenges. DBox adopts the second approach, emphasizing independent thinking and encouraging learners to actively decompose problems \cite{zimmerman2013theories}. While our scaffolding strategies successfully enhanced critical thinking, satisfaction, and perceived usefulness, they also led to increased cognitive effort (Sec. \ref{Effects_on_UX}). This tradeoff underscores the importance of carefully balancing cognitive effort with the promotion of independent thinking.

Future designs could incorporate adaptive scaffolding that adjusts support dynamically based on learner proficiency, reducing unnecessary effort in areas where students have demonstrated competence. Additionally, while incremental scaffolding was effective for algorithmic problem-solving, tailoring strategies to different educational contexts could enhance their applicability in diverse domains. Such adaptive, context-specific approaches could further optimize the balance between support and independence in learning environments.


\subsection{Supporting Personalized Algorithmic Programming Learning}

\subsubsection{\textbf{Prioritizing Learners' Own Solutions Over Optimality}}

Algorithmic problems often have multiple solutions with varying time and space complexities. DBox prioritizes independent exploration by supporting learners' strategies rather than steering them toward a single ``optimal'' solution. Using LLM-driven prompts, it evaluates and guides each step based on the learner's reasoning, preserving their step decomposition and respecting their input—even when errors occur. While some solutions may not be the most efficient, this approach fosters autonomy by aligning feedback with learners’ thought processes instead of enforcing rigid standards.

Our user study showed that this approach improves learning outcomes and is well-received by students. We recommend designing systems that respect personalized problem-solving strategies by aligning feedback with learners' reasoning while allowing for diverse approaches. Designers should balance flexibility and rigor, using prompts and interfaces that support varied strategies while gently guiding learners toward effective solutions.


\subsubsection{\textbf{Catering to Individual Learning Styles and Contextual Needs}}

DBox accommodates diverse problem-solving approaches with two input modes: coding and natural language descriptions. Each mode offers distinct advantages tailored to different learners, stages, and situations. Learners can switch seamlessly between modes, with progress automatically synced across the interface. Features such as verifying code-step alignment ensure strong integration between modes.

Our findings reveal that this flexibility enhances user experience. Participant interaction logs and interviews revealed three usage patterns, highlighting that each mode fits different needs: code mode works well for students with a clear and detailed problem-solving plan already, while the step tree with natural language descriptions helps less experienced students with only a basic idea who are not ready to write code directly, boosting their confidence.


We argue there is no universal “best” mode for programming education—each has unique benefits depending on the learner habits, expertise, and context. Future tools should provide flexibility, like DBox, or use adaptive algorithms to recommend modes based on user needs and context. This flexibility highlights the importance of designing educational tools that accommodate varying levels of expertise and problem-solving styles, which can be generalized to other domains requiring personalized learning \cite{bernacki2021systematic}.

\subsection{Appropriate Usage of LLMs for Supporting Algorithmic Programming Learning}

\subsubsection{\textbf{Caution About LLM Errors}}

Although LLMs have shown strong performance in coding tasks \cite{finnie2023my, leinonen2023using}, they remain prone to errors. Our technical evaluation and user study revealed that even with comprehensive context—such as problem statements, user code, and natural language steps—LLM sometimes misinterprets user descriptions. These errors likely arise from discrepancies between the natural language used by students and the formal, precise language the LLM was trained on, which is primarily sourced from web-based code and comments \cite{liu2023wants}.

Such misinterpretations can hinder learning by causing confusion or frustration. While future improvements to training data and GPT versions may mitigate these issues, design strategies can help address them. \textbf{First}, LLMs should avoid giving direct solutions and instead focus on fostering active problem-solving through explanations and hints. \textbf{Second}, feedback could be paired with interactive features, like a ``Run Code'' option, allowing students to validate their reasoning. \textbf{Third}, simple tutorials could teach users how to phrase their descriptions more clearly, improving LLM's understanding. Additionally, future tools could integrate a ``Language Enhancement'' feature to suggest improvements or assess the clarity of descriptions, aiding LLM in accurately capturing user intent. Most importantly, we recommend designers prioritize technical feasibility, such as conducting rigorous evaluations like ours, before fully integrating LLMs into programming learning tools.
}



\subsubsection{\textbf{Learner-LLM Co-Decomposition of Solutions: Learner as Leader, LLM as Aid}}

A central feature of DBox is the construction of a step tree, where students break solutions into steps and sub-steps. The LLM supports this by mapping code to step descriptions, evaluating them, and offering hints. However, students maintain full control, deciding how to decompose problems and define each step, fostering independent thinking. The LLM acts solely as an aid, using a scaffolding approach to support the development of learners' Zone of Proximal Development (ZPD) \cite{chaiklin2003zone}. Unlike tools like ChatGPT or Copilot that dominate problem-solving, DBox fosters deeper cognitive engagement. Students reported greater accomplishment and found this approach more effective for learning.

This contrasts with existing human-AI collaboration paradigms in non-educational scenarios where AI usually suggest options, leaving final decisions to users \cite{dang2023choice, gao2024collabcoder, gebreegziabher2023patat, ma2019smarteye, ma2022glancee}, such as in human-AI decision-making \cite{ma2023should, ma2024towards, ma2024you}. Some educational tools, like Jin et al. \cite{jin2024teach}, use LLMs to generate solutions for students to evaluate, which aids in syntax learning but such ``LLM-generate then learner-evaluate'' approach is less effective for algorithmic problem-solving, where constructing solutions is key. Just evaluating LLM-generated contents can place a cognitive anchor on learners \cite{furnham2011literature}, limiting independent thinking and creativity. Thus, task allocation between humans and AI should align with the educational context (e.g., whether it is basic knowledge/concept learning or higher-level creative thinking). Future LLM-based educational tools should carefully define the division of roles between LLMs and learners, tailoring it to specific learning contexts and goals.




% \subsubsection{Human-LLM Co-Decomposition of Solution: AI Should Judge Instead of Recommending}

% A core interaction in DBox is the construction of a step tree, where the entire solution is broken down into a series of steps and sub-steps. We refer to this as the human-LLM co-decomposition process. In this process, the LLM behind DBox plays three roles: First, it maps the student's written code into step descriptions. Second, it evaluates the status of each step and sub-step (whether they are correct, incorrect, missing, or need further decomposition). Third, it provides hints for incorrect or missing steps or sub-steps. However, the actual construction of the step tree—such as dividing the solution into steps and sub-steps and determining the content of each node—remains primarily the student's responsibility.

% This division of labor maximizes student engagement in independent thinking and problem-solving. The LLM does not provide any suggestions for decomposition nor directly recommend content for specific steps, aligning with the scaffolding educational approach, where guidance is provided appropriately, but the main task of forming the solution is left to the students.

% In contrast, when students directly seek help from an LLM, such as asking questions in ChatGPT or using Copilot for code completion, the LLM takes too much initiative by directly offering ideas or code. In our co-decomposition design, however, students demonstrated higher cognitive engagement and more active critical thinking. Furthermore, students reported that constructing solutions in this way gave them a greater sense of achievement and made them feel the process was more beneficial for learning, leading to higher satisfaction with the experience.

% Related work has proposed similar approaches. For instance, XXX, in the context of problem-solving, uses the "learning by teaching" concept, where students take on the tasks of judging and teaching, while the LLM generates most of the solutions. Compared to our approach, their division of labor between the student and the LLM is reversed. This method works well in introductory programming, where the focus is on mastering syntax. Having students guide the LLM to generate code or evaluate potentially incorrect code produced by the LLM is an effective way to quiz them. However, in our work, which focuses on algorithmic programming, the key step is constructing a solution from scratch. If the LLM builds the solution, leaving students only to judge it, it hampers their independent thinking.

% Thus, when designing LLM-based educational tools in the future, it is crucial to consider the specific context to effectively allocate tasks between the student and the LLM, ensuring that students derive the maximum benefit from the co-decomposition process.


% \subsection{Future Design Opportunities}

% \emph{Providing Appropriate Generative Assistance:} While DBox promotes independent problem-solving, some users showed interest in features like auto-completion for trivial coding tasks. Future versions could balance promoting independence with targeted assistance by enabling adjustable difficulty levels and offering contextual suggestions when appropriate.

% \emph{Covering All Stages of Algorithmic Programming:} DBox currently lacks a focus on foundational algorithm instruction and problem comprehension. Future iterations could include features like generating distractor solutions, input-output tests, and step-by-step rephrasing to help students grasp key concepts and understand the coding problem.

% \emph{Combining Step Trees with Dialogue:} Users can currently describe their thought processes but cannot ask questions. Adding a dialogue system to the step tree would allow students to share challenges and ask follow-up questions. GPT could then provide guided feedback without giving direct answers, supporting independent problem-solving.





% \emph{Other Important Features.} DBox could offer more control by allowing users to select specific parts of their code for targeted evaluation and guidance. A ``review'' feature could also help students reflect on key stumbling points, understand where their thought process went wrong, and how they eventually solved the problem.


% \subsection{Future Design Opportunities}

% \emph{Providing Appropriate Generative Assistance.} Our tool primarily focuses on encouraging users to create the step tree and write the code independently, with the system mainly serving as a judge. However, users expressed a desire for some intelligent completion features, particularly for repetitive or simple code, allowing them to focus their efforts on learning the key parts. Future improvements should strike a balance between fostering independent thinking and providing appropriate assistance. One approach could be designing basic rules where the tool offers intelligent suggestions and completions for parts unrelated to the core logic, while maintaining the current level of independence for key learning areas. Additionally, the system could offer different modes, allowing users to choose the level of assistance, from basic judgment-only feedback to a combination of judgment, guidance, necessary completions, and even on-demand suggestions.

% \emph{Covering All Stages of Algorithmic Programming.} Currently, our system does not cover the basic teaching of algorithms or the problem comprehension stage. In the future, to address the diversity and uncertainty in solutions and help students grasp multiple approaches, we could expand assistance during the idea formation phase. For example, GPT could generate multiple potential solutions with distractors, prompting students to identify the one that meets the problem's complexity requirements. We could also introduce specialized algorithm training, where students select a specific algorithm, and the system’s guidance focuses solely on that algorithm. To assist with problem comprehension, we could incorporate input-output tests to check students' understanding of the problem and step-by-step rephrasing to help them grasp more complex problems.

% \emph{Combining Interactive Step Trees with Dialogue Boxes.} Sometimes users want to describe their difficulties, and currently, we ask them to outline their thought processes. Additionally, users may want to ask follow-up questions. In the future, we could combine the structured step tree with a small dialogue box. The primary goal would still be to construct the step tree, but users could engage in a conversation with GPT in the context of the current step tree or a specific step. Importantly, GPT should guide the user without revealing direct answers.

% \emph{Other Important Features.} First, DBox could offer learners more control, such as allowing users to select specific parts of the code for targeted evaluation and guidance. We could also introduce a summary feature for key stumbling points, helping students reflect on the challenges they faced, where their thought process went wrong, and how they eventually overcame the problem.




\subsection{Limitations and Future Work}

This study has several limitations. \emph{First}, we tested DBox's effectiveness on only two problem types; future work should examine a broader range of algorithms. \emph{Second}, participants engaged in just one learning session per condition due to time constraints, whereas mastering algorithmic problems typically requires extended practice. Longitudinal studies should explore how DBox supports skill development over time, including changes in mental models and skill retention. \emph{Third}, we assessed learning gains based on correctness in a test session using similar learning and test problems. Future research should evaluate knowledge transfer to less similar problems. Due to time constraints, we conducted a single post-test rather than a pre-post comparison. While pre-test expertise filtering and randomization minimized prior familiarity effects, a more rigorous pre-post design would yield more accurate learning gain measurements. Looking ahead, we plan to release DBox as a Chrome plugin for integration with existing coding platforms, enabling large-scale field studies. This will allow for the collection of long-term usage data and periodic surveys to identify usage patterns and learning experiences over time.



% This study has several limitations. First, in our within-subject design, we selected two types of algorithm problems—Greedy and Binary Search—and randomly assigned them to two conditions (DBox and baseline). However, selection bias may still exist, as some participants might naturally excel at one type of algorithm. Although we addressed this by filtering participants' proficiency through a pre-test and using a Latin Square design, further validation across a broader range of algorithms is needed in future work.

% Second, students experienced only one learning session per condition before the test session. While this allowed for a fair comparison, mastering algorithmic problems typically requires extended practice. Future work should explore how DBox supports students' long-term improvement in algorithmic skills. Longitudinal studies could provide insights into changes in learners' mental models, allowing students more time to deepen their understanding and refine their decomposition methods. Additionally, retention tests could assess whether students can still apply learned problem-solving methods after a time gap.

% We measured learning gains through correctness scores in the test session, with relatively similar learning and test problems. Future work should explore students' ability to transfer their knowledge to problems with lower similarity. Due to time constraints, we opted for a single post-test rather than a pre-post comparison. While we minimized prior familiarity effects by filtering participants and randomizing problem assignments, future studies could adopt a more rigorous pre-post test design for better measurement of learning gains.

% Looking ahead, we plan to release DBox as a Chrome plugin for integration with existing online coding platforms and large-scale real-world testing. In such settings, where students may be more motivated (e.g., preparing for algorithm interviews), we can gather long-term usage data while ensuring privacy. We also plan to conduct periodic surveys to track changes in students' usage patterns and learning experiences over time.



% \subsection{Limitations and Future Work}

% This study has several limitations. First, in our within-subjects study, we selected two types of algorithm problems, Greedy and Binary Search, and randomly assigned them to two conditions, DBox and the baseline. However, there may still be selection bias, where some participants were naturally better at one type of algorithm. While we mitigated this issue to a large extent by filtering participants' proficiency through a pre-test and employing a Latin Square design to randomize the problem-condition assignment, there is still room for improvement. Future work should validate DBox's effectiveness across a broader range of problem types.

% Second, in our experiment, students only experienced one learning session in each condition before moving on to the test session. Although this comparison was fair (as both conditions had only one learning session), mastering an algorithmic problem often requires extended practice. Future work should explore how DBox can help students gradually improve their algorithmic programming skills over time. Longitudinal studies may reveal significant changes in learners' mental models, providing more time for them to understand a specific algorithm and enhance their decomposition methods. Additionally, future studies could include retention tests to measure whether students can still effectively apply previously learned problem-solving methods after a period of time.

% Furthermore, when objectively measuring students' learning gains, we calculated their correctness score in the test session. On the one hand, the learning session and test session problems had a relatively high degree of similarity. Future work should investigate whether students can transfer what they have learned to solve problems of the same algorithm type with lower similarity. On the other hand, due to time constraints, we did not include a pre-post test comparison, opting for a single post-test instead. This result might be influenced by students' pre-existing familiarity with the problems. Although we mitigated this issue by filtering for familiarity (ensuring participants were not too familiar with the problems) and randomizing the problem assignments, future work could include a more rigorous pre-post test design to better calculate students' learning gains.

% Moreover, DBox is currently only applied in algorithmic programming, specifically solving algorithm problems. However, this decomposition-based computational thinking approach could be extended to other learning scenarios, such as project-based learning. Future work could explore how to adapt DBox to broader educational contexts outside of algorithmic programming.

% Looking forward, we aim to deploy DBox in real-world algorithm courses. Since algorithms are a core required subject in undergraduate computer science curricula, we hope to investigate how students who have just learned algorithm concepts use DBox to develop their problem-solving skills. Additionally, we plan to convert DBox into a Chrome plugin and release it in the Chrome Web Store for real-world testing. This would allow DBox to seamlessly integrate with existing online coding platforms, enabling large-scale experiments. In such settings, students' motivation may be stronger (e.g., a graduate preparing for an algorithm interview), leading to more realistic usage patterns. Students could use DBox to tackle a wide variety of algorithm problems. We hope to collect long-term (e.g., six-month) usage data from real-world users while ensuring privacy, and use periodic surveys to capture changes in students' usage patterns and learning experiences over time.





\section{Conclusion}
% In this paper, we introduced Decomposition Box (DBox), a novel tool designed to scaffold learners in decomposing problems during algorithmic programming learning. Based on insights from a formative study, we identified key design goals to address the limitations of existing tools in algorithmic programming education. DBox supports two critical stages of the programming process: idea formation and idea implementation. By offering two modes (code mode and language mode), it encourages users to independently develop their solution strategies. The interactive, visual step tree helps students break down problems and build a structured mental model. DBox provides fine-grained, step-level feedback, enabling students to quickly identify issues, while its multi-level guidance offers targeted support without undermining independent thinking.

% Our user study demonstrated that DBox led to significantly higher learning gains, cognitive engagement, and critical thinking. Students reported a stronger sense of achievement and found the assistance both appropriate and effective for their learning. We identified three main usage patterns, underscoring the importance of respecting students' problem-solving habits and offering them autonomy. The learner-LLM co-decomposition model we designed promotes independent thinking while allowing the LLM to contribute meaningfully, even with occasional imperfections. 

% We hope the formative study, design goals, features, technical evaluation, and key findings from this work will inspire future research on developing educational tools for broader programming learning.
In this paper, we introduced DBox, an interactive tool designed to help learners decompose algorithmic programming problems by supporting both solution formation and implementation. Featuring an intuitive tree-like box widget, DBox accepts input in both code and natural language, fostering independent problem-solving while its step tree structure helps learners develop structured mental models. It provides step-level feedback and layered guidance without compromising learner autonomy.
Our user study showed that DBox significantly improved learning outcomes, cognitive engagement, and critical thinking, with students reporting a greater sense of achievement and finding the support highly effective. Additionally, we identified three key usage patterns, highlighting the importance of accommodating individual problem-solving styles. Moreover, our findings suggest that the learner-LLM co-decomposition approach fosters independent thinking while providing meaningful guidance, even with occasional imperfections.
We hope the insights from our system design will inspire future research on integrating LLMs into educational tools for programming learning.



In this study, we performed the first large-scale analysis of data leakage across 83 software engineering (SE) benchmarks, covering three popular programming languages—Python, Java, and C/C++. By combining an efficient near-duplicate detection algorithm with extensive manual labeling, we ensured the accurate identification of leaked data.



Our findings show that while data leakage is generally low, with average leakage ratios of 4.8\%, 2.8\%, and 0.7\% for Python, Java, and C/C++ benchmarks respectively, some benchmarks exhibit higher leakage that requires attention. We identified four main causes of leakage: direct inclusion of benchmark data in pre-training datasets, overlap between source repositories, reliance on platforms like LeetCode, and shared data sources such as GitHub issues.
We also found that automatic detection methods, like Perplexity-based metrics, struggle to distinguish between leaked and non-leaked samples. Additionally, our experiments reveal that data leakage inflates evaluation metrics, with models performing significantly better on leaked samples. For instance, StarCoder-7b achieved a Pass@1 score 4.9 times higher on leaked samples, underlining the need to address leakage to ensure fair evaluations.
This study offers insights into data leakage status in SE benchmarks and its impact on LLM evaluation.


In the future, we aim to expand the analysis to additional benchmarks and explore new methods to prevent or further reduce data leakage.





\vspace{0.2cm}
\noindent \textbf{Acknowledgement.}  This research / project is supported by the National Research Foundation, under its Investigatorship Grant (NRF-NRFI08-2022-0002). Any opinions, findings and conclusions or recommendations expressed in this material are those of the author(s) and do not reflect the views of National Research Foundation, Singapore.




%%
%% The acknowledgments section is defined using the "acks" environment
%% (and NOT an unnumbered section). This ensures the proper
%% identification of the section in the article metadata, and the
%% consistent spelling of the heading.
\begin{acks}
This work is supported by the U.S. National Science Foundation (NSF) under grant number 2339266.
\end{acks}


%%
%% The next two lines define the bibliography style to be used, and
%% the bibliography file.
\balance
\bibliographystyle{ACM-Reference-Format}
\bibliography{facctref}


%%
%% If your work has an appendix, this is the place to put it.
\appendix
\renewcommand{\thefigure}{S\arabic{figure}}
\setcounter{figure}{0}  
\renewcommand{\thetable}{S\arabic{table}}
\setcounter{table}{0} 


\section{Proofs}
\subsection{Proof of Proposition~\ref{prop:cos_sim_grads}}
\label{prf:prop_grad_grows}
\cosgrads*
\begin{proof}
    We are taking the gradient of $\mathcal{L}^\mathcal{A}_i$ as a function of $z_i$. The principal idea is that the gradient has a term with direction $\hat{z}_j$ and a term with direction $-\hat{z}_i$. We then disassemble the vector with direction $\hat{z}_j$ into its component parallel to $z_i$ and its component orthogonal to $z_i$. In doing so, we find that the two terms with direction $z_i$ cancel, leaving only the one with direction orthogonal to $z_i$.
    
    Writing it out fully, we have $\mathcal{L}^\mathcal{A}_i = -z_i^\top z_j / (\|z_i\| \cdot \|z_j\|)$. Taking the gradient amounts to using the quotient rule, with $f = -z_i^\top z_j$ and $g = \|z_i\| \cdot \|z_j\| = \sqrt{z_i^\top z_i} \cdot \sqrt{z_j^\top z_j}$. Taking the derivative of each, we have
    \begin{align*}
        f' &= -\mathbf{z}_j \\
        g' &= \|z_j\| \frac{z_i}{\sqrt{z_i^\top z_i}} = \|z_j\| \frac{\mathbf{z}_i}{\|z_i\|} \\
        \implies \frac{f' g - g' f}{g^2} &= \frac{- \left(\mathbf{z}_j \cdot \|z_i\| \cdot \|z_j\| \right) + \left(\|z_j\| \frac{\mathbf{z}_i}{\|z_i\|} \cdot z_i^\top z_j \right)}{\|z_i\|^2 \cdot \|z_j\|^2} \\
        &= \frac{-\mathbf{z}_j}{\|z_i\| \cdot \|z_j\|} + \frac{\mathbf{z}_i z_i^\top z_j}{\|z_i\|^3 \|z_j\|},
    \end{align*}
    where we use boldface $\mathbf{z}$ to emphasize which direction each term acts along. We now substitute $\cos(\phi_{ij}) = z_i^\top z_j / (\|z_i\| \cdot \|z_j\|)$ in the second term to get
    \begin{equation}
        \label{eq:quotient_rule}
        \frac{f' g - g' f}{g^2} = \frac{-\hat{z}_j}{\|z_i\|} + \frac{\mathbf{z}_i \cos(\phi)}{\|z_i\|^2}
    \end{equation}

    It remains to separate the first term into its sine and cosine components and perform the resulting cancellations. To do this, we take the projection of $\hat{z}_j = \mathbf{z}_j / \|z_j\|$ onto $\mathbf{z}_i$ and onto the plane orthogonal to $\mathbf{z}_i$. The projection of $\hat{z}_j$ onto $\mathbf{z}_i$ is given by
    \[ \cos \phi_{ij} \frac{\mathbf{z}_i}{\|z_i\|} \]
    while the projection of $\mathbf{z}_j / \|z_j\|$ onto the plane orthogonal to $\mathbf{z}_i$ is
    \[ \left( \mathbf{I} - \frac{z_i z_i^\top}{\|z_i\|^2} \right) \frac{\mathbf{z}_j}{\|z_j\|}. \]
    It is easy to assert that these components sum to $\mathbf{z}_j/\|z_j\|$ by replacing the $\cos \phi_{ij}$ by $\frac{z_i^\top z_j}{\|z_i\|\cdot \|z_j\|}$.

    We plug these into Eq.~\ref{eq:quotient_rule} and cancel the first and third term to arrive at the desired value:
    \begin{align*}
        \frac{f' g - g' f}{g^2} = &-\frac{1}{\|z_i\|} \cos \phi \frac{\mathbf{z}_i}{\|z_i\|} \\
        &- \frac{1}{\|z_i\|} \cdot \left( \mathbf{I} - \frac{z_i z_i^\top}{\|z_i\|^2} \right) \frac{\mathbf{z}_j}{\|z_j\|} \\
        &+ \frac{\mathbf{z}_i \cos(\phi)}{\|z_i\|^2} \\
        = &\frac{-1}{\|z_i\|} \cdot \left( \mathbf{I} - \frac{z_i z_i^\top}{\|z_i\|^2} \right) \frac{\mathbf{z}_j}{\|z_j\|}.
    \end{align*}
\end{proof}

We visualize the loss landscape of the cosine similarity function in Figure \ref{fig:cos_sim_surface}. 

\begin{figure}
    \centering
    \begin{subfigure}{0.45\linewidth}
        \centering 
        \includegraphics[width=1\linewidth]{Images/cosine_similarity_surface_with_circles.pdf}
    \end{subfigure}%
    \begin{subfigure}{0.45\linewidth}
        \centering 
        \includegraphics[width=0.8\linewidth]{Images/cosine_similarity_2D_heatmap.pdf}
    \end{subfigure}
    \caption{Cosine similarity with respect to the direction indicated by the blue line. Three circles of radii 0.5, 1, and 2 are superimposed to show that, for higher norms, the cosine similarity is less steep. Left: 3D Surface plot, right: 2D topview plot.}
    \label{fig:cos_sim_surface}
\end{figure}


\subsection{InfoNCE Gradients}
\label{app:infonce_grads}
\infoncegrads*
\begin{proof}
    We are interested in the gradient of $\mathcal{L}_i^\mathcal{R}$ with respect to $z_i$. By the chain rule, we get
    \begin{align*}
        \nabla_i^\mathcal{R} &= -\frac{\sum_{k \not\sim i} \text{ExpSim}(z_i, z_k) \frac{\partial \frac{z_i^\top z_k}{\|z_i\| \cdot \|z_k\|}}{\partial z_i}}{\sum_{k \not\sim i} \text{ExpSim}(z_i, z_k)} \\
        &= -\frac{\sum_{k \not\sim i} \text{ExpSim}(z_i, z_k) \frac{\partial \frac{z_i^\top z_k}{\|z_i\| \cdot \|z_k\|}}{\partial z_i}}{S_i}
    \end{align*}
    It remains to substitute the result of Prop. \ref{prop:cos_sim_grads} for $\partial \frac{z_i^\top z_k}{\|z_i\| \cdot \|z_k\|} / \partial z_i$.

    We sum this this with the gradients of the attractive term to obtain the full InfoNCE gradient, completing the proof.
\end{proof}

We note that the repulsive force is weighted average over a set of unit vectors. Consequently, the repulsive gradient is smaller than the attractive one. Additionally, we point out that these gradients are symmetric: just like positive and negative samples $z_j$ and $z_k$ affect $z_i$, $z_i$ affects $z_j$ and $z_k$.

\subsection{Proof of Corollary~\ref{cor:embeddings_grow}}
\label{prf:cor_embeddings_grow}
\begin{proof}
    First, consider that we applied the cosine similarity's gradients from Proposition~\ref{prop:cos_sim_grads}. Since $z_i$ and $(z_j)_{\perp z_i}$ are orthogonal, $\|z_i'\|_2^2 = \|z_i\|^2 + \frac{\gamma^2}{\|z_i\|^2}\|(z_j)_{\perp z_i}\|^2$. The second term is positive if $\sin \phi_{ij} > 0$.

    The same exact argument holds for the InfoNCE gradients. The gradient is orthogonal to the embedding, so a step of gradient descent can only increase the embedding's magnitude.
\end{proof}

\subsection{Proof of Theorem~\ref{thm:convergence_rate}}
\label{prf:thm_convergence_rate}
We first restate the theorem:

Let $z_i$ and $z_j$ be positive embeddings with equal norm, i.e. $\|z_i\| = \|z_j\| = \rho$. Let $z_i'$ and $z_j'$ be the embeddings after 1 step of gradient descent with learning rate $\gamma$. Then the change in cosine similarity is bounded from above by:
\begin{equation*}
    \hat{z}_i'^\top \hat{z}_j' - \hat{z}_i^\top \hat{z}_j < \frac{\gamma \sin^2 \phi_{ij}}{\rho^2} \left[ 2 - \frac{\gamma \cos \phi}{\rho^2} \right].
\end{equation*}

\noindent We now proceed to the proof:
\begin{proof}
    Let $z_i$ and $z_j$ be two embeddings with equal norm\footnote{We assume the Euclidean distance for all calculations.}, i.e. $\|z_i\| = \|z_j\| = \rho$. We then perform a step of gradient descent to maximize $\hat{z}_i^\top \hat{z}_j$. That is, using the gradients in \ref{prop:cos_sim_grads} and learning rate $\gamma$, we obtain new embeddings $z_i' = z_i + \frac{\gamma}{\|z_i\|} (\hat{z}_j)_{\perp z_i}$ and $z_j' = z_j + \frac{\gamma}{\|z_j\|} (\hat{z}_i)_{\perp z_j}$. Going forward, we write $\delta_{ij} = (\hat{z}_j)_{\perp z_i}$ and $\delta_{ji} = (\hat{z}_i)_{\perp z_j}$, so $z_i' = z_i + \frac{\gamma}{\rho} \delta_{ij}$ and $z_j' = z_j + \frac{\gamma}{\rho} \delta_{ji}$. Notice that since $z_i$ and $\delta_{ij}$ are orthogonal, by the Pythagorean theorem we have $\|z_i'\|^2 = \|z_i\|^2 + \frac{\gamma^2}{\rho^2}\|\delta_{ij}\|^2 \geq \|z_i\|^2$. Lastly, we define $\rho' = \|z_i'\| = \|z_j'\|$.

    We are interested in analyzing $\hat{z}_i'^\top \hat{z}_j' - \hat{z}_i^\top \hat{z}_j$. To this end, we begin by re-framing $\hat{z}_i'^\top \hat{z}_j'$:
    \begin{align*}
        \hat{z}_i'^\top \hat{z}_j' &= \left(\frac{z_i + \frac{\gamma}{\rho} \delta_{ij}}{\rho'}\right)^\top \left(\frac{z_j + \frac{\gamma}{\rho} \delta_{ji}}{\rho'}\right) \\
        &= \frac{1}{\rho'^2}\left[ z_i^\top z_j + \gamma \frac{z_i^\top \delta_{ji}}{\rho'} + \gamma \frac{z_j^\top \delta_{ij}}{\rho'} + \gamma^2 \frac{\delta_{ij}^\top \delta_{ji}}{\rho'^2} \right].
    \end{align*}

    We now consider that, since $\delta_{ij}$ is the projection of $\hat{z}_j$ onto the subspace orthogonal to $z_i$, we have that the angle between $z_i$ and $\delta_{ji}$ is $\pi/2 - \phi_{ij}$. Plugging this in and simplifying, we obtain
    \begin{align*}
        z_i^\top \delta_{ji} &= \|z_i\| \cdot \|\delta_{ji}\| \cos (\pi/2 - \phi_{ij}) \\
        &= \|z_i\| \cdot \|\delta_{ji}\| \sin \phi_{ij} \\
        &= \rho \sin^2 \phi_{ij}.
    \end{align*}
    By symmetry, the same must hold for $z_j^\top \delta_{ij}$.
    
    Similarly, we notice that the angle $\psi_{ij}$ between $\delta_{ij}$ and $\delta_{ji}$ is $\psi_{ij} = \pi - \phi_{ij}$. The reason for this is that we must have a quadrilateral whose four internal angles must sum to $2\pi$, i.e. $\psi_{ij} + \phi_{ij} + 2 \frac{\pi}{2} = 2 \pi$. Thus, we obtain $\delta_{ij}^\top \delta_{ji} = \|\delta_{ij}\| \cdot \|\delta_{ji}\| \cos(\psi) = -\sin^2 \phi_{ij} \cos \phi_{ij}$.

    We plug these back into our equation for $\hat{z}_i'^\top \hat{z}_j'$ and simplify:
    \begin{align*}
        \hat{z}_i'^\top \hat{z}_j' &= \frac{1}{\rho'^2}\left[ z_i^\top z_j + \gamma \frac{z_i^\top \delta_{ji}}{\rho} + \gamma \frac{z_j^\top \delta_{ij}}{\rho} + \gamma^2 \frac{\delta_{ij}^\top \delta_{ji}}{\rho^2} \right] \\
        &= \frac{1}{\rho'^2}\left[ z_i^\top z_j + \gamma \frac{\rho \sin^2 \phi_{ij}}{\rho} + \gamma \frac{\rho \sin^2 \phi_{ij}}{\rho} - \gamma^2 \frac{\sin^2 \phi_{ij} \cos \phi_{ij}}{\rho^2} \right] \\
        &= \frac{1}{\rho'^2}\left[ z_i^\top z_j + 2 \gamma \sin^2 \phi_{ij} - \gamma^2 \frac{\sin^2 \phi_{ij} \cos \phi_{ij}}{\rho^2} \right].
    \end{align*}

    We now consider the original term in question:
    \begin{align*}
        \hat{z}_i'^\top \hat{z}_j' - \hat{z}_i^\top \hat{z}_j &= \frac{1}{\rho'^2}\left[ z_i^\top z_j + 2 \gamma \sin^2 \phi_{ij} - \gamma^2 \frac{\sin^2 \phi_{ij} \cos \phi_{ij}}{\rho^2} \right] - \frac{z_i^\top z_j}{\rho^2} \\
        &\leq \frac{1}{\rho^2}\left[ z_i^\top z_j + 2 \gamma \sin^2 \phi_{ij} - \gamma^2 \frac{\sin^2 \phi_{ij} \cos \phi_{ij}}{\rho^2} \right] - \frac{z_i^\top z_j}{\rho^2} \\
        &= \frac{1}{\rho^2}\left[ 2 \gamma \sin^2 \phi_{ij} - \gamma^2 \frac{\sin^2 \phi_{ij} \cos \phi_{ij}}{\rho^2} \right] \\
        &= \frac{\gamma \sin^2 \phi_{ij}}{\rho^2}\left[ 2 - \frac{\gamma \cos \phi_{ij}}{\rho^2} \right]\\
        &\leq \frac{2 \gamma \sin^2 \phi_{ij}}{\rho^2}
    \end{align*}
    
    This concludes the proof.
\end{proof}

\section{Simulations}
\label{app:simulations}

\subsection{Aparametric Simulations}

For the simulations in Section \ref{ssec:convergence_simulations}, we produced two datasets, $\mathbf{X}_1$ and $\mathbf{X}_2$, independently by randomly sampling points in $\mathbb{R}^20$ from a standard normal distribution and normalizing them to the hypersphere. The $i$-th point in dataset $\mathbf{X}_1$ is the positive counterpart for the $i$-th point in dataset $\mathbf{X}_2$. The first dataset is then set to be static while the second is modified in order to control for the embedding norms and angles between positive pairs.

We optimize the cosine similarity by performing standard gradient descent on the embeddings themselves with learning rate $10$. We consider a dataset ``converged'' when the average cosine similarity between positive pairs exceeds $0.999$.

\paragraph{Controlling for angles.} In order to control for the angle between positive pairs, we use an interpolation value $\alpha \in [-1, 1]$. Let $x_1$ be a static embedding in $\mathbf{X}_1$ and $x_2$ be the embedding in $\mathbf{X}_2$ whose angle we wish to control. In expectation, $\phi(x_1, x_2)$ will be $\pi / 2$. We therefore define the embedding $x_2$ whose angle has been controlled as 
\[ x_2' = x_2 \cdot (1 - |\alpha|) + x_1 \cdot \alpha. \]

In essence, when $\alpha=0$, $x_2' = x_2$. However, when $\alpha=1$ (resp. $\alpha=-1$), $x_2' = x_1$ (resp. $x_2' = -x_1$).

\paragraph{Controlling for embedding norms.} This setting is simpler than the angles between positive pairs. We simply scale $\mathbf{X}_2$ by the desired value.

\subsection{Parametric Simulations}
\label{app:parametric_sim}

We restate the entire implementation for the simulations in Section \ref{ssec:confidence_simulations} for completeness. We choose centers for 4 latent classes $\{c_1, c_2, c_3, c_4\}$ uniformly at random from $\mathbb{S}^{10}$ by randomly sampling vectors from a standard multivariate normal distribution and normalizing them to the hypersphere. We then obtain the latent samples $\tilde{z}$ around center $c_i$ via $z \sim \mathcal{N}(c_i, 0.1 \cdot \mathbf{I})$ and re-normalizing to the hypersphere. For each center, we produce 1K latent samples; these constitute our latent classes. We depict an example of 8 such latent classes (in 3 dimensions) in Figure \ref{fig:orig_latents}. We finally obtain the dataset by generating a random matrix in $\mathbb{R}^{11 \times 64}$ and applying it to the latent samples.

We train the InfoNCE loss via a 2-layer feedforward neural network with the ReLU activation function in the hidden layer. The network's output dimensionality is $\mathbb{R}^{11}$ so that, after normalization, it can reconstruct the original latent classes. We train the network using the supervised InfoNCE loss with a batch size of 128. Each data point's positive pair is simply another data point from the same latent class.

We visualize the learned (unnormalized) embedding space in Figure \ref{fig:learned_latents}.

\begin{figure}
    \centering
    \begin{subfigure}{0.4\linewidth}
    \includegraphics[width=\linewidth]{Images/orig_latents.png}
    \caption{}
    \label{fig:orig_latents}
    \end{subfigure}
    \quad\quad
    \begin{subfigure}{0.4\linewidth}
    \includegraphics[width=\linewidth]{Images/learned_latents.png}
    \caption{}
    \label{fig:learned_latents}
    \end{subfigure}
    \caption{\emph{Left}: A depiction of $8$ latent classes in $3$D obtained via the description in Section \ref{app:parametric_sim}. Dashed lines represent vectors from the origin to the mean of the distribution. \emph{Right}: A depiction of the learned latent space (unnormalized) using the supervised InfoNCE loss after 50 epochs of training.}
    
\end{figure}


\section{Further Discussion and Experiments}
\label{app:experiments}

\subsection{Experimental Setup}
\label{app:experiment_setup}
Unless otherwise stated, we use a ResNet-50 backbone \cite{resnet} and the default settings outlined in the SimCLR \cite{simclr} and SimSiam \cite{simsiam} papers. We use $1$e-$6$ as the default SimCLR weight decay and $5$e-$4$ as the default SimSiam one. When running on Cifar-10 and Cifar-100, we amend the backbone network's first layer as detailed in \citet{simclr}. We use embedding dimensionality $256$ in SimCLR and $2048$ in SimSiam. When reporting embedding norms, we use the projector's output in SimCLR and the predictor's output in SimSiam: these are the spaces where the loss function is applied and therefore where our theory holds.

Due to computational constraints, we run with batch-size 256 in SimCLR. Although each batch is still 256 samples in SimSiam, we simulate larger batch sizes using gradient accumulation. Thus, our default batch-size for SimSiam is 1024. 

\subsection{Opposite-Halves Effects}
\label{app:opposite_halves_effect}

We devote this section of the Appendix to studying the role of the angle between positive samples on the cosine similarity's convergence under gradient descent. Referring back to Figure~\ref{fig:convergence_sim}, we see that the effect is most impactful when the angle between positive embeddings is close to $\pi$, i.e. $\phi_{ij} > \pi - \varepsilon$ for $\varepsilon \rightarrow 0$. The following result shows that this is exceedingly unlikely for a single pair of embeddings in high-dimensional space:
\begin{proposition}
    \label{prop:unlikely_opp_halves}
    Let $x_i, x_j \sim \mathcal{N}(0, \mathbf{I})$ be $d$-dimensional, i.i.d. random variables and let $0 < \varepsilon < 1$. Then \vspace*{-0.1cm}
    \begin{equation}
    \label{eq:opp_halves_unlikely}
    \mathbb{P}\left[ \hat{x}_i^\top \hat{x}_j \geq 1 - \varepsilon \right] \leq \frac{1}{2d(1-\varepsilon)^2}.
    \end{equation}\vspace*{-0.3cm}
\end{proposition}
\begin{proof}
By \citet{distribution_of_cosine_sim}, the cosine similarity between two i.i.d. random variables drawn from $\mathcal{N}(0, \mathbf{I})$ has expected value $\mu = 0$ and variance $\sigma^2 = 1/d$, where $d$ is the dimensionality of the space. We therefore plug these into Chebyshev's inequality:
\begin{align*}
    &\text{Pr} \left[ \left|\frac{x_i^\top x_j}{\|x_i\|\cdot \|x_j\|} - \mu \right|\geq k \sigma \right] \leq \frac{1}{k^2} \\
    \rightarrow & \text{Pr} \left[ \left |\frac{x_i^\top x_j}{\|x_i\|\cdot \|x_j\|} \right |\geq \frac{k}{\sqrt{d}} \right] \leq \frac{1}{k^2}
\end{align*}

\noindent We now choose $k = \sqrt{d}(1 - \varepsilon)$, giving us
\[ \mathbb{P}\left[ \left |\frac{x_i^\top x_j}{\|x_i\| \cdot \|x_j\|}\right | \geq 1 - \varepsilon \right] \leq \frac{1}{d(1-\varepsilon)^2}.\]

It remains to remove the absolute values around the cosine similarity. Since the cosine similarity is symmetric around $0$, the likelihood that its absolute value exceeds $1 - \varepsilon$ is twice the likelihood that its value exceeds $1- \varepsilon$, concluding the proof.

We note that this is actually an extremely optimistic bound since we have not taken into account the fact that the maximum of the cosine similarity is 1.
\end{proof}

The above proposition represents the likelihood that \emph{one} pair of embeddings has large angle between them. It is therefore \emph{exponentially} unlikely for every pair of embeddings in a dataset to have angle close to $\pi$, as we would require Proposition \ref{prop:unlikely_opp_halves} to hold across every pair of embeddings. Thus, the opposite-halves effect is exceedingly unlikely to occur.

\begin{table}
    \centering
    \quad
    \parbox{.47\linewidth}{
        \begin{tabular}{lrcc}
        \toprule
        Model & Dataset \quad\quad & \makecell{Effect Rate\\Epoch 1} & \makecell{Effect Rate\\Epoch 16} \\
        \midrule
        \multirow{2}{*}{SimCLR} & Imagenet-100 & 2\% & 0\%  \\
        & Cifar-100 & 11\% & 1\% \\
        \cmidrule{1-4}
        \multirow{2}{*}{SimSiam} & Imagenet-100 & 26\% & 1\% \\
        & Cifar-100 & 21\% & 0\% \\
        \cmidrule{1-4}
        \multirow{2}{*}{BYOL} & Imagenet-100 & 28\% & 1\% \\
        & Cifar-100 & 20\% & 0\% \\
        \bottomrule
        \end{tabular}
        \captionof{table}{The rate at which embeddings are on opposite sides of the latent space (angle between a positive pair is greater than $\pi / 2$) for various datasets and SSL models.}
        \label{tbl:opposite_halves_effect}
    }
    \hfill
    \parbox{.38\linewidth}{
        \begin{tabular}{cc ccc}
        \toprule
        \multirow{2}{*}{Epoch} & & \multicolumn{3}{c}{Batch Size}\\
        & & 256 & 512 & 1024 \\
        \cmidrule{3-5}
        \multirow{2}{*}{100} & Default & 46.1 & 41.2 & 32.6 \\
        & Cut ($c=9$) & 43.1 & 46.5 & 44.3 \\
        \cmidrule{2-5}
        \multirow{2}{*}{500} & Default & 59.1 & 60.4 & 61.3\\
        & Cut ($c=9$) & 59.4 & 58.9 & 61.5 \\
        \bottomrule
        \end{tabular}
        \captionof{table}{$k$-nn accuracies for SimSiam trained with various batch sizes. We performed training for both the default and cut-initialized variants and reported $k$-nn accuracies at 100 and 500 epochs.}
        \label{tbl:cut_batch_size}
    }
\end{table}

In accordance with this, Table~\ref{tbl:opposite_halves_effect} shows that, after one epoch of training, embeddings have angle greater than $\pi/2$ at a rate of around $5\%$ and $25\%$ for SimCLR and SimSiam/BYOL, respectively. So even if the `strongest' variant of the opposite-halves effect is not occurring, a weaker one may still be. However, very early into training (epoch 16), every method has a rate of effectively 0 for the opposite-halves effect. Furthermore, the rates in Table~\ref{tbl:opposite_half_effect} measure how often $\phi_{ij} > \frac{\pi}{2}$. This is the absolute weakest version of the opposite-halves effect. Thus, while some weak variant of the opposite-halves effect may occur at the beginning of training, it does not have a strong impact on the convergence dynamics and, in either case, disappears quite quickly.

\subsection{Weight Decay}
\label{app:weight_decay}

We evaluate the effect of weight decay in the imbalanced setting in \ref{fig:weight_decay_imbalanced}, which is an analog of Figure \ref{fig:weight_decay_ablation} for the imbalanced Cifar-10 dataset detailed in Section \ref{sec:convergence}. We again see that using weight decay controls for the embedding norms and improves the convergence of both models. In correspondence with the other results on imbalanced training, we find that stronger control over the embedding norms leads to improved convergence: the high weight decay value does not perform as poorly on SimCLR as in Figure \ref{fig:weight_decay_ablation} and, on SimSiam, outperforms the other weight decay options.

\begin{figure}
    \centering
    \begin{tikzpicture}
    \node () at (0, 0) {\includegraphics[width=0.4\linewidth]{Images/wd_sweep_imbalanced.png}};

    \draw[ballblue, line width=0.07cm] (-4, -3.8) -- (-3.4, -3.8);
    \draw[azure, line width=0.07cm] (-1.3, -3.8) -- (-0.7, -3.8);
    \draw[darkblue, line width=0.07cm] (2, -3.8) -- (2.6, -3.8);

    \node () at (-1.15, -3.33) {\small \textcolor{darkgray}{Train Epoch}};
    \node () at (1.95, -3.33) {\small \textcolor{darkgray}{Train Epoch}};

    \node[inner sep=0pt] () at (-2.5, -3.82) {\textcolor{darkgray}{\scriptsize No weight decay}};
    \node[inner sep=0pt] () at (0.48, -3.82) {\textcolor{darkgray}{\scriptsize Standard weight decay}};
    \node[inner sep=0pt] () at (3.57, -3.82) {\textcolor{darkgray}{\scriptsize High weight decay}};
        
        
    \end{tikzpicture}
    \caption{An analog to Figure \ref{fig:weight_decay_ablation} performed on the exponentially imbalanced Cifar-10 dataset. Weight decays are [$0$, $1$e-$5$, $5$e-$2$] for SimCLR and [$0$, $5$e-$4$, $5$e-$2$] for SimSiam. We plot the effective learning rate in the bottom row, calculated in accordance with Section \ref{sec:convergence}.}
    \label{fig:weight_decay_imbalanced}
\end{figure}

\subsection{Cut-Initialization}
\label{app:cut_init}
We plot the effect of the cut constant on the embedding norms and accuracies over training in Figure~\ref{fig:cut_experiments}. To make the effect more apparent, we use weight-decay $\lambda=5e-4$ in all models. We see that dividing the network's weights by $c>1$ leads to immediate convergence improvements in all models. Furthermore, this effect degrades gracefully: as $c > 1$ becomes $c < 1$, the embeddings stay large for longer and, as a result, the convergence is slower. We also see that cut-initialization has a more pronounced effect in attraction-only models -- a trend that remains consistent throughout the experiments.

We also show the relationship between cut-initialization and the network's batch size on SimSiam in Table \ref{tbl:cut_batch_size}. Consistent with the literature, we see that training with large batches provides improvements to training accuracy. However, we note that larger batch sizes also significantly slow down convergence. However, cut-initialization seems to counteract this and accelerate convergence accordingly. Thus, training with cut-initialization and large batches seems to be the most effective method for SSL training (at least in the non-contrastive setting).

\begin{figure}[t!]
    \centering
    \includegraphics[width=0.95\textwidth]{Images/init_experiments.png}
    \caption{The effect of cut-initialization on Cifar10 SSL representations. $x$-axis and embedding norm's $y$-axis are log-scale. $\lambda=5$e$-4$ in all experiments.}
    \label{fig:cut_experiments}
\end{figure}

\section{More details on gradient scaling layer}
\label{app:grad_scaling}

An implementation of our GradScale layer can be found in Listing \ref{alg:grad_scaling_use}.
We note that this layer is purely a PyTorch optimization trick and does not amount to implicitly choosing a different loss function:

\begin{restatable}{proposition}{nopotential}
    \label{prop:no_potential}
    Let $t\in\mathbb{R}^n$ be a unit vector, $p: \mathbb{R}^n\backslash \{0\} \to [-1, 1], z\mapsto t^\top z/\|z\|$ the cosine similarity with respect to $t$, $\alpha \in \mathbb{R}$, and $\sigma: \mathbb{R}^n \to  \mathbb{R}, z\mapsto \|z\|^\alpha$. Then the vector field $\sigma\nabla p$ has a potential $q$, i.e., $\nabla q = \sigma \nabla p$, only for $\alpha=0$.
\end{restatable}

\begin{proof}
    Suppose $\sigma \nabla p$ has potential. Consider two paths with segments $s_1, s_2$ and $s_3, s_4$ going $t \to 2t \to -2t$ and $t \to -t \to -2t$, where the segments $s_1, s_4$ scaling $\pm t \to \pm2t$ are straight lines and the other segments $s_2, s_3$ follow great circles on $S^{n-1}$. By Proposition~\ref{prop:cos_sim_grads}, we know that $\nabla p(z)=0$ for $z\in \mathbb{R}_{\neq 0}\cdot t$. So $\sigma \nabla p$ is zero on $s_1$ and $s_4$. Moreover, we have
    \begin{align}
        \int_{s_2} \sigma \nabla p \,dz &= \int_{s_2} \|z\|^\alpha \nabla p \,dz
        = \int_{s_2} 2^\alpha \nabla p \,dz 
        = 2^\alpha \int_{s_2} \nabla p \,dz 
        = 2^\alpha \big(p(2t) - p(-2t)\big) = 2^{\alpha+1}
    \end{align}
    and similarly 
    \begin{align}
        \int_{s_3} \sigma \nabla p dz = 1^\alpha \cdot 2 = 2.
    \end{align}
    Since we assume the existence of a potential, we can use path independence to conclude 
    \begin{align}
        2^{\alpha+1} &= \int_{s_2} \sigma \nabla p \,dz 
        = \int_{s_1, s_2} \sigma \nabla p \,dz 
        = \int_{s_3, s_4} \sigma \nabla p \,dz 
        = \int_{s_3} \sigma \nabla p \,dz 
        = 2.
    \end{align}
    Thus, $\alpha=0$ and $\sigma$ does not perform any scaling.
\end{proof}




\begin{figure}
    \begin{lstlisting}[caption={PyTorch code for gradient scaling layer}, label={alg:grad_scaling}]
class scale_grad_by_norm(torch.autograd.Function):
    @staticmethod
    def forward(ctx, z, power=0):
        ctx.save_for_backward(z)
        ctx.power = power
        return z
    @staticmethod
    def backward(ctx, grad_output):
        z = ctx.saved_tensors[0]
        power = ctx.power
        norm = torch.linalg.vector_norm(z, dim=-1, keepdim=True)
        return grad_output * norm**power, None
\end{lstlisting}
\end{figure}

\begin{algorithm}[tb]
   \caption{Pytorch-like pseudo-code using the gradient scaling layer}
   \label{alg:grad_scaling_use}
\begin{algorithmic}
   \STATE {\bfseries Input:} Encoder network $model$, gradient scaling power $p$
   \STATE $z = model(batch)$
   \STATE $z = grad\_scaling\_layer.apply(z, p)$
   \STATE $sim = (\frac{z}{\|z\|})^T \frac{z}{\|z\|}$
   \STATE $loss = InfoNCE(sim)$
   \STATE $loss.backward()$
\end{algorithmic}
\end{algorithm}


\section{Additional figures}
We provide a bar plot analogous to Figure \ref{fig:in_out_violin} in Figure \ref{fig:in_out_distribution_norms}.

\begin{figure}
    \centering
    \begin{tikzpicture}   
        \node[inner sep=0pt] (image) at (0,0) {\includegraphics[width=\textwidth]{Images/Confidence/per_class_norms.pdf}};
    \end{tikzpicture}
    \caption{Bar plot which is analogous to Figure \ref{fig:in_out_violin} showing embedding magnitudes on each dataset split as a function of which dataset the model was trained on. All values are normalized by training set's mean embedding magnitude. Normalized means are represented by black bars. We use the same data augmentations for the train and test sets for consistency.}
    \label{fig:in_out_distribution_norms}
\end{figure}

We also show each Cifar-10 class's 10 highest and 10 lowest embedding-norm samples in Figure \ref{fig:cifar_norms}. These are obtained after training default SimCLR on Cifar-10 for 512 epochs. We see that the high-norm class representatives are prototypical examples of the class while the low-norm representatives are obscure and qualitatively difficult to identify. This property was originally shown by \citet{embed_norm_confidence_2}.

\begin{figure}
    \centering
    \includegraphics[width=0.48\linewidth]{Images/high_norm.png}
    \quad
    \includegraphics[width=0.48\linewidth]{Images/low_norm.png}
    \caption{\emph{Left}: highest-norm representatives (top 10) per class. \emph{Right}: lowest-norm representatives (bottom 10) per class. All from default SimCLR trained on Cifar-10.}
    \label{fig:cifar_norms}
\end{figure}



\end{document}
\endinput

