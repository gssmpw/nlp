\documentclass[acmlarge, imwut]{acmart}

\usepackage[export]{adjustbox}
\usepackage{enumitem}
\usepackage{multirow}
\usepackage{arydshln}
\PassOptionsToPackage{table}{xcolor}
\usepackage{colortbl}
\usepackage[capitalise]{cleveref}


%%
%% \BibTeX command to typeset BibTeX logo in the docs
\AtBeginDocument{%
  \providecommand\BibTeX{{%
    Bib\TeX}}}

\setcopyright{acmlicensed}
\copyrightyear{2025}
\acmYear{2025}
\acmDOI{XXXXXXX.XXXXXXX}

\begin{document}

\title[Data-driven Identification of Sub-activities via Clustering and Visualization for Enhanced Activity Recognition in Smart Homes]{DISCOVER: \underline{D}ata-driven \underline{I}dentification of \underline{S}ub-activities via \underline{C}lustering and \underline{V}isualization for \underline{E}nhanced \underline{R}ecognition of Human Activities in Smart Homes}

%% The "author" command and its associated commands are used to define
%% the authors and their affiliations.
%% Of note is the shared affiliation of the first two authors, and the
%% "authornote" and "authornotemark" commands
%% used to denote shared contribution to the research.
\author{Alexander Karpekov}
\email{alex.karpekov@gatech.edu}
\orcid{0009-0006-5317-2991}
\affiliation{%
  \institution{Georgia Institute of Technology}
  \city{Atlanta}
  \state{Georgia}
  \country{USA}
}

\author{Sonia Chernova}
\email{chernova@gatech.edu}
\orcid{0000-0001-6320-0825}
\affiliation{%
  \institution{Georgia Institute of Technology}
  \city{Atlanta}
  \state{Georgia}
  \country{USA}
}

\author{Thomas Pl{\"o}tz}
\email{thomas.ploetz@gatech.edu}
\orcid{0000-0002-1243-7563}
\affiliation{%
  \institution{Georgia Institute of Technology}
  \city{Atlanta}
  \state{Georgia}
  \country{USA}
}

%%
%% By default, the full list of authors will be used in the page
%% headers. Often, this list is too long, and will overlap
%% other information printed in the page headers. This command allows
%% the author to define a more concise list
%% of authors' names for this purpose.

\renewcommand{\shortauthors}{Karpekov et al.}


\definecolor{rubinered}{HTML}{CE0058}
\newcommand{\thomas}[1]{\textcolor{rubinered}{[#1 -- TP]}}

\definecolor{bleudefrance}{rgb}{0.19, 0.55, 0.91}
\newcommand{\sonia}[1]{{{\textcolor{bleudefrance}{[#1 -- SC]}}}}

\definecolor{green}{rgb}{0.0, 0.65, 0.31}
\newcommand{\alex}[1]{{{\textcolor{green}{[#1 -- AK]}}}}

\definecolor{green}{rgb}{0.0, 0.65, 0.31}
\newcommand{\sourish}[1]{{{\textcolor{red}{[#1 -- SGD]}}}}

\newcommand{\ToolName}{DISCOVER}

\begin{abstract}
\begin{abstract}

% Recent works to jointly reconstruct 3D human and object from a single RGB image, are mostly model-based, that fail to capture the fine details of the clothed human body and object surface. In this paper, we introduce ReCHOR, a novel, model-free, first-method to produce realistic clothed human-object reconstructions from a monocular view. This is extremely challenging due to human-object occlusions, diverse interactions and depth ambiguity, as it needs to infer both 3D spatial awareness and high resolution details. Our core idea is based on estimating neural implicit representations for human and object respectively by an attention-based neural implicit model that attends to pixel-aligned features from both the global human-object image for spatial awareness and  the local separate view of human and object images for high quality details. Additionally, the network is conditioned on semantic features from an initial estimated human-object pose prior and a generative diffusion model that inpaints occluded regions, thus enabling the retrieval of details from them.
% We also propose a synthetic dataset with rendered scenes of diverse, inter-occluded 3D human and object scans, to train our network. We evaluate our method on the synthetic and real world BEHAVE dataset. Our experiments show that our method outperforms the SOTA in achieving realistic clothed human-object reconstructions.
Recent approaches to jointly reconstruct 3D humans and objects from a single RGB image represent 3D shapes with template-based or coarse models, which fail to capture details of loose clothing on human bodies. In this paper, we introduce a novel implicit approach for jointly reconstructing realistic 3D clothed humans and objects from a monocular view. For the first time, we model both the human and the object with an implicit representation, allowing to capture more realistic details such as clothing. This task is extremely challenging due to human-object occlusions and the lack of 3D information in 2D images, often leading to poor detail reconstruction and depth ambiguity. To address these problems, we propose a novel attention-based neural implicit model that leverages image pixel alignment from both the input human-object image for a global understanding of the human-object scene and from local separate views of the human and object images to improve realism with, for example, clothing details. Additionally, the network is conditioned on semantic features derived from an estimated human-object pose prior, which provides 3D spatial information about the shared space of humans and objects. To handle human occlusion caused by objects, we use a generative diffusion model that inpaints the occluded regions, recovering otherwise lost details. For training and evaluation, we introduce a synthetic dataset featuring rendered scenes of inter-occluded 3D human scans and diverse objects. Extensive evaluation on both synthetic and real-world datasets demonstrates the superior quality of the proposed human-object reconstructions over competitive methods.
\end{abstract}
\end{abstract}

\begin{CCSXML}
<ccs2012>
   <concept>
       <concept_id>10003120.10003138</concept_id>
       <concept_desc>Human-centered computing~Ubiquitous and mobile computing</concept_desc>
       <concept_significance>500</concept_significance>
       </concept>
   <concept>
       <concept_id>10010147.10010257.10010293</concept_id>
       <concept_desc>Computing methodologies~Machine learning approaches</concept_desc>
       <concept_significance>500</concept_significance>
       </concept>
 </ccs2012>
\end{CCSXML}

\ccsdesc[500]{Human-centered computing~Ubiquitous and mobile computing}
\ccsdesc[500]{Computing methodologies~Machine learning approaches}


%%
%% Keywords. The author(s) should pick words that accurately describe
%% the work being presented. Separate the keywords with commas.
\keywords{human activity recognition, smart home, ambient sensors, active learning, clustering}

\maketitle

%%% custom footer, remove for submission ----------
 \thispagestyle{fancy}
%... then configure it.
\fancyhead{} % clear all header fields
\fancyfoot{} % clear all footer fields
\pagenumbering{gobble}
\fancyfoot[C]{\textcolor{red}{This manuscript is under review. Please contact alex.karpekov@gatech.edu for up-to-date information}}
%---------------------------------------------------

\section{Introduction}
\label{sec:intro}
% Image editing methods in diffusion models depend on user-defined control directions - users can unlock their creativity using these methods by specifying the desired manipulation through prompts~\cite{gandikota2023concept}, reference images~\cite{ruiz2022dreambooth, kumari2022customdiffusion, gal2022image, chen2024trainingfreeregionalpromptingdiffusion}, or attribute vectors~\cite{parmar2023zero,hertz2022prompt}. In this work, we ask a fundamentally different question: \emph{Can we automatically discover the underlying visual structure of a concept within diffusion model's knowledge?} %Rather than requiring user-specified controls, we aim to decompose the model's internal knowledge into meaningful directions.

% This question touches on a fundamental limitation in how we interact with diffusion models. Current control methods ~\cite{zhang2023addingconditionalcontroltexttoimage, gandikota2023concept, ye2023ipadaptertextcompatibleimage,ye2023ipadaptertextcompatibleimage, hertz2024stylealignedimagegeneration, li2023photomaker, shi2024instantbooth, chen2024trainingfreeregionalpromptingdiffusion} require users to specify their desired manipulations in advance, limiting interactive creativity. This contrasts with natural human artistic workflows, where creators dynamically explore creative ideas while jointly refining them toward meaningful artistic outcomes~\cite{hoffmann2016modeling}. This synergy between specification and exploration is not new to generative models. Early GAN architectures naturally developed disentangled latent spaces that enabled continuous\cite{harkonen2020ganspace,radford2015unsupervised, wu2021stylespace, shen2020interfacegan}, compositional control over generated images. Users could explore these spaces to discover interesting variations that would be difficult to describe in words~\cite{wu2021stylespace}, then combine them to achieve their creative goals~\cite{grabe2022towards}. 


% While diffusion models have largely superseded GANs in conditional image synthesis~\cite{dhariwal2021diffusion},  their underlying structure remains less understood. Diffusion models achieve remarkable diversity through high-dimensional latents, unlike GANs' compact latent spaces.  With a single prompt, diffusion models can generate radically different variations through different random initializations of input noise. We ask - Is it possible to discover interpretable structure within this vast space of variations?

Text-to-image diffusion models are capable of generating remarkable visual variations from a single prompt through different random initializations. However, this vast creative potential remains largely opaque to users---while we can generate diverse images, we lack understanding of the underlying structure of these variations. This presents a fundamental challenge: how can we discover and expose the latent visual capabilities encoded within these models?

\let\thefootnote\relax \footnote{$^{*}$Correspondence to \texttt{gandikota.ro@northeastern.edu}}

The challenge touches on a key limitation in how we interact with diffusion models today. Current control methods require users to explicitly specify their desired edits in advance through prompts~\cite{gandikota2023concept}, reference images~\cite{zhang2023addingconditionalcontroltexttoimage, chen2024trainingfreeregionalpromptingdiffusion, ruiz2022dreambooth,kumari2022customdiffusion, Ryu_lora, hu2021lora}, or attribute vectors~\cite{ye2023ipadaptertextcompatibleimage, hertz2024stylealignedimagegeneration, li2023photomaker, shi2024instantbooth,parmar2023zero,hertz2022prompt}. That contrasts sharply with natural human creative workflows, where artists dynamically explore creative ideas and jointly refine them toward meaningful artistic outcomes~\cite{hoffmann2016modeling}. The need for pre-specified controls creates a barrier between users and the full creative potential of these models.

Interestingly, earlier generative models like GANs~\cite{gans,karras2019style,brock2018large} naturally developed more interpretable internal structures. Their compact latent spaces often exhibited emergent disentanglement~\cite{harkonen2020ganspace,radford2015unsupervised, wu2021stylespace, shen2020interfacegan}, enabling continuous and compositional control over generated images. Users could explore these spaces to discover interesting variations that would be difficult to describe in words~\cite{wu2021stylespace}, then combine them to achieve their creative goals~\cite{grabe2022towards}.

Diffusion models have largely superseded GANs in conditional image synthesis~\cite{dhariwal2021diffusion}, achieving greater diversity through much higher-dimensional latents. And yet an understanding of the underlying structure of these larger latent spaces has remained elusive. In this work, we ask a fundamental question: \emph{Can we automatically discover the visual structure within a diffusion model's knowledge of a concept?} Rather than requiring user-specified controls, we aim to decompose the model's internal representations into expressive directions that users can explore and combine.

To address these needs, we present \textbf{SliderSpace}, a framework that brings systematic explorability to diffusion models. Given just a text prompt, SliderSpace discovers a canonical set of meaningful, diverse, and controllable directions within the model's knowledge of that concept. Each direction is implemented as a low-rank adapter~\cite{hu2021lora} that can be scaled and composed with others, allowing users to explore and smoothly combine different aspects of variation, as shown in Figure~\ref{fig:intro}.

We ground SliderSpace discovery in three key requirements for meaningful decomposition of a diffusion model's visual manifold: 
\begin{enumerate}
    \item \textbf{Unsupervised Discovery:} The decomposition process should emerge from the intrinsic structure of the model's learned representation, rather than being guided by predefined attributes. This ensures we capture the true topology of the model's knowledge space rather than projecting our assumptions onto it.
    
    \item \textbf{Semantic Orthogonality:} Each discovered control must represent a distinct semantic direction. This is enforced in a semantic feature space, like CLIP, where every slider has an orthogonal effect in embeddings. This prevents discovering multiple controls that create similar semantic effects, making the system more efficient and easier.
    
    \item \textbf{Distribution Consistency:} Directions must induce consistent transformations across both random seeds and prompt variations. 
\end{enumerate}

These requirements naturally lead to our proposed framework, which we formalize in Section~\ref{sec:method}. As we show in our experiments, SliderSpace is architecture-agnostic, working with both conventional U-Net based models like Stable Diffusion~\cite{rombach2022high, rombach2022sd20, podell2023sdxl, turbo, dmd} and recent transformer-based architectures like Flux~\cite{flux}.

We demonstrate the expressiveness of SliderSpace through three applications: First, we show how SliderSpace can decompose high-level concepts into diverse and expressive components, revealing the natural axes of variation in the model's understanding. Second, we explore artistic style variation, where SliderSpace discovers directions that match or exceed the diversity of manually curated artist lists while being judged more useful by human evaluators. Finally, we show how SliderSpace can help reverse the mode collapse commonly observed in distilled diffusion models, restoring diversity while maintaining generation speed.

Beyond providing practical creative control, SliderSpace opens new avenues for understanding and utilizing the latent capabilities of diffusion models. By mapping these models' visual potential into intuitive, composable directions, we take a step toward making their creative possibilities more accessible and interpretable to users.

% Image editing methods in diffusion models unlock the creativity of users. In this work we ask an alternate question: \emph{Can we organize and expose what of the diffusion model is already capable of?}.
% Existing methods for controlling image generation typically require users to manually specify edit directions for desired changes. This process is time-consuming, requires technical expertise, and limits the spontaneity of the creative process. For instance, if a user wants to adjust the smile of a generated person, they must explicitly request this edit, often through imprecise prompt engineering or model fine-tuning. This approach of predefined controls or manual specifications restricts users from fully exploring the latent capabilities of the model. There may be interesting stylistic variations or attributes that the model can generate, but users have no easy way to discover or utilize these.

% Natural visual disentanglement was an emergent property in the latent space of Generative Adversarial Models (GANs) \cite{harkonen2020ganspace,radford2015unsupervised, wu2021stylespace, shen2020interfacegan}. In particular, it has been observed that StyleGAN~\cite{karras2019style} stylespace neurons offer detailed control over many meaningful aspects of images that would be difficult to describe in words~\cite{wu2021stylespace}. However, diffusion models do not share such a compact latent space~\cite{park2023unsupervised}; and efforts to uncover such a space in the semantic embeddings of the text conditioning have met with limited success \nik{Nick - is there a specific citation you were thinking about?}.

% In this work we introduce \textbf{SliderSpace}, which takes a step towards uncovering an analogous low dimensional representation of diffusion models' visual breadth; in essence treating the diffusion model as many generators sharing parameters, where a particular generator is defined by a specific prompt. For a given prompt we sample many random seeds (and optionally prompt expansions using an LLM), generate the corresponding images, and apply an off the shelf feature extractor (in this work CLIP, but our method can be applied to any differentiable feature extractor). We use PCA to analyze these features, and for each of the leading $k$ principal components we train a LoRA \cite{} which causes the diffusion model to produces images which increase the feature magnitude along that component when passed back through the same feature extractor. This leads to a 'Slider' for each principal component, because each LoRA can be scaled and applied to the original diffusion model, continuously varying those visual features in the generated results (as measured, in our case, by CLIP).

% There are many other works that enhance the controllability of diffusion models. One common approach is enabling users to add spatial constraints to a generation either manually, or via a reference image \cite{zhang2023addingconditionalcontroltexttoimage, chen2024trainingfreeregionalpromptingdiffusion}, a second is leveraging more abstract embeddings (e.g. identity, style) extracted from a reference image \cite{ye2023ipadaptertextcompatibleimage, hertz2024stylealignedimagegeneration, li2023photomaker, shi2024instantbooth}, a third is finetuning a foundation model to better generate a concept important to the user \cite{ruiz2022dreambooth, kumari2022customdiffusion, Ryu_lora, hu2021lora}, and a fourth (most relevant to this work) is finding low-rank adaptors of the model based on a prompt or small training set which can be scaled to provide continous control over one aspect of generated image (e.g. night vs day, basic vs luxury, etc.) \cite{gandikota2023concept}. SliderSpace is complementary to all of these methods and offers something distinct. All of the other methods we are aware require the user (and / or model designer) to know in advance what type of control they want. In contrast SliderSpace assists users in discovering and controlling hidden capabilities present in the diffusion model's distribution of possible generations.

%We propose that truly intuitive creative control in a text-to-image model should meet three key criteria: \emph{discoverability}, \emph{intuitiveness}, and \emph{specificity}. The model should reveal controllable attributes that may not be immediately obvious, offer controls that are easy to understand and manipulate, and ensure each control affects a distinct attribute of the generated image.

% We demonstrate the utility and power of SliderSpace using three applications built on top of SDXL-DMD \cite{dmd}, because its fast generation speed lends itself well to the continuous control offered by SliderSpace.

% First, we study concept decomposition (Section \ref{sec:concept_exp}), where we learn sliders for a specific concept (e.g. 'monster', 'waterfall', 'car'). Through quantitative metrics of diversity and text alignment we demonstrate that the learned sliders dramatically boost the diversity of generations when randomly applied without harming text alignment; we also ask humans to qualitatively judge these results in a user study where they find the SliderSpace results to be more 'Diverse', 'Useful', and 'Creative' than our baselines.

% Second, we attempt to compare the automatic discoveries of SliderSpace to a large scale manual study of artistic styles (Section \ref{sec:art_exp}), open-sourced by ParrotZone \cite{parrotzone}. In this study SDXL was prompted with over 4300 artist names,  and based on visual inspection the cases of successful stylistic mimicry recorded. Quantitatively SliderSpace more closely matches the distribution of artistic variation discovered by ParrotZone than other baselines, and in our user studies was judged to be significantly more 'Diverse' and 'Useful' than the baselines. To our surprise humans even judged SliderSpace results to be slightly more 'Diverse' than the results generated by the manually discovered artist names of \cite{parrotzone}.

% Third, we attempt to use SliderSpace to reverse the mode collapse commonly observed in distilled few-step diffusion models relative to the original teacher model (Section \ref{sec:diverse_exp}). We quantitatively demonstrate that applying SliderSpace to SDXL-DMD leads to more closely matching the distribution of images by the original teacher, SDXL.

%Through extensive experiments on various state-of-the-art text-to-image models, we demonstrate that SliderSpace significantly enhances user control and creative expression in AI-assisted image generation tasks. Our method enables a range of applications, including concept decomposition and control, diversity improvement in generated images, customization dissection and edits, and the exploration of artistic styles inherent in the model.

% SliderSpace goes beyond providing a practical tool for enhanced creative control. By mapping the visual potential of diffusion models it can open new avenues for generative creativity and deepens our understanding of each model's hidden potential.

\section{Related Work}
\label{sec:related_work}

The original investigation \cite{gibson1979ecological} on the relationship between visual perception and human action defines \emph{affordance} as the opportunities for interaction with the surrounding environment. Behavioral studies on regular and cognitively impaired persons have shown evidence that perception results in both visual and motor signals in the human brain. An extended study \cite{anderson2002attentional} shows that visual attention to the spatial characteristics of the perceived objects initiates automatic motor signals for different actions. In computer vision, human affordance learning involves novel pose prediction such that the estimated pose represents a valid human action within the scene context. The task is fundamental to many problems requiring robust semantic reasoning about the environment, such as human motion synthesis \cite{wang2021scene} and scene-aware human pose generation \cite{wang2017binge, roy2016multi, zhang2022inpaint, yao2023scene}.

Earlier methods of affordance learning have explored knowledge mining \cite{zhu2014reasoning} and multimodal feature cues \cite{roy2016multi} to address the problem. In \cite{zhu2014reasoning}, the authors use a Markov Logic Network for constructing a knowledge base by extracting several object attributes from different image and metadata sources, which can perform various downstream visual inference tasks without any additional classifier, including zero-shot affordance prediction. In \cite{roy2016multi}, the authors use depth map, surface normals, and segmentation map as multimodal cues to train a multi-scale convolutional neural network (CNN) for scene-level semantic label assignment associated with specific human actions. In \cite{do2018affordancenet}, the authors design a multi-branch end-to-end CNN with two separate pathways for object detection and affordance label assignment to achieve high real-time inference throughput. Researchers \cite{chuang2018learning} have also explored socially imposed constraints for affordance learning. In \cite{chuang2018learning}, the authors propose a graph neural network (GNN) to propagate contextual scene information from egocentric views for action-object affordance reasoning.

Probabilistic modeling of scene-aware human motion generation also involves semantic reasoning of human interaction with the environment. Initial works on human motion synthesis have taken different architectural approaches, such as sequence-to-sequence models \cite{barsoum2018hp}, generative adversarial networks (GAN) \cite{barsoum2018hp, cai2018deep, yang2018pose}, graph convolutional networks (GCN) \cite{yan2019convolutional}, and variational autoencoders (VAE) \cite{guo2020action2motion}. However, these methods have mostly ignored the role of environmental semantics. Due to potential uncertainty in human motion, in a recent approach \cite{wang2021scene}, the authors address such motion synthesis with a GAN conditioned on scene attributes and motion trajectory to predict probable body pose dynamics.

One key challenge of human affordance generation in 2D scenes is the lack of large-scale datasets with rich pose annotations. In \cite{wang2017binge}, the authors compile the only public dataset of annotated human body poses in complex 2D indoor scenes by extracting frames from sitcom videos. Aiming to generate a contextually valid human affordance at a user-defined location, the authors propose sampling the scale and deformation parameters for an existing human pose template using a VAE conditioned on the localized image patches as scene context. In \cite{zhang2022inpaint}, the authors introduce a two-stage GAN architecture for achieving a similar goal by estimating the affine bounding box parameters to localize a probable human in the scene and then generating a potential body pose at that location. The method uses the input scene, corresponding depth, and segmentation maps as semantic guidance. In \cite{yao2023scene}, the authors propose a transformer-based approach with knowledge distillation for generating human affordances in 2D indoor scenes.



\section{Privacy-Preserving Intrusion Detection} 
\label{section:approach}

This section provides an overview of our approach towards a privacy-preserving anomaly-based IDS. It describes the proposed data-driven detection model together with problem formalization, and an approach to enable privacy-friendly detection using federated learning.

\subsection{Autoencoder-based Intrusion Detection}

Autoencoders have emerged as an effective tool for anomaly detection, particularly in cyber-physical systems. By learning compressed representations of normal system behavior, these models are efficient at reconstructing input data with lower error under normal conditions. Anomalies, which deviate significantly from learned patterns, result in higher reconstruction errors, enabling their identification. This data-driven approach is particularly advantageous for cyber-physical systems, where anomalies are often rare, and explicit labeling is impractical. In detail, the application of autoencoders involves the following key steps:

\begin{enumerate}
    \item Training: The autoencoder is trained using data representing the normal operation of the system. During training, the network learns to encode input data into a lower-dimensional latent space and reconstruct it back to its original form. The goal is to minimize the reconstruction error.
    \item Reconstruction Error: Once trained, the autoencoder can process new data and attempt to reconstruct it. For normal data, reconstruction errors are typically small, as the autoencoder has learned these patterns during training. However, for anomalous data, which deviates significantly from normal patterns, reconstruction errors tend to be larger.
    \item Setting a Threshold: To distinguish between normal and anomalous behavior, a reconstruction error threshold is determined. This threshold is often set based on statistical measures (e.g., mean and standard deviation of reconstruction errors on the training set) or by analyzing the error distribution during validation.
    \item  Testing and Detection: During testing, the autoencoder processes new data and computes the reconstruction error for each sample. If the error exceeds the predefined threshold, the data point is flagged as an anomaly. This enables both real-time or offline anomaly detection in dynamic systems.
\end{enumerate}
    

\subsection{Problem Definition}

We focus on detecting anomalies in the EC’s operational data using a time-series anomaly detection model based on a reconstruction autoencoder model. The problem is formalized as follows. Let the energy community's operational data be represented as a multivariate time series:

\begin{equation}
\mathcal{X} = \{X_1, X_2, \dots, X_T\},
\end{equation}
%T represents total number of time steps

where each $X_t \in \mathbb{R}^d$ is a feature vector of $d$ dimensions at time step $t$, meaning there are $d$ different numerical measurements recorded at each time step. Each feature represents an energy or power related measurement, such as battery current ($I_{\text{batt}}$), load power ($P_{\text{load}}$), and voltage ($V$). The goal is to distinguish between normal operational data and anomalous data based on probability distributions. The model assumes two types of data distributions, as described below.

\begin{itemize} 
    \item \textbf{Normal Data:} When the energy system operates under normal conditions, its data follows a probability distribution:
    \begin{equation}
    X_t \sim p_{\text{normal}}(X).
    \end{equation}

    \item \textbf{Anomalous Data:} Attacks cause deviations from $p_{\text{normal}}(X)$, generating out-of-distribution samples:
    \begin{equation}
    X_t \sim p_{\text{attack}}(X), \quad \text{where} \quad p_{\text{attack}}(X) \neq p_{\text{normal}}(X).
    \end{equation}


An autoencoder $f_\theta: \mathbb{R}^{d \times T} \to \mathbb{R}^{d \times T}$ where d x T represents operating on an entire segment of length T, learns to reconstruct normal data. The goal is to train the autoencoder to learn an identity mapping:
    \begin{equation}
        f_\theta(X) \approx X.
    \end{equation}
    where \( f_\theta(X) \) is the autoencoder function parametrized by \( \theta \).  Given an input $X$, the model reconstructs it as:
    \begin{equation}
        \hat{X} = f_\theta(X).
    \end{equation}
   

The model is trained only on normal data, so it becomes good at reconstructing normal patterns but struggles with anomalous data. In other words, it has low reconstruction error for normal samples and high reconstruction error for anomalies. The reconstruction error $e$ is computed using the \textit{Mean Absolute Error (MAE)}:
    \begin{equation}
        e(X) = \frac{1}{dT} \sum_{i=1}^{d} \sum_{t=1}^{T} |X_{t, i} - \hat{X}_{t, i}|.
    \end{equation}
This calculates the average difference between the original and reconstructed data. Higher error means the model failed to reconstruct the input well, indicating a potential anomaly. For anomaly classification, a threshold $\tau$ is determined from the reconstruction error $e$ distribution of normal data. A given sample $X$ is thus classified as anomalous if:
    \begin{equation}
        e(X) > \tau.
    \end{equation}
    Otherwise, it is considered normal.

\end{itemize}

\subsection{Private Anomaly Detection}

In traditional and centralized machine learning, data from different sources is aggregated into a single dataset and transferred to a central server, where the model is trained. Since raw data is required to be sent to a central location, this approach suffers from several drawbacks. First, there are privacy risks with collecting and transferring raw sensitive data. Second, this approach requires high network bandwidth, and third, the central server could prove to be a single point of failure. Federated learning (FL) on the other hand is a decentralized machine learning approach which allows multiple devices or nodes to collaboratively train a shared model without sharing their local raw data. Instead of transferring data to a single central server, federated learning allows each node to train the model on its local data and share only the updated model parameters. This approach preserves data privacy as instead of data moving towards the model, the model is essentially moved closer to the data, reduces communication costs, and is particularly valuable in scenarios where data is sensitive or distributed across various locations, such as in ECs.

%each client aims to minimize a loss function related to their data

The FL process begins with the server initializing a global model and distributing its parameters to the clients. Each client trains the model on its local dataset, which consists of normal data, and computes updates to the model parameters. These updates are then sent back to the server, which is responsible for aggregation to improve the global model and distribution of the global model parameters to the clients. This process repeats for a predefined number of communication rounds until the training process is complete, and the trained global model is distributed to all the clients.

In ECs, where multiple households or facilities generate and consume energy, detecting anomalies is critical for ensuring reliable and efficient grid operations. By combining FL with autoencoders, it is possible to build a robust, privacy-preserving anomaly detection system tailored to such communities. Each node in the EC (e.g., a household or facility) can train an autoencoder locally using its energy consumption and generation data. The autoencoder learns to reconstruct normal patterns specific to the node. When new data deviates significantly, resulting in high reconstruction errors, anomalies can be detected locally. This combination offers several advantages: it preserves the privacy of individual nodes by keeping their data local, accommodates the heterogeneity of energy patterns across the community, and ensures a more accurate and adaptive anomaly detection system through collaborative learning. 


%Deployment architecture of the IDPS:    Placement within the REC infrastructure.     Interaction with the local area network (LAN). Real-time detection and mitigation capabilities.

%\begin{figure}
%    \centering
%    \includegraphics[width=0.7\linewidth]{FL-IDS.png}
%    \caption{Federated approach to model training in ECs}
%    \label{fig:FL-IDS}
%\end{figure}




\section{\ourdata}
\label{sec:textbook-exam}

\begin{figure}[t]
    \centering
    \includegraphics[width=\linewidth]{figures/data_framework.pdf}    \caption{\textbf{Overview of  \ourdata curation}. Given a chapter $C$, we use an LM to segment the document $D$ into sections and heuristically extract review questions to form the exam $E$. The LM then classifies each question in $E$ by Bloom’s taxonomy category and maps it to its relevant section.}
    \label{fig:dataset-overview}
    \vspace{-0.3cm}
\end{figure}

In order to evaluate \ours, we curate  \ourdata, a dataset where each entry contains a document \(D\) along with a corresponding set of exam questions \(E\).
An overview of \ourdata is illustrated in \autoref{fig:dataset-overview}.
% In this section, we first describe our data processing pipeline (\secref{ssec:textbook-exam-pipeline}) and then provide data statistics (\secref{ssec:textbook-exam-statistics}).

\subsection{Data Processing}
\label{ssec:textbook-exam-pipeline}
Our pipeline starts with textbooks from the OpenStax repository\footnote{\url{https://github.com/philschatz/textbooks}}.
Each textbook is divided into chapters, where each chapter \(C\) contains learning objectives, main content, and review questions.
For each \(C\), we parse the main content to build \(D\) and the review questions to form \(E\).
% Specifically, only the main content is used to construct \(D\), while the review questions are used to create \(E\).

\paragraph{Extracting sections.}
To simulate a learner incrementally progressing through a chapter, we divide each chapter into sections using an LM-based document structuring method. 
The LM segments \( D \) into \( n \) sections, denoted as \( \{S_1, S_2, \ldots, S_n\} \subset D \), while also extracting the corresponding review questions \( E \). 
However, not all review questions come with ground-truth answers, as some textbooks do not provide them (see Table~\ref{tab:textbook-exam-statistics} for the proportion of \( E \) with answers). 
To ensure consistency in section segmentation across different subjects, we manually annotate the first 2–5 sections from one sample per subject. 
These annotated samples serve as few-shot examples in our LM prompt (see Appendix~\ref{appdx:parsing-sections} for details).

\paragraph{Extracting questions.}
% We developed a custom parsing script using BeautifulSoup4 to extract questions and their corresponding answers. 
To maintain a balance between evaluation depth and computational feasibility, we include only chapters that contain at least 10 questions—ensuring sufficient coverage for assessment—while capping the maximum number of questions at 25 to keep learning simulations computationally manageable.

\subsection{Data Statistics}
\label{ssec:textbook-exam-statistics}
\begin{table}[t!]
    \centering
    \resizebox{\columnwidth}{!}{
        \begin{tabular}{lccccc}
        \toprule
            \textbf{Subject} & \textbf{\# $C$} & \textbf{Split} & \textbf{\# $E$ / $C$} & \textbf{\% $E$ w/ answer} & \textbf{\# $S$ / $C$} \\
        \midrule
            Microbiology & 20 & Train & 12.4 & 64\% & 16.4 \\
                         & 5  & Test  & 13.4 & 58\% & 17.0 \\
        \midrule
            Chemistry    & 20 & Train & 14.2 & 51\% & 11.0 \\
                         & 5  & Test  & 16.2 & 49\% & 6.4 \\
        \midrule
            Economics    & 20 & Train & 12.2 & 23\% & 14.1 \\
                         & 5  & Test  & 12.2 & 23\% & 14.4 \\
        \midrule
            Sociology    & 20 & Train & 10.4 & 62\% & 16.6 \\
                         & 5  & Test  & 11.2 & 67\% & 19.0 \\
        \midrule
            US History   & 20 & Train & 7.2 & 51\% & 14.9 \\
                         & 5  & Test  & 8.4 & 38\% & 13.2 \\
        \bottomrule
        \end{tabular}
    }
    \caption{\textbf{Data Statistics} of \ourdata. \# $ C$: number of chapters, \# $E/C$: avg. number of questions per chapter, \% $E$ w/ answer: proportion of questions that have reference answer, \# $S/C$: avg. number of sections per chapter.}
    \label{tab:textbook-exam-statistics}
\end{table}
For each subject, we curate 25 sequential chapters \(C\), each containing both \(D\) and \(E\).
The chapters are arranged in their natural order, with the first 20 used for training and the last five reserved for evaluation.  
There is the risk that content in later chapters may include information from prior chapters (e.g., revisiting prerequisite knowledge). 
Therefore, preserving this sequential structure between the training and test set is essential for preventing information leakage and fairly assessing a model's learning process.
Table~\ref{tab:textbook-exam-statistics} shows an overview of the statistics of the resulting \ourdata.

\subsection{Distribution of Question Types}
\label{ssec:textbook-exam-bloom}

 \begin{figure}[t]
    \centering
    \includegraphics[width=\linewidth]{figures/bloom_taxonomy_counts_vertical.png}
    \caption{\textbf{Bloom's taxonomy distribution} in \ourdata. \ourdata consists of questions that require a wide variety of cognitive levels and the dominant categories vary for each subject.}
    \label{fig:bloom-distribution}
\end{figure}

To better understand how final exams assess a learner’s comprehension on multiple dimensions, we categorize questions in \ourdata\ based on the revised \textit{Bloom’s Taxonomy}~\cite{krathwohl2002revision_bloom}. 
Using an LM, we assign a cognitive depth \( d_j \) to each question \( E_j \in E \), classifying them into six categories: \textit{Remembering, Understanding, Applying, Analyzing, Evaluating}, and \textit{Creating}.
Additionally, we identify the relevant sections \( S_j \subset D \) that correspond to each question.

The distribution, shown in \autoref{fig:bloom-distribution}, indicates that different subjects emphasize different cognitive skills.
For instance, questions in Microbiology and Sociology primarily focus on \textit{Remembering} and \textit{Understanding}, whereas Chemistry and Economics exhibit a more varied distribution.
This analysis highlights the diverse cognitive demands across subjects and underscores how \ourdata\ provides a multifaceted evaluation of learning outcomes through final exams. 
For further details on data processing, refer to Appendix~\ref{appendix:data_processing}.













% \dongho{Number of textbooks?}

% \dongho{For each textbook, how may $D$?}



% \subsection{Validation}
% \dongho{Let's make a ground truth to see how reliable the data processing pipeline it is. -- for each textbook.}

% \paragraph{Validation of answer for $E_j$.}

% \paragraph{Validation of cognitive depth $d_j$ for $E_j$.}

% \paragraph{Validation of related sections $S_j$ for $E_j$.}

\section{Experimental Evaluation: Clustering Stage}
\label{sec:results:clustering}

In this section, we evaluate the clustering stage of the \ToolName{} pipeline (Steps 0-2) to assess the quality of the clusters generated by our approach.  Although the clusters have not yet been assigned \ToolName{} labels, examining the clusters helps provide insights into the degree to which they match previous hand-annotated labels from CASAS. To validate cluster performance, we use CASAS as a ground truth oracle to temporarily assign a label to each cluster through majority voting. We then perform a direct comparison with fully supervised techniques, and investigate the impact of varying cluster count, examining the trade-off between granularity, label alignment and annotation costs.

\subsection{Setup}
\label{sec:supervised_setup}
We assess the quality of the clusters generated by our approach by comparing them to CASAS labels.  To map \ToolName{} clusters to CASAS labels, we use majority vote to assign a single label to each cluster based on the CASAS labels of its samples. We then assess these cluster-assigned labels, utilizing CASAS labels as ground truth. 

We use two common variants of the F1 metric to evaluate the performance. Let \(\{1, \ldots, C\}\) be the set of activities, and let \(\mathrm{F1}_c\) denote the F1 score for class \(c\). Given the class imbalances in CASAS, we utilize the following metrics:

\begin{itemize}
    \item \textbf{Weighted F1}: Emphasizes frequent classes by weighting each class’s F1 score by its support:
    \[
    \mathrm{F1}_{\mathrm{weighted}} 
    = \sum_{c=1}^{C} \frac{\lvert c \rvert}{N} \,\mathrm{F1}_c,
    \]
    where \(\lvert c \rvert\) is the number of samples belonging to class \(c\), and \(N\) is the total number of samples.
    \item \textbf{Macro F1}: Treats all classes equally by averaging their F1 scores:
    \[
    \mathrm{F1}_{\mathrm{macro}} 
    = \frac{1}{C} \sum_{c=1}^{C} \mathrm{F1}_c.
    \]
\end{itemize}

Temporarily utilizing CASAS labels for our clusters has one additional benefit in that it allows us to directly compare our self-supervised clustering approach to fully-supervised methods.  As noted in earlier discussion, by comparing our cluster-based, majority-vote labels with the outputs of fully supervised baselines, we can gauge how closely our framework approaches traditional, label-intensive methods. While surpassing these baselines is not our primary goal, a comparable \(\mathrm{F1}\) performance signifies a reasonable alignment between our clustering results and CASAS labels.  

Specifically, we compare against two baselines: DeepCASAS \cite{deepcasas2018} and TDOST \cite{tdost2024}. DeepCASAS is a fully supervised deep learning algorithm that leverages LSTMs to capture temporal dependencies in the data. In contrast, TDOST converts sensor triggers into natural language descriptions, 
which provides the model with rich contextual information and enables the transfer of a single model across multiple households.

Note that both DeepCASAS and TDOST, as originally published, were trained and evaluated on pre-segmented datasets. This allowed the authors to compare the isolated algorithmic performance in a controlled setting, but is a less realistic scenario compared to operating over continuous sensor streams. To achieve a fair comparison to our approach, which does not assume pre-segmentation, we re-train both DeepCASAS and TDOST algorithms on sliding windows $W_i$ of length \( l \) to match our sequences. Additionally, given the prevalence of the \textit{Other} label in our datasets (see \cref{sec:datasets}), we conduct experiments both with and without the \textit{Other} label. Specifically, we re-train DeepCASAS and TDOST with and without the \textit{Other} label, which allows us to evaluate how well these methods work with only \textit{relevant} activities.


\subsection{Results}
\label{sec:supervised_results}

\begin{table}[t]
    \centering
    % First part: F1 Weighted
    \textbf{F1 Weighted}
    \vspace{0.5em}
    \begin{adjustbox}{width=0.8\textwidth,center}
        \small
        \begin{tabular}{l | cc | cc | cc}
            \toprule
            & \multicolumn{2}{c|}{Milan} & \multicolumn{2}{c|}{Aruba} & \multicolumn{2}{c}{Cairo} \\
            Model & All Labels & Excl. \textit{Other} & All Labels & Excl. \textit{Other} & All Labels & Excl. \textit{Other} \\
            \midrule
            DeepCASAS & 0.61 & 0.79 & 0.78 & \cellcolor[HTML]{caebc0}0.93 & 0.72 & 0.82 \\
            TDOST     & \cellcolor[HTML]{caebc0}0.62 & \cellcolor[HTML]{caebc0}0.83 & \cellcolor[HTML]{caebc0}0.80 & 0.92 & \cellcolor[HTML]{caebc0}0.74 & 0.82 \\
            \hline
            \hline
            \ToolName & 0.53 & 0.82 & 0.75 & 0.90 & 0.64 & \cellcolor[HTML]{caebc0}0.85 \\
            \bottomrule
        \end{tabular}
    \end{adjustbox}
    
    \vspace{2em} % Add vertical space between the two parts
    \textbf{F1 Macro}
    
    % Second part: F1 Macro
    \vspace{0.5em}
    \begin{adjustbox}{width=0.8\textwidth,center}
        \small
        \begin{tabular}{l | cc | cc | cc}
            \toprule
            & \multicolumn{2}{c|}{Milan} & \multicolumn{2}{c|}{Aruba} & \multicolumn{2}{c}{Cairo} \\
            Model & All Labels & Excl. \textit{Other} & All Labels & Excl. \textit{Other} & All Labels & Excl. \textit{Other} \\
            \midrule
            DeepCASAS & 0.38 & 0.53 & 0.48 & 0.69 & 0.16 & \cellcolor[HTML]{caebc0}0.46 \\
            TDOST     & \cellcolor[HTML]{caebc0}0.40 & \cellcolor[HTML]{caebc0}0.63 & \cellcolor[HTML]{caebc0}0.54 & \cellcolor[HTML]{caebc0}0.69 & \cellcolor[HTML]{caebc0}0.17 & 0.41 \\
            \hline
            \hline            
            \ToolName & 0.32 & 0.50 & 0.24 & 0.49 & 0.12 & 0.39 \\
            \bottomrule
        \end{tabular}
    \end{adjustbox}
    \vspace{0.5em}
    \caption{
        Weighted and macro F1 scores for DeepCASAS, TDOST, and \ToolName{} trained and tested on the Milan, Aruba, and Cairo datasets using CASAS labels as ground truth. The best result for each configuration is highlighted in green. While fully supervised methods yield the highest overall performance, the gap in \ToolName{} performance is moderate—with comparable F1 scores achieved when data associated with the \textit{Other} label is excluded, signifying a reasonable alignment between \ToolName{} clusters and CASAS labels.
    }
    \vspace*{-1em}
    \label{tab:clf_comparison}
\end{table}


In this section we present how well our clusters align with the CASAS labels, and also experiment with the optimal number of clusters needed to achieve useful performance.

\subsubsection{\ToolName{} Cluster Alignment with CASAS Labels}   

\cref{tab:clf_comparison} reports both weighted and macro F1 scores for models trained with and without the \textit{Other} label. The bottom row shows the performance of the \ToolName{} clustering stage compared. Across all three datasets, we observe that the \(\mathrm{F1}_{\mathrm{macro}}\) metric is considerably lower than \(\mathrm{F1}_{\mathrm{weighted}}\). This indicates that less frequent activities, such as \textit{Take\_medicine} and \textit{Leave\_home} are challenging to discover, as has been reported in previous studies \cite{hiremath2022bootstrapping}.  
Additionally, we observe that performance improves by approx.\ 22 points when the \textit{Other} label is excluded, indicating that the availability of \textit{Other} data points during clustering leads to clusters that are less closely aligned with CASAS labeling.  
As we will demonstrate in \cref{sec:tsne}, this effect is due to the non-homogeneous nature of the data captured by the catch-all \textit{Other} label. 


\subsubsection{Comparison to Supervised Baselines}

The top rows of \cref{tab:clf_comparison} report the performance of the DeepCASAS and TDOST baselines\footnote{As noted in \cref{sec:supervised_setup}, both baselines were re-trained and tested on sliding window sequences to remove the pre-segmentation assumption. This change leads to significant drops in performance compared to originally published results. For example, for DeepCASAS, the \(\mathrm{F1}_{\mathrm{weighted}}\) scores dropped by 29 points for Milan from 0.89 to 0.61; by 21 points for Aruba from 0.97 to 0.78; by 13 points for Cairo from 0.85 to 0.72. This highlights the significance of the pre-segmentation assumption.}.  
As with \ToolName{}, we observe that that \(\mathrm{F1}_{\mathrm{macro}}\) is substantially lower than \(\mathrm{F1}_{\mathrm{weighted}}\), and that performance improves with the exclusion of data associated with the \textit{Other} label.  

More generally, we observe that while DeepCASAS and TDOST yield better performance overall, the gap relative to \ToolName{} is moderate, with comparable performance on $\mathrm{F1}_{\mathrm{weighted}}$ when \textit{Other} is excluded.  
These results highlight the misalignment of self-supervised clusters with the \textit{Other} class, suggesting that \textbf{data labeled as \textit{Other} represents many different behaviors rather than a single cohesive activity}.  We explore this finding in greater detail in the following subsection.


\subsubsection{tSNE Analysis}
\label{sec:tsne}

\begin{figure}[t]
    \centering
        \includegraphics[width=0.95\linewidth]{figures/cluster_tsne_with_house_maps.pdf}
    \caption{
        tSNE projection of SCAN embeddings from the Milan household; each point represents a sensor window embedding colored by its original CASAS label. Insets display the deployment environment layout overlaid with a heatmap of sensor activations. Insets (a) and (b) highlight two distinct clusters within the CASAS \textit{Cook} label—cluster 16 showing movement between kitchen and dining areas, and cluster 5 capturing activity near the medicine cabinet. Insets (c) and (d) show clusters from the \textit{Relax} label, corresponding to sitting in the TV room armchair and sitting in the living room armchair. Data associated with the \textit{Other} label (gray) is dispersed across clusters, underscoring its heterogeneous nature. This figure showcases \ToolName{}'s capability to uncover more granular and nuanced activity categories than the original CASAS labels. 
    } 
    \vspace*{-1em}
    \label{fig:cluster_tsne_with_house_maps}
\end{figure}

In this section, we perform a visual inspection of Milan clusters in order to gain more insights into the generated clusters. 
\cref{fig:cluster_tsne_with_house_maps} shows a tSNE projection of the SCAN embeddings from the Milan dataset; each point represents a 768-dimensional sequence embedding colored by its CASAS label. While some activities (e.g., \emph{Work}) form single clusters, data representing other activity labels (e.g., \emph{Cook}, \emph{Relax}) fall into multiple clusters. 
Each point in the scatterplot is annotated with a probability score reflecting SCAN’s confidence in assigning that sequence to its respective cluster, offering interpretability and a level of control that simpler clustering methods (e.g., k-means) lack.

Further, we take the two largest Milan activity labels, \emph{Cook} and \emph{Relax}, and inspect the data within individual clusters, as shown in the figure insets. 
Each inset includes the layout of the deployment environment overlaid with a heatmap representing sensor activation patterns for the data sequences associated with that cluster.  
\cref{fig:cluster_tsne_with_house_maps} illustrates that clustering using \ToolName{} facilitates the discovery and differentiation of granular sub-activities.  
Insets (a) and (b) are associated with the \textit{Cook} CASAS label (orange).  
Although both sets of data are labeled as \textit{Cook} within CASAS, we observe clear differences between the sensor patterns within each cluster.  
Cluster 16 (inset (a)) captures the person leaving the kitchen and moving to the dining room, with motion sensors from the dining room and part of the dining room registering the person’s presence.  
By comparison, cluster 5 (inset (b)) is tightly focused on activity around the medicine cabinet in the kitchen. 
Insets (c) and (d) are associated with the \textit{Relax} CASAS label (burgundy), where cluster 3 (inset (c)) captures a static activity in the TV room armchair while cluster 17 (inset (d)) corresponds to sitting in a living room armchair instead.  

Note that data associated with the \textit{Other} label (gray) is scattered throughout multiple clusters in \cref{fig:cluster_tsne_with_house_maps}, indicating that it does not represent a single sub-activity. Instead, \textit{Other} activities appear to correspond to multiple sub-activities.  

These findings can be interpreted as follows. First, excluding the \textit{Other} label significantly improves performance, particularly when there is no pre-segmentation and with shorter sequence lengths, as the \textit{Other} label tends to overlap with all other categories. Second, the large gap between \(\mathrm{F1}_{\mathrm{weighted}}\) and \(\mathrm{F1}_{\mathrm{macro}}\) reflects the challenges of accurately recognizing low-frequency classes, a point often overlooked when results are reported for the fully supervised methods.
Finally, our method's performance indicates that the clusters align reasonably well with the CASAS labels, showcasing the capabilities of \ToolName{}.
% demonstrating a decent alignment between our clusters and CASAS labels.

\subsubsection{Impact of the Number of Clusters}
\label{sec:varying_clusters_results}

\begin{figure}
    \centering
        \includegraphics[width=0.9\linewidth]{figures/varying_clusters_f1.pdf}
    \caption{
        Macro F1 scores for varying numbers of SCAN clusters (\(k \)) on Milan, Aruba, and Cairo datasets, using all CASAS labels in (a), and without the \textit{Other} label in (b). We chart shows average F1 scores with bootstrapped 95\% confidence intervals . Increasing \(k\) up until 20-40 clusters improves alignment with CASAS labels before the performance improvement stagnates, suggesting that the optimal number of clusters lies in that range.
    } 
    \vspace*{-1em}
    \label{fig:varying_clusters_f1}
\end{figure}

\noindent
We also investigate how the choice of \(k\)—the number of clusters—impacts our model's ability to capture and distinguish fine-grained activities. 
Similar to previous section, we use CASAS labels and apply majority voting per cluster to compute Macro F1 scores. 

\cref{fig:varying_clusters_f1} shows the Macro F1 scores with bootstrapped 95\% confidence intervals (CIs) for Milan, Aruba, and Cairo, when  \( k \in \{10, 15, 20, 30, 40, 50, 60, 100\} \) clusters. 
The left-hand panels in (a) show performance when all CASAS labels are considered; the right-hand panels in (b) exclude \emph{Other}. 
These F1 scores indicate that utilizing more clusters generally leads to better classification performance in (a) for all three datasets, although the improvements appear to plateau after \( k = 30\) for Milan, and \( k = 20 \) for Aruba and Cairo. 
When we exclude the \textit{Other} label, the macro F1 scores for both Milan and Aruba plateau when \( k > 30\) and further increases are not statistically significant; 
Cairo stays relatively flat after \( k = 20 \), and the improvements for \( k >= 50 \) have very wide CIs. 
This leads to a conclusion that increasing the number of clusters beyond 20–30 is not particularly useful.
Refer to \cref{fig:varying_clusters_tsne} in the Appendix to see what these various clusters look like on 2D tSNE projections.

, as the F1 scores plateau. 
In addition, we also visualizing the effect of the number of clusters through a tSNE plot in \cref{fig:varying_clusters_tsne} in the Appendix. 

An optimal \( k \) balances classification performance with the workload of labeling cluster centroids. 
We selected \( k=20 \) because it provides a more granular view than the typical CASAS labels while keeping annotation efforts manageable. 

\section{Experimental Evaluation: Labeling Stage}
\label{sec:label_quality}

For our second set of experiments, we move beyond the CASAS annotations and evaluate the labeling stage of our pipeline (step 2, 3, and 4 in \ref{fig:main_image}). 
This resembles the envisioned use case for our approach where we employ our active learning-like scheme, driven by our self-supervised clustering stage, to guide human annotator to provide reliable labels thereby minimizing manual efforts.

By comparing our human-labeled clusters to CASAS labels, we highlight how our approach provides a more granular and detailed categorization of (sub-)activities, offering finer resolution and capturing nuances that the original labels may miss. 
We also evaluate the quality and consistency of these labels by examining two key metrics: inter-rater agreement, which measures consistency between different labelers, and cluster agreement, which assesses the precision of the labels assigned to each cluster.

\subsection{Setup}
Our objective is to present human labelers with visualizations of a small number of representative data sequences from each cluster, and for labelers to select the appropriate label for each sample.  Below, we detail the labels selected for our work and evaluation metrics.

\subsubsection{Label Set}
The \ToolName{} labeling scheme is organized hierarchically to accommodate different levels of detail. At the highest level, we organize all activities into categories such as \emph{single-room} and \emph{multi-room} events, with subcategories specific to each environment (e.g., \emph{kitchen} vs.\ \emph{bedroom}). By allowing annotators to choose increasingly granular labels, we preserve the flexibility needed for diverse research objectives.

To capture a finer level of granularity than conventional HAR labels such as  \textit{Cook}, \textit{Relax}, or \textit{Work}, we developed a set of \emph{sub-activity} labels by studying the physical layouts of the CASAS homes, examining the replays of actual sensor activations throughout the day, and drawing insights from prior work on structural constructs in activities \cite{shruthi_gameofllms}. We selected a large number of potential labels in order to avoid over-constraining our annotators, leveraging information about furniture placement when it was available. Our labels capture both activities within a single room (e.g., \textit{movement all over kitchen}, \textit{movement near fridge}, \textit{movement near stove}, and \textit{movement near medicine cabinet}) and movement actions between the rooms (e.g., \textit{walking from bedroom to office}, \textit{leaving guest bathroom}). See the full list of sub-activities and their hierarchies in \cref{fig:atomic_activity_tree} in the Appendix.

Note, we have chosen a dataset-specific approach in our label selection, with several labels tailored specifically to the CASAS dataset (e.g., \textit{medicine cabinet}), in order to demonstrate the degree to which \ToolName{} supports customization to a given environment or use case. The granularity of the labels is fully adjustable; depending on the specific application, a user can, for example, choose whether they care to split \textit{armchair sitting} and \textit{couch sitting} into distinct classes, or if they want to group them into a broader \textit{Relax} category.

\subsubsection{Labeling Sample Selection and Metrics}

To obtain cluster labels, we recruited 15 independent raters to each individually label data segments.  For each cluster \( c_k\), we choose \( m\) sequence samples that are closest to the cluster centroid for labeling, with \( m = 5\) in this work.  With $k=20$ clusters per model, this resulted in a total of $300$ samples across all three datasets. Samples were randomized and each sample was labeled by two raters, resulting in $600$ total sample-label pairs. To generate a label, each rater was presented with a selection of $40$ samples, for which they assigned labels using the web-based \ToolName{} annotation tool.  \cref{fig:viz_tool} shows a screenshot of a sample annotation replay for the Milan household. During the replay, the rater can view a temporal playback of the sensor activations captured by the data sequence, and use a drop-down menu to select a label. Note that the \ToolName{} tool is fully customizable and is available in open source at \url{https://anonymized}.

We use two metrics to evaluate the labeling stage.  First, we evaluate \textbf{inter-rater agreement} using Cohen’s Kappa \cite{kappas}.  Higher values represent stronger agreement, with scores over 0.8 representing strong alignment. Observing a high inter-rater agreement on the labeling task would indicate that the data sequences represent consistently interpretable human behavior sequences. Second, we evaluate \textbf{cluster agreement} to examine the uniformity of the \( m\) labels obtained for each cluster.  We use Fleiss’s Kappa \cite{kappas}, which generalizes the idea of Cohen’s Kappa to more than two ratings, to capture the consensus among all 10 labels ($m=5$ by a total of $2$ raters per sample) assigned within a single cluster. A high Fleiss’s Kappa indicates that our raters not only agree between themselves, but also that data sequences captured by the cluster represent a single activity rather than disparate data sequences (i.e., the cluster is internally homogeneous).


Since our sub-activity labels follow a hierarchical, multi-level taxonomy (see \cref{fig:atomic_activity_tree}), disagreements may sometimes stem from different levels of specificity (e.g., \emph{movement near fridge} vs.\ \emph{movement in kitchen}). To account for this, we also compute Kappa scores at a coarser level by ``leveling up'' each annotated label to its parent category. For instance, if one rater selects \emph{movement near fridge} and the other chooses \emph{movement in kitchen}, these would be treated as the same label at the higher level. Comparing Kappa scores at different levels of the hierarchy helps distinguish minor discrepancies in labeling granularity from dramatically different activity interpretations (e.g., \emph{bathroom activity} vs.\ \emph{reading in armchair}). 

\subsection{Results}
\label{sec:label_quality_results}

\begin{table}[ht]
    \centering
    \begin{tabular}{l  cc  cc}
        % \toprule
         & \multicolumn{2}{c}{Inter-Rater \( \kappa \) } & \multicolumn{2}{c}{Cluster Agreement \( \kappa \) } \\
        % \cmidrule(lr){2-3} \cmidrule(lr){4-5}
        Dataset & Original Label & Level-Up Label & Original Label & Level-Up Label \\
        \toprule
        Milan  & 0.850 & 0.884 & 0.600 & 0.775 \\
        Aruba  & 0.890 & 0.918 & 0.627 & 0.757 \\
        Cairo  & 0.872 & 0.916 & 0.630 & 0.810 \\
        \bottomrule
    \end{tabular}
    \vspace{0.5em}
    \caption{
        Sub-Activity label quality: inter-rater (Cohen’s) and cluster-agreement (Fleiss’s) Kappa scores for each of the three CASAS datasets (Milan, Aruba, and Cairo). ``Original Label'' indicates the use of the most specific labels chosen by each rater (e.g., \emph{movement near fridge}), while ``Level-Up Label'' merges them into coarser categories (e.g., \emph{movement in kitchen}). 
        Overall, very high inter-rater agreement indicates very high alignment between annotators; decent cluster agreement also implies relative homogeneity among the ratings.
        The resulting increase in Kappa scores from ``Original Label'' to ``Level-Up Label'' highlights that most disagreements stem from label specificity rather than fundamentally different interpretations of the underlying activity.        
    }
    \vspace*{-2em}
    \label{tab:kappa_scores}
\end{table}


In this section, we first analyze the inter-rater and cluster-agreement scores, then discuss how the obtained labels different from CASAS labels.
\cref{tab:kappa_scores} summarizes the inter-rater and cluster-agreement scores for Milan, Aruba, and Cairo. We also refer to \cref{fig:cluster_majority_label_summary} in the Appendix for a detailed table showing the majority label assigned to each cluster and its corresponding vote count.

Inter-rater agreement, measured as Cohen's Kappa, exceeds 0.85 for all three datasets (Milan \(\kappa=0.85\), Aruba \(\kappa=0.89\), Cairo \(\kappa=0.87\)). These values indicate very high consistency between the two independent annotators in how they interpreted and labeled the same replay samples. Furthermore, when we ``level up'' each label to its parent category in our hierarchical taxonomy (e.g., merging \emph{movement near bed} into \emph{movement in bedroom}), the Kappa scores rise even further—reaching 0.88 in Milan and over 0.90 in Aruba and Cairo. This gain reflects the fact that any disagreements that did arise between raters, were largely in relation to differences in label specificity, such as \emph{sitting in armchair} vs.\ \emph{movement in living room}, rather than fundamentally diverging perceptions of the activity itself.

Cluster agreement, measured as Fleiss' Kappa, results in moderately high scores ranging from 0.60--0.63 when annotators used highly specific sub-activity labels. Once labels are rolled up to a coarser levels, these scores climb to 0.75--0.81 across the three datasets. Here again, the improvement underscores that most labeling discrepancies within a cluster stem from variation in granularity rather than actual activity disagreements. For instance, in one cluster designated \emph{kitchen activity}, half of the annotators specified \emph{movement near medicine cabinet}, while the other half used \emph{movement in kitchen}. In another cluster, some annotators perceived simultaneous sensor firings in two rooms as \emph{multi-room activity}, while others focused on whichever room had the most events.

The high rater and cluster agreements confirm that clusters produced by the \ToolName{} pipeline are both interpretable and relatively homogeneous. Moreover, the gains observed when we unify labels at a higher level indicate that our hierarchical taxonomy successfully accounts for the inherent variation in labeling granularity. This can allow researchers to adopt the level of detail most appropriate for their analysis. 

\begin{table}[ht]
  \centering
  \begin{adjustbox}{width=\textwidth,center}
    \small
    \begin{tabular}{ll|cc|cc|cc}
      \toprule
      \multicolumn{2}{c|}{} & \multicolumn{2}{c|}{Milan} & \multicolumn{2}{c|}{Aruba} & \multicolumn{2}{c}{Cairo} \\
      CASAS Label & \ToolName{} Label & CASAS & \ToolName & CASAS & \ToolName & CASAS & \ToolName \\
      \midrule
      %---------------- Cook ----------------
      \multirow{5}{*}{Cook} 
          & TOTAL                              & 31\%  & 19\%  & 39\%  & 24\%  & 0\%   & 0\%   \\ 
          \cdashline{2-8}[0.5pt/2pt]
          & Movement all over kitchen          &       & 4\%   &       & 24\%  &       &       \\[0.8em]
          & Movement near Fridge               &       &       & --    & --    & --    &  --   \\[0.8em]
          & Movement near Stove                &       & 2\%   & --    & --    & --    &  --   \\[0.8em]
          & Movement near Medicine Cabinet     &       & 13\%  & --    & --    & --    &  --   \\
      \midrule
      %---------------- Eat ----------------
      \multirow{4}{*}{Eat} 
          & TOTAL                              & 2\%   & 6\%   & 4\%   & 6\%   & 72\%  & 67\%  \\ \cdashline{2-8}[0.5pt/2pt]
          & Movement all over Dining Room      &       &       &       &       &       &       \\[0.8em]
          & Movement through Kitchen + Dining  &       & 6\%   &       & 6\%   &       & 32\%  \\[0.8em]
          & Movement through Living + Dining   &       &       &       &       &       & 35\%  \\
      \midrule
      %---------------- Relax ----------------
      \multirow{6}{*}{Relax} 
          & TOTAL                              & 41\%  & 36\%  & 49\%  & 37\%  & 0\%   & 7\%   \\ \cdashline{2-8}[0.5pt/2pt]
          & Motion all over Living room        &       &       &       &  0\%  &       &       \\[0.8em]
          & Motion in TV chair and area        &  --   & --    &       &  6\%  & --    & --    \\[0.8em]
          & Sitting on couch/armchair          &       & 19\%  &       & 29\%  &       & 7\%   \\[0.8em]
          & Sitting in office armchair         &       & 15\%  &       &       & --    & --    \\[0.8em]
          & Movement through Kitchen + Living  &       & 2\%   &       & 2\%   &       &       \\
      \midrule
      %---------------- Work ----------------
      \multirow{2}{*}{Work} 
          & TOTAL                              & 3\%   & 4\%   & 4\%   & 7\%   & 12\%  & 0\%   \\ \cdashline{2-8}[0.5pt/2pt]
          & Movement near office computer/desk &       & 4\%   &       & 7\%   & --    & --    \\
      \midrule
      %---------------- Sleep ----------------
      \multirow{3}{*}{Sleep} 
          & TOTAL                              & 7\%   & 8\%   & 3\%   & 3\%   & 12\%  & 7\%   \\ \cdashline{2-8}[0.5pt/2pt]
          & Motion in all over Master Bedroom  &       & 8\%   &       & 0\%   &       &       \\[0.8em]
          & In Master Bedroom, Movement near bed &       &       &      & 3\%   &       & 7\%   \\
      \midrule
      %---------------- Bathing ----------------
      \multirow{7}{*}{Bathing} 
          & TOTAL                              & 14\%  & 19\%  & 0\%   & 0\%   & 0\%   & 0\%   \\ \cdashline{2-8}[0.5pt/2pt]
          & Entering the Guest Bathroom        &       & 13\%  & --    & --    & --    & --    \\[0.8em]
          & Leaving the Guest Bathroom         &       &       & --    & --    & --    & --    \\[0.8em]
          & Movement inside the Guest Bathroom   &       &       & --    & --    & --    & --    \\[0.8em]
          & Entering Master Bathroom           &       & 1\%   & --    & --    & --    & --    \\[0.8em]
          & Leaving Master Bathroom            &       &       & --    & --    & --    & --    \\[0.8em]
          & Movement in Master bathroom        &       & 5\%   & --    & --    & --    & --    \\
      \midrule
      %---------------- Leave_Home ----------------
      \multirow{1}{*}{Leave\_Home} 
          & TOTAL                              & 2\%   & 6\%   & 0.3\% & 6\%   & 1\%   & 5\%   \\[0.8em]
      \midrule
      %---------------- Other ----------------
      \multirow{1}{*}{Motion Through House} 
          & TOTAL                              & 0\%   & 4\%   & 0\%   & 14\%  & 3\%   & 13\%  \\
      \bottomrule
    \end{tabular}
  \end{adjustbox}
  \caption{
    A side-by-side comparison of original CASAS labels and the more granular sub-activities identified by \ToolName{} in Milan, Aruba, and Cairo.
    CASAS Labels are mapped to \ToolName{} labels. Dashes indicate that this activity cannot be identified in that household (e.g., Aruba and Cairo have no bathroom sensors). These results showcase how \ToolName{} is able to discover much more granular sub-activities for all three CASAS datasets.
  }  
  \vspace*{-1em}  
  \label{tab:subactivity_mapping}
\end{table}


We now want to explore these manually annotated sub-activity labels in greater detail. \cref{tab:subactivity_mapping} illustrates how each discovered sub-activity label maps onto the broader CASAS categories (e.g., \emph{Cook}, \emph{Relax}, \emph{Sleep}), along with the percentage of time-window sequences \(W_i\) that each label occupies in Milan, Aruba, and Cairo. Dashes indicate activities that cannot occur in a given household (e.g., no guest bathroom sensors in Aruba or Cairo).

Notably, the total proportion of each broad CASAS label usually aligns reasonably well with our sub-activity categories. In Milan, for example, \emph{Sleep} comprises around 7\% of the CASAS data and 8\% of our sub-activity labels (\emph{movement in bedroom}, \emph{movement near bed}). Similarly, CASAS marks 41\% of the Milan data as \emph{Relax}, compared to 36\% for our corresponding sub-activities. More importantly, our approach results in a much finer breakdown of these broad labels, such as distinguishing \emph{sitting in living room armchair} vs.\ \emph{sitting in office armchair}, or \emph{movement near the medicine cabinet} vs.\ \emph{movement near the stove}. In Aruba, a single \emph{Relax} label can be further divided into \textit{sitting on the couch} vs.\ \textit{motion in a TV chair}. These finer distinctions highlight one of the key advantages of \ToolName{}: while it can reasonably well replicate higher-level annotations from existing datasets, it can also uncover more granular activities that may carry significant behavioral or clinical relevance.

This paper presents a planning approach for effective and efficient joint motion generation for manipulators to cover a surface, aiming to minimize specific joint space costs.

\textit{Limitations} -- Our work has several limitations that suggest potential directions for future research. First, our method uses a heuristic to accelerate the traditional Joint-GTSP approach. While we provide empirical evidence of its efficiency in producing high-quality solutions, we cannot guarantee consistent performance in all scenarios.
Second, our bi-level hierarchical method reduces the size of GTSP. Future research could extend it to multiple levels to further improve performance, though this may produce misleading guide paths.
Third, we observe that both Joint-GTSP and H-Joint-GTSP tend to generate paths with frequent turns, a pattern also observed in the motions of prior work \cite{kaljaca2020coverage, zhang2024jpmdp}.  Future work should explore strategies to balance joint movements with other objectives such as motion smoothness.

\footnotetext{Visualization tool: \url{https://github.com/uwgraphics/MotionComparator}}
\textit{Implications} -- The hierarchical approach presented in this work enables effective and efficient coverage path planning for robot manipulators. 
This approach is beneficial to applications that require dexterous surface coverage, such as sanding, polishing, wiping, and sensor scanning. 



\section{Summary and Conclusion}
\label{sec:conclusion}


In this paper, we introduced \ToolName{}, a method for discovering fine-grained \emph{sub-activities} from unlabeled smart home sensor data without relying on pre-segmentation. Our pipeline is organized into two core steps: Clustering and Labeling. 
The \textbf{Clustering step} consists of:

\begin{itemize}
    \item \textbf{Encoder Pre-Training:} We leverage a pre-trained BERT model adapted with sensor-specific tokens and train it using a masked language modeling (MLM) objective to generate context-rich embeddings for raw sensor sequences.
    
    \item \textbf{Clustering Model Fine-Tuning:} Using the SCAN loss function, we fine-tune these embeddings to form more homogeneous and distinct clusters of sensor sequences.
\end{itemize}

The \textbf{Labeling step} comprises:

\begin{itemize}
    \item \textbf{Cluster Centroid Annotation:} Representative sequences from each cluster are visualized with a custom tool, enabling expert annotators to assign meaningful sub-activity labels to the centroids.
    
    \item \textbf{Label Propagation:} The centroid labels are propagated to all sequences within their respective clusters, resulting in a fully labeled dataset with minimal manual effort.
    
    \item \textbf{Re-annotation of Original Time-Series Data:} 
    Finally, these propagated labels are mapped back onto the original time-series data, preserving temporal continuity and facilitating the analysis of longitudinal activity patterns.
\end{itemize}


Our approach addresses important challenges in HAR, including the high cost and effort of manual data annotation, the limitations of coarse activity labels, and the need for scalable and generalizable models. \ToolName{} offers an open source tool that facilitates the HAR annotation and re-annotation process and enables the dynamic discovery and validation of sub-activities, thus capturing a broader spectrum of behaviors observed in real homes.

\bibliographystyle{ACM-Reference-Format}
\bibliography{ref}

\newpage
\appendix
\begin{table}[t!]
  \centering
  


% \renewcommand{\arraystretch}{1.2} % 调整行高
% \setlength{\tabcolsep}{10pt}  % 调整列间距
\resizebox{0.48\textwidth}{!}{%
\begin{tabular}{lcrr}
        \toprule
        \textbf{Dataset} & \textbf{Full Size*} & \textbf{Consistency}  & \textbf{\dataset{}} \\
        \midrule
        HotpotQA  & 5,901 & 2,973 {\footnotesize \textcolor{gray}{(50\%)}}  & 1,476 {\footnotesize \textcolor{gray}{(25\%)}}  \\
        NewsQA    & 4,212 & 1,260 {\footnotesize \textcolor{gray}{(30\%)}} & 934  {\footnotesize \textcolor{gray}{(22\%)}}  \\
        NQ        & 7,314 & 4,419 {\footnotesize \textcolor{gray}{(60\%)}}  & 1,479 {\footnotesize \textcolor{gray}{(20\%)}}  \\
        SearchQA  & 16,980 & 12,133 {\footnotesize \textcolor{gray}{(71\%)}} & 1,497 {\footnotesize \textcolor{gray}{(9\%)}}  \\
        SQuAD     & 10,490 & 5,024 {\footnotesize \textcolor{gray}{(48\%)}}  & 2,351 {\footnotesize \textcolor{gray}{(22\%)}}  \\
        TriviaQA  & 7,785 & 6654 {\footnotesize \textcolor{gray}{(85\%)}}  & 792  {\footnotesize \textcolor{gray}{(10\%)}}  \\
        \bottomrule
    \end{tabular}
}




 \caption{Number of instances at each stage in the \dataset{} construction pipeline.}
 \label{tab:our_bench_stats_each_step}
\end{table}
\section{Appendix}
\subsection{License}
We present the licenses of the datasets used in this study: Natural Questions (CC BY-SA 3.0 license), NewsQA (MIT License), SearchQA and TriviaQA (Apache License 2.0), HotpotQA and SQuAD (CC BY-SA 4.0 license).

All these licenses and agreements permit the use of their data for academic purposes.

\subsection{Details of Data Constructing}
\label{append:prompts}
In this section, we detail the two main steps in constructing \dataset{}. The dataset sizes at each stage of the pipeline are shown in Table~\ref{tab:our_bench_stats_each_step}.


\textbf{Parametric Knowledge Elicitation.} First, we elicit the LLM's parametric knowledge by prompting it in a closed-book setting (i.e., without any context). To ensure the reliability of the elicited knowledge, we apply a consistency-based filtering method. Specifically, for each query, the LLM is prompted five times, and the frequency of each response is recorded. The response with the highest frequency is identified as the majority answer. Queries where the majority answer appears fewer than three times are discarded, in order to filter out inconsistent responses and enhance data quality. The following prompt is used to instruct the LLM:
\begin{tcolorbox}
[title=Prompt for eliciting parametric knowledge,colback=blue!10,colframe=blue!50!black,arc=1mm,boxrule=1pt,left=1mm,right=1mm,top=1mm,bottom=1mm]
Answer the question \textcolor{blue}{\{\textit{brevity\_instruction}\}} and provide supporting evidence.

Question: \textcolor{blue}{\{\textit{question}\}}
\end{tcolorbox}
\noindent The ``\textit{brevity\_instruction}'' is used to guide the LLM to generate responses in a more concise form.

\textbf{Conflict Data Selection.} Next, we filter the data to retain only instances where the LLM's parametric knowledge directly conflicts with the contextual answer. Specifically, we categorize the data obtained from the previous step into two groups, conflicting and non-conflicting instances, based on the detailed results of conflict detection. All non-conflicting instances are discarded. GPT-4o-mini is then used to detect the presence of a conflict, using the following prompt:

\begin{tcolorbox}
[title=Prompt for identifying conflict knowledge,colback=blue!10,colframe=blue!50!black,arc=1mm,boxrule=1pt,left=1mm,right=1mm,top=1mm,bottom=1mm]
\small
You are tasked with evaluating the correctness of a model-generated answer based on the given information. 

\small
Context: \textcolor{blue}{\{\textit{context}\}}

Question: \textcolor{blue}{\{\textit{question}\}}

Contextual Answer: \textcolor{blue}{\{\textit{contextual\_answer}\}}

Model-Generated Answer: \textcolor{blue}{\{\textit{Model-Generated\_answer}\}}

\textcolor{blue}{[\textit{Detailed task description...}]}

Output Format:

Evaluate result: (Correct / Partially Correct / Incorrect) 
\end{tcolorbox}




\subsection{Assessing the Reliability of GPT-4o-mini in Knowledge Conflict Identification}
\label{append:human_eval}
In this subsection, we conduct the human evaluation to assess the reliability of GPT-4o-mini in identifying knowledge conflicts, which is a critical task in our data construction process to guarantee the data quality.

We randomly sampled 100 examples from each of the six subsets of \dataset{}, yielding a total of 600 samples. Six senior computational linguistics researchers were then asked to evaluate whether a knowledge conflict was present in each example. For each instance, the evaluators were provided with the question, the contextual answer, the model-generated response, and the corresponding supporting evidence. The results were classified into three categories: No Conflict, Somewhat Conflict, and High Conflict. The detailed annotation instructions are as follows:

\begin{tcolorbox}
[title=Annotation Instruction,colback=blue!10,colframe=blue!50!black,arc=1mm,boxrule=1pt,left=1mm,right=1mm,top=1mm,bottom=1mm]
\small
You are tasked with determining whether the parametric knowledge of LLMs conflicts with the given context to facilitate the study of knowledge conflicts in large language models.

Each data instance contains the following fields: 

Question: \textcolor{blue}{\{\textit{question}\}}


Answers: \textcolor{blue}{\{\textit{answers}\}}


Context: \textcolor{blue}{\{\textit{context}\}}

Parametric\_knowledge: \textcolor{blue}{\{\textit{LLMs' parametric\_knowledge }\}} 

The annotation process consists of two steps. 

\textbf{Step 1}: Compare the model-generated answer with the ground truth answers, based on the given question and context, to determine whether the model’s parametric knowledge conflicts with the context.

\textbf{Step 2}: Classify the results into one of three categories: 

\textcolor{blue}{\{\textit{No Conflict}\}} if the model-generated answer is consistent with the ground truth answers and context, 

\textcolor{blue}{\{\textit{Somewhat Conflict}\}}  if it is partially inconsistent

\textcolor{blue}{\{\textit{High Conflict}\}} if it significantly contradicts the ground truth answers or context.
\end{tcolorbox}


The evaluation results, shown in Table~\ref{tab:append_human_eval}, reveal a high level of agreement between the human annotators and GPT-4o-mini. Over 85\% of the examples reach consensus among the annotators, with an average agreement rate of 85.6\% across all subsets. These findings underscore the reliability of GPT-4o-mini as an effective tool for identifying knowledge conflicts.




\begin{table}[t]
  \centering
  
\centering
\begin{tabular}{l c}
\toprule
\textbf{Subset} & \textbf{Agreement (\%)} \\ \midrule
HotpotQA        & 81.4                        \\
NewsQA          & 72.7                        \\
NQ              & 88.7                        \\
SearchQA        & 95.3                        \\
SQuAD           & 86.1                        \\
TriviaQA        & 90.7                        \\ \midrule
\textbf{Average} & \textbf{85.6}            \\ \bottomrule
\end{tabular}

 \caption{Agreement between human annotators and GPT-4o-mini across different subsets of our \dataset{} benchmark.}
 \label{tab:append_human_eval}
\end{table}



\subsection{Evaluating the Effectiveness of Our Consistency-Based Filtering Method}
\label{append:data_freq}

In this subsection, we evaluate the effectiveness of our consistency-based knowledge conflict filtering method. As described in Appendix~\ref{append:prompts}, for each query, we prompt the model five times and record the most frequently generated answer along with its occurrence frequency. Based on this frequency, we divide the data into sub-datasets, where all queries within each sub-dataset share the same answer frequency. We then apply ``Conflict Data Selection'' to each sub-dataset, retaining only instances where knowledge conflicts occur. Finally, we evaluate ConR and MemR on these sub-datasets.

As shown in Figure~\ref{fig:diff_freq}, a clear trend emerges: as answer frequency increases, ConR consistently decreases, while MemR increases. This pattern indicates that as answer frequency rises, the model becomes increasingly reliant on its internal knowledge. Notably, for data with an answer frequency of 1, MemR is only 3\%, indicating minimal dependence on internal knowledge. Retaining only high-answer-frequency data improves the quality of \dataset{}. This data construction approach distinguishes our methodology from previous studies~\cite{longpre2021entity,xie2023adaptive}.

\begin{figure}[t!]
  \centering
  \includegraphics[width=0.4\textwidth]{figs/diff_freq.pdf}
  \caption{Performance comparison of ConR and MemR across sub-datasets grouped by the answer frequency of LLMs.}
  \label{fig:diff_freq}
\end{figure}





\subsection{Additional Implementation Details of Our Experiments}
\label{append:implementation}
This subsection outlines the training prompt, describes more details of the training data, and provides details of the experimental setup used in our experiments.

\textbf{Training Prompts.}
We adopt a simple QA-format training prompt following~\citet{zhou2023context} for all methods except \attrprompt{} and \oiprompt{}.
\begin{tcolorbox}
[title=Base Prompt ,colback=blue!10,colframe=blue!50!black,arc=1mm,boxrule=1pt,left=1mm,right=1mm,top=1mm,bottom=1mm]
% \small
\textcolor{blue}{\{\textit{context}\}} 
Q: \textcolor{blue}{\{\textit{question}\}} ? 
A: \textcolor{blue}{\{\textit{answer}\}}.
\end{tcolorbox}


\textbf{Training Datasets.} During \method{}, we randomly sample 32,580 instances from the training set of the MRQA 2019 benchmark~\cite{fisch2019mrqa} to construct our training data.



\textbf{Experimental Setup.} In this work, all models are trained for 2,100 steps with a total batch size of 32 and a learning rate of 1e-4. To enhance training efficiency, we implemented \method{} with LoRA~\cite{hu2021lora}, setting both the rank $\text{r}$ and scaling factor $\text{alpha}$ to 64. For \method{}, we set $\alpha$ to 0.1 (Eq.~\ref{eq:selct_layers}), which determines the minimum activation ratio difference required for a layer to be pruned. Additionally, we adopt a dynamic $\gamma$ in $\mathcal{L}_{\text{KC}}$ (Eq.~\ref{eq:kc_loss}), which linearly transitions from an initial margin ($\gamma_{0}=1$) to a final margin ($\gamma^*=5$) as training progresses. This adaptive strategy gradually reduces the model's reliance on internal parametric knowledge, encouraging it to rely more on external knowledge provided by the KAG system.


\subsection{Implementation Details of Baselines}
\label{append:baseline}
This subsection describes the implementation details of all baseline methods.

We adopt two prompt-based baselines: the attributed prompt ($\text{Attr}_{\text{prompt}}$) and a combination of opinion-based and instruction-based prompts ($\text{O\&I}_{\text{prompt}}$). The corresponding prompt templates are as follows:

\begin{tcolorbox}
[title=Attr based prompt ,colback=blue!10,colframe=blue!50!black,arc=1mm,boxrule=1pt,left=1mm,right=1mm,top=1mm,bottom=1mm]
% \small
\textcolor{blue}{\{\textit{context}\}} Q: \textcolor{blue}{\{\textit{question}\}} based on the given text? A: \textcolor{blue}{\{\textit{answer}\}}.
\end{tcolorbox}

\begin{tcolorbox}
[title=O\&I based prompt ,colback=blue!10,colframe=blue!50!black,arc=1mm,boxrule=1pt,left=1mm,right=1mm,top=1mm,bottom=1mm]

Bob said ``\textcolor{blue}{\{\textit{context}\}}'' Q: \textcolor{blue}{\{\textit{question}\}} in Bob's opinion? A: \textcolor{blue}{\{\textit{answer}\}}.
\end{tcolorbox}
For the SFT baseline, we incorporate context during training, similar to \method{}, while keeping the remaining experimental settings identical. To construct preference pairs for DPO training, we use contextually aligned answers from the dataset as ``preferred responses'' to ensure the consistency with the provided context. The ``rejected responses'' are generated by identifying parametric knowledge conflicts through our data construction methodology (Sec.~\ref{sec:benchmark}).

For KAFT, we employ a hybrid dataset containing both counterfactual and factual data. Specifically, we integrate the counterfactual data developed by \citet{xie2023adaptive}, leveraging their advanced data construction framework.

By maintaining equivalent dataset sizes and ensuring comparable data quality across all baselines, we provide a rigorous and fair comparison with our proposed \method{}.




\subsection{Extending \method{} to More LLMs}
\label{append:diff_model_performance}


\begin{figure}[t!]
  \centering
  
\subfigure[ConR Results]{
        \label{fig:diff_model:llama_conr}
        \includegraphics[width=0.462\linewidth]{append_fig/llama_conr.pdf}
    }
    \hspace{0.0005\linewidth} 
    \subfigure[MemR Results]{
        \label{fig:diff_model:llama_memr}
        \includegraphics[width=0.462\linewidth]{append_fig/llama_memr.pdf}
    }


  % \includegraphics[width=0.48\textwidth]{figs/diff_model_double.pdf}
 \caption{Average ConR and MemR across different models implemented by LLMs of LLaMA series, before and after applying \method{}.
 }
 \label{fig:diff_model_double_llama}
\end{figure}

\begin{figure}[t]
  \centering
  \subfigure[ConR Results]{
        \label{fig:diff_model:qwen_conr}
        \includegraphics[width=0.462\linewidth]{append_fig/qwen_conr.pdf}
    }
    \hspace{0.0005\linewidth} 
    \subfigure[MemR Results]{
        \label{fig:diff_model:qwen_memr}
        \includegraphics[width=0.462\linewidth]{append_fig/qwen_memr.pdf}
    }
  % \includegraphics[width=0.48\textwidth]{figs/diff_model_double.pdf}
 \caption{Average ConR and MemR across different models implemented by LLMs of Qwen series, before and after applying \method{}.
 }
 \label{fig:diff_model_double_qwen}
\end{figure}






We extend \method{} to a diverse range of LLMs, encompassing multiple model families and sizes. 

Specifically, our evaluation includes LLaMA3-8B-Instruct, LLaMA3.2-1B-Instruct, LLaMA3.2-3B-Instruct, Qwen2.5-0.5B-Instruct, Qwen2.5-1.5B-Instruct, Qwen2.5-3B-Instruct, Qwen2.5-7B-Instruct, and Qwen2.5-14B-Instruct. The results on ConR and MemR are summarized in Figures~\ref{fig:diff_model_double_llama} and \ref{fig:diff_model_double_qwen}, while Table~\ref{tab:append:all_model_res} presents the average performance of all models on \dataset{} and ConFiQA. Additionally, Table~\ref{tab:diff_model_param} provides detailed parameter information and specifies the layers selected for pruning for each model. This comprehensive evaluation demonstrates the versatility and scalability of \method{} across a wide spectrum of model architectures and sizes.

\begin{table}[!t]
  
    \resizebox{0.48\textwidth}{!}{%
\begin{tabular}{l|c|c|c}
\toprule
\textbf{Models}     & \textbf{Param.} & \textbf{\method{} Param.} & \textbf{Selected Layers} \\
\midrule
\rowcolor{gray!10}
LLaMA3.2-1B        & 1.24B  & 1.08B \small\textcolor{gray}{(87\%)}   & [12, 14]                 \\
LLaMA3.2-3B        & 3.21B  & 2.60B \small\textcolor{gray}{(81\%)}   &  [18, 25]   \\
\rowcolor{gray!10}
LLaMA3-8B          & 8.03B  & 6.97B \small\textcolor{gray}{(87\%)}   & [24, 29]      \\
LLaMA3.1-8B          & 8.03B  & 6.27B \small\textcolor{gray}{(78\%)}   & [20, 29]      \\
\rowcolor{gray!10}
Qwen2.5-0.5B         & 0.49B  & 0.44B \small\textcolor{gray}{(90\%)}   &  [19, 22]       \\
Qwen2.5-1.5B         & 1.54B  & 1.34B \small\textcolor{gray}{(87\%)}   & [21, 25]        \\
\rowcolor{gray!10}
Qwen2.5-3B         & 3.09B  & 2.68B \small\textcolor{gray}{(87\%)}   & [29, 34]        \\
Qwen2.5-7B         & 7.61B  & 7.21B \small\textcolor{gray}{(95\%)}   &   [25, 26 ]     \\
\rowcolor{gray!10}
Qwen2.5-14B        & 14.70B & 12.43B \small\textcolor{gray}{(85\%)}  &  [35, 45]   \\
\bottomrule
\end{tabular}
}

% \end{sidewaystable}

% \end{document}

  \caption{The total number of parameters for various models before and after applying \method{}. \textcolor{gray}{\small$(\cdot)\%$} represents the proportion relative to the original model, and the last column lists the layers selected for pruning.}
   \label{tab:diff_model_param}
\end{table}

These experimental results illustrate several key insights: 1) Larger models tend to rely more on parametric memory. As model size increases in both the LLaMA and Qwen families, MemR also grows, indicating a tendency to overlook external knowledge in favor of internal parameters. \method{} counteracts this behavior, decreasing larger models' MemR score to even below that of smaller models. 2) \method{} consistently benefits all evaluated models. Across both LLaMA and Qwen model families, \method{} outperforms Vanilla-KAG by boosting accuracy and context faithfulness, underscoring its broad applicability and effectiveness. 3) Not all parameters in KAG models are essential. Pruning parametric knowledge not only reduces computation costs but also fosters better generalization without sacrificing accuracy, highlighting the potential of building a parameter-efficient LLM within the KAG framework.




\begin{table*}[!t]
  
\centering
\resizebox{0.96\textwidth}{!}{%
\begin{tabular}{l|c|cccc|cccc}
\toprule
\multirow{2}{*}{\textbf{Models}} & \multirow{2}{*}{\textbf{Param.}} & \multicolumn{4}{c|}{\textbf{\dataset{}}} & \multicolumn{4}{c}{\textbf{ConFiQA}} \\ 
\cmidrule(lr){3-6}  \cmidrule(lr){7-10}
 &  & ConR $\uparrow$ & MemR $\downarrow$ & MR $\downarrow$ & EM $\uparrow$ & ConR $\uparrow$ & MemR $\downarrow$ & MR $\downarrow$ & EM $\uparrow$ \\ 
\midrule
LLaMA3-8B   & 8.03B  & 66.99  & 11.75  & 14.99  & 13.83  & 22.52  & 31.15  & 59.77  & 2.47 \\
\rowcolor{gray!10}
+\method{}    & 6.97B  & 71.50  & 6.48   & 8.41   & 66.19  & 70.43  & 8.82   & 11.32  & 67.29 \\
LLaMA3.1-8B & 8.03B  & 63.15  & 11.69  & 15.93  & 21.85  & 15.38  & 29.97  & 68.98  & 6.69 \\
\rowcolor{gray!10}
+\method{}   & 6.27B  & 70.41  & 6.95   & 9.17   & 63.58  & 71.12  & 9.01   & 11.44  & 66.61 \\
LLaMA3.2-1B & 1.24B  & 39.06  & 10.49  & 21.83  & 5.13   & 32.09  & 18.32  & 36.28  & 7.15 \\
\rowcolor{gray!10}
+\method{}   & 1.08B  & 51.75  & 6.51   & 11.34  & 47.60  & 62.70  & 7.63   & 11.38  & 61.85 \\
LLaMA3.2-3B & 3.21B  & 56.75  & 11.53  & 17.11  & 12.69  & 26.16  & 23.47  & 49.05  & 9.84 \\
\rowcolor{gray!10}
+\method{}   & 2.60B  & 67.00  & 6.80   & 9.35   & 61.59  & 69.61  & 8.39   & 11.09  & 66.53 \\
Qwen2.5-0.5B & 0.49B  & 47.17  & 11.36  & 19.48  & 2.06   & 50.72  & 17.15  & 26.20  & 3.78 \\
\rowcolor{gray!10}
+\method{}   & 0.44B  & 58.13  & 6.63   & 10.41  & 52.56  & 67.54  & 8.04   & 11.03  & 66.33 \\
Qwen2.5-1.5B & 1.54B  & 58.08  & 11.28  & 16.48  & 10.30  & 51.69  & 19.87  & 28.23  & 10.78 \\
\rowcolor{gray!10}
+\method{}   & 1.34B  & 63.78  & 6.74   & 9.76   & 57.67  & 69.61   & 8.35   & 11.05   & 66.04 \\
Qwen2.5-3B   & 3.09B  & 62.22  & 14.45  & 18.88  & 0.10   & 25.47  & 29.34  & 55.70  & 0.01 \\
\rowcolor{gray!10}
+\method{}     & 2.68B  & 66.31  & 6.75   & 9.38   & 59.42  & 66.30   & 8.62  & 11.94   & 63.03 \\
Qwen2.5-7B    & 7.61B  & 65.46  & 14.93  & 18.57  & 0.80   & 24.75  & 33.09  & 59.04  & 0.10 \\
\rowcolor{gray!10}
+\method{}      & 6.60B  & 67.75  & 6.60   & 9.01   & 61.77  & 69.54  & 8.85   & 11.58  & 66.68 \\
Qwen2.5-14B   & 14.70B & 65.75  & 16.13  & 19.75  & 0.00   & 7.86   & 32.88  & 83.71  & 0.01 \\
\rowcolor{gray!10}
+\method{}     & 12.43B & 70.01  & 6.43   & 8.55   & 64.43  & 71.70  & 8.90   & 11.29  & 68.40 \\
\bottomrule
\end{tabular}%
}


  \caption{Average performance of LLMs on \dataset{} and ConFiQA before and after applying \method{}.}
   \label{tab:append:all_model_res}
\end{table*}

\subsection{Neuron Activations in Different LLMs}\label{app:activation}
We present the neuron activations for the LLaMA family models, including LLaMA-3.2-1B-Instruct, LLaMA-3.2-3B-Instruct, LLaMA-3-8B-Instruct, and LLaMA-3.1-8B-Instruct, as well as the Qwen family models, including Qwen-2.5-0.5B-Instruct, Qwen-2.5-1.5B-Instruct, Qwen-2.5-3B-Instruct, Qwen-2.5-7B-Instruct, and Qwen-2.5-14B-Instruct, in Figures~\ref{fig:act_llama} and \ref{fig:act_qwen}, respectively. 
% 我们发现qwen系列模型


\begin{figure*}[t]
  \centering
  \subfigure[Neuron activations of LLaMA-3.2-1B-Instruct]{
        \label{fig:act_llama:3.2-1b}
        \includegraphics[width=0.9\linewidth]{append_fig/act_llama32_1b_all.pdf}
    }
\subfigure[Neuron activations of LLaMA-3.2-3B-Instruct]{
        \label{fig:act_llama:3.2-3b}
        \includegraphics[width=0.9\linewidth]{append_fig/act_llama32_3b_all.pdf}
    }
 \subfigure[Neuron activations of LLaMA-3-8B-Instruct]{
        \label{fig:act_llama:3-8b}
        \includegraphics[width=0.9\linewidth]{append_fig/act_llama_3_8b.pdf}
    }
 \subfigure[Neuron activations of LLaMA-3.1-8B-Instruct]{
        \label{fig:act_llama:3.1-8b}
        \includegraphics[width=0.9\linewidth]{append_fig/act_llama_31_8b.pdf}
    }
 

 \caption{Neuron activations across different layers of the LLaMA series models. We present the inhibition ratio $\Delta R$ under two conditions: with contextual knowledge input (w/ context) and without it (w/o context).}
 \label{fig:act_llama}
\end{figure*}

\begin{figure*}[t]
  \centering
  \subfigure[Neuron activations of Qwen-2.5-0.5B-Instruct]{
        \label{fig:act_qwen:2.5-0.5b}
        \includegraphics[width=0.75\linewidth]{append_fig/act_qwen25_0_5b_all.pdf}
    }
\subfigure[Neuron activations of Qwen-2.5-1.5B-Instruct]{
        \label{fig:act_qwen:2.5-1.5b}
        \includegraphics[width=0.75\linewidth]{append_fig/act_qwen25_1_5b_all.pdf}
    }
\subfigure[Neuron activations of Qwen-2.5-3B-Instruct]{
        \label{fig:act_qwen:2.5-3b}
        \includegraphics[width=0.75\linewidth]{append_fig/act_qwen25_3b_all.pdf}
    }
\subfigure[Neuron activations of Qwen-2.5-7B-Instruct]{
        \label{fig:act_qwen:2.5-7b}
        \includegraphics[width=0.75\linewidth]{append_fig/act_qwen25_7b_all.pdf}
    }
\subfigure[Neuron activations of Qwen-2.5-14B-Instruct]{
        \label{fig:act_qwen:2.5-14b}
        \includegraphics[width=0.75\linewidth]{append_fig/act_qwen25_14b_all.pdf}
    }


 \caption{Neuron activations across different layers of the Qwen series models. We present the inhibition ratio $\Delta R$ under two conditions: with contextual knowledge input (w/ context) and without it (w/o context). }
 \label{fig:act_qwen}
\end{figure*}



\end{document}
 