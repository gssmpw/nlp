\documentclass[acmlarge, imwut]{acmart}

\usepackage[export]{adjustbox}
\usepackage{enumitem}
\usepackage{multirow}
\usepackage{arydshln}
\PassOptionsToPackage{table}{xcolor}
\usepackage{colortbl}
\usepackage[capitalise]{cleveref}


%%
%% \BibTeX command to typeset BibTeX logo in the docs
\AtBeginDocument{%
  \providecommand\BibTeX{{%
    Bib\TeX}}}

\setcopyright{acmlicensed}
\copyrightyear{2025}
\acmYear{2025}
\acmDOI{XXXXXXX.XXXXXXX}

\begin{document}

\title[Data-driven Identification of Sub-activities via Clustering and Visualization for Enhanced Activity Recognition in Smart Homes]{DISCOVER: \underline{D}ata-driven \underline{I}dentification of \underline{S}ub-activities via \underline{C}lustering and \underline{V}isualization for \underline{E}nhanced \underline{R}ecognition of Human Activities in Smart Homes}

%% The "author" command and its associated commands are used to define
%% the authors and their affiliations.
%% Of note is the shared affiliation of the first two authors, and the
%% "authornote" and "authornotemark" commands
%% used to denote shared contribution to the research.
\author{Alexander Karpekov}
\email{alex.karpekov@gatech.edu}
\orcid{0009-0006-5317-2991}
\affiliation{%
  \institution{Georgia Institute of Technology}
  \city{Atlanta}
  \state{Georgia}
  \country{USA}
}

\author{Sonia Chernova}
\email{chernova@gatech.edu}
\orcid{0000-0001-6320-0825}
\affiliation{%
  \institution{Georgia Institute of Technology}
  \city{Atlanta}
  \state{Georgia}
  \country{USA}
}

\author{Thomas Pl{\"o}tz}
\email{thomas.ploetz@gatech.edu}
\orcid{0000-0002-1243-7563}
\affiliation{%
  \institution{Georgia Institute of Technology}
  \city{Atlanta}
  \state{Georgia}
  \country{USA}
}

%%
%% By default, the full list of authors will be used in the page
%% headers. Often, this list is too long, and will overlap
%% other information printed in the page headers. This command allows
%% the author to define a more concise list
%% of authors' names for this purpose.

\renewcommand{\shortauthors}{Karpekov et al.}


\definecolor{rubinered}{HTML}{CE0058}
\newcommand{\thomas}[1]{\textcolor{rubinered}{[#1 -- TP]}}

\definecolor{bleudefrance}{rgb}{0.19, 0.55, 0.91}
\newcommand{\sonia}[1]{{{\textcolor{bleudefrance}{[#1 -- SC]}}}}

\definecolor{green}{rgb}{0.0, 0.65, 0.31}
\newcommand{\alex}[1]{{{\textcolor{green}{[#1 -- AK]}}}}

\definecolor{green}{rgb}{0.0, 0.65, 0.31}
\newcommand{\sourish}[1]{{{\textcolor{red}{[#1 -- SGD]}}}}

\newcommand{\ToolName}{DISCOVER}

\begin{abstract}
\begin{abstract}


The choice of representation for geographic location significantly impacts the accuracy of models for a broad range of geospatial tasks, including fine-grained species classification, population density estimation, and biome classification. Recent works like SatCLIP and GeoCLIP learn such representations by contrastively aligning geolocation with co-located images. While these methods work exceptionally well, in this paper, we posit that the current training strategies fail to fully capture the important visual features. We provide an information theoretic perspective on why the resulting embeddings from these methods discard crucial visual information that is important for many downstream tasks. To solve this problem, we propose a novel retrieval-augmented strategy called RANGE. We build our method on the intuition that the visual features of a location can be estimated by combining the visual features from multiple similar-looking locations. We evaluate our method across a wide variety of tasks. Our results show that RANGE outperforms the existing state-of-the-art models with significant margins in most tasks. We show gains of up to 13.1\% on classification tasks and 0.145 $R^2$ on regression tasks. All our code and models will be made available at: \href{https://github.com/mvrl/RANGE}{https://github.com/mvrl/RANGE}.

\end{abstract}


\end{abstract}

\begin{CCSXML}
<ccs2012>
   <concept>
       <concept_id>10003120.10003138</concept_id>
       <concept_desc>Human-centered computing~Ubiquitous and mobile computing</concept_desc>
       <concept_significance>500</concept_significance>
       </concept>
   <concept>
       <concept_id>10010147.10010257.10010293</concept_id>
       <concept_desc>Computing methodologies~Machine learning approaches</concept_desc>
       <concept_significance>500</concept_significance>
       </concept>
 </ccs2012>
\end{CCSXML}

\ccsdesc[500]{Human-centered computing~Ubiquitous and mobile computing}
\ccsdesc[500]{Computing methodologies~Machine learning approaches}


%%
%% Keywords. The author(s) should pick words that accurately describe
%% the work being presented. Separate the keywords with commas.
\keywords{human activity recognition, smart home, ambient sensors, active learning, clustering}

\maketitle

%%% custom footer, remove for submission ----------
 \thispagestyle{fancy}
%... then configure it.
\fancyhead{} % clear all header fields
\fancyfoot{} % clear all footer fields
\pagenumbering{gobble}
\fancyfoot[C]{\textcolor{red}{This manuscript is under review. Please contact alex.karpekov@gatech.edu for up-to-date information}}
%---------------------------------------------------

\section{Introduction}
Backdoor attacks pose a concealed yet profound security risk to machine learning (ML) models, for which the adversaries can inject a stealth backdoor into the model during training, enabling them to illicitly control the model's output upon encountering predefined inputs. These attacks can even occur without the knowledge of developers or end-users, thereby undermining the trust in ML systems. As ML becomes more deeply embedded in critical sectors like finance, healthcare, and autonomous driving \citep{he2016deep, liu2020computing, tournier2019mrtrix3, adjabi2020past}, the potential damage from backdoor attacks grows, underscoring the emergency for developing robust defense mechanisms against backdoor attacks.

To address the threat of backdoor attacks, researchers have developed a variety of strategies \cite{liu2018fine,wu2021adversarial,wang2019neural,zeng2022adversarial,zhu2023neural,Zhu_2023_ICCV, wei2024shared,wei2024d3}, aimed at purifying backdoors within victim models. These methods are designed to integrate with current deployment workflows seamlessly and have demonstrated significant success in mitigating the effects of backdoor triggers \cite{wubackdoorbench, wu2023defenses, wu2024backdoorbench,dunnett2024countering}.  However, most state-of-the-art (SOTA) backdoor purification methods operate under the assumption that a small clean dataset, often referred to as \textbf{auxiliary dataset}, is available for purification. Such an assumption poses practical challenges, especially in scenarios where data is scarce. To tackle this challenge, efforts have been made to reduce the size of the required auxiliary dataset~\cite{chai2022oneshot,li2023reconstructive, Zhu_2023_ICCV} and even explore dataset-free purification techniques~\cite{zheng2022data,hong2023revisiting,lin2024fusing}. Although these approaches offer some improvements, recent evaluations \cite{dunnett2024countering, wu2024backdoorbench} continue to highlight the importance of sufficient auxiliary data for achieving robust defenses against backdoor attacks.

While significant progress has been made in reducing the size of auxiliary datasets, an equally critical yet underexplored question remains: \emph{how does the nature of the auxiliary dataset affect purification effectiveness?} In  real-world  applications, auxiliary datasets can vary widely, encompassing in-distribution data, synthetic data, or external data from different sources. Understanding how each type of auxiliary dataset influences the purification effectiveness is vital for selecting or constructing the most suitable auxiliary dataset and the corresponding technique. For instance, when multiple datasets are available, understanding how different datasets contribute to purification can guide defenders in selecting or crafting the most appropriate dataset. Conversely, when only limited auxiliary data is accessible, knowing which purification technique works best under those constraints is critical. Therefore, there is an urgent need for a thorough investigation into the impact of auxiliary datasets on purification effectiveness to guide defenders in  enhancing the security of ML systems. 

In this paper, we systematically investigate the critical role of auxiliary datasets in backdoor purification, aiming to bridge the gap between idealized and practical purification scenarios.  Specifically, we first construct a diverse set of auxiliary datasets to emulate real-world conditions, as summarized in Table~\ref{overall}. These datasets include in-distribution data, synthetic data, and external data from other sources. Through an evaluation of SOTA backdoor purification methods across these datasets, we uncover several critical insights: \textbf{1)} In-distribution datasets, particularly those carefully filtered from the original training data of the victim model, effectively preserve the model’s utility for its intended tasks but may fall short in eliminating backdoors. \textbf{2)} Incorporating OOD datasets can help the model forget backdoors but also bring the risk of forgetting critical learned knowledge, significantly degrading its overall performance. Building on these findings, we propose Guided Input Calibration (GIC), a novel technique that enhances backdoor purification by adaptively transforming auxiliary data to better align with the victim model’s learned representations. By leveraging the victim model itself to guide this transformation, GIC optimizes the purification process, striking a balance between preserving model utility and mitigating backdoor threats. Extensive experiments demonstrate that GIC significantly improves the effectiveness of backdoor purification across diverse auxiliary datasets, providing a practical and robust defense solution.

Our main contributions are threefold:
\textbf{1) Impact analysis of auxiliary datasets:} We take the \textbf{first step}  in systematically investigating how different types of auxiliary datasets influence backdoor purification effectiveness. Our findings provide novel insights and serve as a foundation for future research on optimizing dataset selection and construction for enhanced backdoor defense.
%
\textbf{2) Compilation and evaluation of diverse auxiliary datasets:}  We have compiled and rigorously evaluated a diverse set of auxiliary datasets using SOTA purification methods, making our datasets and code publicly available to facilitate and support future research on practical backdoor defense strategies.
%
\textbf{3) Introduction of GIC:} We introduce GIC, the \textbf{first} dedicated solution designed to align auxiliary datasets with the model’s learned representations, significantly enhancing backdoor mitigation across various dataset types. Our approach sets a new benchmark for practical and effective backdoor defense.




\section{Related Work}

\subsection{Large 3D Reconstruction Models}
Recently, generalized feed-forward models for 3D reconstruction from sparse input views have garnered considerable attention due to their applicability in heavily under-constrained scenarios. The Large Reconstruction Model (LRM)~\cite{hong2023lrm} uses a transformer-based encoder-decoder pipeline to infer a NeRF reconstruction from just a single image. Newer iterations have shifted the focus towards generating 3D Gaussian representations from four input images~\cite{tang2025lgm, xu2024grm, zhang2025gslrm, charatan2024pixelsplat, chen2025mvsplat, liu2025mvsgaussian}, showing remarkable novel view synthesis results. The paradigm of transformer-based sparse 3D reconstruction has also successfully been applied to lifting monocular videos to 4D~\cite{ren2024l4gm}. \\
Yet, none of the existing works in the domain have studied the use-case of inferring \textit{animatable} 3D representations from sparse input images, which is the focus of our work. To this end, we build on top of the Large Gaussian Reconstruction Model (GRM)~\cite{xu2024grm}.

\subsection{3D-aware Portrait Animation}
A different line of work focuses on animating portraits in a 3D-aware manner.
MegaPortraits~\cite{drobyshev2022megaportraits} builds a 3D Volume given a source and driving image, and renders the animated source actor via orthographic projection with subsequent 2D neural rendering.
3D morphable models (3DMMs)~\cite{blanz19993dmm} are extensively used to obtain more interpretable control over the portrait animation. For example, StyleRig~\cite{tewari2020stylerig} demonstrates how a 3DMM can be used to control the data generated from a pre-trained StyleGAN~\cite{karras2019stylegan} network. ROME~\cite{khakhulin2022rome} predicts vertex offsets and texture of a FLAME~\cite{li2017flame} mesh from the input image.
A TriPlane representation is inferred and animated via FLAME~\cite{li2017flame} in multiple methods like Portrait4D~\cite{deng2024portrait4d}, Portrait4D-v2~\cite{deng2024portrait4dv2}, and GPAvatar~\cite{chu2024gpavatar}.
Others, such as VOODOO 3D~\cite{tran2024voodoo3d} and VOODOO XP~\cite{tran2024voodooxp}, learn their own expression encoder to drive the source person in a more detailed manner. \\
All of the aforementioned methods require nothing more than a single image of a person to animate it. This allows them to train on large monocular video datasets to infer a very generic motion prior that even translates to paintings or cartoon characters. However, due to their task formulation, these methods mostly focus on image synthesis from a frontal camera, often trading 3D consistency for better image quality by using 2D screen-space neural renderers. In contrast, our work aims to produce a truthful and complete 3D avatar representation from the input images that can be viewed from any angle.  

\subsection{Photo-realistic 3D Face Models}
The increasing availability of large-scale multi-view face datasets~\cite{kirschstein2023nersemble, ava256, pan2024renderme360, yang2020facescape} has enabled building photo-realistic 3D face models that learn a detailed prior over both geometry and appearance of human faces. HeadNeRF~\cite{hong2022headnerf} conditions a Neural Radiance Field (NeRF)~\cite{mildenhall2021nerf} on identity, expression, albedo, and illumination codes. VRMM~\cite{yang2024vrmm} builds a high-quality and relightable 3D face model using volumetric primitives~\cite{lombardi2021mvp}. One2Avatar~\cite{yu2024one2avatar} extends a 3DMM by anchoring a radiance field to its surface. More recently, GPHM~\cite{xu2025gphm} and HeadGAP~\cite{zheng2024headgap} have adopted 3D Gaussians to build a photo-realistic 3D face model. \\
Photo-realistic 3D face models learn a powerful prior over human facial appearance and geometry, which can be fitted to a single or multiple images of a person, effectively inferring a 3D head avatar. However, the fitting procedure itself is non-trivial and often requires expensive test-time optimization, impeding casual use-cases on consumer-grade devices. While this limitation may be circumvented by learning a generalized encoder that maps images into the 3D face model's latent space, another fundamental limitation remains. Even with more multi-view face datasets being published, the number of available training subjects rarely exceeds the thousands, making it hard to truly learn the full distibution of human facial appearance. Instead, our approach avoids generalizing over the identity axis by conditioning on some images of a person, and only generalizes over the expression axis for which plenty of data is available. 

A similar motivation has inspired recent work on codec avatars where a generalized network infers an animatable 3D representation given a registered mesh of a person~\cite{cao2022authentic, li2024uravatar}.
The resulting avatars exhibit excellent quality at the cost of several minutes of video capture per subject and expensive test-time optimization.
For example, URAvatar~\cite{li2024uravatar} finetunes their network on the given video recording for 3 hours on 8 A100 GPUs, making inference on consumer-grade devices impossible. In contrast, our approach directly regresses the final 3D head avatar from just four input images without the need for expensive test-time fine-tuning.



\section{DISCOVER: Data-driven Identification of Sub-activities via Clustering and Visualization for Enhanced Recognition}

\begin{figure}
    \centering
        \includegraphics[width=0.95\linewidth]{figures/main_image.pdf}
    \caption{
        Overview of \ToolName{}, a self-supervised system designed to \textit{discover} fine-grained human activities from unlabeled sensor data without relying on pre-segmentation, consisting of two main stages: clustering and labeling.
        After (0) slicing the raw data into continuous sliding windows of sensor activations \textit{ without assuming any pre-segmentation}, we (1) train a BERT model with mask language modeling task to encode the windows in an embedding space. We then use these embeddings to identify similar activity windows and (2) fine-tune a clustering model using SCAN loss, which results in assigning all data points to \( k \) clusters. We then (3) sample a handful of windows closest to each cluster centroid, replay them on 2D house layouts using our custom built visualization tool, and send these samples to a group of experts for annotation. With minimal labeling effort, we obtain custom granular activity labels for each cluster centroid, and (4) propagate them to the rest of the data points in each cluster. These custom labels are then (5) applied to the original dataset and can later be used for a set of specialized downstream tasks.
    } 
    \vspace*{-1em}
    \label{fig:main_image}
\end{figure}

To address the challenges of human activity recognition in smart homes, we introduce \ToolName{}, a data-driven approach that \textit{discovers} fine-grained human activities from unlabeled sensor data without relying on pre-segmentation. 
%Building on the insights from recent advances in self-supervised learning and clustering, 
\ToolName{} combines unsupervised feature extraction with an interactive annotation tool, enabling more granular and personalized activity labels.

We consider the problem of 
%\emph{self-supervised} 
human activity recognition in smart home environments. Given a time-series dataset of sensor readings collected over a period of time, divided into a collection of \(n\) sliding windows \(\{W_1, W_2, \dots, W_n\}\) of size \( l \), our goal is to assign each of these windows to an activity label. Formally, we seek to learn a mapping
\(
f: W_i \mapsto c_k,
\)
where \(c_k\) represents the activity cluster for window \(W_i\). A natural challenge in this setting is the lack of ground truth labels for the time windows.

\cref{fig:main_image} gives an overview of \ToolName{}, which consists of two main stages: clustering and labeling. The approach starts by pre-processing sensor data into sliding windows in step (0). We then pursue a active learning sub-activity discovery through the following five steps:

\begin{enumerate}
    \item \textbf{Encoder Pre-Training}: Create an embedding representation of sensor readings by training a BERT transformer model with a self-supervised, mask language modeling task. Use these embeddings to create a set of \( m \) nearest neighbors for each window \( W_i \).
    
    \item \textbf{Clustering Model Fine-Tuning}: Fine-tune the pre-trained BERT model using the SCAN loss function \cite{scan2020}, which partitions all windows \( W_i \) into \( k \) distinct clusters, assigning each \( W_i \) to one of them.

    \item \textbf{Centroid Annotation}: Select a handful of windows \( W_i\) closest to each cluster centroid, and send them to expert annotators to have them assign a label to each sample using \ToolName{} custom built visualization tool.
    
    \item \textbf{Label propagation}: Propagate the centroid labels to every data point in the cluster. 
    
    \item \textbf{Re-Annotation of the Original Time-Series Data}: Apply these cluster activity labels to the original time-series data.
\end{enumerate}

\subsection{Encoder Pre-Training}

\begin{figure}
    \centering
        \includegraphics[width=0.95\linewidth]{figures/training_pipeline.pdf}
    \vspace*{-1em}
    \caption{
        \ToolName{} approach -- model training pipeline, consisting of (1) Encoder Pre-Training, and (2) Clustering Model Fine-Tuning. 
        \ToolName{} first trains a BERT model using a Masked Language Modeling head in (1) to obtain initial embeddings for each window \( W_i \). In step 2(a) it uses these embeddings to identify similar activity windows and pairs them together as a a new training set. It then continues training the pre-trained BERT base model with a SCAN loss (2(b)). In the end, the trained SCAN model assigns a cluster \( c_k \) to each input sequence \( W_i \) (2(c)).
    }
    \vspace*{-1em}
    \label{fig:training_pipeline}
\end{figure}

The first step is to learn an informative representation for raw sensor sequences. In this subsection, we describe how we adapt a BERT model through a masked language modeling (MLM) objective to capture context-rich representations of ambient sensor data without any labels.

We build on the approach from \cite{hiremath2022bootstrapping}, which employs a pre-trained BERT Transformer model to derive embeddings for sensor sequences by feeding in new, domain-specific tokens and training for several epochs. However, unlike \cite{hiremath2022bootstrapping}, our pipeline does not rely on pre-segmented data, staying faithful to a more realistic, fully unsupervised scenario. \cref{fig:training_pipeline}(1) shows a diagram of this process. First, we load the original BERT model that was pre-trained on text corpora (BooksCorpus and English Wikipedia) \cite{bert2019} from HuggingFace \cite{huggingface2020} and extend the original vocabulary of size \texttt{30,522} WordPiece tokens by adding sensors and their readings as new tokens to the original vocabulary (which evaluates to roughly \texttt{100} new tokens per dataset). For example, sensor labeled \texttt{M1} and its output signal \texttt{ON} are concatenated into a new token \texttt{M1\_ON}, assigned a new token ID, and initialized with random \texttt{768}-dimensional embedding. Once all training input sequences are processed and encoded, we train BERT model using a self-supervised MLM objective, defined below. 

For every input sequence \( W_i = \{d_1, d_2, \dots, d_l\} \) of sensor tokens \( d \), we choose a proportion \( p \) of these readings to mask. Following \cite{hiremath2020deriving} and the original BERT paper \cite{bert2019}, we set \( p = 0.15 \). Let \( M \subset W_i \) be a subset of positions selected for masking, where \(\lvert M \rvert = \lfloor p \times l \rfloor\). We replace each token in \( M \) with a special \texttt{[MASK]} token, producing a masked sequence \(\hat{W_i}\). The model is then trained to predict the original tokens in \( M \) based on the surrounding context provided by \(\hat{W_i}\).

The training loss \( \mathcal{L}_{MLM} \) is defined as:
\[
\mathcal{L}_{MLM} = - \frac{1}{|M|} \sum_{x \in M} \log P(d_x \mid \hat{W_i})
\]

where \( P(d_x \mid \hat{W_i}) \) is the probability that \(x\)-th masked token \( d \) is correctly predicted by the model.

This loss function incorporates the context from the surrounding, unmasked tokens (i.e., sensor readings) in each sequence. By introducing new tokens for sensor readings (e.g., \texttt{M1\_ON}) into the vocabulary and initializing their embeddings randomly, the model is able to integrate sensor-specific information into the learned representations during pre-training.
Once the training is done, we obtain an embedding for each input sequence \( W_i \) by removing the MLM head and extracting the \texttt{768}-dimensional vector of a special \texttt{[CLS]} token. This \texttt{[CLS]} ("classification") token is specifically designed to represent the entire sequence in a single embedding, a common practice in BERT-based architectures to obtain a fixed-length representation of any input sequence \cite{bert2019}.

\subsection{Fine-Tuning the Clustering Model}
Once we obtain an embedding for each window \(W_i\) in the training data, we use these embeddings to measure the similarity between individual sequence vectors, build a "neighbors" dataset, and train a clustering model using this data.

First, for every input sequence we want to find and store a small set of examples that are most similar to it in the BERT embedding space (\cref{fig:training_pipeline} (2a)). Let \( \text{sim}(W_q, W_r) \) represent the cosine similarity between two embedding vectors \( W_q \) and \( W_r \), defined as:

\[
\text{sim}(W_q, W_r) = \frac{W_q \cdot W_r}{\|W_q\| \|W_r\|}
\]

\noindent For every input sequence \( W_i \), we identify its \( m \)-nearest neighbors using this cosine similarity. Following \cite{scan2020}, we set \( m \) equal to 20. These neighbors will be used as training examples in the following step where the SCAN loss function will be forcing them to be in the same cluster \( c_k\)

Then, we further fine-tune the re-trained BERT model from the previous step by redefining its training objective (\cref{fig:training_pipeline} (2b)). To achieve that, we replace the MLM training head and its cross-entropy loss function with a clustering head paired with the SCAN loss \cite{scan2020}. \textbf{SCAN loss} consists of two components, an \textbf{instance-level contrastive loss} and a \textbf{cluster-level entropy loss}, defined as follows:

\begin{itemize}

    \item 
    \textbf{Instance-Level Contrastive Loss}: Encourages every input sequence and its neighbors to belong to the same cluster. For a sequence \( W_i \), let \( M(W_i) \) denote its set of \( m \)-nearest neighbors. The contrastive loss \( \mathcal{L}_{\text{contrastive}} \) is defined as:

    \[
    \mathcal{L}_{\text{contrastive}} = - \frac{1}{n} \sum_{W_i} \frac{1}{|M(W_i)|} \sum_{W_j \in M(W_i)} \log P(c_i = c_j)
    \]
    
    where \( P(c_i = c_j) \) is the probability that \( W_i \) and \( W_j \) are assigned to the same cluster.

    \item \textbf{Cluster-Level Entropy Loss}: Encourages balanced cluster assignments by ensuring that the distribution of cluster labels is uniform. Let \( P_k \) denote the probability of sequences being assigned to cluster \( k \). The entropy loss \( \mathcal{L}_{\text{entropy}} \) is defined as:
    
    \[
    \mathcal{L}_{\text{entropy}} = - \sum_{k=1}^K P_k \log P_k
    \]

    \item \textbf{Combined SCAN Loss}: The total SCAN loss \( \mathcal{L}_{\text{SCAN}} \) is a sum of the two components:

    \[
    \mathcal{L}_{\text{SCAN}} = \mathcal{L}_{\text{contrastive}} + \lambda \mathcal{L}_{\text{entropy}}
    \]
    
    where \( \lambda \) controls the weight of the entropy loss. We use the default value of \( \lambda = 2\) \cite{scan2020}.

\end{itemize}

SCAN loss enables the BERT model to assign cluster labels 
\( c_i \in \{1, 2, \dots, k\} \) to each sequence \( W_i \). The clustering head is initialized 
with random labels and iteratively learns to group similar sequences together, thus refining the underlying embeddings into more distinct clusters. Along with the cluster labels (\cref{fig:training_pipeline} (2c)), the trained model provides a probability \( P(c_i = k) \), indicating how confidently each sequence 
\( W_i \) is assigned to cluster \( k \). In the end, every input window receives both a cluster assignment and a corresponding probability distribution over all clusters.


\subsection{Cluster Centroid Annotation}

\begin{figure}
    \centering
        \includegraphics[width=0.9\linewidth]{figures/house_map_replays.pdf}
    \caption{
    \ToolName{} custom-built interactive in-browser annotation tool for reviewing sensor activation sequences. The tool displays a 2D house layout, allowing annotators to replay sequences temporally, observe contextual and spatial details, and assign labels via a drop-down menu. The example above shows a sequence of sensor activations following a resident walking to the guest bathroom, that an annotator can label as "Guest Bathroom: Walking In" from the drop-down menu.
    }
    \label{fig:viz_tool}
\end{figure}

Given the set of clusters learned in the previous stage, the next goal is to interpret and contextualize them by assigning meaningful labels. Since the model has so far operated without any supervision, no ready-made labels exist to inform downstream analyses or practical applications. By annotating a small number of representative samples from each cluster, we can translate the resulting clusters into actionable units—such as identifying specific sub-activities or understanding when and where a resident might be performing a certain activity. This process not only validates the clusters internally but also bridges the gap between the automatic discovery of patterns and real-world interpretability.

To minimize the number of sequences we need to review, we will only use \( N \) samples from each cluster that have the highest probability of belonging to that cluster. The probability \( P(c_i = k) \) is taken directly from the SCAN model output. In step (3) in \cref{fig:main_image}, we select \( N \times k \) samples and review them manually using the custom tool that we developed, shown in \cref{fig:viz_tool}. This interactive in-browser annotation tool shows a detailed layout of a given house floor plan and allows to playback the sensor activation sequences temporally, and assign a label from a drop-down menu. This tool lets the annotator see both the contextual information about that sequence (e.g., location of the activity and its surrounding areas), and the temporal dimension (e.g., whether the activity was static, or if it triggered multiple sensor activations throughout the house). For example, \cref{fig:viz_tool} displays a sequence of sensor activations representing a resident walking through the house to the guest bathroom. The annotator can label this sequence as "Guest Bathroom: Walking In." \ToolName{} visualization tool simplifies and speeds up the annotation process and allows for faster iterations. 


\subsection{Label Propagation}
\label{sec:label_propagation}

Once each cluster is assigned a label based on its centroid samples, we propagate labels to all remaining sequences belonging to the cluster. This process creates a fully labeled dataset from initially unlabeled sequences. By grouping each sequence with its dominant cluster label, we enable efficient large-scale annotation of resident activities.

\subsection{Re-Annotation of the Original Time-Series Data}
\label{sec:reannotation}

Having assigned labels to each sequence, we then apply these annotations back to the original time-series data, preserving their chronological order. This comprehensive labeling allows researchers to analyze the temporal progression of activities and identify patterns or shifts in behavior over days, weeks, or longer periods. In doing so, our pipeline not only re-annotates large portions of unlabeled data but also lays the groundwork for an array of downstream tasks, such as behavior monitoring, anomaly detection, and personalized interventions. 


\section{\ourdata}
\label{sec:textbook-exam}

\begin{figure}[t]
    \centering
    \includegraphics[width=\linewidth]{figures/data_framework.pdf}    \caption{\textbf{Overview of  \ourdata curation}. Given a chapter $C$, we use an LM to segment the document $D$ into sections and heuristically extract review questions to form the exam $E$. The LM then classifies each question in $E$ by Bloom’s taxonomy category and maps it to its relevant section.}
    \label{fig:dataset-overview}
    \vspace{-0.3cm}
\end{figure}

In order to evaluate \ours, we curate  \ourdata, a dataset where each entry contains a document \(D\) along with a corresponding set of exam questions \(E\).
An overview of \ourdata is illustrated in \autoref{fig:dataset-overview}.
% In this section, we first describe our data processing pipeline (\secref{ssec:textbook-exam-pipeline}) and then provide data statistics (\secref{ssec:textbook-exam-statistics}).

\subsection{Data Processing}
\label{ssec:textbook-exam-pipeline}
Our pipeline starts with textbooks from the OpenStax repository\footnote{\url{https://github.com/philschatz/textbooks}}.
Each textbook is divided into chapters, where each chapter \(C\) contains learning objectives, main content, and review questions.
For each \(C\), we parse the main content to build \(D\) and the review questions to form \(E\).
% Specifically, only the main content is used to construct \(D\), while the review questions are used to create \(E\).

\paragraph{Extracting sections.}
To simulate a learner incrementally progressing through a chapter, we divide each chapter into sections using an LM-based document structuring method. 
The LM segments \( D \) into \( n \) sections, denoted as \( \{S_1, S_2, \ldots, S_n\} \subset D \), while also extracting the corresponding review questions \( E \). 
However, not all review questions come with ground-truth answers, as some textbooks do not provide them (see Table~\ref{tab:textbook-exam-statistics} for the proportion of \( E \) with answers). 
To ensure consistency in section segmentation across different subjects, we manually annotate the first 2–5 sections from one sample per subject. 
These annotated samples serve as few-shot examples in our LM prompt (see Appendix~\ref{appdx:parsing-sections} for details).

\paragraph{Extracting questions.}
% We developed a custom parsing script using BeautifulSoup4 to extract questions and their corresponding answers. 
To maintain a balance between evaluation depth and computational feasibility, we include only chapters that contain at least 10 questions—ensuring sufficient coverage for assessment—while capping the maximum number of questions at 25 to keep learning simulations computationally manageable.

\subsection{Data Statistics}
\label{ssec:textbook-exam-statistics}
\begin{table}[t!]
    \centering
    \resizebox{\columnwidth}{!}{
        \begin{tabular}{lccccc}
        \toprule
            \textbf{Subject} & \textbf{\# $C$} & \textbf{Split} & \textbf{\# $E$ / $C$} & \textbf{\% $E$ w/ answer} & \textbf{\# $S$ / $C$} \\
        \midrule
            Microbiology & 20 & Train & 12.4 & 64\% & 16.4 \\
                         & 5  & Test  & 13.4 & 58\% & 17.0 \\
        \midrule
            Chemistry    & 20 & Train & 14.2 & 51\% & 11.0 \\
                         & 5  & Test  & 16.2 & 49\% & 6.4 \\
        \midrule
            Economics    & 20 & Train & 12.2 & 23\% & 14.1 \\
                         & 5  & Test  & 12.2 & 23\% & 14.4 \\
        \midrule
            Sociology    & 20 & Train & 10.4 & 62\% & 16.6 \\
                         & 5  & Test  & 11.2 & 67\% & 19.0 \\
        \midrule
            US History   & 20 & Train & 7.2 & 51\% & 14.9 \\
                         & 5  & Test  & 8.4 & 38\% & 13.2 \\
        \bottomrule
        \end{tabular}
    }
    \caption{\textbf{Data Statistics} of \ourdata. \# $ C$: number of chapters, \# $E/C$: avg. number of questions per chapter, \% $E$ w/ answer: proportion of questions that have reference answer, \# $S/C$: avg. number of sections per chapter.}
    \label{tab:textbook-exam-statistics}
\end{table}
For each subject, we curate 25 sequential chapters \(C\), each containing both \(D\) and \(E\).
The chapters are arranged in their natural order, with the first 20 used for training and the last five reserved for evaluation.  
There is the risk that content in later chapters may include information from prior chapters (e.g., revisiting prerequisite knowledge). 
Therefore, preserving this sequential structure between the training and test set is essential for preventing information leakage and fairly assessing a model's learning process.
Table~\ref{tab:textbook-exam-statistics} shows an overview of the statistics of the resulting \ourdata.

\subsection{Distribution of Question Types}
\label{ssec:textbook-exam-bloom}

 \begin{figure}[t]
    \centering
    \includegraphics[width=\linewidth]{figures/bloom_taxonomy_counts_vertical.png}
    \caption{\textbf{Bloom's taxonomy distribution} in \ourdata. \ourdata consists of questions that require a wide variety of cognitive levels and the dominant categories vary for each subject.}
    \label{fig:bloom-distribution}
\end{figure}

To better understand how final exams assess a learner’s comprehension on multiple dimensions, we categorize questions in \ourdata\ based on the revised \textit{Bloom’s Taxonomy}~\cite{krathwohl2002revision_bloom}. 
Using an LM, we assign a cognitive depth \( d_j \) to each question \( E_j \in E \), classifying them into six categories: \textit{Remembering, Understanding, Applying, Analyzing, Evaluating}, and \textit{Creating}.
Additionally, we identify the relevant sections \( S_j \subset D \) that correspond to each question.

The distribution, shown in \autoref{fig:bloom-distribution}, indicates that different subjects emphasize different cognitive skills.
For instance, questions in Microbiology and Sociology primarily focus on \textit{Remembering} and \textit{Understanding}, whereas Chemistry and Economics exhibit a more varied distribution.
This analysis highlights the diverse cognitive demands across subjects and underscores how \ourdata\ provides a multifaceted evaluation of learning outcomes through final exams. 
For further details on data processing, refer to Appendix~\ref{appendix:data_processing}.













% \dongho{Number of textbooks?}

% \dongho{For each textbook, how may $D$?}



% \subsection{Validation}
% \dongho{Let's make a ground truth to see how reliable the data processing pipeline it is. -- for each textbook.}

% \paragraph{Validation of answer for $E_j$.}

% \paragraph{Validation of cognitive depth $d_j$ for $E_j$.}

% \paragraph{Validation of related sections $S_j$ for $E_j$.}

\section{Experimental Evaluation: Clustering Stage}
\label{sec:results:clustering}

In this section, we evaluate the clustering stage of the \ToolName{} pipeline (Steps 0-2) to assess the quality of the clusters generated by our approach.  Although the clusters have not yet been assigned \ToolName{} labels, examining the clusters helps provide insights into the degree to which they match previous hand-annotated labels from CASAS. To validate cluster performance, we use CASAS as a ground truth oracle to temporarily assign a label to each cluster through majority voting. We then perform a direct comparison with fully supervised techniques, and investigate the impact of varying cluster count, examining the trade-off between granularity, label alignment and annotation costs.

\subsection{Setup}
\label{sec:supervised_setup}
We assess the quality of the clusters generated by our approach by comparing them to CASAS labels.  To map \ToolName{} clusters to CASAS labels, we use majority vote to assign a single label to each cluster based on the CASAS labels of its samples. We then assess these cluster-assigned labels, utilizing CASAS labels as ground truth. 

We use two common variants of the F1 metric to evaluate the performance. Let \(\{1, \ldots, C\}\) be the set of activities, and let \(\mathrm{F1}_c\) denote the F1 score for class \(c\). Given the class imbalances in CASAS, we utilize the following metrics:

\begin{itemize}
    \item \textbf{Weighted F1}: Emphasizes frequent classes by weighting each class’s F1 score by its support:
    \[
    \mathrm{F1}_{\mathrm{weighted}} 
    = \sum_{c=1}^{C} \frac{\lvert c \rvert}{N} \,\mathrm{F1}_c,
    \]
    where \(\lvert c \rvert\) is the number of samples belonging to class \(c\), and \(N\) is the total number of samples.
    \item \textbf{Macro F1}: Treats all classes equally by averaging their F1 scores:
    \[
    \mathrm{F1}_{\mathrm{macro}} 
    = \frac{1}{C} \sum_{c=1}^{C} \mathrm{F1}_c.
    \]
\end{itemize}

Temporarily utilizing CASAS labels for our clusters has one additional benefit in that it allows us to directly compare our self-supervised clustering approach to fully-supervised methods.  As noted in earlier discussion, by comparing our cluster-based, majority-vote labels with the outputs of fully supervised baselines, we can gauge how closely our framework approaches traditional, label-intensive methods. While surpassing these baselines is not our primary goal, a comparable \(\mathrm{F1}\) performance signifies a reasonable alignment between our clustering results and CASAS labels.  

Specifically, we compare against two baselines: DeepCASAS \cite{deepcasas2018} and TDOST \cite{tdost2024}. DeepCASAS is a fully supervised deep learning algorithm that leverages LSTMs to capture temporal dependencies in the data. In contrast, TDOST converts sensor triggers into natural language descriptions, 
which provides the model with rich contextual information and enables the transfer of a single model across multiple households.

Note that both DeepCASAS and TDOST, as originally published, were trained and evaluated on pre-segmented datasets. This allowed the authors to compare the isolated algorithmic performance in a controlled setting, but is a less realistic scenario compared to operating over continuous sensor streams. To achieve a fair comparison to our approach, which does not assume pre-segmentation, we re-train both DeepCASAS and TDOST algorithms on sliding windows $W_i$ of length \( l \) to match our sequences. Additionally, given the prevalence of the \textit{Other} label in our datasets (see \cref{sec:datasets}), we conduct experiments both with and without the \textit{Other} label. Specifically, we re-train DeepCASAS and TDOST with and without the \textit{Other} label, which allows us to evaluate how well these methods work with only \textit{relevant} activities.


\subsection{Results}
\label{sec:supervised_results}

\begin{table}[t]
    \centering
    % First part: F1 Weighted
    \textbf{F1 Weighted}
    \vspace{0.5em}
    \begin{adjustbox}{width=0.8\textwidth,center}
        \small
        \begin{tabular}{l | cc | cc | cc}
            \toprule
            & \multicolumn{2}{c|}{Milan} & \multicolumn{2}{c|}{Aruba} & \multicolumn{2}{c}{Cairo} \\
            Model & All Labels & Excl. \textit{Other} & All Labels & Excl. \textit{Other} & All Labels & Excl. \textit{Other} \\
            \midrule
            DeepCASAS & 0.61 & 0.79 & 0.78 & \cellcolor[HTML]{caebc0}0.93 & 0.72 & 0.82 \\
            TDOST     & \cellcolor[HTML]{caebc0}0.62 & \cellcolor[HTML]{caebc0}0.83 & \cellcolor[HTML]{caebc0}0.80 & 0.92 & \cellcolor[HTML]{caebc0}0.74 & 0.82 \\
            \hline
            \hline
            \ToolName & 0.53 & 0.82 & 0.75 & 0.90 & 0.64 & \cellcolor[HTML]{caebc0}0.85 \\
            \bottomrule
        \end{tabular}
    \end{adjustbox}
    
    \vspace{2em} % Add vertical space between the two parts
    \textbf{F1 Macro}
    
    % Second part: F1 Macro
    \vspace{0.5em}
    \begin{adjustbox}{width=0.8\textwidth,center}
        \small
        \begin{tabular}{l | cc | cc | cc}
            \toprule
            & \multicolumn{2}{c|}{Milan} & \multicolumn{2}{c|}{Aruba} & \multicolumn{2}{c}{Cairo} \\
            Model & All Labels & Excl. \textit{Other} & All Labels & Excl. \textit{Other} & All Labels & Excl. \textit{Other} \\
            \midrule
            DeepCASAS & 0.38 & 0.53 & 0.48 & 0.69 & 0.16 & \cellcolor[HTML]{caebc0}0.46 \\
            TDOST     & \cellcolor[HTML]{caebc0}0.40 & \cellcolor[HTML]{caebc0}0.63 & \cellcolor[HTML]{caebc0}0.54 & \cellcolor[HTML]{caebc0}0.69 & \cellcolor[HTML]{caebc0}0.17 & 0.41 \\
            \hline
            \hline            
            \ToolName & 0.32 & 0.50 & 0.24 & 0.49 & 0.12 & 0.39 \\
            \bottomrule
        \end{tabular}
    \end{adjustbox}
    \vspace{0.5em}
    \caption{
        Weighted and macro F1 scores for DeepCASAS, TDOST, and \ToolName{} trained and tested on the Milan, Aruba, and Cairo datasets using CASAS labels as ground truth. The best result for each configuration is highlighted in green. While fully supervised methods yield the highest overall performance, the gap in \ToolName{} performance is moderate—with comparable F1 scores achieved when data associated with the \textit{Other} label is excluded, signifying a reasonable alignment between \ToolName{} clusters and CASAS labels.
    }
    \vspace*{-1em}
    \label{tab:clf_comparison}
\end{table}


In this section we present how well our clusters align with the CASAS labels, and also experiment with the optimal number of clusters needed to achieve useful performance.

\subsubsection{\ToolName{} Cluster Alignment with CASAS Labels}   

\cref{tab:clf_comparison} reports both weighted and macro F1 scores for models trained with and without the \textit{Other} label. The bottom row shows the performance of the \ToolName{} clustering stage compared. Across all three datasets, we observe that the \(\mathrm{F1}_{\mathrm{macro}}\) metric is considerably lower than \(\mathrm{F1}_{\mathrm{weighted}}\). This indicates that less frequent activities, such as \textit{Take\_medicine} and \textit{Leave\_home} are challenging to discover, as has been reported in previous studies \cite{hiremath2022bootstrapping}.  
Additionally, we observe that performance improves by approx.\ 22 points when the \textit{Other} label is excluded, indicating that the availability of \textit{Other} data points during clustering leads to clusters that are less closely aligned with CASAS labeling.  
As we will demonstrate in \cref{sec:tsne}, this effect is due to the non-homogeneous nature of the data captured by the catch-all \textit{Other} label. 


\subsubsection{Comparison to Supervised Baselines}

The top rows of \cref{tab:clf_comparison} report the performance of the DeepCASAS and TDOST baselines\footnote{As noted in \cref{sec:supervised_setup}, both baselines were re-trained and tested on sliding window sequences to remove the pre-segmentation assumption. This change leads to significant drops in performance compared to originally published results. For example, for DeepCASAS, the \(\mathrm{F1}_{\mathrm{weighted}}\) scores dropped by 29 points for Milan from 0.89 to 0.61; by 21 points for Aruba from 0.97 to 0.78; by 13 points for Cairo from 0.85 to 0.72. This highlights the significance of the pre-segmentation assumption.}.  
As with \ToolName{}, we observe that that \(\mathrm{F1}_{\mathrm{macro}}\) is substantially lower than \(\mathrm{F1}_{\mathrm{weighted}}\), and that performance improves with the exclusion of data associated with the \textit{Other} label.  

More generally, we observe that while DeepCASAS and TDOST yield better performance overall, the gap relative to \ToolName{} is moderate, with comparable performance on $\mathrm{F1}_{\mathrm{weighted}}$ when \textit{Other} is excluded.  
These results highlight the misalignment of self-supervised clusters with the \textit{Other} class, suggesting that \textbf{data labeled as \textit{Other} represents many different behaviors rather than a single cohesive activity}.  We explore this finding in greater detail in the following subsection.


\subsubsection{tSNE Analysis}
\label{sec:tsne}

\begin{figure}[t]
    \centering
        \includegraphics[width=0.95\linewidth]{figures/cluster_tsne_with_house_maps.pdf}
    \caption{
        tSNE projection of SCAN embeddings from the Milan household; each point represents a sensor window embedding colored by its original CASAS label. Insets display the deployment environment layout overlaid with a heatmap of sensor activations. Insets (a) and (b) highlight two distinct clusters within the CASAS \textit{Cook} label—cluster 16 showing movement between kitchen and dining areas, and cluster 5 capturing activity near the medicine cabinet. Insets (c) and (d) show clusters from the \textit{Relax} label, corresponding to sitting in the TV room armchair and sitting in the living room armchair. Data associated with the \textit{Other} label (gray) is dispersed across clusters, underscoring its heterogeneous nature. This figure showcases \ToolName{}'s capability to uncover more granular and nuanced activity categories than the original CASAS labels. 
    } 
    \vspace*{-1em}
    \label{fig:cluster_tsne_with_house_maps}
\end{figure}

In this section, we perform a visual inspection of Milan clusters in order to gain more insights into the generated clusters. 
\cref{fig:cluster_tsne_with_house_maps} shows a tSNE projection of the SCAN embeddings from the Milan dataset; each point represents a 768-dimensional sequence embedding colored by its CASAS label. While some activities (e.g., \emph{Work}) form single clusters, data representing other activity labels (e.g., \emph{Cook}, \emph{Relax}) fall into multiple clusters. 
Each point in the scatterplot is annotated with a probability score reflecting SCAN’s confidence in assigning that sequence to its respective cluster, offering interpretability and a level of control that simpler clustering methods (e.g., k-means) lack.

Further, we take the two largest Milan activity labels, \emph{Cook} and \emph{Relax}, and inspect the data within individual clusters, as shown in the figure insets. 
Each inset includes the layout of the deployment environment overlaid with a heatmap representing sensor activation patterns for the data sequences associated with that cluster.  
\cref{fig:cluster_tsne_with_house_maps} illustrates that clustering using \ToolName{} facilitates the discovery and differentiation of granular sub-activities.  
Insets (a) and (b) are associated with the \textit{Cook} CASAS label (orange).  
Although both sets of data are labeled as \textit{Cook} within CASAS, we observe clear differences between the sensor patterns within each cluster.  
Cluster 16 (inset (a)) captures the person leaving the kitchen and moving to the dining room, with motion sensors from the dining room and part of the dining room registering the person’s presence.  
By comparison, cluster 5 (inset (b)) is tightly focused on activity around the medicine cabinet in the kitchen. 
Insets (c) and (d) are associated with the \textit{Relax} CASAS label (burgundy), where cluster 3 (inset (c)) captures a static activity in the TV room armchair while cluster 17 (inset (d)) corresponds to sitting in a living room armchair instead.  

Note that data associated with the \textit{Other} label (gray) is scattered throughout multiple clusters in \cref{fig:cluster_tsne_with_house_maps}, indicating that it does not represent a single sub-activity. Instead, \textit{Other} activities appear to correspond to multiple sub-activities.  

These findings can be interpreted as follows. First, excluding the \textit{Other} label significantly improves performance, particularly when there is no pre-segmentation and with shorter sequence lengths, as the \textit{Other} label tends to overlap with all other categories. Second, the large gap between \(\mathrm{F1}_{\mathrm{weighted}}\) and \(\mathrm{F1}_{\mathrm{macro}}\) reflects the challenges of accurately recognizing low-frequency classes, a point often overlooked when results are reported for the fully supervised methods.
Finally, our method's performance indicates that the clusters align reasonably well with the CASAS labels, showcasing the capabilities of \ToolName{}.
% demonstrating a decent alignment between our clusters and CASAS labels.

\subsubsection{Impact of the Number of Clusters}
\label{sec:varying_clusters_results}

\begin{figure}
    \centering
        \includegraphics[width=0.9\linewidth]{figures/varying_clusters_f1.pdf}
    \caption{
        Macro F1 scores for varying numbers of SCAN clusters (\(k \)) on Milan, Aruba, and Cairo datasets, using all CASAS labels in (a), and without the \textit{Other} label in (b). We chart shows average F1 scores with bootstrapped 95\% confidence intervals . Increasing \(k\) up until 20-40 clusters improves alignment with CASAS labels before the performance improvement stagnates, suggesting that the optimal number of clusters lies in that range.
    } 
    \vspace*{-1em}
    \label{fig:varying_clusters_f1}
\end{figure}

\noindent
We also investigate how the choice of \(k\)—the number of clusters—impacts our model's ability to capture and distinguish fine-grained activities. 
Similar to previous section, we use CASAS labels and apply majority voting per cluster to compute Macro F1 scores. 

\cref{fig:varying_clusters_f1} shows the Macro F1 scores with bootstrapped 95\% confidence intervals (CIs) for Milan, Aruba, and Cairo, when  \( k \in \{10, 15, 20, 30, 40, 50, 60, 100\} \) clusters. 
The left-hand panels in (a) show performance when all CASAS labels are considered; the right-hand panels in (b) exclude \emph{Other}. 
These F1 scores indicate that utilizing more clusters generally leads to better classification performance in (a) for all three datasets, although the improvements appear to plateau after \( k = 30\) for Milan, and \( k = 20 \) for Aruba and Cairo. 
When we exclude the \textit{Other} label, the macro F1 scores for both Milan and Aruba plateau when \( k > 30\) and further increases are not statistically significant; 
Cairo stays relatively flat after \( k = 20 \), and the improvements for \( k >= 50 \) have very wide CIs. 
This leads to a conclusion that increasing the number of clusters beyond 20–30 is not particularly useful.
Refer to \cref{fig:varying_clusters_tsne} in the Appendix to see what these various clusters look like on 2D tSNE projections.

, as the F1 scores plateau. 
In addition, we also visualizing the effect of the number of clusters through a tSNE plot in \cref{fig:varying_clusters_tsne} in the Appendix. 

An optimal \( k \) balances classification performance with the workload of labeling cluster centroids. 
We selected \( k=20 \) because it provides a more granular view than the typical CASAS labels while keeping annotation efforts manageable. 

\section{Experimental Evaluation: Labeling Stage}
\label{sec:label_quality}

For our second set of experiments, we move beyond the CASAS annotations and evaluate the labeling stage of our pipeline (step 2, 3, and 4 in \ref{fig:main_image}). 
This resembles the envisioned use case for our approach where we employ our active learning-like scheme, driven by our self-supervised clustering stage, to guide human annotator to provide reliable labels thereby minimizing manual efforts.

By comparing our human-labeled clusters to CASAS labels, we highlight how our approach provides a more granular and detailed categorization of (sub-)activities, offering finer resolution and capturing nuances that the original labels may miss. 
We also evaluate the quality and consistency of these labels by examining two key metrics: inter-rater agreement, which measures consistency between different labelers, and cluster agreement, which assesses the precision of the labels assigned to each cluster.

\subsection{Setup}
Our objective is to present human labelers with visualizations of a small number of representative data sequences from each cluster, and for labelers to select the appropriate label for each sample.  Below, we detail the labels selected for our work and evaluation metrics.

\subsubsection{Label Set}
The \ToolName{} labeling scheme is organized hierarchically to accommodate different levels of detail. At the highest level, we organize all activities into categories such as \emph{single-room} and \emph{multi-room} events, with subcategories specific to each environment (e.g., \emph{kitchen} vs.\ \emph{bedroom}). By allowing annotators to choose increasingly granular labels, we preserve the flexibility needed for diverse research objectives.

To capture a finer level of granularity than conventional HAR labels such as  \textit{Cook}, \textit{Relax}, or \textit{Work}, we developed a set of \emph{sub-activity} labels by studying the physical layouts of the CASAS homes, examining the replays of actual sensor activations throughout the day, and drawing insights from prior work on structural constructs in activities \cite{shruthi_gameofllms}. We selected a large number of potential labels in order to avoid over-constraining our annotators, leveraging information about furniture placement when it was available. Our labels capture both activities within a single room (e.g., \textit{movement all over kitchen}, \textit{movement near fridge}, \textit{movement near stove}, and \textit{movement near medicine cabinet}) and movement actions between the rooms (e.g., \textit{walking from bedroom to office}, \textit{leaving guest bathroom}). See the full list of sub-activities and their hierarchies in \cref{fig:atomic_activity_tree} in the Appendix.

Note, we have chosen a dataset-specific approach in our label selection, with several labels tailored specifically to the CASAS dataset (e.g., \textit{medicine cabinet}), in order to demonstrate the degree to which \ToolName{} supports customization to a given environment or use case. The granularity of the labels is fully adjustable; depending on the specific application, a user can, for example, choose whether they care to split \textit{armchair sitting} and \textit{couch sitting} into distinct classes, or if they want to group them into a broader \textit{Relax} category.

\subsubsection{Labeling Sample Selection and Metrics}

To obtain cluster labels, we recruited 15 independent raters to each individually label data segments.  For each cluster \( c_k\), we choose \( m\) sequence samples that are closest to the cluster centroid for labeling, with \( m = 5\) in this work.  With $k=20$ clusters per model, this resulted in a total of $300$ samples across all three datasets. Samples were randomized and each sample was labeled by two raters, resulting in $600$ total sample-label pairs. To generate a label, each rater was presented with a selection of $40$ samples, for which they assigned labels using the web-based \ToolName{} annotation tool.  \cref{fig:viz_tool} shows a screenshot of a sample annotation replay for the Milan household. During the replay, the rater can view a temporal playback of the sensor activations captured by the data sequence, and use a drop-down menu to select a label. Note that the \ToolName{} tool is fully customizable and is available in open source at \url{https://anonymized}.

We use two metrics to evaluate the labeling stage.  First, we evaluate \textbf{inter-rater agreement} using Cohen’s Kappa \cite{kappas}.  Higher values represent stronger agreement, with scores over 0.8 representing strong alignment. Observing a high inter-rater agreement on the labeling task would indicate that the data sequences represent consistently interpretable human behavior sequences. Second, we evaluate \textbf{cluster agreement} to examine the uniformity of the \( m\) labels obtained for each cluster.  We use Fleiss’s Kappa \cite{kappas}, which generalizes the idea of Cohen’s Kappa to more than two ratings, to capture the consensus among all 10 labels ($m=5$ by a total of $2$ raters per sample) assigned within a single cluster. A high Fleiss’s Kappa indicates that our raters not only agree between themselves, but also that data sequences captured by the cluster represent a single activity rather than disparate data sequences (i.e., the cluster is internally homogeneous).


Since our sub-activity labels follow a hierarchical, multi-level taxonomy (see \cref{fig:atomic_activity_tree}), disagreements may sometimes stem from different levels of specificity (e.g., \emph{movement near fridge} vs.\ \emph{movement in kitchen}). To account for this, we also compute Kappa scores at a coarser level by ``leveling up'' each annotated label to its parent category. For instance, if one rater selects \emph{movement near fridge} and the other chooses \emph{movement in kitchen}, these would be treated as the same label at the higher level. Comparing Kappa scores at different levels of the hierarchy helps distinguish minor discrepancies in labeling granularity from dramatically different activity interpretations (e.g., \emph{bathroom activity} vs.\ \emph{reading in armchair}). 

\subsection{Results}
\label{sec:label_quality_results}

\begin{table}[ht]
    \centering
    \begin{tabular}{l  cc  cc}
        % \toprule
         & \multicolumn{2}{c}{Inter-Rater \( \kappa \) } & \multicolumn{2}{c}{Cluster Agreement \( \kappa \) } \\
        % \cmidrule(lr){2-3} \cmidrule(lr){4-5}
        Dataset & Original Label & Level-Up Label & Original Label & Level-Up Label \\
        \toprule
        Milan  & 0.850 & 0.884 & 0.600 & 0.775 \\
        Aruba  & 0.890 & 0.918 & 0.627 & 0.757 \\
        Cairo  & 0.872 & 0.916 & 0.630 & 0.810 \\
        \bottomrule
    \end{tabular}
    \vspace{0.5em}
    \caption{
        Sub-Activity label quality: inter-rater (Cohen’s) and cluster-agreement (Fleiss’s) Kappa scores for each of the three CASAS datasets (Milan, Aruba, and Cairo). ``Original Label'' indicates the use of the most specific labels chosen by each rater (e.g., \emph{movement near fridge}), while ``Level-Up Label'' merges them into coarser categories (e.g., \emph{movement in kitchen}). 
        Overall, very high inter-rater agreement indicates very high alignment between annotators; decent cluster agreement also implies relative homogeneity among the ratings.
        The resulting increase in Kappa scores from ``Original Label'' to ``Level-Up Label'' highlights that most disagreements stem from label specificity rather than fundamentally different interpretations of the underlying activity.        
    }
    \vspace*{-2em}
    \label{tab:kappa_scores}
\end{table}


In this section, we first analyze the inter-rater and cluster-agreement scores, then discuss how the obtained labels different from CASAS labels.
\cref{tab:kappa_scores} summarizes the inter-rater and cluster-agreement scores for Milan, Aruba, and Cairo. We also refer to \cref{fig:cluster_majority_label_summary} in the Appendix for a detailed table showing the majority label assigned to each cluster and its corresponding vote count.

Inter-rater agreement, measured as Cohen's Kappa, exceeds 0.85 for all three datasets (Milan \(\kappa=0.85\), Aruba \(\kappa=0.89\), Cairo \(\kappa=0.87\)). These values indicate very high consistency between the two independent annotators in how they interpreted and labeled the same replay samples. Furthermore, when we ``level up'' each label to its parent category in our hierarchical taxonomy (e.g., merging \emph{movement near bed} into \emph{movement in bedroom}), the Kappa scores rise even further—reaching 0.88 in Milan and over 0.90 in Aruba and Cairo. This gain reflects the fact that any disagreements that did arise between raters, were largely in relation to differences in label specificity, such as \emph{sitting in armchair} vs.\ \emph{movement in living room}, rather than fundamentally diverging perceptions of the activity itself.

Cluster agreement, measured as Fleiss' Kappa, results in moderately high scores ranging from 0.60--0.63 when annotators used highly specific sub-activity labels. Once labels are rolled up to a coarser levels, these scores climb to 0.75--0.81 across the three datasets. Here again, the improvement underscores that most labeling discrepancies within a cluster stem from variation in granularity rather than actual activity disagreements. For instance, in one cluster designated \emph{kitchen activity}, half of the annotators specified \emph{movement near medicine cabinet}, while the other half used \emph{movement in kitchen}. In another cluster, some annotators perceived simultaneous sensor firings in two rooms as \emph{multi-room activity}, while others focused on whichever room had the most events.

The high rater and cluster agreements confirm that clusters produced by the \ToolName{} pipeline are both interpretable and relatively homogeneous. Moreover, the gains observed when we unify labels at a higher level indicate that our hierarchical taxonomy successfully accounts for the inherent variation in labeling granularity. This can allow researchers to adopt the level of detail most appropriate for their analysis. 

\begin{table}[ht]
  \centering
  \begin{adjustbox}{width=\textwidth,center}
    \small
    \begin{tabular}{ll|cc|cc|cc}
      \toprule
      \multicolumn{2}{c|}{} & \multicolumn{2}{c|}{Milan} & \multicolumn{2}{c|}{Aruba} & \multicolumn{2}{c}{Cairo} \\
      CASAS Label & \ToolName{} Label & CASAS & \ToolName & CASAS & \ToolName & CASAS & \ToolName \\
      \midrule
      %---------------- Cook ----------------
      \multirow{5}{*}{Cook} 
          & TOTAL                              & 31\%  & 19\%  & 39\%  & 24\%  & 0\%   & 0\%   \\ 
          \cdashline{2-8}[0.5pt/2pt]
          & Movement all over kitchen          &       & 4\%   &       & 24\%  &       &       \\[0.8em]
          & Movement near Fridge               &       &       & --    & --    & --    &  --   \\[0.8em]
          & Movement near Stove                &       & 2\%   & --    & --    & --    &  --   \\[0.8em]
          & Movement near Medicine Cabinet     &       & 13\%  & --    & --    & --    &  --   \\
      \midrule
      %---------------- Eat ----------------
      \multirow{4}{*}{Eat} 
          & TOTAL                              & 2\%   & 6\%   & 4\%   & 6\%   & 72\%  & 67\%  \\ \cdashline{2-8}[0.5pt/2pt]
          & Movement all over Dining Room      &       &       &       &       &       &       \\[0.8em]
          & Movement through Kitchen + Dining  &       & 6\%   &       & 6\%   &       & 32\%  \\[0.8em]
          & Movement through Living + Dining   &       &       &       &       &       & 35\%  \\
      \midrule
      %---------------- Relax ----------------
      \multirow{6}{*}{Relax} 
          & TOTAL                              & 41\%  & 36\%  & 49\%  & 37\%  & 0\%   & 7\%   \\ \cdashline{2-8}[0.5pt/2pt]
          & Motion all over Living room        &       &       &       &  0\%  &       &       \\[0.8em]
          & Motion in TV chair and area        &  --   & --    &       &  6\%  & --    & --    \\[0.8em]
          & Sitting on couch/armchair          &       & 19\%  &       & 29\%  &       & 7\%   \\[0.8em]
          & Sitting in office armchair         &       & 15\%  &       &       & --    & --    \\[0.8em]
          & Movement through Kitchen + Living  &       & 2\%   &       & 2\%   &       &       \\
      \midrule
      %---------------- Work ----------------
      \multirow{2}{*}{Work} 
          & TOTAL                              & 3\%   & 4\%   & 4\%   & 7\%   & 12\%  & 0\%   \\ \cdashline{2-8}[0.5pt/2pt]
          & Movement near office computer/desk &       & 4\%   &       & 7\%   & --    & --    \\
      \midrule
      %---------------- Sleep ----------------
      \multirow{3}{*}{Sleep} 
          & TOTAL                              & 7\%   & 8\%   & 3\%   & 3\%   & 12\%  & 7\%   \\ \cdashline{2-8}[0.5pt/2pt]
          & Motion in all over Master Bedroom  &       & 8\%   &       & 0\%   &       &       \\[0.8em]
          & In Master Bedroom, Movement near bed &       &       &      & 3\%   &       & 7\%   \\
      \midrule
      %---------------- Bathing ----------------
      \multirow{7}{*}{Bathing} 
          & TOTAL                              & 14\%  & 19\%  & 0\%   & 0\%   & 0\%   & 0\%   \\ \cdashline{2-8}[0.5pt/2pt]
          & Entering the Guest Bathroom        &       & 13\%  & --    & --    & --    & --    \\[0.8em]
          & Leaving the Guest Bathroom         &       &       & --    & --    & --    & --    \\[0.8em]
          & Movement inside the Guest Bathroom   &       &       & --    & --    & --    & --    \\[0.8em]
          & Entering Master Bathroom           &       & 1\%   & --    & --    & --    & --    \\[0.8em]
          & Leaving Master Bathroom            &       &       & --    & --    & --    & --    \\[0.8em]
          & Movement in Master bathroom        &       & 5\%   & --    & --    & --    & --    \\
      \midrule
      %---------------- Leave_Home ----------------
      \multirow{1}{*}{Leave\_Home} 
          & TOTAL                              & 2\%   & 6\%   & 0.3\% & 6\%   & 1\%   & 5\%   \\[0.8em]
      \midrule
      %---------------- Other ----------------
      \multirow{1}{*}{Motion Through House} 
          & TOTAL                              & 0\%   & 4\%   & 0\%   & 14\%  & 3\%   & 13\%  \\
      \bottomrule
    \end{tabular}
  \end{adjustbox}
  \caption{
    A side-by-side comparison of original CASAS labels and the more granular sub-activities identified by \ToolName{} in Milan, Aruba, and Cairo.
    CASAS Labels are mapped to \ToolName{} labels. Dashes indicate that this activity cannot be identified in that household (e.g., Aruba and Cairo have no bathroom sensors). These results showcase how \ToolName{} is able to discover much more granular sub-activities for all three CASAS datasets.
  }  
  \vspace*{-1em}  
  \label{tab:subactivity_mapping}
\end{table}


We now want to explore these manually annotated sub-activity labels in greater detail. \cref{tab:subactivity_mapping} illustrates how each discovered sub-activity label maps onto the broader CASAS categories (e.g., \emph{Cook}, \emph{Relax}, \emph{Sleep}), along with the percentage of time-window sequences \(W_i\) that each label occupies in Milan, Aruba, and Cairo. Dashes indicate activities that cannot occur in a given household (e.g., no guest bathroom sensors in Aruba or Cairo).

Notably, the total proportion of each broad CASAS label usually aligns reasonably well with our sub-activity categories. In Milan, for example, \emph{Sleep} comprises around 7\% of the CASAS data and 8\% of our sub-activity labels (\emph{movement in bedroom}, \emph{movement near bed}). Similarly, CASAS marks 41\% of the Milan data as \emph{Relax}, compared to 36\% for our corresponding sub-activities. More importantly, our approach results in a much finer breakdown of these broad labels, such as distinguishing \emph{sitting in living room armchair} vs.\ \emph{sitting in office armchair}, or \emph{movement near the medicine cabinet} vs.\ \emph{movement near the stove}. In Aruba, a single \emph{Relax} label can be further divided into \textit{sitting on the couch} vs.\ \textit{motion in a TV chair}. These finer distinctions highlight one of the key advantages of \ToolName{}: while it can reasonably well replicate higher-level annotations from existing datasets, it can also uncover more granular activities that may carry significant behavioral or clinical relevance.

This paper presents a planning approach for effective and efficient joint motion generation for manipulators to cover a surface, aiming to minimize specific joint space costs.

\textit{Limitations} -- Our work has several limitations that suggest potential directions for future research. First, our method uses a heuristic to accelerate the traditional Joint-GTSP approach. While we provide empirical evidence of its efficiency in producing high-quality solutions, we cannot guarantee consistent performance in all scenarios.
Second, our bi-level hierarchical method reduces the size of GTSP. Future research could extend it to multiple levels to further improve performance, though this may produce misleading guide paths.
Third, we observe that both Joint-GTSP and H-Joint-GTSP tend to generate paths with frequent turns, a pattern also observed in the motions of prior work \cite{kaljaca2020coverage, zhang2024jpmdp}.  Future work should explore strategies to balance joint movements with other objectives such as motion smoothness.

\footnotetext{Visualization tool: \url{https://github.com/uwgraphics/MotionComparator}}
\textit{Implications} -- The hierarchical approach presented in this work enables effective and efficient coverage path planning for robot manipulators. 
This approach is beneficial to applications that require dexterous surface coverage, such as sanding, polishing, wiping, and sensor scanning. 



\section{Conclusion \& Future Work}\label{conclusion}
This work presents XAMBA, the first framework optimizing SSMs on COTS NPUs, removing the need for specialized accelerators. XAMBA mitigates key bottlenecks in SSMs like CumSum, ReduceSum, and activations using ActiBA, CumBA, and ReduBA, transforming sequential operations into parallel computations. These optimizations improve latency, throughput (Tokens/s), and memory efficiency. Future work will extend XAMBA to other models, explore compression, and develop dynamic optimizations for broader hardware platforms.



% This work introduces XAMBA, the first framework to optimize SSMs on COTS NPUs, eliminating the need for specialized hardware accelerators. XAMBA addresses key bottlenecks in SSM execution, including CumSum, ReduceSum, and activation functions, through techniques like ActiBA, CumBA, and ReduBA, which restructure sequential operations into parallel matrix computations. These optimizations reduce latency, enhance throughput, and improve memory efficiency. 
% Experimental results show up to 2.6$\times$ performance improvement on Intel\textregistered\ Core\texttrademark\ Ultra Series 2 AI PC. 
% Future work will extend XAMBA to other models, incorporate compression techniques, and explore dynamic optimization strategies for broader hardware platforms.


% This work presents XAMBA, an optimization framework that enhances the performance of SSMs on NPUs. Unlike transformers, SSMs rely on structured state transitions and implicit recurrence, which introduce sequential dependencies that challenge efficient hardware execution. XAMBA addresses these inefficiencies by introducing CumBA, ReduBA, and ActiBA, which optimize cumulative summation, ReduceSum, and activation functions, respectively, significantly reducing latency and improving throughput. By restructuring sequential computations into parallelizable matrix operations and leveraging specialized hardware acceleration, XAMBA enables efficient execution of SSMs on NPUs. Future work will extend XAMBA to other state-space models, integrate advanced compression techniques like pruning and quantization, and explore dynamic optimization strategies to further enhance performance across various hardware platforms and frameworks.
% This work presents XAMBA, an optimization framework that enhances the performance of SSMs on NPUs. Key techniques, including CumBA, ReduBA, and ActiBA, achieve significant latency reductions by optimizing operations like cumulative summation, ReduceSum, and activation functions. Future work will focus on extending XAMBA to other state-space models, integrating advanced compression techniques, and exploring dynamic optimization strategies to further improve performance across various hardware platforms and frameworks.

% This work introduces XAMBA, an optimization framework for improving the performance of Mamba-2 and Mamba models on NPUs. XAMBA includes three key techniques: CumBA, ReduBA, and ActiBA. CumBA reduces latency by transforming cumulative summation operations into matrix multiplication using precomputed masks. ReduBA optimizes the ReduceSum operation through matrix-vector multiplication, reducing execution time. ActiBA accelerates activation functions like Swish and Softplus by mapping them to specialized hardware during the DPU’s drain phase, avoiding sequential execution bottlenecks. Additionally, XAMBA enhances memory efficiency by reducing SRAM access, increasing data reuse, and utilizing Zero Value Compression (ZVC) for masks. The framework provides significant latency reductions, with CumBA, ReduBA, and ActiBA achieving up to 1.8X, 1.1X, and 2.6X reductions, respectively, compared to the baseline.
% Future work includes extending XAMBA to other state-space models (SSMs) and exploring further hardware optimizations for emerging NPUs. Additionally, integrating advanced compression techniques like pruning and quantization, and developing adaptive strategies for dynamic optimization, could enhance performance. Expanding XAMBA's compatibility with other frameworks and deployment environments will ensure broader adoption across various hardware platforms.

\bibliographystyle{ACM-Reference-Format}
\bibliography{ref}

\newpage


\newpage
\appendix
\section{Applicability of SparseTransX for dense graphs} 
\label{A:density}
Even for fully dense graphs, our KGE computations remain highly sparse. This is because our SpMM leverages the incidence matrix for triplets, rather than the graph's adjacency matrix. In the paper, the sparse matrix $A \in \{-1,0,1\}^{M \times (N+R)}$ represents the triplets, where $N$ is the number of entities, $R$ is the number of relations, and $M$ is the number of triplets. This representation remains extremely sparse, as each row contains exactly three non-zero values (or two in the case of the "ht" representation). Hence, the sparsity of this formulation is independent of the graph's structure, ensuring computational efficiency even for dense graphs.

\section{Computational Complexity}
\label{A:complexity}
 For a sparse matrix $A$ with $m \times k$ having $nnz(A)=$ number of non zeros and dense matrix $X$ with $k \times n$ dimension, the computational complexity of the SpMM is $O(nnz(A) \cdot n)$ since there are a total of $nnz(A)$ number of dot products each involving $n$ components. Since our sparse matrix contains exactly three non-zeros in each row, $nnz(A) = 3m$. Therefore, the complexity of SpMM is $O(3m \cdot n)$ or $O(m \cdot n)$, meaning the complexity increases when triplet counts or embedding dimension is increased. Memory access pattern will change when the number of entities is increased and it will affect the runtime, but the algorithmic complexity will not be affected by the number of entities/relations.

\section{Applicability to Non-translational Models}
\label{A:non_trans}
Our paper focused on translational models using sparse operations, but the concept extends broadly to various other knowledge graph embedding (KGE) methods. Neural network-based models, which are inherently matrix-multiplication-based, can be seamlessly integrated into this framework. Additionally, models such as DistMult, ComplEx, and RotatE can be implemented with simple modifications to the SpMM operations. Implementing these KGE models requires modifying the addition and multiplication operators in SpMM, effectively changing the semiring that governs the multiplication.   

In the paper, the sparse matrix $A \in \{-1,0,1\}^{M \times (N+R)}$ represents the triplets, and the dense matrix $E \in \mathbb{R}^{(N+R) \times d}$ represents the embedding matrix, where $N$ is the number of entities, $R$ is the number of relations, and $M$ is the number of triplets. TransE’s score function, defined as $h + r - t$, is computed by multiplying $A$ and $E$ using an SpMM followed by the L2 norm. This operation can be generalized using a semiring-based SpMM model: $Z_{ij} = \bigoplus_{k=1}^{n} (A_{ik} \otimes E_{kj})$

Here, $\oplus$ represents the semiring addition operator, and $\otimes$ represents the semiring multiplication operator. For TransE, these operators correspond to standard arithmetic addition and multiplication, respectively.

\subsection*{DistMult} 
DistMult’s score function has the expression $h \odot r \odot t$. To adapt SpMM for this model, two key adjustments are required: The sparse matrix $A$ stores $+1$ at the positions corresponding to $h_{\text{idx}}$, $t_{\text{idx}}$, and $r_{\text{idx}}$. Both the semiring addition and multiplication operators are set to arithmetic multiplication. These changes enable the use of SpMM for the DistMult score function.

\subsection*{ComplEx} 
ComplEx’s score function has $h \odot r \odot \bar{t}$, where embeddings are stored as complex numbers (e.g., using PyTorch). In this case, the semiring operations are similar to DistMult, but with complex number multiplication replacing real number multiplication.

\subsection*{RotatE} 
RotatE’s score function has $h \odot r - t$. For this model, the semiring requires both arithmetic multiplication and subtraction for $\oplus$. With minor modifications to our SpMM implementation, the semiring addition operator can be adapted to compute $h \odot r - t$.

\subsection*{Support from other libraries}
Many existing libraries, such as GraphBLAS (Kimmerer, Raye, et al., 2024), Ginkgo (Anzt, Hartwig, et al., 2022), and Gunrock (Wang, Yangzihao, et al., 2017), already support custom semirings in SpMM. We can leverage C++ templates to extend support for KGE models with minimal effort.


\begin{figure*}[t]
\centering     %%% not \center
\includegraphics[width=\textwidth]{figures/all-eval.pdf}
\caption{Loss curve for sparse and non-sparse approach. Sparse approach eventually reaches the same loss value with similar Hits@10 test accuracy.}
\label{fig:loss_curve}
\end{figure*}

\section{Model Performance Evaluation and Convergence}
\label{A:eval}
SpTransX follows a slightly different loss curve (see Figure \ref{fig:loss_curve}) and eventually converges with the same loss as other non-sparse implementations such as TorchKGE. We test SpTransX with the WN18 dataset having embedding size 512 (128 for TransR and TransH due to memory limitation) and run 200-1000 epochs. We compute average Hits@10 of 9 runs with different initial seeds and a learning rate scheduler. The results are shown below. We find that Hits@10 is generally comparable to or better than the Hits@10 achieved by TorchKGE.

\begin{table}[h]
\centering
\caption{Average of 9 Hits@10 Accuracy for WN18 dataset}
\begin{tabular}{|c|c|c|}
\hline
\textbf{Model} & \textbf{TorchKGE} & \textbf{SpTransX} \\ \hline
TransE         & 0.79 ± 0.001700   & 0.79 ± 0.002667   \\ \hline
TransR         & 0.29 ± 0.005735   & 0.33 ± 0.006154   \\ \hline
TransH         & 0.76 ± 0.012285   & 0.79 ± 0.001832   \\ \hline
TorusE         & 0.73 ± 0.003258   & 0.73 ± 0.002780   \\ \hline
\end{tabular}
\label{table:perf_eval}
\end{table}

% We also plot the loss curve for different models in Figure \ref{fig:loss_curve}. We observe that the sparse approach follows a similar loss curve and eventually converges to the same final loss.

\section{Distributed SpTransX and Its Applicability to Large KGs}
\label{A:dist}
SpTransX framework includes several features to support distributed KGE training across multi-CPU, multi-GPU, and multi-node setups. Additionally, it incorporates modules for model and dataset streaming to handle massive datasets efficiently. 

Distributed SpTransX relies on PyTorch Distributed Data Parallel (DDP) and Fully Sharded Data Parallel (FSDP) support to distribute sparse computations across multiple GPUs. 

\begin{table}[h]
\centering
\caption{Average Time of 15 Epochs (seconds). Training time of TransE model with Freebase dataset (250M triplets, 77M entities. 74K relations, batch size 393K)  on 32 NVIDIA A100 GPUs. FSDP enables model training with larger embedding when DDP fails.}
\begin{tabular}{|p{2cm}|p{2.5cm}|p{2.5cm}|}
\hline
\textbf{Embedding Size} & \textbf{DDP (Distributed Data Parallel)} & \textbf{FSDP (Fully Sharded Data Parallel)} \\ \hline
16                      & 65.07 ± 1.641                            & 63.35 ± 1.258                               \\ \hline
20                      & Out of Memory                            & 96.44 ± 1.490                               \\ \hline
\end{tabular}
\end{table}

We run an experiment with a large-scale KG to showcase the performance of distributed SpTransX. Freebase (250M triplets, 77M entities. 74K relations, batch size 393K) dataset is trained using the TransE model on 32 NVIDIA A100 GPUs of NERSC using various distributed settings. SpTransX’s Streaming dataset module allows fetching only the necessary batch from the dataset and enables memory-efficient training. FSDP enables model training with larger embedding when DDP fails.

\section{Scaling and Communication Bottlenecks for Large KG Training}
\label{A:scaling}
Communication can be a significant bottleneck in distributed KGE training when using SpMM. However, by leveraging Distributed Data-Parallel (DDP) in PyTorch, we successfully scale distributed SpTransX to 64 NVIDIA A100 GPUs with reasonable efficiency. The training time for the COVID-19 dataset with 60,820 entities, 62 relations, and 1,032,939 triplets is in Table \ref{table:scaling}. 
% \vspace{-.3cm}
\begin{table}[h]
\centering
\caption{Scaling TransE model on COVID-19 dataset}
\begin{tabular}{|c|c|}
\hline
\textbf{Number of GPUs} & \textbf{500 epoch time (seconds)} \\ \hline
4                       & 706.38                            \\ \hline
8                       & 586.03                            \\ \hline
16                      & 340.00                               \\ \hline
32                      & 246.02                            \\ \hline
64                      & 179.95                            \\ \hline
\end{tabular}
\label{table:scaling}
\end{table}
% \vspace{-.2cm}
It indicates that communication is not a bottleneck up to 64 GPUs. If communication becomes a performance bottleneck at larger scales, we plan to explore alternative communication-reducing algorithms, including 2D and 3D matrix distribution techniques, which are known to minimize communication overhead at extreme scales. Additionally, we will incorporate model parallelism alongside data parallelism for large-scale knowledge graphs.

\section{Backpropagation of SpMM}
\label{A:backprop}
 Our main computational kernel is the sparse-dense matrix multiplication (SpMM). The computation of backpropagation of an SpMM w.r.t. the dense matrix is also another SpMM. To see how, let's consider the sparse-dense matrix multiplication $AX = C$ which is part of the training process. As long as the computational graph reduces to a single scaler loss $\mathfrak{L}$, it can be shown that $\frac{\partial C}{\partial X} = A^T$. Here, $X$ is the learnable parameter (embeddings), and $A$ is the sparse matrix. Since $A^T$ is also a sparse matrix and $\frac{\partial \mathfrak{L}}{\partial C}$ is a dense matrix, the computation $\frac{\partial \mathfrak{L}}{\partial X} = \frac{\partial C}{\partial X} \times \frac{\partial \mathfrak{L}}{\partial C} = A^T \times \frac{\partial \mathfrak{L}}{\partial C} $ is an SpMM. This means that both forward and backward propagation of our approach benefit from the efficiency of a high-performance SpMM.

\subsection*{Proof that $\frac{\partial C}{\partial X} = A^T$}
 To see why $\frac{\partial C}{\partial X} = A^T$ is used in the gradient calculation, we can consider the following small matrix multiplication without loss of generality.
\begin{align*}
A &= \begin{bmatrix}
a_1 & a_2 \\
a_3 & a_4
\end{bmatrix} \\ 
 X &= \begin{bmatrix}
x_1 & x_2 \\
x_3 & x_4
\end{bmatrix} \\
 C &=  \begin{bmatrix}
c_1 & c_2 \\
c_3 & c_4
\end{bmatrix}
\end{align*}
Where $C=AX$, thus-
\begin{align*}
c_1&=f(x_1, x_3) \\
c_2&=f(x_2, x_4) \\
c_3&=f(x_1, x_3) \\
c_4&=f(x_2, x_4) \\
\end{align*}
Therefore-
\begin{align*}
\frac{\partial \mathfrak{L}}{\partial x_1} &= \frac{\partial \mathfrak{L}}{\partial c_1} \times \frac{\partial c_1}{\partial x_1} + \frac{\partial \mathfrak{L}}{\partial c_2} \times \frac{\partial c_2}{\partial x_1} + \frac{\partial \mathfrak{L}}{\partial c_3} \times \frac{\partial c_3}{\partial x_1} + \frac{\partial \mathfrak{L}}{\partial c_4} \times \frac{\partial c_4}{\partial x_1}\\
&= \frac{\partial \mathfrak{L}}{\partial c_1} \times \frac{\partial \mathfrak{c_1}}{\partial x_1} + 0 + \frac{\partial \mathfrak{L}}{\partial c_3} \times \frac{\partial \mathfrak{c_3}}{\partial x_1} + 0\\
&= a_1 \times \frac{\partial \mathfrak{L}}{\partial c_1} + a_3 \times \frac{\partial \mathfrak{L}}{\partial c_3}\\
\end{align*}

Similarly-
\begin{align*}
\frac{\partial \mathfrak{L}}{\partial x_2}
&= a_1 \times \frac{\partial \mathfrak{L}}{\partial c_2} + a_3 \times \frac{\partial \mathfrak{L}}{\partial c_4}\\
\frac{\partial \mathfrak{L}}{\partial x_3}
&= a_2 \times \frac{\partial \mathfrak{L}}{\partial c_1} + a_4 \times \frac{\partial \mathfrak{L}}{\partial c_3}\\
\frac{\partial \mathfrak{L}}{\partial x_4}
&= a_2 \times \frac{\partial \mathfrak{L}}{\partial c_2} + a_4 \times \frac{\partial \mathfrak{L}}{\partial c_4}\\
\end{align*}
This can be expressed as a matrix equation in the following manner-
\begin{align*}
\frac{\partial \mathfrak{L}}{\partial X} &= \frac{\partial C}{\partial X} \times \frac{\partial \mathfrak{L}}{\partial C}\\
\implies \begin{bmatrix}
\frac{\partial \mathfrak{L}}{\partial x_1} & \frac{\partial \mathfrak{L}}{\partial x_2} \\
\frac{\partial \mathfrak{L}}{\partial x_3} & \frac{\partial \mathfrak{L}}{\partial x_4}
\end{bmatrix} &= \frac{\partial C}{\partial X} \times \begin{bmatrix}
\frac{\partial \mathfrak{L}}{\partial c_1} & \frac{\partial \mathfrak{L}}{\partial c_2} \\
\frac{\partial \mathfrak{L}}{\partial c_3} & \frac{\partial \mathfrak{L}}{\partial c_4}
\end{bmatrix}
\end{align*}
By comparing the individual partial derivatives computed earlier, we can say-

\begin{align*}
\begin{bmatrix}
\frac{\partial \mathfrak{L}}{\partial x_1} & \frac{\partial \mathfrak{L}}{\partial x_2} \\
\frac{\partial \mathfrak{L}}{\partial x_3} & \frac{\partial \mathfrak{L}}{\partial x_4}
\end{bmatrix} &= \begin{bmatrix}
a_1 & a_3 \\
a_2 & a_4
\end{bmatrix} \times \begin{bmatrix}
\frac{\partial \mathfrak{L}}{\partial c_1} & \frac{\partial \mathfrak{L}}{\partial c_2} \\
\frac{\partial \mathfrak{L}}{\partial c_3} & \frac{\partial \mathfrak{L}}{\partial c_4}
\end{bmatrix}\\
\implies \begin{bmatrix}
\frac{\partial \mathfrak{L}}{\partial x_1} & \frac{\partial \mathfrak{L}}{\partial x_2} \\
\frac{\partial \mathfrak{L}}{\partial x_3} & \frac{\partial \mathfrak{L}}{\partial x_4}
\end{bmatrix} &= A^T \times \begin{bmatrix}
\frac{\partial \mathfrak{L}}{\partial c_1} & \frac{\partial \mathfrak{L}}{\partial c_2} \\
\frac{\partial \mathfrak{L}}{\partial c_3} & \frac{\partial \mathfrak{L}}{\partial c_4}
\end{bmatrix}\\
\implies \frac{\partial \mathfrak{L}}{\partial X} &= A^T \times \frac{\partial \mathfrak{L}}{\partial C}\\
\therefore \frac{\partial C}{\partial X} &= A^T \qed
\end{align*}



\end{document}
 