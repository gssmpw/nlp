\appendix
\section{Appendix}

\subsection{tSNE Visualization of SCAN Clusters}

Following \cref{sec:tsne}, we present the tSNE projections of \ToolName{} embeddings for all three datasets: Milan on \cref{fig:tsne_label_and_cluster_milan}, Aruba on \cref{fig:tsne_label_and_cluster_aruba}, and Cairo on \cref{fig:tsne_label_and_cluster_cairo}. The left panel shows the embeddings colored by CASAS label; the right one shows the same exact embeddings colored by SCAN cluster labels from 0 to 19. Note how the \textit{Other} label (gray color) is mostly scattered around . Also note that Cairo, which has 81\% of its data points labeled as \textit{Other} (\cref{tab:casas_label_distribution}), has most of its clusters fully gray. 


\begin{figure}
    \centering
        \includegraphics[width=0.8\linewidth]{figures/tsne_label_and_cluster_milan.pdf}
    \caption{
        tSNE projection of SCAN embeddings from the Milan household. Left pane colored by CASAS labels; right pane colored by SCAN cluster labels from 0 to 19.
    } 
    \label{fig:tsne_label_and_cluster_milan}
\end{figure}


\begin{figure}
    \centering
        \includegraphics[width=0.8\linewidth]{figures/tsne_label_and_cluster_aruba.pdf}
    \caption{
        tSNE projection of SCAN embeddings from the Aruba household. Left pane colored by CASAS labels; right pane colored by SCAN cluster labels from 0 to 19.
    } 
    \label{fig:tsne_label_and_cluster_aruba}
\end{figure}


\begin{figure}
    \centering
        \includegraphics[width=0.8\linewidth]{figures/tsne_label_and_cluster_cairo.pdf}
    \caption{
        tSNE projection of SCAN embeddings from the Cairo household. Left pane colored by CASAS labels; right pane colored by SCAN cluster labels from 0 to 19.
    } 
    \label{fig:tsne_label_and_cluster_cairo}
\end{figure}


\subsection{tSNE Projection of Varying Number of Clusters}

Following \cref{sec:label_quality_results}, we present the tSNE projections of \ToolName{} embeddings for the Milan household for various number of cluster, when \( k \in \{10, 15, 20, 30, 40, 50, 60, 100\} \). The visualization uses CASAS labels to color the clusters, even though they were never used during training. As $k$ increases, the resulting clusters appear to be more distinct and increasingly homogeneous, where each point cloud mostly contains only one distinct CASAS label. As shown in \cref{fig:varying_clusters_f1}, the increase in $k$ is associated with diminishing returns to accuracy, and 20-30 range seems like the best choice for the optimal number of clusters.

\begin{figure}
    \centering
        \includegraphics[width=0.9\linewidth]{figures/varying_clusters_tsne.pdf}
    \caption{
        tSNE projections of \ToolName{} embeddings for the Milan household with varying numbers of clusters, when \( k \in \{10, 15, 20, 30, 40, 50, 60, 100\} \). Each point represents a sensor sequence colored by CASAS label for visualization purposes only. As \(k\) increases, clusters become more distinct and homogeneous, leading to standalone cluster even for rare activities such as \emph{Leave\_Home}. 
    } 
    \label{fig:varying_clusters_tsne}
\end{figure}


\subsection{\ToolName{} Sub-Activity Labels}

\cref{fig:atomic_activity_tree} shows the list of all \ToolName{} sub-activity labels that we provided the annotators with. The activities are organized hierarchically by categories, e.g. \textit{Single Room Activity} -> \textit{Kitchen Activity} -> \textit{Movement in Kitchen} -> \textit{Movement Near Stove}. These labels were used to populate the labeling drop-down menu in \ToolName{} visualization tool shown in \cref{fig:viz_tool}.

\begin{figure}
    \centering
        \includegraphics[width=0.8\linewidth]{figures/atomic_activity_tree.pdf}
    \caption{
        \ToolName{} sub-activity hierarchy. This list of manually-curated labels was used to populate the labeling drop-down menu in \ToolName{} visualization and labeling tool.
    } 
    \label{fig:atomic_activity_tree}
\end{figure}

\subsection{Individual Cluster Sub-Activity Labels}

Here we present the list of individual cluster labels for all three CASAS datasets. For every cluster, \cref{fig:cluster_majority_label_summary} shows what was the majority label assigned by the annotators, and how many raters "voted" for that label in the \texttt{votes} column. The \textit{label (level up)} column shows up-leveled labels following the hierarchy in \cref{fig:atomic_activity_tree}. This summary provides a closer look at individual cluster annotations that were used to compute Fleiss' Kappa for cluster agreement in \cref{tab:kappa_scores}.

\begin{figure}
    \centering
        \includegraphics[width=0.9\linewidth]{figures/cluster_majority_label_summary.jpg}
    \caption{
        Sub-activity labels assigned to each cluster  manually labeled clusters for all households. \texttt{votes} column shows how many annotations aligned with this particular label; \texttt{level up} column provides a parent label.
    } 
    \label{fig:cluster_majority_label_summary}
\end{figure}

\subsection{Household Layouts}
\label{sec:household_layouts}

To accurately visualize and analyze sensor data, we utilized the household layouts from \cite{hiremath2022bootstrapping, tdost2024, casas2009}. We re-rendered these layouts to simplify the visuals and ensure compatibility with our \ToolName{} visualization tool.

\begin{figure}
    \centering
    \includegraphics[width=.95\linewidth]{figures/all_house_layouts.pdf}
    \caption{
        Simplified and augmented layout of (a) Milan, (b) Aruba, and (c) Cairo household floor plans, adapted from \cite{casas2009} and \cite{hiremath2022bootstrapping}.
    }
    \label{fig:aruba_layout}
\end{figure}

