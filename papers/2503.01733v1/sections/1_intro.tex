\section{Introduction}

Human Activity Recognition (HAR) using ambient sensors integrated into the built environment is an increasingly dynamic research field that has a potential to revolutionize a number of healthcare and monitoring use cases \cite{alam2012, kientz2008, ransing2015}. Practical applications of HAR include support for independent living for seniors \cite{sasaki2016performance, fi16120476, riboni2015lab} and health monitoring \cite{fritz2022nurse, civitarese2017behavioral}. In elder care contexts, HAR can serve as a foundation for longitudinal assessments of shifts in daily habits and behaviors. Such analyses may enable the early detection of emerging physical and cognitive challenges, therefore supporting timely interventions and improved patient outcomes \cite{morita2023health, riboni2016analysis}.

Deploying Human Activity Recognition requires the instrumentation of target environments (e.g., residential spaces) with multimodal sensor networks, followed by the development of algorithmic approaches to transform the resultant sensor data streams into meaningful activity labels. Existing HAR approaches commonly employ fully supervised learning paradigms \cite{liciotti2020sequential, bouchabou2021using, li2019relation, deepcasas2018}. Several large-scale, annotated datasets have been established to facilitate these approaches, including CASAS \cite{casas2009}, Marble \cite{marble2022}, and Orange4Home \cite{orange2017}.

The ultimate goal of HAR research is to develop techniques that are practical and deployable in a novel real-world setting.  Toward this goal, we argue that successful methodologies should have the following characteristics:
\begin{enumerate}
    \item \textit{Minimal dependence on labeled data:}  Obtaining large quantities of labeled training data for each new environment is prohibitively expensive as data collection and annotation in real homes remains both resource-intensive and logistically challenging. Sensor equipment must be installed for many months to capture diverse human behaviors, and the resulting large-scale sensor logs require painstaking and costly annotation by trained experts or residents.  HAR methodologies should therefore either support generalization and transfer between environments \cite{tdost2024}, be sample efficient, such as by utilizing active learning \cite{hiremath2022bootstrapping, hiremath2024maintenance}, or both. 
    \item \textit{Independence from prior segmentation:} Numerous algorithms \cite{tdost2024, hiremath2022bootstrapping, deepcasas2018} operate under the assumption that sensor streams are pre-segmented into discrete, single-activity intervals. However, the temporal segmentation of continuous activity data presents significant challenges in delineating precise activity boundaries. Therefore, robust HAR approaches should not presuppose the availability of such segmentation.
    \item \textit{Flexible activity granularity:} There is no consensus in existing HAR literature regarding the optimal definition and scope of an "activity," in large part because activity relevance and scope varies substantially across application domains. While some use cases may prioritize macro-level activities (e.g., sleep, cooking), others require fine-grained activity recognition (e.g., walking from fridge to stove). Current supervised learning approaches, including those supporting cross-domain transfer \cite{tdost2024}, are constrained by the predefined coarse activity labels within their training datasets.
\end{enumerate}

In this work, we introduce \ToolName{}, a self-supervised system designed to \textit{discover} fine-grained human sub-activities from unlabeled sensor data without relying on pre-segmentation. \ToolName{}
% In this work, we 
advances the field in two complimentary ways. First, we contribute a novel HAR approach that uniquely satisfies all three of the above requirements.  Second, we contribute an open source tool that facilitates a shift in how HAR annotation is conducted, enabling the dynamic discovery of finer-grained sub-activities, accommodating a wider variety of behaviors observed in real homes.

We propose the use of a minimal-effort, human-in-the-loop annotation workflow, with domain experts only needing to label a small number of representative cluster centroids, corresponding to only $0.05\%$ of our overall dataset. These labeled centroids are then used to automatically propagate annotations to other similar data points. 
This design substantially reduces the manual labeling burden while still preserving interpretability and control over the final labels. 
Beyond bypassing the label scarcity problem, this methodology also relaxes the pre-segmentation assumption, allowing continuous time-series data to be naturally segmented based on the sensor readings themselves.


We demonstrate the effectiveness of our approach through a re-annotation exercise on widely used HAR datasets \cite{casas2009}. By applying our pipeline to an already labeled dataset, we show that our method can uncover finer-grained activities and yield more nuanced, contextually relevant annotations than the original coarse labels.
Our key contributions can be summarized as follows: 

\begin{enumerate}
    \item \textbf{Self-Supervised Pipeline for Human Sub-Activity Discovery}: We design a clustering-based approach that learns directly from raw sensor streams—no large, pre-annotated training sets required—revealing the underlying sub-activities, often overlooked by conventional methods.
    
    \item \textbf{Elimination of Pre-Segmentation Constraints}: By handling continuous, unsegmented time-series, our  framework is applicable to realistic deployment scenarios, and can be easily used in real homes. 
    
    \item \textbf{Human-in-the-Loop Annotation with Minimal Effort}: Employing an active learning–like strategy, annotators focus on only a small set of representative cluster centroids. Labels then propagate to the entire dataset, drastically reducing manual labeling workload by a factor of over 1,000 (requiring annotation of less than 0.01\% of all data points).
    
    \item \textbf{Open-Source Re-Annotation Tool}: We open-source our user-friendly tool that visualizes sensor activations on a 2D house layout, facilitating the labeling of any types of activities.
\end{enumerate}
