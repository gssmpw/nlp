\section{Discussion}

In this section, we discuss how \ToolName{} can be used in real-world deployable applications (\cref{sec:use_cases}). We also outline the limitations of our approach and propose directions for future research (\cref{sec:limitations_future_work}).

\subsection{Potential Use Cases}
\label{sec:use_cases}
In this section, we illustrate the real-world potential of our approach by demonstrating its ability to automatically annotate ambient sensor datasets and re-annotate existing data at finer granularity levels. Finally, we provide some precise implementation details, including a minute-by-minute reconstruction of daily activity patterns.

\subsubsection{Annotation of Massive Unlabeled Ambient Sensor Datasets:}
Smart homes have been studied extensively across academia, industry and medical fields.  While fully annotated datasets such as CASAS \cite{casas2009}, Marble \cite{marble2022}, and Orange4Home \cite{orange2017} are widely used in AI research, they are far from the norm.  Large-scale medical initiatives such as CART \cite{beattie2020collaborative} (272 homes over 3 years) and TIHM \cite{palermo2023tihm} (56 homes over 50 days) produce orders of magnitude more data, but remain less commonly used due to their lack of activity labels resulting from prohibitively high costs of manual annotation. With \ToolName{}, it becomes feasible to automatically annotate such massive datasets. Users can choose their desired level of granularity and exercise control over the annotation process via an interactive visualization tool that displays sensor activation sequences for each cluster. While such labels may not constitute true ground truth, which is inaccessible without camera data or similar information, our analysis has shown that processing with \ToolName{} results in semantically meaningful activity clusters that can be labeled with high degree of agreement, thus providing insight into common everyday activities. We believe that \ToolName{} can expand access to previously unlabeled data, creating opportunities for advancement in the areas such as in-home health monitoring \cite{morita2023health}, early detection of Alzheimer's \cite{alberdi2018smart}, and intelligent support for individuals with dementia \cite{demir2017smart}.

\subsubsection{Re-annotation of Existing Datasets with Finer Granularity:}
As we have seen in case of CASAS \cite{cook2012casas}, existing datasets in the HAR domain often provide only coarse labels that capture broad activities, obscuring the rich variability of human behavior. \ToolName{} addresses this shortcoming by enabling the discovery of sub-activities and variants within the data. For instance, in sleep monitoring, rather than merely recording an eight-hour sleep duration, our method could allow to  differentiate between distinct sub-activities such as tossing and turning, or identifying periods of nocturnal bathroom visits. Similarly, in meal preparation, instead of labeling the entire process as \textit{Cooking}, our approach can reveal detailed steps such as time spent at the stove, visits to the pantry, and travel to/from the dining room.  By re-annotating existing datasets with these finer-grained labels, \ToolName{} provides deeper insights into behavioral patterns, which can be particularly valuable in clinically relevant applications and personalized intervention strategies.

\subsubsection{Temporal Sequence Analysis}
\label{sec:cluster_timeline}

\begin{figure}
    \centering
        \includegraphics[width=.9\linewidth]{figures/atomic_activity_timeline_minutes.pdf}
    \caption{
    A minute by minute view of atomic activities in the Milan household for two hours. Each colored segment represents a specific atomic activity (e.g., \emph{movement near stove}, \emph{entering guest bathroom}), allowing for a fine-grained understanding of daily routines and transitions. This higher-resolution approach goes beyond broad categories like \emph{cooking} or \emph{bathing}, highlighting detailed behaviors and potential health or lifestyle insights.
    }
    \label{fig:atomic_activity_timeline_minutes}
\end{figure}


Building on the above, annotation of HAR data at greater granularity facilitates more detailed temporal analysis of human behavioral data.  
Specifically, our cluster-based approach facilitates the reconstruction of a continuous temporal record of daily activities by assigning each time window to a specific cluster. As illustrated in \cref{fig:atomic_activity_timeline_minutes}, a minute-by-minute snapshot of a Milan household on a given date displays various sub-activity labels (e.g., \emph{movement near stove}, \emph{entering guest bathroom}). The color scale represents the model’s confidence in cluster assignments, transitioning from low confidence (green) to high (red). This figure reveals coherent temporal sequences, such as the resident remaining in the \emph{living room armchair} from 9:00 to 9:15 AM before transitioning to kitchen activities and entering the guest bathroom around 9:25 AM. Lower-confidence activities, often occurring during transitions between rooms, can be filtered out to produce a more stable activity timeline. 

By delineating detailed sequences of sub-activities, our method enables domain experts and researchers to gain deeper insights into residents’ daily routines. This granularity allows for the identification of significant behavioral changes, such as increased bathroom visits indicating potential health issues or prolonged armchair sitting reflecting shifts in mobility or energy levels, factors that are critical for in-home health monitoring \cite{morita2023health}. Additionally, the continuous temporal component supports real-time interventions, adaptive healthcare monitoring, and personalized smart home automations.

\subsection{Future Work}
\label{sec:limitations_future_work}

While our approach demonstrates promising results, it has certain limitations which can be addressed in future work.
First, our current method processes simple sensor activation sequences without considering additional context such as sensor location, time of day, or activation duration. Future work could enrich these input sequences by incorporating spatial and temporal contextual information using TDOST-style sentence embeddings \cite{tdost2024}, which could lead to an improved overall performance. 
Second, the use of fixed sliding windows (i.e., 20 sensor readings) limits contextual understanding, as the model does not account for sensor events immediately preceding or following the window. To address this limitation for model training, future work could either increase the window size or append pre-text and post-text windows to provide additional temporal context. Additionally, integrating this extended context into our visualization tool would help annotators better interpret the sensor sequences. 
Third, our current approach requires manually creating a hierarchical label tree for each dataset, a process that is both labor-intensive and dataset-specific. While we want to preserve researchers' flexibility in choosing the level and granularity of activities, our method could benefit from a unified hierarchical taxonomy of activities, which could be learned from the data by processing multiple heterogeneous datasets all at once.
Fourth, our current manual labeling process assumes that annotators must review every sample and select a label from an extensive list of possibilities, which is both time-consuming and prone to inconsistencies. To address this, we could partially automate the process by leveraging large language models to pre-populate candidate labels. For example, by encoding sensor sequences as descriptive, TDOST-style sentences \cite{tdost2024}, we could automatically summarize cluster centroids, thereby presenting annotators with pre-generated labels for validation rather than requiring them to select the labels from the full list.
Finally, the approach assumes that activity patterns remain stable over time, while in real life significant changes in resident behavior may require periodic re-training or re-clustering. Addressing these limitations in future work will further enhance the adaptability and robustness of our framework for real-world smart home environments.