\section{Related Work}
\textbf{MLLMs.} 
This field has undergone significant evolution~\cite{yin2023survey, fu2023challenger,zhang2024debiasing,fu2024mme}, initially rooted in BERT-based language decoders and later incorporating advancements in LLMs. 
MLLMs exhibit enhanced capabilities and performance, particularly through end-to-end training techniques, by leveraging advanced LLMs such as GPTs~\cite{gpt4,brown2020language},
LLaMA~\cite{touvron2023llama,touvron2023llama2},  and Vicuna~\cite{chiang2023vicuna}. Recent model developments, including Flamingo~\cite{awadalla2023openflamingo}, BLIP-2~\cite{li2023blip}, InstructBLIP~\cite{dai2024instructblip}, LLaVA~\cite{liu2023visual}, Qwen-VL~\cite{bai2023qwen}, Slime~\cite{zhang2024beyond}, and VITA~\cite{fu2024vita}, bring unique perspectives to challenges such as scaling pre-training, enhancing instruction-following capabilities, and overcoming alignment issues. 
However, the performance of these models in the face of real educational scenarios has often not been revealed.

\textbf{Multimodal Benchmark.} 
With the development of MLLMs, a number of benchmarks have been built.
For instance, MME~\cite{fu2023mme} constructs a comprehensive evaluation benchmark that includes a total of 14 perception and cognition tasks. All QA pairs in MME are manually designed to avoid data leakage, and the binary choice format makes it easy to quantify.
MMT-Bench~\cite{mmtbench} scales up the dataset even further, including $31,325$ QA pairs from various scenarios such as autonomous driving and embodied AI. It encompasses evaluations of model capabilities such as visual recognition, localization, reasoning, and planning.
MME-RealWorld~\cite{zhang2024mme} contains over 29K question-answer pairs that cover 43 subtasks across 5 real-world scenarios and is the largest manually annotated benchmark to date. 
Additionally, other benchmarks focus on real-world usage scenarios~\cite{fu2024blink,bitton2023visit}, reasoning capabilities~\cite{yu2024mm,han2023coremm} and mathematical reasoning~\cite{lu2024mathvista} or correctness~\cite{yan2024errorradar}. 
However, there are widespread issues, such as data scale, annotation quality, and task difficulty, in these benchmarks, making it hard to assess the challenges that MLLMs face in the real world.

\begin{table}[htbp]
\centering
\renewcommand{\arraystretch}{1.5} % Adjust line spacing
\caption{Draft Evaluation Criteria and Scoring Standards}\label{tab:draft_cri}
\label{tab:grading-criteria}
\begin{tabularx}{\textwidth}{|p{0.2\textwidth}|X|p{0.05\textwidth}|}
\hline
\textbf{Criteria} & \textbf{Description and Scoring Standards} & \textbf{Weight (\%)} \\ \hline

\textbf{Completeness of Steps} & 
30 points: Complete solution process, including all intermediate steps.  
20-29 points: Most steps present, minor omissions.  
10-19 points: Partial steps, significant omissions.  
0-9 points: Most steps missing, incomplete process. 
& 30 \\ \hline

\textbf{Layout and Clarity} & 
25 points: Logical layout, clear writing, easy to understand.  
18-24 points: Fairly reasonable layout, mostly clear, minor ambiguities.  
10-17 points: Disorganized layout, unclear writing, harder to understand.  
0-9 points: Chaotic layout, illegible writing, incomprehensible.  
& 25 \\ \hline

\textbf{Correctness of Problem-Solving Approach} & 
20 points: Entirely correct solution approach, clear logic.  
15-19 points: Mostly correct, minor logical gaps.  
10-14 points: Significant errors or omissions affecting outcome.  
0-9 points: Incorrect solution approach, no logical reasoning.  
& 20 \\ \hline

\textbf{Logical Consistency and Rigor} & 
15 points: Rigorous logic, well-connected steps.  
10-14 points: Mostly rigorous, minor gaps.  
5-9 points: Weak logic, poor connections between steps.  
0-4 points: Chaotic logic, no clear connections.  
& 15 \\ \hline

\textbf{Unit Annotation and Answer Presentation} & 
5 points: Correct unit annotations, clear final answer.  
4 points: Mostly correct units, minor omissions.  
2-3 points: Incomplete/unclear units, vague final answer.  
0-1 points: Missing/incorrect units, unclear answer.  
& 5 \\ \hline

\textbf{Calculation Accuracy} & 
5 points: All calculations accurate.  
4 points: Most calculations accurate, minor errors.  
2-3 points: Frequent calculation errors.  
0-1 points: Severe errors, incorrect result.  
& 5 \\ \hline

\textbf{Final Score and Feedback} & 
Final score based on weighted criteria, with constructive feedback for improvement. 
& -- \\ \hline

\end{tabularx}
\end{table}


\begin{table}[htbp]
\caption{Categories and Subcategories of Student Errors with Definitions. This table presents a breakdown of the main error categories, their subcategories, and corresponding definitions, highlighting the various challenges students face during problem-solving.}
\label{tab:defi}
\centering
\resizebox{\textwidth}{!}{%
\begin{tabular}{|p{3cm}|p{3cm}|p{8cm}|}
\hline
\rowcolor{gray!20}
\textbf{Category} & \textbf{Subcategory} & \textbf{Definition} \\ \hline
\multirow{1}{4cm}{\textbf{Attitude Issues}} & Messy Drafts & Students' drafts show signs of careless scribbles. \\ \midrule
\multirow{3}{4cm}{\textbf{Misunderstanding}} & Ambiguous Statements & Problems have unclear or ambiguous wording. \\ \cline{2-3}
 & Ignoring Constraints & Students fail to notice constraints in the problem. \\ \cline{2-3}
 & Missing Key Info & Students overlook critical information in the problem. \\ \hline
\multirow{1}{4cm}{\textbf{Logical Reasoning}} & Faulty Reasoning & Students make incorrect conclusions or illogical deductions. \\ \hline
\multirow{1}{4cm}{\textbf{Cognitive Bias Errors}} & Misreading Info & Students misinterpret information due to non-habitual thinking. \\ 

\hline
\multirow{4}{4cm}{\textbf{Answering Technique}} & Improper Format & Students provide answers in an improper format. \\ \cline{2-3}
 & Draft Transcription & Calculations on the draft are correct, but transcription is wrong. \\ \cline{2-3}
 & Misaligned answer & The answer is correct, but the format is wrong. \\ \cline{2-3}
 & Incorrect Order & Students provide answers in the wrong order. \\ \hline
\multirow{1}{4cm}{\textbf{Handwriting Errors}} & Writing A, Thinking B & Students think of answer A but write down answer B. \\ \cline{2-3}
& {Digit Transcription} & Students calculate correctly but copy digits incorrectly. \\
\cline{2-3}
& {Miscounting} & Students make counting mistakes. \\
\cline{2-3}
& {Missing Units} & Students omit units in their answers. \\
\cline{2-3}
& {Incorrect Formula} & Students write down an incorrect formula. \\
\cline{2-3}
& {Extra/Missing Symbol} & Students add or omit symbols during problem-solving. \\
\cline{2-3}
& {Omitting Letters} & Students miss or add unnecessary letters in their answers. \\
\hline
\multirow{4}{4cm}{{\textbf{Attention to Detail}}} & Extra or Missing Zeros & Errors in handling numbers, such as adding or omitting zeros. \\  \cline{2-3}
& {Decimal Point Errors} & Mistakes in decimal point placement. \\  \cline{2-3}
& {Lack of Simplification} & Students fail to simplify fractions or expressions. \\  \cline{2-3}
& {Misplaced Parentheses} & Errors in using parentheses. \\  \cline{2-3}
& {Wrong Sign} & Incorrect sign usage during rearrangement. \\
\hline

\multirow{6}{4cm}{\textbf{Computation Errors}} & Arithmetic Errors & Miscalculations in addition, multiplication, or division. \\   \cline{2-3}
& {Conversion Errors} & Mistakes in converting calculation results into the final answer. \\   \cline{2-3}

& {Division Errors} & Incorrect handling of quotients or remainders. \\ \cline{2-3}
& {Decimal Multiplication} & Errors in aligning or processing decimals in multiplication. \\ \cline{2-3}
& {Fraction Comparison} & Incorrect simplification or comparison of fractions. \\
\cline{2-3}
& {Misapplied Models} & Failure to apply appropriate mathematical strategies or models. \\\hline
\multirow{1}{4cm}{\textbf{Knowledge Gaps}} & Concepts Not Mastered & Insufficient understanding or memory of essential subject. \\ \hline
\end{tabular}%
}

\end{table}

% \begin{longtable}{@{}p{3cm}p{4cm}p{6.5cm}@{}}

% \toprule
% Category & Subcategory & Definition \\
% \midrule
% \endfirsthead

% \toprule
% Category & Subcategory & Definition \\
% \midrule
% \endhead

% \bottomrule
% \endfoot

% % Category: Attitude Issues
% \rowcolor{gray!20}
% Attitude Issues & Messy Drafts & Careless scribbles and lack of seriousness. \\

% % Category: Misunderstanding Problems
% \rowcolor{gray!10}
% Misunderstanding Problems & Ambiguous Statements & Unclear wording requiring interpretation. \\
% \rowcolor{gray!10}
% & Ignoring Constraints & Failing to notice problem constraints. \\
% \rowcolor{gray!10}
% & Missing Key Info & Overlooking critical problem details. \\

% % Category: Logical Reasoning Errors
% \rowcolor{gray!20}
% Logical Reasoning Errors & Faulty Reasoning & Incorrect conclusions due to improper reasoning. \\

% % Category: Cognitive Bias Errors
% \rowcolor{gray!10}
% Cognitive Bias Errors & Misreading Info & Misinterpretation due to non-habitual thinking. \\

% % Category: Answering Technique Errors
% \rowcolor{gray!20}
% Answering Technique Errors & Improper Format & Incorrect format or missing solution steps. \\
% \rowcolor{gray!20}
% & Draft Transcription Errors & Correct draft calculations, but transcription errors. \\
% \rowcolor{gray!20}
% & Misaligned Input and Answer & Correct answer but wrong format. \\
% \rowcolor{gray!20}
% & Incorrect Answer Order & Wrong order in answers. \\

% % Category: Handwriting Errors
% \rowcolor{gray!10}
% Handwriting Errors & Writing A, Thinking B & Thinking one answer but writing another. \\
% \rowcolor{gray!10}
% & Digit Transcription Errors & Correct calculation but copied digits incorrectly. \\
% \rowcolor{gray!10}
% & Miscounting & Counting mistakes. \\
% \rowcolor{gray!10}
% & Missing Units & Omission of units in answers. \\
% \rowcolor{gray!10}
% & Incorrect Formula & Incorrect formula used. \\
% \rowcolor{gray!10}
% & Extra or Missing Symbols & Incorrect addition or omission of symbols. \\
% \rowcolor{gray!10}
% & Omitting or Adding Letters & Missing or adding unnecessary letters. \\

% % Category: Attention to Detail Errors
% \rowcolor{gray!20}
% Attention to Detail Errors & Extra or Missing Zeros & Errors with zeros in numbers. \\
% \rowcolor{gray!20}
% & Decimal Point Errors & Incorrect decimal point placement. \\
% \rowcolor{gray!20}
% & Lack of Simplification & Failing to simplify expressions. \\
% \rowcolor{gray!20}
% & Misplaced Parentheses & Errors with parentheses usage. \\
% \rowcolor{gray!20}
% & Wrong Sign in Rearrangement & Incorrect sign usage during rearrangement. \\

% % Category: Computation Errors
% \rowcolor{gray!10}
% Computation Errors & Basic Arithmetic Errors & Mistakes in arithmetic operations. \\
% \rowcolor{gray!10}
% & Conversion Errors & Errors in converting results into final answers. \\
% \rowcolor{gray!10}
% & Division Errors & Incorrect handling of division. \\
% \rowcolor{gray!10}
% & Decimal Multiplication & Errors in decimal multiplication. \\
% \rowcolor{gray!10}
% & Fraction Comparison & Incorrect fraction simplification. \\
% \rowcolor{gray!10}
% & Misapplied Models & Failure to apply appropriate strategies. \\

% % Category: Knowledge Gaps
% \rowcolor{gray!20}
% Knowledge Gaps & Core Concepts Not Mastered & Lack of understanding of essential concepts. \\

% \caption{Categories and Subcategories of Student Errors with Definitions. The table highlights the various challenges students face during problem-solving.}
% \label{tab:defi}
% \end{longtable}