\section{Related Work}
\label{sec:relWork}
\subsection{Plants for Greenhouse Effect Mitigation}
To limit the carbon accumulation in the atmosphere, three main methods are possible, namely, reducing the use of combustion for energy production, removing $\text{CO}_2$ from the atmosphere, and capturing $\text{CO}_2$ at the source before it enters the atmosphere \cite{benemann1997co2}. Traditional strategies focus on mitigating the emissions at the source \cite{shafique2020overview}; Shafique \textit{et al.} \cite{shafique2020overview} provide an overview of carbon sequestration using plants installed on urban roofs. In contrast, our paper considers the adversarial actions of a source and a sink to regulate the resulting carbon concentration in the atmosphere. Zhang \textit{et al.} \cite{zhang2022urban} assessed whether urban green spaces in China were sinks or sources of carbon using a life-cycle assessment (LCA) approach. They found that trees and shrubs were sinks, whereas lawns were sources due to required maintenance. Marchi \textit{et al.} \cite{marchi2015carbon} developed a model of a vertical greenery system and they compared the sequestration efficiency of different types of plants. Wang \textit{et al.} \cite{wang2021promoting} evaluated urban planting designs considering the sequestration efficiency of plants. Medium-sized evergreen trees were among the best performing species.    
    
The direct injection of $\text{CO}_2$ into the deep oceans, aquifers or depleted oil wells is a method for large-scale carbon sequestration, but it requires particular geological conditions \cite{zhou2017bio}. In contrast, forestation is a more natural process than artificial injection, but it has limited sequestration capacity unless large land areas are covered \cite{zhou2017bio}. Chemical absorption through neutralization of carbonic acid to form carbonates or bicarbonates provides a safe and permanent sequestration, but it is energy and cost intensive \cite{zhou2017bio}. In this paper, we focus on sequestration using microalgae, which is more efficient than forestation. 

Currently, the disadvantages of microalgae are the relatively high costs and their sensitivity to toxic substances in exhaust gases \cite{zhou2017bio}. To address the issue of cost, Ramaraj \textit{et al.} \cite{ramaraj2015biomass} studied the growth of algae in natural water, which is cheaper than their cultivation in an artificial medium; the volumetric carbon uptake rate was $175 \pm 27.86 \text{ mg}\text{L}^{-1}\text{d}^{-1}$. Viswanaathan \textit{et al.} \cite{viswanaathan2022integrated} tabulated the $\text{CO}_2$ uptake of different species of microalgae.  

Another benefit of microalgae-based sequestration is their use as biomass to produce biofuels \cite{arun2021technical}. Hence, plant-based sequestration methods may facilitate a circular flow of material compared to non-biological methods, i.e., they have a life-cycle that easily aligns with the circularity principles detailed in \cite{suarez2019operational}. It is estimated that the major costs in biofuel production from algal cultivations are 77\% from culturing, 12\% from harvesting, and 7.9\% from lipid extraction \cite{sarwer2022algal}.   


 

     





\subsection{Thermodynamical Material Networks for a Circular Economy}
At the core of any engineering subject there is the application of principles of physics and chemistry to develop a mathematical description of the target system. One of the most delicate steps is defining the conditions and simplifications that make the mathematical description sufficiently accurate for the desired purpose, but also simple enough to conclude with quantitative results and meaningful physical interpretations. This engineering approach is systematically used to design machines and processes, whereas it is rarely used for tackling whole-system problems such as the design of circular supply-recovery chains in a circular economy \cite{EMAFund}. Indeed, to date, circular-economy-related models are usually developed with data-analysis techniques such as LCA \cite{amicarelli2022life,walker2020life,xia2022review,cucurachi2019life,sala2021evolution} and material flow analysis (MFA) \cite{luan2021dynamic,li2022uncovering,sieber2020dynamic,liu2021dynamic,eriksen2020dynamic}. 

Thermodynamical material networks (TMNs) were recently proposed \cite{zocco2023thermodynamical,zocco2024unification,zocco2022circularity,zocco2024circular} to develop circular-economy models using the traditional engineering approach described above. Specifically, the methodology of TMNs is a generalization of the design of the Rankine cycle, in which mass and energy balances are applied to each compartment of the cycle and where the compartments are connected to form a network that delivers the desired flow of material in space and time (the material is the working fluid in the Rankine cycle). In contrast with data-analysis techniques such as LCA and MFA, TMNs are based on ordinary differential equations derived from dynamical mass and energy balances applied to each thermodynamic compartment of the network, thus increasing the accuracy of the models while being less data intensive \cite{zocco2023thermodynamical,zocco2022circularity}. Moreover, since based on differential equations, the design of TMNs can include one or more control systems. In this paper, we illustrate the use of the TMN methodology for circular microalgae-based carbon control, where the circularity of carbon dioxide is quantified using the TMN-based definition of circularity given in \cite{zocco2024circular} (specifically, in Definition 4), namely, $\lambda(\mathcal{N})$, with $\mathcal{N}$ the TMN processing the target material (carbon dioxide in this case) and with $\lambda(\mathcal{N}) \in (-\infty,0]$. Hence, the circularity of carbon dioxide reduces to the following arg-max problem \cite{zocco2024circular}:
\begin{equation}\label{eq:circProblem}
\mathcal{N}^* = \arg \max \,\,\, \lambda(\mathcal{N}).  
\end{equation}
      
       



\subsection{Finite-Time Stabilizing Control}
An essential requirement of a reliable process or system is to work in the desired conditions. If the dynamics of the process or system is described by a set of ordinary differential equations (ODEs), then this essential requirement is met if the system state can be stabilized to the desired conditions. Moreover, in practice, a further requirement is to reach the desired point within a finite time rather than merely asymptotically. The satisfaction of these requirement has led to the development of \emph{finite-time stabilizing} controllers, the first of which was proposed in \cite{Roxin1966} and subsequently extended for time-varying nonlinear dynamical systems in \cite{moulay2008finite,haddad2008finite}, for second-order systems in \cite{bhat1998continuous}, and using output feedback in \cite{hong2001output}.    

If, in addition, the controller solves an optimal control problem, the closed-loop nonlinear system is guaranteed to stabilize to the desired conditions both optimally and within a finite time. Such controllers were proposed in \cite{haddad2015finite} for continuous-time systems, \cite{haddad2023finite} for discrete-time systems, and \cite{lee2023finite} for stochastic discrete-time systems. In this paper, we modify the framework for continuous-time dynamical systems \cite{haddad2015finite} to develop a controller that guarantees optimal finite-time stabilization within a \emph{chosen} time.








\subsection{Reinforcement Learning for Continuous-Time Control}          
The advances in deep learning over the last ten years have led to state-of-the-art reinforcement learning (RL) algorithms based on deep neural networks. In 2015, Schulman \emph{et al.} \cite{pmlr-v37-schulman15} proposed the trust region policy optimization (TRPO)  algorithm for optimizing large nonlinear policies using a convolutional neural network with three layers as the policy and processing raw images directly to solve benchmark games. In 2016, Mnih \emph{et al.} \cite{mnih2016asynchronous} introduced asynchronous gradient descent showing that executing multiple agents in parallel on multiple instances of the environment has a stabilizing effect on training and it can be an alternative to the successful, but memory intensive, experience replay. In the same year, Lillicrap \emph{et al.} \cite{lillicrap2019continuouscontroldeepreinforcement} proposed an actor-critic approach based on the policy gradient algorithm which, with the same network architecture and hyperparameters, solved more than twenty simulated physics tasks. In 2017, Schulman \emph{et al.} \cite{schulman2017proximal} combined the data efficiency and reliability of TRPO while using only first-order optimization and alternate sampling data from the environment with optimizing the objective function via stochastic gradient ascent. The following year, Mania \emph{et al.} \cite{NEURIPS2018_7634ea65} aimed to significantly simplify the overall approach to RL since the complexity of the existing algorithms was, at that time, one of the major barriers to the deployment of RL in controlling real physical systems; they named their algorithm augmented random search (ARS) since the work resulted from an augmentation of a basic random search. Recently, Bhatt \emph{et al.} \cite{bhattcrossq} improved sample efficiency by properly using batch normalization and removing the target networks, Kokolakis \emph{et al.} \cite{kokolakis2023fixed} developed a critic-only RL algorithm for learning the solution to the steady-state Hamilton-Jacobi-Bellman equation in fixed-time, while Abel \emph{et al.} \cite{abel2024definition} provided a careful definition of the emerging concept of \emph{continual} RL as an evolution of the common view of learning as ``finding a solution'' to ``endless adaptation''.