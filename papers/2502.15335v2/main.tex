 % This must be in the first 5 lines to tell arXiv to use pdfLaTeX, which is strongly recommended.
\pdfoutput=1
% In particular, the hyperref package requires pdfLaTeX in order to break URLs across lines.

\documentclass[11pt]{article}

% Change "review" to "final" to generate the final (sometimes called camera-ready) version.
% Change to "preprint" to generate a non-anonymous version with page numbers.
\usepackage[preprint]{acl}

% Standard package includes
\usepackage{times}
\usepackage{latexsym}

% For proper rendering and hyphenation of words containing Latin characters (including in bib files)
\usepackage[T1]{fontenc}
% For Vietnamese characters
% \usepackage[T5]{fontenc}
% See https://www.latex-project.org/help/documentation/encguide.pdf for other character sets

% This assumes your files are encoded as UTF8
\usepackage[utf8]{inputenc}

% This is not strictly necessary, and may be commented out,
% but it will improve the layout of the manuscript,
% and will typically save some space.
\usepackage{microtype}

% This is also not strictly necessary, and may be commented out.
% However, it will improve the aesthetics of text in
% the typewriter font.
\usepackage{inconsolata}

%Including images in your LaTeX document requires adding
%additional package(s)
\usepackage{amsthm}
\usepackage{amsmath}
\usepackage{amssymb}
\usepackage{caption}
\usepackage{subcaption}
\usepackage{enumerate}
\usepackage{multirow}
\usepackage{graphicx} 
\usepackage{url}
\usepackage{booktabs}
\usepackage{tabularx}
\usepackage{colortbl}
\usepackage{algorithm}
\usepackage{algorithmic}
\usepackage{makecell}
\usepackage{tcolorbox}
\usepackage{enumitem}
\usepackage{xcolor}

\definecolor{color1}{HTML}{D0E2BF} 
\definecolor{color2}{HTML}{F7E6D8}

\newcommand{\xiang}[1]{\textcolor{red}{[Xiang: #1]}}


% If the title and author information does not fit in the area allocated, uncomment the following
%
%\setlength\titlebox{<dim>}
%
% and set <dim> to something 5cm or larger.

\title{Stepwise Informativeness Search \\
for Effective and Efficient LLM Reasoning}

% Author information can be set in various styles:
% For several authors from the same institution:
% \author{Author 1 \and ... \and Author n \\
%         Address line \\ ... \\ Address line}
% if the names do not fit well on one line use
%         Author 1 \\ {\bf Author 2} \\ ... \\ {\bf Author n} \\
% For authors from different institutions:
% \author{Author 1 \\ Address line \\  ... \\ Address line
%         \And  ... \And
%         Author n \\ Address line \\ ... \\ Address line}
% To start a separate ``row'' of authors use \AND, as in
% \author{Author 1 \\ Address line \\  ... \\ Address line
%         \AND
%         Author 2 \\ Address line \\ ... \\ Address line \And
%         Author 3 \\ Address line \\ ... \\ Address line}

\author{Siyuan Wang\textsuperscript{\rm 1}\footnotemark[1], Enda Zhao\textsuperscript{\rm 2}\thanks{~Equal contribution.}, 
\textbf{Zhongyu Wei}\textsuperscript{\rm 3}, \textbf{Xiang Ren}\textsuperscript{\rm 1} \\
% \thanks{~Corresponding author.}
\textsuperscript{\rm 1}University of Southern California,
\textsuperscript{\rm 2}Tsinghua University,
\textsuperscript{\rm 3}Fudan University. \\
\texttt{sw\_641@usc.edu; zed21@mails.tsinghua.edu.cn} \\
}

%\author{
%  \textbf{First Author\textsuperscript{1}},
%  \textbf{Second Author\textsuperscript{1,2}},
%  \textbf{Third T. Author\textsuperscript{1}},
%  \textbf{Fourth Author\textsuperscript{1}},
%\\
%  \textbf{Fifth Author\textsuperscript{1,2}},
%  \textbf{Sixth Author\textsuperscript{1}},
%  \textbf{Seventh Author\textsuperscript{1}},
%  \textbf{Eighth Author \textsuperscript{1,2,3,4}},
%\\
%  \textbf{Ninth Author\textsuperscript{1}},
%  \textbf{Tenth Author\textsuperscript{1}},
%  \textbf{Eleventh E. Author\textsuperscript{1,2,3,4,5}},
%  \textbf{Twelfth Author\textsuperscript{1}},
%\\
%  \textbf{Thirteenth Author\textsuperscript{3}},
%  \textbf{Fourteenth F. Author\textsuperscript{2,4}},
%  \textbf{Fifteenth Author\textsuperscript{1}},
%  \textbf{Sixteenth Author\textsuperscript{1}},
%\\
%  \textbf{Seventeenth S. Author\textsuperscript{4,5}},
%  \textbf{Eighteenth Author\textsuperscript{3,4}},
%  \textbf{Nineteenth N. Author\textsuperscript{2,5}},
%  \textbf{Twentieth Author\textsuperscript{1}}
%\\
%\\
%  \textsuperscript{1}Affiliation 1,
%  \textsuperscript{2}Affiliation 2,
%  \textsuperscript{3}Affiliation 3,
%  \textsuperscript{4}Affiliation 4,
%  \textsuperscript{5}Affiliation 5
%\\
%  \small{
%    \textbf{Correspondence:} \href{mailto:email@domain}{email@domain}
%  }
%}

\begin{document}
\maketitle
\begin{abstract}
Advances in Large Language Models (LLMs) have significantly improved multi-step reasoning through generating free-text rationales. However, recent studies show that LLMs tend to lose focus over the middle of long contexts. This raises concerns that as reasoning progresses, LLMs may overlook information in earlier steps when decoding subsequent steps, leading to generate unreliable and redundant rationales. To address this, we propose guiding LLMs to generate more accurate and concise step-by-step rationales by (1) proactively referencing information from underutilized prior steps, and (2) minimizing redundant information between new and existing steps. We introduce \textit{stepwise informativeness search}, an inference-time tree search framework incorporating two selection heuristics: grounding-guided selection which prioritizes steps paying higher attention over underutilized steps; and novelty-guided selection which encourages steps with novel conclusions. During rationale generation, we use a self-grounding strategy that prompts LLMs to explicitly reference relevant prior steps to provide premises before deduction at each step. Experimental results on four reasoning datasets demonstrate that our approach improves reasoning accuracy by generating higher-quality rationales with reduced errors and redundancy~\footnote{Code and data are available at \url{https://github.com/SiyuanWangw/Informativeness-Search}.}.
\end{abstract}

\section{Introduction}
Large Language Models (LLMs)~\cite{openai2023gpt4, team2023gemini} have shown remarkable performance in reasoning tasks through Chain-of-Thought (CoT)~\cite{wei2022chain} prompting, which elicits step-by-step rationales to derive answers. However, complex multi-step reasoning remains challenging, particularly for smaller-scale models~\cite{dziri2024faith}. 
\begin{figure}[th!]
    \centering
    \includegraphics[width=1.0\linewidth]{figures/illustration3.pdf}
    \caption{An example illustrating LLMs' difficulty in referencing early-step information (e.g., underutilization of [Step-2,4,5,6]), and the inclusion of redundant steps (e.g., repeated conclusions in [Step-5, 7]). The rightward red arrow indicates the focus is on generating [Step-8] with [Step 1-7] have been generated.
    % \textbf{Bottom-Right:} Among instances involving 8-9 reasoning steps from CLUTRR~\cite{sinha2019clutrr}, performance degrades as the maximum distance between referenced and referencing steps grows.
    % \xiang{to help readers grasp the idea faster, can annotate the steps for why they're selected or skipped -- e.g, ``unused" or ``redundant"}
    % \xiang{(0) to help people understand the situation better, need to indicate we're looking at the generation of step-8, while step 1-7 has been generated. Need to highlight in both figure and caption.
    % (1). ``repetitive deduction" change to "Redundant step". Can remove ``prior" in the ``underutilized step". Use singular tense for ``step".
    % (2). ``Thinking process" change to ``Generated reasoning steps".
    % (3). ``Incorrect derivation" change to ``Incorrect answer". 
    % (4) you didn't mark step 5 as ``redundant step" in the figure. 
    % (5) ``halluciated referencing" change to ``hallucinated reference"; I think you should rather indicate that the useful steps are not utilized, rather than saying ``incorrect inference" so as to motivate our key ideas.}
    }
    \label{fig:illustration}
\end{figure}
Recent advances in tree-search algorithms~\cite{wang2024math, yao2024tree, zhang2024accessing} improve this by generating step-level candidates~\footnote{A reasoning step in this paper refers to a sentence in generated rationales, delimited by the end-of-line token ``/n''.} and using scoring mechanisms to select the most promising ones iteratively, thereby improving overall generated rationales. However, they typically rely on domain-specific reward models or more powerful LLMs to assess candidate validity~\cite{luo2024improve}. 


Moreover, LLMs tend to focus on leading and recent contexts while losing attention in the middle~\cite{hsieh2024found}. As reasoning progresses, this causes difficulty in referencing useful intermediate conclusions from earlier steps when decoding subsequent ones, leading to unreliable and redundant rationales.
For example, in Fig.~\ref{fig:illustration}, [Step 2,4,5,6] provide useful information for deriving the final answer but are not effectively utilized. This results in redundant steps (e.g., [Step-7] and [Step-5] have repeated conclusions) and incorrect answer (e.g., [Step-8]). 
Consequently, LLMs risk getting trapped in repetitive reasoning loops~\cite{chen2024not} and generating unnecessarily lengthy rationales, increasing the likelihood of cumulative errors~\cite{furuta2024exposing}. 
% This prior step reference difficulty is further evident in the bottom right plot of Fig.~\ref{fig:illustration}, where performance degrades as the distance to referenced prior steps grows.

To address this, we propose to guide LLMs in generating more accurate and concise step-by-step rationales by (1) proactively referencing intermediate conclusions generated from underutilized steps, and (2) minimizing redundancy between new and existing steps. With higher-quality rationales generated, we can improve answer accuracy and reduce decoding costs.
Underutilized steps are those whose intermediate conclusions have been less frequently referenced before the current step, suggesting untapped potential to offer useful information for subsequence reasoning. 
% This is further supported by our statistical analysis of CLUTRR and GSM8K's annotated solutions that most earlier steps are typically referenced only once or at most twice in later reasoning. 
Meanwhile, reducing redundancy across steps can contribute novel information, enabling more efficient exploration of the reasoning space toward final answers.
% and explore reasoning space towards final answers more efficiently. 
% deducing new sub-conclusions.
% 统计一下每一步被后面utilize的次数;  实验加一个underexplored准确率统计
% GSM8K / CLUTRR
% 每一步被后面reference的次数,以及被reference的step距离()
% that selects potential next steps based on how well they build upon these underexplored prior steps.
 % with higher utility for subsequent reasoning


% We first introduce a \textit{self-grounding strategy} that elicits LLMs' inherent ability to identify and anchor relevant prior steps to provide necessary premises before deduction at each step. This explicit step-grounding process extend the possibility of connecting with distant underexplored steps by first specifying their step numbers, and also improves the relevance of both grounded and newly generated steps.
% Building on this, we further develop a \textit{utility-based search framework} that selects potential next steps based on how well they build upon these underexplored prior steps.
% By estimating each prior step's reference degree so far as its remaining utility, we introduce a \textit{utility-based search framework} that selects optimal steps from parallel-generated alternatives, prioritizing those derived from underexplored prior steps.


We introduce \textit{stepwise informativeness search}, an inference-time tree search framework that prioritizes steps based on informativeness, either from leveraging underutilized steps or generating novel content.
The framework follows a stepwise beam search paradigm~\cite{xie2024self}, generating a set of candidate steps in parallel at each iteration. Based on standard cumulative step-level likelihood, it incorporates two heuristics to guide candidate selection. (1) \textit{Grounding-guided selection} identifies underutilized steps by computing each step's reference degree so far to estimate its information gain for subsequent reasoning. Since LLMs naturally assign higher attention to their grounding context~\cite{zhang2023tell}, we prioritize candidate steps with the highest attention scores over underutilized steps.
% calculate attention scores of each candidate step over identified underutilized steps and prioritize those with the highest scores. 
(2) \textit{Novelty-guided selection} ranks candidates based on the novelty of their intermediate conclusions relative to prior steps. A trigram-based similarity measure filters out highly similar candidates.
% These two heuristics ensure the generation of utility-rich steps that contribute meaningfully to subsequent reasoning while reducing reasoning path redundancy.
% it leverages an \textit{exploration-guided selection} heuristic which first identifies underexplored prior steps with the highest utilities, by estimating each step's reference degree so far as its remaining utility. Since LLMs naturally assign higher attention score over their grounding context~\cite{zhang2023tell}, this heuristic then calculates attention scores of each candidate over identified underexplored steps and prioritizes those with highest scores. To mitigate redundant deductions with repetitive sub-conclusions across steps, we additional incorporate a \textit{novelty-guided selection} heuristic into our search framework. Candidates are selected based on the novelty of their sub-conclusions relative to prior steps, using trigram-based similarity measures to filter out highly similar alternatives. This ensures the generation of novel and utility-rich steps that contribute meaningfully to subsequent reasoning while reducing the distance to potential useful prior steps.

To prevent grounding-guided selection from focusing on irrelevant prior steps that may emerge during reasoning, we further introduce a \textit{self-grounding strategy}. This approach elicits LLMs' inherent ability to identify relevant prior steps to provide premises before deduction at each step. This process also extend the possibility of connecting with distant underutilized steps by first specifying their step numbers, and reinforcing the generation of well-supported new steps through explicit grounding.
% logical consistency of newly generated steps by enforcing explicit grounding.
% and also improves the relevance of grounded steps and strengthens the logical consistency of newly generated steps.
We implement our informativeness search framework both with and without self-grounding strategy. 
% To handle irrelevant steps that inevitably arise during greedy search, especially when reasoning contexts containing distracting information, we further introduce a self-grounding strategy. This strategy prompts LLMs to explicitly identify the relevant prior steps or queries that provide necessary premises for deduction at each step, improving the relevance of both grounded prior steps and newly generated steps for inference. 
% We experiment with four multi-step reasoning datasets: FOLIO and ProofWriter for deductive reasoning, and MMLU-Pro and GPQA-diamond for diverse complex reasoning tasks. 
Experimental results across four multi-step reasoning datasets demonstrate the effectiveness of both the informativeness search framework and self-grounding strategy when applied to LLMs of varying families and scales.

Overall, our framework can generate more effective solutions with improved accuracy and fewer tokens. Moreover, the two selection heuristics leverage the model's own outputs and attention scores to intrinsically guide step search, making the approach domain-agnostic and minimizing the need for exhaustive interactions with external scorers or self-evaluation at each decoding step. % Notably, our method can also integrate with other scoring mechanisms to further enhance performance. 

% \section{Preliminaries}
\section{Stepwise Beam Search for Reasoning}
% \subsection{Formulation of Stepwise Search} 
In this work, we formulate multi-step reasoning as a stepwise beam search process considering its generation parallelizability can accelerates search process~\cite{xie2024self}. This contrasts with another common tree-search practice, Monte Carlo Tree Search (MCTS) methods~\cite{feng2023alphazero,zhang2024rest}, which involve extensive rollout simulations and are computationally expensive.

Specifically, at each iteration, the model generates a set of reasoning steps in parallel, each delimited by a special end-of-line token ``/\/n''. A beam of the top $N$ steps are selected according to various criteria, where $N$ is the beam size. Unlike step-level evaluation, stepwise beam search ranks candidates by their cumulative rewards (e.g., likelihood) across the sequence generated so far.
% \xiang{I moved up this sentence to first overview what we're following. 
% For the following sentences about the background, let's refine to smooth out the transition.}
% Recent tree search methods, such as Monte Carlo Tree Search (MCTS)~\cite{feng2023alphazero,zhang2024rest} and stepwise beam search~\cite{xie2024self}, 
% enable both exploration and control of reasoning paths through divergent generation and guided selection. 

Formally, the generation of a reasoning sequence $R=[s_1, s_2, \dots, s_T]$ with $T$ steps is formulated as 
$$P(R=s_{1:T}|x)=\prod_{t} P(s_t|s_{1:t-1},x),$$
where $s_t$ is the $t$-th step and $x$ is the input query. Stepwise generation and selection are performed with beam size $N$ and sample size $k$ as follows: starting with $N$ sequences at step $t-1$, it generates $k$ continuations from $P(s_t|s_{1:t-1},x)$ for each sequence $s_{1:t-1}$, forming a candidate set $C_t$ containing $Nk$ reasoning chains of length $t$. The top $N$ sequences are then selected based on a scoring criteria $\phi(C_t, \mathrm{\gamma}(\cdot))=\{s^1, s^2, \dots, s^N\}$. $\phi$ is the selection function (e.g., $topk(\cdot)$) and $\mathrm{\gamma}(s_{1:t})$ evaluates the sequence so far $s_{1:t}$. Initially, given only an input $x$, we generate $Nk$ candidates.

% Subsequently, $N$ reasoning chains are sampled according to different search criteria as $\phi(C_t, \mathrm{score})=\{s^1, s^2, ...,s^N\}$, where $\phi$ is the sampling function and $\mathrm{score}(s^i_{t})$ evaluates $t$-th step, defaultly using the log-probability of the entire generated sequence $s^i_{1:t}$ up to step $t$. 

% This formulation allows us to evaluate and control the generation on step-level by applying several token-level decoding algorithms. For stepwise beam search with beam size $N$ and sample size $k$ where $k\ge N$, at the $t$-th step we sample $k$ sequences from $P(s^t|s^{1:t-1})$ for each beam in total $N$ beams, and we end up with $Nk$ reasoning chain for $s^{1:t}$ which forms the candidate set $C^t$. Then $N$ reasoning beams are sampled as $\Phi(C^t, \mathrm{Score})=\{s_1, s_2, ...,s_N\}$, where the sampling function $\Phi$ varies with different search criteria, $\mathrm{Score}(s^t)$ is the evaluated score for the $t$-th step using different search criteria. 

\begin{figure*}[th!]
    \centering
    \includegraphics[width=1.0\linewidth]{figures/Workflow.pdf}
    \caption{\textbf{Upper:} Overview of our informativeness search framework, illustrated with beam size of 1. Green diagonal-striped blocks represent selected steps while gray blocks are discarded. Cross marks indicate incorrect deductions, and the orange crosshatched block highlights a redundant step that may lead to errors. Italics illustrate our self-grounding strategy. \textbf{Bottom:} While previous methods would accept this redundant [Step-7] as logically valid, our framework filters it out based on its low novelty and poor grounding on underutilized steps.}
    \label{fig:workflow}
\end{figure*}

A standard scoring criteria is the cumulative likelihood of a sequence, defined as: $\mathrm{\gamma}_L(s_{1:t})=\log\prod_{t} P(s_t|s_{1:t-1}, x)$. Alternative scoring functions $\mathrm{\gamma}(s_{1:t})$ are employed in self-evaluation~\cite{xie2024self} and deductive beam search~\cite{zhu2024deductive}. 
The former prompts the backend LLM to provide a correctness score $\mathrm{\gamma}_c(s_t)$ to assess whether $s_{t}$ is correct given $s_{1:t-1}$, which is then combined with likelihood: $\mathrm{\gamma}_{E}(s_{1:t})=\log \prod_t P(s_t|s_{1:t-1}, x) \ \mathrm{\gamma}_c(s_t).$
The latter trains an external deductive verifier $f$ to assess whether each step $s_{t}$ is logically entailed by previous contexts, and replaces the sequence likelihood with a cumulative deductive score: 
$\mathrm{\gamma}_{D}(s_{1:t})=\prod_t f(\text{entails}|s_t, s_{1:t-1}, x).$
% \subsection{Various Selection Criteria}
% \xiang{(1) let's compress this down to a shorter paragraph and make the formula in-line to save space. We should aovid distracting readers to the main idea in the next section.
% (2). We should briefly touch on the limitations of these methods in each part, to transit into your method's section.
% }
% The former prompts the backend LLM to assess whether $s_{t}$ is correct given $s_{1:t-1}$, taking the probability of the positive label $y_c=1$ as its correctness score $\mathrm{score}_c(s_t)=P_{\mathbf{cls}}(\mathbf{y_c=1}| s_{1:t},x)$. The final score combines cumulative likelihood and correctness score up to step $t$ via multiplication.
% $$\mathrm{score}_{E}(s_{1:t})=\prod_t P(s_t|s_{1:t-1}, x) \ \mathrm{score}_c(s_t).$$
% \xiang{What'a the limitation? Your discussion about domain adaptivity should go here.}
% Deductive beam search trains an external deductive verifier $f$ to assess whether a reasoning step $s_{t}$ follows from previous contexts ($\mathbf{y_d=1}$). This method replaces the sequence likelihood with the cumulative deductive score:
% $$\mathrm{score}_{D}(s_{1:t})=\prod_t P_{\mathbf{f}}(\mathbf{y_d=1}|s_t, s_{1:t-1}, x).$$ 

While these methods improve performance, they require additional annotations or prompts to obtain domain-specific scoring models. They also incur interaction overhead by waiting for scorer response at each decoding step, yet failing to address aforementioned grounding and redundancy challenges.
% along with the annotation and computational overhead for training a reward model.

\section{Informativeness Search Framework}
Unlike iteration-based scoring functions described above, we introduce stepwise informativeness search framework with two scoring heuristics that utilize model's intrinsic outputs and attention scores. This reduces reliance on off-the-shelf scorers and iterative interactions during decoding. It prioritizes steps based on informativeness, assessed by grounding-guided and novelty-guided heuristics that determine whether new decoded steps ground on underutilized steps and generate novel content.
% enhance rationale generation by improving prior step utilization and reducing redundancy.
% . These heuristics  proactively encourage grounding underutilized steps and reduce redundancy when decoding new steps. 
% While utilizing model's intrinsic outputs and attention scores, they minimize reliance on off-the-shelf scorers and iterative interactions during decoding.

\subsection{Grounding-Guided Selection}
To ground each deduction upon underutilized steps to maximally leverage useful intermediate information, we design an algorithm to identify underutilized ones among all prior steps. The candidate sequences, denoted as $C_t=\{s_{1:t}^1, s_{1:t}^2, \dots, s_{1:t}^{Nk}\}$, are then evaluated and selected based on whether each current step $s^i_{t}$ is well derived from its corresponding underutilized steps.

\paragraph{Identifying Underutilized Steps} At each reasoning step, underutilized steps are those referenced less frequently up to that point, offering higher untapped potential for contributing information to subsequent reasoning. At the current step $s_t$, the immediately preceding step $s_{t-1}$ is by default considered underutilized since it represents the most recent addition to the reasoning path. For additional underutilized steps, we perform a backward traversal from step $s_{t-2}$ to $s_1$, calculating the reference degree of each step to assess its information gain to subsequent reasoning. 

Specifically, for each prior step $s_{j} \in \{s_{t-2}, ..., s_2, s_1\}$, we first extract its intermediate conclusion $c_j$ by segmenting it using special clause delimiters (e.g., ``so'', ``thus'' and commas). We then compare $c_j$ with each subsequent step $s_m\in \{s_{j+1}, \dots, s_{t-1}\}$ before the current step using a trigram-based similarity measure. The information gain of $s_{j}$ is computed as follows: 
\begin{align*}
    \mathrm{InfoGain}(s_{j}) = 1 - \max\limits_{m \in {j+1, \dots, t-1}}\mathrm{Sim}_{\mathrm{tri}}(c_{j}, s_{m})
\end{align*}
We classify a prior step as underutilized if its information gain exceeds a predefined threshold $\tau$. The set of underutilized steps at step $t$ is:
\begin{align*}
    \mathcal{I}_t = \{s_{t-1}\} \ \cup \ \{s_j \mid \mathrm{InfoGa}&\mathrm{in}(s_j) > \tau\}, \\
    &j \in \{1, \dots, t-2\} 
\end{align*}

\paragraph{Grounding on Underutilized Steps} 
After identifying the set of underutilized steps $\mathcal{I}^i_t$ for each candidate sequence $s^i_{1:t}$ in the candidate set $C_t=\{s^1, s^2, \dots, s^{Nk}\}$ (with subscripts omitted for simplicity), we prioritize candidates that more effectively ground their reasoning in $s^i_t$ upon their respective underutilized steps. 

LLMs typically assign higher attention scores to their grounding context~\cite{zhang2023tell}. We leverage attention scores to evaluate how well each candidate focuses on and utilizes its identified underutilized steps $\mathcal{I}^i_t$ when constructing step $s^i_t$ in the reasoning path. We specifically compute the attention score of $s^i_t$ over $\mathcal{I}^i_t$ as $\mathrm{\gamma}_a(s^i_t)$ by applying mean pooling across all tokens in $s^i_t$ and the highly attended tokens within $\mathcal{I}^i_t$. 
% Detailed Illustration: how, why
We then integrate this attention-based measure into the original cumulative likelihood scoring function to obtain an grounding-enhanced score:
% novelty-based scoring function to obtain the final score:
\begin{align*}
    \mathrm{\gamma}_G(s_{1:t}) & = \mathrm{\gamma}_{L}(s_{1:t}) + \alpha \cdot \mathrm{\gamma}_a(s_t)
\end{align*}
where $\mathrm{\gamma}_L(s_{1:t}) = \log \prod_t P(s_t|s_{1:t-1}, x)$ and $\alpha$ is a weighted hyperparameter. Then $N$ candidates are selected from $C_t=\{s^1, s^2, \dots, s^{Nk}\}$ with the highest $\mathrm{\gamma}_G(s_{1:t})$. We validate this attention-based operation in Sec.~\ref{sec:attention_validation} by analyzing the consistency between highly attended content and actual grounded information.

\subsection{Novelty-Guided Selection}
\label{sec:novelty-guidance}
To reduce redundancy across multiple intermediate steps, we assess the conclusion novelty of each newly generated step $s^i_t$ in a candidate sequence $s^i_{1:t}$, and select candidates with higher novelty.
We extract intermediate conclusions from $s^i_t$ and all its prior steps $\{s^i_1, \dots, s^i_{t-1}\}$ by segmenting the corresponding sentences using special clause delimiters (e.g., ``so'', ``thus'' and commas), forming a set of conclusions $\{c^i_1, \dots, c^i_{t-1}, c^i_{t}\}$. We then calculate the trigram-based similarity between the newly generated conclusion $c^i_{t}$ and all preceding conclusions $\{c^i_1, \dots, c^i_{t-1}\}$. The novelty score of $s^i_t$ is then obtained as follows:
\begin{align*}
    N(s^i_t) = 1 - \max\limits_{j \in {1, \dots, t-1}}\mathrm{Sim}_{\mathrm{tri}}(c^i_{t}, c^i_{j})
\end{align*}
where $\mathrm{Sim}_{\mathrm{tri}}(\cdot, \cdot)$ measures trigram-based similarity. To incorporate novelty into candidate selection, we calibrate the grounding-enhanced scoring function with novelty score. At step $t$, candidates with low-novelty conclusions (i.e., $N(s_t) \leq \theta$) are filtered out, retaining only diverse and meaningful candidates. The adjusted scoring function is defined as below, where $\theta$ is a predefined threshold.
\begin{align*}
    \mathrm{\gamma}_{N}(s_{1:t}) = \begin{cases} 
    \mathrm{\gamma}_G(s_{1:t}), & \text{if} N(s_t)>\theta, \\ 
    -100, &\text{otherwise}.
    \end{cases}
\end{align*}
% where $\mathrm{score}_L(s_{1:t}) = \prod_t P(s_t|s_{1:t-1}, x)$
% \mathrm{I}(N(s_t) > \theta) \prod_t P(s_t|s_{1:t-1}, x) 

\subsection{Self-Grounding Strategy}
To handle irrelevant steps that may arise during reasoning generation and prevent grounding-guided selection from focusing on irrelevant prior steps, especially when contexts contain distracting information, we introduce a self-grounding strategy. This approach leverages LLMs' inherent ability to anchor reasoning in relevant prior information, either from prior steps or the input query, that serve as necessary premises for each new deduction. The strategy explicitly prompts LLMs to reason step by step, structuring each step in the format:
$$
\texttt{``[Step-i] From <source>, <deduction>.''}
$$ 
where ``<source>'' refers to either relevant prior steps or the input query that provide premises for deducing new conclusions in ``<deduction>''. For example, ``\textit{[Step-1] From Query, we know ...}'',
``\textit{[Step-2] From Step-1 and Query, we know ...}'' and ``\textit{[Step-3] From Step-1 and Step-2, because ...}''.
This explicit step-grounding process ensures that each new step directly builds upon established information, maintaining logical coherence while minimizing irrelevant or unsupported conclusions. Moreover, explicitly referencing step numbers facilitates connections with distant underutilized steps. Further details on the prompts and few-shot demonstrations are provided in Appendix~\ref{prompts}.
% To handle irrelevant steps that inevitably arise during greedy search, especially when reasoning contexts containing distracting information, we further introduce a self-grounding strategy. This strategy prompts LLMs to explicitly identify the relevant prior steps or queries that provide necessary premises for deduction at each step, improving the relevance of both grounded prior steps and newly generated steps for inference. 

\begin{table*}[!th]
    \centering
    \setlength\tabcolsep{5pt}
    \resizebox{1.0\textwidth}{!}{
    \begin{tabular}{cc|cccc|c}
    \toprule
    Models & Methods & FOLIO & ProofWriter & MMLU-Pro & GPQA-Diamond & Avg.\\
    \bottomrule
    \addlinespace[1pt]
    \multirow{10}{*}{Llama3.2-3B-Instruct} & 
    \textit{Few-shot CoT} & 38.73\% & 40.00\% & 28.57\% & 21.72\% & 32.25\% \\
    & \textit{Self-Grounding CoT} & 45.59\% & 43.33\% & 28.57\% & 22.73\% & 35.06\% \\
    & \textit{Best-of-N} & 45.59\% & 37.00\% & 30.00\% & 22.73\%  & 33.83\%\\ 
    & \textit{Self-Consistency} & 46.57\% & 47.67\% & 29.64\% & 22.73\% & 36.65\% \\
    \addlinespace[1pt]
    \cline{2-7}
    \addlinespace[1pt]
    & \textit{Tree-of-Thought} & 44.12\% & 44.17\% & 26.43\% & 22.73\% & 34.36\%  \\
    & \textit{Self-Eval Beam Search} & 45.10\% & 47.00\% & 30.71\%& 19.19\%& 35.50\% \\
    & \textit{Deductive Beam Search} & 48.04\% & 38.17\% & 25.71\% & 24.75\% & 34.17\% \\ 
    & \textit{MCTS + Math-PRM} & / & / & 26.07\% & 22.22\% & / \\ 
    & \cellcolor[HTML]{E1E1E1} \textit{Informativeness Search} & \cellcolor[HTML]{E1E1E1} 46.57\% & \cellcolor[HTML]{E1E1E1} 50.33\% & \cellcolor[HTML]{E1E1E1} 33.57\% & \cellcolor[HTML]{E1E1E1} \bf 27.27\% & \cellcolor[HTML]{E1E1E1} 39.44\% \\
    & \cellcolor[HTML]{E1E1E1} \textit{Informativeness Search (w/ SG)} & \cellcolor[HTML]{E1E1E1} \bf 51.96\% & \cellcolor[HTML]{E1E1E1} \bf 53.67\% & \cellcolor[HTML]{E1E1E1} \bf 33.93\% & \cellcolor[HTML]{E1E1E1} 24.24\bf \% & \cellcolor[HTML]{E1E1E1} \bf 40.95\% \\
    \midrule
    \multirow{10}{*}{Llama3-8B-Instruct} & 
    \textit{Few-shot CoT} & 54.90\% & 55.33\% & 37.50\% & 29.29\% & 44.25\% \\
    & \textit{Self-Grounding CoT} & 55.39\% & 57.00\% & 38.57\% & 30.30\% & 45.32\% \\
    & \textit{Best-of-N} & 56.86\% & 50.00\% & 39.29\% & 30.30\% & 44.11\%\\ 
    & \textit{Self-Consistency} & 57.84\% & 60.17\% & 39.29\% & 31.31\% & 47.15\% \\
    \addlinespace[1pt]
    \cline{2-7}
    \addlinespace[1pt]
    & \textit{Tree-of-Thought} & 55.88\% & 53.33\% & 39.29\% & 27.78\% & 44.07\%  \\
    & \textit{Self-Eval Beam Search} & 59.31\% & 56.17\% & 35.00\% & 29.80\% & 45.07\% \\
    & \textit{Deductive Beam Search} & 54.90\% & 48.83\% & 37.50\% & 27.78\% & 42.25\% \\ 
    & \textit{MCTS + Math-PRM} & / & / & 27.14\% & 28.28\% & / \\ 
    & \cellcolor[HTML]{E1E1E1} \textit{Informativeness Search} & \cellcolor[HTML]{E1E1E1} 58.33\% & \cellcolor[HTML]{E1E1E1} 61.33\% & \cellcolor[HTML]{E1E1E1} 40.00\% & \cellcolor[HTML]{E1E1E1} 33.33\% & \cellcolor[HTML]{E1E1E1} 48.25\% \\
    & \cellcolor[HTML]{E1E1E1} \textit{Informativeness Search (w/ SG)} & \cellcolor[HTML]{E1E1E1} \bf 59.80\% & \cellcolor[HTML]{E1E1E1} \bf 62.00\% & \cellcolor[HTML]{E1E1E1} \bf 40.71\% & \cellcolor[HTML]{E1E1E1} \bf 35.35\% & \cellcolor[HTML]{E1E1E1} \bf 49.46\% \\
    \bottomrule
    \end{tabular}
    }
    \caption{Experimental results (accuracy \%) of different methods on Llama3.2-3B-Instruct and LLaMA3-8B-Instruct. \textit{SG} denotes the \textit{Self-Grounding} strategy. Shaded rows present results from our proposed method.}
    \label{overall_result}
\end{table*}
\section{Experiments}
\subsection{Setup}
\textbf{Datasets} We evaluate our framework on four multi-step reasoning datasets: FOLIO~\cite{han2022folio}, ProofWriter~\cite{tafjord2020proofwriter}, MMLU-Pro~\cite{wang2024mmlu} and GPQA~\cite{rein2023gpqa}. FOLIO and ProofWriter focus on deductive reasoning, requiring 1-8 and 1-6 reasoning steps respectively, with test sets of 204 and 600 cases. MMLU-Pro covers 14 domains, including math, physics, chemistry, engineering, law, and psychology, from which we uniformly sample 280 cases. GPQA specializes in biology, physics, and chemistry, and we use its Diamond subset containing 198 expert-answered but non-expert-failed questions.

\vspace{1mm}
\noindent\textbf{Baselines} We evaluate against both sequence-level CoT methods and step-level search methods. Sequence-level methods include: (1) Few-shot CoT~\cite{wei2022chain} performs step-by-step reasoning.
(2) Self-Grounding CoT is our proposed self-grounding strategy without search. 
(3) Best-of-N~\cite{lightman2023let} samples $Nk$ rationales and selects the best via LLM self-evaluation as we lack general reward models for diverse tasks. (4) Self-Consistency~\cite{wang2022self} samples $Nk$ rationales and uses majority voting for the final answer.
Step-level methods include: (5) Tree-of-thought~\cite{yao2024tree} performs breadth-first tree search with self-evaluation at each step.
(6) Self-Eval Beam Search~\cite{xie2024self} and (7) Deductive Beam Search~\cite{zhu2024deductive} both use stepwise beam search, with the former relying on self-evaluation and the latter on deductive scoring trained on synthesized datasets. (8) MCTS~\cite{zhang2024rest} where we use the minimum score across all steps from Qwen2.5-Math-PRM-7B~\cite{zhang2025lessons} to evaluate simulated solutions. As this is a mathematical PRM, we report MCTS results only on MMLU-Pro and GPQA-Diamond. We evaluate our informativeness search with and without the self-grounding (SG) strategy.

\vspace{1mm}
\noindent\textbf{Implementation Details} We evaluate our method and baselines on Llama3.2-3B-Instruct and Llama3-8B-Instruct, using a two-shot prompting strategy with a 1024-token generation limit. We set $N=3$ and $k=2$ for all stepwise beam search methods. The weighted parameter $\alpha$ is set to 2 and the threshold $\tau$ to 0.7. $\theta$ is set to 0.5 for FOLIO and ProofWriter, 0.4 for MMLU-Pro and GPQA-Diamond. Further details and search configurations are provided in Appendix~\ref{sec:implementation}. 
% For BoN and SC-CoT, we adopt a sampling configuration with temperature $T = 1.0$ and top-$40$ token truncation. % For candidate generation, we attempt both temperature sampling and tokenwise beam search, and the latter show better performance and accelerated computational speed. So we adopt tokenwise beam search for all tree-search methods in our reported table.


\begin{figure*}[th!]
    \centering
    \includegraphics[width=0.94\linewidth]{figures/avg_tokens.pdf}
    \caption{Accuracy and average token count (Avg. \# Tokens) of final predicted rationales using different methods on Llama3.2-3B-Instruct.}
    \label{fig:avg_tokens}
\end{figure*}


% \begin{table*}[!th]
%     \centering
%     \setlength\tabcolsep{4pt}
%     \resizebox{0.98\textwidth}{!}{
%     \begin{tabular}{ccccccccc}
%     \toprule
%     \multirow{2}{*}{Method} & \multicolumn{2}{c}{FOLIO} & \multicolumn{2}{c}{ProofWriter} & \multicolumn{2}{c}{MMLU-Pro} & \multicolumn{2}{c}{GPQA-Diamond} \\
%     & Acc. & \# Tokens \ \ & Acc. & \# Tokens \ \ & Acc. & \# Tokens \ \ & Acc. & \# Tokens \ \\
%     \bottomrule
%     \rowcolor[HTML]{E1E1E1} \multicolumn{9}{c}{\textit{w/o Self-Grounding}} \\
%     Few-shot CoT & 38.73\% & 261 / 302 & 40.00\% & 381 / 443 & 28.57\% & 320 / 371 & 21.72\% & 442 / 499  \\
%     Stepwise Beam Search & 41.18\% & 251 / 16660 & 46.33\% & 241 / 27655 & 30.36\% & 295 / 19933 & 23.23\% & 431 / 19465 \\ 
%     Utility-based Beam Search & 46.57\% & 208 / 9657 & 50.33\% & 210 / 18871 & 33.57\% &  274 / 16319 & 27.27\% & 422 / 18403  \\
%     \midrule
%     \rowcolor[HTML]{E1E1E1} \multicolumn{9}{c}{\textit{w/ Self-Grounding}} \\
%     Self-Grounding CoT & 45.59\% & 267 / 314 & 43.67\% & 365 / 477 & 28.57\% & 386 / 451 & 22.73\% & 503 / 572 \\
%     Stepwise Beam Search & 50.49\% & 258 / 9532 & 51.67\% & 289 / 16858 & 32.50\% & 332 / 13336 & 22.22\% & 500 / 17695\\ 
%     Utility-based Beam Search & 51.96\% & 248 / 8378 & 53.67\% & 268 / 13382 & 33.93\% & 324 / 12950 & 24.24\% & 457 / 15875\\
%     \bottomrule
%     \end{tabular}
%     }
%     \caption{Accuracy (Acc.) and token count (\# Tokens) of generated reasoning paths using different methods on Llama3.2-3B-Instruct.}
%     \label{token_num}
% \end{table*} 
\subsection{Main Results}
Table~\ref{overall_result} presents the overall performance comparison across four benchmark datasets. Our method consistently outperforms all baseline methods across both deductive and diverse reasoning datasets when implemented with either Llama3.2-3B-Instruct or Llama3-8B-Instruct. This demonstrates the general superiority of our informativeness search framework and self-grounding strategy. Notably, our method yields more substantial improvements on Llama3.2-3B-Instruct, suggesting its particular effectiveness in enhancing reasoning for lower-performing models. Additionally, self-grounding further enhances informativeness search, except when using Llama3.2-3B-Instruct on GPQA-Diamond. We attribute this to Llama3.2-3B-Instruct's inability to perform self-grounding effectively for the challenging GPQA-Diamond task. Step-level methods like tree-of-thought, deductive beam search and MCTS show moderate performance due to their reliance on specialized reward model or verifiers, limiting their generalizability. In contrast, informativeness search is broadly applicable without requiring task-specific customization. 
% With self-consistency as our main competitors, we further demonstrate our method's effectiveness by comparing their performance in an aggregation setting in Sec.~\ref{}.

% \begin{table}[!th]
%     \centering
%     \setlength\tabcolsep{3pt}
%     \resizebox{0.49\textwidth}{!}{
%     \begin{tabular}{c|cccc}
%     \toprule
%     Method &  FOLIO & ProofWriter & MMLU-Pro & GPQA-D \\
%     \midrule
%     Self-Consistency & 46.57\% & 47.67\% & 29.64\% & 22.73\% \\
%     % Self-Eval SC & 46.57\% & 49.00\% & 33.93\% & 20.71\% \\
%     % Ours & 48.04\% & \% & 31.79\% & 27.27\%\\
%     % Ours w/ SG & 50.98\% & \% & 33.93\% & 25.76\% \\
%     \rowcolor[HTML]{E1E1E1} \textit{Utility-based Search} & 46.08\% & 50.17\% & 33.93\% & \bf 27.27\% \\
%     \rowcolor[HTML]{E1E1E1} \textit{Utility-based Search (w/ SG)} & \bf 51.96\% & \bf 53.17\% & \bf 34.64\% & 24.24\bf \%  \\
%     Ours & 47.06\% & 49.83\% & 35.00\% & 27.27\%\\
%     Ours w/ SG & 52.45\% & \% & 33.93\% & 25.76\%\\
%     % \midrule
%     % Self-Consistency & 57.84\% & 60.17\% & 39.29\% & 31.31\% \\
%     % Self-Eval SC & 60.29\% & 54.00\% & 36.79\% & 31.82\% \\
%     % Ours & 76.67\% & 77.94\% & 33.21\%\\
%     \bottomrule
%     \end{tabular}
%     }
%     \caption{Majority voting results on top of.}
%     \label{majority_voting}
% \end{table}


\subsection{Efficiency Analysis}
\paragraph{Average Rationale Length} We analyze the average token count of final predicted rationales using different methods on Llama3.2-3B-Instruct to examine the relationship between rationale length and accuracy. As shown in Table~\ref{fig:avg_tokens}, our method generates shorter rationales with fewer tokens than few-shot CoT and stepwise beam search while achieving higher accuracy, both with and without the self-grounding strategy. Notably, our approach exhibits greater token reduction in deductive reasoning, correlating with more significant performance improvements. We attribute this to our informativeness search framework can effectively reduce redundancy by combining grounding-guided and novelty-guided selection. This minimizes cumulative errors and prevents circular reasoning loops, ultimately leading to better performance.
% Specifically, our method without self-grounding improves accuracy by 5$\sim$7\% on FOLIO while reducing the average token count by about 40 and achieves an 8\% improvement on ProofWriter with an average reduction of nearly 140 tokens. This demonstrates that our utility-based search framework combining exploration-guided and novelty-guided selection can generate more efficient reasoning paths with fewer tokens, which minimizes cumulative errors and leads to better performance.

\paragraph{Total Token Cost} We further analyze the total token consumption following~\cite{xie2024self}, including all candidate steps during the stepwise beam search process for all methods involving stepwise beam search. As shown in Table~\ref{fig:total_cost}, our method exhibits superior inference efficiency, reducing token usage compared to the baseline and other beam search methods. 
Specifically, both informativeness search and self-grounding progressively reduce token budget compared to baseline stepwise beam search. The high costs of self-eval and deductive beam search stem from additional interactions for obtaining evaluation feedback after each step. Moreover, deductive beam search requires additional computational resources for training a domain-specific deductive verifier. 
% This highlights the efficiency of our method in optimizing reasoning without relying on external verification mechanisms.
\begin{figure}
    \centering
    \includegraphics[width=0.93\linewidth]{figures/total_cost.pdf}
    \caption{Total token costs ($\times k$ tokens) of different stepwise beam search methods. Baseline refers to stepwise beam search using only cumulative likelihood scoring.}
    \label{fig:total_cost}
\end{figure}

\subsection{Results on Additional LLMs}
To further validate the broad effectiveness of our method, we implement it on Phi-4~\cite{abdin2024phi}, a 14B-parameter model from a different model family, and DeepSeek-R1-Distill-Llama-8B~\cite{guo2025deepseek}, a slow-thinking Llama3-8B variant distilled from DeepSeek-R1. We evaluate performance on FOLIO, ProofWriter, and MMLU-Pro, comparing against few-shot CoT, self-grounding, and self-consistency baselines using corresponding backbones. A one-shot prompting strategy is used with $N=3$ and $k=1$, and we extend the generation limit to 2048 tokens to accommodate long CoT from R1-Distill-Llama-8B.
As shown in Table~\ref{phi4-result}, our framework consistently improves performance on more powerful LLMs, though self-grounding fails on R1-Distill-Llama-8B, as it learns to generate free-form CoT and struggles to follow a structured response format. Despite this, our informativeness search still yields significant improvements, notably reducing redundant tokens in final rationales (Table~\ref{r1_llama_tokens}).
This aligns with DeepSeek-R1's over-thinking problems as pointed by~\cite{chen2024not, cuadron2025dangeroverthinkingexaminingreasoningaction}.
These results, along with Table~\ref{overall_result} demonstrate our method's robustness across models. 
% In contrast, self-consistency proves less effective when constrained by a limited number of sampled reasoning paths.
\begin{table}[!th]
    \centering
    \setlength\tabcolsep{2pt}
    \resizebox{0.48\textwidth}{!}{
    \begin{tabular}{c|ccc}
    \toprule
    Method &  FOLIO & ProofWriter & MMLU-Pro \\
    \midrule
    \rowcolor[HTML]{E1E1E1} \multicolumn{4}{c}{\textit{Phi-4}} \\
    % Few-shot CoT & 73.67\% & 71.57\% & 71.79\% \\
    % Self-Grounding & 73.50\% & 71.08\% & 72.14\% \\
    % Self-Consistency & 71.17\% & 71.57\% & 72.50\%  \\
    Few-shot CoT & 73.67\% & 72.55\% & 71.79\% \\
    Self-Grounding CoT & 73.50\% & 72.06\% & 72.14\% \\
    Self-Consistency & 71.17\% & 72.55\% & 72.50\%  \\
    \midrule
    Informativeness Search w/ SG & 76.67\% & 77.94\% & 72.86\%\\
    \midrule
    \rowcolor[HTML]{E1E1E1} \multicolumn{4}{c}{\textit{DeepSeek-R1-Distill-Llama-8B}} \\
    Few-shot CoT & 61.76\% & 48.67\% & 38.57\% \\
    Self-Grounding CoT & 53.92\% & 38.17\% & 35.36\% \\
    Self-Consistency & 62.25\% & 63.50\% & 46.07\%  \\
    \midrule
    Informativeness Search & 70.10\% & 66.50\% & 47.50\%\\
    \bottomrule
    \end{tabular}
    }
    \caption{Results on Phi-4 and R1-Distill-Llama-8B.}
    \label{phi4-result}
\end{table}
\begin{table}[!th]
    \centering
    \vspace{-3mm}
    \setlength\tabcolsep{3pt}
    \resizebox{0.45\textwidth}{!}{
    \begin{tabular}{c|ccc}
    \toprule
    Method &  FOLIO & ProofWriter & MMLU-Pro \\
    \midrule
    Few-shot CoT & 1105 & 1861 & 1636 \\
    Informativeness Search & 588 & 1023 & 1001 \\
    \bottomrule
    \end{tabular}
    }
    \caption{Average token count of the final predicted reasoning paths from R1-Distill-Llama-8B.}
    \label{r1_llama_tokens}
\end{table}

\section{Further Analysis} 

\subsection{Ablation Study}
% To investigate the effectiveness of different components in our method, we conduct an ablation study using LLama3.2-3B-Instruct on FOLIO and MMLU-Pro datasets. Taking stepwise beam search as our baseline, we test the impact of adding: (1) a novelty-guided selection heuristic, (2) a grounding-guided selection heuristic, and (3) both heuristics (which constitutes our informative search framework). We also test the impact of self-grounding strategy on both the baseline and our framework.

% As shown in Table~\ref{ablation_study}, incorporating both selection heuristics improve performance individually, while combining them in our informative search framework yields the best results. Notably, novelty-based selection proves especially effective on FOLIO, indicating that deductive reasoning is more susceptible to repetitive deductions. Furthermore, the self-grounding strategy consistently enhances performance for both the baseline and our framework, with more significant improvement on deductive reasoning where contexts contain verbally similar but redundant information.
% \begin{table}[!th]
%     \centering
%     \setlength\tabcolsep{3.5pt}
%     \resizebox{0.49\textwidth}{!}{
%     \begin{tabular}{lcc}
%     \toprule
%     Methods & FOLIO & MMLU-Pro \\
%     \midrule
%     \rowcolor[HTML]{E1E1E1} \multicolumn{3}{c}{\textit{w/o Self-Grounding}} \\
%     Stepwise Beam Search & 41.18\% & 30.36\% \\
%     \ \ \ \ \ \ \ w/ novelty-guided heuristic & 45.10\% & 32.14\% \\
%     \ \ \ \ \ \ \ w/ grounding-guided heuristic & 42.65\% & *31.79\%* \\  
%     Informative Search & 46.57\% & 33.57\% \\
%     \midrule
%     \rowcolor[HTML]{E1E1E1} \multicolumn{3}{c}{\textit{w/ Self-Grounding}} \\
%     Stepwise Beam Search & 50.49\% & 32.50\% \\
%     % Low-repeating Search & 61.17\% & 58.33\% & 40.36\% \\
%     % Informative Search & 57.67\% & 58.33\% & 40.00\% \\  
%     Informative Search & 51.96\% & 33.93\% \\
%     \bottomrule
%     \end{tabular}
%     }
%     \caption{Ablation study using LLaMA3.2-3B-Instruct on FOLIO and MMLU-Pro. 
%     % Informative search integrates \textit{novelty-guided} and \textit{grounding-guided} selection heuristics on top of stepwise beam search.
%     }
%     \label{ablation_study}
% \end{table}
To investigate the contribution of different components in our method, we conduct an ablation study using LLama3.2-3B-Instruct on FOLIO and MMLU-Pro datasets. Starting with stepwise beam search as our baseline, we progressively add: (1) novelty-guided selection heuristic, (2) grounding-guided selection heuristic (forming our \textit{Informativeness Search Framework}), and (3) self-grounding strategy (resulting in \textit{Informativeness Search w/ SG}). 
As shown in Table~\ref{ablation_study}, incorporating each selection heuristic and self-grounding strategy incrementally improves performance, finally yielding our best-performing informativeness search framework with self-grounding. Notably, novelty-based selection proves especially effective on FOLIO, suggesting that deductive reasoning is more susceptible to redundant step generation. Furthermore, self-grounding achieves more significant improvement on deductive reasoning where contexts contain verbally similar but irrelevant information.
\begin{table}[!th]
    % \vspace{-1mm}
    \centering
    \setlength\tabcolsep{4.5pt}
    \resizebox{0.48\textwidth}{!}{
    \begin{tabular}{lcc}
    \toprule
    Methods & FOLIO & MMLU-Pro \\
    \midrule
    % \rowcolor[HTML]{E1E1E1} \multicolumn{3}{c}{\textit{w/o Self-Grounding}} \\
    Stepwise Beam Search & 41.18\% & 30.36\% \\
    \ + Novelty-Guided Heuristic & 45.10\% & 32.14\% \\
    \ + Grounding-Guided Heuristic & 46.57\% & 33.57\% \\  
    % Informative Search & 46.57\% & 33.57\% \\
    % \midrule
    % \rowcolor[HTML]{E1E1E1} \multicolumn{3}{c}{\textit{w/ Self-Grounding}} \\
    % Stepwise Beam Search & 50.49\% & 32.50\% \\
    % Low-repeating Search & 61.17\% & 58.33\% & 40.36\% \\
    % Informative Search & 57.67\% & 58.33\% & 40.00\% \\  
    \ + Self-Grounding Strategy & 51.96\% & 33.93\% \\
    \bottomrule
    \end{tabular}
    }
    \caption{Ablation study using LLaMA3.2-3B-Instruct. 
    % Informative search integrates \textit{novelty-guided} and \textit{grounding-guided} selection heuristics on top of stepwise beam search.
    }
    \label{ablation_study}
\end{table}


% \begin{table*}[!th]
%     \centering
%     \setlength\tabcolsep{4pt}
%     \resizebox{0.98\textwidth}{!}{
%     \begin{tabular}{ccccccccc}
%     \toprule
%     \multirow{2}{*}{Method} & \multicolumn{2}{c}{FOLIO} & \multicolumn{2}{c}{ProofWriter} & \multicolumn{2}{c}{MMLU-Pro} & \multicolumn{2}{c}{GPQA-Diamond} \\
%     & Acc. & \# Tokens & Acc. & \# Tokens & Acc. & \# Tokens & Acc. & \# Tokens \\
%     \bottomrule
%     \rowcolor[HTML]{E1E1E1} \multicolumn{9}{c}{\textit{Llama3.2-3B-Instruct}} \\
%     \midrule
%     % & Few-shot CoT & 211 & 32.67\% & 221 & 249  \\
%     Stepwise Beam Search & 209/8.45 & & 194/9.28 & 240/8.29 \\ 
%     Utility-based Search & 170/8.0 & 47.33\% & 181/8.63 & 228/7.84  \\
%     \midrule
%     % & Self-grounding CoT & 198 & 42.00\% & 253 & 290 \\
%     Stepwise Beam Search (w/ SG) & 183/6.38 & & 207/6.84 & 241/7.19\\ 
%     Utility-based Search (w/ SG) & 177/6.31 & 52.33\% & 199/6.69 & 237/7.13 \\
%     \bottomrule
%     \rowcolor[HTML]{E1E1E1} \multicolumn{9}{c}{\textit{Llama3-8B-Instruct}} \\
%     \midrule
%     % & Few-shot CoT & 136 & 55.33\% & 110 & 176 \\
%     Stepwise Beam Search & 135/6.98 & & 148/7.18 & 180/8.46 \\ 
%     Utility-based Search & 130/6.75 & 60.67\% & 148/7.42 & 171/8.27 \\
%     \midrule
%     % & Self-grounding CoT & 181 & 57.00\% & 155 & 194 \\
%     Stepwise Beam Search (w/ SG) & 185/6.85 & & 136/5.41 & 214/8.80 \\ 
%     Utility-based Search (w/ SG) & 174/6.25 & \bf 62.67\% & 132/5.35 & 200/8.42  \\
%     \bottomrule
%     \end{tabular}
%     }
%     \caption{Experimental results (accuracy \%) of different methods on top of Llama3.2-3B-Instruct and LLaMA3-8B-Instruct.}
%     \label{overall_result}
% \end{table*}


\subsection{Redundant Step Analysis}
In complex multi-step reasoning tasks, LLMs tend to generate repeated intermediate conclusions, either from same or different premises, which can trap reasoning in circular loops. For detailed investigation, we measure the average number of repeated conclusions across steps per rationale generated by our method compared to few-shot CoT and self-grounding CoT baselines using LLama3.2-3B-Instruct. Specifically, we split rationales into steps using end-of-line token ``/\/n'' and extract intermediate conclusions based on special clause delimiters as operated in Sec.~\ref{sec:novelty-guidance}. A step is considered redundant if its conclusion shares over 70\% tri-word overlap with any previous conclusions in the same rationale. 
As shown in Figure~\ref{fig:repetitive_deduction}, LLMs exhibit a pronounced tendency to produce redundant steps, particularly in
deductive reasoning tasks. This occurs because deductive contexts often contain verbally similar information, causing LLMs to lose
\begin{figure}[ht]
    \centering
    \includegraphics[width=0.94\linewidth]{figures/repetition.pdf}
    \caption{Average count of redundant steps whose conclusions have over 70\% tri-word overlap with any previous conclusions in the same rationale.}
    \vspace{-4mm}
    \label{fig:repetitive_deduction}
\end{figure}
track of logical progression and become stuck in circular reasoning. In contrast, our self-grounding strategy and informativeness search substantially reduce redundant steps, enabling more effective and efficient multi-step reasoning.

\subsection{Validity of Attention-Based Selection}
\label{sec:attention_validation}
To validate our attention-based implementation in grounding-guided selection, we examine whether LLMs naturally assign higher attention to grounded steps than other steps. Using the CLUTRR dataset, which provides well-annotated reasoning paths, we conduct a teacher-forcing analysis where all previous ground-truth steps are fed into the model to prompt the next step. We then compute the average attention score over both grounded and non-grounded steps. This analysis is performed both with and without self-grounding, using Llama3.2-3B-Instruct and Llama3-8B-Instruct. As shown in Fig.~\ref{fig:grounded_scores}, LLMs exhibit significantly higher attention over grounded steps. 
% Statistically, for Llama3-8B-Instruct, 94.46\% (without self-grounding) and 97.26\% (with self-grounding) of instances show higher attention on grounded steps than others. For Llama3.2-3B-Instruct, the figures are 94.31\% and 95.67\%. 
This demonstrates the consistency of LLMs' attention patterns and their grounding behavior, and confirms the validity of our attention-based implementation.
\begin{figure}[ht]
    \centering
    \includegraphics[width=0.98\linewidth]{figures/attention.pdf}
    \caption{Average attention on grounded and other steps.}
    \label{fig:grounded_scores}
\end{figure}


% \begin{table}[!th]
%     \centering
%     \setlength\tabcolsep{4pt}
%     \resizebox{0.47\textwidth}{!}{
%     \begin{tabular}{c|ccc}
%     \toprule
%     Method &  ProofWriter & FOLIO & MMLU-Pro \\
%     \midrule
%     % Few-shot CoT & 73.67\% & 71.57\% & 71.79\% \\
%     % Self-Grounding & 73.50\% & 71.08\% & 72.14\% \\
%     % Self-Consistency & 71.17\% & 71.57\% & 72.50\%  \\
%     Few-shot CoT & 73.67\% & 72.55\% & 71.79\% \\
%     Self-Grounding & 73.50\% & 72.06\% & 72.14\% \\
%     Self-Consistency & 71.17\% & 72.55\% & 72.50\%  \\
%     \midrule
%     Ours & 76.67\% & 77.94\% & 72.86\%\\
%     \bottomrule
%     \end{tabular}
%     }
%     \caption{Results on top of Phi-4. ``Ours'' refers to our integrated approach combining utility-based beam search and self-grounding strategy.}
%     \label{phi4-result}
% \end{table}

% \begin{table}[!th]
%     \centering
%     \setlength\tabcolsep{3pt}
%     \resizebox{0.47\textwidth}{!}{
%     \begin{tabular}{c|ccc}
%     \toprule
%     Method & FOLIO & ProofWriter & MMLU-Pro \\
%     \midrule
%     Few-shot CoT & 61.76\% & 48.67\% & 39.29\% \\
%     Self-Grounding & 53.92\% & 38.17\% & 35.36\% \\
%     Self-Consistency & 62.25\% & 63.50\% & 46.07\%  \\
%     \midrule
%     % Step Beam Search & 64.71\% & & 43.93\% \\
%     Ours & 69.12\% & 64.83\% & 47.86\%\\
%     % Ours + SC & 69.61\% & 65.33\% & 47.86\%\\
%     \bottomrule
%     \end{tabular}
%     }
%     \caption{Results on top of DeepSeek-R1-Distill-Llama-8B. ``Ours'' refers to our utility-based beam search.}
%     \label{r1-result}
% \end{table}


% \subsection{Integration with Other Criteria}

\section{Related Work}
LLMs~\cite{openai2023gpt4, touvron2023llama, abdin2024phi, guo2025deepseek} have demonstrated remarkable performance across diverse tasks. Chain-of-Thought (CoT)~\cite{wei2022chain, zhou2022least} prompting has emerged as an effective strategy for generating step-by-step rationale to derive answers. However, for complex multi-step reasoning problems, LLMs often underutilize critic information from earlier steps as rationale getting longer due to their tendency to lose focus on middle-context information~\cite{peysakhovich2023attention, junqing2023never, hsieh2024found}. Additionally, they frequently generate redundant steps with repeated conclusions, leading to repetitive reasoning loops and error accumulation~\cite{dziri2024faith, furuta2024exposing}. These difficulties are especially pronounced in smaller-scale LLMs with limited reasoning capacity~\cite{fu2023specializing}.
An intuitive method is to prompt LLMs for more concise outputs. However, LLMs often struggle to maintain output quality under length constraints, and simple prompting alone fails to resolve grounding and redundancy issue~\cite{nayab2024concise, han2024token}. 

This inspires generating multiple rationales and determine the most likely solution using majority voting~\cite{wang2022self} or best-of-N~\cite{wang2024math}. However, they are computationally expensive due to the exponentially growing search space when integrating diverse solutions.
To reduce the search space, recent studies have applied tree search techniques with scoring mechanisms to prioritize promising candidates at each step, such as stepwise beam search~\cite{xie2024self}, Tree-of-Thought~\cite{yao2024tree}, and Monte Carlo Tree Search~\cite{jiang2024technical, feng2023alphazero, zhang2024rest}. While effective, they face practical limitations, relying on extensive rollouts~\cite{wang2024math, wang2024q} and intensive annotations~\cite{lightman2023let} for training specialized reward models. Additionally, they introduce latency due to interactions with external or self-evaluators during autoregressive decoding~\cite{xie2024self, yao2024tree}, and fail to address the grounding and redundancy issues we focus on in this work.

\section{Conclusion}
In this work, we address the challenge of LLMs losing focus on intermediate steps during multi-step reasoning, which can lead to unreliable and redundant rationales. To mitigate this issue, we propose an inference-time tree search framework incorporating grounding-guided and novelty-guided selection heuristics, that enhances rationale generation by proactively grounding underutilized prior steps and minimizing redundant conclusions between reasoning steps. We additionally employ a self-grounding strategy, prompting LLMs to explicitly reference relevant prior steps before making deductions. Experimental results demonstrate that our method improves reasoning accuracy by generating higher-quality rationales with fewer errors and reduced redundancy. 

\section*{Limitations}
Our work has several limitations to address in future research. 
First, our experiments primarily focus on four multi-step reasoning datasets covering deductive and diverse-discipline reasoning. Expanding to a broader range of tasks and datasets will further validate our framework’s effectiveness.
Second, due to computational constraints, our main experiments operate within a limited search space with beam size 3 and sample size 2, and use LLM backbones of at most 14B parameters. Future work can explore larger search spaces and more powerful LLMs to further unlock the potential of our framework. Finally, while our method currently relies solely on stepwise beam search with standard cumulative likelihood, incorporating our selection heuristics with other scoring mechanism, such as self-evaluation and process reward models, as well as other tree-search algorithms like MCTS could be potential future work.


% Bibliography entries for the entire Anthology, followed by custom entries
%\bibliography{anthology,custom}
% Custom bibliography entries only
\bibliography{custom}

\appendix

\newpage
\section{Implementation Details}
\label{sec:implementation}

\subsection{Baseline Details}
For Best-of-N and Self-Consistency, we adopt a sampling configuration with temperature $T = 1.0$ and top-$40$ token truncation. For tree-of-thought (ToT) and self-eval beam search (Self-Eval BS), we prompt LLMs to conduct self-evaluation. For deductive beam search that provide a general verifier checkpoint and two data subsets for training a commonsense and a mathematical verifier, we select the best-performing verifier for each dataset. Specifically, we use the general or commonsense verifier for FOLIO, ProofWriter, and MMLU-Pro, and the general or mathematical verifier for GPQA.
For MCTS which operates in a iterative four-stage manner: selection, expansion, simulation and backprogation, we use the minimum score across all steps from Qwen2.5-Math-PRM-7B~\cite{zhang2025lessons} to evaluate simulated rollout. 

\subsection{Varying Search Configurations}
For step-level candidate generation in stepwise beam search, we explore both temperature sampling and tokenwise beam search. 
% The performance with and without majority voting aggregation are shown in Table~\ref{candidate_generation}. 
As shown in Table~\ref{candidate_generation}, our method with grounding and novelty-guided selection consistently outperforms stepwise beam search baseline (with cumulative likelihood scoring), regardless of whether self-grounding is applied. Additionally, tokenwise beam search for candidate generation yields slightly better performance than temperature sampling. 
% However, majority voting does not always enhance performance with tokenwise beam search whereas it generally enhances results with temperature sampling. This is because tokenwise beam search deterministically selects top candidates with the highest probabilities. Therefore, we use tokenwise beam search when majority voting is not applied and temperature sampling when it is.
% Therefore, we adopt tokenwise beam search for all stepwise beam search methods in our reported results (Table~\ref{overall_result}$\sim$~\ref{r1-result}) considering its better performance and accelerated computational speed.
% But self-consistency
% \begin{table}[!th]
%     \centering
%     \setlength\tabcolsep{4.5pt}
%     \resizebox{0.48\textwidth}{!}{
%     \begin{tabular}{lcc}
%     \toprule
%     Methods & FOLIO & MMLU-Pro \\
%     \midrule
%     \rowcolor[HTML]{E1E1E1} \multicolumn{3}{c}{\textit{Beam Search}} \\
%     Stepwise Beam Search & 41.18\% & 30.36\% \\
%     Utility-based Beam Search & 46.57\% & 33.93\% \\
%     Stepwise Beam Search (w/ SG) & 50.49\% & 32.86\% \\
%     Utility-based Beam Search (w/ SG) & 51.47\% & 34.64\% \\
%     \midrule
%     \rowcolor[HTML]{E1E1E1} \multicolumn{3}{c}{\textit{Temperature Sampling}} \\
%     Stepwise Beam Search & 41.67\% & 29.64\% \\
%     Utility-based Beam Search & 47.55\% & 31.43\% \\
%     Stepwise Beam Search (w/ SG) & 44.61\% & 29.64\% \\
%     Utility-based Beam Search (w/ SG) & 47.06\% & 32.50\% \\
%     \midrule
%     \rowcolor[HTML]{E1E1E1} \multicolumn{3}{c}{\textit{Self-Consistency}} \\
%     Stepwise Beam Search & 41.18\% & 31.79\% \\
%     Utility-based Beam Search & 48.04\% & 31.79\% \\
%     Stepwise Beam Search (w/ SG) & 44.61\% & 30.36\% \\
%     Utility-based Beam Search (w/ SG) & 48.04\% & 33.93\% \\
%     \bottomrule
%     \end{tabular}
%     }
%     \caption{Different candidate step generation methods.}
%     \label{ablation_study}
% \end{table}
\begin{table}[!th]
    \centering
    \setlength\tabcolsep{4pt}
    \resizebox{0.49\textwidth}{!}{
    \begin{tabular}{lcc}
    \toprule
    Methods & FOLIO & MMLU-Pro \\
    \midrule
    \rowcolor[HTML]{E1E1E1} \multicolumn{3}{c}{\textit{Beam Search}} \\
    Stepwise Beam Search & 41.18\% & 30.36\% \\
    Informativeness Search & 46.57\% & 33.57\% \\
    Stepwise Beam Search (w/ SG) & 50.49\% & 32.86\% \\
    Informativeness Search (w/ SG) & 51.96\% & 33.93\% \\
    % \rowcolor[HTML]{E1E1E1} \multicolumn{3}{c}{\textit{Beam Search w/ SC}} \\
    % Stepwise Beam Search & 44.12\% & 31.43\% \\
    % Utility-based Beam Search & 44.12\% & 32.86\% \\
    % Stepwise Beam Search (w/ SG) & 50.98\% & 32.50\% \\
    % Utility-based Beam Search (w/ SG) & 52.45\% & 33.21\% \\
    \midrule
    \rowcolor[HTML]{E1E1E1} \multicolumn{3}{c}{\textit{Temperature Sampling}} \\
    Stepwise Beam Search & 41.67\% & 29.64\% \\
    Informativeness Search & 44.12\% & 31.43\% \\
    Stepwise Beam Search (w/ SG) & 47.55\% & 29.64\% \\
    Informativeness Search (w/ SG) & 48.53\% & 32.50\% \\
    % \rowcolor[HTML]{E1E1E1} \multicolumn{3}{c}{\textit{Temperature Sampling w/ SC}} \\
    % Stepwise Beam Search & 41.18\% & 31.79\% \\
    % Utility-based Beam Search & 48.04\% & 31.79\% \\
    % Stepwise Beam Search (w/ SG) & 47.55\% & 30.36\% \\
    % Utility-based Beam Search (w/ SG) & 50.98\% & 33.93\% \\
    \bottomrule
    \end{tabular}
    }
    \caption{Different candidate step generation methods.}
    \label{candidate_generation}
\end{table}

We further evaluate the impact of varying beam sizes in our informativeness search, using both tokenwise beam search and temperature sampling for candidate step generation. 
Specifically, we set the sample size to 2 and vary the beam size from 1 to 4. As shown in Fig.~\ref{fig:beam_size}, both alternatives consistently outperform the few-shot CoT baseline. Additionally, our informativeness search continues to improve as beam size increases. Notably, when the search space is constrained (i.e., with a smaller beam size), tokenwise beam search performs better. Based on these findings, we adopt tokenwise beam search for all stepwise beam search methods in our reported results  (Table~\ref{overall_result}$\sim$~\ref{r1_llama_tokens}) considering its better performance and accelerated computational speed.
% 50  50.49  51.47  51.96
% 45.59 50.00 50.98  50

% 45.59  47.55  48.53  52.45  
% 45.59  45.59  46.57  49.02
\begin{figure}
    \centering
    \includegraphics[width=0.96\linewidth]{figures/beam_size.pdf}
    \caption{The impact of beam size on our utility-based search for the FOLIO dataset on Llama3.2-3B-Instruct.}
    \label{fig:beam_size}
\end{figure}

\subsection{Comparison to Tokenwise Beam Search}
We further compare our informativeness search (beam size $N=3$, sample size $k=2$) with naive tokenwise beam search for whole rationale generation using beam size 3 and 6. Table~\ref{comparison_bs} demonstrate the effectiveness of our method.
\begin{table}[!th]
    \centering
    \setlength\tabcolsep{2pt}
    \resizebox{0.5\textwidth}{!}{
    \begin{tabular}{c|cccc}
    \toprule
    Method &  FOLIO & ProofWriter & MMLU-Pro & GPQA-D \\
    \midrule
    Few-shot CoT & 38.73\% & 40.00\% & 28.57\% & 21.72\% \\
    Tokenwise BS (3) & 43.63\% & 45.00\% & 28.93\% & 21.72\% \\
    Tokenwise BS (6) & 46.08\% & 42.17\% & 31.07\% & 19.19\%\\
    \rowcolor[HTML]{E1E1E1} Informativeness Search & 46.57\% & 50.33\% & 33.93\% & 27.27\% \\
    \bottomrule
    \end{tabular}
    }
    \caption{Comparison with tokenwise beam search using Llama3.2-3B-Instruct for whole rationale generation. Numbers in parentheses denote the beam size.}
    \label{comparison_bs}
\end{table}
% \begin{table*}
% \centering
% \resizebox{0.98\textwidth}{!}{
% \begin{tcolorbox}[colback=blue!5!white,colframe=black!55!black,width=0.98\textwidth,title={Prompt for Self-Evaluation in ToT (FOLIO)}]
% \small
% You are an expert evaluator for logical reasoning in the context of formal logic problems. Your task is to analyze the provided reasoning steps and classify them based on their **correctness** (logical validity and factual accuracy) and **effectiveness** (clarity and relevance in addressing the query).  \\
% \\
% The reasoning steps may consist of multiple lines, where each line represents an individual step in the reasoning process. Your evaluation should consider the reasoning as a whole and provide a single overall judgment.  \\
% \\
% Here are some examples to guide your evaluation:  \\
% \\
% \#\#\# Example 1:  \\
% Query: ... \\  
% Thought: ... \\  
% Overall Evaluation:  \\
% Correct and Effective  \\
% (Reason: The steps are logically valid, factually correct, and clearly lead to the conclusion.)  \\
% \\
% \#\#\# Evaluation Options:  \\
% 1. **Correct and Effective**: All steps are logically valid and factually correct, and the reasoning effectively addresses the query.  \\
% 2. **Correct but Ineffective**: The steps are logically valid but fail to fully address the query or provide a clear resolution.  \\
% 3. **Partially Correct**: Some steps are correct, but others contain logical or factual errors.  \\
% 4. **Incorrect**: The reasoning contains significant logical or factual errors that render it invalid.  \\
% \\
% \#\#\# Your task:  \\
% Evaluate the reasoning below based on its correctness and effectiveness.  \\
% \\
% \{input\}\\
% \\
% Overall Evaluation:
% % According to commonsense knowledge in reality, please list 5 predicates between the given two objects to describe the \{\underline{object affordance}\}. \\
% % Examples: \\
% % Object: Show, Artwork  \\
% % Predicate: CanBeAdaptedFrom(Show X, Artwork Y) \vspace{2mm}\\
% % \textbf{Object}: \{\underline{object$_1$}\}, \{\underline{object$_2$}\} \\
% % \textbf{Predicate}:
% \end{tcolorbox}
% }
% % \vspace{-3mm}
% \caption{The prompt for self-evaluation in ToT on FOLIO.}
% \label{prompt_self_eval_tot_folio}
% \end{table*}

% \begin{table*}
% \centering
% \resizebox{0.98\textwidth}{!}{
% \begin{tcolorbox}[colback=blue!5!white,colframe=black!55!black,width=0.98\textwidth,title={Prompt for Self-Evaluation in ToT (ProofWriter)}]
% \small
% You are an expert evaluator for logical reasoning in formal logic and proof validation. Your task is to analyze the provided reasoning steps and classify them based on their **correctness** (logical validity and adherence to the provided rules) and **effectiveness** (clarity and relevance in addressing the query).  \\
% \\
% The reasoning steps may consist of multiple lines, where each line represents an individual step in the reasoning process. Your evaluation should consider the reasoning as a whole and provide a single overall judgment. \\
% \\
% Here are some examples to guide your evaluation:  \\
% \\
% \#\#\# Example 1:  \\
% Query: ...\\ 
% Thought: ...\\
% Overall Evaluation:  \\
% Correct and Effective  \\
% (Reason: ...)  \\
% \\
% \#\#\# Evaluation Options:  \\
% 1. **Correct and Effective**: All steps are logically valid and factually correct, and the reasoning effectively addresses the query.  \\
% 2. **Correct but Ineffective**: The steps are logically valid but fail to fully address the query or provide a clear resolution.  \\
% 3. **Partially Correct**: Some steps are correct, but others contain logical or factual errors. \\
% 4. **Incorrect**: The reasoning contains significant logical or factual errors that render it invalid.  \\
% \\
% \#\#\# Your task:  \\
% Evaluate the reasoning below based on its correctness and effectiveness.  \\
% \\
% \{input\}  \\
% \\
% Overall Evaluation:
% \end{tcolorbox}
% }
% % \vspace{-3mm}
% \caption{The prompt for self-evaluation in ToT on ProofWriter.}
% \label{prompt_self_eval_tot_proofwriter}
% \end{table*}

% \begin{table*}
% \centering
% \resizebox{0.98\textwidth}{!}{
% \begin{tcolorbox}[colback=blue!5!white,colframe=black!55!black,width=0.98\textwidth,title={Prompt for Self-Evaluation in ToT (MMLU-Pro)}]
% \small
% You are an expert evaluator for logical reasoning in complex problems. Your task is to analyze the provided reasoning steps and classify them based on their **correctness** (logical and factual accuracy) and **effectiveness** (relevance and utility in solving the problem).  \\
% \\
% The reasoning steps may consist of multiple lines, where each line represents an individual step in the thought process. Your evaluation should consider the reasoning as a whole and provide a single overall judgment.  \\
% \\
% Here are some examples to guide your evaluation:  \\
% \\
% \#\#\# Example 1:\\
% Query: ...\\
% Options: ...\\
% Thought: ...\\  
% Overall Evaluation:\\
% Correct and Effective\\  
% (Reason: ...)  \\
% \\
% \#\#\# Evaluation Options:\\
% 1. **Correct and Effective**: All steps are logically correct and directly solve the problem.  \\
% 2. **Correct but Ineffective**: The steps are logically correct but fail to solve the problem or address it effectively.  \\
% 3. **Partially Correct**: Some steps are correct, but others contain errors or inaccuracies.  \\
% 4. **Incorrect**: The steps contain significant logical or factual errors that make them invalid.  \\
% \\
% \#\#\# Your task:\\
% Evaluate the reasoning below based on its correctness and effectiveness.\\
% \\
% \{input\} \\
% \\
% Overall Evaluation:
% \end{tcolorbox}
% }
% % \vspace{-3mm}
% \caption{The prompt for self-evaluation in ToT on MMLU-Pro.}
% \label{prompt_self_eval_tot_mmlu}
% \end{table*}

% \begin{table*}
% \centering
% \resizebox{0.98\textwidth}{!}{
% \begin{tcolorbox}[colback=blue!5!white,colframe=black!55!black,width=0.98\textwidth,title={Prompt for Self-Evaluation in ToT (GPQA-Diamond)}]
% \small
% You are an expert evaluator for mathematical reasoning and problem-solving tasks. Your task is to analyze the provided reasoning steps and classify them based on their **correctness** (mathematical accuracy and logical validity) and **effectiveness** (clarity and relevance in addressing the query).  \\
% \\
% The reasoning steps may consist of multiple lines, where each line represents an individual step in the problem-solving process. In some cases, the reasoning may only include a portion of the full solution. Your evaluation should consider the reasoning as a whole and provide a single overall judgment.  \\
% \\
% Here are some examples to guide your evaluation:  \\
% \\
% \#\#\# Example 1:  \\
% Query: ...\\ 
% Options: ...\\
% Thought: ...\\ 
% Overall Evaluation:  \\
% **Correct and Effective**  \\
% (Reason: ....)  \\
% \\
% \#\#\# Evaluation Options:  \\
% 1. **Correct and Effective**: All steps are mathematically accurate and logically valid, and the reasoning effectively addresses the query.  \\
% 2. **Correct but Ineffective**: The steps are mathematically valid but fail to fully address the query or provide a clear resolution.  \\
% 3. **Partially Correct**: Some steps are correct, but others contain mathematical or logical errors.  \\
% 4. **Incorrect**: The reasoning contains significant mathematical or logical errors that render it invalid.  \\
% \\
% \#\#\# Your task:  \\
% Evaluate the reasoning below based on its correctness and effectiveness.  \\
% \\
% \{input\}  \\
% \\
% Overall Evaluation:
% \end{tcolorbox}
% }
% % \vspace{-3mm}
% \caption{The prompt for self-evaluation in ToT on GPQA-Diamond.}
% \label{prompt_self_eval_tot_gpqa}
% \end{table*}

% \begin{table*}
% \centering
% \resizebox{0.98\textwidth}{!}{
% \begin{tcolorbox}[colback=blue!5!white,colframe=black!55!black,width=0.98\textwidth,title={Prompt for Self-Evaluation in Self-Eval BS (FOLIO)}]
% \small
% Query: ...\\
% Thought:\\
% \{Step 1\}\\
% \# Is the above step of reasoning:\\
% \# (A) Correct\\
% \# (B) Incorrect\\
% \# The above step of reasoning is: (A)\\
% \{Step 2\}\\
% \# Is the above step of reasoning:\\
% \# (A) Correct\\
% \# (B) Incorrect\\
% \# The above step of reasoning is: (B), because ...
% \end{tcolorbox}
% }
% % \vspace{-3mm}
% \caption{The prompt for self-evaluation in Self-Eval on FOLIO.}
% \label{prompt_self_eval_self_eval_folio}
% \end{table*}

% \begin{table*}
% \centering
% \resizebox{0.98\textwidth}{!}{
% \begin{tcolorbox}[colback=blue!5!white,colframe=black!55!black,width=0.98\textwidth,title={Prompt for Self-Evaluation in Self-Eval BS (ProofWriter)}]
% \small
% Query: ...\\
% Thought:\\
% \{Step 1\}\\
% \# Is the above step of reasoning:\\
% \# (A) Correct\\
% \# (B) Incorrect\\
% \# The above step of reasoning is: (A)\\
% \{Step 2\}\\
% \# Is the above step of reasoning:\\
% \# (A) Correct\\
% \# (B) Incorrect\\
% \# The above step of reasoning is: (B), because ...
% \end{tcolorbox}
% }
% % \vspace{-3mm}
% \caption{The prompt for self-evaluation in Self-Eval on ProofWriter.}
% \label{prompt_self_eval_self_eval_proofwriter}
% \end{table*}

% \begin{table*}
% \centering
% \resizebox{0.98\textwidth}{!}{
% \begin{tcolorbox}[colback=blue!5!white,colframe=black!55!black,width=0.98\textwidth,title={Prompt for Self-Evaluation in Self-Eval BS (MMLU-Pro)}]
% \small
% Query: ...\\
% Options: ...\\
% Thought:\\
% \{Step 1\}\\
% \# Is the above step of reasoning:\\
% \# (A) Correct\\
% \# (B) Incorrect\\
% \# The above step of reasoning is: (A)\\
% \{Step 2\}\\
% \# Is the above step of reasoning:\\
% \# (A) Correct\\
% \# (B) Incorrect\\
% \# The above step of reasoning is: (B), because ...
% \end{tcolorbox}
% }
% % \vspace{-3mm}
% \caption{The prompt for self-evaluation in Self-Eval on MMLU-Pro.}
% \label{prompt_self_eval_self_eval_mmlu}
% \end{table*}

% \begin{table*}
% \centering
% \resizebox{0.98\textwidth}{!}{
% \begin{tcolorbox}[colback=blue!5!white,colframe=black!55!black,width=0.98\textwidth,title={Prompt for Self-Evaluation in Self-Eval BS (GPQA-Diamond)}]
% \small
% Query: ...\\
% Options: ...\\
% Thought: \\
% \{Step 1\}\\
% \# Is the above step of reasoning:\\
% \# (A) Correct\\
% \# (B) Incorrect\\
% \# The above step of reasoning is: (A)\\
% \{Step 2\}\\
% \# Is the above step of reasoning:\\
% \# (A) Correct\\
% \# (B) Incorrect\\
% \# The above step of reasoning is: (B), because ...
% \end{tcolorbox}
% }
% % \vspace{-3mm}
% \caption{The prompt for self-evaluation in Self-Eval on GPQA-Diamond.}
% \label{prompt_self_eval_self_eval_gpqa}
% \end{table*}


\section{Framework Prompts}
\label{prompts}
Table~\ref{prompt_no_grounding_folio},~\ref{prompt_no_grounding_proofwriter},~\ref{prompt_no_grounding_mmlu} and~\ref{prompt_no_grounding_gpqa} present the prompts used in our informativeness search framework without self-grounding strategy for the FOLIO, ProofWriter, MMLU-Pro and GPQA-Diamond datasets. For illustration, Table~\ref{prompt_grounding_gpqa} provides the prompt used in our informativeness search framework with self-grounding strategy on GPQA-Diamond~\footnote{We use GPT-4o and Claude to adjust prompts manually.}.


\section{Illustration of the Grounding Challenge}
We provide a detailed illustration of the challenge LLM face in  grounding prior reasoning steps. Specifically, we analyze all instances involving 8-9 reasoning steps from CLUTRR~\cite{sinha2019clutrr}, a dataset with well-annotated rationales. We evaluate the performance of Llama3-8B-Instruct
across instances with varying maximum distances between referencing and referenced steps. As shown in Fig.~\ref{fig:acc_distance}, performance degrades as the distance to the referenced prior steps grows. This demonstrate the inherent difficulty of grounding prior step, with longer distances (steps accumulating) making grounding progressively harder.
\begin{figure}[ht]
    \centering
    \includegraphics[width=0.98\linewidth]{figures/acc_distance.pdf}
    \caption{Accuracy versus maximum distance between referencing and referenced steps on CLUTRR.}
    \label{fig:acc_distance}
\end{figure}

\begin{table*}
\centering
\resizebox{0.98\textwidth}{!}{
\begin{tcolorbox}[colback=blue!5!white,colframe=black!55!black,width=0.98\textwidth,title={Prompt without Self-Grounding (FOLIO)}]
\small
You are a helpful assistant.\\
You will receive a query. Your task is to answer the query.\\
\\
\#\#\#\# Examples\\
Query: LanguageA is a universal language. If a universal language exists, then for every two people if they both know the same universal language they can communicate. Katya cannot communicate with Danil. Katya knows LanguageA. Based on the above information, is the following statement true, false, or uncertain? Danil knows LanguageA.\\
Thought: \\
Because LanguageA is a universal language, and if a universal language exists, then for every two people if they both know the same universal language they can communicate, so every two people that know LanguageA can communicate.\\
Because every two people that know LanguageA can communicate, and Katya knows LanguageA, so Katya can communicate with others that know LanguageA.\\
Because Katya can communicate with others that knows LanguageA, and Katya cannot communicate with Danil, so Danil does not know LanguageA.\\
Therefore, the statement "Danil knows LanguageA." is False.\\
END.\\\
So the answer is: False.\\
------\\
Query: All eels are fish. No fish are plants. A thing is either a plant or animal. Nothing that breathes is paper. All animals breathe. If a sea eel is either an eel or a plant, then a sea eel is an eel or an animal. Based on the above information, is the following statement true, false, or uncertain? Sea eel breathes or is paper.\\
Thought: \\
Because all eels are fish, so a sea eel is a fish.\\
Because no fish are plants, a thing is either a plant or animal, so a fish is an animal. \\
Because a sea eel is a fish, and a fish is an animal, so a sea eel is an animal.\\
Because a sea eel is an animal, and all animals breathe, so a sea eel breathes.\\
Because a sea eel breathes and nothing that breathes is paper, so a sea eel is not paper.\\
Therefore, the statement "Sea eel breathes or is paper." is True.\\
END.\\
So the answer is: True.\\
\\
\#\#\#\# Here's what you need to do. Please first think step-by-step, give out each of your step in a newline, then end your thought with "END.". Finally respond "True", "False" or "Uncertain" in a newline, strictly starting with "So the answer is: ".
\end{tcolorbox}
}
% \vspace{-3mm}
\caption{The prompt without self-grounding on FOLIO.}
\label{prompt_no_grounding_folio}
\end{table*}

\begin{table*}
\centering
\resizebox{0.98\textwidth}{!}{
\begin{tcolorbox}[colback=blue!5!white,colframe=black!55!black,width=0.98\textwidth,title={Prompt without Self-Grounding (ProofWriter)}]
\small
You are a helpful assitant.\\
You will receive a query. Your task is to answer the query.\\
\\
\#\#\#\# Examples\\
Query: Bob is big. Dave is big. Dave is rough. Erin is nice. Erin is white. Gary is nice. Gary is white. Red things are white. All big things are green. All red, white things are nice. All green things are blue. If something is nice then it is big. All blue, green things are rough. All rough things are red. If something is blue then it is nice. If something is red then it is blue. Based on the above information, is the following statement true, false, or unknown? Gary is not red.\\
Thought: \\
Because Gary is nice, and if something is nice then it is big, so Gary is big.\\
Because Gary is big and all big things are green, so Gary is green.\\
Because Gary is green and all green things are blue, so Gary is blue.\\
Because Gary is green and Gary is blue, and all blue, green things are rough, so Gary is rough.\\
Because Gary is rough and all rough things are red, so Gary is red. \\
Therefore, the statement "Gary is not red." is false.\\
END.\\
So the answer is: False.\\
------\\
Query: Anne is nice. Anne is smart. Charlie is green. Fiona is nice. Fiona is round. Fiona is white. Harry is blue. White, kind things are nice. If something is smart and kind then it is green. If something is round and kind then it is white. Smart things are kind. Nice, white things are kind. Round things are kind. If something is nice then it is smart. All white things are round. If Charlie is green then Charlie is white. Based on the above information, is the following statement true, false, or unknown? Charlie is smart.\\
Thought: \\
Because Charlie is green, and if Charlie is green then Charlie is white, so Charlie is white.\\
Because Charlie is white and all white things are round, so Charlie is round. \\
Because Charlie is round and round things are kind, so Charlie is kind. \\
Because Charlie is white and Charlie is kind, and white, kind things are nice, so Charlie is nice.\\
Because Charlie is nice, and if something is nice then it is smart, so Charlie is smart. \\
Therefore, the statement "Charlie is smart." is true.\\
END.\\
So the answer is: True.\\
\\
\#\#\#\# Here's what you need to do. Please first think step-by-step, give out each of your step in a newline, then end your thought with "END.". Finally respond "True", "False" or "Unknown" in a newline, strictly starting with "So the answer is: ".
\end{tcolorbox}
}
% \vspace{-3mm}
\caption{The prompt without self-grounding on ProofWriter.}
\label{prompt_no_grounding_proofwriter}
\end{table*}

\begin{table*}
\centering
\resizebox{0.98\textwidth}{!}{
\begin{tcolorbox}[colback=blue!5!white,colframe=black!55!black,width=0.98\textwidth,title={Prompt without Self-Grounding (MMLU-Pro)}]
\small
You will receive a query and ten options. Your task is to select an option to answer the query.\\
\\
\#\#\#\# Examples\\
Query: Kylar went to the store to buy glasses for his new apartment. One glass costs \$5, but every second glass costs only 60\% of the price. Kylar wants to buy 16 glasses. How much does he need to pay for them?\\
Options: A.24, B.54, C.40, D.32, E.64, F.8, G.16, H.60, I.100, J.74\\
Thought: \\
Because one glass costs \$5, and every second glass costs only 60\% of the price, so the discount price of every second glass is 60/100 * 5 = \$3.\\
Because every second glass is discounted at \$3, and Kylar wants to buy 16 glasses, so Kylar is going to buy 16 / 2 = 8 discounted glasses and 16 - 8 = 8 regular-priced glasses.\\
Because Kylar is going to buy 8 discounted glasses, and every discounted glass is \$3, so Kylar is going to pay 8 * 3 = \$24.\\
Because Kylar is also going to buy 8 regular-priced glasses, and one glass costs \$5, so Kylar will pay 8 * 5 = \$40.\\
Because Kylar will pay \$24 for 8 discounted glasses, and \$40 for 8 regular-priced glasses, so in total Kylar needs to pay 24 + 40 = \$64 for the glasses he wants to buy.\\
END.\\
So the answer is: E.\\
------\\
Query: A refracting telescope consists of two converging lenses separated by 100 cm. The eye-piece lens has a focal length of 20 cm. The angular magnification of the telescope is ?\\
Options: A.10, B.40, C.6, D.25, E.15, F.50, G.30, H.4, I.5, J.20\\
Thought: \\
Because in a refracting telescope both lenses are converging, so their focus must be between the two lenses.\\
Because the focus of both lenses must lie between them, so their focal lengths must add up to their separation. \\
Because the two lenses are separated by 100 cm, and one lens has a focal length of 20 cm, so the other lens must have a focal length of 80 cm.\\
Because one lens has a focal length of 20 cm and the other 80 cm, so the magnification is the ratio of their focal lengths, which is 4.\\
END.\\
So the answer is: H. \\
\\
\#\#\#\# Here's what you need to do. Please first think step-by-step, presenting each of your step in a new line. Then end your thought with "END.". Finally respond with an option from "A", "B", "C", "D", "E", "F", "G", "H", "I" or "J" in a newline, strictly starting with "So the answer is: ".
\end{tcolorbox}
}
% \vspace{-3mm}
\caption{The prompt without self-grounding on MMLU-Pro.}
\label{prompt_no_grounding_mmlu}
\end{table*}

\begin{table*}
\centering
\resizebox{0.98\textwidth}{!}{
\begin{tcolorbox}[colback=blue!5!white,colframe=black!55!black,width=0.98\textwidth,title={Prompt without Self-Grounding (GPQA-Diamond)}]
\small
You will receive a query along with four options. Your task is to select an option to answer the query.\\
\\
\#\#\#\# Examples\\
Query: Kylar went to the store to buy glasses for his new apartment. One glass costs \$5, but every second glass costs only 60\% of the price. Kylar wants to buy 16 glasses. How much does he need to pay for them?\\
Options: \\
(A) 24 \\
(B) 54 \\
(C) 40 \\
(D) 64\\
Thought: \\
Because one glass costs \$5, and every second glass costs only 60\% of the price, so the discount price of every second glass is 60/100 * 5 = \$3.\\
Because every second glass is discounted at \$3, and Kylar wants to buy 16 glasses, so Kylar is going to buy 16 / 2 = 8 discounted glasses and 16 - 8 = 8 regular-priced glasses.\\
Because Kylar is going to buy 8 discounted glasses, and every discounted glass is \$3, so Kylar is going to pay 8 * 3 = \$24.\\
Because Kylar is also going to buy 8 regular-priced glasses, and one glass costs \$5, so Kylar will pay 8 * 5 = \$40.\\
Because Kylar will pay \$24 for 8 discounted glasses, and \$40 for 8 regular-priced glasses, so in total Kylar needs to pay 24 + 40 = \$64 for the glasses he wants to buy.\\
END.\\
So the answer is: D.\\
------\\
Query: A refracting telescope consists of two converging lenses separated by 100 cm. The eye-piece lens has a focal length of 20 cm. The angular magnification of the telescope is ?\\
Options: \\
(A) 10 \\
(B) 6 \\
(C) 4 \\
(D) 25\\
Thought: \\
Because in a refracting telescope both lenses are converging, so their focus must be between the two lenses.\\
Because the focus of both lenses must lie between them, so their focal lengths must add up to their separation. \\
Because the two lenses are separated by 100 cm, and one lens has a focal length of 20 cm, so the other lens must have a focal length of 80 cm.\\
Because one lens has a focal length of 20 cm and the other 80 cm, so the magnification is the ratio of their focal lengths, which is 4.\\
END.\\
So the answer is: C. \\
\\
\#\#\#\# Here's what you need to do. Please first think step-by-step, give out each of your step in a newline. Then end all your thought with "END.". Finally respond with an option from "A", "B", "C" or "D" in a newline, strictly starting with "So the answer is: ".
\end{tcolorbox}
}
% \vspace{-3mm}
\caption{The prompt without self-grounding on GPQA-Diamond.}
\label{prompt_no_grounding_gpqa}
\end{table*}

\begin{table*}
\centering
\resizebox{0.98\textwidth}{!}{
\begin{tcolorbox}[colback=blue!5!white,colframe=black!55!black,width=0.98\textwidth,title={Prompt with Self-Grounding (GPQA-Diamond)}]
\small
You will receive a query along with four options. Your task is to select an option to answer the query.\\
\\
\#\#\#\# Examples\\
Query: Kylar went to the store to buy glasses for his new apartment. One glass costs \$5, but every second glass costs only 60\% of the price. Kylar wants to buy 16 glasses. How much does he need to pay for them?\\
Options: \\
(A) 24 \\
(B) 54 \\
(C) 40 \\
(D) 64\\
Thought: \\
\text{[Step-1]} From Query, because one glass costs \$5, and every second glass costs only 60\% of the price, so the discount price of every second glass is 60/100 * 5 = \$3. \\
\text{[Step-2]} From Step-1 and Query, because every second glass is discounted at \$3, and Kylar wants to buy 16 glasses, so Kylar is going to buy 16 / 2 = 8 discounted glasses and 16 - 8 = 8 regular-priced glasses. \\
\text{[Step-3]} From Step-1 and Step-2, because Kylar is going to buy 8 discounted glasses, and every discounted glass is \$3, so Kylar is going to pay 8 * 3 = \$24. \\
\text{[Step-4]} From Step-2 and Query, because Kylar is also going to buy 8 regular-priced glasses, and one glass costs \$5, so Kylar will pay 8 * 5 = \$40. \\
\text{[Step-5]} From Step-3 and Step-4, because Kylar will pay \$24 for 8 discounted glasses, and \$40 for 8 regular-priced glasses, so in total Kylar needs to pay 24 + 40 = \$64 for the glasses he wants to buy. \\
END.\\
So the answer is: D. \\
------\\
Query: A refracting telescope consists of two converging lenses separated by 100 cm. The eye-piece lens has a focal length of 20 cm. The angular magnification of the telescope is ?\\
Options: \\
(A) 10 \\
(B) 6 \\
(C) 4 \\
(D) 25\\
Thought: \\
\text{[Step-1]} From Query, because in a refracting telescope both lenses are converging, so their focus must be between the two lenses. \\
\text{[Step-2]} From Step-1, because the focus of both lenses must lie between them, so their focal lengths must add up to their separation. \\
\text{[Step-3]} From Step-2 and Query, because the two lenses are separated by 100 cm, and one lens has a focal length of 20 cm, so the other lens must have a focal length of 80 cm. \\
\text{[Step-4]} From Step-3 and Query, because one lens has a focal length of 20 cm and the other 80 cm, so the magnification is the ratio of their focal lengths, which is 4. \\
END.\\
So the answer is: C. \\
\\
\#\#\#\# Here's what you need to do. Please first think step-by-step, give out each of your step in a newline starting with \text{[Step-i]}, and cite the sources (e.g., Step-i, Query) of your premises at the beginning of each step. Then end all your thought with "END.". Finally respond with an option from "A", "B", "C" or "D" in a newline, strictly starting with "So the answer is: ".
\end{tcolorbox}
}
% \vspace{-3mm}
\caption{The prompt with self-grounding on GPQA-Diamond.}
\label{prompt_grounding_gpqa}
\end{table*}

\end{document}
