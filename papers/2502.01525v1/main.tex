\documentclass[acmlarge, nonacm=true]{acmart}
%\documentclass{article}

\usepackage{fnpct}
\usepackage{multirow}
\usepackage{algpseudocode}
\usepackage{algorithm}
\usepackage{graphicx}
\usepackage{subcaption}
\usepackage{array}
\usepackage[export]{adjustbox}
\usepackage{flushend}
\usepackage{hyperref}
\usepackage{listings}
\usepackage{xcolor}
\usepackage[utf8]{inputenc}

%\usepackage{cite}
%\usepackage{biblatex} 
%\addbibresource{refs.bib}
%\usepackage[pdftex,colorlinks=true,urlcolor=blue,citecolor=black,anchorcolor=black,linkcolor=black,bookmarks=false,hidelinks]{hyperref}

%Example from: https://tex.stackexchange.com/questions/89574/language-option-supported-in-listings
\lstdefinelanguage{JavaScript}{
  keywords={typeof, new, true, false, catch, function, return, null, catch, switch, var, if, in, while, do, else, case, break},
  keywordstyle=\color{blue}\bfseries,
  ndkeywords={class, export, boolean, throw, implements, import, this},
  ndkeywordstyle=\color{darkgray}\bfseries,
  identifierstyle=\color{black},
  sensitive=false,
  comment=[l]{//},
  morecomment=[s]{/*}{*/},
  commentstyle=\color{purple}\ttfamily,
  stringstyle=\color{red}\ttfamily,
  morestring=[b]',
  morestring=[b]"
}

%Customize listings
%% Example from: https://www.overleaf.com/learn/latex/Code_listing
\definecolor{codegreen}{rgb}{0,0.6,0}
\definecolor{codegray}{rgb}{0.5,0.5,0.5}
\definecolor{codepurple}{rgb}{0.58,0,0.82}
\definecolor{backcolour}{rgb}{0.95,0.95,0.92}

\lstdefinestyle{mystyle}{
    backgroundcolor=\color{backcolour},   
    commentstyle=\color{codegreen},
    keywordstyle=\color{magenta},
    numberstyle=\tiny\color{codegray},
    stringstyle=\color{codepurple},
    basicstyle=\ttfamily\footnotesize,
    breakatwhitespace=false,         
    breaklines=true,                 
    captionpos=b,                    
    keepspaces=true,                 
    numbers=left,                    
    numbersep=5pt,                  
    showspaces=false,                
    showstringspaces=false,
    showtabs=false,                  
    tabsize=2
}

\lstset{style=mystyle}


% Other Comments: https://docs.google.com/document/d/1jz74jsTioKlvNpB9bz3tlwSFooDaBLc2/edit?usp=sharing&ouid=104511984749472138562&rtpof=true&sd=true
\begin{document}
%\title{Problems With Archiving and Replaying Current Web Advertisements}

\title{Archiving and Replaying Current Web Advertisements: Challenges and Opportunities}
%I would suggest we retitle the report so that we suggest not only the problems or challenges, but also the opportunities/possibilities. How about something to the effect of "Archiving and Replaying Current Web Advertisements: Challenges and Opportunities" 

\author{Travis Reid} \orcid{0000-0003-1360-7963} \affiliation{ \department{Department of Computer Science} \institution{Old Dominion University} \city{Norfolk} \state{VA} \postcode{23529} \country{USA} } \email{treid003@odu.edu} 

\author{Alex H. Poole} \affiliation{ \department{Department of Information Science} \institution{Drexel University} \city{Philadelphia} \state{PA} \postcode{19104} \country{USA} } \email{ahp56@drexel.edu}

\author{Hyung Wook Choi} \orcid{0000-0002-4075-0768} \affiliation{ \department{Department of Information Science} \institution{Drexel University} \city{Philadelphia} \state{PA} \postcode{19104} \country{USA} } \email{hc685@drexel.edu} 

\author{Christopher Rauch} \affiliation{ \department{Department of Information Science} \institution{Drexel University} \city{Philadelphia} \state{PA} \postcode{19104} \country{USA} } \email{cr625@drexel.edu}  

\author{Mat Kelly} \orcid{0000-0002-0236-7389} \affiliation{ \department{Department of Information Science} \institution{Drexel University} \city{Philadelphia} \state{PA} \postcode{19104} \country{USA} } \email{mkelly@drexel.edu} 

\author{Michael L. Nelson} \orcid{0000-0003-3749-8116} \affiliation{ \department{Department of Computer Science} \institution{Old Dominion University} \city{Norfolk} \state{VA} \postcode{23529} \country{USA} } \email{mln@cs.odu.edu} 

\author{Michele C. Weigle} \orcid{0000-0002-2787-7166} \affiliation{ \department{Department of Computer Science} \institution{Old Dominion University} \city{Norfolk} \state{VA} \postcode{23529} \country{USA} } \email{mweigle@cs.odu.edu}

%\author{Travis Reid \and Hyung Wook Choi \and Christopher Rauch \and Alex H. Poole \and Mat Kelly \and Michael L. Nelson \and Michele C. Weigle}

%\date{November 2024}

\begin{abstract}
Although web advertisements represent an inimitable part of digital cultural heritage, serious archiving and replay challenges persist.
%
To explore these challenges, we created a dataset of 279 archived ads. We encountered five problems in archiving and replaying them.
%To explore these challenges, we archived and replayed 279 ads. We encountered five problems in doing so.
%
For one, prior to August 2023, Internet Archive’s Save Page Now service excluded not only well-known ad services' ads, but also URLs with ad related file and directory names. Although after August 2023, Save Page Now still blocked the archiving of ads loaded on a web page, it permitted the archiving of an ad’s resources if the user directly archived the URL(s) associated with the ad. 
%
Second, Brozzler’s incompatibility with Chrome prevented ads from being archived.
%
Third, during crawling and replay sessions, Google's and Amazon's ad scripts generated URLs with different random values. This precluded archived ads' replay. Updating replay systems' fuzzy matching approach should enable the replay of these ads.
%
Fourth, when loading Flashtalking web page ads outside of ad iframes, the ad script requested a non-existent URL. This, prevented the replay of ad resources. But as was the case with Google and Amazon ads, updating replay systems' fuzzy matching approach should enable Flashtalking ads' replay.
%
Finally, successful replay of ads loaded in iframes with the \texttt{src} attribute of ``\texttt{about:blank}'' depended upon a given browser's service worker implementation. A Chromium bug stopped service workers from accessing resources inside of this type of iframe, which in turn prevented replay. Replacing the ``\texttt{about:blank}'' value for the iframe's \texttt{src} attribute with a blob URL before an ad was loaded solved this problem.
%
%Problems with replaying Google, Amazon, and Flashtalking ads can be resolved by updating the fuzzy matching approach used by replay systems. The replay problem that involved Chromium based browsers' service worker implementation can be fixed by replacing the ``\texttt{about:blank}'' value for the \texttt{src} attribute for an iframe with a blob URL before an ad is loaded.
%
Resolving these replay problems will improve the replay of ads and other dynamically loaded embedded web resources that use random values or ``\texttt{about:blank}'' iframes.
%
%Once these replay problems are resolved, it will improve the replay of ads and other dynamically loaded embedded web resources that use random values or ``\texttt{about:blank}'' iframes.
\end{abstract}

%Web advertisements represent a significant and rapidly evolving aspect of digital cultural heritage and the need to preserve them is increasing, but there are serious problems with archiving and replaying current web ads.  This research identified key obstacles, including ad exclusion by Internet Archive's Save Page Now, browser incompatibility with Chrome that prevented ads from being archived, and discrepancies in random number generation during the crawl and replay session that prevented ads from being loaded.

% Before August 2023, Save Page Now excluded ads from well known ad services, as well as URLs with ad related file and directory names in the URL's path. After August 2023, Save Page Now still blocked ads that were loaded on a web page from being archived, but it did allow an ad's resource(s) to be archived when the user directly archived the URL(s) associated with the ad.

\maketitle

\section{Introduction}
%Start off an article by providing broader context (web archiving is important because...and we face steep challenges because...), then segue into explaining the importance of your specific topic (we care about saving web ads because...). Then you state your problem focus, i.e. the obstacles with which crawlers contend. For example, we can set the table here by pulling from my abortive conference paper and the grant proposal itself

%Sample text: 
Brewster Kahle, founder of the Internet Archive, noted \cite{Kahle-sa97} as early as 1997 that the web constituted ``a storehouse of valuable scientific, cultural and historical information'' (p. 82). Almost a quarter century later, Webster \cite{webster-dpc20} characterized the web in similar terms, but also stressed its potential as ``a vast but underused scholarly resource for the study of almost every possible aspect of the last two decades'' (p. 1). Despite calls to action from numerous scholars, however, web content has been hemorrhaged \cite{cohen-dc10, nelson-arxiv12, pennock-dpc13}. 
% (\cite{cohen-dc10, nelson-arxiv12, pennock-dpc13})
%% (Cohen, 2010; Nelson, 2012; Pennock, 2013)

Whether impelled by legal obligation, business purposes (e.g., marketing), social or cultural interest, or scholarly and/or historical research, web archiving involves collecting, storing, preserving, and providing long-term access to content \cite{ball-dcc10, cook-ala18, niu-dlib12, pennock-dpc13}. Web archives may be used as evidence about the activities of its creator(s), its users, or about the period in which the archived content was created or modified. 
% (\cite{ball-dcc10, cook-ala18, niu-dlib12, pennock-dpc13})
%% (Ball, 2010; Cook, 2018; Niu, 2012; Pennock, 2013)

%Archiving websites with dynamic content, frequent updates, and complex structures is inherently challenging, especially with the added complexity of JavaScript, embedded media, and interactive interface elements \cite{kelly_change_2013}. These challenges intensify when archiving online advertisements, which are ephemeral and often integrated from external and dynamic sources \cite{bainotti_archive_2021}. The variability of targeted ads, influenced by user behavior and context, complicates consistent archiving. The use of sophisticated algorithms and real-time bidding for ad integration further hinders high-fidelity preservation \cite{wang_real-time_2016}. These difficulties underscore the need for tools and methodologies for archiving web ads effectively, ensuring a comprehensive digital historical record that reflects the concurrent evolution of web technology, culture, and commerce \cite{mcdonald_interplay_2021}.

% \cite{hwang20}
%% (Hwang, 2020)
Because the web depends upon advertising revenue \cite{hwang20}, web ads constitute a particularly important type of dynamic content. Just as physical ephemera in libraries, archives, and museums undergird compelling research, so do online ads illuminate not only the contemporary objectives of advertisers, but also social norms, values, and ideals in ways that curated news stories cannot. Advertisements provide foundational source material for political, social, cultural, and business scholarship, especially in unpacking research questions concerning race, ethnicity, gender, and socioeconomic class \cite{cohen-vb04, ewen-bb01, leach-vb94, marchand-cp86, packard-lgc57}. According to historian Jackson Lears \cite{lears-bb93}, advertisements validate certain worldviews and structures and marginalize others. Advertising, he contends, promotes ``the dominant aspirations, anxieties, even notions of personal identity, in the modern United States'' (p. 2).
% \cite{cohen-vb04, ewen-bb01, leach-vb94, marchand-cp86, packard-lgc57}
%% (L. Cohen, 2004; Ewen, 2001; Leach, 1994; Marchand, 1986; Packard, 1957)
% \cite{lears-bb93}

%The juxtaposition of ads and other genres of web content (e.g., politics, news, food) can enhance our understanding of their interplay, and thus of market segmentation and its continuity or change over time. Moreover, by catering to market segmentation, web advertisers’ delivery practices often reflect and promote prejudice, even discrimination \cite{sweeney-acm13}. Such content extends rich possibilities for scholarly research. 
% \cite{sweeney-acm13}
%% (Sweeney, 2013)
%I would cut this - the paper says nothing about market segmentation, so it is a non-sequitur here

Despite web ads' importance, little has been done to build collections of web ads. In large measure, this gap results from the novel and complex technical work it demands.
%% (now we have arrived at our problem statement). Next, we introduce our RQs and voila, away we go)
%% This exploratory study addresses the following research questions. 
%%% 1) for current tools for archiving/replay, how well can they archive/replay web ads?
%%% 2) what problems crop up with archiving/replay?
%Even though browser-based web archive crawlers can archive web advertisements, many web advertisements will be missing from archived web pages that should have loaded ads. Three reasons explain these omissions. First, web archiving services like Save Page Now block web ads from being archived. Second, web archiving services like archive.today cannot archive the embedded web page associated with the ad iframe, which is needed to load the ads. Finally, replay systems do not generate the same random number that was generated during the crawling session, which prevents the replay of some archived advertisements that use Google's or Amazon's ad service. 
%
%This is the roadmap: 
%% This is overly general. What are the main points of this tech report? "In this paper, we analyze and highlight..." Give a summary of the findings so that people are interested in reading more about it.
This exploratory research therefore addresses the following question: what are the key obstacles to archiving and replaying web ads? 
First, we define and situate the core concepts that underpin our work. Next, we explain our methods for archiving, replaying, and identifying resources associated with the 279 ads from our dataset, as well as our tool we developed to find difficult to replay ads. Finally, we describe the five technical challenges we encountered in archiving and replaying web ads. The first problem was the Internet Archive's Save Page Now \cite{ia-savePageNow} service excluding ads from being archived. Second, Brozzler \cite{Levitt-Brozzler-gh14} became incompatible with Chrome after March 2023, which prevented ads from being archived. Third, Google's and Amazon's ad script generated a URL with a random value that differed during the crawling and replay sessions, because replay systems overwrote the random number generator's seed. Fourth, the JavaScript for Flashtalking web page ads requested an unarchived URL when an ad was loaded outside of an iframe, which prevented the resources from loading. Fifth, the ability to replay an ad depended on the browser because the implementation of service workers differed among browsers.


\section{Background}
%Key concepts include web archiving, archived web page replay, and loading web advertisements.  
% This section will go over the terms that will be used throughout this paper. The topics include web archiving, replay systems, and loading web ads on the live web.
%The topics associated with this work are web archiving and web advertisements.
%
To understand the archiving and replay problems we identified, we first describe terms and concepts related to web archive crawlers, replay systems, and loading web advertisements on the live web.

%\subsection{Web Crawler}
%A web crawler is a program that can download web pages, extract hyperlinks from web pages, and recursively download the web pages associated with these extracted hyperlinks \cite{najork18}. A crawling session occurs when the web crawler is given a set of seed URIs, visits these seed URIs, and downloads the files associated with the web pages.
%
%
%In this section we have described what a web crawler is in general and in the next section we will discuss how web crawlers are used for web archiving and how the archived resources are loaded by replay systems.

\subsection{Web Archive Crawlers and Replay Systems}
%This section addresses  web archiving, Memento, and replay.
%
Web archiving involves using a crawler to collect content from the World Wide Web and preserve it in an archival format such as WARC (Web ARChive) \cite{warc-iipc13} or WACZ (Web Archive Collection Zipped) \cite{wacz-wr21}. An example crawling session is shown in Figure \ref{fig:crawlingSessionExample}. After the content is archived, a web archive replay system can display the archived version of the web resource(s) in a web browser.
%
\begin{figure}[tbp]
    \centering
     \begin{subfigure}[b]{1.0\textwidth}
        \centering
         \includegraphics[width=\textwidth]{Images/Example_Crawling_and_Replay_Sessions_Browsertrix_Crawler.png}
         \caption{A crawling session where Browsertrix Crawler is archiving a web page}
         \label{fig:crawlingSessionExample}        
     \end{subfigure}
     \vfill
     \begin{subfigure}[b]{1.0\textwidth}
        \centering
         \includegraphics[width=\textwidth]{Images/Example_Crawling_and_Replay_Sessions_ReplayWebPage.png}
         \caption{A replay session where ReplayWeb.page is replaying the web page archived by Browsertrix Crawler}
         \label{fig:replaySessionExample}        
     \end{subfigure}
    \caption{Example of a crawling session and replay session.}
    \label{fig:crawlingAndReplaySession}
\end{figure}
%
Some web archive crawlers use a graphical user interface (GUI)-based web browser, such as Chrome (bottom of Figure \ref{fig:exampleCrawlers}), or a headless browser, which operates without a GUI (top of Figure \ref{fig:exampleCrawlers}). 
%Figure \ref{fig:exampleCrawlers} shows web pages being crawled by three different web crawlers. When a web crawler does not use a regular web browser (top and center of Figure \ref{fig:exampleCrawlers}), the user cannot see the interactions occurring during the crawl. 
%Some browser-based web crawlers like Squidwarc \cite{berlin-gh17}, Brozzler \cite{Levitt-Brozzler-gh14}, and Browsertrix Crawler \cite{kreymer-BrowsertrixCrawler-gh20} can  use a headless browser or a regular browser.
%
\begin{figure}[tbp]
    \centering
     \begin{subfigure}[b]{0.75\textwidth}
        \centering
         \includegraphics[width=\textwidth]{Images/wget_Web_Crawlers_Vertical_cropped.png}
         \caption{wget is a web crawler that does \emph{not} use a web browser when archiving web pages}
         \label{fig:wget}        
     \end{subfigure}
     \vfill
     \begin{subfigure}[b]{0.75\textwidth}
        \centering
         \includegraphics[width=\textwidth]{Images/squidwarc_Web_Crawlers_Vertical_cropped.png}
         \caption{Squidwarc is a web crawler that can use a headless web browser when archiving web pages}
         \label{fig:squidwarc}
     \end{subfigure}
     \vfill
     \begin{subfigure}[b]{0.75\textwidth}
        \centering
         \includegraphics[width=\textwidth]{Images/Brozzler_Web_Crawlers_Vertical.png}
         \caption{Brozzler is a web crawler that can use a regular web browser (Google Chrome) when archiving web pages}
         \label{fig:Brozzler}
     \end{subfigure}
    \caption{Examples of web crawlers}
    \label{fig:exampleCrawlers}
\end{figure}

Replaying an archived web page involves using a web archive replay system to load the archived content in a browser, allowing users to view a previously stored version. An example is shown in Figure \ref{fig:replaySessionExample}. 
%
%\begin{figure}[tbp] \centering \includegraphics[width=\textwidth]{Images/URL_Rewriting_Before_and_After.png} \caption{Example of URL rewriting performed by Wayback Machine for a form on an archived web page (URI-M: \url{https://web.archive.org/web/20221220095518/https://www.google.com/})}  \label{fig:urlRewrite} \end{figure}
%
Replay systems use a URL rewriting system like Wombat \cite{wombat-github23} to modify URLs referenced by an archived webpage's HTML, CSS, and JavaScript files. This rewriting (Listings \ref{urlRewriteLiveWebPage} and \ref{urlRewriteArchivedWebPage}) changes a URI-R (the URI of an Original Resource [i.e., the state of a web resource on the live web at the time it was archived \cite{memento-framework23}]) to a URI-M (the URI for a memento [a previous state of an Original Resource]). In doing so, it prevents live web resources from loading during replay. 
%Replay systems use a URL rewriting system like Wombat \cite{wombat-github23} to rewrite URLs referenced by an archived webpage's HTML, CSS, and JavaScript files. This rewriting changes a URI-R to a URI-M (Figure \ref{fig:urlRewrite}) and prevents live web resources from being loaded during replay. 
%
%A URI-R is the URI of an Original Resource (i.e., the state of a web resource on the live web at the time it was archived \cite{memento-framework23}), and it is associated with the current version of the web resource that is available on the Web.
%
%A URI-M is the URI for a memento (a previous state of an Original Resource).
%
An example URI-M from Wayback Machine is \url{https://web.archive.org/web/20221220095518/https://www.google.com/}. 
%Figure~\ref{fig:uriRandUriM} shows an example URI-M~\footnote{Example URI-M from Wayback Machine: \url{https://web.archive.org/web/20221220095518/https://www.google.com/}} from the Wayback Machine \cite{ia-waybackMachine}. 
%
This URI-M includes the datetime indicating when the web page was archived (the \textit{Memento-Datetime}): ``\texttt{20221220095518}'' (2022-12-20T09:55:18Z) and the URI-R for the archived web page: ``\url{https://www.google.com/}".
%this is an important part of the rewriting process, and not everything follows. Wayback Machine conventions
Some web archives, such as Perma.cc~\cite{dulin-permacc-dlib17} and Archive.today~\cite{nelson-wsdlBlog13}, do not include the datetime or the URI-R in their URI-Ms (example URI-M for \url{https://www.google.com/} from Perma.cc is \url{https://perma.cc/V2KT-MYA6} and from Archive.today is \url{https://archive.is/SSsjK}).


Some replay systems, such as ReplayWeb.page, use service workers \cite{archibald-w3c22} to perform client-side URL rewriting. Service workers intercept HTTP requests made by archived web pages during replay \cite{Alam-ServiceWorker-jcdl17, berlin-acm23, berlin-odu18}. In contrast, server-side URL rewriting modifies the URLs in archived HTML, CSS, and JavaScript files before it sends the archived resources to the user.
%After the initial server-side URL rewriting, service workers can be used to perform client-side URL rewriting \cite{berlin-odu18}. 
This type of rewriting is useful for rewriting URLs in HTML and CSS files. By contrast, in client-side URL rewriting, a service worker or a client-side rewriting JavaScript library like Wombat changes the URL \cite{berlin-odu18}. Wombat, for example, performs the same URL rewriting that is done server-side and overrides the JavaScript API \cite{berlin-odu18}. Client-side URL rewriting outperforms server-side URL rewriting for URLs dynamically generated by JavaScript.

%\subsection{Memento}
%An Original Resource is a web resource that ``exists or used to exist, and for which access to one of its prior states may be required'' \cite{memento-framework23}. 
%
%Figure \ref{fig:originalResource} shows a representation of an Original Resource for ODU's homepage during 2023-11-12, and Figure \ref{fig:memento} shows a memento of ODU's homepage from 1999. 
%
%\begin{figure}[tbp] \centering \includegraphics[width=\textwidth]{Images/Original_Resource_example_2023-11-12.png} \caption{A representation of an original resource of ODU's homepage from the live web during 2023-11-12.}    \label{fig:originalResource} \end{figure}
%
% 
%\begin{figure}[tbp] \centering \includegraphics[scale=0.28]{Images/condensed_Archived_Resource_Example_1999-04-18.png} \caption{A memento of ODU's homepage archived during 1999-04-18 and replayed using the Internet Archive's Wayback Machine. URI-M: \url{https://web.archive.org/web/19990418080134/http://web.odu.edu/}} \label{fig:memento} \end{figure}
%Some web archives, such as Perma.cc~\cite{dulin-permacc-dlib17} and Archive.today~\cite{nelson-wsdlBlog13}, do not include the Datetime or the URI-R in their URI-Ms (examples\footnote{Example URI-M from Perma.cc: \url{https://perma.cc/V2KT-MYA6}}\footnote{Example URI-M from Archive.today: \url{https://archive.is/SSsjK} } shown at the bottom of Figure~\ref{fig:uriRandUriM}).
%
%\begin{figure}[tbp] \centering \includegraphics[width=\textwidth]{Images/URI_R_and_URI_M_Example.png} \caption{Example of one URI-R and multiple URI-Ms.} \label{fig:uriRandUriM} \end{figure}
%
%TimeMaps are used to view a list of URIs of mementos for an Original Resource. A URI for a TimeMap is a URI-T. TimeGates are able to perform datetime negotiation for an Original Resource, which helps with accessing a previous state of an Original Resource. A URI for a TimeGate is a URI-G.
%Associated with the currently available version of the web resource, a URI-R is the URI of an Original Resource. 
%Some web archives, such as Perma.cc~\cite{dulin-permacc-dlib17} and Archive.today~\cite{nelson-wsdlBlog13}, do not follow the Wayback Machine's convention for their URI-Ms and do not include the Datetime or the URI-R in their URI-Ms (examples shown at the bottom of Figure~\ref{fig:uriRandUriM}).

\noindent\begin{minipage}[tb]{\textwidth}
\begin{lstlisting}[language=html, breaklines=false, label=urlRewriteLiveWebPage, caption=Script element before URL rewriting]
<script src="https://treid003.github.io/displayAds.js"> </script>
\end{lstlisting}
\end{minipage}
%
\noindent\begin{minipage}[tb]{\textwidth}
\begin{lstlisting}[language=html, breaklines=true, label=urlRewriteArchivedWebPage, caption=Script element after URL rewriting]
<script src="https://web.archive.org/web/20240524092904js_/https://treid003.github.io/displayAds.js"> </script>
\end{lstlisting}
\end{minipage}

%\subsection{Archived Web Page Replay}
%
%This section focused on the tools used for archiving and replaying web resources, in the next section we will discuss a dynamic web resource that we focused on for this project: web advertisements.  


\subsection{Loading Web Advertisements on the Live Web} \label{Google_SafeFrame_Section}
The process of loading web advertisements involves three steps, as illustrated in Figures \ref{fig:loadWebAds} and \ref{fig:requestsWhenloadingWebAds}. 
%
First, the publisher (website owner) adds advertisement code, which employs HTML elements such as div and iframe, to create ad spaces~\footnote{Other terms denoting ad space include ad unit \cite{adunit-google24}, ad slot \cite{ad-glossary-google24}, and ad inventory \cite{admanager-google24}.} on their website \cite{adsense-google24}. 
%
Example code for creating an ad slot is shown at the top row of Figure \ref{fig:loadWebAds}.
% %First row of Figure \ref{fig:loadWebAds}.
%
During this step, the JavaScript files needed to use the ad service's API, like Google's gpt.js~\footnote{\url{https://securepubads.g.doubleclick.net/tag/js/gpt.js}} and pubads\_impl.js~\footnote{Archived version: \url{https://web.archive.org/web/20230821142108id_/https://securepubads.g.doubleclick.net/pagead/managed/js/gpt/m202308210101/pubads_impl.js?cb=31077272}}, are loaded. The first and second requests shown in Figure \ref{fig:requestsWhenloadingWebAds} were required to use Google Publisher Tag API.
%
Second, the publisher selects the ad(s) to load either by holding an auction \cite{adsense-google24, bids-amazon24} (example code for requesting auction bids is shown on the left side of the second row of Figure \ref{fig:loadWebAds}) or agreeing to host a sponsored ad \cite{ad-transactions-google24} (the Ashoka-sponsored ad selected by IGN is shown on the right side of the second row of Figure \ref{fig:loadWebAds}).
% The right side of this row shows the ad details from a sponsored ad that was selected by publisher
%This is not essential for the tech report, but what does "holding an auction" entail? When is this done? At webpage delivery time? In advance for a certain amount of time or for certain client conditions?
%% The auction can be initiated at different times depending on the ad service. For Google AdSense the ad script is loaded immediately and will start the auction and create the ad spaces. For Google Ad Manager, the process for starting the auction can be initiated at different times depending on the website publisher's JavaScript.
% Is this code in the webpage (via JS) and is run when the viewer loads the page?
%% Yes, for this example IGN loads multiple JavaScript files (ad scripts) when the web page is loaded.
%
The ad service selects an auction's winning bid based on metrics such as cost per mile (CPM) (the cost per 1,000 ad impressions~\cite{cpm-amazon24, cpm-google24}).
%Steps:
%% Request JavaScript file for GPT 
%% Create ad space using GPT API
%%% Create a div
%%% iframe might be created
%%% Request a Google SafeFrame if the ad requires it
%%% Request the resources needed for the ad
%
In the final step, the ad script dynamically retrieves and renders the selected ad(s) into designated ad space(s) on the web page, ensuring proper display and interaction capabilities. The third and fourth HTTP requests in Figure \ref{fig:requestsWhenloadingWebAds} were used to retrieve the code and web resources needed to load the advertisement into an ad slot.
% 
In Figure \ref{fig:loadWebAds}, the Intel and Fortnite ads were selected through auctions, while the website publisher (IGN) selected the Ashoka-sponsored ad.
%
\begin{figure}[tbp]
    \centering
    \includegraphics[width=\textwidth]{Images/Loading_Ads_Into_Web_Page.png}
    \caption{Process for loading advertisements into a web page. IGN's web page (\url{https://www.ign.com/tv/the-last-of-us-the-series}) was used for this example. The code at the top is from \url{https://cdn.ziffstatic.com/pg/ign.js}. the leftmost code is from \url{https://cdn.ziffstatic.com/pg/ign.js}, and the rightmost code is a post message that occurred after IGN requested Disney's ad. WACZ: \url{https://zenodo.org/records/10373131/files/safeframe-example.wacz?download=1}}
    \label{fig:loadWebAds}
\end{figure}

\begin{figure}[tbp]
    \centering
    \includegraphics[width=\textwidth]{Images/Requests_When_Loading_Google_Ads.png}
    \caption{HTTP requests used when loading the Fortnite image ad that was shown in Figure \ref{fig:loadWebAds}.}
    \label{fig:requestsWhenloadingWebAds}
\end{figure}


%Google Publisher Tag (GPT): Google Publisher Tag is the ad tag for ad spaces created using Google Ad Manager.
%Amazon Ad Server: This service will be sunset during the end of 2024.

%\newpage
%\subsubsection{Google Ad iframe} \label{Google_SafeFrame_Section}
Some Google ads, such as embedded web page ads, use SafeFrame \cite{safeframe-overview-google24}---an iframe based on Interactive Advertising Bureau (IAB) specifications \cite{safeFrame-iab24}---to enhance communication and security between the ad and the hosting web page. Unlike regular iframes, SafeFrames enable controlled communication between the publisher's web page and the ad, while limiting potentially harmful interactions \cite{safeFrame-iab24}.
%% "provides a single, unified mechanism for communication between advertiser and publisher content."
%% "preventing external access to sensitive data and providing more granular control over which creatives are rendered using the SafeFrame container with GPT."
%% "SafeFrame is a managed API-enabled iframe that opens a line of communication between the publisher page and the iframe-contained ad creative. While an iframe restricts any activity between the ad and the page, the communication protocol available with SafeFrame enables rich interaction without risking page security."
%% "To minimize the chances of malicious creatives serving"
%% SafeFrames are used instead of regular iframes, because SafeFrames allow the publisher's web page to communicate with the ad while also restricting the activity between the ad and the web page \cite{safeFrame-iab24}.
%
There are two types of Google SafeFrames. One type does not include a random value in the URL. % and it is easier to archive and replay. 
However, this type of Google SafeFrame is disallowed, since Google Publisher Tag's (GPT) API's SafeFrame configuration option \texttt{useUniqueDomain} \cite{uniqeDomain-gpt} is deprecated \cite{gpt-overview-google24, gptAPI-gpt24}.
%
%The other type does include a random value in the URL's subdomain (Figures \ref{fig:safeFrameURI} and \ref{fig:googleAdLiveVsReplay}). 
In contrast, the other type of SafeFrame includes a random value in the URL's subdomain. This sequesters SafeFrame content, thereby strengthening security measures \cite{safeframe-overview-google24}.  Figure \ref{fig:safeFrameURI} shows an example Google SafeFrame URL with a dynamically generated random subdomain. 
This type of SafeFrame remains difficult to replay because the random value generated during replay by Google's \href{https://web.archive.org/web/20230113005605id_/https://securepubads.g.doubleclick.net/gpt/pubads_impl_2023010901.js?cb=31071543}{pubads\_impl\_2023020201.js}~\footnote{URI-M: \url{https://web.archive.org/web/20230113005605id_/https://securepubads.g.doubleclick.net/gpt/pubads_impl_2023010901.js?cb=31071543}} script will differ from the random value generated at crawl time. This replay problem will be discussed in more detail in Section \ref{Replaying_Google_SafeFrame}.
%
Figure \ref{fig:googleAdLiveVsReplay} shows an example ad that failed to load because of this problem. 
%
\begin{figure}[tbp]
    \centering
    \includegraphics[width=\textwidth]{Images/SafeFrame_URI_Example.png}
    \caption{Example URI for a Google SafeFrame. The subdomain contains a random value that is dynamically generated when loading an ad. }
    \label{fig:safeFrameURI}
\end{figure}
%
\begin{figure}[tbp]
    \centering
    \includegraphics[width=\textwidth]{Images/Loading_Google_ad_live_vs_replay.png}
    \caption{Different SafeFrame URLs during crawl and replay sessions. Google's pubads\_impl.js (URI-R: \url{https://securepubads.g.doubleclick.net/pagead/managed/js/gpt/m202308210101/pubads_impl.js?cb=31077272} | 
    WACZ: \url{https://zenodo.org/records/10373131/files/safeframe-example.wacz?download=1}) generates the random SafeFrame URL.}
    \label{fig:googleAdLiveVsReplay}
\end{figure}
%
%Loading an ad inside of a Google SafeFrame prevents the replay of the ad because Google SafeFrames use a random value in the subdomain (Figure \ref{fig:safeFrameRandomSubdomain}) for the URL. 
%The replay systems we have tested (pywb\cite{pywb-gh13}, OpenWayback\cite{OpenWayback-gh12}, ReplayWeb.page\cite{ReplayWebPage-gh20}, Conifer\cite{rhizome-conifer}, Wayback Machine\cite{ia-waybackMachine}, Archive.today\cite{nelson-wsdlBlog13} and Arquivo.pt\cite{gomes-arquivopt08}) cannot generate the same random value that was generated during crawl time (Figure \ref{fig:safeFrameLivevsReplay}). 


%transition: sum up 
%\subsubsection{Amazon ad iframe}
Like Google's SafeFrame, Amazon's ad iframe uses a random value in the iframe's URL, albeit one  located in the query string (Figure \ref{fig:amazonAdIframeURI}) instead of the subdomain. 
%
A random value's presence in the query string results in the replay system generating an unarchived Amazon ad iframe URL. This can prevent an Amazon ad from loading during replay.
%
\begin{figure}[tbp]
    \centering
    \includegraphics[width=\textwidth]{Images/Amazon_Ad_URI_Example.png}
    \caption{Example URI for an Amazon ad iframe. The \texttt{rnd} parameter in the query string contains a random value that is dynamically generated when loading an ad.}
    \label{fig:amazonAdIframeURI}
\end{figure}


%In this section, we reviewed the concepts that are necessary to understand the problems we identified related to archiving and replaying web advertisements. Next, we discuss related work that focuses on the challenges with archiving and replaying dynamic web resources.


%Methods section
%% We focused on Amazon and Google ad services because they are most commonly use platforms by advertisers. 
% We explored/analyzed between X month and X month
%% We used ads that were archived in the dataset; to construct it, went to top/popular websites; went through web pages on those 
%% Consider: what tools did you use? Why were they the most appropriate tools (i.e. why did you choose those tools over others)? During what time frame did you conduct your work? What was your sample (in this case the domains from which you pulled the ads) and how/why did you choose it? Try to be transparent about and justify each of your decisions. Then at the end of the paper, you acknowledge the limitations of your study that result from these decisions and perhaps tie them in with key questions for future research. In this paper, we want to emphasize the unique challenges with web ads AND link the findings/lessons learned to more general technical problems and solutions facing dynamic content.
\section{Methods}
% Before describing what's in this section, we need more of an intro into what we're trying to demonstrate and why it's important. We've just had a lot of Background and Related Work, so we need to bring people back to our work. At a high level, what are we trying to do?
% This exploratory research therefore addresses the following question: what are the key obstacles to archiving and replaying web ads?
% Despite web ads' importance, little has been done to build collections of web ads. In large measure, this gap results from the novel technical work it demands.
%At a high level, what are we trying to do?
%% Describe how we created our dataset
%%% Which web archiving and replay tools were used
%%% Which tools were used to find the advertisements in our dataset
%% Describe how we categorized ads
%This section describes our process for creating a dataset of 279 recently archived ads. 
%
In this work, we created a dataset~\footnote{We created a web page to display ads from our dataset: \url{https://savingads.github.io/themed_ad_collections.html}} \cite{ad-dataset-gh24} of 279 unique recent (January-June, 2023) advertisements culled from the live web. We archived them with appropriate tools, and identified problems with both their archiving and  their replay.

%\newpage
\subsection{Creating a Dataset of Recently Archived Ads} \label{section:creatingDataset}
%We created a dataset~\footnote{We created a web page to display ads from our dataset: \url{https://savingads.github.io/themed_ad_collections.html}} \cite{ad-dataset-gh24} of 279 unique advertisements culled from the live web.
%
%We created a dataset \cite{ad-dataset-gh24} of unique advertisements from the live web in support of our project, ``Saving Ads: Assessing and Improving Web Archives' Holdings of Online Advertisements''~\footnote{\url{https://www.imls.gov/grants/awarded/lg-252362-ols-22}}, which has been supported by the Institute of Museum and Library Services (IMLS).
%
%The goal was to archive at least 250 ads, but we got 279 ads after archiving 17 web pages that included ads.
%
%To construct this exemplary dataset of approximately 250 ads ...
%% Thematic content
%%% Current themes from the dataset (did not specify a theme for all of the ads yet)
%%%% There are currently 22 themes for the ads in our dataset.
%%%% Art, Beauty and Cosmetics, Shopping, Travel, Entertainment, Gaming, Gambling and Fantasy Sports, Fitness and Sports, Health, Food and Drink, Internet and Mobile Service Provider, Cybersecurity, Pets and Animals, News, Business, Finance, Real Estate, Military, Charity, Funeral Services,  Sponsored Brand, Vehicle and Automotive
%
%We then developed a list of top websites \cite{top-websites-23} from SimilarWeb's categories and randomly sorted that list. We selected a web page to archive by scanning the list until we found a web page that included ads.
To construct our dataset, we randomly selected websites from Similarweb's top websites worldwide \cite{similarweb} (including all categories except ``Adult''~\footnote{The ``Adult'' category was excluded because we planned on recording and uploading videos of the crawling and replay sessions to YouTube. Websites from this category cannot be included in a video, as it would violate YouTube's Community Guidelines (\url{https://support.google.com/youtube/answer/2802002?hl=en&ref_topic=9282679}).}), rendered a web page from each website, and if the page loaded ads, archived it. We repeated this process until we had collected at least 250 ads.
%Each webpage was loaded in a web browser and manually scanned to see if it contained ads. We continued with this process until we had enough webpages to collect at least 250 ads.

\begin{table}[tbp]
\begin{tabular}{|l|r|l|l|}
\hline
\multicolumn{1}{|c|}{\textbf{Web Page}}                                                           & \multicolumn{1}{c|}{\textbf{\begin{tabular}[c]{@{}c@{}}Number of \\ Ads Archived\end{tabular}}} & \multicolumn{1}{c|}{\textbf{\begin{tabular}[c]{@{}c@{}}Web Archiving \\ Tool\end{tabular}}} & \multicolumn{1}{c|}{\textbf{Replay System}} \\ \hline
\href{https://canalturf.com}{https://canalturf.com}                                                                             & 66                                                                                             & Save Page Now                                                                               & Wayback Machine                             \\ \hline
\begin{tabular}[c]{@{}l@{}}\href{https://www.lequipe.fr/Tous-sports/Actualites/Le-flash-sports-du-5-avril/1389820}{https://www.lequipe.fr}\\ \href{https://www.lequipe.fr/Tous-sports/Actualites/Le-flash-sports-du-5-avril/1389820}{/Tous-sports/.../1389820}\end{tabular}         & 37                                                                                              & ArchiveWeb.page                                                                             & ReplayWeb.page                              \\ \hline
\begin{tabular}[c]{@{}l@{}}\href{https://www.leroymerlin.com.br/}{https://www.leroymerlin}\\ \href{https://www.leroymerlin.com.br/}{.com.br}/\end{tabular}                        & 31                                                                                              & Arquivo.pt                                                                                  & Arquivo.pt                                  \\ \hline
\begin{tabular}[c]{@{}l@{}}\href{https://www.ign.com/articles/the-last-of-us-season-1-review}{https://www.ign.com}\\ \href{https://www.ign.com/articles/the-last-of-us-season-1-review}{/articles/the-last-...-review}\end{tabular}       & 24                                                                                              & ArchiveWeb.page                                                                             & ReplayWeb.page                              \\ \hline
\begin{tabular}[c]{@{}l@{}}\href{https://www.facebook.com/}{https://www.facebook}\\ \href{https://www.facebook.com/}{.com/}\end{tabular}                              & 24                                                                                              & ArchiveWeb.page                                                                             & ReplayWeb.page                              \\ \hline
\url{https://www.cnn.com/}                                                                              & 23                                                                                              & ArchiveWeb.page                                                                             & ReplayWeb.page                              \\ \hline
\begin{tabular}[c]{@{}l@{}}\href{https://www.marketwatch.com/}{https://www.marketwatch}\\ \href{https://www.marketwatch.com/}{.com/}\end{tabular}                           & 18                                                                                              & Brozzler                                                                                    & ReplayWeb.page                              \\ \hline
\url{https://www.diy.com/}                                                                              & 13                                                                                              & Conifer                                                                                     & Conifer                                     \\ \hline
\begin{tabular}[c]{@{}l@{}}\href{https://www.realtor.com/news/unique-homes/frank-lloyd-wright-designed-home-in-tulsa-ok-lists-for-7-9m/}{https://www.realtor.com}\\ \href{https://www.realtor.com/news/unique-homes/frank-lloyd-wright-designed-home-in-tulsa-ok-lists-for-7-9m/}{/news/.../frank-...7-9m/}\end{tabular}        & 11                                                                                              & Browsertrix Crawler                                                                         & ReplayWeb.page                              \\ \hline
\begin{tabular}[c]{@{}l@{}}\href{https://mortalkombat.fandom.com/wiki/Tag_Team_Ladder}{https://mortalkombat.fandom}\\ \href{https://mortalkombat.fandom.com/wiki/Tag_Team_Ladder}{.com/wiki/Tag\_Team\_Ladder}\end{tabular} & 8                                                                                               & Browsertrix Crawler                                                                         & ReplayWeb.page                              \\ \hline
\url{https://tokopedia.com}                                                                             & 6                                                                                               & Arquivo.pt                                                                                  & Arquivo.pt                                  \\ \hline
\begin{tabular}[c]{@{}l@{}}\href{https://www.deviantart.com/kvacm/art/Hellstone-Ruins-860415274}{https://www.deviantart.com}\\ \href{https://www.deviantart.com/kvacm/art/Hellstone-Ruins-860415274}{/kvacm/art/Hellstone-...274}\end{tabular}  & 5                                                                                               & Browsertrix Crawler                                                                         & ReplayWeb.page                              \\ \hline
\begin{tabular}[c]{@{}l@{}}\href{https://unsplash.com/t/wallpapers}{https://unsplash.com/t}\\ \href{https://unsplash.com/t/wallpapers}{/wallpapers}\end{tabular}                      & 3                                                                                               & archive.today                                                                               & archive.today                               \\ \hline
\url{https://sports.yahoo.com}                                                                          & 3                                                                                               & Conifer                                                                                     & Conifer                                     \\ \hline
\begin{tabular}[c]{@{}l@{}}\href{https://www.vidal.ru/novosti/kak-potreblenie-razlichnyh-doz-alkogolya-vliyaet-na-smertnost-11744}{https://www.vidal.ru/novosti}\\ \href{https://www.vidal.ru/novosti/kak-potreblenie-razlichnyh-doz-alkogolya-vliyaet-na-smertnost-11744}{/kak-potreblenie-...744}\end{tabular}    & 2                                                                                               & Save Page Now                                                                               & Wayback Machine                             \\ \hline
\url{https://canalturf.com}                                                                             & 2                                                                                               & ArchiveBot                                                                                  & Wayback Machine                             \\ \hline
\url{https://canalturf.com}                                                                             & 1                                                                                               & Perma.cc                                                                                    & Wayback Machine                             \\ \hline
\begin{tabular}[c]{@{}l@{}}\href{https://www.youtube.com/watch?v=PZShwWiepeY}{https://www.youtube.com}\\ \href{https://www.youtube.com/watch?v=PZShwWiepeY}{/watch?v=PZShwWiepeY}\end{tabular}            & 1                                                                                               & Browsertrix Crawler                                                                         & ReplayWeb.page                              \\ \hline
\begin{tabular}[c]{@{}l@{}}\href{https://www.tripadvisor.it/Tourism-g186338-London_England-Vacations.html}{https://www.tripadvisor.it}\\ \href{https://www.tripadvisor.it/Tourism-g186338-London_England-Vacations.html}{/Tourism-...-Vacations.html}\end{tabular}  & 1                                                                                               & archive.today                                                                               & archive.today                               \\ \hline
\end{tabular}
\caption{The number of archived ads from each web page that we archived when creating the dataset.}
\label{table:adsPerWebpage}
\end{table}


Ultimately, we selected 17 web pages and 279 ads to archive (Table \ref{table:adsPerWebpage}). We used Internet Archive's Save Page Now, Arquivo.pt, archive.today, and Conifer because these web archiving services permit the archiving of an unlimited number of web pages cost-free. We also used three browser-based tools (ArchiveWeb.page \cite{kreymer-ArchiveWebpage-gh20}, Browsertrix Crawler, and Brozzler) that facilitate archiving dynamically loaded web resources. ArchiveWeb.page and Browsertrix Crawler archived four web pages each, Brozzler archived one web page, and four web archiving services (Save Page Now~\footnote{When replaying a web page, \url{https://canalturf.com}, archived by Save Page Now, we found that some of the ads that were loaded by Wayback Machine were archived by ArchiveBot and Perma.cc}, Arquivo.pt, archive.today, and Conifer) archived two web pages each. We did not archive four web pages with each web archiving tool because we had reached our goal of 250 advertisements.
%%This footnote is for a particular URL, not in general, right? If it's just for this URL, surround it by commas, not parens.
%
We successfully archived (captured all the resources needed) nearly all (273) of these ads.
%Nearly all (273) of these ads were successfully archived, that is, they contained all of the resources needed for replay. 
% 
%
%The ``Adult'' category was excluded, because we planned on recording and uploading videos of the crawling and replay sessions to YouTube and websites from this category cannot be included in a video since it would violate YouTube's Community Guidelines~\footnote{\url{https://support.google.com/youtube/answer/2802002?hl=en&ref_topic=9282679}}.   
%

%URI-M for SimilarWeb for top 50: https://archive.is/7cdUl
%% The individual categories were not archived and most web archives failed the verification test to access SimilarWeb's web page
% We used SimilarWeb, because Alexa's Top Websites service no longer exists.
%We used Internet Archive's Save Page Now, Arquivo.pt, archive.today, and Conifer because these web archiving services permit the archiving of an unlimited number of web pages cost-free. We also used three browser-based tools, which facilitate archiving dynamically loaded web resources. All three browser-based tools (ArchiveWeb.page, Browsertrix Crawler, and Brozzler) proved effective: the ads displayed during the crawling sessions were archived successfully.


Six ads were not fully archived. One required a specific user interaction (clicking on a play button) to load all of the ad resources. Three ads requested unarchived JavaScript and HTML files during replay. We used ReplayWeb.page's URL prefix search and removed the query string from the requested URI to see if a resource with a similar URI was archived, but did not find these JavaScript and HTML files in ArchiveWeb.page's output file (WACZ). For one Flashtalking ad, it was impossible to determine if the ad  was successfully archived because this type of ad cannot be replayed outside of its ad iframe. This prevented us from comparing the ad resources that loaded on the live web and (presumably) would load during replay. (This problem with Flashtalking ads will be discussed in Section \ref{section:flashtalking}.) The last partially archived ad's dynamically generated URL (during crawl time) included an  \texttt{e} query string parameter that prevented\footnote{Ad URI with \texttt{e} parameter: \url{https://s0.2mdn.net/sadbundle/14073241274752696320/970x250-HBO_Max/index.html?e=69&leftOffset=0&topOffset=0&c=qipO0hxSN3&t=1&renderingType=2&ev=01_247}}\footnote{Ad URI without \texttt{e} parameter: \url{https://s0.2mdn.net/sadbundle/14073241274752696320/970x250-HBO_Max/index.html?leftOffset=0&topOffset=0&c=qipO0hxSN3&t=1&renderingType=2&ev=01_247}} some of the images from loading. There were seven more images on the live version of the ad, but the live web ad was checked months after crawl time.
%


%during replay the ad was not requested and did not load inside of its ad iframe. Flashtalking ads that are used with Google or Amazon ad service cannot be viewed outside of an ad iframe (this problem will be discussed in section \ref{section:flashtalking}).
%
%The last partially archived ad did not display the image for the movie or TV show from HBO MAX.

%Ultimately, we selected 17 web pages to archive, resulting in the collection of 279 advertisements. We employed  ArchiveWeb.page and Browsertrix Crawler to archive four web pages each, Brozzler to archive one web page, and the web archiving services (Save Page Now, Arquivo.pt, archive.today, and Conifer) to archive two web pages each. We did not archive four web pages with each web archiving tool because we went over our goal for 250 advertisements. This left us with a dataset of 279 ads archived between January and June of 2023. Nearly all (273) of these ads were successfully archived, that is, they contained all of the resources needed for replay. 
%
%The ``Adult'' category was excluded, because we planned on recording and uploading videos of the crawling and replay sessions to YouTube and websites from this category cannot be included in a video since it would violate YouTube's Community Guidelines~\footnote{\url{https://support.google.com/youtube/answer/2802002?hl=en&ref_topic=9282679}}.   
%

%URI-M for SimilarWeb for top 50: https://archive.is/7cdUl
%% The individual categories were not archived and most web archives failed the verification test to access SimilarWeb's web page
% We used SimilarWeb, because Alexa's Top Websites service no longer exists.

%\subsubsection{Archiving Web Pages With Ads}

%\subsubsection{Identifying Archived Ads}
\subsubsection{Finding Archived Ads' Web Resources}
By using ReplayWeb.page's URL search feature \cite{replayWebPage-search-24} and our own bespoke Display Archived Ads tool \cite{display-archived-ads-gh24}, we found that 55 out of 279 advertisements were not replayable when loading the archived containing web page~\footnote{A containing web page is the web page that loaded the ad during a crawling session.}. 
% Should switch the number with always fail to replay (include partial ads as well)  and mention that the ads that did not load during replay of the containing web page so we had to use these tools to discover these ads in the WARC or WACZ
%
%\subsubsubsection{ReplayWeb.page’s URL Search}
We used ReplayWeb.page's URL search feature to identify ads in our WARC and WACZ files. This feature allows users to specify the MIME type, which facilitates searching for a specific ad type like image ads. It also allows for a prefix search, which we used to identify resources associated with services like Flashtalking, Innovid, Amazon display ads, and Google AdSense. 
% Will add description below after I find a video timestamp for an example
%%One issue with ReplayWeb.page’s URL search during 2023 was that some web resources associated with a URL prefix would not be shown in the search results. This is one of the reasons why our tool for displaying potential ads was created to make sure that all ads in the WARC or WACZ file were identified.

% Removed from JASIST version
The steps we used for ReplayWeb.page's URL search are listed below~\footnote{Video example: \url{https://youtu.be/QXUDINMJ5Ec?t=238s}}: \begin{enumerate} \item Load WARC or WACZ file with ReplayWeb.page \item Select ``URLs'' tab \item For the button beside ``Search'', select the MIME type for the ad \item Enter a URL prefix associated with an ad service, such as \url{https://s0.2mdn.net/} \item Click on the search result items to load the resource (the image, video, or HTML file) and see if the resource is an ad \end{enumerate}

%\subsubsection{Our Tool For Displaying Potential Ads}
Our Display Archived Ads tool~\footnote{Demo video for our tool: \url{https://youtu.be/Bc2T6ZZd210?t=30}} (Figure \ref{fig:displayPotentialAds}) was used to display most of the HTML, image, and video files inside of a given WARC file.
% Will use sentence below after I find an example video timestamp where the search results did not show an image that was in the archive
%%There are three reasons why our tool for displaying potential ads was created. 
%
Our tool depends on the warcio~\cite{kreymer-warcio-gh24}, pywb~\cite{pywb-gh13}, and Selenium~\cite{selenium} software packages. Warcio retrieves the URLs for the web resources from the WARC file. Pywb replays the archived ads within an iframe~\footnote{We replayed the archived resources in an iframe in order to display multiple ads on the same web page.}, and Selenium opens a web browser and executes the JavaScript necessary to display the ads. 
%
%It is possible to use ReplayWeb.page to load the advertisement, but it would require a dynamically generated web page that embeds ReplayWeb.page. Also, the WARC file must be accessible from python's localhost server (it is more steps than using pywb).
%
Our tool offered two affordances. 
%
%%The first reason is that ReplayWeb.page’s URL search feature does not always show all of the web resources inside of the WACZ file when searching for a specific MIME type like images. Another reason for creating this tool is to filter out some of the known ad resources that are not visible during replay so that it is quicker to go through all of the files associated with a MIME type.
%%The last reason for creating this tool is for displaying the live version of an ad beside the archived version, which allows us to see if there is an issue with replaying the ad.
%
It enabled us to filter out some of the known ad resources that remain invisible during replay~\footnote{An example filtered file is pixel.gif, a commonly used image for ad services that only shows a few white pixels.}, thereby speeding up the review. Further, by allowing us to display an ad's live version beside its archived version, the tool showed problems with replay.
% Example of using pixel elements or images:
%% https://support.google.com/admanager/answer/2508388
%% Pixels are used for tracking: https://developers.google.com/third-party-ads/
%
% Removed from JASIST version
%\newpage
The command for running our tool requires a WARC file and the seed URL, which is used to exclude the containing web page from the HTML category. Listing \ref{displayAdsGeneral}  shows the general structure for the command. Listing \ref{displayAdsExample} shows an example command for an IGN web page that includes ads.

\begin{figure}[tbp]
    \centering
    \includegraphics[width=\textwidth]{Images/square_fortnite_shell.png}
    \caption{Our tool for displaying potential ads loading the live web ad beside the archived version of the same ad.}
    \label{fig:displayPotentialAds}
\end{figure}

% Removed from JASIST version
\noindent\begin{minipage}[tb]{\textwidth} \begin{lstlisting}[language=bash, breaklines=false, label=displayAdsGeneral, caption=General structure for the command used for our tool] 
python3 Display_Archived_Ads.py /path/to/WARC/file.warc <seed URL> \end{lstlisting} \end{minipage}
%
% Removed from JASIST version
\noindent\begin{minipage}[tb]{\textwidth} \begin{lstlisting}[language=bash, breaklines=true, label=displayAdsExample, caption=Example command used for an IGN web page] 
python3 Display_Archived_Ads.py data.warc.gz https://www.ign.com/tv/the-last-of-us-the-series \end{lstlisting} \end{minipage}


\subsubsection{Replaying Archived Ads}
To replay the archived advertisements, we used four web archiving services (Internet Archive's Wayback Machine, Arquivo.pt, archive.today, and Conifer). We replayed the web ads that we archived with a web archiving service with a service from the same web archive. We also used three other replay systems (ReplayWeb.page, pywb, and OpenWayback \cite{OpenWayback-gh12}). 
%
%The web archiving services were used to replay the web ads that we archived using the same web archiving service.
%The pywb replay system is also used by Arquivo.pt and Conifer. 
We used ReplayWeb.page to replay the archived ads from our WARC and WACZ files, because at the beginning of 2023, ReplayWeb.page could replay more dynamically loaded web resources than pywb and OpenWayback. Pywb (version 2.7.3) failed to replay archived web ads that relied upon an Amazon ad iframe. During replay, OpenWayback (version 2.4.0) \cite{ruest-owb-240-gh19} loaded live web advertisements instead of the archived ads.  Brunelle \cite{brunelle-zombies-wsdl12} and Lerner et al. \cite{Lerner-acm17} discuss this problem of live web resources being loaded during replay.  
%We did not select OpenWayback (version 2.4.0) \cite{ruest-owb-240-gh19} as the preferred replay system for the web pages we archived, because during replay it loaded live web advertisements instead of the archived ads.
%, but ReplayWeb.page (version 1.7.13) \cite{kreymer-rpw-1713-gh23} was able to replay more ads successfully at the beginning of 2023 (around February).
%Brunelle et al. and Lerner have also described cases 
%

\subsection{Categorizing Advertisements}

%If this table is not shown before the example ads it will be multiple pages after the ad themes section
\begin{table}[tbp]
\begin{tabular}{|l|r|}
\hline
\multicolumn{1}{|c|}{\textbf{Theme}} & \multicolumn{1}{l|}{\textbf{Number of Ads}} \\ \hline
Shopping                             & 85                                          \\ \hline
Finance                              & 27                                          \\ \hline
Vehicle and Automotive               & 23                                          \\ \hline
Business Services                    & 21                                          \\ \hline
Travel                               & 19                                          \\ \hline
Entertainment                        & 16                                          \\ \hline
Health                               & 15                                          \\ \hline
Real Estate                          & 15                                          \\ \hline
News                                 & 14                                          \\ \hline
Unknown                              & 6                                           \\ \hline
Internet and Mobile Service Provider & 5                                           \\ \hline
Art                                  & 4                                           \\ \hline
Beauty and Cosmetics                 & 4                                           \\ \hline
Gaming                               & 4                                           \\ \hline
Military                             & 4                                           \\ \hline
Food and Drink                       & 3                                           \\ \hline
Gambling and Fantasy Sports          & 3                                           \\ \hline
Computer Security                    & 2                                           \\ \hline
Fitness and Sports                   & 2                                           \\ \hline
Pets and Animals                     & 2                                           \\ \hline
Sponsored Brand                      & 2                                           \\ \hline
Charity                              & 1                                           \\ \hline
Funeral Services                     & 1                                           \\ \hline
Politics                             & 1                                           \\ \hline
\end{tabular}
\caption{List of themes used for the ads in our dataset.}
\label{table:adThemes}
\end{table}

We categorized our 279 ads into five categories: image, video, embedded web page, text-only, or a combination.
%
The first three types are associated with one web resource that is viewable outside of the containing web page (provided the user knows the URI associated with the resource). Figures \ref{fig:imageAd} (image ad), \ref{fig:videoAd} (video ad), and \ref{fig:embeddedWebPageAd} (embedded web page ad) show examples.
%
Text-only ads (Figure \ref{fig:textAd}) cannot be viewed outside of the containing web page because the web page loads the text.
%
The combination category comprises ads (Figure \ref{fig:combinationAd}) that rely upon multiple resources and are constructed inside of the containing web page or ad iframe. Like text ads, combination ads cannot be viewed outside of the containing web page.

%
\begin{figure}[tbp]
    \centering
    \includegraphics[width=\textwidth]{Images/image_ad_example.png}
    \caption{An example image ad loaded outside of the containing web page. Ad's URI-M: \url{https://conifer.rhizome.org/treid003/2023-05-16-archiving-ads-on-sportsyahoocom/https://s.yimg.com/bx/adb/20230310154032235.jpg} }
    \label{fig:imageAd}
\end{figure}

%
\begin{figure}[tbp]
    \centering
    \includegraphics[scale=0.25]{Images/video_ad_example.png}
    \caption{An example video ad loaded outside of the containing web page. WACZ: \url{https://zenodo.org/record/8057942/files/2023_06_07_archiving_ads_on_lequipe_ArchiveWeb_page.wacz?download=1} | Ad's URI-R: \url{https://azv.adsrvr.org/thetradedesk-ads-video/2nwniqr/f2rcr8v/g5gzcxa29115fa8fdeed4ef88089cec513d745e4.mp4} }
    \label{fig:videoAd}
\end{figure}

%
\begin{figure}[tbp]
    \centering
    \includegraphics[scale=0.25]{Images/embedded_web_page_example.png}
    \caption{An example web page ad loaded outside of the containing web page. WARC: \url{https://zenodo.org/record/7601187/files/2023-01-11_00-59-34_ads_on_fandom_browsertrix_crawler.warc.gz?download=1} | Ad's URI-R: \url{https://s0.2mdn.net/sadbundle/13045786678919115269/CCD2C_5568424_300x600_MF_CP_APPLY_NA_NR_EN_V1_H5_BD_2022_042025/index.html} }
    \label{fig:embeddedWebPageAd}
\end{figure}

\begin{figure}[tbp]
    \centering
    \includegraphics[scale=0.3]{Images/text_only_ad_example.png}
    \caption{An example text-only ad for a sponsored news article loaded in the containing web page. WACZ: \url{https://zenodo.org/record/8057942/files/2023_06_07_archiving_ads_on_lequipe_ArchiveWeb_page.wacz?download=1} | Containing web page's URI-R: \url{https://www.lequipe.fr/Tous-sports/Actualites/Le-flash-sports-du-5-avril/1389820} }
    \label{fig:textAd}
\end{figure}

%
\begin{figure}[tbp]
    \centering
    \includegraphics[scale=0.3]{Images/combination_ad_example.png}
    \caption{An example combination ad loaded in the containing web page. This ad uses three images and one video. WACZ: \url{https://zenodo.org/record/8000975/files/2023-02-07-ads-on-ign_ArchiveWeb_page.wacz?download=1} | Containing web page's URI-R: \url{https://www.ign.com/articles/the-last-of-us-season-1-review} }
    \label{fig:combinationAd}
\end{figure}


%\subsubsection{Ad Themes}
Next, we coded each ad topically.
%There were 24 different themes we used when labeling the ads (Table \ref{table:adThemes}). 
% 
Table \ref{table:adThemes} shows the 24 themes and the corresponding number of ads for each.
%Table \ref{table:adThemes} shows the thematic composition of our dataset and the number of ads for each theme. 
%
Most themes (17 of 24) aligned with SimilarWeb's website categories. The other themes included ``Internet and Mobile Service Provider'', ``Politics'', ``Funeral Services'', ``Charity'', ``Military'', ``Sponsored Brand'', and ``Unknown''\footnote{``Unknown'' refers to ads that we were not able to replay and could not view on the live web.}.
%
Notably, the Military theme was exclusively video ads. In contrast, the other themes with more than three ads included multiple ad types.
%One themes (with several ads) featured one ad type: Military theme only had video ads and the ``Unknown'' theme exclusively had embedded web page ads.
%
%The most common ad theme was Shopping (30\% of all ads). Even without the 50 ads archived from shopping websites, it would still lead with 35 ads.
%Shopping was the most common ad theme (30\% of all ads), even without the 50 ads archived from shopping websites, it would still lead with 35 ads.
%
%\newpage
%Next, we discuss the problems we found while archiving and replaying recent web ads.
After this thematic coding, we sought to identify the problems with archiving and replaying the 279 ads. 
%Shopping ads were very common (30\% of the ads). Three of the 17 web pages archived were online stores and 50 ads were archived from these web pages. Even without these 50 ads, this would still be the most common theme with 35 ads. 
% All of the military ads were videos.
% All of the ads from the Unknown theme are embedded web page ads.
% The ads from the Unknown theme are either Flashtalking or Amazon ads.
% Themes without a majority of combination ads:
%% Military
%% Unknown
%% X Shopping (39 out of 85)
%% Charity
%% Entertainment
%% Internet and Mobile Service Provider
%% Finance (11 out of 27)
%% Gambling and Fantasy Sports
%% Art
%% Politics
%% Beauty and Cosmetics (0 combination ads)
% Most of the news ads (12 out of 14) and gaming were combination ads with an image and the article's title.
%% Other ad themes with mostly combination ads
%%% Business services (17 out of 21)
%%% Vehicles and Automotive (19 out of 23)
%%% Gaming (3 out of 4)
%%% Fitness and Sports Ads Collection (2 out of 2)
%%% Food and Drink Ads Collection (2 out of 3)
%%% Health (10 out of 15)
%% Pets and Animals (2 of 2)
%%% Funeral Services (1 of 1)
%%% Travel (18 out of 19)
%%% Computer Security (2 out of 2)
%%% Sponsored Brand (2 out of 2)
% Real Estate theme had 8 out of 15 combination ads.
%Over half of the ad themes (15 out of 24) had less than 10 ads. 





%\newpage
%In general, I'm not a fan of these types of transitions. I'd rather see something that logically leads to the next section. We're building up a narrative. What's the next step that's needed to set up our work?
\section{Findings}
We identified five key archiving and replay problems. First, we discuss two archiving problems that involved the Internet Archive's Save Page Now excluding ads and recent versions of Chrome being incompatible with Brozzler. Second, we describe three replay problems caused by random values in URLs, by a non-existent URL being requested, and by a Chromium bug that prevented service workers from accessing resources in an  ``\texttt{about:blank}'' iframe. 
%Internet Archive's web archiving service (Save Page Now), recent versions of Chrome, and ad services all caused archiving or replay problems.
%First, we will discuss the archiving problems that were caused by Internet Archive's web archiving service (Save Page Now) and Brozzler's incompatibility problem with recent versions of Chrome. Second, we will focus on the replay problems caused by ad services and Chrome's implementation of its service workers.



%Wikipedia does have a section that mentions that ad servers started being blocked during January 2022: https://en.wikipedia.org/wiki/Wayback_Machine#Recent_event_history 
%
\subsection{Archiving Problems}
Two problems prevented web ads from being archived: Internet Archive's Save Page Now intentionally excluding ads and Brozzler being incompatible with recent versions of Google Chrome.
%While creating our dataset, we identified five problems that can prevent a web ad from being archived or replayed. The first problem was caused by Internet Archive's Save Page Now blocking ads from being archived. The second problem was Brozzler's incompatibility with recent versions of Google Chrome. The third problem involves Google and Amazon ad services dynamically generating URLs with random values that prevent an ad from being replayed. The Flashtalking ad service caused the fourth problem of replaying an embedded web page ad outside of an ad iframe. The last problem was ads not loading during replay depending on the browser.
%Ad services, Save Page Now, and recent versions of Chrome caused five problems with archiving and replaying web advertisements from our dataset. 
%Ad services, Save Page Now, and recent versions of Chrome caused five problems that prevented ads from being archived or replayed. 



%\newpage
\subsubsection{Internet Archive’s Save Page Now Blocks Ads From Being Archived}
%When viewing ads that were available on Wayback Machine using URL search, most of the advertisements were archived or uploaded by Perma.cc, ArchiveTeam, and Archive-It users.
%It is possible to replay archived ads from the Wayback Machine, but certain users are blocked from archiving some advertisements.
%Another issue we noticed when trying to archive ads during January through June of 2023, was that Internet Archive's Save Page Now blocked ads from being archived. 
%The Internet Archive's Save Page Now service blocked ads from being archived. 
%
%It is possible to replay archived ads from the Wayback Machine, but Save Page Now users are blocked from archiving some advertisements using Save Page Now.
%
%When viewing ads that were available on Wayback Machine using URL search for ads with the prefix ``https://s0.2mdn.net/''~\footnote{Prefix search URL for some Google ads: \url{https://web.archive.org/web/*/https://s0.2mdn.net/*}}, ``https://s-static.innovid.com/''~\footnote{Prefix search URL for Innovid ads: \url{https://web.archive.org/web/*/https://s-static.innovid.com/*}}, and ``https://cdn.flashtalking.com''~\footnote{Prefix search URL for Flashtalking ads: \url{https://web.archive.org/web/*/https://cdn.flashtalking.com*}}, most of the recent advertisements (before August 2023) were archived or uploaded by Perma.cc, Archive Team, and Archive-It users. Save Page Now users were not able to archive most ads. Image, video, and web page ads were getting blocked. Google SafeFrames and other ad iframes that are needed to load the ads were also getting blocked. There were some cases where we successfully archive ads using Save Page Now (Figure \ref{fig:adArchivedBySPN}), but most ads were getting blocked. Around August of 2023, Save Page Now allowed more ads to be archived when the user archives the URI-R associated with the ad. Save Page Now is still blocking Google ads from being archived if someone archives a web page that loads Google ads (Figure \ref{fig:adBlockedBySPN}). The image ad in Figure~\ref{fig:adIdentifiedBySPN} was not archived after the crawling session and was still not archived when searching for this ad a week later (Figure \ref{fig:adNotArchivedBySPN}).
We inspected ads available on Wayback Machine using three URL searches with the prefixes ``https://s0.2mdn.net/''~\footnote{Prefix search URL for some Google ads: \url{https://web.archive.org/web/*/https://s0.2mdn.net/*}}, ``https://s-static.innovid.com/''~\footnote{Prefix search URL for Innovid ads: \url{https://web.archive.org/web/*/https://s-static.innovid.com/*}}, and ``https://cdn.flashtalking.com''~\footnote{Prefix search URL for Flashtalking ads: \url{https://web.archive.org/web/*/https://cdn.flashtalking.com*}}.
%
Perma.cc, Archive Team, and Archive-It users archived or uploaded most of the recent (before August 2023) ads, but Internet Archive prevented Save Page Now users from archiving image, video, and web page ads, as well as the Google SafeFrames and other ad iframes needed to load them. 
%
%There were some cases where we successfully archive ads using Save Page Now (Figure \ref{fig:adArchivedBySPN}), but most ads were getting blocked. 
%By contrast, Save Page Now successfully archived ads using services like Outbrain (Figure \ref{fig:adArchivedBySPN}). 
% By contrast, ads that used the Outbrain ad service were not excluded from being archived by Save Page Now (Figure \ref{fig:adArchivedBySPN}). 


After August 2023, Save Page Now allowed the archiving of more ads, provided the user directly archived the URI-R associated with the ad. 
%Save Page Now is still blocking Google ads from being archived if someone archives a web page that loads Google ads (Figure \ref{fig:adBlockedBySPN}). The image ad in Figure~\ref{fig:adIdentifiedBySPN} was not archived after the crawling session and was still not archived when searching for this ad a week later (Figure \ref{fig:adNotArchivedBySPN}).
But in October 2023, we archived a web page that loaded Google ads and found that Save Page Now still blocked those ads from being archived (Figure \ref{fig:adBlockedBySPN}). The image ad in Figure \ref{fig:adIdentifiedBySPN} was not archived (Figure \ref{fig:adNotArchivedBySPN}).

%was not archived after the crawling session or a week later (Figure \ref{fig:adNotArchivedBySPN}).
%%It was not archived:  https://web.archive.org/web/*/https://s0.2mdn.net/simgad/4399882211496353364
%
%\begin{figure}[tbp] \centering \includegraphics[scale=0.20]{Images/Ad_Archived_by_Save_Page_Now.png} \caption{Archived web page with ads archived by Save Page Now. Video: \url{https://youtu.be/D3Xt8zJpEFU?t=2419}} \label{fig:adArchivedBySPN} \end{figure}

% Moreover, it was still not archived when we searched for it a week later (Figure \ref{fig:adNotArchivedBySPN})

\begin{figure}[tbp]
     \centering
     \begin{subfigure}[b]{\textwidth}
        \centering
         \includegraphics[scale=0.35]{Images/Left_window_Save_Page_Now_did_not_archive_Google_Ad.png}
         \caption{URI-Rs identified during a crawling session. The URI-R for an image ad is highlighted.}
         \label{fig:adIdentifiedBySPN}        
     \end{subfigure}
     \vfill
     \begin{subfigure}[b]{\textwidth}
        \centering
         \includegraphics[scale=0.35]{Images/Right_Window_Save_Page_Now_did_not_archive_Google_Ad.png}
         \caption{The URI-R for the image ad was not archived.}
         \label{fig:adNotArchivedBySPN}
     \end{subfigure}
     \caption{Save Page Now identified the URI-R for an image associated with an ad during a crawling session, but did not archive the image.}
     \label{fig:adBlockedBySPN}
     %\hfill
\end{figure}

%\begin{figure}[tbp] \centering \includegraphics[scale=0.20]{Images/Save_Page_Now_did_not_archive_Google_Ad.png} \caption{Save Page Now identified the URI-R for an image associated with an ad during the crawling session, but did not archive the image. The left part of the figure shows the URI-Rs identified during the crawling session and the URI-R for the image ad is highlighted. The right part of the figure shows that the URI-R for the image ad was not archived.} \label{fig:adNotArchivedBySPN} \end{figure}

% For this example, we created files with a name that included ad related words like ``ad'' and ``advertisement'' to see if Save Page Now would block the resource from being archived.

%\newpage
Save Page Now also blocked URLs that included an ad-related file or directory name in the URL's path (Figure \ref{fig:webPageBlockedByFileName}). 
%
To explore these findings further, we created a video playlist~\footnote{Video playlist: \url{https://www.youtube.com/playlist?list=PLYiVfucTlg-MZYrYfLyKF2OR6DbVrDnvf}} to show ad related file names and directory names that caused Save Page Now to block their URLs. 
%
Blocked file names included ``imgAd.jpg'', ``displayAds.js'', ``videoAd.mp4'', and ``webAd.png''. However, if an ad-related file name did not include a file extension (like a file named ``imgAd''), Save Page Now did not block it (Figure \ref{fig:imgAdExample}).  
%
Blocked directory names included ``Advertisement\_files'' (Figure \ref{fig:adDirectoryExample}), ``displayAds'', ``videoAd'', ``webAd'', and ``ads''. 

\begin{figure}[tbp]
     \centering
     \begin{subfigure}[b]{\textwidth}
        \centering
         \includegraphics[scale=0.3]{Images/left_window_example_file_name_blocked_by_IA_SPN_cropped.png}
         \caption{Web page with ``Advertisement'' in the file name. URI-R: \url{https://treid003.github.io/Block_Ads_By_Regular_Expression/Advertisement_one.htm}}
         \label{fig:webPageWithAdRelatedFileName}        
     \end{subfigure}
     \vfill
     \begin{subfigure}[b]{\textwidth}
        \centering
         \includegraphics[scale=0.3]{Images/right_window_example_file_name_blocked_by_IA_SPN_cropped.png}
         \caption{The web page was blocked from being archived}
         \label{fig:SPNblocksWebPageWithAdRelatedName}
     \end{subfigure}
     \caption{Save Page Now previously blocked some web resources from being archived when ad related file names were used}
     \label{fig:webPageBlockedByFileName}
\end{figure}

\begin{figure}[tbp]
     \centering
     \begin{subfigure}[b]{\textwidth}
        \centering
         \includegraphics[width=\textwidth]{Images/imgAd_jpg_blocked.png}
         \caption{imgAd.jpg was blocked by Save Page Now (video: \url{https://youtu.be/LmGPc7KdcL4?t=264}). This URL is no longer blocked. URI-M: \url{https://web.archive.org/web/20240403212831/https://savingads.github.io/files/temp/imgAd.jpg}}
         \label{fig:imgAdWithExtensionBlocked}        
     \end{subfigure}
     \vfill
     \begin{subfigure}[b]{\textwidth}
        \centering
         \includegraphics[width=\textwidth]{Images/imgAd_no_extension_cropped.png}
         \caption{imgAd was not blocked by Save Page Now. URI-M: \url{https://web.archive.org/web/20230621000547/https://savingads.github.io/no_extension/imgAd}}
         \label{fig:imgAdWithoutExtensionArchived}
     \end{subfigure}
     \caption{If the file name does not include a file extension, the ad related file name will not cause the URL to be blocked.}
     \label{fig:imgAdExample}
\end{figure}

% allAds.csv
%Archived: https://youtu.be/mxep4RzKUUQ?t=2486
%Blocked: https://youtu.be/MflGE016o28?t=4816


%Block_Ads_By_Regular_Expression.html
% Archived: https://youtu.be/MflGE016o28?t=3135
% Blocked: https://youtu.be/MflGE016o28?t=3301
\begin{figure}[tbp]
     \centering
     \begin{subfigure}[b]{\textwidth}
        \centering
         \includegraphics[width=\textwidth]{Images/url_blocked_with_ad_directory.png}
         \caption{When Block\_Ads\_By\_Regular\_Expression.html was in a directory named Advertisement\_files the URL was blocked by Save Page Now (video: \url{https://youtu.be/MflGE016o28?t=3301}). This URL is no longer blocked. URI-M: \url{https://web.archive.org/web/20240403234751/https://treid003.github.io/Advertisement_files/Block_Ads_By_Regular_Expression.html}} 
         \label{fig:directoryWithAdName}        
     \end{subfigure}
     \vfill
     \begin{subfigure}[b]{\textwidth}
        \centering
         \includegraphics[width=\textwidth]{Images/url_archived_without_ad_directory.png}
         \caption{Block\_Ads\_By\_Regular\_Expression.html was archived when the directory it is in did not have an ad related name. URI-M: \url{https://web.archive.org/web/20230607173846/https://treid003.github.io/files/Block_Ads_By_Regular_Expression.html}}
         \label{fig:regularDirectoryName}
     \end{subfigure}
     \caption{When a directory in the URL's path has an ad related name, Save Page Now had blocked the URL.}
     \label{fig:adDirectoryExample}
\end{figure}


%\begin{figure}[tbp] \centering \includegraphics[scale=0.20]{Images/imgAds_blocked.jpeg} \caption{imgAd.jpg blocked by Save Page Now} \label{fig:imgAdBlockedBySPN} \end{figure}

%\begin{figure}[tbp] \centering \includegraphics[width=\textwidth]{Images/example_file_name_blocked_by_IA_SPN_cropped.png} \caption{} \label{fig:fileNameBlocked} \end{figure}

As of June 2023, Save Page Now also blocked social media accounts with an ad-related username if the social media platform used the username as a directory in the URL's path (Figure \ref{fig:twitterAccountBlockedBySPN}). For example, Save Page Now blocked our effort to archive \texttt{displayads}'s tweet (Figure \ref{fig:twitterAccountBlockedBySPN}) because the name ``displayads'' was a directory in the URL's path. We communicated with Wayback Machine staff about this issue (in August of 2023) and they made this account archivable (Figure \ref{fig:twitterAccountArchivedBySPN}). 
%
Save Page Now also blocked ads on social media websites like Instagram (Figures \ref{fig:videoAdInstagramLive} and \ref{fig:videoAdInstagramBlocked}), Twitch (Figures \ref{fig:webAdTwitchLive} and \ref{fig:webAdTwitchBlocked}), and Facebook (Figures \ref{fig:videoAdFacebookLive} and \ref{fig:videoAdFacebookBlocked}). 

\begin{figure}[tbp]
     \centering
     \begin{subfigure}[b]{\textwidth}
        \centering
         \includegraphics[width=\textwidth]{Images/social_media_account_blocked_twitter_cropped.jpeg}
         \caption{Social media accounts with ad related usernames were getting blocked by Save Page Now.}
         \label{fig:twitterAccountBlockedBySPN}        
     \end{subfigure}
     \vfill
     \begin{subfigure}[b]{\textwidth}
        \centering
         \includegraphics[width=\textwidth]{Images/displayAds_social_media_account_archived.png}
         \caption{It is now possible to archive social media accounts with an ad related username. URI-M: \url{https://web.archive.org/web/20230809162244/https://twitter.com/displayads/status/128664060186214400}}
         \label{fig:twitterAccountArchivedBySPN}
     \end{subfigure}
     \caption{Before June 2023, some social media accounts with ad related usernames were blocked by Save Page Now}
     \label{fig:socialMediaAccountWithAdUsernames}
\end{figure}


%Instagram example
%% Live web: https://youtu.be/vskyvNrdjqw?si=P3ZBaU99Iu5d3wp3&t=2698
%% Blocked: https://youtu.be/vskyvNrdjqw?t=2653
% Twitch example
%% Live web: https://youtu.be/vskyvNrdjqw?si=XJCrD1Jax3nevY88&t=2996
%% Blocked: https://youtu.be/vskyvNrdjqw?t=3046
% Facebook example
%% Live web: https://youtu.be/vskyvNrdjqw?si=5nazfeXL2DXmM38p&t=2752
%% Blocked: https://youtu.be/vskyvNrdjqw?t=2760
\begin{figure}[tbp]
     \centering
     % Instagram
     \begin{subfigure}[b]{0.4\textwidth}
        \centering
         \includegraphics[width=\textwidth]{Images/video_ad_account_instagram.png}
         \caption{videoad account on Instagram. URI-R: \url{https://www.instagram.com/videoAd/}}
         \label{fig:videoAdInstagramLive}        
     \end{subfigure}
     \hfill
     \begin{subfigure}[b]{0.5\textwidth}
        \centering
         \includegraphics[width=\textwidth, valign=T]{Images/instagram_account_blocked.png}
         \caption{videoad account on Instagram was blocked by Save Page Now. Video: \url{https://youtu.be/vskyvNrdjqw?t=2653}}
         \label{fig:videoAdInstagramBlocked}
     \end{subfigure}
     \vfill
     %Twitch
     \begin{subfigure}[b]{0.4\textwidth}
        \centering
         \includegraphics[width=\textwidth]{Images/webad_twitch_account.png}
         \caption{webad account on Twitch. URI-R: \url{https://www.twitch.tv/webAd/about}}
         \label{fig:webAdTwitchLive}        
     \end{subfigure}
     \hfill
     \begin{subfigure}[b]{0.5\textwidth}
        \centering
         \includegraphics[width=\textwidth]{Images/twitch_account_blocked.png}
         \caption{webad account on Twitch was blocked by Save Page Now. Video: \url{https://youtu.be/vskyvNrdjqw?t=3046}}
         \label{fig:webAdTwitchBlocked}
     \end{subfigure}    
      \vfill
     %Facebook
     \begin{subfigure}[b]{0.4\textwidth}
        \centering
         \includegraphics[width=\textwidth]{Images/videoAd_facebook_account.png}
         \caption{videoad account on Facebook. URI-R: \url{https://www.facebook.com/videoAd/}}
         \label{fig:videoAdFacebookLive}        
     \end{subfigure}
     \hfill
     \begin{subfigure}[b]{0.5\textwidth}
        \centering
         \includegraphics[width=\textwidth]{Images/facebook_account_blocked.png}
         \caption{videoad account on Facebook was blocked by Save Page Now. Video: \url{https://youtu.be/vskyvNrdjqw?t=2760}}
         \label{fig:videoAdFacebookBlocked}
     \end{subfigure} 
     \caption{Example of other social media accounts that were previously blocked by Save Page Now}
     \label{fig:otherSocialMediaAccountsBlocked}
\end{figure}

%In the past, Save Page Now has blocked ad related URLs from being archived. This prevented ads and other web pages like social media accounts from being archived. We next describe the last archiving problem we noticed when trying to ads for our dataset. 

\subsubsection{Brozzler’s Incompatibility With Recent Versions of Google Chrome}
Brozzler became incompatible~\footnote{\url{https://github.com/internetarchive/brozzler/issues/256}} with versions of Google Chrome that released after March 2023.
%Beginning with the March 2023 version (111.0.5563.110) of Google Chrome \cite{chrome-v111} was incompatible with Brozzler.\footnote{\url{https://github.com/internetarchive/brozzler/issues/256}} 
%The March 2023 version (111.0.5563.110) of Google Chrome \cite{chrome-v111} was incompatible with Brozzler \cite{Levitt-Brozzler-gh14}.
% 
%
%Removed from JASIST version
When we attempted to archive a web page using the \texttt{brozzle-page} command (Listing \ref{BrozzlerCommands}), Brozzler failed to load the web page (Figure \ref{fig:webPageFailedToLoad}), which prevented it from being archived (Figure \ref{fig:failedToArchive}). 
%When we attempted to archive a web page using the \texttt{brozzle-page} command, Brozzler failed to load it (Figure \ref{fig:webPageFailedToLoad}), which  prevented the web page from being archived (Figure \ref{fig:failedToArchive}). 
%
\begin{figure}[tbp]
    \centering
    \includegraphics[scale=0.175]{Images/web_page_failed_to_load.png}
    \caption{The web page does not load when Brozzler is archiving a web page with a Chrome version released after version 110. GitHub issue: \url{https://github.com/internetarchive/Brozzler/issues/256}}
    \label{fig:webPageFailedToLoad}
\end{figure}
%
%This results in the web page not being archived successfully (Figure \ref{fig:failedToArchive}). 
%
\begin{figure}[tbp]
    \centering
    \includegraphics[scale=0.19]{Images/web_page_not_archived_cropped.png}
    \caption{Brozzler failed to archive a web page (https://www.ign.com/articles/the-last-of-us-season-1-review) with Chrome version 113.0.5672.126. WARC file: \url{https://zenodo.org/records/10373135/files/WARCPROX-20230519163909687-00000-0so5t1md.warc?download=1}}
    \label{fig:failedToArchive}
\end{figure}
%
%The commands used for this example are shown in Listing \ref{BrozzlerCommands}. 
%
When we executed these commands on Ubuntu (22.04.2 and 20.04.6 LTS) and macOS (Ventura 13.3.1), a ``WebSocketBadStatusException: Handshake status 403 Forbidden'' error occurred (Figure \ref{fig:webSocketBadStatusException}).  
%
When we employed these same commands in early 2023, however, Brozzler loaded the desired web page (Figure \ref{fig:loadedWebPage}).  

% Removed from JASIST version
\noindent\begin{minipage}[tb]{\textwidth} \begin{lstlisting}[language=bash, breaklines=true, label=BrozzlerCommands, caption=The commands that were used for this example. Video: \url{https://youtu.be/A-zr6zVTZSo?t=4888}]
warcprox -p 8081 -d ./warcs/IGN/brozzle_page/2023_05_19 
--dedup-db-file /dev/null
export Brozzler_EXTRA_CHROME_ARGS="--ignore-certificate-errors" 
brozzle-page --chrome-exe '/usr/bin/google-chrome' --proxy localhost:8081 'https://www.ign.com/articles/the-last-of-us-season-1-review' \end{lstlisting} \end{minipage}


\begin{figure}[tbp]
    \centering
    \includegraphics[scale=0.20]{Images/webSocketBadStatusError.png}
    \caption{A WebSocketBadStatusException occurs when using Brozzler to archive web pages with versions of Chrome after version 110.}
    \label{fig:webSocketBadStatusException}
\end{figure}

\newpage
After testing Brozzler on a range of web pages, we concluded that browser incompatibility is to blame.
%This browser compatibility problem also occurred when trying to archive other web pages. We selected the web pages below because Brozzler successfully archived them at the beginning of 2023~\footnote{Video: \url{https://www.youtube.com/live/0n_CcYm1Z90}}:
%
%\begin{itemize} \item \url{https://www.tomshardware.com/reviews/nvidia-geforce-rtx-4090-review/3} \item \url{https://armorgames.com/} \item \url{https://www.w3schools.com/tags/default.asp} \item \url{https://www.geeksforgeeks.org/python-classes-and-objects/?ref=lbp} %\item \url{https://mortalkombat.fandom.com/wiki/Tag_Team_Ladder} %\item \url{https://www.youtube.com/watch?v=PZShwWiepeY} %\item \url{https://arquivo.pt/services/complete-page?url=https://pt.ign.com/&timestamp=20220202150304} %\item \url{https://www.bloomberg.com/news/newsletters/2023-01-08/when-will-apple-launch-the-reality-pro-mixed-reality-headset-apple-2023-devices-lcnfzkc7}  %\item \url{https://www.theverge.com/23496511/ces-2023-news-products-announcements-tvs-laptops-fitness-trackers} 
%\end{itemize}
%
%At the end of 2023, the latest Google Chrome version (120.0.6099.109) \cite{chrome-v120} was also incompatible with Brozzler~\footnote{Video: \url{https://youtu.be/6eIa8A87Rq8}}. 
%
%When these commands were used at the beginning of that year (January 10, 2023), Brozzler was able to load the web page that needed to be archived (Figure \ref{fig:loadedWebPage}).  
%
\begin{figure}[tbp]
    \centering
    \includegraphics[scale=0.25]{Images/archive_successful_earlier_this_year.png}
    \caption{Brozzler was able to load web pages during crawl time with earlier versions of Chrome released at the beginning of 2023. Video: \url{https://www.youtube.com/live/0n_CcYm1Z90?feature=share&t=41} }
    \label{fig:loadedWebPage}
\end{figure}
%
The last stable version of Google Chrome that worked with Brozzler is version 110.0.5481.177, (released\footnote{Chrome 110.0.5481.177: \url{https://chromereleases.googleblog.com/2023/02/stable-channel-desktop-update_22.html}} on February 22, 2023). Originally released in mid-December 2023, the Chrome version (120.0.6099.109)\footnote{Chrome 120.0.6099.109: \url{https://chromereleases.googleblog.com/2023/12/stable-channel-update-for-desktop_12.html}} remained incompatible with Brozzler~\footnote{Video: \url{https://youtu.be/6eIa8A87Rq8}}. 


%Cut this.  In general, cut all of these "wrapup" paragraphs. The narrative should flow such that these aren't needed. If you need an introduction to the next problem, it should come as the first sentence of the next section.
%Brozzler's incompatibility with Chrome was the last archiving problem that we identified. 
%
%The next section will discuss the first replay problem we identified which involved Google's ad service dynamically generating URLs that prevented the replay of ads.


\subsection{Replay Problems}
When we sought to replay ads from our dataset, we encountered three obstacles. First, the JavaScript for Google and Amazon ad services dynamically generated URLs with random values. Second, the JavaScript for Flashtalking ad service prevented the replay of an embedded web page ad outside of an ad iframe. Finally, some ads did not load during replay depended on the browser.
%Ad services triggered most of the replay problems that we identified. For example, Google and Amazon ad services dynamically generate random values in the ad iframe's URL. Second, the FlashTalking ad service returns an HTTP status code of 404 if the ad is not loaded in the correct iframe. Third, depending upon the ad service used and how the Save Page Now user archives the ad, Save Page Now blocks ads in some cases. Fourth, encountering variable replay depending upon the browser used. Fifth, recent versions of Google Chrome are incompatible with Brozzler, which prevents ads from being archived.

%\newpage
% Summarize the comments below and reference the background section
\subsubsection{Replaying Google SafeFrames} \label{Replaying_Google_SafeFrame}
% Example for domain rewriting: https://www.ndss-symposium.org/ndss-paper/melting-pot-of-origins-compromising-the-intermediary-web-services-that-rehost-websites/
% Example for replacing JavaScript: https://www.usenix.org/conference/osdi22/presentation/goel 
Google SafeFrame uses a random value in the iframe's URL's subdomain (Figure \ref{fig:safeFrameURI}), which prevents replay.
%
%One problem we identified with replaying an ad that uses a Google SafeFrame is that Google uses a random value in the subdomain (Figure \ref{fig:safeFrameURI}) for the iframe's URL.
%
%On the live web, Google SafeFrames that use the \texttt{useUniqueDomain} option generate a random value that is different each time the web page is loaded.   
%When loading a Google SafeFrame on the live web, the random value should be different each time the web page is loaded.
%When loading a Google SafeFrame on the live web, it is expected for the random value to be different each time the web page is loaded.
%This also occurred when replaying an archived web page that loaded a Google SafeFrame for an ad (Table \ref{table:SafeFrameSubdomainChangesWhenLoadingAdDuringReplay}), which resulted in the ad failing to replay, because an unarchived SafeFrame URL was requested.
%% This occurs for a different reason, which is that the execution order for the random number functions can be different each time the web page is replayed when multiple JavaScript files are using the Random API.
%This resulted in an unarchived SafeFrame URL being requested during replay time, which prevents the ad content from being loaded.
\begin{table}[tbp]
\begin{tabular}{|l|l|}
\hline
\multicolumn{1}{|c|}{\begin{tabular}[c]{@{}c@{}}Replay Session \\ Number\end{tabular}} & \multicolumn{1}{c|}{Random Value in Google SafeFrame URL} \\ \hline
1                                                                                      & af393d3d232450caab92d97eaefb484e                \\ \hline
2                                                                                      & 36dc52191b8e81186b187c938af4b280                \\ \hline
3                                                                                      & 5f68c90c97e25bf663f52ef786eb49b8                \\ \hline
4                                                                                      & f663f52ef786eb49b8d803369fd0abea                \\ \hline
5                                                                                      & 6d18ef14a03123734d453dee25a8be6e                \\ \hline
6                                                                                      & 7a9317739c43b98082f4e77ed17fe3fa                \\ \hline
7                                                                                      & 4f8332dc6d18ef14a03123734d453dee                \\ \hline
8                                                                                      & 5bf663f52ef786eb49b8d803369fd0ab                \\ \hline
9                                                                                      & 227cd10c62e4b3ea26c2bd42e587e2c5                \\ \hline
10                                                                                     & 173e9e9bad424f8332dc6d18ef14a031                \\ \hline
\end{tabular}
\caption{When loading the ad iframe (Google SafeFrame) for a Google ad, the random value in the Google SafeFrame URL differed each time the archived web page was replayed. (URI-R: \url{https://mortalkombat.fandom.com/wiki/Tag_Team_Ladder} | WARC: \url{https://zenodo.org/record/7601187/files/2023-01-11_00-59-34_ads_on_fandom_browsertrix_crawler.warc.gz?download=1})} \label{table:SafeFrameSubdomainChangesWhenLoadingAdDuringReplay}
\end{table}
%Loading an ad inside of a Google SafeFrame prevents the replay of the ad because Google SafeFrames use a random value in the subdomain (Figure \ref{fig:safeFrameRandomSubdomain}) for the URL. 
%The seven replay systems we tested (pywb, OpenWayback, ReplayWeb.page, Conifer, Wayback Machine, and Arquivo.pt) cannot generate the same random value generated during crawl time (Figure \ref{fig:safeFrameLivevsReplay}). 
%
When the seven replay systems we tested (pywb, OpenWayback, ReplayWeb.page, Conifer, Wayback Machine, and Arquivo.pt) executed JavaScript code that generated a random number, the random value differed from the random value generated during crawl time. These replay systems also generated different random values on each replay. Drawing upon public web archives, Aturban et al. \cite{aturban-plosone-2023} discussed this problem of inconsistent replay of archived web pages.
%% Not a footnote. Let's cite relevant work in the main body of the paper.
%
%\begin{figure}[tbp] \centering \includegraphics[scale=0.20]{Images/SafeFrame_Random_Subdomain_example_1_v1_Annotated.png} \caption{Google SafeFrames have a URL with a random value in the subdomain} \label{fig:safeFrameRandomSubdomain} \end{figure}
\begin{figure}[tbp]
    \centering
    \includegraphics[scale=0.20]{Images/SafeFrame_random_value_live_vs_replay.png}
    \caption{Currently web archive replay systems cannot generate the same random subdomain that was used during crawl time}
    \label{fig:safeFrameLivevsReplay}
\end{figure}
%
Since we successfully archived the Google SafeFrame and the ad content (Figures \ref{fig:safeFrameArchived} and \ref{fig:safeFrameAdArchived}), this problem was caused by the replay system, not the crawler. 
%
When ReplayWeb.page, pywb, Conifer, or Wayback Machine attempted to load a Google SafeFrame for an advertisement, an HTTP status code of 404 (Not Found) was returned. 
%Save Page Now blocks this type of Google SafeFrame URL so this type of Google ad cannot be replayed using Wayback Machine unless an Archive-It user archived a web page with Google ads. 
%% SPN now allows users to archive the Google SafeFrame if they specify the SafeFrame URL. Will probably mention the issue of SPN users only recently being able to archive ads directly in a later section
In contrast, when loading a Google SafeFrame with Arquivo.pt, an HTTP status code of 307 (Temporary Redirect) was returned. Arquivo.pt's replay system changed the timestamp for the archived Safeframe's URI-M, but failed to load the ad content into the successfully archived SafeFrame. 

\begin{figure}[tbp]
    \centering
    \includegraphics[width=\textwidth]{Images/cropped_SafeFrame_Archived_annotated_v2.png}
    \caption{Google SafeFrame successfully archived by ArchiveWeb.page. WACZ: \url{https://zenodo.org/records/10373131/files/safeframe-example.wacz?download=1} | SafeFrame URI-R: \url{https://e76308bcf1c30aa4c853507f4b382285.safeframe.googlesyndication.com/safeframe/1-0-40/html/container.html}}
    \label{fig:safeFrameArchived}
\end{figure}
%
\begin{figure}[tbp]
    \centering
    \includegraphics[scale=0.2]{Images/cropped_SafeFrame_Ad_Content_Archived.png}
    \caption{Ad content successfully archived by ArchiveWeb.page}
    \label{fig:safeFrameAdArchived}
\end{figure}

%\newpage 
%First, we will show the difficulties with replaying random numbers and replaying Google ads that use a SafeFrame. 
%\subsubsection{Generating Random Numbers and SafeFrame URLs}
Since the random value in a Google SafeFrame URL changed each time a Google ad was replayed (Table \ref{table:SafeFrameSubdomainChangesWhenLoadingAdDuringReplay}), we created an example web page \footnote{Demo web page for generating random numbers and Google SafeFrames: \url{https://treid003.github.io/random_Values_external_JS_with_async.html}} (Figure \ref{fig:randomValuesAndSafeFrameCrawlTime}) that used ad code from Google's \href{https://web.archive.org/web/20230113005605id_/https://securepubads.g.doubleclick.net/gpt/pubads_impl_2023010901.js?cb=31071543}{pubads\_impl\_2023020201.js}~\footnote{URI-M: \url{https://web.archive.org/web/20230113005605id_/https://securepubads.g.doubleclick.net/gpt/pubads_impl_2023010901.js?cb=31071543}} script to determine how the random values were generated for a Google SafeFrame. 
%
%Our example web page \cite{} (Figure _) uses ad code from Google's \href{https://web.archive.org/web/20230113005605/https://securepubads.g.doubleclick.net/gpt/pubads_impl_2023010901.js?cb=31071543}{pubads\_impl\_2023020201.js} script \cite{pubads-gpt23} to generate random numbers and Google SafeFrame URLs.
%
First, we generated random numbers using the same random functions as those used for Google SafeFrame (\texttt{Math.random()} \cite{math-random-mozillat24} and \texttt{window.crypto.getRandomValues()} \cite{crypto-random-mozillat24} functions). This permitted us to determine not only if the replay system generated the same random numbers as those generated during crawl time, but if the random numbers differed on each replay. 
%If the JavaScript code was executed synchronyously, then the random value would be the same for the SafeFrame. Since the JavaScript code is executed asynchronyously, the random number generator can be called a different number of times before loading the ad and this results in a different SafeFrame URL each time the web page is replayed. 
%
To check the random values generated during crawl time, we used ArchiveWeb.page and recorded a video of the crawling session~\footnote{Crawling session: \url{https://youtu.be/IzGMVmLyYGQ?t=2694}}.
%\cite{generating-random-numbers-yt23}. 
%
Second, we used the code from Google's pubads\_impl\_2023020201.js script to generate the random value for each SafeFrame URL. In the first section of the web page, two SafeFrame URLs were generated: the first URL used \texttt{Math.random()} and the other used \texttt{window.crypto.getRandomValues()}.
%
%The random value in the SafeFrame URL was generated using either the \texttt{Math.random()} or \texttt{window.crypto.getRandomValues()} function. 
%
As in the first step, we compared the values generated during crawl and replay time.
%
Lastly, to check if the replay system successfully replayed the archived Google SafeFrames, we created and embedded iframes into the web page for each SafeFrame URL~\footnote{When a Google SafeFrame is loaded without an ad it will be a blank iframe with no content inside of the iframe.}.

\begin{figure}[tbp]
    \centering
    \includegraphics[width=0.9\textwidth]{Images/random_values_and_safeframes_cropped_crawling_session_cropped_with_Border.png}
    \caption{Demo web page (shown during the crawling session) that generates random values and Google SafeFrames. URI-R: \url{https://treid003.github.io/random_Values_external_JS_with_async.html}}
    \label{fig:randomValuesAndSafeFrameCrawlTime}
\end{figure}


\begin{table}[tbp]
\begin{tabular}{lll}
\multicolumn{3}{c}{Random Values in Google SafeFrame URLs}                                                                                                            \\ \cline{2-3} 
\multicolumn{1}{l|}{}             & \multicolumn{1}{c|}{Generated Using Math.random()}    & \multicolumn{1}{c|}{Generated Using crypto.getRandomValues()} \\ \hline
\multicolumn{1}{|l|}{Crawl Time}  & \multicolumn{1}{l|}{c614e85f7b54b53618165b0c0c7a3ed3} & \multicolumn{1}{l|}{9235d575f33febc001f7e53ba7c7c876}         \\ \hline
\multicolumn{1}{|l|}{Replay Time} & \multicolumn{1}{l|}{328a37205b691401e879e520ba757b42} & \multicolumn{1}{l|}{70b75c8cd9262cb991fc0248a75df5d6}         \\ \hline
\end{tabular}
\caption{Random values in Google SafeFrame URLs generated during crawl time and replay time. Video: \url{https://youtu.be/IzGMVmLyYGQ?t=2697}} \label{table:SafeFrameSubdomainsCrawlTimeVsReplayTime}
\end{table}

Table \ref{table:SafeFrameSubdomainsCrawlTimeVsReplayTime} shows the random values generated for the Google SafeFrame URLs during crawl and replay sessions. 
%
%The random values are not the same when comparing the values generated during replay time and crawl time. 
The random values generated during replay and crawl time differed (Figure \ref{fig:randomValuesAndSafeFrameReplayTime}), because the seeds used during the crawling session for the random number generators differed from the seeds used by ReplayWeb.page. Replay systems employ rewriting tools like Wombat.js to overwrite the seed for the random number generators. This results in a more consistent replay where the random values generated should be the same upon each replay. 



Listings \ref{wombatMathRandom} and \ref{wombatCryptoRandom} show the JavaScript code used to initialize the \texttt{Math.random()} and \texttt{crypto.getRandomValues()}. When initializing \texttt{Math.random()} (Listing \ref{wombatMathRandom}), Wombat.js overwrote the random number generator's seed (on line 7) with an expression that includes the time of the resource's archiving. Wombat.js initialized \texttt{crypto.getRandomValues()} (Listing \ref{wombatCryptoRandom}) by overwriting the function (lines 6-10).
%
Seed selection will differ during the crawling and replay sessions because of different implementations for \texttt{Math.random()} and \texttt{crypto.getRandomValues()}.
%
For \texttt{Math.random()}, the seed selection is an ``implementation-defined'' strategy~\footnote{\url{https://tc39.es/ecma262/multipage/overview.html}}. This allows an external source to define its approach without recommendations from the standard specification \cite{math-random-ecma24}. The W3C API specification \cite{crypto-interface-w3c24} states that the random number generator for \texttt{crypto.getRandomValues()} should be seeded with a high-quality entropy source like ``/dev/urandom'', an operating system entropy source that retrieves environmental noise from device drivers \cite{random-linux-man24}.
%
%V8 might be the JavaScript engine used for Chromium browsers and they do have some documentation about the Math.random() implementation: https://v8.dev/blog/math-random
%This is different from replaying an archived Google ad that uses a Google SafeFrame, since the random subdomain is different each time the archived web page is loaded (Table \ref{table:SafeFrameSubdomainChangesWhenLoadingAdDuringReplay}).

\begin{figure}[tbp]
    \centering
    \includegraphics[width=\textwidth]{Images/random_values_and_safeframes_cropped_replay_session_cropped_with_Border.png}
    \caption{Demo web page shown during the replay session. The values generated during replay differ from the values during crawl time (Figure \ref{fig:randomValuesAndSafeFrameCrawlTime}). WACZ: \url{https://zenodo.org/records/10932857/files/2023-02-21-random-values-with-safeframes.wacz?download=1} | URI-R: \url{https://treid003.github.io/random_Values_external_JS_with_async.html}}
    \label{fig:randomValuesAndSafeFrameReplayTime}
\end{figure}


%
\noindent\begin{minipage}[tbp]{\textwidth}
\begin{lstlisting}[language=JavaScript, breaklines=true, label=wombatMathRandom, caption=Wombat.js overriding the seed for Math.random()]
//From Wombat.js. URI-R: https://replayweb.page/static/wombat.js
//Initializing Math.random() by overwriting the seed
f.prototype.initSeededRandom = function(t) {
    this.$wbwindow.Math.seed = parseInt(t);
    var e = this;
    this.$wbwindow.Math.random = function() {
        return e.$wbwindow.Math.seed = (9301 * e.$wbwindow.Math.seed + 49297) % 233280,
        e.$wbwindow.Math.seed / 233280
    }
}

/*...*/

f.prototype.wombatInit = function() { /*...*/ 
    this.initSeededRandom(this.wb_info.wombat_sec) 
    /*...*/}

/*...*/

function f(e, i) {/*...*/ this.wb_info = i, /*...*/}

/*...*/

const b = f; /*...*/   
window._WBWombatInit = function(t) {/*...*/ var e = new b(this,t); /*...*/}

//From the archived web page. URI-R: https://www.ign.com/tv/the-last-of-us-the-series
if (window && window._WBWombatInit) {window._WBWombatInit(wbinfo);}
/*...*/
//wombat_sec is the time when the web page was archived 2023-08-22
wbinfo.wombat_sec = "1692720944";
\end{lstlisting}
\end{minipage}
%

%\newpage
\noindent\begin{minipage}[t]{\textwidth}
\begin{lstlisting}[language=JavaScript, breaklines=true, label=wombatCryptoRandom, caption=Wombat.js overriding the crypto.getRandomValues() function]
//From Wombat.js. URI-R: https://replayweb.page/static/wombat.js
//Initializing crypto.getRandomValues() by overwriting the function
f.prototype.initCryptoRandom = function() {
    if (this.$wbwindow.crypto && this.$wbwindow.Crypto) {
        var t = this
          , e = function(e) {
            for (var r = 0; r < e.length; r++)
                e[r] = parseInt(4294967296 * t.$wbwindow.Math.random());
            return e
        };
        this.$wbwindow.Crypto.prototype.getRandomValues = e,
        this.$wbwindow.crypto.getRandomValues = e
    }
}
\end{lstlisting}
\end{minipage}

\begin{figure}[tbp]
    \centering
     \begin{subfigure}[b]{0.45\textwidth}
        \centering
         \includegraphics[width=\textwidth]{Images/random_values_generated_by_button_click.png}
         \caption{Random values are generated after clicking a button}
         \label{fig:rvButtonClick}        
     \end{subfigure}
     \hfill
     \begin{subfigure}[b]{0.45\textwidth}
        \centering
         \includegraphics[width=\textwidth]{Images/random_values_generate_by_scrolling.png}
         \caption{Random values generated after scrolling to this part of the web page}
         \label{fig:rvScrolling}
     \end{subfigure}
    \caption{New sections added to the demo web page that requires user interaction to generate the random values.}
    \label{fig:updatedGSFDemo}
\end{figure}

\begin{table}[tbp]
\begin{tabular}{lll}
                                                                                                                                   & \multicolumn{2}{l}{\begin{tabular}[c]{@{}l@{}}Random Values in Google SafeFrame URLs During Replay for the \\Section  Updated By Scrolling to the Bottom of the Web Page\end{tabular}}                   \\ \hline

%Total function calls
% Without button clicks there are 2 function calls for the first section and 1 call in the last section before generating the SafeFrame URL
% Two extra function calls for each button click
%\multicolumn{1}{|c|}{\begin{tabular}[c]{@{}c@{}}Number of function calls for \\ Math.random() and \\ crypto.getRandomValues() \\ Before Scrolling to the last section\end{tabular}} & \multicolumn{1}{c|}{\begin{tabular}[c]{@{}c@{}}Generated Using \\ Math.random()\end{tabular}} & \multicolumn{1}{c|}{\begin{tabular}[c]{@{}c@{}}Generated Using\\ crypto.getRandomValues()\end{tabular}} \\ \hline
%\multicolumn{1}{|l|}{3}                                                                                                            & \multicolumn{1}{l|}{75edc22fa0ea2f68a1e0ae51aedf6203}                                         & \multicolumn{1}{l|}{aa89e23772c94a19508c2923a84bed49}                                                   \\ \hline
%\multicolumn{1}{|l|}{5}                                                                                                            & \multicolumn{1}{l|}{cfb20651dc8e7d295dbd03dc7dd57827}                                         & \multicolumn{1}{l|}{4f693309e7fad65b456071c053d21559}                                                   \\ \hline
%\multicolumn{1}{|l|}{7}                                                                                                            & \multicolumn{1}{l|}{bfc665544160a197051235bd1083f606}                                         & \multicolumn{1}{l|}{c941958a67339666e399eb523c9e282c}                                                   \\ \hline
%\multicolumn{1}{|l|}{9}                                                                                                            & \multicolumn{1}{l|}{0772e7dea6a02819114c37a4b768c6d3}                                         & \multicolumn{1}{l|}{b86e07e787554508f07e9c813beeb327}                                                   \\ \hline
%\multicolumn{1}{|l|}{11}                                                                                                            & \multicolumn{1}{l|}{0c2dfbe3bba19197b87d2cab13e2e4be}                                         & \multicolumn{1}{l|}{e8b3cf26ef3723cbce88f440ff133635}                                                   \\ \hline
%
%Button clicks
\multicolumn{1}{|c|}{\begin{tabular}[c]{@{}c@{}}Total Buttons \\ Clicked Before \\ Scrolling to the \\ last section\end{tabular}} & \multicolumn{1}{c|}{\begin{tabular}[c]{@{}c@{}}Generated Using \\ Math.random()\end{tabular}} & \multicolumn{1}{c|}{\begin{tabular}[c]{@{}c@{}}Generated Using\\ crypto.getRandomValues()\end{tabular}} \\ \hline
\multicolumn{1}{|l|}{0}                                                                                                            & \multicolumn{1}{l|}{75edc22fa0ea2f68a1e0ae51aedf6203}                                         & \multicolumn{1}{l|}{aa89e23772c94a19508c2923a84bed49}                                                   \\ \hline
\multicolumn{1}{|l|}{1}                                                                                                            & \multicolumn{1}{l|}{cfb20651dc8e7d295dbd03dc7dd57827}                                         & \multicolumn{1}{l|}{4f693309e7fad65b456071c053d21559}                                                   \\ \hline
\multicolumn{1}{|l|}{2}                                                                                                            & \multicolumn{1}{l|}{bfc665544160a197051235bd1083f606}                                         & \multicolumn{1}{l|}{c941958a67339666e399eb523c9e282c}                                                   \\ \hline
\multicolumn{1}{|l|}{3}                                                                                                            & \multicolumn{1}{l|}{0772e7dea6a02819114c37a4b768c6d3}                                         & \multicolumn{1}{l|}{b86e07e787554508f07e9c813beeb327}                                                   \\ \hline
\multicolumn{1}{|l|}{4}                                                                                                            & \multicolumn{1}{l|}{0c2dfbe3bba19197b87d2cab13e2e4be}                                         & \multicolumn{1}{l|}{e8b3cf26ef3723cbce88f440ff133635}                                                   \\ \hline
\end{tabular}
\caption{This table shows that the random value in a Google SafeFrame URL can change on each replay if the number of function calls to the \texttt{Math.random()} and \texttt{crypto.getRandomValues()} differs before creating the SafeFrame URL. In this example, each button click resulted in two extra function calls to \texttt{Math.random()} and \texttt{crypto.getRandomValues()}. If the number of function calls to the \texttt{Math.random()} and \texttt{crypto.getRandomValues()} are the same on each replay, then the random values generated will also be the same as in Table \ref{table:SafeFrameSubdomainsReplayTime}. (URL: \url{https://treid003.github.io/random_Values_external_JS_with_async.html} | WACZ: \url{https://zenodo.org/records/13695291/files/2024-09-04-random-values-with-safeframes.wacz?download=1})}
\label{table:differentRandomSafeFrameURLs}
\end{table}


\begin{table}[tbp]
\begin{tabular}{lll}
\multicolumn{3}{c}{Random Values in Google SafeFrame URLs During Replay}                                                                                                                                                                                                                                     \\ \hline
\multicolumn{1}{|c|}{\begin{tabular}[c]{@{}c@{}}Replay Session \\ Number\end{tabular}} & \multicolumn{1}{c|}{\begin{tabular}[c]{@{}c@{}}Generated Using \\ Math.random()\end{tabular}} & \multicolumn{1}{c|}{\begin{tabular}[c]{@{}c@{}}Generated Using \\ crypto.getRandomValues()\end{tabular}} \\ \hline
\multicolumn{1}{|l|}{1}                                                                & \multicolumn{1}{l|}{328a37205b691401e879e520ba757b42}                                         & \multicolumn{1}{l|}{70b75c8cd9262cb991fc0248a75df5d6}                                                    \\ \hline
\multicolumn{1}{|l|}{2}                                                                & \multicolumn{1}{l|}{328a37205b691401e879e520ba757b42}                                         & \multicolumn{1}{l|}{70b75c8cd9262cb991fc0248a75df5d6}                                                    \\ \hline
\multicolumn{1}{|l|}{3}                                                                & \multicolumn{1}{l|}{328a37205b691401e879e520ba757b42}                                         & \multicolumn{1}{l|}{70b75c8cd9262cb991fc0248a75df5d6}                                                    \\ \hline
\multicolumn{1}{|l|}{4}                                                                & \multicolumn{1}{l|}{328a37205b691401e879e520ba757b42}                                         & \multicolumn{1}{l|}{70b75c8cd9262cb991fc0248a75df5d6}                                                    \\ \hline
\multicolumn{1}{|l|}{5}                                                                & \multicolumn{1}{l|}{328a37205b691401e879e520ba757b42}                                         & \multicolumn{1}{l|}{70b75c8cd9262cb991fc0248a75df5d6}                                                    \\ \hline
\multicolumn{1}{|l|}{6}                                                                & \multicolumn{1}{l|}{328a37205b691401e879e520ba757b42}                                         & \multicolumn{1}{l|}{70b75c8cd9262cb991fc0248a75df5d6}                                                    \\ \hline
\multicolumn{1}{|l|}{7}                                                                & \multicolumn{1}{l|}{328a37205b691401e879e520ba757b42}                                         & \multicolumn{1}{l|}{70b75c8cd9262cb991fc0248a75df5d6}                                                    \\ \hline
\multicolumn{1}{|l|}{8}                                                                & \multicolumn{1}{l|}{328a37205b691401e879e520ba757b42}                                         & \multicolumn{1}{l|}{70b75c8cd9262cb991fc0248a75df5d6}                                                    \\ \hline
\multicolumn{1}{|l|}{9}                                                                & \multicolumn{1}{l|}{328a37205b691401e879e520ba757b42}                                         & \multicolumn{1}{l|}{70b75c8cd9262cb991fc0248a75df5d6}                                                    \\ \hline
\multicolumn{1}{|l|}{10}                                                               & \multicolumn{1}{l|}{328a37205b691401e879e520ba757b42}                                         & \multicolumn{1}{l|}{70b75c8cd9262cb991fc0248a75df5d6}                                                    \\ \hline
\end{tabular}
\caption{Random values in Google SafeFrame URLs are the same when replaying the archived web page multiple times with the same number of function calls to the random number functions before creating the Google SafeFrames. (URI-R: \url{https://treid003.github.io/random_Values_external_JS_with_async.html} | WACZ: \url{https://zenodo.org/records/10931734/files/2023-02-21-random-values-with-safeframes.wacz?download=1})} \label{table:SafeFrameSubdomainsReplayTime}
\end{table}

Loading Google ads in a containing web page is an example in which overwriting the seed for the random number generators did not result in the same random numbers being generated during each replay.
%
%This is different from replaying an archived Google ad that uses a Google SafeFrame. 
%
In this case, the Google SafeFrame URL's random subdomain differed each time the archived web page was loaded (Table \ref{table:SafeFrameSubdomainChangesWhenLoadingAdDuringReplay}).
%
The Google SafeFrame subdomain differed in Table \ref{table:SafeFrameSubdomainChangesWhenLoadingAdDuringReplay} in part because a Google SafeFrame was loaded when the ad slot is close to the viewport, not immediately upon replay.  
%
Delaying the ads' loading can produce different timings in network communications, which ``lead[s] to a varying execution order and thus a different order of pop-requests from the `random' number sequence'' \cite{kiesel-wasp-desires18}.
%different timings in the network communications lead to a varying execution order and thus a different order of pop-requests from the “random” number sequence.
%
%Since the Google SafeFrame is not immediately being loaded upon replay, this could be the reason why the Google SafeFrame subdomain differed in Table \ref{table:SafeFrameSubdomainChangesWhenLoadingAdDuringReplay}. %This differs from our example, where the random values are immediately generated.
%The Google SafeFrame subdomain may be changing in Table \ref{table:SafeFrameSubdomainChangesWhenLoadingAdDuringReplay}, because the Google SafeFrame is not immediately being loaded upon replay. This differs from our example, where the random values are immediately generated. 
%
%Also, some Google ads are only loaded when the ad slot is near the viewport, which can change the time when the Google SafeFrame is loaded.
%This may be happening for Google ads, since the ads are usually loaded when the ad slot is near the viewport.
%
To explore this phenomenon, we augmented the demo web page (Figure \ref{fig:updatedGSFDemo}) with two sections in which the random values were dynamically generated based on user interaction. The first section featured buttons that, when clicked, generated the random values (Figure \ref{fig:rvButtonClick}). The second section generated random values when the user scrolled to it (Figure \ref{fig:rvScrolling}).
%
We replayed the updated demo web page five times and changed the number of buttons clicked on during each replay; therefore, the number of function calls to \texttt{Math.random()} and \texttt{crypto.getRandomValues()} differed before reaching the last section. Varying the number of function calls to the random number functions before generating the last two Google SafeFrame URLs produced different random values in the subdomain for the URLs (shown in Table \ref{table:differentRandomSafeFrameURLs}), which is similar to what happened in Table \ref{table:SafeFrameSubdomainChangesWhenLoadingAdDuringReplay}. 


If there are multiple JavaScript files making function calls for \texttt{Math.random()} and \texttt{crypto.getRandomValues()} (e.g., running multiple ad services on a web page) before a Google ad is loaded, then the random number sequence will change, which causes variance with the random value included in the Google SafeFrame URL.
%
In contrast, if the number of function calls to the random number functions is the same, then the random values will be consistent, as in the first version of our demo web page (Table \ref{table:SafeFrameSubdomainsReplayTime}).
%When replaying the archived web page multiple times (Table \ref{table:SafeFrameSubdomainsReplayTime}), the same random values were generated.  

%This section shows that web archive replay systems cannot generate the same random value that was generated during crawl time. This problem will impact any dynamically loaded resources that use random values. The next section shows another example where the random values used by an ad service prevents some web ads from replaying.
Overall, replay systems cannot generate the same random value that was generated during crawl time. Notably, \textit{any} dynamically loaded resources that use random values will encounter this problem. 
%% this feels like a bumpersticker that we should make sure is surfaced in the abstract, intro, and conclusions. 

%The next section shows another example where the random values used by an ad service prevents some web ads from replaying.
%% good! try to do more of this throughout the paper

%\newpage
\subsubsection{Amazon Ad iframe}
To replay an Amazon ad, a crawler must archive both the embedded web page for the Amazon ad iframe and the ad loaded inside it.
%
%Our work between February 2023 and April 2023 indicates that 
%Amazon’s ad service prevented pywb 2.7.3 from replaying ads. By contrast, ReplayWeb.page successfully replayed these ads. 
%
%We encountered three challenges with Amazon ads. 
We encountered two challenges with Amazon ads. 
%First, when archiving web pages that contain Amazon ads, Conifer and Arquivo.pt failed to archive the Amazon ad iframe, which prevents the Amazon ads from replaying. 
%% Notes: 
%%%Currently it is possible to archive Amazon ad iframe with Conifer and Arquivo.pt. 
%%% Example Amazon ad iframe URI-R: https://aax-us-east.amazon-adsystem.com/e/dtb/admi?b=JAkcZL99KtnLJUwYJ8dHHdIAAAGGLy8dDAEAAAxWAQBhcHNfdHhuX2JpZDEgICBOL0EgICAgICAgICAgICA_KuR0&rnd=8954498773591675828862700&pp=1wff280&p=e44jk0&crid=arcgnw6w
%%% Arquivo.pt URI-M: https://arquivo.pt/wayback/20240716165600/https://aax-us-east.amazon-adsystem.com/e/dtb/admi?b=JAkcZL99KtnLJUwYJ8dHHdIAAAGGLy8dDAEAAAxWAQBhcHNfdHhuX2JpZDEgICBOL0EgICAgICAgICAgICA_KuR0&rnd=8954498773591675828862700&pp=1wff280&p=e44jk0&crid=arcgnw6w
%%% Conifer URI-M: https://conifer.rhizome.org/treid003/archiveready-examples/20240716170421/https://aax-us-east.amazon-adsystem.com/e/dtb/admi?b=JAkcZL99KtnLJUwYJ8dHHdIAAAGGLy8dDAEAAAxWAQBhcHNfdHhuX2JpZDEgICBOL0EgICAgICAgICAgICA_KuR0&rnd=8954498773591675828862700&pp=1wff280&p=e44jk0&crid=arcgnw6w
%%% The problem that occurred with pywb may be the same problem for Arquivo.pt and Confier, because these web archives both use pywb.
%
First, although the ad resources were successfully archived, pywb version 2.7.3 \cite{kreymer-pywb-273-gh23} failed to replay some Amazon ads because a URL for the ad bid contained incorrect \texttt{ws} and \texttt{pid} query string parameters (Figure \ref{fig:differentBidURLs}).
%The \texttt{ws} parameter might be the size of the ad container and can vary depending on the size of the browser window. The \textt{pid} parameter
The dynamically generated values for these parameters differed during the crawling and replay sessions.
% The ws parameter is based on the size of the window.
% The pid parameter is either a random value or is based on the current time when the URL is requested.
%% The code for these parameters is in ign.prebid.js, which is the customized script used by IGN to perform auctions for the ad spaces and determine which ads to load.
%
%For pywb, it seems like the auction is failing which prevents the ad from being loaded. The pid parameter is incorrect during replay time. During crawl time the pid was "guLkRIOevEsbJ". When using pywb the pid was "LexIBtLcDvOKR". ReplayWeb.page also generates an incorrect pid ("TencIlcbstzWs"), but it can still successfully replay the ad bid URL. The pid parameter is different each time I replay the IGN's web page so it is more than likely using a random value. Some of IGN's code uses secure tokens and XMLHttpRequests. I have not found the section of the code where a random function is used.
%%The ws and pid parameters are different. ws may be associated with the dimensions of the container or the screen. pid may be a random value since it changes each time the web page is replayed.
%http://localhost:8080/my-web-archive/20230208035919mp_/https://aax-dtb-cf.amazon-adsystem.com/e/dtb/bid?src=3158&u=https%3A%2F%2Fwww.ign.com%2Farticles%2Fthe-last-of-us-season-1-review&pid=LexIBtLcDvOKR&cb=0&ws=1360x565&v=23.127.1625&t=700&slots=%5B%7B%22sd%22%3A%22top-1%22%2C%22s%22%3A%5B%22970x250%22%2C%221900x500%22%2C%22970x500%22%2C%22970x66%22%2C%22728x90%22%2C%22100x1%22%2C%22120x30%22%5D%2C%22sn%22%3A%22%2F5691%2Fign_desktop_web_display%2Farticle%22%7D%2C%7B%22sd%22%3A%22main-1%22%2C%22s%22%3A%5B%22300x250%22%2C%22300x420%22%5D%2C%22sn%22%3A%22%2F5691%2Fign_desktop_web_display%2Farticle%22%7D%2C%7B%22sd%22%3A%22sidebar-1%22%2C%22s%22%3A%5B%22300x250%22%2C%22300x600%22%2C%22300x1050%22%2C%22300x420%22%5D%2C%22sn%22%3A%22%2F5691%2Fign_desktop_web_display%2Farticle%22%7D%2C%7B%22sd%22%3A%22main-2%22%2C%22s%22%3A%5B%22728x90%22%2C%22970x250%22%2C%22970x500%22%2C%221900x500%22%2C%22970x90%22%2C%22100x1%22%5D%2C%22sn%22%3A%22%2F5691%2Fign_desktop_web_display%2Farticle%22%7D%5D&pj=%7B%22us_privacy%22%3A%221YNY%22%7D&gdprl=%7B%22status%22%3A%22no-cmp%22%7D
%
Pywb 2.7.3 only replayed Amazon ads initially loaded inside of a Google SafeFrame. To enable this replay, however, we not only needed to know the URL for the Amazon ad iframe, but also loaded the ad outside of the containing web page.
%
Second, since the Internet Archive blocked Save Page Now users from archiving the URLs associated with Amazon ad iframes, the Wayback Machine will not replay most Amazon ads.

\begin{figure}[tbp]
    \centering
    \includegraphics[width=\textwidth]{Images/Different_pid_and_ws_values_original_vs_replay.png}
    \caption{Pywb 2.7.3 failed to load Amazon ads because an ad bid URL failed to load. WACZ: \url{https://zenodo.org/record/8000975/files/2023-02-07-ads-on-ign_ArchiveWeb_page.wacz?download=1} | URI-R: \href{https://aax-dtb-cf.amazon-adsystem.com/e/dtb/bid?src=3158&u=https\%3A\%2F\%2Fwww.ign.com\%2Farticles\%2Fthe-last-of-us-season-1-review&pid=guLkRIOevEsbJ&cb=0&ws=1745x845&v=23.127.1625&t=700&slots=\%5B\%7B\%22sd\%22\%3A\%22top-1\%22\%2C\%22s\%22\%3A\%5B\%22970x250\%22\%2C\%221900x500\%22\%2C\%22970x500\%22\%2C\%22970x66\%22\%2C\%22728x90\%22\%2C\%22100x1\%22\%2C\%22120x30\%22\%5D\%2C\%22sn\%22\%3A\%22\%2F5691\%2Fign_desktop_web_display\%2Farticle\%22\%7D\%2C\%7B\%22sd\%22\%3A\%22main-1\%22\%2C\%22s\%22\%3A\%5B\%22300x250\%22\%2C\%22300x420\%22\%5D\%2C\%22sn\%22\%3A\%22\%2F5691\%2Fign_desktop_web_display\%2Farticle\%22\%7D\%2C\%7B\%22sd\%22\%3A\%22sidebar-1\%22\%2C\%22s\%22\%3A\%5B\%22300x250\%22\%2C\%22300x600\%22\%2C\%22300x1050\%22\%2C\%22300x420\%22\%5D\%2C\%22sn\%22\%3A\%22\%2F5691\%2Fign_desktop_web_display\%2Farticle\%22\%7D\%2C\%7B\%22sd\%22\%3A\%22sidebar-2\%22\%2C\%22s\%22\%3A\%5B\%22300x250\%22\%2C\%22300x600\%22\%2C\%22300x420\%22\%2C\%22100x1\%22\%5D\%2C\%22sn\%22\%3A\%22\%2F5691\%2Fign_desktop_web_display\%2Farticle\%22\%7D\%2C\%7B\%22sd\%22\%3A\%22main-2\%22\%2C\%22s\%22\%3A\%5B\%22728x90\%22\%2C\%22970x250\%22\%2C\%22970x500\%22\%2C\%221900x500\%22\%2C\%22970x90\%22\%2C\%22100x1\%22\%5D\%2C\%22sn\%22\%3A\%22\%2F5691\%2Fign_desktop_web_display\%2Farticle\%22\%7D\%5D&pj=\%7B\%22us_privacy\%22\%3A\%221YNY\%22\%7D&gdprl=\%7B\%22status\%22\%3A\%22no-cmp\%22\%7D}{https://aax-dtb-cf.amazon-adsystem.com/e/dtb/bid?src=3158\&u=https\%3A\%2F\%2Fwww.ign.com\%2Farticles\ ...}}
    \label{fig:differentBidURLs}
\end{figure}

%URI without query string: \url{https://aax-dtb-cf.amazon-adsystem.com/e/dtb/bid?src=3158&u=https%3A%2F%2Fwww.ign.com%2Farticles%2Fthe-last-of-us-season-1-review&pid=guLkRIOevEsbJ&cb=0&ws=1745x845&v=23.127.1625&t=700&slots=%5B%7B%22sd%22%3A%22top-1%22%2C%22s%22%3A%5B%22970x250%22%2C%221900x500%22%2C%22970x500%22%2C%22970x66%22%2C%22728x90%22%2C%22100x1%22%2C%22120x30%22%5D%2C%22sn%22%3A%22%2F5691%2Fign_desktop_web_display%2Farticle%22%7D%2C%7B%22sd%22%3A%22main-1%22%2C%22s%22%3A%5B%22300x250%22%2C%22300x420%22%5D%2C%22sn%22%3A%22%2F5691%2Fign_desktop_web_display%2Farticle%22%7D%2C%7B%22sd%22%3A%22sidebar-1%22%2C%22s%22%3A%5B%22300x250%22%2C%22300x600%22%2C%22300x1050%22%2C%22300x420%22%5D%2C%22sn%22%3A%22%2F5691%2Fign_desktop_web_display%2Farticle%22%7D%2C%7B%22sd%22%3A%22sidebar-2%22%2C%22s%22%3A%5B%22300x250%22%2C%22300x600%22%2C%22300x420%22%2C%22100x1%22%5D%2C%22sn%22%3A%22%2F5691%2Fign_desktop_web_display%2Farticle%22%7D%2C%7B%22sd%22%3A%22main-2%22%2C%22s%22%3A%5B%22728x90%22%2C%22970x250%22%2C%22970x500%22%2C%221900x500%22%2C%22970x90%22%2C%22100x1%22%5D%2C%22sn%22%3A%22%2F5691%2Fign_desktop_web_display%2Farticle%22%7D%5D&pj=%7B%22us_privacy%22%3A%221YNY%22%7D&gdprl=%7B%22status%22%3A%22no-cmp%22%7D} | WACZ: \url{https://zenodo.org/record/8000975/files/2023-02-07-ads-on-ign_ArchiveWeb_page.wacz?download=1}

%Web pages that mention ReplayWeb.page's approach for handling requests
%%Papers that mention wabac.js:
%%% https://www.researchgate.net/profile/Tino-Fritsch/publication/360849737_Digital_Preservation_Environments_-_A_Qualitative_Case_Study_of_Rhizome/links/628e61bc55273755ebb51651/Digital-Preservation-Environments-A-Qualitative-Case-Study-of-Rhizome.pdf
%% https://github.com/webrecorder/replayweb.page/issues/69
%%% Fuzzy matching is done by wabac.js by performing a prefix query from the index server
%%% There is also custom domain fuzzy matching rules
%%% Fake CDX entries are included when it is not possible to use prefix querying
%% https://github.com/webrecorder/replayweb.page/issues/61
%% https://github.com/webrecorder/replayweb.page/issues/31
%%% Problem: "Overview page contains a link to showflat.php?Number=5970002, but it will take you to showflat.php?Number=1005878."
%%% "the replay system also has a fuzzy matching system that attempts to match inexact responses. The system is necessary match inexact timestamps or other params."
%%% I should test the WACZ provided and see if the requested URL is rewritten or is it fuzzy matched with an incorrect URL
%% https://github.com/webrecorder/wabac.js
%% wabac.js is the server-side rewriting portion of ReplayWeb.page
%% Fuzzy match script for wabac.js: https://github.com/webrecorder/wabac.js/blob/ad059dd0bde293e2a921b3a61cffe86280c8b827/src/fuzzymatcher.js#L15
%%Papers that mentions fuzzy matching:
%%% WASP: web archiving and search personalized
%%%% https://repository.ubn.ru.nl/bitstream/handle/2066/197597/197597.pdf
%%%% "To greater effect, pywb employs a fuzzy matching of GET parameters that ignores some of the parameters that it assumes to have random values (eg, session ids),"
%%%% Also mentions how the execution of the random functions result in a different random number sequence: "While pywb replaces the JavaScript random number generator by a deterministic one, this only affects the archived page and does not fully solve the problem: different timings in the network communications lead to a varying execution order and thus a different order of pop-requests from the “random” number sequence."
%%% Reproducible web corpora: interactive archiving with automatic quality assessment
%%%% https://dl.acm.org/doi/pdf/10.1145/3239574?casa_token=PFnycAg4wbQAAAAA:kVZshlrFstHcJUSscb6T_RK2wzXiUp87W5vbdCqoqx2xWZ6aehtPllcaOhXBUfI3cceLCQCVMkphXQ
%%%% "When reproducing a web page, all requests sent by the archived web page must be answered with the same responses that have been captured during archiving. However, the requests sent during reproduction may differ in subtle ways from the ones originally sent, requiring a fuzzy match between request and response. For example, requests made via JavaScript that involve random numbers or that are based on the current time will be dissimilar from their original."
Alone among the replay systems we tested, ReplayWeb.page successfully replayed this type of Amazon ad. Amazon ads use a random value in the URL's query string stored in the \texttt{rnd} parameter.
%
ReplayWeb.page's approach for fuzzy matching made it possible to replay these Amazon ads even when the \texttt{rnd} parameter generated during replay differed from the one generated during crawl time (Figure \ref{fig:differentRndValues}). Requests that have different query string parameters during replay than during crawl time are handled by fuzzy matching to match requests during replay with responses that were captured during crawl time \cite{kiesel-acm18}. 
%
%ReplayWeb.page replayed these Amazon ads even when the \texttt{rnd} parameter generated during replay differed from the one generated during crawl time (Figure \ref{fig:differentRndValues}).
%
%Requests during replay that have dynamically generated query string parameters are handled by using fuzzy matching. 
%
%Requests during replay that have different query string parameters than during crawl time are handled by using fuzzy matching to send responses that were captured during crawl time \cite{kiesel-acm18}. 
%
%Fuzzy matching is used to handle requests during replay by sending responses that have been captured during crawl time. One example where that include.
%
While the random value generated in the query string of the URL did not prevent ReplayWeb.page from loading an Amazon ad, a random value used in the subdomain of a Google ad's URL did.
%Notably, the random value generated in the query string of the URL does not prevent ReplayWeb.page from loading an Amazon ad, but when a random value is used in the subdomain of a URL for a Google ad it does prevent an ad from loading.

\begin{figure}[tbp]
    \centering
    \includegraphics[width=\textwidth]{Images/Different_rnd_values_original_vs_replay.png}
    \caption{When replaying an Amazon ad iframe, the \texttt{rnd} parameter is not the same as the original value that is in the URI-R. Even though an incorrect URI-M is generated, ReplayWeb.page is able to load the ad. URI-R: \url{https://aax-us-east.amazon-adsystem.com/e/dtb/admi?b=JEs-gAH7EaH2UKbdDLn5qMwAAAGGLy2RRQEAAAxWAQBhcHNfdHhuX2JpZDEgICBOL0EgICAgICAgICAgICCW8VTU&rnd=4734766067051675828791974&pp=q44zcw&p=1kaetq8&crid=lm7xjkp3} | WACZ: \url{https://zenodo.org/record/8000975/files/2023-02-07-ads-on-ign_ArchiveWeb_page.wacz?download=1}}
    \label{fig:differentRndValues}
\end{figure}

Amazon ads' use of random numbers in their iframe URLs caused another problem. Multiple ads may use the same base ad iframe URL, albeit with different query strings. 
%
This prevents some of the ads from being shown during replay because of how ReplayWeb.page uses fuzzy matching. If multiple ad iframe URLs only differ by their query string, then the same ad will be selected and replayed when loading an unarchived ad iframe URL.
%
%Therefore, some of the ads will not be shown during replay because the random numbers generated during replay will differ from those generated during the crawl. As a result, the same Amazon ad -- but not the other Amazon ads using the same iframe -- will be replayed in the containing web page.

%This section discussed a case where using a dynamically generated random value in a URL's query string can prevent the replay of an archived ad. The next section will describe the last example we found where an ad service used a dynamically generated URL that prevented the replay of an ad.
%The random value used in the URI for Google and Amazon ad services ad iframes triggered replay problems. The ad service in the next section (Flashtalking) requests a missing resource which prevents a web page ad from replaying outside of an ad iframe.
% How is Flashtalking related
%% It is another ad service that caused problems with replaying an archived advertisement.
%% 

\subsubsection{Loading Embedded Web Page Ads Outside of an Ad iframe} \label{section:flashtalking}
%Other notes: The index.html for the embedded web page returns a 200, but another resource returns 404. For these ads, The index.html will load another embedded web page and that web page is returning a 404. When the ad is not loaded in the iframe, it attempts to load a different embedded web page that does not exist and will return a status code of 404. The web page that is loaded does not use the ad id so it may be a general web page that is loaded when the ad is accessed outside of the iframe.
%
%When loading an embedded web  page ad that uses Flashtalking outside of the ad iframe, an incorrect web page is loaded that is not associated with the current ad and this prevents the ad resources from being replayed. 
%% Rich load when the ad is out of iframe: https://cdn.flashtalking.com/richLoads/300x600_Master_Richload/index.html
%% This URL is missing the ad ID and the URL is used on an about:blank iframe and the URL structure is different
%% Rich load when loaded in iframe: https://cdn.flashtalking.com/173980/300x600_Master_Richload_Compressed/index.html
%
%Flashtalking may be intentionally blocking these ads from loading outside of an ad iframe so that the ads are not loaded on a website that the advertisers do not want to be associated with.
%Some embedded web page ads cannot be successfully replayed outside of an ad iframe. 
The JavaScript for Flashtalking’s ad service also dynamically generated a URL that prevented ad replay. This prevented us from replaying the embedded web page ads that we archived during 2023 outside the containing web page because the JavaScript for Flashtalking’s ad service loaded an unarchived web resource. 
% May also mention the percentage of embedded web page ads are affected by this problem. 
%% How many embedded web page ads that I archived were from Flashtalking?
%
\begin{figure}[tbp]
    \centering
    \includegraphics[width=\textwidth]{Images/flashtalking_ad_outside_of_iframe_v2.png}
    \caption{Replaying a successfully archived embedded web page ad outside of its ad iframe. This ad uses Flashtalking and Amazon ad services. WACZ file: \url{https://zenodo.org/record/8000975/files/2023-02-07-ads-on-ign_ArchiveWeb_page.wacz?download=1} | URI-R: \url{https://cdn.flashtalking.com/173980/4163777/index.html}}
    \label{fig:replayFlashtalkingAdOutsideIframe}
\end{figure}
%
Figure \ref{fig:replayFlashtalkingAdOutsideIframe} shows an example of an embedded web page ad outside of its ad iframe. 
%
The error message shown in Figure \ref{fig:replayFlashtalkingAdOutsideIframe} is associated with an incorrect resource being loaded that prevents the ad from being replayed. The URI-R \url{https://cdn.flashtalking.com/richLoads/300x600_Master_Richload/index.html} is not associated with the current ad. The correct URI-R that should have been loaded is \url{https://cdn.flashtalking.com/173980/300x600_Master_Richload_Compressed/index.html}, which includes the ad id (173980). The Richload URI includes the ad id when replaying the embedded web page ad in an Amazon ad iframe, which  enables replay of the other ad resources (Figure \ref{fig:replayFlashtalkingAdInsideIframe}).
%
\begin{figure}[tbp]
    \centering
    \includegraphics[width=\textwidth]{Images/flashtalking_ad_inside_of_iframe.png}
    \caption{This embedded web page ad will successfully replay when it is loaded inside of an Amazon ad iframe. The correct Richload URI will be loaded when the ad is in the iframe. This ad uses Flashtalking and Amazon ad services. WACZ file: \url{https://zenodo.org/record/8000975/files/2023-02-07-ads-on-ign_ArchiveWeb_page.wacz?download=1} | Ad iframe URI-R: \url{https://aax-us-east.amazon-adsystem.com/e/dtb/admi?b=JAkcZL99KtnLJUwYJ8dHHdIAAAGGLy8dDAEAAAxWAQBhcHNfdHhuX2JpZDEgICBOL0EgICAgICAgICAgICA_KuR0&rnd=8954498773591675828862700&pp=1wff280&p=e44jk0&crid=arcgnw6w} | Richload URI-R: \url{https://cdn.flashtalking.com/173980/300x600_Master_Richload_Compressed/index.html}}
    \label{fig:replayFlashtalkingAdInsideIframe}
\end{figure}
%
However, even if we try to access this web page ad outside of the ad iframe on the live web, the web page will use an incorrect Richload URI and the ad will not load~\footnote{Flashtalking might have done this to block ads from loading on a website with which the advertisers do not want to be associated.} (Figure \ref{fig:liveFlashtalkingAdOutsideIframe}). 
%
\begin{figure}[tbp]
    \centering
    \includegraphics[width=\textwidth]{Images/live_flashtalking_ad_outside_iframe.png}
    \caption{The live web version of this embedded web page ad also fails to load outside of the ad iframe. URI-R: \url{https://cdn.flashtalking.com/173980/4163777/index.html}}
    \label{fig:liveFlashtalkingAdOutsideIframe}
\end{figure}
%The issue below has been fixed by a recent update to ReplayWeb.page 
%Another issue with loading an embedded web page ad outside of its iframe is that size of the content is not adjusted properly and the ad looks different from how it looked during crawl time. 
% Examples
%% FlashTalking
%%% URI-R: https://cdn.flashtalking.com/173980/4163777/index.html
%%% WACZ: https://zenodo.org/record/8000975/files/2023-02-07-ads-on-ign_ArchiveWeb_page.wacz?download=1
%%% Video where it loaded successfully:  https://youtu.be/sYIle82V7O4?t=557
%%% Ad iframe: https://aax-us-east.amazon-adsystem.com/e/dtb/admi?b=JAkcZL99KtnLJUwYJ8dHHdIAAAGGLy8dDAEAAAxWAQBhcHNfdHhuX2JpZDEgICBOL0EgICAgICAgICAgICA_KuR0&rnd=8954498773591675828862700&pp=1wff280&p=e44jk0&crid=arcgnw6w
%
%Flashtalking may be intentionally blocking these ads from loading outside of an ad iframe so that the ads are not loaded on a website with which the advertisers do not want to be associated.
%
%\newpage
%Although Flashtalking, Google, and Amazon ad services make it difficult to replay some ads, we successfully archived the necessary ad resources. 

%In the next section, we will discuss the last replay problem we identified which involved a replay system that used service workers.

%In the next section, we provide a case in which most ads were not replayable because a web archiving service prevented the archiving of ad resources.
% The problems can be fixed by matching the requested URIs with the URIs that were archived similar to Jawa paper.
%% Overide the dynamically generated URIs and replace with archived URIs that are similar to the requested URI.

\newpage
\subsubsection{Replay of An Ad Can Differ Depending On The Web Browser Used}
% Video: https://www.youtube.com/watch?v=gCW15i-5teQ
Finally, we identified a replay problem with a replay system (ReplayWeb.page) that used service workers. In January 2023, we archived and replayed a web page~\footnote{\url{https://www.scmp.com/news/china/society/article/3049489/coronavirus-outpouring-grief-and-anger-after-death-whistle}} that included an ad (Figure \ref{fig:replayDifferentBasedOnBrowser}) whose successful replay depended upon the browser used~\footnote{Video: \url{https://youtu.be/gCW15i-5teQ?t=40}}. 
%In January of 2023, we found an example ad (Figure \ref{fig:replayDifferentBasedOnBrowser}) whose successful replay depended upon the browser used to load the archived web page~\footnote{Video: \url{https://youtu.be/gCW15i-5teQ?t=40}} \cite{rwp-different-browser-issue-yt23}.
When replay systems use service workers, the replay of an archived web page can differ depending upon a browser's implementation of service workers. We observed this problem when an ad used an iframe with a \texttt{src} attribute value of ``\texttt{about:blank}''.
%\caption{Successful replay in Firefox}
\begin{figure}[tbp]
    \centering
    \begin{subfigure}[b]{0.45\textwidth}
        \centering
        \includegraphics[width=\textwidth]{Images/Firefox_Replay_Different_Depending_on_Browser.png} 
        \label{fig:FirefoxReplay}
    \end{subfigure}
    \hfill
    \begin{subfigure}[b]{0.45\textwidth}
        \centering
        \includegraphics[width=\textwidth]{Images/Chrome_Replay_Different_Depending_on_Browser.png}
        \label{fig:ChromeReplay}
    \end{subfigure}
    \caption{The replay of an ad differed depending on the web browser used. WACZ: \url{https://zenodo.org/records/10373131/files/2023-01-26-ads-on-scmp_ArchiveWeb_page.wacz?download=1} | URI-R: \url{https://www.scmp.com/news/china/society/article/3049489/coronavirus-outpouring-grief-and-anger-after-death-whistle} | Video: \url{https://www.youtube.com/watch?v=gCW15i-5teQ}}
    \label{fig:replayDifferentBasedOnBrowser}
\end{figure}

When we used ReplayWeb.page with Firefox version 109.0 \cite{firefox-109-mozilla23}, the image ad loaded (left side of Figure \ref{fig:replayDifferentBasedOnBrowser}). Conversely, it failed to load when using Chrome version 109.0.5414 \cite{chrome-109} (right side of Figure \ref{fig:replayDifferentBasedOnBrowser}). 
%
%Based on a comment from this GitHub issue
After identifying this problem, we created a GitHub issue \cite{reid-rwp-issue23} on ReplayWeb.page's GitHub repository. One of the comments~\footnote{GitHub issue: \url{https://github.com/webrecorder/replayweb.page/issues/157}} mentioned that the service worker had not gained control of the ad iframe, which led to leaked requests. These leaked requests resulted in a 404 status code during replay for a successfully archived resource. There is a Chromium bug~\footnote{Chromium bug: \url{https://issues.chromium.org/issues/41411856}} related to this issue, where the service worker is unable to access the resources loaded in an ``\texttt{about:blank}'' iframe. 
% The document.write() method may have caused this issue
However, a ReplayWeb.page update fixed this by overriding the \texttt{document.write()} \cite{document-write-mozillat24} method with a blob URL~\footnote{A blob URL is another way to access a File object and it can be used as a \texttt{src} or \texttt{href} attribute \cite{bidelman-11}.} created by the service worker. 
% add description of blob URL here or in the background
%% https://books.google.com/books?hl=en&lr=&id=j_guWrO8ih0C&oi=fnd&pg=PR5&dq=blob+url&ots=jbIKv1d5qR&sig=2iRT72WGtWCvKkODqfYts9UZokA#v=onepage&q=blob%20url&f=false
%%% Bidelman, Eric. Using the HTML5 filesystem API. " O'Reilly Media, Inc.", 2011.
%"override document.write with a URL loaded from service worker that loads a blob". 

% Another browser related problem will be discussed in the next section.

%This browser related problem prevents the replay of successfully archived ads. The next section describes a browser compatibility problem that prevents ads from being archived by Brozzler.


\section{Conclusions}
%%there's more to be said here.  need to: 1) tell me what you're going to tell me (intro), 2) tell me, and 3) tell me what you told me (conclusions)
%% restate the contributions, lessons learned, change in SPN behavior, and data set creation. 
Web advertisements represent a significant and rapidly evolving aspect of digital cultural heritage. The need to preserve them is ever-increasing. But serious problems with archiving and replaying current web ads persist.
%
This paper explored the process of creating a dataset of 279 recent (January--June, 2023) web ads and discussed the problems we encountered while archiving and replaying these ads.
%Dataset description
This dataset was created by archiving 17 web pages from SimilarWeb's top websites worldwide. When archiving these web pages, we utilized four web archiving services (Internet Archive's Save Page Now, Arquivo.pt, archive.today, and Conifer) and three browser-based tools (ArchiveWeb.page, Browsertrix Crawler, and Brozzler). We replayed these archived web pages with four web archiving services (Internet Archive's Wayback Machine, Arquivo.pt, archive.today, and Conifer) and three other replay systems (ReplayWeb.page, pywb, and OpenWayback). 

%
%This paper explored the process of archiving and replaying 279 unique web ads and yielded five key findings. 
The process of archiving and replaying these 279 unique web ads yielded five key findings. First, Internet Archive's Save Page Now feature excluded web ads from being archived. Before August 2023, Save Page Now prevented users from archiving ads from well-known ad services like Google AdSense and Amazon Ad Server and it excluded URLs with ad related file and directory names in the URL's path. After August 2023, Save Page Now still excluded ads that were loaded on a web page from being archived, but it did allow an ad's resource(s) to be archived when the user directly archived the URL(s) associated with the ad.
%
Second, Brozzler was incompatible with recent versions of Google Chrome released after March 2023. This incompatibility prevented Chrome from loading web pages during the crawling session which prevented web resources from being archived.
%
Third, when executing Google's and Amazon's ad script, the random values generated are not the same during the crawling and replay sessions, because the seed for the random number generator is set by the replay system and the value will not be the same as during crawl time. This resulted in a request for an incorrect URL during replay that was not archived and prevented the ad from loading. 
%
Fourth, the JavaScript for Flashtalking's ad service prevented the replay of embedded web page ads outside of an ad iframe, because the ad script dynamically generated an incorrect URL that does not exist on the live web. When this incorrect URL is requested, it stops the remaining ad resources from being loaded which resulted in a web page that displayed a 404 error message.
%
Fifth, some web ads were not loaded during replay depending on the web browser used, because the service worker implementation can differ between browsers. Chromium had a bug that prevented service workers from accessing resources inside an iframe with the \texttt{src} attribute of ``\texttt{about:blank}''. This resulted in leaked requests, which prevented the replay of a successfully archived ad.  
%
Complementing these findings, we created the Display Archived Ads tool to help find advertisements that were not able to replay when loading the containing web page. Our tool filters out known ad resources that remain invisible during replay, and it displays the live version of an ad alongside the archived version.

%Two of the replay problems we identified can be fixed when replay systems improve their approach for fuzzy matching and the last replay problem can be resolved by using blob URLs. 
Three replay problems need to be fixed.
%
First, dynamically generated URLs with random values need to be matched with the URLs that were successfully crawled. When performing a URL match, the random value should be removed. Google SafeFrame URLs include the random value in the subdomain and Amazon ad iframe URLs include the random value in the query string. 
% 
Second, when a Flashtalking ad requests a non-existent URL that includes ``\texttt{Richload}'' in the path, it needs to be matched with a different URL that includes ``\texttt{Richload}'' and the ad ID in the URL's path. The ad ID can be retrieved from the URL for the web page ad that initiated the request.
%
Third, when a replay system uses service workers and is replaying an ad loaded in an iframe with a \texttt{src} attribute value of ``\texttt{about:blank}'', it could replace the \texttt{src} attribute with a blob URL. Using blob URLs for ``\texttt{about:blank}'' iframes is the workaround that ReplayWeb.page used to resolve this problem for Chromium-based browsers.
%
Fixing these replay problems will not only improve the replay of ads, but also improve the replay of other dynamically loaded embedded resources that use random values or ``\texttt{about:blank}'' iframes.

% https://cdn.flashtalking.com/richLoads/300x600_Master_Richload/index.html
% https://cdn.flashtalking.com/173980/300x600_Master_Richload_Compressed/index.html
%Updating the fuzzy matching rules used by replay systems might resolve the replay problems with Google, Amazon, and Flashtalking ad services.

%% Combine with the info above
%Web advertisements have been representing a significant and rapidly evolving aspect of digital cultural heritage and the need to preserve them grows more. 

%However, web archiving faces substantial challenges in capturing the dynamic content. Ads change rapidly 

%This paper explored the process of archiving and reproducing 279 unique web ads and identified some key findings. First, Internet Archive's Save Page Now feature excluded web ads from being archived. Second, Brozzler was incompatible with recent versions of Google Chrome. Third, when executing Google's and Amazon's ad script, the random values generated are not the same during crawl time and replay time, which results in requesting an incorrect URL during replay that was not archived. Fourth, the Flashtalking ad service prevented the replay of embedded web page ads outside of an ad iframe. Fifth, some web ads were not loaded during replay depending on the web browser used.To fix the replay problems we identified, replay systems need to improve their approach for fuzzy matching so that dynamically generated URLs with random values are matched with the URLs that were successfully crawled. Another future work is to identify problems with archiving personalized web advertisements.
 

%For this work, we have created a dataset of 279 unique archived web ads and we identified key problems with archiving and replaying recent web ads.
%This research explored key problems in archiving and replaying recent web advertisements. 
%% Problems:
%%% 1) Replay systems incorrectly generating random values in URL
%%% 2) Loading Flashtalking ads outside of ad iframe
%%% 3) Internet Archiving blocking Save Page Now users from archiving ads and URLs that include ad keywords
%%% 4) Browser replay being different depending on web browser
%%% 5) Brozzler's incompatibility with recent versions of Chrome
%
%When archiving web pages with ads, most of the ads were successfully archived but failed to replay because of issues with replaying ads that use certain ad services.
%When archiving web pages with ads, most of the ads were successfully archived but failed to replay because of certain ad services' constraints.
%
%Google and Amazon ad services use random values in the URL. This prevents an ad from being replayed successfully since replay systems do not generate the same random value that was generated during the crawling session. The Flashtalking ad service also causes problems with replaying an embedded web page ad outside of its ad iframe, which can make some ads impossible to replay since Google SafeFrames and Amazon ad iframes usually have replay problems. Another problem is that Internet Archive's Save Page Now block ad resources from being archived. 
%This will reduce the number of web advertisements that historians and researchers will have access to in the future.
%This will reduce the number of web advertisements that we will have access to in the future.

%Our dataset~\footnote{\url{https://github.com/savingads/Recently_Archived_Ads/blob/main/All_Ads.csv}} of 279 web advertisements suggests that ads commonly use a combination of web resources. To be replayable, these combination ads must be constructed in a containing web page that would have loaded the ad during the crawling session. Save Page Now blocking ads that are loaded when archiving the containing web page means that many current web advertisements will not be archived. Another finding is that most of the successfully archived web advertisements (224 out of 279) can be replayed in the containing web page depending on the web browser used and the version of the replay system.

\section{Acknowledgments}
This research was made possible through the support of the Institute of Museum and Library Services (IMLS) \href{https://www.imls.gov/grants/awarded/lg-252362-ols-22}{\#LG-252362-OLS-22}.

%\section{Research Questions (to incorporate)}
%This project will tackle three primary research questions (RQ):

%\begin{itemize}
%\item RQ1: To what extent do institutional web archives capture personalized advertisements on the web?
%\item RQ2: Do scholarly or lay web users prefer (re)using  archived web pages including personalized ads, a generic comprehensive capture, or a capture with web ads missing?
%\item RQ3: In what ways might the strategic use of diverse personas surface types of web content that would otherwise go unarchived?
%\end{itemize}

\bibliographystyle{acm}
\bibliography{refs.bib}
\end{document}