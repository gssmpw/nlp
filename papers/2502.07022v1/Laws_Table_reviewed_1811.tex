\begin{document}

\section{Comparison of Modern Slavery Reporting Criteria and Metrics}

\begin{landscape}
\begin{table}
\centering
\caption{Comparison of Modern Slavery Reporting Criteria and Metrics}
\vspace{0.1mm}
\fontsize{7}{7}\selectfont
\label{tab:appendix:examples}
%\begin{adjustbox}{width=\textwidth,height=\textheight} 
\begin{adjustbox}{scale=0.93}
    \begin{tabularx}{26cm}{|X|X|X|X|X|}
    \hline  % horizontal line at the top
         \textbf{AIMS.au Dataset Annotation Specification Questions} & \textbf{Australian Modern Slavery Act Mandatory Reporting Criteria} & \textbf{UK Modern Slavery Act Reporting Suggestions} & \textbf{Canadian Fighting Against Forced Labour and Child Labour in Supply Chains Act Reporting Obligations} & \textbf{The "Beyond Compliance" Walk Free \& WikiRate Metrics} \\
    \hline  % horizontal line after header

      \textcolor{darkgreen}{  Question: Is the statement approved by the entity's principal governing body?} & \textcolor{darkgreen}{Ensure that the statement is approved by the board.} & \textcolor{darkgreen}{Approval from the board of directors (or equivalent management body)} & \textcolor{darkgreen}{Approval by the organization’s governing body.} & \textcolor{darkgreen}{MSA Statement Approval: Was the company’s Modern Slavery Act statement explicitly approved by the board of directors (or equivalent management body)?} \\
    \hline

         \textcolor{darkgreen}{Question: Is the statement signed by a responsible member of the reporting entity?} & \textcolor{darkgreen}{The statement is signed by a responsible member of the organization. }&\textcolor{darkgreen}{ Signature from a director (or equivalent) or designated member }&\textcolor{darkgreen}{ Signature of one or more members of the governing body of each entity that approved the report.} & \textcolor{darkgreen}{MSA Statement Signed: Was the company’s Modern Slavery Act statement signed by an appropriate person?} \\
    \hline

        \textcolor{darkgreen}{ Question: Does the statement clearly identify which entities covered by the statement are the relevant reporting entities?}  & \textcolor{darkgreen}{Mandatory Criterion 1: The statement clearly identifies the Reporting Entity.} & \textcolor{darkred}{N/A} & \textcolor{darkred}{N/A } & \textcolor{darkred}{N/A} \\
    \hline

         \textcolor{darkgreen}{Question: Does the reporting entity describe its structure? \newline Question: Does the reporting entity describe its operations? \newline Question: Does the reporting entity describe its supply chains?} & \textcolor{darkgreen}{Mandatory Criterion 2: Describe the reporting entity’s structure, operations, and supply chains.} & \textcolor{darkgreen}{The organisation’s structure, business and supply chains.} & \textcolor{darkgreen}{Description of the organisation’s structure, activities and supply chains.} & \textcolor{darkgreen}{ MSA Organizational structure and operations: Does the company disclose the ownership structure(s) and/or business model(s) of each of its brands, subsidiaries, and other businesses covered by their Modern Slavery statement? \newline MSA Supply Chain Disclosure: Does the company’s statement identify the suppliers in their supply chain and/or the geographic regions where their supply chain operates? }\\
   \hline

         \textcolor{darkgreen}{Question: Does the reporting entity describe its modern slavery risks?} & \textcolor{darkgreen}{Mandatory Criterion 3: Describe the risks of modern slavery practices in the operations and supply chains of the reporting entity and any entities the reporting entity owns or controls.} & \textcolor{darkgreen}{Risk assessment and management.} & \textcolor{darkgreen}{Description of the parts of its business and supply chains that carry a risk of forced labour or child labour being used and the steps it has taken to assess and manage that risk.} & \textcolor{darkgreen}{MSA Identification of Risks?} \\
    \hline

         \textcolor{darkgreen}{Question: Does the reporting entity describe the actions applied to identify, assess, and mitigate the modern slavery risks it identified?} & \textcolor{darkgreen}{ Mandatory Criterion 4: Describe the actions taken by the reporting entity and any entities it owns or controls to assess and address these risks, including due diligence and remediation processes.} & \textcolor{darkorange}{ Description of the organisation’s policies in relation to slavery and human trafficking. \newline Description of the organisation’s due diligence processes in relation to slavery and human trafficking in its business and supply chains. \newline Description of the parts of the organisation’s business and supply chains where there is a risk of slavery and human trafficking taking place, and the steps it has taken to assess and manage that risk. \newline The training and capacity building about slavery and human trafficking available to its staff.} & \textcolor{darkorange}{  Description of the organisation’s policies and due diligence processes in relation to forced labour and child labour. \newline Description of the parts of organisation’s activities and supply chains that carry a risk of forced labour or child labour being used and the steps it has taken to assess and manage that risk. \newline The training provided to employees on forced labour and child labour.} & \textcolor{darkorange}{ MSA Policy (Revised): Does the company’s statement detail specific, organisational policies or actions to combat slavery in their direct (tier one) and/or in-direct (beyond tier one) supply chain? \newline MSA Risk assessment: How does the company assess the risks of modern slavery and trafficking in their supply chain? \newline MSA Risk management: Does the company continuously monitor suppliers to ensure that they comply with the company’s policies and local laws? \newline MSA Whistleblowing Mechanism (revised): Does the company have a grievance mechanism in place to facilitate whistleblowing or the reporting of suspected incidents of slavery or trafficking? \newline MSA Training (revised): Does the statement describe training for staff that is specifically geared towards detecting signs of slavery or trafficking?} \\
    \hline

         \textcolor{darkgreen}{ Question: Does the reporting entity describe remediation actions for modern slavery cases?} & \textcolor{darkgreen}{ Mandatory Criterion 4: Describe the actions taken by the reporting entity and any entities it owns or controls to assess and address these risks, including due diligence and remediation processes.}& \textcolor{darkgreen}{The organisation should paint a detailed picture of all the steps it has taken to address and remedy modern slavery, and the effectiveness of all such steps.} & \textcolor{darkgreen}{Description of any measures taken to remediate any forced labour or child labour.} & \textcolor{darkgreen}{MSA Incidents Remediation (revised): Did the company explain the corrective steps it has taken (or would take) in response to modern slavery incidents in their own operations and/or supply chain?} \\
    \hline

         \textcolor{darkgreen}{Question: Does the reporting entity describe how it assesses the effectiveness of its actions?} & \textcolor{darkgreen}{Mandatory Criterion 5: Describe how the reporting entity assesses the effectiveness of these actions.} & \textcolor{darkgreen}{Description of the organisation’s effectiveness in ensuring that slavery and human trafficking is not taking place in its business or supply chains, measured against such performance indicators as it considers appropriate. \newline The organisation should paint a detailed picture of all the steps it has taken to address and remedy modern slavery, and the effectiveness of all such steps. }& \textcolor{darkgreen}{ Description of how the entity assesses its effectiveness in ensuring that forced labour and child labour are not being used in its business and supply chains.} & \textcolor{darkgreen}{MSA Performance Indicators: Does the company define performance indicators that measure the effectiveness of their actions to combat slavery and trafficking?} \\
    \hline

         \textcolor{darkgreen}{Question: Does the reporting entity describe how it consulted on its statement with any entities it owns or controls? } & \textcolor{darkgreen}{Mandatory Criterion 6: Describe the process of consultation with any entities the reporting entity owns or controls.} & \textcolor{darkred}{N/A} & \textcolor{darkred}{N/A} & \textcolor{darkred}{N/A} \\
    \hline

         \textcolor{darkred}{N/A} & \textcolor{darkred}{Mandatory Criterion 7: Provide any other relevant information.} & \textcolor{darkred}{N/A} & \textcolor{darkred}{Any measures taken to remediate the loss of income to the most vulnerable families that results from any measure taken to eliminate the use of forced labour or child labour in its activities and supply chains.} & \textcolor{darkred}{MSA Impact on Company Behaviour: Does the company’s statement describe a change in their policy that occurred as a direct result of the Australian Modern Slavery Act? \newline MSA Business Performance Indicators: Has the company reviewed business KPIs to ensure they are not increasing risk of modern slavery? \newline MSA Historic Record: Does the company provide a historic record of their modern slavery statements (either on their website or in their current statement)?} \\
    \hline  % horizontal line at the bottom
    \end{tabularx}
\end{adjustbox}
\end{table}
\end{landscape}

\end{document}
