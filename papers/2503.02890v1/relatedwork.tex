\section{Related Works}
% Urban infrastructure networks mainly include electric networks, road networks, and communication networks, etc. A lot of work has been devoted to the prediction of CFs in infrastructure networks.
Existing CF prediction methods can be categorized into methods based on single networks and interdependent networks.

\subsection{CF prediction on single network}

% Most of the existing work on CF prediction in urban infrastructure networks focuses on single networks, e.g., given an initially destroyed power plant, predicting the power plants that will be destroyed afterwards. In such approaches, each component is mostly viewed as a homogeneous node, and the prediction of CFs is based on the relationships between different components.

% \subsubsection{Model-based methods}

% These methods are often based on expert knowledge in the domain to propose CF prediction algorithms based on some rule that is presented in the form of differential equations. For example, Dobson et al. proposed OPA~\cite{dobson2001initial} based on DC flow calculation analysis, which enables cascading failure modeling and analysis of the grid, and Riot et al. proposed the Manchester model~\cite{rios2002value} to address the lack of AC features in the model by focusing on AC flows. Thereafter, a large number of DC and AC models have been proposed to model cascading failures in electric networks~\cite{mei2009improved,ciapessoni2013cascading}. In addition, there are also similar approaches in other types of networks; for example, Su et al. predicted the process of traffic redistribution after node overload in a transportation network and analyzed the transmission efficiency and cost~\cite{su2014robustness}; Ren et al. predicted the failure of nodes in a communication network based on a stochastic model of state transfer theory~\cite{ren2018stochastic}. Although the model-based approach is quite robust, it requires accurate knowledge of the load transfer between the network nodes, which leads to a significant cost~\cite{nakhaei2022solution}.

% \subsubsection{Data-driven methods}

% Recently, more and more approaches have begun to predict cascading failures in single networks through machine learning models. For example, Nakarmi et al used Markov chain model to predict the cascade size distribution based on the location of initial faults in each community identified within the power system~\cite{nakarmi2020markov}; Zhu et al proposed a physically informative GNN based on a data-driven model to solve the tidal equations so that the output satisfies the laws of physics and makes the results more physically interpretable~\cite{zhu2022cascading}; Basak et al designed a model based on LSTM that can capture spatial information and interconnection of transportation networks and predict cascading failures~\cite{basak2019data}.

% Although CF prediction algorithms on single networks can better take into account the features within the network, they are unable to take into account the impact of other kinds of infrastructures on the current network, which makes the prediction results suboptimal.

Most of the existing work on CF prediction in urban infrastructure networks focuses on single networks, e.g., given an initially destroyed power plant, predicting the power plants that will be destroyed afterward. These methods are often based on expert knowledge in the domain to propose CF prediction algorithms based on some rule that is presented in the form of differential equations. For example, Dobson et al. proposed OPA~\cite{dobson2001initial} based on DC(Direct Current) flow calculation analysis, which enables cascading failure modeling and analysis of the grid, and Riot et al. proposed the Manchester model~\cite{rios2002value} to address the lack of AC(Alternating Current) features in the model by focusing on AC flows. Thereafter, a large number of DC and AC models have been proposed to model cascading failures in electric networks~\cite{mei2009improved,ciapessoni2013cascading}. 

Recently, more and more approaches have begun to predict cascading failures in single networks through machine learning models. For example, Nakarmi et al used the Markov chain model to predict the cascade size distribution based on the location of initial faults in each community identified within the power system~\cite{nakarmi2020markov}. Zhu et al proposed a physically informative GNN based on a data-driven model to solve the tidal equations so that the output satisfies the laws of physics and makes the results more physically interpretable~\cite{zhu2022cascading}. Basak et al designed a model based on LSTM that can capture spatial information and interconnection of transportation networks and predict cascading failures~\cite{basak2019data}.

Although CF prediction algorithms on single networks can better take into account the features within the network, they are unable to capture the impact of other kinds of infrastructures on the current network, which makes the prediction results suboptimal.

\subsection{CF prediction on interdependent network}

% Real-world infrastructure networks are not isolated; elements of different networks are interdependent. If a small failure occurs in one network, it can cause components of other networks to fail and propagate, leading to the collapse of the entire system~\cite{valdez2020cascading}. This dependency between different kinds of networks cannot be captured by modeling methods for single networks, thus, more and more attention is being paid to the prediction of CFs on interdependent networks.

% \subsubsection{Model-based methods}
% Buldyrev et al developed a simple model of interdependence between networks and showed that such systems may suddenly collapse under random faults~\cite{parandehgheibi2014mitigating}; Parandehgheibi et al predicted CF on a electric network-communication network and proposed a two-stage control strategy to mitigate cascading faults; Sturaro et al focused on the interdependence between grid-communication networks and proposed the HINT model for CF prediction~\cite{sturaro2018realistic}.

% The above methods consider the interactions between different networks, but they are poorly generalizable; e.g., some networks do not consider the communication network, while others do not consider the traffic network, which makes it challenging to predict CFs in real cities.
% \vspace{-0.25cm}
% \subsubsection{Data-driven methods}

% Cassottana et al. developed a resilience prediction model using the infrastructure simulation model ML algorithm for CF prediction in water-grid-transportation networks~\cite{cassottana2022predicting}; Rahnamay-Naeini et al. used an interdependent Markov chain framework to capture the interdependence between two critical infrastructures to predict interdependent networks in CF~\cite{rahnamay2016cascading}; Mao et al. developed a graph neural network system based on reinforcement learning to detect vulnerable nodes in infrastructure networks~\cite{mao2023detecting}.

% Most of the above methods are based on traditional machine learning models and do not use graph structures to capture the coupling between different networks, resulting in limited expressive power. In turn, GCN-based methods face the problem of over-smoothing, which makes it challenging to use GCN while retaining the information of the initial nodes.
% \vspace{-0.25cm}

Real-world infrastructure networks are not isolated and elements of different networks are interdependent. If a small failure occurs in one network, it can cause components of other networks to fail and propagate, leading to the collapse of the entire system~\cite{valdez2020cascading}. Thus, more and more attention is being paid to the prediction of CFs on interdependent networks.

Buldyrev et al developed a simple model of interdependence between networks and showed that such systems might suddenly collapse under random faults~\cite{parandehgheibi2014mitigating}. Parandehgheibi et al predicted CF on an electric network-communication network and proposed a two-stage control strategy to mitigate cascading faults. Sturaro et al focused on the interdependence between grid-communication networks and proposed the HINT model for CF prediction~\cite{sturaro2018realistic}.

There are also many data-driven methods for predicting CF in interdependent networks. Cassottana et al. developed a resilience prediction model using the infrastructure simulation model ML algorithm for CF prediction in water-grid-transportation networks~\cite{cassottana2022predicting}. Rahnamay-Naeini et al. used an interdependent Markov chain framework to capture the interdependence between two critical infrastructures to predict interdependent networks in CF~\cite{rahnamay2016cascading}. Mao et al. developed a graph neural network system based on reinforcement learning to detect vulnerable nodes in infrastructure networks~\cite{mao2023detecting}.

These coupled network-based approaches either model only part of the network, e.g. not considering the communication network or the road network, resulting in them being one-sided, or do not use a graph structure to extract the relationships between different infrastructures.