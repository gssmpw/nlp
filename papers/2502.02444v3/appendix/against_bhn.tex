\section{Comparative Analysis Against ValueLex \citep{biedma2024beyond}}

\label{app:against_bhn}

\citet{biedma2024beyond} proposed a lexical approach to constructing value systems for LLMs. While their work offers valuable contributions, we believe that core aspects of their approach could benefit from further theoretical and empirical development. In the following, we present a detailed comparative analysis of \our{} in relation to their work.

\subsection{Lexicon Collection}

\citet{biedma2024beyond} employ direct prompts such as "List the words that most accurately represent your value system" to extract value lexicons. However, direct questioning may not fully capture an LLM's complete spectrum of values. This work, in contrast, uses indirect, contextually guided questions to elicit a more comprehensive expression of values from LLMs.

In human psychology, self-reported values can be incomplete or skewed due to 1) social desirability bias \cite{randall1993social, larson2019controlling}, the tendency for people to self-report values that they believe are more socially acceptable; and 2) unconscious or implicit values \cite{greenwald1995implicit, hofer2006congruence}, where individuals may not recognize some deeply rooted values until certain situations bring them into play. LLM literature also reveals the unreliability of self-report results \cite{dominguez2023questioning, rottger2024political, ye2025gpv}.


Empirically, we examine all the collected lexicons in \cite{biedma2024beyond} and find that, exemplified using Schwartz's values, three values are missing: Achievement, Self-Direction, and Hedonism, according to the embedding similarity criteria established in \cite{sorensen2024value} (cosine similarity < 0.53). Using our indirect, contextually guided questions, we can capture these values, as shown in the following elicited LLM perceptions.

\begin{itemize}[leftmargin=2em]
    \item Achievement:
    \begin{itemize}
        \item Encouragement to take a challenging course for long-term goals and career development.
        \item Recognition of the importance of personal growth for professional and personal success.
        \item Emphasis on evaluating personal skills and experience for career development.
    \end{itemize}
    \item Self-Direction:
    \begin{itemize}
        \item The importance of aligning decisions with personal values and causes.
        \item Desire to take on the project independently and communicate openly with the manager.
    \end{itemize}
    \item Hedonism:
    \begin{itemize}
        \item Consideration of lifestyle enjoyment in the new city.
        \item Belief in following one's heart and pursuing joyful projects.
        \item The belief that art should bring joy rather than financial stress.
    \end{itemize}
\end{itemize}

\subsection{Computing Value Correlations}
The structure of a value system is derived from correlations between different values. In psychology research, researchers measure the value hierarchies of the participants to gather data, then use correlation derived from the data to evaluate interrelationships among these values. High positive correlations indicate values that are often endorsed together, while negative correlations show contrasting values. From these correlations, researchers can map and cluster values, revealing underlying value structures and hidden value factors (\cref{sec:approach}, Value Measurement and System Construction).

The method proposed by \citet{biedma2024beyond} derives correlations based on the co-occurrence of value lexicons in LLM self-reports in response to direct prompts. However, this approach lacks a theoretical foundation in psychology. We hypothesize that the co-occurrence of value lexicons in LLM responses does not necessarily indicate a true correlation between values. To test this hypothesis, we examine the correlation derived from their method for Schwartz's values.

\paragraph{Experimental Setup.}
We pair each of the collected value lexicons in \cite{biedma2024beyond} with the most semantically similar Schwartz's value according to the embedding model \cite{openai2024textembedding3large}. Then, we can compute the correlation between Schwartz's values using the normalized co-occurrence frequency of the original lexicons, following the method in \cite{biedma2024beyond}.

\paragraph{Results.}
\cref{fig:heatmap} illustrates the correlation heatmap of Schwartz's values based on the co-occurrence frequency of the lexicons. The values are ordered along the x and y axes according to Schwartz's circular structure. Except for the two values of Self-Enhancement, correlations between values generally contradict the theoretical structure of Schwartz's values. The results suggest that the co-occurrence frequency of lexicons in responses to direct prompts does not necessarily reflect the true value correlation. Value measurements using GPV are more theoretically grounded and empirically validated \cite{ye2025gpv}.

\begin{figure}[h]
    \centering
    \includegraphics[width=0.7\textwidth]{figures/heatmap_freq.pdf}
    \caption{Correlation heatmap derived from lexicon co-occurrence frequency \cite{biedma2024beyond}, after Min-Max normalization.}
    \label{fig:heatmap}
\end{figure}


\subsection{LLM Subjects}

\citet{biedma2024beyond} tune the generation hyperparameters (e.g., temperature, top-p) for different LLMs, attempting to generate diverse LLM subjects. However, simply tuning the generation hyperparameters is not sufficient to ensure the diversity and coverage of the LLM subjects, as they are not effective in steering the LLMs toward certain value orientations \cite{rozen2024llms}. In this work, we employ the validated value-anchoring prompts \cite{rozen2024llms} to steer the LLMs toward specific value orientations, ensuring the best possible coverage of the value spectrum and the practical relevance of our measurement results, since steering LLMs toward certain roles is common in public-facing applications nowadays.