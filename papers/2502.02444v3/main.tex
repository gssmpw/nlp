\documentclass{article}


% if you need to pass options to natbib, use, e.g.:
    \PassOptionsToPackage{numbers, compress}{natbib}
    
% before loading neurips_2024


% ready for submission
% \usepackage{neurips_2024}


% to compile a preprint version, e.g., for submission to arXiv, add add the
% [preprint] option:
    \usepackage[preprint]{neurips_2024}


% to compile a camera-ready version, add the [final] option, e.g.:
%     \usepackage[final]{neurips_2024}


% to avoid loading the natbib package, add option nonatbib:
%    \usepackage[nonatbib]{neurips_2024}


\usepackage[utf8]{inputenc} % allow utf-8 input
\usepackage[T1]{fontenc}    % use 8-bit T1 fonts
% \usepackage{hyperref}       % hyperlinks
\usepackage{url}            % simple URL typesetting
\usepackage{booktabs}       % professional-quality tables
\usepackage{amsfonts}       % blackboard math symbols
\usepackage{nicefrac}       % compact symbols for 1/2, etc.
\usepackage{microtype}      % microtypography
\usepackage{xcolor}         % colors


% Added by authors
\usepackage{graphicx}
% \usepackage{subfigure}
\usepackage{floatrow}
\usepackage{subcaption}
\usepackage{listings}
\usepackage{longtable}

% \usepackage{subfig}

% To make smaller captions
% \captionsetup{font=footnotesize}


% hyperref makes hyperlinks in the resulting PDF.
% If your build breaks (sometimes temporarily if a hyperlink spans a page)
% please comment out the following usepackage line and replace
% \usepackage{icml2024} with \usepackage[nohyperref]{icml2024} above.

%% Note: you can modify the below for colors. Green is now a bit more friendly to the eye
\usepackage[bookmarks=true,         colorlinks=true,linkcolor=red,urlcolor=magenta,citecolor=green!80!black]{hyperref}
% \hypersetup{colorlinks=true}

\usepackage{multirow}
\usepackage{color}
\definecolor{deepblue}{rgb}{0,0,0.5}
\definecolor{deepred}{rgb}{0.6,0,0}
\definecolor{deepgreen}{rgb}{0,0.5,0}

% Attempt to make hyperref and algorithmic work together better:
\newcommand{\theHalgorithm}{\arabic{algorithm}}
\usepackage{amsmath}
\usepackage{amssymb}
\usepackage{mathtools}
\usepackage{amsthm}
\theoremstyle{plain}
\newtheorem{theorem}{Theorem}[section]
\newtheorem{proposition}[theorem]{Proposition}
\newtheorem{lemma}[theorem]{Lemma}
\newtheorem{corollary}[theorem]{Corollary}
\theoremstyle{definition}
\newtheorem{definition}[theorem]{Definition}
\newtheorem{assumption}[theorem]{Assumption}
\theoremstyle{remark}
\newtheorem{remark}[theorem]{Remark}
% \usepackage[capitalize,noabbrev]{cleveref}
% \usepackage{subfigure}
\usepackage{bm} 
\usepackage{algorithm, algorithmic}
% \usepackage{caption}
% Set the table caption on top
\floatsetup[table]{capposition=bottom}

\usepackage{enumitem}


% \usepackage{authblk} % added
\DeclareMathOperator{\BN}{BN}

% Change the bibliography style
\usepackage[numbers]{natbib}
\bibliographystyle{abbrvnat}

% This must be in the first 5 lines to tell arXiv to use pdfLaTeX, which is strongly recommended.
\pdfoutput=1
% In particular, the hyperref package requires pdfLaTeX in order to break URLs across lines.

\documentclass[11pt]{article}

% Change "review" to "final" to generate the final (sometimes called camera-ready) version.
% Change to "preprint" to generate a non-anonymous version with page numbers.
\usepackage[preprint]{acl}

% Standard package includes
\usepackage{times}
\usepackage{latexsym}
\usepackage{booktabs}
\usepackage{multirow} 
\usepackage{amsmath}
\usepackage{amsfonts}
\usepackage{colortbl}
\usepackage{graphicx} % 用于插入图片
\usepackage{subcaption} % 用于分割小图和添加子标题
% For proper rendering and hyphenation of words containing Latin characters (including in bib files)
\usepackage[T1]{fontenc}
% For Vietnamese characters
% \usepackage[T5]{fontenc}
% See https://www.latex-project.org/help/documentation/encguide.pdf for other character sets

% This assumes your files are encoded as UTF8
\usepackage[utf8]{inputenc}

% This is not strictly necessary, and may be commented out,
% but it will improve the layout of the manuscript,
% and will typically save some space.
\usepackage{microtype}

% This is also not strictly necessary, and may be commented out.
% However, it will improve the aesthetics of text in
% the typewriter font.
\usepackage{inconsolata}

%Including images in your LaTeX document requires adding
%additional package(s)
\usepackage{graphicx}

% If the title and author information does not fit in the area allocated, uncomment the following
%
%\setlength\titlebox{<dim>}
%
% and set <dim> to something 5cm or larger.

\title{RGAR: Recurrence Generation-augmented Retrieval for Factual-aware Medical Question Answering}

% Author information can be set in various styles:
% For several authors from the same institution:
% \author{Author 1 \and ... \and Author n \\
%         Address line \\ ... \\ Address line}
% if the names do not fit well on one line use
%         Author 1 \\ {\bf Author 2} \\ ... \\ {\bf Author n} \\
% For authors from different institutions:
% \author{Author 1 \\ Address line \\  ... \\ Address line
%         \And  ... \And
%         Author n \\ Address line \\ ... \\ Address line}
% To start a separate ``row'' of authors use \AND, as in
% \author{Author 1 \\ Address line \\  ... \\ Address line
%         \AND
%         Author 2 \\ Address line \\ ... \\ Address line \And
%         Author 3 \\ Address line \\ ... \\ Address line}

% \author{First Author \\
%   Affiliation / Address line 1 \\
%   Affiliation / Address line 2 \\
%   Affiliation / Address line 3 \\
%   \texttt{email@domain} \\\And
%   Second Author \\
%   Affiliation / Address line 1 \\
%   Affiliation / Address line 2 \\
%   Affiliation / Address line 3 \\
%   \texttt{email@domain} \\}

\author{
 \textbf{Sichu Liang \textsuperscript{1}}\thanks{Equal Contribution},
 \textbf{Linhai Zhang \textsuperscript{2}}\footnotemark[1],
 \textbf{Hongyu Zhu \textsuperscript{3}}\footnotemark[1],
 \textbf{Wenwen Wang\textsuperscript{4}},
 \textbf{Yulan He\textsuperscript{2, 5}},
 \textbf{Deyu Zhou \textsuperscript{1}}\thanks{Corresponding author} 
%  \textbf{Seventh Author\textsuperscript{1}},
%  \textbf{Eighth Author \textsuperscript{1,2,3,4}},
% \\
%  \textbf{Ninth Author\textsuperscript{1}},
%  \textbf{Tenth Author\textsuperscript{1}},
%  \textbf{Eleventh E. Author\textsuperscript{1,2,3,4,5}},
%  \textbf{Twelfth Author\textsuperscript{1}},
% \\
%  \textbf{Thirteenth Author\textsuperscript{3}},
%  \textbf{Fourteenth F. Author\textsuperscript{2,4}},
%  \textbf{Fifteenth Author\textsuperscript{1}},
%  \textbf{Sixteenth Author\textsuperscript{1}},
% \\
%  \textbf{Seventeenth S. Author\textsuperscript{4,5}},
%  \textbf{Eighteenth Author\textsuperscript{3,4}},
%  \textbf{Nineteenth N. Author\textsuperscript{2,5}},
%  \textbf{Twentieth Author\textsuperscript{1}}
% \\
\\
 \textsuperscript{1}School of Computer Science and Engineering, Key Laboratory of New Generation Artificial Intelligence \\Technology and Its Interdisciplinary Applications, Southeast University, Ministry of Education, China
 \\
\textsuperscript{2}Department of Informatics, King's College London, UK\\
 \textsuperscript{3}	School of Electronic Information and Electrical Engineering, Shanghai Jiao Tong University, China\\
 \textsuperscript{4}School of Electrical and Computer Engineering, Carnegie Mellon University, USA\\
 \textsuperscript{5}The Alan Turing Insitute, UK\\
}

\begin{document}
\maketitle
\begin{abstract}
%Medical question answering demands substantial access to specialized conceptual knowledge. The current paradigm, Retrieval-Augmented Generation (RAG), acquires medical knowledge through large-scale corpus retrieval and transfers this knowledge to a general-purpose large language model (LLM) for generating answers. 
Medical question answering requires extensive access to specialized \textit{conceptual knowledge}. The current paradigm, Retrieval-Augmented Generation (RAG), acquires expertise medical knowledge through large-scale corpus retrieval and uses this knowledge to guide a general-purpose large language model (LLM) for generating answers. 
%However, existing retrieval approaches lack dedicated attention to and consideration of factual knowledge, limiting the relevance and effectiveness of conceptual knowledge retrieval and hindering applications in real-world scenarios such as clinical decision-making based on Electronic Health Records (EHRs).
% However, existing retrieval approaches lack dedicated consideration of \textit{factual knowledge}, limiting the relevance of retrieved conceptual knowledge and hindering applications in real-world scenarios such as clinical decision-making based on Electronic Health Records (EHRs).
However, existing retrieval approaches often overlook the importance of \textit{factual knowledge}, which limits the relevance of retrieved conceptual knowledge and restricts its applicability in real-world scenarios, such as clinical decision-making based on Electronic Health Records (EHRs).
%This paper presents RGAR, a recurrence generation-augmented retrieval framework that retrieve relevant factual knowledge and conceptual knowledge from dual ends, allowing them to interact and update one another.
This paper introduces RGAR, a recurrence generation-augmented retrieval framework that retrieves both relevant \textit{factual} and \textit{conceptual} knowledge from dual sources (i.e., EHRs and the corpus), allowing them to interact and refine each another.
% This paper presents RGAR, a recurrence generation-augmented retrieval framework that leverages retrieved medical knowledge to continuously extract question-relevant factual knowledge from queries and transform it into retrieval-optimized representations, ultimately facilitating more relevant medical knowledge retrieval. 
% Through extensive evaluation across three factual-aware medical question answering benchmarks, RGAR sets a new state-of-the-art performance among medical RAG systems.
Through extensive evaluation across three factual-aware medical question answering benchmarks, RGAR establishes a new state-of-the-art performance among medical RAG systems.
Notably, the Llama-3.1-8B-Instruct model with RGAR surpasses the considerably larger, RAG-enhanced GPT-3.5. 
%Our findings reveal that extracting factual knowledge significantly enhances system performance, consistently yielding improved retrieval accuracy.
Our findings demonstrate the benefit of extracting factual knowledge for retrieval, which consistently yields improved generation quality.
\end{abstract}

\section{Introduction}
Large Language Models (LLMs) have demonstrated remarkable capabilities in general question answering (QA) tasks, achieving impressive performance across diverse scenarios \cite{achiam2023gpt}. However, when facing domain-specific questions that require specialized expertise, from medical diagnosis \cite{jin2021disease} to legal charge prediction \cite{wei-etal-2024-mud}, these models face significant challenges, often generating unreliable conclusions due to both hallucinations \cite{ji2023survey} and potentially stale knowledge embedded in their parameters \cite{wang2024knowledge}. % These issues can be especially dangerous in high-stakes domains such as healthcare, where incorrect information could lead to serious consequences \cite{tian2024opportunities}.

% Addressing the task of question answering (QA) presents significant challenges, as it demands intricate reasoning involving both the explicit constraints articulated in the questions and the implicit domain knowledge \cite{frisoni-etal-2024-generate}. Such difficult tasks effectively reflect the complexities of real-life scenarios and are prevalent in fields requiring specialized knowledge, ranging from medical diagnostics \cite{jin2021disease} to predictions of criminal charges \cite{wei-etal-2024-mud}. 

% Open-domain question answering (OpenQA) \cite{chen-etal-2017-reading} aims to deal with real-world queries without relying on expert knowledge from any predefined domain. This approach effectively reflects the complexities of real-life scenarios, where each potential question lacks a pre-labeled body of text containing the answer. As one of the most challenging forms of question answering, OpenQA has been widely applied in professional knowledge reasoning tasks, ranging from medical diagnostics \cite{jin2021disease} to criminal charge predictions \cite{wei-etal-2024-mud}.

\begin{figure}
    \centering
    \includegraphics[width=\linewidth]{small.pdf}
    \caption{a) Medical AI Systems from the Perspective of Bloom's Taxonomy. b) Two Types of Medical Question Answering Tasks.}
    \label{fig:enter-label}
\end{figure}

\textbf{Retrieval-Augmented Generation (RAG)} \cite{lewis2020retrieval} has emerged as a promising approach to address these challenges by leveraging extensive, trustworthy knowledge bases to support LLM reasoning. The effectiveness of this approach, however, heavily depends on the relevance of retrieved documents. Recent advances, such as \textbf{Generation-Augmented Retrieval (GAR)} \cite{mao-etal-2021-generation}, focus on enhancing retrieval performance by generating relevant context for query expansion.

% By retrieving relevant document chunks from extensive, trustworthy knowledge bases to assist LLMs, \textbf{Retrieval-Augmented Generation (RAG)} \cite{lewis2020retrieval} has shown promise in tackling the above challenges. However, the relevance of the retrieved documents plays a crucial role in the model's reasoning capability. Recent approaches, such as \textbf{Generation-Augmented Retrieval (GAR)} \cite{mao-etal-2021-generation}, 
% focus on query formulation and propose query expansion methods to enhance generation performance.
% focus on how query formulation influences document relevance, proposing methods expending queries into multiple variations to enhance generation performance.

In the medical domain, current RAG approaches concatenate all available contextual information from a given example into a single basic query for retrieval, aiming to provide comprehensive context for model reasoning \cite{xiong-etal-2024-benchmarking}. While this method has demonstrated substantial improvements on early \textit{knowledge-intensive} medical QA datasets such as PubMedQA \cite{jin-etal-2019-pubmedqa}, its limitations have become increasingly apparent with the emergence of EHR-integrated datasets that better reflect real-world clinical practices \cite{kweon2024ehrnoteqa}. Electronic Health Records (EHRs) typically contain extensive patient data, including comprehensive diagnostic test results and medical histories \cite{pang2021cehr}. However, for any specific medical query, only a small subset of this information is typically relevant, and retrieval performance can be significantly degraded when queries are diluted with extraneous EHR content \cite{johnson2023mimic, lovon-melgarejo-etal-2024-revisiting}.

% The current RAG approach to solving medical problems concatenates all contextual information from a given example into a basic query for retrieval, aiming to capture the most comprehensive content for model reasoning \cite{xiong-etal-2024-benchmarking}. This method has achieved significant improvements on early knowledge-intensive medical QA datasets like PubMedQA \cite{jin-etal-2019-pubmedqa}.
% However, the emergence of EHR-integrated datasets, which better align with real-world clinical practices \cite{kweon2024ehrnoteqa}, reveals critical limitations of this paradigm. Electronic Health Records (EHRs) typically contain extensive patient data, including all diagnostic test results and medical histories \cite{pang2021cehr}, yet only a small fraction of this data is relevant to a specific question. Retrieval performance can be impaired when queries contain lengthy, irrelevant texts from EHRs \cite{johnson2023mimic, lovon-melgarejo-etal-2024-revisiting}.

% We highlight that current \textit{retrieval methods} often fail to adequately consider factual information. Real-world medical scenarios are inherently \textbf{factual-aware}, emphasizing the importance of factual information, such as EHRs, which are crucial for providing personalized and accurate medical advice for a specific query.

We highlight that current \textit{retrieval methods} often fail to adequately consider \textit{factual information} in real-world medical scenarios. Crucially, even when applying query expansion with GAR, the persistent oversight of factual information fundamentally limits their ability to retrieve real relevant documents.

% In \textbf{real-world} medical QA scenarios, it is crucial to consider not only \textit{professionally knowledge} but also \textit{factual-aware information}. \textbf{Factual-aware information} consists of essential factual content, such as electronic health records (EHRs), which are essential for delivering personalized and accurate medical advice. Early medical QA datasets primarily focused on professional knowledge \cite{jin-etal-2019-pubmedqa}, while more recent ones have recognized the significance of factual information, designing resources that better align with real-world clinical practice \cite{kweon2024ehrnoteqa}.


% However, current \textit{retrieval methods} often fail to adequately account for factual information. Existing studies simply concatenate EHR data with the query, assuming that retrieving relevant professional knowledge is sufficient for LLMs to solve the corresponding problem \cite{xiong-etal-2024-benchmarking}. This approach relies on the oversimplified assumption that all EHR content is relevant to the specific question, which is rarely the case in practice. EHRs typically contain comprehensive patient data, including all diagnostic test results and medical histories \cite{pang2021cehr}, of which only a small fraction is relevant to any specific question. Furthermore, the verbosity of EHRs can hinder retrieval performance when irrelevant, lengthy texts are included \cite{johnson2023mimic, lovon-melgarejo-etal-2024-revisiting}.

Inspired by \textbf{Bloom's taxonomy} \cite{forehand2010bloom,markus2001toward}, we categorize the knowledge required to address real-world medical QA problems into four types: \textit{Factual Knowledge}, \textit{Conceptual Knowledge}, \textit{Procedural Knowledge}, and \textit{Metacognitive Knowledge}.
The latter two represent higher-order knowledge typically embedded within advanced RAG systems. Specifically, \textit{Procedural Knowledge} refers to the processes and strategies required to solve problems, such as problem decomposition and retrieval \cite{wei2022chain, zhou2023leasttomost}, while \textit{Metacognitive Knowledge} pertains to an LLM's ability to assess whether it has sufficient knowledge or evidence to perform effective reasoning \cite{kim-etal-2023-tree, wang-etal-2023-self-knowledge}.

\textit{Factual Knowledge} and \textit{Conceptual Knowledge} require retrieval from large databases containing substantial amounts of irrelevant content, corresponding to the EHRs of patients and medical corpora in answering medical questions. Unfortunately, current RAG systems do not differentiate between these types of \textit{retrieval targets}, overlooking the necessity of retrieval from EHRs.

% \textit{Factual Knowledge} and \textit{Conceptual Knowledge} involve processing information from extensive databases that contain significant amounts of irrelevant content, corresponding to the EHRs of patients and the corpora of medical knowledge, to answer medical questions. Unfortunately, current RAG systems do not distinguish between these types of \textit{retrieval targets}, overlooking the necessity of retrieving information specifically from EHRs.

To overcome this limitation, we propose \textbf{RGAR}, a system designed to simultaneously retrieves \textit{Factual Knowledge} and \textit{Conceptual Knowledge} through a recurrent query generation and interaction mechanism. This approach iteratively refines queries to enhance the relevance of retrieved professional and factual knowledge, thereby improving performance on \textit{knowledge-intensive} and \textit{factual-aware} medical QA tasks.

Our key contributions are listed as follows:
\begin{itemize}
\item We are the first to analyze RAG systems through the lens of Bloom's taxonomy, addressing the current underrepresentation of \textit{Factual Knowledge} in existing frameworks.
\item We introduce RGAR, a dual-end retrieval system that facilitates recurrent interactions between \textit{Factual} and \textit{Conceptual} Knowledge, bridging the gap between LLMs and real-world clinical applications.
\item Through extensive experiments on three medical QA datasets involving \textit{Factual Knowledge}, we demonstrate that RGAR achieves superior average performance compared to state-of-the-art (SOTA) methods, enabling Llama-3.1-8B-Instruct model to outperform the considerably larger RAG-enhanced GPT-3.5-turbo.
\end{itemize}

\section{Related Work}
\textbf{RAG Systems. } RAG systems are characterized as a "Retrieve-then-Read" framework \cite{gao2023retrieval}. The development of Naive RAG has primarily focused on retriever optimization, evolving from discrete retrievers such as BM25 \cite{friedman1977algorithm} to more sophisticated and domain-specific dense retrievers, including DPR \cite{karpukhin-etal-2020-dense} and MedCPT \cite{jin2023medcpt}, which demonstrate superior performance.

In recent years, numerous advanced RAG systems have emerged. Advanced RAG systems focus on designing multi-round retrieval structures, including iterative retrieval \cite{sun2019pullnet}, recursive retrieval \cite{sarthi2024raptor}, and adaptive retrieval \cite{jeong-etal-2024-adaptive}. A notable work in medical QA is MedRAG \cite{xiong-etal-2024-benchmarking}, which analyzes retrievers, corpora, and LLMs, offering practical guidelines. Follow-up work, $i$-MedRAG \cite{xiong2024improving}, improved performance through multi-round decomposition and iteration, albeit with significant computational costs.

These approaches focus solely on optimizing the retrieval process, overlooking the retrievability of \textit{factual knowledge}. In contrast, RGAR introduces a recurrent structure, enabling continuous query optimization through dual-end retrieval and extraction from EHRs and professional knowledge corpora, thereby enhancing access to both knowledge types.

\textbf{Query Optimization. } As the core interface in human-AI interaction, query optimization (also known as prompt optimization) is the key to improving AI system performance. It is widely applied in tasks such as text-to-image generation \cite{liu2022compositional, wu-etal-2024-universal} and code generation \cite{nazzal2024promsec}.

In the era of large language models, query optimization for retrieval tasks has gained increasing attention. Representative work includes GAR \cite{mao-etal-2021-generation}, which improves retrieval performance through query expansion using fine-tuned BERT models \cite{devlin-etal-2019-bert}. GENREAD \cite{yu2023generate} further explored whether LLM-generated contexts could replace retrieved professional documents as reasoning evidence. MedGENIE \cite{frisoni-etal-2024-generate} extended this approach to medical QA.

Another line of work focuses on query transformation and decomposition, breaking down original queries into multiple sub-queries tailored to specific tasks, enhancing retrieval alignment with model needs \cite{dhuliawala2023chain}. Subsequent work has reinforced the effectiveness of query decomposition through fine-tuning \cite{ma2023query}.

Using expanded queries directly as reasoning evidence lacks the transparency of RAG, as RAG relies on retrievable documents that provide traceable and trustworthy reasoning, which is crucial in the medical field.
Besides, the effectiveness of query expansion and query decomposition approaches is heavily dependent on fine-tuning LLMs, which limits scalability.

%Additionally, domain-specific LLMs that generate reasoning evidence face challenges in knowledge updating \cite{wang2024knowledge}, making RAG a more robust solution.

In contrast, our work focuses on query optimization without fine-tuning LLMs. Specifically, retrieval from EHRs can be seen as query filtering that eliminates irrelevant information, thereby obtaining pertinent \textit{factual knowledge}. Extracting factual knowledge enhances the effectiveness of retrieval from the corpus.

%\subsection{Medical Question Answering}

%Recent medical QA datasets such as MMLU-Med (Measuring Massive Multitask Language Understanding), PubMedQA (PubMedQA: A Dataset for Biomedical Research Question Answering), and BioASQ-Y/N (An Overview of the BIOASQ Large-Scale Biomedical Semantic Indexing and Question Answering Competition) require models to master vast amounts of medical knowledge not provided within the question context, exemplifying the challenges of open-domain question answering. The MIRAGE benchmark adopts a Question-Only Retrieval (QOR) paradigm, aligning with real-world cases of medical QA, where answer options should not be presented as input during retrieval.

%To better approximate clinical diagnosis scenarios, some datasets, such as MedQA-US (What Disease Does This Patient Have? A Large-Scale Open Domain Question Answering Dataset from Medical Exams) and MedMCQA (MedMCQA: A Large-Scale Multi-Subject Multi-Choice Dataset for Medical Domain Question Answering), incorporate specific patient cases within their questions, demanding that models apply medical knowledge to resolve practical issues. This represents a simplified form of factual-aware medical question answering. The latest dataset, EHRNoteQA, utilizes original EHR data from MIMIC-IV, necessitating that models accurately identify which factual information within the EHR aligns with the posed question and leverage specialized knowledge to formulate answers.

%Our approach adopts the MIRAGE benchmark's framework, focusing on enhancing models' capabilities in factual-aware medical question answering.

\begin{figure*}
    \centering
    \includegraphics[width=\linewidth]{pipeline.pdf}
    \caption{The Overall Framework of RGAR. a) The Recurrence Pipeline in § \ref{sec:pipeline}; b) Conceptual Knowledge Retrieval in § \ref{sec:Train-free}; c) Factual Knowledge Extraction in § \ref{sec:Extraction}; d) Response Template in § \ref{sec:pipeline}.}
    \label{fig:pipeline}
\end{figure*}

\section{Methodology}
% 开头整段都要改
% 已经改过了
In this section, we introduce RGAR framework, as illustrated in Figure \ref{fig:pipeline}. It begins by prompting a general-purpose LLM to generate multiple queries from an initial basic query. These multiple queries are then used to \textbf{retrieve conceptual knowledge} from the corpus (§ \ref{sec:Train-free}). Then retrieved conceptual knowledge is subsequently used to \textbf{extract factual knowledge} from the electronic health records (EHRs) and transform it into retrieval-optimized representations (§ \ref{sec:Extraction}). The \textbf{recurrence pipeline} continuously updates the basic query and iteratively executes the two aforementioned components. This process optimizes the retrieved results, ultimately improving the quality of responses.(§ \ref{sec:pipeline}).
\subsection{Task Formulation}
In \textit{factual-aware} medical QA, each data sample comprises the following elements: a patient's natural language query $\mathcal{Q}$, the electronic health record (EHR) as factual knowledge $\mathcal{F}$, and a set of candidate answer $\mathcal{A} = \{a_1, ..., a_{|\mathcal{A}|}\}$. The overall goal is to identify the correct answer $\hat{a}$ from $\mathcal{A}$.

A \textit{non-retrieval} approach directly prompts an LLM to act as a \textbf{reader}, processing the entire context and generating an answer, formulated as:

\begin{equation}
\hat{a}=\textbf{LLM}(\mathcal{F},\mathcal{Q},\mathcal{A}|\mathcal{T}_r)
\end{equation}

where $\mathcal{T}_r$ is the prompts. However, this approach relies exclusively on the conceptual knowledge encoded within LLM, without leveraging external, trustworthy medical knowledge sources.

To overcome this limitation, recent studies have explored \textit{retrieval-based} approaches, which enhance the model’s knowledge by retrieving a specified number $N$ of chunks, denoted as $\mathcal{C} = \{c_1, ..., c_N\}$, from a chunked corpus (knowledge base) $\mathcal{K}$. This answering process is expressed as:

\begin{equation}
\hat{a}=\textbf{LLM}(\mathcal{F},\mathcal{Q},\mathcal{A},\mathcal{C}|\mathcal{T}_r).
\label{eq:retrieval-augmented}
\end{equation}

\subsection{Conceptual Knowledge Retrieval (CKR)}
\label{sec:Train-free}
To maintain consistency with the \textit{option-free retrieval approach} proposed by \cite{xiong-etal-2024-benchmarking}, we do not incorporate the answer options $\mathcal{A}$ during retrieval. This design is in line with real-world medical quality assurance scenarios, where answer choices are typically not available in advance.

Following their method, we construct the \textbf{basic query} by concatenating the EHR and the patient's query, formally defined as $q_b = \mathcal{Q} \oplus \mathcal{F}$, where $\oplus$ denotes text concatenation.

Traditional dense retrievers, such as Dense Passage Retrieval (DPR) \cite{karpukhin-etal-2020-dense}, identify the top-$N$ relevant chunks $C$ from the knowledge base $\mathcal{K}$ by computing similarity scores using an encoder $E$:

\begin{equation}
\begin{split}
    &\text{sim}(q_b, c_i) = E(q_b)^\top E(c_i), \\
    &\mathcal{C} = \text{top-}N(\{\text{sim}(q_b, c_i)\}).
\end{split}
\end{equation}


Vanilla GAR \cite{mao-etal-2021-generation} expands $q_b$ using a fine-tuned BERT \cite{devlin-etal-2019-bert} to produce three types of content that enhance retrieval: potential answers $q_e^a$, contexts $q_e^c$, and titles $q_e^t$.
With the growing zero-shot generation capabilities of LLMs \cite{kojima2022large}, a common practice is to prompt LLMs to serve as train-free query \textbf{generators}, producing expanded content $\tilde{q}_e$ using prompt templates $\mathcal{T}_g$ \cite{frisoni-etal-2024-generate}. The three types of content generation process can be formulated as:

\begin{equation}
\label{eq:query-generation}
\begin{array}{l}
\tilde{q}_e^a = \textbf{LLM}(q_b |\mathcal{T}^a_g), \\[1ex]
\tilde{q}_e^c = \textbf{LLM}(q_b |\mathcal{T}^c_g), \\[1ex]
\tilde{q}_e^t = \textbf{LLM}(q_b |\mathcal{T}^t_g).
\end{array}
\end{equation}

% 这样改了一下,不知道合不合适
%非常合适
%The final score $Sc$ to get retrieved $\mathcal{C}$ is then obtained by normalizing and averaging the similarities of these expanded queries:
The final score $Sc$ for retrieving $\mathcal{C}$ is then computed by normalizing and averaging the similarities of these expanded queries:
\begin{equation}
\label{eq:normalized-retrieval-score}
\text{Sc}(c_i) = \sum_{\tilde{q}_e \in \{\tilde{q}_e^a, \tilde{q}_e^c, \tilde{q}_e^t\}} \frac{\exp(\text{sim}(\tilde{q}_e, c_i))}{\sum_{c_j} \exp(\text{sim}(\tilde{q}_e, c_j))}.
\end{equation}


\subsection{Factual Knowledge Extraction (FKE)}
\label{sec:Extraction}

In EHR, only a small portion of necessary information constitutes problem-relevant factual knowledge \cite{d2004evaluation}. Direct input of lengthy EHR content containing substantial irrelevant information into dense retrievers can degrade retrieval performance \cite{ren-etal-2023-thorough}. While a straightforward approach would be to retrieve EHR content based on question $\mathcal{Q}$ \cite{factual_aware}, this fails to fully utilize conceptual knowledge obtained from previous Conceptual Knowledge Retrieval Stage. Furthermore, the necessary chunking of EHR for retrieval introduces content discontinuity \cite{luo-etal-2024-landmark}.

Given that EHRs more closely resemble long passages from the Needle in a Haystack task \cite{kamradt2024needle} rather than necessarily chunked corpus, and inspired by large language models' capability to precisely locate answer spans in reading comprehension tasks \cite{cheng2024adapting}, we propose leveraging LLMs for text span tasks \cite{rajpurkar-etal-2016-squad} on EHR to filter relevant factual knowledge efficiently and effectively using conceptual knowledge. We define this filtered factual knowledge as $\mathcal{F}_s$, with prompts $\mathcal{T}_s$, expressed as:
\begin{equation}
    \mathcal{F}_s=\textbf{LLM}(\mathcal{F},\mathcal{Q},\mathcal{C}|\mathcal{T}_s).  
\end{equation}


In addition, EHRs often contain numerical report results \cite{lovon-melgarejo-etal-2024-revisiting} that require conceptual knowledge to interpret their significance. Furthermore, medical QA involves multi-hop questions \cite{pal2022medmcqa}, where retrieved conceptual knowledge can generate explainable new factual knowledge conducive to reasoning. Drawing from LLM zero-shot summarization prompting strategies \cite{wu2025towards}, we analyze and summarize the filtered EHR $\mathcal{F}_s$ with prompts $\mathcal{T}_e$, yielding an enriched representation $\mathcal{F}_e$:
\begin{equation}
    \mathcal{F}_e=\textbf{LLM}(\mathcal{F}_s,\mathcal{Q},\mathcal{C}|\mathcal{T}_e).  
\end{equation}



This process, which we refer to as the LLM \textbf{Extractor}, completes the extraction of original EHR information. In practice, RGAR implements these two phases using single-stage prompting to reduce time overhead. 

% This new reliable factual knowledge enables deeper reasoning and analysis in the GAR, generating multi-queries for multi-hop knowledge retrieval.
% The length and complexity of EHR documents often pose significant challenges when it comes to efficiently extracting relevant information \cite{d2004evaluation}. Feeding lengthy, question-irrelevant EHR content directly into dense retrievers can degrade retrieval performance. A straightforward solution is to segment the content into chunks and use $\mathcal{Q}$ to retrieve necessary information from these chunks. However, dense retrievers primarily measure textual similarity, and the query often lacks the direct semantic links needed to connect $\mathcal{Q}$ with the underlying medical concepts in the original EHR content $\mathcal{F}$. This highlights the importance of the retrieved conceptual knowledge $C = \{c_1, c_2, \dots, c_N\}$, which serves as a vital bridge between $\mathcal{Q}$ and $\mathcal{F}$.

% Consider two straightforward query construction strategies:

% 1. Single unified query:  
%    \begin{equation}
%    \label{eq:single-query}
%    q_{\text{all}} = \mathcal{Q} \oplus c_1 \oplus c_2 \oplus \cdots \oplus c_N
%    \end{equation}  
%    In this approach, all conceptual knowledge is concatenated directly to the query. However, since each \( c_i \) corresponds to a distinct medical concept, the resulting embeddings blend multiple types of information, making it difficult for the retriever to focus on the single relevant signal.

% 2. Separate queries for each concept \( c_i \):  
%    \begin{equation}
%    \label{eq:separate-query}
%    q_{c_i} = \mathcal{Q} \oplus c_i
%    \end{equation}  
%    Here, each \( c_i \) is used to create a separate query. Retrieval is performed for each query independently, and the final result is obtained by averaging the normalized similarity scores of all \( q_{c_i} \). While this approach more precisely captures the relationship between each concept and the corresponding EHR segments, it introduces substantial computational overhead, requiring \( N \) independent retrieval operations. As such, it is impractical for real-world deployment scenarios.

% Inspired by the ability of large language models (LLMs) to locate answers within lengthy text passages \cite{cheng2024adapting} and their role in relevance assessment in RAG systems \cite{es-etal-2024-ragas}, we propose leveraging LLMs’ conditional generation capabilities to \textit{approximate} the retrieval task. Specifically, we design a set of retrieval prompts $\mathcal{T}^r$ that guide the LLM to produce an output distribution approximating the results of traditional dense retrievers:  
% \begin{equation}
% \label{eq:conceptual-retrieval}
% p_\theta(\mathcal{F}_r \mid \mathcal{F}, \mathcal{Q}, \mathcal{T}^r) \approx p_r(\mathcal{F}_r \mid \mathcal{Q}, \mathcal{F}),
% \end{equation}
% where $\mathcal{F}_r \subseteq \mathcal{F}$ represents EHR fragments relevant to $\mathcal{Q}$, and $\mathcal{T}^r$ represents prompts specifically designed for the retrieval task.

% Since large language models possess strong contextual retrieval and understanding capabilities, they can mitigate the information loss that may occur when embedding all $C$ into a single query. Thus, we approximate:  
% \begin{equation}
% \begin{split}
% p_\theta(\mathcal{F}_r \mid \mathcal{F}, Q, C, \mathcal{T}^r) &\approx \frac{1}{N}\sum_{i=1}^{N} \\
% &p_\theta\Big(\mathcal{F}_r \,\big|\, \mathcal{F}, Q \oplus c_i, \mathcal{T}^r\Big).
% \end{split}
% \end{equation}

% In other words, the performance of a single concatenated query approximately matches the normalized average results of individual queries $q_{c_i}=Q\oplus c_i$. The accuracy of this approximation depends on $p_\theta$.

% EHR documents often contain numerous numerical test results, which can be difficult for retrievers to match conceptually. For example, “Platelet count 14,200/mm³” might correspond to “low platelet count” in medical literature. To address this, we use the concept knowledge $C$ as a supplementary condition and employ specialized rewriting prompts $\mathcal{T}^{\text{rew}}$ to guide the LLM in rewriting retrieved EHR fragments. The process is formalized as:
% \begin{equation}
% p_\theta(\mathcal{F}_{re} \mid \mathcal{F}_r, \mathcal{Q}, \mathcal{T}^{\text{rew}})
% \end{equation}
% The model is expected to achieve the following objectives under the guidance of the prompt template and the provided concept knowledge $C$:  

% \textbf{First}, normalization of numerical information: Transform numerical expressions into standardized text descriptions. For instance, rewriting “Platelet count 14,200/mm³” into “low platelet count” facilitates retrieving truly relevant articles.  
% \textbf{Second}, information fusion: When certain indicators are scattered across multiple document fragments, the model can integrate them into a more comprehensive interpretation. For example, if one fragment mentions “elevated white blood cell count” and another mentions “decreased platelet count,” the model might generate “the patient shows signs of an inflammatory response accompanied by thrombocytopenia.” This provides a more complete context for generating question-relevant contexts or answers in the GAR process.


\subsection{The Recurrence Pipeline and Response}
\label{sec:pipeline}
% The extracted EHR information serves as question-relevant Factual knowledge $\mathcal{F}_e$, updating the Basic query $q_b$ through $\mathcal{Q} \oplus \mathcal{F}_e$.
% Building on the $\mathcal{F}_r$ obtained in the previous stage, we update the basic query $q_b = \mathcal{Q} \oplus \mathcal{F}_r$. \textit{Training-free Generation-augmented Retrieval} and \textit{Conditional Generating Retrieved and Rewritten EHR} stages are then iteratively performed until a predetermined number of iterations is reached. Ultimately, this iterative optimization yields the final retrieved conceptual knowledge $C^*$. 

Building on the \(\mathcal{F}_e\), we \textbf{update} the basic query for Conceptual Knowledge Retrieval as \(q_b = \mathcal{Q} \oplus \mathcal{F}_e\). This establishes a \textbf{recurrence interaction} between factual and conceptual knowledge, guiding next retrieval toward more relevant content. Iterative execution enhances the stability of both retrieval and extraction. The entire pipeline recurs for a predefined number of iterations, ultimately yielding the final retrieved conceptual knowledge $\mathcal{C}^*$.

% 这里改了一下
% During the response phase, we adhere to the approach outlined in Equation \ref{eq:retrieval-augmented} to produce answers. In particular, the $\mathcal{F}_e$ are confined to the retrieval phase and are not utilized in the response phase. The only difference lies in the retrieved chunks, which allows us to clearly demonstrate the impact of retrieval quality on the response phase.
During the response phase, we follow the approach in Equation \ref{eq:retrieval-augmented} to generate answers. Notably, the $\mathcal{F}_e$ are restricted to the retrieval phase and are not used in the response phase. The sole difference lies in the retrieved chunks, highlighting the impact of retrieval quality on the responses.

\section{Experiments}
\subsection{Experimental Setup}
\subsubsection{Benchmark Datasets}

We evaluated RGAR on three \textit{factual-aware} medical QA benchmarks featuring multiple-choice questions that require human-level reading comprehension and expert reasoning to analyze patients' clinical conditions.

% We evaluated RGAR on three \textit{factual-aware} medical QA benchmarks featuring multiple-choice questions that require multi-hop reasoning and human-level reading comprehension.  

% We evaluated RGAR on three \textit{factual-aware} medical QA benchmarks. Table \ref{tab:qa_benchmarks} illustrates the statistical differences between these factual-aware questions and traditional question types. All three datasets are multiple-choice OpenQA benchmarks that require multi-hop reasoning and human-level reading comprehension capabilities.


\textbf{MedQA-USMLE} \cite{jin2021disease} and \textbf{MedMCQA} \cite{pal2022medmcqa} consist of questions derived from professional medical exams, evaluating specialized expertise such as disease symptom diagnosis and medication dosage requirements. The problems frequently involve patient histories, vital signs (e.g., blood pressure, temperature), and final diagnostic evaluations (e.g., CT scans), making it necessary to retrieve relevant medical knowledge tailored to the patient’s specific circumstances. However, due to their exam-oriented format, the provided information has already been filtered, reducing the difficulty of extracting factual knowledge from EHR.

\textbf{EHRNoteQA} \cite{kweon2024ehrnoteqa} is a recently introduced benchmark that provides authentic, complex EHR data derived from MIMIC-IV \cite{johnson2023mimic}. This dataset encompasses a wide range of topics and demands that models emulate genuine clinical consultations, ultimately generating accurate discharge recommendations. Consequently, EHRNoteQA challenges models to identify which \textit{factual details} within the EHR are relevant to the questions at hand and apply domain-specific knowledge to address them.

\begin{table}[htbp]
  \centering
  \caption{Medical QA Benchmark Statistics.}
  \resizebox{\linewidth}{!}{ % 让表格适应页面宽度
    \begin{tabular}{lccc}
      \toprule
      Benchmarks & Max. Len & Avg. Len & Min. Len \\
            \midrule
      \rowcolor{gray!20} \multicolumn{4}{c}{Non-EHR QA Benchmarks} \\
      \midrule
        BioASQ-Y/N & 52 & 17 & 9  \\
      PubMedQA & 57 & 23 & 10  \\
      \midrule
      \rowcolor{gray!20} \multicolumn{4}{c}{EHR QA Benchmarks} \\
      \midrule
      MedMCQA & 207 & 41 & 11  \\
      MedQA-USMLE & 872 & 197 & 50  \\
      EHRNoteQA & 5782 & 3061 & 667  \\

      \bottomrule
    \end{tabular}
  }
  \label{tab:qa_benchmarks}
\end{table}

Table \ref{tab:qa_benchmarks} highlights that the chosen datasets, which include EHR information, tend to have significantly \textbf{longer} content compared to datasets without EHRs. Notably, the EHRNoteQA dataset has a maximum length exceeding 4,000 tokens. This raises concerns about the reasonableness of directly employing these EHRs for retrieval.

\begin{table*}[htbp]
  \centering
  \caption{Comparison of RGAR with Other Methods on Three Factual-Aware Datasets. $\Delta$ Indicates Improvement Over Custom, \textbf{Bold} Represents the Best, and \underline{Underline} Indicates the Second-Best.}
  \resizebox{\linewidth}{!}{%
    \begin{tabular}{llcccccc|cc}
      \toprule
      \multicolumn{2}{c}{\multirow{2}{*}{Method}} & \multicolumn{2}{c}{MedQA-USMLE (\# 1273)} & \multicolumn{2}{c}{MedMCQA(\# 4183)} & \multicolumn{2}{c|}{EHRNoteQA(\# 962)} & \multicolumn{2}{c}{Average(↓)} \\
      \cmidrule(lr){3-4} \cmidrule(lr){5-6} \cmidrule(lr){7-8} \cmidrule(lr){9-10}
      \multicolumn{2}{c}{} & Acc. & $\Delta$ & Acc. & $\Delta$ & Acc. & $\Delta$ & Acc. & $\Delta$ \\
      \midrule
      \multirow{2}{*}{w/o Retrieval} & Custom  & 50.20 & 0.00  & 50.01 & 0.00  & 47.19 & 0.00  & 49.13 & 0.00  \\
                               & CoT     & 51.45 & 1.25  & 44.53 & -5.48 & 62.89 & 15.70 & 52.96 & 3.82  \\
      \midrule
      \multirow{5}{*}{w/ Retrieval}  & RAG     & 53.50 & 3.30  & \underline{50.54} & \underline{0.53}  & 61.12 & 13.93 & 55.05 & 5.92  \\
                               & MedRAG  & 50.27 & 0.07  & 47.53 & -2.48 & 70.58 & 23.39 & 56.13 & 6.99  \\
                               & GAR     & \underline{57.97} & \underline{7.77}  & 50.42 & 0.41  & 65.48 & 18.29 & 57.96 & 8.82  \\
                               & $i$-MedRAG & 56.24 & 6.04  & 44.94 & -5.07 & \textbf{74.22} & \textbf{27.03} & \underline{58.47} & \underline{9.33}  \\
                               & RGAR    & \textbf{58.83} & \textbf{8.63}  & \textbf{51.02} & \textbf{1.01}  & \underline{73.28} & \underline{26.09} & \textbf{61.04} & \textbf{11.91} \\
      \bottomrule
    \end{tabular}%
  }
  \label{tab:mian_results}
\end{table*}


% In line with MIRAGE \cite{xiong-etal-2024-benchmarking}, we implement the following evaluation framework:
% \begin{itemize}
%     \item \textbf{Option-Free Retrieval:} As mentioned in § \ref{sec:Train-free}, to replicate real-world medical QA conditions, no answer options are provided as input during retrieval.
%     \item \textbf{Zero-Shot Learning:} Given that real-world medical questions often lack similar exemplars, our benchmark evaluates RAG systems in a zero-shot setting, without in-context few-shot learning.
%     \item \textbf{Metrics:} We use Accuracy—the proportion of questions correctly answered—as the main evaluation metric across all benchmarks. We extract model outputs through regular expression matching applied to complete generated answers \cite{wang-etal-2024-answer-c}.
% \end{itemize}

\subsubsection{Retriever and Corpus}
To ensure a fair comparison, we adopt the same retriever, corpus, and parameter settings as previous work \cite{xiong-etal-2024-benchmarking}. We use MedCPT \cite{jin2023medcpt}, a dense retriever specialized for the biomedical domain, configured to retrieve 32 chunks by default. For the corpus, we employ the Textbooks dataset \cite{jin-etal-2019-pubmedqa}, a lightweight collection of 125.8k chunks derived from medical textbooks, with an average length of 182 tokens.
% To ensure a fair comparison, we follow MIRAGE \cite{xiong-etal-2024-benchmarking} in terms of the retrievers, corpus, and parameter settings. We employ MedCPT \cite{jin2023medcpt}, a dense retriever tailored to the biomedical domain, configured to retrieve 32 chunks by default. For the corpus, we use the Textbooks dataset \cite{jin-etal-2019-pubmedqa}, a lightweight collection derived from medical textbooks, consisting of 125.8k chunks with an average length of 182 tokens.




\subsubsection{LLMs and Baselines}
We focus on the effect of RGAR on general-purpose LLMs without domain-specific knowledge. Therefore, we exclude LLMs fine-tuned on the medical domain, such as PMC-Llama \cite{wu2024pmc}. %We focus on models with fewer than 8 billion parameters. 
Our primary experiments utilize Llama-3.2-3B-Instruct, while ablation studies include a range of models from the Llama-3.1/3.2 \cite{dubey2024llama} and Qwen-2.5 \cite{yang2024qwen2} families, ranging from 1.5B to 8B parameters. All selected models feature a context length of approximately 128K tokens.
Temperatures are set to zero to ensure reproducibility through greedy decoding. % To mitigate repetitive generation in smaller models, we use a repetition penalty of 1.2 and limit the maximum generation length to 8K tokens.

For \textit{non-retrieval methods}, we consider a zero-shot approach Custom \cite{kojima2022large} as a baseline and evaluate improvements relative to it. To fully exploit the reasoning capabilities of the LLMs, we incorporate chain-of-thought (CoT) reasoning \cite{wei2022chain}.
For \textit{retrieval-based methods}, we evaluate the classic RAG model \cite{lewis2020retrieval}, the domain-adapted MedRAG \cite{xiong-etal-2024-benchmarking}, and $i$-MedRAG \cite{xiong2024improving}, a medical-domain RAG system designed to decompose questions and iteratively provide answers.

We adopt GAR \cite{mao-etal-2021-generation} as a representative \textit{query-optimized RAG method}, implemented train-free in accordance with § \ref{sec:Train-free}. RGAR defaults to \textbf{2} rounds of recurrence.


\subsubsection{Evaluation Settings}
Following MIRAGE \cite{xiong-etal-2024-benchmarking}, we adopt the following evaluation framework. In \textbf{Option-Free Retrieval}, no answer options are provided for retrieval (§\ref{sec:Train-free}), ensuring a more realistic medical QA scenario. In \textbf{Zero-Shot Learning}, RAG systems are evaluated without in-context few-shot learning, reflecting the lack of similar exemplars in real-world medical questions. For \textbf{Metrics}, we employ Accuracy, defined as the proportion of correctly answered questions, and we extract model outputs by applying regular expression matching to the entire generated responses \cite{wang-etal-2024-answer-c}.


% We adopt GAR \cite{mao-etal-2021-generation} as a representative \textit{query-optimized RAG approach}. Rather than the original strategy of training a BERT model to generate queries, we instead leverage prompt-based query generation with LLMs \cite{yu2023generate}. Our RGAR approach maintains this same strategy and defaults to two retrieval–generation cycles.
% 我文章读错了,确实是要微调BERT,但是它也是答案是生成的而非原始的Notably, the original GAR retrieves ground-truth answer options; in our comparison, these options are also generated via prompts.















\subsection{Main Results}
\subsubsection{Cross-Dataset Performance Improvement}
\label{cross-dataset}
\begin{figure*}[ht]
    \centering
    \begin{subfigure}{0.32\textwidth}
        \includegraphics[width=\textwidth]{line_1.pdf}
        \caption{Effect of Using Original Options.}
        \label{fig:sub1}
    \end{subfigure}
    \hfill
    \begin{subfigure}{0.32\textwidth}
        \includegraphics[width=\textwidth]{line_2.pdf}
        \caption{Effect of RGAR's Two Components.}
        \label{fig:sub2}
    \end{subfigure}
    \hfill
    \begin{subfigure}{0.32\textwidth}
        \includegraphics[width=\textwidth]{line_3.pdf}
        \caption{Effect of Rounds in RGAR.}
        \label{fig:sub3}
    \end{subfigure}
    \caption{Accuracy with Different Numbers of Retrieved Chunks on EHRNoteQA Dataset.}
    \label{fig:three_sub_figures}
\end{figure*}

We evaluate RGAR with the Llama-3.2-3B-Instruct across three factual-aware medical datasets, comparing it with several competitive baselines. Table~\ref{tab:mian_results} presents the results of all methods, along with their relative improvements over the Custom baseline. RGAR achieves the highest average performance across the three datasets, surpassing the second-best method, $i$-MedRAG, by 2\%. The retrieval-based methods, even the lowest-performing RAG, consistently outperform the non-retrieval methods Custom and CoT. This highlights the importance of retrieving specialized medical knowledge when using general-purpose LLMs to answer professional medical queries. Comparing different retrieval methods, GAR outperforms vanilla RAG by approximately 3\% on average, with a maximum improvement of 4.37\% across datasets. This indicates that generating multiple queries for retrieval provides consistent benefits. However, while performing well on EHRNoteQA, MedRAG demonstrates a negative effect on the other two datasets compared to vanilla RAG.

Notably, the improvements achieved by our RGAR over GAR exhibit a positive correlation with the average length of the dataset’s context. On EHRNoteQA, which has an average context length exceeding 3000 tokens, our approach achieved a 7.8\% improvement. This validates the advantage of our \textit{Factual knowledge Extraction} in enhancing retrieval effectiveness. Consequently, our method is particularly well-suited to real-world scenarios where complete electronic health records must be analyzed to provide medical advice. This indicates that our approach is promising for real-life applications in assisting physicians with clinical recommendations.

When analyzing performance across different datasets, we find that retrieval-based methods perform significantly better on MedQA-USMLE and EHRNoteQA, while MedMCQA showa a negative effect—consistent with results reported by MedRAG \cite{xiong-etal-2024-benchmarking}. A closer analysis reveals that MedMCQA incorporates arithmetic reasoning questions (roughly 7\% of the total), and the addition of extensive retrieved contexts diminishes the model’s numerical reasoning capabilities, which could potentially be fixed with larger base LLMs \cite{mirzadeh2025gsmsymbolic}. Nonetheless, among retrieval-based methods, our RGAR stands out as the only approach that outperforms vanilla RAG on this dataset, delivering an improvement of more than 1\% over Custom.
On EHRNoteQA, while RGAR’s performance is slightly below that of $i$-MedRAG, \textbf{the latter’s inference time is approximately 4 times longer, establishing RGAR as a more efficient and cost-effective alternative}.


\subsubsection{Base LLMs with Different Sizes and Model Families}
\begin{table}[htbp]
  \centering
  \caption{Comparison of LLMs on MedQA-USMLE.}
  \resizebox{\linewidth}{!}{ % 调整表格宽度适应页面
    \begin{tabular}{lcccc}
      \toprule
      Model & \multicolumn{1}{c}{Custom} & \multicolumn{1}{c}{RAG} & \multicolumn{1}{c}{GAR} & \multicolumn{1}{c}{RGAR} \\
      \midrule

      Llama-3.2-1B-Instruct & 38.96 & 29.30 & 30.79 & 29.85 \\
      Llama-3.2-3B-Instruct & 50.20 & 53.50 & 57.97 & 58.83 \\
      Llama-3.1-8B-Instruct & 60.80 & 62.14 & 67.39 & 69.52 \\
        \midrule
      Qwen2.5-1.5B-Instruct & 43.99 & 41.48 & 43.42 & 42.58 \\
      Qwen2.5-3B-Instruct & 48.23 & 49.96 & 53.50 & 54.28 \\
      Qwen2.5-7B-Instruct & 59.46 & 58.83 & 63.39 & 63.86 \\
      \midrule
      Average   & 50.27 & 49.20 & 52.74 & 53.15 \\
      \bottomrule
    \end{tabular}
  }
  \label{tab:performance}
\end{table}
To further assess the versatility of RGAR, we conduct evaluations on MedQA-USMLE, a widely used medical dataset, by utilizing base LLMs of various sizes and model families, specifically from Llama and Qwen. The results in Table \ref{tab:performance} show that RGAR consistently achieves the best average performance.

When considering model size, we find that retrieval-based approaches fall short of the non-retrieval Custom baseline for smaller models, such as Llama-3.2-1B-Instruct and Qwen2.5-1.5B-Instruct. These smaller models, constrained by their weaker performance, are not well-suited to leverage retrieval-enhanced information. As the model size increases, however, all retrieval-enhanced approaches exhibit notable performance gains, with RGAR yielding the most significant improvements. This trend becomes particularly pronounced for larger models. For example, RGAR achieves a 7.38\% improvement over RAG on Llama-8B, 5.33\% on Llama-3B, 5.03\% on Qwen-8B, and 4.32\% on Qwen-3B.

%While we did not test commercial closed-source models like GPT due to their high API costs
Moreover, we find that under the same experimental conditions, \textbf{Llama-3.1-8B-Instruct achieves a performance of 69.52\% with RGAR, surpassing the 66.22\% reported by MedRAG for GPT-3.5-16k-0613} \cite{achiam2023gpt}. This significant improvement underscores the practicality of using well-optimized retrieval methods with smaller models, enabling performance rivals those of proprietary large-scale foundational models in real-world medical recommendation tasks.

\subsection{Ablation Study}
% Due to the absence of ground-truth retrieval chunks across all three datasets, it is not feasible to evaluate retrieval performance using metrics such as nDCG@10 or Recall@100 like the BEIR benchmark \cite{thakur2021beir}. Instead, we assess retrieval effectiveness through QA performance, varying the number of retrieved items \(N\) from 4 to 32. A lower retrieval count more rigorously tests retrieval quality. We investigate three primary factors: the effect of options generated by GAR versus those originally provided by the dataset, the contributions of GAR and enhanced EHR components, and the impact of RGAR’s iterative rounds.

Due to the absence of ground-truth retrieval chunks, we evaluate retrieval effectiveness through QA performance, systematically varying the number of retrieved chunks \(N\) from 4 to 32. A reduced retrieval number serves as a more stringent assessment of retrieval quality. We investigate three primary factors in Figure \ref{fig:three_sub_figures}: the effect of options generated by GAR versus those originally provided by the dataset, the contributions of CKR and FKE components, and the impact of RGAR’s recurrence rounds.

We first compare the retrieval performance between LLM-generated options and original dataset options. Figure \ref{fig:sub1} shows how RGAR and GAR perform across different values of \(N\). Both approaches maintain stable performance across different \(N\), indicating reliable retrieval quality. While using original options shows slightly higher average Accuracy, the difference is minimal. This suggests that even when GAR generates options that differ from the originals, it achieves similar retrieval results as long as the core topics align. 

We then examine the impact of RGAR's two main components—CKR and FKE—as shown in Figure \ref{fig:sub2}. When we remove the conceptual knowledge interaction from the FKE phase, the system shows only moderate improvements when extracting factual knowledge from EHR without conceptual knowledge, demonstrating the importance of integrating both types of knowledge. %When we remove the multi-query generation step from CKR, performance decreases as \(N\) increases, indicating unstable retrieval quality. This highlights the necessity of generating multiple queries during the CKR phase to maintain stable retrieval.
% 这里也改了一下
Removing the multi-query generation step from CKR causes performance to degrade as \(N\) increases, indicating that multiple queries are necessary to maintain stable retrieval.

Finally, we analyze the effect of rounds in RGAR (Round 0 means GAR), as illustrated in Figure \ref{fig:sub3}. Our results show that even a single iteration significantly improves performance by enabling interaction between factual and conceptual knowledge. Multiple rounds work similarly to a reranking mechanism \cite{mao-etal-2021-reader}, improving the ranking of important chunks and showing substantial gains even with relatively small \(N\). With \(N = 8\) , the default two-round setup achieves a performance of 75.78\%, almost 1\% better than using a single round. However, adding more rounds shows no clear benefits, as they tend to generate multi-hop factual knowledge during the FKE phase, leading CKR to retrieve multi-hop conceptual knowledge, which may cause LLMs to over-infer ~\cite{yang-etal-2024-large-language-models}. Given that each round involves one reasoning step from both the LLM extractor and LLM query generator, two rounds sufficiently support multi-hop reasoning needs \cite{lv-etal-2021-multi}.

% We begin by comparing the retrieval performance of using LLM-generated options to that of directly using the original dataset-provided options. Figure \ref{fig:sub1} illustrates how the performance of RGAR and GAR changes with different values of \(N\). Both configurations exhibit relatively stable performance across the range of \(N\), indicating consistent retrieval quality. While the approach relying on original options shows slightly higher average accuracy (Acc) at various \(N\), the difference is negligible. Even when GAR generates options that differ in content from the originals, it achieves similar retrieval outcomes as long as the underlying topics are aligned. This suggests that GAR-generated options, despite their differences, remain conducive to effective retrieval.

% Next, we analyze the role of RGAR’s two main components in Figure \ref{fig:sub2}: GAR itself and the enhanced EHR retrieval process. We examine the impact of using original EHR data directly for retrieval instead of leveraging our LLM-generated approach. When relying solely on original EHR data, even with multiple iterations, the performance shows only modest improvements and remains capped. This is because such approaches can only enhance the relevance of concept-level information and key content, without encouraging the synthesis of new information or exploration beyond the original data. This limitation is especially pronounced for medical queries requiring multi-hop reasoning. When GAR is removed entirely, we observe a performance decline as \(N\) increases, highlighting the instability of retrieval quality. While the most relevant information may still be retrieved, the absence of auxiliary context hampers reasoning, and the introduction of irrelevant information as \(N\) grows leads to further performance degradation.

% Finally, we assess the effect of iterative rounds in RGAR in Figure \ref{fig:sub3}.The experimental results demonstrate that implementing even a single recurrence iteration yields significant performance improvements, as it facilitates the interaction between factual and conceptual knowledge domains. Multiple iterations function analogously to a reranking mechanism, elevating the relevance of truly pertinent chunks and achieving substantial enhancement when operating with relatively modest N. Notably, with N=8, the default two-iteration configuration achieved a performance of 75.78\%, representing nearly a 1\% improvement over the single-iteration baseline. However, excessive iteration rounds demonstrate no discernible advantages, as they tend to generate multi-hop factual knowledge during the FKE phase, potentially leading to over-inference by LLMs ~\cite{yang-etal-2024-large-language-models}. 

% Finally, we assess the effect of iterative retrieval rounds in RGAR. Setting the iteration count to zero effectively results in the GAR approach. Figure~X depicts how performance evolves as the number of retrieval rounds increases from zero to three. The results show that even a single round yields notable improvements, with subsequent rounds offering diminishing returns. We adopt two retrieval rounds, based on the assumption that most questions can be answered within three reasoning steps~\cite{lv-etal-2021-multi}, and excessive rounds provide no clear advantage~\cite{yang-etal-2024-large-language-models}. RGAR’s approach resembles bidirectional breadth-first search, simultaneously exploring from both the original context and possible answers. With two rounds, RGAR allows for up to four multi-hop steps, which proves to be sufficient.

\subsection{Fine-Grained Performance Analysis}

While the previous sections examined overall dataset performance and established preliminary findings, this section provides a detailed analysis of specific aspects of our results. In § \ref{cross-dataset}, we showed that RGAR performs better on real-world medical recommendation tasks involving comprehensive EHRs. To verify this finding, we conduct a detailed analysis of EHRNoteQA by grouping questions based on context length and dividing them into four bins. Within each bin, we compare the performance of RGAR, GAR, and Custom. As shown in Figure \ref{fig:bar}, Custom shows decreasing accuracy with increasing context length. GAR improves accuracy across all bins, with RGAR achieving further performance gains. Notably, the improvements are more significant in the three bins with longer contexts compared to the first bin. The results show that RGAR maintains consistent average performance across different context length.

% The previous sections focused on overall dataset performance and provided some preliminary conclusions. Here, we delve deeper into specific aspects of those findings. In earlier sections, we concluded that our approach is better suited to real-world medical recommendation tasks involving comprehensive EHRs. To validate this, we further analyze EHRNoteQA by sorting questions based on average context length and dividing them into four bins. Within each bin, we compare the performance of our approach and GAR against the Custom baseline. As shown in Figure \ref{fig:bar}, Acc declines with increasing question length for the Custom baseline. GAR improves Acc across all bins, and RGAR further enhances performance. Notably, the improvement is more pronounced in the three bins with longer contexts compared to the first bin. Overall, the average performance across all bins is similar.

\begin{figure}[htbp]
    \centering
    \includegraphics[width=1\linewidth]{bar.pdf}
    \caption{Fine-Grained Accuracy of EHRNoteQA After Sorting by Length and Dividing into Four Equal Parts.}
    \label{fig:bar}
\end{figure}

%We also revisit the findings, which suggest that GAR stabilizes retrieval. 
It is also important to note that generating multiple queries from different aspects within RGAR helps stabilize retrieval.
Figure \ref{fig:tsne} presents a t-SNE visualization of different queries and their individually retrieved chunks for a sample question (details provided in Appendix~\ref{case}). The basic query shows limited suitability for retrieval, as its coverage area differs from that of the three queries generated by RGAR. RGAR clearly introduces some variation in retrieval content. Although the regions corresponding to the three generated queries overlap, the specific chunks retrieved do not overlap significantly. This underscores the need to average the retrieval similarities of these three queries to achieve more stable retrieval results.

\begin{figure}[htbp]
    \centering
    \includegraphics[width=1\linewidth]{t-sne.pdf}
    \caption{t-SNE Visualization of Different Queries and the Retrieved Chunks.}
    \label{fig:tsne}
\end{figure}

% We also revisit the conclusion from our ablation study, which suggested that GAR stabilizes retrieval. Figure \ref{fig:tsne} presents a t-SNE visualization of different queries and their individually retrieved chunks for a sample question (details provided in Appendix~\ref{case}). The basic query shows limited suitability for retrieval, as its coverage area differs from that of the three queries generated by GAR. GAR clearly introduces some variation in retrieval content. Although the regions corresponding to the three GAR-generated queries overlap, the specific chunks retrieved do not overlap significantly. This underscores the need to average the retrieval similarities of these three queries to achieve more stable retrieval results.

\section{Conclusion}
In this work, we propose RGAR, a novel RAG system that distinguishes two types of retrievable knowledge. Through comprehensive evaluation across three factual-aware medical benchmarks, RGAR demonstrates substantial improvements over existing methods, emphasizing the significant impact of in-depth factual knowledge extraction and its interaction with conceptual knowledge on enhancing retrieval performance. %Notably, our system enables 8B parameter models to outperform proprietary large-scale commercial models with retrieval capabilities.
Notably, our RGAR enables the Llama-3.1-8B-Instruct model to outperform the considerably larger, RAG-enhanced proprietary GPT-3.5.
%From a broader perspective, RGAR represents a promising approach for enhancing general LLMs in real-world clinical diagnostic scenarios that demand extensive factual knowledge processing. This framework shows potential for extension to other professional domains where factual awareness is crucial, offering a viable solution for specialized applications requiring precise factual knowledge management.
From a broader perspective, RGAR offers a promising approach for enhancing general-purpose LLMs in clinical diagnostic scenarios where extensive factual knowledge is crucial, with potential for extension to other professional domains demanding precise factual awareness. 
\newpage
\section*{Limitations}
%Despite RGAR achieving superior average performance, several limitations warrant discussion. Our RGAR necessitates corpus retrieval, with time complexity scaling proportionally with corpus size, this is a problem inherent in the RAG paradigm. Approaches that generate reasoning evidence directly through domain-specified LLMs \cite{yu2023generate, frisoni-etal-2024-generate} avoid the inference-time computational issue, however, they are illed in updating LLMs to follow new medical knowledge, which induces frequency updation and training costs. %Additionally, while the multiple LLM generations required by RGAR's retrieval process showed negligible additional time overhead on Llama3.2-3B in our primary experiments, this overhead becomes significant when scaling to larger models.
Despite RGAR achieving superior average performance, several limitations warrant discussion. Our RGAR requires corpus retrieval, and its time complexity scales proportionally with the size of the corpus, which is an inherent issue within the RAG paradigm. Approaches that generate reasoning evidence directly through domain-specific LLMs \cite{yu2023generate, frisoni-etal-2024-generate} avoid the computational challenges at inference time. However, they face difficulties in updating LLMs to incorporate new medical knowledge, which results in frequent updates and training costs.

Comparative approaches such as MedRAG \cite{xiong-etal-2024-benchmarking} and $i$-MedRAG \cite{xiong2024improving} explore integration possibilities with prompting techniques like Chain-of-Thought \cite{wei2022chain} and Self-Consistency \cite{wang2023selfconsistency} to enhance reasoning capabilities. Our investigation focused specifically on validating how additional factual knowledge processing improves retrieval performance, without examining the impact of these prompting strategies. %Furthermore, unlike multi-round methods such as %Adaptive RAG \cite{jeong-etal-2024-adaptive} and 
%$i$-MedRAG \cite{xiong2024improving} that implement LLM-based early stopping to reduce computational costs, our system operates with fixed time complexity.
%However, it is noteworthy that由于i-medrag每轮都要分解若干query检索并回答再汇总,RGAR的实际时间开销远小于i-medrag。
Furthermore, unlike multi-round methods such as $i$-MedRAG \cite{xiong2024improving} that implement LLM-based early stopping to reduce computational costs, our system operates with fixed time complexity. However, it is noteworthy that, because $i$-MedRAG requires multiple rounds of query decomposition, retrieval, and answer aggregation, the actual time overhead of RGAR is significantly smaller than that of $i$-MedRAG.


Our EHR extraction approach assumes LLMs can process complete EHR contextual input, justified by current mainstream LLMs exceeding 128K context windows with anticipated growth. However, in extreme cases where EHR content exceeds LLM context limits, integration with chunk-free approaches may be necessary \cite{luo-etal-2024-landmark, qian-etal-2024-grounding}. Finally, as RGAR operates in a zero-shot setting without instruction fine-tuning, its effectiveness is partially contingent on the model's instruction-following capabilities—which we cannot fully mitigate.

\section*{Ethical Statement}
This research adheres to the ACL Code of Ethics. All medical datasets utilized in this study are either open access or obtained through credentialed access protocols. To ensure patient privacy protection, all datasets have undergone comprehensive anonymization procedures.
While Large Language Models (LLMs) present considerable societal benefits, particularly in healthcare applications, they also introduce potential risks that warrant careful consideration. Although our work advances the relevance of retrieved content for medical queries, we acknowledge that LLM-generated responses based on retrieved information may still be susceptible to errors or perpetuate existing biases.
Given the critical nature of medical information and its potential impact on healthcare decisions, we strongly advocate for a conservative implementation approach. Specifically, we recommend that all system outputs undergo rigorous validation by qualified medical professionals before any practical application. This stringent verification process is essential to maintain the integrity of clinical and scientific discourse and prevent the propagation of inaccurate or potentially harmful information in healthcare settings.
These ethical safeguards reflect our commitment to responsible AI development in the medical domain, where the stakes of misinformation are particularly high and the need for reliability is paramount.

\bibliography{custom}

\appendix

\section{Implementation Details}
\subsection{Hardware Configuration}
All experiments were conducted on an in-house workstation equipped with \textit{dual} NVIDIA GeForce RTX 4090 GPUs, % (24GB VRAM \textit{each}),
128GB RAM, and an Intel® Core i9-13900K CPU.

Time cost across all methods on EHRNoteQA are shown in Table \ref{tab:method_time_comparison}.

\begin{table}[htbp]
  \centering
  \caption{Comparison of different methods in terms of execution time (hours).}
  \resizebox{\linewidth}{!}{
  \begin{tabular}{lccccccc}
    \toprule
    Method & Custom & CoT & RAG & MedRAG & GAR & $i$-MedRAG & RGAR \\
    \midrule
    Time (h) & 0.5 & 0.5 & 1 & 1 & 2 & 22 & 6 \\
    \bottomrule
  \end{tabular}
  }
  \label{tab:method_time_comparison}
\end{table}

\subsection{Code and Results}
The core implementation of the RGAR framework and the output json files can be accessed via the \textbf{Anonymous Repository}: \url{https://anonymous.4open.science/r/RGAR-C613}


\section{Prompt Template and Case Study}
\label{case}
For simplicity, we merged EHR and question in the prompt words of the answer and treated them as question in the prompt words.
Table \ref{tab:prompts} shows the prompts template of RGAR and compared work (Using CoT ones). Table \ref{tab:input} shows the input of a sample, Table \ref{tab:output} shows the final output of RGAR.

\begin{table*}[h]
\centering
\begin{tabular}{p{0.3\textwidth}|p{0.7\textwidth}}
\toprule
Type & Prompt Template \\
\midrule
System prompts for Non-CoT & You are a helpful medical expert, and your task is to answer a multi-choice medical question using the relevant documents. Organize your output in a json formatted as Dict \{"answer\_choice": Str\{A/B/C/...\}\}. Your responses will be used for research purposes only, so please have a definite answer. Please just give me the json of the answer. \\
\midrule
System prompts for using CoT  & You are a helpful medical expert, and your task is to answer a multi-choice medical question. Please first think step-by-step and then choose the answer from the provided options. Organize your output in a json formatted as Dict\{"step\_by\_step\_thinking": Str(explanation), "answer\_choice": Str\{A/B/C/...\}\}. Your responses will be used for research purposes only, so please have a definite answer. Please just give me the json of the answer. \\
\midrule
Answer prompts for Non-CoT &Here are the relevant documents:
\{\{context\}\}
\newline
Here is the question:
\{\{question\}\}
\newline
Here are the potential choices:
\{\{options\}\}
\newline
Please just give me the json of the answer. Generate your output in json:\\

\midrule
Answer prompts for Using CoT &Here are the relevant documents:
\{\{context\}\}
\newline
Here is the question:
\{\{question\}\}
\newline
Here are the potential choices:
\{\{options\}\}
\newline
Please think step-by-step and generate your output in one json:\\
\midrule
Extracting EHR prompts & Here are the relevant knowledge sources:
\{\{context\}\}
\newline
Here are the electronic health records:
\{\{ehr\}\}
\newline
Here is the question:
\{\{question\}\}
\newline
Please analyze and extract the key factual information in the electronic health records relevant to solving this question and present it as a Python list. 
Use concise descriptions for each item, formatted as ["key detail 1", ..., "key detail N"]. Please only give me the list. Here is the list: \\
\midrule
Generating Possible Answer prompts & Please give 4 options for the question. Each option should be a concise description of a key detail, formatted as: A. "key detail 1" B. "key detail 2" C. "key detail 3" D. "key detail 4\\
\midrule
Generating Possible Title prompts & Please generate some titles of references that might address the above question. Please give me only the titles, formatted as: ["title 1", "title 2", ..., "title N"]. Please be careful not to give specific content and analysis, just the title.\\
\midrule
Generating Possible Contexts prompts & Please generate some knowledge that might address the above question. please give me only the knowledge. \\
\bottomrule
\end{tabular}
\caption{Prompt templates used in RGAR and Compared Methods.}
\label{tab:prompts}
\end{table*}

\begin{table*}[h]
\centering
\begin{tabular}{p{0.3\textwidth}|p{0.7\textwidth}}
\toprule
Type & Texts \\
\midrule
EHR & A 39-year-old woman is brought to the emergency department because of fevers, chills, and left lower quadrant pain. Her temperature is 39.1°C (102.3°F), pulse is 126/min, respirations are 28/min, and blood pressure is 80/50 mm Hg. There is blood oozing around the site of a peripheral intravenous line. Pelvic examination shows mucopurulent discharge from the cervical os and left adnexal tenderness. Laboratory studies show: Platelet count 14,200/mm3 Fibrinogen 83 mg/mL (N = 200–430 mg/dL) D-dimer 965 ng/mL (N < 500 ng/mL)\\
\midrule
Questions & When phenol is applied to a sample of the patient's blood at 90°C, a phosphorylated N-acetylglucosamine dimer with 6 fatty acids attached to a polysaccharide side chain is identified. A blood culture is most likely to show which of the following? \\
\midrule
Options & A. Coagulase-positive, gram-positive cocci forming mauve-colored colonies on methicillin-containing agar \newline
B. Encapsulated, gram-negative coccobacilli forming grey-colored colonies on charcoal blood agar \newline
C. Spore-forming, gram-positive bacilli forming yellow colonies on casein agar \newline
\textbf{D. Lactose-fermenting, gram-negative rods forming pink colonies on MacConkey agar} \\
\bottomrule
\end{tabular}
\caption{Case Study of a Sample Input.}
\label{tab:input}
\end{table*}

\begin{table*}[h]
\centering
\begin{tabular}{p{0.3\textwidth}|p{0.7\textwidth}}
\toprule
Type & Texts \\
\midrule
Generated Title & "Identification of Streptococcus pneumoniae in Clinical Specimens",\newline "Phosphorylcholine-Binding Protein A of Staphylococcus aureus",\newline "D-Dimer Assay for Diagnosing Deep Vein Thrombosis",\newline "The Role of Fibrin Degradation Products in Hemostasis",\newline "Infection Control Measures for Hospital-Acquired Bloodstream Infections" \\
\midrule
Generate Context & The description suggests Neisseria gonorrhoeae infection. Neisseria gonorrhoeae produces lipopolysaccharides in its outer membrane, specifically LPS or endotoxin. When heated, these lipopolysaccharides undergo cross-linking reactions resulting in the formation of phosphorylcholine-attached oligomers such as those described in the question stem. These structures can be detected by ELISA using antibodies against specific epitopes on the surface of N. gonorrhoeae cells. The presence of this structure would suggest an active infection due to N. gonorrhoeae. Blood cultures for N. gonorrhoeae typically require specialized media containing antibiotics like vancomycin, colistin, and nystatin to inhibit competing bacteria \\
\midrule
Generate Answer & Option 1: Streptococcus pneumoniae The presence of a high fever, tachycardia, hypotension, and low platelets suggests sepsis or severe infection. The laboratory findings also indicate disseminated intravascular coagulation (DIC). The identification of a phosphorylated N-acetylglucosane dimer with 6 fatty acid chains attached to a polysaccharide side chain after applying phenol at 90°C is consistent with pneumococcal capsular polysacarcharides.\newline
Option 2: Staphylococcus aureus Staphylococci can cause various infections but do not typically produce this type of capsule structure when exposed to heat.\newline
Option 3: Escherichia coli E. coli does have capsules that may be detected by certain methods such as latex agglutination tests; however, they don't match the description given in the prompt.\newline
Option 4: Neisseria meningitidis Neisseria species like N. meningitidis would more commonly present with symptoms related to central nervous system involvement rather than pelvic signs described in the scenario.\\
\midrule
Extract EHR & ['Patient age: 39 years', 'Symptoms: fever, chills, left lower quadrant pain', 'Vital signs: \textbf{high} temperature (39.1°C), tachycardia (126/min), tachypnea (28/min), hypotension (80/50 mmHg)', 'Physical exam findings: mucopurulent discharge from cervix, left adnexal tenderness', 'Laboratory results: \textbf{low} platelet count (14,200/mm\^{}3), \textbf{elevated} D-dimer (965 ng/mL)', 'Phenol test result: phosphorylated N-acetylglucosame dimmer with 6 fatty acids attached to a polysaccharide side chain'] \\
\midrule
Output Json & \{"answer\_choice": "D"\}\\
\bottomrule
\end{tabular}
\caption{Case Study of output in RGAR.}
\label{tab:output}
\end{table*}

\section{Framework Insight}
\subsection{Another View of the Recurrence Pipeline}
We conceptualize the Recurrence Pipeline as an exploration-exploitation process within the reinforcement learning framework \cite{10.1023/A:1013689704352}. In GAR, even when generated content is only partially accurate (or potentially inaccurate), it remains valuable for retrieval if it correlates with passages containing correct information (e.g., co-occurrence with correct answers), thus representing an exploratory phase. Conversely, EHR extraction serves as an exploitation phase, thoroughly utilizing explored knowledge by selecting relevant components and synthesizing new evidence (factual knowledge). Based on this newly derived evidence, subsequent iterations can initiate fresh exploration-exploitation cycles, creating a continuous knowledge transmission process \cite{10446501}.

In scenarios where additional factual knowledge is not required, the retrieved content tends to remain relatively constant, and utilizing this content under identical prompting conditions would likely yield similar factual knowledge through extraction and summarization. However, when conceptual knowledge is needed to derive new factual knowledge through reasoning from existing factual information, the updated basic query facilitates easier retrieval of conceptual knowledge supporting current reasoned factual knowledge, thereby maintaining the integrity of reasoning chains. Furthermore, leveraging current factual knowledge for retrieval enables the exploration and discovery of novel knowledge domains.

\subsection{Why No Flexible Stopping Criteria}
Similar multiround RAG systems have adopted more flexible stopping criteria. For instance, Adaptive RAG \cite{jeong-etal-2024-adaptive} determines whether to retrieve further % or how many rounds of retrieval are needed
by consulting the model itself. $i$-MedRAG \cite{xiong2024improving}, while setting a maximum number of retrieval iterations, also supports early stopping.

In our RGAR framework, we do not adopt such settings. On the one hand, we focus on evaluating how additional processing of \textit{factual knowledge} enhances retrieval performance, raising awareness of this often-overlooked type of knowledge in previous RAG systems, while flexible stopping criteria mainly showcase procedural knowledge and metacognitive knowledge. On the other hand, the metacognitive capabilities of current LLMs remain under question, as a model’s self-evaluation of the need for additional retrieval information often does not match actual requirements \cite{kumar-etal-2024-confidence}.

\subsection{Future Work}
% % 我们的RGAR framework利用检索到的medical domain专业知识,在medical OpenQA任务重提供了卓越的回答质量。但是,我们担忧这种强大的生成能力一旦被恶意利用,也可能带来安全隐患。比如,当被检索的corpus包含私隐私信息或版权内容时,恶意的提问者可能利用LLM的回答提取并泄露corpus中的敏感信息 \cite{carlini2021extracting}。此外,恶意的提问者可能通过收集大量提问-问答对来尝试replicate我们的base LLM \cite{tramer2016stealing, zhu2024efficient}, 或推断我们检索生成框架的内部信息 \cite{carlinistealing} 作为泄露的商业秘密或未来攻击的基石。我们将在未来尽最大努力阻止这些恶意攻击,比如检查query是否合法 \cite{inan2023llama} 和通过水印标识RGAR所使用的模型 \cite{zhu2024reliable}, 从而保证RAGR被合理、合法地使用。
Our RGAR framework leverages retrieved medical domain knowledge to deliver exceptional answer quality% in medical OpenQA tasks
. However, we are concerned that such powerful generative capabilities, if maliciously exploited, could pose security risks. For instance, when the retrieved corpus contains private or copyrighted information, malicious users could exploit the LLM's responses to extract and disclose sensitive data from the corpus \cite{carlini2021extracting}. 
% Additionally, malicious users might attempt to replicate our base LLM \cite{tramer2016stealing, zhu2024efficient} by collecting large volumes of question-answer pairs or infer internal details of our retrieval-based generation framework \cite{carlinistealing}. 
Additionally, malicious users might attempt to replicate our base LLM \cite{tramer2016stealing, zhu2024efficient} by collecting large volumes of question-answer pairs or infer internal details of our retrieval-based generation framework \cite{carlinistealing}. 
%, potentially exposing proprietary information or providing a foundation for future attacks. 
% %To mitigate these risks, 
We will make every effort to mitigate these risks, such as verifying the legitimacy of queries \cite{inan2023llama} and watermarking the models used in RGAR \cite{zhu2024reliable}.
, ensuring that RGAR is used responsibly and legally.

\section{Comparative Analysis of Dataset Length Distributions}
In this section, we present additional visualizations comparing the two categories of datasets we described, and explain our rationale for excluding the MMLU-med dataset \cite{hendrycks2021measuring}. We plotted smoothed Kernel Density Estimation (KDE) curves for these datasets, as shown in Figure \ref{fig:kde}. Our analysis confirms that datasets containing Electronic Health Records (EHR) consistently demonstrate greater length compared to those without EHR content. However, certain datasets exhibit complex question sources and types. For instance, while the MMLU dataset shows a considerable mean length of 84 tokens and a maximum length of 961 tokens, as illustrated in the figure, the vast majority of its questions lack EHR content and are predominantly shorter in length. This characteristic led to our decision to exclude it from our experimental evaluation.
\ref{fig:kde}
\begin{figure*}[htbp]
        \centering
    \includegraphics[width=1\linewidth]{KDE.pdf}
    \caption{Length Distribution Analysis of Medical QA Datasets with and without EHR.}
    \label{fig:kde}
\end{figure*}
\end{document}






\title{
Generative Psycho-Lexical Approach for Constructing Value Systems in Large Language Models
}




\author{%
  Haoran Ye\thanks{Equal contribution.} \textsuperscript{ 1}, Tianze Zhang\footnotemark[1] \textsuperscript{ 1 2},
  Yuhang Xie\footnotemark[1] \textsuperscript{ 1}, \\
  \textbf{Liyuan Zhang\textsuperscript{1},
  Yuanyi Ren\textsuperscript{1},  
  Xin Zhang\textsuperscript{3 4}, Guojie Song\thanks{Corresponding author.} \textsuperscript{ 1 5}}
  \\[0.5em]
\textsuperscript{1}State Key Laboratory of General Artificial Intelligence,\\School of Intelligence Science and Technology, Peking University\\
\textsuperscript{2}Yuanpei College, Peking University\\
% \textsuperscript{3}School of Software and
% Microelectronics, Peking University\\
\textsuperscript{3}School of Psychological and Cognitive Sciences, Peking University\\
\textsuperscript{4}Key Laboratory of Machine Perception (Ministry of Education), Peking University\\
\textsuperscript{5}PKU-Wuhan Institute for Artificial Intelligence\\[0.5em]
\small \texttt{\{hrye, ericzhang, yuhangxie, zly2003\}@stu.pku.edu.cn} \\
\small \texttt{
\{yyren, zhang.x, gjsong\}@pku.edu.cn} 
}
\begin{document}


\maketitle


End-to-end imitation learning offers a promising approach for training robot policies. However, generalizing to new settings—such as unseen scenes, tasks, and object instances—remains a significant challenge. Although large-scale robot demonstration datasets have shown potential for inducing generalization, they are resource-intensive to scale. In contrast, human video data is abundant and diverse, presenting an attractive alternative. Yet, these human-video datasets lack action labels, complicating their use in imitation learning. Existing methods attempt to extract grounded action representations (e.g., hand poses), but resulting policies struggle to bridge the embodiment gap between human and robot actions.
% our approach
We propose an alternative approach: leveraging language-based reasoning from human videos - essential for guiding robot actions - to train generalizable robot policies. Building on recent advances in reasoning-based policy architectures, we introduce Reasoning through Action-free Data (RAD). RAD learns from both robot demonstration data (with reasoning and action labels) and action-free human video data (with only reasoning labels). The robot data teaches the model to map reasoning to low-level actions, while the action-free data enhances reasoning capabilities. Additionally, we will release a new dataset of 3,377 human-hand demonstrations compatible with the Bridge V2 benchmark. This dataset includes chain-of-thought reasoning annotations and hand-tracking data to help facilitate future work on reasoning-driven robot learning.
% experiments
Our experiments demonstrate that RAD enables effective transfer across the embodiment gap, allowing robots to perform tasks seen only in action-free data. Furthermore, scaling up action-free reasoning data significantly improves policy performance and generalization to novel tasks. These results highlight the promise of reasoning-driven learning from action-free datasets for advancing generalizable robot control. 
% releasing dataset
Website: \href{https://rad-generalization.github.io}{here}.

\begin{figure}[ht]
    \centering
    \includegraphics[width=0.8\linewidth]{graphs/greater_than_naive.pdf}
    \vspace{0.5cm}
    \includegraphics[width=0.8\linewidth]{graphs/p1_bottom.png}
    \vspace{-5pt}
    \caption{\textcolor{positional}{Positional} vs.\ \textcolor{nonpositional}{non-positional} circuits. In a \textcolor{nonpositional}{non-positional} circuit, the same edges must be included at all positions. A \textcolor{positional}{positional} circuit can distinguish between the same edge at different positions. This specificity yields better trade-offs between circuit size and faithfulness. It can also increase both precision and recall.}
    \label{fig:p1}
    \vspace{-5pt}
\end{figure}

\section{Introduction}

\looseness=-1
A primary goal of interpretability research is to characterize the internal mechanisms in language models (LMs) and other NLP models. 
A core approach in this area is \textbf{circuit discovery}---identifying the minimal subgraph within the model's computation graph that performs a specific task \citep{olah2021framework,olah-mech}.
Typically, the nodes of a circuit represent model components (e.g., attention heads, neurons, or layers).
While manual circuit discovery methods can yield position-specific insights \citep{wanginterpretability,goldowskydill2023localizingmodelbehaviorpath}, \emph{automatic methods often overlook positional information}, treating components as uniformly relevant across all input token positions \citep{conmytowards,syed2023attribution}. 
For instance, if an attention head is included in a circuit, it is assumed to contribute equally to the computation for every position in the input sequence.
The assumption that circuits are position-invariant ignores the fact that different positions often require distinct computations.
By ignoring positions, current methods limit their ability to capture mechanisms that operate across positions, such as interactions between attention heads across positions.

In this study, we start by demonstrating that positional agnosticism is a significant limitation (\S\ref{sec:motivating}). Then, to address these limitations, we introduce a new approach: position-aware edge attribution patching (PEAP; \S\ref{sec:full_circ_discovery}; Figure~\ref{fig:p1}). Current approaches  assume that if an edge is in a circuit, then the same edge will be in the circuit at all positions, thus leading to low precision. It is also assumed that an edge's importance should be aggregated across positions before deciding whether it should be included in the circuit; this can lead to cancellation effects, and thus low recall. PEAP instead allows us to compute the importance of cross-positional edges, and separately evaluates edge importance at each position. We show that this leads to smaller and more accurate circuits; see Figure~\ref{fig:p1}.

Incorporating positional information into circuit discovery is straightforward when inputs have the same length and structure across examples.

However, realistic datasets are not nearly this templatic.
How, then, can we incorporate positional information into automatic circuit discovery?
To address this challenge, we propose \textbf{schemas} (\S\ref{sec:schema}). 
Schemas assign semantic labels to spans of tokens, enabling information aggregation across examples even when the spans differ in length.

For example, in the input ``The \textcolor{positional}{war} lasted from 1453 to 14\underline{\hspace{1em}},'' the span ``\textcolor{positional}{war}'' could be labeled as ``\emph{Subject}''.
This enables handling spans with varying lengths: the phrase ``\textcolor{positional}{Black Plague}'' in another example can be treated as a single positional span with the same role as ``\textcolor{positional}{war}''.
In experiments with two LMs and three tasks, we find that circuits discovered using schemas achieve a better trade-off between circuit size and faithfulness to the model's behavior than position-agnostic circuits.
Importantly, position-aware circuits offer a more precise representation of the underlying mechanisms, providing a more concise foundation for mechanistic explanations.

We also present a fully automated pipeline for schema generation and application (\S\ref{sec:schema-generation}) using large language models (LLMs). 
We evaluate the quality of the generated schemas and their utility in discovering position-aware circuits (\S\ref{sec:schema-eval}).
Notably, circuits derived using automatically generated and applied schemas achieve comparable faithfulness scores to circuits discovered with human-designed and manually applied schemas.

We summarize our contributions as follows:
\begin{itemize}[noitemsep,leftmargin=*,topsep=1pt,parsep=1pt]
    \item Introduce a position-aware circuit discovery method, which obtains better faithfulness than position-agnostic discovery.  
    \item Introduce dataset schemas,  facilitating positional circuit discovery in more naturalistic settings. 
    \item Develop an automated schema generation and application pipeline with LLMs, yielding schemas that are comparable to manually-annotated ones.
\end{itemize}


\section{Related work}


Recent advances in single-image animatable head avatar generation can be categorized into mainly 2D-based and 3D-based approaches. 

\paragraph{\bf Image to 2D Animatable Avatar.}
2D-based methods, leveraging the power of convolutional neural networks (CNNs)~\cite{DBLP:conf/cvpr/KarrasLAHLA20,DBLP:conf/cvpr/IsolaZZE17,DBLP:conf/nips/GoodfellowPMXWOCB14}, often employ generative adversarial networks (GANs)~\cite{DBLP:conf/cvpr/StyleGAN} for direct image synthesis. Early approaches~\cite{DBLP:conf/cvpr/WangDYSW23,DBLP:conf/cvpr/BurkovPGL20,DBLP:conf/iccv/ZakharovSBL19} focus on injecting expression and pose features into the generator network, often utilizing architectures like U-Net or StyleGAN~\cite{DBLP:conf/cvpr/StyleGAN}.
Some other 2D methods~\cite{DBLP:journals/corr/abs-2407-03168,DBLP:conf/cvpr/ZhangQZZW0CW023,DBLP:conf/cvpr/HongZS022,DBLP:conf/mm/DrobyshevCKILZ22,DBLP:conf/cvpr/BurkovPGL20,DBLP:conf/nips/SiarohinLT0S19} represent expressions and poses as warping fields applied to the source image. 
Benefiting from advances in image and video diffusion networks, more recent 2D-based works~\cite{DBLP:journals/corr/abs-2410-07718,DBLP:journals/corr/abs-2406-08801,DBLP:conf/eccv/TianWZB24} get improved results with diffusion techniques. 
However, these methods still face challenges related to long generation times and significant computational resource demands. Audio-driven 2D control methods~\cite{DBLP:conf/cvpr/ZhangCWZSGSW23,DBLP:journals/corr/abs-2211-12368,DBLP:conf/iccv/GuoCLLBZ21} are easy to use but cannot explicitly control facial expressions and poses. 2D-based techniques often struggle with large pose or expression variations due to the lack of an explicit 3D structure, sometimes producing unrealistic distortions or identity changes. While some 2D methods~\cite{SadTalker,StyleHEAT,Pirenderer,DBLP:conf/cvpr/WangM021,MegaPortraits} incorporate 3D Morphable Models (3DMMs)~\cite{DBLP:conf/fgr/GerigMBELSV18,DBLP:journals/tog/LiBBL017,DBLP:conf/avss/PaysanKARV09,DBLP:conf/siggraph/BlanzV99} to mitigate these issues, they typically cannot achieve free-viewpoint rendering. 

\vspace{-0.1in}

\begin{figure*}[h]
    \centering
    \includegraphics[width=0.9\linewidth]{images/framework.pdf}
    \caption{\textbf{Overall Framework.} Our framework utilizes learnable query features attached to FLAME vertices to perform cross-attention with the extracted multi-level image features. The extracted features are then decoded to reconstruct the Gaussian avatar in the canonical space, which can be animated utilizing standard linear blend skinning (LBS) and corrective blendshapes as the FLAME model did and rendered in real-time on various platforms.}
    \label{fig:framework}
\end{figure*}

\paragraph{\bf Image to 3D Animatable Avatar.}
3D-aware methods offer improved geometric consistency and free-viewpoint rendering capabilities. Early 3D approaches~\cite{DBLP:conf/eccv/KhakhulinSLZ22,DBLP:conf/cvpr/XuYCWDJT20} utilize 3DMMs for head avatar reconstruction. With the advent of Neural Radiance Fields (NeRFs)~\cite{DBLP:conf/eccv/MildenhallSTBRN20}, many recent methods~\cite{DBLP:conf/siggraph/YuFZWYBCSWSW23,DBLP:conf/cvpr/MaZQLZ23,DBLP:conf/cvpr/LiZWZ0CZWB023,GPAvatar,ye2024real3d,deng2024portrait4d,deng2024portrait4d2,DBLP:conf/eccv/KiMC24,DBLP:conf/cvpr/BaiFWZSYS23,PointAvatar,Nerfies,INSTA} have adopted this representation for higher fidelity, particularly in modeling fine details like hair. However, NeRF-based~\cite{DBLP:conf/cvpr/ZhangZLHLWGCL024,HAvatar,DBLP:conf/cvpr/BaiTHSTQMDDOPTB23,AD-NeRF,DBLP:journals/tog/GaoZXHGZ22,DBLP:journals/tog/ParkSHBBGMS21,DBLP:conf/cvpr/AtharXSSS22,DBLP:journals/corr/abs-2112-05637,DBLP:conf/iccv/TretschkTGZLT21,DBLP:conf/cvpr/GafniTZN21,DBLP:conf/eccv/KiMC24,DBLP:conf/cvpr/BaiFWZSYS23,PointAvatar,Nerfies,DBLP:conf/siggraph/YuFZWYBCSWSW23,DBLP:conf/cvpr/MaZQLZ23,DBLP:conf/cvpr/LiZWZ0CZWB023} approaches often require extensive training data, including multi-view or single-view videos, raising privacy concerns and limiting generalization to unseen identities. Some methods~\cite{DBLP:conf/cvpr/SunWWLZZL23,DBLP:conf/3dim/ZhuangMKS22,DBLP:journals/pami/SunWZHWL24,DBLP:journals/tvcg/TangZYZCMW24,DBLP:conf/iclr/XuZLZBFS23} bypass this data requirement by training generators with random noise and then inverting them for identity-specific reconstruction, but inversion accuracy remains a challenge. Test-time optimization offers another alternative, but its computational cost limits practical applications. Several recent works~\cite{goha2023,hidenerf2023,gpavatar2024,ye2024real3d,ma2024cvthead,deng2024portrait4d,deng2024portrait4d2,GGHead} have explored one-shot 3D head reconstruction to address the limitations of data requirements and computational cost. These methods employ various techniques, such as tri-plane features, deformation fields, point-based expression fields, and vertex-feature transformers. Despite these advancements, NeRF-based methods often struggle with real-time rendering. 
Recently, 3D Gaussian Splatting~\cite{GaussianSplatting} has emerged as a promising alternative, offering both high-quality results and fast rendering speeds. However, existing Gaussian Splatting methods~\cite{GaussianAvatar,DBLP:conf/cvpr/XuCL00ZL24} typically rely on video data for training for each person, limiting their ability to generalize to new identities. Instead, the most recent work, GAGAvatar~\cite{GAGAvatar}, proposes a one-shot 3D Gaussian-based head avatar generation method. However, it still relies heavily on complex 2D neural post-processing to achieve optimal animation outcomes, thus it is not a pure 3D solution and the extra neural network hinders its application on various platforms. In contrast, our work generates Gaussian heads that are immediately animatable and renderable without additional networks or post-processing steps, enabling seamless integration into existing rendering pipelines for real-time animation and rendering across a wide range of platforms, including mobile phones. 
\section{Approach}\label{sec:approach} 

In this section, we first formally define the OOD code detection problem for (NL, PL) models (Sec. IV-A), then introduce the overall proposed framework (Sec. IV-B), and finally present details of unsupervised COOD and weakly-supervised COOD+ in Sec. IV-C and Sec. IV-D, respectively.
\subsection{Problem Statement}

Since current state-of-the-art code-related models~\cite{guo2020graphcodebert, guo2022unixcoder} typically extract code semantics by capturing the semantic connection between NL (\ie comment) and PL (\ie code) modalities, we formally defined OOD samples involving these two modalities in the SE context by following the convention in ML~\cite{yang2021generalized, zhou2021contrastive}. Consider a dataset comprising training samples $((t_1, c_1), y_1), ((t_2, c_2), y_2), ...$ from the joint distribution $P((T, C),Y)$ over the space $\mathcal{(T, C)}\times \mathcal{Y}$, and a neural-based code model is trained to learn this distribution. Here, $((t_1, c_1), y_1)$ represents the first input pair of (comment, code) along with its ground-truth prediction in the training corpus. $T$, $C$ and $Y$ are random variables on an input (comment, code) space $\mathcal{(T, C)}$ and a output (semantic) space $\mathcal{Y}$, respectively. OOD code samples refer to instances that typically deviate from the overall training distribution due to distribution shifts. The concept of distribution shift is very \textit{broad}~\cite{yang2021generalized, salehiunified} and can occur in either the marginal distribution $P(T, C)$, or both $P(Y)$ and $P(T, C)$.


We then formally define the OOD code detection task following~\cite{hsu2020generalized, liu2020energy, tian2020few, kim-etal-2023-pseudo} as follows. Given a main code-related task (\eg clone detection, code search, \etc), the objective here is to develop an \textit{auxiliary} scoring function $g: \mathcal{(T, C)} \rightarrow \mathcal{R}$ that assigns higher scores to normal instances where $ ((t, c), y) \in P((T, C),Y)$, and lower scores to OOD instances where $((t, c), y) \notin P((T, C),Y)$. Based on whether to use OOD instances during the main-task training of pre-trained NL-PL models, we define OOD for code in two settings, namely unsupervised and weakly-supervised learning. For the unsupervised setting, only normal data is used in the main-task training. Conversely, weakly-supervised approaches utilize ID and a tiny collection of OOD data (\eg 1\% of ID data)~\cite{tian2020few} in training. In this context, the output space $\mathcal{Y}$ is typically a binary set, indicating normal or abnormal, which is probably unknown during inference. Due to the small number of training OOD data, the OOD samples required by our COOD+ and other existing weakly-supervised approaches~\cite{ruff2020deep, tian2020few} in ML can be generated at minimal cost and feasibly verified by human experts when necessary.

% \textcolor{blue}{In line with prior work~\cite{ming2022delving, DBLP:journals/corr/abs-2104-08812}, a threshold should be set during inference to filter out OOD code instances and retain most ID code instances (e.g., 95\%), considering that real-world deployment usually involves a small number of OODs. Ensuring a high proportion of ID data is crucial to prevent the OOD auxiliary scoring from adversely affecting the performance of the models on their main code-related tasks.}

%By systematically generating OOD data representative of realistic SE scenarios, we aim to show the merits of OOD detection in SE deployment settings. Moreover, weak supervision provides guarantees that the performance of the main downstream task is not adversely affected by the OOD-aware finetuning process~\cite{majhi2021weakly}.

% When OOD samples are drawn from a distribution resulted from distribution shifts on $P(X,Y)$, they do not conform to the patterns, characteristics, or statistical properties of the original $P(X,Y)$ previously used to develop, train, and validate the model, leading to performance and reliability concerns. Hence, it is important to detect when SE models encounter OOD samples for appropriate counter measures, safeguarding their deployment in practical settings.

\subsection{Overview}
\begin{figure*}[!thb]
\begin{center}
 \includegraphics[width=0.98\textwidth]{figs/COOD.pdf}
    \caption{The Overview of Our Proposed COOD and COOD+ Approaches for OOD Detection %The COOD+ approach consists of two main components: a contrastive learning module and a binary OOD rejection network. (1) For unsupervised COOD, only the contrastive learning module is utilized, focusing on maximizing the cosine similarity difference between ID and OOD data. (2)The weakly-supervised COOD includes both the contrastive learning module and the binary OOD rejection network which identifies OOD samples with low ID probabilities using a threshold.
    }
    \label{fig:cad}
\end{center}

\end{figure*}

Overall, there are two versions of our COOD approach: unsupervised COOD and weakly-supervised COOD+. Given a multi-modal (NL, PL) input, the unsupervised COOD learns distinct representations based on a contrastive learning module by utilizing a pre-trained Transformer-based code representation model (\ie GraphCodeBERT~\cite{guo2020graphcodebert}). Then, these representations are mapped to distance-based OOD detection scores in order to indicate whether the test samples are OODs during inference. The weakly-supervised COOD+ further integrates a improved contrastive learning module with a binary OOD rejection module to enhance the detection performance by using a very tiny number of OOD data during model training. The OOD samples are then identified by the detection scores produced by the contrastive learning module as well as the prediction probabilities of the binary OOD rejection module. 

\subsection{Unsupervised COOD}

Our unsupervised COOD approach consists of a contrastive learning (CL) module trained only on ID samples. Specifically, given (comment, code) pairs as input, we fine-tune a comment encoder and a code encoder through a contrastive objective to learn discriminative features, which are expected to help identify OOD samples based on a scoring function. 

The (comment, code) pairs are first converted into the comment and code representations, which are processed by the comment and code encoder, respectively. We use the pre-trained GraphCodeBERT model~\cite{guo2020graphcodebert} as the encoder architecture (\ie backbone). GraphCodeBERT is a Transformer-based model pre-trained on six PLs by taking the (comment, code) pairs as well as the data flow graph of the code as input, which has shown superior performance on code understanding and generation tasks. All the representations of the last hidden states of the GraphCodeBERT encoder are averaged to obtain the sequence-level features of comment and code.


\textbf{Contrastive Learning Module.} To achieve the contrastive learning objective, we fine-tune the base (GraphCodeBERT) encoders with the InfoNCE loss~\cite{oord2018representation}. The comment and code encoders follow the Siamese architecture~\cite{guo2022unixcoder} since they are designed to be identical subnetworks with the same GraphCodeBERT backbones, in which their parameters (\ie weights and biases) are shared during fine-tuning. Parameter sharing can reduce the model size and has shown state-of-the-art performance for the code search task~\cite{shi2023cocosoda}. To extract discriminative features for (comment, code) pairs, we organize them into functionally-similar positive pairs and dissimilar negative (unpaired) pairs. Through a contrastive objective, positive pairs are drawn together, while unpaired comment and code are pulled apart. Specifically, for each positive (comment, code) pair $(t_i, c_i)$ in the batch, the code in each of other pairs and $t_i$ are constructed as in-batch negatives, similarly for the comment side. The loss function then formulates the contrastive learning as a classification task, which maximizes the probability of selecting positives along the diagonal of the similarity matrix (as shown in Fig. \ref{fig:cad}) by taking the \textit{softmax} of projected embedding similarities across the batch.  The loss function can be summarized as follows: 

  
\begin{align}\small
\label{eq:infonce}
\begin{split}
    \mathcal{L}^{CL} &= -\frac{1}{2N}(\sum_{n=1}^{N}\log \frac{e^{sim(v_{t_i},v_{c_i})/\tau}}{\sum_{j=1}^{N}e^{sim(v_{t_i},v_{c_j})/\tau}}\\ &+ \sum_{n=1}^{N}\log \frac{e^{sim(v_{t_i},v_{c_i})/\tau}}{\sum_{j=1}^{N}e^{sim(v_{t_j},v_{c_i})/\tau}})
\end{split}
\end{align}
where $v_{t_i}$ and $v_{c_i}$ represent the extracted features of the comment $t_i$ and the code ${c_i}$. $\tau$ is the temperature hyperparameter, which is set to 0.07 following previous work~\cite{shi2023cocosoda}.  $sim(v_{c_i},v_{t_i})$ and $sim(v_{t_i},v_{c_j})/sim(v_{t_j},v_{c_i})$ represent the cosine similarities between comment and code features for positive and negative pairs, respectively. $N$ is the number of input pairs in the batch. InfoNCE loss is designed for self-supervised learning and learns to distinguish positive pairs from in-batch negatives. Compared to other contrastive losses~\cite{khosla2020supervised, zhou2021contrastive}, it can take advantage of large batch size to automatically construct many diverse in-batch negatives for robustness representation learning, which is more effective to capture the alignment information between comment and code. % Despite~\cite{khosla2020supervised, zhou2021contrastive} being able to learn class-aware representations, they are designed for supervised learning which requires labeled data, and specifically for the code search task, a large amount of negative pairs. 


\textbf{Scoring Function.} Existing OOD detection techniques in ML derive scoring functions based on model's output, which typically map the learned class-probabilistic distributions to OOD detection scores for testing samples. Maximum Softmax Probability (MSP)~\cite{hendrycks2016baseline} is commonly used for OOD scoring. This method uses the maximum classification probability $\max_{l\in L}\textit{softmax}(f(v_{t}, v_{c}))$, where $f(v_{t}, v_{c})$ is the output of the classification model, with low scores indicating low likelihoods of being OOD. However, NL-PL code search models typically utilize the similarity retrieval scores of NL-PL output representations to make predictions. Therefore, to enable simultaneous similarity and OOD inference, we alternatively extract cosine similarity scores of testing NL-PL pairs as OOD detection scores, denoted as $P^{CL}=sim(v_{c},v_{t})$. The underlying intuition behind this scoring metric is that OOD testing samples should receive low retrieval confidence from the model fine-tuned on ID data, which establishes a closer relationship between ID (comment, code) pairs. Hence, this scoring function also assigns higher scores to ID data and lower scores to OOD data similar to previous scoring methods.

\subsection{Weakly-Supervised COOD+}

To further enhance the performance of unsupervised COOD, we extend it to a weakly-supervised detection model, called COOD+, which takes advantage of a few OOD examples. Inspired by~\cite{duong2024general}, our COOD+ combines an improved contrastive learning (CL) and a binary OOD rejection classifier (BC). The improved CL module adopts a margin-based loss~\cite{xue2009svm} which enforces a margin of difference between the cosine similarities of aligned and unaligned (comment, code) pairs, and constrains the cosine similarities of OOD pairs below another margin. The BC module integrates features from both comments and code to calculate the probabilities of OOD pairs. The OOD scoring function is then designed by combining the cosine similarity scores from the CL module and the prediction probabilities from the BC module. Below, we detail each component of our weakly-supervised COOD+.

\textbf{Improved Contrastive Learning (CL) Module.} Given a batch of $N$ input pairs (comprising $N-K$ ID pairs and $K$ OOD pairs), the latent representations are first obtained from the comment and code encoders. Then the margin-based loss is leveraged in the CL module to distinguish representations of ID and OOD data by constraining the cosine similarity. Specifically, the margin-based contrastive loss is first applied to $N-K$ ID code to maximize the difference between aligned (comment, code) pairs and incorrect pairs for each batch: 


%%%
\begin{equation}\scriptsize
\label{eq:contrast_id}
    \hspace{-0.2cm} \mathcal{L}^{ID} = \sum_{i=1}^{N-K} \left(\frac{1}{N}\sum_{j=1, j \neq i}^N \max \left(0,m - s(v_{t_i^+}, v_{c_i^+}) + s(v_{t_j^-}, v_{c_i^+})\right)\right)
\end{equation}


\noindent $s(v_{t_i^+}, v_{c_i^+})$ represents the cosine similarity of representations between each aligned ID pair from all the $N-K$ aligned pairs, and $s(v_{t_j^-},v_{c_i^+})$ represents the cosine similarity of representations between each ID code and all the other $N-1$ comments (\ie the comment is either not aligned with the ID code or from OOD comments). Thus, this margin-based loss encourages the difference between the aligned pairs and the incorrect pairs greater than margin $m$. 

Regarding the $K$ OOD code, we enforce a constraint on the cosine similarity between each OOD code and all the comments, ensuring that the similarity remains below a margin $m$. This constraint is necessary because each OOD code should not align with its corresponding comment, nor with any of the other $K-1$ OOD comments and the $N-K$ ID comments. The loss function is denoted as follows:
\begin{equation}\small
\label{eq:contrast_ood}
\mathcal{L}^{OOD} = \sum_{k=1}^{K}\left(\frac{1}{N}\sum_{i=1}^N \max \left(0,-m + sim(t_j^-, c_k^-)\right)\right),
\end{equation}

\noindent where $sim(t_j^-, c_k^-)$ represents the cosine similarity between each of the $K$ OOD code and all $N$ comments.
Finally, the overall loss for the contrastive module can be expressed as: 


\begin{equation}\small
\label{eq:contrast}
    \mathcal{L}^{CL} = \frac{1}{N}\left(\mathcal{L}^{ID}+ \mathcal{L}^{OOD}\right).
\end{equation}


\textbf{Binary OOD Rejection (BC) Module.}
Besides the CL module, we also introduce a classification module under weakly-supervision for identifying OOD samples. %The key challenge lies in effectively integrating features from both NL and PL to improve classification accuracy.
Inspired by the Replaced Token Detection (RTD) objective utilized in~\cite{feng2020codebert}, we bypass the generation phase since our OOD data are generated prior to training. Therefore, we directly train a rejection network responsible for determining whether (comment, code) pairs are OOD or not, which can be framed as a binary classification problem. Our binary OOD rejection network comprises a 3-layer fully-connected neural network with \textit{Tanh} activation, and the input is based on the concatenation of features from the comment and code encoders: 
$v_i = (v_{t_i}, v_{c_i}, v_{t_i} - v_{c_i}, v_{t_i} + v_{c_i}).$ Apart from utilizing the comment and code features, we also incorporate feature subtraction $v_{t_i} - v_{c_i}$ and aggregation $v_{t_i} + v_{c_i}$.  Additionally, we apply the sigmoid function to the output layer, producing a prediction probability that indicates whether the sample is OOD. We then use binary cross entropy loss for this module:

\begin{equation}\small
    \mathcal{L}^{BC} = \frac{1}{N}\sum_{i=1}^{N-K}(y_i\log p(v_i)+(1-y_i)\log(1-p(v_i))),
\end{equation}


\noindent where $p(v_i)$ is the output probability of the BC module, and $y_i \in [0, 1]$ is the ground-truth label. $y_i = 1$ indicates the input sample is an inlier, while $y_i = 0$ signifies it is an outlier. 

Hence, for weakly-supervised COOD+, we combine the objectives of the CL and the BC modules to jointly train our model, where $\lambda$ is a weight used to balance the loss functions:

\begin{equation}\small
\label{eq:loss}
    \mathcal{L} = \mathcal{L}^{CL} + \lambda \mathcal{L}^{BC}.
\end{equation}

\noindent\textbf{Combined Scoring Function.}
Similar to the unsupervised COOD approach, we utilize the diagonals of the similarity matrix as the OOD detection scores obtained from the CL module. To further improve the detection performance of the weakly-supervised version, we combine these $P^{CL}$ scores with the output probabilities of the BC module, denoted as $P^{BC}$. Here, we convert cosine similarity scores into probabilities using the sigmoid function $P^{CL*}=\sigma(sim(v_{c},v_{t}))$, then use multiplication to create the overall scoring function, yielding $P^{ID}=P^{CL*}\times P^{BC}$. We anticipate that higher scores will be assigned to ID pairs, while lower scores will be assigned to OOD pairs. This combined scoring function aims to enhance the discrimination between inliers and outliers, leading to more effective OOD detection.
\section{Results}\label{sec:experiments}

This section presents the results of our experiments, including the proposed LLM value system, the analysis of the system, and the benchmarking of our system against the well-established Schwartz's values \cite{schwartz2012overview}.


\subsection{Proposed Value System}
\cref{fig:dendrogram} visualizes the value system constructed by \our{}.
\cref{tab:factor loadings} gathers its factor loadings, Cronbach's Alpha, and confidence intervals. The factor loadings indicate the strength of the relationship between each factor and its atomic values, with higher loadings suggesting more contribution to the factor. Cronbach's Alpha measures the internal consistency of each factor, with higher values indicating greater reliability. 
Some atomic values are removed to ensure clear loading patterns and desirable factor reliability \cite{aavik2002structure}. After that, all our factors reach the standard psychometric threshold of 0.7, indicating strong internal consistency \cite{taber2018use}.

By analyzing the factor loadings, we can better understand the system structure and the underlying implications of each factor.

\paragraph{Factor 1: Social Responsibility.}
This factor reflects values centered on collective well-being and ethical social engagement. The high loadings on Equity (0.890), Empathy (0.885), and Teamwork (0.872) highlight its emphasis on fairness and collaboration. Values such as Equality (0.827) and Unity (0.825) indicate a strong focus on inclusivity. Public Benefit (0.820) and Democracy (0.813) support the broader societal perspective and prioritize the common good.

\paragraph{Factor 2: Risk-Taking.}
This factor embodies a preference for dynamism, exploration, and adaptability. High loadings for Challenge (0.812), Boldness (0.804), and Adventure (0.798) illustrate a willingness to confront uncertainty and seek new experiences. Values such as Change (0.730), Flexibility (0.721), and negatively loaded Stability (-0.701) underscore openness to transformation, while Thrill-seeking (0.728) further conveys a desire for excitement.

\paragraph{Factor 3: Rule-Following.}
This factor prioritizes order, discipline, and dependability. Strong loadings for Realism (0.708), Order (0.694), Responsibility (0.682), and Reliability (0.598) reflect a grounded, pragmatic, and conscientious approach to life. Values such as Prudence (0.676), Efficiency (0.669), and Timeliness (0.658) emphasize structured and deliberate actions.

\paragraph{Factor 4: Self-Competence.}
This factor represents personal growth and self-efficacy. Loadings for Confidence (0.601), Impact (0.527), and Proactivity (0.460) indicate a focus on self-assurance and initiative. Values such as Achievement (0.455), Recognition (0.455), and Excellence (0.455) highlight aspirations for acknowledgment and high performance.

\paragraph{Factor 5: Rationality.}
This factor centers on logical and evidence-based decision-making. Loadings for Objectivity (0.705) and Evidence-based (0.649) demonstrate a preference for impartiality and reliance on empirical data; Logic (0.618) and Neutrality (0.522) further reinforce this analytic perspective.
 

\begin{table}[h!]
    \centering
    \caption{Factor loadings and Cronbach's Alpha for our value system. CI denotes the 95\% confidence interval. Full results are available in \cref{app:pca}.}
    \begin{tabular}{llcc}
        \toprule
        Factor & Value & Loading & Cronbach's Alpha (CI) \\
        \midrule
        \multirow{8}{*}{Social Responsibility}
                                  & Equity        & 0.890 & \multirow{8}{*}{0.957 (0.952 -- 0.961)}\\
                                  & Empathy       & 0.885 & \\
                                  & Teamwork     & 0.872 & \\
                                  & Equality     & 0.827 & \\
                                  & Unity        & 0.825 & \\
                                  & Public Benefit  & 0.820 & \\
                                  & Democracy  & 0.813 & \\
                                  \multicolumn{4}{c}{\textit{...}} \\
        \midrule
        \multirow{8}{*}{Risk-Taking} 
                                  & Challenge      & 0.812 & \multirow{8}{*}{0.919 (0.910 -- 0.928)} \\
                                  & Boldness     & 0.804 & \\
                                  & Adventure     & 0.798 & \\
                                  & Change   & 0.730 & \\
                                  & Thrill-seeking & 0.728 & \\
                                  & Flexibility & 0.721 & \\
                                  & Stability  & -0.701 & \\
                                  \multicolumn{4}{c}{\textit{...}} \\
        \midrule
        \multirow{8}{*}{Rule-Following}
                                  &  Realism    & 0.708 & \multirow{8}{*}{0.842 (0.824 -- 0.859)} \\
                                  &  Order      & 0.694 & \\
                                  &  Responsibility  & 0.682 & \\
                                  &  Prudence  & 0.676 & \\
                                  &  Efficiency  & 0.669 & \\
                                  &  Timeliness  & 0.658 & \\
                                  &  Reliability  & 0.598 & \\
                                  \multicolumn{4}{c}{\textit{...}} \\
        \midrule
        \multirow{6}{*}{Self-Competence}
                                  & Confidence   & 0.601 & \multirow{6}{*}{0.761 (0.732 -- 0.787)} \\
                                  & Impact       & 0.527 & \\
                                  & Proactivity  & 0.460 & \\
                                  & Achievement  & 0.455 & \\
                                  & Recognition  & 0.455 & \\
                                  & Excellence   & 0.455 & \\
        \midrule
        \multirow{4}{*}{Rationality}
                                  & Objectivity & 0.705 & \multirow{4}{*}{0.722 (0.686 -- 0.754)} \\
                                  & Evidence-based & 0.649 & \\
                                  & Logic & 0.618 & \\
                                  & Neutrality & 0.522 & \\
        \bottomrule
    \end{tabular}
    \label{tab:factor loadings}
\end{table}



\begin{figure*}[htbp]
    \begin{floatrow}
        \ffigbox[\FBwidth]
        {\includegraphics[width=0.62\textwidth]{figures/system_dendrogram.pdf}}
        {\caption{Dendrogram of our value system. Values with a "*" are negatively loaded.} \label{fig:dendrogram}}
        \killfloatstyle
        \ffigbox[\FBwidth]
        {%
            \begin{tabular}{@{}c@{}}
                \includegraphics[width=0.34\textwidth]{figures/correlation_heatmap_our_system.pdf}
                 \\\vspace{-6mm}
                \includegraphics[width=0.34\textwidth]{figures/circumplex_analysis.pdf} \\[-0.5em]
                \caption{Correlation heatmap and circumplex analysis.}\label{fig:corr-circ}
            \end{tabular}
        }
        {}
    \end{floatrow}
\end{figure*}


\subsection{Analyzing Value System}

\paragraph{Value Correlation Analysis.}
\cref{fig:corr-circ} (Top) presents the correlations between the factors in our value system. Similar to Schwartz's theory of basic human values \cite{schwartz2012overview}, LLM values also exhibit both compatible and opposing relationships. Notably, social responsibility, rule-following, and rationality show positive correlations with one another, while all of them are negatively correlated with risk-taking.

\paragraph{Circumplex Analysis.}
Circumplex analysis is a statistical method that examines whether the underlying structure of variables aligns with a circumplex pattern, and, if so, the positions of variables on a circle. The stronger the correlation between variables, the shorter their distance on the circumference.
We conduct circumplex analysis based on the correlations between factors. \cref{fig:corr-circ} (Bottom) illustrates the analysis results based on Browne's circular stochastic process model \cite{browne1992circumplex, grassi2010circe}. The compatible values are closer on the circle (e.g., Social Responsibility and Rule-Following) while opposing values are positioned diagonally (e.g., Risk-Taking and Rule-Following). The results verify the presence of a circumplex structure in the value system.

\paragraph{Consistency Across Datasets.} To evaluate the consistency and robustness of our multivariate system structure (i.e., the 5-dimensional relationship), we measure LLM values using two distinct prompt datasets: the psychometric prompts from GPV \cite{ye2025gpv} and the red-teaming prompts from SALAD-Bench \cite{li2024salad}, which feature distinct prompt distributions. Their measurements yield an average intra-LLM correlation of 0.87; here we use intra-LLM correlations because the relative value hierarchy within an LLM is more important than their absolute measurements. This result indicates a high level of consistency in the value structure across prompt distributions. We also find a high correlation (0.73) between intra-LLM value consistency and LLM safety scores \cite{li2024salad}. It suggests that LLMs with higher value consistency tend to be safer. Complete results are available in \cref{app:consistency}.


\subsection{Comparing Value Systems}
\label{sec:comparing value systems}

We benchmark the proposed value system against Schwartz's value system \cite{schwartz2012overview}, the most established framework for human values and commonly used in LLM value studies.

\paragraph{Confirmatory Factor Analysis for Evaluating Structure Validity.}
We follow the standard validation procedures of CFA \cite{schwartz2004cfa} to evaluate the structure validity of different value systems. For our value system, we construct it using half of the measurement data and bootstrap the data to ensure its sufficiency (\#data points \( \ge 5 \times \) \#variables). The other half of the data is held out for CFA. For Schwartz's value system, we map the observed value variables (the atomic values in our system) to its four high-level values or ten low-level values, according to the semantic relevance \cite{schwartz2004cfa} measured by an embedding model \cite{openai2024textembedding3large}; all data is used for CFA. \cref{tab:cfa results} displays the CFA results. Our value system demonstrates a better fit for the data, capturing the underlying values behind LLM generations.

\begin{table}[H]
    \centering
    \begin{tabular}{c|c|ccccc}
        \toprule
        Value system & \#Values & CFI \( \uparrow \) & GFI \( \uparrow \) & RMSEA \( \downarrow \) & AIC \( \downarrow \) & BIC \( \downarrow \) \\
        \midrule
        Schwartz (H) & 4 & 0.56 & 0.52 & \textbf{0.10} & 340 & 1484 \\
        Schwartz (L) & 10 & 0.23 & 0.22 & 0.11 & 324 & 1464 \\
        Ours & 5 & \textbf{0.68} & \textbf{0.65} & 0.12 & \textbf{265} & \textbf{1145} \\
        \bottomrule
    \end{tabular}
    \caption{CFA results of different value systems. H and L denote high and low-level Schwartz values, respectively.}
    \label{tab:cfa results}
\end{table}
    


\paragraph{LLM Safety Prediction for Evaluating Predictive Validity.}
We follow the experimental setup in \cite[Section 5.2]{ye2025gpv} to predict LLM safety based on their value orientations. \cref{tab:llm safety pred acc} presents the prediction accuracy under different value systems. The higher accuracy of our value system indicates its superior predictive validity.
In addition, according to the parameters of well-trained linear classifiers, we find that Social Responsibility, Rule-Following, and Rationality enhance safety, whereas Risk-taking and Self-Competence undermine it; see \cref{app:safety_prediction} for details.


\begin{wraptable}[9]{r}{0.36\linewidth}
        \centering
        \vspace{-4mm}
        \begin{tabular}{c|c}
            \toprule
            Value system & Acc (\%) \\
            \midrule
            Schwartz (H) & 81\tiny{$\pm$ 15} \\
            Schwartz (L) & 74\tiny{$\pm$ 16 } \\
            Ours & \textbf{87}\tiny{$\pm$ 9 } \\
            \bottomrule
        \end{tabular}
        \caption{Accuracy of LLM safety prediction based on values.}
        \label{tab:llm safety pred acc}
    \end{wraptable}


\paragraph{LLM Value Alignment for Evaluating Representation Power.}
We follow the experimental setup in \cite[Section 6.2]{yao2023value_fulcra} to perform LLM value alignment. Different value systems are respectively used to represent LLM outputs and desired human values. We employ GPV \cite{ye2025gpv} as an open-vocabulary value evaluator for all value systems, but also include the original results of \cite{yao2023value_fulcra} using its Schwartz-specific evaluator for comparison. \cref{tab:llm value alignment} shows the alignment performance of different value systems. Alignment under our value system converges to the lowest harmlessness and the highest helpfulness, establishing its superior representation power. Full experimental details are available in \cref{app:llm value alignment}.


\begin{table}[H]
    \centering
    \begin{tabular}{c|cc}
        \toprule
        Value system
        & Harmlessness
        & Helpfulness
         \\
        \midrule
        Schwartz* \cite{yao2023value_fulcra} & -1.52 & 2.15 \\
        Schwartz & -1.40 & 2.13 \\
        Ours & \textbf{-1.26} & \textbf{2.16} \\
        \bottomrule
    \end{tabular}
    \caption{Alignment performance of different value systems. Schwartz* denotes the original results drawn from \cite{yao2023value_fulcra} using a Schwartz-specific value evaluator. Both Schwartz baselines are based on the 10-dimensional Schwartz values \cite{yao2023value_fulcra}.}
    \label{tab:llm value alignment}
\end{table}


% \qy{In this paper, we propose an efficient single-stage framework called \nickname{} for 3D object detection. Considering the task of object detection inherently focuses on the foreground points, we propose an instance-aware learning-based downsampling way to automatically select the sparse yet important instance points. In addition, a dedicated contextual centroid perception module is proposed to fully exploit the geometrical structure around the bounding boxes. Extensive experiments conducted on the KITTI detection benchmark demonstrated the superior efficiency and accuracy of the proposed \nickname{}. \revise{In future work, we will further tackle extreme cases such as overlapped bounding boxes.}}

%This paper presents a new point-based single-stage 3D object detection networks, named \nickname{}. With novel instance-aware downsampling strategy and centroid rally module, we can effectively and efficiently achieve muti-class 3D object detection in a bottom-up manner.  Our \nickname{} achieves the best results among pure point-based methods, and provides a state-of-the-art efficiency than existing LiDAR detectors. In the future, we will focus on designing an efficient network to achieve real-time and robust 3D detection in 360-degree LiDAR scenes.

\qy{In this paper, we propose an efficient solution termed \nickname{} for point-based 3D object detection in LiDAR point clouds. Considering the task of object detection inherently focuses on the foreground information, we propose an instance-aware learning-based downsampling way to automatically select the sparse yet important instance points. Additionally, a dedicated contextual centroid perception module is proposed to fully exploit the geometrical structure around the bounding boxes. Extensive experiments conducted on three detection benchmarks demonstrated the superior efficiency and accuracy of the proposed \nickname{}. 
}

\smallskip\noindent\textbf{Limitations.} Although the proposed \nickname{} can achieve remarkable efficiency in object detection of large-scale LiDAR points clouds, it also has limitations. \textit{e.g.,} the instance-aware sampling relies on the semantic prediction of each point, which is susceptible to class imbalances distribution. For future work, we will further explore advanced techniques to alleviate the imbalanced issue.





\clearpage
\bibliography{main}


\clearpage
\appendix



\section{Experimental Details}
\label{app:experimental_details}

\subsection{Loss of plasticity with Dropout}
\label{app:loss_pl_dropout}

\subsubsection{Trainability for Dropout}
\label{app:loss_pl_dropout_trainability}
We employed an 8-layer MLP featuring 512 hidden units and trained it on 1400 samples from the MNIST dataset.
The model is trained for 50 different tasks, with each task running for 100 epochs.
To evaluate subnetwork plasticity, we extracted 10 subnetworks at the conclusion of each epoch, training these on new tasks for an additional 100 epochs and then calculated the mean final accuracy.
Adam optimizer was utilized for model optimization.


\subsubsection{Warm-start}
We used the ResNet-18 architecture described in Appendix \ref{app:generalizability}.
In the warm-start scenario, the model was pre-trained on 10\% of the CIFAR100 dataset for 1,000 epochs and continued training on the full dataset for 100 epochs, with the optimizer reset at the start of full dataset training.
In the cold-start scenario, the model was trained on the full dataset for 100 epochs. Adam optimizer with learning rate $0.001$ was utilized.


\subsection{Trainability}
\label{app:trainability}
\textbf{Permuted MNIST. }
We followed the setup of \cite{dohare2024loss}.
It consists of a total of 800 tasks that 60,000 images are fed into models only once with 512 batch size.
In the beginning of each task, the pixel of images are permuted arbitrarily.
We trained fully connected neural networks with three hidden layers.
Each hidden layer has 2,000 units and followed by ReLU activaiton function.

\textbf{Random Label MNIST. }
We conducted a variant of \cite{kumar2023maintaining}.
It consists of a total of 200 tasks that 1,600 images are fed into models with 64 batch size.
In the beginning of each task, the label class of images are changed to other class arbitrarily.
We trained fully connected neural networks with three hidden layers 100 epochs per each task.
Each hidden layer has 2,000 units and followed by ReLU activaiton function.


\subsection{Generalizability}
\label{app:generalizability}
We conducted experiments on CIFAR-10, CIFAR-100 \cite{krizhevsky2009learning}, and TinyImageNet \cite{le2015tiny} datasets using a 4-layer CNN, ResNet-18, and VGG-16, respectively, to evaluate the effectiveness of AID across different model architectures and datasets.
We provide detailed model architecture below:
\begin{itemize}
    \item \textbf{CNN}: We employed a convolutional neural network (CNN), which is used in relatively small image classification.
    The model includes two convolutional layers with a $5\times5$ kernel and 16 channels and max-pooling layer is followed after activation function.
    The two fully connected layers follow with 100 hidden units. 
    % residual block에도 잘 적용되는지 확인하기 위해 modern architecture인 resnet-18을 실험에 사용 
    \item \textbf{Resnet-18} \citep{he2016deep}: We utilized ResNet-18 to examine how well AID integrates with modern deep architectures featuring residual connections.
    Following \cite{lee2024slow}, the stem layers were removed to accommodate the smaller image size of the dataset. Additionally, a gradient clipping threshold of 0.5 was applied to ensure stable training.
    \item \textbf{VGG-16} \citep{simonyan2014very}: We adopted VGG-16 with batch normalization to investigate whether AID adapts properly in large-size models. The number of hidden units of the classifiers was set to 4096 without dropout.
\end{itemize}

In addition, we replaced the max-pooling layer with an average-pooling layer for methods such as Fourier activation, DropReLU, and AID, where large values may not necessarily represent important features.
Next, we provide the detailed experimental settings below.

\textbf{Continual Full. } Similar to the setup provided by \cite{shen2024step}, the entire data is randomly split into 10 chunks.
The training process consists of 10 stages and the model gains access up to the $k$-th chunk in each stage $k$.
In each stage, the dataset is fed into models with a batch size of 256.
We trained the model for 100 epochs per each stage, and we reset the optimizer when each stage starts training.

\textbf{Continual Limited. } This setting follows the same configuration as continual full, with the key distinction that the model does not retain access to previously seen data chunks.
At each stage, the model is trained only on the current chunk, without revisiting earlier data, simulating real-world constraints such as memory limitations and privacy concerns.

\textbf{Class-Incremental. } For CIFAR100 and TinyImageNet, the dataset was divided into 20 class-based chunks, with new classes introduced incrementally at each stage.
Unless otherwise specified, test accuracy is evaluated based on the corresponding classes encountered up to each stage.
The rest of the setup, including the batch size, and training epochs per stage, follows the Continual Full setting.










\subsection{Reinforcement Learning}
\label{app:reinforcement_learning}
To evaluate whether AID enhances sample efficiency on reinforcement learning, we conducted experiments on 17 Atari games from the Arcade Learning Environment \cite{bellemare2013arcade}, selected based on prior studies \cite{kumar2020implicit,sokar2023dormant}. We trained a DQN model following \citet{mnih2015human} using the CleanRL framework \cite{huang2022cleanrl}. The replay ratio was set to 1, as adopted in \citet{sokar2023dormant, elsayed2024weight}. We followed the hyperparameter settings for environment from \citet{sokar2023dormant}, with details provided in Table \ref{tab:hyperparameter_rl_env}.



\begin{table}[h]
    \centering
    \caption{Hyperparameters used in reinforcement learning environment}
    \begin{tabular}{l r}
        \toprule
        \textbf{Parameter} & \textbf{Value} \\
        \midrule
        Optimizer & Adam \cite{kingma2014adam} \\
        Optimizer: $\epsilon$ & $1.5\mathrm{e}-4$ \\
        Optimizer: Learning rate & $6.25\mathrm{e}-4$ \\
        \midrule
        Minimum $\epsilon$ for training & $0.01$ \\
        Evaluation $\epsilon$ & $0.001$\\
        Discount Factor $\gamma$ & $0.99$ \\
        Replay buffer size & $10^6$ \\
        Minibatch size & $32$ \\
        Initial collect steps & $20000$ \\
        Training iterations & $10$ \\
        Training environment steps per iteration & $250K$ \\
        Updates per environment step (Replay Ratio) & $1$ \\
        Target network update period & $2000$\\
        Loss function & Huber Loss \cite{huber1992robust} \\
        
        \bottomrule
    \end{tabular}
    \label{tab:hyperparameter_rl_env}
\end{table}







\subsection{Standard Supervised Learning}
For the same model architecture and dataset used in the generalizability experiments, we trained with Adam optimizer for 200 epochs, applying learning rate decay at the 100th and 150th epochs. The initial learning rate was set to $0.001$ and was reduced by a factor of $10$ at each decay step.



\subsection{Hyperparameter Search Space}
\label{app:hyperparameter_seach_space}
We present the hyperparameter search space considered for each experiment in Table \ref{tab:hyperparameter_search_space}.
Without mentioned, we performed a sweep over 5 different seeds for all experiments, except for VGG-16 model on the TinyImageNet dataset, where we used only 3 seeds due to computational cost.
\cref{tab:hyperparameter_permuted_mnist,tab:hyperparameter_random_label_mnist,tab:hyperparameter_continual_full,tab:hyperparameter_continual_limited,tab:hyperparameter_class_incremental,tab:hyperparameter_rl,tab:hyperparameter_sl} shows the best hyperparameter set that we found in various experiments. 

\begin{table}[H]
    \centering
    \caption{Hyperparameter search space for every experiment}
    \begin{tabular}{l l l l}
        \toprule
        \textbf{Experiment} & \textbf{Method} & \textbf{Hyperparameters} & \textbf{Search Space}\\
        \midrule
        Warm-Start & Dropout & $p$ & $0.1, 0.3, 0.5$ \\
        & DIA & $p$ & $0.7, 0.8, 0.9$ \\
        \midrule
        Trainability & ADAM & learning rate & $1\mathrm{e}-3, 1\mathrm{e}-4$ \\
                     & SGD & learning rate & $3\mathrm{e}-2, 3\mathrm{e}-3$ \\
                     & L2 & $\lambda$ & $1\mathrm{e}-2, 1\mathrm{e}-3, 1\mathrm{e}-4, 1\mathrm{e}-5, 1\mathrm{e}-6$ \\
                     & L2 Init & $\lambda$ & $1\mathrm{e}-2, 1\mathrm{e}-3, 1\mathrm{e}-4, 1\mathrm{e}-5, 1\mathrm{e}-6$ \\
                     & Dropout & $p$ & $0.01, 0.05, 0.1, 0.15, 0.2, 0.25, 0.3$ \\
                     & S\&P & $\lambda$ & $0.1, 0.2, 0.3, 0.4, 0.5, 0.6, 0.7, 0.8, 0.9$ \\
                     & ReDo & \text{threshold} & $0.0, 1.0, 5.0, 10.0, 50.0$ \\
                     &      & \text{period} & $2500$ \\
                     & CBP & \text{replacement rate}($\rho$) & $1\mathrm{e}-1, 1\mathrm{e}-2, 1\mathrm{e}-3, 1\mathrm{e}-4, 1\mathrm{e}-5$ \\
                     &     & \text{maturity threshold} & $100$ \\
                     & RReLU & \text{lower} & $0.0625, 0.125, 0.25$ \\
                     &       & \text{upper} & $0.125, 0.25, 0.333, 0.5$ \\
                     & DropReLU & $p$ & $0.1, 0.2, 0.3, 0.4, 0.5, 0.6, 0.7, 0.8, 0.9, 0.99$ \\
                     & DIA & $p$ & $0.1, 0.2, 0.3, 0.4, 0.5, 0.6, 0.7, 0.8, 0.9, 0.99$ \\
        \midrule
        Generalizability & ADAM & learning rate & $1\mathrm{e}-3, 1\mathrm{e}-4$ \\
                         & S\&P & $\lambda$ & $0.2, 0.4, 0.6, 0.8$ \\
                         & CBP & \text{replacement rate}($\rho$) & $1\mathrm{e}-4, 1\mathrm{e}-5$\\
                         &     & \text{maturity threshold} & $100, 1000$\\
                         & DropReLU & $p$ & $0.7, 0.8, 0.9$ \\
                         & L2 & $\lambda$ & $1\mathrm{e}-2, 1\mathrm{e}-3, 1\mathrm{e}-4, 1\mathrm{e}-5$ \\
                         & Dropout & $p$ & $0.1, 0.3, 0.5$\\
                         & RReLU & lower & $0.125$\\
                         &       & upper & $0.333$\\
                         & DIA & $p$ & $0.7, 0.8, 0.9$ \\
        \hdashline
        Continual Full   & DASH & $\alpha$ & $0.1, 0.3$\\
                         &      & $\lambda$ & $0.05, 0.1, 0.3$\\
        \hdashline
        Class-Incremental & DASH & $\alpha$ & $0.1, 0.3$\\
                         &      & $\lambda$ & $0.05, 0.1, 0.3$\\
        \hdashline
        Continual Limited & DASH & $\alpha$ & $0.1, 0.3, 0.7, 1.0$\\
                         &      & $\lambda$ & $0.3$\\
                         & S-EWC & $\lambda$ & $100, 10, 1, 0.1, 0.01$\\
        \midrule
        Standard SL & L2   & $\lambda$ & $1\mathrm{e}-2, 1\mathrm{e}-3, 1\mathrm{e}-4, 1\mathrm{e}-5$ \\
                    & Dropout & $p$ & $0.1, 0.3, 0.5$ \\
                    & DIA & $p$ & $0.8, 0.9, 0.95$ \\
        \midrule
        Reinforcement Learning & DIA & $p$ & $0.99, 0.999$ \\
        \bottomrule
    \end{tabular}
    \label{tab:hyperparameter_search_space}
\end{table}



\begin{table}[p]
    \centering
    \caption{Hyperparameter set of each method on permuted MNIST}
    \begin{tabular}{l|l|l|l}
        \toprule
        \textbf{Method} & \textbf{Optimizer} & \textbf{Learning Rate} & \textbf{Optimal Hyperparameters} \\
        \midrule
        Baseline                & Adam & $1\mathrm{e}{-3}$ & -- \\
        Dropout                 & Adam & $1\mathrm{e}{-3}$ & $p = 0.05$ \\
        L2                      & Adam & $1\mathrm{e}{-3}$ & $\lambda = 1\mathrm{e}{-5}$ \\
        L2 Init                 & Adam & $1\mathrm{e}{-3}$ & $\lambda = 1\mathrm{e}{-4}$ \\
        Shrink \& Perturb       & Adam & $1\mathrm{e}{-3}$ & $\lambda = 0.2$ \\
        ReDo                    & Adam & $1\mathrm{e}{-3}$ & $\text{recycle period} = 118, \text{recycle threshold} = 50$ \\
        Continual Backprop      & Adam & $1\mathrm{e}{-3}$ & $\rho = 1\mathrm{e}{-4}, \text{maturity threshold} = 100$ \\
        Concat ReLU             & Adam & $1\mathrm{e}{-3}$ & -- \\
        RReLU                   & Adam & $1\mathrm{e}{-3}$ & $\text{lower} = 0.0625, \text{upper} = 0.333$ \\
        DropReLU                & Adam & $1\mathrm{e}{-3}$ & $p = 0.99$ \\
        DIA                     & Adam & $1\mathrm{e}{-3}$ & $p = 0.99$ \\
        \midrule
        Baseline                & Adam & $1\mathrm{e}{-4}$ & -- \\
        Dropout                 & Adam & $1\mathrm{e}{-4}$ & $p = 0.05$ \\
        L2                      & Adam & $1\mathrm{e}{-4}$ & $\lambda = 1\mathrm{e}{-5}$ \\
        L2 Init                 & Adam & $1\mathrm{e}{-4}$ & $\lambda = 1\mathrm{e}{-4}$ \\
        Shrink \& Perturb       & Adam & $1\mathrm{e}{-3}$ & $\lambda = 0.1$ \\
        ReDo                    & Adam & $1\mathrm{e}{-4}$ & $\text{recycle period} = 118, \text{recycle threshold} = 50$ \\
        Continual Backprop      & Adam & $1\mathrm{e}{-4}$ & $\rho = 1\mathrm{e}{-4}, \text{maturity threshold} = 100$ \\
        Concat ReLU             & Adam & $1\mathrm{e}{-4}$ & -- \\
        RReLU                   & Adam & $1\mathrm{e}{-4}$ & $\text{lower} = 0.0625, \text{upper} = 0.333$ \\
        DropReLU                & Adam & $1\mathrm{e}{-4}$ & $p = 0.99$ \\
        DIA                     & Adam & $1\mathrm{e}{-4}$ & $p = 0.99$ \\
        \midrule
        Baseline                & SGD  & $3\mathrm{e}{-2}$ & -- \\
        Dropout                 & SGD  & $3\mathrm{e}{-2}$ & $p = 0.25$ \\
        L2                      & SGD  & $3\mathrm{e}{-2}$ & $\lambda = 1\mathrm{e}{-5}$ \\
        L2 Init                 & SGD  & $3\mathrm{e}{-2}$ & $\lambda = 1\mathrm{e}{-3}$ \\
        Shrink \& Perturb       & SGD  & $3\mathrm{e}{-3}$ & $\lambda = 0.1$ \\
        ReDo                    & SGD  & $3\mathrm{e}{-2}$ & $\text{recycle period} = 118, \text{recycle threshold} = 5$ \\
        Continual Backprop      & SGD  & $3\mathrm{e}{-2}$ & $\rho = 1\mathrm{e}{-3}, \text{maturity threshold} = 100$ \\
        Concat ReLU             & SGD  & $3\mathrm{e}{-2}$ & -- \\
        RReLU                   & SGD  & $3\mathrm{e}{-2}$ & $\text{lower} = 0.0625, \text{upper} = 0.333$ \\
        DropReLU                & SGD  & $3\mathrm{e}{-2}$ & $p = 0.99$ \\
        DIA                     & SGD  & $3\mathrm{e}{-2}$ & $p = 0.99$ \\
        \midrule
        Baseline                & SGD  & $3\mathrm{e}{-3}$ & -- \\
        Dropout                 & SGD  & $3\mathrm{e}{-3}$ & $p = 0.01$ \\
        L2                      & SGD  & $3\mathrm{e}{-3}$ & $\lambda = 1\mathrm{e}{-5}$ \\
        L2 Init                 & SGD  & $3\mathrm{e}{-3}$ & $\lambda = 1\mathrm{e}{-3}$ \\
        Shrink \& Perturb       & SGD  & $3\mathrm{e}{-3}$ & $\lambda = 0.1$ \\
        ReDo                    & SGD  & $3\mathrm{e}{-3}$ & $\text{recycle period} = 118, \text{recycle threshold} = 1$ \\
        Continual Backprop      & SGD  & $3\mathrm{e}{-3}$ & $\rho = 1\mathrm{e}{-5}, \text{maturity threshold} = 100$ \\
        Concat ReLU             & SGD  & $3\mathrm{e}{-3}$ & -- \\
        RReLU                   & SGD  & $3\mathrm{e}{-3}$ & $\text{lower} = 0.0625, \text{upper} = 0.333$ \\
        DropReLU                & SGD  & $3\mathrm{e}{-3}$ & $p = 0.99$ \\
        DIA                     & SGD  & $3\mathrm{e}{-3}$ & $p = 0.99$ \\
        \bottomrule
    \end{tabular}
    \label{tab:hyperparameter_permuted_mnist}
\end{table}




\begin{table}[p]
    \centering
    \caption{Hyperparameter set of each method on random label MNIST}
    \begin{tabular}{l|l|l|l}
        \toprule
        \textbf{Method} & \textbf{Optimizer} & \textbf{Learning Rate} & \textbf{Optimal Hyperparameters} \\
        \midrule
        Baseline                & Adam & $1\mathrm{e}{-3}$ & -- \\
        Dropout                 & Adam & $1\mathrm{e}{-3}$ & $p = 0.15$ \\
        L2                      & Adam & $1\mathrm{e}{-3}$ & $\lambda = 1\mathrm{e}{-4}$ \\
        L2 Init                 & Adam & $1\mathrm{e}{-3}$ & $\lambda = 1\mathrm{e}{-3}$ \\
        Shrink \& Perturb       & Adam & $1\mathrm{e}{-3}$ & $\lambda = 0.8$ \\
        ReDo                    & Adam & $1\mathrm{e}{-3}$ & $\text{recycle period} = 2500, \text{recycle threshold} = 0$ \\
        Continual Backprop      & Adam & $1\mathrm{e}{-3}$ & $\rho = 1\mathrm{e}{-3}, \text{maturity threshold} = 100$ \\
        Concat ReLU             & Adam & $1\mathrm{e}{-3}$ & -- \\
        RReLU                   & Adam & $1\mathrm{e}{-3}$ & $\text{lower} = 0.0625, \text{upper} = 0.125$ \\
        DropReLU                & Adam & $1\mathrm{e}{-3}$ & $p = 0.9$ \\
        DIA                     & Adam & $1\mathrm{e}{-3}$ & $p = 0.9$ \\
        \midrule
        Baseline                & Adam & $1\mathrm{e}{-4}$ & -- \\
        Dropout                 & Adam & $1\mathrm{e}{-4}$ & $p = 0.25$ \\
        L2                      & Adam & $1\mathrm{e}{-4}$ & $\lambda = 1\mathrm{e}{-3}$ \\
        L2 Init                 & Adam & $1\mathrm{e}{-4}$ & $\lambda = 1\mathrm{e}{-4}$ \\
        Shrink \& Perturb       & Adam & $1\mathrm{e}{-3}$ & $\lambda = 0.7$ \\
        ReDo                    & Adam & $1\mathrm{e}{-4}$ & $\text{recycle period} = 2500, \text{recycle threshold} = 0$ \\
        Continual Backprop      & Adam & $1\mathrm{e}{-4}$ & $\rho = 1\mathrm{e}{-4}, \text{maturity threshold} = 100$ \\
        Concat ReLU             & Adam & $1\mathrm{e}{-4}$ & -- \\
        RReLU                   & Adam & $1\mathrm{e}{-4}$ & $\text{lower} = 0.25, \text{upper} = 0.333$ \\
        DropReLU                & Adam & $1\mathrm{e}{-4}$ & $p = 0.99$ \\
        DIA                     & Adam & $1\mathrm{e}{-4}$ & $p = 0.99$ \\
        \midrule
        Baseline                & SGD  & $3\mathrm{e}{-2}$ & -- \\
        Dropout                 & SGD  & $3\mathrm{e}{-2}$ & $p = 0.15$ \\
        L2                      & SGD  & $3\mathrm{e}{-2}$ & $\lambda = 1\mathrm{e}{-3}$ \\
        L2 Init                 & SGD  & $3\mathrm{e}{-2}$ & $\lambda = 1\mathrm{e}{-3}$ \\
        Shrink \& Perturb       & SGD  & $3\mathrm{e}{-3}$ & $\lambda = 0.2$ \\
        ReDo                    & SGD  & $3\mathrm{e}{-2}$ & $\text{recycle period} = 2500, \text{recycle threshold} = 1$ \\
        Continual Backprop      & SGD  & $3\mathrm{e}{-2}$ & $\rho = 1\mathrm{e}{-5}, \text{maturity threshold} = 100$ \\
        Concat ReLU             & SGD  & $3\mathrm{e}{-2}$ & -- \\
        RReLU                   & SGD  & $3\mathrm{e}{-2}$ & $\text{lower} = 0.0625, \text{upper} = 0.125$ \\
        DropReLU                & SGD  & $3\mathrm{e}{-2}$ & $p = 0.99$ \\
        DIA                     & SGD  & $3\mathrm{e}{-2}$ & $p = 0.99$ \\
        \midrule
        Baseline                & SGD  & $3\mathrm{e}{-3}$ & -- \\
        Dropout                 & SGD  & $3\mathrm{e}{-3}$ & $p = 0.01$ \\
        L2                      & SGD  & $3\mathrm{e}{-3}$ & $\lambda = 1\mathrm{e}{-2}$ \\
        L2 Init                 & SGD  & $3\mathrm{e}{-3}$ & $\lambda = 1\mathrm{e}{-3}$ \\
        Shrink \& Perturb       & SGD  & $3\mathrm{e}{-3}$ & $\lambda = 0.1$ \\
        ReDo                    & SGD  & $3\mathrm{e}{-3}$ & $\text{recycle period} = 2500, \text{recycle threshold} = 1$ \\
        Continual Backprop      & SGD  & $3\mathrm{e}{-3}$ & $\rho = 1\mathrm{e}{-3}, \text{maturity threshold} = 100$ \\
        Concat ReLU             & SGD  & $3\mathrm{e}{-3}$ & -- \\
        RReLU                   & SGD  & $3\mathrm{e}{-3}$ & $\text{lower} = 0.0625, \text{upper} = 0.25$ \\
        DropReLU                & SGD  & $3\mathrm{e}{-3}$ & $p = 0.99$ \\
        DIA                     & SGD  & $3\mathrm{e}{-3}$ & $p = 0.99$ \\
        \bottomrule
    \end{tabular}
    \label{tab:hyperparameter_random_label_mnist}
\end{table}




\begin{table}[p]
    \centering
    \caption{Hyperparameter set of each method on continual full setting.}
    \begin{tabular}{l|l|c|c}
        \toprule
        \textbf{Dataset (Model)} & \textbf{Method} & \textbf{Optimal Learning Rate} & \textbf{Optimal Hyperparameters} \\
        \midrule
        CIFAR10 & Baseline                & $1\mathrm{e}{-4}$ & -- \\
        (CNN)&Full Reset              & $1\mathrm{e}{-4}$ & -- \\
        &L2                      & $1\mathrm{e}{-3}$ & $\lambda = 1\mathrm{e}{-2}$ \\
        &Dropout                 & $1\mathrm{e}{-4}$ & $p = 0.3$ \\
        &RReLU                   & $1\mathrm{e}{-4}$ & -- \\
        &CReLU                   & $1\mathrm{e}{-4}$ & -- \\
        &Fourier                 & $1\mathrm{e}{-4}$ & -- \\
        &S\&P                    & $1\mathrm{e}{-4}$ & $\lambda = 0.8$ \\
        &CBP                     & $1\mathrm{e}{-4}$ & $\rho = 1\mathrm{e}{-5}, \text{maturity threshold} = 100$ \\
        &DASH                    & $1\mathrm{e}{-4}$ & $\alpha=0.1$, $\lambda=0.3$ \\
        &DropReLU                & $1\mathrm{e}{-4}$ & $p = 0.9$ \\
        &AID                     & $1\mathrm{e}{-3}$ & $p=0.9$ \\
        \midrule
        CIFAR100 & Baseline                & $1\mathrm{e}{-3}$ & -- \\
        (Resnet-18)&Full Reset              & $1\mathrm{e}{-3}$ & -- \\
        &L2                      & $1\mathrm{e}{-3}$ & $\lambda = 1\mathrm{e}{-2}$ \\
        &Dropout                 & $1\mathrm{e}{-4}$ & $p = 0.3$ \\
        &RReLU                   & $1\mathrm{e}{-3}$ & -- \\
        &CReLU                   & $1\mathrm{e}{-3}$ & -- \\
        &Fourier                 & $1\mathrm{e}{-3}$ & -- \\
        &S\&P                    & $1\mathrm{e}{-3}$ & $\lambda = 0.8$ \\
        &CBP                     & $1\mathrm{e}{-3}$ & $\rho = 1\mathrm{e}{-5}, \text{maturity threshold} = 1000$ \\
        &DASH                    & $1\mathrm{e}{-4}$ & $\alpha=0.1$, $\lambda=0.05$ \\
        &DropReLU                & $1\mathrm{e}{-3}$ & $p = 0.8$ \\
        &AID                     & $1\mathrm{e}{-3}$ & $p=0.7$ \\
        \midrule
        TinyImageNet & Baseline                & $1\mathrm{e}{-3}$ & -- \\
        (VGG-16)&Full Reset              & $1\mathrm{e}{-3}$ & -- \\
        &L2                      & $1\mathrm{e}{-3}$ & $\lambda = 1\mathrm{e}{-4}$ \\
        &Dropout                 & $1\mathrm{e}{-4}$ & $p = 0.1$ \\
        &RReLU                   & $1\mathrm{e}{-4}$ & -- \\
        &CReLU                   & $1\mathrm{e}{-3}$ & -- \\
        &Fourier                 & $1\mathrm{e}{-4}$ & -- \\
        &S\&P                    & $1\mathrm{e}{-3}$ & $\lambda = 0.8$ \\
        &CBP                     & $1\mathrm{e}{-3}$ & $\rho = 1\mathrm{e}{-4}, \text{maturity threshold} = 100$ \\
        &DASH                    & $1\mathrm{e}{-4}$ & $\alpha=0.3$, $\lambda=0.1$ \\
        &DropReLU                & $1\mathrm{e}{-4}$ & $p = 0.7$ \\
        &AID                     & $1\mathrm{e}{-4}$ & $p=0.7$ \\
        \bottomrule
    \end{tabular}
    \label{tab:hyperparameter_continual_full}
\end{table}






\begin{table}[p]
    \centering
    \caption{Hyperparameter set of each method on continual limited setting.}
    \begin{tabular}{l|l|c|c}
        \toprule
        \textbf{Dataset (Model)} & \textbf{Method} & \textbf{Optimal Learning Rate} & \textbf{Optimal Hyperparameters} \\
        \midrule
        CIFAR10 & Baseline                & $1\mathrm{e}{-4}$ & -- \\
        (CNN)&L2                      & $1\mathrm{e}{-4}$ & $\lambda = 1\mathrm{e}{-5}$ \\
        &Dropout                 & $1\mathrm{e}{-3}$ & $p = 0.5$ \\
        &S-EWC                 & $1\mathrm{e}{-4}$ & $\lambda = 0.01$ \\
        &RReLU                   & $1\mathrm{e}{-4}$ & -- \\
        &CReLU                   & $1\mathrm{e}{-4}$ & -- \\
        &Fourier                 & $1\mathrm{e}{-4}$ & -- \\
        &S\&P                    & $1\mathrm{e}{-4}$ & $\lambda = 0.2$ \\
        &CBP                     & $1\mathrm{e}{-4}$ & $\rho = 1\mathrm{e}{-4}, \text{maturity threshold} = 100$ \\
        &DASH                    & $1\mathrm{e}{-3}$ & $\alpha=0.1$, $\lambda=0.3$ \\
        &DropReLU                & $1\mathrm{e}{-4}$ & $p = 0.9$ \\
        &AID                     & $1\mathrm{e}{-3}$ & $p=0.8$ \\
        \midrule
        CIFAR100 & Baseline                & $1\mathrm{e}{-3}$ & -- \\
        (Resnet-18)&L2                      & $1\mathrm{e}{-3}$ & $\lambda = 1\mathrm{e}{-5}$ \\
        &Dropout                 & $1\mathrm{e}{-4}$ & $p = 0.1$ \\
        &S-EWC                 & $1\mathrm{e}{-3}$ & $\lambda = 1$ \\
        &RReLU                   & $1\mathrm{e}{-3}$ & -- \\
        &CReLU                   & $1\mathrm{e}{-3}$ & -- \\
        &Fourier                 & $1\mathrm{e}{-3}$ & -- \\
        &S\&P                    & $1\mathrm{e}{-3}$ & $\lambda = 0.2$ \\
        &CBP                     & $1\mathrm{e}{-3}$ & $\rho = 1\mathrm{e}{-5}, \text{maturity threshold} = 1000$ \\
        &DASH                    & $1\mathrm{e}{-4}$ & $\alpha=0.1$, $\lambda=0.3$ \\
        &DropReLU                & $1\mathrm{e}{-4}$ & $p = 0.7$ \\
        &AID                     & $1\mathrm{e}{-3}$ & $p=0.8$ \\
        \midrule
        TinyImageNet & Baseline                & $1\mathrm{e}{-4}$ & -- \\
        (VGG-16)&L2                      & $1\mathrm{e}{-3}$ & $\lambda = 1\mathrm{e}{-4}$ \\
        &Dropout                 & $1\mathrm{e}{-4}$ & $p = 0.1$ \\
        &S-EWC                 & $1\mathrm{e}{-3}$ & $\lambda = 100$ \\
        &RReLU                   & $1\mathrm{e}{-4}$ & -- \\
        &CReLU                   & $1\mathrm{e}{-3}$ & -- \\
        &Fourier                 & $1\mathrm{e}{-4}$ & -- \\
        &S\&P                    & $1\mathrm{e}{-3}$ & $\lambda = 0.4$ \\
        &CBP                     & $1\mathrm{e}{-4}$ & $\rho = 1\mathrm{e}{-4}, \text{maturity threshold} = 1000$ \\
        &DASH                    & $1\mathrm{e}{-3}$ & $\alpha=0.1$, $\lambda=0.3$ \\
        &DropReLU                & $1\mathrm{e}{-4}$ & $p = 0.8$ \\
        &AID                     & $1\mathrm{e}{-4}$ & $p=0.8$ \\
        \bottomrule
    \end{tabular}
    \label{tab:hyperparameter_continual_limited}
\end{table}




\input{tables/hyperparameter_class_incremental}
\newpage


\begin{table}[H]
    \vspace{2.5cm}
    \centering
    \caption{\centering Hyperparameter set of AID on reinforcement learning setting.}
    \begin{tabular}{l|c}
        \toprule
        \textbf{Game} &\textbf{Optimal Hyperparameters} \\
        \midrule
        Seaquest & $p=0.999$\\
        DemonAttack & $p=0.99$\\
        SpaceInvaders & $p=0.99$\\
        Qbert & $p=0.999$\\
        DoubleDunk & $p=0.99$\\
        MsPacman & $p=0.999$\\
        Enduro & $p=0.99$\\
        BeamRider & $p=0.99$\\
        WizardOfWor & $p=0.999$\\
        Jamesbond & $p=0.99$\\
        RoadRunner & $p=0.999$\\
        Asterix & $p=0.99$\\
        Pong & $p=0.999$\\
        Zaxxon & $p=0.999$\\
        YarsRevenge & $p=0.99$\\
        Breakout & $p=0.99$\\
        IceHockey & $p=0.99$\\
        
        \bottomrule
    \end{tabular}
    \label{tab:hyperparameter_rl}
\end{table}
\vfill




\begin{table}[H]
    \centering
    \caption{Hyperparameter set of each method on standard supervised learning setting.}
    \begin{tabular}{l|l|c}
        \toprule
        \textbf{Dataset(Model)} & \textbf{Method} & \textbf{Optimal Hyperparameters} \\
        \midrule
        CIFAR10 &L2     & $\lambda = 1\mathrm{e}{-2}$ \\
        (CNN)
        &Dropout               & $p = 0.3$ \\
        &AID                    & $p=0.95$ \\
        \midrule
        CIFAR100 &L2     & $\lambda = 1\mathrm{e}{-5}$ \\
        (Resnet-18)
        
        &Dropout              & $p = 0.1$ \\
        &AID                  & $p=0.8$ \\
        \midrule
        TinyImageNet &L2 & $\lambda = 1\mathrm{e}{-5}$ \\
        (VGG-16)
        &Dropout            & $p = 0.1$ \\
        &AID               & $p=0.9$ \\
        \bottomrule
    \end{tabular}
    \label{tab:hyperparameter_sl}
\end{table}
\vfill




\newpage
\section{Extended Results}


\begin{table*}[t]
\centering
\small
\setlength{\tabcolsep}{0.6mm}
\begin{tabular}{ll|lr|rl|rl|rl|rl|rl|rl|ll|ll}
\hline
\multicolumn{2}{c|}{Dataset} & \multicolumn{2}{c|}{ECL} & \multicolumn{2}{c|}{Weather} & \multicolumn{2}{c|}{Traffic} & \multicolumn{2}{c|}{Solar} & \multicolumn{2}{c|}{Exchange} & \multicolumn{2}{c|}{ETTh1} & \multicolumn{2}{c|}{ETTh2} & \multicolumn{2}{c|}{ETTm1} & \multicolumn{2}{c}{ETTm2} \\ \cline{3-20} 
\multicolumn{2}{c|}{Models} & MSE & MAE & MSE & MAE & MSE & MAE & MSE & MAE & MSE & MAE & MSE & MAE & MSE & MAE & MSE & MAE & MSE & MAE \\ \hline
\parbox[t]{2mm}{\multirow{4}{*}{\rotatebox[origin=c]{90}{z.s. }}} & TimePFN & \textbf{0.315} & \textbf{0.383} & \textbf{0.209} & \textbf{0.255} & \textbf{1.108} & \textbf{0.613} & 0.941 & \textbf{0.730} & 0.105 & 0.229 & \textbf{0.453} & \textbf{0.439} & \textbf{0.328} & \textbf{0.362} & \textbf{0.637} & \textbf{0.512} & \textbf{0.212}  & \textbf{0.291} \\
 & Naive & 1.587 & 0.945 & 0.259 & 0.254 & 2.714 & 1.077 & 1.539 & 0.815 & \textbf{0.081} & \textbf{0.196} & 1.294 & 0.713 & 0.431 & 0.421  & 1.213 & 0.664 & 0.266 & 0.327 \\ 
  & SeasonalNaive & 1.618 & 0.964 & 0.268 & 0.263 & 2.774 & 1.097 & 1.599 & 0.844 & 0.086 & 0.204 & 1.325 & 0.727 & 0.445 & 0.431  & 1.227 & 0.673 & 0.274 & 0.334 \\ 
  & Mean & 0.845 & 0.761 & 0.215 & 0.271 & 1.410 & 0.804 & \textbf{0.910} & 0.734 & 0.139 & 0.269 & 0.700 & 0.558 & 0.352 & 0.387 & 0.693 & 0.547 & 0.229 & 0.307 \\ \hline \hline
\parbox[t]{2mm}{\multirow{7}{*}{\rotatebox[origin=c]{90}{Budget =50 }}} & TimePFN & \textbf{0.235} & \textbf{0.322} & \textbf{0.190} & \textbf{0.235} & \textbf{0.746} & \textbf{0.468} & \textbf{0.429} & \textbf{0.450} & \textbf{0.096} & \textbf{0.218}  & \textbf{0.438} & \textbf{0.429} & \textbf{0.324} & \textbf{0.359} &  \textbf{0.419} & \textbf{0.418} & \textbf{0.195} & \textbf{0.276} \\
 & iTransformer & 0.278 & 0.360 & 0.237 & 0.278 & 0.801 & 0.499 & 0.513 & 0.479  & 0.145 & 0.275 & 0.838 & 0.617 & 0.410 & 0.422  & 0.884   & 0.608  & 0.268 & 0.337 \\
 & PatchTST & 0.667 & 0.646 & 0.221 & 0.269 & 1.295 & 0.746 & 0.810 & 0.669 & 0.127 & 0.255 & 0.778 & 0.587 & 0.372 & 0.401 & 0.656 & 0.528 & 0.231 & 0.310 \\
 & DLinear & 0.406 & 0.463 & 0.742 & 0.612 & 1.888 & 0.937 & 0.956 & 0.813 & 3.432 & 1.349 & 1.404 & 0.881 & 3.928 & 1.383 & 1.332 & 0.846 & 3.484 & 1.290 \\
 & FEDFormer & 0.908 & 0.758 & 0.306 & 0.381 & 1.587 & 0.874 & 0.972 & 0.757 & 0.165 & 0.300 & 0.676 & 0.570 & 0.424 & 0.468  & 0.745 & 0.589 & 0.291 & 0.387 \\ 
& Informer & 1.226 & 0.896 & 0.464 & 0.511 & 1.714 & 0.901 & 0.887 & 0.783 & 1.470 & 1.007 & 1.172 & 0.819 & 2.045 & 1.093  & 1.003 & 0.745 & 1.590 & 0.995 \\ 
  & Autoformer & 0.729 & 0.675 & 0.322 & 0.401 & 1.600 & 0.883 & 1.065 & 0.808 & 0.213 & 0.351 & 0.607 & 0.560 & 0.492 & 0.506 & 0.763 & 0.592 & 0.316 & 0.407 \\ \hline \hline





  

\parbox[t]{2mm}{\multirow{7}{*}{\rotatebox[origin=c]{90}{Budget =100 }}} & TimePFN & \textbf{0.221} & \textbf{0.309} & \textbf{0.187} & \textbf{0.232} & \textbf{0.644} & \textbf{0.424} & \textbf{0.351} & \textbf{0.383} & \textbf{0.083} & \textbf{0.205}  & \textbf{0.441} & \textbf{0.429} & \textbf{0.322} & \textbf{0.356} &  \textbf{0.412} & \textbf{0.411} & \textbf{0.196} & \textbf{0.273} \\
 & iTransformer & 0.253 & 0.337 & 0.220 & 0.263 & 0.740 & 0.468 & 0.369 & 0.387  & 0.138 & 0.268 & 0.728 & 0.574 & 0.401 & 0.418  & 0.816   & 0.586  & 0.260 & 0.331 \\
 & PatchTST & 0.361 & 0.432 & 0.216 & 0.256 & 0.982 & 0.592 & 0.575 & 0.524 & 0.102 & 0.227 & 0.757 & 0.579 & 0.371 & 0.400 & 0.502 & 0.461 & 0.215 & 0.298 \\
 & DLinear & 0.332 & 0.409 & 0.636 & 0.562 & 1.770 & 0.897 & 0.887 & 0.784 & 2.712 & 1.172 & 1.256 & 0.826 & 3.237 & 1.246 & 1.214 & 0.799 & 2.810 & 1.140 \\
 & FEDformer & 0.597 & 0.598 & 0.264 & 0.344 & 1.350 & 0.775 & 0.951 & 0.752 & 0.158 & 0.291 & 0.562 & 0.513 & 0.362 & 0.407  & 0.724 & 0.572 & 0.290 & 0.387 \\ 
  & Informer & 1.056 & 0.837 & 0.369 & 0.432 & 1.609 & 0.860 & 0.731 & 0.702 & 0.883 & 0.774 & 1.038 & 0.757 & 1.279 & 0.916  & 0.883 & 0.664 & 1.040 & 0.802 \\ 
  & Autoformer & 0.468 & 0.519 & 0.251 & 0.330 & 1.344 & 0.773 & 0.960 & 0.762 & 0.187 & 0.321 & 0.550 & 0.524 & 0.379 & 0.422 & 0.704 & 0.554 & 0.258 & 0.351 \\ \hline \hline





  

\parbox[t]{2mm}{\multirow{7}{*}{\rotatebox[origin=c]{90}{Budget =500 }}} & TimePFN & \textbf{0.190} & \textbf{0.283} & \textbf{0.178} & \textbf{0.222} & \textbf{0.487} & \textbf{0.335} & \textbf{0.269} & \textbf{0.305} & 0.083 & 0.203 & \textbf{0.401} & \textbf{0.412} & \textbf{0.311} & \textbf{0.352} & \textbf{0.360} & \textbf{0.386} & \textbf{0.185} & \textbf{0.268} \\
 & iTransformer & 0.200 & 0.284 & 0.211 & 0.248 & 0.514 & 0.354  & 0.307 & 0.334 & 0.113 & 0.239 & 0.489 & 0.470 & 0.361 & 0.394 & 0.569 & 0.494 & 0.231 & 0.310  \\
 & PatchTST & 0.236 & 0.320 & 0.210 & 0.246 & 0.740 & 0.455 & 0.321 & 0.353 & \textbf{0.081} & \textbf{0.198} & 0.596 & 0.515 & 0.358 & 0.392 & 0.369  & 0.386 & 0.190 & 0.275  \\
 & DLinear & 0.235 & 0.328 & 0.335 & 0.394 & 1.312 & 0.727 & 0.622 & 0.656 & 0.655 & 0.551 & 0.749 & 0.609 & 1.098 & 0.712 & 0.817 & 0.621 & 0.870 & 0.626 \\
 & FEDformer & 0.317 & 0.407 & 0.265 & 0.341 & 0.888 & 0.548 & 0.821 & 0.706 & 0.157 & 0.288 & 0.444 & 0.452 & 0.358 & 0.401  & 0.674 & 0.542 & 0.238 & 0.322 \\ 
   & Informer & 0.869 & 0.760 & 0.320 & 0.393 & 1.411 & 0.774 & 0.318 & 0.385 & 0.699 & 0.694 & 0.913 & 0.713 & 1.311 & 0.940  & 0.704 & 0.595 & 1.121 & 0.803 \\ 
  & Autoformer & 0.303 & 0.396 & 0.237 & 0.312 & 0.896 & 0.549 & 0.950 & 0.787 & 0.158 & 0.290 & 0.456 & 0.456 & 0.339 & 0.384 & 0.672 & 0.534 & 0.223 & 0.308 \\ \hline \hline




\parbox[t]{2mm}{\multirow{7}{*}{\rotatebox[origin=c]{90}{Budget = 1000 }}} & TimePFN & \textbf{0.173} & \textbf{0.268} & \textbf{0.175} & \textbf{0.219} & \textbf{0.452} & \textbf{0.310} & 0.243 & 0.288 & 0.084 & 0.204 & \textbf{0.405} & \textbf{0.415} & \textbf{0.304} & \textbf{0.351} & \textbf{0.344} & \textbf{0.378} & \textbf{0.180} & \textbf{0.262} \\
 & iTransformer & 0.184 & 0.271 & 0.206 & 0.242 & 0.469 & 0.324  & 0.276 & 0.309 & 0.100 & 0.223 & 0.433 & 0.436 & 0.336 & 0.379 & 0.464 & 0.444 & 0.211 & 0.294  \\
 & PatchTST & 0.219 & 0.304 & 0.198 & 0.237 & 0.683 & 0.420 & 0.280 & 0.324 & \textbf{0.082} & \textbf{0.200} & 0.490 & 0.467 & 0.337 & 0.378 & 0.353  & 0.375 & 0.187 & 0.272  \\
 & DLinear & 0.218 & 0.310 & 0.254 & 0.331 & 1.076 & 0.627 & 0.488 & 0.569 & 0.193 & 0.330 & 0.562 & 0.513 & 0.528 & 0.507 & 0.629 & 0.528 & 0.380 & 0.437 \\
 & FEDformer & 0.284 & 0.379 & 0.269 & 0.341 & 0.806 & 0.486 & 0.545 & 0.546 & 0.157 & 0.287 & 0.402 & 0.435 & 0.341 & 0.383  & 0.436 & 0.456 & 0.228 & 0.312 \\ 
   & Informer & 0.693 & 0.647 & 0.341 & 0.413 & 1.231 & 0.678 & \textbf{0.229} & \textbf{0.294} & 0.689 & 0.666 & 0.887 & 0.710 & 1.357 & 0.939  & 0.682 & 0.596 & 0.615 & 0.596 \\ 
  & Autoformer & 0.270 & 0.367 & 0.239 & 0.314 & 0.787 & 0.492 & 0.926 & 0.742 & 0.156 & 0.285 & 0.427 & 0.442 & 0.341 & 0.383 & 0.617 & 0.521 & 0.218 & 0.301 \\ \hline \hline




  
 \parbox[t]{2mm}{\multirow{7}{*}{\rotatebox[origin=c]{90}{Budget = All }}} & TimePFN & \textbf{0.138} & \textbf{0.137} & \textbf{0.166} & \textbf{0.208} & \textbf{0.392} & \textbf{0.260} & 0.203 & 0.219 & 0.100 & 0.223 & 0.402 & 0.417 & \textbf{0.293} & \textbf{0.343} & 0.392 & 0.402 & 0.180 & 0.262 \\
 & iTransformer & 0.147 & 0.239 & 0.175 & 0.215  & 0.393 & 0.268 & 0.201 & 0.233 & 0.086& 0.206  & \textbf{0.387} & 0.405 & 0.300 & 0.349 & 0.342 & 0.376 & 0.185 & 0.272 \\
 & PatchTST & 0.185 & 0.267 & 0.177 & 0.218 & 0.517 & 0.334 & 0.222 & 0.267 & \textbf{0.080} & \textbf{0.196} & 0.392 & \textbf{0.404} & \textbf{0.293} & \textbf{0.343} & \textbf{0.318} & \textbf{0.357} & \textbf{0.177} & \textbf{0.260} \\
 & DLinear & 0.195 & 0.278 & 0.341 & 0.412 & 0.690 & 0.432 & 0.286 & 0.375 & 0.101 & 0.237 & 0.400 & 0.412 & 0.357 & 0.406  & 0.344 & 0.371 & 0.195 & 0.293 \\
 & FEDformer & 0.196 & 0.310 & 0.227 & 0.313 & 0.573 & 0.357 & 0.242 & 0.342 & 0.148 & 0.280 & 0.380 & 0.417 & 0.340 & 0.386  & 0.363 & 0.408 & 0.191 & 0.286 \\ 
& Informer & 0.327 & 0.413 & 0.455 & 0.481 & 0.735 & 0.409 & \textbf{0.190} & \textbf{0.216} & 0.921 & 0.774 & 0.930 & 0.763 & 2.928 & 1.349  & 0.623 & 0.559 & 0.396 & 0.474 \\ 
  & Autoformer & 0.214 & 0.327 & 0.273 & 0.344 & 0.605 & 0.376 & 0.455 & 0.480 & 0.141 & 0.271 & 0.440 & 0.446 & 0.364 & 0.408 & 0.520 & 0.490 & 0.233 & 0.311 \\ \hline \hline



 \parbox[t]{2mm}{\multirow{7}{*}{\rotatebox[origin=c]{90}{Avg Acc. Budgets }}} & TimePFN & \textbf{0.191}	& \textbf{0.264}	&\textbf{0.179}	&\textbf{0.223}&	\textbf{0.544}	&\textbf{0.359}	&\textbf{0.299}	&\textbf{0.329}&	\textbf{0.089}&	\textbf{0.211}&	\textbf{0.417}&	\textbf{0.420}&	\textbf{0.311}	&\textbf{0.352}&	\textbf{0.385}	&\textbf{0.399}&	\textbf{0.187}&	\textbf{0.268 } \\
&iTransformer &0.212	&0.298&	0.210&	0.249	&0.583	&0.383&	0.333	&0.348	&0.116&0.242&	0.575	&0.500	&0.362&	0.392&	0.615&	0.502	&0.231&	0.309 \\
&PatchTST & 0.334	&0.394	&0.204&	0.245&	0.843&	0.509&	0.442&	0.427	&0.094&	0.215	&0.603&	0.510&	0.346	&0.383&	0.440&	0.421	&0.200	&0.283 \\
& DLinear& 0.277 & 0.358	&0.462&	0.462&	1.347&	0.724&	0.648	&0.639	&1.419	&0.728&	0.874&	0.648&	1.830&	0.851&	0.867&	0.633&	1.548	&0.757 \\
& FEDformer &0.460&	0.490&	0.266	&0.344	&1.041&	0.608&	0.706	&0.621&	0.157&	0.289&	0.493&	0.477&	0.365&	0.409&	0.588	&0.513&	0.248&	0.339 \\
&Informer &0.834&	0.711	&0.390&	0.446	&1.340	&0.724&	0.471&	0.476	&0.932	&0.783&	0.988&	0.752&	1.784&	1.047&	0.779&	0.632&	0.952&	0.734 \\
&Autoformer& 0.397&	0.457	&0.264&	0.340	&1.046&	0.615&	0.871	&0.716	&0.171&	0.304&	0.496&	0.486&	0.383&	0.421&	0.655&	0.538&	0.250&	0.336 \\
 \hline 
\multicolumn{2}{c|}{\# of Variates} & \multicolumn{2}{c|}{321} & \multicolumn{2}{c|}{21} & \multicolumn{2}{c|}{862} & \multicolumn{2}{c|}{137} & \multicolumn{2}{c|}{8} & \multicolumn{2}{c|}{7} & \multicolumn{2}{c|}{7} & \multicolumn{2}{c|}{7} & \multicolumn{2}{c}{7} \\ 
 
\end{tabular}
\caption{Results of \name on various benchmarks, compared to baseline models. \name has been fine-tuned using specified data budgets, with MSE and MAE scores reported. The best results are highlighted in bold, and both input and prediction lengths are set at 96. \name demonstrates remarkable performance in budget-limited settings, as well as with the full dataset, particularly in scenarios involving a large number of variates. }
\label{tbl:mse_main_app}
\end{table*}


\begin{table*}[t]
\centering
\small
\setlength{\tabcolsep}{0.6mm}
\begin{tabular}{ll|lr|rl|rl|rl|rl|rl|rl|ll|}
\hline
\multicolumn{2}{c|}{Prediction Length} & \multicolumn{2}{c|}{6} & \multicolumn{2}{c|}{8} & \multicolumn{2}{c|}{14} & \multicolumn{2}{c|}{18} & \multicolumn{2}{c|}{24} & \multicolumn{2}{c|}{36} & \multicolumn{2}{c|}{48} & \multicolumn{2}{c|}{Average} \\ \cline{3-18} 
\multicolumn{2}{c|}{Metric} & MSE & MAE & MSE & MAE & MSE & MAE & MSE & MAE & MSE & MAE & MSE & MAE & MSE & MAE & MSE & MAE \\ \hline
{\multirow{6}{*}{\rotatebox[origin=c]{90}{TimePFN-96 }}} & Exchange & 0.015 & 0.094 & 0.017 & 0.102 & 0.026 & 0.124 & 0.030 & 0.136 & 0.037 & 0.149 & 0.052 & 0.174 & 0.065 & 0.195 & 0.034 & 0.139 \\
 & Weather & 0.020 & 1.006 & 0.023 & 1.068 & 0.034 & 1.280 & 0.041 & 1.411 & 0.051 & 1.582 & 0.072 & 1.885 & 0.087 & 2.108 & 0.046 & 1.477 \\
 & Traffic & 0.388 & 0.489 & 0.397 & 0.496 & 0.409 & 0.506 & 0.408 & 0.499 & 0.413 & 0.499 & 0.436 & 0.516 & 0.449 & 0.520 & 0.414 & 0.503 \\
 & ECL & 0.368 & 0.471 & 0.410 & 0.497 & 0.497 & 0.550 & 0.518 & 0.560 & 0.539 & 0.571 & 0.602 & 0.600 & 0.629 & 0.600 & 0.509 & 0.549 \\
 & ETTh1 & 0.017 & 0.101 & 0.020 & 0.109 & 0.026 & 0.126 & 0.030 & 0.134 & 0.034 & 0.143 & 0.041 & 0.156 & 0.045 & 0.163 & 0.030 & 0.133 \\
 & ETTh2 & 0.059 & 0.185 & 0.067 & 0.198 & 0.082 & 0.220 & 0.087 & 0.227 & 0.091 & 0.234 & 0.104 & 0.249 & 0.112 & 0.260 & 0.086 & 0.224 \\ \hline



 \hline \parbox[t]{2mm}{\multirow{6}{*}{\rotatebox[origin=c]{90}{TimePFN-36 }}} & Exchange & 0.012 & 0.087 & 0.014 & 0.094 & 0.020 & 0.112 & 0.024 & 0.122 & 0.030 & 0.134 & 0.041 & 0.155 & 0.052 & 0.174 & 0.027 & 0.125 \\
 & Weather $\times 10^{2}$ & 0.017 & 0.932 & 0.020 & 0.991 & 0.029 & 1.174 & 0.036 & 1.301 & 0.046 & 1.470 & 0.065 & 1.775 & 0.081 & 2.024 & 0.042 & 1.381 \\
 & Traffic & 1.393 & 1.008 & \textbf{1.528} & \textbf{1.051} & \textbf{1.644} & \textbf{1.084} & \textbf{1.520} & \textbf{1.031} & \textbf{1.403} & \textbf{0.988} & \textbf{1.538} & \textbf{1.039} & \textbf{1.495} & \textbf{1.027} & \textbf{1.503} & \textbf{1.032} \\
 & ECL & 0.585 & 0.621 & \textbf{0.640} & 0.649 & \textbf{0.745} & \textbf{0.701} & \textbf{0.747} & \textbf{0.702} & \textbf{0.760} & \textbf{0.712} & \textbf{0.878} & \textbf{0.764} & \textbf{0.909} & \textbf{0.772} & \textbf{0.752} & \textbf{0.703} \\
 & ETTh1 & 0.018 & 0.100 & 0.020 & 0.107 & 0.025 & 0.121 & \textbf{0.028} & \textbf{0.128} & \textbf{0.032} & \textbf{0.137} & \textbf{0.040} & \textbf{0.153} & \textbf{0.045} & \textbf{0.164} & \textbf{0.029} & 0.130 \\
 & ETTh2 & 0.100 & 0.241 & \textbf{0.110} & 0.253 & \textbf{0.126} & \textbf{0.274} & \textbf{0.125} & \textbf{0.274} & \textbf{0.126} & \textbf{0.275} & \textbf{0.145} & \textbf{0.295} & \textbf{0.152} & \textbf{0.302} & \textbf{0.126} & \textbf{0.273} \\ \hline
 
 
 \hline \parbox[t]{2mm}{\multirow{6}{*}{\rotatebox[origin=c]{90}{ForecastPFN }}} & Exchange & 0.041 & 0.154 & 0.042 & 0.158 & 0.049 & 0.169 & 0.054 & 0.177 & 0.061 & 0.187 & 0.072 & 0.201 & 0.084 & 0.215 & 0.057 & 0.180 \\
 & Weather $\times 10^{2}$ & 0.062 & 1.668 & 0.065 & 1.719 & 0.074 & 1.865 & 0.080 & 1.952 & 0.089 & 2.073 & 0.103 & 2.278 & 0.115 & 2.443 & 0.084 & 1.999 \\
 & Traffic & 4.690 & 1.779 & 4.712 & 1.790 & 4.572 & 1.765 & 4.428 & 1.724 & 4.348 & 1.698 & 4.504 & 1.735 & 4.394 & 1.703 & 4.521 & 1.742 \\
 & ECL & 1.430 & 0.962 & 1.444 & 0.969 & 1.406 & 0.955 & 1.360 & 0.935 & 1.356 & 0.935 & 1.453 & 0.973 & 1.467 & 0.977 & 1.416 & 0.958 \\
 & ETTh1 & 0.085 & 0.216 & 0.087 & 0.220 & 0.093 & 0.228 & 0.097 & 0.232 & 0.104 & 0.239 & 0.119 & 0.256 & 0.131 & 0.270 & 0.102 & 0.237 \\
 & ETTh2 & 0.409 & 0.504 & 0.418 & 0.510 & 0.424 & 0.513 & 0.421 & 0.509 & 0.426 & 0.511 & 0.462 & 0.532 & 0.481 & 0.540 & 0.434 & 0.517 \\ \hline





\hline \parbox[t]{2mm}{\multirow{6}{*}{\rotatebox[origin=c]{90}{Chronos-Small }}} & Exchange & 0.026 & 0.072 & 0.048 & 0.081 & 0.079 & 0.104 & 0.020 & 0.107 & 0.059 & 0.124 & \textbf{0.034} & 0.141 & 0.075 & 0.165 & 0.049 & 0.113 \\
 & Weather $\times 10^{2}$ & 0.013 & 0.623 & \textbf{0.014} & 0.703 & \textbf{0.023} & 0.920 & 0.030 & 1.055 & 0.041 & 1.244 & \textbf{0.058} & 1.568 & 0.078 & 1.842 & 0.036 & 1.136 \\
 & Traffic & \textbf{1.298} & \textbf{0.819} & 1.997 & 1.056 & 3.738 & 1.530 & 4.063 & 1.642 & 3.545 & 1.502 & 3.434 & 1.482 & 3.646 & 1.519 & 3.103 & 1.364 \\
 & ECL & \textbf{0.473} & \textbf{0.488} & 0.698 & \textbf{0.601} & 1.313 & 0.856 & 1.443 & 0.914 & 1.310 & 0.865 & 1.371 & 0.893 & 1.458 & 0.931 & 1.152 & 0.792 \\
 & ETTh1 & 0.045 & 0.114 & 0.044 & 0.121 & 0.062 & 0.151 & 0.065 & 0.159 & 0.065 & 0.168 & 0.073 & 0.184 & 0.076 & 0.194 & 0.061 & 0.155 \\
 & ETTh2 & \textbf{0.089} & \textbf{0.188} & 0.134 & \textbf{0.238} & 0.227 & 0.337 & 0.251 & 0.365 & 0.235 & 0.358 & 0.250 & 0.374 & 0.266 & 0.393 & 0.207 & 0.321 \\ \hline


  \hline \parbox[t]{2mm}{\multirow{6}{*}{\rotatebox[origin=c]{90}{SeasonalNaive }}} & Exchange & 0.015 & 0.096 & 0.016 & 0.100 & 0.021 & 0.114 & 0.025 & 0.124 & 0.030 & 0.135 & 0.039 & 0.154 & 0.050 & 0.172 & 0.028 & 0.128 \\
 & Weather $\times 10^{2}$ & 0.021 & 0.907 & 0.023 & 0.965 & 0.031 & 1.137 & 0.039 & 1.278 & 0.048 & 1.445 & 0.067 & 1.740 & 0.084 & 1.989 & 0.045 & 1.352 \\
 & Traffic & 4.354 & 1.850 & 4.581 & 1.891 & 5.263 & 2.016 & 4.416 & 1.784 & 3.756 & 1.614 & 4.104 & 1.691 & 3.631 & 1.548 & 4.301 & 1.771 \\
 & ECL & 1.427 & 0.962  & 1.523 & 0.994 & 1.810 & 1.092 & 1.590 & 1.004 & 1.427 & 0.942 & 1.600 & 1.001 & 1.533 & 0.973 & 1.559 & 0.995 \\
 & ETTh1 & 0.027 & 0.126 & 0.029 & 0.131 & 0.035 & 0.145 & 0.037 & 0.149 & 0.040 & 0.156 & 0.049 & 0.171 & 0.055 & 0.181 & 0.039 & 0.151 \\
 & ETTh2 &0.272 & 0.394 & 0.283 & 0.405 & 0.313 & 0.437 & 0.278 & 0.409 & 0.254 & 0.390 & 0.279 & 0.413  & 0.273 & 0.406 & 0.279 & 0.408 \\ \hline


 \hline \parbox[t]{2mm}{\multirow{6}{*}{\rotatebox[origin=c]{90}{Naive }}} & Exchange & \textbf{0.008} & \textbf{0.064} & \textbf{0.010} & \textbf{0.073} & \textbf{0.015} & \textbf{0.093} & \textbf{0.019} & \textbf{0.104} & \textbf{0.024} & \textbf{0.118} & \textbf{0.034} & \textbf{0.140} & \textbf{0.045} & \textbf{0.160} & \textbf{0.022} & \textbf{0.107} \\
 & Weather $\times 10^{2}$ & \textbf{0.011} & \textbf{0.598} & \textbf{0.014} & \textbf{0.685} & \textbf{0.023} & \textbf{0.910} & \textbf{0.029} & \textbf{1.044} & \textbf{0.038} & \textbf{1.232} & \textbf{0.058} & \textbf{1.561} & \textbf{0.075} & \textbf{1.834} & \textbf{0.035} & \textbf{1.123} \\
 & Traffic & 1.759 & 1.041 & 2.495 & 1.263 & 4.090 & 1.661 & 4.245 & 1.716 & 3.524 & 1.504 & 3.622 & 1.536 & 3.574 & 1.517 & 3.330 & 1.463 \\
 & ECL & 0.586 & 0.560 & 0.827 & 0.672 & 1.400 & 0.904 & 1.479 & 0.941 & 1.309 & 0.874 & 1.406 & 0.914 & 1.471 & 0.938 & 1.211 & 0.829 \\
 & ETTh1 & \textbf{0.014} & \textbf{0.084} & \textbf{0.018} & \textbf{0.095} & \textbf{0.027} & \textbf{0.120} & 0.031 & 0.131 & 0.034 & 0.139 & 0.043 & 0.157 & 0.050 & 0.171 & 0.031 & \textbf{0.128} \\
 & ETTh2 &0.114&0.226& 0.157 & 0.272 & 0.240 & 0.357 & 0.256 & 0.376 & 0.229 & 0.357 & 0.248 & 0.377  & 0.259 & 0.390 & 0.215 & 0.336 \\ \hline
 
 \hline \parbox[t]{2mm}{\multirow{6}{*}{\rotatebox[origin=c]{90}{Mean }}} & Exchange & 0.027 & 0.131 & 0.028 & 0.135 & 0.033 & 0.146 & 0.037 & 0.152 & 0.042 & 0.161 & 0.052 & 0.177 & 0.062 & 0.192 & 0.040 & 0.156 \\
 & Weather $\times 10^{2}$ & 0.047 & 1.546 & 0.050 & 1.599 & 0.059 & 1.775 & 0.064 & 1.840 & 0.074 & 1.966 & 0.089 & 2.178 & 0.101 & 2.345 & 0.069 & 1.893 \\
 & Traffic & 2.293 & 1.332 & 2.350 & 1.343 & 2.233 & 1.287 & 2.049 & 1.221 & 1.920 & 1.183 & 2.078 & 1.234 & 1.955 & 1.192 & 2.125 & 1.256 \\
 & ECL & 0.923 & 0.793 & 0.955 & 0.805 & 0.960 & 0.805 & 0.929 & 0.790 & 0.923 & 0.788 & 1.025 & 0.828 & 1.029 & 0.827 & 0.963 & 0.805 \\
 & ETTh1 & 0.031 & 0.135 & 0.033 & 0.138 & 0.036 & 0.146 & 0.038 & 0.151 & 0.041 & 0.157 & 0.048 & 0.170 & 0.052 & 0.178 & 0.040 & 0.154 \\
 & ETTh2 & 0.161 & 0.314 & 0.166 & 0.320 & 0.167 & 0.321 & 0.162 & 0.315 & 0.162 & 0.314 & 0.179 & 0.330 & 0.182 & 0.333 & 0.168 & 0.321 \\ \hline
 
\end{tabular}
\caption{Zero-shot results of TimePFN on univariate time-series forecasting with input length = 36. TimePFN-96 has input length of 96. All other baselines have input length 36. Meta-N-Beats is not included as it is not our implementation. }
\label{tbl:uni_zero_shot}
\end{table*}


In addition to the budget scenarios presented in the main body, we also conducted experiments with data budgets of 100 and 1,000 to fully characterize our experimental results. Furthermore, the average accuracy across these data budgets is provided for reference. Table 5 showcases all these evaluations. In Table 6, we present the raw results of the univariate forecasting task for zero-shot forecasting.

\subsection{Multivariate Forecasting}
As shown in Table 5, \name consistently achieves the best results with a data budget of 100 and significantly outperforms all other models with a budget of 1,000, leading in 7 out of 9 datasets. \name excels particularly in datasets with a multivariate nature. Consider that PatchTST \cite{Yuqietal-2023-PatchTST} assumes channel independence, whereas iTransformer \cite{liu2023itransformer} treats each variate as a token, demonstrating extreme channel dependence. In the full budget scenario, where the entire dataset is utilized, the difference in forecasting performance between iTransformer and PatchTST is revealing, particularly in detecting datasets with high inter-channel dependencies. For instance, in the ECL and Traffic datasets, which have a large number of variates (which does not mean high channel dependence by itself), iTransformer shows superior forecasting performance compared to PatchTST. Conversely, in the ETT datasets, PatchTST performs comparatively better. Extrapolating from there, we realize that \name excels in datasets with a high multivariate nature, even in full budget scenarios, and also yields good and competitive performance in datasets with comparatively low multivariate characteristic in full budget scenarios. With limited budgets, we see that \name is the leading model among the baselines.  

\subsection{Univariate Forecasting}
Although \name is specifically designed for multivariate time series forecasting, we also assessed its performance in zero-shot univariate forecasting, compared to similar models. See Table 6 for more details. On average, \name-36 is the most successful model among other models, and uniformly better than all other deep-learning based baselines in our setting. Generally, Chronos-small \cite{ansari2024chronos} outperforms \name-36 with shorter prediction lengths, while TimePFN-36 excels at longer prediction lengths, outperforming the other models. This outcome is expected, as \name is specifically trained to handle an input length of 96 and predict the same distance ahead. For these comparisons, we trimmed \name's predictions to match the given prediction lengths. Given \name's focus on longer prediction horizons, it's no surprise that Chronos-small performs better at shorter lengths. For \name-36, we padded the first 60 sequences of the 96-sequence input with the average of a 36-sequence input to minimize distribution shift. We also included results for \name-96, which uses the full 96-sequence input length without padding, to demonstrate our model’s complete performance. 















\section{Comparative Analysis Against ValueLex \citep{biedma2024beyond}}

\label{app:against_bhn}

\citet{biedma2024beyond} proposed a lexical approach to constructing value systems for LLMs. While their work offers valuable contributions, we believe that core aspects of their approach could benefit from further theoretical and empirical development. In the following, we present a detailed comparative analysis of \our{} in relation to their work.

\subsection{Lexicon Collection}

\citet{biedma2024beyond} employ direct prompts such as "List the words that most accurately represent your value system" to extract value lexicons. However, direct questioning may not fully capture an LLM's complete spectrum of values. This work, in contrast, uses indirect, contextually guided questions to elicit a more comprehensive expression of values from LLMs.

In human psychology, self-reported values can be incomplete or skewed due to 1) social desirability bias \cite{randall1993social, larson2019controlling}, the tendency for people to self-report values that they believe are more socially acceptable; and 2) unconscious or implicit values \cite{greenwald1995implicit, hofer2006congruence}, where individuals may not recognize some deeply rooted values until certain situations bring them into play. LLM literature also reveals the unreliability of self-report results \cite{dominguez2023questioning, rottger2024political, ye2025gpv}.


Empirically, we examine all the collected lexicons in \cite{biedma2024beyond} and find that, exemplified using Schwartz's values, three values are missing: Achievement, Self-Direction, and Hedonism, according to the embedding similarity criteria established in \cite{sorensen2024value} (cosine similarity < 0.53). Using our indirect, contextually guided questions, we can capture these values, as shown in the following elicited LLM perceptions.

\begin{itemize}[leftmargin=2em]
    \item Achievement:
    \begin{itemize}
        \item Encouragement to take a challenging course for long-term goals and career development.
        \item Recognition of the importance of personal growth for professional and personal success.
        \item Emphasis on evaluating personal skills and experience for career development.
    \end{itemize}
    \item Self-Direction:
    \begin{itemize}
        \item The importance of aligning decisions with personal values and causes.
        \item Desire to take on the project independently and communicate openly with the manager.
    \end{itemize}
    \item Hedonism:
    \begin{itemize}
        \item Consideration of lifestyle enjoyment in the new city.
        \item Belief in following one's heart and pursuing joyful projects.
        \item The belief that art should bring joy rather than financial stress.
    \end{itemize}
\end{itemize}

\subsection{Computing Value Correlations}
The structure of a value system is derived from correlations between different values. In psychology research, researchers measure the value hierarchies of the participants to gather data, then use correlation derived from the data to evaluate interrelationships among these values. High positive correlations indicate values that are often endorsed together, while negative correlations show contrasting values. From these correlations, researchers can map and cluster values, revealing underlying value structures and hidden value factors (\cref{sec:approach}, Value Measurement and System Construction).

The method proposed by \citet{biedma2024beyond} derives correlations based on the co-occurrence of value lexicons in LLM self-reports in response to direct prompts. However, this approach lacks a theoretical foundation in psychology. We hypothesize that the co-occurrence of value lexicons in LLM responses does not necessarily indicate a true correlation between values. To test this hypothesis, we examine the correlation derived from their method for Schwartz's values.

\paragraph{Experimental Setup.}
We pair each of the collected value lexicons in \cite{biedma2024beyond} with the most semantically similar Schwartz's value according to the embedding model \cite{openai2024textembedding3large}. Then, we can compute the correlation between Schwartz's values using the normalized co-occurrence frequency of the original lexicons, following the method in \cite{biedma2024beyond}.

\paragraph{Results.}
\cref{fig:heatmap} illustrates the correlation heatmap of Schwartz's values based on the co-occurrence frequency of the lexicons. The values are ordered along the x and y axes according to Schwartz's circular structure. Except for the two values of Self-Enhancement, correlations between values generally contradict the theoretical structure of Schwartz's values. The results suggest that the co-occurrence frequency of lexicons in responses to direct prompts does not necessarily reflect the true value correlation. Value measurements using GPV are more theoretically grounded and empirically validated \cite{ye2025gpv}.

\begin{figure}[h]
    \centering
    \includegraphics[width=0.7\textwidth]{figures/heatmap_freq.pdf}
    \caption{Correlation heatmap derived from lexicon co-occurrence frequency \cite{biedma2024beyond}, after Min-Max normalization.}
    \label{fig:heatmap}
\end{figure}


\subsection{LLM Subjects}

\citet{biedma2024beyond} tune the generation hyperparameters (e.g., temperature, top-p) for different LLMs, attempting to generate diverse LLM subjects. However, simply tuning the generation hyperparameters is not sufficient to ensure the diversity and coverage of the LLM subjects, as they are not effective in steering the LLMs toward certain value orientations \cite{rozen2024llms}. In this work, we employ the validated value-anchoring prompts \cite{rozen2024llms} to steer the LLMs toward specific value orientations, ensuring the best possible coverage of the value spectrum and the practical relevance of our measurement results, since steering LLMs toward certain roles is common in public-facing applications nowadays.






\end{document}