\section{Background and Related Work}

\subsection{Values, Value Measurement, and Theory-Driven Value Systems}

\paragraph{Values.}
Values are broad, fundamental beliefs or principles that guide individuals' behaviors, cognition, and decisions \cite{schwartz1987toward}. They represent what people consider important in life and serve as motivational forces when evaluating actions, people, and events \cite{schwartz2012overview}. Unlike other psychological constructs such as personalities, values are stable, motivational, and transituational \cite{sagiv2022personal}. These characteristics make values uniquely powerful for understanding human behaviors \cite{bardi2003values}, evaluating individual and cultural differences \cite{schwartz2006theory}, and resolving conflicts rooted in clashing priorities \cite{sen1986social}. 

\paragraph{Value Measurement.}
Values are measurable constructs \cite{schwartz2001extending}. Value measurement seeks a quantitative assessment of the importance individuals or organizations assign to specific values. Traditionally, it relies on tools like self-report questionnaires \cite{schwartz2001extending, maio2010mental}, behavioral observation \cite{lonnqvist2013personal, maio2009changing, fischer2011whence}, and experimental techniques \cite{sagiv2011compete, yamagishi2013behavioral, murphy2014social}. However, these tools are hindered by response biases, resource demands, inaccuracies in capturing authentic behaviors, and inability to handle historical or subjective data \cite{ponizovskiy2020development, boyd2015values, bardi2008new}. Recently proposed Generative Psychometrics for Values (GPV) \cite{ye2025gpv} leverage LLMs to dynamically parse unstructured texts into perceptions and measure values therein. GPV is shown to be more scalable, reliable, and flexible than traditional tools in measuring LLM values, and is utilized for non-reactive value measurement in this work.

\paragraph{Theory-Driven Value Systems.}
Isolated values are structured into a value system for a holistic understanding of their relationships and implications \cite{schwartz2012overview}. Theory-driven value systems like Schwartz's theory of basic human values \cite{schwartz1992universals} and functional theory of values \cite{gouveia2014functional} are rooted in theoretical hypotheses and preconceptions of psychologists. These systems are established prior to data collection and analysis, and are later validated through empirical studies. However, theory-driven approaches are constrained by the subjectivity of the theorists, limited coverage, and an inability to adapt to evolving values or specific contexts \cite{ponizovskiy2020development, raad2017reply}.

\subsection{Psycho-Lexical Approach}
In contrast to the theory-driven approaches, the psycho-lexical approach organizes psychological constructs such as values, personality traits, and social attitudes \cite{aavik2002structure, crectu2012psycho, Klages1929-KLATSO-5, saucier2000isms}, under a data-driven paradigm. It operates on the fundamental principle that language naturally evolves to capture salient and socially relevant individual differences \cite{de2016values}.
The pioneering attempts date back to the early 20th century \cite{allport1936trait, galton1950character}, and the paradigm has then been developed and extended to different psychological constructs
for decades \cite{cattell1957personality, john1988lexical, aavik2002structure,ponizovskiy2020development, mai2023exploring, garrashi2024personality}.

Most of these works, taking values as an example, involve the following steps: 1) compilation of value-descriptive terms, mostly from dictionaries and thesauruses; 2) refinement and reduction to eliminate synonyms, ambiguous terms, and those that are not commonly used or understood; 3) value measurement, through collecting self-reports and peer ratings, to identify underlying correlations between the value descriptors; and 4) uncovering the hidden value factors or dimensions through statistical methods like principal component analysis or factor analysis.

In this work, we aim to harness the extensive knowledge and semantic understanding of LLMs to address the limitations of traditional psycho-lexical approaches. These limitations include: 1) extensive manual labor required to compile and refine the lexicons, 2) insufficient coverage of values of different linguistic forms, 3) lack of criteria for prioritizing values lexicons when filtering, 4) the responses bias and inscalability of self-report value measurement, and 5) the inability to adapt to changing values or specific contexts.


\subsection{From Human Values to LLM Values}
\label{sec:human_to_llm_values}
The rise of LLMs also brings the critical need to evaluate, understand, and align their values. Extensive works focus on evaluating the values of LLMs using traditional self-report tools \cite{li2022gpt,li2024evaluatingpsychologicalsafetylarge, huang2024humanity, safdari2023-personality-traits-in-llm, safdari2023personality}, static customized inventories \cite{ren2024valuebench,meadows2024localvaluebench, jiang2024evaluating}, dynamically generated inventories \cite{jiang2024raising}, or directly from the model's free-form outputs \cite{ye2025gpv, yao2024clave,yao2023value_fulcra, yao2025leaderboard}.
Other works attempt to understand the LLM values in aspects like the consistency of shown values \cite{rozen2024llms, moore2024large,rottger2024political}, the ability to reason about human values \cite{ganesan2023systematic,sorensen2024value, jiang2024can, strachan2024testing}, and the ability to express or role-play human values \cite{jiang-etal-2024-personallm, zhang2023measuring, kang2024causal, zhang2025extrapolating, li2024quantifying, huang2024social}.
Another line of research aligns LLM values with human values, setting the alignment goal as specific risk metrics \cite{lin2024towards, gunjal2024detecting}, human demonstrations \cite{dubois2024alpacafarm, kopf2024openassistant, alpaca}, implicit preference modeling \cite{ouyang2022training, rafailov2024dpo, zhong2024panacea}, or intrinsic values \cite{bai2022hh, yao2023value_fulcra, bai2022constitutional}. Among different alignment goals, intrinsic values, as transituational decision-guiding principles, demonstrate unique advantages \cite{yao2023alignment_goals}. However, prior works are mostly based on Schwartz's value system, which is established and validated for humans and may not necessarily capture the unique LLM psychology. Therefore, it is essential to construct a value system for LLMs grounded in psychological theories and principles, based on which we can more effectively evaluate, understand, and align the values of LLMs.

One preliminary attempt \cite{biedma2024beyond} of constructing an LLM value system faces several limitations that undermine its psychological grounding: 1) the value lexicon collection is prone to response bias and lacks comprehensive coverage; 2) the derivation of value correlations is not theoretically grounded and lacks empirical validity; and 3) the resulting value system has not been evaluated in terms of its reliability, validity, or utility. \cref{app:against_bhn} presents further discussion and experimental evidence.
