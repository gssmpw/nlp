\section{Results}\label{sec:experiments}

This section presents the results of our experiments, including the proposed LLM value system, the analysis of the system, and the benchmarking of our system against the well-established Schwartz's values \cite{schwartz2012overview}.


\subsection{Proposed Value System}
\cref{fig:dendrogram} visualizes the value system constructed by \our{}.
\cref{tab:factor loadings} gathers its factor loadings, Cronbach's Alpha, and confidence intervals. The factor loadings indicate the strength of the relationship between each factor and its atomic values, with higher loadings suggesting more contribution to the factor. Cronbach's Alpha measures the internal consistency of each factor, with higher values indicating greater reliability. 
Some atomic values are removed to ensure clear loading patterns and desirable factor reliability \cite{aavik2002structure}. After that, all our factors reach the standard psychometric threshold of 0.7, indicating strong internal consistency \cite{taber2018use}.

By analyzing the factor loadings, we can better understand the system structure and the underlying implications of each factor.

\paragraph{Factor 1: Social Responsibility.}
This factor reflects values centered on collective well-being and ethical social engagement. The high loadings on Equity (0.890), Empathy (0.885), and Teamwork (0.872) highlight its emphasis on fairness and collaboration. Values such as Equality (0.827) and Unity (0.825) indicate a strong focus on inclusivity. Public Benefit (0.820) and Democracy (0.813) support the broader societal perspective and prioritize the common good.

\paragraph{Factor 2: Risk-Taking.}
This factor embodies a preference for dynamism, exploration, and adaptability. High loadings for Challenge (0.812), Boldness (0.804), and Adventure (0.798) illustrate a willingness to confront uncertainty and seek new experiences. Values such as Change (0.730), Flexibility (0.721), and negatively loaded Stability (-0.701) underscore openness to transformation, while Thrill-seeking (0.728) further conveys a desire for excitement.

\paragraph{Factor 3: Rule-Following.}
This factor prioritizes order, discipline, and dependability. Strong loadings for Realism (0.708), Order (0.694), Responsibility (0.682), and Reliability (0.598) reflect a grounded, pragmatic, and conscientious approach to life. Values such as Prudence (0.676), Efficiency (0.669), and Timeliness (0.658) emphasize structured and deliberate actions.

\paragraph{Factor 4: Self-Competence.}
This factor represents personal growth and self-efficacy. Loadings for Confidence (0.601), Impact (0.527), and Proactivity (0.460) indicate a focus on self-assurance and initiative. Values such as Achievement (0.455), Recognition (0.455), and Excellence (0.455) highlight aspirations for acknowledgment and high performance.

\paragraph{Factor 5: Rationality.}
This factor centers on logical and evidence-based decision-making. Loadings for Objectivity (0.705) and Evidence-based (0.649) demonstrate a preference for impartiality and reliance on empirical data; Logic (0.618) and Neutrality (0.522) further reinforce this analytic perspective.
 

\begin{table}[h!]
    \centering
    \caption{Factor loadings and Cronbach's Alpha for our value system. CI denotes the 95\% confidence interval. Full results are available in \cref{app:pca}.}
    \begin{tabular}{llcc}
        \toprule
        Factor & Value & Loading & Cronbach's Alpha (CI) \\
        \midrule
        \multirow{8}{*}{Social Responsibility}
                                  & Equity        & 0.890 & \multirow{8}{*}{0.957 (0.952 -- 0.961)}\\
                                  & Empathy       & 0.885 & \\
                                  & Teamwork     & 0.872 & \\
                                  & Equality     & 0.827 & \\
                                  & Unity        & 0.825 & \\
                                  & Public Benefit  & 0.820 & \\
                                  & Democracy  & 0.813 & \\
                                  \multicolumn{4}{c}{\textit{...}} \\
        \midrule
        \multirow{8}{*}{Risk-Taking} 
                                  & Challenge      & 0.812 & \multirow{8}{*}{0.919 (0.910 -- 0.928)} \\
                                  & Boldness     & 0.804 & \\
                                  & Adventure     & 0.798 & \\
                                  & Change   & 0.730 & \\
                                  & Thrill-seeking & 0.728 & \\
                                  & Flexibility & 0.721 & \\
                                  & Stability  & -0.701 & \\
                                  \multicolumn{4}{c}{\textit{...}} \\
        \midrule
        \multirow{8}{*}{Rule-Following}
                                  &  Realism    & 0.708 & \multirow{8}{*}{0.842 (0.824 -- 0.859)} \\
                                  &  Order      & 0.694 & \\
                                  &  Responsibility  & 0.682 & \\
                                  &  Prudence  & 0.676 & \\
                                  &  Efficiency  & 0.669 & \\
                                  &  Timeliness  & 0.658 & \\
                                  &  Reliability  & 0.598 & \\
                                  \multicolumn{4}{c}{\textit{...}} \\
        \midrule
        \multirow{6}{*}{Self-Competence}
                                  & Confidence   & 0.601 & \multirow{6}{*}{0.761 (0.732 -- 0.787)} \\
                                  & Impact       & 0.527 & \\
                                  & Proactivity  & 0.460 & \\
                                  & Achievement  & 0.455 & \\
                                  & Recognition  & 0.455 & \\
                                  & Excellence   & 0.455 & \\
        \midrule
        \multirow{4}{*}{Rationality}
                                  & Objectivity & 0.705 & \multirow{4}{*}{0.722 (0.686 -- 0.754)} \\
                                  & Evidence-based & 0.649 & \\
                                  & Logic & 0.618 & \\
                                  & Neutrality & 0.522 & \\
        \bottomrule
    \end{tabular}
    \label{tab:factor loadings}
\end{table}



\begin{figure*}[htbp]
    \begin{floatrow}
        \ffigbox[\FBwidth]
        {\includegraphics[width=0.62\textwidth]{figures/system_dendrogram.pdf}}
        {\caption{Dendrogram of our value system. Values with a "*" are negatively loaded.} \label{fig:dendrogram}}
        \killfloatstyle
        \ffigbox[\FBwidth]
        {%
            \begin{tabular}{@{}c@{}}
                \includegraphics[width=0.34\textwidth]{figures/correlation_heatmap_our_system.pdf}
                 \\\vspace{-6mm}
                \includegraphics[width=0.34\textwidth]{figures/circumplex_analysis.pdf} \\[-0.5em]
                \caption{Correlation heatmap and circumplex analysis.}\label{fig:corr-circ}
            \end{tabular}
        }
        {}
    \end{floatrow}
\end{figure*}


\subsection{Analyzing Value System}

\paragraph{Value Correlation Analysis.}
\cref{fig:corr-circ} (Top) presents the correlations between the factors in our value system. Similar to Schwartz's theory of basic human values \cite{schwartz2012overview}, LLM values also exhibit both compatible and opposing relationships. Notably, social responsibility, rule-following, and rationality show positive correlations with one another, while all of them are negatively correlated with risk-taking.

\paragraph{Circumplex Analysis.}
Circumplex analysis is a statistical method that examines whether the underlying structure of variables aligns with a circumplex pattern, and, if so, the positions of variables on a circle. The stronger the correlation between variables, the shorter their distance on the circumference.
We conduct circumplex analysis based on the correlations between factors. \cref{fig:corr-circ} (Bottom) illustrates the analysis results based on Browne's circular stochastic process model \cite{browne1992circumplex, grassi2010circe}. The compatible values are closer on the circle (e.g., Social Responsibility and Rule-Following) while opposing values are positioned diagonally (e.g., Risk-Taking and Rule-Following). The results verify the presence of a circumplex structure in the value system.

\paragraph{Consistency Across Datasets.} To evaluate the consistency and robustness of our multivariate system structure (i.e., the 5-dimensional relationship), we measure LLM values using two distinct prompt datasets: the psychometric prompts from GPV \cite{ye2025gpv} and the red-teaming prompts from SALAD-Bench \cite{li2024salad}, which feature distinct prompt distributions. Their measurements yield an average intra-LLM correlation of 0.87; here we use intra-LLM correlations because the relative value hierarchy within an LLM is more important than their absolute measurements. This result indicates a high level of consistency in the value structure across prompt distributions. We also find a high correlation (0.73) between intra-LLM value consistency and LLM safety scores \cite{li2024salad}. It suggests that LLMs with higher value consistency tend to be safer. Complete results are available in \cref{app:consistency}.


\subsection{Comparing Value Systems}
\label{sec:comparing value systems}

We benchmark the proposed value system against Schwartz's value system \cite{schwartz2012overview}, the most established framework for human values and commonly used in LLM value studies.

\paragraph{Confirmatory Factor Analysis for Evaluating Structure Validity.}
We follow the standard validation procedures of CFA \cite{schwartz2004cfa} to evaluate the structure validity of different value systems. For our value system, we construct it using half of the measurement data and bootstrap the data to ensure its sufficiency (\#data points \( \ge 5 \times \) \#variables). The other half of the data is held out for CFA. For Schwartz's value system, we map the observed value variables (the atomic values in our system) to its four high-level values or ten low-level values, according to the semantic relevance \cite{schwartz2004cfa} measured by an embedding model \cite{openai2024textembedding3large}; all data is used for CFA. \cref{tab:cfa results} displays the CFA results. Our value system demonstrates a better fit for the data, capturing the underlying values behind LLM generations.

\begin{table}[H]
    \centering
    \begin{tabular}{c|c|ccccc}
        \toprule
        Value system & \#Values & CFI \( \uparrow \) & GFI \( \uparrow \) & RMSEA \( \downarrow \) & AIC \( \downarrow \) & BIC \( \downarrow \) \\
        \midrule
        Schwartz (H) & 4 & 0.56 & 0.52 & \textbf{0.10} & 340 & 1484 \\
        Schwartz (L) & 10 & 0.23 & 0.22 & 0.11 & 324 & 1464 \\
        Ours & 5 & \textbf{0.68} & \textbf{0.65} & 0.12 & \textbf{265} & \textbf{1145} \\
        \bottomrule
    \end{tabular}
    \caption{CFA results of different value systems. H and L denote high and low-level Schwartz values, respectively.}
    \label{tab:cfa results}
\end{table}
    


\paragraph{LLM Safety Prediction for Evaluating Predictive Validity.}
We follow the experimental setup in \cite[Section 5.2]{ye2025gpv} to predict LLM safety based on their value orientations. \cref{tab:llm safety pred acc} presents the prediction accuracy under different value systems. The higher accuracy of our value system indicates its superior predictive validity.
In addition, according to the parameters of well-trained linear classifiers, we find that Social Responsibility, Rule-Following, and Rationality enhance safety, whereas Risk-taking and Self-Competence undermine it; see \cref{app:safety_prediction} for details.


\begin{wraptable}[9]{r}{0.36\linewidth}
        \centering
        \vspace{-4mm}
        \begin{tabular}{c|c}
            \toprule
            Value system & Acc (\%) \\
            \midrule
            Schwartz (H) & 81\tiny{$\pm$ 15} \\
            Schwartz (L) & 74\tiny{$\pm$ 16 } \\
            Ours & \textbf{87}\tiny{$\pm$ 9 } \\
            \bottomrule
        \end{tabular}
        \caption{Accuracy of LLM safety prediction based on values.}
        \label{tab:llm safety pred acc}
    \end{wraptable}


\paragraph{LLM Value Alignment for Evaluating Representation Power.}
We follow the experimental setup in \cite[Section 6.2]{yao2023value_fulcra} to perform LLM value alignment. Different value systems are respectively used to represent LLM outputs and desired human values. We employ GPV \cite{ye2025gpv} as an open-vocabulary value evaluator for all value systems, but also include the original results of \cite{yao2023value_fulcra} using its Schwartz-specific evaluator for comparison. \cref{tab:llm value alignment} shows the alignment performance of different value systems. Alignment under our value system converges to the lowest harmlessness and the highest helpfulness, establishing its superior representation power. Full experimental details are available in \cref{app:llm value alignment}.


\begin{table}[H]
    \centering
    \begin{tabular}{c|cc}
        \toprule
        Value system
        & Harmlessness
        & Helpfulness
         \\
        \midrule
        Schwartz* \cite{yao2023value_fulcra} & -1.52 & 2.15 \\
        Schwartz & -1.40 & 2.13 \\
        Ours & \textbf{-1.26} & \textbf{2.16} \\
        \bottomrule
    \end{tabular}
    \caption{Alignment performance of different value systems. Schwartz* denotes the original results drawn from \cite{yao2023value_fulcra} using a Schwartz-specific value evaluator. Both Schwartz baselines are based on the 10-dimensional Schwartz values \cite{yao2023value_fulcra}.}
    \label{tab:llm value alignment}
\end{table}

