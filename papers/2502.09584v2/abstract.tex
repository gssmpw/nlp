\begin{abstract}
We initiate the study of differentially private data-compression schemes motivated by the insecurity of the popular ``Compress-Then-Encrypt'' framework. Data compression is a useful tool which exploits redundancy in data to reduce storage/bandwidth when files are stored or transmitted. However, if the contents of a file are confidential then the \emph{length} of a compressed file might leak confidential information about the content of the file itself. Encrypting a compressed file does not eliminate this leakage as data encryption schemes are only designed to hide the \emph{content} of confidential message instead of the \emph{length} of the message. In our proposed \emph{Differentially Private Compress-Then-Encrypt} framework, we add a random positive amount of padding to the compressed file to ensure that any leakage satisfies the rigorous privacy guarantee of $(\epsilon,\delta)$-differential privacy. The amount of padding that needs to be added depends on the sensitivity of the compression scheme to small changes in the input, i.e., to what degree can changing a single character of the input message impact the length of the compressed file. While some popular compression schemes are highly sensitive to small changes in the input, we argue that effective data compression schemes do not necessarily have high sensitivity. Our primary technical contribution is analyzing the fine-grained sensitivity of the LZ77 compression scheme (IEEE Trans. Inf. Theory 1977) which is one of the most common compression schemes used in practice. We show that the global sensitivity of the LZ77 compression scheme has the upper bound $\mathcal{O}(W^{2/3}\log n)$ where $W\leq n$ denotes the size of the sliding window. When $W=n$, we show the lower bound $\Omega(n^{2/3}\log^{1/3}n)$ for the global sensitivity of the LZ77 compression scheme which is tight up to a sublogarithmic factor.


%    Data compression enables the reduction of the size of data representations while preserving the essential information. Lossless compression schemes, which do not lose any information about input upon compression, identify and eliminate statistical redundancy of input. This could become problematic in the Compress-Then-Encrypt framework as encryption schemes often leak the length of compressed message and the attacker could gain useful information about the input by looking at the compression length.

 %   In this paper, we consider the problem of designing compression schemes in a differentially private way. In particular, we present an algorithm that translates any compression algorithm to another compression algorithm that is differentially private. Along the way, we initiate the study of analyzing the sensitivity of compression schemes which is a crucial element of differentially private compression schemes. In particular, we analyze the fine-grained sensitivity of the LZ77 compression scheme (IEEE Trans. Inf. Theory 1977) and show that for strings of length $n$, the global sensitivity of the LZ77 compression scheme has the upper bound $\mathcal{O}(n^{2/3}\log n)$ and the lower bound $\Omega(n^{2/3}\log^{1/3}n)$ which are tight up to a sublogarithmic factor.
    
    
    % It is important to analyze the sensitivity of compression algorithms as understanding the sensitivity of compression schemes plays a crucial role in designing differentially private compression algorithms.
    
    % In this work, we initiate the study of analyzing the sensitivity of compression schemes and introduce the first differentially private compression scheme framework that translates any compression scheme to a differentially private one. In particular, we analyze the fine-grained sensitivity of the LZ77 compression scheme (IEEE Trans. Inf. Theory 1977) and show that for strings of length $n$, the global sensitivity of the LZ77 compression scheme has the upper bound $\mathcal{O}(n^{2/3}\log n)$ and the lower bound $\Omega(n^{2/3}\log^{1/3}n)$ which are tight up to a sublogarithmic factor.

   % On the other hand, we also show that good compression schemes do not necessarily have high sensitivity. We present a computable variant of the Kolmogorov compression scheme and show that the scheme is both efficient and has low global sensitivity $\mathcal{O}(\log n)$.
    
    %  Lagarde and Perifel (SODA 2018) proved that the LZ78 compression scheme (IEEE Trans. Inf. Theory 1978) has high sensitivity, but it is unknown whether a good compression scheme necessarily has high sensitivity in general.

    % In this work, we dispute this conjecture by providing a counterexample: a computable variant of Kolmogorov compression that is efficient and has low sensitivity. We further analyze the sensitivity of a more practical compression scheme and present the upper and lower bound of the sensitivity of LZ77 compression scheme (IEEE Trans. Inf. Theory 1977) which are tight up to sublogarithmic factor. We also initiate the study of differentially private compression scheme and provide the first construction that translates any compression scheme to a differentially private compression scheme.
\end{abstract}