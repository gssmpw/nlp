% !TEX root = main.tex
\usepackage[
	n,
	operators,
	advantage,
	sets,
	adversary,
	landau,
	probability,
	notions,	
	logic,
	ff,
	mm,
	primitives,
	events,
	complexity,
	asymptotics,
	keys]{cryptocode}
	
% \let\proof\relax
% \let\endproof\relax
% \usepackage{amsmath,amssymb,amsfonts,amsthm}

% \usepackage[hyphens,spaces,obeyspaces]{url}
% \urldef{\footurl}\url{https://engineering.fb.com/2014/04/10/core-data/scaling-the-facebook-data-warehouse-to-300-pb/}

% \usepackage[final]{changes}
% Below is to fix weird strikeouts within the math mode
% \newcommand{\stkout}[1]{\ifmmode\text{\sout{\ensuremath{#1}}}\else\sout{#1}\fi}
% \setdeletedmarkup{\stkout{#1}}

\let\comment\undefined
\usepackage{skak}
\usepackage{thm-restate}

\usepackage{makeidx}
% \usepackage{euscript}

\usepackage{dsfont}
\usepackage{latexsym}
%\usepackage{subfigure}
\usepackage{graphicx}
%\usepackage{fancybox}
%\usepackage[margin=1in]{geometry}
\usepackage[breakable,skins]{tcolorbox}

\usepackage{todonotes}
\setuptodonotes{size=\scriptsize}

% \usepackage[breaklinks=true]{hyperref} % add option [backref] if you want to add hyperlink to the main body in the reference list
% \usepackage{paralist}
% \usepackage{xcolor}
%\usepackage{wrapfig}
\usepackage{tikz}
\usetikzlibrary{shapes.callouts, decorations.pathreplacing, patterns, patterns.meta}
%\usepackage{setspace}
\usepackage{algorithm}
\usepackage[noend]{algpseudocode}
\usepackage[framemethod=tikz]{mdframed}
\usepackage{xspace}
\usepackage{pgfplots}
\usepackage{framed}
% \usepackage{enumitem}
\pgfplotsset{compat=1.5}
\usepackage{standalone}

\usepackage{breakcites}

\usepackage{mathtools}
\DeclarePairedDelimiter\pars{\lparen}{\rparen}

%\newenvironment{proof}{\noindent{\bf Proof : \ }}{\hfill$\Box$\par\medskip}

% \newenvironment{proofhack}{\noindent {\bf Proof}.\ }{}
% \newtheorem{theorem}{Theorem}[section]
% \newtheorem{corollary}[theorem]{Corollary}
% \newtheorem{lemma}[theorem]{Lemma}
% \newtheorem{proposition}[theorem]{Proposition}
% \newtheorem{definition}[theorem]{Definition}
% \newtheorem{claim}[theorem]{Claim}
% \newtheorem{remark}[theorem]{Remark}
% \newtheorem{example}[theorem]{Example}
% \newtheorem{conjecture}[theorem]{Conjecture}
% \newtheorem{problem}[theorem]{Problem}

\DeclareMathAlphabet{\mathpzc}{OT1}{pzc}{m}{it}

\newcommand{\qedmath}{\tag*{$\square$}}
\newcommand{\qedremark}{\hfill $\triangleleft$}
\newcommand{\qedremarkmath}{\tag*{$\triangleleft$}}

\newenvironment{proofof}[1]{\begin{trivlist} \item {\bf Proof
#1:~~}}
  {\qed\end{trivlist}}
\renewenvironment{proofof}[1]{\par\medskip\noindent{\color{lipicsGray}\sffamily\bfseries Proof of #1. \ }}{\qed\par\medskip}
\newenvironment{proofsketch}[1]{\begin{trivlist} \item {\it Proof
#1:~~}}
  {\qed\end{trivlist}}
\renewenvironment{proofsketch}[1]{\par\medskip\noindent{\it Proof Sketch of #1: \ }}{\qed\par\medskip}
\newcommand{\namedref}[2]{\hyperref[#2]{#1~\ref*{#2}}}
\newcommand{\eqnamedref}[2]{\hyperref[#2]{#1~(\ref*{#2})}}
%\newcommand{\namedref}[2]{#1~\ref{#2}}
\newcommand{\thmlab}[1]{\label{thm:#1}}
\newcommand{\thmref}[1]{\namedref{Theorem}{thm:#1}}
\newcommand{\lemlab}[1]{\label{lem:#1}}
\newcommand{\lemref}[1]{\namedref{Lemma}{lem:#1}}
\newcommand{\claimlab}[1]{\label{claim:#1}}
\newcommand{\claimref}[1]{\namedref{Claim}{claim:#1}}
\newcommand{\corlab}[1]{\label{cor:#1}}
\newcommand{\corref}[1]{\namedref{Corollary}{cor:#1}}
\newcommand{\seclab}[1]{\label{sec:#1}}
\newcommand{\secref}[1]{\namedref{Section}{sec:#1}}
\newcommand{\applab}[1]{\label{app:#1}}
\newcommand{\appref}[1]{\namedref{Appendix}{app:#1}}
\newcommand{\factlab}[1]{\label{fact:#1}}
\newcommand{\factref}[1]{\namedref{Fact}{fact:#1}}
\newcommand{\remlab}[1]{\label{rem:#1}}
\newcommand{\remref}[1]{\namedref{Remark}{rem:#1}}
\newcommand{\figlab}[1]{\label{fig:#1}}
\newcommand{\figref}[1]{\namedref{Figure}{fig:#1}}
\newcommand{\alglab}[1]{\label{alg:#1}}
\renewcommand{\algref}[1]{\namedref{Algorithm}{alg:#1}}
\newcommand{\tablelab}[1]{\label{tab:#1}}
\newcommand{\tableref}[1]{\namedref{Table}{tab:#1}}
\newcommand{\deflab}[1]{\label{def:#1}}
\newcommand{\defref}[1]{\namedref{Definition}{def:#1}}
\newcommand{\eqnlab}[1]{\label{eq:#1}}
\newcommand{\eqnref}[1]{\eqnamedref{Equation}{eq:#1}}
\newcommand{\constref}[1]{\namedref{Construction}{const:#1}}
\newcommand{\constlab}[1]{\label{const:#1}}
\newcommand{\obslab}[1]{\label{obs:#1}}
\newcommand{\obsref}[1]{\namedref{Observation}{obs:#1}}
\newcommand{\exlab}[1]{\label{ex:#1}}
\newcommand{\exref}[1]{\namedref{Example}{ex:#1}}
\newcommand{\stepref}[1]{\namedref{Step}{step:#1}}
\newcommand{\steplab}[1]{\label{step:#1}}
\newcommand{\propref}[1]{\namedref{Proposition}{prop:#1}}
\newcommand{\proplab}[1]{\label{prop:#1}}
\newcommand{\probref}[1]{\namedref{Problem}{prob:#1}}
\newcommand{\problab}[1]{\label{prob:#1}}

\newenvironment{remindertheorem}[1]{\medskip \noindent {\textcolor{lipicsGray}{$\blacktriangleright$}\nobreakspace\sffamily\bfseries Reminder of #1.}\em}{}
\newenvironment{reminderclaim}[1]{\medskip \noindent {\textcolor{black}{$\vartriangleright$}\nobreakspace\sffamily Reminder of #1.}\em}{}
% macros for the new paper
\usepackage{skak}
\newcommand{\lcompress}{\mathtt{LCompress}}
\newcommand{\exval}{\mathbb{E}}
\newcommand{\compress}{\mathtt{Compress}}
\newcommand{\kc}{\mathtt{KC}}
\newcommand{\ckc}{\mathtt{CKC}}
\newcommand{\score}{\mathsf{Score}}
\newcommand{\dpcompress}{\mathtt{DPCompress}}
\newcommand{\ldpcompress}{\mathtt{LDPCompress}}
\newcommand{\lapalg}{\mathcal{T}}
\newcommand{\decompress}{\mathtt{Decompress}}
%\newcommand{\pad}{\mathtt{Pad}}
\newcommand{\unpad}{\mathtt{Unpad}}
\newcommand{\startinside}{\mathsf{StartInside}}
\newcommand{\counter}{\mathsf{ct}}
\newcommand{\length}{\mathsf{Length}}
\newcommand{\B}{\mathcal{B}}
\newcommand{\M}{\mathcal{M}}
\newcommand{\BAD}{\mathsf{BAD}}
\newcommand{\GOOD}{\mathsf{GOOD}}
\newcommand{\GS}{\mathtt{GS}}   
\newcommand{\Lap}{\mathrm{Lap}}
\newcommand{\Ham}{\mathtt{Ham}}


\newcommand{\CT}{\mathsf{ct}}
\newcommand{\ctb}{\mathsf{ctb}}
\newcommand{\ctc}{\mathsf{ctc}}

\newcommand{\superscript}[1]{\ensuremath{^{\mbox{\scriptsize{\textit{#1}}}}}}
\def \th {\superscript{th}} 

\renewcommand{\S}{\mathcal{S}}

\newcommand{\Enc}{\mathsf{Enc}}
\newcommand{\QuinStr}{\mathsf{QuinStr}}
\newcommand{\str}{\mathsf{str}}

\renewcommand{\vec}{\mathbf}

\newcommand{\Reachable}{\mathsf{Reachable}}
\newcommand{\Overlay}{\mathsf{Overlay}}
\newcommand{\Hb}{\textup{H}_\textup{b}}

\def \rank  {\mdef{\mathrm{rank}}}


\def \A {\mdef{\mathcal{A}}}
\def \B {\mdef{\mathcal{B}}}
\def \D {\mdef{\mathcal{D}}}
\def \E {\mdef{\mathsf{E}}}
\def \F {\mdef{\mathcal{F}}}

%%%% pebbling complexity notions
\newcommand{\peb}{\Pi} % pebbling complexity
\newcommand{\Peb}{{\cal P}} % set of legal pebblings

%\newcommand{\parallel}{\|}

\newcommand{\ppeb}{\peb^{\parallel}} % parallel versions of def above
\newcommand{\pPeb}{\Peb^{\parallel}}

\newcommand{\Cspace}{s}
\newcommand{\Ctime}{t}
\newcommand{\Ccc}{{cc}}
\newcommand{\Cspacetime}{{st}}
\newcommand{\Cevery}{\{\Cspace,\Ctime,\Cspacetime,\Ccc\}}

%\def \pcc {\ppeb_\Ccc} % shortcut for Parallel Cumulative Complexity
\def \pcc {\cc} % shortcut for Parallel Cumulative Complexity
\def \sst {\peb_{\Cspacetime}} % shourtcut for Sequential Space Time


%---------------------------------------------------------------------------------------------------------------------------------------------------

%---------------------------------------------------------------------------------------------------------------------------------------------------
%\DeclarePairedDelimiterX{\norm}[1]{\lVert}{\rVert}{#1}
%%\newcommand\norm[1]{\left\lVert#1\right\rVert}
\newcommand{\PPr}[1]{\ensuremath{\mathbf{Pr}\left[#1\right]}}
\newcommand{\Ex}[1]{\ensuremath{\mathbb{E}\left[#1\right]}}
\newcommand{\EEx}[2]{\ensuremath{\underset{#1}{\mathbb{E}}\left[#2\right]}}
\renewcommand{\O}[1]{\ensuremath{\mathcal{O}\pars*{#1}}}
\newcommand{\eps}{\varepsilon}

\renewcommand{\labelenumi}{(\arabic{enumi})}
\renewcommand\labelitemii{\mbox{$\circ$}}
\renewcommand\labelitemiii{\mbox{\tiny$\spadesuit$}}
\renewcommand{\figurename}{Figure}
\renewcommand{\poly}{\mathsf{poly}}

%%% Unified Formating
\newcommand{\mdef}[1]{{\ensuremath{#1}}\xspace}  % Math Def which can also be used in normal text.
\newcommand{\mydist}[1]{\mdef{\mathcal{#1}}}     % Distribution should use mathcal.
\newcommand{\myset}[1]{\mdef{\mathbb{#1}}}       % (Important) Sets should use mathbb.
\newcommand{\myalg}[1]{\mdef{\mathcal{#1}}}       % Algorithms should use mathtt.
\newcommand{\myfunc}[1]{\mdef{\mathsf{#1}}}      % Functions denoted with text (e.g. size()) should use mathsf.
\newcommand{\myvec}[1]{\mathbf{#1}}               % Vector
\def \ds {\displaystyle}                         % Shorthand for ``displaystyle''.
%\DeclareMathOperator*{\argmin}{arg\,min}
%\DeclareMathOperator*{\argmax}{arg\,max}
\DeclareMathOperator*{\polylog}{polylog}

%%% Text notation
%\newcommand{\superscript}[1]{\ensuremath{^{\mbox{\tiny{\textit{#1}}}}}\xspace}
\def \th {\superscript{th}}     % 'The i-th entry it a list...' --> i\th
\def \st {\superscript{st}}     % 'The 1-st entry it a list...' --> 1\st
\def \nd {\superscript{nd}}     % 'The 2-nd entry it a list...' --> 2\nd
\def \rd {\superscript{rd}}     % 'The 3-rd entry it a list...' --> 3\rd
\def \etal{{\it et~al.}}

%%% Common Sets & Symbols & Functions
\def \size     {\mdef{\myfunc{size}}}                % Function outputing vector of cardinalities of a given vector of sets.
\def \negl     {\mdef{\myfunc{negl}}}                % Negliglbe function
\def \polylog  {\mdef{\myfunc{polylog}}}             % Polylogarithm function
\def \polyloglog  {\mdef{\myfunc{polyloglog}}}       % Any function that is polynomial in the logarithm of the logarithm of the argument
%%\newcommand{\abs}[1]{\mdef{\left|#1\right|}}         % Absolute value
\newcommand{\flr}[1]{\mdef{\left\lfloor#1\right\rfloor}}              % Absolute value
%%\newcommand{\ceil}[1]{\mdef{\left\lceil#1\right\rceil}}               % Absolute value
\newcommand{\rdim}[1]{\mdef{\dim\left(#1\right)}}                     % Dimension of a vector
%%\newcommand{\set}[1]{\mdef{\left\{#1\right\}}}                        % Absolute value
%\newcommand{\E}[2][]{\mdef{\underset{#1}{\mathbb{E}}\left[#2\right]}} % Expected value

\newcommand{\ignore}[1]{}

\newif\ifnotes\notestrue %set this to true if notes are visible and to false (next line) if they should be hidden
% \newif\ifnotes\notesfalse
\ifnotes
\newcommand{\samson}[1]{\textcolor{purple}{{\bf (Samson:} {#1}{\bf ) }} \marginpar{\tiny\bf
             \begin{minipage}[t]{0.5in}
               \raggedright S:
            \end{minipage}}}  
% \newcommand{\jeremiah}[1]{\textcolor{red}{{\bf (Jeremiah:} {#1}{\bf ) }} \marginpar{\tiny\bf
%              \begin{minipage}[t]{0.5in}
%                \raggedright J:
%             \end{minipage}}}
\newcommand{\jeremiah}[1]{
\todo[color=crimson!30]{\textbf{Jeremiah:} #1}
}
% \newcommand{\seunghoon}[1]{\textcolor{blue}{{\bf (Seunghoon:} {#1}{\bf ) }} \marginpar{\tiny\bf
%              \begin{minipage}[t]{0.5in}
%                \raggedright SL:
%             \end{minipage}}}
\newcommand{\seunghoon}[1]{
\todo[color=dodgerblue!30]{\textbf{Seunghoon:} #1}
}
\else
\newcommand{\samson}[1]{}
\newcommand{\jeremiah}[1]{}
\fi


% \hypersetup{
%      colorlinks   = true,
%      citecolor    = blue,
% 		 linkcolor		= red
% }

\makeatletter
\renewcommand*{\@fnsymbol}[1]{\textcolor{blue}{\ensuremath{\ifcase#1\or *\or \dagger\or \ddagger\or
 \mathsection\or \triangledown\or \mathparagraph\or \|\or **\or \dagger\dagger
   \or \ddagger\ddagger \else\@ctrerr\fi}}}
\makeatother

\providecommand{\email}[1]{\href{mailto:#1}{\nolinkurl{#1}\xspace}}

\definecolor{mahogany}{rgb}{0.75, 0.25, 0.0}
\definecolor{darkblue}{rgb}{0.0, 0.0, 0.55}
\definecolor{darkpastelgreen}{rgb}{0.01, 0.75, 0.24}
\definecolor{darkgreen}{rgb}{0.0, 0.2, 0.13}
\definecolor{darkgoldenrod}{rgb}{0.72, 0.53, 0.04}
\definecolor{forestgreen}{rgb}{0.13, 0.55, 0.13}
\definecolor{darkred}{rgb}{0.55, 0.0, 0.0}
\definecolor{blueviolet}{RGB}{138,43,226}
\definecolor{dodgerblue}{RGB}{30,144,255}
\definecolor{crimson}{RGB}{220,20,60}
\hypersetup{
    colorlinks  = true,
    citecolor   = dodgerblue,
	linkcolor	= crimson
}	