\section{Related Works}
\subsection{Advanced Retrieval-Augmented Generation}
Advanced retrieval-augmented generation (RAG) methods enhance document retrieval through pre-processing techniques (e.g., sliding window chunking, context-enriched chunking, small-to-big, parent-child chunking, and reverse HyDe chunking) \cite{anthropic_contextual_2024,setty_improving_2024,yang_advanced_2023}, intra-retrieval strategies (e.g., query rewriting, decomposition, and transformation) \cite{ma_query_2023,tang_multihop-rag_2024,trivedi_interleaving_2023,yao_react_2023,khattab_demonstrate-search-predict_2023,joshi_reaper_2024,xu_rewoo_2023,zheng_take_2024}, and post-retrieval approaches (e.g., reranking and corrective RAG) \cite{raudaschl_forget_2023,asai_self-rag_2023,yan_corrective_2024,sun_is_2023}. While Graph RAG is within the advanced RAG family, traditional RAG relies on vector databases, whereas Graph RAG uses knowledge graphs and vector retrieval \cite{gao2024retrievalaugmentedgenerationlargelanguage}. Retrieved documents from Graph RAG can also be concatenated with vector search results \cite{sarmah2024hybridragintegratingknowledgegraphs,raudaschl_forget_2023}. Two-pass document retrieval strategies like small-to-big and parent-child chunking \cite{langchain_parent_document_retriever_2023}, which map smaller chunks to larger chunks, require maintaining child-parent ID mappings and performing inefficient vector retrieval, whereas structured knowledge graphs naturally manage the indexing, mapping, and retrieval during inference-time graph traversal. Our Graph RAG-Tool Fusion is a novel plug-and-play approach extending Graph RAG from document retrieval to scalable tool selection for agents, leveraging advanced RAG strategies for the initial vector retrieval.

\subsection{Knowledge Graphs and LLMs} Integrating knowledge graphs (KGs) and LLMs enhances RAG by leveraging structured relationships between document chunks and entities to improve reasoning, retrieval, Q\&A, and summarization tasks \cite{peng2024graphretrievalaugmentedgenerationsurvey, neo4j_graphrag_2024}. Common approaches utilize an LLM or agent to decompose queries, iteratively explore a knowledge graph, extract relevant subgraphs, and solve multi-hop queries \cite{sun2024thinkongraphdeepresponsiblereasoning, li2024decodinggraphsfaithfulsound, jin2024graphchainofthoughtaugmentinglarge,hu2024graggraphretrievalaugmentedgeneration}. Combining various data structures such as graphs, chunks, tables, or retrievers can further tailor queries to correct document sources \cite{li2024structragboostingknowledgeintensive,sarmah2024hybridragintegratingknowledgegraphs}. Knowledge graphs improve LLM long-term memory \cite{gutiérrez2024hipporagneurobiologicallyinspiredlongterm}, text-to-SQL and text-to-Cypher translation tasks \cite{sequeda2023benchmarkunderstandroleknowledge,ozsoy2024text2cypherbridgingnaturallanguage}, and domain-specific approaches \cite{wu2024medicalgraphragsafe}. Graph learning approaches encode structural and relational information from graphs to support node classification, link prediction, and graph-level analysis \cite{he2024gretrieverretrievalaugmentedgenerationtextual,jiang2023diffkgknowledgegraphdiffusion,zaratiana2024grapherstructureawaretexttographmodel}.

Unlike these approaches, Graph RAG-Tool Fusion uses a knowledge graph for tool selection rather than document Q\&A or schema creation, avoiding reliance on graph learning, automatic schema generation, or text-to-Cypher prompting.

\subsection{Tool Selection or Retrieval}
Tool selection aims to retrieve a subset of relevant tools from a large corpus of tools. Baseline approaches utilize lexical term matching for storage and retrieval. Recently, state-of-the-art retriever-based and LLM-based approaches rely on traditional RAG and vector databases for tool selection. Retriever-based methods utilize neural networks to learn the semantic relationships between user queries and tools \cite{anantha_protip_2023,chen2024reinvoketoolinvocationrewriting,lumer2024toolshedscaletoolequippedagents,moon_efficient_2024}. LLM-based methods use an LLM or agent to plan, retrieve, and select relevant tools \cite{yuan_easytool_2024,li_api_bank_2023,du_anytool_2024}. Graph-based methods aim to plan out multi-hop queries with available APIs, but do not consider direct or indirect tool dependencies and benefits of vector retrieval \cite{liu2024toolnetconnectinglargelanguage,liu2023controlllmaugmentlanguagemodels,zhang2023graphtoolformerempowerllmsgraph}.

Our approach, Graph RAG-Tool Fusion, falls within retriever-based tool selection, but introduces a novel use of knowledge graphs to enhance retrieval. To our knowledge, it is the first method to combine knowledge graphs with vector retrieval to improve tool selection, enabling the efficient indexing, maintenance, and retrieval of structured direct or indirect dependencies between tools without relying on query planning strategies.