\section{Experiment}
\label{sec:experiments}
\noindent We validate the proposed approach within a pick-and-place scenario, confirming its effectiveness.

\subsection{Software Libraries}
In this work, 
the SV computation~\eqref{eq:sweptVol} is based on the algorithm described in~\cite{sellan2021swept}.
It is implemented within the \textsc{gpytoolbox} 
library\footnote{S. Sellán, O. Stein et al., “gptyoolbox: 
	A python geometry processing toolbox,” 2023, \url{https://gpytoolbox.org/}.}, 
which leverages the \textsc{libigl} 
library\footnote{A. Jacobson, D. Panozzo et al., “libigl: 
	A simple C++ geometry processing library,” 2018, \url{https://libigl.github.io/}.}.
The algorithm inputs are the triangle mesh of the solid of interest and its discretized sequence of poses,
resulting in a high-quality 3D mesh.

We selected the \textsc{visualization toolkit}~\cite{vtkBook} for the decimation algorithm.
The input is a triangle mesh, 
and the parameters defining the decimation process 
(e.g., the target percentage of the triangle reduction, 
the maximum allowed error, and whether the mesh topology shall be preserved).
It returns a triangle mesh.

Finally, 
we chose the software library \textsc{libigl} 
for the Boolean difference operation, 
which is based on~\cite{zhou2016mesh} and 
\textsc{cgal} library\footnote{"CGAL, Computational Geometry Algorithms Library," \url{https://www.cgal.org/}.}.
The algorithm inputs are two triangle meshes, % to be subtracted, 
provided as a collection of vertices and faces.

\subsection{Hardware Setup}
The experimental platform is a KUKA LBR iisy 3 R760 with $\numLinks=6$ non-static links.
It is equipped with a SCHUNK change system (FWA series), 
enabling a quick switch between the exploration tool and the gripper required for the task.
The latter is a SCHUNK gripper (GEI FWA-50 series) with parallel 3D-printed fingertips.  Overall, it approximately measures $\distInCm{\left( 7 \times 7 \times 12 \right)}$.
We designed a cube-shaped exploration tool to speed up the exploration phase 
and avoid high computation times or numerical instabilities. 
\refFig{fig:exploration_frames} shows the exploration tool, 
a lightweight cardboard box of $\distInCm{\left( 20 \times 20 \times 20 \right)}$, which encloses the gripper entirely.
\refFig{fig:sweptvols_comparison} highlights the advantages of using this tool
by comparing the robot SV mesh for a simple trajectory in three different scenarios: 
the robot as it is, 
with a parallel gripper, 
and with the exploration tool.
The SV mesh generated with the gripper shows imperfections such as holes and irregularities. 
In contrast, 
the use of the exploration tool results in a smoother and more expansive SV.
This underlines our secondary contribution.

Utilizing Robot Operating System (ROS)~2 %(Humble version) 
to exchange data,
all computations are performed on a laptop system with an Intel Core i7-12800H (2.4 GHz) CPU and 32 GB of RAM.

\begin{figure}[!t]
	\centering
	\includegraphics[width=0.99\linewidth]{./figures/sweptvols_comparison_8k}
	\caption{KUKA LBR iisy swept volume of a simple trajectory: 
			 robot as it is (left), 
			 with a parallel gripper (middle), 
			 and with a cube-shaped exploration tool (right).}
	\label{fig:sweptvols_comparison}
\end{figure}

\subsection{Constrained Robot Cell Environment}
\refFig{fig:demonstrator} shows an industrial cart positioned in front of the robot,
with three glasses and a ramp holding six orange blocks. 
The cell contains four additional obstacles: 
two boxes between the ramp and the glasses and two boxes next to the robot, 
limiting its elbow motions. 
While operating, 
the robot must avoid self-collisions and collisions with the environment.

\subsection{Pick-and-Place Task}
\label{subsec:pickAndPlaceTask}
The robot must pick the orange blocks from the ramp and place them into the glasses. 
Each time the block at the bottom of the ramp is picked, 
the remaining ones slide down. 
This process is repeated six times until the ramp is empty. 
The first three blocks go into different glasses, 
and the next three follow the same order. 
In the end, 
each glass contains two blocks.

\begin{figure}[!t]
	\centering
	\includegraphics[width=0.99\linewidth]{./figures/demonstrator}
	\caption{Constrained cell with a KUKA LBR iisy performing a pick-and-place task. 
	The robot must pick the orange blocks from the ramp and place them into the three glasses 
	by exploiting the proposed obstacle representation.}
	\label{fig:demonstrator}
\end{figure}

\subsubsection{Workspace Exploration}
The operator explores the collision-free space with the collaborative robot, 
equipped with the exploration tool, 
utilizing the hand guidance mode proposed in~\cite{osorio2019physical}.
We record the trajectory at $25$ Hz 
and collect $4009$ joint configurations in less than three minutes.
\refFig{fig:exploration_frames} shows selected video frames of the supplementary material.

\begin{figure*}[!t]
	\centering
	\includegraphics[width=0.99\linewidth]{./figures/exploration_8fr}
	\caption{Exploration phase through hand guidance of a KUKA LBR iisy equipped 
	with a cube-shaped exploration tool: 
	front view (top row) and side view (bottom row).
	The corresponding swept volume is shown in~\refFig{fig:sv_pick_and_place}.}
	\label{fig:exploration_frames}
\end{figure*}

\subsubsection{Robot Link Swept Volumes and Decimation}
The six swept volumes of the non-static links are decimated and jointly visualized in \refFig{fig:sv_pick_and_place} together with the static robot base.
The volume decimation step has reduced both the number of vertices and faces by $63.64 \,\%$.

\subsubsection{Obstacle Representation}
We represent the bounding volume $\volume{BV}$ as a cube whose dimensions have been chosen according to the maximum robot length.
The mesh of the obstacle representation volume $\volume{O}$ is obtained by iteratively subtracting the mesh representing the link swept volumes $\decimatedVolume{i}$ from the mesh of the bounding volume $\volume{BV}$.

\subsubsection{Motion Planning and Control}
Thanks to the change system,
once the exploratory phase has been performed, 
the gripper quickly replaces the exploration tool. 
The pick-and-place task is modeled as a simple finite state machine,
specifying the pick-and-place poses through the hand guidance mode.
Moreover, 
since a collision is likely detected when the robot picks up the orange block, 
we record an additional \textit{pre-pick} joint configuration $16$ cm above the ramp.
%Multiple robot configurations can be recorded to exploit the robot's dexterity and facilitate the motion planning.

To generate optimal collision-free trajectories,
we rely on the framework presented in \cite{osorio2020unilateral} used with the \textsc{flexible collision library (fcl)}~\cite{pan2012fcl} 
for fast collision checks. 
Potential collisions are evaluated given the mesh of the robot, 
the gripper, 
and the environment represented as $\volume{O}$.

\subsubsection{Task Execution}
The robot executes the planned trajectories without collisions, 
moving consistently within the previously explored space. 
For further details, 
refer also to the video in the supplementary material.

\subsection{Execution Time Analysis}
\label{subsec:exTimeAnalysis}
A thorough exploration (performed by the first author) of the free space surrounding the robot took $\timeInSec{162.27}$ ($\approx\timeInMin{2.70}$).
\refTab{tab:exTimesPickAndPlace} reports the execution times of each pipeline step
(see \refSec{subsec:pickAndPlaceTask}) averaged over ten independent pipeline executions. 
From a computational perspective, 
the most time-consuming operation is the computation of the robot link swept volumes, 
which takes $\timeInSec{185.81}$ ($\approx\timeInMin{3.1}$). 
On the other hand, 
the computation of the volume decimation is almost negligible, 
taking only a few seconds.
Overall, 
the pipeline generated the obstacle and unexplored space representation~$\volume{O}$ in $\timeInSec{368.49}$ ($\approx\timeInMin{6.14}$). 
Additionally, 
programming the particular application took about $\timeInMin{1.5}$ to 
record the home robot configuration and the pick-and-place poses as well as to generate a collision-free trajectory. 
In summary, 
the robot operator spent $\timeInMin{7.64}$ to model the robot cell and setup the collision-free robot program. 
%Notice that the computational optimization may be further improved by considering a subset of the recorded joint configurations. 
%However, 
%in this case, 
%a trade-off between computational efficiency and accuracy must be found.

\begin{table}[t!]
	\caption{Execution times in seconds averaged over ten independent executions
			 of the pipeline steps for a pick-and-place task with a kuka lbr iisy collaborative robot.
			 }
	\centering
	\label{tab:exTimesPickAndPlace}
	\begin{tabularx}{0.49\textwidth}{C|C|C|C|C}
			\hline
			\multirow{2}{*}{Exploration} & Swept & Volume & Obstacle & \multirow{2}{*}{Total}\\ 
			& Volume & Decimation & Repr. & \\
			\hline
			$\timeInSec{162.27}$ & $\timeInSec{185.81}$ & $\timeInSec{2.16}$ & $\timeInSec{18.25}$ & $\timeInSec{368.49}$ \\
			\hline
		\end{tabularx}
\end{table}

\subsection{The Role of the Volume Decimation}
\label{subsec:roleOfVolDeci}
In this ablation study, 
we evaluate the role of the volume decimation step in the pipeline from a computational load point of view. 
Therefore, 
we have repeated the execution time analysis conducted in \refSec{subsec:exTimeAnalysis} without performing the volume decimation step. 
The obstacle representation step works directly with the SV meshes $\volume{i}$ produced by the previous step.
The execution times to obtain the obstacle representation increase from $\timeInSec{18.25}$ to $\timeInSec{37.40}$. 
Hence, 
the optional and almost costless execution of $\timeInSec{2.16}$ for the volume decimation step implies a reduction of $\timeInSec{16.99}$ ($\approx 4.41\,\%$) on the whole execution time. 
It is worth noticing that computational optimization may be significantly higher in the case of longer explorations or a more complex SV mesh.
