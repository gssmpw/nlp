\section{Conclusion}
\label{sec:conclusion}
\noindent This letter proposes a novel data-driven and robot-agnostic approach to modeling obstacles within a cluttered robot cell. 
The method does not rely on additional external sensors, 
making the environment modeling process cost-effective and immediate. 
It supports novice users, who can gather the necessary data by hand guiding the robot.
After performing exploratory robot motions for few minutes,
the unexplored and potentially occupied space is modeled by leveraging the robot's kinematic structure and the swept volume of the non-static links.
Our method is capable of effectively managing clustered or heterogeneously distributed data.
We obtain a triangular mesh, 
which is finally used to plan and execute collision-free trajectories safely.
Showcasing the method's potential to streamline industrial processes,
we validated the intuitive interface in a pick-and-place scenario.
Our execution time analysis highlighted that the user can model a robot cell and perform a task in less than eight minutes. 
Moreover, 
the ablation study showed the beneficial role of the optional volume decimation (step 3), 
which further optimizes the computational efficiency almost at no cost. 

Future work involves integrating an autonomous exploration mode, 
where the robot changes its direction of motion upon contact detection.
Furthermore, 
we would like to evaluate less accurate but faster AI-accelerated techniques similar to \cite{baxter2020deep,joho2024neural}
for visualizing the swept volume in real-time during an exploratory session (e.g., utilizing immersive augmented reality hardware).
Additionally, 
the non-convex obstacle representation can be decomposed 
into multiple convex meshes using existing techniques, 
potentially accelerating collision checks for motion generation.