
%% Basic document, font, and layout
\usepackage[utf8]{inputenc}
\usepackage[T1]{fontenc}
\usepackage[letterpaper,margin=1in]{geometry}

% AMS packages
\usepackage{amsmath}
\usepackage{amsthm}
\usepackage{amssymb}
\usepackage{amsfonts}
\usepackage[normalem]{ulem}
\ifnum\usenix=0
\usepackage[numbers]{natbib}
\fi
\usepackage{xspace}
\usepackage{hyphenat}
%other packages
\usepackage{graphicx}
\usepackage[dvipsnames]{xcolor}
\usepackage{hyperref, bbm}
\usepackage[inline,shortlabels]{enumitem}
\newcommand{\msf}{\mathsf}
\usepackage{cmll}
\usepackage{mathtools}
\usepackage{algorithm}          % Floating Environments
\usepackage{algpseudocodex}     % Procedure Formatting
\usepackage[inline,nomargin,multiuser]{fixme}
\usepackage{tikz}
\usepackage{cleveref}
\renewcommand{\cref}{\Cref}
\renewcommand\qedsymbol{$\blacksquare$}
\usepackage{thm-restate} %allows for them restatement

% Simple AMSthm environments, numbered together.
\newtheorem{lemma}{Lemma}
\newtheorem{theorem}{Theorem}
\newtheorem{conjecture}{Conjecture}
\newtheorem{claim}{Claim}
\newtheorem{remark}{Remark}
\newtheorem{corollary}{Corollary}
\newtheorem{definition}{Definition}

%basic macros
\newcommand{\zo}{\{0,1\}}
\newcommand{\N}{\mathbb{N}}
\newcommand{\R}{\mathbb{R}}
\newcommand{\E}{\mathbb{E}}
\newcommand{\setrandomly}{\ensuremath{\xleftarrow{\$}}}

%Marco's
\newcommand{\st}{\mid}
\newcommand{\nat}{\mathbb{N}}
\newcommand{\godel}[1]{\ulcorner#1\urcorner}
\newcommand{\pair}[2]{\langle #1, #2 \rangle}
\DeclareMathOperator*{\argmax}{arg\,max}
\DeclareMathOperator*{\argmin}{arg\,min}
\DeclarePairedDelimiter\abs{\vert}{\rvert}
\DeclarePairedDelimiter\norm{\lVert}{\rVert}
\DeclareMathOperator{\MD}{MD}
\newcommand{\SIZE}{\mathsf{SIZE}}
\newlist{todolist}{itemize}{2}
\setlist[todolist]{label=$\square$}

% Curly Letters
\newcommand{\cL}{\mathcal{L}}
\newcommand{\cS}{\mathcal{S}}
\newcommand{\cA}{\mathcal{A}}
\newcommand{\cD}{\mathcal{D}}
\newcommand{\cC}{\mathcal{C}}
\newcommand{\cF}{\mathcal{F}}
\newcommand{\cG}{\mathcal{G}}
\newcommand{\cH}{\mathcal{H}}
\newcommand{\cE}{\mathcal{E}}
\newcommand{\cM}{\mathcal{M}}
\newcommand{\cO}{\mathcal{O}}
\newcommand{\cP}{\mathcal{P}}
\newcommand{\cX}{\mathcal{X}}
\newcommand{\cY}{\mathcal{Y}}
\newcommand{\alg}{\cA}
\newcommand{\wdist}{\dist_{\mathcal{W}_1}}
% Blackboard Bold Letters
\newcommand{\bB}{\mathbb{B}}

%project specific macros
\newcommand{\adv}{\cA}
\newcommand{\eps}{\varepsilon}
\newcommand{\Sim}{\cS}
\newcommand{\sample}{\mathcal{S}}
\newcommand{\learning}{\mathsf{L}}
\newcommand{\learningalgorithm}{\learning}
\newcommand{\census}{\mathsf{C}}
\newcommand{\uid}{\msf{uid}}
\newcommand{\dset}{D}%dataset
\newcommand{\query}{q} %query
\newcommand{\qset}{\mathcal{Q}} %query set
\newcommand{\matchrate}{\msf{match}\text{-}\msf{rate}}
\newcommand{\publicfeatures}{\mathcal{X}_{\msf{pub}}}
\newcommand{\cat}{\mathcal{C}}
\newcommand{\dms}{{\color{orange} n}}
\newcommand{\lens}{\rho}
\newcommand{\inp}{\mathcal{X}}
\newcommand{\pred}{\gamma}
\newcommand{\ind}{\mathbbm{1}}
\newcommand{\uids}{U}
\newcommand{\selection}{\rho }

\newcommand{\univ}{\mathcal{X}}
\newcommand{\infouniv}{\mathcal{U}^{\mathbb{N} \times \mathbb{N}}}
\newcommand{\coherenceExp}{\mathsf{DemCoh}}


\newcommand{\distance}{\msf{dist}}
\newcommand{\dist}{\msf{dist}}
\newcommand{\bin}{\msf{b}}
\newcommand{\fspace}{\cF}
\newcommand{\complete}{\msf{ALL}}
\newcommand{\linkit}[2]{\textcolor{blue}{\textit{\href{#1}{#2}}}}
%%%%%%%%%% comments %%%%%%%%%%%%%%%%%%
\definecolor{azure}{rgb}{0,0.49,1}

\ifnum\notes=1
\newcommand{\gsk}[1]{\textsf{\textbf{\textcolor{azure}{[[GPK: #1]]}}}}
% \newcommand{\gsknote}[1]
% {{\color{azure}\footnote{\textcolor{azure}{\textbf{GPK:} #1}}}}
\newcommand{\gsknote}[1]{{\color{azure}\footnote{{\color{azure} {\bf GPK:} #1}}}}
\else
\newcommand{\gsknote}[1]{}
\newcommand{\gsk}[1]{}
\fi

\definecolor{mypurple}{rgb}{0.58,0.23,0.94}


\ifnum\notes=1
\newcommand{\pj}[1]
{{\color{mypurple}#1}}
\newcommand{\pjinline}[1]{{\color{mypurple} #1}}
\newcommand{\pjnote}[1]{{\color{mypurple}\footnote{{\color{mypurple} {\bf P:} #1}}}}
\newcommand{\pjtodo}[2]{{\color{mypurple}\sf
{\bf[[P:} #1{\bf]]} #2}
\footnote{{\color{mypurple} {[[\bf P:} #1 {\bf]]}}}
}
\else
\newcommand{\pj}[1]{{\color{black}#1}}
\newcommand{\pjinline}[1]{}
\newcommand{\pjnote}[1]{}
\newcommand{\pjtodo}[2]{}%{{\color{mypurple}\sf
%{\bf[[P:} #1{\bf]]} #2}
%\footnote{{\color{mypurple} {[[\bf P:} #1 {\bf]]}}}
%}
\fi

\ifnum\notes=1
\newcommand{\mc}[1]
{{\color{orange}\sf\bf [[MC: #1]]}}
\newcommand{\mcinline}[1]{{\color{orange} #1}}
\newcommand{\mcnote}[1]{{\color{orange}\footnote{{\color{orange} {\bf MC:} #1}}}}
\else
\newcommand{\mc}[1]{}
\newcommand{\mcinline}[1]{}
\newcommand{\mcnote}[1]{}
\fi

% \newcommand{\mb}[1]{{\color{red} #1}}
\ifnum\notes=1
\newcommand{\mb}{\mbnote}
% \newcommand{\mbnote}[1]
% {{\color{red}\footnote{\textsf{\textbf{\textcolor{red}{MB: #1}}}}}}
\newcommand{\mbnote}[1]{{\color{red}\footnote{{\color{red} {\bf MB:} #1}}}}
\else
\newcommand{\mb}{\mbnote}
% \newcommand{\mbnote}[1]
% {{\color{red}\footnote{\textsf{\textbf{\textcolor{red}{MB: #1}}}}}}
\newcommand{\mbnote}[1]{}
\fi

% \newcommand{\sstext}[1]{{\textcolor{purple} #1 }}
\ifnum\notes=1
\newcommand{\sstext}[1]
{{\color{purple}#1}}
% \newcommand{\ssnote}[1]
% {{\color{purple}\footnote{\textbf{\textcolor{purple}{SS: #1}}}}}
\newcommand{\ssnote}[1]{{\color{purple}\footnote{{\color{purple} {\bf SS:} #1}}}}
\else
\newcommand{\sstext}[1]{{{\color{black}#1}}}
% \newcommand{\ssnote}[1]
% {{\color{purple}\footnote{\textbf{\textcolor{purple}{SS: #1}}}}}
\newcommand{\ssnote}[1]{}
\fi

\ifnum\notes=0
\renewcommand{\sout}[1]{}
\fi

\newcommand{\eg}[0]{\emph{e.g.,}\xspace}
\newcommand{\ie}[0]{\emph{i.e.,}\xspace}

 \newcommand{\cdf}{\msf{cdf}}
    \newcommand{\distw}{\dist_{\mathcal{W}_1}}
    \newcommand{\wass}{\distw}

\DeclareMathAlphabet{\mathbbold}{U}{bbold}{m}{n}

\newcommand{\coherence}{\xspace demographic coherence\xspace}
\newcommand{\coherent}{\xspace demographically coherent\xspace}

\newcommand{\restrict}[2]{\raisebox{0.17ex}{$\mathop{\boldsymbol{\large\pi}_{#2}}$}\left(#1\right)}

% \newcommand{\restrict}[2]{\raisebox{0.17ex}{$\large\pi_{#2}$}\left(#1\right)}



\newcommand{\defeq}{\stackrel{\text{def}}{=}}

\newcommand{\ignore}[1]{{ }}
\newcommand{\bigmid}{\,\,\,\Big\vert\,}

\newcommand{\1}{\mathbbm{1}}
\newcommand{\ex}[1]{\mathbb{E}\left[#1\right]}
\DeclareMathOperator*{\Expectation}{\mathbb{E}}
\newcommand{\Ex}[2]{\Expectation_{#1}\left[#2\right]}
\DeclareMathOperator*{\Probability}{\mathrm{Pr}}
\newcommand{\prob}[1]{\mathrm{Pr}\left[#1\right]}
\newcommand{\Prob}[2]{\Probability_{#1}\left[#2\right]}
\DeclareMathOperator*{\smin}{s-\mathrm{min}}

\newcommand{\bbx}{\mathbf{x}}
\newcommand{\bX}{\mathbf{X}}

\newcommand\numberthis{\addtocounter{equation}{1}\tag{\theequation}}

\newcommand\allbold[1]{{\boldmath\textbf{#1}}}

\newcommand{\mybox}[4]{
\ifnum\usenix=1
\begin{figure}[t]
\else
\begin{figure}[H]
\fi
\begin{center}
\fcolorbox{black}{mypurple!3}
{
\small
\hbox{\quad
\ifnum\usenix=1
\begin{minipage}{.75\columnwidth}
\else
\begin{minipage}{6in}
\fi
\vspace{0.3em}
\begin{center}
{\allbold{#1}}
\end{center}
#4
\vspace{0.2em}
\end{minipage}
}
}
\caption{\label{#3} #2}
\end{center}
\end{figure}
\vspace{-1em}
}



\ifnum\notes=1%% uncomment before submitting
\usepackage{showlabels}
\renewcommand{\showlabelfont}{\tiny\ttfamily}
\fi