% \section{Conclusion}
% To facilitate robotics and computer vision applications, a LiDAR-based mapping and localization method is needed that can robustly handle unordered, low overlap, multiview point clouds simultaneously, similar to how Structure-from-Motion (SfM) for cameras processes unordered images all at once.
% However, unlike in the field of SfM, where the use of visual fiducial markers (VFMs) has been widely studied, the development of the LiDAR fiducial marker (LFM) and its utilization in such a framework remains a technological gap. This dissertation addresses these pressing issues from the following perspectives.
% \par
% First, to address the lack of a general LFM system that can be applied to different LiDAR models and is compatible with popular VFM patterns, a novel intensity image-based LiDAR fiducial marker (IFM) system is proposed. This system only requires an unstructured point cloud with intensity as the input. A marker detection method that locates the predefined 3D fiducials in the point cloud through the intensity image is introduced. Then, an approach that utilizes the detected 3D fiducials to estimate the LiDAR 6-DOF pose that describes the transmission from the world coordinate system to the LiDAR coordinate system is developed. The main novelties of IFM are its convenience and extensibility. Convenience refers to the ability for users to print the marker on letter-sized paper and place it freely, as with VFM systems, unlike existing LiDAR fiducial objects that are placed on tripods as extra 3D objects, altering the environment's 3D geometry. Extensibility means that popular VFM systems, such as AprilTag \cite{ap3}, ArUco \cite{aruco}, CCTag \cite{cctag}, and future ones, can easily be integrated into the proposed system. Thus, the system imposes no restrictions on marker shape. Experimental results demonstrate that the proposed system is applicable to various solid-state and mechanical LiDARs with decent efficiency (approximately 40 Hz on the Livox Mid-40 and 143 Hz on the VLP-16) and offers better pose estimation accuracy than VFM systems.
% \par
% Second, given that the vanilla IFM has two limitations: (1) it detects fiducials only in single-view point clouds, not in 3D LiDAR maps, and (2) it has a limited detectable range as the tag's projection size shrinks with increasing distance, a novel method is developed to improve the localization of thin-sheet LFMs. 
% %
% First, LFM localization is extended from single-view point clouds to 3D maps, facilitating downstream tasks such as map merging. Second, the valid range for LFM localization is significantly increased while achieving better accuracy with the proposed method. Moreover, this method is generalizable and applicable to 3D maps constructed by various LiDAR-based SLAM approaches. Qualitative and quantitative experiments demonstrate the effectiveness and superiority of the proposed method.
% \par
% Third, a novel framework for mapping and localization using LFMs is developed, achieving mapping and localization by registering unordered multiview point clouds, even with low overlap. In particular, two-level graphs are designed to process the unordered set of point clouds and to efficiently and optimally estimate the relative pose between them. An adaptive threshold LFM detection method is proposed to address the unstable detection of the vanilla IFM in the wild when the viewpoint changes dramatically. Qualitative and quantitative experiments are conducted to demonstrate the superiority of the proposed method over previous state-of-the-art methods and its diverse robotic and computer vision applications, including 3D asset collection, training data collection, reconstruction of a degraded scene, localization in a GPS-denied environment, and 3D map merging.
% %
% Specifically, a new dataset named Livox-3DMatch is collected using the proposed framework to support learning-based point cloud registration methods.
% %
% The Livox-3DMatch dataset expands the original 3DMatch training data from 14,400 pairs to 17,700 pairs, representing a 22.91\% increase. Training on this augmented dataset improves the performance of the state-of-the-art learning-based method SGHR \cite{sghr} by 2.90\% on 3DMatch \cite{3dmatch}, 4.29\% on ETH \cite{eth}, and 22.72\% (translation) / 11.19\% (rotation) on ScanNet \cite{scan}.

\section{Conclusion}
To facilitate robotics and computer vision applications, a LiDAR-based mapping and localization method is needed that can robustly handle unordered, low overlap, multiview point clouds simultaneously, similar to how Structure-from-Motion (SfM) for cameras processes unordered images all at once.
However, unlike in the field of SfM, where the use of visual fiducial markers (VFMs) has been widely studied, the development of the LiDAR fiducial marker (LFM) and its utilization in such a framework remains a technological gap. This dissertation addresses these pressing issues from the following perspectives.
% \subsection{Intensity Image-based LiDAR Fiducial Marker System}
\par
To address the lack of a general LFM system that can be applied to different LiDAR models and is compatible with popular VFM patterns, a novel intensity image-based LiDAR fiducial marker (IFM) system is proposed. 
\begin{itemize}
\item Unlike LiDARTag \cite{lt} which requires extra 3D objects to be added to the environment, the usage of the IFM is as convenient and easy as the VFM systems \cite{ap3,aruco}. Namely, the users can produce the marker by printing the VFM on regular letter-size paper with a regular printer and then place the marker anywhere they like. 
\item A novel marker detection method is proposed to detect 3D fiducials through the intensity image. Thanks to this, the VFM systems proposed in the past, present, and even future can be easily embedded into the IFM system. This is a superiority of the proposed system over LiDARTag \cite{lt} which only supports square markers with patterns from AprilTag 3 \cite{ap3}. 
\item A pose estimation approach for the LiDAR via the proposed IFM is introduced, which has better accuracy than the VFM-based pose estimation for the camera. In addition, unlike the VFM system, the proposed pose estimation is free from the rotation ambiguity problem \cite{ippe,yibo} and is robust to changes in ambient light. 
\end{itemize} \par
Due to the adoption of 3D-to-2D spherical projection, the vanilla IFM exhibits two limitations: (i) IFM can only detect fiducials in a single-view point cloud and does not apply to a 3D LiDAR map, and (ii) as the distance between the tag and the LiDAR increases, the projection size of the tag decreases until it is too small to be detected. In response to these limitations, a novel method has been developed to improve the localization of thin-sheet LFMs.
\begin{itemize}
	\item The LFM localization is extended from the single-view point cloud to the 3D LiDAR map and the detectable distance of LFMs is enhanced. 
	\item A new pipeline for jointly analyzing a 3D point cloud from both intensity and geometry perspectives is proposed, as fiducial tags are planar objects with high-intensity contrast and are indistinguishable from the plane to which they are attached. This differs from conventional 3D object detection methods that rely merely on 3D geometric features and can only detect spatially distinguishable objects.
\end{itemize} \par
Finally, building upon the developed algorithms for LFM detection, this dissertation discusses how to exploit LFMs for mapping and localization.
\begin{itemize}
	\item A novel framework for mapping and localization using LFMs is designed, where mapping and localization are achieved by registering unordered multiview point clouds, even with low overlap. The proposed framework serves as an efficient and reliable tool for diverse robotics and computer vision applications, including 3D asset collection, training data collection, reconstruction of degraded scenes, localization in GPS-denied environments, and 3D map merging.
	\item A new training dataset called Livox-3DMatch using the proposed framework is collected, which augments the original 3DMatch training data \cite{3dmatch} from 14,400 pairs to 17,700 pairs (a 22.91\% increase). Training on this augmented dataset improves the performance of the SOTA learning-based methods on various benchmarks. In particular, the RR of SGHR \cite{sghr} is increased by 2.90\% on 3DMatch \cite{3dmatch} and 4.29\% on ETH \cite{eth}. The translation error and rotation error on ScanNet \cite{scan} are decreased by 22.72\% and 11.19\%, respectively. The RR of MDGD \cite{mdgd} increases by 1.71\% on 3DMatch \cite{3dmatch} and 2.89\% on ETH \cite{eth}. The translation error and rotation error on ScanNet \cite{scan} decrease by 22.45\% and 7.80\%, respectively.
\item An adaptive threshold LFM detection algorithm that is robust to viewpoint changes in the wild is developed.
 
\end{itemize}

\section{Future Work}
In this dissertation, a framework for mapping and localization using LFMs is proposed to benefit various robotic and computer vision applications. While the results are promising, several issues remain that require further exploration. The following are potential future research directions to enhance this work.
\par	
The first issue is the utilization of learning-based methods in LFM detection. Since the proposed IFM is open-source, much inspiring user feedback has been received, which also motivated the reconsideration of LFM detection. For example, both LiDARTag \cite{lt} and IFM \cite{iilfm} propose introducing the encoding-decoding algorithms of VFM systems in LFM detection. However, since the 'code' of a marker's pattern typically exceeds 16 bits—meaning more than 16 black-and-white blocks—the LiDAR must have a high resolution to fully capture the code/bit information. Based on feedback from users in the autonomous driving industry who use LFM for calibration, marker detection sometimes fails as the bit information is lost when the distance increases. Unfortunately, this is caused by the sparse features of the LiDAR data, which have not been adequately addressed so far. Despite this, a notable technology called LiDAR intensity densification \cite{dense}, which learns to generate densified LiDAR intensity data from sparse and occluded LiDAR data, has the potential to mitigate this problem.
Another solution that has a similar effect is Neural LiDAR Fields \cite{lidarnerf}, which learns LiDAR-based novel view synthesis and supports the upsampling of LiDAR data. Moreover, considering that many 3D object detection and segmentation models are designed to learn directly from sparse and occluded data, it is possible to tackle this problem using a learning-based method. Nevertheless, it should be noted that the lack of corresponding training data presents a significant challenge. In particular, the training data should include a substantial number of annotated thin-sheet LFMs, and the model should be designed to learn detection by utilizing intensity information.
\par
The second issue is the need to collect more data using the proposed framework. Although a new dataset named Livox-3DMatch has been collected for point cloud registration research in this dissertation, it would be beneficial to use the proposed framework to collect additional data and a greater variety of datasets. For example, 3D assets of irregular vehicles, such as construction trucks equipped with various machines, are rare but crucial for autonomous driving simulation. Thus, collecting more 3D assets of irregular vehicles using the proposed framework is advantageous. LiDAR-based novel view synthesis is an emerging topic. In camera-based research, large-scale datasets like OmniObject3D \cite{omni} provide extensive object-level posed images. Developing a similar dataset for LiDAR that offers posed point clouds would enhance LiDAR-based novel view synthesis \cite{lidarnerf,lidarnerf2}, which is a significant downstream and future application of the proposed framework.
\par
The third issue is to explore other types of materials and formats to construct markers for better utilization of the intensity information. In this research, we only study black-and-white markers made of paper, as we aim to develop a low-cost solution. However, for scenes where the distance between the fiducial and the sensor is hundreds of meters, it is beneficial to introduce fiducial objects such as prisms. Moreover, by leveraging other materials and formats, such as prisms, to construct LFMs, it is feasible to eliminate the reliance on 2D space, while keeping the system as unified as prisms work for different LiDAR models.
\par
The fourth issue is to fuse LFM detection with VFM detection for a robust autonomous system. Although utilizing the algorithms proposed in this research, it is feasible to detect thin-sheet markers with LiDARs alone, redundant design is considered necessary in autonomous systems nowadays. As introduced in the dissertation, LFM detection can help address rotational ambiguity and unideal illumination conditions, which are challenging for VFM detection. On the other hand, there are many mature computer vision algorithms to derain, defog, and desnow, which are challenging for LFM detection. In addition, fusing LFM and VFM detection and constructing the pose estimation of an autonomous system as a global optimization problem has the potential to provide better accuracy.