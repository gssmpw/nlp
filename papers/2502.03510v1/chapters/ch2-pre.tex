% \section{Three-dimensional Transformation} \label{threedimen}
% Suppose that $\mathbf{p}_{a} = [x,y,z]^{T}$ is a 3D point expressed in the coordinate system $\{a\}$. To express the point in another coordinate system $\{b\}$ as $\mathbf{p}_{b} = [x^{\prime},y^{\prime},z^{\prime}]^{T}$, translation and rotation transformations need be applied to $\mathbf{p}_{a}$. In particular, the extrinsic matrix, $\mathbf{T}\in\mathbb{R}^{4\times4}$, is employed to describe the 6-DOF pose:

% \begin{equation}
% \mathbf{T} = \left[\begin{array}{cc}
% \mathbf{R} & \mathbf{t} \\
% \mathbf{0}^{1 \times 3} & 1 \label{transform}
% \end{array}\right],
% \end{equation} 
% where $\mathbf{t}\in\mathbb{R}^{3\times1}$ denotes the translation vector and $\mathbf{R}\in \mathbb{R}^{3\times3}$ denotes the rotation matrix. Afterwards, $\mathbf{p}_{b}$ is obtained as follows:
% \begin{equation}
% \left[\begin{array}{c}
% \mathbf{p}_b \\
% 1
% \end{array}\right]=\mathbf{T}\left[\begin{array}{c}
% \mathbf{p}_a \\
% 1
% \end{array}\right]. \label{dot}
% \end{equation} \par
% To simplify the notation, the operator ($\cdot$) is introduced to denote:
% \begin{equation}
% \mathbf{p}_b=\mathbf{T} \cdot \mathbf{p}_a. 
% \end{equation} \par
% Please note that ($\cdot$) is not a dot product as $\mathbf{T}$ is $4 \times 4$ and $\mathbf{p}$ is $3\times 1$. So it is different from a regular dot product. Eq. (\ref{dot}) shows what it means.

% \section{Lie Group and Lie Algebra} \label{lie}
% The rotation matrix $\mathbf{R}\in SO(3)$, which is the special orthogonal group representing rotations \cite{barfoot}:
% \begin{equation}
% SO(3) = \{ \mathbf{R}\in\mathbb{R}^{3\times3} | \ \mathbf{R}\mathbf{R}^{T}=\mathbf{1},det(\mathbf{R}) =1 \}.
% \end{equation} \par
% The pose matrix $\mathbf{T} \in SE(3)$, which is the special Euclidean group representing poses \cite{barfoot}:
% \begin{equation}
% SE(3) = \{ \mathbf{T} = \left[\begin{array}{cc}
% \mathbf{R} & \mathbf{t} \\
% 0 & 1 
% \end{array}\right] \in \mathbb{R}^{4 \times 4} | \ \mathbf{R} \in SO(3), \mathbf{t}\in \mathbb{R}^{3 \times 1}\}.
% \end{equation} \par
% $SO(3)$ and $SE(3)$ are two specific Lie groups. Every matrix Lie group is associated with a Lie algebra \cite{barfoot}. The Lie algebra $\mathfrak{so(\mathrm{3})}$ that is associated with $SO(3)$ is defined as follows:
% \begin{equation}
% \begin{aligned}
% & SO(3) \rightarrow \mathfrak{so(\mathrm{3})}: \\
% & \log(\mathbf{R}) = \frac{\theta}{2\mathrm{sin}(\theta)}(\mathbf{R}-\mathbf{R}^{T}),  \\ 
% & \xi = \log(\mathbf{R})_{\vee}, 
% \end{aligned}
% \end{equation}
% where $\theta= \arccos \frac{1}{2}(Trace(\mathbf{R})-1)$. $\log(\cdot)$ is the matrix logarithm and $\xi \in \mathbb{R}^{3\times1}$ is the Lie algebra coordinates. $\vee$ is the \textit{vee} map operator that finds the unique vector $\xi \in \mathbb{R}^{3\times1}$ corresponding to a given skew-symmetric matrix $\log(\mathbf{R})\in \mathbb{R}^{3 \times 3}$ \cite{barfoot,tagslam}.

% \subsection{LiDAR Technology}
% LiDAR is a technology that uses laser pulses to measure the distance between the sensor and objects in the environment. It creates precise three-dimensional representations of the scanned scene. 
% %
% The LiDAR sensor emits laser beams, and by measuring the time it takes for the reflected light to return to the sensor, the positions of 3D points in space are calculated.
% %
% In addition to the geometric data, a LiDAR sensor can also capture intensity values, which reflect the strength of the returned signal. These intensity values are influenced by factors such as surface material, texture, and angle of incidence. The inclusion of intensity information enhances the ability to distinguish between different types of surfaces, improving tasks like object detection, scene understanding, and feature recognition. Fig.~\ref{lidarwork} illustrates the schematic diagram of the working principle of LiDAR.
% \begin{figure}[H] 
% 	\centering
% 	\includegraphics[width=1.0\linewidth]{figs/intro/lidarwork.png}
% 	\caption{ The schematic diagram of the working principle of LiDAR.
%  }
% 	\label{lidarwork}
% \end{figure}