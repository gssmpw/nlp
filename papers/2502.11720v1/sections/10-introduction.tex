\section{Introduction}
A spoken sentence is often not limited to its literal meaning. The question \textit{``Can you pass the salt?''} implicitly requests an action, while literally questioning the listener's physical ability to handover the salt. Alternatively, \textit{``This soup needs salt''} both asserts an opinion about the soup and, depending on the context and surrounding objects, may be requesting someone to pass the salt~\cite{clark1979responding}. These are examples of indirect speech acts (ISAs), which~\citet{searle1975indirect} defined as utterances where one speech act is performed indirectly by carrying out another, transforming direct intents into implicatures. They are complex, multi-faceted, and require shared context and interpretation~\cite{searle1975indirect}. They are also an optimized way to communicate that commonly occurs in collaboration settings where teammates build a shared understanding of the task~\cite{clark1986referring}. Similar to human-human interaction, understanding ISA in human-robot interaction is crucial, since interactions are often based on language to achieve a certain task or goal. 

The inherent naturalness, ease of production, and flexibility of indirect speech make it well-suited for effective human-robot collaboration (HRC)~\cite{obaigbena2024ai, liu2019review}. Through this lens, the robot is envisioned as a social, intelligent collaborator, where politeness, social etiquette, and discussion become factors of shared tasks. In emerging social collaborative robotic (cobotic) scenarios, such as with personal assistants in healthcare and accessibility~\cite{carros2020exploring, zhang2023follower}, then, ISAs are seen as a suitable method of interaction~\cite{williams2020excuse}. 

In performance- and task-oriented settings, however, the appropriateness of ISAs is less obvious. If the cobot partner is performing critical tasks, there may be little room for interpretation or lack of clarity. Direct speech -- \textit{``pass the salt''} -- is clear and timely. This is akin to more traditional interactions, where robots were clearly subservient and commands needed to be learned and delivered correctly. However, this learning creates barriers to natural and intuitive interaction, and the influence on user experience lacks evidence from comparative user studies. 
 
Even though the recent advances in large language models (LLMs) enhance the potential for natural speech interactions with robots in physical tasks~\cite{singh2023progprompt, macdonald2024language, liang2023code, zhao2024large}, to date, much of the attention is still on direct, explicit commands. This often oversimplifies communication, stripping away the naturalness observed in genuine human collaboration~\cite{lamm2017pragmatics}. The rate of LLMs' advances makes it likely that indirect speech will be supported through these models, before which, however, it remains essential to understand the role and impact of ISAs in human-robot collaboration.

Previous work has shown that humans tend to use ISAs when interacting with robots at frequencies similar to those used with other humans~\cite{lee2010receptionist, bennett2017differences}. This highlights the need to develop human-centred verbal communication interfaces for cobots that can accommodate the naturalistic and varied ways in which people express themselves. However, despite the indispensability of ISAs in collaborative communication, there remains a gap in empirical evidence regarding the impact of ISAs on human-robot collaboration, especially in tabletop manipulation tasks. 

To address this gap, we conducted a study with 36 participants, comparing two speech modes of a real robot in a laboratory setting: one capable of understanding ISAs and another without this capability, on three collaborative tasks. Given that natural language communication is a barrier preventing human-robot teams from outperforming human-human teams~\cite{schelble2023investigating}, we theorise that the use of ISAs can contribute to the effectiveness and naturalness of communication, thereby improving perceived team performance and user experience. Specifically, to assess the impact of ISAs on collaboration and communication, we evaluated four key metrics commonly used in HRC. \textit{Team fluency} reflects seamless coordination, which is critical for user satisfaction and acceptance of cobots~\cite{hoffman2019evaluating, duan2021bridging}. \textit{Goal alignment} measures the success and efficiency of collaboration~\cite{salehzadeh2022purposeful, salas1995situation}. \textit{Trust} is essential for preventing misuse or disuse of the robot, ultimately enhancing collaboration effectiveness~\cite{abbass2018foundations, zhang2023investigating}. Moreover, enabling the robot's ability to understand ISAs could serve as a means to induce \textit{anthropomorphism}, thereby improving collaborative engagement by fostering a sense of partnership and enhancing the collaborative experience~\cite{zhang2021ideal, rezwana2022understanding}.

In summary, our research addresses the following questions:
\begin{description}
    \item [RQ1] How does a robot's capability to understand indirect speech acts influence the perceived \textit{team's performance}?
    \begin{description}
        \item [RQ1.1] How does a robot's capability to understand indirect speech acts influence the \textit{fluency} of human-robot teamwork?
        \item [RQ1.2] How does a robot's capability to understand indirect speech acts influence the establishment of \textit{goal alignment} among the human-robot team?
    \end{description}
    \item [RQ2] How does a robot's capability to understand indirect speech acts influence a human teammate's \textit{trust} in the robot's performance?
    \item [RQ3] How does a robot's capability to understand indirect speech acts influence a human teammate's perception of the robot's \textit{anthropomorphism}?
\end{description}

Our findings show that while ISAs are beneficial in human-robot collaboration, their effectiveness can vary depending on the context. The quantitative results show the robot's ability to comprehend ISAs significantly enhances participants' perceived team performance, trust, and anthropomorphism. The use of ISAs fosters a deeper cognitive engagement, making the robot appear more as a collaborative partner rather than a mere tool. However, qualitative results suggest that the usage of ISA can be task- and context-dependent in human-robot collaboration, with inappropriate use potentially leading to negative impacts on trust and user perception. These insights highlight the inherent limitations of relying solely on direct command-based interactions, which lack the subtlety required for establishing shared understanding and the sense of teaming. They also emphasise the importance of using indirect requests in a contextually adaptive and appropriate manner. Therefore, the careful integration of direct and indirect verbal communication emerges as a critical factor in optimising the performance and overall experience of human-robot collaboration. We advocate for the human-computer interaction (HCI) and human-robot interaction (HRI) community to develop human-centred LLMs for collaborative robots, recognising the critical role of ISAs in achieving this goal.