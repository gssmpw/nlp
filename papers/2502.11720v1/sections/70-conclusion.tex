\section{Conclusion}
In this study, we investigated the impacts of indirect speech acts on human-robot collaboration. Our findings highlight that the robot's ability to interpret ISAs plays a crucial role in verbal communication, though the implications of this ability vary depending on context and task. Our results suggest that ISAs hold significant potential as a communication tool to facilitate team fluency, goal alignment, and trust in HRC when applied appropriately. Robots with the ability to understand indirect requests can also increase human perception of anthropomorphism, which enhances the sense of partnership and results in a better collaborative experience. We further explored the human motivations for using indirect requests and the underlying factors driving these impacts using qualitative analysis.  

Future research should focus on assessing language models' ability to interpret implicatures in indirect requests, provide appropriate ISAs, and develop large language models capable of nuanced, context-aware interactions for robotic systems. Moreover, given the inherent ambiguity of ISAs, designing effective backchanneling mechanisms to prevent misunderstandings and convey uncertainty is equally important. We advocate for careful integration of both direct and indirect verbal communication into the design and evaluation of collaborative robots, ensuring that ISAs are neither overlooked nor overused in inappropriate contexts.
