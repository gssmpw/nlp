\section{Related Work}
\subsection{Speaking to Embodied Agents}
%voice assistant, anthropomorphism, more capable agent
Voice assistants (VA), like Siri and Alexa, can have a noticeable influence on user behaviour, as these voice command interfaces are increasingly integrated into daily interactions through devices like phones, computers, and cars.~\cite{amershi2019guidelines, tsoli2018interactive}. The reach of VAs extends beyond simple task execution, influencing users' linguistic habits and potentially shaping social norms surrounding technology use~\cite{motta2021users, williams2020excuse}. Early research primarily addressed the technical challenges associated with speech detection and dialogue systems, focusing on improving the accuracy and efficiency of voice recognition technologies~\cite{cohen1995role}. As VAs became more commercialised, researchers observed that people adapt their language when interacting with VAs, using direct commands, simplified sentences, and keywords to mitigate the risk of misinterpretation~\cite{jaber2024cooking, myers2018patterns}. This adaptation reflects the users' low expectations of language processing and voice interfaces, as well as the inherent limitation of VAs' ability to comprehend and execute complex commands accurately. However, with the advancements in natural language processing, there has been a shift away from command-based paradigms towards more nuanced and complex verbal interactions due to its increased ability to infer intention and understand context~\cite{tanneberg2024help, ning2024user}, allowing for more natural and fluid dialogues between users and machines~\cite{abdul2015survey, yi2024survey}. 

Voice command interfaces have been implemented in embodied agents, such as social and collaborative robots, offering significant advantages in making these systems more human-like assistants capable of supporting real-world tasks. The advantages of incorporating voice interfaces into robots are evident, particularly in scenarios where natural and intuitive communication is essential. Besides, voice interfaces make AI and digital information more accessible to specific populations, such as children and the elderly, who may otherwise struggle with traditional interaction methods~\cite{shiomi2015effectiveness, ruggiero2022companion}. To make the voice interface more capable and better accommodate human activities, recent studies have increasingly focused on elements such as vocal fillers~\cite{ohshima2015conversational}, voice-matching~\cite{de2024your}, and social norms, particularly language politeness~\cite{williams2020excuse}. Among these, most elements contribute to making interactions feel more natural and human-like. Anthropomorphism remains one of the most extensively studied characteristics in human-agent verbal interaction~\cite{seaborn2021voice}. 

A considerable amount of existing research has explored the potential impact of robots' voices on human perception of human-likeness, trust, and capability. For instance, studies have shown that a human-like voice can increase trust in the robot~\cite{xu2019first}, with this effect being more noticeable when the robot's voice matches the gender of the participant~\cite{eyssel2012if}. Another study found that participants issued more commands to robots with artificial voices compared to those with human-like speech, suggesting that a less human-like voice may lead users to perceive the robot as a less capable machine rather than as a competent human~\cite{sims2009robots}. While the anthropomorphism of robots' speech can enhance user experience, it also increases the risk of participants overestimating the robot's intelligence and abilities~\cite{cha2015perceived}. Other factors like politeness, humor, and directness also shape a robot's perceived anthropomorphism ~\cite{emnett2024using}. Robots using indirect language in social interactions often seem more human-like ~\cite{saunderson2021robots}.

\subsection{Verbal Communication during Human-Robot Collaboration}
Verbal communication offers distinct advantages due to its naturalness and efficiency~\cite{liu2019review}. Previous research has demonstrated that humans communicating task-related information to robots can enhance the robot's understanding of goals and intentions, thereby improving overall performance~\cite{breazeal2004designing}. Additionally, studies have shown that robots equipped with communication abilities and verbal feedback can improve team performance by reducing task completion times and being perceived as better teammates~\cite{st2015robot}. Furthermore, explicitly incorporating context into communication enhances clarity, reduces ambiguity, and improves mutual understanding~\cite{mavridis2005grounded, tellex2014asking}.

In natural language processing for human-robot collaboration, several methods exist to parse commands from explicit utterances. The most direct approach involves extracting semantic features and mapping them to predefined robot controllers~\cite{tellex2006spatial}. However, research has shown that participants often provide instructions at varying levels of abstraction~\cite{anderson2018vision}. To interpret more abstract commands lacking specific keywords, association models are used to combine literal linguistic features and extract semantic meaning, typically relying on probability-based methods~\cite {misra2016tell, liu2016natural, matuszek2013learning}. Additionally, to generalize across new tasks and enable contextual understanding, researchers are exploring the usage of large language models for controlling robots in physical tasks~\cite{singh2023progprompt, macdonald2024language, liang2023code, zhao2024large}. While LLMs have the potential to comprehend implicit verbal commands, most studies focus on explicit, direct commands, which provide clear instructions but do not capture the nuanced and indirect nature of human communication in real-world scenarios.

However, relying primarily on direct commands that explicitly convey human requests oversimplifies interactions. Indirect speech acts are a natural feature of human communication, contributing to enhancing robots' anthropomorphism, which has been shown to be an important factor in creating an ideal AI teammate~\cite{zhang2021ideal}. ISAs also serve as an implicit and important means for humans to express their intentions. When the ISAs are misinterpreted during collaboration, the potential for long-term efficiency gains is compromised. Therefore, equipping robots with the ability to interpret ISAs enables them to respond more naturally and effectively, closely mimicking human-like communication patterns. This capability is particularly important in tasks that require high levels of coordination and mutual understanding, such as cooperative manipulation tasks~\cite{shah2011improved}.

\subsection{Indirect Speech Acts in Human-Robot Interaction}
Research has shown that humans tend to use indirect verbal requests when interacting with robots at frequencies similar to those used in human-human interactions, demonstrating the necessity of enabling ISAs in HRI~\cite{lee2010receptionist, bennett2017differences}. Several studies have focused on providing robots with the ability to interpret indirect requests. For example, Briggs and Scheutz~\cite{briggs2013hybrid} created a hybrid system to comprehend indirect requests and provide appropriate responses. Another studies~\cite{williams2015going, wen2020dempster} introduced a probabilistic algorithm for robots to learn sociocultural norms, infer intentions from human utterances, and generate clarification requests.

Moreover, the impact of ISAs on human-robot interaction has been examined from various perspectives. Research shows that robots employing ISAs are perceived as more likeable~\cite{torrey2013robot, strait2014let}, trustworthy~\cite{saunderson2021robots}, and willing to help~\cite{srinivasan2016help}. Conversely, a robot's inability to understand conventionalised ISAs (e.g., \textit{``Can you..?''}, \textit{``I need you to..''}) during social interactions has been found to negatively affect its performance and human perception~\cite{williams2018thank}. Even when participants are aware that robots may not fully comprehend ISAs, they tend to continue using them, which can impact the interaction fluency~\cite{briggs2017enabling}. While current research largely focuses on social interactions, there is still a limited exploration of ISAs in the context of physical collaboration, where conversations tend to be more collaborative, continuous, and shaped by physical context, rather than purely by politeness and social norms.

Although ISAs enable robots to engage in more nuanced and contextually rich social interactions, there is still a gap in the empirical evaluation of ISAs' impact on perceived task performance and user experience in HRC across physical collaborative tasks. In this study, we investigate the effects of a robot's ability to understand ISAs on team fluency, goal alignment, and human perception based on the task taxonomy for robotic manipulators concluded by~\cite{semeraro2023human}.

\subsection{Hypotheses}
Based on prior literature, we outline several hypotheses to address the research questions.

Previous research highlights that implicitly conveying contextual information through language can foster mutual understanding and facilitate smoother teamwork~\cite{frank2012predicting, clark1996using}. Therefore, we hypothesise that enabling the robot to understand ISAs will positively influence team fluency and goal alignment.
\begin{description}
    \item [H1a] Perceived team fluency will be better when the robot has the capability to understand ISAs. 
    \item [H1b] Perceived goal alignment will be better when the robot has the capability to understand ISAs.
\end{description}

Existing literature suggests that ISAs can increase trustworthiness in social interaction scenarios~\cite{saunderson2021robots}. However, there is a lack of research on their impact in physical collaborative scenarios. In this study, we hypothesise that a robot's ability to understand ISAs will positively impact trust in physical collaboration contexts.
\begin{description}
    \item [H2] Participants will perceive the robot as more trustworthy when it has the capability to understand ISAs.
\end{description}

Research shows that more human-like robots are perceived as better teammates~\cite{zhang2021ideal}. Moreover, human-like communication has been statistically proven to be highly effective in enhancing the impact of anthropomorphism compared to other anthropomorphic morphologies~\cite{roesler2021meta}. Thus, we hypothesise that the robot's ability to understand ISAs will enhance the user experience by increasing its perceived anthropomorphism, making interactions feel more human-like. 
\begin{description}
    \item [H3] Participants will perceive the robot as exhibiting greater anthropomorphism when it has the capability to understand ISAs.
\end{description}
