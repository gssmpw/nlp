\section{Formal Definition of Counterfactual Stability}\label{app:stability}

Counterfactual stability is a desirable property of SCMs~\citep{oberst2019counterfactual} that has previously been used in the context of autoregressive generation of LLMs~\cite{chatzi2024counterfactual}. In the following, we provide its formal definition along with a simple example to explain the intuition behind it. Throughout this section, $\PP^{\Ccal \,;\, do(\cdot)}$ denotes the probability of the interventional distribution entailed by an SCM $\Ccal$ under an intervention $do(\cdot)$.
% and $do(\cdot)$ denotes interventions to random variables. 
Moreover, $\PP^{\Ccal \,|\, \star \,;\, do(\cdot)}$ denotes the probability of the counterfactual distribution entailed by an SCM $\Ccal$ under an intervention $do(\cdot)$ given that an observed event $\star$ has already occurred.

\begin{definition}
    A sampling mechanism defined by $f_T$ and $P_U$ satisfies counterfactual stability if for all LLMs $m,m'\in \Mcal$, $i\in\{1,2,\ldots,K\}$ and tokens $t_1,t_2 \in V$ with $t_1\neq t_2$, the condition 
    \begin{equation}\label{eq:stability_condition}
        \frac{\PP^{\Ccal \,;\, do(M=m')}[T_i=t_1 \given D_i]}{\PP^{\Ccal \,;\, do(M=m)}[T_i=t_1 \given D_i]} \geq \frac{\PP^{\Ccal \,;\, do(M=m')}[T_i=t_2 \given D_i]}{\PP^{\Ccal \,;\, do(M=m)}[T_i=t_2 \given D_i]}
    \end{equation}
    implies that $\PP^{\Ccal \,|\, D_i,M=m,T_i=t_1 \,;\, do(M=m')}[T_i=t_2]=0$.
\end{definition}

The property of counterfactual stability has an intuitive interpretation that can be best understood via a simple example. Assume that the vocabulary contains $2$ tokens ``A'' and ``B'' and, using LLM $m$, the next-token distribution at a time step $i$ assigns values $0.6$, $0.4$ to the two tokens, respectively. Moreover, the realized noise value $\ub_i$ is such that the token ``A'' is sampled.
% , that is, $\text{``A''}=f_T(d_i, u_i)$.
Now, consider that, while keeping the noise value $\ub_i$ fixed, we change the LLM to $m'$, resulting in a next-token distribution that assigns values $0.7$, $0.3$ to the two tokens, respectively. Counterfactual stability ensures that, since the noise value $\ub_i$ led to ``A'' being sampled under $m$ at $0.6$ to $0.4$ odds, the same value
cannot lead to ``B'' being sampled under $m'$ where its relative odds are lower (\ie, $0.3$ to $0.7$).
% also leads to ``A'' being sampled under $m'$ where its relative odds are higher (\ie, $0.7$ to $0.3$).

\section{Proofs} \label{sec:proofs}

\subsection{Proof of Proposition~\ref{prop:variance}}
 We can rewrite the variance of the difference in scores under independent generation in terms of the variance of the difference in scores under coupled generation as follows:
 %
    \begin{align*}
        \Var[R_m(\Ub,S_q)-R_{m'}(\Ub',S_q)] ] 
        &= \Var[R_m(\Ub,S_q)-R_{m'}(\Ub,S_q) + R_{m'}(\Ub,S_q) -R_{m'}(\Ub',S_q)] ] \nonumber
        \\ &= \Var[R_m(\Ub,S_q)-R_{m'}(\Ub,S_q)] + \Var[R_{m'}(\Ub,S_q)-R_{m'}(\Ub',S_q)] \nn
            \\ &+ 2 \cdot \Cov[R_m(\Ub,S_q)-R_{m'}(\Ub,S_q), R_{m'}(\Ub,S_q)-R_{m'}(\Ub',S_q) ]. 
    \end{align*}
%
For the variance of the difference in scores for the same LLM under independent noise values, we have that
%
    \begin{align*}
        \Var[R_{m'}(\Ub,S_q)-R_{m'}(\Ub',S_q)]
        & \overset{(a)}{=} \EE[(R_{m'}(\Ub,S_q)-R_{m'}(\Ub',S_q))^2] -
         \EE[R_{m'}(\Ub,S_q)-R_{m'}(\Ub',S_q)]^2 % \label{eq:defvar1}
        \\& \overset{(b)}{=} \EE[R_{m'}(\Ub,S_q)^2-2\cdot R_{m'}(\Ub,S_q)R_{m'}(\Ub',S_q) + R_{m'}(\Ub',S_q)^2] % \label{eq:zeroterm}
        \\& \overset{(c)}{=} 2 \cdot \EE[R_{m'}(\Ub,S_q)^2] - 2 \cdot \EE[R_{m'}(\Ub,S_q)R_{m'}(\Ub',S_q)], % \label{eq:rewrite}
    \end{align*}
%
where (a) holds by the definition of variance, (b) is due to the subtraction term being $0$, and (c) is due to the linearity of expectation.
%
Further, for the covariance of the difference in scores under independent generation and the difference in scores under coupled generation, we have that
%
\begingroup
\allowdisplaybreaks
\begin{align*}
    \Cov[R_m(\Ub,S_q)-R_{m'}&(\Ub,S_q), R_{m'}(\Ub,S_q)-R_{m'}(\Ub',S_q)]\nonumber
    \\ % \begin{split}\label{eq:covdef}
    &\overset{(a)}{=} \EE[\left(R_m(\Ub,S_q)-R_{m'}(\Ub,S_q)\right) \cdot \left( R_{m'}(\Ub,S_q)-R_{m'}(\Ub',S_q) \right)] 
    \\ &\quad \quad - \EE[R_m(\Ub,S_q)-R_{m'}(\Ub,S_q)]\cdot \EE[R_{m'}(\Ub,S_q)-R_{m'}(\Ub',S_q)]
    % \end{split}
    \\ % \begin{split}\label{eq:zeroexp}
    &\overset{(b)}{=} \EE[R_m(\Ub,S_q)R_{m'}(\Ub,S_q)] - \EE[R_m(\Ub,S_q)R_{m'}(\Ub',S_q)]
    \\ &\quad \quad - \EE[R_{m'}(\Ub,S_q)R_{m'}(\Ub,S_q)] + \EE[R_{m'}(\Ub,S_q)R_{m'}(\Ub',S_q)] 
    \\ % \end{split}
    & \overset{(c)}{=} \Cov[R_m(\Ub,S_q),R_{m'}(\Ub,S_q)] - \EE[R_{m'}(\Ub,S_q)^2] + \EE[R_{m'}(\Ub,S_q)R_{m'}(\Ub',S_q)]] % \label{eq:covfinal}
\end{align*}
\endgroup
where (a) and (c) hold by the definition of covariance and (b) is due to the last term being zero and by the expansion of the first term.

Putting all the above results together, it follows that
%
\begin{align*}
    \Var[R_m(\Ub,S_q)-R_{m'}(\Ub',S_q)] ]
            % \\ \begin{split}
        & = \Var[R_m(\Ub,S_q)-R_{m'}(\Ub,S_q)] 
        + 2\cdot \Cov\left[R_m(\Ub,S_q),R_{m'}(\Ub,S_q)\right] \\
         & \quad+ 2\cdot \EE[R_{m'}(\Ub,S_q)R_{m'}(\Ub',S_q)] 
         - 2\cdot \EE[R_{m'}(\Ub,S_q)^2] 
         + 2 \cdot \EE[R_{m'}(\Ub,S_q)^2] \\
         & \quad - 2 \cdot \EE[R_{m'}(\Ub,S_q)R_{m'}(\Ub',S_q)]
       %  \end{split}
        \\&= \Var[R_m(\Ub,S_q)-R_{m'}(\Ub,S_q)] 
        + 2\cdot \Cov[R_m(\Ub,S_q),R_{m'}(\Ub,S_q)] 
\end{align*}
which concludes the proof.


\subsection{Proof of Proposition~\ref{prop:var_stability}}

Due to Proposition~\ref{prop:variance}, to show that Eq.~\ref{eq:binary_var} holds, it suffices to show that the covariance between the scores of the different LLMs under coupled generation is non-negative, \ie, $\Cov[R_m(\Ub,S_q),R_{m'}(\Ub,S_q)]\geq 0$.

To this end, we first rewrite the covariance as
    \begin{multline}
            \Cov[R_m(\Ub,S_q),R_{m'}(\Ub,S_q)] = \PP[R_m(\Ub,S_q)=1, R_{m'}(\Ub,S_q)=1] -\PP[R_m(\Ub,S_q)=1]\cdot \PP[R_{m'}(\Ub,S_q)=1] \\ \label{eq:covariance_rewrite}
            =\sum_{s_q} \PP[S_q=s_q] \cdot \left( \PP[R_m(\Ub,s_q)=1, R_{m'}(\Ub,s_q)=1]-\PP[R_m(\Ub,s_q)=1]\cdot \PP[R_{m'}(\Ub,s_q)=1] \right) 
        % \end{split}
    \end{multline}
%
Next, we note that the event $R_m(\Ub,s_q)=1$ is equivalent to LLM $m$ sampling the ground truth token for prompt $s_q$. 
%
Without loss of generality, assume $t_1$ is the ground truth token, \ie, $\textsc{c}(s_q)=t_1$. 
%
Then, since only tokens $\{t_1, t_2\}$ have positive probability under $m$ and $m'$, it must hold that either (i) one LLM assigns a greater probability to $t_1$ and the other LLM assigns a greater probability to $t_2$, 
%
or (ii) both LLMs assign the same probabilities. 
%
Further, since the sampling mechanism defined by $f_T$ and $P_U$ satisfies counterfactual stability, we have that the condition in Eq.~\ref{eq:stability_condition} holds in both (i) and (ii) and, under coupled generation, the LLM with greater (or equal) probability for $t_1$ will always sample $t_1$ when the LLM with lower (or equal) probability does. 
%
This implies that
%
\begin{equation}
    \PP[R_m(\Ub,s_q)=1, R_{m'}(\Ub,s_q)=1] = \min\{ \PP[R_m(\Ub,s_q)=1], \PP[R_{m'}(\Ub,s_q)=1]\}
\end{equation}
%
Finally, since it holds that
%
\begin{equation}
    \min\{ \PP[R_m(\Ub,s_q)=1], \PP[R_{m'}(\Ub,s_q)=1]\} \geq \PP[R_m(\Ub,s_q)=1] \PP[R_{m'}(\Ub,s_q)=1]
\end{equation}
because $\PP[R_m(\Ub,s_q)=1] \in (0,1)$ and $\PP[R_{m'}(\Ub,s_q)=1]\in(0,1)$ by assumption,
% (with strict inequality holding if in $(0,1)$),
we can conclude from Eq.~\ref{eq:covariance_rewrite} that
    \begin{equation}
        \begin{split}
            &\Cov[R_m(\Ub,S_q),R_{m'}(\Ub,S_q)] > 0.
        \end{split}
    \end{equation}
%
% Additionally, we can also conclude that, if $\PP[R_m(\Ub,s_q)=1] \in (0,1)$ 
% and $\PP[R_{m'}(\Ub,s_q)=1]\in (0,1)$, the above inequality is strict.
    
\subsection{Proof of Proposition~\ref{prop:var_gumbel}}
%
Using Proposition~\ref{prop:variance}, we have that
%
    \begin{align*}
            \Cov[R_m(\Ub,S_q),R_{m'}(\Ub,S_q)] &= \EE[R_m(\Ub,S_q)\cdot R_{m'}(\Ub,S_q)]-\EE[R_m(\Ub,S_q)]\cdot  \EE[R_{m'}(\Ub,S_q)]\\
            &= \underbrace{\PP[R_m(\Ub,S_q)=1, R_{m'}(\Ub,S_q)=1
            ]}_{(i)}-\underbrace{\PP[R_m(\Ub,S_q)=1) \cdot \PP[R_{m'}(\Ub,S_q)=1]}_{(ii)}.
    \end{align*}
%
In the remainder of the proof, we will bound each term (i) and (ii) separately and, since $|\textsc{c}(s_q)|=1$ for all $s_q \sim P_{Q}$, assume without loss of generality that the correct token is single-token sequence $t_1$.

To bound the term (ii), first note that, using the definition of the Gumbel-Max SCM, we have that, for each $k \in \{2, \ldots, |V|\}$, it holds that
%
\begin{align*}
            R_{m}(\Ub,s_q)=1 &\iff U_1 + \log ( [f_D(s_q,m)]_{t_1} ) \geq U_k + \log ( [f_D(s_q, m)]_{t_k} ), \\
            R_{m'}(\Ub,s_q)=1 &\iff U_1 + \log ( [f_D(s_q,m')]_{t_1} ) \geq U_k + \log ( [f_D(s_q, m')]_{t_k} ).
\end{align*}
%
Next, let $\varepsilon^*>0$ be an arbitrary constant that we will determine later such that
%
\begin{equation} \label{eq:bound}
    |\log ( [f_D(S_q,m)]_{t_k}) -\log ([f_D(S_q,m')]_{t_k}) |\leq \varepsilon^*,
\end{equation}
%
and note that since, by assumption, $D_{t_k} > 0$ for all $k \in \{1, \ldots, |V|\}$, any bound on the absolute difference of log-probabilities $|\log ( [f_D(S_q,m)]_{t_k}) -\log ([f_D(S_q,m')]_{t_k}) |$ uniformly implies a bound on the difference of probabilities $|[f_D(S_q,m)]_{t_k} -[f_D(S_q,m')]_{t_k} |$ and vice versa. 
%
For simplicity, we prove the result in the log-domain.

Now, using the bound defined by Eq.~\ref{eq:bound}, we have that
%
\begin{multline*}
        \bigcap_{k\neq1} \left\{ U_1 + \log ( [f_D(S_q, m')]_{t_1} ) \geq U_k + \log ( [f_D(S_q, m')]_{t_k} ) \right\}\\
            \subset \bigcap_{k\neq1} \left\{U_1 + \log ( [f_D(S_q, m)]_{t_1} )+\varepsilon^* \geq U_k + \log ( [f_D(S_q, m)]_{t_k} )-\varepsilon^* \right\},
\end{multline*}
%
and we can then bound the term (ii) as follows:
    \begin{align*}
            \PP[R_m(\Ub,S_q)=1] &\cdot \PP[R_{m'}(\Ub,S_q)=1] \\
            &=
            \PP[\cap_{k\neq1} \{U_1 + \log ( [f_D(S_q,m)]_{t_1} ) \geq U_k + \log ( [f_D(S_q,m)]_{t_k} )\}]\\
            & \qquad \times \PP[\cap_{k\neq1}\{ U_1 + \log ( [f_D(S_q,m')]_{t_1} ) \geq U_k + \log ( [f_D(S_q,m')]_{t_k} ) \}]\\
            &\leq \PP[\cap_{k\neq1} \{U_1 + \log ( [f_D(S_q,m)]_{t_1} ) \geq U_k + \log ( [f_D(S_q,m)]_{t_k} )\}]\\
            & \qquad \times \PP[\cap_{k\neq1} \{U_1 + \log ( [f_D(S_q,m)]_{t_1} )+\varepsilon^* \geq U_k + \log ( [f_D(S_q,m)]_{t_k} )-\varepsilon^*\}].
    \end{align*}
%
To bound the term (i), first note that, using the bound defined by Eq.~\ref{eq:bound}, we have that
    \begin{multline*}
        \bigcap_{k\neq1} \left\{U_1 + \log ( [f_D(S_q,m')]_{t_1} ) \geq U_k + \log ( [f_D(S_q,m')]_{t_k} ) \right\}\\
         \supset \bigcap_{k\neq1} \left\{U_1 + \log ( [f_D(S_q,m)]_{t_1} )-\varepsilon^* \geq U_k + \log ( [f_D(S_q,m)]_{t_k} )+\varepsilon^* \right\}.
    \end{multline*} 
%        
Thus, we can bound the term (i) as follows:
%
    \begin{align*}
        \PP[R_m(\Ub,S_q)=1, R_{m'}(\Ub,S_q)=1] &=\PP\Big[\cap_{k\neq1} \{U_1 + \log ( [f_D(S_q,m)]_{t_1} ) \geq U_k + \log ([f_D(S_q,m)]_{t_k} )\}\\
            & \qquad \cap \{ U_1 + \log ( [f_D(S_q,m')]_{t_1} ) \geq U_k + \log ( [f_D(S_q,m')]_{t_k} ) \}\Big] \\
            & \geq \PP\Big[\cap_{k\neq1} \{U_1 + \log ( [f_D(S_q,m)]_{t_1} ) \geq U_k + \log ( [f_D(S_q,m)]_{t_k} )\}\\
            & \qquad \cap \{ U_1 + \log ( [f_D(S_q,m)]_{t_1} ) \geq U_k + \log ( [f_D(S_q,m)]_{t_k} ) +2\varepsilon^* \}\Big] \\
            & \overset{(a)}{=} \sum_{s_q} \PP[S_q=s_q]\cdot \PP[\cap_{k\neq1} \{ U_1 + \log ( [f_D(S_q,m)]_{t_1} ) \\
            & \geq U_k + \log ( [f_D(S_q,m)]_{t_k} ) +2\varepsilon^* \}],
    \end{align*}
where (a) follows from the fact that
%
\begin{multline*}
      \left\{ U_1 + \log ( [f_D(S_q,m)]_{t_1} ) \geq U_k + \log ( [f_D(S_q,m)]_{t_k} ) +2\varepsilon^* \right\} \\
      \subset \left\{ U_1 + \log ( [f_D(S_q,m)]_{t_1} ) \geq U_k + \log ( [f_D(S_q,m)]_{t_k} ) \right\}.
\end{multline*}
%
Now, note that, for $k \in \{2, \ldots, |V|\}$, the variable $X_k \equiv U_1-U_k \sim \text{Logistic}(0,1)$ (for $k=1$, define $X_k \equiv 0$). 
%
Therefore, we can rewrite the bound for (i) as
%  
\begin{multline*}
      \PP[R_m(\Ub,S_q)=1, R_{m'}(\Ub,S_q)=1]\\
      \geq\sum_{s_q} \PP[S_q=s_q] \cdot \prod_{k\neq1}\cdot \PP[ \{ X_k \geq \log ( [f_D(S_q,m)]_{t_k} )-\log ( [f_D(S_q,m)]_{t_1} ) +2\varepsilon^* \}]
\end{multline*}
%
and we can rewrite the bound for (ii) as
%
\begin{multline*}
    \PP[R_m(\Ub,S_q)=1] \PP[R_{m'}(\Ub,S_q)=1]\leq \\
    \sum_{s_q}\PP[S_q=s_q]\cdot 
            \left\{
            \prod_{k\neq1}
            \PP[ \{ X_k \geq \log ( [f_D(S_q,m)]_{t_k} )-\log ( [f_D(S_q,m)]_{t_k} ) - 2\varepsilon^* \} ]
            \right\}\\
            \times \PP[\cap_{k\neq1} \{X_k \geq \log ( [f_D(S_q,m)]_{t_k} )-\log ( [f_D(S_q,m)]_{t_k} )\}].
\end{multline*}
%
As a consequence, to prove that $\PP[R_m(\Ub,S_q)=1, R_{m'}(\Ub,S_q)=1] > \PP[R_m(\Ub,S_q)=1] \PP[R_{m'}(\Ub,S_q)=1]$, it suffices to show that
    \begin{multline}\label{eq:condition-covariance-binary-scores}
            \sum_{s_q}P[S_q=s_q] \prod_{k\neq1}\cdot \PP[ \{ X_k \geq \log ( [f_D(S_q,m)]_{t_k} )-\log ( [f_D(S_q,m)]_{t_1} ) +2\varepsilon^* \}]\\
            > \sum_{s_q}\PP[S_q=s_q] \prod_{k\neq1}\cdot \PP[ \{ X_k \geq \log ( [f_D(S_q,m)]_{t_k} )-\log ( [f_D(S_q,m)]_{t_1} ) - 2\varepsilon^* \}]
            \\
            \times \PP[\cap_{k\neq1} \{X_k \geq \log ( [f_D(S_q,m)]_{t_k} )-\log ( [f_D(S_q,m)]_{t_1} )\}]
    \end{multline}
%
To do so, note that Eq.~\ref{eq:condition-covariance-binary-scores} holds trivially for $\varepsilon^*=0$ since
%
\begin{equation*}
      \PP[\cap_{k\neq1} \{X_k \geq \log ( [f_D(S_q,m)]_{t_k} )-\log ( [f_D(S_q,m)]_{t_1} )\}] < 1,
\end{equation*}
%
which is a fixed term independent of $m'$. Since all terms in Eq.~\ref{eq:condition-covariance-binary-scores} are continuous in $\varepsilon^*$, there exists $\varepsilon^*(m) >0 $, possibly dependent of $m$ but independent of $m'$, such that Eq.~\ref{eq:condition-covariance-binary-scores} holds if
\begin{equation*}
    \sup_{s_q} \norm{\log(f_D(s_q,m))-\log(f_D(s_q, m'))}_\infty < \varepsilon^*(m).
\end{equation*}
Since by assumption $D_t>0$ for all $t\in V$, there exists $\varepsilon(m)>0$ in probability space such that Eq.~\ref{eq:condition-covariance-binary-scores} holds if
\begin{equation*}
    \sup_{s_q} \norm{f_D(s_q,m)-f_D(s_q, m')}_\infty < \varepsilon(m).
\end{equation*}
%
This concludes the proof.

\subsection{Proof of Proposition~\ref{prop:gap_win_rates_stability}}

% We first compute the win-rates under coupled autoregressive generation.
%
Under coupled autoregressive generation, if the LLM $m$ samples the preferred token $t_+$, then the LLM $m'$ must also sample $t_{+}$ because $t_{+}$ is more likely under $m'$ than under $m$ and the sampling mechanism defined by $f_T$ and $P_U$ satisfies counterfactual stability. 
%
This implies that the win-rate achieved by $m$ against $m'$ is
%
    \begin{equation}\label{eq:prop_greater_shared}
            \EE_{\Ub\sim P_{\Ub}} [\one\{R_m(\Ub,s_q)>R_{m'}(\Ub,s_q)\}] =
            \PP[f_T(f_D(s_q, m),\Ub)= t_+, f_T(f_D(s_q, m'),\Ub)= t_-]=0 
    \end{equation}
and that
    \begin{equation}\label{eq:prop_equal_probs+}
            \PP[f_T(f_D(s_q, m),\Ub)= t_+, f_T(f_D(s_q, m'),\Ub)= t_+]= \PP[f_T(f_D(s_q, m),\Ub)= t_+] =p_m.
    \end{equation}
%
Using the same reasoning, if the LLM $m'$ samples the non-preferred token $t_{-}$, then, $m$ must also sample $t_{-}$ because $t_{-}$ is more likely under $m$ than under $m'$. This implies that
%
    \begin{equation}\label{eq:prop_equal_probs-}
            \PP[f_T(f_D(s_q, m),\Ub)= t_-, f_T(f_D(s_q, m'),\Ub)= t_-]= \PP[f_T(f_D(s_q, m'),\Ub)= t_-] = 1-p_{m'}
    \end{equation} 
%
Then, from Eq.~\ref{eq:prop_equal_probs+} and Eq.~\ref{eq:prop_equal_probs-}, we can conclude that
%
    \begin{equation}\label{eq:prop_equal_shared}
        \begin{aligned}
            \EE_{\Ub\sim P_{\Ub}} [\one\{R_m(\Ub,s_q)=R_{m'}(\Ub,s_q)\}] = p_m + (1-p_{m'})
        \end{aligned}
    \end{equation}
%
Finally, from Eq.~\ref{eq:prop_greater_shared} and Eq.~\ref{eq:prop_equal_shared}, we can conclude that the win-rate achieved by $m'$ against $m$ is
%
\begin{multline*}
    \EE_{\Ub\sim P_{\Ub}} [\one\{R_m(\Ub,s_q)<R_{m'}(\Ub,s_q)\}] \\
        = 1 - \EE_{\Ub\sim P_{\Ub}} [\one\{R_m(\Ub,s_q)>R_{m'}(\Ub,s_q)\}] -\EE_{\Ub\sim P_{\Ub}} [\one\{R_m(\Ub,s_q)=R_{m'}(\Ub,s_q)\}] = p_{m'}-p_m.
\end{multline*}

% Next, we compute the win-rates under independent autoregressive generation.
%
Under independent autoregressive generation, the LLMs $m$ and $m'$ sample tokens independently from each other, \ie, $f_T(f_D(s_q, m),\Ub) \perp f_T(f_D(s_q, m'),\Ub')$. 
%
Thus, we can factorize all joint probabilities when computing the win-rates and obtain
%
\begin{align*}\nonumber
    % \begin{split}
        \EE_{\Ub, \Ub'\sim P_{\Ub}} [\one\{R_m(\Ub,s_q)>R_{m'}(\Ub',s_q)\}] &=\PP[f_T(f_D(s_q, m),\Ub)= t_+] \cdot \PP[f_T(f_D(s_q, m'),\Ub')= t_-] \\ &=
        p_m \cdot (1-p_{m'})
    % \end{split}
\end{align*}
and
\begin{equation}\nonumber
    \begin{split}
        \EE_{\Ub, \Ub'\sim P_{\Ub}} & [\one\{R_m(\Ub,s_q)<R_{m'}(\Ub',s_q)\}] 
         = p_{m'} \cdot (1-p_m).
        % we do not write it the = rates in the proposition 
        %\\
        % \text{and} \quad \EE_{\Ub\sim P_{\Ub}, \Ub'\sim P_{\Ub}, S_q\sim P_{Q}} & [\one\{R_m(\Ub,S_q)=R_{m'}(\Ub',S_q)\}] 
        %  = p_m(p_m+\delta) + (1-p_m-\delta)(1-p_m)
    \end{split}
\end{equation}


\subsection{Proof of Proposition~\ref{prop:gumbel_ties}}

We follow the notations and technique of Proposition~\ref{prop:var_gumbel}.
Fix query $s_q$ and consider first the case of independent autoregressive generation. Since each LLM can only assign a non-zero probability to single-token sequences, we have:
\begin{equation*}
    \begin{split}
        \PP[R_m(\Ub, s_q)=R_{m'}(\Ub', s_q)]&=\sum_{k=1}^{|V|}\PP[f_T(f_D(s_q,m),\Ub)=t_k] \PP[f_T(f_D(s_q,m'),\Ub)=t_k]\\
        &<\sum_{k=1}^{|V|}\PP[f_T(f_D(s_q,m),\Ub)=t_k],
    \end{split}
\end{equation*}
%
In the case of coupled autoregressive generation, since
\begin{equation*}
    \PP\left[\{f_T(f_D(s_q,m),\Ub)=t_k\} \cap \{f_T(f_D(s_q,m),\Ub)=t_j \}\right]=0,\; i\neq j,
\end{equation*}
%
we obtain:
\begin{equation*}
    \begin{split}
        &\PP[R_m(\Ub, s_q)=R_{m'}(\Ub, s_q)]\\
        &=\PP\left[\cup_i \{ f_T(f_D(s_q,m),\Ub)=t_k, f_T(f_D(s_q,m'),\Ub)=t_k \} \right]\\
        &=\sum_k\PP[\{f_T(f_D(s_q,m),\Ub)=t_k, f_T(f_D(s_q,m'),\Ub)=t_k \}]\\
        &=\sum_k \PP[f_T(f_D(s_q,m),\Ub)=t_k]\PP[f_T(f_D(s_q,m'),\Ub)=t_k|f_T(f_D(s_q,m),\Ub)=t_k].\\
    \end{split}
\end{equation*}
We now follow \cite{9729603} and expand the posterior Gumbels, $\PP[f_T(f_D(s_q,m'),\Ub)=t_k|f_T(f_D(s_q,m),\Ub)=t_k]$, as truncated Gumbel distributions. In particular, we leverage the fact that
\begin{equation}\label{eq:max of gumbels}
    \max_{t\in V}\{U_t+\log([f_D(s_q,\bullet)]_{t})\} \sim \text{Gumbel}(0,1),
\end{equation}
and that a Gumbel distribution, with parameter $\log (\theta)$, truncated at $b\sim \text{Gumbel}(0,1)$  can be sampled as
\begin{equation}\label{eq: truncated gumbel}
    -\log(\exp(-b)-\log(\eta)/\theta),\; \eta \sim U(0,1).
\end{equation}
Furthermore, by assumption, $D_{t_k} > 0$ for all $k \in \{1, \ldots, |V|\}$, so that any bound on the absolute difference of log-probabilities $|\log ( [f_D(s_q,m)]_{t_k}) -\log ([f_D(s_q,m')]_{t_k}) |$ uniformly implies a bound on the difference of probabilities $|[f_D(s_q,m)]_{t_k} -[f_D(s_q,m')]_{t_k} |$ and vice versa. Using the bound 
\begin{equation*}
    |\log([f_D(s_q,m)]_{t_k})-\log([f_D(s_q,m')]_{t_k})|\leq \varepsilon^*
\end{equation*}
and the Gumbel properties in Eq.~\ref{eq:max of gumbels} and Eq.~\ref{eq: truncated gumbel}, we obtain:
%
\begin{align}
        &\PP[R_m(\Ub, s_q)=R_{m'}(\Ub, s_q)] \nn \\
        &=\sum_k \PP[f_T(f_D(s_q,m),\Ub)=t_k] \nn \\
            & \qquad \quad \times \PP\Bigg[
            \bigcap_k \Big\{
            \log( [f_D(s_1, m')]_{t_k}) - \log([f_D(s_1, m)]_{t_k})-\log(-\log (\eta_k)) \nn \\
            & \qquad \qquad \qquad \qquad \geq \log([f_D(s_1, m')]_{t_j}) - \log([f_D(s_1, m)]_{t_j})-\log(-\log(\eta_k) -\log(\eta_j)/[f_D(s_1,m')]_{t_j})
            \Big\}
            \Bigg] \nn \\
            &\geq
            \sum_k \PP[f_T(f_D(s_q,m),\Ub)=t_k] \nn \\
            &\qquad \quad \times\PP\left[
            \cap_k \{
            -\log(-\log (\eta_k)) \geq -2\varepsilon^* -\log(-\log(\eta_k) -\log(\eta_j)/[f_D(s_1,m')]_{t_j})
            \}
            \right] \label{eq:final-bound}
\end{align}
where $\eta_k\sim \text{U}(0,1)$ are independently distributed uniform random variables.
%
Now, note that the claim holds for $\varepsilon^*=0$ since, in that case, we have that
%
\begin{equation*}
   \PP\left[
        \bigcap_k \left\{
        -\log(-\log (\eta_k)) \geq -\log(-\log(\eta_k) -\log(\eta_k)/[f_D(s_1,m')]_{t_k})
        \right\} \right]=1,
\end{equation*}
%
using that $x\mapsto -\log(x)$ is strictly decreasing. 
%
Since all terms in Eq.~\ref{eq:final-bound} are continuous in $\varepsilon^*$, there exists $\varepsilon^*(m) >0 $, possibly dependent on $m$ but independent of $m'$, such that 
\begin{equation}\label{eq:prop5 claim prob}
    \PP[R_m(\Ub, s_q)=R_{m'}(\Ub, s_q)] > \PP[R_m(\Ub, s_q)=R_{m'}(\Ub', s_q)]
\end{equation}
holds if
\begin{equation*}
    \sup_{s_q} \norm{\log(f_D(s_q,m))-\log(f_D(s_q, m'))}_\infty < \varepsilon^*(m).
\end{equation*}
Since by assumption $D_t>0$ for all $t\in V$, there exists $\varepsilon(m)>0$ in probability space such that Eq.~\ref{eq:prop5 claim prob} holds if

\begin{equation*}
    \sup_{s_q} \norm{f_D(s_q,m)-f_D(s_q, m')}_\infty < \varepsilon(m).
\end{equation*}
%
This concludes the proof.






\subsection{Calculation of average win-rates in the example used in Sections~\ref{sec:intro} and~\ref{sec:pairwise}} \label{app:example-ranking}
%
In this section, we provide detailed calculations of the win-rates for the example in Sections~\ref{sec:intro} and~\ref{sec:pairwise}. Recall that in this example, we are given three LLMs $m_1$, $m_2$ and $m_3$, and we need to rank them according to their ability to answer correctly two types of input prompts, $q$ and $q'$, picked uniformly at random.
%
We assume that the true probability that each LLM answers correctly each type of input prompt is given by:
%
\begin{table}[H]
    \centering
    \begin{tabular}{lccc}
        \toprule
         & $m_1$ & $m_2$ & $m_3$ \\
        \midrule
        $q$           & $p_1=0.4$             & $p_2=0.48$           & $p_3=0.5$             \\
        $q'$           & $p'_1=1$           & $p'_2=0.9$         & $p'_3=0.89$              \\
        \bottomrule
 \end{tabular}
 \vspace{-8mm}
 \caption*{}
 \end{table}

Using Proposition~\ref{prop:gap_win_rates_stability}, 
% under the assumption of counterfactual stability, 
the win-rates under independent autoregressive generation are given, for each LLM $m_k$, by:
\begin{equation}
    \begin{split}
       \frac{1}{2}\sum_{j\neq k}\EE_{\Ub, \Ub' \sim P_{\Ub}, S_q\sim P_Q}[\one\{R_{m_k}(\Ub, S_q)>R_{m_j}(\Ub', S_q)\}] & =\frac{\sum_{j\neq k}p_k(1-p_j) + \sum_{j\neq k}p'_k(1-p'_j)}{4}.
    \end{split}
\end{equation}
%
Substituting the numerical values we obtain:

 \begin{equation}
    \begin{split}
       \frac{1}{2}\sum_{j\neq 1}\EE_{\Ub, \Ub' \sim P_{\Ub}, S_q\sim P_Q}[\one\{R_{m_1}(\Ub, S_q)>R_{m_j}(\Ub', S_q)\}] & =0.1545,\\
       \frac{1}{2}\sum_{j\neq 2}\EE_{\Ub, \Ub' \sim P_{\Ub}, S_q\sim P_Q}[\one\{R_{m_2}(\Ub, S_q)>R_{m_j}(\Ub', S_q)\}] & =0.15675,\\
       \frac{1}{2}\sum_{j\neq 3}\EE_{\Ub, \Ub' \sim P_{\Ub}, S_q\sim P_Q}[\one\{R_{m_3}(\Ub, S_q)>R_{m_j}(\Ub', S_q)\}] & =0.16225\\
    \end{split}
\end{equation}

Similarly, using Proposition~\ref{prop:gap_win_rates_stability}, the win-rates using coupled autoregressive generation can be written, for each LLM $m_k$, as:
\begin{equation}
    \begin{split}
       \frac{1}{2}\sum_{j\neq k}\EE_{\Ub \sim P_{\Ub}, S_q\sim P_Q}[\one\{R_{m_k}(\Ub, S_q)>R_{m_j}(\Ub, S_q)\}] & =\frac{\sum_{j\neq k}(p_k-p_j)_{+} + \sum_{j\neq k}(p'_k-p'_j)_{+}}{4},
    \end{split}
\end{equation}
where $(\bullet)_{+}=\max(0, \bullet)$ denotes the positive part. Substituting the numerical values we obtain:

 \begin{align*}
       \frac{1}{2}\sum_{j\neq 1}\EE_{\Ub \sim P_{\Ub}, S_q\sim P_Q}[\one\{R_{m_1}(\Ub, S_q)>R_{m_j}(\Ub, S_q)\}] & = 
       % \frac{(p_1-p_2)_{+} +(p_1-p_3)_{+} +(p'_1-p'_2)_{+} +(p'_1-p'_3)_{+}}{4}
       0.0525, \\
       \frac{1}{2}\sum_{j\neq 2}\EE_{\Ub \sim P_{\Ub}, S_q\sim P_Q}[\one\{R_{m_2}(\Ub, S_q)>R_{m_j}(\Ub, S_q)\}] &=
       % \frac{(p_2-p_1)_{+} +(p_2-p_3)_{+} +(p'_2-p'_1)_{+} +(p'_2-p'_3)_{+}}{4}=
       0.0225, \\
       \frac{1}{2}\sum_{j\neq 3}\EE_{\Ub \sim P_{\Ub}, S_q\sim P_Q}[\one\{R_{m_3}(\Ub, S_q)>R_{m_j}(\Ub, S_q)\}] &=
       % \frac{(p_3-p_2)_{+} +(p_3-p_1)_{+} +(p'_3-p'_2)_{+} +(p'_3-p'_1)_{+}}{4}=
       0.03.
\end{align*}






\newpage

\section{Additional Experimental Details} \label{app:add_exp}
\vspace{-2mm}

\xhdr{Hardware setup} Our experiments are executed on a compute server equipped with 2 $\times$ Intel Xeon Gold 5317 CPU, $1{,}024$ GB main memory, and $2$ $\times$ A100 Nvidia Tesla GPU ($80$ GB, Ampere Architecture). In each experiment a single Nvidia A100 GPU is used.


\xhdr{Datasets} 
As a benchmark dataset, we use Measuring Massive Multitask Language Understanding dataset (MMLU)~\cite{hendrycks2021measuring} consisting of $14{,}042$ questions covering $52$ diverse knowledge areas with each question offering four possible choices indexed from A to D, and a ground-truth answer. For pairwise comparison tasks, we use the first $500$ questions from the LMSYS-Chat-1M dataset~\cite{zheng2024lmsys}.

\xhdr{Models} In our experiments, we 
% evaluate popular large language models from the \texttt{Llama} family sharing 
% the same token vocabulary.  
%Specifically, we 
use \texttt{Llama-3.1-8B-Instruct}, its quantized variants \texttt{Llama-3.1-8B-Instruct-\{AWQ-INT4, bnb-4bit, bnb-8bit\}} and \texttt{Llama-3.2-\{1B, 3B\}-Instruct} models. The models are obtained from \texttt{Hugging Face}, and the quantised LLM variants \texttt{Llama-3.1-8B-Instruct-\{bnb-4bit, bnb-8bit\}} are built using the \texttt{bitsandbytes} library~\cite{bitsnbytes2024bits}.

\xhdr{Prompts} To instruct LLMs for generating output, we use the system prompt in Table~\ref{app:sys_prompt_2} for the MMLU dataset and Table~\ref{app:sys_prompt_3} for the LMSYS-Chat-1M dataset. Further, to perform pairwise comparisons of outputs of different LLMs, we use the system prompt in Table~\ref{app:sys_prompt_1}, which is adapted from~\cite{chiang2024chatbot}, to prompt the strong LLM.

\begin{table}[h]
\centering
\begin{tcolorbox}[
    colframe=white,      % Border color
    colback=gray!14,     % Background color
    boxrule=0.5mm,       % Border thickness
    arc=4mm,             % Rounded corners
    left=3mm,            % Left margin
    right=3mm,           % Right margin
    top=3mm,             % Top margin
    bottom=3mm           % Bottom margin
]
\begin{tabular}{ m{15.2cm} }
\rowcolor{gray!14} 
    \textbf{System:} Please act as an impartial judge and evaluate the quality of the responses provided by two AI assistants to the user prompt displayed below. Your job is to evaluate which assistant's answer is better. When evaluating the assistants' answers, compare both assistants' answers. You must identify and correct any mistakes or inaccurate information. Then consider if the assistant's answers are helpful, relevant, and concise. Helpful means the answer correctly responds to the prompt or follows the instructions. Note when user prompt has any ambiguity or more than one interpretation, it is more helpful and appropriate to ask for clarifications or more information from the user than providing an answer based on assumptions. Relevant means all parts of the response closely connect or are appropriate to what is being asked. Concise means the response is clear and not verbose or excessive. Then consider the creativity and novelty of the assistant's answers when needed. Finally, identify any missing important information in the assistants' answers that would be beneficial to include when responding to the user prompt. do not provide any justification or explanation for your response. You must output only one of the following choices as your final verdict: 

    \vspace{2mm}
    `A' if the response of assistant A is better
    
    \vspace{1mm}
    `B' if the response of assistant B is better
    
    \vspace{1mm}
    `Tie' if the responses are tied
\end{tabular}
\end{tcolorbox}
\vspace{-3mm}
\caption{\label{app:system_prompt}System prompt used for obtaining pairwise preferences using \texttt{GPT-4o-2024-11-20} as the judge.}
\label{app:sys_prompt_1}
\end{table}


\vspace{-3mm}

\begin{table}[h]
\centering
\begin{tcolorbox}[
    colframe=white,      % Border color
    colback=gray!14,     % Background color
    boxrule=0.5mm,       % Border thickness
    arc=4mm,             % Rounded corners
    left=3mm,            % Left margin
    right=3mm,           % Right margin
    top=3mm,             % Top margin
    bottom=3mm           % Bottom margin
]
\begin{tabular}{ m{15.2cm} }
    \rowcolor{gray!14}
    \textbf{System:} You will be given multiple choice questions. Please reply with a single character `A', `B', `C', or `D' only. DO NOT explain your reply.
\end{tabular}
\end{tcolorbox}
\vspace{-3mm}
\caption{System prompt used for the MMLU dataset.}
\label{app:sys_prompt_2}
\end{table}


\vspace{-3mm}

\begin{table}[h]
\centering
\begin{tcolorbox}[
    colframe=white,      % Border color
    colback=gray!14,     % Background color
    boxrule=0.5mm,       % Border thickness
    arc=4mm,             % Rounded corners
    left=3mm,            % Left margin
    right=3mm,           % Right margin
    top=3mm,             % Top margin
    bottom=3mm           % Bottom margin
]
\begin{tabular}{ m{15.2cm} }
    \rowcolor{gray!14}
    \textbf{System:} Keep your responses short and to the point.
\end{tabular}
\end{tcolorbox}
\vspace{-3mm}
\caption{System prompt used for the LMSYS Chatbot Arena dataset.}
\label{app:sys_prompt_3}
\end{table}






\clearpage
\newpage

% \section{Additional Experimental Results} \label{app:exp_results}
%In this section, we present experimental results supplementing those from Section~\ref{sec:experiments}, first for the MMLU dataset, and then the LMSYS-Chat-1M dataset.

\section{Additional Experimental Results on the MMLU Dataset} \label{app:mmlu}
% Here, we present experimental results supplementing the results from 
% Section~\ref{sec:experiments}, comparing different pairs of models in 
% Figure~\ref{fig:mmlu-1B-vs-3B-models} and different knowledge areas in 
% Figure~\ref{fig:mmlu-1B-vs-3B-areas}. 

\begin{figure}[h]
\centering
\begin{tabular}{c c c}
    \multicolumn{3}{c}{\texttt{Llama-3.2-1B-Instruct} vs. \texttt{Llama-3.2-3B-Instruct}}\\
    \includegraphics[width=0.27\linewidth]{./figures/mmlu/covariance/college_computer_science/cov_Llama-3.2-3B-Instruct_Llama-3.2-1B-Instruct_kde.pdf} &
    \includegraphics[width=0.27\linewidth]{./figures/mmlu/variance/college_computer_science/var_Llama-3.2-3B-Instruct_Llama-3.2-1B-Instruct_kde.pdf} &
    \includegraphics[width=0.27\linewidth]{./figures/mmlu/error/college_computer_science/error_Llama-3.2-3B-Instruct_Llama-3.2-1B-Instruct.pdf} \\ \\
%
    \multicolumn{3}{c}{\texttt{Llama-3.2-1B-Instruct} vs. \texttt{Llama-3.1-8B-Instruct}}\\
    \includegraphics[width=0.27\linewidth]{./figures/mmlu/covariance/college_computer_science/cov_Llama-3.2-1B-Instruct_Llama-3.1-8B-Instruct_kde.pdf} &
    \includegraphics[width=0.27\linewidth]{./figures/mmlu/variance/college_computer_science/var_Llama-3.1-8B-Instruct_Llama-3.2-1B-Instruct_kde.pdf} &
    \includegraphics[width=0.27\linewidth]{./figures/mmlu/error/college_computer_science/error_Llama-3.1-8B-Instruct_Llama-3.2-1B-Instruct.pdf} \\ \\
%
    \multicolumn{3}{c}{\texttt{Llama-3.2-3B-Instruct} vs. \texttt{Llama-3.1-8B-Instruct}}\\
    \includegraphics[width=0.27\linewidth]{./figures/mmlu/covariance/college_computer_science/cov_Llama-3.1-8B-Instruct_Llama-3.2-3B-Instruct_kde.pdf} &
    \includegraphics[width=0.27\linewidth]{./figures/mmlu/variance/college_computer_science/var_Llama-3.1-8B-Instruct_Llama-3.2-3B-Instruct_kde.pdf} &
    \includegraphics[width=0.27\linewidth]{./figures/mmlu/error/college_computer_science/error_Llama-3.1-8B-Instruct_Llama-3.2-3B-Instruct.pdf} \\ \\
%
    (a) Score covariance & (b) Variance of the score difference & (c) Estimation error vs. \# samples \\ 
    
\end{tabular}
    \caption{\textbf{Comparison between three pairs of LLMs on multiple-choice questions from the ``college computer science'' knowledge area of the MMLU dataset.}
    Panels in column (a) show the kernel density estimate (KDE) of the covariance between the scores of the two LLMs on each question under coupled generation; the dashed lines correspond to average values. Panels in column (b) show the KDE of the variance of the difference between the scores of the LLMs on each question under coupled and independent generation; the highlighted points correspond to median values. Panels in column (c) show the absolute error in the estimation of the expected difference between the scores of the LLMs against the number of samples; for each point on the x-axis, we perform $1{,}000$ sub-samplings and shaded areas correspond to $95\%$ confidence intervals.}
    \label{fig:mmlu-1B-vs-3B-models}
\end{figure}


\begin{figure}[h]
\centering
\begin{tabular}{c c c}
%
    \multicolumn{3}{c}{\textbf{College chemistry}} \\
    \includegraphics[width=0.27\linewidth]{./figures/mmlu/covariance/college_chemistry/cov_Llama-3.2-3B-Instruct_Llama-3.2-1B-Instruct_kde.pdf} &
    \includegraphics[width=0.27\linewidth]{./figures/mmlu/variance/college_chemistry/var_Llama-3.2-3B-Instruct_Llama-3.2-1B-Instruct_kde.pdf} &
    \includegraphics[width=0.27\linewidth]{./figures/mmlu/error/college_chemistry/error_Llama-3.2-3B-Instruct_Llama-3.2-1B-Instruct.pdf} \\
%
    \multicolumn{3}{c}{\textbf{Professional accounting}} \\
    \includegraphics[width=0.27\linewidth]{./figures/mmlu/covariance/professional_accounting/cov_Llama-3.2-3B-Instruct_Llama-3.2-1B-Instruct_kde.pdf} &
    \includegraphics[width=0.27\linewidth]{./figures/mmlu/variance/professional_accounting/var_Llama-3.2-3B-Instruct_Llama-3.2-1B-Instruct_kde.pdf} &
    \includegraphics[width=0.27\linewidth]{./figures/mmlu/error/professional_accounting/error_Llama-3.2-3B-Instruct_Llama-3.2-1B-Instruct.pdf} \\
% 
    \multicolumn{3}{c}{\textbf{Professional law}}\\
    \includegraphics[width=0.27\linewidth]{./figures/mmlu/covariance/professional_law/cov_Llama-3.2-3B-Instruct_Llama-3.2-1B-Instruct_kde.pdf} &
    \includegraphics[width=0.27\linewidth]{./figures/mmlu/variance/professional_law/var_Llama-3.2-3B-Instruct_Llama-3.2-1B-Instruct_kde.pdf} &
    \includegraphics[width=0.27\linewidth]{./figures/mmlu/error/professional_law/error_Llama-3.2-3B-Instruct_Llama-3.2-1B-Instruct.pdf} \\ \\
%
    \multicolumn{3}{c}{\textbf{Professional medicine}} \\   
    \includegraphics[width=0.27\linewidth]{./figures/mmlu/covariance/professional_medicine/cov_Llama-3.2-3B-Instruct_Llama-3.2-1B-Instruct_kde.pdf} &
    \includegraphics[width=0.27\linewidth]{./figures/mmlu/variance/professional_medicine/var_Llama-3.2-3B-Instruct_Llama-3.2-1B-Instruct_kde.pdf} &
    \includegraphics[width=0.27\linewidth]{./figures/mmlu/error/professional_medicine/error_Llama-3.2-3B-Instruct_Llama-3.2-1B-Instruct.pdf} \\
%
    (a) Score covariance & (b) Variance of the score difference & (c) Estimation error vs. \# samples \\ 
\end{tabular}
    \caption{\textbf{Comparison between \texttt{Llama-3.2-1B-Instruct} and \texttt{Llama-3.2-3B-Instruct} on multiple-choice questions from four knowledge areas of the MMLU dataset.}
    Panels in column (a) show the kernel density estimate (KDE) of the covariance between the scores of the two LLMs on each question under coupled generation; the dashed lines correspond to average values. Panels in column (b) show the KDE of the variance of the difference between the scores of the LLMs on each question under coupled and independent generation; the highlighted points correspond to median values. Panels in column (c) show the absolute error in the estimation of the expected difference between the scores of the LLMs against the number of samples; for each point on the x-axis, we perform $1{,}000$ sub-samplings and shaded areas correspond to $95\%$ confidence intervals. We observe qualitatively similar results for other knowledge areas.}
    \label{fig:mmlu-1B-vs-3B-areas}
\end{figure}

\clearpage
\newpage

\section{Additional Experimental Results on the LMSYS-Chat-1M Dataset} \label{app:lmsys}
% Here, we report the empirical win-rate of each LLM against other LLMs using 
% coupled and independent generation in Figure~\ref{fig:lmsys-all-llms}.
%
\begin{figure}[h]
\centering
\includegraphics[width=0.72\linewidth]{figures/lmsys/all.pdf}
%
\caption{
\textbf{Empirical win-rate of each LLM against other LLMs 
on questions from the LMSYS-Chat-1M dataset.} 
%
Empirical estimate of the win-rate under coupled autoregressive generation as given by Eq.~\ref{eq:coupled-generation-win-rates} and under independent generation generation as given by  Eq.~\ref{eq:independent-generation-win-rates}. 
%
Each empirical win-rate is computed using pairwise comparisons between the outputs of each LLM and any other LLM over $500$ questions with $10$ (different) random seeds.
%
The error bars correspond to $95\%$ confidence intervals.
%
For each pair of empirical win-rates, we conduct a two-tailed test, to test the hypothesis that the empirical win-rates are the same; (\fourstars, \threestars, \twostars, \onestar) indicate $p$-values ($<0.0001$, $<0.001$, $<0.01$, $<0.05$), respectively.
}
\label{fig:lmsys-all-llms}
\end{figure}
