\section{Alternative Views}

While this position paper argues that Scaling Up encounters significant bottlenecks and that future trends will shift towards Scaling Down and Scaling Out, an alternative view is that \textit{Scaling Up remains a viable trajectory despite the challenges}. Supporters argue that addressing these challenges is necessary and feasible through interdisciplinary innovation. Key challenges must be overcome include data quality, computational demands, and energy consumption.

Firstly, although data quantity has grown, their quality has not kept pace. Synthetic data offers a controlled alternative but may introduce biases and lack real-world applicability.
As for computational demands, computational power is a bottleneck, as traditional hardware faces physical and economic limits. Alternatives like quantum, optical, and neuromorphic computing might help.
Concern about energy consumption emphasizes the need for sustainable AI. Low-power chips and renewable energy integration 
could reduce the environmental impact of large-scale computing.

Unlike Scaling Down and Scaling Out, which provide immediate solutions, Scaling Up requires extensive interdisciplinary collaboration. This challenge extends beyond AI research to fields such as hardware engineering, quantum mechanics, and sustainable energy solutions. The long research and development cycles make Scaling Up a long-term strategy. A concern regarding it is technological breakthroughs are unpredictable. 
Therefore, a balanced strategy is necessary, where both short- and long-term solutions are invested, rather than exclusively focusing on one of them.
