\section{Scaling Out: Advancing  AI Ecosystems} \label{sec:scaling_out}
Scaling Up and Scaling Down represent two complementary approaches to AI scaling, yet neither fully realizes AI’s potential in real-world applications. Scaling Up builds larger, generalized models like GPT and BERT, but their resource demands limit accessibility and task-specific adaptability. Scaling Down optimizes models for efficiency, enabling deployment in resource-constrained environments, but struggles with adaptability, collaboration, and decentralized intelligence. To address these gaps, AI must evolve into a distributed ecosystem where multiple AI entities interact, specialize, and collectively enhance intelligence.

We propose Scaling Out as the next step in AI evolution. Scaling Out expands AI’s reach by deploying interfaces—mechanisms that enable AI to interact with users, devices, and other systems. \emph{These \textbf{interfaces}, powered by specialized sub-models derived from foundation models, form an expandable AI ecosystem}. Unlike Scaling Up’s focus on size or Scaling Down’s focus on efficiency, Scaling Out emphasizes accessibility and adaptability. For example, in a smart city, AI interfaces for traffic, energy, and safety could collaborate to create a seamless urban experience, showcasing Scaling Out’s transformative potential.

\subsection{Scaling Out builds an AI Ecosystem}
Scaling Out transforms isolated AI models into a diverse, interconnected ecosystem by expanding foundation models like LLaMA~\cite{touvron2023llama} and Stable Diffusion~\cite{rombach2022high} into specialized variants equipped with structured interfaces. Foundation models provide generalized intelligence, while specialized models, fine-tuned for tasks like legal contract analysis or medical diagnosis, ensure domain-specific adaptability. For instance, ControlNet~\cite{zhang2023adding} enables structured image generation by conditioning outputs on additional inputs, demonstrating how foundation models can be adapted for specific use cases.

Interfaces bridge specialized models with users, applications, and other AI systems. These range from simple APIs for task-specific queries to intelligent agents capable of multi-turn reasoning and decision-making. For example, the GPT Store hosts specialized GPTs, which are sub-models derived from the GPT Foundation Model that perform tasks like coding assistance and creative writing. Similarly, Hugging Face’s ecosystem fine-tunes LLaMA variants for tasks such as sentiment analysis and summarization, showcasing how Scaling Out extends AI’s reach across domains.

\textit{By combining foundation models, specialized variants, and well-designed interfaces, Scaling Out creates a dynamic AI ecosystem.} This ecosystem fosters collaboration, enables large-scale deployment, and continuously expands AI’s capabilities, marking a shift toward open, scalable, and domain-adaptive AI infrastructure.

\subsection{Technical Foundations}
Scaling Out relies on efficiently adapting foundation models into specialized models for different tasks and domains. Traditional fine-tuning requires extensive computational resources, but \textbf{Parameter-Efficient Fine-Tuning} (PEFT) techniques allow models to be adapted efficiently while preserving the original knowledge. Methods like LoRA~\cite{hu2021lora} and Adapter Layers enable adding task-specific knowledge without modifying the entire model. Prompt Tuning and Prefix Tuning~\cite{li2021prefix} further optimize the behavior of the model by modifying inputs rather than parameters. These techniques are widely used in HuggingFace’s Transformers and applications like BloomZ, which enables multilingual fine-tuning of large models with minimal computational cost~\cite{muennighoff2022crosslingual}.

\textbf{Condition control} enables a single foundation model to dynamically adapt to multiple tasks without the need for retraining distinct models. Instead of fine-tuning a model separately for every task, condition control allows AI models to modify their behavior through additional input constraints, making them more flexible. ControlNet~\cite{zhang2023building} extends Stable Diffusion by incorporating structural guidance (\textit{e.g.}, edge maps, depth maps and pose estimation) to generate context-aware images while maintaining the efficiency of the original model. Similarly, in large language models, FLAN-T5~\cite{chung2024scaling} demonstrates how conditioning input prompts can alter model outputs for diverse tasks like summarization, translation, and reasoning without fine-tuning. In speech synthesis, VALL-E~\cite{wang2023neural} utilizes audio conditions to generate highly expressive speech from a short sample, enabling personalized voice generation without retraining on new data. 

\textbf{Federated learning} (FL) enables the collaborative training of AI models across distributed devices or systems without centralizing data. This decentralized approach ensures data privacy and security, as raw data remains on local devices while only model updates (\textit{e.g.}, gradients) are shared. FL allows specialized sub-models to be trained on diverse, domain-specific datasets, enhancing their adaptability to local conditions and tasks. For example, in healthcare, FL enables hospitals to collaboratively train diagnostic models without sharing sensitive patient data, ensuring compliance with privacy regulations~\cite{yang2019federated}. Techniques like Federated Averaging ~\cite{mcmahan2017communication} optimize communication efficiency, making FL scalable across millions of devices. Additionally, advancements such as Federated Transfer Learning~\cite{saha2021federated} and Personalized Federated Learning~\cite{smith2017federated} further enhance the adaptability of models to heterogeneous data distributions, a key requirement for Scaling Out. 

\subsection{Future Trends}

\textbf{Blockchain.}
Just as App stores in Android/iOS provide diverse applications, an AI model store will emerge, enabling users to access, customize, and deploy specialized AI models. For example, the recently launched foundation model DeepSeek-v3~\cite{liu2024deepseek3} has already surpassed 100 variations in just one month, demonstrating how foundational models can rapidly evolve into specialized versions. To ensure security, transparency, and intellectual property protection in decentralized AI marketplaces, blockchain can serve as a trust layer, recording all modifications, ownership changes, and interactions on an immutable ledger. Every fine-tuning adjustment, API call, or derivative model creation would leave a verifiable trace, ensuring credit attribution, preventing unauthorized modifications, and securing proprietary AI advancements. This decentralized framework will safeguard AI innovations and ensures a collaborative, accountable AI ecosystem, where Scaling Out thrives on trustworthy, trackable, and openly governed AI interfaces.

\textbf{Edge Computing.}
Edge computing processes data locally on devices, such as smartphones, IoT sensors, or edge servers, minimizing the need to send information to centralized data centers. Federated learning complements this by allowing distributed devices to collaboratively train machine learning models without sharing raw data, enhancing privacy and efficiency. Together, these technologies reduce latency, improve real-time decision-making, and ensure scalability by distributing computation across a network of edge nodes. For Scaling Out, this decentralized architecture allows billions of lightweight, specialized AI agents to operate independently while sharing collective insights, as seen in applications like personalized healthcare monitoring or real-time traffic management. This synergy fosters ecosystems where agents adapt locally while contributing to a globally optimized intelligence network.


