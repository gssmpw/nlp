%%%%%%%% ICML 2025 EXAMPLE LATEX SUBMISSION FILE %%%%%%%%%%%%%%%%%

\documentclass{article}

% Recommended, but optional, packages for figures and better typesetting:
\usepackage{microtype}
\usepackage{graphicx}
\usepackage{subfigure}
\usepackage{booktabs} % for professional tables

% hyperref makes hyperlinks in the resulting PDF.
% If your build breaks (sometimes temporarily if a hyperlink spans a page)
% please comment out the following usepackage line and replace
% \usepackage{icml2025} with \usepackage[nohyperref]{icml2025} above.
\usepackage{hyperref}


% Attempt to make hyperref and algorithmic work together better:
\newcommand{\theHalgorithm}{\arabic{algorithm}}

% Use the following line for the initial blind version submitted for review:
% \usepackage{icml2025}

% If accepted, instead use the following line for the camera-ready submission:
\usepackage[accepted]{icml2025}

% For theorems and such
\usepackage{amsmath}
\usepackage{amssymb}
\usepackage{mathtools}
\usepackage{amsthm}

% if you use cleveref..
\usepackage[capitalize,noabbrev]{cleveref}

%%%%%%%%%%%%%%%%%%%%%%%%%%%%%%%%
% THEOREMS
%%%%%%%%%%%%%%%%%%%%%%%%%%%%%%%%
\theoremstyle{plain}
\newtheorem{theorem}{Theorem}[section]
\newtheorem{proposition}[theorem]{Proposition}
\newtheorem{lemma}[theorem]{Lemma}
\newtheorem{corollary}[theorem]{Corollary}
\theoremstyle{definition}
\newtheorem{definition}[theorem]{Definition}
\newtheorem{assumption}[theorem]{Assumption}
\theoremstyle{remark}
\newtheorem{remark}[theorem]{Remark}

% Todonotes is useful during development; simply uncomment the next line
%    and comment out the line below the next line to turn off comments
%\usepackage[disable,textsize=tiny]{todonotes}
\usepackage[textsize=tiny]{todonotes}


% The \icmltitle you define below is probably too long as a header.
% Therefore, a short form for the running title is supplied here:
\icmltitlerunning{AI Scaling: From Up to Down and Out}

% fix title space
\usepackage[compact]{titlesec}
% Arguments: \titlespacing*{command}{left}{before-sep}{after-sep}[right-sep]
\titlespacing{\section}{0px}{1px}{0px}
\titlespacing{\subsection}{0px}{1px}{0px}
\titlespacing{\subsubsection}{0px}{0px}{0px}
\titlespacing{\paragraph}{0px}{0px}{0px}

\begin{document}

\twocolumn[
\icmltitle{AI Scaling: From Up to Down and Out}

% It is OKAY to include author information, even for blind
% submissions: the style file will automatically remove it for you
% unless you've provided the [accepted] option to the icml2025
% package.

% List of affiliations: The first argument should be a (short)
% identifier you will use later to specify author affiliations
% Academic affiliations should list Department, University, City, Region, Country
% Industry affiliations should list Company, City, Region, Country

% You can specify symbols, otherwise they are numbered in order.
% Ideally, you should not use this facility. Affiliations will be numbered
% in order of appearance and this is the preferred way.
\icmlsetsymbol{equal}{*}

\begin{icmlauthorlist}
\icmlauthor{Yunke Wang}{yyy}
\icmlauthor{Yanxi Li}{yyy}
\icmlauthor{Chang Xu}{yyy}
%\icmlauthor{}{sch}
%\icmlauthor{}{sch}
\end{icmlauthorlist}

\icmlaffiliation{yyy}{School of Computer Science, The University of Sydney, Sydney, Australia. Emails: \{yunke.wang@, yali0722@uni., c.xu@\}sydney.edu.au}

%\icmlcorrespondingauthor{}{c.xu@sydney.edu.au}

% You may provide any keywords that you
% find helpful for describing your paper; these are used to populate
% the "keywords" metadata in the PDF but will not be shown in the document
% \icmlkeywords{Machine Learning, ICML}

\vskip 0.3in
]

% this must go after the closing bracket ] following \twocolumn[ ...

% This command actually creates the footnote in the first column
% listing the affiliations and the copyright notice.
% The command takes one argument, which is text to display at the start of the footnote.
% The \icmlEqualContribution command is standard text for equal contribution.
% Remove it (just {}) if you do not need this facility.

\printAffiliationsAndNotice{}  % leave blank if no need to mention equal contribution
% \printAffiliationsAndNotice{\icmlEqualContribution} % otherwise use the standard text.

\begin{abstract}
AI Scaling has traditionally been synonymous with Scaling Up, which builds larger and more powerful models. However, the growing demand for efficiency, adaptability, and collaboration across diverse applications necessitates a broader perspective. This position paper presents a holistic framework for AI scaling, encompassing Scaling Up, Scaling Down, and Scaling Out. It argues that \textbf{while Scaling Up of models faces inherent bottlenecks, the future trajectory of AI scaling lies in Scaling Down and Scaling Out}. These paradigms address critical technical and societal challenges, such as reducing carbon footprint, ensuring equitable access, and enhancing cross-domain collaboration. We explore transformative applications in healthcare, smart manufacturing, and content creation, demonstrating how AI Scaling can enable breakthroughs in efficiency, personalization, and global connectivity. Additionally, we highlight key challenges, including balancing model complexity with interpretability, managing resource constraints, and fostering ethical development. By synthesizing these approaches, we propose a unified roadmap that redefines the future of AI research and application, paving the way for advancements toward Artificial General Intelligence (AGI).
\vspace{-1em}
\end{abstract}


\section{Introduction}
\label{sec:intro}
% Image editing methods in diffusion models depend on user-defined control directions - users can unlock their creativity using these methods by specifying the desired manipulation through prompts~\cite{gandikota2023concept}, reference images~\cite{ruiz2022dreambooth, kumari2022customdiffusion, gal2022image, chen2024trainingfreeregionalpromptingdiffusion}, or attribute vectors~\cite{parmar2023zero,hertz2022prompt}. In this work, we ask a fundamentally different question: \emph{Can we automatically discover the underlying visual structure of a concept within diffusion model's knowledge?} %Rather than requiring user-specified controls, we aim to decompose the model's internal knowledge into meaningful directions.

% This question touches on a fundamental limitation in how we interact with diffusion models. Current control methods ~\cite{zhang2023addingconditionalcontroltexttoimage, gandikota2023concept, ye2023ipadaptertextcompatibleimage,ye2023ipadaptertextcompatibleimage, hertz2024stylealignedimagegeneration, li2023photomaker, shi2024instantbooth, chen2024trainingfreeregionalpromptingdiffusion} require users to specify their desired manipulations in advance, limiting interactive creativity. This contrasts with natural human artistic workflows, where creators dynamically explore creative ideas while jointly refining them toward meaningful artistic outcomes~\cite{hoffmann2016modeling}. This synergy between specification and exploration is not new to generative models. Early GAN architectures naturally developed disentangled latent spaces that enabled continuous\cite{harkonen2020ganspace,radford2015unsupervised, wu2021stylespace, shen2020interfacegan}, compositional control over generated images. Users could explore these spaces to discover interesting variations that would be difficult to describe in words~\cite{wu2021stylespace}, then combine them to achieve their creative goals~\cite{grabe2022towards}. 


% While diffusion models have largely superseded GANs in conditional image synthesis~\cite{dhariwal2021diffusion},  their underlying structure remains less understood. Diffusion models achieve remarkable diversity through high-dimensional latents, unlike GANs' compact latent spaces.  With a single prompt, diffusion models can generate radically different variations through different random initializations of input noise. We ask - Is it possible to discover interpretable structure within this vast space of variations?

Text-to-image diffusion models are capable of generating remarkable visual variations from a single prompt through different random initializations. However, this vast creative potential remains largely opaque to users---while we can generate diverse images, we lack understanding of the underlying structure of these variations. This presents a fundamental challenge: how can we discover and expose the latent visual capabilities encoded within these models?

\let\thefootnote\relax \footnote{$^{*}$Correspondence to \texttt{gandikota.ro@northeastern.edu}}

The challenge touches on a key limitation in how we interact with diffusion models today. Current control methods require users to explicitly specify their desired edits in advance through prompts~\cite{gandikota2023concept}, reference images~\cite{zhang2023addingconditionalcontroltexttoimage, chen2024trainingfreeregionalpromptingdiffusion, ruiz2022dreambooth,kumari2022customdiffusion, Ryu_lora, hu2021lora}, or attribute vectors~\cite{ye2023ipadaptertextcompatibleimage, hertz2024stylealignedimagegeneration, li2023photomaker, shi2024instantbooth,parmar2023zero,hertz2022prompt}. That contrasts sharply with natural human creative workflows, where artists dynamically explore creative ideas and jointly refine them toward meaningful artistic outcomes~\cite{hoffmann2016modeling}. The need for pre-specified controls creates a barrier between users and the full creative potential of these models.

Interestingly, earlier generative models like GANs~\cite{gans,karras2019style,brock2018large} naturally developed more interpretable internal structures. Their compact latent spaces often exhibited emergent disentanglement~\cite{harkonen2020ganspace,radford2015unsupervised, wu2021stylespace, shen2020interfacegan}, enabling continuous and compositional control over generated images. Users could explore these spaces to discover interesting variations that would be difficult to describe in words~\cite{wu2021stylespace}, then combine them to achieve their creative goals~\cite{grabe2022towards}.

Diffusion models have largely superseded GANs in conditional image synthesis~\cite{dhariwal2021diffusion}, achieving greater diversity through much higher-dimensional latents. And yet an understanding of the underlying structure of these larger latent spaces has remained elusive. In this work, we ask a fundamental question: \emph{Can we automatically discover the visual structure within a diffusion model's knowledge of a concept?} Rather than requiring user-specified controls, we aim to decompose the model's internal representations into expressive directions that users can explore and combine.

To address these needs, we present \textbf{SliderSpace}, a framework that brings systematic explorability to diffusion models. Given just a text prompt, SliderSpace discovers a canonical set of meaningful, diverse, and controllable directions within the model's knowledge of that concept. Each direction is implemented as a low-rank adapter~\cite{hu2021lora} that can be scaled and composed with others, allowing users to explore and smoothly combine different aspects of variation, as shown in Figure~\ref{fig:intro}.

We ground SliderSpace discovery in three key requirements for meaningful decomposition of a diffusion model's visual manifold: 
\begin{enumerate}
    \item \textbf{Unsupervised Discovery:} The decomposition process should emerge from the intrinsic structure of the model's learned representation, rather than being guided by predefined attributes. This ensures we capture the true topology of the model's knowledge space rather than projecting our assumptions onto it.
    
    \item \textbf{Semantic Orthogonality:} Each discovered control must represent a distinct semantic direction. This is enforced in a semantic feature space, like CLIP, where every slider has an orthogonal effect in embeddings. This prevents discovering multiple controls that create similar semantic effects, making the system more efficient and easier.
    
    \item \textbf{Distribution Consistency:} Directions must induce consistent transformations across both random seeds and prompt variations. 
\end{enumerate}

These requirements naturally lead to our proposed framework, which we formalize in Section~\ref{sec:method}. As we show in our experiments, SliderSpace is architecture-agnostic, working with both conventional U-Net based models like Stable Diffusion~\cite{rombach2022high, rombach2022sd20, podell2023sdxl, turbo, dmd} and recent transformer-based architectures like Flux~\cite{flux}.

We demonstrate the expressiveness of SliderSpace through three applications: First, we show how SliderSpace can decompose high-level concepts into diverse and expressive components, revealing the natural axes of variation in the model's understanding. Second, we explore artistic style variation, where SliderSpace discovers directions that match or exceed the diversity of manually curated artist lists while being judged more useful by human evaluators. Finally, we show how SliderSpace can help reverse the mode collapse commonly observed in distilled diffusion models, restoring diversity while maintaining generation speed.

Beyond providing practical creative control, SliderSpace opens new avenues for understanding and utilizing the latent capabilities of diffusion models. By mapping these models' visual potential into intuitive, composable directions, we take a step toward making their creative possibilities more accessible and interpretable to users.

% Image editing methods in diffusion models unlock the creativity of users. In this work we ask an alternate question: \emph{Can we organize and expose what of the diffusion model is already capable of?}.
% Existing methods for controlling image generation typically require users to manually specify edit directions for desired changes. This process is time-consuming, requires technical expertise, and limits the spontaneity of the creative process. For instance, if a user wants to adjust the smile of a generated person, they must explicitly request this edit, often through imprecise prompt engineering or model fine-tuning. This approach of predefined controls or manual specifications restricts users from fully exploring the latent capabilities of the model. There may be interesting stylistic variations or attributes that the model can generate, but users have no easy way to discover or utilize these.

% Natural visual disentanglement was an emergent property in the latent space of Generative Adversarial Models (GANs) \cite{harkonen2020ganspace,radford2015unsupervised, wu2021stylespace, shen2020interfacegan}. In particular, it has been observed that StyleGAN~\cite{karras2019style} stylespace neurons offer detailed control over many meaningful aspects of images that would be difficult to describe in words~\cite{wu2021stylespace}. However, diffusion models do not share such a compact latent space~\cite{park2023unsupervised}; and efforts to uncover such a space in the semantic embeddings of the text conditioning have met with limited success \nik{Nick - is there a specific citation you were thinking about?}.

% In this work we introduce \textbf{SliderSpace}, which takes a step towards uncovering an analogous low dimensional representation of diffusion models' visual breadth; in essence treating the diffusion model as many generators sharing parameters, where a particular generator is defined by a specific prompt. For a given prompt we sample many random seeds (and optionally prompt expansions using an LLM), generate the corresponding images, and apply an off the shelf feature extractor (in this work CLIP, but our method can be applied to any differentiable feature extractor). We use PCA to analyze these features, and for each of the leading $k$ principal components we train a LoRA \cite{} which causes the diffusion model to produces images which increase the feature magnitude along that component when passed back through the same feature extractor. This leads to a 'Slider' for each principal component, because each LoRA can be scaled and applied to the original diffusion model, continuously varying those visual features in the generated results (as measured, in our case, by CLIP).

% There are many other works that enhance the controllability of diffusion models. One common approach is enabling users to add spatial constraints to a generation either manually, or via a reference image \cite{zhang2023addingconditionalcontroltexttoimage, chen2024trainingfreeregionalpromptingdiffusion}, a second is leveraging more abstract embeddings (e.g. identity, style) extracted from a reference image \cite{ye2023ipadaptertextcompatibleimage, hertz2024stylealignedimagegeneration, li2023photomaker, shi2024instantbooth}, a third is finetuning a foundation model to better generate a concept important to the user \cite{ruiz2022dreambooth, kumari2022customdiffusion, Ryu_lora, hu2021lora}, and a fourth (most relevant to this work) is finding low-rank adaptors of the model based on a prompt or small training set which can be scaled to provide continous control over one aspect of generated image (e.g. night vs day, basic vs luxury, etc.) \cite{gandikota2023concept}. SliderSpace is complementary to all of these methods and offers something distinct. All of the other methods we are aware require the user (and / or model designer) to know in advance what type of control they want. In contrast SliderSpace assists users in discovering and controlling hidden capabilities present in the diffusion model's distribution of possible generations.

%We propose that truly intuitive creative control in a text-to-image model should meet three key criteria: \emph{discoverability}, \emph{intuitiveness}, and \emph{specificity}. The model should reveal controllable attributes that may not be immediately obvious, offer controls that are easy to understand and manipulate, and ensure each control affects a distinct attribute of the generated image.

% We demonstrate the utility and power of SliderSpace using three applications built on top of SDXL-DMD \cite{dmd}, because its fast generation speed lends itself well to the continuous control offered by SliderSpace.

% First, we study concept decomposition (Section \ref{sec:concept_exp}), where we learn sliders for a specific concept (e.g. 'monster', 'waterfall', 'car'). Through quantitative metrics of diversity and text alignment we demonstrate that the learned sliders dramatically boost the diversity of generations when randomly applied without harming text alignment; we also ask humans to qualitatively judge these results in a user study where they find the SliderSpace results to be more 'Diverse', 'Useful', and 'Creative' than our baselines.

% Second, we attempt to compare the automatic discoveries of SliderSpace to a large scale manual study of artistic styles (Section \ref{sec:art_exp}), open-sourced by ParrotZone \cite{parrotzone}. In this study SDXL was prompted with over 4300 artist names,  and based on visual inspection the cases of successful stylistic mimicry recorded. Quantitatively SliderSpace more closely matches the distribution of artistic variation discovered by ParrotZone than other baselines, and in our user studies was judged to be significantly more 'Diverse' and 'Useful' than the baselines. To our surprise humans even judged SliderSpace results to be slightly more 'Diverse' than the results generated by the manually discovered artist names of \cite{parrotzone}.

% Third, we attempt to use SliderSpace to reverse the mode collapse commonly observed in distilled few-step diffusion models relative to the original teacher model (Section \ref{sec:diverse_exp}). We quantitatively demonstrate that applying SliderSpace to SDXL-DMD leads to more closely matching the distribution of images by the original teacher, SDXL.

%Through extensive experiments on various state-of-the-art text-to-image models, we demonstrate that SliderSpace significantly enhances user control and creative expression in AI-assisted image generation tasks. Our method enables a range of applications, including concept decomposition and control, diversity improvement in generated images, customization dissection and edits, and the exploration of artistic styles inherent in the model.

% SliderSpace goes beyond providing a practical tool for enhanced creative control. By mapping the visual potential of diffusion models it can open new avenues for generative creativity and deepens our understanding of each model's hidden potential.


\section{Scaling Up: Expanding Foundation Models}
    Scaling Up is critical for advancing AI research and applications as it pushes the boundaries of what AI systems can achieve. Larger models act as high-quality foundational models for both academia and industry, further setting benchmarks and inspiring further innovations. These models are capable of solving a wide range of tasks while they can serve as a foundation for creating specialized and diverse AI interfaces through fine-tuning as well. 
    
\subsection{Scaling in AI models}
The past experience in AI Scaling Up is mostly based on increasing data size, model size and computational resources.

\textbf{Data Size.}
Expanding dataset size is a fundamental aspect of Scaling Up AI models, as it directly impacts the quality of the system. Large and diverse datasets expose models to a wide variety of knowledge, thereby enabling them to perform effectively across multiple domains. For instance, GPT-3~\cite{brown2020language} was trained on 570GB of cleaned and curated text data drawn from sources such as Common Crawl, BooksCorpus, and Wikipedia, which enabled it to generate human-like responses across diverse contexts. 
More recently, multi-modal datasets such as LAION-5B~\cite{schuhmann2022laion} have been used to scale vision-language models like Stable Diffusion~\cite{rombach2022high}, showcasing the impact of data size on model's capabilities. 
    
\textbf{Model Size.}
Larger models have greater representational power, allowing them to capture complex relationships within data. For example, the 175B parameters of GPT-3 significantly outperform their predecessors in tasks that require learning of few shots or zero shots~\cite{brown2020language}. Similarly, GLaM~\cite{du2022glam} scaled to 1.2 trillion parameters using a mixture of experts, activating only a subset of parameters per task, which reduced computational costs while maintaining high performance. The scaling laws proposed by~\cite{kaplan2020scaling} highlight that model performance improves predictably with increased size. This insight has guided the development of increasingly large models, unlocking capabilities like in-context learning and cross-modal understanding.


\textbf{Computational Resources.}
The process of scaling computational resources has evolved dramatically alongside advancements in AI, particularly in computer vision and NLP. Early in the development of these fields, training models required only a few plain GPUs. For instance, AlexNet~\cite{krizhevsky2012imagenet}, which revolutionized computer vision in 2012, was trained using just two GTX 580 GPUs. Similarly, early NLP models such as Word2Vec~\cite{church2017word2vec} were trained on modest computational setups. However, the era of LLMs has accelerated in unprecedented demands for computational resources. For example, OpenAI’s GPT-3~\cite{brown2020language} has 175B parameters and requires 10,000 NVIDIA V100 GPUs for training, consuming an estimated 1,287 MWh of electricity. Meanwhile, Meta’s LLaMA 3~\cite{touvron2023llama} scaled training to utilize thousands of NVIDIA A100 GPUs, representing the latest generation of high-performance accelerators optimized for AI workloads. This progression highlights the critical role of computational scaling in AI model's performance.


\subsection{Bottleneck}
From the \textit{data} perspective, as pointed out by many researchers, large-scale pretraining has already utilized most of the high-quality publicly available data on the web. The remaining data is either low-quality or consists of AI-generated content, which risks model degradation due to data contamination and reinforcement of biases~\cite{shumailov2024ai}. Simply increasing the dataset size will no longer yield the same level of improvement as before. From the \textit{model} perspective, while increasing parameters has led to substantial performance gains in recent years, the returns on scaling have shown diminishing improvements, and larger models suffer from inefficiencies such as redundancy in representation, overfitting to training distributions, and difficulties in interpretability and controllability. Additionally, the training and inference of massive models introduce challenges in optimization stability and robustness~\cite{dai2024enhancing}. From the \textit{computational resource} aspect, the exponential growth in required hardware, energy consumption, and costs is reaching unsustainable levels. The marginal benefit of adding more compute is decreasing while the environmental impact is rising~\cite{wu2024beyond}. The availability of high-performance GPUs poses financial constraints that limit the feasibility of further scaling. 
Together, these bottlenecks indicate that the traditional approach of scaling up is approaching its practical limits.

\subsection{Future Trends}
Despite bottlenecks in AI scaling, Scaling Up remains essential for pushing AI model's performance boundary. The future of Scaling Up should lie in balancing efficiency, adaptability and sustainability to meet the demands of larger models. Innovations in dataset optimization, efficient training, and test-time scaling will redefine AI Scaling Up.

\textbf{Dataset Optimization.}
As AI continues to scale, data optimization will become a cornerstone for advancing model efficiency and robustness. Future trends will focus on data-efficient training using smaller, high-quality datasets for faster learning. Curriculum learning~\cite{bengio2009curriculum} and active learning~\cite{settles2009active} will help models acquire knowledge incrementally and prioritize impactful samples. Techniques for handling noisy data, such as noise-robust loss functions and data augmentation, will enhance model resilience. Additionally, leveraging proprietary, domain-specific datasets will drive breakthroughs by providing richer insights beyond public data.
 
\textbf{Efficient Training.}
Another trend is developing efficient training methods to address the growing computational and environmental costs of training large models. Progressive training, where models gradually scale from smaller sub-models to full-capacity systems, will become a standard approach to reduce resource demands in the initial stages. Distributed optimization techniques, such as asynchronous training paradigms, will improve scalability across large computational infrastructures. Advances in mixed-precision training, sparse updates, and activation checkpointing will further minimize memory and compute overhead, making AI development more sustainable and scalable. 
% These trends are set to redefine how foundational models are trained, enabling powerful capabilities without imposing unsustainable hardware and energy requirements.
    
\textbf{Test-time Scaling.}
Recent research has highlighted the potential of scaling up test-time computing to enhance the performance of large language models (LLMs), providing an alternative to solely scaling up model parameters. For example,  
\citet{snell2024scaling} explore two strategies: adaptive output distribution and verifier-based search mechanisms, both improving model performance dynamically. Unlike previous inference-time optimization attempts, this approach tailors compute allocation to problem complexity, enabling smaller models to outperform larger ones on certain prompts. Adaptive test-time scaling presents a promising direction for optimizing efficiency without excessive pretraining.


\section{Scaling Down: Refining Core Functions} \label{sec:scaling_down}
As models become increasingly large and complex through \textit{Scaling Up}, their training, deployment, and maintenance demand significant computational, memory, and energy resources. These challenges limit accessibility and scalability.
A critical question emerges: \textit{how can we maintain or improve model effectiveness while reducing size and computational requirements?}
Drawing inspiration from the human brain, where specialized small units handle essential functions while auxiliary components support adaptability and memory, the \textbf{Scaling Down} concept offers a novel approach. By identifying and extracting the essential functional modules of large models, Scaling Down makes it possible to reduce the model size and computation costs significantly while retaining or even enhancing key capabilities.
Scaling Down can be approached in \textit{two distinct ways}. The first involves directly reducing the model size by decreasing the number or precision of parameters (Section \ref{sec:scaling_down:small}). 
Alternatively, minimizing redundant or unnecessary computations can enhance computational efficiency without altering the number or precision of parameters (Section \ref{sec:scaling_down:computing}).

\subsection{Reducing the Size of Large Models} \label{sec:scaling_down:small}

    The most straightforward approach to reducing model size involves reducing the number of parameters within a model. \textbf{Pruning} achieves this by simplifying neural networks through the removal of less significant components \cite{lecun1989optimal, han2015deep, molchanov2016pruning}.
    LLM-Pruner \cite{ma2023llm} proposes a task-agnostic approach to structural pruning by selectively removing non-critical structures using gradient information.
    Wanda \cite{sun2023simple} emphasizes simplicity and efficiency by pruning weights based on the product of weight magnitudes and corresponding input activations without the need for retraining or weight updates.
        
    An alternative to directly removing parameters is the use of \textbf{low-rank approximations}, which employ smaller matrices to approximate larger ones \citet{sainath2013low}. 
    Low-Rank Adaptation (LoRA) \cite{hu2021lora} tackles the inefficiency of fine-tuning all model parameters by introducing trainable low-rank decomposition matrices into Transformer layers while keeping the pre-trained model weights frozen.
    Linformer \cite{wang2020linformer} leverages the observation that self-attention mechanisms in Transformers exhibit low-rank structures. By approximating the self-attention matrix with a low-rank factorization, Linformer reduces the time and space complexity of self-attention to a linear scale.
    
    Another effective strategy focuses on reducing parameter precision rather than quantity. \textbf{Quantization} reduces the bit-width of weights and activations by substituting floating-point parameters with integers \cite{gupta2015deep, nagel2020up}.
    GPTQ \cite{frantar2022gptq} introduces an efficient one-shot weight quantization method based on approximate second-order information. 
    AWQ \cite{lin2024awq} focuses on activation-aware weight quantization, leveraging the unequal importance of weights and optimal per-channel scaling to protect salient weights.
    QLoRA \cite{dettmers2024qlora} introduces a memory-efficient fine-tuning approach by combining 4-bit quantization with LoRA. 

    Rather than modifying existing models, \textbf{knowledge distillation} (KD) facilitates the transfer of knowledge from large and complex teachers to small and efficient students \cite{hinton2015distilling}. The students are trained to replicate the behavior of the teachers.
    \citet{yu2024distilling} propose to distill System 2 reasoning processes—such as Chain-of-Thought and System 2 Attention—into a single-step System 1 model, eliminating intermediate reasoning while retaining or improving task performance.
    Program-aided Distillation (PaD) \cite{zhu2024pad} introduces a new KD paradigm that uses reasoning programs to verify and refine synthetic CoT data, enhancing distilled reasoning quality. PaD automates error-checking, incorporates iterative self-refinement to address faulty reasoning chains, and employs step-wise beam search to validate reasoning steps progressively.

\subsection{Optimizing Computational Efficiency} \label{sec:scaling_down:computing}

    \textbf{Speculative decoding} optimizes the inference process by dynamically adapting decoding strategies.
    \citet{leviathan2023fast} introduce speculative decoding as a method that leverages more efficient approximation models to propose candidate tokens, which are then verified by the target model in parallel. 
    Similarly, \citet{chen2023accelerating} propose speculative sampling, employing a draft model to generate multiple token candidates, which are then validated using a modified rejection sampling scheme. 
    These methods underscore the potential of speculative execution to mitigate the inherent inefficiencies of autoregressive decoding, enabling faster inference without retraining or compromising output quality.
    
    \textbf{Key-value cache} is a pivotal strategy in autoregressive decoding, where intermediate states of attention mechanisms are stored to avoid recomputation in subsequent inference steps. This technique significantly accelerates the generation of long sequences by leveraging stored key-value pairs from previous layers. However, it introduces additional memory overhead, which must be carefully managed. 
    Sparse attention mechanisms \cite{zhang2023h2o, anagnostidis2024dynamic, liu2024scissorhands} use specialized sparsity patterns that prevent unnecessary token access. They use KV cache eviction and compression strategies to achieve significant improvements in latency, throughput, and memory savings.
    Block-wise KV cache management \cite{kwon2023efficient, prabhu2024vattention} adopts memory fragmentation techniques inspired by paged memory systems, offering efficient runtime memory allocation and reallocation.


    \textbf{Mixture of Experts} (MoE) introduced distributed specialization, enabling efficient scaling through task-specific sub-models controlled by a gating mechanism \citep{jacobs1991adaptive}.
    Early dense MoE models suffered computational inefficiencies \citep{jordan1994hierarchical}.
    Sparse architectures \citep{shazeer2017outrageously} improved efficiency by selectively activating relevant experts. Models like GShard \citep{lepikhin2020gshard}, Switch Transformer \citep{fedus2022review}, and GLaM \citep{du2022glam} leveraged MoE for state-of-the-art performance with reduced computation. Recent advances, including Mixtral \citep{jiang2024mixtral} and DeepSeekMoE \citep{dai2024deepseekmoe}, further optimized efficiency.

\subsection{Small Models for Large Impacts}

    Designing high-efficiency architectures is fundamental to developing small-scale models. 
    The most computationally intensive and memory-intensive component of Transformer-based models is the Attention mechanism. 
    Extensive research efforts have been devoted to enhancing the efficiency of Attention mechanisms. Notable advancements include \textit{Flash Attention} \cite{dao2022flashattention}, which is utilized by models such as Phi-1.5 \cite{li2023textbooks} and DeepSeek-LLM \cite{bi2024deepseek}, \textit{Grouped Query Attention} \cite{ainslie2023gqa}, which is utilized by MiniCPM \cite{hu2024minicpm}, Mistral \cite{jiang2023mistral}, Phi-3 \cite{abdin2024phi}, DeepSeek-LLM \cite{bi2024deepseek}, and DeepSeek-V2 \cite{liu2024deepseek}, and \textit{Multi-Head Latent Attention}, which was first introduced by DeepSeek-V2 \cite{liu2024deepseek} and has been adopted in its successor, DeepSeek-V3 \cite{liu2024deepseek3}.

    While such innovations enable the development of highly efficient small models, further improvements are necessary to bridge the performance gap between small and large-scale models. Key directions for achieving this include curating high-quality training data, designing scalable training strategies, and leveraging techniques such as mixture-of-experts (MoE), which allow for the selective activation of model components to optimize efficiency and performance.

    \textbf{High-Quality Training Data.}
    The Phi family of models \cite{gunasekar2023textbooks, li2023textbooks, javaheripi2023phi, abdin2024phi} highlights the importance of high-quality training data. Rather than relying on vast amounts of noisy web-scraped text, these models are trained on curated, synthetically generated textbook-style data, including structured exercises and carefully filtered educational content. This approach enhances efficiency and mitigates common issues such as hallucination and bias.

    \textbf{Scalable Training Strategies.}
    Training efficiency is another critical factor in developing compact yet powerful models. Mini-CPM \cite{hu2024minicpm} introduces Model Wind Tunnel Experiments (MWTE) to optimize hyperparameter selection, ensuring that smaller models are trained in a computationally efficient manner. Additionally, it employs the Warmup-Stable-Decay (WSD) learning rate scheduler, which segments training into distinct phases to maximize hardware utilization and improve convergence.
    
    \textbf{More Parameters but Less Activation.}
    A crucial trend in optimizing smaller models for efficiency is the adoption of Mixture-of-Experts (MoE) architectures, where a subset of model parameters is activated per token, reducing computation while maintaining a large overall parameter pool. Several recent models exemplify this technique:
    Mixtral \cite{jiang2024mixtral} consists of 8 expert models with a total of 56B parameters, while each token is processed by only 2 experts.
    Phi-3.5-MoE \cite{abdin2024phi} comprises 16 experts totaling 60.8B parameters but activates only 6.6B (10.9\%).
    DeepSeek-V2 \cite{liu2024deepseek} is a 236B-parameter model with 21B parameters activated per token (8.9\%).
    DeepSeek-V3 \cite{liu2024deepseek3} scales further to 671B total parameters while activating only 37B per token (5.5\%).
    These MoE-based models show that large parameter spaces can be efficiently leveraged without incurring excessive computational costs, allowing smaller models to achieve performance comparable to their larger counterparts with only a fraction of the active compute requirements.
    
\subsection{Future Trends}

    \textbf{Core Functional Module Refinement.}
    A promising direction for future research in Scaling Down models lies in refining core functional modules. While existing methods predominantly emphasize the balance between efficiency and effectiveness, a critical gap remains in identifying the minimal functional module within large models. This minimal module would represent the smallest possible unit that retains all essential functionalities without compromising performance. Future investigations may focus on developing systematic approaches to detect and characterize such modules, potentially leveraging advancements in model pruning and knowledge distillation. Establishing rigorous criteria for defining and verifying minimal functional modules could significantly contribute to optimizing model architectures while maintaining their operational integrity.
    
    \textbf{External Assistance.}
    Leveraging external assistance enables small-scale core models to dynamically extend their capacity to handle complex tasks. 
    Retrieval-Augmented Generation (RAG) \cite{lewis2020retrieval} is a method for \textit{external knowledge} augmentation. RAG combines pre-trained parametric memory with non-parametric memory,    which enables models to fetch contextually relevant information dynamically.
    Integrating \textit{external tools} allows models to assign specialized operations to certain systems.
    Toolformer \cite{schick2023toolformer} can autonomously learn to invoke external APIs, such as calculators, search engines, and translation systems.
    Beyond merely utilizing external tools, recent advancements suggest that models can also generate tools to extend their own capabilities. VISPROG \cite{gupta2023visual} can leverage in-context learning to produce modular, Python-like programs and execute them for complex visual reasoning tasks. 

\section{Scaling Out: Advancing  AI Ecosystems} \label{sec:scaling_out}
Scaling Up and Scaling Down represent two complementary approaches to AI scaling, yet neither fully realizes AI’s potential in real-world applications. Scaling Up builds larger, generalized models like GPT and BERT, but their resource demands limit accessibility and task-specific adaptability. Scaling Down optimizes models for efficiency, enabling deployment in resource-constrained environments, but struggles with adaptability, collaboration, and decentralized intelligence. To address these gaps, AI must evolve into a distributed ecosystem where multiple AI entities interact, specialize, and collectively enhance intelligence.

We propose Scaling Out as the next step in AI evolution. Scaling Out expands AI’s reach by deploying interfaces—mechanisms that enable AI to interact with users, devices, and other systems. \emph{These \textbf{interfaces}, powered by specialized sub-models derived from foundation models, form an expandable AI ecosystem}. Unlike Scaling Up’s focus on size or Scaling Down’s focus on efficiency, Scaling Out emphasizes accessibility and adaptability. For example, in a smart city, AI interfaces for traffic, energy, and safety could collaborate to create a seamless urban experience, showcasing Scaling Out’s transformative potential.

\subsection{Scaling Out builds an AI Ecosystem}
Scaling Out transforms isolated AI models into a diverse, interconnected ecosystem by expanding foundation models like LLaMA~\cite{touvron2023llama} and Stable Diffusion~\cite{rombach2022high} into specialized variants equipped with structured interfaces. Foundation models provide generalized intelligence, while specialized models, fine-tuned for tasks like legal contract analysis or medical diagnosis, ensure domain-specific adaptability. For instance, ControlNet~\cite{zhang2023adding} enables structured image generation by conditioning outputs on additional inputs, demonstrating how foundation models can be adapted for specific use cases.

Interfaces bridge specialized models with users, applications, and other AI systems. These range from simple APIs for task-specific queries to intelligent agents capable of multi-turn reasoning and decision-making. For example, the GPT Store hosts specialized GPTs, which are sub-models derived from the GPT Foundation Model that perform tasks like coding assistance and creative writing. Similarly, Hugging Face’s ecosystem fine-tunes LLaMA variants for tasks such as sentiment analysis and summarization, showcasing how Scaling Out extends AI’s reach across domains.

\textit{By combining foundation models, specialized variants, and well-designed interfaces, Scaling Out creates a dynamic AI ecosystem.} This ecosystem fosters collaboration, enables large-scale deployment, and continuously expands AI’s capabilities, marking a shift toward open, scalable, and domain-adaptive AI infrastructure.

\subsection{Technical Foundations}
Scaling Out relies on efficiently adapting foundation models into specialized models for different tasks and domains. Traditional fine-tuning requires extensive computational resources, but \textbf{Parameter-Efficient Fine-Tuning} (PEFT) techniques allow models to be adapted efficiently while preserving the original knowledge. Methods like LoRA~\cite{hu2021lora} and Adapter Layers enable adding task-specific knowledge without modifying the entire model. Prompt Tuning and Prefix Tuning~\cite{li2021prefix} further optimize the behavior of the model by modifying inputs rather than parameters. These techniques are widely used in HuggingFace’s Transformers and applications like BloomZ, which enables multilingual fine-tuning of large models with minimal computational cost~\cite{muennighoff2022crosslingual}.

\textbf{Condition control} enables a single foundation model to dynamically adapt to multiple tasks without the need for retraining distinct models. Instead of fine-tuning a model separately for every task, condition control allows AI models to modify their behavior through additional input constraints, making them more flexible. ControlNet~\cite{zhang2023building} extends Stable Diffusion by incorporating structural guidance (\textit{e.g.}, edge maps, depth maps and pose estimation) to generate context-aware images while maintaining the efficiency of the original model. Similarly, in large language models, FLAN-T5~\cite{chung2024scaling} demonstrates how conditioning input prompts can alter model outputs for diverse tasks like summarization, translation, and reasoning without fine-tuning. In speech synthesis, VALL-E~\cite{wang2023neural} utilizes audio conditions to generate highly expressive speech from a short sample, enabling personalized voice generation without retraining on new data. 

\textbf{Federated learning} (FL) enables the collaborative training of AI models across distributed devices or systems without centralizing data. This decentralized approach ensures data privacy and security, as raw data remains on local devices while only model updates (\textit{e.g.}, gradients) are shared. FL allows specialized sub-models to be trained on diverse, domain-specific datasets, enhancing their adaptability to local conditions and tasks. For example, in healthcare, FL enables hospitals to collaboratively train diagnostic models without sharing sensitive patient data, ensuring compliance with privacy regulations~\cite{yang2019federated}. Techniques like Federated Averaging ~\cite{mcmahan2017communication} optimize communication efficiency, making FL scalable across millions of devices. Additionally, advancements such as Federated Transfer Learning~\cite{saha2021federated} and Personalized Federated Learning~\cite{smith2017federated} further enhance the adaptability of models to heterogeneous data distributions, a key requirement for Scaling Out. 

\subsection{Future Trends}

\textbf{Blockchain.}
Just as App stores in Android/iOS provide diverse applications, an AI model store will emerge, enabling users to access, customize, and deploy specialized AI models. For example, the recently launched foundation model DeepSeek-v3~\cite{liu2024deepseek3} has already surpassed 100 variations in just one month, demonstrating how foundational models can rapidly evolve into specialized versions. To ensure security, transparency, and intellectual property protection in decentralized AI marketplaces, blockchain can serve as a trust layer, recording all modifications, ownership changes, and interactions on an immutable ledger. Every fine-tuning adjustment, API call, or derivative model creation would leave a verifiable trace, ensuring credit attribution, preventing unauthorized modifications, and securing proprietary AI advancements. This decentralized framework will safeguard AI innovations and ensures a collaborative, accountable AI ecosystem, where Scaling Out thrives on trustworthy, trackable, and openly governed AI interfaces.

\textbf{Edge Computing.}
Edge computing processes data locally on devices, such as smartphones, IoT sensors, or edge servers, minimizing the need to send information to centralized data centers. Federated learning complements this by allowing distributed devices to collaboratively train machine learning models without sharing raw data, enhancing privacy and efficiency. Together, these technologies reduce latency, improve real-time decision-making, and ensure scalability by distributing computation across a network of edge nodes. For Scaling Out, this decentralized architecture allows billions of lightweight, specialized AI agents to operate independently while sharing collective insights, as seen in applications like personalized healthcare monitoring or real-time traffic management. This synergy fosters ecosystems where agents adapt locally while contributing to a globally optimized intelligence network.




%%%%%%%%%%%%%%%%%%%%%%%%%%%%%%%%%%%%%%%%%%%%%%%%%%%%
% Section: Future Scenarios and Application Prospects
%%%%%%%%%%%%%%%%%%%%%%%%%%%%%%%%%%%%%%%%%%%%%%%%%%%%
\section{Future Application Prospects}
The true potential of AI Scaling lies in the future scenarios it can enable. This section explores two use cases that illustrate the transformative capabilities of AI scaling: human-AI creative communities and smart manufacturing ecosystems. 
% These examples not only highlight the practical significance of scaling but also point toward an inspiring vision of the AGI era.

\subsection{Human-AI Creative Communities}
Content creation platforms like TikTok, YouTube, and Instagram showcase how AI scaling transforms creativity and engagement. Scaling Up integrates vast multimodal datasets, enabling foundation models to analyze trends, predict preferences, and optimize recommendations on a global scale. These models, trained on billions of interactions, continuously evolve to match audience demands. Scaling Down brings AI closer to users, with lightweight models enabling real-time video, music, and AR generation on personal devices. On-device AI also enhances content moderation, ensuring platform safety without heavy computational costs. Scaling Out redefines these platforms as AI-driven ecosystems where specialized AI agents actively participate alongside human users. These AI contributors focus on education, sports, music, and niche domains, generating and engaging with content just as human creators do. For example, an education AI produces real-time tutorials, while a sports AI provides live commentary. AI bots collaborate, such as a music AI partnering with a graphic-design AI to create immersive audiovisual content.

At scale, these platforms evolve into hybrid ecosystems where human and AI creators collaborate seamlessly. The interaction between human and AI creators fosters a dynamic, participatory environment where creativity flourishes without boundaries. As AI bots continuously adapt to cultural shifts and audience feedback, they contribute to a globally inclusive and interactive digital space. Such platforms no longer merely host content but become thriving communities of hybrid human-AI interaction, where collaboration and innovation redefine the boundaries of creativity.


\subsection{Smart Manufacturing Ecosystems}

Manufacturing ecosystems differ from traditional multi-agent systems due to their open, dynamic nature and massive scale, involving suppliers, manufacturers, and distributors as autonomous AI interfaces adapting to constant change. Scaling Up builds foundational models that integrate vast, heterogeneous datasets across sourcing, logistics, production, and consumer behavior, equipping agents with advanced predictive capabilities. Scaling Down tailors these global models into lightweight, task-specific AI, optimizing factory operations, equipment monitoring, and localized supply chain decisions. Scaling Out expands the ecosystem’s reach, enabling thousands of AI interfaces to collaborate and compete, such as supplier interfaces negotiating contracts or distributor interfaces optimizing delivery schedules. The synergy between these scaling paradigms creates a self-optimizing, adaptive network, where AI continuously integrates new entrants, eliminates inefficiencies, and responds dynamically to global challenges. It transforms manufacturing into an intelligent, resilient ecosystem.

\section{Challenges and Opportunities}

Scaling Up, Down, and Out collectively offers both significant opportunities and notable challenges on the path toward AGI. This section explores these dual aspects, outlining key areas where transformative advancements can occur while addressing critical hurdles that must be overcome.

% \subsection{Cross-disciplinary research and collaboration}
\textbf{Cross-disciplinary research and collaboration.}
AI scaling demands cross-disciplinary collaboration. Cognitive science can inspire efficient model architectures, such as modular designs that selectively activate components based on input complexity \cite{laird2017standard}. Integrating neuroscience, hardware engineering, and data science is key to achieving adaptive computation at scale.
Advancements in hardware efficiency must align with AI scaling. Energy-efficient processors tailored for AI can reduce carbon footprints, while co-developing sparse computation chips enhances Scaling Down, enabling AI in resource-limited settings \cite{james2022agi_chip}.
Data science defines metrics for AI scaling, establishing benchmarks that balance model size, computational cost, and real-world performance \cite{kaplan2020scaling}. Standardizing these trade-offs provides a shared framework for innovation, guiding future research and deployment.

% \subsection{Quantitative metrics and standards for scaling}
\textbf{Quantitative metrics and standards for scaling.}
Effectively scaling AI requires quantitative models to predict performance and resource trade-offs. Developing scaling metrics for Scaling Down and Scaling Out can help assess efficiency, such as measuring performance gains relative to changes in model size, data, or compute \cite{kaplan2020scaling}. Formalized metrics also address industry concerns by providing predictable cost-benefit analyses. Scaling laws can estimate energy savings from replacing large models with smaller, task-specific AI, encouraging broader adoption of Scaling Down. Additionally, open benchmarks for Scaling Out should evaluate how distributed models communicate, adapt, and collaborate in real-world tasks, ensuring AI ecosystems remain robust and efficient \cite{dou2023agi_iot}.

% \subsection{Building open ecosystems for lightweight AI}
\textbf{Building open ecosystems for lightweight AI.}
Scaling Down fosters open and accessible AI ecosystems by enabling lightweight core models as flexible building blocks for diverse applications. Open-source initiatives supported by research and industry can accelerate innovation in this space \cite{wang2021revise}.
Releasing modular AI components with flexible APIs allows developers to adapt models for specific needs, such as edge AI in healthcare or resource-efficient industrial applications. These ecosystems also encourage hybrid scaling strategies, combining pre-trained models with task-specific fine-tuning.
Industry partnerships are essential for real-world impact. Sectors like agriculture and logistics can benefit from domain-specific AI, and fostering cross-industry collaboration will drive adoption and scalable innovation \cite{schmidt2014how_google_works}.


% \subsection{Scaling for sustainability and global equity}
\textbf{Scaling for sustainability and global equity.}
As AI systems expand, their environmental impact grows, making Scaling Down crucial for sustainability. Smaller models can match larger models' performance while consuming less energy \cite{schwartz2020green_ai}. Deploying lightweight AI on solar-powered edge devices reduces reliance on energy-intensive data centers, especially in infrastructure-limited regions.
Beyond sustainability, Scaling Out improves AI accessibility, enabling distributed intelligence to serve education, healthcare, and agriculture in underserved areas. For example, offline AI models can assist smallholder farmers with crop management or provide diagnostic tools in rural clinics \cite{pal2021ai_social_good}.
Achieving this vision requires aligning AI scaling with societal goals. Governments and organizations should fund scalable AI research that prioritizes sustainability and equity, ensuring AI benefits are broadly and fairly distributed.


% \subsection{A unified vision toward AGI}
\textbf{A unified vision toward AGI.}
The convergence of Scaling Up, Scaling Down, and Scaling Out forms a cohesive path toward AGI, balancing generalization, efficiency, and adaptability \cite{bostrom2014superintelligence}. Scaling Up builds foundational knowledge, Scaling Down optimizes AGI for diverse environments, and Scaling Out fosters collaboration among specialized intelligence to tackle complex, multidisciplinary challenges \cite{goertzel2006agi}.
Achieving this vision requires addressing technical, ethical, and societal challenges. Scaling Up must ensure interpretability and robustness, Scaling Down must prioritize privacy and security, and Scaling Out must foster fairness and accountability in collaborative AI. Cross-disciplinary efforts drawn from cognitive science, hardware engineering, and policy frameworks are essential for sustainable and ethical AGI.

\section{Alternative Views}
\label{sec:alt-views}

While our position highlights the distinct risks posed by LLM agents in finance and the need for risk-aware evaluations, we acknowledge two prevalent perspectives that question this stance. 

\subsection{View 1: Standard Benchmarks Already Capture LLM Agent Reliability}

One perspective states that traditional NLP evaluation metrics, such as accuracy, F1 scores, and even domain-specific financial indicators (e.g., sentiment classification accuracy in earnings call analyses), are sufficient for assessing the overall reliability of LLMs in real-world financial applications.
Advocates of this perspective argue:
\begin{itemize}
    \item \emph{Correlation with Agent Competence:} 
    Strong performance on established benchmarks is interpreted as a proxy for an LLM agent's overall capability, implying that models excelling on standard datasets exhibit fewer intrinsic failure modes (e.g., hallucination, flawed reasoning) in production~\citep{xiao2025tradingagentsmultiagentsllmfinancial,wu2023bloomberggpt}.
    
    \item \emph{Incremental Benchmark Evolution:} Standardized benchmarks can be continuously refined by incorporating new challenge sets (e.g., adversarial question-answering tasks)~\citep{xie2024finben,islam2023financebench}, progressively mitigating known weaknesses in LLMs.

    \item \emph{Ease of Comparison:} 
    The well-established metrics enable efficient performance comparisons across multiple LLM agents, which simplifies evaluation and adoption for financial stakeholders seeking to identify the most suitable model without the complexity of domain-specific assessment methodologies~\citep{xie2024pixiu,li2024investorbenchbenchmarkfinancialdecisionmaking,xie2024finben}.
    \end{itemize}

\paragraph{Counterarguments.}
We do not deny the usefulness of these benchmarks for basic proficiency. However, purely accuracy-driven evaluations overlook LLM-specific vulnerabilities, such as:
\vspace{-5pt}
\begin{itemize}
    \item \emph{Language-Induced Attack Vectors:} Subtle rhetorical or ``jailbreak'' prompts can bypass standard filters, leading to unsafe outputs even if the model scores highly on standardized datasets~\citep{shen2024anything}.
    \item \emph{Systematic Hallucination Patterns:} Traditional metrics rarely detect when an LLM concocts facts under complex or misleading instructions~\citep{kang2023deficiency}. A flaw that can lead to severe decisions if the agent is tasked with autonomously analyzing financial data~\citep{roychowdhury2023hallucination}.
    \item \emph{Unpredictable Decision Pathways:} CoT or ReAct reasoning remains opaque in standard evaluations~\citep{Wei2022ChainOT,yao2022react}. Agents can arrive at correct answers for the ``wrong reasons,'' which masks potential misalignment~\citep{turpin2024language}.
\end{itemize}
Therefore, while standard benchmarks remain a valuable baseline, they do not fully expose the agent-level risks that arise from an LLM's generative and self-reasoning capabilities.

\subsection{View 2: Engineering Solutions Can Mitigate LLM Agent Risks}

Another perspective argues that direct engineering interventions can mitigate the inherent risks associated with LLM agents, such as guardrail modules~\citep{chu2024causal}, advanced prompt filtering~\citep{gao2024brief}, and extended fine-tuning~\citep{qi2024finetuning}. Thus, it is sufficient to rely on these solutions to ensure the safety of LLMs in financial applications. Advocates reason that:
\vspace{-5pt}
\begin{itemize}
    \item \emph{Agent-Level Guardrails:} 
    Mechanisms such as content filtering systems and CoT verification modules can intercept and correct potentially unsafe model outputs~\citep{ling2023deductive}, thereby reducing the likelihood of misinformation or harmful decision-making.
    \item \emph{Alignment with Minimal Overhead:} 
    Approaches such as reinforcement learning from human feedback (RLHF)~\citep{ouyang2022training} or fine-tuning with domain-specific datasets~\citep{li2023large} can enhance factual reliability, without requiring stress testing of every decision-making pathway within the model.
    \item \emph{Rapid Iteration:} Continuous updates to the LLM's weights, and carefully engineered prompts may adapt the agent's behaviors faster than risk metrics~\citep{chen2023unleashing}. 
    
\end{itemize}

\paragraph{Counterarguments.}
We support robust engineering interventions. However, even comprehensive alignment strategies may not capture all vulnerabilities inherent to LLMs, unless they are repeatedly tested under realistic, agent-centric scenarios:
\vspace{-5pt}
\begin{itemize}
    \item \emph{Emerging Exploitation Strategies:} 
    Attackers continuously develop new techniques to manipulate an LLM's conversational logic, such as chaining multi-step misleading queries~\citep{xhonneux2024efficient}. Without specialized evaluations designed to simulate these adversarial interactions, critical security loopholes may remain undetected.
    
    \item \emph{Hallucination in Edge Cases:} 
    When exposed to incomplete data, LLMs may exhibit misplaced confidence in their reasoning, leading to erroneous conclusions~\citep{Chen2025EngagingWA}. Engineering solutions typically address known failure modes but may overlook emergent anomalies that arise when the model is required to make complex, interdependent decisions.
    \item \emph{The Challenge of Continuous Adaptation:} Real-world usage of LLM agents evolves rapidly; the system may be extended for new tasks or connected to external APIs. Without an ongoing risk-centric evaluation pipeline, engineering fixes become outdated as the agent's operational scope expands.
\end{itemize}
Therefore, while engineering solutions are essential, they are not sufficient to guarantee the safety. Agent-level vulnerabilities require domain-sensitive validation to ensure that new functionalities do not introduce dangerous behaviors.



\section*{Conclusion}
This paper aims to enhance our understanding of the computational complexity of computing various Shapley value variants. We found that for various ML models --- including decision trees, regression tree ensembles, weighted automata, and linear regression --- both local and global interventional and baseline SHAP can be computed in polynomial time under HMM modeled distributions. This extends popular algorithms, such as TreeSHAP, beyond their empirical distributional scope. We also establish strict complexity gaps between the various SHAP variants (baseline, interventional, and conditional) and prove the intractability of computing SHAP for tree ensembles and neural networks in simplified scenarios. Overall, we present SHAP as a versatile framework whose complexity depends on four key factors: \begin{inparaenum}[(i)] \item model type, \item SHAP variant, \item distribution modeling approach, \item and local vs. global explanations\end{inparaenum}. We believe this perspective provides deeper insight into the computational complexity of SHAP, paving the way for future work.




%We believe that our framework provides a more intricate understanding of SHAP computation complexity across different models, distributions, and variants, paving the way for further research.

Our work opens promising directions for future research. First, expanding our computational analysis to other SHAP-related metrics, such as asymmetric SHAP~\citep{frye20} and SAGE~\citep{covert2020understanding}, would be valuable. Additionally, we aim to explore more expressive distribution classes and relaxed assumptions beyond those in Section \ref{sec:tractable} while maintaining tractable SHAP computation. Finally, when exact computation is intractable (Section \ref{sec:intractable}), investigating the approximability of SHAP metrics through approximation and parameterized complexity theory~\citep{downey2012parameterized} is an important direction.

%Our work opens several promising avenues for future research on the computational properties of explainable AI methods, with a particular focus on SHAP. First, it would be interesting to broaden the computational analysis conducted in this work to include other popular SHAP-related metrics in the literature, such as asymmetric SHAP \cite{frye20} and SAGE \cite{covert2020understanding}. Also, in the future, we aim to explore more expressive distribution classes and relaxed distributional assumptions—extending beyond those examined in Section \ref{sec:tractable} —that still yield tractable SHAP computation. Finally, when exact computation proves intractable (Section \ref{sec:intractable}), it is worthwhile to theoretically investigate the question of the approximability of computing the SHAP metrics across various configurations, through the lens of approximation and parametrized complexity theory \cite{arora2009computational}.

%This paper aims to deepen our understanding of the computational complexity involved in obtaining different Shapley value variants. We found that for a variety of ML models, including decision trees, tree ensembles for regression, weighted automata, and linear regression models — computing both local and global interventional and baseline SHAP can be done in polynomial time when distributions are modeled by HMMs. This extends the distributional scope of popular algorithms like TreeSHAP, which is limited to empirical distributions. Additionally, we demonstrate a strict complexity gap between SHAP variants, showing that interventional and baseline SHAP can be strictly easier to compute than conditional SHAP. Despite these positive results, we uncovered intractability for various SHAP variants in neural networks and tree ensembles. Finally, we provided generalized complexity relations across SHAP variants. We believe that our framework offers a deeper understanding of the complexity involved in computing SHAP across various variants, models, distributions, as well as in both local and global computations, laying the groundwork for future research.



% \clearpage
\section*{Impact Statement}
This paper advances AI Scaling by integrating Scaling Up, Scaling Down, and Scaling Out to build efficient, adaptive, and decentralized AI ecosystems. While Scaling Out democratizes AI access across domains like education and healthcare, it also raises concerns about privacy, security, and fairness in decentralized AI marketplaces.


\nocite{langley00}

\bibliography{example_paper,use_cases,down}
\bibliographystyle{icml2025}


\end{document}

