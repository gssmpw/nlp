
Hypotheses are central to information acquisition, decision-making, and discovery. However, many real-world hypotheses are abstract, high-level statements that are difficult to validate directly. 
This challenge is further intensified by the rise of hypothesis generation from Large Language Models (LLMs), which are prone to hallucination and produce hypotheses in volumes that make manual validation impractical. Here we propose \mname, an agentic framework for rigorous automated validation of free-form hypotheses. 
Guided by Karl Popper's principle of falsification, \mname validates a hypothesis using LLM agents that design and execute falsification experiments targeting its measurable implications. A novel sequential testing framework ensures strict Type-I error control while actively gathering evidence from diverse observations, whether drawn from existing data or newly conducted procedures.
We demonstrate \mname on six domains including biology, economics, and sociology. \mname delivers robust error control, high power, and scalability. Furthermore, compared to human scientists, \mname achieved comparable performance in validating complex biological hypotheses while reducing time by 10 folds, providing a scalable, rigorous solution for hypothesis validation. \mname is freely available at \url{https://github.com/snap-stanford/POPPER}.



