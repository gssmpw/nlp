\section{Background}
\label{sec:background}

\subsection{ROS2 and ADS Midlewares}
Middleware solutions are crucial for Automated Driving Systems (ADS) software, providing the communication infrastructure necessary for seamless data exchange between various components. ROS2 has emerged as a leading middleware framework by leveraging the Data Distribution Service (DDS) \cite{pardo-castelloteOMGDataDistributionService2003} for decentralized communication, offering enhanced scalability, real-time performance, and security compared to ROS 1 \cite{quigleyROSOpensourceRobot}.

In ROS2, topics, publishers, and subscribers form the core of its publish-subscribe communication model:
\begin{itemize}
    \item \textbf{Topics} are named channels that facilitate the transmission of data between nodes.
    \item \textbf{Publishers} are nodes that send messages to specific topics.
    \item \textbf{Subscribers} are nodes that receive messages from topics they are interested in.
\end{itemize}

This structure allows multiple nodes to asynchronously exchange information efficiently. Additionally, ROS2 supports \emph{services} for synchronous, request-response interactions between nodes. Services enable one node to request specific actions or information from another, ensuring direct and immediate communication when needed.

Autoware \cite{autowarefoundationAutoware} operates on the ROS 2 middleware framework and serves as an open-source software stack for ADS applications. It encompasses functionalities such as localization, perception, prediction, and planning without delving into specific module implementations. Autoware supports features like valet parking, shuttle buses and cargo delivery. Its modular architecture facilitates scalability and the integration of future enhancements, leveraging ROS 2’s real-time processing and robust communication capabilities.

Baidu's Apollo \cite{baiduApolloOpenAutonomous} runs on a middleware called Cyber RT \cite{baiduIntroductionCyberRT}. Similar to ROS 2, Cyber RT is based on a publish-subscribe model but operates without a central core component, with each module running independently. It uses Google Protocol Buffers for efficient data serialization. In contrast to ROS2, Apollo's middleware is tailored for automated driving, integrating with tools for mapping, localization, traffic management, and vehicle control. Both ROS 2 and Cyber RT provide robust software development kits but differ in approach. Studies suggest Apollo's modules are more robust, though this deviates from the broader ROS2 ecosystem \cite{rajuPerformanceOpenAutonomous2019}. Conversely, ROS2's flexibility and extensive community support make it adaptable for various AV applications, especially in research and development.

Other middleware such as ZeroMQ \cite{ZeroMQ} are used in ADS development \cite{novickisFunctionalArchitectureAutonomous2020}. However, in our view,  ROS2 presents advantages by offering specialized real-time communication and reliability essential for AV safety and performance. Unlike ZeroMQ, which requires additional layers to achieve similar real-time capabilities, ROS2 provides built-in deterministic message delivery and robust quality of service settings. Additionally, ROS2’s ecosystem, supported by the community, accelerates prototyping, allowing developers to quickly integrate and test complex AV systems. 

Both Autoware and Apollo provide a comprehensive environment for ADS development. While these ecosystems are powerful for full-scale AV development, their complexity can hinder the rapid testing and validation of smaller proofs of concept, as they are designed for complete vehicle deployment \cite{rajuPerformanceOpenAutonomous2019}. Their extensive feature sets and deployment capabilities may introduce unnecessary overhead for early-stage development, where simplicity and agility are paramount.

\subsection{Service-Oriented Architecture}

As opposed to traditional monolithic architectures, Service-Oriented Architecture (SOA)  is a paradigm that employs discrete software components, or services, to assemble system applications. Each service encapsulates a distinct functionality, enabling interoperability across diverse platforms and programming languages. SOA facilitates the reuse of services across multiple systems or the integration of independent services to execute more complex tasks. This approach has been recognized by industry experts over several years \cite{10011506} \cite{7930217}. Its benefits have been empirically validated not only for Advanced Driver Assistance Systems (ADAS) functionalities \cite{8712376} but also across other domains \cite{8354415}.


In the automotive context, building on the SOA principles discussed above, services are represented as topics within a data-centric publisher-subscriber model using the DDS, the middleware framework used by ROS2. This structure has been effectively implemented in containerized environments, providing a highly controllable and configurable environment \cite{8417118}.


%Currently, microservices stand as the predominant SOA implementation for cloud-based applications \cite{SOAVsMicroservices}. However, this approach is progressively being extended to support complex, real-world robotic applications \cite{10611586}, which adhere to DevOps methodologies and have demonstrated superior performance relative to other architectural styles \cite{betzContainerizedMicroserviceArchitecture2024}.



\subsection{Simulation Platform}

Simulation platforms are an important part of the ADS development lifecycle, providing virtual environments for testing and refining vehicle behaviors without the substantial risks and costs associated with real-world trials. These platforms enable developers to model and evaluate vehicle behavior under diverse conditions, from everyday traffic scenarios to rare and extreme events that are challenging to replicate physically. Leveraging simulations allows for rapid iteration, early issue identification, and optimization of AV algorithms in a controlled and repeatable setting.

The evolution of simulators has seen distinct phases. Initially, from the 1990s to the early 2000s, the focus was primarily on simulating vehicle dynamics, leading to the creation of widely adopted tools such as CarSim \cite{MechanicalSimulationCorporation} and IPG CarMaker \cite{CarMakerIPGAutomotive}. These early platforms enabled detailed modeling of vehicle mechanics but were limited in their ability to simulate complex driving scenarios. The 2000-2015 period brought more sophisticated simulators, like rFpro \cite{AutomotiveSimulationDriving}, a critical tool for testing advanced driver assistance systems (ADAS) and autonomous driving technologies. Since 2015, a surge of open-source simulators has transformed AV research, driven by the demand for high-fidelity and versatile simulation platforms. CARLA \cite{pmlr-v78-dosovitskiy17a} has become one of the most widely adopted platforms, valued for its flexibility and compatibility with plugins like SUMO \cite{SUMO2018} for traffic simulation. Yet, CARLA's limitations in V2X communication simulations have led researchers to supplement it with network-focused simulators.

To address these gaps, several specialized simulators have emerged. Veins \cite{sommerVeinsOpenSource2019}, integrated with OMNeT++ \footnote{https://omnetpp.org/}, focuses on vehicular network (VANET) simulations, making it ideal for Connected Autonomous Vehicle (CAV) applications where communication fidelity is essential. Similarly, Eclipse MOSAIC \cite{schrabModelingITSManagement2023} offers a multi-domain approach, integrating SUMO for traffic flow and ns-3\footnote{https://www.nsnam.org/} for network simulation.

%Notable simulators under active development include VISTA 2.0 \cite{aminiVISTA20Open2021} and MetaDrive \cite{liMetaDriveComposingDiverse2021}. VISTA 2.0 enables end-to-end perception-to-control policy training, focusing on closed-loop testing in simulated environments, although it is currently optimized for single-vehicle tasks rather than cooperative driving. MetaDrive, emphasizing scenario generation, is advantageous for researchers focusing on diverse environments for reinforcement learning (RL) training, though its strengths lie more in scenario variety than in high-fidelity vehicle dynamics or cooperative traffic interactions.

Given its open-source nature, community outreach and developing environment, we selected CARLA as an initial target simulator. However, APIKS service-oriented architecture is agnostic to the simulating platforms through a bridge system. This means, ROS2 messages exchanged within APIKS are not from the nodes exposed by the simulators, but rather internal.

%\todo[inline]{Start - Add related work}

%As highlighted in \cite{10461065}, the development of autonomous driving simulators has evolved through several distinct phases. The initial phase, spanning from 1990 to 2000, predominantly focused on simulating vehicle dynamics, resulting in the creation of widely used tools such as CarSim \cite{MechanicalSimulationCorporation} and IPG CarMaker \cite{CarMakerIPGAutomotive}. The subsequent phase, from 2000 to 2015, witnessed only moderate advancements, with notable developments including the 2007 release of rFpro \cite{AutomotiveSimulationDriving}, which is now regarded as one of the most sophisticated driving simulators. The current phase, beginning in 2015, marks a resurgence in simulation technologies, driven especially by the emergence of open-source driving simulators.

%Autonomous driving simulators are frequently designed to emphasize particular functions, such as vehicle dynamics, driving policy development, traffic simulation, and sensory data generation. Nevertheless, only a limited number of simulators offer a fully integrated approach encompassing these diverse aspects, which is critical for accurately simulating the end-to-end autonomous driving process, from perception through to control.

%Each simulator exhibits unique strengths and constraints, with the most actively developed platforms increasingly striving for higher levels of realism. Among open-source simulators, CARLA \cite{pmlr-v78-dosovitskiy17a} has emerged as a leading choice, recognized for its versatility and extensive compatibility with various plug-ins, rendering it one of the most flexible simulation platforms in this domain. CARLA, for instance, enables vehicle dynamics simulations via tools such as CarSim or the open-source Gazebo \cite{1389727}, and can seamlessly integrate with MATLAB. Other simulators under continuous development include VISTA 2.0 \cite{aminiVISTA20Open2021}, tailored specifically for end-to-end testing and training of perception-to-control policies, and MetaDrive \cite{liMetaDriveComposingDiverse2021}, which prioritizes scenario generation \cite{liScenarionetOpensourcePlatform2024} to support reinforcement learning (RL) agent training.

%While earlier simulators incorporated Hardware-in-the-Loop (HIL) testing capabilities \cite{shahAirSimHighFidelityVisual2017}, no open-source simulator currently provides this feature natively. Some autonomous driving software stacks, notably Autoware Core/Universe \cite{HomePage} and Apollo \cite{GitHubApolloAutoApollo}, stand out as advanced, modular, open-source stacks and have been engineered to support HIL testing. Autoware, in particular, is recognized for its enhanced development tools and capabilities \cite{9033734}. Due to their modular design, these stacks support the testing of individual modules within a broader software architecture and are compatible with external frameworks \cite{10588748} and simulators \cite{10588623}. This modularity facilitates multilevel testing, ranging from basic HIL \cite{10186585} \cite{8443742} to full-stack validation on actual vehicles \cite{karleEDGARAutonomousDriving2023} \cite{8813784}.


%In recent years, numerous architectures have been developed to support HIL testing. For instance, AutoDRIVE \cite{samakAutodriveComprehensiveFlexible2023} offers a comprehensive digital twin ecosystem encompassing a development environment, simulator, and cost-effective physical components. It also features close integration with Autoware Universe \cite{samakAutonomyOrientedDigital2024}, which also underpins F1-TENTH \cite{okellyF110Opensource2019}—a platform designed for autonomous racing applications but also applicable to commercially available vehicle testing.



\subsection{ISO/TR 4804}

ISO/TR 4804:2020 \cite{isoRoadVehiclesSafety2020} provides detailed guidelines for ensuring the safety and cybersecurity of automated driving systems. The technical report emphasizes a systematic approach to the design, verification, and validation processes, aiming to guarantee reliable ADS operation across diverse scenarios. By employing a risk-based methodology, ISO/TR 4804 advocates for the early identification and mitigation of potential hazards throughout the development lifecycle.

The report integrates safety and cybersecurity considerations into the fundamental architecture of ADS, aligning with standards such as ISO 26262 \cite{isoRoadVehiclesFunctional} for functional safety and ISO/SAE 21434 \cite{isoRoadVehiclesCybersecurity} for cybersecurity engineering. This integration facilitates a cohesive framework that addresses the multifaceted challenges inherent in automated driving technologies. Key aspects include modular design principles, rigorous testing protocols, and the incorporation of cybersecurity measures from the outset. The principles outlined in the report support the development of platforms with adaptable architectures that can accommodate continuous updates and enhancements without compromising safety or security.

We chose to follow the architectural depiction outlined in the standard because of its emphasis on modularity, which significantly enhances safety. This modularity favors a design where individual components can be developed, tested, and validated independently, reducing the risk of systematic failures. It also allows for the seamless integration of software components, which is particularly relevant for SDVs.




