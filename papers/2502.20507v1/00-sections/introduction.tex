\section{Introduction}
\label{sec:intro}
The rapid advancement of autonomous vehicle (AV) technologies holds the promise of transforming transportation systems by enhancing safety, efficiency, and convenience. Autonomous vehicles have the potential to significantly reduce traffic accidents caused by human error \cite{wangQuantificationSafetyImprovements2024}, optimize traffic flow to alleviate congestion, and provide accessible transportation options for those unable to drive. They can revolutionize logistics and public transportation, leading to economic benefits and improved quality of life. However, ensuring the reliability and safety of AV software in complex and dynamic environments remains a significant challenge. These vehicles must navigate unpredictable real-world conditions, interpret vast amounts of sensory data in real-time, and make critical decisions that should prioritize the safety of all road users. Comprehensive testing and validation are essential to address these challenges, necessitating tools and frameworks that can efficiently simulate a wide array of real-world scenarios. Such simulations must encompass diverse environmental conditions, traffic patterns, and potential hazards to thoroughly evaluate the performance and safety of AV systems before deployment.

Simulation tools have become indispensable in the development and testing of AV systems. They enable extensive testing across diverse conditions much faster than real-world trials, achieving speeds up to $10^3$ to $10^5$ times faster \cite{fengDenseReinforcementLearning2023}. Existing simulators often emphasize specific functions—such as vehicle dynamics, driving policy development, traffic simulation, or sensory data generation. However, there is a scarcity of platforms offering an integrated approach for end-to-end AV testing that also supports rapid prototyping and streamlined development.

Comprehensive platforms like Apollo \cite{baiduApolloOpenAutonomous} and Autoware \cite{autowarefoundationAutoware} provide robust solutions capable of deployment in actual vehicles. While these platforms are invaluable for full-scale AV development, their complexity can hinder rapid testing and validation of small proofs of concept. The extensive feature sets and deployment capabilities introduce overhead that may not be necessary for early-stage development, making it challenging to quickly iterate and test new ideas.

In response to these challenges, we introduce the APIKS\footnote{Following the double-blind process, we cannot divulge the meaning of the accronym} automotive platform, a modular framework based on Robot Operating System 2 (ROS2) middleware \cite{macenskiRobotOperatingSystem2022} designed for the efficient testing and validation of autonomous vehicle software. APIKS offers a simpler architecture tailored for small proofs of concept, enabling rapid prototyping without the complexities associated with full-scale deployment platforms. By basing our development on industry standards, including alignment with the ISO/TR 4804 guidelines \cite{isoRoadVehiclesSafety2020}, APIKS builds on automotive best practices in safety and reliability. The framework streamlines development by being modular and software-centric, allowing developers to focus on specific components and functionalities as needed.

Our contributions are the following:

\begin{itemize}
    \item We present APIKS, a modular framework that facilitates efficient testing and validation of AV software, particularly suited for rapid prototyping and small proofs of concept.
    \item We showcase the potential of APIKS in a Construction Zone Assist (CZA) use case, illustrating its effectiveness in enabling the deployment and testing of new automated driving functionalities in a simulated environment.
\end{itemize}

The remainder of this paper is organized as follows: Section~\ref{sec:background} provides a background on the core concepts behind APIKS. Section~\ref{sec:framework} outlines the details of our proposed platform. Section~\ref{sec:cza} 
introduces the Construction Zone Assist, an exemplary use case of an automated vehicle function deployed in APIKS. Finally, Section~\ref{sec:conclusion} concludes the paper by summarizing our contributions and discussing potential avenues for future work in this domain.
