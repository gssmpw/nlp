\section{Construction Zone Assist}
\label{sec:cza}

To demonstrate APIKS, we deployed a widely relevant application for AV: the \emph{Construction Zone Assist} (CZA) system. Figure This automated functionality is conceived to detect highway construction zones with a high degree of safety and reliability. Construction zones present unique challenges due to their unpredictable and dynamic nature, which may include irregular road geometries, temporary signage, and a variety of static and dynamic obstacles such as construction machinery and workers. By leveraging depth camera technology and YOLOv8 \cite{ultralytics2023yolov8} for object detection, the CZA system enables automated vehicles to accurately perceive and navigate these complex environments.

The system is based on the premise that an existing automated driving feature, such as an \emph{autopilot}, is already integrated into the vehicle. When the vehicle operates in this automated mode and a construction zone is detected, the system assesses whether the current ODD permits a mode transition. If the conditions are favourable, the system switches to an alternative mode to activate the appropriate functionality for navigating the construction zone. The primary modifications occur within the vehicle's drive planning algorithms, tailored to navigate the static objects limiting the route, constraint target velocities and safety distances. This is supported by using Frenetix \cite{trauthFRENETIXHighPerformanceModular2024} to plan the trajectories according to \cite{werlingOptimalTrajectoryGeneration2010}. 

Figure \ref{fig:cza} illustrates a scenario demonstrating the operation of the CZA functionality. The green line represents the initially computed target path without knowledge of the temporary deviation. The red line depicts the real-time computed path that allows navigation through the construction zone. This example was deployed and tested using the APIKS platform connected to the CARLA simulator via its ROS-bridge. Additionally, we utilized the scenario runner feature, which allows us to abstract the testing of the functionality using a standardized format—namely, OpenSCENARIO from ASAM. This scenario demonstrates new functionalities are able to be deployed using previously developed modules within the APIKS platform to adapt the vehicle's path in real-time. It shows how the system effectively navigates through temporary deviations like construction zones, validating the functionality of the CZA feature.

