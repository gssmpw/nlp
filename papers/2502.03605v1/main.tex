\pdfoutput=1
\newcommand{\red}[1]{\textcolor{red}{ #1}}
\newcommand{\blue}[1]{\textcolor{blue}{ #1}}
\newcommand{\green}[1]{\textcolor{black}{ #1}}
\renewcommand{\baselinestretch}{1}

% \titlespacing*{\section}{0pt}{3pt}{3pt}
% \titlespacing*{\subsection}{0pt}{3pt}{2pt}
% \titlespacing*{\subsubsection}{0pt}{3pt}{2pt}

\documentclass[9pt, conference]{IEEEtran}
\usepackage{balance}
\usepackage{nopageno}
% \usepackage{graphicx}
% \usepackage{epstopdf}
\usepackage{float}
\usepackage{tabularx}
\usepackage{mathtools}
\usepackage{amsmath}
\usepackage{hhline}
\usepackage{silence}
\usepackage{graphicx}
\usepackage{soul}
% \usepackage{xcolor}
\usepackage{multirow}
\usepackage{graphicx}
\usepackage{multirow}
\usepackage{amsmath}
\usepackage[english]{babel}
\usepackage{booktabs}
\usepackage{array}
\usepackage{paralist}
\usepackage{threeparttable}
\usepackage{lipsum}
\usepackage{flushend}
\usepackage{cuted}
\usepackage{algpseudocode}
\usepackage{algorithm}
\usepackage{amssymb}
\usepackage{multicol}
\usepackage{subfig}
\usepackage{cite}
% \usepackage{theorem}
\usepackage{makecell}
\usepackage{url}
\usepackage[table]{xcolor}
\usepackage[normalem]{ulem}
% \usepackage{color}
% \usepackage{colortbl}
% \definecolor{mygray4}{rgb}{0.86,0.86,0.86}
% \usepackage{balance}

% \usepackage[justification=centering,font=small,labelfont=bf]{caption}
% \usepackage{subcaption}
% \usepackage{silence}
% \WarningFilter{caption}{Unsupported document class}
% \usepackage{enumitem}

% \definecolor{gray1}{gray}{0.90}
% \definecolor{gray2}{gray}{0.98}
% \definecolor{light-gray}{gray}{0.95}
% \def\bibfont{\small}
%Adds extra padding to the table cells
% \setlength\extrarowheight{2pt}

\newcommand{\ignore}[1]{}
\newcommand{\redHL}[1]{\textcolor{red}{#1}}
\newcommand{\blueHL}[1]{{\textcolor{blue}{#1}}}
\newcommand{\greenHL}[1]{\textcolor{green!75!black}{#1}}
\newcommand{\blackHL}[1]{\textcolor{black}{#1}}
\newcommand{\grayHL}[1]{\textcolor{gray1}{#1}}
\newcommand{\Alter}[2]{\sout{#1}\redHL{#2}}
\newcommand{\redfn}[1]{\redHL{\footnote{\redHL{#1}}}}
\newcommand{\bluefn}[1]{\blueHL{\footnote{\blueHL{#1}}}}
\newcommand{\txtoverline}[1]{$\overline{\mbox{{#1}}}$}
% \newcommand{\TMR}{\mbox{TMR}}
% \newcommand{\sTMR}{\scriptsize{\mbox{TMR}}}
\pagestyle{plain}
% \renewcommand{\bibfont}{\small}
% \usepackage{tikz}
% \usetikzlibrary{tikzmark}
%\usetikzlibrary{shapes,arrows}
% \renewcommand{\baselinestretch}{0.93}
%\selectcolormodel{gray}

\newcommand\blfootnote[1]{%
  \begingroup
  \renewcommand\thefootnote{}\footnote{#1}%
  \addtocounter{footnote}{-1}%
  \endgroup
}

\usepackage{listings}
\lstset{
   breaklines=true,
   basicstyle=\ttfamily}

\DeclareMathOperator{\sigmoid}{sigmoid}
\DeclareMathOperator{\ReLU}{ReLU}
\DeclareMathOperator{\Addition}{Addition}
\DeclareMathOperator{\Subtraction}{Subtraction}
\DeclareMathOperator{\Multiplication}{Multiplication}
\DeclareMathOperator{\Average}{Average}
\DeclareMathOperator{\Int}{Int}


\begin{document}

\title{Accelerating OTA Circuit Design: Transistor Sizing Based on~a Transformer Model and Precomputed Lookup Tables}

% \author{\IEEEauthorblockN{Subhadip Ghosh}
% \IEEEauthorblockA{\textit{University of Minnesota}\\
% Minneapolis, MN, USA}
% \and
% \IEEEauthorblockN{Endalk Y. Gebru}
% \IEEEauthorblockA{\textit{University of Minnesota}\\
% Minneapolis, MN, USA}
% \and
% \IEEEauthorblockN{Chandramouli Kashyap}
% \IEEEauthorblockA{\textit{Cadence Design Systems}\\
% Portland, OR, USA}
% \and
% \IEEEauthorblockN{Ramesh Harjani}
% \IEEEauthorblockA{\textit{University of Minnesota}\\
% Minneapolis, MN, USA}
% \and
% \IEEEauthorblockN{Sachin S. Sapatnekar}
% \IEEEauthorblockA{\textit{University of Minnesota}\\
% Minneapolis, MN, USA}
% }

\author{
Subhadip Ghosh$^1$, Endalk Y. Gebru$^1$, Chandramouli V. Kashyap$^2$, Ramesh Harjani$^1$, Sachin S. Sapatnekar$^1$ \\
$^1$\text{Department of Electrical and Computer Engineering, University of Minnesota}, Minneapolis, MN, USA \\
$^2$\text{Cadence Design Systems}, Portland, OR, USA
}


\maketitle

\begin{abstract}
Device sizing is crucial for meeting performance specifications in operational transconductance amplifiers (OTAs), and this work proposes an automated sizing framework based on a transformer model. The approach first leverages the driving-point signal flow graph (DP-SFG) to map an OTA circuit and its specifications into transformer-friendly sequential data. A specialized tokenization approach is applied to the sequential data to expedite the training of the transformer on a diverse range of OTA topologies, under multiple specifications. Under specific performance constraints, the trained transformer model is used to accurately predict DP-SFG parameters in the inference phase. The predicted DP-SFG parameters are then translated to transistor sizes using a precomputed look-up table-based approach inspired by the $g_m/I_d$ methodology. In contrast to previous conventional or machine-learning-based methods, the proposed framework achieves significant improvements in both speed and computational efficiency by reducing the need for expensive SPICE simulations within the optimization loop; instead, almost all SPICE simulations are confined to the one-time training phase. The method is validated on a variety of unseen specifications, and the sizing solution demonstrates over 90\% success in meeting specifications with just one SPICE simulation for validation, and 100\% success with 3--5 additional SPICE simulations. 
% \redHL{Can we quantify ``small'' -- e.g., ``after 3-5 iterations''? Can we change ``iteration'' to ``SPICE simulation''?\textbf{YES, How about ``additional'' with that?}}

\end{abstract}

\section{Introduction}

\subsection{Background and Motivation}
Integrating Deep Reinforcement Learning (DRL) in financial market analysis significantly evolved investment analysis with Deep Learning. DRL combines deep learning and reinforcement learning to offer a sophisticated framework for adapting strategies in the dynamic financial domain. It allows a deep learning model to effectively decipher complex patterns in historical market data often overlooked by traditional quantitative models.
It is no secret that financial markets are inherently complex and influenced by economic trends and geopolitical events. Therefore, traditional financial modeling often struggles to adapt to these ever-changing conditions. However, with its direct learning from data and adaptive strategies, DRL presents a promising solution to these challenges. With its autonomous learning ability and continual adaptation to the financial environment, it leverages historical market data to identify complex relationships and patterns.


\subsection{Overview of Our Previous Work}
In recent years, significant progress has been made in applying deep reinforcement learning (DRL) to stock trading strategies. For instance, Wang et al. proposed a parallel multi-module DRL algorithm that effectively captures both current market conditions and long-term dependencies using fully connected and LSTM layers \cite{parallel_drl_stock_trading}. Zhang et al. introduced an automated stock trading system based on the Proximal Policy Optimization algorithm, modeling trading as a Markov decision process \cite{novel_drl_stock_trading}. Additionally, Huang et al. demonstrated the importance of integrating market sentiment data to enhance the performance of DRL models in trading \cite{market_sentiment_drl_stock_trading}. Liu et al. developed an end-to-end trading strategy using a multi-view environment representation neural network, incorporating a Long Memory mechanism to improve decision-making \cite{drl_end_to_end_stock_trading}. Lastly, Li et al. focused on adaptive trading strategies using Gated Recurrent Units to capture time-series data effectively \cite{adaptive_drl_stock_trading}. These studies collectively highlight the potential of DRL in creating robust and adaptive trading strategies.

Liu et al. significantly advanced Deep Reinforcement Learning in Finance by developing platforms such as FinRL-Meta \cite{Liu2022FinRL}. This platform is a comprehensive tool for training and evaluating data-driven reinforcement learning agents within several simulated financial markets, offering a robust benchmarking system for algorithm comparison and facilitating the simulation of complex market conditions. The FinRL platform enables researchers to refine and test the efficacy of various DRL strategies, and it has been pivotal in democratizing access to sophisticated financial simulation tools and propelling research in financial analysis.

FinRL uses environments that offer broad simulation capabilities. These specialized environments, such as ABIDES-Gym \cite{Vyawahare2020}, provide the necessary infrastructure that allows FinRL to create discrete event simulations explicitly tailored for financial markets. ABIDES-Gym extends the OpenAI Gym interface to accommodate the complex dynamics of financial trading, allowing for a nuanced replication of market mechanisms and agent interactions. This level of detail will enable researchers and practitioners to explore the impact of individual agent behaviors and market responses, thus enhancing the understanding of market microstructure and agent-based modeling. The framework also streamlines the model training process on financial datasets, epitomizing the intersection of DRL and high-performance computing. It Leverages distributed computing resources to reduce training times significantly and optimizes computational workflows to enable the application of complex DRL models to extensive financial tasks. Their efforts have led to the creation of scalable and efficient financial models.

Our previous work \cite{Montazeri2023} demonstrated the efficiency and capability of CNNs when used as policies for deep reinforcement learning. We utilized the FinRL platform to conduct experiments on CNNs as a significantly improved policy to FinRL's original proposition. We also showed \cite{Montazeri2024, Montazeri2024GradientRC} that rearranging the stock market features used in the FinRL platform to group them per company could benefit the mode's performance. This study also utilizes the FinRL platform with its original dataset, containing features generated through traditional Technical Analysis used in Finance. It also uses the new dataset introduced in FinRL Meta, which contains statistically engineered features such as Simple Moving Average (SMA), momentum, and rate of change.

Building upon these foundational studies, our research aims to bridge the gap between CNN architecture optimization and financial market analysis. By introducing a systematic approach to temporal window selection, we seek to enhance the adaptability and performance of DRL models in capturing complex market dynamics.
    
\section{Objectives of the Current Study}
So far, we have presented the literature and the setting in which our study operates. The primary objective of our research is to explore the effects of changing the temporal window of a Convolutional Neural Network (CNN) used as a policy in a FinRL. By progressively expanding the observation period, beginning with a concise two-week window and incrementally enlarging it by two weeks in each iteration and culminating in twelve weeks, we aim to observe and analyze the performance of our model as its temporal window changes in the FinRL platform. This iterative window expansion is designed to examine the impact of different temporal scales on the model's performance. This process enables a comprehensive analysis of how varying lengths of financial data affect the model's predictive capabilities, offering insights and an opportunity to optimize the temporal granularity for financial market analysis. Our study also examines the arrangement of feature vectors within these expanding windows to better understand the model-market dynamics.

Furthermore, we contrast the model's performance across these different temporal windows to discern patterns in market behavior and model performance. In our study, short-term windows, particularly the initial two-week period, are hypothesized to be critical for understanding the model's ability to capture immediate market changes and short-term trends, which are essential for timely and accurate trading predictions. As the window expands, the model is expected to integrate a broader spectrum of market conditions, capturing longer-term trends and patterns. This bi-weekly expansion strategy is designed to balance the benefits of short-term immediacy and long-term historical perspective, ensuring the model remains adaptable and responsive to transient market fluctuations and enduring trends. We hope to contribute to financial analytics by demonstrating the efficacy of CNNs in a DRL setting and by providing new insights into the role of temporal dynamics in financial modeling.
\subsection{Gene Expression Classification with ML models}
Gene expression classification \cite{do2024enhancing,do2023ensemble,huynh2019novel} lies at the forefront of biomedical research, offering profound insights into the molecular mechanisms underlying various diseases. ML models have become indispensable in this domain, as they can uncover complex patterns within vast and high-dimensional gene expression datasets. However, these datasets often contain a plethora of features, many of which are redundant or irrelevant, potentially obscuring the most critical biological signals and leading to overfitting. Consequently, feature selection becomes imperative—it refines the dataset by isolating the most informative genes, thereby enhancing model accuracy, interpretability, and computational efficiency. By focusing solely on the pivotal biomarkers, this research is able to achieve more reliable predictive outcomes. In this paper, we investigate and evaluate the classification with various ML techniques. Namely, we experiment our selected features with ML algorithms, i.e., SVM \cite{vapnik1995support}, Random Forest \cite{breiman2001random}, XGB \cite{chen2015xgboost}, Gradient Boosting \cite{friedman2002stochastic}.

\begin{definition}[Classification]
Let \( D = (X, y) \) be a dataset where \( X \subseteq \mathbb{R}^n \) is the feature space and \( y \in \mathcal{Y} = \{1,2,\dots,k\} \) represents the class labels. A classifier is a function
\[
f: X \to \mathcal{Y},
\]
that assigns a predicted label \( \hat{y} = f(x) \) to each input \( x \in X \). The function \( f \) is learned from the labeled examples
\[
D = \{(x_i, y_i) \mid x_i \in X,\; y_i \in \mathcal{Y},\; i = 1, \dots, N\},
\]
by minimizing a loss function \( \ell: \mathcal{Y} \times \mathcal{Y} \to \mathbb{R}_{\ge 0} \) that quantifies the error between the predicted and true labels. Once trained, \( f \) is used to classify new, unseen inputs.
\end{definition}

% \begin{definition}[Classification Using Machine Learning]
% Let \( D_{\text{selected}} = (X_{\text{selected}}, y) \) be the dataset with features \( X_{\text{selected}} \subseteq X^* \) as determined by LIME. A classifier is a function 
% \[
% f: X_{\text{selected}} \to \mathcal{Y},
% \]
% that assigns a predicted label \( \hat{y} = f(x) \) to each input \( x \in X_{\text{selected}} \). The classifier is trained on the labeled examples
% \[
% D_{\text{selected}} = \{(x_i, y_i) \mid x_i \in X_{\text{selected}},\; y_i \in \mathcal{Y},\; i = 1, \dots, N\},
% \]
% by minimizing a loss function \( \ell: \mathcal{Y} \times \mathcal{Y} \to \mathbb{R}_{\ge 0} \) that measures the discrepancy between the predicted and true labels. The trained classifier is then used to predict the classes of new, unseen instances.
% \end{definition}


Feature selection is crucial before classification begins. Our study focuses on two techniques: Boruta and LIME. 
% Boruta is chosen for its robustness in identifying all relevant features in high-dimensional datasets, ensuring no important predictor is missed. LIME is used for its ability to provide interpretable, local explanations of model predictions, which is essential for evaluating feature importance. 
We now introduce Boruta and LIME in the following sections.

\subsection{Leveraging Boruta for Robust Feature Extraction}
Boruta \cite{kursa2010boruta,zhou2023diabetes} is a powerful wrapper-based feature selection algorithm designed to identify all truly relevant variables in a dataset. By comparing the importance of actual features with that of randomly generated ``shadow'' features, Boruta systematically filters out irrelevant variables while preserving essential predictors. This rigorous selection process is particularly valuable in high-dimensional applications, such as gene expression classification, where capturing meaningful signals is crucial. For clarity, we formally define Boruta as follows:
\begin{definition}[Boruta Feature Selection]
Let \( D = (X, y) \) be a dataset with features \( X = \{x_1, x_2, \dots, x_p\} \) and target \( y \). The Boruta algorithm identifies all relevant features in \( X \) as follows:
\begin{enumerate}
    \item \textbf{Shadow Feature Generation:} For each \( x_i \in X \), create a shadow feature \( x_i^{\text{shadow}} \) by randomly permuting its values, forming the set \( X^{\text{shadow}} \).
    \item \textbf{Importance Estimation:} Train a classifier (e.g., Random Forest) on the combined set \( X \cup X^{\text{shadow}} \) and compute the importance score \( I(z) \) for each \( z \).
    \item \textbf{Feature Comparison:} For each \( x_i \), define
    \[
    I^{\text{shadow}}_{\max} = \max_{z \in X^{\text{shadow}}} I(z).
    \]
    Then classify \( x_i \) as \emph{relevant} if \( I(x_i) \) is significantly greater than \( I^{\text{shadow}}_{\max} \), \emph{irrelevant} if significantly lower, or \emph{tentative} otherwise.
    \item \textbf{Iteration:} Remove irrelevant and tentative features and repeat until all features are decisively classified.
\end{enumerate}
The final selected subset \( X^* \subseteq X \) comprises all features deemed relevant.
\end{definition}

After applying the Boruta algorithm, we retain only the relevant features (confirmed) and excluded the tentative and irrelevant features (rejected). To further enhance the selection of features in \(X^*\), we employed the AI explanation technique outlined in the following section.

% \begin{definition}[Boruta Feature Selection]
% Given a dataset \( D = (X, y) \) with original features \( X = \{ x_1, x_2, \dots, x_p \} \), Boruta augments \( X \) by creating shadow features \( X^{\text{shadow}} = \{ x_1^{\text{shadow}}, \dots, x_p^{\text{shadow}} \} \) via random permutation. A model \( M \) (e.g., Random Forest) is then trained on \( X \cup X^{\text{shadow}} \) to compute importance scores \( I(z) \) for every feature \( z \). For each \( x_i \in X \), if \( I(x_i) \) is significantly greater than the maximum shadow importance \( I^{\text{shadow}}_{\max} = \max_{z \in X^{\text{shadow}}} I(z) \), then \( x_i \) is marked as relevant; otherwise, it is rejected or considered tentative. Iterating this process yields the final set of selected features \( X^* \subseteq X \).
% \end{definition}

\subsection{XAI for Feature Selection}
Explainable AI (XAI) \cite{dwivedi2023explainable,zacharias2022designing} represents a forefront of AI research, aiming to elucidate the decision-making processes of complex models. In the context of gene expression classification, where feature selection is pivotal to model performance and interpretability, our study leverages LIME—Local Interpretable Model-Agnostic Explanations—to demystify and select critical features. LIME approximates the behavior of a sophisticated, black-box model with a simpler, locally interpretable surrogate, thereby pinpointing the most influential predictors in the vicinity of a given instance. This approach enhances the transparency of the model's predictions and facilitates a more informed and rigorous feature selection process, ultimately contributing to both improved accuracy and trustworthiness of the classification system.  Now, we provide a formal definition of LIME as follows:

% \begin{definition}[LIME-based Feature Selection]
% Let \( D = (X, y) \) be a dataset and \( f: X \to \mathcal{Y} \) a trained black-box classifier, where \( X \subseteq \mathbb{R}^p \) and \( \mathcal{Y} = \{1,2,\dots,k\} \). For a given instance \( x \in X \), LIME constructs an interpretable surrogate model \( g \) from a simple model class \( G \) (typically linear), expressed as
% \[
% g(z) = w_0 + \sum_{j=1}^{p} w_j z_j.
% \]
% The surrogate \( g \) is fitted by minimizing the weighted loss
% \[
% \min_{g \in G} \sum_{z \in Z_x} \pi_x(z) \left( f(z) - g(z) \right)^2 + \Omega(g),
% \]
% where \( Z_x \) is a set of perturbed samples around \( x \), \( \pi_x(z) \) is a proximity measure between \( z \) and \( x \), and \( \Omega(g) \) is a regularization term enforcing simplicity. The absolute coefficients \( |w_j| \) quantify the local importance of each feature, thus guiding feature selection.
% \end{definition}
\begin{definition}[LIME-based Feature Selection]
Let \( D^* = (X^*, y) \) be the dataset resulting from Boruta, where \( X^* \subseteq \mathbb{R}^{p^*} \) is the set of relevant features. Given a trained black-box classifier \( f: X^* \to \mathcal{Y} \) and an instance \( x \in X^* \), LIME constructs an interpretable surrogate model \( g \in G \) (typically linear), expressed as
\[
g(z) = w_0 + \sum_{j=1}^{p^*} w_j z_j,
\]
by solving the optimization problem
\[
\min_{g \in G} \sum_{z \in Z_x} \pi_x(z) \left( f(z) - g(z) \right)^2 + \Omega(g),
\]
where \( Z_x \) is a set of perturbed samples in the neighborhood of \( x \), \( \pi_x(z) \) is a proximity measure, and \( \Omega(g) \) enforces simplicity. The absolute coefficients \( |w_j| \) indicate the local importance of each feature, enabling a further refined selection \( X_{\text{selected}} \subseteq X^* \) for classification.
\end{definition}


To clarify, our choice of LIME for feature selection arises from the critical question of determining the optimal number of features for the model. In this context, assessing the local importance of each vector proves to be the most effective strategy, leading us to introduce the BOLIMES algorithm. The following section will provide a comprehensive explanation of the BOLIMES algorithm and its application.

%--------------------





\section{Proposed Methodology}
\label{sec:sizing_framework}

\subsection{Overview of the solution}

\noindent
We leverage transformer models to capture the complex relationships between device attributes and circuit performance. We conceptualize the transistor sizing problem as a language translation task, where the input sequence consists of a DP-SFG representation for an OTA circuit, together with performance specifications. The transformer model generates an enhanced DP-SFG representation with the device characteristics necessary to meet the specifications.

\begin{figure}[b]
  \vspace{-5mm}
  \centering
  \includegraphics[width=0.85\linewidth,bb=0 0 295 180]{fig/Toplevel.pdf} % Replace example-image with your image file
  % \vspace{-0.5cm}
  \caption{Overall sizing flow using our transformer-based method.}
  \label{fig:toplevel}
  % \vspace{-2mm}
\end{figure}

Fig.~\ref{fig:toplevel} illustrates the workflow of our framework, with four stages. Stage I performs preprocessing, generating the DP-SFG from the circuit netlist.  The DP-SFG and the designer-specified performance constraints are then tokenized into a combined sequence. Next, in Stage II, a transformer model processes these tokens to predict circuit parameters that meet performance specifications; these are then translated to individual device widths in Stage III using the precomputed LUTs and a $g_m/I_d$ methodology. Finally, in Stage~IV, the performance of the predicted sized circuit is verified using SPICE simulation. In a vast majority of cases, we will show that the performance criteria are satisfied; if not, the designer is brought into the loop to provide tighter specifications, and procedure is reinvoked so that the original specifications are met. The remainder of this section discusses the detailed implementation of each stage.

% 
\small
\begin{algorithm*}[t]
\vspace{-0.5cm}
\caption{Automated DP-SFG Generation for single phase analog circuits}
\label{algo:dpsfg}
\begin{multicols}{2}
\begin{algorithmic}[1]
\State \textbf{Input:} Netlist of circuit components (RCs, transitors, etc.)
\State \textbf{Output:} DP-SFG of the circuit
\State $V \gets \text{All terminals in the netlist as nodes}$  \label{algo1:init_begin}
\State $D_{nc} \gets \text{Initialize node-component dictionary including parasitics}$
\State $D_{cn} \gets \text{Initialize component-node dictionary including parasitics}$
\State $T_{n} \gets \text{All transistors and corresponding terminals}$ \label{algo1:init_end}

\Statex \textit{// \textbf{Step 1}: Initialization of auxiliary nodes}
\For{node $n$ in $D_{nc}$} \textit{// Iterate over all nodes} \label{algo1:step1_begin}
    % \If{$n$ is not connected to $Ground$}
        % \State $C \gets$ All components $c$ connected to $n$
        % \If {$\nexists$ no voltage in $C$}        
    %         \State $F_{aux}[n] \gets 1$ \textit{//node $n$ auxiliary flag assigned 1}
    %         \State Create auxiliary node $\Ddot{n}$
    %         \State Create edge $(\Ddot{n}, n)$ = driving-point impedance at $n$
    %     \EndIf
    % \EndIf
    \If{ $n \not =$ ground \textbf{or} connected to voltage source}   
        \State $F_{aux}[n] \gets 1$; Create auxiliary node $\hat{n}$
        \State Add edge $(\hat{n}, n)$; weight = driving-point impedance at $n$
    \EndIf    
\EndFor \label{algo1:step1_end}
\Statex \textit{// \textbf{Step 2}: Adding branches due to passive components}
\For{component $c$ connected between terminals $i, j \in D_{cn}$} \label{algo1:step2_begin}
    \State $G_c \gets$ admittance of component $c$ 
    \If {node $t \in \{i,j\}$ connects to an auxiliary source node $\hat{t}$}
    \State Add an edge from the other terminal to $\hat{t}$ with weight $G_c$
    \Else
    \State Add an edge from $i$ to $j$ with weight $G_c$ 
    % \redHL{Not clear. These are directed edges (I think) - so why $i$ to $j$, and not $j$ to $i$. The two terminals of a resistor are indistinguishable in terms of direction. \textbf{I did it according to the terminals ...like PLUS terminal to MINUS terminal of Resistors from netlist convention.....to be exact Higher potential to lower} {\em How do you know the potential? You have not solved the circuit and you don't know any voltages.}}
    \EndIf
    % \State {\em Is this what you want to say? \textbf{The edge should be from the non-aux node to the auxnode } OK now?yes YES (If so, pl delete this comment and remove the lines below. If not, please let me know how to correct it. As you will see, this is much more human-readable. The goal of pseudocode is to explain an algorithm clearly, rather than to be semantically correct and exhaustive wrt listing all if-then-else cases. Please use this guidance to adjust Step 3 (I have not seen your most recent version. FYI, I am landing soon and may be offline till ~10pm CT. Meanwhile, can you look at how I have rewritten Step 1, and write Steps 2 and 3 in that way? I first explain the {\em principle} and then go to the example of Fig. 2 to provide a particular case. Thanks. Also: trans-conductance $\rightarrow$ transconductance (``trans'' is not a stand-alone word (I can hyphenate ``stand-alone'' because ``stand'' and ``alone'' are both valid words.). Pl. search for all hyphenated words and fix.} OK PROF
    
    % \State $F_{aux}[s] == 1$ ? $(d, \hat{s})$ $\gets$ $G_c$ 
    % \State $F_{aux}[d] == 1$ ? $(s, \hat{d})$ $\gets$ $G_c$ 
    %\State If neither node is auxiliary, create edge $(s, d)$ $\gets$ $G_c$
\EndFor \label{algo1:step2_end}

\Statex\textit{// \textbf{Step 3}: Adding branches due to transconductance $g_m$}

\For{$t$ in $T_n$} \textit{// Iterate over all transistors} \label{algo1:step3_begin}
    \Statex \textit{// $I_d[t]$: Drain current of t; $g$: Gate; $s$: Source, $d$: Drain}

    \If{$V_{gs} \neq 0$}
        \If{transistor $t$ is NMOS}
            \State For node $i$, $j$  in $\{s,d\}$ terminals:
            \If{Voltage at node $i \propto I_d[t]$}
                \State Add edges from $i$ to $\hat{i}$ with weight $-g_m$ 
                \State Add an edge from $i$ to $\hat{j}$ with weight $+g_m$ 
            \Else  
                \State (Same as above, but negate the edge weights)
            \EndIf
            \State Add an edge from $g$ to $\hat{d}$ with weight $-g_m$ 
            \State Add an edge from $g$ to $\hat{s}$ with weight $+g_m$ 
        \Else \textit{// For PMOS transistors}
            \State (Same as above, but exchange ``then'' and ``else'')
        \EndIf
\EndIf
\EndFor \label{algo1:step3_end}


% \For{$t$ in $T_n$} \textit{// Iterate over all transistors}
%     \If{$V_{gs} \neq 0$}
%         \If{transistor $t$ is NMOS}
%             \State For node $t$ in \{s,d,g\} terminals:
%             \If{Voltage at node $t \propto I_d[t]$}
%                     \State $(t, \hat{x}) \gets (I_d[t] \propto V_{x} \text{ and } x \neq g) \ ? \ -g_m : +g_m$
%                     \State $(x, \hat{y}) \gets (I_d[t] \propto V_{x}) \ ? \ +g_m : -g_m$
%                 \Else
%                     \State $(g, \hat{d}) \gets -g_m$
%                     \State $(g, \hat{s}) \gets +g_m$
%                 \EndIf
%         \Else
%             \State For PMOS transistors, the logic for edge creation would be opposite.
%         \EndIf
% \EndIf
% \EndFor





\end{algorithmic}
\end{multicols}
\vspace{-0.3cm}
\end{algorithm*}
\normalsize


\vspace{-2mm}
\subsection{Circuit-to-sequence mapping using the DP-SFG}
\label{sec:dp-sfg}


\noindent
The procedure for creating the DP-SFG formalizes the approach in~\cite{schmid_18,schmid_yt}. We use the running example of an active inductor circuit, whose DP-SFG is shown in Fig.~\ref{fig:DP-SFG_ex}(b) to illustrate the method.


\noindent
\textbf{Step 0: Initial bookkeeping and node initialization}
The algorithm begins with initializing the vertex (or node) set $V$, and initializing data structures for fast access to connectivity information between circuit components (RCs, transistors, etc.) from the netlist. 

\noindent
\textbf{Step 1: Insertion of auxiliary nodes.} 
For nodes that are not connected to voltage sources, we create auxiliary voltage sources. These sources are described by $V = z I$, where $z$ is the driving point impedance (DPI) at the node, i.e., the the inverse sum of all conductances connected to the node. In Fig.~\ref{fig:DP-SFG_ex}, auxiliary sources are added at nodes~1 and~2. The sources replicate node voltages without introducing additional current into the circuit.  This establishes relationships $V_1 = z_1 I_1 $ and $V_2 = z_2 I_2$, where 
\begin{equation}
z_1 = \frac{1}{sC+sC_{\textit{ds}}+sC_{\textit{gs}}+g_{\textit{ds}}}, \; \;
z_2 = \frac{1}{sC+sC_{\textit{gs}}+G}
\end{equation}

\noindent
\textbf{Step 2: Adding branches due to passive components.} Next, we add the edges associated with passive components. We consider how each terminal connects to auxiliary sources. If a terminal connects to an auxiliary source, we connect it to the auxiliary node of the other terminal using its admittance as the edge weight. If neither terminal connects to an auxiliary source, we connect them with an edge using the admittance of the component. In Fig.~\ref{fig:DP-SFG_ex}, the terminals of capacitor \( C \) are both connected to auxiliary sources carrying currents \( I_1=s(C + C_{\textit{gs}}) V_2 \) and \( I_2=s(C + C_{\textit{gs}}) V_1 \) through the associated edges.

\noindent
\textbf{Step 3: Adding branches due to transistor transconductances.} 
The voltage at each terminal of a transistor influences its drain current. Based on these terminal voltages, we establish connections that directly or indirectly affect the auxiliary node currents. In the example of Fig.~\ref{fig:DP-SFG_ex}, if $V_1$ increases, the drain current $I_d$ increases, and hence current flowing to $I_1$ decreases in the opposite direction. This is reflected by setting the weight on the branch from $V_1$ to $I_1$ to $-g_m$. Similarly, the branch $V_2$ to $I_1$ has weight $+g_m$, reflecting the positive dependence of drain current $I_d$ with $V_2$. 

\begin{figure}[t]
  \centering
  \includegraphics[width=0.9\linewidth, bb=0 0 240 170]{fig/dpsfg_seq.pdf} % Replace example-image with your image file
  \caption{\textbf{Input:} DP-SFG paths with desired performance specifications, \textbf{Output:} Predicted sequences with device parameter values.}
  \label{fig:seq}
  \vspace{-0.4cm}
\end{figure}

At the end of Step~3, we obtain the final DP-SFG in Fig.~\ref{fig:DP-SFG_ex}(b) for the active inductor circuit. We will encode such a DP-SFG into sequential data that encapsulates the functionality of the circuit as well as the parameters of circuit components, including parasitics. This sequence representation is provided as an input to the transformer and is eventually used to size transistors in the circuit. We utilize the NetworkX Python package to process the final DP-SFG. This approach utilizes Johnson's algorithm ($O(V^2 \log V + VE)$ complexity) to identify all cycles, and the depth-first search algorithm ($O(V + E)$ complexity) to find all forward paths, where $V$ represents the number of nodes and $E$ represents the number of edges in the graph. 
For our running example, Fig.~\ref{fig:seq} shows the sequences obtained from the DPSFG. Specifically, the path outlined by red dotted rectangle represents the forward path between input and output nodes, while the blue and green outlined paths denote the cycles.
% \bluefn{(1)~This does not make sense. There are no paths in this figure: the paths are in Fig. 2.  \textbf{FIXED} Additionally, ``blue/green/red marked paths'' does not make sense because the paths are only outlined with a dotted rectangle, not ``marked'' by these colors. \textbf{FIXED} See change to next paragraph for the specs, and please change this accordingly. (2)~The blue path in Fig. 2 does not seem to correspond to the path shown with the blue dotted rectangle that you reference here, which is supposed to be a forward path between input and output nodes -- that path is a cycle.  Same for the path marked by the green dotted rectangle in Fig. 2. Are they supposed to be related? If so, please make consistent. And if not, please change because it is confusing. \textbf{FIXED}} SSS_NOTE

















\ignore{
\noindent
\textbf{Step 0: Initial bookkeeping and node initialization (lines~\ref{algo1:init_begin} to~\ref{algo1:init_end})}
The algorithm begins with initializing the vertex set $V$,
and initializing data structures for fast access to connectivity information between circuit components (RCs, transistors, etc.) from the netlist. 

\noindent
\textbf{Step 1: Insertion of auxiliary nodes (lines~\ref{algo1:step1_begin} to~\ref{algo1:step1_end})} 
Following the DP-SFG construction procedure in~\cite{schmid_18}, for nodes that are not connected to voltage sources, we create auxiliary voltage sources. These sources are described by $V = z I$, where $z$ is the driving point impedance (DPI) at the node, i.e., the
the inverse sum of all conductances connected to the node. In Fig.~\ref{fig:DP-SFG_ex}, auxiliary sources are added at nodes~1 and~2. The sources replicate node voltages without introducing additional current into the circuit. 
 
\redHL{CK: I haven't checked the math as it is due today. This is the part I worry about. Also SFG fomulations apply to only "linearizable" circuits like an op-amp. We should say something about handling general non-linearities. You could say that usually there is linear model lurking somewhere by transformation for example a PLL in phase domain. For others, this could used for studying stability by perturbation around a steady state which can be modeled linearly.}
This establishes relationships $V_1 = z_1 I_1 $ and $V_2 = z_2 I_2$, where 

\begin{equation}
z_1 = \frac{1}{sC+sC_{\textit{ds}}+g_{\textit{ds}}}, \; \;
z_2 = \frac{1}{sC+G}
\end{equation}

\noindent
\textbf{Step 2: Adding branches due to passive components and external sources (lines~\ref{algo1:step2_begin} to~\ref{algo1:step2_end})} Next, we add the edges due to passive components. We consider how each terminal connects to auxiliary sources. If a terminal connects to an auxiliary source, we connect it to the auxiliary node of the other terminal using the admittance of the component as the edge weight. If neither terminal connects to an auxiliary source, we connect them with an edge using the admittance of the component. In Fig.~\ref{fig:DP-SFG_ex}, both terminals of capacitor \( C \) are connected to auxiliary sources, allowing currents \( I_2 = (sC) V_1 \) and \( I_1 = (sC) V_2 \) to flow through the edges.


\noindent
\textbf{Step 3: Adding branches due to transistor transconductance (lines~\ref{algo1:step3_begin} to~\ref{algo1:step3_end})} 
In this step, we analyze the impact of the transistor transconductance. The voltage at each terminal of a transistor influences its drain current. Based on these terminal voltages, we establish connections that directly or indirectly affect the auxiliary node currents. In the example of Fig.~\ref{fig:DP-SFG_ex}, if $V_1$ increases, the drain current $I_d$ increases, and hence current flowing to $I_1$ decreases. Due to this the weight on the branch from $V_1$ to $I_1$ is set to $-g_m$. Similarly, the branch $V_2$ to $I_1$ has weight $+g_m$, reflecting the positive dependence of drain current $I_d$ with $V_2$. 

\begin{figure}[ht]
  \centering
  \includegraphics[width=\linewidth]{fig/DPSFG_paths.pdf} % Replace example-image with your image file
  \vspace{-0.5cm}
  \caption{Sequential paths of DP-SFG, Performance specifications, and transformer predicted sequence with device parameters}
  \label{fig:seq}
  \vspace{-0.4cm}
\end{figure}

At the end of Step~3, we obtain the final DP-SFG in Fig.~\ref{fig:DP-SFG_ex}(b) for the active inductor circuit. We will encode such a DP-SFG into sequential data that encapsulates the functionality of the circuit as well as the parameters of circuit components, including parasitics. This sequence representation is provided as an input to the transformer and is eventually used to size transistors in the circuit. We utilize the NetworkX Python package to process the final DP-SFG. This approach utilizes Johnson's algorithm ($O(V^2 \log V + VE)$ complexity) to identify all cycles and the depth-first search algorithm ($O(V + E)$ complexity) to find all forward paths, where $V$ represents the number of nodes and $E$ represents the number of edges in the graph.

For our running example, the sequence is derived from the paths 
marked in the DP-SFG in Fig.~\ref{fig:seq}.
Specifically, the blue marked path represents the forward path between input and output nodes, while the red and green marked paths denote the cycles. 
}

\subsection{Implementation of the transformer}

\noindent
\textbf{Transformer inputs.} The transformer model comprehends the interdependencies between device parameters and circuit performance metrics. We frame the paths by concatenating the nodes and edge weights from the DPSFG for every forward path and loop. The transformer takes in the list of paths extracted from the DP-SFG, each augmented with the desired set of specifications outlined by the black dotted rectangle in Fig.~\ref{fig:seq}. The transformer acts on the sequences and predicts the device parameters $g_m$, $g_{ds}$, $C_{ds}$, and $C_{gs}$ which will satisfy the desired specifications.


\noindent
\textbf{Overall transformer architecture.}
We implement the transformer in Python, leveraging the PyTorch library. The architecture of the transformer model closely resembles the one proposed by~\cite{vaswani_17}, with minor modifications. We use a 720-dimensional input embedding with 12 heads of parallel attention layers, while keeping the rest of the parameters unchanged.

\noindent
\textbf{Tokenization and byte-pair encoding}.
Tokenization is a crucial step for optimizing transformer efficacy. It breaks down a long sequence into smaller entities called tokens. Unlike in traditional NLP, where individual words and sub-words are treated as tokens, we use specific groups of individual characters such as the key device parameters $g_m, g_{ds}$, $C_{ds}$, $C_{gs}$, edge weights, and the names of the transistors, as tokens that convey necessary information about the circuit. 

For our problem, character-level tokenization (CLT), where each character is treated as a single token, is found to lead to long sequence lengths, i.e., a large number of tokens within a single sequence, resulting in computational inefficiency.
To overcome this problem, we employ the byte-pair encoding (BPE) approach~\cite{rico_16}.
This approach iteratively combines the most frequently occurring tokens (bytes) into a single token,
and dynamically adapts the vocabulary of the training data to capturing a common group of characters conveying essential information. By applying BPE, we achieve a 3.77$\times$ compression of the sequence lengths compared to the use of CLT, 
% \bluefn{Compared to what baseline? CLT? This is meaningless without stating the baseline. \textbf{FIXED}} 
leading to substantial savings in training time and memory requirements.

To demonstrate tokenization and for an actual DP-SFG path, we choose a partial path of a five-transistor operational transconductance amplifier (5T-OTA). The results of CLT and BPE are:
% character-level tokenization $(A)$ and BPE $(B)$.

\noindent
% \textit{Character-level encoding:} Character-level coding treats every single character as an independent individual token, as shown below:\\ 
CLT: \textcolor{white}{\sethlcolor{blue}\hl{3}\sethlcolor{red}\hl{2} \sethlcolor{orange}\hl{g}\sethlcolor{teal}\hl{m}\sethlcolor{violet}\hl{P}\sethlcolor{brown}\hl{1} \sethlcolor{red}\hl{-}\sethlcolor{pink}\hl{1}\sethlcolor{gray}\hl{6}
\sethlcolor{blue}\hl{1}\sethlcolor{black}\hl{/}\sethlcolor{cyan}\hl{(}\sethlcolor{orange}\hl{g}\sethlcolor{lightgray}\hl{d}\sethlcolor{blue}\hl{s}\sethlcolor{teal}\hl{M}\sethlcolor{purple}\hl{0}\sethlcolor{pink}\hl{+}\sethlcolor{magenta}\hl{s}\sethlcolor{teal}\hl{C}\sethlcolor{lightgray}\hl{d}\sethlcolor{brown}\hl{s}\sethlcolor{red}\hl{M}\sethlcolor{blue}\hl{0}\sethlcolor{pink}\hl{+}\sethlcolor{gray}\hl{s}\sethlcolor{black}\hl{C}\sethlcolor{lightgray}\hl{d}\sethlcolor{orange}\hl{s}\sethlcolor{violet}\hl{P}\sethlcolor{blue}\hl{1}\sethlcolor{pink}\hl{+}\sethlcolor{orange}\hl{g}\sethlcolor{teal}\hl{m}\sethlcolor{violet}\hl{P}\sethlcolor{blue}\hl{1}\sethlcolor{cyan}\hl{)}} 

% \textcolor{white}{\sethlcolor{blue}\hl{32} \sethlcolor{orange}\hl{gmP1} \sethlcolor{red}\hl{-16}
% \sethlcolor{blue}\hl{1/(}\sethlcolor{orange}\hl{gdsM0}\sethlcolor{pink}\hl{+}\sethlcolor{lightgray}\hl{s}\sethlcolor{teal}\hl{CdsM0}\sethlcolor{pink}\hl{+}\sethlcolor{lightgray}\hl{s}\sethlcolor{black}\hl{CdsP1}\sethlcolor{pink}\hl{+}\sethlcolor{orange}\hl{gmP1}\sethlcolor{cyan}\hl{)}} 


\noindent
BPE: 
% \textcolor{white}{\sethlcolor{blue}\hl{3}\sethlcolor{pink}\hl{2}\sethlcolor{pink} \hl{2}\sethlcolor{orange}\hl{.}\sethlcolor{violet}\hl{5}\sethlcolor{black}\hl{m}\sethlcolor{magenta}\hl{S}\sethlcolor{orange}\hl{P}\sethlcolor{cyan}\hl{1} \sethlcolor{red}\hl{-}\sethlcolor{cyan}\hl{1}\sethlcolor{brown}\hl{6}
% \sethlcolor{cyan}\hl{1}\sethlcolor{orange}\hl{/}\sethlcolor{red}\hl{(}\sethlcolor{violet}\hl{5}\sethlcolor{black}\hl{6}\sethlcolor{cyan}\hl{7}\sethlcolor{teal}\hl{u}\sethlcolor{magenta}\hl{S}\sethlcolor{violet}\hl{M}\sethlcolor{olive}\hl{0}\sethlcolor{pink}\hl{+}\sethlcolor{magenta}\hl{s}\sethlcolor{olive}\hl{0}\sethlcolor{orange}\hl{.}\sethlcolor{cyan}\hl{7}\sethlcolor{black}\hl{a}\sethlcolor{red}\hl{F}\sethlcolor{teal}\hl{M}\sethlcolor{olive}\hl{0}\sethlcolor{pink}\hl{+}\sethlcolor{magenta}\hl{s}\sethlcolor{violet}\hl{5}\sethlcolor{red}\hl{4}\sethlcolor{gray}\hl{1}\sethlcolor{black}\hl{a}\sethlcolor{red}\hl{F}\sethlcolor{darkgray}\hl{P}\sethlcolor{cyan}\hl{1}\sethlcolor{pink}\hl{+}\sethlcolor{pink}\hl{2}\sethlcolor{orange}\hl{.}\sethlcolor{violet}\hl{5}\sethlcolor{black}\hl{m}\sethlcolor{magenta}\hl{S}\sethlcolor{orange}\hl{P}\sethlcolor{cyan}\hl{1}\sethlcolor{purple}} 
\textcolor{white}{\sethlcolor{blue}\hl{32}\sethlcolor{pink} \hl{2}\sethlcolor{orange}\hl{.}\sethlcolor{violet}\hl{5}\sethlcolor{black}\hl{mS}\sethlcolor{orange}\hl{P1} \sethlcolor{red}\hl{-16}
\sethlcolor{blue}\hl{1/(}\sethlcolor{violet}\hl{5}\sethlcolor{black}\hl{6}\sethlcolor{cyan}\hl{7}\sethlcolor{orange}\hl{uS}\sethlcolor{violet}\hl{M0}\sethlcolor{pink}\hl{+}\sethlcolor{magenta}\hl{s}\sethlcolor{pink}\hl{0}\sethlcolor{violet}\hl{.}\sethlcolor{cyan}\hl{7}\sethlcolor{black}\hl{aF}\sethlcolor{teal}\hl{M0}\sethlcolor{pink}\hl{+}\sethlcolor{gray}\hl{s}\sethlcolor{violet}\hl{5}\sethlcolor{red}\hl{4}\sethlcolor{gray}\hl{1}\sethlcolor{black}\hl{aF}\sethlcolor{darkgray}\hl{P1}\sethlcolor{pink}\hl{+}\sethlcolor{pink}\hl{2}\sethlcolor{orange}\hl{.}\sethlcolor{violet}\hl{5}\sethlcolor{black}\hl{mS}\sethlcolor{orange}\hl{P1}} 

% \hfill--$(B)$

\noindent
The CLT sequence colors each neighboring character differently, denoting the tokenization of each unique character. However, CLT cannot easily comprehend the relation to device parameters or device names. The BPE approach overcomes this by iteratively combining frequently-occurring tokens. For example, BPE combines tokens such as
gmP1, gdsM0, and CdsM0 to represent $g_m$ of transistor P1, and $g_{ds}$ and $C_{ds}$ of transistor M0, respectively. Similarly, character-level tokens for the units of circuit parameters (e.g., ``mS'' or ``aF'') are combined into tokens of multiple characters. However, all purely numeric strings are left uncombined, as shown in the BPE sequence. For instance, for the value 2.5mS, which corresponds to the $g_m$ of transistor P1, the tokens representing 2.5 are maintained as character-level tokens, enabling the transformer to predict each digit relative to performance metrics independently, but the two character-level tokens for ``mS'' are combined into a single token. This restricted BPE representation thus enables the transformer model to better comprehend circuit relationships, as compared with CLT.





\noindent
\textbf{Loss Function.}
To enhance the learning of device parameter prediction from specified inputs, we utilize a weighted cross-entropy loss function for the transformer. Each token is treated as a separate class, with the loss function assigning greater importance to classes critical for accurate predictions. We focus on tokens representing numerical values of device characteristics (e.g., $g_m$, $g_{ds}$, $C_{ds}$, and $C_{gs}$), ensuring they receive more attention during training. This approach allows the transformer to grasp the significance of these characteristics and their impact on performance. Our experiments compared unweighted and weighted loss functions with varying weights, revealing that applying a 20\% increased weight on the numerical tokens yielded optimal performance.


\vspace{-0.02cm}
\subsection{Translating circuit parameters to device widths}
\label{sec:precomputedLUTs}

\noindent
After the trained transformer predicts the values of circuit parameters, they must be transformed to device widths. 
In this section, we describe a methodology 
% that uses the LUT from Section~\ref{sec:LUT} 
for this purpose.

\subsubsection{\textbf{Device characterization}}

In older technologies, the square-law model for MOS transistors could be used to perform a translation between circuit parameters and transistor widths, but square-law behavior is inadequate for capturing the complexities of modern MOS transistor models. In this work, we use a precomputed lookup table (LUT) that rapidly performs the mapping to device sizes while incorporating the complexities of advanced MOS models.

\begin{figure}[t]
% \vspace{-0.4cm}
\centering
\includegraphics[height=3cm, , bb=0 0 210 100]{fig/lut_fig_new.pdf}
% \vspace{-0.55cm}
\caption{LUT generation and characterization.}
\label{fig:lutgen}
\vspace{-5mm}
\end{figure}

The LUT is indexed by the $V_{gs}$, $V_{ds}$, and length $L$ of the transistor, and provides five outputs: the drain current ($I_d$), transconductance ($g_m$), drain-source conductance ($g_{ds}$), drain-source capacitance ($C_{ds}$), and gate-source capacitance ($C_{gs}$) all computed per unit transistor width. The entries of the LUT are computed by performing a nested DC sweep simulation across the input indices for the MOSFET with a specific reference width, $W_{ref}$, as shown in Fig.~\ref{fig:lutgen}, and for each input combination, the five outputs are recorded. Since the five quantities all vary linearly with the width of the transistor, we store their corresponding values per unit width. 
% Empirically, we see that the impact of $V_{sb}$ is small enough
% \bluefn{For my info: over what range of $V_{sb}$ have you done this, and why is that sufficient? \textbf{I chose the range between (0 - $V_{dd}/3$), this is due to having around three mosfets b/n $V_{dd}$ and ground in our circuits. For example, in the 5T OTA the $V_{sb}$ for the DP is non-zero since the source is sitting on top of the tail mosfet, and the voltage (Vds) across the tail mos is usually $<$vdd/3}} SSS_NOTE
% that it can be neglected, and therefore we set $V_{sb} = 0$ in the sweeps used to create the LUT. 
The LUT stores the vector-valued function
% \vspace{-2mm}
\begin{align}
[I_d \;\; g_m \;\; g_{ds} \;\; C_{ds} \;\; C_{gs}] = f(V_{gs}, V_{ds})
\end{align}

We have constructed a lookup table for a 65nm technology with a reference transistor width of 700nm, with $V_{gs}$ and $V_{ds}$ values ranging from 0--1.2V with a 60mV step. Given the relatively coarse granularity of data points in the LUT, we have implemented cubic spline interpolation to enhance accuracy at intermediate values. These LUT granularity, together with interpolation, ensures that it provides accurate predictions, and yet has a reasonable size.  

Our methodology uses this LUT, together with the $g_m/I_d$ methodology~\cite{silviera_96,jespers_17}, to translate circuit parameters predicted by the transformer to transistor widths. The cornerstone of this methodology relies on the inherent width independence of the ratio $g_m/I_d$ to estimate the unknown device width: this makes it feasible to use an LUT characterized for a reference width $W_{ref}$. 





\ignore{
The device characterization stage involves studying how transistor small signal parameters vary with the device size and external voltages applied between the terminals: gate-to-source voltage ($V_{gs}$), drain-to-source voltage ($V_{ds}$), and source-to-bulk voltage ($V_{sb}$). This analysis explores the relationship between the DoFs and output parameters such as drain current $I_d$, $g_m$, $g_{ds}$, and $C_{ds}$. 
\begin{align}
I_d, g_m, g_{ds}, C_{ds}, \dots & \Rightarrow f(V_{gs}, V_{ds}, V_{sb}, L, W)
\end{align}
\begin{equation}
I_d  = \mu_nC_{ox}\frac{W}{L} \times f(V_{gs},V_{ds},V_{sb})
\label{eq:width_indp}
\end{equation}
From equation \eqref{eq:width_indp}, it is evident that the drain current ($Id$) exhibits a linearly proportional relationship with the width (W). This relationship holds for the quadratic behavior of the $I_d$ both in the linear and saturation regions. It is also generally acceptable for other parameters, i.e., $g_m$, $g_{ds}$, $C_{ds}$. Although there may be slight deviations in practical scenarios, these can usually be overlooked for simplification~\cite{jespers_17}.

Our method primarily focuses on $V_{gs}$ and $V_{ds}$, and $L$ with the assumption of a fixed $V_{sb}$ value of zero; this assumption gives a reasonably accurate approximation, reducing analysis to a model with three degrees of freedom: $f(V_{ds}, V_{gs}, L)$. We constructed a lookup table with a reference width of $700nm$ based on $V_{gs}$ and $V_{ds}$ values ranging from $0$ to $1.2V$ with a $60mV$ step, and five lengths starting from $100nm$ to $180nm$ with $20nm$ step. These chosen values ensure that the LUT remains reasonably sized for efficient lookup operations.  

Given the relatively coarse granularity of our data points in the LUT, we implemented cubic spline interpolation to enhance accuracy when retrieving values from the lookup table.
}

\subsubsection{\textbf{Width estimation}} 

The width estimation process uses the recorded LUT and transformer-predicted MOSFET parameters to compute the optimal width. The pseudocode for the algorithm employed is presented in Algorithm~\ref{algo:width_estimation}. 
After initialization on line~\ref{algo2:init}, the input is converted to the desired $g_m/I_d$ ratio.  Lines~\ref{algo2:while_begin}--\ref{algo2:while_end} iterate over the LUT to find the $W$ that matches the transformer-supplied parameters. Specifically, line~\ref{algo2:gmId} finds the value of $V_{gs}$ at which the $g_m/I_d$ ratio is met. For this value, lines~\ref{algo2:wcalc_begin}--\ref{algo2:wcalc_end} determine candidate values of $W$, $w_1, \cdots, w_5$, by ratioing $I_d^{in}$, $g_m^p$, $g_{ds}^p$, $C_{ds}^p$, and $C_{gs}^p$, respectively, with the corresponding LUT outputs. We iterate over $V_{ds}$ until $w_1, \cdots, w_5$ are as close as possible. Line~\ref{algo2:while_end} takes a step in this direction using the empirically chosen factor $\alpha = 10^{-4}$. The iterations continue until the candidate width values converge.



\ignore{
\\
\noindent\textbf{Step 1: Operating points calculation} (lines 5 to 11), using the parameters obtained from the transformer, i.e., $g_m^p$ and $I_d$ the $g_m/I_d$ operating point is calculated. Then in lines 7 to 11, the $g_m$ and $I_d$ values are read from the table with initial $V_{ds}$ and $L$, as a function of $V_{gs}$. $V_{gs}$ value that satisfies the calculated $g_m/I_d$ operating point is then obtained from the ratio of the $V_{gs}$ dependent $g_m$ and $I_d$ functions.  

\noindent\textbf{Step 2: Reading parameters as a function of $V_{ds}$} (line 12), by treating these parameters as a function of $V_{ds}$ we eliminate dependency on the initial guess $V_{ds}$ value, which unlike the $g_m/I_d$, significantly affect the other four parameters.

\noindent\textbf{Step 3: Normalization of parameters} (line 13) Based on the width proportionality property stated in equation \eqref{eq:width_indp} the parameters are made to be width independent by normalizing them with $W_{ref}$.

\noindent\textbf{Step 4: Calculating width and the total cost} (lines 14 to 17) next, the widths corresponding to each transformer-predicted parameter are calculated. These widths are used to determine the total cost, defined as the total deviation of the computed widths as a function of $V_{ds}$. Our objective is to find a width value that ensures the predicted parameters. Therefore, the minimum point of the cost function, where most widths align, is taken as the minimum cost.

\noindent\textbf{Step 5: Iterate until optimal accuracy is achieved}, the difference, $\Delta$, between the minimum cost obtained from the above step and the previous cost. The iteration continues until the cost no longer improves, which is controlled by a minimum value, $\epsilon$. Alongside this process, the $V_{ds}$ value is also updated with the direction determined by the sign of $\Delta$ and the magnitude by another variable, $\alpha$. Here $\alpha$ and $\epsilon$ are user-defined parameters; their values are obtained by manually tuning them using the training dataset as a reference to assess the convergence behavior. In the experiment, we observe that the loop converges rapidly for a wide range of $\alpha$ and $\epsilon$ values. Empirically, we find the choice $\alpha = 0.01$, $\epsilon = 1$x$10^{-8}$ to be effective.

Finally, once the loop is done the $V_{ds}$ value that minimizes the cost function is identified as the optimal drain-to-source voltage, $V_{ds}^*$, of the MOSFET resulting in the desired parameters. The optimal width $W^*$ can then be obtained from one of the width functions defined in lines 14 to 17 evaluated at the optimum $V_{ds}$ value.
}

    \begin{algorithm}[t]
    \footnotesize
        \caption{Width Estimation}
        \label{algo:width_estimation}
    \begin{algorithmic}[1]
    \State \textbf{Inputs:} Transformer-predicted $g_{m}^p$, $g_{ds}^p$, $C_{ds}^p$, $C_{gs}^p$ and current $I_d^{in}$ for a MOSFET; LUT for the device type (PMOS/NMOS); 
    % reference width, $W_{ref}$, used for recording the LUT; 
    tolerances $\alpha$ and $\epsilon$ 
    \State \textbf{Outputs:} Optimal width, $W$, of the MOSFET
    % DC operating points and width of the MOSFET, namely $V_{gs}^*$, $V_{ds}^*$ and $W^*$
    \State $V_{ds,curr} \gets V_{dd}/2$ , $cost_{curr} \gets \infty$, $\Delta \gets \infty$  \label{algo2:init}
    
    \State $g_m\_I_d  \gets g_{m}^p/I_d$ \textit{// Compute the $g_m\_I_d$ operating point}
    
    \While{$|\Delta| > \epsilon$} \textit{// until current cost $\approx$ previous cost}
        \State mincost$_{prev} \gets$ mincost$_{curr}$, $V_{ds,prev} \gets V_{ds,curr}$ \label{algo2:while_begin}
        % \Statex \hspace{1.3em} \textit{// Getting $V_{gs}$ from the LUT}
        \State Find LUT entry $[I_d \;\; g_m \;\; g_{ds} \;\; C_{ds} \;\; C_{gs}] = f(V_{gs}, V_{ds})$ \label{algo2:gmId}
        \Statex \hspace*{10mm} at which $g_m/I_d = g_m\_I_d$. Report $V_{gs}^p$ for this entry.

        \State At $V_{gs}^p$, the LUT for $[I_d \;\; g_m \;\; g_{ds} \;\; C_{ds} \;\; C_{gs}]$ is $f(V_{ds})$.
        \State \textit{// Find $w_1 \cdots w_5$ as functions of $V_{ds}$}
        \State $w_1$$\gets$$g_{m}^p/g_m$, $w_2$$\gets$$g_{ds}^p/g_{ds}$, $w_3$$\gets$$C_{ds}^p/c_{ds}$, 
        \Statex \hspace*{10mm} $w_4$$\gets$$C_{gs}^p/c_{gs}$, $w_5$$\gets$$I_d^{in}/I_d$
        

        \Statex \hspace{1.3em} \textit{// Find $V_{ds}$ at which $w_i$s are closest}
        \State cost$(V_{ds}) \gets \sum_{n=1}^{3} \sum_{m=n+1}^{4} |w_n - w_m|$   \label{algo2:wcalc_begin}
        \State mincost$_{curr} = \min_{V_{ds}}$ (cost)
        \State $\Delta \gets$ mincost$_{prev}$ $-$ mincost$_{curr}$ \label{algo2:wcalc_end}
        \Statex \hspace{1.3em} \textit{// Updating the initial guess $V_{ds}$ value}
        \State $V_{ds,curr} \gets V_{ds,curr} + \mbox{sgn}(\Delta).\alpha.V_{ds, prev}$ \label{algo2:while_end}
        
        % \State $g_m = f(V_{gs}) \gets \text{lookup}\_g_{m}(V_{ds,curr},L) $
        % \State $I_d = f(V_{gs}) \gets \text{lookup}\_I_{d}(V_{ds,curr},L) $
        % \State $V_{gs} \gets$ \text{get\ $V_{gs}$ where $g_{m}/I_{d}$ == $g_m\_I_d$} \redHL{Find Vgs corresponding to gm\_Id; lines 9 and 10 provide the function, line 5 provides the value to be looked up}
        % \Statex \hspace{1.3em} \textit{// Reading the four parameters as a function of $V_{ds}$}
        % \State $I_d,g_m,g_{ds},C_{ds} =\ f(V_{ds})\ \gets\ \text{lookup}(V_{gs},L)$   
        
        % \Statex \hspace{1.3em} \textit{// Normalizing the parameters with $W_{ref}$}
        % \State $I_d\_w,\ g_m\_w,\ g_{ds}\_w, c_{ds}\_w\gets f(V_{ds})/W_{ref}$
        % \Statex \hspace{1.3em} \textit{// Width calculation using the normalized functions}
        % \Statex \hspace{1.3em} \textit{// Each width is a function of $V_{ds}$}
        % \State $w_1 \gets g_{m}^p/g_m\_w$
        % \State $w_2 \gets g_{ds}^p/g_{ds}\_w$
        % \State $w_3 \gets C_{ds}^p/c_{ds}\_w$
        % \State $w_4 \gets I_d/I_d\_w$

        % \Statex \hspace{1.3em} \textit{// Determining cost function from the widths}
        % \State $cost\_func \redHL{(Vds)} \gets \sum_{n=1}^{3} \sum_{m=n+1}^{4} |w_n - w_m|$   
        % \State $min\_cost_{curr} = min_{\redHL{all Vds}} (cost\_func)$
        % \State $\Delta \gets min\_cost_{prev}-min\_cost_{curr}$
        % \Statex \hspace{1.3em} \textit{// Updating the initial guess $V_{ds}$ value}
        % \State $V_{ds,curr} \gets V_{ds,curr} + sgn(\Delta).\alpha.V_{ds, prev}$
    \EndWhile
    
    % \State $V_{ds}^* \gets \arg\min(cost\_func)$
    \State $W \gets\ w_1(V_{ds})$
    % \State $V_{gs}^* \gets V_{gs}$    

    \end{algorithmic}
    \end{algorithm}


    

\subsection{SPICE verification and margin allocation}

\noindent
Finally, we perform just one SPICE simulation to verify compliance with all specifications. If any specification deviates from the requirements, the model modifies the specifications and repeats the inference step to obtain a new set of device sizes. For example, if the gain of the sized OTA is 10\% below the desired value, the model iteratively tightens the specifications to accommodate this 10\% difference in the gain requirement until all specifications are satisfied.
% \bluefn{Do you also do this if the performance is much better than the specification? You would reduce the power/area by using smaller device sizes.  Also: why don't you report power/area anywhere? \textbf{No, I don't do this if the performance is much better than the specifications. However, doing this is very much possible. But, its not guaranteed that reducing size will bring down the performance closer to the specifications.}}SSS_NOTE
\section{Experimental Setup and Results}
\label{sec:results}
\vspace{-2mm}
\noindent
% \blueHL{\sout{In this section, we delve into our experimental setup, covering the process of data generation and pre-processing. Additionally, we present details on model training and validation procedures. Finally, we present the efficacy of our sizing assistant in predicting device sizes with some unseen circuit performance specifications.}}
% \bluefn{Can delete this without much loss of continuity, if we need to save space. \textbf{OK}} SSS_NOTE

\subsection{Data generation and preprocessing}

\noindent
To demonstrate the efficacy of our framework, we employ three distinct OTA topologies: five-transistor OTA (5T-OTA), current-mirror OTA (CM-OTA), and two-stage OTA (2S-OTA), each implemented using a 65nm technology node. In Fig.~\ref{fig:schemas}, we show the OTA schematics along with matching constraints under consideration. For clarity in our demonstration, we focus on three performance metrics: gain, 3dB-bandwidth (BW), and unity-gain frequency (UGF), and aim to meet the given performance specifications.
 
Table~\ref{tab:dataset} shows the range of different specifications for the OTAs considered for our training set. We assume that the length of all the devices in a circuit is set to 180nm with a load capacitor $C_L$ of 500fF for all the topologies. To ensure reliable model analysis, we start with precise data generation for each OTA topology using OCEAN scripting. This involves the following steps:
\begin{itemize}
    \item Generating multiple designs with varying transistor sizes by nested sweeps of widths ranging from 0.7$\mu$m to 50$\mu$m.
    \item Enforcing matching constraints for active load current mirror (CM), and differential pair (DP).
    \item Sweeping the DC voltage to determine the input common-mode range (ICMR) of the designs.
    \item Ensuring that the CMs operate in the strong inversion region while the DPs function in the weak inversion region.
    % \bluefn{(1)~Strange choice of words -- what is ``irrelevant''? It either meets specs or it does not. How can it be irrelevant? \textbf{Can I use ``invalid'' ? Cause, I am using ICMR just for filtering out the OTAs which can practically function for a valid input common mode voltage range.} (2)~If you use ICMR as a spec, why isn't it ever reported in your results? \textbf{I am not using ICMR as a SPEC}} SSS_NOTE
    \item Filtering out designs that falls out of the predefined specification range for the dataset outlined in Table~\ref{tab:dataset}.
    \item Capturing the device parameters -- specifically, $g_m, g_{ds}$, $C_{ds}$, and $C_{gs}$ -- for the final legal designs.
\end{itemize}

% \bluefn{What does ``inadequate'' mean? Elaborate. \textbf{I think ``functional'' would be better. The circuits which have a valid ICMR,  and giving out practically useful gain, BW, UGF, are choosen}} SSS_NOTE

\begin{figure}[t]
    \centering
    \subfloat[]{
        \includegraphics[width=19.5mm, bb = 0 0 100 110]{fig/5tota.pdf}
    }
    \hspace{-2mm} % Adjust spacing between subfigures
    \subfloat[]{
        \includegraphics[width=30.5mm, bb = 0 0 200 110]{fig/cmota.pdf}
    }
    \hspace{-1mm} % Adjust spacing between subfigures
    \subfloat[]{
        \includegraphics[width=29mm, bb = 0 0 160 110]{fig/2sota.pdf}
    }
    
    \caption{Schematic of (a) 5T-OTA, (b) CM-OTA, and (c) 2S-OTA.}
    \label{fig:schemas}
    \vspace{-0.2cm}
\end{figure}

Next, we focus on generating appropriate DP-SFG paths for each circuit topology. Table~\ref{tab:dataset} shows the number of sequential paths for each topology. The DP-SFGs are small and the cost of path enumeration is small; for more complex examples, if the number of paths grows large, it is possible to devise other string representations of the DP-SFG.
Finally, in the preprocessing stage, we generate two sets of sequential data, one each for the encoder and the decoder.
% \bluefn{\textbf{***I think at this stage SEQUENTIAL DATA makes more sense***} Is the use of ``sequential paths'' confusing? You are talking about a sequence of tokens that feeds the transformer. The paths are not tokens. Maybe use some other term instead of ``sequential paths''?}
\begin{itemize}
    \item The sequential data at the encoder comprises DP-SFG paths that maintain consistency across all designs within a specific topology. It also includes performance metrics for each design, encompassing gain, BW, and UGF parameters, associated with each unique set of transistor sizes.
    \item The sequential data at the decoder covers the same DP-SFG paths, but with device parameters replaced by values obtained during data generation. These values are unique to each design, aligning with the performance metrics in the encoder sequence.
\end{itemize}

\begin{table}[t]
    \caption{Dataset information.}
    \centering
    % \begin{tabular}{|>{\centering\arraybackslash}m{1.1cm}|>{\centering\arraybackslash}m{1cm}|>{\centering\arraybackslash}m{1.3cm}|>{\centering\arraybackslash}m{1.2cm}|>{\centering\arraybackslash}m{1cm}|>{\centering\arraybackslash}m{0.7cm}|}
    \resizebox{1\linewidth}{!}{\begin{tabular}{|l|c|c|c|c|c|}
        \hline
        \textbf{Topology} & \makecell{\textbf{Gain}\\\textbf{(dB)} \\\textit{min-max}} & \makecell{\textbf{3dB bandwidth}\\\textbf{(MHz)} \\\textit{min-max}} & \makecell{\textbf{UGF} \\\textbf{(MHz)} \\\textit{min-max}}& \makecell{\textbf{\#forward} \\\textbf{paths}} & \textbf{\#cycles} \\
        \hline
        5T-OTA & 18 -- 23 &  7 -- 54 & 80 -- 871  &  9 & 4  \\
        \hline
        CM-OTA & 19 -- 25 & 17.5 -- 86 & 57 -- 1185 & 26 & 5  \\
        \hline
        2S-OTA & 28 -- 54 & 0.01 -- 0.32 & 1.8 -- 370  &  2 & 11 \\
        \hline
    \end{tabular}}
    \label{tab:dataset}
    \vspace{-5mm}
\end{table}


We train a single transformer model that works across all three OTA topologies. By considering all performance criteria and all DP-SFG paths, we convey complete information about each circuit to the transformer. Our dataset comprises 17,000 designs for 5T-OTA, 25,000 designs for CM-OTA, and 8,000 designs for 2S-OTA, each with a different set of transistor sizes. This diverse dataset trains the model across multiple design specification requirements.


% \begin{table}[t]
%     \vspace{-0.2cm}
%     \caption{Dataset Information}
%     \centering
%     \begin{tabular}{|>{\centering\arraybackslash}p{1.1cm}|>{\centering\arraybackslash}p{0.9cm}|>{\centering\arraybackslash}p{0.7cm}|>{\centering\arraybackslash}p{0.5cm}|>{\centering\arraybackslash}p{0.5cm}|>{\centering\arraybackslash}p{0.5cm}|>{\centering\arraybackslash}p{0.5cm}|>{\centering\arraybackslash}p{0.5cm}|>{\centering\arraybackslash}p{0.5cm}|}
%         \hline
%         \multirow{2}{*}{\textbf{Topology}} & \multirow{2}{*}{\textbf{Forward}} & \multirow{2}{*}{\textbf{Cycles}} & \multicolumn{2}{c|}{\makecell{\textbf{Gain}\\\textbf{(dB)}}} & \multicolumn{2}{c|}{\makecell{\textbf{Bandwidth}\\\textbf{(MHz)}}} & \multicolumn{2}{c|}{\makecell{\textbf{UGF} \\\textbf{(MHz)}}}\\
%         \cline{4-9}
%          & \textbf{paths} &  & \textbf{Min} & \textbf{Max} & \textbf{Min} & \textbf{Max} & \textbf{Min} & \textbf{Max} \\
%         \hline
%         5T-OTA & 9 & 4 & 18 & 29 & 7 & 54 & 80 & 871\\
%         \hline
%         CM-OTA & 26 & 5 & 20 & 32 & 59 & 86 & 57 & 1185\\
%         \hline
%         2S-OTA & 2 & 11 & 30 & 52 & 13 & 32 & 18 & 370\\
%         \hline
%     \end{tabular}
%     \label{tab:DP-SFG}
% \end{table}

\subsection{Training and validation}

% \redfn{CK: The number of paths in a more complex circuit could be very large? How do we propose to handle it? \textbf{Since, the inferencing is quite fast, the overall process will still be fast. But, yes the size of the dataset is proportional to number of paths in DPSFG. {\em Added text below Table 1. Please check blueHL.}YES MAKES SENSE}}

\noindent
For our experiments, we employ an Nvidia L40S GPU equipped with 45GB of memory. The dataset is split into an 80:20 ratio for training and validation across each OTA topology. We train a single model using datasets from all three topologies for 40 epochs, employing an adaptive learning rate strategy with the Adam optimizer, beginning with an initial learning rate of $10^{-4}$. Subsequently, our framework is validated against unseen performance specifications across all three OTA topologies. For a given topology and performance specifications, the validation phase rapidly predicts a sequence of tokens corresponding to circuit parameters.The transformer takes in the encoder sequences list and predicts output sequences containing the device parameter values that satisfy the specifications. This is followed by the LUT-based estimator that translates the predicted device parameters to transistor widths. 

% In the subsequent subsection, we comprehensively analyze each aspect of our framework's performance.

\begin{figure}[t]
\vspace{-4mm}
\centering
\subfloat[]{
  \includegraphics[width=0.49\linewidth, bb = 0 0 1100 1000]{fig/scatter.pdf}
}
\hspace{-0.4cm}
\subfloat[]{
  \includegraphics[width=0.49\linewidth, bb = 0 0 1100 1000]{fig/gSCATTER.pdf}
}
% \vspace{-2mm}
\caption{For 5T-OTA: Scatter plots showing comparisons between predicted and simulation-based device parameters (a) $g_m$ and (b) $g_{ds}$.}
\label{fig:plots}
\vspace{-5mm}
\end{figure}

\subsection{Performance of the framework}

\noindent
We conduct comprehensive performance evaluations to assess the effectiveness of the transformer model and LUT-based width estimator for each OTA topology. Our method sizes 100 unique designs per topology, each with distinct performance specifications not included in the training set. For each specification, the transformer model predicts the key device parameters, which are then converted to transistor sizes using the LUT-based method. The performance of the final design is validated through Spectre simulation of the sized OTA circuit for each topology. Additionally, we ensure the optimized devices operate in the desired region of operation. 


\begin{table}[b]
    \vspace{-4mm}
    \centering
    % First table
    \begin{minipage}{\linewidth}
        \centering
        \caption{\centering Correlation coefficient of device parameters between validation data and model outputs for the 5T-OTA.}
        \resizebox{1\linewidth}{!}{
            \begin{tabular}{|>{\centering\arraybackslash}p{0.9cm}|>{\centering\arraybackslash}p{2.3cm}|c|c|c|c|}
                \hline
                \multirow{2}{*}{\makecell{\textbf{MOS} \\ \textbf{devices}}} & \multirow{2}{*}{\makecell{\textbf{Transistor} \\ \textbf{information}}} & \multicolumn{4}{c|}{\makecell{\textbf{Correlation coefficient}}} \\ \cline{3-6}
                & & \textbf{$g_m$} & \textbf{$g_{ds}$} & \textbf{$C_{ds}$} & \textbf{$C_{gs}$} \\
                \hline
                M1/M2 & Active load & 0.982 & 0.993 & 0.962 & 0.964 \\ \cline{1-6} 
                M3/M4 & DP & 0.999 & 0.991 & 0.997 & 0.998 \\ \cline{1-6} 
                M5 & Tail MOS & 0.999 & 0.997 & 0.997 & 0.997 \\ \cline{1-6} 
                \hline
            \end{tabular}
        }
        \label{tab:5t_corr}
    \end{minipage}
    
    \vspace{1mm} % Space between tables

    % Second table
    \begin{minipage}{\linewidth}
        \centering
        \caption{\centering Comparison of optimized design performance with target specifications for the 5T-OTA}
        \resizebox{1\linewidth}{!}{
            \begin{tabular}{|c|c|c|c|c|c|}
                \hline
                \multicolumn{2}{|c|}{\textbf{Gain (dB)}} & \multicolumn{2}{c|}{\makecell{\textbf{UGF (MHz)}}} & \multicolumn{2}{c|}{\makecell{\textbf{3dB bandwidth (MHz)}}} \\ \cline{1-6}
                \makecell{\textbf{Target}} & \makecell{\textbf{Optimized}} & \makecell{\textbf{Target}} & \makecell{\textbf{Optimized}} & \makecell{\textbf{Target}} & \makecell{\textbf{Optimized}} \\
                \hline
                20.13 & 20.6 & 118.78 & 144.64 & 11.38 & 13.33 \\ \cline{1-6} 
                21.23 & 21.37 & 181.25 & 185.38 & 15.31 & 15.49 \\ \cline{1-6} 
                22.78 & 22.79 & 281.75 & 288.54 & 20.18 & 20.48 \\ \cline{1-6} 
                \hline
            \end{tabular}
        }
        \label{tab:5t_specs}
    \end{minipage}

    % \vspace{-2mm}
\end{table}

\noindent
\textbf{5T-OTA}
The 5T-OTA topology includes a matched active current-mirror load (M1/M2), a differential pair (M3/M4), and a tail transistor (M5), all requiring precise sizing to meet performance targets. We assess the prediction accuracy of the transformer by correlating predicted device parameters with SPICE-based validation results. Fig.~\ref{fig:plots} shows a strong correlation between predicted \(g_{m}\) and \(g_{ds}\) values and their SPICE counterparts along the 45° line,
% , where we use single variable for matched transistors, 
and Table~\ref{tab:5t_corr} summarizes the correlation coefficients of all the parameters, highlighting model accuracy. We show the results of applying the transformer model for three sets of unseen target specifications in Table~\ref{tab:5t_specs}: the optimized circuit can be seen to meet all requirements.



\noindent
\textbf{CM-OTA}
The CM-OTA topology incorporates a differential input stage, succeeded by three current mirror loads. A total of nine devices require sizing in this configuration. The correlation coefficient between the device parameters predicted by the transformer model and the SPICE-based validation data are shown in Table~\ref{tab:cmota_corr} and display high accuracy. Finally, Table~\ref{tab:cmota_specs} delineates the target specifications for three randomly selected designs from the validation set. As in the case of the 5T-OTA, the output of the transformer yields optimized circuits that meet all performance requirements. 

\begin{table}[t]
    \centering
    % \vspace{-2mm}
    % First table
    \begin{minipage}{\linewidth}
        \centering
        \caption{\centering Correlation coefficient of device parameters between validation data and model outputs for the CM-OTA.}
        \resizebox{1\linewidth}{!}{
            \begin{tabular}{|>{\centering\arraybackslash}p{0.9cm}|>{\centering\arraybackslash}p{2.3cm}|c|c|c|c|}
                \hline
                \multirow{2}{*}{\makecell{\textbf{MOS} \\ \textbf{devices}}} & \multirow{2}{*}{\makecell{\textbf{Transistor} \\ \textbf{information}}} & \multicolumn{4}{c|}{\makecell{\textbf{Correlation coefficient}}} \\ \cline{3-6}
                & & \textbf{$g_m$} & \textbf{$g_{ds}$} & \textbf{$C_{ds}$} & \textbf{$C_{gs}$} \\
                \hline
                M1/M2 & Matched CM load & 0.811 & 0.838 & 0.871 & 0.875 \\ \cline{1-6}
                M3/M4 & DP & 0.798 & 0.683 & 0.878 & 0.883 \\ \cline{1-6} 
                M5 & Tail MOS & 0.817 & 0.867 & 0.601 & 0.760 \\ \cline{1-6} 
                M6/M7 & Matched CM load & 0.893 & 0.803 & 0.881 & 0.895 \\ \cline{1-6} 
                M8/M9 & Matched CM load & 0.912 & 0.914 & 0.891 & 0.892 \\ \cline{1-6} 
                \hline
            \end{tabular}
        }
        \label{tab:cmota_corr}
    \end{minipage}
    
    \vspace{0.5mm} % Space between tables

    % Second table
    \begin{minipage}{\linewidth}
        \centering
        \caption{\centering Comparison of optimized design performance with target specifications for the CM-OTA}
        \resizebox{1\linewidth}{!}{
            \begin{tabular}{|c|c|c|c|c|c|}
                \hline
                \multicolumn{2}{|c|}{\textbf{Gain (dB)}} & \multicolumn{2}{c|}{\makecell{\textbf{UGF (MHz)}}} & \multicolumn{2}{c|}{\makecell{\textbf{3dB bandwidth (MHz)}}} \\ \cline{1-6}
                \makecell{\textbf{Target}} & \makecell{\textbf{Optimized}} & \makecell{\textbf{Target}} & \makecell{\textbf{Optimized}} & \makecell{\textbf{Target}} & \makecell{\textbf{Optimized}} \\
                \hline
                20.83 & 21.99 & 345.9 & 475.74 & 30.84 & 37.65 \\ \cline{1-6} 
                21.55 & 23.25 & 247.98 & 408.11 & 20.15 & 27.48 \\ \cline{1-6} 
                23.8 & 24.3 & 1033.77 & 1478.5 & 71.47 & 104.24 \\ \cline{1-6} 
                \hline
            \end{tabular}
        }
        \label{tab:cmota_specs}
    \end{minipage}

    \vspace{-4mm}
\end{table}



\noindent
\textbf{2S-OTA}
The 2S-OTA topology includes a 5T-OTA in the first stage, followed by a common source amplifier comprising seven devices. Table~\ref{tab:2sota_corr} provides a summary of the correlation coefficient between the device parameters predicted by the transformer model and those generated by SPICE, thereby affirming the accuracy of the model. Furthermore, Table~\ref{tab:2sota_specs} presents the target specifications for three randomly selected designs from the validation set. Again, the transformer delivers optimized circuits that meet all specifications.

\begin{table}[b]
    \centering
    \vspace{-4mm}
    % First table
    \hspace*{-0.03\linewidth}
    \begin{minipage}{1.0\linewidth}
        \centering
        \caption{\centering Correlation coefficient of device parameters between validation data and model outputs for the 2S-OTA.}
        \resizebox{1\linewidth}{!}{
            \begin{tabular}{|>{\centering\arraybackslash}p{0.9cm}|>{\centering\arraybackslash}p{2.3cm}|c|c|c|c|}
                \hline
                \multirow{2}{*}{\makecell{\textbf{MOS} \\ \textbf{devices}}} & \multirow{2}{*}{\makecell{\textbf{Transistor} \\ \textbf{information}}} & \multicolumn{4}{c|}{\makecell{\textbf{Correlation coefficient}}} \\ \cline{3-6}
                & & \textbf{$g_m$} & \textbf{$g_{ds}$} & \textbf{$C_{ds}$} & \textbf{$C_{gs}$} \\
                \hline
                M1/M2 & 1\textsuperscript{st} stage active load & 0.942 & 0.936 & 0.876 & 0.879 \\ \cline{1-6}
                M3/M4 & 1\textsuperscript{st} stage DP & 0.988 & 0.945 & 0.913 & 0.915 \\ \cline{1-6} 
                M5 & 1\textsuperscript{st} stage tail MOS & 0.928 & 0.989 & 0.918 & 0.922 \\ \cline{1-6} 
                M6 & 2\textsuperscript{nd} stage tail MOS & 0.856 & 0.881 & 0.843 & 0.798 \\ \cline{1-6} 
                M7 & 2\textsuperscript{nd} stage CS & 0.892 & 0.887 & 0.785 & 0.880 \\ 
                \hline
            \end{tabular}
        }
        \label{tab:2sota_corr}
    \end{minipage}
    
    % \vspace{0.5mm} % Space between tables

    % Second table
    \begin{minipage}{\linewidth}
        \centering
        \caption{\centering Comparison of optimized design performance with target specifications for the 2S-OTA}
        \resizebox{1\linewidth}{!}{
            \begin{tabular}{|c|c|c|c|c|c|}
                \hline
                 \multicolumn{2}{|c|}{\textbf{Gain (dB)}} & \multicolumn{2}{c|}{\makecell{\textbf{UGF (MHz)}}} & \multicolumn{2}{c|}{\makecell{\textbf{3dB bandwidth (kHz)}}} \\ \cline{1-6}
                \makecell{\textbf{Target}} & \makecell{\textbf{Optimized}} & \makecell{\textbf{Target}} & \makecell{\textbf{Optimized}} & \makecell{\textbf{Target}} & \makecell{\textbf{Optimized}} \\
                \hline
                43.6 & 45.61 & 13.33 & 13.4 & 90 & 140 \\ \cline{1-6} 
                47.17 & 47.93 & 11.09 & 11.77 & 80 & 90 \\ \cline{1-6} 
                55.19 & 46.04 & 9.42 & 10.11 & 60 & 91 \\ 
                \hline
            \end{tabular}
        }
        \label{tab:2sota_specs}
    \end{minipage}

    \vspace{-2mm}
\end{table}



From the correlation coefficient analysis, we observe that in some cases the coefficients can be relatively lower (e.g., $<$0.8). These cases correspond to scenarios where the corresponding parameter does not impact the performance metrics significantly, while the other parameter has a more dominant influence. This behavior is attributed to the attention mechanism of the transformer which weights the importance of different parameters based on their level of influence on the performance specifications. As a result, the contributions of less impactful parameters may be overshadowed, leading to a lower correlation coefficient in those specific cases. 

% \bluefn{When I read this entire results section, all I see in the text is ``5T-OTA'': you don't even mention the other OTAs. If the reviewer is in a hurry, s/he will think you really have only done 5T-OTAs. Need to write this better to bring out the other two OTA types. Your paper is already weak because of the simplicity of the circuits (compare with any of the other papers on OTA sizing, which show much more complex circuits). Don't shoot yourself in the foot even further.}









\noindent
%\redHL{~{\em 2) LUT-based transistor width estimation.} For each OTA topology,\bluefn{List them by name so that the reader does not just see ``5T OTA'' in the results section.} we use our approach to determine the transistor sizes for 100 distinct performance specifications that are unseen in the training set. For each set of performance specifications, the transformer model predicts the circuit parameters, which are translated to transistor sizes using the LUT-based method. We report the performance of the design based on a Spectre simulation of the sized OTA circuit. Table~\ref{tab:specs} shows the list of target specs and obtained specs\bluefn{No! You don't \underline{obtain} a spec. See earlier comment. \textbf{FIXED}} for three designs from the validation set for each topology. \textbf{IN MY OPINION, WE DON'T NEED THIS RED PART ANYMORE}}



\begin{table}[t]
    % \vspace{-0.2cm}
    \caption{Runtime analysis of training and inferencing stages.} 
    % \blueHL{This table is poorly explained. You never say anywhere that you run 100 designs. I have changed 95 to 95/100, etc., to make this more obvious in the paper, but you also need to say this in the text.  Please read the paper adversarially, from the point of view of the reviewer. Right now, there are many items that will provoke instant rejection from the reviewer, and you have not bothered to try and address these. I've caught what I could, but I am sure there are other issues because I am starting from almost zero. \textbf{NOTED}}}
    \centering
    \resizebox{1\linewidth}{!}{\begin{tabular}{|c|c|c|c|c|c|c|c|}
        \hline
        \multirow{2}{*}{\makecell{\textbf{OTA} \\ \textbf{topology}}} & \multirow{2}{*}{\makecell{\textbf{One-time} \\\textbf{training} \\\textbf{duration}}}  & \multicolumn{2}{l|}{\makecell{\textbf{Single iteration}}} & \multicolumn{3}{l|}{\makecell{\textbf{Multiple iterations}}} \\ \cline{3-7}
        & & \makecell{\textbf{\#designs}\\ \textbf{optimized}} & \makecell{\textbf{Average} \\\textbf{time}} & \makecell{\textbf{\#designs} \\ \textbf{optimized}} &  \makecell{\textbf{Average} \\\textbf{time}} & \makecell{\textbf{Average}\\\textbf{\#iterations}} \\
        \hline
        5T-OTA & 8.5h  & 95/100 & 37s & 5/100 & 111s & 3\\
        \hline
        CM-OTA & 22h  & 98/100 & 46s & 2/100 & 230s & 5\\
        \hline
        2S-OTA & 11h & 90/100 & 36s & 10/100 & 180s & 5\\
        \hline
    \end{tabular}}
    \label{tab:runtime}
    \vspace{-5mm}
\end{table}

% \vspace{-2mm}

\subsection{Runtime analysis}
\noindent
Table~\ref{tab:runtime} provides a detailed runtime analysis, including both the one-time SPICE-based training duration and the average runtime per design optimization by the trained model. The reported runtime encompasses the entire process, from sequence inference by the trained transformer model (taking approximately 0.5s per sequence)
% -- where each sequence takes approximately 0.5 seconds -- 
to the LUT-based estimation and subsequent SPICE simulation verification. 
In cases where performance criteria are not fully met due to minor prediction inaccuracies, a ``copilot'' mode is activated, performing iterative refinements with progressively tighter specifications to introduce a design margin that compensates for errors. This ensures that all design specifications are ultimately satisfied, typically requiring only a few additional iterations, thereby balancing model accuracy with computational efficiency to achieve reliable design convergence. The runtime ranges from just above 30s to just under four minutes, significantly lower than competing methods.
% \vspace{-2mm}

\subsection{Qualitative comparison with prior approaches.}

\noindent
Table~\ref{tab:comparison} compares our approach for OTA sizing with prior methods, including simulated annealing (SA)~\cite{gielen_90}, particle swarm optimization (PSO)~\cite{vural_12}, graph convolutional network-based RL (GCN-RL)~\cite{Wang_2020}, weighted expected improvement-based Bayesian optimization (WEIBO)~\cite{lyu_18}, and differential evolutionary (DE) algorithm~\cite{liu_09}.
% \redfn{You never answered my question re. comparing against~\cite{budak_21}.\textbf{I haven not included ~\cite{budak_21}, because, firstly its RL, and it needs SPICE in the loop for convergence. Although it has significantly reduced the no. of simulations, it still needs $>100$ SPICE simulations. If I want to compare with this, I have to show the exact no. of SPICE Simulations required for our topologies. } {\em (1)~I don't understand your logic here. Have you shown the exact number of SPICE simulations for any of the methods in Table~\ref{tab:comparison}? I don't see it. So, why talk about the number for~\cite{budak_21}? \textbf{By looking at the numbers that Budak and rest of the papers have shown in their comparison, the number of simulations in ~\cite{budak_21} is way less than the rest of all the papers. Not only that, the previous papers using stochastic methods don't even have 100\% success rate. So in summary, Budak paper ~\cite{budak_21} is a close competitor and to beat that, solid number is necessary}. {\em If you need 3-5 iterations (as shown in your table), you need 3-5 SPICE simulations, right? If so, why is a paper with 100 simulations a ``close competitor''?} \textbf{Because their circuits are more complicated. They may not need 100 simulations for 5T OTA. So, its hard to claim that.} {\em OK, I buy your argument. Let's leave it out.} RESOLVED. OK(2)~Also -- I just noticed that you \underline{never} refer to Table~\ref{tab:comparison} anywhere in the paper!! Please double-check that all figures and tables are cited at least once in the text of the paper. It is pointless to drop in a table without citing it in the paper. This section needs a clear pointer to the table (which I have now added in the first sentence).} \textbf{Even I am surprised with this ignorance of mine. I mentioned it initially, but later was rephrasing and deleted it somehow. I checked the rest of the paper for all the figures and tables, and they all are cited at their designated sections.} RESOLVED}

We utilize various metrics for comparison. \textit{SPICE simulation dependency} gauges the reliance on costly simulations for convergence: lower dependence indicates greater efficiency. \textit{Sizing accuracy} measures how well the approach satisfies all design specifications.
% \textit{Optimization efficiency} captures the trade-off between resource usage and the ability to find optimal designs, while sizing accuracy evaluates how reliably each method determines correct transistor sizes.
% \redfn{What is your basis for saying this method is high? You never evaluate whether the sizing is performed at low area or power -- so what is your metric for deciding that this is true? \textbf{I probably thought about this metric in the wrong way. I thought of it as a function of the performance of the optimized design and computational cost for the optimization.} {\em So how did you think about it? What is your basis for saying that the optimization efficiency is high? Can you use a different term that captures what you were trying to say?} \textbf{I feel like this metric is kind of redundant now. We have SPICE simulation dependency metric and Convergence cost, which pretty much covers this metric as well.} {\em I am ok with removing this row of the table.} RESOLVED (from my end).} 
\textit{Runtime} reflects the time required to reach a solution, and \textit{memory utilization} pertains to the amount of memory resources consumed during the optimization process.
% \redHL{\sout{resource usage} runtime}.
% \redfn{Unclear what a ``resource'' is in this context. \textbf{Time and memory due to SPICE simulations} {\em How is this different from ``SPICE simulation dependency'' which seems to already cover this issue?} \textbf{I thought of mentioning this explicitly because at the end overall sizing duration is the most important factor in favor of us.} {\em Can we maybe focus just on runtime and skip memory (which is harder to quantify anyway)? If so, could this row by ``Runtime'' and then you could list it as hours/days/... for others and 0.5s for us. Again, there is the issue of more complicated circuits -- not sure if it makes it hard to use their numbers directly?} \textbf{Prof, I don't think using their numbers would work. } {\em OK. Can we still change it to ``runtime'' or are you trying to say something different?}\textbf{We can definitely use ``runtime''} {\em OK, please make that change. It feels more concrete and is in line with our claims.} RESOLVED OK} 
% \redfn{I had provided feedback on this table, but it is not clear if it was read. There is no change to the table, and I see no email explaining why this form is better. You should not ignore my feedback. If you disagree, that's ok -- but make an argument against what I am saying rather than just ignoring my feedback and throwing it in the trash. \textbf{Prof, I considered your feedback and included the modified comparison by using the terms "Moderate to low", "Very slow" just to distinguish between two "High" and "High" which I did previously. Unfortunately reporting approx no. of SPICE simulations may raise questions. Cause in that case we have to tell about our implementation of the other approaches and then check for no. of simulations for our OTA topologies.} {\em OK, this is a reasonable argument. But you need to make such arguments to me, not unilaterally make decisions without consultation. Your logic is right in this case, but there may be other cases when it is not. We cannot have decisions without discussion.} \textbf{Yes Prof. I totally agree. } RESOLVED} 
Our approach, using a trained transformer model with precomputed LUTs, significantly reduces SPICE simulation dependency -- achieving over 90\% of sizing without simulations -- while improving accuracy and reducing runtime from hours to seconds, positioning it as a highly efficient solution.
% \redfn{One more thing: what is the difference between the last two rows -- runtime and computational cost? And also ``convergence speed''? All seem to be very similar. \textbf{Computational cost mean the total memory utilization due to Simulations } That is not obvious. Maybe use the terms you mean -- e.g., memory, runtime, etc.? I don't think the average reader will understand computational cost == runtime. \textbf{Should I use just ``Memory'' or ``Memory utilization or requirement''?} Either is ok. Maybe ``Memory utilization'' is ok. RESOLVED.}
% \redfn{I don't really see the relationship between the table and these statements. You connect it directly to ``SPICE simulation dependency'' but I don't see any of the other keywords in column 1 of the table. \textbf{I connect to SPICE simulation dependency, convergence speed, and computational cost, because those are the only metrics which are in favor of us, which I have shown in BOLD in the table} {\em Yes, but this text does not use any of those terms. If you want to connect it to the table, use the same terms in the text.} RESOLVED}

% \blueHL{Many comments on the table: (1)~When you say that the training cost is ``High'' in all cases -- is it equally high? Can you be more quantitative by reporting the number of hours? \textbf{I will think about this} (2)~Unclear how ``Handle layout parasitics'' is a yes for us -- we also work on a schematic.  I don't see anything in your paper about handling layout parasitic. \textbf{We haven't demonstrated that, but from principle, the way we use DPSFG, will allow inclusion of more R's and C's into the graph} (3)~``90\% SPICE-independent sizing'' is poorly stated and will not convey what you want to say.  Can you report the number of SPICE simulations instead? Similarly, ``SPICE-based convergence'' is a poor descriptor and the reader will have no idea what you are saying. The footnote clarifies it a bit, but you can put some effort into making yourself more clear in the table too. (4)~``Sizing duration'' sounds amateurish.  Something like ``Runtime of the sizer'' would be better.}

\begin{table}[ht]
    \centering
    \vspace{-2mm}
    \caption{Qualitative comparison with prior sizing approaches. }
    % \redHL{Why are some items bf and others not? If we are trying to bf our approach, why is sizing accuracy = high not in bf?} \textbf{Because, its high for others also, so nothing special in our method.}}
    \resizebox{1\linewidth}{!}{
    \begin{tabular}{|c|c|c|c|c|c|c|}
        \hline
        \makecell{\textbf{Sizing method}} & \makecell{\cite{gielen_90}} & 
        \makecell{\cite{vural_12}} &
        \makecell{\cite{Wang_2020}} &
        \makecell{\cite{lyu_18}} &  
         \makecell{\cite{liu_09}} &
         \makecell{\textbf{Our approach}}
         \\
        \hline
        \textbf{Algorithm} & SA & PSO & \makecell{GCN \\ + RL} & WEIBO & DE & \makecell{Transformer \\ + LUT } \\
        \hline
        \makecell{\textbf{SPICE simulation} \\ \textbf{dependency}} & \makecell{Very high} & \makecell{Very high} & \makecell{Low to \\ moderate}& \makecell{High} &  \makecell{Very high} & \makecell{\textbf{Very low\textsuperscript{*}}}\\
        % \hline
        % \makecell{\textbf{Optimization} \\ \textbf{efficiency}} & \makecell{Moderate \\ to low} & \makecell{Moderate \\ to low} & Moderate &  High & High & High\\
        \hline
        \makecell{\textbf{Sizing accuracy}}& Variable & \makecell{Moderate \\ to high}& High & High & High & \makecell{ High}\\
        \hline
        \makecell{\textbf{Runtime}} & \makecell{Very slow}& \makecell{Very slow}& \makecell{Moderate}& \makecell{Moderate}  & \makecell{Slow}& \makecell{\textbf{Very fast}} \\
        \hline
        \makecell{ \textbf{Memory} \\ \textbf{utilization}} & Moderate & Moderate & \makecell{Moderate \\ to high}& \makecell{High} & \makecell{Very high} & \makecell{\textbf{Moderate} \\ \textbf{to low}}\\
        
        
        \hline
    \end{tabular}}
    \label{tab:comparison}
    \scriptsize\textsuperscript{*} $>$90\% of sizing is performed without SPICE simulations, as shown in Table~\ref{tab:runtime}.
    \vspace{-3mm}
\end{table}





% %Original
% \begin{table}[ht]
%     \centering
%     \vspace{-1mm}
%     \caption{Qualitative comparison with prior sizing approaches.}
%     \resizebox{1\linewidth}{!}{
%     \begin{tabular}{|c|c|c|c|c|c|c|}
%         \hline
%         \makecell{\textbf{Sizing method}} & \makecell{\cite{gielen_90}} & \makecell{\cite{liu_09}} &  
%         \makecell{\cite{vural_12}} & \makecell{\cite{lyu_18}} &
%         \makecell{\cite{Wang_2020}} & \makecell{\textbf{Our} \\ \textbf{approach}}
%          \\
%         \hline
%         \textbf{Algorithm} & SA & DE & PSO & WEIBO & \makecell{GCN \\ + RL} & \makecell{Transformer \\ + LUT } \\
%         \hline
%         \makecell{\textbf{SPICE simulation} \\ \textbf{dependency}} & \makecell{Very \\ high} & \makecell{Very \\ high} & \makecell{Very \\ high} &  \makecell{High} & \makecell{Low to \\ moderate}& \makecell{\textbf{Very} \\ \textbf{low\textsuperscript{*}}}\\
%         % \hline
%         % \makecell{\textbf{Optimization} \\ \textbf{efficiency}} & \makecell{Moderate \\ to low} & \makecell{Moderate \\ to low} & Moderate &  High & High & High\\
%         \hline
%         \makecell{\textbf{Sizing accuracy}}& Variable & High & \makecell{Moderate \\ to high} & High & High & \makecell{ High}\\
%         \hline
%         \makecell{\textbf{Runtime}} & \makecell{Very \\ slow}& \makecell{Slow}& \makecell{Very \\ slow}  & \makecell{Moderate} & \makecell{Moderate} & \makecell{\textbf{Very} \\ \textbf{fast}} \\
%         \hline
%         \makecell{ \textbf{Memory} \\ \textbf{utilization}} & Moderate & \makecell{Very \\ high} & Moderate & High & \makecell{Moderate \\ to high} & \makecell{\textbf{Moderate} \\ \textbf{to low}}\\
        
        
%         \hline
%     \end{tabular}}
%     \label{tab:comparison}
%     \scriptsize\textsuperscript{*}$ >90\%$ of sizing is done without SPICE simulations, as shown in Table~\ref{tab:runtime}.
%     % \vspace{-4.5mm}
% \end{table}


% \redfn{\textbf{**FIXED**}The table has numerous problems: (1)~``next few iterations'' is not a scientific statement. How many iterations? (2)~What is the criterion for tightening the specifications? (3)~Presentation issues: (a)~1st (informal usage) $\rightarrow$ first; (b)~No. $\rightarrow$ number (this is also pointed out elewhere); (c)~Successfully sized designs $\rightarrow$ Number of designs that meet all specifications}










\section{Conclusion and Discussion}
In this paper, we introduce 3DMolFormer for structure-based drug discovery, a dual-channel transformer-based framework designed to process parallel sequences of tokens and numerical values representing pocket-ligand complexes. Through self-supervised large-scale pre-training and supervised fine-tuning, 3DMolFormer can accurately and efficiently predict the binding poses of ligands to protein pockets. Furthermore, through reinforcement learning fine-tuning, 3DMolFormer can generate drug candidates that exhibit high binding affinities for a given protein target, along with favorable drug-likeness and synthesizability. Above all, 3DMolFormer is the first machine learning framework that can simultaneously address both protein-ligand docking and pocket-aware 3D drug design, and it outperforms previous baselines in both tasks.

It is noteworthy that many recent deep learning models for 3D molecules, such as Uni-Mol, Pocket2Mol, TargetDiff, and DecompDiff, which serve as baselines in our experiments, adhere to the concept of "equivariance" introduced by geometric deep learning~\citep{Equivariance,Equivariance2}. However, the 3DMolFormer model does not explicitly enforce SE(3)-symmetry. It appears that through the normalization of 3D coordinates and random rotations during data augmentation, 3DMolFormer has acquired the SE(3)-equivariance by training on a sufficiently large and diverse dataset. This approach aligns with recent successful methods in the field, including AlphaFold3~\citep{AlphaFold3}, which also does not rely on SE(3)-equivariant architectures.

Admittedly, our approach still has some limitations. First, 3DMolFormer does not account for the flexibility of proteins during ligand binding, which may affect the accuracy of subsequent binding affinity prediction. Second, protein-ligand binding is a dynamic process, but 3DMolFormer struggles to capture this dynamism effectively. Finally, 3DMolFormer does not consider environmental factors such as temperature and pH, which can significantly influence the 3D conformation of the binding complex. These issues represent core challenges in current computational methods for structure-based drug discovery, and we look forward to future work addressing these limitations. Furthermore, the implementation details in 3DMolFormer have the potential to be further optimized, for example, advanced methods of multi-objective reinforcement learning~\citep{MORL} may be introduced into the drug design process.

% \redHL{The number of references is quite small and most of the last page is blank. This was understandable when you were page-limited, but it looks bad when the last page is mostly blank. (Don't you think about these things without being told?) I have tried to mitigate this by using the full form of each journal/conference to take up more space (also changed to IEEEtran instead of ieeetr2, which allows me to turn off ``{\em et al.}''). I am not even sure you have run a good enough literature review. There are no references from Helmut Graeb's group, and he has done a LOT of work on OTAs -- missing his work is a glaring omission. Nothing from Georges Gielen's group. Plus I am sure there are other groups you have missed. {\bf Please answer this question: have you really made the effort to read all papers on OTA sizing?} I suspect not. Don't you think 2 years is enough time to show the will to conduct a full literature review? What do you do with your time??} SSS_NOTE

\newpage
% \bstctlcite{IEEEexample:BSTcontrol}
% \bibliographystyle{alpha}
% \bibliographystyle{misc/ieeetr2}
\bibliographystyle{misc/IEEEtran}
% \bibliography{main.bib}
% Generated by IEEEtran.bst, version: 1.14 (2015/08/26)
\begin{thebibliography}{10}
\providecommand{\url}[1]{#1}
\csname url@samestyle\endcsname
\providecommand{\newblock}{\relax}
\providecommand{\bibinfo}[2]{#2}
\providecommand{\BIBentrySTDinterwordspacing}{\spaceskip=0pt\relax}
\providecommand{\BIBentryALTinterwordstretchfactor}{4}
\providecommand{\BIBentryALTinterwordspacing}{\spaceskip=\fontdimen2\font plus
\BIBentryALTinterwordstretchfactor\fontdimen3\font minus \fontdimen4\font\relax}
\providecommand{\BIBforeignlanguage}[2]{{%
\expandafter\ifx\csname l@#1\endcsname\relax
\typeout{** WARNING: IEEEtran.bst: No hyphenation pattern has been}%
\typeout{** loaded for the language `#1'. Using the pattern for}%
\typeout{** the default language instead.}%
\else
\language=\csname l@#1\endcsname
\fi
#2}}
\providecommand{\BIBdecl}{\relax}
\BIBdecl

\bibitem{harjani_89}
R.~Harjani, R.~Rutenbar, and L.~Carley, ``{OASYS}: A framework for analog circuit synthesis,'' \emph{IEEE Transactions on Computer-Aided Design of Integrated Circuits and Systems}, vol.~8, no.~12, pp. 1247--1266, Dec. 1989.

\bibitem{koza_96}
J.~R. Koza, F.~H. Bennett, D.~Andre, and M.~A. Keane, ``Automated design of both the topology and sizing of analog electrical circuits using genetic programming,'' in \emph{Artificial Intelligence in Design '96}, J.~S. Gero and F.~Sudweeks, Eds.\hskip 1em plus 0.5em minus 0.4em\relax Dordrecht, Netherlands: Springer, 1996, pp. 151--170.

\bibitem{Kruiskamp_95}
W.~Kruiskamp and D.~Leenaerts, ``{DARWIN}: {CMOS} opamp synthesis by means of a genetic algorithm,'' in \emph{Proceedings of the ACM/IEEE Design Automation Conference}, 1995, pp. 433--438.

\bibitem{gielen_90}
G.~Gielen, H.~Walscharts, and W.~Sansen, ``Analog circuit design optimization based on symbolic simulation and simulated annealing,'' \emph{IEEE Journal of Solid-State Circuits}, vol.~25, no.~3, pp. 707--713, Jun. 1990.

\bibitem{vural_12}
R.~A.~Vural and T.~Yildirim, ``Analog circuit sizing via swarm intelligence,'' \emph{AEU -- International Journal of Electronics and Communications}, vol.~66, p. 732–740, Sep. 2012.

\bibitem{abel_22}
I.~Abel and H.~Graeb, ``{FUBOCO}: Structure synthesis of basic op-amps by functional block composition,'' \emph{ACM Transactions on Design Automation of Electronic Systems}, vol.~27, no.~6, Jun. 2022.

\bibitem{abel_22_2}
I.~Abel, M.~Neuner, and H.~E. Graeb, ``A hierarchical performance equation library for basic op-amp design,'' \emph{IEEE Transactions on Computer-Aided Design of Integrated Circuits and Systems}, vol.~41, no.~7, pp. 1976--1989, 2022.

\bibitem{hershenson_01}
M.~Hershenson, S.~Boyd, and T.~Lee, ``Optimal design of a {CMOS} op-amp via geometric programming,'' \emph{IEEE Transactions on Computer-Aided Design of Integrated Circuits and Systems}, vol.~20, no.~1, pp. 1--21, Jan. 2001.

\bibitem{budak_21}
A.~F. Budak, P.~Bhansali, B.~Liu, N.~Sun, D.~Z. Pan, and C.~V. Kashyap, ``{DNN-Opt}: An {RL} inspired optimization for analog circuit sizing using deep neural networks,'' in \emph{Proceedings of the ACM/IEEE Design Automation Conference}, 2021, pp. 1219--1224.

\bibitem{settaluri_20}
K.~Settaluri, A.~Haj-Ali, Q.~Huang, K.~Hakhamaneshi, and B.~Nikolic, ``{AutoCkt}: Deep reinforcement learning of analog circuit designs,'' in \emph{Proceedings of the Design, Automation \& Test in Europe}, 2020, pp. 490--495.

\bibitem{Wang_2020}
H.~Wang, K.~Wang, J.~Yang, L.~Shen, N.~Sun, H.-S. Lee, and S.~Han, ``{GCN-RL} circuit designer: Transferable transistor sizing with graph neural networks and reinforcement learning,'' in \emph{Proceedings of the ACM/IEEE Design Automation Conference}, 2020, pp. 1--6.

\bibitem{choi_23}
M.~Choi, Y.~Choi, K.~Lee, and S.~Kang, ``Reinforcement learning-based analog circuit optimizer using {$g_m/I_D$} for sizing,'' in \emph{Proceedings of the ACM/IEEE Design Automation Conference}, 2023.

\bibitem{vaswani_17}
A.~Vaswani, N.~Shazeer, N.~Parmar, J.~Uszkoreit, L.~Jones, A.~N. Gomez, L.~Kaiser, and I.~Polosukhin, ``Attention is all you need,'' in \emph{Advances in Neural Information Processing Systems}, vol.~30, Dec. 2017, pp. 5998--6008.

\bibitem{ochoa_98}
A.~Ochoa, ``A systematic approach to the analysis of general and feedback circuits and systems using signal flow graphs and driving-point impedance,'' \emph{IEEE Transactions on Circuits and Systems II}, vol.~45, no.~2, pp. 187--195, Feb. 1998.

\bibitem{schmid_18}
H.~Schmid and A.~Huber, ``Analysis of switched-capacitor circuits using driving-point signal-flow graphs,'' \emph{Analog Integrated Circuits and Signal Processing}, vol.~96, pp. 495--507, Sep. 2018.

\bibitem{Mason53}
S.~J. Mason, ``Feedback theory-some properties of signal flow graphs,'' \emph{Proceedings of the IRE}, vol.~41, no.~9, pp. 1144--1156, 1953.

\bibitem{schmid_yt}
{H. Schmid}, ``{{HT FHNW EIT}: Analog and mixed-signal circuits and signal processing},'' \url{https://tube.switch.ch/channels/d206c96c}.

\bibitem{rico_16}
R.~Sennrich, B.~Haddow, and A.~Birch, ``Neural machine translation of rare words with subword units,'' in \emph{Annual Meeting of the Association for Computational Linguistics}, Aug. 2016, pp. 1715--1725.

\bibitem{silviera_96}
F.~Silveira, D.~Flandre, and P.~Jespers, ``A {$g_m$/$I_D$} based methodology for the design of {CMOS} analog circuits and its application to the synthesis of a silicon-on-insulator micropower {OTA},'' \emph{IEEE Journal of Solid-State Circuits}, vol.~31, no.~9, pp. 1314 -- 1319, Oct. 1996.

\bibitem{jespers_17}
P.~Jespers and B.~Murmann, \emph{Systematic Design of Analog {CMOS} Circuits: Using Pre-Computed Lookup Tables}.\hskip 1em plus 0.5em minus 0.4em\relax Cambridge, UK: Cambridge University Press, 2017.

\bibitem{lyu_18}
W.~Lyu, P.~Xue, F.~Yang, C.~Yan, Z.~Hong, X.~Zeng, and D.~Zhou, ``An efficient {Bayesian} optimization approach for automated optimization of analog circuits,'' \emph{IEEE Transactions on Circuits and Systems I}, vol.~65, no.~6, pp. 1954--1967, Jun. 2018.

\bibitem{liu_09}
B.~Liu, Y.~Wang, Z.~Yu, L.~Liu, M.~Li, Z.~Wang, J.~Lu, and F.~V. Fernández, ``Analog circuit optimization system based on hybrid evolutionary algorithms,'' \emph{Integration}, vol.~42, no.~2, pp. 137--148, Apr 2009.

\end{thebibliography}

%misc/cram,misc/main.bib,misc/pim.bib}

%\newpage
\appendix
\onecolumn

\part{}
\section*{\centering \LARGE{Appendix}}
\mtcsettitle{parttoc}{Contents}
\parttoc

\clearpage

\section{Related Work}
\label{sec:relatedwork}
% \paragraph{Tool Usage and Toolchain Management} Research in this area focuses on how intelligent agents design and optimize tool networks to effectively execute complex tasks, particularly by dynamically generating, selecting, and combining tools based on task requirements.This includes methods for automated tool generation and optimization, emphasizing systems that can adaptively choose and adjust tool combinations according to different task needs.

% \paragraph{Multi-Agent Systems and Collaboration} Research in Multi-Agent Systems has explored how multiple intelligent agents can collaboratively solve complex tasks in dynamic environments. One significant contribution is the development of decentralized algorithms that allow agents to autonomously form beneficial collaborations and adapt to changing tasks without the need for a central server (DeLAMA) ~\citep{tang2024decentralizedlifelongadaptivemultiagentcollaborative}. Another key area of study focuses on collaboration among heterogeneous agents, where different agents with varied capabilities work together on complex tasks, such as cleaning large spaces, using hierarchical decision models to allocate sub-tasks effectively~\citep{liu2023heterogeneousembodiedmultiagentcollaboration}. Additionally, collaborative learning approaches like Collaborative Q-learning (CollaQ) enhance agent teamwork by decomposing the Q-function and introducing reward attribution techniques to improve performance in multi-agent environments, such as the StarCraft challenge ~\citep{zhang2020multiagentcollaborationrewardattribution}. Finally, research has also examined how multi-agent collaboration can enhance the performance of large language models (LLMs) in tasks like simulations and software development, highlighting the potential of intelligent agent collaboration to improve task outcomes~\citep{talebirad2023multiagentcollaborationharnessingpower}.

\paragraph{Code Generation and Task Solving with LLMs} Large Language Models (LLMs) have demonstrated remarkable potential in generating code to solve complex tasks. Prior studies highlight their effectiveness in mathematical computation ~\citep{zhou2023solving, wang2023mathcoder, gou2023tora}, tabular reasoning ~\citep{chen2022program, lyu2023faithful, lu2024chameleon}, and visual understanding ~\citep{suris2023vipergpt, choudhury2023zero, gupta2023visual}. Frameworks such as AutoGen ~\citep{wu2023autogen} and CodeActAgent~\citep{wang2024executable} extend this capability to agent-based tasks by interpreting executable code as actions. These models dynamically invoke basic tools based on environmental feedback, significantly expanding their utility. Despite their successes, these approaches often treat program generation processes independently, failing to model shared task features and limiting the reusability of functional modules across tasks.

\paragraph{Reusable Tool Creation and Abstraction} To address the limitations of single-use program generation, recent efforts have focused on creating reusable tools. CREATOR ~\citep{qian2023creator} separates the processes of planning (tool creation) and execution, while LATM ~\citep{cai2023large} and CRAFT ~\citep{yuan2023craft} pre-build tools using training and validation sets for task solving. However, these methods often generate a large number of tools, presenting challenges for their efficient reuse. Furthermore, while abstraction-based approaches like REGAL ~\citep{stengel2024regal} focus on extracting reusable tools from primitive programs, they primarily construct simple tools with limited functional complexity. Similarly, Trove ~\citep{wang2024trove} adopts a training-free approach by dynamically composing high-level tools during testing, but its reliance on self-consistency can lead to hallucinated knowledge, reducing accuracy in complex tasks.

\paragraph{Tool Selection for Complex Task Solving} Currently, research on tool selection and retrieval methods primarily focuses on selecting appropriate tools through retrieval mechanisms and LLM-based approaches. ToolRerank ~\citep{zheng2024toolrerank} uses adaptive truncation and hierarchy-aware reranking to improve retrieval results, while Re-Invoke ~\citep{chen2024reinvoketool} introduces an unsupervised framework with synthetic queries and multi-view ranking, enhancing both single-tool and multi-tool retrieval. COLT ~\citep{Qu_2024COLT} combines semantic matching with graph-based collaborative learning to capture relationships among tools, outperforming larger models in some cases. AvaTaR~\citep{wu2024avataroptimizingllmagents} automates the optimization of LLM prompts for better tool utilization, and DRAFT~\citep{qu2024DAFT} refines tool documentation through iterative feedback and exploration, helping LLMs better understand external tools. Despite progress, existing methods generally overlook cost-effectiveness and scalability in tool selection, and often struggle to efficiently adapt to new tools and task requirements in dynamic environments, leading to performance and efficiency bottlenecks. In contrast, our approach dynamically prioritizes tools by combining their relevance and structural importance, ensuring computational efficiency and scalability, thus enabling more effective solutions for complex tasks.
\section{Experimental Details}
\label{app:apexp}
\subsection{Open-ended Task}
\label{subsec:open}
\paragraph{Benchmark} We employed the benchmark proposed by Voyager~\citep{wang2023voyager}, using Minecraft as the experimental platform. Minecraft provides a sandbox environment where players gather resources and craft tools to achieve various goals. The simulation is built on MineDojo~\citep{fan2022minedojo} and uses Mineflayer~\citep{PrismarineJS2013} JavaScript APIs for motor control. 

\paragraph{Baselines}
We conducted a comprehensive comparison with four baselines. Except for Voyager, these methods were originally designed for NLP tasks without embodiment. Therefore, we had to reinterpret and adapt them for execution within the MineDojo environment, ensuring compatibility with the specific requirements of our experimental setup.
\begin{itemize}
    \item \textbf{ReAct:} ReAct~\citep{yao2022react} uses chain-of-thought prompting [46] by generating both reasoning traces and action
plans with LLMs. We provide it with our environment feedback and the agent states as observations.
    \item \textbf{Reflexion:} Reflexion~\citep{shinn2023reflexion} is built on top of ReAct~\citep{yao2022react}with self-reflection to infer more intuitive future actions.
    \item \textbf{AutoGPT:} AutoGPT~\citep{richardssignificant} is a popular software tool that automates NLP tasks by decomposing a high-level
goal into multiple subgoals and executing them in a ReAct-style loop. We re-implement AutoGPT by using GPT-4O to do task decomposition and provide it with the agent states, environment feedback,
and execution errors as observations for subgoal execution
We provide it with execution errors and our self-verification module.
    \item \textbf{Voyager:} Voyager~\citep{wang2023voyager} is a system that integrates an automated curriculum, a scalable skill library, and an iterative prompting framework based on environmental feedback to explore, store, and accumulate skill library within the Minecraft environment.
\end{itemize}


\paragraph{Metric}
The evaluation metric is based on the number of iterations required to progress through tool upgrades, from wooden to stone, iron, and finally diamond tools. Each execution of code is considered one iteration.

\paragraph{Model}
We leverage GPT-4o for text completion, along with the text-embedding-ada-002 API for text embedding. We set all temperatures to
0 except for the automatic curriculum, which uses temperature = 0.1 to encourage task diversity. 

\paragraph{Setting}
We set the maximum number of iterations to 160. For both \ours\ and Voyager, all agents are controlled by GPT-4o, with the number of tools retrieved per iteration set to 5. To ensure a fairer comparison, we removed the Tool Requirement Stage and bug-free checks in \ours\ , and allowed a maximum of 3 self-checks per iteration.

\paragraph{Item Types and Levels}
In the Minecraft task, there are different types and levels of items. Diamond tools are the highest level, and rare items such as golden apples also exist. High-level tools require some lower-level items to craft. Table \ref{tab:toollist} lists the key items in the Minecraft task.
\begingroup
\begin{table}[H]
\caption{List of item types and levels in the Minecraft task.}
\label{tab:toollist}
\vskip -0.1in
\setlength{\tabcolsep}{10pt} % 调整列间距
\begin{center}
\begin{small}
\begin{sc}
\begin{tabular}{l|c|c}
\toprule
\textnormal{\textbf{Category}} & \textnormal{\textbf{level}} & \textnormal{\textbf{Items}} \\
\midrule         
\midrule
\multirow{4}{*}{\multirow{3}{*}{\normalfont Tools}} 
              & \normalfont Wooden Tools & \normalfont Wooden\_Shovel,Wooden\_Pickaxe,Wooden\_Axe,Wooden\_Hoe,Wooden\_Sword \\
              \cmidrule{2-3}
              & \normalfont Stone Tools &\normalfont stone\_pickaxe, stone\_shovel,Stone\_Axe,Stone\_Hoe,Stone\_Sword   \\
              \cmidrule{2-3}
              & \normalfont Iron Tools &\normalfont iron\_pickaxe, iron\_axe, iron\_sword, iron\_shovel, iron\_hoe    \\
              \cmidrule{2-3}
              & \normalfont Diamond Tools &\normalfont diamond\_pickaxe, diamond\_sword, diamond\_axe, diamond\_shovel    \\
             
\midrule
\multirow{2}{*}{\multirow{1}{*}{\normalfont  Armor}} 
              & \normalfont Iron Armor &\normalfont iron\_chestplate, iron\_helmet, iron\_leggings  \\
              \cmidrule{2-3}
              & \normalfont Diamond Armor &\normalfont diamond\_chestplate, diamond\_helmet, diamond\_leggings, diamond\_boots     \\

\midrule
\multirow{3}{*}{\multirow{2}{*}{\normalfont  Food}} 
              & \normalfont Raw Food &\normalfont chicken, mutton, porkchop, rabbit, raw\_rabbit, spider\_eye, bone  \\
              \cmidrule{2-3}
              & \normalfont Cooked Food &\normalfont cooked\_beef, cooked\_chicken, cooked\_mutton, cooked\_porkchop, cooked rabbit  \\
              \cmidrule{2-3}
              & \normalfont Advanced Food &\normalfont golden apple    \\

\bottomrule
\end{tabular}
\end{sc}
\end{small}
\end{center}
\vskip -0.1in
\end{table}
\endgroup


\subsection{Agent Task}
\label{subsec:agent}
\paragraph{Benchmark}
We conducted experiments on two types of agent tasks, demonstrating {\ours}'s capabilities in both game-related and data science tasks.
\begin{itemize}
     \item \textbf{TextCraft:} We evaluate {\ours} on the TextCraft dataset~\citep{futuyma1988evolution}, which challenges agents to craft Minecraft items in a text-only environment~\citep{cote2019textworld}. Each task instance provides a goal and a sequence of crafting commands, which include distractors. We use depth-2 splits for testing and reserve a subset of depth-1 recipes for development, resulting in a 99/77 train/test split.
    \item \textbf{InfiAgent-DABench:} We also test {\ours} on the InfiAgent-DABench benchmark~\citep{hu2024infiagent}, which evaluates LLM-based agents on end-to-end data analysis tasks. This benchmark consists of 257 questions across 52 CSV files, with each question corresponding to a unique CSV file. Agents are required to generate code to analyze data and produce the specified output format. We randomly selected 20 CSV files and their associated question-answer pairs as training data, resulting in a train/test split of 98/159 instances.
\end{itemize}

\paragraph{Baselines}
We compare \ours\ with three methods described below.
\begin{itemize}
     \item \textbf{ReAct:} In this setting, we employ the executor to interact iteratively with the environment, adopting the think-act-observe prompting style from ReAct~\citep{yao2022react}.
     \item \textbf{Plan-Execution:} In contrast, the Plan-and-Execute approach~\citep{shridhar2023art, yang2023intercode} generates a plan upfront and assigns each sub-task to the executor. To ensure each step is executable without further decomposition, we provide new prompts with more detailed planning instructions.
    \item \textbf{Reflexion:} In the Reflection setting~\citep{shinn2023reflexion}, the agent engages in self-reflection after each step, drawing on environmental feedback and exploration history. 
\end{itemize}

\paragraph{Metric} 
The most practically important aspect of the solutions is correctness. For Textcraft, we verify whether the agent’s inventory contains the goal item. For DABench, we check if the agent’s final answer matches the ground truth.

\paragraph{Model}
During training, we use GPT-4o to construct the tool library with a temperature setting of 0. In the testing phase, we conduct a comprehensive comparison of various open-source and closed-source models. The open-source models include \textit{Qwen2.5-7B-Instruct, Qwen-Coder-7B-Instruct, Qwen2.5-14B-Instruct, Deepseeker-Coder-6.7B-Instruct, and Deepseeker-Coder-33B-Instruct}, while the closed-source models primarily include \textit{gpt-3.5-turbo-1106} and \textit{Claude-3-haiku}. During testing, the temperature is set to 0.3, and each experiment is repeated 3 times, with the average result reported.

\paragraph{Setting} 
For ReAct, Reflexion, and \ours\ , the maximum number of steps is set to 20. For Plan-Execution, the maximum number of steps for each sub-task is set to 8. In \ours\ , the number of tools retrieved during testing is limited to 3.



\subsection{Single-turn Code Task}
\label{subsec:code}
\paragraph{Benchmark}
To further explore {\ours}'s potential, we evaluated it on single-turn code generation tasks spanning mathematical reasoning, date comprehension, and tabular reasoning:
 \begin{itemize}
     \item \textbf{MATH:} We used a subset of the MATH dataset~\citep{hendrycks2021measuring}, focusing on 405 level-4 and level-5 algebra problems (MATH contains 5 levels of difficulty) that require textual understanding and advanced reasoning. We randomly selected 200 examples from the test set of the MATH dataset to construct the tool network, resulting in a train/test split of 200/405.
     \item \textbf{Date:} We use the date understanding task from BigBenchHard~\citep{srivastava2022beyond}, which consists of short word problems requiring date understanding. We follow the data splits provided by REGAL\citep{stengel2024regal}, resulting in a train/test split of 66/180.
     \item \textbf{TabMWP:} We further extend our general experiments on MATH by testing on TabMWP~\citep{grand2023learning}, a tabular reasoning dataset consisting of math word problems about tabular data. Based on the CRAFT~\citep{yuan2023craft} splits, we selected 470 problems from levels 7 and 8 (TabMWP contains 8 levels) from the 1,000 test examples. Additionally, we randomly selected 200 examples from the TabMWP training set, resulting in a train/test split of 200/470.
\end{itemize}

\paragraph{Baselines}
For these tasks, we use Programs of Thoughts (PoT)~\citep{chen2022program} and other existing tool-making methods as baselines for comparison.

\begin{itemize}
    \item \textbf{PoT:} The LLM utilizes a program to reason through the problem step by step~\citep{chen2022program}.
   \item \textbf{LATM:} LATM~\citep{cai2023large} samples 3 examples from the training set and create a tool for the task, which is further verified by 3 samples from the validation set. The created tool is then applied to all test cases.
    \item \textbf{CREATOR:} CREATOR~\citep{qian2023creator} disentangle planning (tool making) from execution, enabling Large Language Models (LLMs) to autonomously create a specific tool for each test case during inference.
     \item \textbf{CRAFT:} CRAFT~\citep{yuan2023craft} constructs task-specific toolsets by generating a tool for each training example. During testing, it utilizes a tool retrieval module and a reasoning process akin to CREATOR, generating a function first and then producing the corresponding invocation code. 
      % \item \textbf{Trove:} Trove~\citep{wang2024trove} introduces a training-free method based on self-consistency, where LMs interact with the toolbox through three modes—IMPORT, SKIP, and CREATE. Each mode is executed K times, and from the 3K responses, the function from the most consistent and optimal response is added to the toolbox.
      \item \textbf{REGAL:} During training, REGAL~\citep{stengel2024regal} refines primitive programs by extracting functions. In the testing phase, it retrieves both tools and refactored programs—comprising original and refactored versions—to generate a program that effectively solves the task. 
\end{itemize}
\paragraph{Metric}
We use correctness as the evaluation metric, measuring whether the execution outcome of the solution program exactly matches the ground-truth answer(s).
\paragraph{Model}
The models for the single-turn code generation task are the same as those used for the Agent Task, as presented in Section \ref{subsec:agent}.
\paragraph{Setting}
To ensure a fair comparison, we make slight adjustments to each method. For all methods, we allow up to 3 times for format checking and correction, as small models may not always follow the required output format. For PoT, we use 6 fixed examples of basic tool usage as few-shot. CREATOR employs the rectifying process, while for CRAFT, we use the same training set as our method and construct the tool library with GPT-4o, retrieving 3 tools during testing. For Regal, we use PoT along with GPT-4o to obtain ground-truth code, select the correct program, and have GPT-4o reconstruct it. To maintain fairness in tool generation quality, we standardize the few-shot examples of basic tools and retrieve 3 tools, along with 3 usage examples from the current tool library, avoiding errors from pruned tools. For our method, we train with GPT-4o, retrieving 3 tools and their corresponding usage examples during testing, while fixing the basic tool few-shot examples to 3, ensuring consistency with PoT’s total few-shot count.
\section{More Results}
\label{app:apresults}
\subsection{Open-ended Task}
\label{subsec:open-results}
\paragraph{More complex tools} 
Our hierarchical graph architecture offers significant advantages in handling complex tasks and large-scale systems. As shown in Figure \ref{fig:toolnet1}, Trial 1 starts with five nodes occupying three layers, and evolves into a five-layer network, with an increasing number of inter-tool calls. As shown in Figure \ref{fig:toolnet2}, Trial 2 starts with four nodes occupying four layers, and evolves into a five-layer network with more inter-tool calls. As shown in Figure \ref{fig:toolnet3}, Trial 3 starts with four nodes occupying three layers, and evolves into a six-layer network structure, with a growing number of inter-tool calls. Our tool graph becomes progressively more complex, flexibly expanding and optimizing its components. These results demonstrate that our method can generate tools that call each other, and combine them into more complex tools. This not only enhances scalability but also facilitates the creation of more sophisticated tools, enabling the solution of increasingly complex problems.


\paragraph{More types of inventory} Our method is able to generate more inventory types than Voyager. As shown in Table \ref{tab:Number}, we can see that {\ours} produces more inventory types in all three trials compared to Voyager.

The inventory collected by {\ours} in each trial is

\begin{itemize}
    \item \textbf{Trial 1:}  \textit{oak\_log, birch\_log, oak\_planks, birch\_planks, crafting\_table, stick, wooden\_pickaxe, dirt, cobblestone, coal, stone\_pickaxe, raw\_copper, furnace, copper\_ingot, andesite, raw\_iron, granite, iron\_ingot, iron\_pickaxe, shield, diorite, raw\_gold, lapis\_lazuli, redstone, diamond, diamond\_pickaxe, bucket, gold\_ingot, iron\_chestplate, arrow, iron\_sword, iron\_helmet, diamond\_sword, diamond\_helmet, lightning\_rod, chest, iron\_axe, iron\_leggings, sandstone, dandelion, spider\_eye, string, iron\_shovel, copper\_block, iron\_door, iron\_hoe, kelp, bow, dried\_kelp, torch, cooked\_beef, gray\_wool, cobbled\_deepslate, tuff, diamond\_leggings, bone, diamond\_chestplate, chicken, white\_banner, cooked\_chicken, egg, feather, oak\_sapling, apple, acacia\_log, golden\_apple, diamond\_axe}

    \item \textbf{Trial 2:}  \textit{oak\_sapling, oak\_log, stick, oak\_planks, crafting\_table, wooden\_pickaxe, dirt, cobblestone, stone\_pickaxe, diorite, raw\_iron, coal, lapis\_lazuli, gravel, furnace, iron\_ingot, raw\_copper, sandstone, granite, iron\_pickaxe, andesite, raw\_gold, gold\_ingot, diamond, diamond\_pickaxe, redstone, cobbled\_deepslate, bucket, iron\_sword, arrow, bow, bone, birch\_log, chest, amethyst\_block, calcite, smooth\_basalt, iron\_chestplate, diamond\_sword, diamond\_helmet, iron\_leggings, diamond\_boots, water\_bucket, string, orange\_tulip, mutton, white\_wool, porkchop, dandelion, cooked\_porkchop, cooked\_mutton}

    \item \textbf{Trial 3:}  \textit{jungle\_log, stick, oak\_sapling, jungle\_planks, crafting\_table, dirt, wooden\_pickaxe, cobblestone, stone\_pickaxe, raw\_iron, raw\_copper, furnace, iron\_ingot, iron\_pickaxe, coal, diorite, lapis\_lazuli, andesite, moss\_block, clay\_ball, redstone, raw\_gold, cobbled\_deepslate, granite, diamond, diamond\_pickaxe, copper\_ingot, gunpowder, bucket, gravel, gold\_ingot, oak\_log, iron\_sword, iron\_chestplate, chest, diamond\_sword, spruce\_sapling, rotten\_flesh, bone, rose\_bush, water\_bucket, string, oak\_planks, grass\_block, diamond\_helmet, iron\_leggings, emerald, snowball, rabbit\_hide, rabbit, spruce\_log, cooked\_rabbit, diamond\_boots}
\end{itemize}


The inventory collected by Voyager in each trial is
\begin{itemize}
    \item \textbf{Trial 1:}  \textit{oak\_log, birch\_log, oak\_sapling, birch\_sapling, oak\_planks, stick, crafting\_table, wooden\_pickaxe, dirt, cobblestone, stone\_pickaxe, raw\_copper, white\_tulip, coal, furnace, copper\_ingot, granite, raw\_iron, iron\_ingot, lightning\_rod, iron\_pickaxe, pink\_tulip, orange\_tulip, sandstone, shears, shield, diorite, cobbled\_deepslate, iron\_block, chest, tuff, lapis\_lazuli, redstone, diamond, raw\_gold, gold\_ingot, diamond\_pickaxe, diamond\_helmet, diamond\_sword, sand, andesite, arrow, bone, iron\_chestplate, beef, leather, oak\_leaves, porkchop, cooked\_beef, leather\_leggings}

    \item \textbf{Trial 2:}  \textit{dirt, oak\_log, oak\_planks, crafting\_table, stick, oak\_sapling, wooden\_pickaxe, cobblestone, coal, stone\_pickaxe, raw\_iron, granite, lapis\_lazuli, raw\_copper, furnace, iron\_ingot, copper\_ingot, iron\_helmet, iron\_pickaxe, diorite, andesite, salmon, ink\_sac, iron\_chestplate, lightning\_rod, cooked\_salmon, stone, stonecutter, rotten\_flesh, gravel, flint, chest, iron\_leggings, copper\_block, cobbled\_deepslate, tuff, diamond, diamond\_pickaxe, raw\_gold, gold\_ingot, redstone, diamond\_sword, egg, diamond\_boots, diamond\_axe}

    \item \textbf{Trial 3:}  \textit{jungle\_log, jungle\_planks, oak\_sapling, oak\_log, crafting\_table, stick, wooden\_pickaxe, dirt, cobblestone, coal, stone\_pickaxe, raw\_copper, furnace, copper\_ingot, magma\_block, lightning\_rod, stone\_axe, jungle\_boat, kelp, sand, sandstone, glass, raw\_iron, granite, lapis\_lazuli, diorite, iron\_ingot, bucket, iron\_pickaxe, chest, andesite, redstone, dried\_kelp, iron\_chestplate, wooden\_sword, shield, iron\_sword}
\end{itemize}

\vskip -0.2in
\begin{table}[H]
\caption{Number of different inventory types produced by each trial}
\label{tab:Number}
% \vskip 0.1in
\setlength{\tabcolsep}{12pt} % 调整列间距
\renewcommand{\arraystretch}{1.0} % 调整行间距
\begin{center}
% \resizebox{\textwidth}{!}{ % 自动调整表格宽度以适应页面
\begin{small}
\begin{sc}
\begin{tabular}{lccc} % 确保列数与标题一致
\toprule
\textnormal{\textbf{Method}} & \textnormal{\textbf{Trial 1}} & \textnormal{\textbf{Trial 2}} & \textnormal{\textbf{Trial 3}}  \\
\midrule
\normalfont Voyager     & 50  & 45  & 37    \\
\normalfont AETG(Ours)  & 67  & 51  & 53    \\
\bottomrule
\end{tabular}
\end{sc}
\end{small}
% }
\end{center}
\vskip -0.1in
\end{table}


\paragraph{Longer exploration path} To better demonstrate the exploration capabilities of the agent, we compared the exploration trajectories and their lengths. As shown in Figure \ref{fig:linermap}, our agent exhibits longer and more persistent exploration capabilities than Voyager. In Table \ref{tab:length}, the trajectory lengths of our agent are consistently much greater than those of Voyager. {\ours}is able to traverse across multiple terrains, with an average distance 2.66 times longer than Voyager. Additionally, {\ours} can explore across different continental plates, while Voyager remains confined to a single plate, highlighting the exceptional exploration capability of {\ours}.

% \vskip -0.2in
\begin{table}[H]
\caption{Exploration trajectory length in each trial, where \textit{Performance Gain} = $\textit{ours}/\textit{voyager}$.}
\label{tab:length}
% \vskip 0.1in
\setlength{\tabcolsep}{12pt} % 调整列间距
% \renewcommand{\arraystretch}{1.0} % 调整行间距
\begin{center}
% \resizebox{\textwidth}{!}{ % 自动调整表格宽度以适应页面
\begin{small}
\begin{sc}
\begin{tabular}{lcccc} % 确保列数与标题一致
\toprule
\textnormal{\textbf{Method}} & \textnormal{\textbf{Trial 1}} & \textnormal{\textbf{Trial 2}} & \textnormal{\textbf{Trial 3}} & \textnormal{\textbf{\textit{Avg}}}\\
\midrule
\normalfont Voyager     & 1925.74  & 4102.99  & 902.13  & 2310.29   \\
\normalfont {\ours}(Ours)  & 5665.75  & 8908.57  & 3895.06 & 6156.46  \\
\midrule
\normalfont \textit{Performance Gain} & 2.94  & 2.17   & 4.32    & 2.66 \\
\bottomrule
\end{tabular}
\end{sc}
\end{small}
% }
\end{center}
\vskip -0.1in
\end{table}


\vskip -0.2in
\begin{figure}[H]
\vskip 0.2in
\begin{center}
\centerline{\includegraphics[width=1\linewidth]{trial-map.png}}
% \vskip -0.2in
\caption{Map coverage: Three bird’s eye views of Minecraft maps. The trajectories are plotted based on the position coordinates where each agent interacts.}
\label{fig:trialmap}
\end{center}
\vskip -0.3in
\end{figure}


\vskip -0.2in
\begin{figure}[H]
\vskip 0.2in
\begin{center}
\centerline{\includegraphics[width=1\linewidth]{liner-map.png}}
% \vskip -0.2in
\caption{Movement trajectory Map: Three bird’s eye views of Minecraft maps. The trajectories are plotted based on the position coordinates where each agent interacts.}
\label{fig:linermap}
\end{center}
\vskip -0.3in
\end{figure}



\paragraph{Efficient Zero-Shot Generalization to Unseen Tasks} Based on the results presented in Table \ref{tab:newtechtree} and Figure \ref{fig:diamon and compass}, we can clearly observe the significant advantages of {\ours} in the open-ended task. Table \ref{tab:newtechtree} shows the number of iterations required for different methods to complete various tasks (Gold Sword, Compass, Diamond Hoe, Lava Bucket), where fewer iterations indicate higher efficiency. Compared to Voyager and {\ours} (w/o toolnet), {\ours} consistently requires significantly fewer iterations across all tasks, demonstrating substantial improvements in efficiency. Notably, in the Gold Sword task, {\ours} (ours) completes the task in just 14.00±1.73 iterations, whereas Voyager requires 46.33±14.57 iterations, showcasing its superior performance.

Figure \ref{fig:diamon and compass} further visualizes the intermediate progress of different methods on the "Craft a Compass" and "Craft a Diamond Hoe" tasks. It is evident that {\ours} learns and masters the necessary skills for crafting items more quickly. As the number of prompting iterations increases, {\ours} reaches the task objectives significantly earlier than the other methods. Additionally, while {\ours}(w/o Tool Graph) performs better than Voyager, it still lags behind {\ours}, indicating that the ToolNet component plays a crucial role in enhancing the model's capability.

Overall, these experimental results demonstrate that {\ours} not only learns new skills and crafting techniques more efficiently but also that its key module, Tool Graph, is essential for overall performance improvement. This further validates the effectiveness of our approach in self-driven exploration and task generalization.


\begingroup
\begin{table}[H]
\caption{The mastery of the tech tree in the Open-ended Task. The number indicates the number of iterations. The fewer the iterations, the more efficient the method. "N/A" indicates that the number of iterations for obtaining the current type of tool is not available.}
\label{tab:newtechtree}
\vskip 0.1in
\setlength{\tabcolsep}{12pt} % 调整列间距
% \renewcommand{\arraystretch}{1.0} % 调整行间距
\begin{center}
% \resizebox{\textwidth}{!}{ % 自动调整表格宽度以适应页面
\begin{small}
\begin{sc}
\begin{tabular}{lccccc} % 确保列数与标题一致
\toprule
\textnormal{\textbf{Method}} & \textnormal{\textbf{Trial}} & \textnormal{\textbf{Gold Sword}} & \textnormal{\textbf{Compass}} & \textnormal{\textbf{Diamond Pickaxe}} & \textnormal{\textbf{Lava Bucket}} \\
\midrule
\multirow{4}{*}{\multirow{2}{*}{\normalfont Voyager}} 
              & \normalfont Trial 1 & 48 & 16 &  24 & N/A         \\
              & \normalfont Trial 2 & 31 & 17 &  25 & 39         \\
              & \normalfont Trial 3 & 60 & 20 & 18  & N/A         \\
              \cmidrule{2-6}
              & \textit{Average} & 46.33$\pm$14.57 & 17.67$\pm$2.08 & 22.33$\pm$3.79 & 39.00$\pm$0.00 \\
\midrule
\multirow{4}{*}{\multirow{2}{*}{\normalfont {\ours}\textit{\small(w/o toolnet)}}} 
               & \normalfont Trial 1 & 26 & 27 & 23  & N/A         \\
              & \normalfont Trial 2 & 18 & 22 & 18  & N/A        \\
              & \normalfont Trial 3 & 56 & 15 & 30  & N/A          \\
              \cmidrule{2-6}
              & \textit{Average} & 33.33$\pm$20.03 & 21.33$\pm$6.03 & 23.67$\pm$6.03 & N/A$\pm$N/A \\
\midrule
\multirow{4}{*}{\multirow{2}{*}{\normalfont {\ours}\textit{\small(ours)}}} 
              & \normalfont Trial 1 & 13 & 28 & 16  & 19       \\
              & \normalfont Trial 2 & 13 & 10 & 14  & 27       \\
              & \normalfont Trial 3 & 16 & 13  & 13  & 18      \\
              \cmidrule{2-6}
              & \textit{Average} & \textbf{14.00$\pm$1.73} & \textbf{17.00$\pm$9.64} & \textbf{14.33$\pm$1.53} & \textbf{21.33$\pm$4.93} \\
             

\bottomrule
\end{tabular}
\end{sc}
\end{small}
% }
\end{center}
\vskip -0.1in
\end{table}
\endgroup



\begin{figure}[H]
\vskip 0.2in
\begin{center}
\centerline{\includegraphics[width=1\linewidth]{compass_and_diamond.png}}
% \vskip -0.2in
\caption{Zero-shot generalization to unseen tasks. Here, we visualize the intermediate progress of each method on the tasks "Craft a Compass" and "Craft a Diamond Hoe."}
\label{fig:diamon and compass}
\end{center}
\vskip -0.3in
\end{figure}



\subsection{Agent Task}
\label{subsec:agent-results}

Figures \ref{fig:toolnet-dabench} and \ref{fig:toolnet-textcraft} present the tool network evolution diagrams of DA-Bench and TextCraft, which visually reflect the call relationships between different tool functions. In these diagrams, each node represents a specific tool function, edges indicate the call dependencies between tools, and the shading of the nodes reflects the frequency of tool calls—darker colors indicate higher call frequency. From Figure \ref{fig:toolnet-dabench}, it can be observed that in DA-Bench, the tool network expands progressively as the task advances, forming multiple core nodes with higher call frequencies. This suggests that certain key tools are frequently called during the task execution, playing a central role. Additionally, the tool call relationships exhibit a hierarchical and well-organized structure, reflecting DA-Bench's efficiency in tool dependency management.

In contrast, Figure \ref{fig:toolnet-textcraft} illustrates the tool network evolution of TextCraft, which also shows a similar expansion trend overall. However, compared to DA-Bench, the tool call frequency in TextCraft is more evenly distributed across multiple nodes, meaning that the system calls a wider variety of tools during task execution, rather than relying on a few core tools. This distribution pattern may suggest that TextCraft adopts a more diverse tool usage strategy in task execution.

A comparative analysis of the two figures reveals that, although both DA-Bench and TextCraft exhibit certain hierarchical and expansive characteristics in their tool call patterns, DA-Bench relies more heavily on a few core tools, whereas TextCraft displays a more dispersed tool call pattern. This contrast not only highlights the differences in tool usage between the two, but also emphasizes the importance and effectiveness of ToolNet.





\subsection{Single-turn Code Task}
\label{subsec:code-results}

As shown in the Figure\ref{fig:toolnet-math} \ref{fig:toolnet-tabmwp}, this illustrates the evolution of the tool graph for the Math and TabMWP tasks. It is evident that the tool graph gradually becomes more complex, creating multiple layers of tools, making the tool graph more intricate. Since the Date task can be solved with fewer tools, there is no evolution of the tool graph. However, the generated tools can still effectively solve the task, while there exists a multi-level calling relationship.


\section{More Ablations}
\label{app:apablation}
\subsection{Open-ended Task}
\label{subsec:open-ablation}

As shown in Figure \ref{fig:ablation}, AETG significantly outperforms methods that lack certain functional modules in discovering new Minecraft items and skills. It can be observed that the performance is worst when "w/o retrieval" is used, indicating that the absence of retrieval has the greatest impact on overall functionality and plays a crucial role, thereby validating the effectiveness of our retrieval method. The performance with "w/o duplication" is slightly better, indicating its importance is weaker than that of "w/o retrieval." The performance of "w/o check" and "w/o pruning" is better, but still far behind AETG, which further demonstrates the importance and effectiveness of each functional component.

\vskip -0.1in
\begin{figure}[H]
% \vskip 0.2in
\begin{center}
\centerline{\includegraphics[width=0.6\linewidth]{toolnumber-ablation.png}}
% \vskip -0.2in
\caption{Ablation study of the iterative prompting mechanism. AETN surpasses all other options, highlighting the essential significance of each functional module in the iterative prompting mechanism.}
\label{fig:ablation}
\end{center}
\vskip -0.3in
\end{figure}


\subsection{Closed-Ended Task}
\label{subsec:closed-ended}
For the Closed-Ended Task, we select Textcraft from the Agent Task and Date from the Single-turn Code Task to evaluate the effectiveness of several components in our method. The results are shown in the Table \ref{tab:closed-toolnumber}.

\begingroup
\begin{table}[H]
\caption{The number of tools in Close-Ended Task.}
\label{tab:closed-toolnumber}
\vskip -0.1in
\setlength{\tabcolsep}{10pt} % 调整列间距
\begin{center}
\begin{small}
\begin{sc}
\begin{tabular}{l|cc}
\toprule
\textnormal{\textbf{Method}} & \textnormal{\textbf{TextCraft}}  & \textnormal{\textbf{Date}} \\
\midrule         

\normalfont W/o Self-Check & 42 & 9 \\
\midrule  
\normalfont W/o Merging & 49 & 11\\
\midrule  
\normalfont W/o pruning & 46 & 9 \\
\midrule  
\normalfont GATE & 44 & 4 \\


\bottomrule
\end{tabular}
\end{sc}
\end{small}
\end{center}
\vskip -0.1in
\end{table}
\endgroup

\section{Tool Making}
\label{app:toolgarph}
\subsection{Basic Tools}
\label{subsec:basic-tools}
As shown in the Table \ref{tab:basictool} , the basic tools generated by each method are displayed.

\begingroup
\begin{table}[H]
\caption{Basic tools in various methods.}
\label{tab:basictool}
\vskip -0.1in
\setlength{\tabcolsep}{10pt} % 调整列间距
\begin{center}
\begin{small}
\begin{sc}
\begin{tabular}{l|p{12cm}}
\toprule
\textnormal{\textbf{Tasks}} & \textnormal{\textbf{Basic Tools}}  \\
\midrule         

\normalfont Other Tasks & \normalfont ToolRequest, NotebookBlock, Terminate, CreateTool, EditTool, Python, Feedback, SendAPI, Feedback, Retrieval \\
\midrule  
\normalfont Minecraft & \normalfont smeltItem, killMob, waitForMobRemoved, givePlacedItemBack, useChest, exploreUntil, craftItem, mineBlock, shoot, placeItem, craftHelper, smeltItem, mineflayer, killMob, useChest, exploreUntil, craftItem, mineBlock, placeItem \\

\bottomrule
\end{tabular}
\end{sc}
\end{small}
\end{center}
\vskip -0.1in
\end{table}
\endgroup


\subsection{Tool construction Lists}
\label{subsec:tool construction}

\paragraph{CREATOR:}
\begin{itemize}[noitemsep, topsep=0pt]
    \item \textbf{MATH:}  \textit{sum of areas, find largest won matches, find K, total distance after bounces, find common ratio sum, count lattice points with distance squared, find c for radius, find circle equation and constants, polynomial degree product, calculate cells, find fiftieth term, find non domain values, inverse function product, find m and n, sum of fractions from roots, find roots of quadratic, main, find coefficients, compute expression, prime factors, find x y, find second largest angle, find y coordinate, find constants, evaluate expression, find b for one solution, find c, find minimum value, find possible s, solve expression, find cone height, solve abc, find minimum expression, \dots, time to hit ground, sum of reciprocals of roots, solve x floor x product, sum of possible x, find constant a, sum of squares of solutions, find cost per extra hour, is triangular number, find smallest b greater than 2011, solve exponential equation, solve club suit equation, find degree of h, f, find vertical asymptotes, domain width, maximize revenue, future value, total savings, find min interest rate, equation, find integers, sum of x coordinates squared, find integer values of a, smallest c for real domain, smallest integer c, find m, required investment, simplify expression, g, distance between midpoints, compute x and power, greatest possible a, find continued fraction value, find a b, solve mnp, compute sum, sum of integers in range,
    }

    \item \textbf{Date:}  \textit{get us thanksgiving date, get date one week from first monday of 2019, calculate anniversary date, calculate yesterday from last day of january, calculate one week ago from first monday, get first monday of 2019, calculate yesterday, calculate yesterday from rescheduled meeting, calculate date a month ago from rescheduled meeting, calculate yesterday from first monday of 2019, get date 10 days before us thanksgiving, calculate one week ago from egg runout, calculate one week ago from end of first quarter, calculate date 24 hours later, calculate date a month ago, calculate date 24 hours after anniversary, calculate one week from today from rescheduled meeting, \dots, get tomorrow from us thanksgiving, calculate yesterday from day before yesterday, calculate yesterday from anniversary, calculate date 10 days ago, calculate one year ago from egg run out date, calculate tomorrow from yesterday, calculate one week from last day of january, calculate one week from anniversary, calculate yesterday from eggs run out, calculate tomorrow from today, calculate tomorrow from day before yesterday, calculate one week ago from today, calculate one week ago, calculate date one month ago from anniversary, calculate one year ago from given date, calculate one week from given date}

    \item \textbf{TabMWP:}  \textit{calculate total cost, smallest points, price difference, cost of river rafts, calculate median, calculate range, calculate total spent, rate of change, cost difference, cost for rides, rate of change vacation days, total participants, calculate mean glasses, find mode of states visited, rate of change straight A students, calculate median basketball hoops, count bins with toys in range, people with at least 3 trips, count teams with fewer than 80 swimmers, calculate median clubs, count exact pushups, children with less than 2 necklaces, people played exactly 3 times, count people with fewer than 80 pullups, range of states visited, find spent amount, \dots, calculate median miles, people with fewer than 3 seashells, calculate median glasses, cost to buy cockatiels, largest broken lights, calculate spent, calculate ice cream cost, range of soccer fields, patrons with at least 2 books, count bushes with 20 roses, total people played golf, range of articles, count shipments with exactly 60 broken plates, total cost for lip balms, rate of change scholarships, count teams with fewer than 50 members, count tests with 34 problems, find mode of soccer fields, rate of change hockey games, find lowest score, count pizzas with exactly 48 pepperoni, count people with at least 30 points, cost of wooden benches, rate of change students, patients with fewer than 2 trips, find mode, total cost for hazelnuts, calculate mean fan letters, readers with at least 4 hats, count classrooms with 41 desks}
\end{itemize}

\paragraph{CRAFT:}
\begin{itemize}[noitemsep, topsep=0pt]
    \item  \textbf{MATH:}  \textit{find pack size, count distinct solutions, calculate points, find tank capacity, solve exponential log equation, total energy equilateral triangle, inverse square law force, find max value, total logs in stack, sum of multiples of 13, calculate exponential growth, gravitational force, find x for piecewise composition, positive difference, specific piecewise func, day exceeds 200 cents, find lattice points, count integer parameters for integer solutions, count zeros in square of power of ten minus one, energy stored, sum of squares of roots, sum odd integers, find d minus e squared, compute complex series sum, total energy configuration, sum of areas, \dots, max item price, solve two variable system, inverse variation power, total distance hopped, is prime, total distance, find constant term of polynomial, total distance moved, find perpendicular slope, calculate inverse proportionality, find value of A, count integer a, find min items for higher score, apply r n times, find min x, day exceeds threshold, calculate area in square yards, solve log equation, total items produced, find variable for distance condition, solve time at speeds, find largest solution, find weight of object, calculate proportional value, calculate material cost, solve for variable, total elements in arithmetic sequence, transformed domain, find day for algae coverage, calculate energy stored, least value of y, solve bowling ball weight, find min froods}

    \item \textbf{Date:} \textit{get today date, calculate one week ago, calculate n days from future date, calculate n days from date in format, calculate date days ago, calculate n months from date, calculate one week from today, calculate date after event, find palindrome day, calculate date a month ago, calculate date after days and months, calculate relative date, calculate n days from reference, calculate one year ago from today, calculate n hours from date, calculate date n days from, get date today, calculate date 10 days ago from deadline, calculate n weeks from date, \dots, calculate n units from date, calculate n years from date, calculate n weeks from first weekday of year, calculate today from tomorrow, find special day, calculate date 10 days ago from future, calculate n days after event, calculate date from days passed, calculate one week from christmas eve, calculate one year ago, calculate date 24 hours later, calculate n weeks from anniversary, calculate tomorrow from uk format date, calculate n days from date, is palindrome, calculate one week from first monday of year, calculate one week ago from anniversary}

    \item \textbf{TabMWP:} \textit{get frequency, calculate volleyballs in lockers, calculate total cost from package prices, calculate total items from group counts, calculate mode, calculate donation difference for person, count bags with 20 to 40 broken cookies, calculate total items from groups and items per group, count commutes of 50 minutes, get received amount, calculate total items for groups, find probability, calculate vacation cost, calculate rate of change, find received amount for transaction, calculate vote difference between two items for group, count customers, find minimum value in stem leaf, calculate metric wrenches, find smallest number, count books with 30 to 50 characters, \dots, count people with 67 pullups, calculate difference in donations for person, calculate total cost from unit price and weight, calculate total items from ratio, calculate total cost from unit weight prices and weight, calculate donation difference between causes, calculate difference, calculate net income, calculate grasshoppers on twigs, count total members in group, calculate expenses on date, find lightest child, calculate difference in amounts, count votes for item from groups, calculate probability from count table, get table cell value, calculate jeans in hampers, count instances with specific value in stem leaf, calculate donation difference for person and causes, calculate total from frequency and additional count, calculate range, calculate total reviews}
\end{itemize}


\paragraph{REGAL:}
\begin{itemize}[noitemsep, topsep=0pt]
    \item \textbf{MATH:}  \textit{solve for largest side, apply function sequence, solve rational equation, calculate expression sum, max sum of products, find b for perpendicular bisector, vertex of quadratic, calculate work days, calculate c for zero coefficient, simplify and rationalize sympy, find a for binomial square, compound interest, calculate inverse variation, expand expression, calculate average speed, calculate rs, sum sequence, solve for p, max consecutive integers, find x intercept, day exceeding threshold, find smallest sum, solve for ac pair, constant function, sum of distances, evaluate expression, sum finite geometric series, factor expression, find common difference, total coins pirates, calculate geometric first term, calculate closest whole number, calculate x minus y squared, solve letter values, find circle center v2, evaluate expression with sqrt, calculate sum of equations, \dots, calculate x3 plus y3, find negative intervals, calculate floor and abs, solve quadratic and find min, calculate y, solve for a, check equations, rationalize and simplify, calculate xyz, calculate distance, solve for x in simplified equation, calculate expression, calculate exponent, sum arithmetic series, complete square form, calculate x2 plus y2
    }

    \item \textbf{Date:}  \textit{subtract weeks from date, add weeks to date, format date, add days to date, subtract months from date, subtract days from date, subtract years from date, calculate date, calculate days between weekdays}

    \item \textbf{TabMWP:}  \textit{count range, find mode, total participants, count bushes with fewer roses, find max frequency, total items, count in range, calculate total items, count below threshold, count teams with minimum size, calculate total, calculate range, calculate fraction, sum frequencies below threshold, sum frequencies, calculate difference, calculate median, total outcomes, count specific height, count numbers in range, difference between groups, access frequency, calculate proportionality constant, count values below threshold, find median, calculate probability, calculate mode, get frequency, convert stem leaf to numbers, find minimum, get total items, count scores above, rate of change, calculate mean}
\end{itemize}



\subsection{The tool graph evolution diagrams of {\ours} for various tasks.}
\label{subsec:tool-graph}
Below are the tool graph evolution diagrams for various tasks. The Date task does not have a tool network evolution diagram, as date reasoning does not heavily rely on tool diversity.


\begin{figure}[H]
\vskip 0.2in
\begin{center}
\centerline{\includegraphics[width=1\linewidth]{toolnet-trial1.png}}
% \vskip -0.2in
\caption{
The tool graph evolution diagram for Minecraft Trial 1. In this diagram, each node represents a tool function, and the edges represent the invocation relationships between tools. The darker the color, the more frequently the tool is invoked. The network consists of a total of 6 layers, with layers 2 to 6 shown here from top to bottom.}
\label{fig:toolnet1}
\end{center}
\vskip -0.3in
\end{figure}

\vskip -0.2in
\begin{figure}[H]
\vskip 0.2in
\begin{center}
\centerline{\includegraphics[width=1\linewidth]{toolnet-trial2.png}}
% \vskip -0.2in
\caption{The tool graph evolution diagram for Minecraft Trial 2. In this diagram, each node represents a tool function, and the edges represent the invocation relationships between tools. The darker the color, the more frequently the tool is invoked. The network consists of a total of 6 layers, with layers 2 to 6 shown here from top to bottom.}
\label{fig:toolnet2}
\end{center}
\vskip -0.3in
\end{figure}

\vskip -0.2in
\begin{figure}[H]
\vskip 0.2in
\begin{center}
\centerline{\includegraphics[width=1\linewidth]{toolnet-trial3.png}}
% \vskip -0.2in
\caption{The tool graph evolution diagram for Minecraft Trial 3. In this diagram, each node represents a tool function, and the edges represent the invocation relationships between tools. The darker the color, the more frequently the tool is invoked. The network consists of a total of 6 layers, with layers 2 to 7 shown here from top to bottom.}
\label{fig:toolnet3}
\end{center}
\vskip -0.3in
\end{figure}


\begin{figure}[H]
\vskip 0.2in
\begin{center}
\centerline{\includegraphics[width=1\linewidth]{toolnet-dabench.png}}
% \vskip -0.2in
\caption{The tool graph evolution diagram of DA-Bench. In this diagram, each node represents a tool function, and the edges represent the invocation relationships between tools. The darker the color, the more frequently the tool is invoked.}
\label{fig:toolnet-dabench}
\end{center}
\vskip -0.3in
\end{figure}

\begin{figure}[H]
\vskip 0.2in
\begin{center}
\centerline{\includegraphics[width=1\linewidth]{toolnet-textcraft.png}}
% \vskip -0.2in
\caption{The tool graph evolution diagram of TextCraft. In this diagram, each node represents a tool function, and the edges represent the invocation relationships between tools. The darker the color, the more frequently the tool is invoked.}
\label{fig:toolnet-textcraft}
\end{center}
\vskip -0.3in
\end{figure}


\begin{figure}[H]
\vskip 0.2in
\begin{center}
\centerline{\includegraphics[width=1\linewidth]{toolnet-math.png}}
% \vskip -0.2in
\caption{The tool graph evolution diagram of MATH. In this diagram, each node represents a tool function, and the edges represent the invocation relationships between tools. The darker the color, the more frequently the tool is invoked.}
\label{fig:toolnet-math}
\end{center}
\vskip -0.3in
\end{figure}

\begin{figure}[H]
\vskip 0.2in
\begin{center}
\centerline{\includegraphics[width=1\linewidth]{toolnet-tabmwp.png}}
% \vskip -0.2in
\caption{The tool graph evolution diagram of TabMWP. In this diagram, each node represents a tool function, and the edges represent the invocation relationships between tools. The darker the color, the more frequently the tool is invoked.}
\label{fig:toolnet-tabmwp}
\end{center}
\vskip -0.3in
\end{figure}
\section{Prompt Template}
\label{app:prompt}
In this section, we provide the prompt templates of different types used throughout our experiment. These prompts were carefully crafted to ensure that the model's output aligns with the specific objectives of each task.

\subsection{Construction Stage}
In open-ended task online training, we made slight modifications to their prompts based on Voyager~\citep{wang2023voyager}. For close-ended tasks, the prompts used during the construction process are as follows:
\begin{tcolorbox}[title=Task Solver's Prompt, breakable, width=\textwidth,top=0mm]
\begin{Verbatim}[breaklines, fontsize=\footnotesize]
# Instruction #
You are the Task Solver in a collaborative team, specializing in reasoning and Python programming. Your role is to analyze tasks, collaborate with the Tool Manager, and solve problems step by step.
Directly solving tasks without tool analysis is not allowed. Request necessary tools before proceeding when needed, based on the task analysis.

# WORKFLOW #
You can decide which step to take based on the environment and current situation, adapting dynamically as the task progresses.
Stage 1. Tool Requests:
    Requesting tool is mandatory. Request generalized and reusable tools to solve the task. Focus on abstract functionality rather than task-specific details to enhance flexibility and adaptability.
Stage 2. Code and Interact: 
    Write notebook blocks incrementally, executing and interacting with the environment step by step. Avoid bundling all steps into a single block; instead, adjust dynamically based on feedback after each interaction.
Stage 3: Validate and Conclude: 
    When confident in the solution, review your work, validate the results, and conclude the task.

# Custom Library #
===api===

# NOTICE #
1. You must fully understand the action space and its parameters before using it.
2. If code execution fails, you should analyze the error and try to resolve it. If you find that the error is caused by the API, please promptly report the error information to the Tool Manager.
3. Regardless of how simple the issue may seem, you should always aim to summarize and refine the tool requirements.


# Tool Request Guidelines #
1. Keep It Simple: Design tools with single and simple functionality to ensure they are easy to implement, understand, and use. Avoid unnecessary complexity.
2. Define Purpose: Clearly outline the tool’s role within broader workflows. Focus on creating reusable tools that solve abstract problems rather than task-specific ones.
3. Specify Input and Output: Define the required input and expected output formats, prioritizing generic structures (e.g., dictionaries or lists) to enhance flexibility and adaptability.
4. Generalize Functionality: Ensure the tool is not tied to a specific task. Abstract its functionality to make it applicable to similar problems in other contexts.


# ACTION SPACE #
You should Only take One action below in one RESPONSE:
## NotebookBlock Action
* Signature: 
NotebookBlock():
```python
executable python script
```
* Description: The NotebookBlock action allows you to create and execute a Jupyter Notebook cell. The action will add a code block to the notebook with the content wrapped inside the paired ``` symbols. If the block already exists, it can be overwritten based on the specified conditions (e.g., execution errors). Once added or replaced, the block will be executed immediately.
* Restrictions: Only one notebook block can be managed or executed per action.
* Example
- Example1: 
NotebookBlock():
```python
# Calculate the area of a circle with a radius of 5
radius = 5
area = 3.1416 * radius ** 2
print(area)
```

## Tool_request Action
* Signature:
{
    "action_name": "tool_request",
    "argument": {
         "request": [
             ...
         ]
    }
}
* Description: The Tool Request Action allows you to send tool requirements to the Tool Manager and request it to create appropriate tools. You need to provide the action in a JSON format, where the argument field contains a request parameter that accepts a list. Each element in the list is a string describing the desired tool.
* Note:
* Examples:
- Example 1:
{
    "action_name": "tool_request",
    "argument": {
        "request": [
            "I need a tool that calculates the average value of a specified column in a dataset. The input should include the column name."
        ]
    }
}
- Example 2:
{
    "action_name": "tool_request",
    "argument": {
        "request": [
            "I need a tool that filters rows in a dataset based on a specified condition. The input should include the column name and the condition to filter by."
        ]
    }
}


## Terminate Action
* Signature: Terminate(result=the result of the task)
* Description: The Terminate action ends the process and provides the task result. The `result` argument contains the outcome or status of task completion.
* Examples:
  - Example1: Terminate(result="A")
  - Example2: Terminate(result="1.23")

# RESPONSE FORMAT #
For each task input, your response should contain:
1. One RESPONSE should only contain One Stage, One Thought and One Action.
2. An current phase of task completion, outlining the steps from planning to review, ensuring progress and adherence to the workflow.  (prefix "Stage: ").
3. An analysis of the task and the current environment, including reasoning to determine the next action based on your role as a SolvingAgent. (prefix "Thought: ").
4. An action from the **ACTION SPACE** (prefix "Action: "). Specify the action and its parameters for this step.

# RESPONSE EXAMPLE #
Observation: ...(the output of last actions, as provided by the environment and the code output, you don't need to generate it)

Stage:...(One Stage from `WORKFLOW`)
Thought: ...
Action: ...(Use an action from the ACTION SPACE no more than once per response.)

# TASK #
===task===
\end{Verbatim}
\end{tcolorbox}

\begin{tcolorbox}[title=Tool Manager's Prompt, breakable, width=\textwidth,top=0mm]
\begin{Verbatim}[breaklines, fontsize=\footnotesize]
# Instruction #
You are a Tool Manager in a collaborative team, specializing in assembling existing APIs to construct hierarchical and reusable abstract tools based on predefined criteria.
You will be provided with a custom library, similar to Python’s built-in modules, containing various functions related to date reasoning. For each task, you will receive:
1. Tool request: The specific goal or functionality the new tool must achieve.
2. Existing tools: A list of available functions from the custom library that you can utilize.
Your task is to analyze the given request and create a reusable tool by effectively leveraging the relevant functions from the existing tools or utilizing basic tools to achieve the desired functionality. 
If an existing tool from the provided library already fully satisfies the requirements, simply return that tool instead of duplicating functionality. Ensure all responses align with reusability and efficiency principles.

# Custom Library #
===api===

# Creation Criteria #
- **Reusability**: The function could be resued for more complex function.
- **Innovation**: Tools should offer innovation, not merely wrap or replicate existing APIs. Simply re-calling an API without significant enhancements does not qualify as innovation.
- **Completeness**: The function should handle potential edge cases to ensure completeness.
- **Leveraging Existing Functions**: The function should effectively utilize existing functions to enhance efficiency and avoid redundancy.
- **Functionality**: Ensure the tool runs successfully and is bug-free, guaranteeing full functionality.

# ACTION SPACE #
You should Only take One action below in one RESPONSE:
## Create tool Action
* Description: The Create Tool action allows you to develop a new tool and temporarily store it in a private repository accessible only to you. Each invocation creates a single tool at a time. You can repeatedly use this action to build smaller components, which can later be assembled into the final tool.
* Signature: 
Create_tool(tool_name=The name of the tool you want to create):
```python
The source code of tool
```
* Example:
Create_tool(tool_name=“calculate_column_statistics”):
```python
def calculate_column_statistics(dataset: pd.DataFrame, column_name: str) -> Dict[str, float]:
    """
    Calculates basic statistics (mean, median, standard deviation) for a specified column in a dataset.
    Parameters:
    - dataset: A pandas DataFrame containing the data.
    - column_name: The name of the column to calculate statistics for.
    Returns:
    - A dictionary containing the mean, median, and standard deviation of the column.
    """
    if column_name not in dataset.columns:
        raise ValueError(f"Column '{column_name}' not found in the dataset.")
    
    column_data = dataset[column_name]
    stats = {
        "mean": column_data.mean(),
        "median": column_data.median(),
        "std_dev": column_data.std()
    }
    return stats
```
## Edit tool Action
* Description: The Edit Tool action allows you to modify an existing tool and temporarily store it in a private repository that only you can access. You must provide the name of the tool to be updated along with the complete, revised code. Please note that only one tool can be edited at a time.
* Signature: 
Edit_tool(tool_name=The name of the tool you want to create):
```python
The edited source code of tool
```
* Examples:
Edit_tool(tool_name="filter_rows_by_condition"):
```python
def filter_rows_by_condition(dataset: pd.DataFrame, column_name: str, condition: str) -> pd.DataFrame:
    """
    Filters rows in a dataset based on a specified condition for a given column.
    Parameters:
    - dataset: A pandas DataFrame containing the data.
    - column_name: The name of the column to apply the condition to.
    - condition: A string representing the condition, e.g., 'value > 10'.
    Returns:
    - A filtered DataFrame containing only the rows that satisfy the condition.
    """
    if column_name not in dataset.columns:
        raise ValueError(f"Column '{column_name}' not found in the dataset.")
    
    try:
        filtered_dataset = dataset.query(f"{column_name} {condition}")
    except Exception as e:
        raise ValueError(f"Invalid condition: {condition}. Error: {e}")
    
    return filtered_dataset
```

# RESPONSE FORMAT #
For each task input, your response should contain:
1. Each response should contain only one "Thought," and one "Action."
2. Determine how to construct your tool to meet tool request and function creation criteria. Check if any functions in the Existing Tool can be invoked to assist in the tool’s development and ensure alignment with the criteria.(prefix "Thought: ").
3. An action dict from the **ACTION SPACE** (prefix "Action: "). Specify the action and its parameters for this step. 

# RESPONSE EXAMPLE  #
1. If you determine that the tool request cannot be solved using existing tools, choose this mode to provide a clear and complete code solution.

Thought: ...
Action: ...

2. If you determine that the tool request is already satisfied by existing tools, choose this mode to directly reference and return the relevant tool without creating additional solutions.
Thought: ...
Tool: {  
    "tool_name": "Name of Existing tools"
}

# NOTICE #
1. You can directly call and use the tool in the custom library in your code or tool without importing it.
2. You can only create or edit one tool per response, so take it one step at a time.

# TASK #
===task===
\end{Verbatim}
\end{tcolorbox}


\begin{tcolorbox}[title=Prompt of Self-Check Step 1, breakable, width=\textwidth,top=0mm]
\begin{Verbatim}[breaklines, fontsize=\footnotesize]
# Instruction #
You are evaluating whether the tools provided by the Tool Manager meet the required standards. 
You follow a defined workflow, take actions from the ACTION SPACE, and apply the evaluation criteria. 

# Evaluation Criteria #
- **Reusability**: The function should be designed for reuse in more complex scenarios. For instance, in the case of the `craft_wooden_sword()` tool, it would be more versatile if it could accept a quantity as an input parameter.
- **Innovation**: Tools should offer innovation, not merely wrap or replicate existing APIs. Simply re-calling an API without significant enhancements does not qualify as innovation. If an existing tool from the provided library already fully satisfies the requirements, simply return that tool instead of duplicating functionality. Ensure all responses align with reusability and efficiency principles.
- **Completeness**: The function should handle potential edge cases to ensure completeness.
- **Leveraging Existing Functions**: Check if any function in "Existing Function" is helpful for completing the task. If such functions exist but are not invoked in the provided code, relevant feedback should be given.

## Tool Abstraction ##
Tool abstraction is essential for enabling tools to adapt to diverse tasks. Key principles include:
- Design generic functions to handle queries of the same type, based on shared reasoning steps, avoiding specific object names or terms.
- Name functions and write docstrings to reflect the core reasoning pattern and data organization, without referencing specific objects.
- Use general variable names and pass all column names as arguments to enhance adaptability.

# ACTION SPACE #
You should Only take One action below in one RESPONSE:
# Feedback Action
* Signature: {
    "action_name": "Feedback",
    "argument": {
        "feedback": ...
        "passed": true/false
    }
}
* Description: The Feedback Action is represented as a JSON string that provides feedback to the Tool Manager or SolvingAgent. The feedback field contains comments or suggestions, while pass indicates whether the tool meets the requirements (true for approval, false for rejection). Feedback should be concise, constructive, and relevant. If pass is true, the feedback can be left empty; otherwise, it must be provided.
* Example:
- Example1:
{
    "action_name": "Feedback",
    "argument": {
        "feedback": "",
        "passed": true
    }
}
- Example2:
{
    "action_name": "Feedback",
    "argument": {
        "feedback": "The tool correctly solves the equation for small numbers, but fails when the coefficients are very large. Consider optimizing the algorithm for handling larger values and improving computational efficiency.",
        "passed": false
    }
}

# RESPONSE FORMAT #
For each task input, your response should contain:
1. One RESPONSE should ONLY contain One Thought and One Action.
2. An comprehensive analysis of the tool code based on the evaluation criteria.(prefix "Thought: ").
3. An action from the **ACTION SPACE** (prefix "Action: "). 

# EXAMPLE RESPONSE #
Observation: ...(output from the last action, provided by the environment and task input, no need for you to generate it)

Thought: 1. Reusability: ...
2. Innovation: ...
3. Completeness: ...
4. Leveraging Existing Functions: ...

Action: ...(Use an action from the ACTION SPACE once per response.)

# Custom Library #
===api===

# TASK #
===task===
\end{Verbatim}
\end{tcolorbox}

\begin{tcolorbox}[title=Prompt of Self-Check Step 2, breakable, width=\textwidth,top=0mm]
\begin{Verbatim}[breaklines, fontsize=\footnotesize]
# Instruction #
You are verifying whether the tools provided by the Tool Manager execute without runtime errors.
You will use a custom library, similar to the built-in library, which provides everything necessary for the tasks. Your task is only to execute the provided tool code and check for runtime errors, not to evaluate the tool’s functionality or correctness.

# Stage and Workflow #
1. **Ensure Bug-Free Tool Operation**:
	- Execute the tool to ensure it runs without any runtime bugs.
	- You don’t need to verify the function’s functionality; simply call it to check for any runtime errors.
	- If the tool is a retrieved API, skip this step and proceed.
2. **Send Feedback**:
	- After executing the code, provide feedback based on the output, indicating whether the operation was successful or not.

# Notice #
1. If any issues with the tool are found, promptly provide clear and critical feedback to the Tool Manager for resolution. 
2. You should not create or edit functions (tools) with the same name as the Existing Functions in the code.
3. You can directly call the APIs from the custom library without needing to import or declare any external libraries.
4. You don’t need to verify the function’s functionality or set up its standard output; simply call it to check for any errors.

# ACTION SPACE #
You should Only take One action below in one RESPONSE:
## Python Action
* Signature: 
Python(file_path=python_file):
```python
executable_python_code
```
* Description: The Python action will create a python file in the field `file_path` with the content wrapped by paired ``` symbols. If the file already exists, it will be overwritten. After creating the file, the python file will be executed. Remember You can only create one python file.
* Examples:
- Example1
Python(file_path="solution.py"):
```python
# Calculate the area of a circle with a radius of 5
radius = 5
area = 3.1416 * radius ** 2
print(f"The area of the circle is {area} square units.")
```
- Example2
Python(file_path="solution.py"):
```python
# Calculate the perimeter of a rectangle with length 8 and width 3
length = 8
width = 3
perimeter = 2 * (length + width)
print(f"The perimeter of the rectangle is {perimeter} units.")
```

# Feedback Action
* Signature: {
    "action_name": "Feedback",
    "argument": {
        "feedback": ...
        "passed": true/false
    }
}
* Description: The Feedback Action is used to provide feedback to the Tool Manager. The feedback field contains detailed comments or suggestions. If the tool encounters an error, you should set passed to false and provide a detailed feedback. If the tool runs without errors, you can set passed to true and leave feedback as an empty string.
* Examples:
- Example 1:
{
    "action_name": "Feedback",
    "argument": {
        "feedback": ""
        "passed": true
    }
}
- Example 2:
{
    "action_name": "Feedback",
    "argument": {
        "feedback": "The tool encountered an error while executing. The variable 'height' is missing in the function call. Please ensure that all required parameters are provided.",
        "passed": false
    }
}

# RESPONSE FORMAT #
For each task input, your response should contain:
1. One RESPONSE should ONLY contain One Thought and One Action.
2. An analysis of the task and current environment, reasoning through the next evaluation step based on your role as CheckingAgent.(prefix "Thought: ").
3. An action from the **ACTION SPACE** (prefix "Action: "). Specify the action and its parameters for this step.

# EXAMPLE RESPONSE #
Observation: ...(output from the last action, provided by the environment and task input, no need for you to generate it)

Thought: ...
Action: ...(Use an action from the ACTION SPACE once per response.)

# Custom Library #
You can use pandas, sklearn, or other Python libraries as part of the custom library.

* Note: You can directly call these tools without importing or redefining them in your code.

Let's think step by step.
# TASK #
===task===
\end{Verbatim}
\end{tcolorbox}

\subsection{Test Stage}
\label{appsub:test_prompt}
During the test stage, the prompts used for different datasets are as follows:
\begin{tcolorbox}[title=Prompt on DABench, breakable, width=\textwidth,top=0mm]
\begin{Verbatim}[breaklines, fontsize=\footnotesize]
# Instruction #
You are a helpful assistant, skilled in data science tasks.
You will be provided with a task description and related files. 
You should complete tasks by writing notebook code to interact with the environment containing the task files.
Additionally, you must strictly adhere to the task constraints. 
Once the task is completed, you need to format the answer as specified in the answer format and invoke the Terminate action to conclude.
You should use actions from the ACTION SPACE, follow the Response Format, and complete the task within 20 steps.

You may also leverage the following helper functions if needed.
===api===


===example===


# Response Format #
Your each response should contain:
1. One RESPONSE should only contain ONLY One Thought and ONLY One Action.
2. Only an analysis of the task and the current environment, including reasoning to determine the next action. (prefix "Thought: ").
3. Only an action from the **ACTION SPACE** (prefix "Action: "). Specify the action and its parameters for this step.

Observation: ...(Provided by the environment, no need for you to generate it.))

Thought: ...
Action: ...

# ACTION SPACE #
## NotebookBlock Action
* Signature: 
NotebookBlock():
```python
executable python script
```
* Description: The NotebookBlock action allows you to create and execute a Jupyter Notebook cell. The action will add a code block to the notebook with the content wrapped inside the paired ``` symbols. If the block already exists, it can be overwritten based on the specified conditions (e.g., execution errors). Once added or replaced, the block will be executed immediately.
* Restrictions: Each response must contain only one notebook block.
* Note: In a single block, you may call multiple tools or single.
* Example:
Action: NotebookBlock():
```python
# Calculate the area of a circle with a radius of 5
radius = 5
area = 3.1416 * radius ** 2
print(area)
```

# Terminate Action
* Signature: Terminate(result="the result of the task")
* Description: The Terminate action marks the completion of a task and presents the final result. It is a formatting guideline, not an executable Python function. The result parameter must contain a clear, specific answer that strictly complies with the task’s output format, with all required values explicitly provided.
Tips:
    - Ensure the result parameter provides a definite and concrete final answer.
    - Do not include unresolved Python expressions, placeholders, or variables (e.g., @value[{x + y}] or @result[{variable_name}] or "@result[{variable_name}]".format(variable_name)).
    - The output must adhere precisely to the task’s formatting specifications, ensuring clarity and consistency.
* Examples:
- Example 1: 
Answer Format: @shapiro_wilk_statistic[test_statistic] @shapiro_wilk_p_value[p_value]
Action: Terminate(result="@shapiro_wilk_statistic[0.56] @shapiro_wilk_p_value[0.0002]")
- Example 2: 
Answer Format: @total_votes_outliers_num[outlier_num]
where "outlier_num" is an integer representing the number of values considered outliers in the 'total_votes' column.
Action: Terminate(result="@total_votes_outliers[10]")
- Example3:
Action: Terminate(result="@normality_test_result[Not Normal] @p_value[0.000]")

## Response Example
Here are four examples of responses.
## Example1
Thought: The dataset has been loaded successfully and it contains the "Close Price" column. Now, we need to calculate the mean of the "Close Price" column using pandas.
Action: NotebookBlock():
```python
# Calculate the mean of the "Close Price" column
mean_close_price = data["Close Price"].mean()
# Round the result to two decimal places
mean_close_price_rounded = round(mean_close_price, 2)
print(mean_close_price_rounded)
```
## Example2
Thought: We need to filter the dataset to only include rows where the “Volume” is greater than 100,000. This will help focus on high-volume trades.
Action: NotebookBlock():
```python
# Filter rows where "Volume" is greater than 100,000
filtered_data = data[data["Volume"] > 100000]
# Display the filtered dataset
print(filtered_data)
```
## Example3
Thought: To analyze the correlation between “Open Price” and “Close Price,” we will calculate the Pearson correlation coefficient using pandas.
Action: NotebookBlock():
```python
# Calculate the correlation between "Open Price" and "Close Price"
correlation = data["Open Price"].corr(data["Close Price"])
# Print the correlation result
print(correlation)
```
## Example4
Thought: To check for missing values in the dataset, we need to check for null values in each column using pandas.
Action: NotebookBlock():
```python
# Check for missing values in each column
missing_values = data.isnull().sum()
# Display the result
print(missing_values)
```

# Begin #
Let's Begin.
## Task 
===task===
\end{Verbatim}
\end{tcolorbox}


\begin{tcolorbox}[title=Prompt on TextCraft, breakable, width=\textwidth,top=0mm]
\begin{Verbatim}[breaklines, fontsize=\footnotesize]
# Instruction #
You are provided with a set of useful crafting recipes to create items in Minecraft.
Crafting commands follow the format: "craft [target object] using [input ingredients]".
You can either "fetch" an object (ingredient) from the inventory or the environment or "craft" the target object using the provided crafting commands.
You are allowed to use only the crafting commands provided; do not invent or use your own crafting commands.
If a crafting command specifies a generic ingredient, such as "planks", you can substitute it with a specific type of that ingredient (e.g., “dark oak planks”).
To complete the crafting tasks, you will write notebook code utilizing tools from the "Custom Library". You should carefully read and understand the tool’s docstrings and code to fully grasp their functionality and usage.
The tools should be invoked by outputting a block of Python code. Additionally, you may incorporate Python constructs such as for-loops, if-statements, and other logic where necessary.
Please always use actions from the ACTION SPACE and follow the Response Format.


# ACTION SPACE #
## NotebookBlock Action
* Signature: 
NotebookBlock():
```python
executable python script
```
* Description: The NotebookBlock action creates and executes a Jupyter Notebook cell. It adds a code block wrapped in ``` symbols, overwriting existing blocks if specified (e.g., after execution errors). The block is executed immediately after being added or replaced.
* Note: In a single block, you may call multiple tools.

## Terminate Action
* Signature: Terminate(result=the result of the task)
* Description: The Terminate action ends the process and provides the task result. The `result` argument contains the outcome or status of task completion. Only the CheckingAgent has the authority to decide whether a task is finished.
* Examples:
  - Example1: Action: Terminate(result="3")
  - Example2: Action: Terminate(result="Successfully craft 2 oak planks")
  - Example3: Action: Terminate(result="Successfully craft 1 milk")


# Response Format #
For each task input, your response should contain:
1. One RESPONSE should only contain ONLY One Thought and ONLY One Action.
2. An analysis of the task and the current environment, including reasoning to determine the next action. (prefix "Thought: ").
3. An action from the **ACTION SPACE** (prefix "Action: "). Specify the action and its parameters for this step.

## Response Example
### Example1
Thought: To craft a crimson fence, I need to use the crafting command: “craft 3 crimson fence using 2 stick, 4 crimson planks.” This means I need two ingredients: 2 sticks and 4 crimson planks. To obtain 4 crimson planks, I can use the command “craft 4 crimson planks using 1 crimson stems,” so I need to fetch 1 crimson stem first. For the sticks, I can use the command “craft 4 stick using 2 planks.” Since the crafting of sticks requires planks, I can directly use the crimson planks I craft earlier for this purpose.
Steps: 
1. Fetch 1 crimson stems.
2. Craft 4 crimson planks using 1 crimson stems.
3. Craft 4 sticks using 2 crimson planks.
4. Craft 3 crimson fences using 2 sticks and 4 crimson planks.
Action: NotebookBlock()
```python
get_object('1 crimson stems')
```
### Example2
Thought: The inventory is currently empty, which means I need to retrieve the necessary spruce logs from the environment. Since we need spruce planks to craft the spruce trapdoor, and each spruce trapdoor requires 6 spruce planks, I should start by fetching an adequate number of spruce logs. Since each spruce log can craft 4 spruce planks, we need at least 2 spruce logs to ensure we can craft 8 spruce planks (more than enough for 2 spruce trapdoors if needed in the future).
Action: NotebookBlock()
```python
# Step 2: Fetch 2 spruce logs from the environment
get_object('2 spruce logs')
```
### Example3
Thought: We have successfully retrieved 2 spruce logs from the environment. 
Action: Terminate(result="Successfully crafted 8 spruce planks")


# Custom Library #
### Tool `check_inventory`
Source Code:
```python
def check_inventory() -> str:
    """
    Retrieves the current inventory state from the environment.
    The function sends an 'inventory' command to the environment
    and processes the observation to return a string representation
    of the inventory, listing items and their quantities.
    Returns:
        str: A string describing the inventory in the format:
             "Inventory: [item_name] (quantity)"
    """
    obs, _ = step('inventory')
    return obs
```
Usage Example:
```python
check_inventory() 
# If the environment has no items, Output: Inventory: You are not carrying anything.
# If the environment contains 2 oak planks, Output: Inventory: [oak planks] (2)
```
### Tool `get_object`
Source Code:
```python
def get_object(target: str) -> None:
    """
    Retrieves an item from the environment.

    The function prints the response message from the environment, 
    indicating whether the retrieval was successful or not.

    Args:
        target (str): The name of the item to be retrieved.

    Returns:
        None
    """
    obs, _ = step("get " + target)
    print(obs)
```
Usage Example:
Craft Command:
craft 2 yellow dye using 1 sunflower
craft 8 yellow carpet using 8 white carpet, 1 yellow dye
```python
get_object("1 sunflower") # Ouput: Got 1 sunflower
get_object("2 sunflower") # Ouput: Got 2 sunflower
# Note: You cannot retrieve yellow dye directly from the environment; it must first be crafted using sunflowers.
get_object("1 yellow dye") # Output: Could not find yellow dye
```
### Tool `craft_object`
Source Code:
```python
def craft_object(target: str, ingredients: List[str]) -> None:
    """
    Crafts a specified item using the given ingredients.

    This function's `target` and `ingredients` parameters correspond to the craft command: 
    "Craft 'target' using [ingredients]".
    
    **Note:** The `ingredients` must exactly match the command format. For example, if the command requires 
    '1 oak logs', providing '1 oak log' instead will not be recognized.

    Prints the environment's response to indicate whether the crafting operation was successful or not.

    Args:
        target (str): The item to craft along with its quantity (e.g., '4 oak planks').
        ingredients (List[str]): A list of required ingredients with their respective quantities 
                                (e.g., ['1 oak logs']).

    Returns:
        None

    """
    obs, _ = step("craft " + target + " using " + ", ".join(ingredients))
    print(obs)
```
Usage Example:
- Example1
Query: Craft 1 black dye using 1 wither rose
Craft Command:
craft 1 black dye using 1 wither rose
craft 1 black dye using 1 ink sac

```python
get_object('1 wither rose')
craft_object('1 black dye', ['1 wither rose'])
# Ouput:
# Got 1 wither rose
# Crafted 1 minecraft:black dye
```
- Example2
Craft Commands: 4 oak planks using 1 oak logs
```python
get_object("1 oak logs") # Got 1 oak logs
check_inventory() # Inventory: ["oak logs"] (1)
craft_object('4 oak planks', ['1 oak logs'])
# Crafted 4 minecraft:oak planks
```

===api===


===example===

# Example #
Here is a demo of how to fetch and craft objects.
Crafting commands:
craft 3 dark oak sign using 6 dark oak planks, 1 stick
craft 4 dark oak planks using 1 dark oak logs
craft 1 stick using 1 planks
craft 4 stick using 2 bamboo
craft 4 oak planks using 1 oak logs
craft 1 dark oak fence using 2 stick, 4 dark oak planks
craft 1 warped stairs using 6 warped planks
craft 3 oak sign using 6 oak planks, 1 stick

craft 2 oak planks

Thought: To craft 2 oak planks, I need to first check if I can fetch them directly from the environment or my inventory. If they are not available, I will craft them using the appropriate crafting recipe. Based on the crafting commands, I know that I can craft 4 oak planks using 1 oak logs. My plan is as follows:
1. Check if 2 oak planks are already in my inventory or can be fetched from the environment.
2. If not, verify if I have the required ingredient (1 oak logs).
3. Gather the missing ingredient, if necessary.
4. Craft 4 oak planks and ensure 2 are available for the task.
Action: NotebookBlock():
```python
check_inventory()
```

Observation: Inventory: [stick] (1) 

Thought: I could not find 2 oak planks or the required ingredient (1 oak logs) in my inventory. I need to fetch 1 oak logs first.
Action: NotebookBlock():
```
get_object("1 oak logs")
```

Observation: Got 1 oak logs

Thought: I now have the required ingredient to craft oak planks. I will use the crafting command: craft 4 oak planks using 1 oak logs.
Action: NotebookBlock():
```
craft_object("4 oak planks", ["1 oak logs"])
check_inventory()
```

Observation: Crafted 4 minecraft:oak_planks
Inventory: [stick] (1) [oak_planks] (4)

Thought: I now have 2 oak planks in my inventory. Task Completed!
Action: Terminate(result=“Successfully craft 2 oak planks”)


# Begin #
Let's Begin. Please follow the Instruction and focus on solve the task.
# Current items in the inventory #
These are the items currently in your inventory. You cannot use items that are not in your inventory to craft new items.
===inventory===

# Task #
===task===
\end{Verbatim}
\end{tcolorbox}

\begin{tcolorbox}[title=Prompt on MATH, breakable, width=\textwidth,top=0mm]
\begin{Verbatim}[breaklines, fontsize=\footnotesize]
Your task is to solve math competition problems by writing Python programs.

You may also leverage the following helper functions, but must avoid fabricating and calling undefined function names.
```python
===api===
```

Examples: 

Examples: 
Query: Point $P$ lies on the line $x= -3$ and is 10 units from the point $(5,2)$. Find the product of all possible $y$-coordinates that satisfy the given conditions.
Program: 
```python
from sympy import symbols, Eq, solve
# Define symbolic variable for y-coordinate of point P
y = symbols('y')
# Step 1: Given conditions
x1 = -3  # Point P lies on the vertical line x = -3
x2, y2 = 5, 2  # Coordinates of the given point (5, 2)
d = 10  # Distance between point P and (5,2)
# Step 2: Apply the distance formula
# Distance formula: sqrt((x2 - x1)^2 + (y - y2)^2) = d
# Squaring both sides to eliminate the square root:
# (x2 - x1)^2 + (y - y2)^2 = d^2
distance_equation = Eq((x2 - x1)**2 + (y - y2)**2, d**2)
# Step 3: Solve for possible values of y
y_solutions = solve(distance_equation, y)
# Step 4: Compute the product of all possible y-values
product = y_solutions[0] * y_solutions[1]
# Step 5: Output the final result
print("Final Answer:", product)
```

Query: If $3p+4q=8$ and $4p+3q=13$, what is $q$ equal to?
Program:
```python
from sympy import symbols, Eq, solve
# Define symbolic variables for the unknowns p and q
p, q = symbols('p q')
# Step 1: Define the given system of equations
eq1 = Eq(3 * p + 4 * q, 8)  # Equation 1: 3p + 4q = 8
eq2 = Eq(4 * p + 3 * q, 13)  # Equation 2: 4p + 3q = 13
# Step 2: Solve the system of equations for p and q
solution = solve((eq1, eq2), (p, q))
# Step 3: Extract and output the value of q
print("Final Answer:", solution[q])
```

Query: Simplify $\frac{3^4+3^2}{3^3-3}$. Express your answer as a common fraction.
Program:
```python
from sympy import symbols, simplify
# Define the variable
x = symbols('x')
# Define the expression
numerator = 3**4 + 3**2
denominator = 3**3 - 3
fraction = numerator / denominator
# Simplify the fraction
simplified_fraction = simplify(fraction)
# Print the result
print("Final Answer:", simplified_fraction)
```

===example===

## Begin !
Please generate ONLY the code wrapped in ```python...``` to solve the query below.

Query: ===task===
Program:
\end{Verbatim}
\end{tcolorbox}



\begin{tcolorbox}[title=Prompt on Date, breakable, width=\textwidth,top=0mm]
\begin{Verbatim}[breaklines, fontsize=\footnotesize]
Your task is to solve simple word problems by creating Python programs.

You may also leverage the following helper functions, but must avoid fabricating and calling undefined function names, such as `calculate_date_by_years`.
```python
===api===
```

Examples:

Query: In the US, Thanksgiving is on the fourth Thursday of November. Today is the US Thanksgiving of 2001. What is the date one week from today in MM/DD/YYYY?
Program:
```python
# import relevant packages
from datetime import date, time, datetime
from dateutil.relativedelta import relativedelta
import calendar
# 1. What is the date of the first Thursday of November? (independent, support: [])
date_1st_thu = date(2001,11,1)
while date_1st_thu.weekday() != calendar.THURSDAY:
    date_1st_thu += relativedelta(days=1)
# 2. How many days are there in a week? (independent, support: ["External knowledge: There are 7 days in a week."])
n_days_of_a_week = 7
# 3. What is the date today? (depends on 1 and 2, support: ["Today is the US Thanksgiving of 2001", "Thanksgiving is on the fourth Thursday of November"])
days_from_1st_to_4th_thu = (4-1) * n_days_of_a_week
date_today = date_1st_thu + relativedelta(days=days_from_1st_to_4th_thu)
# 4. What is the date one week from today? (depends on 3, support: [])
date_1week_from_today = date_today + relativedelta(weeks=1)
# 5. Final Answer: What is the date one week from today in MM/DD/YYYY? (depends on 4, support: [])
answer = date_1week_from_today.strftime("%m/%d/%Y")
# print the answer
print(answer)
```

Query: Yesterday was 12/31/1929. Today could not be 12/32/1929 because December has only 31 days. What is the date tomorrow in MM/DD/YYYY?
Program:
```python
# import relevant packages
from datetime import date, time, datetime
from dateutil.relativedelta import relativedelta
# 1. What is the date yesterday? (independent, support: ["Yesterday was 12/31/1929"])
date_yesterday = date(1929,12,31)
# 2. What is the date today? (depends on 1, support: ["Today could not be 12/32/1929 because December has only 31 days"])
date_today = date_yesterday + relativedelta(days=1)
# 3. What is the date tomorrow? (depends on 2, support: [])
date_tomorrow = date_today + relativedelta(days=1)
# 4. Final Answer: What is the date tomorrow in MM/DD/YYYY? (depends on 3, support: [])
answer = date_tomorrow.strftime("%m/%d/%Y")
# print the answer
print(answer)
```

Query: The day before yesterday was 11/23/1933. What is the date one week from today in MM/DD/YYYY?
Program:
```python
# import relevant packages
from datetime import date, time, datetime
from dateutil.relativedelta import relativedelta
# 1. What is the date the day before yesterday? (independent, support: ["The day before yesterday was 11/23/1933"])
date_day_before_yesterday = date(1933,11,23)
# 2. What is the date today? (depends on 1, support: [])
date_today = date_day_before_yesterday + relativedelta(days=2)
# 3. What is the date one week from today? (depends on 2, support: [])
date_1week_from_today = date_today + relativedelta(weeks=1)
# 4. Final Answer: What is the date one week from today in MM/DD/YYYY? (depends on 3, support: [])
answer = date_1week_from_today.strftime("%m/%d/%Y")
# print the answer
print(answer)
```

===example===

## Begin !
Please generate ONLY the code wrapped in ```python...``` to solve the query below.

Query: ===task===
Program:
\end{Verbatim}
\end{tcolorbox}



\begin{tcolorbox}[title=Prompt on TabMWP, breakable, width=\textwidth,top=0mm]
\begin{Verbatim}[breaklines, fontsize=\footnotesize]
Your task is to solve table-reasoning problems by writing Python programs.
You are given a table. The first row is the name for each column. Each column is seperated by "|" and each row is seperated by "\n".
Pay attention to the format of the table, and what the question asks.

You may also leverage the following helper functions, but must avoid fabricating and calling undefined function names.
```python
===api===
```


Examples: 
### Table
Name: None
Unit: $
Content:
Date | Description | Received | Expenses | Available Funds
 | Balance: end of July | | | $260.85
8/15 | tote bag | | $6.50 | $254.35
8/16 | farmers market | | $23.40 | $230.95
8/22 | paycheck | $58.65 | | $289.60
### Question
This is Akira's complete financial record for August. How much money did Akira receive on August 22?
### Solution code
```python
records = {
    "7/31": {"Description": "Balance: end of July", "Received": "", "Expenses": "", "Available Funds": 260.85},
    "8/15": {"Description": "tote bag", "Received": "", "Expenses": 6.5, "Available Funds": ""},
    "8/16": {"Description": "farmers market", "Received": "", "Expenses": 23.4, "Available Funds": ""},
    "8/22": {"Description": "paycheck", "Received": 58.65, "Expenses": "", "Available Funds": ""}
}
# Access the amount received on August 22
received_aug_22 = records["8/22"]["Received"]
print("Final Answer: ", received_aug_22)
```

### Table
Name: Orange candies per bag
Unit: bags
Content:
Stem | Leaf 
2 | 2, 3, 9
3 | 
4 | 
5 | 0, 6, 7, 9
6 | 0
7 | 1, 3, 9
8 | 5
### Question
A candy dispenser put various numbers of orange candies into bags. How many bags had at least 32 orange candies?
### Solution code
```python
data = {
    2: [2, 3, 9],
    3: [],
    4: [],
    5: [0, 6, 7, 9],
    6: [0],
    7: [1, 3, 9],
    8: [5]
}
# Initialize the count to zero
count = 0
# Iterate over the keys in the dictionary
for key in data:
    # Combine tenth digit and unit digit
    if key * 10 + data[key] >= 32:
        # Increment the count
        count += 1

# Output the result
print("Final Answer: ", count)
```

### Table
Name: Monthly Savings  
Unit: $  
Content:  
Date  | Description       | Received | Expenses | Available Funds |
       | Balance: end of May |   |   | $500.00 |
6/10  | groceries        |   | $45.75 | $454.25 |
6/15  | gas refill       |   | $30.20 | $424.05 |
6/25  | salary           | $1200.00 |   | $1624.05 |
### Question
How much money did Akira receive on June 25?
### Solution code
```python
import pandas as pd
records = {
    "5/31": {"Description": "Balance: end of May", "Received": "", "Expenses": "", "Available Funds": 500.00},
    "6/10": {"Description": "groceries", "Received": "", "Expenses": 45.75, "Available Funds": ""},
    "6/15": {"Description": "gas refill", "Received": "", "Expenses": 30.2, "Available Funds": ""},
    "6/25": {"Description": "salary", "Received": 1200.00, "Expenses": "", "Available Funds": ""}
}
# Access the amount received on June 25
received_june_25 = records["6/25"]["Received"]
print("Final Answer: ", received_june_25)
```

===example===

## Begin!
Please solve the task below and enclose your code within a single code block using ```python```  format.

===task===
### Solution code
\end{Verbatim}
\end{tcolorbox}









\section{Examples}
\label{app:example}
\subsection{Generated Tools}

\textbf{The tools generated for the Open-ended Tasks are as follows:}
\begin{tcolorbox}[title=CraftDiamondHelmet, width=\textwidth,top=0mm,  breakable]
\begin{Verbatim}[breaklines=true, breakanywhere=true, fontsize=\footnotesize]
async function craftDiamondHelmet(bot) {
  const mcData = require('minecraft-data')(bot.version);
  const Vec3 = require('vec3').Vec3;

  // Check inventory for the number of diamonds
  const diamondCount = bot.inventory.count(mcData.itemsByName["diamond"].id);
  const requiredDiamonds = 5;

  // If not enough diamonds, mine diamond ores
  if (diamondCount < requiredDiamonds) {
    const diamondsToMine = requiredDiamonds - diamondCount;
    bot.chat(`Need ${diamondsToMine} more diamond(s). Mining...`);
    await mineBlock(bot, "diamond_ore", diamondsToMine);
  }

  // Ensure a crafting table is placed nearby
  let craftingTable = bot.findBlock({
    matching: mcData.blocksByName.crafting_table.id,
    maxDistance: 32
  });
  if (!craftingTable) {
    bot.chat("Placing crafting table...");
    await placeItem(bot, "crafting_table", bot.entity.position.offset(1, 0, 0));
    craftingTable = bot.findBlock({
      matching: mcData.blocksByName.crafting_table.id,
      maxDistance: 32
    });
  }

  // Craft the diamond helmet
  bot.chat("Crafting diamond helmet...");
  await craftItem(bot, "diamond_helmet", 1);
  bot.chat("Diamond helmet crafted successfully.");
}
\end{Verbatim}
\end{tcolorbox}


\begin{tcolorbox}[title=CraftItemWithMaterials, width=\textwidth,top=0mm,  breakable]
\begin{Verbatim}[breaklines=true, breakanywhere=true, fontsize=\footnotesize]
async function craftItemWithMaterials(bot, itemName, requiredMaterials) {
  const mcData = require('minecraft-data')(bot.version);
  const Vec3 = require('vec3').Vec3;

  // Check inventory for required materials
  for (const material of requiredMaterials) {
    let itemCount = bot.inventory.count(mcData.itemsByName[material.name].id);
    if (itemCount < material.count) {
      const requiredCount = material.count - itemCount;
      bot.chat(`Need ${requiredCount} more ${material.name}(s).`);
      if (material.name === "diamond") {
        let diamondOre = await bot.findBlock({
          matching: mcData.blocksByName["diamond_ore"].id,
          maxDistance: 32
        });
        if (!diamondOre) {
          bot.chat("No diamond ore found nearby. Exploring...");
          diamondOre = await exploreUntil(bot, new Vec3(1, 0, 1), 60, () => {
            return bot.findBlock({
              matching: mcData.blocksByName["diamond_ore"].id,
              maxDistance: 32
            });
          });
        }
        if (diamondOre) {
          await mineBlock(bot, "diamond_ore", requiredCount);
        } else {
          bot.chat("Failed to find diamond ore after exploring.");
          return;
        }
      } else if (material.name === "stick") {
        const woodenPlanksCount = bot.inventory.count(mcData.itemsByName["oak_planks"].id) + bot.inventory.count(mcData.itemsByName["birch_planks"].id);
        if (woodenPlanksCount < 2) {
          const requiredLogs = Math.ceil((2 - woodenPlanksCount) / 4);
          bot.chat(`Need more wooden planks. Gathering ${requiredLogs} logs...`);
          await obtainWoodLogs(bot, requiredLogs);
          await craftItem(bot, "oak_planks", requiredLogs);
        }
        bot.chat("Crafting sticks...");
        await craftItem(bot, "stick", 1);
      }
    }
  }

  // Ensure a crafting table is placed nearby
  let craftingTable = bot.findBlock({
    matching: mcData.blocksByName.crafting_table.id,
    maxDistance: 32
  });
  if (!craftingTable) {
    bot.chat("Placing crafting table...");
    await placeItem(bot, "crafting_table", bot.entity.position.offset(1, 0, 0));
    craftingTable = bot.findBlock({
      matching: mcData.blocksByName.crafting_table.id,
      maxDistance: 32
    });
  }

  // Craft the item
  bot.chat(`Crafting ${itemName}...`);
  await craftItem(bot, itemName, 1);
  bot.chat(`${itemName} crafted successfully.`);
}

async function craftDiamondAxe(bot) {
  const requiredMaterials = [{
    name: "diamond",
    count: 3
  }, {
    name: "stick",
    count: 2
  }];
  await craftItemWithMaterials(bot, "diamond_axe", requiredMaterials);
}
\end{Verbatim}
\end{tcolorbox}


\textbf{The tools generated for the Agent Tasks are as follows:}
Here, we can clearly see the call relationships between functions, thus forming more complex tools.
\begin{tcolorbox}[title=Tools for DA-Bench, width=\textwidth,top=0mm,  breakable]
\begin{Verbatim}[breaklines=true, breakanywhere=true, fontsize=\footnotesize]
def filter_rows_by_non_null(df: pd.DataFrame, column_name: str) -> pd.DataFrame:
    """
    Filters rows in a dataset based on non-null values in a specified column.
    
    Parameters:
    - df (pd.DataFrame): The input DataFrame.
    - column_name (str): The name of the column to filter by non-null values.
    
    Returns:
    - pd.DataFrame: A DataFrame with rows containing non-null values in the specified column.
    
    Raises:
    - ValueError: If the specified column is not found in the DataFrame.
    """
    # Check if the column exists in the DataFrame
    if column_name not in df.columns:
        raise ValueError(f"Column '{column_name}' not found in the DataFrame.")
    
    # Filter rows based on non-null values in the specified column
    filtered_df = df.dropna(subset=[column_name])
    
    return filtered_df

def convert_column_to_numeric(df: pd.DataFrame, column_name: str) -> pd.DataFrame:
    """
    Converts a specified column in a DataFrame to numeric values, handling non-numeric values appropriately.
    
    Parameters:
    - df (pd.DataFrame): The input DataFrame.
    - column_name (str): The name of the column to convert to numeric values.
    
    Returns:
    - pd.DataFrame: The DataFrame with the specified column converted to numeric values.
    
    Raises:
    - ValueError: If the specified column is not found in the DataFrame.
    """
    # Check if the column exists in the DataFrame
    if column_name not in df.columns:
        raise ValueError(f"Column '{column_name}' not found in the DataFrame.")
    
    # Convert the specified column to numeric values, setting non-numeric values to NaN
    df[column_name] = pd.to_numeric(df[column_name], errors='coerce')
    
    # Filter out rows with non-numeric values in the specified column using the existing tool
    df = filter_rows_by_non_null(df, column_name)
    
    return df

def create_sum_feature(df: pd.DataFrame, new_column_name: str, columns_to_sum: list) -> pd.DataFrame:
    """
    Creates a new feature by summing specified columns in a DataFrame.
    
    Parameters:
    - df (pd.DataFrame): The input DataFrame.
    - new_column_name (str): The name of the new column to be created.
    - columns_to_sum (list): A list of column names to sum.
    
    Returns:
    - pd.DataFrame: The DataFrame with the new feature added.
    
    Raises:
    - ValueError: If any of the specified columns are not found in the DataFrame.
    """
    # Check if all specified columns exist in the DataFrame
    for column in columns_to_sum:
        if column not in df.columns:
            raise ValueError(f"Column '{column}' not found in the DataFrame.")
    
    # Convert specified columns to numeric values
    for column in columns_to_sum:
        df = convert_column_to_numeric(df, column)
    
    # Create the new feature by summing the specified columns
    df[new_column_name] = df[columns_to_sum].sum(axis=1)
    
    return df
\end{Verbatim}
\end{tcolorbox}


\begin{tcolorbox}[title=Tools for TextCraft, width=\textwidth,top=0mm, breakable]
\begin{Verbatim}[breaklines=true, breakanywhere=true, fontsize=\footnotesize]
def gather_materials_for_dye(required_materials: dict) -> bool:
    """
    Gathers the required materials for crafting any dye.
    
    Parameters:
    - required_materials (dict): A dictionary where keys are material names and values are the required quantities.
    
    The tool checks the inventory for these materials and gathers them if they are missing.
    
    Returns:
    - bool: True if all materials were successfully gathered, False otherwise.
    """
    # Gather the required materials
    if not gather_materials(required_materials):
        return False
    
    # Check if we have white dye, if not gather bone meal or lily of the valley to craft it
    inventory = check_inventory()
    if "white dye" in required_materials and "white dye" not in inventory:
        if not gather_materials({"bone meal": 1}) and not gather_materials({"lily of the valley": 1}):
            return False
        # Craft white dye using bone meal or lily of the valley
        if "bone meal" in inventory:
            craft_object("1 white dye", ["1 bone meal"])
        elif "lily of the valley" in inventory:
            craft_object("1 white dye", ["1 lily of the valley"])
    
    # Recheck the inventory to ensure all materials are gathered
    missing_items = check_missing_items([f"{qty} {item}" for item, qty in required_materials.items()])
    if missing_items:
        print(f"Missing items: {missing_items}")
        return False
    
    # Successfully gathered all materials
    return True

def craft_orange_dye(quantity: int) -> bool:
    """
    Crafts the specified quantity of orange dye.
    
    Parameters:
    - quantity (int): The number of orange dye to craft.
    
    Returns:
    - bool: True if the orange dye was successfully crafted, False otherwise.
    """
    # Define the required materials for crafting orange dye
    required_materials = {"orange tulip": quantity, "red dye": quantity, "yellow dye": quantity}
    
    # Gather the required materials using the existing gather_materials_for_dye function
    if not gather_materials_for_dye(required_materials):
        return False
    
    # Check the inventory for available materials
    inventory = check_inventory()
    
    # Craft orange dye using orange tulip if available
    if "orange tulip" in inventory:
        craft_object(f"{quantity} orange dye", [f"{quantity} orange tulip"])
        print(f"Crafted {quantity} orange dye using {quantity} orange tulip")
        return True
    
    # Craft orange dye using red dye and yellow dye if available
    if "red dye" in inventory and "yellow dye" in inventory:
        craft_object(f"{quantity} orange dye", [f"{quantity} red dye", f"{quantity} yellow dye"])
        print(f"Crafted {quantity} orange dye using {quantity} red dye and {quantity} yellow dye")
        return True
    
    # If neither method was successful, return False
    print("Failed to craft orange dye.")
    return False
\end{Verbatim}
\end{tcolorbox}


\textbf{The tools generated for the Single-turn Code Task are as follows:}
\begin{tcolorbox}[title=Tools for MATH, width=\textwidth,top=0mm, breakable]
\begin{Verbatim}[breaklines=true, breakanywhere=true, fontsize=\footnotesize]
def find_integer_satisfying_condition(condition):
    """
    Find the smallest positive integer that satisfies the given condition.

    Parameters:
        condition (function): A lambda function representing the condition to be checked.

    Returns:
        int: The smallest positive integer that satisfies the condition.
    """
    x = 1
    while True:
        if condition(x):
            return x
        x += 1

def calculate_min_correct_answers(total_problems, passing_percentage):
    """
    Calculate the minimum number of correct answers required to pass a test based on the total number of problems and the passing percentage.

    Parameters:
        total_problems (int): The total number of problems on the test.
        passing_percentage (float): The passing percentage required to pass the test.

    Returns:
        int: The minimum number of correct answers required to pass the test.
    """
    if total_problems <= 0:
        return "Total number of problems must be greater than zero."
    if not (0 <= passing_percentage <= 100):
        return "Passing percentage must be between 0 and 100."

    required_correct_answers = (passing_percentage / 100) * total_problems

    # Use find_integer_satisfying_condition to find the minimum integer satisfying the condition
    min_correct_answers = find_integer_satisfying_condition(lambda x: x >= required_correct_answers)
    
    return min_correct_answers
\end{Verbatim}
\end{tcolorbox}

\begin{tcolorbox}[title=Tools for Date, width=\textwidth,top=0mm, breakable]
\begin{Verbatim}[breaklines=true, breakanywhere=true, fontsize=\footnotesize]
def calculate_date_by_days(start_date_str: str, days_to_add: int, date_format="%m/%d/%Y") -> str:
    """
    Calculates the date a specified number of days before or after a given date.

    Parameters:
    - start_date_str (str): The starting date as a string in the format MM/DD/YYYY.
    - days_to_add (int): The number of days to add (positive) or subtract (negative) from the start date.
    - date_format (str): The format of the input and output date string. Default is 'MM/DD/YYYY'.

    Returns:
    - str: The resulting date in the format MM/DD/YYYY.
    
    Raises:
    - ValueError: If the input date string does not match the specified format.
    - OverflowError: If the resulting date is out of the valid range for dates.
    """
    from datetime import datetime, timedelta

    try:
        # Parse the input date string into a date object using the provided format
        start_date = datetime.strptime(start_date_str, date_format).date()

        # Calculate the new date by adding the specified number of days
        new_date = start_date + timedelta(days=days_to_add)

        # Format the new date back into the desired string format
        result_date_str = new_date.strftime(date_format)

        return result_date_str
    except ValueError as e:
        raise ValueError("Incorrect date format. Please ensure the date string matches the provided format.") from e
    except OverflowError as e:
        raise OverflowError("The resulting date is out of the valid range for dates.") from e

def calculate_date_by_days_uk_format(start_date_str: str, days_to_add: int) -> str:
    """
    Calculates the date a specified number of days before or after a given date in UK format (DD/MM/YYYY).

    Parameters:
    - start_date_str (str): The starting date as a string in the format DD/MM/YYYY.
    - days_to_add (int): The number of days to add (positive) or subtract (negative) from the start date.

    Returns:
    - str: The resulting date in the format MM/DD/YYYY.
    
    Raises:
    - ValueError: If the input date string does not match the specified format.
    """
    from datetime import datetime

    try:
        # Convert the input date from DD/MM/YYYY to MM/DD/YYYY
        start_date = datetime.strptime(start_date_str, "%d/%m/%Y")
        
        # Use the existing tool to calculate the new date
        result_date_str = calculate_date_by_days(start_date.strftime("%m/%d/%Y"), days_to_add, "%m/%d/%Y")
        
        return result_date_str
    except ValueError as e:
        raise ValueError("Incorrect date format. Please ensure the date string matches the provided format.") from e
\end{Verbatim}
\end{tcolorbox}


\begin{tcolorbox}[title=Tools for TabMWP, width=\textwidth,top=0mm, breakable]
\begin{Verbatim}[breaklines=true, breakanywhere=true, fontsize=\footnotesize]
import pandas as pd

def stem_and_leaf_to_dataframe(stem_leaf_dict: dict) -> pd.DataFrame:
    """
    Converts a stem-and-leaf plot into a DataFrame.

    Parameters:
    - stem_leaf_dict (dict): A dictionary where keys are the stems and values are lists of leaves.

    Returns:
    - pd.DataFrame: A DataFrame with a single column containing the combined values of stems and leaves.
    """
    # Initialize an empty list to store the combined values
    combined_values = []

    # Iterate through the dictionary to combine stems and leaves
    for stem, leaves in stem_leaf_dict.items():
        for leaf in leaves:
            combined_value = int(f"{stem}{leaf}")
            combined_values.append(combined_value)

    # Create a DataFrame from the combined values
    df = pd.DataFrame(combined_values, columns=["Values"])
    
    return df

import pandas as pd

def count_value_occurrences(stem_leaf_dict: dict, value) -> int:
    """
    Counts the occurrences of a specific value in a DataFrame column created from a stem-and-leaf plot.

    Parameters:
    - stem_leaf_dict (dict): A dictionary where keys are the stems and values are lists of leaves.
    - value: The value to count in the DataFrame.

    Returns:
    - int: The count of the specified value in the DataFrame.
    """
    # Convert the stem-and-leaf plot to a DataFrame using the existing tool
    df = stem_and_leaf_to_dataframe(stem_leaf_dict)
    
    # Count the occurrences of the specified value in the DataFrame
    count = df["Values"].value_counts().get(value, 0)
    
    return count
\end{Verbatim}
\end{tcolorbox}
%\clearpage
% \newpage
%\input{sec/10-supplementary_material}
% \blueHL{checks}
% \begin{itemize}
%     \item Please use $\sqrt{xyz}$ instead of $\sqrt xyz$. Your sqrt sign only appears over x if you do the latter.
% \item Please number the first equation in Sec IIIA.
% \item Please use $\log$ instead of log within an equation.
% \item Also use $\left ( and \right )$ in scenarios where the parentheses are too small, e.g., in (2).
% \item Please double-check that notation is suitably italicized in the text, (e.g., use $B$ in the line after eq. (1) instead of B; there are several other instances).
% \item When you write "where..." after an equation, it is a continuation of the sentence that contains the equation. So, you should use no indentation and should not capitalize it ("Where...")
% \item Remove single period that is alone on a line after the P\_NOISE equation.
% \item PJ -> pJ
% \item Never start a sentence with "E.g" - it just does not work that way. Use "For example, ..." if you need to say that.
% \item I assume you will do this anyway, but just to be complete
% Pay attention to figure sizes, text readability, etc.
% \item Make your captions more descriptive (e.g., all figures on p4 have very short/insufficiently descriptive captions). Check all figures.
% \item Check for "( xxx" - extra space after "(" - this is a common issue in your writing and easy to fix with search-replace.
% \item There are numerous easy-to-find issues in the reference section (capitalization problems, at least one incorrect journal name, capitalization problems in the month, etc.)
% \end{itemize}
\end{document}
