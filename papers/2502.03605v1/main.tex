\pdfoutput=1
\newcommand{\red}[1]{\textcolor{red}{ #1}}
\newcommand{\blue}[1]{\textcolor{blue}{ #1}}
\newcommand{\green}[1]{\textcolor{black}{ #1}}
\renewcommand{\baselinestretch}{1}

% \titlespacing*{\section}{0pt}{3pt}{3pt}
% \titlespacing*{\subsection}{0pt}{3pt}{2pt}
% \titlespacing*{\subsubsection}{0pt}{3pt}{2pt}

\documentclass[9pt, conference]{IEEEtran}
\usepackage{balance}
\usepackage{nopageno}
% \usepackage{graphicx}
% \usepackage{epstopdf}
\usepackage{graphicx}
\usepackage{multirow}
\usepackage{multicol}
\usepackage{color}
\usepackage{url}
\usepackage{array}
\newcolumntype{P}[1]{>{\centering\arraybackslash}p{#1}}
\usepackage{algorithm}
\usepackage{comment}
\usepackage{algpseudocode}
\usepackage{mathtools} 
\usepackage[export]{adjustbox}
\usepackage{textcomp}
\usepackage{cite}
\usepackage{amsmath}
\usepackage{amssymb}
\usepackage{mwe}
\edef\mttopnumber{\arabic{topnumber}}
\setcounter{topnumber}{1}

\usepackage{subfigure}
\usepackage[normalem]{ulem}
\usepackage{enumitem}
\usepackage{float}
\usepackage{enumitem}
\definecolor{gray1}{gray}{0.7}
\definecolor{gray2}{gray}{0.98}
\definecolor{light-gray}{gray}{0.95}
\setlength\extrarowheight{2pt}

\newcommand{\ignore}[1]{}
\newcommand{\redHL}[1]{\textcolor{red}{#1}}
\newcommand{\redfn}[1]{\textcolor{red}{\footnote{\textcolor{red}{#1}}}}
\newcommand{\blueHL}[1]{{\textcolor{blue}{#1}}}
\newcommand{\magentaHL}[1]{{\textcolor{magenta}{#1}}}
\newcommand{\greenHL}[1]{\textcolor{green}{#1}}
\newcommand{\blackHL}[1]{\textcolor{black}{#1}}
\newcommand{\grayHL}[1]{\textcolor{gray1}{#1}}
\newcommand{\bluesout}[1]{{\blueHL{\sout{#1}}}}
\newcommand{\redsout}[1]{{\redHL{\sout{#1}}}}
\newcommand{\Alter}[2]{\sout{#1}\redHL{#2}}
\newcommand{\phase}[3][]{\phi_{#1#2}^{#3}}
\newcommand{\period}[3][]{T_{#1#2}^{#3}}
\newcommand{\freq}[3][]{\omega_{#1#2}^{#3}}
\newcommand{\pluseq}{\mathrel{+}=}
\newcommand{\asteq}{\mathrel{*}=}
\pagestyle{plain}
\renewcommand{\baselinestretch}{0.990}

\newcommand{\red}[1]{\textcolor{red}{ #1}}
\newcommand{\blue}[1]{\textcolor{blue}{ #1}}
\newcommand{\green}[1]{\textcolor{green}{ #1}}
\newcommand{\cyan}[1]{\textcolor{cyan}{ #1}}
\newcommand{\bluefn}[1]{\blue{\footnote{\blue{#1}}}}
\newcommand{\abhimanyu}[1]{{\color{blue}[Abhimanyu: #1]}}
\newcommand{\sachin}[1]{{\color{red}[Sachin: #1]}}

\setlength{\textfloatsep}{2mm}
\setlength{\abovedisplayskip}{2pt}
\setlength{\belowdisplayskip}{2pt}


\begin{document}

\title{Accelerating OTA Circuit Design: Transistor Sizing Based on~a Transformer Model and Precomputed Lookup Tables}

% \author{\IEEEauthorblockN{Subhadip Ghosh}
% \IEEEauthorblockA{\textit{University of Minnesota}\\
% Minneapolis, MN, USA}
% \and
% \IEEEauthorblockN{Endalk Y. Gebru}
% \IEEEauthorblockA{\textit{University of Minnesota}\\
% Minneapolis, MN, USA}
% \and
% \IEEEauthorblockN{Chandramouli Kashyap}
% \IEEEauthorblockA{\textit{Cadence Design Systems}\\
% Portland, OR, USA}
% \and
% \IEEEauthorblockN{Ramesh Harjani}
% \IEEEauthorblockA{\textit{University of Minnesota}\\
% Minneapolis, MN, USA}
% \and
% \IEEEauthorblockN{Sachin S. Sapatnekar}
% \IEEEauthorblockA{\textit{University of Minnesota}\\
% Minneapolis, MN, USA}
% }

\author{
Subhadip Ghosh$^1$, Endalk Y. Gebru$^1$, Chandramouli V. Kashyap$^2$, Ramesh Harjani$^1$, Sachin S. Sapatnekar$^1$ \\
$^1$\text{Department of Electrical and Computer Engineering, University of Minnesota}, Minneapolis, MN, USA \\
$^2$\text{Cadence Design Systems}, Portland, OR, USA
}


\maketitle

\begin{abstract}
Device sizing is crucial for meeting performance specifications in operational transconductance amplifiers (OTAs), and this work proposes an automated sizing framework based on a transformer model. The approach first leverages the driving-point signal flow graph (DP-SFG) to map an OTA circuit and its specifications into transformer-friendly sequential data. A specialized tokenization approach is applied to the sequential data to expedite the training of the transformer on a diverse range of OTA topologies, under multiple specifications. Under specific performance constraints, the trained transformer model is used to accurately predict DP-SFG parameters in the inference phase. The predicted DP-SFG parameters are then translated to transistor sizes using a precomputed look-up table-based approach inspired by the $g_m/I_d$ methodology. In contrast to previous conventional or machine-learning-based methods, the proposed framework achieves significant improvements in both speed and computational efficiency by reducing the need for expensive SPICE simulations within the optimization loop; instead, almost all SPICE simulations are confined to the one-time training phase. The method is validated on a variety of unseen specifications, and the sizing solution demonstrates over 90\% success in meeting specifications with just one SPICE simulation for validation, and 100\% success with 3--5 additional SPICE simulations. 
% \redHL{Can we quantify ``small'' -- e.g., ``after 3-5 iterations''? Can we change ``iteration'' to ``SPICE simulation''?\textbf{YES, How about ``additional'' with that?}}

\end{abstract}

\section{Introduction}

In sensor networks, maintaining data freshness is crucial to support diverse applications such as environmental monitoring, industrial automation, and smart cities \cite{kandris2020applications}. A critical metric for quantifying data freshness is the Age of Information (AoI), which measures the time elapsed since the last received update was generated \cite{yates2012}. Minimizing AoI is essential in dynamic environments, where obsolete information can result in inaccurate decisions or missed opportunities. Efficient AoI management involves balancing update frequency, data relevance, and network resource constraints to ensure decision-makers have timely and accurate information when required \cite{yates2021age}. The significance of AoI has led to extensive research on its optimization across various domains, including single-server systems with one or multiple sources \cite{modiano2015,mm1,sun2016,najm2018,soysal2019,9137714,yates2019,zou2023costly}, scheduling strategies \cite{modiano-sch-1,9007478,sch-igor-1,9241401,sch-li,sch-sun}, and analysis of resource-constrained systems \cite{const-ulukus,const-biyikoglu,const-arafa,const-farazi,const-parisa}. 

%\ali{A good transition here would be: one particular area that has been garnering focus by the AoI researchers and that is correalted systems. In fact, sensor networks often handle...}

Among the strategies for AoI minimization, packet preemption is regarded as a cornerstone approach for ensuring the timeliness of information in communication networks, especially when resources such as service rates are limited \cite{yates2021age}. By prioritizing critical updates, preemption ensures that the most valuable data reaches its destination promptly, as demonstrated in the context of single-sensor, memoryless systems \cite{kaul2012status,inoue2019general}. Beyond this specific scenario, numerous studies have extensively investigated its role in optimizing AoI across diverse settings. For example, \cite{maatouk2019age} analyzes systems with prioritized information streams sharing a common server, where lower-priority packets may be buffered or discarded. Similarly, \cite{wang2019preempt} and \cite{kavitha2021controlling} examine preemption strategies for rate-limited links and lossy systems, identifying in the process the optimal policies for minimizing the AoI.

On the other hand, one particular area that has been garnering focus among AoI researchers is correlated systems. In fact, sensor networks often handle correlated data streams, where relationships between data collected by different sensors can be leveraged to enhance decision-making, reduce redundancy, and improve overall system performance \cite{mahmood2015reliability,yetgin2017survey}. This correlation often arises when multiple sensors monitor overlapping areas or related phenomena, allowing them to collaboratively exchange information and optimize resource usage. The role of correlation in sensor networks has further been explored in studies focusing on its potential to optimize system efficiency and effectiveness \cite{he2018,tong2022,popovski2019,modiano2022,ramakanth2023monitoring,erbayat2024}.











% The importance of AoI and correlation in sensor networks has motivated extensive research into optimizing AoI within correlated sensor systems. For example, \cite{he2018} studied sensor networks with overlapping fields of view, proposing a joint optimization framework for fog node assignment and transmission scheduling to reduce the AoI of multi-view image data. Similarly, \cite{tong2022} focused on overlapping camera networks, introducing scheduling algorithms for multi-channel systems designed to minimize AoI. Other works, such as \cite{popovski2019, modiano2022}, leveraged probabilistic correlation models to formulate sensor scheduling strategies aimed at lowering AoI. Additionally, \cite{ramakanth2023monitoring} treated the correlation of status updates as a discrete-time Wiener process, developing a scheduling policy that balances AoI reduction with monitoring accuracy. Furthermore, \cite{erbayat2024} analyzed the impact of optimal correlation probabilities under varying environmental conditions, addressing the interplay between error minimization and AoI.

%\ali{On the other hand, Preemption in AoI systems has been widely studied...Also, Id say reduce the size of this paragraph} 



%\ali{I don't like this transition here. Talk about correlated systems in the previous paragraph and how AoI is of interest. Then, switch here to preemption is still open question. Do not focus on your paper as you did here}
As part of ongoing efforts in this area, the potential of leveraging interdependencies between sensors to reduce the AoI in correlated systems has been studied, but the benefits and challenges of employing preemption in multi-sensor systems with correlated data streams remain an open question. While preemption is a potential strategy to minimize AoI in a network, it is not always the optimal strategy \cite{yates2019}. This approach must account for the specific features of the packets being transmitted since preempting leads to prioritization. For example, a sensor with a lower arrival rate may track a unique process that no other sensor monitors, making its packets particularly valuable and critical to retain. On the other hand, preempting a packet from a sensor with a high arrival rate may not significantly reduce AoI, as the frequent updates from such sensors diminish the impact of losing a single packet.


%\ali{Here you make the connection between preemption and multi-sensor correlated systems}

%\ali{Its good to emphasize that we have correlation here so it is different than typical AoI system}.

To address this gap, this paper introduces adaptable and probabilistic preemption mechanisms that dynamically balance priorities across sensors, considering their unique correlation characteristics and resource demands. To that end, the main contributions of this paper are summarized as follows:

%To address these challenges, we propose a system where the ability of a packet to preempt an ongoing transmission depends on its source, allowing for a more adaptable approach to managing updates. We also introduce the concept of probabilistic preemption, where preemption decisions are guided by source-specific probabilities rather than fixed or deterministic rules. This probabilistic method improves efficiency by giving higher-priority updates a better chance to preempt, keeping the information more up-to-date. By incorporating stochastic hybrid system modeling, we derive a closed-form expression for the AoI, providing a theoretical foundation to analyze the impact of probabilistic preemption on network performance. Building on this system, we explore how varying preemption probabilities can influence the total AoI in multi-sensor systems, considering the interplay between diverse sensors and their shared resources. Furthermore, we establish that the problem of deciding optimal preemption strategies can be framed as a sum of linear ratios problem. We derive an upper bound on the number of iterations required using a branch-and-bound algorithm, ensuring computational efficiency in identifying optimal solutions. Through this analysis, we identify optimal preemption strategies that minimize the total AoI, balancing the timeliness and relevance of updates across all monitored processes to achieve an efficient and well-coordinated system.

%Interestingly, the results show how the system adjusts priorities between sensors to keep the AoI as low as possible. For example, if one sensor spreads its updates more evenly across multiple processes, the system tends to rely on it more, even if another sensor is sending updates less often. As arrival rates or service rates change, the system shifts its strategy to stay efficient.\footnote{Due to size limitations, we present the proof details in \url{https://github.com/erbayat/xxxx}}.


\begin{itemize}
    \item As a first step, we propose a system where the ability of a packet to preempt an ongoing transmission probabilistically depends on its source rather than being fixed or following deterministic rules. Subsequently, using stochastic hybrid system modeling, we derive a closed-form expression for AoI to analyze the impact of probabilistic preemption on network performance.
    
    %enabling a more adaptable approach to manage updates by giving higher-priority updates a better chance to preempt, ensuring information remains up-to-date.

    \item Following that, we investigate optimizing the total AoI in multi-sensor systems, considering the interplay between diverse sensors and shared resources. Building on this, we frame the problem of deciding optimal preemption strategies as a sum of linear ratios problem, which is generally an NP-Hard problem\cite{freund2001solving}. However, by analyzing its unique characteristics, we derive an upper bound on the number of iterations required to identify optimal preemption strategies using a branch-and-bound algorithm, thus ensuring computational efficiency in finding the optimal solution.
    %\ali{You are using a lot the , ensuring... it sounds very chatgpt liky, try to minimize those when possible. Also, talk about the bounds and the impact of these results on getting an efficient solution}
    \item Lastly, we validate our findings with numerical results and evaluate optimal preemption strategies to minimize AoI. Our findings demonstrate how correlation influences preemption strategies. Notably, when a source provides a lower aggregate number of updates while distributing them more evenly, the system prioritizes it for preemption, even if another sensor updates less frequently.\ifthenelse{\boolean{withappendix}}
{}
{\footnote{Due to space limitations, we present the proof details in \cite{technicalNote}.}}
 %\ali{Dont forget to put the right link}
\end{itemize}


%These results not only support the theory but also offer practical ideas for real-world use, such as in IoT networks, factories, or autonomous systems, where staying up-to-date is very important.

%The remainder of this paper is structured as follows. Section \ref{system-model} introduces the system model and key assumptions. In Section \ref{aoi-S}, we derive the closed-form expression for the AoI within the proposed system. Section \ref{aoi-opt} outlines the optimization problem and details the process of determining the optimal preemption probabilities. The numerical results are presented in Section \ref{numerical}, and the paper concludes with a summary and discussion in Section \ref{conc}.



\section{Background}
\label{sec:background}

\noindent
In this section, we first overview the principles governing transformer architecture. Next, we present a concise overview of DP-SFGs, which we employ to map OTA circuits into transformer-friendly sequential data. Finally, we describe a precomputed LUT-based width estimator to translate DP-SFG parameters to transistor widths.
\vspace{-1mm}
\subsection{The transformer architecture}

\noindent
The transformer~\cite{vaswani_17} is viewed as one of the most promising deep learning architectures for sequential data prediction in NLP.  It relies on an attention mechanism that reveals interdependencies among sequence elements, even in long sequences. The architecture takes a series of inputs \((x_1, x_2, x_3, \ldots, x_n\)) and generates corresponding outputs \((y_1, y_2, y_3, \ldots, y_n\)).

\begin{figure}[b]
\vspace{-5mm}
\centering
\includegraphics[width=0.5\textwidth, bb=0 0 370 190]{fig/TransformermODEL.pdf}
\vspace{-5mm}
\caption{Architecture of a transformer.}
\label{fig:simpleTrans}
% \vspace{-2mm}
\end{figure}

The simplified architecture shown in Fig.~\ref{fig:simpleTrans} consists of $N$ identical stacked encoder blocks, followed by $N$ identical stacked decoder blocks. The encoder and decoder is fed by an input embedding block, which converts a discrete input sequence to a continuous representation for neural processing. Additionally, a positional encoding block encodes the relative or absolute positional details of each element in the sequence using sine-cosine encoding functions at different frequencies. This allows the model to comprehend the position of each element in the sequence, thus understanding its context. Each encoder block comprises a multi-head self-attention block and a position-wise feed-forward network (FFN); each decoder block, which has a similar structure to the encoder, consists of an additional multi-head cross-attention block, stacked between the multi-head self-attention and feed-forward blocks. The attention block tracks the correlation between elements in the sequence and builds a contextual representation of interdependencies using a scaled dot-product between the query ($Q$), key ($K$), and value ($V$) vectors:
\begin{equation}
\text{{Attention}}(Q, K, V) = \text{softmax}\left(\frac{QK^T}{\sqrt{d_k}}\right)V,
\end{equation}
where $d_k$ is the dimension of the query and key vectors. The FFN consists of two fully connected networks with an activation function and dropout after each network to avoid overfitting. The model features residual connections across the attention blocks and FFN to mitigate vanishing gradients and facilitate information flow.

\subsection{Driving-point signal flow graphs}

\noindent
The input data sequence to the transformer must encode information that relates the parameters of a circuit to its performance metrics.  Our method for representing circuit performance is based on the signal flow graph (SFG).  The classical SFG proposed by Mason~\cite{Mason53} provides a graph representation of linear time-invariant (LTI) systems, and maps on well to the analysis of linear analog circuits such as amplifiers. In our work, we employ the driving-point signal flow graph (DP-SFG)~\cite{ochoa_98,schmid_18}. The vertices of this graph are the set of excitations (voltage and current sources) in the circuit and internal states (e.g., voltages) in the circuit.  
% An edge is drawn between vertices that have an electrical relationship, and the weight on each edge is the gain of the edge;
An edge connects vertices with an electrical relationship, and the edge weight is the gain; 
for example, if a vertex $z$ has two incoming edges from vertices $x$ and $y$, with gains $a$ and $b$, respectively, then $z = ax + by$, using the principle of superposition in LTI systems.  To effectively use superposition to assess the impact of each node on every other node, the DP-SFG introduces auxiliary voltages at internal nodes of the circuit that are not connected to excitations. These auxiliary sources are structured to not to alter any currents or voltages in the original circuit, and simplifies the SFG formulation for circuit analysis.
% enable easy formulation of the SFG to analyze circuit behavior. 

\begin{figure}[t]
% \vspace{-6mm}
\centering
\includegraphics[width=0.9\linewidth, bb=0 0 320 140]{fig/DPSFG.pdf}
\vspace{-0.25cm}
\caption{~(a) Schematic and (b) DP-SFG for an active inductor.}
\label{fig:DP-SFG_ex}
\vspace{-5mm}
\end{figure}

Fig.~\ref{fig:DP-SFG_ex}(a) shows a circuit of an active inductor, which is an inductor-less circuit that replicates the behavior of an inductor over a certain range of frequencies. Fig.~\ref{fig:DP-SFG_ex}(b) shows the equivalent DP-SFG. In Section~\ref{sec:dp-sfg}, we provide a detailed explanation that shows how a circuit may be mapped to its equivalent DP-SFG. 


\ignore{
\subsection{Lookup table for MOSFET sizing}
\label{sec:LUT}

\noindent
As seen in Fig.~\ref{fig:DP-SFG_ex}, the edge weights in a DP-SFG include circuit parameters such as the transistor transconductance, $g_m$, and various capacitances in the circuit.  The circuit may be optimized to find values of these parameters that meet specifications, but ultimately these must be translated into physical transistor parameters such as the transistor width.   In older technologies, the square-law model for MOS transistors could be used to perform a translation between DP-SFG parameters and transistor widths, but square-law behavior is inadequate for capturing the complexities of modern MOS transistor models.
In this work, we use a precomputed lookup table (LUT) that rapidly performs the mapping to device sizes while incorporating the complexities of advanced MOS models.

\begin{figure}[htbp]
\vspace{-0.4cm}
\centering
\includegraphics[height=4cm]{fig/lut_fig_1.pdf}
\vspace{-0.55cm}
\caption{LUT generation using three DOFs, $V_{gs}$, $V_{ds}$ and $L$.}
\label{fig:lutgen}
\vspace{-0.1cm}
\end{figure}

The LUT is indexed by the $V_{gs}$, $V_{ds}$, and length $L$ of the transistor, and provides four outputs: the drain current ($I_d$), transconductance ($g_m$), source-drain conductance ($g_{ds}$), and drain-source capacitance ($C_{ds}$).
The entries of the LUT are computed by performing a nested DC sweep simulation across the three input indices for the MOSFET with a specific reference width, $W_{ref}$, as shown in Fig.~\ref{fig:lutgen}, and for each input combination, the four outputs are recorded.
\blueHL{Empirically, we see that the impact of $V_{sb}$ is small enough that it can be neglected, and therefore we set $V_{sb} = 0$ in the sweeps used to create the LUT.}

Our methodology uses this LUT, together with the $g_m/I_d$ methodology~\cite{silviera_96}, to translate circuit parameters predicted by the transformer to transistor widths. The cornerstone of this methodology relies on the inherent width independence of the ratio $g_m/I_d$ to estimate the unknown device width: this makes it feasible to use an LUT characterized for a reference width $W_{ref}$. 
We will elaborate on this procedure further in Section~\ref{sec:precomputedLUTs}, and show how the LUT, together with the $g_m/I_d$ method, can effectively estimate the device widths corresponding to the transformer outputs.
% \redHL{\sout{required to achieve equivalent DC operating characteristics within the circuit. Section III D \redHL{Do not hardcode section numbers!!} provides an in-depth explanation of the implementation details of this methodology.}}
}
\section{Proposed Methodology}
\label{sec:sizing_framework}

\subsection{Overview of the solution}

\noindent
We leverage transformer models to capture the complex relationships between device attributes and circuit performance. We conceptualize the transistor sizing problem as a language translation task, where the input sequence consists of a DP-SFG representation for an OTA circuit, together with performance specifications. The transformer model generates an enhanced DP-SFG representation with the device characteristics necessary to meet the specifications.

\begin{figure}[b]
  \vspace{-5mm}
  \centering
  \includegraphics[width=0.85\linewidth,bb=0 0 295 180]{fig/Toplevel.pdf} % Replace example-image with your image file
  % \vspace{-0.5cm}
  \caption{Overall sizing flow using our transformer-based method.}
  \label{fig:toplevel}
  % \vspace{-2mm}
\end{figure}

Fig.~\ref{fig:toplevel} illustrates the workflow of our framework, with four stages. Stage I performs preprocessing, generating the DP-SFG from the circuit netlist.  The DP-SFG and the designer-specified performance constraints are then tokenized into a combined sequence. Next, in Stage II, a transformer model processes these tokens to predict circuit parameters that meet performance specifications; these are then translated to individual device widths in Stage III using the precomputed LUTs and a $g_m/I_d$ methodology. Finally, in Stage~IV, the performance of the predicted sized circuit is verified using SPICE simulation. In a vast majority of cases, we will show that the performance criteria are satisfied; if not, the designer is brought into the loop to provide tighter specifications, and procedure is reinvoked so that the original specifications are met. The remainder of this section discusses the detailed implementation of each stage.

% 
\small
\begin{algorithm*}[t]
\vspace{-0.5cm}
\caption{Automated DP-SFG Generation for single phase analog circuits}
\label{algo:dpsfg}
\begin{multicols}{2}
\begin{algorithmic}[1]
\State \textbf{Input:} Netlist of circuit components (RCs, transitors, etc.)
\State \textbf{Output:} DP-SFG of the circuit
\State $V \gets \text{All terminals in the netlist as nodes}$  \label{algo1:init_begin}
\State $D_{nc} \gets \text{Initialize node-component dictionary including parasitics}$
\State $D_{cn} \gets \text{Initialize component-node dictionary including parasitics}$
\State $T_{n} \gets \text{All transistors and corresponding terminals}$ \label{algo1:init_end}

\Statex \textit{// \textbf{Step 1}: Initialization of auxiliary nodes}
\For{node $n$ in $D_{nc}$} \textit{// Iterate over all nodes} \label{algo1:step1_begin}
    % \If{$n$ is not connected to $Ground$}
        % \State $C \gets$ All components $c$ connected to $n$
        % \If {$\nexists$ no voltage in $C$}        
    %         \State $F_{aux}[n] \gets 1$ \textit{//node $n$ auxiliary flag assigned 1}
    %         \State Create auxiliary node $\Ddot{n}$
    %         \State Create edge $(\Ddot{n}, n)$ = driving-point impedance at $n$
    %     \EndIf
    % \EndIf
    \If{ $n \not =$ ground \textbf{or} connected to voltage source}   
        \State $F_{aux}[n] \gets 1$; Create auxiliary node $\hat{n}$
        \State Add edge $(\hat{n}, n)$; weight = driving-point impedance at $n$
    \EndIf    
\EndFor \label{algo1:step1_end}
\Statex \textit{// \textbf{Step 2}: Adding branches due to passive components}
\For{component $c$ connected between terminals $i, j \in D_{cn}$} \label{algo1:step2_begin}
    \State $G_c \gets$ admittance of component $c$ 
    \If {node $t \in \{i,j\}$ connects to an auxiliary source node $\hat{t}$}
    \State Add an edge from the other terminal to $\hat{t}$ with weight $G_c$
    \Else
    \State Add an edge from $i$ to $j$ with weight $G_c$ 
    % \redHL{Not clear. These are directed edges (I think) - so why $i$ to $j$, and not $j$ to $i$. The two terminals of a resistor are indistinguishable in terms of direction. \textbf{I did it according to the terminals ...like PLUS terminal to MINUS terminal of Resistors from netlist convention.....to be exact Higher potential to lower} {\em How do you know the potential? You have not solved the circuit and you don't know any voltages.}}
    \EndIf
    % \State {\em Is this what you want to say? \textbf{The edge should be from the non-aux node to the auxnode } OK now?yes YES (If so, pl delete this comment and remove the lines below. If not, please let me know how to correct it. As you will see, this is much more human-readable. The goal of pseudocode is to explain an algorithm clearly, rather than to be semantically correct and exhaustive wrt listing all if-then-else cases. Please use this guidance to adjust Step 3 (I have not seen your most recent version. FYI, I am landing soon and may be offline till ~10pm CT. Meanwhile, can you look at how I have rewritten Step 1, and write Steps 2 and 3 in that way? I first explain the {\em principle} and then go to the example of Fig. 2 to provide a particular case. Thanks. Also: trans-conductance $\rightarrow$ transconductance (``trans'' is not a stand-alone word (I can hyphenate ``stand-alone'' because ``stand'' and ``alone'' are both valid words.). Pl. search for all hyphenated words and fix.} OK PROF
    
    % \State $F_{aux}[s] == 1$ ? $(d, \hat{s})$ $\gets$ $G_c$ 
    % \State $F_{aux}[d] == 1$ ? $(s, \hat{d})$ $\gets$ $G_c$ 
    %\State If neither node is auxiliary, create edge $(s, d)$ $\gets$ $G_c$
\EndFor \label{algo1:step2_end}

\Statex\textit{// \textbf{Step 3}: Adding branches due to transconductance $g_m$}

\For{$t$ in $T_n$} \textit{// Iterate over all transistors} \label{algo1:step3_begin}
    \Statex \textit{// $I_d[t]$: Drain current of t; $g$: Gate; $s$: Source, $d$: Drain}

    \If{$V_{gs} \neq 0$}
        \If{transistor $t$ is NMOS}
            \State For node $i$, $j$  in $\{s,d\}$ terminals:
            \If{Voltage at node $i \propto I_d[t]$}
                \State Add edges from $i$ to $\hat{i}$ with weight $-g_m$ 
                \State Add an edge from $i$ to $\hat{j}$ with weight $+g_m$ 
            \Else  
                \State (Same as above, but negate the edge weights)
            \EndIf
            \State Add an edge from $g$ to $\hat{d}$ with weight $-g_m$ 
            \State Add an edge from $g$ to $\hat{s}$ with weight $+g_m$ 
        \Else \textit{// For PMOS transistors}
            \State (Same as above, but exchange ``then'' and ``else'')
        \EndIf
\EndIf
\EndFor \label{algo1:step3_end}


% \For{$t$ in $T_n$} \textit{// Iterate over all transistors}
%     \If{$V_{gs} \neq 0$}
%         \If{transistor $t$ is NMOS}
%             \State For node $t$ in \{s,d,g\} terminals:
%             \If{Voltage at node $t \propto I_d[t]$}
%                     \State $(t, \hat{x}) \gets (I_d[t] \propto V_{x} \text{ and } x \neq g) \ ? \ -g_m : +g_m$
%                     \State $(x, \hat{y}) \gets (I_d[t] \propto V_{x}) \ ? \ +g_m : -g_m$
%                 \Else
%                     \State $(g, \hat{d}) \gets -g_m$
%                     \State $(g, \hat{s}) \gets +g_m$
%                 \EndIf
%         \Else
%             \State For PMOS transistors, the logic for edge creation would be opposite.
%         \EndIf
% \EndIf
% \EndFor





\end{algorithmic}
\end{multicols}
\vspace{-0.3cm}
\end{algorithm*}
\normalsize


\vspace{-2mm}
\subsection{Circuit-to-sequence mapping using the DP-SFG}
\label{sec:dp-sfg}


\noindent
The procedure for creating the DP-SFG formalizes the approach in~\cite{schmid_18,schmid_yt}. We use the running example of an active inductor circuit, whose DP-SFG is shown in Fig.~\ref{fig:DP-SFG_ex}(b) to illustrate the method.


\noindent
\textbf{Step 0: Initial bookkeeping and node initialization}
The algorithm begins with initializing the vertex (or node) set $V$, and initializing data structures for fast access to connectivity information between circuit components (RCs, transistors, etc.) from the netlist. 

\noindent
\textbf{Step 1: Insertion of auxiliary nodes.} 
For nodes that are not connected to voltage sources, we create auxiliary voltage sources. These sources are described by $V = z I$, where $z$ is the driving point impedance (DPI) at the node, i.e., the the inverse sum of all conductances connected to the node. In Fig.~\ref{fig:DP-SFG_ex}, auxiliary sources are added at nodes~1 and~2. The sources replicate node voltages without introducing additional current into the circuit.  This establishes relationships $V_1 = z_1 I_1 $ and $V_2 = z_2 I_2$, where 
\begin{equation}
z_1 = \frac{1}{sC+sC_{\textit{ds}}+sC_{\textit{gs}}+g_{\textit{ds}}}, \; \;
z_2 = \frac{1}{sC+sC_{\textit{gs}}+G}
\end{equation}

\noindent
\textbf{Step 2: Adding branches due to passive components.} Next, we add the edges associated with passive components. We consider how each terminal connects to auxiliary sources. If a terminal connects to an auxiliary source, we connect it to the auxiliary node of the other terminal using its admittance as the edge weight. If neither terminal connects to an auxiliary source, we connect them with an edge using the admittance of the component. In Fig.~\ref{fig:DP-SFG_ex}, the terminals of capacitor \( C \) are both connected to auxiliary sources carrying currents \( I_1=s(C + C_{\textit{gs}}) V_2 \) and \( I_2=s(C + C_{\textit{gs}}) V_1 \) through the associated edges.

\noindent
\textbf{Step 3: Adding branches due to transistor transconductances.} 
The voltage at each terminal of a transistor influences its drain current. Based on these terminal voltages, we establish connections that directly or indirectly affect the auxiliary node currents. In the example of Fig.~\ref{fig:DP-SFG_ex}, if $V_1$ increases, the drain current $I_d$ increases, and hence current flowing to $I_1$ decreases in the opposite direction. This is reflected by setting the weight on the branch from $V_1$ to $I_1$ to $-g_m$. Similarly, the branch $V_2$ to $I_1$ has weight $+g_m$, reflecting the positive dependence of drain current $I_d$ with $V_2$. 

\begin{figure}[t]
  \centering
  \includegraphics[width=0.9\linewidth, bb=0 0 240 170]{fig/dpsfg_seq.pdf} % Replace example-image with your image file
  \caption{\textbf{Input:} DP-SFG paths with desired performance specifications, \textbf{Output:} Predicted sequences with device parameter values.}
  \label{fig:seq}
  \vspace{-0.4cm}
\end{figure}

At the end of Step~3, we obtain the final DP-SFG in Fig.~\ref{fig:DP-SFG_ex}(b) for the active inductor circuit. We will encode such a DP-SFG into sequential data that encapsulates the functionality of the circuit as well as the parameters of circuit components, including parasitics. This sequence representation is provided as an input to the transformer and is eventually used to size transistors in the circuit. We utilize the NetworkX Python package to process the final DP-SFG. This approach utilizes Johnson's algorithm ($O(V^2 \log V + VE)$ complexity) to identify all cycles, and the depth-first search algorithm ($O(V + E)$ complexity) to find all forward paths, where $V$ represents the number of nodes and $E$ represents the number of edges in the graph. 
For our running example, Fig.~\ref{fig:seq} shows the sequences obtained from the DPSFG. Specifically, the path outlined by red dotted rectangle represents the forward path between input and output nodes, while the blue and green outlined paths denote the cycles.
% \bluefn{(1)~This does not make sense. There are no paths in this figure: the paths are in Fig. 2.  \textbf{FIXED} Additionally, ``blue/green/red marked paths'' does not make sense because the paths are only outlined with a dotted rectangle, not ``marked'' by these colors. \textbf{FIXED} See change to next paragraph for the specs, and please change this accordingly. (2)~The blue path in Fig. 2 does not seem to correspond to the path shown with the blue dotted rectangle that you reference here, which is supposed to be a forward path between input and output nodes -- that path is a cycle.  Same for the path marked by the green dotted rectangle in Fig. 2. Are they supposed to be related? If so, please make consistent. And if not, please change because it is confusing. \textbf{FIXED}} SSS_NOTE

















\ignore{
\noindent
\textbf{Step 0: Initial bookkeeping and node initialization (lines~\ref{algo1:init_begin} to~\ref{algo1:init_end})}
The algorithm begins with initializing the vertex set $V$,
and initializing data structures for fast access to connectivity information between circuit components (RCs, transistors, etc.) from the netlist. 

\noindent
\textbf{Step 1: Insertion of auxiliary nodes (lines~\ref{algo1:step1_begin} to~\ref{algo1:step1_end})} 
Following the DP-SFG construction procedure in~\cite{schmid_18}, for nodes that are not connected to voltage sources, we create auxiliary voltage sources. These sources are described by $V = z I$, where $z$ is the driving point impedance (DPI) at the node, i.e., the
the inverse sum of all conductances connected to the node. In Fig.~\ref{fig:DP-SFG_ex}, auxiliary sources are added at nodes~1 and~2. The sources replicate node voltages without introducing additional current into the circuit. 
 
\redHL{CK: I haven't checked the math as it is due today. This is the part I worry about. Also SFG fomulations apply to only "linearizable" circuits like an op-amp. We should say something about handling general non-linearities. You could say that usually there is linear model lurking somewhere by transformation for example a PLL in phase domain. For others, this could used for studying stability by perturbation around a steady state which can be modeled linearly.}
This establishes relationships $V_1 = z_1 I_1 $ and $V_2 = z_2 I_2$, where 

\begin{equation}
z_1 = \frac{1}{sC+sC_{\textit{ds}}+g_{\textit{ds}}}, \; \;
z_2 = \frac{1}{sC+G}
\end{equation}

\noindent
\textbf{Step 2: Adding branches due to passive components and external sources (lines~\ref{algo1:step2_begin} to~\ref{algo1:step2_end})} Next, we add the edges due to passive components. We consider how each terminal connects to auxiliary sources. If a terminal connects to an auxiliary source, we connect it to the auxiliary node of the other terminal using the admittance of the component as the edge weight. If neither terminal connects to an auxiliary source, we connect them with an edge using the admittance of the component. In Fig.~\ref{fig:DP-SFG_ex}, both terminals of capacitor \( C \) are connected to auxiliary sources, allowing currents \( I_2 = (sC) V_1 \) and \( I_1 = (sC) V_2 \) to flow through the edges.


\noindent
\textbf{Step 3: Adding branches due to transistor transconductance (lines~\ref{algo1:step3_begin} to~\ref{algo1:step3_end})} 
In this step, we analyze the impact of the transistor transconductance. The voltage at each terminal of a transistor influences its drain current. Based on these terminal voltages, we establish connections that directly or indirectly affect the auxiliary node currents. In the example of Fig.~\ref{fig:DP-SFG_ex}, if $V_1$ increases, the drain current $I_d$ increases, and hence current flowing to $I_1$ decreases. Due to this the weight on the branch from $V_1$ to $I_1$ is set to $-g_m$. Similarly, the branch $V_2$ to $I_1$ has weight $+g_m$, reflecting the positive dependence of drain current $I_d$ with $V_2$. 

\begin{figure}[ht]
  \centering
  \includegraphics[width=\linewidth]{fig/DPSFG_paths.pdf} % Replace example-image with your image file
  \vspace{-0.5cm}
  \caption{Sequential paths of DP-SFG, Performance specifications, and transformer predicted sequence with device parameters}
  \label{fig:seq}
  \vspace{-0.4cm}
\end{figure}

At the end of Step~3, we obtain the final DP-SFG in Fig.~\ref{fig:DP-SFG_ex}(b) for the active inductor circuit. We will encode such a DP-SFG into sequential data that encapsulates the functionality of the circuit as well as the parameters of circuit components, including parasitics. This sequence representation is provided as an input to the transformer and is eventually used to size transistors in the circuit. We utilize the NetworkX Python package to process the final DP-SFG. This approach utilizes Johnson's algorithm ($O(V^2 \log V + VE)$ complexity) to identify all cycles and the depth-first search algorithm ($O(V + E)$ complexity) to find all forward paths, where $V$ represents the number of nodes and $E$ represents the number of edges in the graph.

For our running example, the sequence is derived from the paths 
marked in the DP-SFG in Fig.~\ref{fig:seq}.
Specifically, the blue marked path represents the forward path between input and output nodes, while the red and green marked paths denote the cycles. 
}

\subsection{Implementation of the transformer}

\noindent
\textbf{Transformer inputs.} The transformer model comprehends the interdependencies between device parameters and circuit performance metrics. We frame the paths by concatenating the nodes and edge weights from the DPSFG for every forward path and loop. The transformer takes in the list of paths extracted from the DP-SFG, each augmented with the desired set of specifications outlined by the black dotted rectangle in Fig.~\ref{fig:seq}. The transformer acts on the sequences and predicts the device parameters $g_m$, $g_{ds}$, $C_{ds}$, and $C_{gs}$ which will satisfy the desired specifications.


\noindent
\textbf{Overall transformer architecture.}
We implement the transformer in Python, leveraging the PyTorch library. The architecture of the transformer model closely resembles the one proposed by~\cite{vaswani_17}, with minor modifications. We use a 720-dimensional input embedding with 12 heads of parallel attention layers, while keeping the rest of the parameters unchanged.

\noindent
\textbf{Tokenization and byte-pair encoding}.
Tokenization is a crucial step for optimizing transformer efficacy. It breaks down a long sequence into smaller entities called tokens. Unlike in traditional NLP, where individual words and sub-words are treated as tokens, we use specific groups of individual characters such as the key device parameters $g_m, g_{ds}$, $C_{ds}$, $C_{gs}$, edge weights, and the names of the transistors, as tokens that convey necessary information about the circuit. 

For our problem, character-level tokenization (CLT), where each character is treated as a single token, is found to lead to long sequence lengths, i.e., a large number of tokens within a single sequence, resulting in computational inefficiency.
To overcome this problem, we employ the byte-pair encoding (BPE) approach~\cite{rico_16}.
This approach iteratively combines the most frequently occurring tokens (bytes) into a single token,
and dynamically adapts the vocabulary of the training data to capturing a common group of characters conveying essential information. By applying BPE, we achieve a 3.77$\times$ compression of the sequence lengths compared to the use of CLT, 
% \bluefn{Compared to what baseline? CLT? This is meaningless without stating the baseline. \textbf{FIXED}} 
leading to substantial savings in training time and memory requirements.

To demonstrate tokenization and for an actual DP-SFG path, we choose a partial path of a five-transistor operational transconductance amplifier (5T-OTA). The results of CLT and BPE are:
% character-level tokenization $(A)$ and BPE $(B)$.

\noindent
% \textit{Character-level encoding:} Character-level coding treats every single character as an independent individual token, as shown below:\\ 
CLT: \textcolor{white}{\sethlcolor{blue}\hl{3}\sethlcolor{red}\hl{2} \sethlcolor{orange}\hl{g}\sethlcolor{teal}\hl{m}\sethlcolor{violet}\hl{P}\sethlcolor{brown}\hl{1} \sethlcolor{red}\hl{-}\sethlcolor{pink}\hl{1}\sethlcolor{gray}\hl{6}
\sethlcolor{blue}\hl{1}\sethlcolor{black}\hl{/}\sethlcolor{cyan}\hl{(}\sethlcolor{orange}\hl{g}\sethlcolor{lightgray}\hl{d}\sethlcolor{blue}\hl{s}\sethlcolor{teal}\hl{M}\sethlcolor{purple}\hl{0}\sethlcolor{pink}\hl{+}\sethlcolor{magenta}\hl{s}\sethlcolor{teal}\hl{C}\sethlcolor{lightgray}\hl{d}\sethlcolor{brown}\hl{s}\sethlcolor{red}\hl{M}\sethlcolor{blue}\hl{0}\sethlcolor{pink}\hl{+}\sethlcolor{gray}\hl{s}\sethlcolor{black}\hl{C}\sethlcolor{lightgray}\hl{d}\sethlcolor{orange}\hl{s}\sethlcolor{violet}\hl{P}\sethlcolor{blue}\hl{1}\sethlcolor{pink}\hl{+}\sethlcolor{orange}\hl{g}\sethlcolor{teal}\hl{m}\sethlcolor{violet}\hl{P}\sethlcolor{blue}\hl{1}\sethlcolor{cyan}\hl{)}} 

% \textcolor{white}{\sethlcolor{blue}\hl{32} \sethlcolor{orange}\hl{gmP1} \sethlcolor{red}\hl{-16}
% \sethlcolor{blue}\hl{1/(}\sethlcolor{orange}\hl{gdsM0}\sethlcolor{pink}\hl{+}\sethlcolor{lightgray}\hl{s}\sethlcolor{teal}\hl{CdsM0}\sethlcolor{pink}\hl{+}\sethlcolor{lightgray}\hl{s}\sethlcolor{black}\hl{CdsP1}\sethlcolor{pink}\hl{+}\sethlcolor{orange}\hl{gmP1}\sethlcolor{cyan}\hl{)}} 


\noindent
BPE: 
% \textcolor{white}{\sethlcolor{blue}\hl{3}\sethlcolor{pink}\hl{2}\sethlcolor{pink} \hl{2}\sethlcolor{orange}\hl{.}\sethlcolor{violet}\hl{5}\sethlcolor{black}\hl{m}\sethlcolor{magenta}\hl{S}\sethlcolor{orange}\hl{P}\sethlcolor{cyan}\hl{1} \sethlcolor{red}\hl{-}\sethlcolor{cyan}\hl{1}\sethlcolor{brown}\hl{6}
% \sethlcolor{cyan}\hl{1}\sethlcolor{orange}\hl{/}\sethlcolor{red}\hl{(}\sethlcolor{violet}\hl{5}\sethlcolor{black}\hl{6}\sethlcolor{cyan}\hl{7}\sethlcolor{teal}\hl{u}\sethlcolor{magenta}\hl{S}\sethlcolor{violet}\hl{M}\sethlcolor{olive}\hl{0}\sethlcolor{pink}\hl{+}\sethlcolor{magenta}\hl{s}\sethlcolor{olive}\hl{0}\sethlcolor{orange}\hl{.}\sethlcolor{cyan}\hl{7}\sethlcolor{black}\hl{a}\sethlcolor{red}\hl{F}\sethlcolor{teal}\hl{M}\sethlcolor{olive}\hl{0}\sethlcolor{pink}\hl{+}\sethlcolor{magenta}\hl{s}\sethlcolor{violet}\hl{5}\sethlcolor{red}\hl{4}\sethlcolor{gray}\hl{1}\sethlcolor{black}\hl{a}\sethlcolor{red}\hl{F}\sethlcolor{darkgray}\hl{P}\sethlcolor{cyan}\hl{1}\sethlcolor{pink}\hl{+}\sethlcolor{pink}\hl{2}\sethlcolor{orange}\hl{.}\sethlcolor{violet}\hl{5}\sethlcolor{black}\hl{m}\sethlcolor{magenta}\hl{S}\sethlcolor{orange}\hl{P}\sethlcolor{cyan}\hl{1}\sethlcolor{purple}} 
\textcolor{white}{\sethlcolor{blue}\hl{32}\sethlcolor{pink} \hl{2}\sethlcolor{orange}\hl{.}\sethlcolor{violet}\hl{5}\sethlcolor{black}\hl{mS}\sethlcolor{orange}\hl{P1} \sethlcolor{red}\hl{-16}
\sethlcolor{blue}\hl{1/(}\sethlcolor{violet}\hl{5}\sethlcolor{black}\hl{6}\sethlcolor{cyan}\hl{7}\sethlcolor{orange}\hl{uS}\sethlcolor{violet}\hl{M0}\sethlcolor{pink}\hl{+}\sethlcolor{magenta}\hl{s}\sethlcolor{pink}\hl{0}\sethlcolor{violet}\hl{.}\sethlcolor{cyan}\hl{7}\sethlcolor{black}\hl{aF}\sethlcolor{teal}\hl{M0}\sethlcolor{pink}\hl{+}\sethlcolor{gray}\hl{s}\sethlcolor{violet}\hl{5}\sethlcolor{red}\hl{4}\sethlcolor{gray}\hl{1}\sethlcolor{black}\hl{aF}\sethlcolor{darkgray}\hl{P1}\sethlcolor{pink}\hl{+}\sethlcolor{pink}\hl{2}\sethlcolor{orange}\hl{.}\sethlcolor{violet}\hl{5}\sethlcolor{black}\hl{mS}\sethlcolor{orange}\hl{P1}} 

% \hfill--$(B)$

\noindent
The CLT sequence colors each neighboring character differently, denoting the tokenization of each unique character. However, CLT cannot easily comprehend the relation to device parameters or device names. The BPE approach overcomes this by iteratively combining frequently-occurring tokens. For example, BPE combines tokens such as
gmP1, gdsM0, and CdsM0 to represent $g_m$ of transistor P1, and $g_{ds}$ and $C_{ds}$ of transistor M0, respectively. Similarly, character-level tokens for the units of circuit parameters (e.g., ``mS'' or ``aF'') are combined into tokens of multiple characters. However, all purely numeric strings are left uncombined, as shown in the BPE sequence. For instance, for the value 2.5mS, which corresponds to the $g_m$ of transistor P1, the tokens representing 2.5 are maintained as character-level tokens, enabling the transformer to predict each digit relative to performance metrics independently, but the two character-level tokens for ``mS'' are combined into a single token. This restricted BPE representation thus enables the transformer model to better comprehend circuit relationships, as compared with CLT.





\noindent
\textbf{Loss Function.}
To enhance the learning of device parameter prediction from specified inputs, we utilize a weighted cross-entropy loss function for the transformer. Each token is treated as a separate class, with the loss function assigning greater importance to classes critical for accurate predictions. We focus on tokens representing numerical values of device characteristics (e.g., $g_m$, $g_{ds}$, $C_{ds}$, and $C_{gs}$), ensuring they receive more attention during training. This approach allows the transformer to grasp the significance of these characteristics and their impact on performance. Our experiments compared unweighted and weighted loss functions with varying weights, revealing that applying a 20\% increased weight on the numerical tokens yielded optimal performance.


\vspace{-0.02cm}
\subsection{Translating circuit parameters to device widths}
\label{sec:precomputedLUTs}

\noindent
After the trained transformer predicts the values of circuit parameters, they must be transformed to device widths. 
In this section, we describe a methodology 
% that uses the LUT from Section~\ref{sec:LUT} 
for this purpose.

\subsubsection{\textbf{Device characterization}}

In older technologies, the square-law model for MOS transistors could be used to perform a translation between circuit parameters and transistor widths, but square-law behavior is inadequate for capturing the complexities of modern MOS transistor models. In this work, we use a precomputed lookup table (LUT) that rapidly performs the mapping to device sizes while incorporating the complexities of advanced MOS models.

\begin{figure}[t]
% \vspace{-0.4cm}
\centering
\includegraphics[height=3cm, , bb=0 0 210 100]{fig/lut_fig_new.pdf}
% \vspace{-0.55cm}
\caption{LUT generation and characterization.}
\label{fig:lutgen}
\vspace{-5mm}
\end{figure}

The LUT is indexed by the $V_{gs}$, $V_{ds}$, and length $L$ of the transistor, and provides five outputs: the drain current ($I_d$), transconductance ($g_m$), drain-source conductance ($g_{ds}$), drain-source capacitance ($C_{ds}$), and gate-source capacitance ($C_{gs}$) all computed per unit transistor width. The entries of the LUT are computed by performing a nested DC sweep simulation across the input indices for the MOSFET with a specific reference width, $W_{ref}$, as shown in Fig.~\ref{fig:lutgen}, and for each input combination, the five outputs are recorded. Since the five quantities all vary linearly with the width of the transistor, we store their corresponding values per unit width. 
% Empirically, we see that the impact of $V_{sb}$ is small enough
% \bluefn{For my info: over what range of $V_{sb}$ have you done this, and why is that sufficient? \textbf{I chose the range between (0 - $V_{dd}/3$), this is due to having around three mosfets b/n $V_{dd}$ and ground in our circuits. For example, in the 5T OTA the $V_{sb}$ for the DP is non-zero since the source is sitting on top of the tail mosfet, and the voltage (Vds) across the tail mos is usually $<$vdd/3}} SSS_NOTE
% that it can be neglected, and therefore we set $V_{sb} = 0$ in the sweeps used to create the LUT. 
The LUT stores the vector-valued function
% \vspace{-2mm}
\begin{align}
[I_d \;\; g_m \;\; g_{ds} \;\; C_{ds} \;\; C_{gs}] = f(V_{gs}, V_{ds})
\end{align}

We have constructed a lookup table for a 65nm technology with a reference transistor width of 700nm, with $V_{gs}$ and $V_{ds}$ values ranging from 0--1.2V with a 60mV step. Given the relatively coarse granularity of data points in the LUT, we have implemented cubic spline interpolation to enhance accuracy at intermediate values. These LUT granularity, together with interpolation, ensures that it provides accurate predictions, and yet has a reasonable size.  

Our methodology uses this LUT, together with the $g_m/I_d$ methodology~\cite{silviera_96,jespers_17}, to translate circuit parameters predicted by the transformer to transistor widths. The cornerstone of this methodology relies on the inherent width independence of the ratio $g_m/I_d$ to estimate the unknown device width: this makes it feasible to use an LUT characterized for a reference width $W_{ref}$. 





\ignore{
The device characterization stage involves studying how transistor small signal parameters vary with the device size and external voltages applied between the terminals: gate-to-source voltage ($V_{gs}$), drain-to-source voltage ($V_{ds}$), and source-to-bulk voltage ($V_{sb}$). This analysis explores the relationship between the DoFs and output parameters such as drain current $I_d$, $g_m$, $g_{ds}$, and $C_{ds}$. 
\begin{align}
I_d, g_m, g_{ds}, C_{ds}, \dots & \Rightarrow f(V_{gs}, V_{ds}, V_{sb}, L, W)
\end{align}
\begin{equation}
I_d  = \mu_nC_{ox}\frac{W}{L} \times f(V_{gs},V_{ds},V_{sb})
\label{eq:width_indp}
\end{equation}
From equation \eqref{eq:width_indp}, it is evident that the drain current ($Id$) exhibits a linearly proportional relationship with the width (W). This relationship holds for the quadratic behavior of the $I_d$ both in the linear and saturation regions. It is also generally acceptable for other parameters, i.e., $g_m$, $g_{ds}$, $C_{ds}$. Although there may be slight deviations in practical scenarios, these can usually be overlooked for simplification~\cite{jespers_17}.

Our method primarily focuses on $V_{gs}$ and $V_{ds}$, and $L$ with the assumption of a fixed $V_{sb}$ value of zero; this assumption gives a reasonably accurate approximation, reducing analysis to a model with three degrees of freedom: $f(V_{ds}, V_{gs}, L)$. We constructed a lookup table with a reference width of $700nm$ based on $V_{gs}$ and $V_{ds}$ values ranging from $0$ to $1.2V$ with a $60mV$ step, and five lengths starting from $100nm$ to $180nm$ with $20nm$ step. These chosen values ensure that the LUT remains reasonably sized for efficient lookup operations.  

Given the relatively coarse granularity of our data points in the LUT, we implemented cubic spline interpolation to enhance accuracy when retrieving values from the lookup table.
}

\subsubsection{\textbf{Width estimation}} 

The width estimation process uses the recorded LUT and transformer-predicted MOSFET parameters to compute the optimal width. The pseudocode for the algorithm employed is presented in Algorithm~\ref{algo:width_estimation}. 
After initialization on line~\ref{algo2:init}, the input is converted to the desired $g_m/I_d$ ratio.  Lines~\ref{algo2:while_begin}--\ref{algo2:while_end} iterate over the LUT to find the $W$ that matches the transformer-supplied parameters. Specifically, line~\ref{algo2:gmId} finds the value of $V_{gs}$ at which the $g_m/I_d$ ratio is met. For this value, lines~\ref{algo2:wcalc_begin}--\ref{algo2:wcalc_end} determine candidate values of $W$, $w_1, \cdots, w_5$, by ratioing $I_d^{in}$, $g_m^p$, $g_{ds}^p$, $C_{ds}^p$, and $C_{gs}^p$, respectively, with the corresponding LUT outputs. We iterate over $V_{ds}$ until $w_1, \cdots, w_5$ are as close as possible. Line~\ref{algo2:while_end} takes a step in this direction using the empirically chosen factor $\alpha = 10^{-4}$. The iterations continue until the candidate width values converge.



\ignore{
\\
\noindent\textbf{Step 1: Operating points calculation} (lines 5 to 11), using the parameters obtained from the transformer, i.e., $g_m^p$ and $I_d$ the $g_m/I_d$ operating point is calculated. Then in lines 7 to 11, the $g_m$ and $I_d$ values are read from the table with initial $V_{ds}$ and $L$, as a function of $V_{gs}$. $V_{gs}$ value that satisfies the calculated $g_m/I_d$ operating point is then obtained from the ratio of the $V_{gs}$ dependent $g_m$ and $I_d$ functions.  

\noindent\textbf{Step 2: Reading parameters as a function of $V_{ds}$} (line 12), by treating these parameters as a function of $V_{ds}$ we eliminate dependency on the initial guess $V_{ds}$ value, which unlike the $g_m/I_d$, significantly affect the other four parameters.

\noindent\textbf{Step 3: Normalization of parameters} (line 13) Based on the width proportionality property stated in equation \eqref{eq:width_indp} the parameters are made to be width independent by normalizing them with $W_{ref}$.

\noindent\textbf{Step 4: Calculating width and the total cost} (lines 14 to 17) next, the widths corresponding to each transformer-predicted parameter are calculated. These widths are used to determine the total cost, defined as the total deviation of the computed widths as a function of $V_{ds}$. Our objective is to find a width value that ensures the predicted parameters. Therefore, the minimum point of the cost function, where most widths align, is taken as the minimum cost.

\noindent\textbf{Step 5: Iterate until optimal accuracy is achieved}, the difference, $\Delta$, between the minimum cost obtained from the above step and the previous cost. The iteration continues until the cost no longer improves, which is controlled by a minimum value, $\epsilon$. Alongside this process, the $V_{ds}$ value is also updated with the direction determined by the sign of $\Delta$ and the magnitude by another variable, $\alpha$. Here $\alpha$ and $\epsilon$ are user-defined parameters; their values are obtained by manually tuning them using the training dataset as a reference to assess the convergence behavior. In the experiment, we observe that the loop converges rapidly for a wide range of $\alpha$ and $\epsilon$ values. Empirically, we find the choice $\alpha = 0.01$, $\epsilon = 1$x$10^{-8}$ to be effective.

Finally, once the loop is done the $V_{ds}$ value that minimizes the cost function is identified as the optimal drain-to-source voltage, $V_{ds}^*$, of the MOSFET resulting in the desired parameters. The optimal width $W^*$ can then be obtained from one of the width functions defined in lines 14 to 17 evaluated at the optimum $V_{ds}$ value.
}

    \begin{algorithm}[t]
    \footnotesize
        \caption{Width Estimation}
        \label{algo:width_estimation}
    \begin{algorithmic}[1]
    \State \textbf{Inputs:} Transformer-predicted $g_{m}^p$, $g_{ds}^p$, $C_{ds}^p$, $C_{gs}^p$ and current $I_d^{in}$ for a MOSFET; LUT for the device type (PMOS/NMOS); 
    % reference width, $W_{ref}$, used for recording the LUT; 
    tolerances $\alpha$ and $\epsilon$ 
    \State \textbf{Outputs:} Optimal width, $W$, of the MOSFET
    % DC operating points and width of the MOSFET, namely $V_{gs}^*$, $V_{ds}^*$ and $W^*$
    \State $V_{ds,curr} \gets V_{dd}/2$ , $cost_{curr} \gets \infty$, $\Delta \gets \infty$  \label{algo2:init}
    
    \State $g_m\_I_d  \gets g_{m}^p/I_d$ \textit{// Compute the $g_m\_I_d$ operating point}
    
    \While{$|\Delta| > \epsilon$} \textit{// until current cost $\approx$ previous cost}
        \State mincost$_{prev} \gets$ mincost$_{curr}$, $V_{ds,prev} \gets V_{ds,curr}$ \label{algo2:while_begin}
        % \Statex \hspace{1.3em} \textit{// Getting $V_{gs}$ from the LUT}
        \State Find LUT entry $[I_d \;\; g_m \;\; g_{ds} \;\; C_{ds} \;\; C_{gs}] = f(V_{gs}, V_{ds})$ \label{algo2:gmId}
        \Statex \hspace*{10mm} at which $g_m/I_d = g_m\_I_d$. Report $V_{gs}^p$ for this entry.

        \State At $V_{gs}^p$, the LUT for $[I_d \;\; g_m \;\; g_{ds} \;\; C_{ds} \;\; C_{gs}]$ is $f(V_{ds})$.
        \State \textit{// Find $w_1 \cdots w_5$ as functions of $V_{ds}$}
        \State $w_1$$\gets$$g_{m}^p/g_m$, $w_2$$\gets$$g_{ds}^p/g_{ds}$, $w_3$$\gets$$C_{ds}^p/c_{ds}$, 
        \Statex \hspace*{10mm} $w_4$$\gets$$C_{gs}^p/c_{gs}$, $w_5$$\gets$$I_d^{in}/I_d$
        

        \Statex \hspace{1.3em} \textit{// Find $V_{ds}$ at which $w_i$s are closest}
        \State cost$(V_{ds}) \gets \sum_{n=1}^{3} \sum_{m=n+1}^{4} |w_n - w_m|$   \label{algo2:wcalc_begin}
        \State mincost$_{curr} = \min_{V_{ds}}$ (cost)
        \State $\Delta \gets$ mincost$_{prev}$ $-$ mincost$_{curr}$ \label{algo2:wcalc_end}
        \Statex \hspace{1.3em} \textit{// Updating the initial guess $V_{ds}$ value}
        \State $V_{ds,curr} \gets V_{ds,curr} + \mbox{sgn}(\Delta).\alpha.V_{ds, prev}$ \label{algo2:while_end}
        
        % \State $g_m = f(V_{gs}) \gets \text{lookup}\_g_{m}(V_{ds,curr},L) $
        % \State $I_d = f(V_{gs}) \gets \text{lookup}\_I_{d}(V_{ds,curr},L) $
        % \State $V_{gs} \gets$ \text{get\ $V_{gs}$ where $g_{m}/I_{d}$ == $g_m\_I_d$} \redHL{Find Vgs corresponding to gm\_Id; lines 9 and 10 provide the function, line 5 provides the value to be looked up}
        % \Statex \hspace{1.3em} \textit{// Reading the four parameters as a function of $V_{ds}$}
        % \State $I_d,g_m,g_{ds},C_{ds} =\ f(V_{ds})\ \gets\ \text{lookup}(V_{gs},L)$   
        
        % \Statex \hspace{1.3em} \textit{// Normalizing the parameters with $W_{ref}$}
        % \State $I_d\_w,\ g_m\_w,\ g_{ds}\_w, c_{ds}\_w\gets f(V_{ds})/W_{ref}$
        % \Statex \hspace{1.3em} \textit{// Width calculation using the normalized functions}
        % \Statex \hspace{1.3em} \textit{// Each width is a function of $V_{ds}$}
        % \State $w_1 \gets g_{m}^p/g_m\_w$
        % \State $w_2 \gets g_{ds}^p/g_{ds}\_w$
        % \State $w_3 \gets C_{ds}^p/c_{ds}\_w$
        % \State $w_4 \gets I_d/I_d\_w$

        % \Statex \hspace{1.3em} \textit{// Determining cost function from the widths}
        % \State $cost\_func \redHL{(Vds)} \gets \sum_{n=1}^{3} \sum_{m=n+1}^{4} |w_n - w_m|$   
        % \State $min\_cost_{curr} = min_{\redHL{all Vds}} (cost\_func)$
        % \State $\Delta \gets min\_cost_{prev}-min\_cost_{curr}$
        % \Statex \hspace{1.3em} \textit{// Updating the initial guess $V_{ds}$ value}
        % \State $V_{ds,curr} \gets V_{ds,curr} + sgn(\Delta).\alpha.V_{ds, prev}$
    \EndWhile
    
    % \State $V_{ds}^* \gets \arg\min(cost\_func)$
    \State $W \gets\ w_1(V_{ds})$
    % \State $V_{gs}^* \gets V_{gs}$    

    \end{algorithmic}
    \end{algorithm}


    

\subsection{SPICE verification and margin allocation}

\noindent
Finally, we perform just one SPICE simulation to verify compliance with all specifications. If any specification deviates from the requirements, the model modifies the specifications and repeats the inference step to obtain a new set of device sizes. For example, if the gain of the sized OTA is 10\% below the desired value, the model iteratively tightens the specifications to accommodate this 10\% difference in the gain requirement until all specifications are satisfied.
% \bluefn{Do you also do this if the performance is much better than the specification? You would reduce the power/area by using smaller device sizes.  Also: why don't you report power/area anywhere? \textbf{No, I don't do this if the performance is much better than the specifications. However, doing this is very much possible. But, its not guaranteed that reducing size will bring down the performance closer to the specifications.}}SSS_NOTE
\section{Experimental Setup and Results}
\label{sec:results}
\vspace{-2mm}
\noindent
% \blueHL{\sout{In this section, we delve into our experimental setup, covering the process of data generation and pre-processing. Additionally, we present details on model training and validation procedures. Finally, we present the efficacy of our sizing assistant in predicting device sizes with some unseen circuit performance specifications.}}
% \bluefn{Can delete this without much loss of continuity, if we need to save space. \textbf{OK}} SSS_NOTE

\subsection{Data generation and preprocessing}

\noindent
To demonstrate the efficacy of our framework, we employ three distinct OTA topologies: five-transistor OTA (5T-OTA), current-mirror OTA (CM-OTA), and two-stage OTA (2S-OTA), each implemented using a 65nm technology node. In Fig.~\ref{fig:schemas}, we show the OTA schematics along with matching constraints under consideration. For clarity in our demonstration, we focus on three performance metrics: gain, 3dB-bandwidth (BW), and unity-gain frequency (UGF), and aim to meet the given performance specifications.
 
Table~\ref{tab:dataset} shows the range of different specifications for the OTAs considered for our training set. We assume that the length of all the devices in a circuit is set to 180nm with a load capacitor $C_L$ of 500fF for all the topologies. To ensure reliable model analysis, we start with precise data generation for each OTA topology using OCEAN scripting. This involves the following steps:
\begin{itemize}
    \item Generating multiple designs with varying transistor sizes by nested sweeps of widths ranging from 0.7$\mu$m to 50$\mu$m.
    \item Enforcing matching constraints for active load current mirror (CM), and differential pair (DP).
    \item Sweeping the DC voltage to determine the input common-mode range (ICMR) of the designs.
    \item Ensuring that the CMs operate in the strong inversion region while the DPs function in the weak inversion region.
    % \bluefn{(1)~Strange choice of words -- what is ``irrelevant''? It either meets specs or it does not. How can it be irrelevant? \textbf{Can I use ``invalid'' ? Cause, I am using ICMR just for filtering out the OTAs which can practically function for a valid input common mode voltage range.} (2)~If you use ICMR as a spec, why isn't it ever reported in your results? \textbf{I am not using ICMR as a SPEC}} SSS_NOTE
    \item Filtering out designs that falls out of the predefined specification range for the dataset outlined in Table~\ref{tab:dataset}.
    \item Capturing the device parameters -- specifically, $g_m, g_{ds}$, $C_{ds}$, and $C_{gs}$ -- for the final legal designs.
\end{itemize}

% \bluefn{What does ``inadequate'' mean? Elaborate. \textbf{I think ``functional'' would be better. The circuits which have a valid ICMR,  and giving out practically useful gain, BW, UGF, are choosen}} SSS_NOTE

\begin{figure}[t]
    \centering
    \subfloat[]{
        \includegraphics[width=19.5mm, bb = 0 0 100 110]{fig/5tota.pdf}
    }
    \hspace{-2mm} % Adjust spacing between subfigures
    \subfloat[]{
        \includegraphics[width=30.5mm, bb = 0 0 200 110]{fig/cmota.pdf}
    }
    \hspace{-1mm} % Adjust spacing between subfigures
    \subfloat[]{
        \includegraphics[width=29mm, bb = 0 0 160 110]{fig/2sota.pdf}
    }
    
    \caption{Schematic of (a) 5T-OTA, (b) CM-OTA, and (c) 2S-OTA.}
    \label{fig:schemas}
    \vspace{-0.2cm}
\end{figure}

Next, we focus on generating appropriate DP-SFG paths for each circuit topology. Table~\ref{tab:dataset} shows the number of sequential paths for each topology. The DP-SFGs are small and the cost of path enumeration is small; for more complex examples, if the number of paths grows large, it is possible to devise other string representations of the DP-SFG.
Finally, in the preprocessing stage, we generate two sets of sequential data, one each for the encoder and the decoder.
% \bluefn{\textbf{***I think at this stage SEQUENTIAL DATA makes more sense***} Is the use of ``sequential paths'' confusing? You are talking about a sequence of tokens that feeds the transformer. The paths are not tokens. Maybe use some other term instead of ``sequential paths''?}
\begin{itemize}
    \item The sequential data at the encoder comprises DP-SFG paths that maintain consistency across all designs within a specific topology. It also includes performance metrics for each design, encompassing gain, BW, and UGF parameters, associated with each unique set of transistor sizes.
    \item The sequential data at the decoder covers the same DP-SFG paths, but with device parameters replaced by values obtained during data generation. These values are unique to each design, aligning with the performance metrics in the encoder sequence.
\end{itemize}

\begin{table}[t]
    \caption{Dataset information.}
    \centering
    % \begin{tabular}{|>{\centering\arraybackslash}m{1.1cm}|>{\centering\arraybackslash}m{1cm}|>{\centering\arraybackslash}m{1.3cm}|>{\centering\arraybackslash}m{1.2cm}|>{\centering\arraybackslash}m{1cm}|>{\centering\arraybackslash}m{0.7cm}|}
    \resizebox{1\linewidth}{!}{\begin{tabular}{|l|c|c|c|c|c|}
        \hline
        \textbf{Topology} & \makecell{\textbf{Gain}\\\textbf{(dB)} \\\textit{min-max}} & \makecell{\textbf{3dB bandwidth}\\\textbf{(MHz)} \\\textit{min-max}} & \makecell{\textbf{UGF} \\\textbf{(MHz)} \\\textit{min-max}}& \makecell{\textbf{\#forward} \\\textbf{paths}} & \textbf{\#cycles} \\
        \hline
        5T-OTA & 18 -- 23 &  7 -- 54 & 80 -- 871  &  9 & 4  \\
        \hline
        CM-OTA & 19 -- 25 & 17.5 -- 86 & 57 -- 1185 & 26 & 5  \\
        \hline
        2S-OTA & 28 -- 54 & 0.01 -- 0.32 & 1.8 -- 370  &  2 & 11 \\
        \hline
    \end{tabular}}
    \label{tab:dataset}
    \vspace{-5mm}
\end{table}


We train a single transformer model that works across all three OTA topologies. By considering all performance criteria and all DP-SFG paths, we convey complete information about each circuit to the transformer. Our dataset comprises 17,000 designs for 5T-OTA, 25,000 designs for CM-OTA, and 8,000 designs for 2S-OTA, each with a different set of transistor sizes. This diverse dataset trains the model across multiple design specification requirements.


% \begin{table}[t]
%     \vspace{-0.2cm}
%     \caption{Dataset Information}
%     \centering
%     \begin{tabular}{|>{\centering\arraybackslash}p{1.1cm}|>{\centering\arraybackslash}p{0.9cm}|>{\centering\arraybackslash}p{0.7cm}|>{\centering\arraybackslash}p{0.5cm}|>{\centering\arraybackslash}p{0.5cm}|>{\centering\arraybackslash}p{0.5cm}|>{\centering\arraybackslash}p{0.5cm}|>{\centering\arraybackslash}p{0.5cm}|>{\centering\arraybackslash}p{0.5cm}|}
%         \hline
%         \multirow{2}{*}{\textbf{Topology}} & \multirow{2}{*}{\textbf{Forward}} & \multirow{2}{*}{\textbf{Cycles}} & \multicolumn{2}{c|}{\makecell{\textbf{Gain}\\\textbf{(dB)}}} & \multicolumn{2}{c|}{\makecell{\textbf{Bandwidth}\\\textbf{(MHz)}}} & \multicolumn{2}{c|}{\makecell{\textbf{UGF} \\\textbf{(MHz)}}}\\
%         \cline{4-9}
%          & \textbf{paths} &  & \textbf{Min} & \textbf{Max} & \textbf{Min} & \textbf{Max} & \textbf{Min} & \textbf{Max} \\
%         \hline
%         5T-OTA & 9 & 4 & 18 & 29 & 7 & 54 & 80 & 871\\
%         \hline
%         CM-OTA & 26 & 5 & 20 & 32 & 59 & 86 & 57 & 1185\\
%         \hline
%         2S-OTA & 2 & 11 & 30 & 52 & 13 & 32 & 18 & 370\\
%         \hline
%     \end{tabular}
%     \label{tab:DP-SFG}
% \end{table}

\subsection{Training and validation}

% \redfn{CK: The number of paths in a more complex circuit could be very large? How do we propose to handle it? \textbf{Since, the inferencing is quite fast, the overall process will still be fast. But, yes the size of the dataset is proportional to number of paths in DPSFG. {\em Added text below Table 1. Please check blueHL.}YES MAKES SENSE}}

\noindent
For our experiments, we employ an Nvidia L40S GPU equipped with 45GB of memory. The dataset is split into an 80:20 ratio for training and validation across each OTA topology. We train a single model using datasets from all three topologies for 40 epochs, employing an adaptive learning rate strategy with the Adam optimizer, beginning with an initial learning rate of $10^{-4}$. Subsequently, our framework is validated against unseen performance specifications across all three OTA topologies. For a given topology and performance specifications, the validation phase rapidly predicts a sequence of tokens corresponding to circuit parameters.The transformer takes in the encoder sequences list and predicts output sequences containing the device parameter values that satisfy the specifications. This is followed by the LUT-based estimator that translates the predicted device parameters to transistor widths. 

% In the subsequent subsection, we comprehensively analyze each aspect of our framework's performance.

\begin{figure}[t]
\vspace{-4mm}
\centering
\subfloat[]{
  \includegraphics[width=0.49\linewidth, bb = 0 0 1100 1000]{fig/scatter.pdf}
}
\hspace{-0.4cm}
\subfloat[]{
  \includegraphics[width=0.49\linewidth, bb = 0 0 1100 1000]{fig/gSCATTER.pdf}
}
% \vspace{-2mm}
\caption{For 5T-OTA: Scatter plots showing comparisons between predicted and simulation-based device parameters (a) $g_m$ and (b) $g_{ds}$.}
\label{fig:plots}
\vspace{-5mm}
\end{figure}

\subsection{Performance of the framework}

\noindent
We conduct comprehensive performance evaluations to assess the effectiveness of the transformer model and LUT-based width estimator for each OTA topology. Our method sizes 100 unique designs per topology, each with distinct performance specifications not included in the training set. For each specification, the transformer model predicts the key device parameters, which are then converted to transistor sizes using the LUT-based method. The performance of the final design is validated through Spectre simulation of the sized OTA circuit for each topology. Additionally, we ensure the optimized devices operate in the desired region of operation. 


\begin{table}[b]
    \vspace{-4mm}
    \centering
    % First table
    \begin{minipage}{\linewidth}
        \centering
        \caption{\centering Correlation coefficient of device parameters between validation data and model outputs for the 5T-OTA.}
        \resizebox{1\linewidth}{!}{
            \begin{tabular}{|>{\centering\arraybackslash}p{0.9cm}|>{\centering\arraybackslash}p{2.3cm}|c|c|c|c|}
                \hline
                \multirow{2}{*}{\makecell{\textbf{MOS} \\ \textbf{devices}}} & \multirow{2}{*}{\makecell{\textbf{Transistor} \\ \textbf{information}}} & \multicolumn{4}{c|}{\makecell{\textbf{Correlation coefficient}}} \\ \cline{3-6}
                & & \textbf{$g_m$} & \textbf{$g_{ds}$} & \textbf{$C_{ds}$} & \textbf{$C_{gs}$} \\
                \hline
                M1/M2 & Active load & 0.982 & 0.993 & 0.962 & 0.964 \\ \cline{1-6} 
                M3/M4 & DP & 0.999 & 0.991 & 0.997 & 0.998 \\ \cline{1-6} 
                M5 & Tail MOS & 0.999 & 0.997 & 0.997 & 0.997 \\ \cline{1-6} 
                \hline
            \end{tabular}
        }
        \label{tab:5t_corr}
    \end{minipage}
    
    \vspace{1mm} % Space between tables

    % Second table
    \begin{minipage}{\linewidth}
        \centering
        \caption{\centering Comparison of optimized design performance with target specifications for the 5T-OTA}
        \resizebox{1\linewidth}{!}{
            \begin{tabular}{|c|c|c|c|c|c|}
                \hline
                \multicolumn{2}{|c|}{\textbf{Gain (dB)}} & \multicolumn{2}{c|}{\makecell{\textbf{UGF (MHz)}}} & \multicolumn{2}{c|}{\makecell{\textbf{3dB bandwidth (MHz)}}} \\ \cline{1-6}
                \makecell{\textbf{Target}} & \makecell{\textbf{Optimized}} & \makecell{\textbf{Target}} & \makecell{\textbf{Optimized}} & \makecell{\textbf{Target}} & \makecell{\textbf{Optimized}} \\
                \hline
                20.13 & 20.6 & 118.78 & 144.64 & 11.38 & 13.33 \\ \cline{1-6} 
                21.23 & 21.37 & 181.25 & 185.38 & 15.31 & 15.49 \\ \cline{1-6} 
                22.78 & 22.79 & 281.75 & 288.54 & 20.18 & 20.48 \\ \cline{1-6} 
                \hline
            \end{tabular}
        }
        \label{tab:5t_specs}
    \end{minipage}

    % \vspace{-2mm}
\end{table}

\noindent
\textbf{5T-OTA}
The 5T-OTA topology includes a matched active current-mirror load (M1/M2), a differential pair (M3/M4), and a tail transistor (M5), all requiring precise sizing to meet performance targets. We assess the prediction accuracy of the transformer by correlating predicted device parameters with SPICE-based validation results. Fig.~\ref{fig:plots} shows a strong correlation between predicted \(g_{m}\) and \(g_{ds}\) values and their SPICE counterparts along the 45° line,
% , where we use single variable for matched transistors, 
and Table~\ref{tab:5t_corr} summarizes the correlation coefficients of all the parameters, highlighting model accuracy. We show the results of applying the transformer model for three sets of unseen target specifications in Table~\ref{tab:5t_specs}: the optimized circuit can be seen to meet all requirements.



\noindent
\textbf{CM-OTA}
The CM-OTA topology incorporates a differential input stage, succeeded by three current mirror loads. A total of nine devices require sizing in this configuration. The correlation coefficient between the device parameters predicted by the transformer model and the SPICE-based validation data are shown in Table~\ref{tab:cmota_corr} and display high accuracy. Finally, Table~\ref{tab:cmota_specs} delineates the target specifications for three randomly selected designs from the validation set. As in the case of the 5T-OTA, the output of the transformer yields optimized circuits that meet all performance requirements. 

\begin{table}[t]
    \centering
    % \vspace{-2mm}
    % First table
    \begin{minipage}{\linewidth}
        \centering
        \caption{\centering Correlation coefficient of device parameters between validation data and model outputs for the CM-OTA.}
        \resizebox{1\linewidth}{!}{
            \begin{tabular}{|>{\centering\arraybackslash}p{0.9cm}|>{\centering\arraybackslash}p{2.3cm}|c|c|c|c|}
                \hline
                \multirow{2}{*}{\makecell{\textbf{MOS} \\ \textbf{devices}}} & \multirow{2}{*}{\makecell{\textbf{Transistor} \\ \textbf{information}}} & \multicolumn{4}{c|}{\makecell{\textbf{Correlation coefficient}}} \\ \cline{3-6}
                & & \textbf{$g_m$} & \textbf{$g_{ds}$} & \textbf{$C_{ds}$} & \textbf{$C_{gs}$} \\
                \hline
                M1/M2 & Matched CM load & 0.811 & 0.838 & 0.871 & 0.875 \\ \cline{1-6}
                M3/M4 & DP & 0.798 & 0.683 & 0.878 & 0.883 \\ \cline{1-6} 
                M5 & Tail MOS & 0.817 & 0.867 & 0.601 & 0.760 \\ \cline{1-6} 
                M6/M7 & Matched CM load & 0.893 & 0.803 & 0.881 & 0.895 \\ \cline{1-6} 
                M8/M9 & Matched CM load & 0.912 & 0.914 & 0.891 & 0.892 \\ \cline{1-6} 
                \hline
            \end{tabular}
        }
        \label{tab:cmota_corr}
    \end{minipage}
    
    \vspace{0.5mm} % Space between tables

    % Second table
    \begin{minipage}{\linewidth}
        \centering
        \caption{\centering Comparison of optimized design performance with target specifications for the CM-OTA}
        \resizebox{1\linewidth}{!}{
            \begin{tabular}{|c|c|c|c|c|c|}
                \hline
                \multicolumn{2}{|c|}{\textbf{Gain (dB)}} & \multicolumn{2}{c|}{\makecell{\textbf{UGF (MHz)}}} & \multicolumn{2}{c|}{\makecell{\textbf{3dB bandwidth (MHz)}}} \\ \cline{1-6}
                \makecell{\textbf{Target}} & \makecell{\textbf{Optimized}} & \makecell{\textbf{Target}} & \makecell{\textbf{Optimized}} & \makecell{\textbf{Target}} & \makecell{\textbf{Optimized}} \\
                \hline
                20.83 & 21.99 & 345.9 & 475.74 & 30.84 & 37.65 \\ \cline{1-6} 
                21.55 & 23.25 & 247.98 & 408.11 & 20.15 & 27.48 \\ \cline{1-6} 
                23.8 & 24.3 & 1033.77 & 1478.5 & 71.47 & 104.24 \\ \cline{1-6} 
                \hline
            \end{tabular}
        }
        \label{tab:cmota_specs}
    \end{minipage}

    \vspace{-4mm}
\end{table}



\noindent
\textbf{2S-OTA}
The 2S-OTA topology includes a 5T-OTA in the first stage, followed by a common source amplifier comprising seven devices. Table~\ref{tab:2sota_corr} provides a summary of the correlation coefficient between the device parameters predicted by the transformer model and those generated by SPICE, thereby affirming the accuracy of the model. Furthermore, Table~\ref{tab:2sota_specs} presents the target specifications for three randomly selected designs from the validation set. Again, the transformer delivers optimized circuits that meet all specifications.

\begin{table}[b]
    \centering
    \vspace{-4mm}
    % First table
    \hspace*{-0.03\linewidth}
    \begin{minipage}{1.0\linewidth}
        \centering
        \caption{\centering Correlation coefficient of device parameters between validation data and model outputs for the 2S-OTA.}
        \resizebox{1\linewidth}{!}{
            \begin{tabular}{|>{\centering\arraybackslash}p{0.9cm}|>{\centering\arraybackslash}p{2.3cm}|c|c|c|c|}
                \hline
                \multirow{2}{*}{\makecell{\textbf{MOS} \\ \textbf{devices}}} & \multirow{2}{*}{\makecell{\textbf{Transistor} \\ \textbf{information}}} & \multicolumn{4}{c|}{\makecell{\textbf{Correlation coefficient}}} \\ \cline{3-6}
                & & \textbf{$g_m$} & \textbf{$g_{ds}$} & \textbf{$C_{ds}$} & \textbf{$C_{gs}$} \\
                \hline
                M1/M2 & 1\textsuperscript{st} stage active load & 0.942 & 0.936 & 0.876 & 0.879 \\ \cline{1-6}
                M3/M4 & 1\textsuperscript{st} stage DP & 0.988 & 0.945 & 0.913 & 0.915 \\ \cline{1-6} 
                M5 & 1\textsuperscript{st} stage tail MOS & 0.928 & 0.989 & 0.918 & 0.922 \\ \cline{1-6} 
                M6 & 2\textsuperscript{nd} stage tail MOS & 0.856 & 0.881 & 0.843 & 0.798 \\ \cline{1-6} 
                M7 & 2\textsuperscript{nd} stage CS & 0.892 & 0.887 & 0.785 & 0.880 \\ 
                \hline
            \end{tabular}
        }
        \label{tab:2sota_corr}
    \end{minipage}
    
    % \vspace{0.5mm} % Space between tables

    % Second table
    \begin{minipage}{\linewidth}
        \centering
        \caption{\centering Comparison of optimized design performance with target specifications for the 2S-OTA}
        \resizebox{1\linewidth}{!}{
            \begin{tabular}{|c|c|c|c|c|c|}
                \hline
                 \multicolumn{2}{|c|}{\textbf{Gain (dB)}} & \multicolumn{2}{c|}{\makecell{\textbf{UGF (MHz)}}} & \multicolumn{2}{c|}{\makecell{\textbf{3dB bandwidth (kHz)}}} \\ \cline{1-6}
                \makecell{\textbf{Target}} & \makecell{\textbf{Optimized}} & \makecell{\textbf{Target}} & \makecell{\textbf{Optimized}} & \makecell{\textbf{Target}} & \makecell{\textbf{Optimized}} \\
                \hline
                43.6 & 45.61 & 13.33 & 13.4 & 90 & 140 \\ \cline{1-6} 
                47.17 & 47.93 & 11.09 & 11.77 & 80 & 90 \\ \cline{1-6} 
                55.19 & 46.04 & 9.42 & 10.11 & 60 & 91 \\ 
                \hline
            \end{tabular}
        }
        \label{tab:2sota_specs}
    \end{minipage}

    \vspace{-2mm}
\end{table}



From the correlation coefficient analysis, we observe that in some cases the coefficients can be relatively lower (e.g., $<$0.8). These cases correspond to scenarios where the corresponding parameter does not impact the performance metrics significantly, while the other parameter has a more dominant influence. This behavior is attributed to the attention mechanism of the transformer which weights the importance of different parameters based on their level of influence on the performance specifications. As a result, the contributions of less impactful parameters may be overshadowed, leading to a lower correlation coefficient in those specific cases. 

% \bluefn{When I read this entire results section, all I see in the text is ``5T-OTA'': you don't even mention the other OTAs. If the reviewer is in a hurry, s/he will think you really have only done 5T-OTAs. Need to write this better to bring out the other two OTA types. Your paper is already weak because of the simplicity of the circuits (compare with any of the other papers on OTA sizing, which show much more complex circuits). Don't shoot yourself in the foot even further.}









\noindent
%\redHL{~{\em 2) LUT-based transistor width estimation.} For each OTA topology,\bluefn{List them by name so that the reader does not just see ``5T OTA'' in the results section.} we use our approach to determine the transistor sizes for 100 distinct performance specifications that are unseen in the training set. For each set of performance specifications, the transformer model predicts the circuit parameters, which are translated to transistor sizes using the LUT-based method. We report the performance of the design based on a Spectre simulation of the sized OTA circuit. Table~\ref{tab:specs} shows the list of target specs and obtained specs\bluefn{No! You don't \underline{obtain} a spec. See earlier comment. \textbf{FIXED}} for three designs from the validation set for each topology. \textbf{IN MY OPINION, WE DON'T NEED THIS RED PART ANYMORE}}



\begin{table}[t]
    % \vspace{-0.2cm}
    \caption{Runtime analysis of training and inferencing stages.} 
    % \blueHL{This table is poorly explained. You never say anywhere that you run 100 designs. I have changed 95 to 95/100, etc., to make this more obvious in the paper, but you also need to say this in the text.  Please read the paper adversarially, from the point of view of the reviewer. Right now, there are many items that will provoke instant rejection from the reviewer, and you have not bothered to try and address these. I've caught what I could, but I am sure there are other issues because I am starting from almost zero. \textbf{NOTED}}}
    \centering
    \resizebox{1\linewidth}{!}{\begin{tabular}{|c|c|c|c|c|c|c|c|}
        \hline
        \multirow{2}{*}{\makecell{\textbf{OTA} \\ \textbf{topology}}} & \multirow{2}{*}{\makecell{\textbf{One-time} \\\textbf{training} \\\textbf{duration}}}  & \multicolumn{2}{l|}{\makecell{\textbf{Single iteration}}} & \multicolumn{3}{l|}{\makecell{\textbf{Multiple iterations}}} \\ \cline{3-7}
        & & \makecell{\textbf{\#designs}\\ \textbf{optimized}} & \makecell{\textbf{Average} \\\textbf{time}} & \makecell{\textbf{\#designs} \\ \textbf{optimized}} &  \makecell{\textbf{Average} \\\textbf{time}} & \makecell{\textbf{Average}\\\textbf{\#iterations}} \\
        \hline
        5T-OTA & 8.5h  & 95/100 & 37s & 5/100 & 111s & 3\\
        \hline
        CM-OTA & 22h  & 98/100 & 46s & 2/100 & 230s & 5\\
        \hline
        2S-OTA & 11h & 90/100 & 36s & 10/100 & 180s & 5\\
        \hline
    \end{tabular}}
    \label{tab:runtime}
    \vspace{-5mm}
\end{table}

% \vspace{-2mm}

\subsection{Runtime analysis}
\noindent
Table~\ref{tab:runtime} provides a detailed runtime analysis, including both the one-time SPICE-based training duration and the average runtime per design optimization by the trained model. The reported runtime encompasses the entire process, from sequence inference by the trained transformer model (taking approximately 0.5s per sequence)
% -- where each sequence takes approximately 0.5 seconds -- 
to the LUT-based estimation and subsequent SPICE simulation verification. 
In cases where performance criteria are not fully met due to minor prediction inaccuracies, a ``copilot'' mode is activated, performing iterative refinements with progressively tighter specifications to introduce a design margin that compensates for errors. This ensures that all design specifications are ultimately satisfied, typically requiring only a few additional iterations, thereby balancing model accuracy with computational efficiency to achieve reliable design convergence. The runtime ranges from just above 30s to just under four minutes, significantly lower than competing methods.
% \vspace{-2mm}

\subsection{Qualitative comparison with prior approaches.}

\noindent
Table~\ref{tab:comparison} compares our approach for OTA sizing with prior methods, including simulated annealing (SA)~\cite{gielen_90}, particle swarm optimization (PSO)~\cite{vural_12}, graph convolutional network-based RL (GCN-RL)~\cite{Wang_2020}, weighted expected improvement-based Bayesian optimization (WEIBO)~\cite{lyu_18}, and differential evolutionary (DE) algorithm~\cite{liu_09}.
% \redfn{You never answered my question re. comparing against~\cite{budak_21}.\textbf{I haven not included ~\cite{budak_21}, because, firstly its RL, and it needs SPICE in the loop for convergence. Although it has significantly reduced the no. of simulations, it still needs $>100$ SPICE simulations. If I want to compare with this, I have to show the exact no. of SPICE Simulations required for our topologies. } {\em (1)~I don't understand your logic here. Have you shown the exact number of SPICE simulations for any of the methods in Table~\ref{tab:comparison}? I don't see it. So, why talk about the number for~\cite{budak_21}? \textbf{By looking at the numbers that Budak and rest of the papers have shown in their comparison, the number of simulations in ~\cite{budak_21} is way less than the rest of all the papers. Not only that, the previous papers using stochastic methods don't even have 100\% success rate. So in summary, Budak paper ~\cite{budak_21} is a close competitor and to beat that, solid number is necessary}. {\em If you need 3-5 iterations (as shown in your table), you need 3-5 SPICE simulations, right? If so, why is a paper with 100 simulations a ``close competitor''?} \textbf{Because their circuits are more complicated. They may not need 100 simulations for 5T OTA. So, its hard to claim that.} {\em OK, I buy your argument. Let's leave it out.} RESOLVED. OK(2)~Also -- I just noticed that you \underline{never} refer to Table~\ref{tab:comparison} anywhere in the paper!! Please double-check that all figures and tables are cited at least once in the text of the paper. It is pointless to drop in a table without citing it in the paper. This section needs a clear pointer to the table (which I have now added in the first sentence).} \textbf{Even I am surprised with this ignorance of mine. I mentioned it initially, but later was rephrasing and deleted it somehow. I checked the rest of the paper for all the figures and tables, and they all are cited at their designated sections.} RESOLVED}

We utilize various metrics for comparison. \textit{SPICE simulation dependency} gauges the reliance on costly simulations for convergence: lower dependence indicates greater efficiency. \textit{Sizing accuracy} measures how well the approach satisfies all design specifications.
% \textit{Optimization efficiency} captures the trade-off between resource usage and the ability to find optimal designs, while sizing accuracy evaluates how reliably each method determines correct transistor sizes.
% \redfn{What is your basis for saying this method is high? You never evaluate whether the sizing is performed at low area or power -- so what is your metric for deciding that this is true? \textbf{I probably thought about this metric in the wrong way. I thought of it as a function of the performance of the optimized design and computational cost for the optimization.} {\em So how did you think about it? What is your basis for saying that the optimization efficiency is high? Can you use a different term that captures what you were trying to say?} \textbf{I feel like this metric is kind of redundant now. We have SPICE simulation dependency metric and Convergence cost, which pretty much covers this metric as well.} {\em I am ok with removing this row of the table.} RESOLVED (from my end).} 
\textit{Runtime} reflects the time required to reach a solution, and \textit{memory utilization} pertains to the amount of memory resources consumed during the optimization process.
% \redHL{\sout{resource usage} runtime}.
% \redfn{Unclear what a ``resource'' is in this context. \textbf{Time and memory due to SPICE simulations} {\em How is this different from ``SPICE simulation dependency'' which seems to already cover this issue?} \textbf{I thought of mentioning this explicitly because at the end overall sizing duration is the most important factor in favor of us.} {\em Can we maybe focus just on runtime and skip memory (which is harder to quantify anyway)? If so, could this row by ``Runtime'' and then you could list it as hours/days/... for others and 0.5s for us. Again, there is the issue of more complicated circuits -- not sure if it makes it hard to use their numbers directly?} \textbf{Prof, I don't think using their numbers would work. } {\em OK. Can we still change it to ``runtime'' or are you trying to say something different?}\textbf{We can definitely use ``runtime''} {\em OK, please make that change. It feels more concrete and is in line with our claims.} RESOLVED OK} 
% \redfn{I had provided feedback on this table, but it is not clear if it was read. There is no change to the table, and I see no email explaining why this form is better. You should not ignore my feedback. If you disagree, that's ok -- but make an argument against what I am saying rather than just ignoring my feedback and throwing it in the trash. \textbf{Prof, I considered your feedback and included the modified comparison by using the terms "Moderate to low", "Very slow" just to distinguish between two "High" and "High" which I did previously. Unfortunately reporting approx no. of SPICE simulations may raise questions. Cause in that case we have to tell about our implementation of the other approaches and then check for no. of simulations for our OTA topologies.} {\em OK, this is a reasonable argument. But you need to make such arguments to me, not unilaterally make decisions without consultation. Your logic is right in this case, but there may be other cases when it is not. We cannot have decisions without discussion.} \textbf{Yes Prof. I totally agree. } RESOLVED} 
Our approach, using a trained transformer model with precomputed LUTs, significantly reduces SPICE simulation dependency -- achieving over 90\% of sizing without simulations -- while improving accuracy and reducing runtime from hours to seconds, positioning it as a highly efficient solution.
% \redfn{One more thing: what is the difference between the last two rows -- runtime and computational cost? And also ``convergence speed''? All seem to be very similar. \textbf{Computational cost mean the total memory utilization due to Simulations } That is not obvious. Maybe use the terms you mean -- e.g., memory, runtime, etc.? I don't think the average reader will understand computational cost == runtime. \textbf{Should I use just ``Memory'' or ``Memory utilization or requirement''?} Either is ok. Maybe ``Memory utilization'' is ok. RESOLVED.}
% \redfn{I don't really see the relationship between the table and these statements. You connect it directly to ``SPICE simulation dependency'' but I don't see any of the other keywords in column 1 of the table. \textbf{I connect to SPICE simulation dependency, convergence speed, and computational cost, because those are the only metrics which are in favor of us, which I have shown in BOLD in the table} {\em Yes, but this text does not use any of those terms. If you want to connect it to the table, use the same terms in the text.} RESOLVED}

% \blueHL{Many comments on the table: (1)~When you say that the training cost is ``High'' in all cases -- is it equally high? Can you be more quantitative by reporting the number of hours? \textbf{I will think about this} (2)~Unclear how ``Handle layout parasitics'' is a yes for us -- we also work on a schematic.  I don't see anything in your paper about handling layout parasitic. \textbf{We haven't demonstrated that, but from principle, the way we use DPSFG, will allow inclusion of more R's and C's into the graph} (3)~``90\% SPICE-independent sizing'' is poorly stated and will not convey what you want to say.  Can you report the number of SPICE simulations instead? Similarly, ``SPICE-based convergence'' is a poor descriptor and the reader will have no idea what you are saying. The footnote clarifies it a bit, but you can put some effort into making yourself more clear in the table too. (4)~``Sizing duration'' sounds amateurish.  Something like ``Runtime of the sizer'' would be better.}

\begin{table}[ht]
    \centering
    \vspace{-2mm}
    \caption{Qualitative comparison with prior sizing approaches. }
    % \redHL{Why are some items bf and others not? If we are trying to bf our approach, why is sizing accuracy = high not in bf?} \textbf{Because, its high for others also, so nothing special in our method.}}
    \resizebox{1\linewidth}{!}{
    \begin{tabular}{|c|c|c|c|c|c|c|}
        \hline
        \makecell{\textbf{Sizing method}} & \makecell{\cite{gielen_90}} & 
        \makecell{\cite{vural_12}} &
        \makecell{\cite{Wang_2020}} &
        \makecell{\cite{lyu_18}} &  
         \makecell{\cite{liu_09}} &
         \makecell{\textbf{Our approach}}
         \\
        \hline
        \textbf{Algorithm} & SA & PSO & \makecell{GCN \\ + RL} & WEIBO & DE & \makecell{Transformer \\ + LUT } \\
        \hline
        \makecell{\textbf{SPICE simulation} \\ \textbf{dependency}} & \makecell{Very high} & \makecell{Very high} & \makecell{Low to \\ moderate}& \makecell{High} &  \makecell{Very high} & \makecell{\textbf{Very low\textsuperscript{*}}}\\
        % \hline
        % \makecell{\textbf{Optimization} \\ \textbf{efficiency}} & \makecell{Moderate \\ to low} & \makecell{Moderate \\ to low} & Moderate &  High & High & High\\
        \hline
        \makecell{\textbf{Sizing accuracy}}& Variable & \makecell{Moderate \\ to high}& High & High & High & \makecell{ High}\\
        \hline
        \makecell{\textbf{Runtime}} & \makecell{Very slow}& \makecell{Very slow}& \makecell{Moderate}& \makecell{Moderate}  & \makecell{Slow}& \makecell{\textbf{Very fast}} \\
        \hline
        \makecell{ \textbf{Memory} \\ \textbf{utilization}} & Moderate & Moderate & \makecell{Moderate \\ to high}& \makecell{High} & \makecell{Very high} & \makecell{\textbf{Moderate} \\ \textbf{to low}}\\
        
        
        \hline
    \end{tabular}}
    \label{tab:comparison}
    \scriptsize\textsuperscript{*} $>$90\% of sizing is performed without SPICE simulations, as shown in Table~\ref{tab:runtime}.
    \vspace{-3mm}
\end{table}





% %Original
% \begin{table}[ht]
%     \centering
%     \vspace{-1mm}
%     \caption{Qualitative comparison with prior sizing approaches.}
%     \resizebox{1\linewidth}{!}{
%     \begin{tabular}{|c|c|c|c|c|c|c|}
%         \hline
%         \makecell{\textbf{Sizing method}} & \makecell{\cite{gielen_90}} & \makecell{\cite{liu_09}} &  
%         \makecell{\cite{vural_12}} & \makecell{\cite{lyu_18}} &
%         \makecell{\cite{Wang_2020}} & \makecell{\textbf{Our} \\ \textbf{approach}}
%          \\
%         \hline
%         \textbf{Algorithm} & SA & DE & PSO & WEIBO & \makecell{GCN \\ + RL} & \makecell{Transformer \\ + LUT } \\
%         \hline
%         \makecell{\textbf{SPICE simulation} \\ \textbf{dependency}} & \makecell{Very \\ high} & \makecell{Very \\ high} & \makecell{Very \\ high} &  \makecell{High} & \makecell{Low to \\ moderate}& \makecell{\textbf{Very} \\ \textbf{low\textsuperscript{*}}}\\
%         % \hline
%         % \makecell{\textbf{Optimization} \\ \textbf{efficiency}} & \makecell{Moderate \\ to low} & \makecell{Moderate \\ to low} & Moderate &  High & High & High\\
%         \hline
%         \makecell{\textbf{Sizing accuracy}}& Variable & High & \makecell{Moderate \\ to high} & High & High & \makecell{ High}\\
%         \hline
%         \makecell{\textbf{Runtime}} & \makecell{Very \\ slow}& \makecell{Slow}& \makecell{Very \\ slow}  & \makecell{Moderate} & \makecell{Moderate} & \makecell{\textbf{Very} \\ \textbf{fast}} \\
%         \hline
%         \makecell{ \textbf{Memory} \\ \textbf{utilization}} & Moderate & \makecell{Very \\ high} & Moderate & High & \makecell{Moderate \\ to high} & \makecell{\textbf{Moderate} \\ \textbf{to low}}\\
        
        
%         \hline
%     \end{tabular}}
%     \label{tab:comparison}
%     \scriptsize\textsuperscript{*}$ >90\%$ of sizing is done without SPICE simulations, as shown in Table~\ref{tab:runtime}.
%     % \vspace{-4.5mm}
% \end{table}


% \redfn{\textbf{**FIXED**}The table has numerous problems: (1)~``next few iterations'' is not a scientific statement. How many iterations? (2)~What is the criterion for tightening the specifications? (3)~Presentation issues: (a)~1st (informal usage) $\rightarrow$ first; (b)~No. $\rightarrow$ number (this is also pointed out elewhere); (c)~Successfully sized designs $\rightarrow$ Number of designs that meet all specifications}










\section{Conclusion}

We justify that a flow matching generative model can produce dense and reliable rewards for training LLMs to explain the decisions of RL agents and other LLMs. 
Looking into the future, we envision extending this method to a general LLM training approach, automatically generating high-quality dense rewards, and ultimately reducing the reliance on human feedback. 

% Our method has the potential to facilitate human-AI collaboration applications, such as transportation, education, and security defense.

\newpage
\section{Impact Statements}
This paper presents work whose goal is to advance the field of machine learning by developing a model-agnostic explanation generator for intelligent agents, enhancing transparency and interpretability in agent decision prediction. The ability to generate effective and interpretable explanations has the potential to foster trust in AI systems, improving effectiveness in high-stakes applications such as healthcare, finance, and autonomous systems. Overall, we believe our work contributes positively to the broader AI ecosystem by promoting more explainable and trustworthy AI.

% \redHL{The number of references is quite small and most of the last page is blank. This was understandable when you were page-limited, but it looks bad when the last page is mostly blank. (Don't you think about these things without being told?) I have tried to mitigate this by using the full form of each journal/conference to take up more space (also changed to IEEEtran instead of ieeetr2, which allows me to turn off ``{\em et al.}''). I am not even sure you have run a good enough literature review. There are no references from Helmut Graeb's group, and he has done a LOT of work on OTAs -- missing his work is a glaring omission. Nothing from Georges Gielen's group. Plus I am sure there are other groups you have missed. {\bf Please answer this question: have you really made the effort to read all papers on OTA sizing?} I suspect not. Don't you think 2 years is enough time to show the will to conduct a full literature review? What do you do with your time??} SSS_NOTE

\newpage
% \bstctlcite{IEEEexample:BSTcontrol}
% \bibliographystyle{alpha}
% \bibliographystyle{misc/ieeetr2}
\bibliographystyle{misc/IEEEtran}
% \bibliography{main.bib}
% Generated by IEEEtran.bst, version: 1.14 (2015/08/26)
\begin{thebibliography}{10}
\providecommand{\url}[1]{#1}
\csname url@samestyle\endcsname
\providecommand{\newblock}{\relax}
\providecommand{\bibinfo}[2]{#2}
\providecommand{\BIBentrySTDinterwordspacing}{\spaceskip=0pt\relax}
\providecommand{\BIBentryALTinterwordstretchfactor}{4}
\providecommand{\BIBentryALTinterwordspacing}{\spaceskip=\fontdimen2\font plus
\BIBentryALTinterwordstretchfactor\fontdimen3\font minus \fontdimen4\font\relax}
\providecommand{\BIBforeignlanguage}[2]{{%
\expandafter\ifx\csname l@#1\endcsname\relax
\typeout{** WARNING: IEEEtran.bst: No hyphenation pattern has been}%
\typeout{** loaded for the language `#1'. Using the pattern for}%
\typeout{** the default language instead.}%
\else
\language=\csname l@#1\endcsname
\fi
#2}}
\providecommand{\BIBdecl}{\relax}
\BIBdecl

\bibitem{harjani_89}
R.~Harjani, R.~Rutenbar, and L.~Carley, ``{OASYS}: A framework for analog circuit synthesis,'' \emph{IEEE Transactions on Computer-Aided Design of Integrated Circuits and Systems}, vol.~8, no.~12, pp. 1247--1266, Dec. 1989.

\bibitem{koza_96}
J.~R. Koza, F.~H. Bennett, D.~Andre, and M.~A. Keane, ``Automated design of both the topology and sizing of analog electrical circuits using genetic programming,'' in \emph{Artificial Intelligence in Design '96}, J.~S. Gero and F.~Sudweeks, Eds.\hskip 1em plus 0.5em minus 0.4em\relax Dordrecht, Netherlands: Springer, 1996, pp. 151--170.

\bibitem{Kruiskamp_95}
W.~Kruiskamp and D.~Leenaerts, ``{DARWIN}: {CMOS} opamp synthesis by means of a genetic algorithm,'' in \emph{Proceedings of the ACM/IEEE Design Automation Conference}, 1995, pp. 433--438.

\bibitem{gielen_90}
G.~Gielen, H.~Walscharts, and W.~Sansen, ``Analog circuit design optimization based on symbolic simulation and simulated annealing,'' \emph{IEEE Journal of Solid-State Circuits}, vol.~25, no.~3, pp. 707--713, Jun. 1990.

\bibitem{vural_12}
R.~A.~Vural and T.~Yildirim, ``Analog circuit sizing via swarm intelligence,'' \emph{AEU -- International Journal of Electronics and Communications}, vol.~66, p. 732–740, Sep. 2012.

\bibitem{abel_22}
I.~Abel and H.~Graeb, ``{FUBOCO}: Structure synthesis of basic op-amps by functional block composition,'' \emph{ACM Transactions on Design Automation of Electronic Systems}, vol.~27, no.~6, Jun. 2022.

\bibitem{abel_22_2}
I.~Abel, M.~Neuner, and H.~E. Graeb, ``A hierarchical performance equation library for basic op-amp design,'' \emph{IEEE Transactions on Computer-Aided Design of Integrated Circuits and Systems}, vol.~41, no.~7, pp. 1976--1989, 2022.

\bibitem{hershenson_01}
M.~Hershenson, S.~Boyd, and T.~Lee, ``Optimal design of a {CMOS} op-amp via geometric programming,'' \emph{IEEE Transactions on Computer-Aided Design of Integrated Circuits and Systems}, vol.~20, no.~1, pp. 1--21, Jan. 2001.

\bibitem{budak_21}
A.~F. Budak, P.~Bhansali, B.~Liu, N.~Sun, D.~Z. Pan, and C.~V. Kashyap, ``{DNN-Opt}: An {RL} inspired optimization for analog circuit sizing using deep neural networks,'' in \emph{Proceedings of the ACM/IEEE Design Automation Conference}, 2021, pp. 1219--1224.

\bibitem{settaluri_20}
K.~Settaluri, A.~Haj-Ali, Q.~Huang, K.~Hakhamaneshi, and B.~Nikolic, ``{AutoCkt}: Deep reinforcement learning of analog circuit designs,'' in \emph{Proceedings of the Design, Automation \& Test in Europe}, 2020, pp. 490--495.

\bibitem{Wang_2020}
H.~Wang, K.~Wang, J.~Yang, L.~Shen, N.~Sun, H.-S. Lee, and S.~Han, ``{GCN-RL} circuit designer: Transferable transistor sizing with graph neural networks and reinforcement learning,'' in \emph{Proceedings of the ACM/IEEE Design Automation Conference}, 2020, pp. 1--6.

\bibitem{choi_23}
M.~Choi, Y.~Choi, K.~Lee, and S.~Kang, ``Reinforcement learning-based analog circuit optimizer using {$g_m/I_D$} for sizing,'' in \emph{Proceedings of the ACM/IEEE Design Automation Conference}, 2023.

\bibitem{vaswani_17}
A.~Vaswani, N.~Shazeer, N.~Parmar, J.~Uszkoreit, L.~Jones, A.~N. Gomez, L.~Kaiser, and I.~Polosukhin, ``Attention is all you need,'' in \emph{Advances in Neural Information Processing Systems}, vol.~30, Dec. 2017, pp. 5998--6008.

\bibitem{ochoa_98}
A.~Ochoa, ``A systematic approach to the analysis of general and feedback circuits and systems using signal flow graphs and driving-point impedance,'' \emph{IEEE Transactions on Circuits and Systems II}, vol.~45, no.~2, pp. 187--195, Feb. 1998.

\bibitem{schmid_18}
H.~Schmid and A.~Huber, ``Analysis of switched-capacitor circuits using driving-point signal-flow graphs,'' \emph{Analog Integrated Circuits and Signal Processing}, vol.~96, pp. 495--507, Sep. 2018.

\bibitem{Mason53}
S.~J. Mason, ``Feedback theory-some properties of signal flow graphs,'' \emph{Proceedings of the IRE}, vol.~41, no.~9, pp. 1144--1156, 1953.

\bibitem{schmid_yt}
{H. Schmid}, ``{{HT FHNW EIT}: Analog and mixed-signal circuits and signal processing},'' \url{https://tube.switch.ch/channels/d206c96c}.

\bibitem{rico_16}
R.~Sennrich, B.~Haddow, and A.~Birch, ``Neural machine translation of rare words with subword units,'' in \emph{Annual Meeting of the Association for Computational Linguistics}, Aug. 2016, pp. 1715--1725.

\bibitem{silviera_96}
F.~Silveira, D.~Flandre, and P.~Jespers, ``A {$g_m$/$I_D$} based methodology for the design of {CMOS} analog circuits and its application to the synthesis of a silicon-on-insulator micropower {OTA},'' \emph{IEEE Journal of Solid-State Circuits}, vol.~31, no.~9, pp. 1314 -- 1319, Oct. 1996.

\bibitem{jespers_17}
P.~Jespers and B.~Murmann, \emph{Systematic Design of Analog {CMOS} Circuits: Using Pre-Computed Lookup Tables}.\hskip 1em plus 0.5em minus 0.4em\relax Cambridge, UK: Cambridge University Press, 2017.

\bibitem{lyu_18}
W.~Lyu, P.~Xue, F.~Yang, C.~Yan, Z.~Hong, X.~Zeng, and D.~Zhou, ``An efficient {Bayesian} optimization approach for automated optimization of analog circuits,'' \emph{IEEE Transactions on Circuits and Systems I}, vol.~65, no.~6, pp. 1954--1967, Jun. 2018.

\bibitem{liu_09}
B.~Liu, Y.~Wang, Z.~Yu, L.~Liu, M.~Li, Z.~Wang, J.~Lu, and F.~V. Fernández, ``Analog circuit optimization system based on hybrid evolutionary algorithms,'' \emph{Integration}, vol.~42, no.~2, pp. 137--148, Apr 2009.

\end{thebibliography}

%misc/cram,misc/main.bib,misc/pim.bib}

%\newpage
\appendix
\section{Appendix}

\subsection{Conversational agent prompts for generating stable diffusion prompts in art-making phase}

\textbf{Role:} You will be able to capture the essence of the sessions and drawings in the recordings based on the art therapy session recordings I have given you and summarize them into a short sentence that will be used to guide the PROMPT for the Stable Diffusion model.

\vspace{0.5em} % 添加一些垂直间距

\textbf{Example input:}

\begin{itemize}[leftmargin=*]
    \item \textbf{USER:} [user-drawn] I drew the ocean. [canvas content] There is nothing on the canvas right now.
    \item \textbf{ASSISTANT:} What kind of ocean is this?
    \item \textbf{USER:} [user-drawn] I drew grass. [canvas content] Now there is an ocean on the canvas.
    \item \textbf{ASSISTANT:} What kind of grass is this?
    \item \textbf{USER:} [user-drawn] I drew the sky. [canvas content] Now there is grass and ocean on the canvas.
    \item \textbf{ASSISTANT:} What kind of sky is this?
    \item \textbf{USER:} [user-drawn] I drew mountains. [canvas content] Now there is sky, grass, and ocean on the canvas.
    \item \textbf{ASSISTANT:} What kind of mountain is this?
    \item \textbf{USER:} [user-drawn] I drew clouds. [canvas content] Now there is sky, mountain, grass, and ocean on the canvas.
    \item \textbf{ASSISTANT:} What kind of cloud is this?
    \item \textbf{USER:} [user dialogue] Colorful clouds, emerald green mountains and grass, choppy ocean
\end{itemize}

\vspace{0.5em} % 添加一些垂直间距

\textbf{Task:}

\begin{enumerate}[label=\textbf{Step \arabic*:}]
    \item \textbf{[Step 0]} Read the given transcript of the art therapy session, focusing on the content of \texttt{user: [user drawing]} and \texttt{user: [user dialog]}: Go to \textbf{[Step 1]}.
    \item \textbf{[Step 1]} Based on the input, find the last entry of user's input with \texttt{[canvas content]}, find the keywords of the screen elements that the canvas now contains (in the example input above, it is: sky, grass, sea), separate the keywords of each element with a comma, and add them to the generated result. Examples: [keyword1], [keyword2], [keyword3], \dots, [keyword n].
    \item \textbf{[Step 2]} Find whether there are more specific descriptions of the keywords of the painting elements in \texttt{[Step 1]} in \texttt{[User Dialog]} according to the input. If not, this step ends into \textbf{[Step 3]}; if there are, combine these descriptions and the keywords corresponding to the descriptions into a new descriptive phrase, and replace the previous keywords with the new phrases. Examples: [description of keyword 1] [keyword 1], [keyword 2 description of keyword 2], [description of keyword 3], \dots. Based on the above example input, the output is: rough sea, lush grass, blue sky.
    \item \textbf{[Step 3]} Based on the input, find out if there is a description of the painting style in the \texttt{[User Dialog]} in the dialog record, and if there is, add the style of the picture as a separate phrase after the corresponding phrase generated in \texttt{[Step 2]}, separated by commas. For example: [description of keyword 1] [keyword 1], [description of keyword 2] [keyword 2], \dots, [screen style phrase 1], [screen style phrase 2], [screen style phrase 3], \dots, [Picture Style Phrase n].
\end{enumerate}

\vspace{0.5em} % 添加一些垂直间距

\textbf{Output:} 

Only need to output the generated result of \textbf{[Step 3]}.

\vspace{0.5em} % 添加一些垂直间距

\textbf{Example output:} 

\emph{Rough sea, lush grass}

\subsection{Conversational agent prompts for discussion phase}

\textbf{Role:} <therapist\_name>, Professional Art Therapist

\textbf{Characteristics:} Flexible, empathetic, honest, respectful, trustworthy, non-judgmental.

\vspace{0.5em} % 添加垂直间距

\textbf{Task:} Based on the user's dialogic input, start sequentially from step [A], then step [B], to step [C], step [D], step [E] \dots Step [N] will be asked in a dialogical order, and after step [N], you can go to \textbf{Concluding Remarks}. You can select only one question to be asked at a time from the sample output display of step [N]! You have the flexibility to ask up to one round of extended dialog questions at step [N] based on the user's answers. Lead the user to deeper self-exploration and emotional expression, rather than simply asking questions.

\vspace{0.5em} % 添加垂直间距

\textbf{Operational Guidelines:}

\begin{enumerate}
    \item You must start with the first question and proceed sequentially through the steps in the conversational process (step [A], step [B], step [C], step [D], step [E], \dots, step [N]).
    \item Do not include references like step '[A]', step '[B]' directly in your reply text.
    \item You may include one round of extended dialog questions at any step [N] depending on the user's responses and situation. After that, move on to the next step.
    \item Always ensure empathy and respect are present in your responses, e.g., re-telling or summarizing the user's previous answer to show empathy and attention.
\end{enumerate}

\vspace{0.5em} % 添加垂直间距

\textbf{Therapist’s Configuration:}

\textbf{Principle 1:}  
\textit{Sample question:} How are you feeling about what you are creating in this moment?

\vspace{0.5em}

\textbf{Principle 2:}  
\textit{Sample question:} Can you share with me what this artwork represents to you personally? 

\vspace{0.5em}

\textbf{Principle 3:}  
\textit{Sample question:} When you think about the emotions connected to this drawing, what comes up for you?

\vspace{0.5em}

\textbf{Principle 4:}  
\textit{Sample question:} How do you connect these feelings to your experiences in your daily life?

\vspace{0.5em} % 添加垂直间距

\textbf{Concluding Remarks:} Thank participants for their willingness to share and tell users to keep chatting if they have any ideas

\vspace{1em} % 添加额外的间距

\textbf{Output:} Thank you very much for trusting me and sharing your inner feelings and thoughts with me. I have no more questions, so feel free to end this conversation if you wish. Or, if you wish, we can continue to talk.

\subsection{AI summary prompts}

\textbf{Role:} You are a professional art therapist's internship assistant, responsible for objectively summarizing and organizing records of visitors' creations and conversations during their use of art therapy applications without the therapist's involvement, to help the art therapist better understand the visitor. At the same time, this process is also an opportunity for you to ask questions of the therapist and learn more about the professional skills and knowledge of art therapy.

\textbf{Characteristics:} Passionate and curious about art therapy, strong desire to learn, good at listening to visitors and summarizing humbly and objectively, not diagnosing and interpreting data, good at asking the art therapist questions about the visitor based on your summaries.

\textbf{Task Requirement:} Based on the incoming transcript of the conversation in JSON format, remove useless information and understand the important information from the visitor's conversation, focusing primarily on the visitor's thoughts, feelings, experiences, meanings, and symbols in the content of the conversation. Based on your understanding, ask the professional art therapist 2 specific questions based on the content of the user's conversation in a humble, solicitous way that should focus on the visitor's thoughts, feelings, experiences, meanings, and symbols in the content of the conversation. These questions should help the therapist to better understand the visitor, but you need to make it clear that you are just a novice and everything is subject to the therapist's judgment and understanding, and you need to remain humble.

\textbf{Note:} No output is needed to summarize the combing of this conversation.



%\clearpage
% \newpage
%\input{sec/10-supplementary_material}
% \blueHL{checks}
% \begin{itemize}
%     \item Please use $\sqrt{xyz}$ instead of $\sqrt xyz$. Your sqrt sign only appears over x if you do the latter.
% \item Please number the first equation in Sec IIIA.
% \item Please use $\log$ instead of log within an equation.
% \item Also use $\left ( and \right )$ in scenarios where the parentheses are too small, e.g., in (2).
% \item Please double-check that notation is suitably italicized in the text, (e.g., use $B$ in the line after eq. (1) instead of B; there are several other instances).
% \item When you write "where..." after an equation, it is a continuation of the sentence that contains the equation. So, you should use no indentation and should not capitalize it ("Where...")
% \item Remove single period that is alone on a line after the P\_NOISE equation.
% \item PJ -> pJ
% \item Never start a sentence with "E.g" - it just does not work that way. Use "For example, ..." if you need to say that.
% \item I assume you will do this anyway, but just to be complete
% Pay attention to figure sizes, text readability, etc.
% \item Make your captions more descriptive (e.g., all figures on p4 have very short/insufficiently descriptive captions). Check all figures.
% \item Check for "( xxx" - extra space after "(" - this is a common issue in your writing and easy to fix with search-replace.
% \item There are numerous easy-to-find issues in the reference section (capitalization problems, at least one incorrect journal name, capitalization problems in the month, etc.)
% \end{itemize}
\end{document}
