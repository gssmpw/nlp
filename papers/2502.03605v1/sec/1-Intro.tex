\vspace{-1mm}
\section{Introduction}
\label{sec:Intro}

\noindent
% \bluefn{It does not appear that you have learned very much from my comments on the abstract. You have taken the same lazy approach as in the abstract: the entire introduction contains just one instance of the term ``OTA.'' Exactly why do you believe this is a good idea, even when you have been told very clearly that it is not? \textbf{Prof, I thought, in Abstract it has to be specific to OTA, cause it is a concise version of the paper. But, in the introduction, I thought of starting in general in Intro } It's the same paper, isn't it? And if you job is to convince the reviewers, you should try to tell a coherent story that is to the point. Otherwise, why not talk about VLSI design in general? Or electrical engineering in general? You need to convince the reviewer that you have a good contribution and a good story. It is not helpful if your presentation is not to-the-point. \textbf{Ok prof, I agree. I should make a more coherent approach. I am working on this now.}} SSS_NOTE
% \bluefn{I will insert comments here and rewrite this. The logic is questionable, the grammar has problems, and we don't have time for you to fix this. re you really using Grammarly/equivalent? \textbf{Yes Prof, I use Grammarly. I will modify this section} \redHL{PLEASE DO NOT!!!! I DON'T HAVE TIME FOR YOUR ERRORS NOW. THE DEADLINE IS TOMORROW MORNING. THIS SHOULD HAVE BEEN WRITTEN WELL A WEEK AGO.} The grammar here is pretty bad, and much worse than what other students turn in when they use grammar aids. I spent a lot of effort in writing this part well, and your changes have ruined things. I will look for the old version.  This just does not cut it and I am not going to waste my time rewriting something that I had already polished.\textbf{I modified this just to comply with the new title of OTA sizing method. But I have included the older version also now.}}
% \redfn{Prof, this is the older version. \textbf{
% Transistor sizing in analog circuits has long been a time-consuming process, relying strongly on human experts. However, a “perfect storm” of events is driving the need for greater automation: increased demand for analog circuits within larger systems; a diminishing analog designer workforce that is hard-pressed to fulfill this demand; and increased design complexity due to increasingly complex circuit models in advanced nodes. In these circumstances, it is imperative to develop automation techniques that improve designer productivity.}}
% Operational transconductance amplifiers (OTAs) play a crucial in analog circuits, performing tasks such as amplification, filtering, and signal conditioning.\bluefn{Are you really using Grammarly/equivalent? The grammar here is pretty bad, and much worse than what other students turn in when they use grammar aids.} Their ability to convert differential input voltages into an output current\bluefn{Is the output really a current? Isn't it a voltage? Are you talking about a DP here or an OTA?} with tunable gain makes them essential in analog systems.\bluefn{Incomplete argument. How does this ability make them essential in analog systems?} \blueHL{\sout{such as phase-locked loops and analog-to-digital converters.}}\bluefn{Is this really true? My impression is that a PLL has a phase comparator and VCO. Even if this were true, this also is possibly a self-defeating argument: you could automate the design of the OTA, but you still need to build a VCO or a matched R/2R array, or a capacitor array, or other parts of the filters involved in these circuits. Can we try to avoid talking about these examples completely?} \blueHL{\sout{Despite their broad applicability,}}\bluefn{Poorly constructed argument: what does the broad applicability of OTA design have to do with the complexity of their design? There is no causal relationship between the two.} The process of designing OTAs requires the optimization of conflicting performance metrics within strict constraints, resulting in a time-consuming design process that requires intensive intervention from the expert human designer. automated sizing, which together with a shrinking pool of analog designers, and increasing \blueHL{\sout{circuit}}\bluefn{The circuit is largely the same. It's the device models/nanometer scale effects that cause complexity} complexity in advanced technology nodes, all underscore the need for automation techniques to improve design productivity and efficiency.

\noindent
A ``perfect storm'' of events is driving the need for greater automation in analog design: increased demand for analog circuits within larger systems; a diminishing designer workforce that is hard-pressed to fulfill this demand; and increased design complexity due to the complexity of circuit models in advanced nodes. Today, it is imperative to develop automation techniques that improve designer productivity.\blfootnote{This work was supported in part by NSF (award 2212345) and SRC.} 

One of the most critical blocks in many analog systems is the operational transconductance amplifier (OTA), which performs tasks such as amplification, filtering, and signal conditioning.  Transistor sizing for OTAs has long been a time-consuming process, relying on human experts to optimize circuit metrics under strict performance constraints. Early automated approaches employed knowledge-based methods~\cite{harjani_89}, but codifying expertise into exhaustive rules is challenging, especially with the need for updates across technology nodes. 

% Equation-based approaches in~\cite{abel_22, abel_22_2} are suitable for older technology nodes, but design in more recent nodes involves more complex device models and SPICE simulation. \redfn{Can remove some if required. I am not sure if I should cite multiple papers from the same author. Can keep 2, remove 1. Use your judgment on which ones to keep. And I still see inconsistencies in the refs (e.g., [7] provides an online link for some reason. The ref list needs more careful inspection.)}, yet they often become intractable for complex topologies\bluefn{Not a good way to criticize this. You have the same failings for the results you show.} and deviate from actual performance. 
% \blueHL{\sout{Methods based on stochastic search, such as genetic algorithms~\cite{Kruiskamp_95}, simulated annealing~\cite{gielen_90}, evolutionary algorithms~\cite{liu_09}, and particle swarm optimization~\cite{vural_12}, rely heavily on SPICE simulations, leading to slow optimization and challenges with convergence.}}
% \bluefn{Again, you have messed up my corrections here. I had SPECIFICALLY pointed out that ``heavily'' is not appropriate usage in formal writing. \textbf{Ok Prof.}}
Methods based on stochastic search (genetic algorithms~\cite{koza_96, Kruiskamp_95}, simulated annealing~\cite{gielen_90}, 
and particle swarm optimization~\cite{vural_12}) have been explored for OTA optimization, but accuracy requires numerous expensive SPICE simulations within the optimization loop.
Equation-based techniques~\cite{abel_22, abel_22_2} and approaches based on convex optimization, such as geometric programming using posynomial-form models~\cite{hershenson_01}, work for simplified MOS models, but can falter in the face of the complex device models in advanced nodes. 
% Methods such as\bluefn{NEVER use ``like'' in formal writing. Use ``such as'' instead. I am pretty sure I have said this before to you. Please search and replace all such instances. If I work at such low-level issues, I will never get around to giving you useful feedback. I am a professor, not a secretary. Use my abilities as a professor, not as a proofreader for your errors and careless work.} GASPAD~\cite{liu_14}, developed for mm wave IC synthesis,\bluefn{This appears to be complete nonsense. Does GASPAD use OTAs???? What are you thinking? And can you not take the trouble to fix the references? Why do you have ``Gaspad'' in the references instead of ``GASPAD''?? I have laboriously fixed all your errors, but you put in zero effort on your part and ruin the reference section with you poor work. This really is very poor.} use surrogate models such as Gaussian process regression (GPR)~\cite{rasmussen_2004}. While GPR has advantages, its cubic computational complexity, $O(N^3)$, limits its scalability as the sample size grows. Additionally, Bayesian optimization has been used with GPR ~\cite{lyu_18}, but they require numerous circuit simulations, rendering them computationally inefficient.\bluefn{I am not sure how much of this paragraph makes sense beyond the description of GASPAD. Let me come back to this later. If you can fix it, it will be helpful. Please, use your intelligence. Why are you talking about mm-wave circuits in a paper about a low-frequency circuit such as an OTA???? If you leave your brains at home, you may as well not bother writing a paper.}
Recent works have proposed ML-based techniques such as DNN-Opt~\cite{budak_21}, AutoCkt~\cite{settaluri_20}, and GCN-RL~\cite{Wang_2020}, and an RL approach using
% $g_m/I_D$ method~\cite{silviera_96} for
sensitivity analysis~\cite{choi_23}. 
% While these methods reduce the number of simulations compared to older black-box approaches, 
For these methods, OTA sizing under each new set of performance specifications  requires numerous SPICE simulations.
% \bluefn{Please check that these all handle OTAs. There is zero effort in the writeup to link any of these references to OTAs. At this point, I give up. I have no time to fix this, and I have no time to fix any ``corrections'' you may make, because those are poorly thought through and require substantial effort on my part. Let's go with this since it seems to be the best you want to do, and you don't think it is worth doing better. I give up. I cannot teach you to WANT to do better. I will just do the secretarial work of fixing your errors. Apparently that is what you need because you don't want to put any effort into doing better.\textbf{No Prof, I will do a quick check.}}
% \bluefn{Unclear. You also face challenges due to circuit topology -- you need to retrain for each topology. So why are these methods being criticized here? \textbf{The biggest drawback of these approaches, which we are solving is SPICE dependency. So, I think I should focus on that only. Even my approach is topology dependent. I commented that part.}} SSS_NOTE 

We present a transistor sizing approach for OTA circuits using a transformer architecture~\cite{vaswani_17}. This method employs an attention mechanism to capture complex nonlinear relationships between device parameters and circuit performance, addressing the intricacies of modern technology nodes. A key feature of our approach, in contrast with prior methods, is that the cost of SPICE simulations is confined to a one-time training phase. \textit{In contrast with prior methods, for a given set of specifications, in the inference phase, the sizing solution can be obtained rapidly with very few SPICE simulations.}

The transformer requires the circuit characteristics to be represented as a character sequence that forms a set of input tokens to the transformer. We achieve this by utilizing the driving-point signal flow graph (DP-SFG) from~\cite{ochoa_98}, later modified in~\cite{schmid_18}, to facilitate direct mapping of the schematic to a graph. We note that the DP-SFG representation of a circuit serves as a descriptive language, translating the behavior of the circuit into a character string that encapsulates its parameters and structure. Based on this observation, we map the transistor sizing problem to a language-translation task akin to natural language processing (NLP). For a given query representing desired performance specifications, the transformer is trained to output a string with the DP-SFG parameter values that meet these specifications. We translate DP-SFG parameters to predict the transistor sizes based on precomputed look-up tables (LUTs). 

An ML-based approach does not guarantee perfect accuracy, and occasionally, the predicted design point may show minor violations in the specifications in some test cases. 
% \redHL{I had specifically changed the language to fix these sentences. Do you have the old version that I had fixed? Please do not overwrite my changes and make me duplicate effort. {\em That is not it. I had removed the part that said ``No ML-based approach...'' Why do you overwrite my changes when you don't know how to write?? I will have to do this again.} \textbf{PROF, I won't do that again.} You are going to cause me to have a late night because of your thoughtlessness. I am not at all happy.\textbf{Prof, this will never happen again.} It's ok to change, but at least don't DELETE my changes. Put them in comments, or save a copy, or do something that allows recoverability. Now I have to waste my time on doing this all over again.\textbf{I will keep that in mind.} 
% \redHL{I will keep this as is. I don't really have time to fix it now, and while the new wording would have made this better, nothing that is stated here is incorrect.} 
In these instances, the designer can re-invoke the fast transformer-based method with tighter design specifications until all requirements are met. Thus, the method acts as a copilot, keeping the designer in the loop. 

The key contributions of our work are as follows:
\vspace{-0.05cm}
\begin{itemize}
\setlength\itemsep{0.05cm}
    \item We implement an automated framework to map an OTA circuit to its equivalent DP-SFG. The paths of this DP-SFG are encoded and concatenated in a language that describes circuit behavior.

    \item We create a labeled training set for our ML model, corresponding to a range of device sizes, over multiple OTA topologies, and evaluate performance metrics using SPICE simulations. 
    
    \item We customize a transformer-based encoder-decoder to encode paths in the DP-SFG
    % ; based on the training data, the transformer 
    to predict the DP-SFG parameters such as transconductance $(g_m)$ and capacitance values.
    
    \item We utilize our LUTs, precharacterized using SPICE simulations, to translate the outputs of the transformer into device sizes, thus providing the sizing solution that meet specifications.
\end{itemize}
Our approach is flexible enough to be used within a layout optimization loop. After sizing, a layout engine updates parasitics, updating the parasitic values in the DP-SFG.  Our model, trained on a range of values, can then be re-invoked without further SPICE simulations.

The paper is structured as follows. Section~\ref{sec:background} overviews a set of core building blocks used in our approach, and is followed by Section~\ref{sec:sizing_framework}, which details our proposed sizing framework and its implementation. Next, Section~\ref{sec:results} details our experimental setup and findings, and finally, Section~\ref{sec:conclusion} concludes the paper.