\section{Conclusion}
\label{sec:conclusion}

\noindent
We have introduced a fully automated rapid-sizing tool for OTA circuits that utilizes a transformer-based attention mechanism. Our framework successfully meets stringent design specifications across multiple unseen designs for three distinct OTA topologies, achieving a success rate exceeding 90\% of the designs on the first attempt. Unlike much prior research, our method eliminates the need for extensive and costly SPICE simulations for each design, enhancing computational efficiency and accelerating the design process for OTA circuits. 
% \sout{The effectiveness of our model is primarily due to the attention mechanism of the transformer, which enables the integration of amplifier circuit behavior derived from complex SPICE~device~models.}\redfn{I don't really see what the basis for this last sentence is. Is it an opinion? If so, it does not belong here. Is it an inference from the data? If so, the logic should be explained more clearly. \textbf{It is an inference from the data which is validating the principle of attention mechanism.} {\em How does the data show this at all? To show it, you need to run an ablation study of some sort, where you remove the attention mechanism and add it, and then show that there is a difference. I don't even know how you would do this, because I don't know how you would remove the attention mechanism from a transformer.  But in the absence of such an experiment, I cannot see how your data leads you to this inference. Please explain clearly.} \textbf{I think, we should remove this sentence from Conclusion. If you look at the para just before Section IV - D, there I talked about the reason behind low correlation values for some parameters. There I think this sentence is more apt.} {\em Agreed. In the conclusion, the scope is much broader and it is harder to justify without concrete experimental data.}RESOLVED}

% Leveraging this framework minimizes the need for iterative and expensive simulations. Moreover, it accurately discerns dependencies among individual device parameters, enhancing overall performance.   
% \blueHL{\sout{Our approach is demonstrated on amplifiers and may be extended to switched capacitor circuits, or even to circuits such as PLLs that are nonlinear in the voltage domain, but admit linear representations in the phase domain for perturbations near a steady-state. We expect to explore this in future work.}}\blueHL{No benefit to talking about this, now that the title limits the work to OTAs.} SSS_NOTE

