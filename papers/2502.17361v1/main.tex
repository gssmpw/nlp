
\documentclass{article}


\usepackage{enumitem}
\usepackage{algorithm}
\usepackage{algorithmic}
\usepackage{textcomp}
% \usepackage{csquotes}
\usepackage{microtype}
\usepackage{graphicx}
\usepackage{booktabs}
\usepackage{hyperref}
\usepackage{amsmath}
\usepackage{amssymb}
\usepackage{mathtools}
\usepackage{amsthm}
\usepackage[capitalize,noabbrev]{cleveref}
\usepackage[title]{appendix}
\usepackage{subcaption}
\usepackage{wrapfig}
\usepackage[utf8]{inputenc} % allow utf-8 input
\usepackage[T1]{fontenc}    % use 8-bit T1 fonts
\usepackage{url}            % simple URL typesetting
\usepackage{xspace}
\usepackage[export]{adjustbox}
\usepackage{amsfonts}       % blackboard math symbols
\usepackage{nicefrac}       % compact symbols for 1/2, etc.
\usepackage{microtype}      % microtypography
\usepackage{wasysym}
\usepackage{enumitem}
\usepackage[american]{babel}
\usepackage{multirow}
\usepackage{adjustbox}
\usepackage{hyperref}       % hyperlinks
\usepackage{booktabs}       % professional-quality tables
\usepackage{xcolor}         % colors
\usepackage{flushend}
\usepackage{tabularx}
\usepackage{longtable}

\usepackage{blindtext}
\usepackage{tikz} % For drawing the color blocks
\usepackage{xcolor} % For defining custom colors

% Define custom colors based on colors_type
\definecolor{CoralOrange}{HTML}{FF7043} % B
\definecolor{VibrantGreen}{HTML}{66BB6A} % C
\definecolor{RichRed}{HTML}{EF5350} % D
\definecolor{SoftIndigo}{HTML}{5C6BC0} % E
\definecolor{EmeraldTeal}{HTML}{26A69A} % F
\definecolor{VividPurple}{HTML}{AB47BC} % G
\definecolor{BrightCyan}{HTML}{26C6DA} % H
\definecolor{FreshLime}{HTML}{D4E157} % I
\definecolor{EmeraldTeal}{HTML}{26A69A}
% Attempt to make hyperref and algorithmic work together better:
\newcommand{\theHalgorithm}{\arabic{algorithm}}

% Use the following line for the initial blind version submitted for review:
% \usepackage{icml2025}

% If accepted, instead use the following line for the camera-ready submission:
\usepackage[accepted]{sections/icml2025}

%%%%% NEW MATH DEFINITIONS %%%%%

\usepackage{amsmath,amsfonts,bm}
\usepackage{derivative}
% Mark sections of captions for referring to divisions of figures
\newcommand{\figleft}{{\em (Left)}}
\newcommand{\figcenter}{{\em (Center)}}
\newcommand{\figright}{{\em (Right)}}
\newcommand{\figtop}{{\em (Top)}}
\newcommand{\figbottom}{{\em (Bottom)}}
\newcommand{\captiona}{{\em (a)}}
\newcommand{\captionb}{{\em (b)}}
\newcommand{\captionc}{{\em (c)}}
\newcommand{\captiond}{{\em (d)}}

% Highlight a newly defined term
\newcommand{\newterm}[1]{{\bf #1}}

% Derivative d 
\newcommand{\deriv}{{\mathrm{d}}}

% Figure reference, lower-case.
\def\figref#1{figure~\ref{#1}}
% Figure reference, capital. For start of sentence
\def\Figref#1{Figure~\ref{#1}}
\def\twofigref#1#2{figures \ref{#1} and \ref{#2}}
\def\quadfigref#1#2#3#4{figures \ref{#1}, \ref{#2}, \ref{#3} and \ref{#4}}
% Section reference, lower-case.
\def\secref#1{section~\ref{#1}}
% Section reference, capital.
\def\Secref#1{Section~\ref{#1}}
% Reference to two sections.
\def\twosecrefs#1#2{sections \ref{#1} and \ref{#2}}
% Reference to three sections.
\def\secrefs#1#2#3{sections \ref{#1}, \ref{#2} and \ref{#3}}
% Reference to an equation, lower-case.
\def\eqref#1{equation~\ref{#1}}
% Reference to an equation, upper case
\def\Eqref#1{Equation~\ref{#1}}
% A raw reference to an equation---avoid using if possible
\def\plaineqref#1{\ref{#1}}
% Reference to a chapter, lower-case.
\def\chapref#1{chapter~\ref{#1}}
% Reference to an equation, upper case.
\def\Chapref#1{Chapter~\ref{#1}}
% Reference to a range of chapters
\def\rangechapref#1#2{chapters\ref{#1}--\ref{#2}}
% Reference to an algorithm, lower-case.
\def\algref#1{algorithm~\ref{#1}}
% Reference to an algorithm, upper case.
\def\Algref#1{Algorithm~\ref{#1}}
\def\twoalgref#1#2{algorithms \ref{#1} and \ref{#2}}
\def\Twoalgref#1#2{Algorithms \ref{#1} and \ref{#2}}
% Reference to a part, lower case
\def\partref#1{part~\ref{#1}}
% Reference to a part, upper case
\def\Partref#1{Part~\ref{#1}}
\def\twopartref#1#2{parts \ref{#1} and \ref{#2}}

\def\ceil#1{\lceil #1 \rceil}
\def\floor#1{\lfloor #1 \rfloor}
\def\1{\bm{1}}
\newcommand{\train}{\mathcal{D}}
\newcommand{\valid}{\mathcal{D_{\mathrm{valid}}}}
\newcommand{\test}{\mathcal{D_{\mathrm{test}}}}

\def\eps{{\epsilon}}


% Random variables
\def\reta{{\textnormal{$\eta$}}}
\def\ra{{\textnormal{a}}}
\def\rb{{\textnormal{b}}}
\def\rc{{\textnormal{c}}}
\def\rd{{\textnormal{d}}}
\def\re{{\textnormal{e}}}
\def\rf{{\textnormal{f}}}
\def\rg{{\textnormal{g}}}
\def\rh{{\textnormal{h}}}
\def\ri{{\textnormal{i}}}
\def\rj{{\textnormal{j}}}
\def\rk{{\textnormal{k}}}
\def\rl{{\textnormal{l}}}
% rm is already a command, just don't name any random variables m
\def\rn{{\textnormal{n}}}
\def\ro{{\textnormal{o}}}
\def\rp{{\textnormal{p}}}
\def\rq{{\textnormal{q}}}
\def\rr{{\textnormal{r}}}
\def\rs{{\textnormal{s}}}
\def\rt{{\textnormal{t}}}
\def\ru{{\textnormal{u}}}
\def\rv{{\textnormal{v}}}
\def\rw{{\textnormal{w}}}
\def\rx{{\textnormal{x}}}
\def\ry{{\textnormal{y}}}
\def\rz{{\textnormal{z}}}

% Random vectors
\def\rvepsilon{{\mathbf{\epsilon}}}
\def\rvphi{{\mathbf{\phi}}}
\def\rvtheta{{\mathbf{\theta}}}
\def\rva{{\mathbf{a}}}
\def\rvb{{\mathbf{b}}}
\def\rvc{{\mathbf{c}}}
\def\rvd{{\mathbf{d}}}
\def\rve{{\mathbf{e}}}
\def\rvf{{\mathbf{f}}}
\def\rvg{{\mathbf{g}}}
\def\rvh{{\mathbf{h}}}
\def\rvu{{\mathbf{i}}}
\def\rvj{{\mathbf{j}}}
\def\rvk{{\mathbf{k}}}
\def\rvl{{\mathbf{l}}}
\def\rvm{{\mathbf{m}}}
\def\rvn{{\mathbf{n}}}
\def\rvo{{\mathbf{o}}}
\def\rvp{{\mathbf{p}}}
\def\rvq{{\mathbf{q}}}
\def\rvr{{\mathbf{r}}}
\def\rvs{{\mathbf{s}}}
\def\rvt{{\mathbf{t}}}
\def\rvu{{\mathbf{u}}}
\def\rvv{{\mathbf{v}}}
\def\rvw{{\mathbf{w}}}
\def\rvx{{\mathbf{x}}}
\def\rvy{{\mathbf{y}}}
\def\rvz{{\mathbf{z}}}

% Elements of random vectors
\def\erva{{\textnormal{a}}}
\def\ervb{{\textnormal{b}}}
\def\ervc{{\textnormal{c}}}
\def\ervd{{\textnormal{d}}}
\def\erve{{\textnormal{e}}}
\def\ervf{{\textnormal{f}}}
\def\ervg{{\textnormal{g}}}
\def\ervh{{\textnormal{h}}}
\def\ervi{{\textnormal{i}}}
\def\ervj{{\textnormal{j}}}
\def\ervk{{\textnormal{k}}}
\def\ervl{{\textnormal{l}}}
\def\ervm{{\textnormal{m}}}
\def\ervn{{\textnormal{n}}}
\def\ervo{{\textnormal{o}}}
\def\ervp{{\textnormal{p}}}
\def\ervq{{\textnormal{q}}}
\def\ervr{{\textnormal{r}}}
\def\ervs{{\textnormal{s}}}
\def\ervt{{\textnormal{t}}}
\def\ervu{{\textnormal{u}}}
\def\ervv{{\textnormal{v}}}
\def\ervw{{\textnormal{w}}}
\def\ervx{{\textnormal{x}}}
\def\ervy{{\textnormal{y}}}
\def\ervz{{\textnormal{z}}}

% Random matrices
\def\rmA{{\mathbf{A}}}
\def\rmB{{\mathbf{B}}}
\def\rmC{{\mathbf{C}}}
\def\rmD{{\mathbf{D}}}
\def\rmE{{\mathbf{E}}}
\def\rmF{{\mathbf{F}}}
\def\rmG{{\mathbf{G}}}
\def\rmH{{\mathbf{H}}}
\def\rmI{{\mathbf{I}}}
\def\rmJ{{\mathbf{J}}}
\def\rmK{{\mathbf{K}}}
\def\rmL{{\mathbf{L}}}
\def\rmM{{\mathbf{M}}}
\def\rmN{{\mathbf{N}}}
\def\rmO{{\mathbf{O}}}
\def\rmP{{\mathbf{P}}}
\def\rmQ{{\mathbf{Q}}}
\def\rmR{{\mathbf{R}}}
\def\rmS{{\mathbf{S}}}
\def\rmT{{\mathbf{T}}}
\def\rmU{{\mathbf{U}}}
\def\rmV{{\mathbf{V}}}
\def\rmW{{\mathbf{W}}}
\def\rmX{{\mathbf{X}}}
\def\rmY{{\mathbf{Y}}}
\def\rmZ{{\mathbf{Z}}}

% Elements of random matrices
\def\ermA{{\textnormal{A}}}
\def\ermB{{\textnormal{B}}}
\def\ermC{{\textnormal{C}}}
\def\ermD{{\textnormal{D}}}
\def\ermE{{\textnormal{E}}}
\def\ermF{{\textnormal{F}}}
\def\ermG{{\textnormal{G}}}
\def\ermH{{\textnormal{H}}}
\def\ermI{{\textnormal{I}}}
\def\ermJ{{\textnormal{J}}}
\def\ermK{{\textnormal{K}}}
\def\ermL{{\textnormal{L}}}
\def\ermM{{\textnormal{M}}}
\def\ermN{{\textnormal{N}}}
\def\ermO{{\textnormal{O}}}
\def\ermP{{\textnormal{P}}}
\def\ermQ{{\textnormal{Q}}}
\def\ermR{{\textnormal{R}}}
\def\ermS{{\textnormal{S}}}
\def\ermT{{\textnormal{T}}}
\def\ermU{{\textnormal{U}}}
\def\ermV{{\textnormal{V}}}
\def\ermW{{\textnormal{W}}}
\def\ermX{{\textnormal{X}}}
\def\ermY{{\textnormal{Y}}}
\def\ermZ{{\textnormal{Z}}}

% Vectors
\def\vzero{{\bm{0}}}
\def\vone{{\bm{1}}}
\def\vmu{{\bm{\mu}}}
\def\vtheta{{\bm{\theta}}}
\def\vphi{{\bm{\phi}}}
\def\va{{\bm{a}}}
\def\vb{{\bm{b}}}
\def\vc{{\bm{c}}}
\def\vd{{\bm{d}}}
\def\ve{{\bm{e}}}
\def\vf{{\bm{f}}}
\def\vg{{\bm{g}}}
\def\vh{{\bm{h}}}
\def\vi{{\bm{i}}}
\def\vj{{\bm{j}}}
\def\vk{{\bm{k}}}
\def\vl{{\bm{l}}}
\def\vm{{\bm{m}}}
\def\vn{{\bm{n}}}
\def\vo{{\bm{o}}}
\def\vp{{\bm{p}}}
\def\vq{{\bm{q}}}
\def\vr{{\bm{r}}}
\def\vs{{\bm{s}}}
\def\vt{{\bm{t}}}
\def\vu{{\bm{u}}}
\def\vv{{\bm{v}}}
\def\vw{{\bm{w}}}
\def\vx{{\bm{x}}}
\def\vy{{\bm{y}}}
\def\vz{{\bm{z}}}

% Elements of vectors
\def\evalpha{{\alpha}}
\def\evbeta{{\beta}}
\def\evepsilon{{\epsilon}}
\def\evlambda{{\lambda}}
\def\evomega{{\omega}}
\def\evmu{{\mu}}
\def\evpsi{{\psi}}
\def\evsigma{{\sigma}}
\def\evtheta{{\theta}}
\def\eva{{a}}
\def\evb{{b}}
\def\evc{{c}}
\def\evd{{d}}
\def\eve{{e}}
\def\evf{{f}}
\def\evg{{g}}
\def\evh{{h}}
\def\evi{{i}}
\def\evj{{j}}
\def\evk{{k}}
\def\evl{{l}}
\def\evm{{m}}
\def\evn{{n}}
\def\evo{{o}}
\def\evp{{p}}
\def\evq{{q}}
\def\evr{{r}}
\def\evs{{s}}
\def\evt{{t}}
\def\evu{{u}}
\def\evv{{v}}
\def\evw{{w}}
\def\evx{{x}}
\def\evy{{y}}
\def\evz{{z}}

% Matrix
\def\mA{{\bm{A}}}
\def\mB{{\bm{B}}}
\def\mC{{\bm{C}}}
\def\mD{{\bm{D}}}
\def\mE{{\bm{E}}}
\def\mF{{\bm{F}}}
\def\mG{{\bm{G}}}
\def\mH{{\bm{H}}}
\def\mI{{\bm{I}}}
\def\mJ{{\bm{J}}}
\def\mK{{\bm{K}}}
\def\mL{{\bm{L}}}
\def\mM{{\bm{M}}}
\def\mN{{\bm{N}}}
\def\mO{{\bm{O}}}
\def\mP{{\bm{P}}}
\def\mQ{{\bm{Q}}}
\def\mR{{\bm{R}}}
\def\mS{{\bm{S}}}
\def\mT{{\bm{T}}}
\def\mU{{\bm{U}}}
\def\mV{{\bm{V}}}
\def\mW{{\bm{W}}}
\def\mX{{\bm{X}}}
\def\mY{{\bm{Y}}}
\def\mZ{{\bm{Z}}}
\def\mBeta{{\bm{\beta}}}
\def\mPhi{{\bm{\Phi}}}
\def\mLambda{{\bm{\Lambda}}}
\def\mSigma{{\bm{\Sigma}}}

% Tensor
\DeclareMathAlphabet{\mathsfit}{\encodingdefault}{\sfdefault}{m}{sl}
\SetMathAlphabet{\mathsfit}{bold}{\encodingdefault}{\sfdefault}{bx}{n}
\newcommand{\tens}[1]{\bm{\mathsfit{#1}}}
\def\tA{{\tens{A}}}
\def\tB{{\tens{B}}}
\def\tC{{\tens{C}}}
\def\tD{{\tens{D}}}
\def\tE{{\tens{E}}}
\def\tF{{\tens{F}}}
\def\tG{{\tens{G}}}
\def\tH{{\tens{H}}}
\def\tI{{\tens{I}}}
\def\tJ{{\tens{J}}}
\def\tK{{\tens{K}}}
\def\tL{{\tens{L}}}
\def\tM{{\tens{M}}}
\def\tN{{\tens{N}}}
\def\tO{{\tens{O}}}
\def\tP{{\tens{P}}}
\def\tQ{{\tens{Q}}}
\def\tR{{\tens{R}}}
\def\tS{{\tens{S}}}
\def\tT{{\tens{T}}}
\def\tU{{\tens{U}}}
\def\tV{{\tens{V}}}
\def\tW{{\tens{W}}}
\def\tX{{\tens{X}}}
\def\tY{{\tens{Y}}}
\def\tZ{{\tens{Z}}}


% Graph
\def\gA{{\mathcal{A}}}
\def\gB{{\mathcal{B}}}
\def\gC{{\mathcal{C}}}
\def\gD{{\mathcal{D}}}
\def\gE{{\mathcal{E}}}
\def\gF{{\mathcal{F}}}
\def\gG{{\mathcal{G}}}
\def\gH{{\mathcal{H}}}
\def\gI{{\mathcal{I}}}
\def\gJ{{\mathcal{J}}}
\def\gK{{\mathcal{K}}}
\def\gL{{\mathcal{L}}}
\def\gM{{\mathcal{M}}}
\def\gN{{\mathcal{N}}}
\def\gO{{\mathcal{O}}}
\def\gP{{\mathcal{P}}}
\def\gQ{{\mathcal{Q}}}
\def\gR{{\mathcal{R}}}
\def\gS{{\mathcal{S}}}
\def\gT{{\mathcal{T}}}
\def\gU{{\mathcal{U}}}
\def\gV{{\mathcal{V}}}
\def\gW{{\mathcal{W}}}
\def\gX{{\mathcal{X}}}
\def\gY{{\mathcal{Y}}}
\def\gZ{{\mathcal{Z}}}

% Sets
\def\sA{{\mathbb{A}}}
\def\sB{{\mathbb{B}}}
\def\sC{{\mathbb{C}}}
\def\sD{{\mathbb{D}}}
% Don't use a set called E, because this would be the same as our symbol
% for expectation.
\def\sF{{\mathbb{F}}}
\def\sG{{\mathbb{G}}}
\def\sH{{\mathbb{H}}}
\def\sI{{\mathbb{I}}}
\def\sJ{{\mathbb{J}}}
\def\sK{{\mathbb{K}}}
\def\sL{{\mathbb{L}}}
\def\sM{{\mathbb{M}}}
\def\sN{{\mathbb{N}}}
\def\sO{{\mathbb{O}}}
\def\sP{{\mathbb{P}}}
\def\sQ{{\mathbb{Q}}}
\def\sR{{\mathbb{R}}}
\def\sS{{\mathbb{S}}}
\def\sT{{\mathbb{T}}}
\def\sU{{\mathbb{U}}}
\def\sV{{\mathbb{V}}}
\def\sW{{\mathbb{W}}}
\def\sX{{\mathbb{X}}}
\def\sY{{\mathbb{Y}}}
\def\sZ{{\mathbb{Z}}}

% Entries of a matrix
\def\emLambda{{\Lambda}}
\def\emA{{A}}
\def\emB{{B}}
\def\emC{{C}}
\def\emD{{D}}
\def\emE{{E}}
\def\emF{{F}}
\def\emG{{G}}
\def\emH{{H}}
\def\emI{{I}}
\def\emJ{{J}}
\def\emK{{K}}
\def\emL{{L}}
\def\emM{{M}}
\def\emN{{N}}
\def\emO{{O}}
\def\emP{{P}}
\def\emQ{{Q}}
\def\emR{{R}}
\def\emS{{S}}
\def\emT{{T}}
\def\emU{{U}}
\def\emV{{V}}
\def\emW{{W}}
\def\emX{{X}}
\def\emY{{Y}}
\def\emZ{{Z}}
\def\emSigma{{\Sigma}}

% entries of a tensor
% Same font as tensor, without \bm wrapper
\newcommand{\etens}[1]{\mathsfit{#1}}
\def\etLambda{{\etens{\Lambda}}}
\def\etA{{\etens{A}}}
\def\etB{{\etens{B}}}
\def\etC{{\etens{C}}}
\def\etD{{\etens{D}}}
\def\etE{{\etens{E}}}
\def\etF{{\etens{F}}}
\def\etG{{\etens{G}}}
\def\etH{{\etens{H}}}
\def\etI{{\etens{I}}}
\def\etJ{{\etens{J}}}
\def\etK{{\etens{K}}}
\def\etL{{\etens{L}}}
\def\etM{{\etens{M}}}
\def\etN{{\etens{N}}}
\def\etO{{\etens{O}}}
\def\etP{{\etens{P}}}
\def\etQ{{\etens{Q}}}
\def\etR{{\etens{R}}}
\def\etS{{\etens{S}}}
\def\etT{{\etens{T}}}
\def\etU{{\etens{U}}}
\def\etV{{\etens{V}}}
\def\etW{{\etens{W}}}
\def\etX{{\etens{X}}}
\def\etY{{\etens{Y}}}
\def\etZ{{\etens{Z}}}

% The true underlying data generating distribution
\newcommand{\pdata}{p_{\rm{data}}}
\newcommand{\ptarget}{p_{\rm{target}}}
\newcommand{\pprior}{p_{\rm{prior}}}
\newcommand{\pbase}{p_{\rm{base}}}
\newcommand{\pref}{p_{\rm{ref}}}

% The empirical distribution defined by the training set
\newcommand{\ptrain}{\hat{p}_{\rm{data}}}
\newcommand{\Ptrain}{\hat{P}_{\rm{data}}}
% The model distribution
\newcommand{\pmodel}{p_{\rm{model}}}
\newcommand{\Pmodel}{P_{\rm{model}}}
\newcommand{\ptildemodel}{\tilde{p}_{\rm{model}}}
% Stochastic autoencoder distributions
\newcommand{\pencode}{p_{\rm{encoder}}}
\newcommand{\pdecode}{p_{\rm{decoder}}}
\newcommand{\precons}{p_{\rm{reconstruct}}}

\newcommand{\laplace}{\mathrm{Laplace}} % Laplace distribution

\newcommand{\E}{\mathbb{E}}
\newcommand{\Ls}{\mathcal{L}}
\newcommand{\R}{\mathbb{R}}
\newcommand{\emp}{\tilde{p}}
\newcommand{\lr}{\alpha}
\newcommand{\reg}{\lambda}
\newcommand{\rect}{\mathrm{rectifier}}
\newcommand{\softmax}{\mathrm{softmax}}
\newcommand{\sigmoid}{\sigma}
\newcommand{\softplus}{\zeta}
\newcommand{\KL}{D_{\mathrm{KL}}}
\newcommand{\Var}{\mathrm{Var}}
\newcommand{\standarderror}{\mathrm{SE}}
\newcommand{\Cov}{\mathrm{Cov}}
% Wolfram Mathworld says $L^2$ is for function spaces and $\ell^2$ is for vectors
% But then they seem to use $L^2$ for vectors throughout the site, and so does
% wikipedia.
\newcommand{\normlzero}{L^0}
\newcommand{\normlone}{L^1}
\newcommand{\normltwo}{L^2}
\newcommand{\normlp}{L^p}
\newcommand{\normmax}{L^\infty}

\newcommand{\parents}{Pa} % See usage in notation.tex. Chosen to match Daphne's book.

\DeclareMathOperator*{\argmax}{arg\,max}
\DeclareMathOperator*{\argmin}{arg\,min}

\DeclareMathOperator{\sign}{sign}
\DeclareMathOperator{\Tr}{Tr}
\let\ab\allowbreak


\theoremstyle{plain}
\newtheorem{theorem}{Theorem}[section]
\newtheorem{proposition}[theorem]{Proposition}
\newtheorem{lemma}[theorem]{Lemma}
\newtheorem{corollary}[theorem]{Corollary}
\theoremstyle{definition}
\newtheorem{definition}[theorem]{Definition}
\newtheorem{assumption}[theorem]{Assumption}
\theoremstyle{remark}
\newtheorem{remark}[theorem]{Remark}

\makeatletter
\DeclareRobustCommand\onedot{\futurelet\@let@token\@onedot}
\def\@onedot{\ifx\@let@token.\else.\null\fi\xspace}

\def\eg{\emph{e.g}\onedot} \def\Eg{\emph{E.g}\onedot}
\def\ie{\emph{i.e}\onedot} \def\Ie{\emph{I.e}\onedot}
\def\cf{\emph{c.f}\onedot} \def\Cf{\emph{C.f}\onedot}
\def\etc{\emph{etc}\onedot} \def\vs{\emph{vs}\onedot}
\def\wrt{w.r.t\onedot} \def\dof{d.o.f\onedot}
\def\etal{\emph{et al}\onedot}
\makeatother

\begin{document}



\icmltitlerunning{A Closer Look at \ours: Strength, Limitation, and Extension}

\twocolumn[
\icmltitle{A Closer Look at \ours: Strength, Limitation, and Extension}




\begin{icmlauthorlist}
\icmlauthor{Han-Jia Ye}{nju,lab}
\icmlauthor{Si-Yang Liu}{nju,lab}
\icmlauthor{Wei-Lun Chao}{osu}
\end{icmlauthorlist}

\icmlaffiliation{nju}{School of Artificial Intelligence, Nanjing University}
\icmlaffiliation{lab}{National Key Laboratory for Novel Software Technology, Nanjing University}
\icmlaffiliation{osu}{The Ohio State University}

\icmlcorrespondingauthor{Han-Jia Ye}{yehj@lamda.nju.edu.cn}


\icmlkeywords{Machine Learning, ICML}

\vskip 0.3in
]
\printAffiliationsAndNotice{}  % leave blank if no need to mention equal contribution


\begin{abstract}  
Test time scaling is currently one of the most active research areas that shows promise after training time scaling has reached its limits.
Deep-thinking (DT) models are a class of recurrent models that can perform easy-to-hard generalization by assigning more compute to harder test samples.
However, due to their inability to determine the complexity of a test sample, DT models have to use a large amount of computation for both easy and hard test samples.
Excessive test time computation is wasteful and can cause the ``overthinking'' problem where more test time computation leads to worse results.
In this paper, we introduce a test time training method for determining the optimal amount of computation needed for each sample during test time.
We also propose Conv-LiGRU, a novel recurrent architecture for efficient and robust visual reasoning. 
Extensive experiments demonstrate that Conv-LiGRU is more stable than DT, effectively mitigates the ``overthinking'' phenomenon, and achieves superior accuracy.
\end{abstract}  

\section{Introduction}
\label{sec:introduction}
The business processes of organizations are experiencing ever-increasing complexity due to the large amount of data, high number of users, and high-tech devices involved \cite{martin2021pmopportunitieschallenges, beerepoot2023biggestbpmproblems}. This complexity may cause business processes to deviate from normal control flow due to unforeseen and disruptive anomalies \cite{adams2023proceddsriftdetection}. These control-flow anomalies manifest as unknown, skipped, and wrongly-ordered activities in the traces of event logs monitored from the execution of business processes \cite{ko2023adsystematicreview}. For the sake of clarity, let us consider an illustrative example of such anomalies. Figure \ref{FP_ANOMALIES} shows a so-called event log footprint, which captures the control flow relations of four activities of a hypothetical event log. In particular, this footprint captures the control-flow relations between activities \texttt{a}, \texttt{b}, \texttt{c} and \texttt{d}. These are the causal ($\rightarrow$) relation, concurrent ($\parallel$) relation, and other ($\#$) relations such as exclusivity or non-local dependency \cite{aalst2022pmhandbook}. In addition, on the right are six traces, of which five exhibit skipped, wrongly-ordered and unknown control-flow anomalies. For example, $\langle$\texttt{a b d}$\rangle$ has a skipped activity, which is \texttt{c}. Because of this skipped activity, the control-flow relation \texttt{b}$\,\#\,$\texttt{d} is violated, since \texttt{d} directly follows \texttt{b} in the anomalous trace.
\begin{figure}[!t]
\centering
\includegraphics[width=0.9\columnwidth]{images/FP_ANOMALIES.png}
\caption{An example event log footprint with six traces, of which five exhibit control-flow anomalies.}
\label{FP_ANOMALIES}
\end{figure}

\subsection{Control-flow anomaly detection}
Control-flow anomaly detection techniques aim to characterize the normal control flow from event logs and verify whether these deviations occur in new event logs \cite{ko2023adsystematicreview}. To develop control-flow anomaly detection techniques, \revision{process mining} has seen widespread adoption owing to process discovery and \revision{conformance checking}. On the one hand, process discovery is a set of algorithms that encode control-flow relations as a set of model elements and constraints according to a given modeling formalism \cite{aalst2022pmhandbook}; hereafter, we refer to the Petri net, a widespread modeling formalism. On the other hand, \revision{conformance checking} is an explainable set of algorithms that allows linking any deviations with the reference Petri net and providing the fitness measure, namely a measure of how much the Petri net fits the new event log \cite{aalst2022pmhandbook}. Many control-flow anomaly detection techniques based on \revision{conformance checking} (hereafter, \revision{conformance checking}-based techniques) use the fitness measure to determine whether an event log is anomalous \cite{bezerra2009pmad, bezerra2013adlogspais, myers2018icsadpm, pecchia2020applicationfailuresanalysispm}. 

The scientific literature also includes many \revision{conformance checking}-independent techniques for control-flow anomaly detection that combine specific types of trace encodings with machine/deep learning \cite{ko2023adsystematicreview, tavares2023pmtraceencoding}. Whereas these techniques are very effective, their explainability is challenging due to both the type of trace encoding employed and the machine/deep learning model used \cite{rawal2022trustworthyaiadvances,li2023explainablead}. Hence, in the following, we focus on the shortcomings of \revision{conformance checking}-based techniques to investigate whether it is possible to support the development of competitive control-flow anomaly detection techniques while maintaining the explainable nature of \revision{conformance checking}.
\begin{figure}[!t]
\centering
\includegraphics[width=\columnwidth]{images/HIGH_LEVEL_VIEW.png}
\caption{A high-level view of the proposed framework for combining \revision{process mining}-based feature extraction with dimensionality reduction for control-flow anomaly detection.}
\label{HIGH_LEVEL_VIEW}
\end{figure}

\subsection{Shortcomings of \revision{conformance checking}-based techniques}
Unfortunately, the detection effectiveness of \revision{conformance checking}-based techniques is affected by noisy data and low-quality Petri nets, which may be due to human errors in the modeling process or representational bias of process discovery algorithms \cite{bezerra2013adlogspais, pecchia2020applicationfailuresanalysispm, aalst2016pm}. Specifically, on the one hand, noisy data may introduce infrequent and deceptive control-flow relations that may result in inconsistent fitness measures, whereas, on the other hand, checking event logs against a low-quality Petri net could lead to an unreliable distribution of fitness measures. Nonetheless, such Petri nets can still be used as references to obtain insightful information for \revision{process mining}-based feature extraction, supporting the development of competitive and explainable \revision{conformance checking}-based techniques for control-flow anomaly detection despite the problems above. For example, a few works outline that token-based \revision{conformance checking} can be used for \revision{process mining}-based feature extraction to build tabular data and develop effective \revision{conformance checking}-based techniques for control-flow anomaly detection \cite{singh2022lapmsh, debenedictis2023dtadiiot}. However, to the best of our knowledge, the scientific literature lacks a structured proposal for \revision{process mining}-based feature extraction using the state-of-the-art \revision{conformance checking} variant, namely alignment-based \revision{conformance checking}.

\subsection{Contributions}
We propose a novel \revision{process mining}-based feature extraction approach with alignment-based \revision{conformance checking}. This variant aligns the deviating control flow with a reference Petri net; the resulting alignment can be inspected to extract additional statistics such as the number of times a given activity caused mismatches \cite{aalst2022pmhandbook}. We integrate this approach into a flexible and explainable framework for developing techniques for control-flow anomaly detection. The framework combines \revision{process mining}-based feature extraction and dimensionality reduction to handle high-dimensional feature sets, achieve detection effectiveness, and support explainability. Notably, in addition to our proposed \revision{process mining}-based feature extraction approach, the framework allows employing other approaches, enabling a fair comparison of multiple \revision{conformance checking}-based and \revision{conformance checking}-independent techniques for control-flow anomaly detection. Figure \ref{HIGH_LEVEL_VIEW} shows a high-level view of the framework. Business processes are monitored, and event logs obtained from the database of information systems. Subsequently, \revision{process mining}-based feature extraction is applied to these event logs and tabular data input to dimensionality reduction to identify control-flow anomalies. We apply several \revision{conformance checking}-based and \revision{conformance checking}-independent framework techniques to publicly available datasets, simulated data of a case study from railways, and real-world data of a case study from healthcare. We show that the framework techniques implementing our approach outperform the baseline \revision{conformance checking}-based techniques while maintaining the explainable nature of \revision{conformance checking}.

In summary, the contributions of this paper are as follows.
\begin{itemize}
    \item{
        A novel \revision{process mining}-based feature extraction approach to support the development of competitive and explainable \revision{conformance checking}-based techniques for control-flow anomaly detection.
    }
    \item{
        A flexible and explainable framework for developing techniques for control-flow anomaly detection using \revision{process mining}-based feature extraction and dimensionality reduction.
    }
    \item{
        Application to synthetic and real-world datasets of several \revision{conformance checking}-based and \revision{conformance checking}-independent framework techniques, evaluating their detection effectiveness and explainability.
    }
\end{itemize}

The rest of the paper is organized as follows.
\begin{itemize}
    \item Section \ref{sec:related_work} reviews the existing techniques for control-flow anomaly detection, categorizing them into \revision{conformance checking}-based and \revision{conformance checking}-independent techniques.
    \item Section \ref{sec:abccfe} provides the preliminaries of \revision{process mining} to establish the notation used throughout the paper, and delves into the details of the proposed \revision{process mining}-based feature extraction approach with alignment-based \revision{conformance checking}.
    \item Section \ref{sec:framework} describes the framework for developing \revision{conformance checking}-based and \revision{conformance checking}-independent techniques for control-flow anomaly detection that combine \revision{process mining}-based feature extraction and dimensionality reduction.
    \item Section \ref{sec:evaluation} presents the experiments conducted with multiple framework and baseline techniques using data from publicly available datasets and case studies.
    \item Section \ref{sec:conclusions} draws the conclusions and presents future work.
\end{itemize}

\putsec{related}{Related Work}

\noindent \textbf{Efficient Radiance Field Rendering.}
%
The introduction of Neural Radiance Fields (NeRF)~\cite{mil:sri20} has
generated significant interest in efficient 3D scene representation and
rendering for radiance fields.
%
Over the past years, there has been a large amount of research aimed at
accelerating NeRFs through algorithmic or software
optimizations~\cite{mul:eva22,fri:yu22,che:fun23,sun:sun22}, and the
development of hardware
accelerators~\cite{lee:cho23,li:li23,son:wen23,mub:kan23,fen:liu24}.
%
The state-of-the-art method, 3D Gaussian splatting~\cite{ker:kop23}, has
further fueled interest in accelerating radiance field
rendering~\cite{rad:ste24,lee:lee24,nie:stu24,lee:rho24,ham:mel24} as it
employs rasterization primitives that can be rendered much faster than NeRFs.
%
However, previous research focused on software graphics rendering on
programmable cores or building dedicated hardware accelerators. In contrast,
\name{} investigates the potential of efficient radiance field rendering while
utilizing fixed-function units in graphics hardware.
%
To our knowledge, this is the first work that assesses the performance
implications of rendering Gaussian-based radiance fields on the hardware
graphics pipeline with software and hardware optimizations.

%%%%%%%%%%%%%%%%%%%%%%%%%%%%%%%%%%%%%%%%%%%%%%%%%%%%%%%%%%%%%%%%%%%%%%%%%%
\myparagraph{Enhancing Graphics Rendering Hardware.}
%
The performance advantage of executing graphics rendering on either
programmable shader cores or fixed-function units varies depending on the
rendering methods and hardware designs.
%
Previous studies have explored the performance implication of graphics hardware
design by developing simulation infrastructures for graphics
workloads~\cite{bar:gon06,gub:aam19,tin:sax23,arn:par13}.
%
Additionally, several studies have aimed to improve the performance of
special-purpose hardware such as ray tracing units in graphics
hardware~\cite{cho:now23,liu:cha21} and proposed hardware accelerators for
graphics applications~\cite{lu:hua17,ram:gri09}.
%
In contrast to these works, which primarily evaluate traditional graphics
workloads, our work focuses on improving the performance of volume rendering
workloads, such as Gaussian splatting, which require blending a huge number of
fragments per pixel.

%%%%%%%%%%%%%%%%%%%%%%%%%%%%%%%%%%%%%%%%%%%%%%%%%%%%%%%%%%%%%%%%%%%%%%%%%%
%
In the context of multi-sample anti-aliasing, prior work proposed reducing the
amount of redundant shading by merging fragments from adjacent triangles in a
mesh at the quad granularity~\cite{fat:bou10}.
%
While both our work and quad-fragment merging (QFM)~\cite{fat:bou10} aim to
reduce operations by merging quads, our proposed technique differs from QFM in
many aspects.
%
Our method aims to blend \emph{overlapping primitives} along the depth
direction and applies to quads from any primitive. In contrast, QFM merges quad
fragments from small (e.g., pixel-sized) triangles that \emph{share} an edge
(i.e., \emph{connected}, \emph{non-overlapping} triangles).
%
As such, QFM is not applicable to the scenes consisting of a number of
unconnected transparent triangles, such as those in 3D Gaussian splatting.
%
In addition, our method computes the \emph{exact} color for each pixel by
offloading blending operations from ROPs to shader units, whereas QFM
\emph{approximates} pixel colors by using the color from one triangle when
multiple triangles are merged into a single quad.



\section{Background}\label{sec:backgrnd}

\subsection{Cold Start Latency and Mitigation Techniques}

Traditional FaaS platforms mitigate cold starts through snapshotting, lightweight virtualization, and warm-state management. Snapshot-based methods like \textbf{REAP} and \textbf{Catalyzer} reduce initialization time by preloading or restoring container states but require significant memory and I/O resources, limiting scalability~\cite{dong_catalyzer_2020, ustiugov_benchmarking_2021}. Lightweight virtualization solutions, such as \textbf{Firecracker} microVMs, achieve fast startup times with strong isolation but depend on robust infrastructure, making them less adaptable to fluctuating workloads~\cite{agache_firecracker_2020}. Warm-state management techniques like \textbf{Faa\$T}~\cite{romero_faa_2021} and \textbf{Kraken}~\cite{vivek_kraken_2021} keep frequently invoked containers ready, balancing readiness and cost efficiency under predictable workloads but incurring overhead when demand is erratic~\cite{romero_faa_2021, vivek_kraken_2021}. While these methods perform well in resource-rich cloud environments, their resource intensity challenges applicability in edge settings.

\subsubsection{Edge FaaS Perspective}

In edge environments, cold start mitigation emphasizes lightweight designs, resource sharing, and hybrid task distribution. Lightweight execution environments like unikernels~\cite{edward_sock_2018} and \textbf{Firecracker}~\cite{agache_firecracker_2020}, as used by \textbf{TinyFaaS}~\cite{pfandzelter_tinyfaas_2020}, minimize resource usage and initialization delays but require careful orchestration to avoid resource contention. Function co-location, demonstrated by \textbf{Photons}~\cite{v_dukic_photons_2020}, reduces redundant initializations by sharing runtime resources among related functions, though this complicates isolation in multi-tenant setups~\cite{v_dukic_photons_2020}. Hybrid offloading frameworks like \textbf{GeoFaaS}~\cite{malekabbasi_geofaas_2024} balance edge-cloud workloads by offloading latency-tolerant tasks to the cloud and reserving edge resources for real-time operations, requiring reliable connectivity and efficient task management. These edge-specific strategies address cold starts effectively but introduce challenges in scalability and orchestration.

\subsection{Predictive Scaling and Caching Techniques}

Efficient resource allocation is vital for maintaining low latency and high availability in serverless platforms. Predictive scaling and caching techniques dynamically provision resources and reduce cold start latency by leveraging workload prediction and state retention.
Traditional FaaS platforms use predictive scaling and caching to optimize resources, employing techniques (OFC, FaasCache) to reduce cold starts. However, these methods rely on centralized orchestration and workload predictability, limiting their effectiveness in dynamic, resource-constrained edge environments.



\subsubsection{Edge FaaS Perspective}

Edge FaaS platforms adapt predictive scaling and caching techniques to constrain resources and heterogeneous environments. \textbf{EDGE-Cache}~\cite{kim_delay-aware_2022} uses traffic profiling to selectively retain high-priority functions, reducing memory overhead while maintaining readiness for frequent requests. Hybrid frameworks like \textbf{GeoFaaS}~\cite{malekabbasi_geofaas_2024} implement distributed caching to balance resources between edge and cloud nodes, enabling low-latency processing for critical tasks while offloading less critical workloads. Machine learning methods, such as clustering-based workload predictors~\cite{gao_machine_2020} and GRU-based models~\cite{guo_applying_2018}, enhance resource provisioning in edge systems by efficiently forecasting workload spikes. These innovations effectively address cold start challenges in edge environments, though their dependency on accurate predictions and robust orchestration poses scalability challenges.

\subsection{Decentralized Orchestration, Function Placement, and Scheduling}

Efficient orchestration in serverless platforms involves workload distribution, resource optimization, and performance assurance. While traditional FaaS platforms rely on centralized control, edge environments require decentralized and adaptive strategies to address unique challenges such as resource constraints and heterogeneous hardware.



\subsubsection{Edge FaaS Perspective}

Edge FaaS platforms adopt decentralized and adaptive orchestration frameworks to meet the demands of resource-constrained environments. Systems like \textbf{Wukong} distribute scheduling across edge nodes, enhancing data locality and scalability while reducing network latency. Lightweight frameworks such as \textbf{OpenWhisk Lite}~\cite{kravchenko_kpavelopenwhisk-light_2024} optimize resource allocation by decentralizing scheduling policies, minimizing cold starts and latency in edge setups~\cite{benjamin_wukong_2020}. Hybrid solutions like \textbf{OpenFaaS}~\cite{noauthor_openfaasfaas_2024} and \textbf{EdgeMatrix}~\cite{shen_edgematrix_2023} combine edge-cloud orchestration to balance resource utilization, retaining latency-sensitive functions at the edge while offloading non-critical workloads to the cloud. While these approaches improve flexibility, they face challenges in maintaining coordination and ensuring consistent performance across distributed nodes.



\definecolor{darkgreen}{rgb}{0.0, 0.5, 0.0}
\definecolor{violet}{rgb}{0.56, 0.0, 1.0}
\section{Evaluation}
We apply our methodology to derive counterfactual policies for various MDPs, addressing three main research questions: (1) how does our policy's performance compare to the Gumbel-max SCM approach; (2) how do the counterfactual stability and monotonicity assumptions impact the probability bounds; and (3) how fast is our approach compared with the Gumbel-max SCM method?

\begin{figure*}
    \centering
    %
    \resizebox{0.6\textwidth}{!}{
        \begin{tikzpicture}[scale=1.0, every node/.style={scale=1.0}]
            \draw[thick, black] (-3, -0.25) rectangle (10, 0.25);
            %
            \draw[black, line width=1pt] (-2.5, 0.0) -- (-2,0.0);
            \fill[black] (-2.25,0.0) circle (2pt); %
            \node[right] at (-2,0.0) {\small Observed Path};
            
            %
            \draw[blue, line width=1pt] (1.0,0.0) -- (1.5,0.0);
            \node[draw=blue, circle, minimum size=4pt, inner sep=0pt] at (1.25,0.0) {}; %
            \node[right] at (1.5,0.0) {\small Interval CFMDP Policy};
            
            %
            \draw[red, line width=1pt] (5.5,0) -- (6,0);
            \node[red] at (5.75,0) {$\boldsymbol{\times}$}; %
            \node[right] at (6,0) {\small Gumbel-max SCM Policy};
        \end{tikzpicture}
    }\\
    %
    \subfigure[\footnotesize Lowest cumulative reward: Interval CFMDP ($312$), Gumbel-max SCM ($312$)]{%
        \resizebox{0.76\columnwidth}{!}{
             \begin{tikzpicture}
                \begin{axis}[
                    xlabel={$t$},
                    ylabel={Mean reward at time step $t$},
                    title={Optimal Path},
                    grid=both,
                    width=20cm, height=8.5cm,
                    every axis/.style={font=\Huge},
                    %
                ]
                \addplot[
                    color=black, %
                    mark=*, %
                    line width=2pt,
                    mark size=3pt,
                    error bars/.cd,
                    y dir=both, %
                    y explicit, %
                    error bar style={line width=1pt,solid},
                    error mark options={line width=1pt,mark size=4pt,rotate=90}
                ]
                coordinates {
                    (0, 0.0)  +- (0, 0.0)
                    (1, 0.0)  +- (0, 0.0) 
                    (2, 1.0)  +- (0, 0.0) 
                    (3, 1.0)  +- (0, 0.0)
                    (4, 2.0)  +- (0, 0.0)
                    (5, 3.0) +- (0, 0.0)
                    (6, 5.0) +- (0, 0.0)
                    (7, 100.0) +- (0, 0.0)
                    (8, 100.0) +- (0, 0.0)
                    (9, 100.0) +- (0, 0.0)
                };
                %
                \addplot[
                    color=blue, %
                    mark=o, %
                    line width=2pt,
                    mark size=3pt,
                    error bars/.cd,
                    y dir=both, %
                    y explicit, %
                    error bar style={line width=1pt,solid},
                    error mark options={line width=1pt,mark size=4pt,rotate=90}
                ]
                 coordinates {
                    (0, 0.0)  +- (0, 0.0)
                    (1, 0.0)  +- (0, 0.0) 
                    (2, 1.0)  +- (0, 0.0) 
                    (3, 1.0)  +- (0, 0.0)
                    (4, 2.0)  +- (0, 0.0)
                    (5, 3.0) +- (0, 0.0)
                    (6, 5.0) +- (0, 0.0)
                    (7, 100.0) +- (0, 0.0)
                    (8, 100.0) +- (0, 0.0)
                    (9, 100.0) +- (0, 0.0)
                };
                %
                \addplot[
                    color=red, %
                    mark=x, %
                    line width=2pt,
                    mark size=6pt,
                    error bars/.cd,
                    y dir=both, %
                    y explicit, %
                    error bar style={line width=1pt,solid},
                    error mark options={line width=1pt,mark size=4pt,rotate=90}
                ]
                coordinates {
                    (0, 0.0)  +- (0, 0.0)
                    (1, 0.0)  +- (0, 0.0) 
                    (2, 1.0)  +- (0, 0.0) 
                    (3, 1.0)  +- (0, 0.0)
                    (4, 2.0)  +- (0, 0.0)
                    (5, 3.0) +- (0, 0.0)
                    (6, 5.0) +- (0, 0.0)
                    (7, 100.0) +- (0, 0.0)
                    (8, 100.0) +- (0, 0.0)
                    (9, 100.0) +- (0, 0.0)
                };
                \end{axis}
            \end{tikzpicture}
         }
    }
    \hspace{1cm}
    \subfigure[\footnotesize Lowest cumulative reward: Interval CFMDP ($19$), Gumbel-max SCM ($-88$)]{%
         \resizebox{0.76\columnwidth}{!}{
            \begin{tikzpicture}
                \begin{axis}[
                    xlabel={$t$},
                    ylabel={Mean reward at time step $t$},
                    title={Slightly Suboptimal Path},
                    grid=both,
                    width=20cm, height=8.5cm,
                    every axis/.style={font=\Huge},
                    %
                ]
                \addplot[
                    color=black, %
                    mark=*, %
                    line width=2pt,
                    mark size=3pt,
                    error bars/.cd,
                    y dir=both, %
                    y explicit, %
                    error bar style={line width=1pt,solid},
                    error mark options={line width=1pt,mark size=4pt,rotate=90}
                ]
              coordinates {
                    (0, 0.0)  +- (0, 0.0)
                    (1, 1.0)  +- (0, 0.0) 
                    (2, 1.0)  +- (0, 0.0) 
                    (3, 1.0)  +- (0, 0.0)
                    (4, 2.0)  +- (0, 0.0)
                    (5, 3.0) +- (0, 0.0)
                    (6, 3.0) +- (0, 0.0)
                    (7, 2.0) +- (0, 0.0)
                    (8, 2.0) +- (0, 0.0)
                    (9, 4.0) +- (0, 0.0)
                };
                %
                \addplot[
                    color=blue, %
                    mark=o, %
                    line width=2pt,
                    mark size=3pt,
                    error bars/.cd,
                    y dir=both, %
                    y explicit, %
                    error bar style={line width=1pt,solid},
                    error mark options={line width=1pt,mark size=4pt,rotate=90}
                ]
              coordinates {
                    (0, 0.0)  +- (0, 0.0)
                    (1, 1.0)  +- (0, 0.0) 
                    (2, 1.0)  +- (0, 0.0) 
                    (3, 1.0)  +- (0, 0.0)
                    (4, 2.0)  +- (0, 0.0)
                    (5, 3.0) +- (0, 0.0)
                    (6, 3.0) +- (0, 0.0)
                    (7, 2.0) +- (0, 0.0)
                    (8, 2.0) +- (0, 0.0)
                    (9, 4.0) +- (0, 0.0)
                };
                %
                \addplot[
                    color=red, %
                    mark=x, %
                    line width=2pt,
                    mark size=6pt,
                    error bars/.cd,
                    y dir=both, %
                    y explicit, %
                    error bar style={line width=1pt,solid},
                    error mark options={line width=1pt,mark size=4pt,rotate=90}
                ]
                coordinates {
                    (0, 0.0)  +- (0, 0.0)
                    (1, 1.0)  +- (0, 0.0) 
                    (2, 1.0)  +- (0, 0.0) 
                    (3, 1.0)  +- (0, 0.0)
                    (4, 2.0)  += (0, 0.0)
                    (5, 3.0)  += (0, 0.0)
                    (6, 3.17847) += (0, 0.62606746) -= (0, 0.62606746)
                    (7, 2.5832885) += (0, 1.04598233) -= (0, 1.04598233)
                    (8, 5.978909) += (0, 17.60137623) -= (0, 17.60137623)
                    (9, 5.297059) += (0, 27.09227512) -= (0, 27.09227512)
                };
                \end{axis}
            \end{tikzpicture}
         }
    }\\[-1.5pt]
    \subfigure[\footnotesize Lowest cumulative reward: Interval CFMDP ($14$), Gumbel-max SCM ($-598$)]{%
         \resizebox{0.76\columnwidth}{!}{
             \begin{tikzpicture}
                \begin{axis}[
                    xlabel={$t$},
                    ylabel={Mean reward at time step $t$},
                    title={Almost Catastrophic Path},
                    grid=both,
                    width=20cm, height=8.5cm,
                    every axis/.style={font=\Huge},
                    %
                ]
                \addplot[
                    color=black, %
                    mark=*, %
                    line width=2pt,
                    mark size=3pt,
                    error bars/.cd,
                    y dir=both, %
                    y explicit, %
                    error bar style={line width=1pt,solid},
                    error mark options={line width=1pt,mark size=4pt,rotate=90}
                ]
                coordinates {
                    (0, 0.0)  +- (0, 0.0)
                    (1, 1.0)  +- (0, 0.0) 
                    (2, 2.0)  +- (0, 0.0) 
                    (3, 1.0)  +- (0, 0.0)
                    (4, 0.0)  +- (0, 0.0)
                    (5, 1.0) +- (0, 0.0)
                    (6, 2.0) +- (0, 0.0)
                    (7, 2.0) +- (0, 0.0)
                    (8, 3.0) +- (0, 0.0)
                    (9, 2.0) +- (0, 0.0)
                };
                %
                \addplot[
                    color=blue, %
                    mark=o, %
                    line width=2pt,
                    mark size=3pt,
                    error bars/.cd,
                    y dir=both, %
                    y explicit, %
                    error bar style={line width=1pt,solid},
                    error mark options={line width=1pt,mark size=4pt,rotate=90}
                ]
                coordinates {
                    (0, 0.0)  +- (0, 0.0)
                    (1, 1.0)  +- (0, 0.0) 
                    (2, 2.0)  +- (0, 0.0) 
                    (3, 1.0)  +- (0, 0.0)
                    (4, 0.0)  +- (0, 0.0)
                    (5, 1.0) +- (0, 0.0)
                    (6, 2.0) +- (0, 0.0)
                    (7, 2.0) +- (0, 0.0)
                    (8, 3.0) +- (0, 0.0)
                    (9, 2.0) +- (0, 0.0)
                };
                %
                \addplot[
                    color=red, %
                    mark=x, %
                    line width=2pt,
                    mark size=6pt,
                    error bars/.cd,
                    y dir=both, %
                    y explicit, %
                    error bar style={line width=1pt,solid},
                    error mark options={line width=1pt,mark size=4pt,rotate=90}
                ]
                coordinates {
                    (0, 0.0)  +- (0, 0.0)
                    (1, 0.7065655)  +- (0, 0.4553358) 
                    (2, 1.341673)  +- (0, 0.67091621) 
                    (3, 1.122926)  +- (0, 0.61281824)
                    (4, -1.1821935)  +- (0, 13.82444042)
                    (5, -0.952399)  +- (0, 15.35195457)
                    (6, -0.72672) +- (0, 20.33508414)
                    (7, -0.268983) +- (0, 22.77861454)
                    (8, -0.1310835) +- (0, 26.31013314)
                    (9, 0.65806) +- (0, 28.50670214)
                };
                %
            %
            %
            %
            %
            %
            %
            %
            %
            %
            %
            %
            %
            %
            %
            %
            %
            %
            %
                \end{axis}
            \end{tikzpicture}
         }
    }
    \hspace{1cm}
    \subfigure[\footnotesize Lowest cumulative reward: Interval CFMDP ($-698$), Gumbel-max SCM ($-698$)]{%
         \resizebox{0.76\columnwidth}{!}{
            \begin{tikzpicture}
                \begin{axis}[
                    xlabel={$t$},
                    ylabel={Mean reward at time step $t$},
                    title={Catastrophic Path},
                    grid=both,
                    width=20cm, height=8.5cm,
                    every axis/.style={font=\Huge},
                    %
                ]
                \addplot[
                    color=black, %
                    mark=*, %
                    line width=2pt,
                    mark size=3pt,
                    error bars/.cd,
                    y dir=both, %
                    y explicit, %
                    error bar style={line width=1pt,solid},
                    error mark options={line width=1pt,mark size=4pt,rotate=90}
                ]
                coordinates {
                    (0, 1.0)  +- (0, 0.0)
                    (1, 2.0)  +- (0, 0.0) 
                    (2, -100.0)  +- (0, 0.0) 
                    (3, -100.0)  +- (0, 0.0)
                    (4, -100.0)  +- (0, 0.0)
                    (5, -100.0) +- (0, 0.0)
                    (6, -100.0) +- (0, 0.0)
                    (7, -100.0) +- (0, 0.0)
                    (8, -100.0) +- (0, 0.0)
                    (9, -100.0) +- (0, 0.0)
                };
                %
                \addplot[
                    color=blue, %
                    mark=o, %
                    line width=2pt,
                    mark size=3pt,
                    error bars/.cd,
                    y dir=both, %
                    y explicit, %
                    error bar style={line width=1pt,solid},
                    error mark options={line width=1pt,mark size=4pt,rotate=90}
                ]
                coordinates {
                    (0, 0.0)  +- (0, 0.0)
                    (1, 0.504814)  +- (0, 0.49997682) 
                    (2, 0.8439835)  +- (0, 0.76831917) 
                    (3, -8.2709165)  +- (0, 28.93656754)
                    (4, -9.981082)  +- (0, 31.66825363)
                    (5, -12.1776325) +- (0, 34.53463233)
                    (6, -13.556076) +- (0, 38.62845372)
                    (7, -14.574418) +- (0, 42.49603359)
                    (8, -15.1757075) +- (0, 46.41913968)
                    (9, -15.3900395) +- (0, 50.33563368)
                };
                %
                \addplot[
                    color=red, %
                    mark=x, %
                    line width=2pt,
                    mark size=6pt,
                    error bars/.cd,
                    y dir=both, %
                    y explicit, %
                    error bar style={line width=1pt,solid},
                    error mark options={line width=1pt,mark size=4pt,rotate=90}
                ]
                coordinates {
                    (0, 0.0)  +- (0, 0.0)
                    (1, 0.701873)  +- (0, 0.45743556) 
                    (2, 1.1227805)  +- (0, 0.73433129) 
                    (3, -8.7503255)  +- (0, 30.30257976)
                    (4, -10.722092)  +- (0, 33.17618589)
                    (5, -13.10721)  +- (0, 36.0648089)
                    (6, -13.7631645) +- (0, 40.56553451)
                    (7, -13.909043) +- (0, 45.23829402)
                    (8, -13.472517) +- (0, 49.96270296)
                    (9, -12.8278835) +- (0, 54.38618735)
                };
                %
            %
            %
            %
            %
            %
            %
            %
            %
            %
            %
            %
            %
            %
            %
            %
            %
            %
            %
                \end{axis}
            \end{tikzpicture}
         }
    }
    \caption{Average instant reward of CF paths induced by policies on GridWorld $p=0.4$.}
    \label{fig: reward p=0.4}
\end{figure*}

\subsection{Experimental Setup}
To compare policy performance, we measure the average rewards of counterfactual paths induced by our policy and the Gumbel-max policy by uniformly sampling $200$ counterfactual MDPs from the ICFMDP and generating $10,000$ counterfactual paths over each sampled CFMDP. \jl{Since the interval CFMDP depends on the observed path, we select $4$  paths of varying optimality to evaluate how the observed path impacts the performance of both policies: an optimal path, a slightly suboptimal path that could reach the optimal reward with a few changes, a catastrophic path that enters a catastrophic, terminal state with low reward, and an almost catastrophic path that was close to entering a catastrophic state.} When measuring the average probability bound widths and execution time needed to generate the ICFMDPs, we averaged over $20$ randomly generated observed paths
\footnote{Further training details are provided in Appendix \ref{app: training details}, and the code is provided at \href{https://github.com/ddv-lab/robust-cf-inference-in-MDPs}{https://github.com/ddv-lab/robust-cf-inference-in-MDPs}
%
%
.}.

\subsection{GridWorld}
\jl{The GridWorld MDP is a $4 \times 4$ grid where an agent must navigate from the top-left corner to the goal state in the bottom-right corner, avoiding a dangerous terminal state in the centre. At each time step, the agent can move up, down, left, or right, but there is a small probability (controlled by hyper-parameter $p$) of moving in an unintended direction. As the agent nears the goal, the reward for each state increases, culminating in a reward of $+100$ for reaching the goal. Entering the dangerous state results in a penalty of $-100$. We use two versions of GridWorld: a less stochastic version with $p=0.9$ (i.e., $90$\% chance of moving in the chosen direction) and a more stochastic version with $p=0.4$.}

\paragraph{GridWorld ($p=0.9$)}
When $p=0.9$, the counterfactual probability bounds are typically narrow (see Table \ref{tab:nonzero_probs} for average measurements). Consequently, as shown in Figure \ref{fig: reward p=0.9}, both policies are nearly identical and perform similarly well across the optimal, slightly suboptimal, and catastrophic paths.
%
However, for the almost catastrophic path, the interval CFMDP path is more conservative and follows the observed path more closely (as this is where the probability bounds are narrowest), which typically requires one additional step to reach the goal state than the Gumbel-max SCM policy.
%

\paragraph{GridWorld ($p=0.4$)}
\jl{When $p=0.4$, the GridWorld environment becomes more uncertain, increasing the risk of entering the dangerous state even if correct actions are chosen. Thus, as shown in Figure \ref{fig: reward p=0.4}, the interval CFMDP policy adopts a more conservative approach, avoiding deviation from the observed policy if it cannot guarantee higher counterfactual rewards (see the slightly suboptimal and almost catastrophic paths), whereas the Gumbel-max SCM is inconsistent: it can yield higher rewards, but also much lower rewards, reflected in the wide error bars.} For the catastrophic path, both policies must deviate from the observed path to achieve a higher reward and, in this case, perform similarly.
%
%
%
%
\subsection{Sepsis}
The Sepsis MDP \citep{oberst2019counterfactual} simulates trajectories of Sepsis patients. Each state consists of four vital signs (heart rate, blood pressure, oxygen concentration, and glucose levels), categorised as low, normal, or high.
and three treatments that can be toggled on/off at each time step (8 actions in total). Unlike \citet{oberst2019counterfactual}, we scale rewards based on the number of out-of-range vital signs, between $-1000$ (patient dies) and $1000$ (patient discharged). \jl{Like the GridWorld $p=0.4$ experiment, the Sepsis MDP is highly uncertain, as many states are equally likely to lead to optimal and poor outcomes. Thus, as shown in Figure \ref{fig: reward sepsis}, both policies follow the observed optimal and almost catastrophic paths to guarantee rewards are no worse than the observation.} However, improving the catastrophic path requires deviating from the observation. Here, the Gumbel-max SCM policy, on average, performs better than the interval CFMDP policy. But, since both policies have lower bounds clipped at $-1000$, neither policy reliably improves over the observation. In contrast, for the slightly suboptimal path, the interval CFMDP policy performs significantly better, shown by its higher lower bounds. 
Moreover, in these two cases, the worst-case counterfactual path generated by the interval CFMDP policy is better than that of the Gumbel-max SCM policy,
indicating its greater robustness.
%
\begin{figure*}
    \centering
     \resizebox{0.6\textwidth}{!}{
        \begin{tikzpicture}[scale=1.0, every node/.style={scale=1.0}]
            \draw[thick, black] (-3, -0.25) rectangle (10, 0.25);
            %
            \draw[black, line width=1pt] (-2.5, 0.0) -- (-2,0.0);
            \fill[black] (-2.25,0.0) circle (2pt); %
            \node[right] at (-2,0.0) {\small Observed Path};
            
            %
            \draw[blue, line width=1pt] (1.0,0.0) -- (1.5,0.0);
            \node[draw=blue, circle, minimum size=4pt, inner sep=0pt] at (1.25,0.0) {}; %
            \node[right] at (1.5,0.0) {\small Interval CFMDP Policy};
            
            %
            \draw[red, line width=1pt] (5.5,0) -- (6,0);
            \node[red] at (5.75,0) {$\boldsymbol{\times}$}; %
            \node[right] at (6,0) {\small Gumbel-max SCM Policy};
        \end{tikzpicture}
    }\\
    \subfigure[\footnotesize Lowest cumulative reward: Interval CFMDP ($8000$), Gumbel-max SCM ($8000$)]{%
         \resizebox{0.76\columnwidth}{!}{
             \begin{tikzpicture}
                \begin{axis}[
                    xlabel={$t$},
                    ylabel={Mean reward at time step $t$},
                    title={Optimal Path},
                    grid=both,
                    width=20cm, height=8.5cm,
                    every axis/.style={font=\Huge},
                    %
                ]
                \addplot[
                    color=black, %
                    mark=*, %
                    line width=2pt,
                    mark size=3pt,
                ]
                coordinates {
                    (0, -50.0)
                    (1, 50.0)
                    (2, 1000.0)
                    (3, 1000.0)
                    (4, 1000.0)
                    (5, 1000.0)
                    (6, 1000.0)
                    (7, 1000.0)
                    (8, 1000.0)
                    (9, 1000.0)
                };
                %
                \addplot[
                    color=blue, %
                    mark=o, %
                    line width=2pt,
                    mark size=3pt,
                    error bars/.cd,
                    y dir=both, %
                    y explicit, %
                    error bar style={line width=1pt,solid},
                    error mark options={line width=1pt,mark size=4pt,rotate=90}
                ]
                coordinates {
                    (0, -50.0)  +- (0, 0.0)
                    (1, 50.0)  +- (0, 0.0) 
                    (2, 1000.0)  +- (0, 0.0) 
                    (3, 1000.0)  +- (0, 0.0)
                    (4, 1000.0)  +- (0, 0.0)
                    (5, 1000.0) +- (0, 0.0)
                    (6, 1000.0) +- (0, 0.0)
                    (7, 1000.0) +- (0, 0.0)
                    (8, 1000.0) +- (0, 0.0)
                    (9, 1000.0) +- (0, 0.0)
                };
                %
                \addplot[
                    color=red, %
                    mark=x, %
                    line width=2pt,
                    mark size=6pt,
                    error bars/.cd,
                    y dir=both, %
                    y explicit, %
                    error bar style={line width=1pt,solid},
                    error mark options={line width=1pt,mark size=4pt,rotate=90}
                ]
                coordinates {
                    (0, -50.0)  +- (0, 0.0)
                    (1, 50.0)  +- (0, 0.0) 
                    (2, 1000.0)  +- (0, 0.0) 
                    (3, 1000.0)  +- (0, 0.0)
                    (4, 1000.0)  +- (0, 0.0)
                    (5, 1000.0) +- (0, 0.0)
                    (6, 1000.0) +- (0, 0.0)
                    (7, 1000.0) +- (0, 0.0)
                    (8, 1000.0) +- (0, 0.0)
                    (9, 1000.0) +- (0, 0.0)
                };
                %
                \end{axis}
            \end{tikzpicture}
         }
    }
    \hspace{1cm}
    \subfigure[\footnotesize Lowest cumulative reward: Interval CFMDP ($-5980$), Gumbel-max SCM ($-8000$)]{%
         \resizebox{0.76\columnwidth}{!}{
            \begin{tikzpicture}
                \begin{axis}[
                    xlabel={$t$},
                    ylabel={Mean reward at time step $t$},
                    title={Slightly Suboptimal Path},
                    grid=both,
                    width=20cm, height=8.5cm,
                    every axis/.style={font=\Huge},
                    %
                ]
               \addplot[
                    color=black, %
                    mark=*, %
                    line width=2pt,
                    mark size=3pt,
                ]
                coordinates {
                    (0, -50.0)
                    (1, 50.0)
                    (2, -50.0)
                    (3, -50.0)
                    (4, -1000.0)
                    (5, -1000.0)
                    (6, -1000.0)
                    (7, -1000.0)
                    (8, -1000.0)
                    (9, -1000.0)
                };
                %
                \addplot[
                    color=blue, %
                    mark=o, %
                    line width=2pt,
                    mark size=3pt,
                    error bars/.cd,
                    y dir=both, %
                    y explicit, %
                    error bar style={line width=1pt,solid},
                    error mark options={line width=1pt,mark size=4pt,rotate=90}
                ]
                coordinates {
                    (0, -50.0)  +- (0, 0.0)
                    (1, 50.0)  +- (0, 0.0) 
                    (2, -50.0)  +- (0, 0.0) 
                    (3, 20.0631)  +- (0, 49.97539413)
                    (4, 71.206585)  +- (0, 226.02033693)
                    (5, 151.60797) +- (0, 359.23292559)
                    (6, 200.40593) +- (0, 408.86185176)
                    (7, 257.77948) +- (0, 466.10372804)
                    (8, 299.237465) +- (0, 501.82579506)
                    (9, 338.9129) +- (0, 532.06124996)
                };
                %
                \addplot[
                    color=red, %
                    mark=x, %
                    line width=2pt,
                    mark size=6pt,
                    error bars/.cd,
                    y dir=both, %
                    y explicit, %
                    error bar style={line width=1pt,solid},
                    error mark options={line width=1pt,mark size=4pt,rotate=90}
                ]
                coordinates {
                    (0, -50.0)  +- (0, 0.0)
                    (1, 20.00736)  +- (0, 49.99786741) 
                    (2, -12.282865)  +- (0, 267.598755) 
                    (3, -47.125995)  +- (0, 378.41755832)
                    (4, -15.381965)  +- (0, 461.77616558)
                    (5, 41.15459) +- (0, 521.53189262)
                    (6, 87.01595) +- (0, 564.22243126 )
                    (7, 132.62376) +- (0, 607.31338037)
                    (8, 170.168145) +- (0, 641.48013693)
                    (9, 201.813135) +- (0, 667.29441777)
                };
                %
                %
                %
                %
                %
                %
                %
                %
                %
                %
                %
                %
                %
                %
                %
                %
                %
                %
                %
                \end{axis}
            \end{tikzpicture}
         }
    }\\[-1.5pt]
    \subfigure[\footnotesize Lowest cumulative reward: Interval CFMDP ($100$), Gumbel-max SCM ($100$)]{%
         \resizebox{0.76\columnwidth}{!}{
             \begin{tikzpicture}
                \begin{axis}[
                    xlabel={$t$},
                    ylabel={Mean reward at time step $t$},
                    title={Almost Catastrophic Path},
                    grid=both,
                    every axis/.style={font=\Huge},
                    width=20cm, height=8.5cm,
                    %
                ]
               \addplot[
                    color=black, %
                    mark=*, %
                    line width=2pt,
                    mark size=3pt,
                ]
                coordinates {
                    (0, -50.0)
                    (1, 50.0)
                    (2, 50.0)
                    (3, 50.0)
                    (4, -50.0)
                    (5, 50.0)
                    (6, -50.0)
                    (7, 50.0)
                    (8, -50.0)
                    (9, 50.0)
                };
                %
                %
                \addplot[
                    color=blue, %
                    mark=o, %
                    line width=2pt,
                    mark size=3pt,
                    error bars/.cd,
                    y dir=both, %
                    y explicit, %
                    error bar style={line width=1pt,solid},
                    error mark options={line width=1pt,mark size=4pt,rotate=90}
                ]
                coordinates {
                    (0, -50.0)  +- (0, 0.0)
                    (1, 50.0)  +- (0, 0.0) 
                    (2, 50.0)  +- (0, 0.0) 
                    (3, 50.0)  +- (0, 0.0)
                    (4, -50.0)  +- (0, 0.0)
                    (5, 50.0) +- (0, 0.0)
                    (6, -50.0) +- (0, 0.0)
                    (7, 50.0) +- (0, 0.0)
                    (8, -50.0) +- (0, 0.0)
                    (9, 50.0) +- (0, 0.0)
                };
                %
                \addplot[
                    color=red, %
                    mark=x, %
                    line width=2pt,
                    mark size=6pt,
                    error bars/.cd,
                    y dir=both, %
                    y explicit, %
                    error bar style={line width=1pt,solid},
                    error mark options={line width=1pt,mark size=4pt,rotate=90}
                ]
                coordinates {
                    (0, -50.0)  +- (0, 0.0)
                    (1, 50.0)  +- (0, 0.0) 
                    (2, 50.0)  +- (0, 0.0) 
                    (3, 50.0)  +- (0, 0.0)
                    (4, -50.0)  +- (0, 0.0)
                    (5, 50.0) +- (0, 0.0)
                    (6, -50.0) +- (0, 0.0)
                    (7, 50.0) +- (0, 0.0)
                    (8, -50.0) +- (0, 0.0)
                    (9, 50.0) +- (0, 0.0)
                };
                %
                %
                %
                %
                %
                %
                %
                %
                %
                %
                %
                %
                %
                %
                %
                %
                %
                %
                %
                \end{axis}
            \end{tikzpicture}
         }
    }
    \hspace{1cm}
    \subfigure[\footnotesize Lowest cumulative reward: Interval CFMDP ($-7150$), Gumbel-max SCM ($-9050$)]{%
         \resizebox{0.76\columnwidth}{!}{
            \begin{tikzpicture}
                \begin{axis}[
                    xlabel={$t$},
                    ylabel={Mean reward at time step $t$},
                    title={Catastrophic Path},
                    grid=both,
                    width=20cm, height=8.5cm,
                    every axis/.style={font=\Huge},
                    %
                ]
               \addplot[
                    color=black, %
                    mark=*, %
                    line width=2pt,
                    mark size=3pt,
                ]
                coordinates {
                    (0, -50.0)
                    (1, -50.0)
                    (2, -1000.0)
                    (3, -1000.0)
                    (4, -1000.0)
                    (5, -1000.0)
                    (6, -1000.0)
                    (7, -1000.0)
                    (8, -1000.0)
                    (9, -1000.0)
                };
                %
                %
                \addplot[
                    color=blue, %
                    mark=o, %
                    line width=2pt,
                    mark size=3pt,
                    error bars/.cd,
                    y dir=both, %
                    y explicit, %
                    error bar style={line width=1pt,solid},
                    error mark options={line width=1pt,mark size=4pt,rotate=90}
                ]
                coordinates {
                    (0, -50.0)  +- (0, 0.0)
                    (1, -50.0)  +- (0, 0.0) 
                    (2, -50.0)  +- (0, 0.0) 
                    (3, -841.440725)  += (0, 354.24605512) -= (0, 158.559275)
                    (4, -884.98225)  += (0, 315.37519669) -= (0, 115.01775)
                    (5, -894.330425) += (0, 304.88572805) -= (0, 105.669575)
                    (6, -896.696175) += (0, 301.19954514) -= (0, 103.303825)
                    (7, -897.4635) += (0, 299.61791279) -= (0, 102.5365)
                    (8, -897.77595) += (0, 298.80392585) -= (0, 102.22405)
                    (9, -897.942975) += (0, 298.32920557) -= (0, 102.057025)
                };
                %
                \addplot[
                    color=red, %
                    mark=x, %
                    line width=2pt,
                    mark size=6pt,
                    error bars/.cd,
                    y dir=both, %
                    y explicit, %
                    error bar style={line width=1pt,solid},
                    error mark options={line width=1pt,mark size=4pt,rotate=90}
                ]
            coordinates {
                    (0, -50.0)  +- (0, 0.0)
                    (1, -360.675265)  +- (0, 479.39812699) 
                    (2, -432.27629)  +- (0, 510.38620897) 
                    (3, -467.029545)  += (0, 526.36009628) -= (0, 526.36009628)
                    (4, -439.17429)  += (0, 583.96638919) -= (0, 560.82571)
                    (5, -418.82704) += (0, 618.43027478) -= (0, 581.17296)
                    (6, -397.464895) += (0, 652.67322574) -= (0, 602.535105)
                    (7, -378.49052) += (0, 682.85407033) -= (0, 621.50948)
                    (8, -362.654195) += (0, 707.01412023) -= (0, 637.345805)
                    (9, -347.737935) += (0, 729.29076479) -= (0, 652.262065)
                };
                %
                %
                %
                %
                %
                %
                %
                %
                %
                %
                %
                %
                %
                %
                %
                %
                %
                %
                %
                \end{axis}
            \end{tikzpicture}
         }
    }
    \caption{Average instant reward of CF paths induced by policies on Sepsis.}
    \label{fig: reward sepsis}
\end{figure*}

%
%
%
\subsection{Interval CFMDP Bounds}
%
%
Table \ref{tab:nonzero_probs} presents the mean counterfactual probability bound widths (excluding transitions where the upper bound is $0$) for each MDP, averaged over 20 observed paths. We compare the bounds under counterfactual stability (CS) and monotonicity (M) assumptions, CS alone, and no assumptions. This shows that the assumptions marginally reduce the bound widths, indicating the assumptions tighten the bounds without excluding too many causal models, as intended.
\renewcommand{\arraystretch}{1}

\begin{table}
\centering
\caption{Mean width of counterfactual probability bounds}
\resizebox{0.8\columnwidth}{!}{%
\begin{tabular}{|c|c|c|c|}
\hline
\multirow{2}{*}{\textbf{Environment}} & \multicolumn{3}{c|}{\textbf{Assumptions}} \\ \cline{2-4}
 & \textbf{CS + M} & \textbf{CS} & \textbf{None\tablefootnote{\jl{Equivalent to \citet{li2024probabilities}'s bounds (see Section \ref{sec: equivalence with Li}).}}} \\ \hline
\textbf{GridWorld} ($p=0.9$) & 0.0817 & 0.0977 & 0.100 \\ \hline
\textbf{GridWorld} ($p=0.4$) & 0.552  & 0.638  & 0.646 \\ \hline
\textbf{Sepsis} & 0.138 & 0.140 & 0.140 \\ \hline
\end{tabular}
}
\label{tab:nonzero_probs}
\end{table}


\subsection{Execution Times}
Table \ref{tab: times} compares the average time needed to generate the interval CFMDP vs.\ the Gumbel-max SCM CFMDP for 20 observations.
The GridWorld algorithms were run single-threaded, while the Sepsis experiments were run in parallel.
Generating the interval CFMDP is significantly faster as it uses exact analytical bounds, whereas the Gumbel-max CFMDP requires sampling from the Gumbel distribution to estimate counterfactual transition probabilities. \jl{Since constructing the counterfactual MDP models is the main bottleneck in both approaches, ours is more efficient overall and suitable for larger MDPs.}
\begin{table}
\centering
\caption{Mean execution time to generate CFMDPs}
\resizebox{0.99\columnwidth}{!}{%
\begin{tabular}{|c|c|c|}
\hline
\multirow{2}{*}{\textbf{Environment}} & \multicolumn{2}{c|}{\textbf{Mean Execution Time (s)}} \\ \cline{2-3} 
                                      & \textbf{Interval CFMDP} & \textbf{Gumbel-max CFMDP} \\ \hline
\textbf{GridWorld ($p=0.9$) }                  & 0.261                   & 56.1                      \\ \hline
\textbf{GridWorld ($p=0.4$)  }                 & 0.336                   & 54.5                      \\ \hline
\textbf{Sepsis}                                 & 688                     & 2940                      \\ \hline
\end{tabular}%
}
\label{tab: times}
\end{table}


\section{Analysis}
\label{sec:analysis}
In the following sections, we will analyze European type approval regulation\footnote{Strictly speaking, the German enabling act (AFGBV) does not regulate type-approval, but how test \& operating permits are issued for SAE-Level-4 systems. Type-approval regulation for SAE-Level-3 systems follows UN Regulation No. 157 (UN-ECE-ALKS) \parencite{un157}.} regarding the underlying notions of ``safety'' and ``risk''.
We will classify these notions according to their absolute or relative character, underlying risk sources, or underlying concepts of harm.

\subsection{Classification of Safety Notions}
\label{sec:safety-notions}
We will refer to \emph{absolute} notions of safety as conceptualizations that assume the complete absence of any kind of risk.
Opposed to this, \emph{relative} notions of safety are based on a conceptualization that specifically includes risk acceptance criteria, e.g., in terms of ``tolerable'' risk or ``sufficient'' safety.

For classifying notions of safety by their underlying risk (or rather ``hazard'') sources, and different concepts of harm, \Cref{fig:hazard-sources} provides an overview of our reasoning, which is closely in line with the argumentation provided by Waymo in \parencite{favaro2023}.
We prefer ``hazard sources'' over ``risk sources'', as a risk must always be related to a \emph{cause} or \emph{source of harm} (i.e., a hazard \parencite[p.~1, def. 3.2]{iso51}).
Without a concrete (scenario) context that the system is operating in, a hazard is \emph{latent}: E.g., when operating in public traffic, there is a fundamental possibility that a \emph{collision with a pedestrian} leads to (physical) harm for that pedestrian. 
However, only if an automated vehicle shows (potentially) hazardous behavior (e.g., not decelerating properly) \emph{and} is located near a pedestrian (context), the hazard is instantiated and leads to a hazardous event.
\begin{figure*}
    \includeimg[width=.9\textwidth]{hazard-sources0.pdf}
    \caption{Graphical summary of a taxonomy of risk related to automated vehicles, extended based on ISO 21448 (\parencite{iso21448}) and \parencite{favaro2023}. Top: Causal chain from hazard sources to actual harm; bottom: summary of the individual elements' contributions to a resulting risk. Graphic translated from \parencite{nolte2024} \label{fig:hazard-sources}}
\end{figure*}
If the hazardous event cannot be mitigated or controlled, we see a loss event in which the pedestrian's health is harmed.
Note that this hypothetical chain of events is summarized in the definition of risk:
The probability of occurrence of harm is determined by a) the frequency with which hazard sources manifest, b) the time for which the system operates in a context that exposes the possibility of harm, and c) by the probability with which a hazardous event can be controlled.
A risk can then be determined as a function of the probability of harm and the severity of the harm potentially inflicted on the pedestrian.

In the following, we will apply this general model to introduce different types of hazard sources and also different types of harm.
\cref{fig:hazard-sources} shows two distinct hazard sources, i.e., functional insufficiencies and E/E-failures that can lead to hazardous behavior.
ISO~21488 \parencite{iso21448} defines functional insufficiencies as insufficiencies that stem from an incomplete or faulty system specification (specification insufficiencies).
In addition, the standard considers insufficiencies that stem from insufficient technical capability to operate inside the targeted Operational Design Domain (performance insufficiencies).
Functional insufficiencies are related to the ``Safety of the Intended Functionality (SOTIF)'' (according to ISO~21448), ``Behavioral Safety'' (according to Waymo \parencite{waymo2018}), or ``Operational Safety'' (according to UN Regulation No. 157 \parencite{un157}).
E/E-Failures are related to classic functional safety and are covered exhaustively by ISO~26262 \parencite{iso2018}.
Additional hazard sources can, e.g., be related to malicious security attacks (ISO~21434), or even to mechanical failures that should be covered (in the US) in the Federal Motor Vehicle Safety Standards (FMVSS).

For the classification of notions of safety by the related harm, in \parencite{salem2024, nolte2024}, we take a different approach compared to \parencite{koopman2024}:
We extend the concept of harm to the violation of stakeholder \emph{values}, where values are considered to be a ``standard of varying importance among other such standards that, when combined, form a value pattern that reduces complexity for stakeholders [\ldots] [and] determines situational actions [\ldots].'' \parencite{albert2008}
In this sense, values are profound, personal determinants for individual or collective behavior.
The notion of values being organized in a weighted value pattern shows that values can be ranked according to importance.
For automated vehicles, \emph{physical wellbeing} and \emph{mobility} can, e.g., be considered values which need to be balanced to achieve societal acceptance, in line with the discussion of required tradeoffs in \cref{sec:terminology}.
For the analysis of the following regulatory frameworks, we will evaluate if the given safety or risk notions allow tradeoffs regarding underlying stakeholder values. 

\subsection{UN Regulation No. 157 \& European Implementing Regulation (EU) 2022/1426}
\label{sec:enabling-act}
UN Regulation No. 157 \parencite{un157} and the European Implementing Regulation 2022/1426 \parencite{eu1426} provide type approval regulation for automated vehicles equipped with SAE-Level-3 (UN Reg. 157) and Level 4 (EU 2022/1426) systems on an international (UN Reg. 157) and European (EU 2022/1426) level.

Generally, EU type approval considers UN ECE regulations mandatory for its member states ((EU) 2018/858, \parencite{eu858}), while the EU largely forgoes implementing EU-specific type approval rules, it maintains the right to alter or to amend UN ECE regulation \parencite{eu858}.

In this respect, the terminology and conceptualizations in the EU Implementing Act closely follow those in UN Reg. No. 157.
The EU Implementing Act gives a clear reference to UN Reg. No. 157 \parencite[][Preamble,  Paragraph 1]{eu1426}.
Hence, the documents can be assessed in parallel.
Differences will be pointed out as necessary.

Both acts are written in rather technical language, including the formulation of technical requirements (e.g., regarding deceleration values or speeds in certain scenarios).
While providing exhaustive definitions and terminology, neither of both documents provide an actual definition of risk or safety.
The definition of ``unreasonable'' risk in both documents does not define risk, but only what is considered \emph{unreasonable}. It states that the ``overall level of risk for [the driver, (only in UN Reg. 157)] vehicle occupants and other road users which is increased compared to a competently and carefully driven manual vehicle.''
The pertaining notions of safety and risk can hence only be derived from the context in which they are used.

\subsubsection{Absolute vs. Relative Notions of Safety}
In line with the technical detail provided in the acts, both clearly imply a \emph{relative} notion of safety and refer to the absence of \emph{unreasonable} risk throughout, which is typical for technical safety definitions.

Both acts require sufficient proof and documentation that the to-be-approved automated driving systems are ``free of unreasonable safety risks to vehicle occupants and other road users'' for type approval.\footnote{As it targets SAE-Level-3 systems, UN Reg. 157 also refers to the driver, where applicable.}
In this respect, both acts demand that the manufacturers perform verification and validation activities for performance requirements that include ``[\ldots] the conclusion that the system is designed in such a way that it is free from unreasonable risks [\ldots]''.
Additionally, \emph{risk minimization} is a recurring theme when it comes to the definition of Minimum Risk Maneuvers (MRM).

Finally, supporting the relative notions of safety and risk, UN Reg. 157 introduces the concept of ``reasonable foreseeable and preventable'' \parencite[Article 1, Clause 5.1.1.]{un157} collisions, which implies that a residual risk will remain with the introduction of automated vehicles.
\parencite[][Appendix 3, Clause 3.1.]{un157} explicitly states that only \emph{some} scenarios that are unpreventable for a competent human driver can actually be prevented by an automated driving system.
While this concept is not applied throughout the EU Implementing Act, both documents explicitly refer to \emph{residual} risks that are related to the operation of automated driving systems (\parencite[][Annex I, Clause 1]{un157}, \parencite[][Annex II, Clause 7.1.1.]{eu1426}).

\subsubsection{Hazard Sources}
Hazard sources that are explicitly differentiated in UN Reg. 157 and (EU) 2022/1426 are E/E-failures that are in scope of functional safety (ISO~26262) and functional insufficiencies that are in scope of behavioral (or ``operational'') safety (ISO~21448).
Both documents consistently differentiate both sources when formulating requirements.

While the acts share a common definition of ``operational'' safety (\parencite[][Article 2, def. 30.]{eu1426}, \parencite[][Annex 4, def. 2.15.]{un157}), the definitions for functional safety differ.
\parencite{un157} defines functional safety as the ``absence of unreasonable risk under the occurrence of hazards caused by a malfunctioning behaviour of electric/electronic systems [\ldots]'', \parencite{eu1426} drops the specification of ``electric/electronic systems'' from the definition.
When taken at face value, this definition would mean that functional safety included all possible hazard sources, regardless of their origin, which is a deviation from the otherwise precise usage of safety-related terminology.

\subsubsection{Harm Types}
As the acts lack explicit definitions of safety and risk, there is no consistent and explicit notion of different harm types that could be differentiated.

\parencite{un157} gives little hints regarding different considered harm types.
``The absence of unreasonable risk'' in terms of human driving performance could hence be related to any chosen performance metric that allows a comparison with a competent careful human driver including, e.g., accident statistics, statistics about rule violations, or changes in traffic flow.

In \parencite{eu1426}, ``safety'' is, implicitly, attributed to the absence of unreasonable risk to life and limb of humans.
This is supported by the performance requirements that are formulated:
\parencite[][Annex II, Clause 1.1.2. (d)]{eu1426} demands that an automated driving system can adapt the vehicle behavior in a way that it minimizes risk and prioritizes the protection of human life.

Both acts demand the adherence to traffic rules (\parencite[][Annex 2, Clause 1.3.]{eu1426}, \parencite[][Clause 5.1.2.]{un157}).
\parencite[][Annex II, Clause 1.1.2. (c)]{eu1426} also demands that an automated driving system shall adapt its behavior to surrounding traffic conditions, such as the current traffic flow.
With the relative notion of risk in both acts, the unspecific clear statement that there may be unpreventable accidents \parencite{un157}, and a demand of prioritization of human life in \parencite{eu1426}, both acts could be interpreted to allow developers to make tradeoffs as discussed in \cref{sec:terminology}.


\subsubsection{Conclusion}
To summarize, the UN Reg. 157 and the (EU) 2022/1426 both clearly support the technical notion of safety as the absence of unreasonable risk.
The notion is used consistently throughout both documents, providing a sufficiently clear terminology for the developers of automated vehicles.
Uncertainty remains when it comes to considered harm types: Both acts do not explicitly allow for broader notions of safety, in the sense of \parencite{koopman2024} or \parencite{salem2024}.
Finally, a minor weak spot can be seen in the definition of risk acceptance criteria: Both acts take the human driving performance as a baseline.
While (EU) 2022/1426 specifies that these criteria are specific to the systems' Operational Design Domain \parencite[][Annex II, Clause 7.1.1.]{eu1426}, the reference to the concrete Operational Design Domain is missing in UN Reg. 157.
Without a clearly defined notion of safety, however, it remains unclear, how aspects beyond net accident statistics (which are given as an example in \parencite[][Annex II, Clause 7.1.1.]{eu1426}), can be addressed practically, as demanded by \parencite{koopman2024}.

\subsection{German Regulation (StVG \& AFGBV)}
\label{sec:afgbv}
The German L3 (Automated Driving Act) and L4 (Act on Autonomous Driving) Acts from 2017 and 2021,\footnote{Formally, these are amendments to the German Road Traffic Act (StVG): 06/21/2017, BGBl. I p. 1648, 07/12/2021 BGBl. I p. 3108.} respectively, provide enabling regulation for the operation of SAE-Level-3 and 4 vehicles on German roads.
The German Implementing Regulation (\parencite{afgbv}, AFGBV) defines how this enabling regulation is to be implemented for granting testing permits for SAE-Level-3 and -4 and driving permits for SAE-Level-3 and -4 automated driving systems.\footnote{Note that these permits do not grant EU-wide type approval, but serve as a special solution for German roads only. At the same time, the AFGBV has the same scope as (EU) 2022/1426.}
With all three acts, Germany was the first country to regulate the approval of automated vehicles for a domestic market.
All acts are subject to (repeated) evaluation until the year 2030 regarding their impact on the development of automated driving technology.
An assessment of the German AFGBV and comparisons to (EU) 2022/1426 have been given in \cite{steininger2022} in German.

Just as for UN Reg. 157 and (EU) 2022/1426, neither the StVG nor the AFGBV provide a clear definition of ``safety'' or ``risk'' -- even though the "safety" of the road traffic is one major goal of the StVG and StVO.
Again, different implicit notions of both concepts can only be interpreted from the context of existing wording.
An additional complication that is related to the German language is that ``safety'' and ``security'' can both be addressed as ``Sicherheit'', adding another potential source of unclarity.
Literal Quotations in this section are our translations from the German act.

\subsubsection{Absolute vs. Relative Notions of Safety}
For assessing absolute vs. relative notions of safety in German regulation, it should be mentioned that the main goal of the German StVO is to ensure the ``safety and ease of traffic flow'' -- an already diametral goal that requires human drivers to make tradeoffs.\footnote{For human drivers, this also creates legal uncertainty which can sometimes only be settled in a-posteriori court cases.}
While UN and EU regulation clearly shows a relative notion of safety\footnote{And even the StVG contains sections that use wording such as ``best possible safety for vehicle occupants'' (§1d (4) StVG) and acknowledges that there are unavoidable hazards to human life (§1e (2) No. 2c)).}, the German AFGBV contains ambiguous statements in this respect:
Several paragraphs contain a demand for a hazard free operation of automated vehicles.
§4 (1) No. 4 AFGBV, e.g., states that ``the operation of vehicles with autonomous driving functions must neither negatively impact road traffic safety or traffic flow, nor endanger the life and limb of persons.''
Additionally, §6 (1) AFGBV states that the permits for testing and operation have to be revoked, if it becomes apparent that a ``negative impact on road traffic safety or traffic flow, or hazards to the life and limb of persons cannot be ruled out''.
The same wording is used for the approval of operational design domains regulated in §10 (1) No. 1.
A particularly misleading statement is made regarding the requirements for technical supervision instances which are regulated in §14 (3) AFGBV which states that an automated vehicle has to be  ``immediately removed from the public traffic space if a risk minimal state leads to hazards to road traffic safety or traffic flow''.
Considering the argumentation in \cref{sec:terminology}, that residual risks related to the operation of automated driving systems are inevitable, these are strong statements which, if taken at face value, technically prohibit the operation of automated vehicles.
It suggests an \emph{absolute} notion of safety that requires the complete absence of risk.  
The last statement above is particularly contradictory in itself, considering that a risk \emph{minimal} state always implies a residual risk.

In addition to these absolute safety notions, there are passages which suggest a relative notion of safety:
The approval for Operational Design Domains is coupled to the proof that the operation of an automated vehicle ``neither negatively impacts road traffic safety or traffic flow, nor significantly endangers the life and limb of persons beyond the general risk of an impact that is typical of local road traffic'' (§9 (2) No. 3 AFGBV).
The addition of a relative risk measure ``beyond the general risk of an impact'' provides a relaxation (cf. also \cite{steininger2022}, who criticizes the aforementioned absolute safety notion) that also yields an implicit acceptance criterion (\emph{statistically as good as} human drivers) similar to the requirements stated in UN Reg. 157 and (EU) 2022/1426.

Additional hints for a relative notion of safety can be found in Annex 1, Part 1, No. 1.1 and Annex 1, Part 2, No. 10.
Part 1, No 1.1 specifies collision-avoidance requirements and acknowledges that not all collisions can be avoided.\footnote{The same is true for Part 2, No. 10, Clause 10.2.5.}
Part 2, No. 10 specifies requirements for test cases.
It demands that test cases are suitable to provide evidence that the ``safety of a vehicle with an autonomous driving function is increased compared to the safety of human-driven vehicles''.
This does not only acknowledge residual risks, but also yields an acceptance criterion (\emph{better} than human drivers) that is different from the implied acceptance criterion given in §9 (2) No. 3 AFGBV.

\subsubsection{Hazard Sources}
Regarding hazard sources, Annex 1 and 3 AFGBV explicitly refer to ISO~26262 and ISO~21448 (or rather its predecessor ISO/PAS~21448:2019).
However, regarding the discussion of actual hazard sources, the context in which both standards are mentioned is partially unclear:
Annex 1, Clause 1.3 discusses requirements for path and speed planning.
Clause 1.3 d) demands that in intersections, a Time to Collision (TTC) greater than 3 seconds must be guaranteed.
If manufacturers deviate from this, it is demanded that ``state-of-the-art, systematic safety evaluations'' are performed.
Fulfillment of the state of the art is assumed if ``the guidelines of ISO~26262:2018-12 Road Vehicles -- Functional Safety are fulfilled''.
Technically, ISO~26262 is not suitable to define the state of the art in this context, as the requirements discussed fall in the scope of operational (or behavioral) safety (ISO~21448).
A hazard source ``violated minimal time to collision'' is clearly a functional insufficiency, not an E/E-failure.

Similar unclarity presents itself in Annex 3, Clause 1 AFGBV: 
Clause 1 specifies the contents of the ``functional specification''.
The ``specification of the functionality'' is an artifact which is demanded in ISO~21448:2022 (Clause 5.3) \parencite{iso21448}.
However, Annex 3, Clause 1 AFGBV states that the ``functional specification'' is considered to comply to the state of the art, if the ``functional specification'' adheres to ISO~26262-3:2018 (Concept Phase).
Again, this assumes SOTIF-related contents as part of ISO~26262, which introduces the ``Item Definition'' as an artifact, which is significantly different from the ``specification of the functionality'' which is demanded by ISO~21448.
Finally, Annex 3, Clause 3 AFGBV demands a ``documentation of the safety concept'' which ``allows a functional safety assessment''.
A safety concept that is related to operational / behavioral safety is not demanded.
Technically, the unclarity with respect to the addressed harm types lead to the fact that the requirements provided by the AFGBV do not comply with the state of the art in the field, providing questionable regulation.

\subsubsection{Harm Types}
Just like UN Reg. 157 and (EU) 2022/1426, the German StVG and AFGBV do not explicitly differentiate concrete harm types for their notions of safety.
However, the AFGBV mentions three main concerns for the operation of automated vehicles which are \emph{traffic flow} (e.g., §4 (1) No. 4 AFGBV), compliance to \emph{traffic law} (e.g., §1e (2) No. 2 StVG), and the \emph{life and limb of humans} (e.g., §4 (1) No. 4 AFGBV).

Again, there is some ambiguity in the chosen wording:
The conflict between traffic flow and safety has already been argued in \cref{sec:terminology}.
The wording given in §4 (1) No. 4 and §6 (1) AFGBV  demand to ensure (absolute) safety \emph{and} traffic flow at the same time, which is impossible (cf. \cref{sec:terminology}) from an engineering perspective.
§1e (2) No. 2 StVG defines that ``vehicles with an autonomous driving function must [\ldots] be capable to comply to [\ldots] traffic rules in a self-contained manner''.
Taken at face value, this wording implies that an automated driving system could lose its testing or operating permit as soon as it violates a traffic rule.
A way out could be provided by §1 of the German Traffic Act (StVO) which demands careful and considerate behavior of all traffic participants and by that allows judgement calls for human drivers.
However, if §1 is applicable in certain situations is often settled in court cases. 
For developers, the application of §1 StVO during system design hence remains a legal risk.

While there are rather absolute statements as mentioned above, sections of the AFGBV and StVG can be interpreted to allow tradeoffs:
§1e (2) No. 2 b) demands that a system,  ``in case of an inevitable, alternative harm to legal objectives, considers the significance of the legal objectives, where the protection of human life has highest priority''.
This exact wording \emph{could} provide some slack for the absolute demands in other parts of the acts, enabling tradeoffs between (tolerable) risk and mobility as discussed in \cref{sec:terminology}.
However, it remains unclear if this interpretation is legally possible.

\subsubsection{Conclusion}
Compared to UN Reg. 157 and (EU) 2022/1426, the German StVG and AFGBV introduce openly inconsistent notions of safety and risk which are partially directly contradictory:
The wording partially implies absolute and relative notions of safety and risk at the same time.
The implied validation targets (``better'' or ``as good as'' human drivers) are equally contradictory. 
The partially implied absolute notions of safety, when taken at face value, prohibit engineers from making the tradeoffs required to develop a system that is safe and provides customer benefit at the same time. 
In consequence, the wording in the acts is prone to introducing legal uncertainty.
This uncertainty creates additional clarification need and effort for manufacturers and engineers who design and develop SAE-Level-3 and -4 automated driving systems. The use of undefined legal terms not only makes it more difficult for engineers to comply with the law, but also complicates the interpretation of the law and leads to legal uncertainty.

\subsection{UK Automated Vehicles Act 2024 (2024 c. 10)}
The UK has issued a national enabling act for regulating the approval of automated vehicles on the roads in the UK.
To the best of our knowledge, concrete implementing regulation has not been issued yet.
Regarding terminology, the act begins with a dedicated terminology section to clarify the terms used in the act \parencite[Part 1, Chapter 1, Section 1]{ukav2024}.
In that regard, the act defines a vehicle to drive ```autonomously' if --- (a)
it is being controlled not by an individual but by equipment of the vehicle, and (b) neither the vehicle nor its surroundings are being monitored by an individual with a view to immediate intervention in the driving of the vehicle.''
The act hence covers SAE-Level-3 to SAE-Level-5 automated driving systems.

\subsubsection{Absolute vs. Relative Notions of Safety}
While not providing an explicit definition of safety and risk, the UK Automated Vehicles Act (``UK AV Act'') \parencite{ukav2024} explicitly refers to a relative notion of safety.
Part~1, Chapter~1, Section~1, Clause (7)~(a) defines that an automated vehicle travels ```safely' if it travels to an acceptably safe standard''.
This clarifies that absolute safety is not achievable and that acceptance criteria to prove the acceptability of residual risk are required, even though a concrete safety definition is not given.
The act explicitly tasks the UK Secretary of State\footnote{Which means, that concrete implementation regulation needs to be enacted.} to install safety principles to determine the ``acceptably safe standard'' in Part~1, Chapter~1, Section~1, Clause (7)~(a).
In this respect, the act also provides one general validation target as it demands that the safety principles must ensure that ``authorized automated vehicles will achieve a level of safety equivalent to, or higher than, that of careful and competent human drivers''.
Hence, the top-level validation risk acceptance criterion assumed for UK regulation is ``\emph{at least as good} as human drivers''.

\subsubsection{Hazard Sources}
The UK AV Act contains no statements that could be directly related to different hazard sources.
Note that, in contrast to the rest of the analyzed documents, the UK AV Act is enabling rather than implementing regulation.
It is hence comparable to the German StVG, which does not refer to concrete hazard sources as well.

\subsubsection{Types of Harm}
Even though providing a clear relative safety notion, the missing definition of risk also implies a lack of explicitly differentiable types of harm.
Implicitly, three different types of harm can be derived from the wording in the act.
This includes the harm to life and limb of humans\footnote{Part~1, Chapter~3, Section~25 defines ``aggravated offence where death or serious injury occurs'' \parencite{ukav2024}.}, the violation of traffic rules\footnote{Part~1, Chapter~1, Clause~(7)~(b) defines that an automated vehicle travels ```legally' if it travels with an acceptably low risk of committing a traffic infraction''}, and the cause of inconvenience to the public \parencite[Part~1, Chapter~1, Section~58, Clause (2)~(d)]{ukav2024}.

The act connects all the aforementioned types of harm to ``risk'' or ``acceptable safety''.
While the act generally defines criminal offenses for providing ``false or misleading information about safety'', it also acknowledges possible defenses if it can be proven that ``reasonable precautions'' were taken and that ``due diligence'' was exercised to ``avoid the commission of the offence''.
This statement could enable tradeoffs within the scope of ``reasonable risk'' to the life and limb of humans, the violation of traffic rules, or to the cause of inconvenience to the public, as we argued in \cref{sec:terminology}.

\subsubsection{Conclusion}
From the set of reviewed documents, the current UK AV Act is the one with the most obvious relative notions of safety and risk and the one that seems to provide a legal framework for permitting tradeoffs.
In our review, we did not spot major inconsistency beyond a missing definitions of safety and risk\footnote{Note that with the Office for Product Safety and Standards (OPSS), there is a British government agency that maintains an exhaustive and widely focussed ``Risk Lexicon'' that provides suitable risk definitions. For us, it remains unclear, to what extent this terminology is assumed general knowledge in British legislation.}.
The general, relative notion of safety and the related alleged ability for designers to argue well-founded development tradeoffs within the legal framework could prove beneficial for the actual implementation of automated driving systems.
While the act thus appears as a solid foundation for the market introduction of automated vehicles, without accompanying implementing regulation, it is too early to draw definite conclusions.

\section{\ours as a Feature Encoder}
To better understand \ours, we examine whether its in-context learning capability also produces separable and meaningful feature representations.

\subsection{Challenges in Embedding Extraction}
As described in~\autoref{sec:relimiary}, the output tokens from the multiple transformer layers in \ours correspond one-to-one with the input tokens, forming a tensor of size $(N+1) \times (d+1) \times k$. The final token, corresponding to the (dummy) label embedding $\tilde{\vy}^*$ of the test instance, is passed through an MLP block to produce the output. An intuitive idea is to treat the output tokens associated with the training label embeddings $\{\vy_i\}_{i=1}^N$—prior to the MLP block—as the extracted embeddings for the training data.

However, this straightforward method has one fundamental limitation due to the distinct roles of labeled training and unlabeled test data in \ours's in-context learning process. Specifically, the label embeddings for the training instances are derived from their true labels, while those for the test instances rely on dummy labels. This discrepancy results in output embeddings that are inherently {\em non-comparable} between the training and test instances. 

\subsection{Leave-one-fold-out Feature Extraction}
To address this challenge, we propose a leave-one-fold-out strategy that enables the extraction of comparable embeddings for training and test data. Within the \ours framework, we define the support set $\gS$ as the examples with true labels and the query set $\gQ$ as the examples with dummy labels. Under the standard configuration, $\gS$ corresponds to the labeled training set, while $\gQ$ corresponds to the unlabeled test instances.

A key trade-off arises: $\gS$ needs to include as much labeled data as possible to propagate knowledge from $\gS$ to $\gQ$ effectively in the in-context learning process. However, to ensure that embeddings for training and test instances are comparable, training instances must also be included in $\gQ$ and paired with dummy label embeddings.

To reconcile these requirements, we split the training set into multiple folds (\eg, 10 folds). One fold is used as $\gQ$ for embedding extraction, while the remaining folds constitute $\gS$ with true training labels. This ensures that $\gS$ retains sufficient label information while allowing embeddings to be extracted for the training data in $\gQ$. An extreme version of this strategy operates in a leave-one-out fashion, with only one training instance placed in $\gQ$ at a time.

Results in~\autoref{fig:our_extraction}~(c)-(f) demonstrate that embeddings extracted using this strategy (with 10 folds) more effectively reveal dataset properties. We observe that \ours simplifies tabular data distributions and transforms datasets into nearly separable embedding spaces, particularly in the embeddings after intermediate transformer layers.
\begin{table}[t]
\vspace{-3mm}
\caption{Average rank (lower is better) of \ours and a linear classifier trained on the extracted embeddings across 29 classification datasets.  ``Combined'' refers to an approach where embeddings from up to three layers (from the 12 available layers) are selected based on the validation set performance and concatenated.
}
\small
\label{tab:linear_probing}
\tabcolsep 1.5pt
\begin{tabular}{ccccccc}
\addlinespace
\toprule
$\downarrow$ & \ours &Vanilla & Layer 6 & Layer 9 & Layer 12 & Combined \\     
\midrule
Rank & 3.12 & 3.43& 5.03 & 4.72 & 2.45 & \textbf{2.24} \\
\bottomrule
\end{tabular}
\vspace{-3mm}
\end{table}

\subsection{Validation of Embedding Quality}
To validate the quality of the extracted embeddings, we train a logistic regression on top of the extracted embeddings. Specifically, the classifier is trained on embeddings derived from the training set and evaluated on test set embeddings. The average rank across 29 classification datasets from the tiny benchmark2 in~\citet{Ye2024Closer} is reported in~\autoref{tab:linear_probing}.

Remarkably, training a linear classifier on these extracted embeddings yields performance comparable to \ours's in-context learner. Embeddings from intermediate layers (or their selected concatenations) sometimes achieve even better results. These findings highlight \ours's potential as a robust feature encoder, offering valuable insights into its architecture and paving the way for broader applications in tabular data analysis and representation learning.

\section{Improving Fleetwide Efficiency}\label{sec:improvements}
 \begin{figure}[t]
    \centering
    \includegraphics[height=2in]{final_figs/pg_cl.png}
    \caption{This figure demonstrates the effect of an XLA algebraic simplification optimization on Program Goodput (PG) across a benchmark of the top 150 fleet workloads. Looking at the PG in this way allows us to bisect which code changes improved or regressed overall fleet efficiency.}
    \label{fig:pg_cl}
\end{figure}

We show how \mpg is a robust quantifier of ML Fleet performance through optimization examples from Google's ML Fleet in production. We present a breakdown of the MPG components using segmented fleet data and demonstrate how this procedure can help identify potential optimization techniques. Additionally, we showcase the effects of deploying these optimizations and how MPG helps verify and track performance improvements.

Looking at the aggregated MPG of the fleet does not necessarily help ML practitioners identify what kinds of improvements will make the largest impact on the fleet; this is where the decomposability of the metric comes into play, as shown in \autoref{tab:improvements}. By breaking MPG into its three components; Program Goodput, Runtime Goodput, and Scheduling Goodput, we can diagnose fleetwide issues and identify the types of optimizations that would most improve fleet efficiency. Furthermore, we can segment the fleet using the characteristics described in Section~\ref{sec:fleet} in order to identify issues with specific workload types and propose model-level optimizations. 


\renewcommand*{\arraystretch}{1.25}
\begin{table*}[h!]
\caption{Optimizing different components of \mpg.}
\label{tab:improvements}
\resizebox{\textwidth}{!}{%
\scriptsize
\begin{tabular}{@{}lp{2cm}p{3cm}p{3cm}p{4cm}@{}}
\toprule
\textbf{ML Fleet Stack Layer} & \textbf{Program Goodput $\times$} & \textbf{Runtime Goodput $\times$} & \textbf{Scheduling Goodput $=$} & \textbf{Workload \mpg} \\ \bottomrule
\textbf{Compiler:} \\ On-duty step time decreases &  \textbf{Increases} & Decreases if device-bound \newline Decreases if host-bound & Decreases if device-bound \newline No change if host-bound & \textbf{Increases if device-bound}  \newline No change if host-bound
 \\ \midrule
\textbf{Runtime:} \\ Off-duty time or preemption waste decreases & No change & \textbf{Increases} & Decreases & \textbf{Increases} \\ \midrule
\textbf{Scheduler:} \\ Partially-allocated time decreases & No change & No change & \textbf{Increases}  & \textbf{Increases} 
\\ \bottomrule
\end{tabular}%
}
\end{table*}

\subsection{Program Goodput Optimizations}
We present various techniques and strategies that have been employed at \google over recent years for improving Program Goodput in our ML fleets, ranging from parallelization methods to compiler optimizations. Recall that PG measures the effective utilization of computational resources. As ML models grow in size and complexity, optimizing PG becomes increasingly important to make efficient use of hardware and reduce compute times. With PG instrumentation, we have been able to pinpoint which segments of the fleet require further optimization at the compiler or ML model level. 

 \begin{figure}[t]
    \centering
    \includegraphics[height=2in]{final_figs/pg_chip.png}
    \caption{Tracking the Program Goodput (PG) versus allocation trends for a particular domain-specific chip in an ML fleet. Looking at the disaggregated segments of MPG can help reveal fleetwide trends and interactions between different layers in the ML fleet stack, informing future design decisions.}
    \label{fig:pg_chip_type}
\end{figure}


 



 \textbf{Overlapping communication and computation.}
To identify potential system optimizations that can improve fleet efficiency, we can look at the PG of workloads segmented by performance characteristics. In other words, how many of the workloads in the fleet are compute-bound versus communication-bound? By segmenting the PG in this way, it is possible for us to identify that many high-cost workloads are communication-bound. 

To address this issue at the high-level operation (HLO) level, a technique that overlaps communication with computation was developed and deployed in our production fleet (described in ~\citet{wang2022overlap}). This technique decomposes communication collectives, along with the dependent computation operations, into a sequence of finer-grained operations to hide data transfer latency so that better system utilization is achieved. This approach improved the overall system throughput by up to 1.38$\times$ and achieves 72\% FLOPS utilization on 1024 TPU chips for a large language model with 500 billion parameters~\cite{wang2022overlap}.

\textbf{Compiler autotuning.}
At the fleet level, we have also developed and deployed optimizations that improve code-generation quality and can be generalized to any workload in the fleet. XTAT~\cite{phothilimthana2021flexible} is an autotuner for production ML compilers that tunes multiple compiler stages, including tensor layouts, operator fusion decisions, tile sizes and code generation parameters. Evaluated over 150 ML training and inference models on TPUs, XTAT offers speedups over the heavily-optimized XLA compiler in the fleet.

\textbf{Example: Quantifying the impact of an XLA optimization on the TPU fleet.}
It is rare for any single optimization to have a significant impact on overall fleet-wide PG. But we can track the impact of these optimizations by looking at the change in PG for a fixed set of benchmarked workloads or segment of the production fleet over time. For example, looking at a benchmark of the top 150 most costly workloads in the fleet, \autoref{fig:pg_cl} pinpoints the effect of a code change that was submitted to the XLA compiler - in this case, an algebraic simplification in the compiler graph. The dramatic increase in PG for the benchmark of 150 workloads suggests that the positive impact of this optimization can be generalized to the ML fleet as a whole. 

It is also helpful to look at PG fluctuations across hardware segments of the fleet. \autoref{fig:pg_chip_type} illustrates a notional example where looking at segmented PG can uncover insights that would otherwise be hidden by looking at aggregate metrics. In this case, the segmented data suggests that when a new ML accelerator chip is introduced to the fleet, the workloads running on that chip may initially have a low PG, since the model / compiler code has not been fully tailored for that chip yet. As user adoption increases and accelerator-specific software optimizations are rolled out to the fleet, PG gets closer to theoretical peak efficiency. In other words, hardware accelerator maturity tends to yield greater PG over time. As the chip nears the end of its lifecycle (represented by decreasing allocation in the fleet, and illustrated in \autoref{fig:pg_chip_type} by the ``Chip decommissioned'' label), the PG decreases due to lower chip usage and natural workload/compiler drift. This highlights the importance of co-design across all layers of the ML fleet to make sure that both the software (compiler) and hardware (chips) are optimized for the latest workloads.
\subsection{Runtime Goodput Optimizations}
Outside of device time, host overhead and pre-emptions can be major bottlenecks for some workloads, and can be tracked by measuring Runtime Goodput. For example, training jobs usually use input pipelines to ingest and transform input data, which could be bottlenecks for certain models that ingest large amounts of data. Some solutions have been proposed to reduce host overhead, such as Plumber~\cite{kuchnik2022plumber}, a tool to find bottlenecks in ML input pipelines. 

We can improve the RG of the ML fleet with asynchronous strategies such as sharding the dataflow graph, as proposed by Pathways \cite{barham2022pathways}. The segmented analysis of RG shows that the particular workloads on Pathways tend to have higher RG scores over time, validating the benefits of Pathways for our particular ML Fleet. Also, techniques such as asynchronous checkpointing \citep{maurya2024datastates, nicolae2020deepfreeze} can reduce the time spent fetching previous model training checkpoints where the accelerators temporarily pause training and are completely idle.  


Other strategies, such as ahead-of-time compilation, where programs are compiled on less expensive hardware such as CPUs and then executed on TPUs, can also improve RG. By offloading compilation to a less expensive chip and storing the results in a compilation cache, we can reduce the total runtime of more specialized accelerators. These techniques are often implemented in common ML frameworks, such as TensorFlow \cite{aotTF} and JAX \cite{aotJAX}.


To demonstrate the benefits of these framework-specific optimizations, we can examine the RG of the ML Fleet at the framework level of the system stack described in \autoref{fig:ml_stack}. This can help us understand which frameworks or runtime strategies may be better suited for which workloads.
\autoref{fig:segmented_runtime_goodput} shows RG scores from a sample of Google's ML fleet for internal workloads, segmented based on characteristics such as model architecture, product area or workload phase (training, real-time serving, or bulk inference), and compared to a baseline of top fleet workloads. Although the segments in \autoref{fig:segmented_runtime_goodput} are not explicitly identified, we demonstrate how segmenting RG based on workload characteristics (Segment~A, Segment~B, and Segment~C) can reveal trends that would otherwise be hidden by aggregate fleet metrics (represented by the "Top Fleet Workloads" segment). For example, training workloads running JAX with Pathways may tend to have a higher RG, possibly due to the fact that Pathways is single-client \cite{barham2022pathways} and therefore better optimized for training than multi-client frameworks.

\begin{figure}[t]
    \centering
    \includegraphics[width=\columnwidth]{final_figs/segmented_runtime_goodput.png}
    \caption{Runtime goodput speedups over the course of one quarter, segmented by fleet workload types. Speedup is normalized to the top N workloads in the fleet, measured at the beginning of the quarter. }
    \label{fig:segmented_runtime_goodput}
\end{figure}

Examining the data along a different axis, training versus real-time serving versus bulk inference, can also be helpful. Using a sample from Google's ML fleet, \autoref{fig:train_inf_rg} illustrates that training workloads tend to have a higher RG than serving workloads. This is most likely due to the inherently different nature of serving and training.  Typically, training workloads have more constant computational demands, while real-time serving can fluctuate based on user demand. The slight decrease in serving RG can be attributed to transitory demands on the fleet, but it remains relatively stable in comparison to the bulk inference segment. The huge fluctuation in RG for bulk inference highlights the changing nature of production fleet demands. Previously, the bulk inference segment of the fleet was dominated by workloads running on a single core, where each chip contained a replica of the model, resulting in more easily accessible checkpoints/data and less accelerator wait time. However, as we move to larger models, the weights must be sharded across multiple chips, resulting in more expensive data reads. Additionally, the rise of expert-based models \cite{shazeer2017moe} has made bulk inference runtime much more complex to optimize, as some machines must wait for others for distillation of weight updates in a student-teacher model. This has resulted in an temporary decrease of RG for the bulk inference segment between "Month 3" and "Month 6" of \autoref{fig:train_inf_rg}. This example illustrates how analyzing disaggregated RG can allow ML fleet architects to make informed decisions about their runtime stack by pinpointing segments that may be more susceptible to shifting fleet demands.

\begin{figure}[t]
    \centering
    \includegraphics[height=1.8in]{final_figs/train_inf_rg.png}
    \caption{Runtime Goodput trends for a notional slice of a sample ML fleet over a period of six months, segmented by workload phase.}
    \label{fig:train_inf_rg}
\end{figure}



\subsection{Scheduling Goodput Optimizations}

The optimization of Scheduling Goodput (SG) can be presented as a bin packing problem, as described in \autoref{sec:scheduler}. Users launch workloads with varying TPU topology sizes, and the scheduling algorithm must determine how to best fit these workloads into the existing fleet of allocated chips. This process presents numerous challenges, primarily due to the wide variety of job sizes in the fleet, ranging from single-chip to multipod configurations \cite{kumar2021exploring}.

A significant complication in job scheduling is that it requires more than mere availability in the fleet. The topology of the available hardware must also satisfy the topology requirements of the workload, which is sometimes impossible without first pre-empting other jobs. Consequently, suboptimal scheduling can have a cascading effect on other components of \mpg.



We can identify availability issues in the ML Fleet by looking at the SG for jobs with different chip allocation requirements, as shown in \autoref{fig:sg_job_size}. The data shows that the overall SG is already close to optimal, due to defragmentation techniques and scheduling optimizations. However, it is interesting to note which jobs tend to have the highest SG: extra-large jobs which require the greatest number of chips or possibly multiple TPU pods, as well as smaller jobs which require only a single chip or a few chips. 

This is likely due to the way the scheduler deals with evictions; evicting extremely large jobs would have a severe negative impact on the overall \mpg score due to their huge startup overhead. Once the extra-large job is running, it is also immensely dependent on checkpointing and data sharding, affecting the Runtime and Program Goodput components as well. In short, evicting extra-large jobs from the hardware would present a cascading series of failures, strongly incentivizing the scheduler to reduce churn for these jobs and evict medium-sized jobs instead. 

On the other hand, extremely small jobs usually do not get prematurely evicted since they are more likely to finish quickly, and if pre-empted, it is usually quicker to find topologically matching availability. With extremely small jobs, the scheduler has more flexibility to intelligently allocate the workloads to optimal compute cells in order to defragment the overall ML fleet availability. It is also important to note that for workloads of all sizes, the SG is greater than 95\% due to the particular pre-emption preferences of the scheduler. The pre-emption preferences of the scheduler can be tuned, e.g. to require a SG of greater than 95\% for medium-sized jobs, but this could reduce the SG for other segments of the fleet.


\begin{figure}[t]
    \centering
    \includegraphics[height=1.8in]{final_figs/sg_job_size.png}
    \caption{Scheduling goodput by job size. Extra-large and small jobs tend to have better scheduling goodput due to the scheduler's preemption algorithm.}
    \label{fig:sg_job_size}
\end{figure}



\section{Conclusion}
In this work, we propose a simple yet effective approach, called SMILE, for graph few-shot learning with fewer tasks. Specifically, we introduce a novel dual-level mixup strategy, including within-task and across-task mixup, for enriching the diversity of nodes within each task and the diversity of tasks. Also, we incorporate the degree-based prior information to learn expressive node embeddings. Theoretically, we prove that SMILE effectively enhances the model's generalization performance. Empirically, we conduct extensive experiments on multiple benchmarks and the results suggest that SMILE significantly outperforms other baselines, including both in-domain and cross-domain few-shot settings.


\bibliography{main}
\bibliographystyle{sections/icml2025}

\appendix
\onecolumn
\subsection{Lloyd-Max Algorithm}
\label{subsec:Lloyd-Max}
For a given quantization bitwidth $B$ and an operand $\bm{X}$, the Lloyd-Max algorithm finds $2^B$ quantization levels $\{\hat{x}_i\}_{i=1}^{2^B}$ such that quantizing $\bm{X}$ by rounding each scalar in $\bm{X}$ to the nearest quantization level minimizes the quantization MSE. 

The algorithm starts with an initial guess of quantization levels and then iteratively computes quantization thresholds $\{\tau_i\}_{i=1}^{2^B-1}$ and updates quantization levels $\{\hat{x}_i\}_{i=1}^{2^B}$. Specifically, at iteration $n$, thresholds are set to the midpoints of the previous iteration's levels:
\begin{align*}
    \tau_i^{(n)}=\frac{\hat{x}_i^{(n-1)}+\hat{x}_{i+1}^{(n-1)}}2 \text{ for } i=1\ldots 2^B-1
\end{align*}
Subsequently, the quantization levels are re-computed as conditional means of the data regions defined by the new thresholds:
\begin{align*}
    \hat{x}_i^{(n)}=\mathbb{E}\left[ \bm{X} \big| \bm{X}\in [\tau_{i-1}^{(n)},\tau_i^{(n)}] \right] \text{ for } i=1\ldots 2^B
\end{align*}
where to satisfy boundary conditions we have $\tau_0=-\infty$ and $\tau_{2^B}=\infty$. The algorithm iterates the above steps until convergence.

Figure \ref{fig:lm_quant} compares the quantization levels of a $7$-bit floating point (E3M3) quantizer (left) to a $7$-bit Lloyd-Max quantizer (right) when quantizing a layer of weights from the GPT3-126M model at a per-tensor granularity. As shown, the Lloyd-Max quantizer achieves substantially lower quantization MSE. Further, Table \ref{tab:FP7_vs_LM7} shows the superior perplexity achieved by Lloyd-Max quantizers for bitwidths of $7$, $6$ and $5$. The difference between the quantizers is clear at 5 bits, where per-tensor FP quantization incurs a drastic and unacceptable increase in perplexity, while Lloyd-Max quantization incurs a much smaller increase. Nevertheless, we note that even the optimal Lloyd-Max quantizer incurs a notable ($\sim 1.5$) increase in perplexity due to the coarse granularity of quantization. 

\begin{figure}[h]
  \centering
  \includegraphics[width=0.7\linewidth]{sections/figures/LM7_FP7.pdf}
  \caption{\small Quantization levels and the corresponding quantization MSE of Floating Point (left) vs Lloyd-Max (right) Quantizers for a layer of weights in the GPT3-126M model.}
  \label{fig:lm_quant}
\end{figure}

\begin{table}[h]\scriptsize
\begin{center}
\caption{\label{tab:FP7_vs_LM7} \small Comparing perplexity (lower is better) achieved by floating point quantizers and Lloyd-Max quantizers on a GPT3-126M model for the Wikitext-103 dataset.}
\begin{tabular}{c|cc|c}
\hline
 \multirow{2}{*}{\textbf{Bitwidth}} & \multicolumn{2}{|c|}{\textbf{Floating-Point Quantizer}} & \textbf{Lloyd-Max Quantizer} \\
 & Best Format & Wikitext-103 Perplexity & Wikitext-103 Perplexity \\
\hline
7 & E3M3 & 18.32 & 18.27 \\
6 & E3M2 & 19.07 & 18.51 \\
5 & E4M0 & 43.89 & 19.71 \\
\hline
\end{tabular}
\end{center}
\end{table}

\subsection{Proof of Local Optimality of LO-BCQ}
\label{subsec:lobcq_opt_proof}
For a given block $\bm{b}_j$, the quantization MSE during LO-BCQ can be empirically evaluated as $\frac{1}{L_b}\lVert \bm{b}_j- \bm{\hat{b}}_j\rVert^2_2$ where $\bm{\hat{b}}_j$ is computed from equation (\ref{eq:clustered_quantization_definition}) as $C_{f(\bm{b}_j)}(\bm{b}_j)$. Further, for a given block cluster $\mathcal{B}_i$, we compute the quantization MSE as $\frac{1}{|\mathcal{B}_{i}|}\sum_{\bm{b} \in \mathcal{B}_{i}} \frac{1}{L_b}\lVert \bm{b}- C_i^{(n)}(\bm{b})\rVert^2_2$. Therefore, at the end of iteration $n$, we evaluate the overall quantization MSE $J^{(n)}$ for a given operand $\bm{X}$ composed of $N_c$ block clusters as:
\begin{align*}
    \label{eq:mse_iter_n}
    J^{(n)} = \frac{1}{N_c} \sum_{i=1}^{N_c} \frac{1}{|\mathcal{B}_{i}^{(n)}|}\sum_{\bm{v} \in \mathcal{B}_{i}^{(n)}} \frac{1}{L_b}\lVert \bm{b}- B_i^{(n)}(\bm{b})\rVert^2_2
\end{align*}

At the end of iteration $n$, the codebooks are updated from $\mathcal{C}^{(n-1)}$ to $\mathcal{C}^{(n)}$. However, the mapping of a given vector $\bm{b}_j$ to quantizers $\mathcal{C}^{(n)}$ remains as  $f^{(n)}(\bm{b}_j)$. At the next iteration, during the vector clustering step, $f^{(n+1)}(\bm{b}_j)$ finds new mapping of $\bm{b}_j$ to updated codebooks $\mathcal{C}^{(n)}$ such that the quantization MSE over the candidate codebooks is minimized. Therefore, we obtain the following result for $\bm{b}_j$:
\begin{align*}
\frac{1}{L_b}\lVert \bm{b}_j - C_{f^{(n+1)}(\bm{b}_j)}^{(n)}(\bm{b}_j)\rVert^2_2 \le \frac{1}{L_b}\lVert \bm{b}_j - C_{f^{(n)}(\bm{b}_j)}^{(n)}(\bm{b}_j)\rVert^2_2
\end{align*}

That is, quantizing $\bm{b}_j$ at the end of the block clustering step of iteration $n+1$ results in lower quantization MSE compared to quantizing at the end of iteration $n$. Since this is true for all $\bm{b} \in \bm{X}$, we assert the following:
\begin{equation}
\begin{split}
\label{eq:mse_ineq_1}
    \tilde{J}^{(n+1)} &= \frac{1}{N_c} \sum_{i=1}^{N_c} \frac{1}{|\mathcal{B}_{i}^{(n+1)}|}\sum_{\bm{b} \in \mathcal{B}_{i}^{(n+1)}} \frac{1}{L_b}\lVert \bm{b} - C_i^{(n)}(b)\rVert^2_2 \le J^{(n)}
\end{split}
\end{equation}
where $\tilde{J}^{(n+1)}$ is the the quantization MSE after the vector clustering step at iteration $n+1$.

Next, during the codebook update step (\ref{eq:quantizers_update}) at iteration $n+1$, the per-cluster codebooks $\mathcal{C}^{(n)}$ are updated to $\mathcal{C}^{(n+1)}$ by invoking the Lloyd-Max algorithm \citep{Lloyd}. We know that for any given value distribution, the Lloyd-Max algorithm minimizes the quantization MSE. Therefore, for a given vector cluster $\mathcal{B}_i$ we obtain the following result:

\begin{equation}
    \frac{1}{|\mathcal{B}_{i}^{(n+1)}|}\sum_{\bm{b} \in \mathcal{B}_{i}^{(n+1)}} \frac{1}{L_b}\lVert \bm{b}- C_i^{(n+1)}(\bm{b})\rVert^2_2 \le \frac{1}{|\mathcal{B}_{i}^{(n+1)}|}\sum_{\bm{b} \in \mathcal{B}_{i}^{(n+1)}} \frac{1}{L_b}\lVert \bm{b}- C_i^{(n)}(\bm{b})\rVert^2_2
\end{equation}

The above equation states that quantizing the given block cluster $\mathcal{B}_i$ after updating the associated codebook from $C_i^{(n)}$ to $C_i^{(n+1)}$ results in lower quantization MSE. Since this is true for all the block clusters, we derive the following result: 
\begin{equation}
\begin{split}
\label{eq:mse_ineq_2}
     J^{(n+1)} &= \frac{1}{N_c} \sum_{i=1}^{N_c} \frac{1}{|\mathcal{B}_{i}^{(n+1)}|}\sum_{\bm{b} \in \mathcal{B}_{i}^{(n+1)}} \frac{1}{L_b}\lVert \bm{b}- C_i^{(n+1)}(\bm{b})\rVert^2_2  \le \tilde{J}^{(n+1)}   
\end{split}
\end{equation}

Following (\ref{eq:mse_ineq_1}) and (\ref{eq:mse_ineq_2}), we find that the quantization MSE is non-increasing for each iteration, that is, $J^{(1)} \ge J^{(2)} \ge J^{(3)} \ge \ldots \ge J^{(M)}$ where $M$ is the maximum number of iterations. 
%Therefore, we can say that if the algorithm converges, then it must be that it has converged to a local minimum. 
\hfill $\blacksquare$


\begin{figure}
    \begin{center}
    \includegraphics[width=0.5\textwidth]{sections//figures/mse_vs_iter.pdf}
    \end{center}
    \caption{\small NMSE vs iterations during LO-BCQ compared to other block quantization proposals}
    \label{fig:nmse_vs_iter}
\end{figure}

Figure \ref{fig:nmse_vs_iter} shows the empirical convergence of LO-BCQ across several block lengths and number of codebooks. Also, the MSE achieved by LO-BCQ is compared to baselines such as MXFP and VSQ. As shown, LO-BCQ converges to a lower MSE than the baselines. Further, we achieve better convergence for larger number of codebooks ($N_c$) and for a smaller block length ($L_b$), both of which increase the bitwidth of BCQ (see Eq \ref{eq:bitwidth_bcq}).


\subsection{Additional Accuracy Results}
%Table \ref{tab:lobcq_config} lists the various LOBCQ configurations and their corresponding bitwidths.
\begin{table}
\setlength{\tabcolsep}{4.75pt}
\begin{center}
\caption{\label{tab:lobcq_config} Various LO-BCQ configurations and their bitwidths.}
\begin{tabular}{|c||c|c|c|c||c|c||c|} 
\hline
 & \multicolumn{4}{|c||}{$L_b=8$} & \multicolumn{2}{|c||}{$L_b=4$} & $L_b=2$ \\
 \hline
 \backslashbox{$L_A$\kern-1em}{\kern-1em$N_c$} & 2 & 4 & 8 & 16 & 2 & 4 & 2 \\
 \hline
 64 & 4.25 & 4.375 & 4.5 & 4.625 & 4.375 & 4.625 & 4.625\\
 \hline
 32 & 4.375 & 4.5 & 4.625& 4.75 & 4.5 & 4.75 & 4.75 \\
 \hline
 16 & 4.625 & 4.75& 4.875 & 5 & 4.75 & 5 & 5 \\
 \hline
\end{tabular}
\end{center}
\end{table}

%\subsection{Perplexity achieved by various LO-BCQ configurations on Wikitext-103 dataset}

\begin{table} \centering
\begin{tabular}{|c||c|c|c|c||c|c||c|} 
\hline
 $L_b \rightarrow$& \multicolumn{4}{c||}{8} & \multicolumn{2}{c||}{4} & 2\\
 \hline
 \backslashbox{$L_A$\kern-1em}{\kern-1em$N_c$} & 2 & 4 & 8 & 16 & 2 & 4 & 2  \\
 %$N_c \rightarrow$ & 2 & 4 & 8 & 16 & 2 & 4 & 2 \\
 \hline
 \hline
 \multicolumn{8}{c}{GPT3-1.3B (FP32 PPL = 9.98)} \\ 
 \hline
 \hline
 64 & 10.40 & 10.23 & 10.17 & 10.15 &  10.28 & 10.18 & 10.19 \\
 \hline
 32 & 10.25 & 10.20 & 10.15 & 10.12 &  10.23 & 10.17 & 10.17 \\
 \hline
 16 & 10.22 & 10.16 & 10.10 & 10.09 &  10.21 & 10.14 & 10.16 \\
 \hline
  \hline
 \multicolumn{8}{c}{GPT3-8B (FP32 PPL = 7.38)} \\ 
 \hline
 \hline
 64 & 7.61 & 7.52 & 7.48 &  7.47 &  7.55 &  7.49 & 7.50 \\
 \hline
 32 & 7.52 & 7.50 & 7.46 &  7.45 &  7.52 &  7.48 & 7.48  \\
 \hline
 16 & 7.51 & 7.48 & 7.44 &  7.44 &  7.51 &  7.49 & 7.47  \\
 \hline
\end{tabular}
\caption{\label{tab:ppl_gpt3_abalation} Wikitext-103 perplexity across GPT3-1.3B and 8B models.}
\end{table}

\begin{table} \centering
\begin{tabular}{|c||c|c|c|c||} 
\hline
 $L_b \rightarrow$& \multicolumn{4}{c||}{8}\\
 \hline
 \backslashbox{$L_A$\kern-1em}{\kern-1em$N_c$} & 2 & 4 & 8 & 16 \\
 %$N_c \rightarrow$ & 2 & 4 & 8 & 16 & 2 & 4 & 2 \\
 \hline
 \hline
 \multicolumn{5}{|c|}{Llama2-7B (FP32 PPL = 5.06)} \\ 
 \hline
 \hline
 64 & 5.31 & 5.26 & 5.19 & 5.18  \\
 \hline
 32 & 5.23 & 5.25 & 5.18 & 5.15  \\
 \hline
 16 & 5.23 & 5.19 & 5.16 & 5.14  \\
 \hline
 \multicolumn{5}{|c|}{Nemotron4-15B (FP32 PPL = 5.87)} \\ 
 \hline
 \hline
 64  & 6.3 & 6.20 & 6.13 & 6.08  \\
 \hline
 32  & 6.24 & 6.12 & 6.07 & 6.03  \\
 \hline
 16  & 6.12 & 6.14 & 6.04 & 6.02  \\
 \hline
 \multicolumn{5}{|c|}{Nemotron4-340B (FP32 PPL = 3.48)} \\ 
 \hline
 \hline
 64 & 3.67 & 3.62 & 3.60 & 3.59 \\
 \hline
 32 & 3.63 & 3.61 & 3.59 & 3.56 \\
 \hline
 16 & 3.61 & 3.58 & 3.57 & 3.55 \\
 \hline
\end{tabular}
\caption{\label{tab:ppl_llama7B_nemo15B} Wikitext-103 perplexity compared to FP32 baseline in Llama2-7B and Nemotron4-15B, 340B models}
\end{table}

%\subsection{Perplexity achieved by various LO-BCQ configurations on MMLU dataset}


\begin{table} \centering
\begin{tabular}{|c||c|c|c|c||c|c|c|c|} 
\hline
 $L_b \rightarrow$& \multicolumn{4}{c||}{8} & \multicolumn{4}{c||}{8}\\
 \hline
 \backslashbox{$L_A$\kern-1em}{\kern-1em$N_c$} & 2 & 4 & 8 & 16 & 2 & 4 & 8 & 16  \\
 %$N_c \rightarrow$ & 2 & 4 & 8 & 16 & 2 & 4 & 2 \\
 \hline
 \hline
 \multicolumn{5}{|c|}{Llama2-7B (FP32 Accuracy = 45.8\%)} & \multicolumn{4}{|c|}{Llama2-70B (FP32 Accuracy = 69.12\%)} \\ 
 \hline
 \hline
 64 & 43.9 & 43.4 & 43.9 & 44.9 & 68.07 & 68.27 & 68.17 & 68.75 \\
 \hline
 32 & 44.5 & 43.8 & 44.9 & 44.5 & 68.37 & 68.51 & 68.35 & 68.27  \\
 \hline
 16 & 43.9 & 42.7 & 44.9 & 45 & 68.12 & 68.77 & 68.31 & 68.59  \\
 \hline
 \hline
 \multicolumn{5}{|c|}{GPT3-22B (FP32 Accuracy = 38.75\%)} & \multicolumn{4}{|c|}{Nemotron4-15B (FP32 Accuracy = 64.3\%)} \\ 
 \hline
 \hline
 64 & 36.71 & 38.85 & 38.13 & 38.92 & 63.17 & 62.36 & 63.72 & 64.09 \\
 \hline
 32 & 37.95 & 38.69 & 39.45 & 38.34 & 64.05 & 62.30 & 63.8 & 64.33  \\
 \hline
 16 & 38.88 & 38.80 & 38.31 & 38.92 & 63.22 & 63.51 & 63.93 & 64.43  \\
 \hline
\end{tabular}
\caption{\label{tab:mmlu_abalation} Accuracy on MMLU dataset across GPT3-22B, Llama2-7B, 70B and Nemotron4-15B models.}
\end{table}


%\subsection{Perplexity achieved by various LO-BCQ configurations on LM evaluation harness}

\begin{table} \centering
\begin{tabular}{|c||c|c|c|c||c|c|c|c|} 
\hline
 $L_b \rightarrow$& \multicolumn{4}{c||}{8} & \multicolumn{4}{c||}{8}\\
 \hline
 \backslashbox{$L_A$\kern-1em}{\kern-1em$N_c$} & 2 & 4 & 8 & 16 & 2 & 4 & 8 & 16  \\
 %$N_c \rightarrow$ & 2 & 4 & 8 & 16 & 2 & 4 & 2 \\
 \hline
 \hline
 \multicolumn{5}{|c|}{Race (FP32 Accuracy = 37.51\%)} & \multicolumn{4}{|c|}{Boolq (FP32 Accuracy = 64.62\%)} \\ 
 \hline
 \hline
 64 & 36.94 & 37.13 & 36.27 & 37.13 & 63.73 & 62.26 & 63.49 & 63.36 \\
 \hline
 32 & 37.03 & 36.36 & 36.08 & 37.03 & 62.54 & 63.51 & 63.49 & 63.55  \\
 \hline
 16 & 37.03 & 37.03 & 36.46 & 37.03 & 61.1 & 63.79 & 63.58 & 63.33  \\
 \hline
 \hline
 \multicolumn{5}{|c|}{Winogrande (FP32 Accuracy = 58.01\%)} & \multicolumn{4}{|c|}{Piqa (FP32 Accuracy = 74.21\%)} \\ 
 \hline
 \hline
 64 & 58.17 & 57.22 & 57.85 & 58.33 & 73.01 & 73.07 & 73.07 & 72.80 \\
 \hline
 32 & 59.12 & 58.09 & 57.85 & 58.41 & 73.01 & 73.94 & 72.74 & 73.18  \\
 \hline
 16 & 57.93 & 58.88 & 57.93 & 58.56 & 73.94 & 72.80 & 73.01 & 73.94  \\
 \hline
\end{tabular}
\caption{\label{tab:mmlu_abalation} Accuracy on LM evaluation harness tasks on GPT3-1.3B model.}
\end{table}

\begin{table} \centering
\begin{tabular}{|c||c|c|c|c||c|c|c|c|} 
\hline
 $L_b \rightarrow$& \multicolumn{4}{c||}{8} & \multicolumn{4}{c||}{8}\\
 \hline
 \backslashbox{$L_A$\kern-1em}{\kern-1em$N_c$} & 2 & 4 & 8 & 16 & 2 & 4 & 8 & 16  \\
 %$N_c \rightarrow$ & 2 & 4 & 8 & 16 & 2 & 4 & 2 \\
 \hline
 \hline
 \multicolumn{5}{|c|}{Race (FP32 Accuracy = 41.34\%)} & \multicolumn{4}{|c|}{Boolq (FP32 Accuracy = 68.32\%)} \\ 
 \hline
 \hline
 64 & 40.48 & 40.10 & 39.43 & 39.90 & 69.20 & 68.41 & 69.45 & 68.56 \\
 \hline
 32 & 39.52 & 39.52 & 40.77 & 39.62 & 68.32 & 67.43 & 68.17 & 69.30  \\
 \hline
 16 & 39.81 & 39.71 & 39.90 & 40.38 & 68.10 & 66.33 & 69.51 & 69.42  \\
 \hline
 \hline
 \multicolumn{5}{|c|}{Winogrande (FP32 Accuracy = 67.88\%)} & \multicolumn{4}{|c|}{Piqa (FP32 Accuracy = 78.78\%)} \\ 
 \hline
 \hline
 64 & 66.85 & 66.61 & 67.72 & 67.88 & 77.31 & 77.42 & 77.75 & 77.64 \\
 \hline
 32 & 67.25 & 67.72 & 67.72 & 67.00 & 77.31 & 77.04 & 77.80 & 77.37  \\
 \hline
 16 & 68.11 & 68.90 & 67.88 & 67.48 & 77.37 & 78.13 & 78.13 & 77.69  \\
 \hline
\end{tabular}
\caption{\label{tab:mmlu_abalation} Accuracy on LM evaluation harness tasks on GPT3-8B model.}
\end{table}

\begin{table} \centering
\begin{tabular}{|c||c|c|c|c||c|c|c|c|} 
\hline
 $L_b \rightarrow$& \multicolumn{4}{c||}{8} & \multicolumn{4}{c||}{8}\\
 \hline
 \backslashbox{$L_A$\kern-1em}{\kern-1em$N_c$} & 2 & 4 & 8 & 16 & 2 & 4 & 8 & 16  \\
 %$N_c \rightarrow$ & 2 & 4 & 8 & 16 & 2 & 4 & 2 \\
 \hline
 \hline
 \multicolumn{5}{|c|}{Race (FP32 Accuracy = 40.67\%)} & \multicolumn{4}{|c|}{Boolq (FP32 Accuracy = 76.54\%)} \\ 
 \hline
 \hline
 64 & 40.48 & 40.10 & 39.43 & 39.90 & 75.41 & 75.11 & 77.09 & 75.66 \\
 \hline
 32 & 39.52 & 39.52 & 40.77 & 39.62 & 76.02 & 76.02 & 75.96 & 75.35  \\
 \hline
 16 & 39.81 & 39.71 & 39.90 & 40.38 & 75.05 & 73.82 & 75.72 & 76.09  \\
 \hline
 \hline
 \multicolumn{5}{|c|}{Winogrande (FP32 Accuracy = 70.64\%)} & \multicolumn{4}{|c|}{Piqa (FP32 Accuracy = 79.16\%)} \\ 
 \hline
 \hline
 64 & 69.14 & 70.17 & 70.17 & 70.56 & 78.24 & 79.00 & 78.62 & 78.73 \\
 \hline
 32 & 70.96 & 69.69 & 71.27 & 69.30 & 78.56 & 79.49 & 79.16 & 78.89  \\
 \hline
 16 & 71.03 & 69.53 & 69.69 & 70.40 & 78.13 & 79.16 & 79.00 & 79.00  \\
 \hline
\end{tabular}
\caption{\label{tab:mmlu_abalation} Accuracy on LM evaluation harness tasks on GPT3-22B model.}
\end{table}

\begin{table} \centering
\begin{tabular}{|c||c|c|c|c||c|c|c|c|} 
\hline
 $L_b \rightarrow$& \multicolumn{4}{c||}{8} & \multicolumn{4}{c||}{8}\\
 \hline
 \backslashbox{$L_A$\kern-1em}{\kern-1em$N_c$} & 2 & 4 & 8 & 16 & 2 & 4 & 8 & 16  \\
 %$N_c \rightarrow$ & 2 & 4 & 8 & 16 & 2 & 4 & 2 \\
 \hline
 \hline
 \multicolumn{5}{|c|}{Race (FP32 Accuracy = 44.4\%)} & \multicolumn{4}{|c|}{Boolq (FP32 Accuracy = 79.29\%)} \\ 
 \hline
 \hline
 64 & 42.49 & 42.51 & 42.58 & 43.45 & 77.58 & 77.37 & 77.43 & 78.1 \\
 \hline
 32 & 43.35 & 42.49 & 43.64 & 43.73 & 77.86 & 75.32 & 77.28 & 77.86  \\
 \hline
 16 & 44.21 & 44.21 & 43.64 & 42.97 & 78.65 & 77 & 76.94 & 77.98  \\
 \hline
 \hline
 \multicolumn{5}{|c|}{Winogrande (FP32 Accuracy = 69.38\%)} & \multicolumn{4}{|c|}{Piqa (FP32 Accuracy = 78.07\%)} \\ 
 \hline
 \hline
 64 & 68.9 & 68.43 & 69.77 & 68.19 & 77.09 & 76.82 & 77.09 & 77.86 \\
 \hline
 32 & 69.38 & 68.51 & 68.82 & 68.90 & 78.07 & 76.71 & 78.07 & 77.86  \\
 \hline
 16 & 69.53 & 67.09 & 69.38 & 68.90 & 77.37 & 77.8 & 77.91 & 77.69  \\
 \hline
\end{tabular}
\caption{\label{tab:mmlu_abalation} Accuracy on LM evaluation harness tasks on Llama2-7B model.}
\end{table}

\begin{table} \centering
\begin{tabular}{|c||c|c|c|c||c|c|c|c|} 
\hline
 $L_b \rightarrow$& \multicolumn{4}{c||}{8} & \multicolumn{4}{c||}{8}\\
 \hline
 \backslashbox{$L_A$\kern-1em}{\kern-1em$N_c$} & 2 & 4 & 8 & 16 & 2 & 4 & 8 & 16  \\
 %$N_c \rightarrow$ & 2 & 4 & 8 & 16 & 2 & 4 & 2 \\
 \hline
 \hline
 \multicolumn{5}{|c|}{Race (FP32 Accuracy = 48.8\%)} & \multicolumn{4}{|c|}{Boolq (FP32 Accuracy = 85.23\%)} \\ 
 \hline
 \hline
 64 & 49.00 & 49.00 & 49.28 & 48.71 & 82.82 & 84.28 & 84.03 & 84.25 \\
 \hline
 32 & 49.57 & 48.52 & 48.33 & 49.28 & 83.85 & 84.46 & 84.31 & 84.93  \\
 \hline
 16 & 49.85 & 49.09 & 49.28 & 48.99 & 85.11 & 84.46 & 84.61 & 83.94  \\
 \hline
 \hline
 \multicolumn{5}{|c|}{Winogrande (FP32 Accuracy = 79.95\%)} & \multicolumn{4}{|c|}{Piqa (FP32 Accuracy = 81.56\%)} \\ 
 \hline
 \hline
 64 & 78.77 & 78.45 & 78.37 & 79.16 & 81.45 & 80.69 & 81.45 & 81.5 \\
 \hline
 32 & 78.45 & 79.01 & 78.69 & 80.66 & 81.56 & 80.58 & 81.18 & 81.34  \\
 \hline
 16 & 79.95 & 79.56 & 79.79 & 79.72 & 81.28 & 81.66 & 81.28 & 80.96  \\
 \hline
\end{tabular}
\caption{\label{tab:mmlu_abalation} Accuracy on LM evaluation harness tasks on Llama2-70B model.}
\end{table}

%\section{MSE Studies}
%\textcolor{red}{TODO}


\subsection{Number Formats and Quantization Method}
\label{subsec:numFormats_quantMethod}
\subsubsection{Integer Format}
An $n$-bit signed integer (INT) is typically represented with a 2s-complement format \citep{yao2022zeroquant,xiao2023smoothquant,dai2021vsq}, where the most significant bit denotes the sign.

\subsubsection{Floating Point Format}
An $n$-bit signed floating point (FP) number $x$ comprises of a 1-bit sign ($x_{\mathrm{sign}}$), $B_m$-bit mantissa ($x_{\mathrm{mant}}$) and $B_e$-bit exponent ($x_{\mathrm{exp}}$) such that $B_m+B_e=n-1$. The associated constant exponent bias ($E_{\mathrm{bias}}$) is computed as $(2^{{B_e}-1}-1)$. We denote this format as $E_{B_e}M_{B_m}$.  

\subsubsection{Quantization Scheme}
\label{subsec:quant_method}
A quantization scheme dictates how a given unquantized tensor is converted to its quantized representation. We consider FP formats for the purpose of illustration. Given an unquantized tensor $\bm{X}$ and an FP format $E_{B_e}M_{B_m}$, we first, we compute the quantization scale factor $s_X$ that maps the maximum absolute value of $\bm{X}$ to the maximum quantization level of the $E_{B_e}M_{B_m}$ format as follows:
\begin{align}
\label{eq:sf}
    s_X = \frac{\mathrm{max}(|\bm{X}|)}{\mathrm{max}(E_{B_e}M_{B_m})}
\end{align}
In the above equation, $|\cdot|$ denotes the absolute value function.

Next, we scale $\bm{X}$ by $s_X$ and quantize it to $\hat{\bm{X}}$ by rounding it to the nearest quantization level of $E_{B_e}M_{B_m}$ as:

\begin{align}
\label{eq:tensor_quant}
    \hat{\bm{X}} = \text{round-to-nearest}\left(\frac{\bm{X}}{s_X}, E_{B_e}M_{B_m}\right)
\end{align}

We perform dynamic max-scaled quantization \citep{wu2020integer}, where the scale factor $s$ for activations is dynamically computed during runtime.

\subsection{Vector Scaled Quantization}
\begin{wrapfigure}{r}{0.35\linewidth}
  \centering
  \includegraphics[width=\linewidth]{sections/figures/vsquant.jpg}
  \caption{\small Vectorwise decomposition for per-vector scaled quantization (VSQ \citep{dai2021vsq}).}
  \label{fig:vsquant}
\end{wrapfigure}
During VSQ \citep{dai2021vsq}, the operand tensors are decomposed into 1D vectors in a hardware friendly manner as shown in Figure \ref{fig:vsquant}. Since the decomposed tensors are used as operands in matrix multiplications during inference, it is beneficial to perform this decomposition along the reduction dimension of the multiplication. The vectorwise quantization is performed similar to tensorwise quantization described in Equations \ref{eq:sf} and \ref{eq:tensor_quant}, where a scale factor $s_v$ is required for each vector $\bm{v}$ that maps the maximum absolute value of that vector to the maximum quantization level. While smaller vector lengths can lead to larger accuracy gains, the associated memory and computational overheads due to the per-vector scale factors increases. To alleviate these overheads, VSQ \citep{dai2021vsq} proposed a second level quantization of the per-vector scale factors to unsigned integers, while MX \citep{rouhani2023shared} quantizes them to integer powers of 2 (denoted as $2^{INT}$).

\subsubsection{MX Format}
The MX format proposed in \citep{rouhani2023microscaling} introduces the concept of sub-block shifting. For every two scalar elements of $b$-bits each, there is a shared exponent bit. The value of this exponent bit is determined through an empirical analysis that targets minimizing quantization MSE. We note that the FP format $E_{1}M_{b}$ is strictly better than MX from an accuracy perspective since it allocates a dedicated exponent bit to each scalar as opposed to sharing it across two scalars. Therefore, we conservatively bound the accuracy of a $b+2$-bit signed MX format with that of a $E_{1}M_{b}$ format in our comparisons. For instance, we use E1M2 format as a proxy for MX4.

\begin{figure}
    \centering
    \includegraphics[width=1\linewidth]{sections//figures/BlockFormats.pdf}
    \caption{\small Comparing LO-BCQ to MX format.}
    \label{fig:block_formats}
\end{figure}

Figure \ref{fig:block_formats} compares our $4$-bit LO-BCQ block format to MX \citep{rouhani2023microscaling}. As shown, both LO-BCQ and MX decompose a given operand tensor into block arrays and each block array into blocks. Similar to MX, we find that per-block quantization ($L_b < L_A$) leads to better accuracy due to increased flexibility. While MX achieves this through per-block $1$-bit micro-scales, we associate a dedicated codebook to each block through a per-block codebook selector. Further, MX quantizes the per-block array scale-factor to E8M0 format without per-tensor scaling. In contrast during LO-BCQ, we find that per-tensor scaling combined with quantization of per-block array scale-factor to E4M3 format results in superior inference accuracy across models. 


\end{document}