\section{Conclusion}
Our analysis highlights TabPFN v2's potential as a pre-trained foundation model for tabular data. By leveraging randomized feature tokens, it effectively handles dataset heterogeneity and excels in in-context learning, outperforming other models on small- to medium-scale datasets. Additionally, its ability to create nearly separable embeddings suggests its potential role as a feature encoder.
To address TabPFN v2's limitations on high-dimensional, large-scale, and many-category datasets, we propose post-hoc divide-and-conquer strategies that improve scalability, expanding applicability while preserving its core strengths.