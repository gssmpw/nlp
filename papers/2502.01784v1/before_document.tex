\usepackage{graphicx}
\usepackage[percent]{overpic}%added by Zhengtong
\usepackage{subfig}
\usepackage{amsmath}
\let\proof\relax \let\endproof\relax % to avoid confliction of amsthm with the packages such as ieeeconf
\usepackage{amsthm} % definte the proof environment: block letters + italic letters
\usepackage{amssymb}
\usepackage[font=footnotesize]{caption} % set the captain font size to 8 (i.e. footnotesize)
\usepackage{subfig} % uses subfloats within a single float MUST after the package {caption}!!
\usepackage[noadjust]{cite} % alphabetize grouped citations automatically [3, 1, 2] to [1, 2, 3]
\usepackage{color}
\usepackage{algorithm} % options: boxed [section]
\usepackage{algpseudocode} % for algorithm
\usepackage{enumerate}
\usepackage[hidelinks,colorlinks=false]{hyperref}
%\usepackage{setspace} \setstretch{1.5}
\usepackage[left=0.75in, right=0.75in, top=0.75in, bottom=0.75in]{geometry}
\usepackage{authblk} % for authors' affiliation \affil{xxx}
\usepackage{arydshln} % for dashline in table or matrix


\usepackage{bm}
\usepackage{tikz}
\usetikzlibrary{calc} % for calculation functions in Tikz let, in commands in Tikz
\usetikzlibrary{shapes} % for block diagram
\usetikzlibrary{chains}
\usetikzlibrary{fit}
\usetikzlibrary{arrows}
\usetikzlibrary{decorations.text} % text along path
\usetikzlibrary{decorations.markings}
\usetikzlibrary{decorations.pathmorphing} % for zigzag arrow
\usetikzlibrary{shadows}
\usetikzlibrary{patterns}
\usetikzlibrary{matrix}
\usepackage{pgfplots}
\usepackage[europeanresistors]{circuitikz}
\usepackage[outline]{contour} % white surround text
\contourlength{1.5pt}

\newtheorem{theorem}{Theorem}
\newtheorem{lemma}{Lemma}
\newtheorem{assumption}{Assumption}
\newtheorem{remark}{Remark}
\newtheorem{proposition}{Proposition}
\newtheorem{corollary}{Corollary}
\newtheorem{example}{Example}
\newtheorem{definition}{Definition}
\newtheorem{problem}{Problem}

\newcommand{\tr}{\mathrm{tr}}
\newcommand{\myvec}{\mathrm{vec}}
\newcommand{\Null}{\mathrm{Null}}
\newcommand{\Range}{\mathrm{Col}}
\newcommand{\one}{\mathbf{1}}
\newcommand{\rank}{\mathrm{rank}}
\newcommand{\myspan}{\mathrm{span}}
\newcommand{\mydiag}{\mathrm{diag}}
\newcommand{\diff}{\mathrm{d}}
\newcommand{\blkdiag}{\mathrm{blkdiag}}
\newcommand{\sgn}{\mathrm{sgn}}
\newcommand{\T}{\mathrm{T}}
\newcommand{\myqed}{\hfill$\blacksquare$}
\newcommand{\ep}{\varepsilon}
\newcommand{\sig}{\mathrm{sig}_a}
%\newcommand{\sigep_}[1]{\sig(\ep_{#1})}
\newcommand{\R}{\mathbb{R}}
%\newcommand{\G}{\mathcal{G}}
\newcommand{\E}{\mathcal{E}}
\newcommand{\V}{\mathcal{V}}
\newcommand{\N}{\mathcal{N}}
\newcommand{\M}{\mathcal{M}}
\newcommand{\A}{\mathcal{A}}
\newcommand{\D}{\mathcal{D}}
\renewcommand{\L}{\mathcal{B}}
\renewcommand{\H}{\mathcal{H}}
\renewcommand{\S}{\Omega}
\newcommand{\xe}{x_{\text{e}}}
%\newcommand{\Null}[1]{\mathrm{Null}\left(#1\right)}
\newcommand{\sk}[1]{\left[#1\right]_\times} % skew symmetric operator
\newcommand{\dia}[1]{\mathrm{diag}\left(#1\right)} % block diagnal matrix
%\renewcommand{\span}[1]{\mathrm{span}\left\{#1\right\}} % ERROR when redefine \span
\newcommand{\Db}{D_{\bar{b}}}
\newcommand{\smin}{\sigma_{\min}}
\newcommand{\smax}{\sigma_{\max}}
\newcommand{\lmin}{\lambda_{\min}}
\newcommand{\lmax}{\lambda_{\max}}
\newcommand{\emax}{\ener_{\max}}
\newcommand{\emin}{\ener_{\min}}
\renewcommand{\Re}{\mathrm{Re}}
%\newcommand{\C}{\mathbb{C}}
\newcommand{\e}{\mathrm{e}}
\newcommand{\ener}{\varepsilon}
\newcommand{\edgeN}{n_{e}}
\newcommand{\din}{{d^{\mathrm{in}}}}
\newcommand{\dout}{{d^{\mathrm{out}}}}
\graphicspath{{figures/}}
