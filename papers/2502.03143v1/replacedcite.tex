\section{Related works}
\subsection{Machine learning in student performance prediction}
Research on predicting student performance using machine learning is a hot topic that continues to attract a wide range of researchers ____. For instance, Bujang et al. ____ conducted a comprehensive analysis of machine learning techniques and predicted the final grades of first semester based on the model achieving best accuracy. Similarly, Ko and Leu ____ applied supervised and unsupervised machine learning techniques to identify the attributes of successful learners in a computer course, discovering that the Naïve Bayes classifier was the most effective for predicting student performance, and revealed the importance of weekly progress and self-efficacy beliefs in influencing final outcomes. Xu et al. ____ gathered data describing student learning behavior and used it to test its predictability over 1/4, 1/2, and 3/4 semesters via multiple regression models, which experimentally proved to be more consistent and reliable as the course progressed. Said et al. ____ proposed a ML-based method to predict undergraduate academic performance and dropout risk, focusing on early identification of students needing attention from the first semester due to academic weaknesses. It identifies key factors such as demographic, pre-admission, and academic, that influence academic performance, supporting educational decision-making. Accordingly, feature selection is an important research subject in machine learning and data mining, and the prediction performance of the model can be effectively improved by appropriate feature selection methods. 

To thoroughly investigate the impact of feature selection on the performance of different prediction models, Zhou et al. ____ conducted a series of experiments, and the empirical results showed that the feature selection methods played a significant role in predicting student performance. Building on this, Peng et al. ____ employed Random forest to pinpoint the key information and communication technology related factors influencing reading performance in blended learning, highlighting AI's role in educational technology optimization. Talebi et al. ____ performed feature selection by eliminating redundant features using Recursive Feature Elimination (RFE), and selecting the most predictive features based on feature importance scores from machine learning models, thereby optimizing the feature set. Despite these advancements, current research on predicting student performance with machine learning has largely failed to extended the application of these model-driven insights to real-world educational settings. As a result, bridging the gap between research findings and their practical implementation in educational contexts remains a pressing issue. 

\subsection{Tiered instruction}
The tiered instruction ____ refers to the design of multi-level learning content based on learners' existing skills and knowledge, aiming to support each student in engaging in effective learning at a level appropriate to their abilities, thereby optimizing their learning experience and achievement ____. This innovative educational method is an innovative educational method commonly used in today's educational environment, has drawn significant attention. For instance, Pullen et al. ____ provided additional instruction to at-risk students through the design and use of tiered interventions in the general education classroom and assists students at risk for dyslexia. Additionally, Freeman-Green et al. ____ effectively used the tiered approach at the students' secondary school level to enhance students' skills and abilities in a targeted manner according to the teaching requirements, and achieved excellent teaching results. Magableh et al. ____ investigated the effects of differentiated instruction on enhancing the overall academic achievement of Jordanian students. By employing tiered assignments and group-based teaching that adjusted instructional content according to students' abilities and interests, the study demonstrated a significant improvement in student performance. Vojinovic et al. ____ proposed a tiered lab programming model that helped students gradually overcome learning challenges through tiered assignments and deliberately set learning obstacles. The study demonstrated that this approach significantly enhanced student motivation and exam pass rates, with particularly notable results among previously unmotivated students. 

However, as one of the most effective classroom teaching methods, the tiered instruction has not been well integrated with modern technology to demonstrate better performance. In addition, although many existing machine learning methods have shown good performance in educational data mining, the predictive outcomes of the models have not been effectively utilized in actual teaching scenarios. Based on the existing researches, we combine the analysis results of machine learning with actual teaching scenarios by introducing tiered teaching method and propose a specific implementation plan, which provides a novel insight on the application of machine learning in modern education. To the best of our knowledge, this is the first work to combine machine learning methods with tiered instruction.