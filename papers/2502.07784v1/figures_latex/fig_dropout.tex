\begin{figure}
    \centering
    \setlength{\tabcolsep}{1pt}
    \renewcommand{\arraystretch}{0}
    \newcommand{\imgsize}{.10\textwidth}
    \newcommand{\imgsizehalf}{.05\textwidth}

    \newcommand{\row}[2]{%
        & \includegraphics[width=\imgsizehalf]{images/dropout/texture_#1.png}%
        & \multirow{2}{*}[#2]{\includegraphics[width=\imgsize]{example-image-a}}%
        & \multirow{2}{*}[#2]{\includegraphics[width=\imgsize]{images/dropout/ours_#1_drop_E_masked.png}}%
        & \multirow{2}{*}[#2]{\includegraphics[width=\imgsize]{example-image-a}}%
        & \multirow{2}{*}[#2]{\includegraphics[width=\imgsize]{images/dropout/ours_#1_masked.png}}%
        & \multirow{2}{*}[#2]{\includegraphics[width=\imgsize]{example-image-a}}%
        & \multirow{2}{*}[#2]{\includegraphics[width=\imgsize]{images/dropout/ours_#1_drop_E.png}}%
        & \multirow{2}{*}[#2]{\includegraphics[width=\imgsize]{example-image-a}}%
        & \multirow{2}{*}[#2]{\includegraphics[width=\imgsize]{images/dropout/ours_#1.png}}\\

        & \includegraphics[width=\imgsizehalf]{images/dropout/mask_#1.png} &&&& \\
    }

	\resizebox{.99\linewidth}{!}{%
    \begin{tabular}{c cS{0.2ex} HcHc | HcHc}


        & & \multicolumn{4}{c}{target: $\target\cdot\mask$} & \multicolumn{4}{c}{target: $\target$}\\

        \cmidrule[1pt](lr){3-6} \cmidrule[1pt](lr){7-10}
        & Src. & $\xx \smallsetminus \{ \normal \}$ & $\xx \smallsetminus \{ \irra \}$ & $\xx \smallsetminus \{ \normal,\irra \}$ & $\xx$ & $\xx \smallsetminus \{ \normal \}$ & $\xx \smallsetminus \{ \irra \}$ & $\xx \smallsetminus \{ \normal,\irra \}$ & $\xx$ \\[.7ex]

        \row{e3a0ce32-image_007_B_ms_paving_stones_092__grass_001}{6.5ex}\\
        \row{4ac74304-image_043_A_acg_wood_014}{6.6ex}\\[.2ex]

         & \multicolumn{1}{r}{$\xx=$} &  & $\{\target{\cdot}\mask, \normal, \mask\}$ & & $\{\target{\cdot}\mask, \normal, \irra, \mask\}$ & & $\{\target, \normal, \mask\}$ & & $\{\target, \normal, \irra, \mask\}$\\[.7ex]
    \end{tabular}}

    \caption{Ablation study on lighting cues. When deprived of lighting cues by masking out the target image \target{} (\ie, providing $\target\!\cdot\!\mask$ as target) and removing the irradiance map \irra{}, our method produces results with flat, implausible shading (leftmost). Reintroducing either the irradiance \irra{} (second column) or the masked region (third column) restores the light effects. Providing all lighting cues (\method{}) provides the best result (rightmost).
    For clarity, we report the conditioning used below.
    }
    \label{fig:dropout}
\end{figure}