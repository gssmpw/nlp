



    
    

\begin{figure*}
    \centering
    \setlength{\tabcolsep}{0pt}
    \renewcommand{\arraystretch}{0}
    \newcommand{\imgsize}{.14\textwidth}
    \newcommand{\imgsizehalf}{.07\textwidth}
    \newcommand{\shift}{9.25ex}
    \newcommand{\prompt}{\cellcolor{gray!40}}
    
    \newcommand{\img}[1]{%
        \multirow{2}{*}[\shift]{\includegraphics[width=\imgsize,height=\imgsize]{#1}}%
    }
    \newcommand{\row}[2]{%
        &\includegraphics[width=\imgsizehalf,height=\imgsizehalf]{images/baselines/texture_#1.png}%
        & \img{images/baselines/sd2_#1.png}%
        & \multirow{2}{*}[\shift]{\includegraphics[width=\imgsize,height=\imgsize]{images/baselines/sdxl_inpaint_#1.png}}%
        & \multirow{2}{*}[\shift]{\includegraphics[width=\imgsize,height=\imgsize]{images/baselines/flux_#1.png}}%
        & \multirow{2}{*}[\shift]{\includegraphics[width=\imgsize,height=\imgsize]{images/baselines/bld_#1.png}}%
        & \multirow{2}{*}[\shift]{\includegraphics[width=\imgsize,height=\imgsize]{images/baselines/zest_#1.png}}%
        & \multirow{2}{*}[\shift]{\includegraphics[width=\imgsize,height=\imgsize]{images/baselines/mocka_v3_a100_512px_#1.png}} \\
        
        \multirow{2}{*}[18.5ex]{\includegraphics[width=\imgsize,height=\imgsize]{images/baselines/image_#1.png}} & \includegraphics[width=\imgsizehalf,height=\imgsizehalf]{images/baselines/mask_#1.png}  \\[0.1ex]
        
        && \multicolumn{4}{c}{\footnotesize \prompt{} \texttt{``#2''}} \\
        \\
    }
    
	\resizebox{.99\linewidth}{!}{%
    \begin{tabular}{cS{.5ex}c S{.5ex}|S{.5ex} c S{.5ex} c S{.5ex} c S{.5ex} c S{.5ex} c S{.5ex} c}
        \multicolumn{1}{c}{Image} & Source & SD v2.1 & SD-XL inpaint & FLUX.1 Fill & Blended LD &  ZeST &\textbf{ours} \\[.7ex]
        \row{296cada7-image_019_A_st_plaster_028}{A distressed gray concrete wall texture with white stains.}\\[.4ex]
        \row{67a44738-image_036_A_th_brick_wall_005}{Aged brick wall with varied earth tones and faded mortar.}\\[.4ex]
        \row{49be7783-image_021_C_acg_leather_005}{Dark marled leather texture with visible grain.}\\[.4ex]
        \row{36c7a88c-image_042_A_cgbc_granite_005_small}{Dark speckled stone texture with irregular black flecks.}\\[.4ex]
        \row{22f65695-image_023_D_acg_paving_stones_009}{Old grey stone brick texture with moss on the edges.}\\[.4ex]
        \row{4c2dfc3f-image_037_A_acg_bricks_073_b}{Red brick wall texture with moss growth.}\\[.4ex]
        \row{d61d19eb-image_047_B_acg_paving_stones_048}{Grey brick tile texture with light mottled specks.}\\[.4ex]
        \row{8035c3ed-image_031_C_js_bricks_clay_001}{Weathered reddish-brown brick wall texture.}
    \end{tabular}}
    \vspace{-3mm}
    \caption{Comparison to baselines. All methods use the input image and mask as conditioning: SD-v2.1, SD-XL, and ZeST use it to compose the final image at the latent space, while SD-XL, FLUX and ours use it as input. ZeST uses a depth map via its parallel ControlNet branch. Instead, we rely on both irradiance and normal maps at the input level of the UNet. We can see that text-based methods cannot precisely reproduce the desired appearance, as text descriptions tend to be too vague. Our approach better preserves the target image lighting while maintaining the transferred material appearance.}
    \label{fig:baselines}
\end{figure*}

\begin{figure*}
    \centering
	\footnotesize
    \setlength{\tabcolsep}{0pt}
    \renewcommand{\arraystretch}{0}
    \newcommand{\imgsize}{.09\textwidth}
    \newcommand{\imgsizehalf}{.045\textwidth}

    \newcommand{\rowvertical}[7]{%
        &\includegraphics[width=\imgsizehalf]{images/hue/irradiance_#1.png}%
        & \multirow{2}{*}[#2]{\includegraphics[width=\imgsize]{images/hue/image_#1.png}}%
        & \includegraphics[width=\imgsizehalf]{images/hue/texture_#1_#3.png}%
        && \multirow{2}{*}[#2]{\includegraphics[width=\imgsize]{images/hue/ours_#1_#3.png}}%
        & \multirow{2}{*}[#2]{\includegraphics[width=\imgsize]{images/hue/ours_#1_#4.png}}%
        & \multirow{2}{*}[#2]{\includegraphics[width=\imgsize]{images/hue/ours_#1_#5.png}}%
        & \multirow{2}{*}[#2]{\includegraphics[width=\imgsize]{images/hue/ours_#1_#6.png}}%
        & \multirow{2}{*}[#2]{\includegraphics[width=\imgsize]{images/hue/ours_#1_#7.png}}%
        \\
        
        & \includegraphics[width=\imgsizehalf]{images/hue/normals_#1.png}%
        && \includegraphics[width=\imgsizehalf]{images/hue/mask_#1.png}%
        \\
    }

    \newcommand{\rowhorizontal}[5]{%
        &\includegraphics[width=\imgsizehalf]{images/hue/irradiance_#1.png}%
        & \multirow{2}{*}[#2]{\includegraphics[width=\imgsize]{images/hue/image_#1.png}}%
        & \includegraphics[width=\imgsizehalf]{images/hue/texture_#1_#3.png}%
        && \multirow{2}{*}[#2]{\includegraphics[width=\imgsize]{images/hue/ours_#1_#3.png}}%
        & \multirow{2}{*}[#2]{\includegraphics[width=\imgsize]{images/hue/ours_#1_#4.png}}%
        & \multirow{2}{*}[#2]{\includegraphics[width=\imgsize]{images/hue/ours_#1_#5.png}}%
        \\
        
        & \includegraphics[width=\imgsizehalf]{images/hue/normals_#1.png}%
        && \includegraphics[width=\imgsizehalf]{images/hue/mask_#1.png}%
        \\
    }

    \newcommand{\cbull}[2]{\textcolor[HTML]{#2}{\raisebox{-2ex}{\scalebox{#1}{$\bullet$}}}}
    
	\resizebox{.99\linewidth}{!}{%
    \begin{tabular}{c ccS{0.5ex}c C{0.5ex} ccccc}

        &\multicolumn{2}{c}{Conditions} & Source &&\cbull{1.5}{804916} & \cbull{1.5}{388016} & \cbull{1.5}{1f1680} & \cbull{1.5}{5e1680} & \cbull{1.5}{801623} \\
        \rowvertical{36515637-image_010_A_st_camouflage_021}{11.5ex}{h000_804916}{h036_388016}{h108_1f1680}{h126_5e1680}{h162_801623}
        
        
        && &  && \cbull{1.5}{b8bf9e} & \cbull{1.5}{9dc0ab} & \cbull{1.5}{9eabc0} & \cbull{1.5}{b89ebf} & \cbull{1.5}{c0a09f} \\
        \rowvertical{36c7a88c-image_042_A_acg_terrazzo_009}{11ex}{h018_b8bf9e}{h054_9dc0ab}{h090_9eabc0}{h126_b89ebf}{h162_c0a09f}
        \midrule
    \end{tabular}}
    
    \newcommand{\cbullo}[2]{\textcolor[HTML]{#2}{\raisebox{-.7ex}{\scalebox{#1}{$\bullet$}}}}
    
    \resizebox{.99\linewidth}{!}{%
    \tiny
    \begin{tabular}{c ccS{0.5ex}c C{0.5ex} ccc}
        &\multicolumn{2}{c}{Conditions} & Source && \cbullo{1.5}{aa9982} & \cbullo{1.5}{aa8383} & \cbullo{1.5}{a182a9} \\
        \rowhorizontal{tdbb975b3-image_064_B_acg_bricks_035}{7.1ex}{h000_aa9982}{h162_aa8383}{h126_a182a9}\\[1ex]

        &&&&& \cbullo{1.5}{c0bed8} & \cbullo{1.5}{c7d8be} & \cbullo{1.5}{d8bec2} \\
        \rowhorizontal{67a44738-image_036_B_st_marble_040}{7.2ex}{h126_c0bed8}{h054_c7d8be}{h000_d8bec2}\\[1ex]
        
        &&&&& \cbullo{1.5}{9bb9bd} & \cbullo{1.5}{bdb39b} & \cbullo{1.5}{b99bbd} \\
        \rowhorizontal{t6bc7c953-image_063_B_acg_leather_035_a}{7.2ex}{h072_9bb9bd}{h000_bdb39b}{h126_b99bbd}\\[1ex]
        
        &&&&& \cbullo{1.5}{4e373b} & \cbullo{1.5}{37374e} & \cbullo{1.5}{374e3c} \\
        \rowhorizontal{7b330187-image_035_A_acg_leather_001}{8.1ex}{h162_4e373b}{h108_37374e}{h054_374e3c}
    \end{tabular}}
    
    \caption{Adherence to texture conditioning. We provide different hue variations of the exemplar material as input and observe that our method correctly adapts, maintaining realism in the generated image.}
    \label{fig:variations}
\end{figure*}