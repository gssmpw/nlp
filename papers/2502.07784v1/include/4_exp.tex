

%\subsection{Seizure Detection}


\begin{table*}[h]
\renewcommand{\arraystretch}{1.2}
\caption{Summary of deep learning frameworks for seizure detection}
\label{tab:seizures}
\footnotesize
\begin{tabular}{p{0.4cm}p{2.8cm}p{2cm}p{1.5cm}p{1.9cm}p{1.9cm}p{0.8cm}p{1.8cm}p{1.5cm}}
\hline
\textbf{Ref.} & \textbf{Preprocessing} & \textbf{Feature} & \textbf{Backbone} & \textbf{Training} & \textbf{Dataset} & \textbf{Task} & \textbf{Partitioning} & \textbf{Accuracy} \\
\hline
~\cite{Seizure1} & Image generation & 2D Image & 2D-CNN & supervised & Bern-Barcelona,\newline private  & binary & mixed-subject & 100\% \\
~\cite{zhou2018epileptic} & FFT & Frequency-domain features & 2D-CNN & supervised & Freiburg,\newline CHB-MIT & binary \newline 
3-class & subject-specific & 98.2\%-99.4\% \newline
95.3\% \\
~\cite{Seizure3} & Filtering,Downsampling & Raw & 2D-CNN & supervised & private & binary & cross-subject & AUC=0.94 \\
~\cite{Seizure4} & FSST,WSST & Time-Frequency matrix & 2D-CNN & supervised & Bern-Barcelona & binary & mixed-subject & 99.94\% \\
~\cite{Seizure5} & Filtering,EMD,FWT,FT & Raw,IMFs,\newline Wavelet Coefficients,\newline Module Values & 2D-CNN & supervised & Bern-Barcelona,\newline private  & binary & mixed-subject & 98.9\% \\
~\cite{Seizure6} & Z-norm,STFT & 2D Spectrograms & 2D-CNN & supervised & Bern-Barcelona,\newline private  & binary & mixed-subject & 91.8\% \\
~\cite{Seizure7} & Filtering & 2D Images & 2D-CNN & supervised & Bonn & binary & mixed-subject & 99.6\% \\
~\cite{turk2019epilepsy} & CWT & 2D Scalograms & 2D-CNN & supervised & Bonn & binary \newline 3-class \newline
5-class & mixed-subject & 93.60\% \\
~\cite{Seizure9} & Windowing & Raw Segments & 2D-CNN & supervised & CHB-MIT & binary & mixed-subject & 99.07\% \\
~\cite{Seizure10} & Image construction & intensity Image & 2D-CNN & supervised & CHB-MIT & binary & mixed-subject & 99.48\% \\
~\cite{hossain2019applying} & Windowing,Normalization & Raw & 2D-CNN & supervised & CHB-MIT & binary & cross-subject & 98.05\% \\
~\cite{Seizure12} & FFT,WPD & Time-Frequency  features & 2D-CNN & supervised & CHB-MIT & binary & subject-specific & 98.33\% \\
~\cite{Seizure13} & STFT,Filtering,\newline MAS calculation & MAS Map Image & 2D-CNN & supervised & CHB-MIT,\newline private & binary \newline
3-class \newline
5-class & mixed-subject & 99.33\% \newline
98.62\% \newline
87.95\% \\
~\cite{Seizure14} & MPS & 2D Spectrograms & 2D-CNN & supervised & CHB-MIT,\newline private & binary & mixed-subject & SEN\textgreater 90\% \\
~\cite{Seizure15} & Filtering,Segmentation & 2D Image & 2D-CNN & supervised & private & binary & cross-subject & TPR=74\% \\
~\cite{Seizure16} & Filtering,Normalization,\newline Image generation & 2D Image & 2D-CNN & supervised & private & binary & mixed-subject & 87.65\% \\
~\cite{Seizure17} & FFT & 2D Spectrograms & 2D-CNN & supervised & TUSZ & binary & cross-subject & F1=59.2\% \\
~\cite{Seizure18} & Segmentation,\newline Image generation & RPS Image & 2D-CNN & supervised & Bonn & binary \newline
3-class & mixed-subject & 98.5\% \newline
95\% \\
~\cite{Seizure19} & CWT & Scalograms & 2D-CNN & supervised & Bonn & binary & mixed-subject & 72.49\% \\
~\cite{Seizure20} & Hilbert Transform,\newline GASF,GADF & 2D Images & 2D-CNN & supervised & Bonn & binary & mixed-subject & 98\% \\
~\cite{Seizure21} & Filtering,DWT & 2D Image & 2D-CNN & supervised & Bonn & binary & mixed-subject & 97.74\% \\
~\cite{Seizure22} & Segmentation & Raw Segments & 2D-CNN & supervised & CHB-MIT & binary & cross-subject & 99.72\% \\
~\cite{Seizure23} & Segmentation,DWT & PSDED & 2D-CNN & supervised & CHB-MIT & 4-class & mixed-subject & 92.6\% \\
~\cite{Seizure24} & Channel selection,\newline Image generation & 2D Image & 2D-CNN & supervised & CHB-MIT & 3-class & mixed-subject & 94.98\% \\
~\cite{Seizure25} & Segmentation,STFT & 2D Spectrograms & 2D-CNN & supervised & CHB-MIT & binary & subject-specific & 95.65\% \\
~\cite{Seizure26} & Filtering,Segmentation,\newline STFT & 2D Spectrograms & 2D-CNN & supervised & CHB-MIT & binary & subject-specific & SEN=92.7\% \\
~\cite{Seizure27} & Segmentation,STFT & 2D Spectrograms & 2D-CNN & supervised & CHB-MIT & binary & mixed-subject & 98.26\% \\
~\cite{Seizure28} & Filtering,FT,Welch’s,WPD & Fusion feature Image & 2D-CNN & supervised & CHB-MIT,\newline private & 5-class & mixed-subject & 98.97\% \\
~\cite{Seizure29} & Normalization,DWT,\newline S-Transform & 2D Spectrograms & 2D-CNN & supervised & Freiburg & binary & subject-specific & 98.12\% \\
~\cite{Seizure30} & Segmentation,CWT & scalograms & 2D-CNN & supervised & Melbourne & binary & mixed-subject & AUC=0.928 \\
~\cite{Seizure31} & Filtering,STFT & 2D Spectrograms & 2D-CNN & supervised & TUSZ & binary & cross-subject & 88.3\% \\
~\cite{Seizure32} & Filtering,Segmentation & Raw Segments & 2D-CNN & supervised & TUSZ & binary & cross-subject & 70.38\% \\
~\cite{Seizure33} & Segmentation,STFT,CWT & 2D Spectrogram,\newline Scalogram & 2D-CNN & supervised & Bonn & binary & mixed-subject & 99.21\% \\
\hline
\end{tabular}
\end{table*}


\begin{table*}[ht]
\renewcommand{\arraystretch}{1.2}
\caption*{(Continued) Summary of deep learning frameworks for seizure detection}
\footnotesize
\begin{tabular}{p{0.4cm}p{2.8cm}p{2cm}p{1.5cm}p{1.9cm}p{1.9cm}p{0.8cm}p{1.8cm}p{1.5cm}}
\hline
\textbf{Ref.} & \textbf{Preprocessing} & \textbf{Feature} & \textbf{Backbone} & \textbf{Training} & \textbf{Dataset} & \textbf{Task} & \textbf{Partitioning} & \textbf{Accuracy} \\
\hline
 ~\cite{Seizure34} & Segmentation,
FNSW & 2D Image & 2D-CNN & supervised & Bonn & binary
3-class
5-class & mixed-subject & 100\% \\
~\cite{Seizure35} & Segmentation,EMD,WOG & Graph\newline  representation & 2D-CNN & supervised & Bonn,\newline private & binary & mixed-subject & 100\% \newline
97.65\% \\
~\cite{Seizure36} & Z-norm,Windowing & RPS Image & 2D-CNN & supervised & Bonn & binary & mixed-subject & 92.3\% \\
~\cite{Seizure37} & Filtering,CWT & 2D Scalograms & 2D-CNN & supervised & Bonn & binary & mixed-subject,\newline cross-subject & 99.5\% \\
 ~\cite{yuan2018novel} & STFT & 2D Spectrograms & 2D-CNN\newline +Attention & supervised & CHB-MIT & binary & mixed-subject & 96.61\% \\
~\cite{Seizure39} & Filtering,Z-norm & Raw Segments & 2D-CNN\newline +Attention & supervised & SWEC-ETHZ,private & binary & subject-specific & AUC=0.92
AUC=0.96 \\
~\cite{choi2019novel} & STFT & Spectrograms & 3D-CNN & supervised & CHB-MIT,\newline private & binary & cross-subject & 99.4\% \\
~\cite{zhang2022epileptic} & Segmentation,WT & Relative Energy matrix & Bi-GRU & supervised & CHB-MIT,\newline private & binary & cross-subject,\newline subject-specific & SEN=95.49 \\
~\cite{Seizure42} & Segmentation,\newline Time-GAN & Enhanced Segments & BiLSTM & supervised & private & binary & cross-subject & 78.5\% \\
~\cite{Seizure43} & Filtering,Frequency feature extraction & Linear features & Bi-LSTM & supervised & Bonn & binary & mixed-subject & 98.56\% \\
~\cite{Seizure44} & Z-norm,Filtering,\newline Segmentation & Raw Segments & Bi-LSTM & supervised & Bern-Barcelona & binary & mixed-subject & 99.6\% \\
~\cite{Seizure45} & Segmentation,LMD & Statistical features & Bi-LSTM & supervised & CHB-MIT & binary & subject-specific & SEN=93.61\% \\
~\cite{Seizure46} & Segmentation,\newline S-transform & Spectrogram & Bi-LSTM & supervised & Freiburg & binary & subject-specific & 98.69\% \\
~\cite{Seizure47} & Segmentation & Segments & Bi-LSTM & supervised & CHB-MIT & binary & mixed-subject,\newline cross-subject & 87.8\% \\
~\cite{Seizure48} & Normalization,\newline Instantaneous frequency & Spectral entropy & Bi-LSTM & supervised & Bonn & binary \newline
5-class & mixed-subject & 100\% \newline
96\% \\
~\cite{Seizure49} & Baseline Correction,\newline Windowing,linear detrending & Raw Segments & CNN & supervised & private & binary & cross-subject & 87.51\% \\
~\cite{Seizure50} & Downsampling,
Filtering & Raw & CNN & supervised & private & binary & cross-subject & 97.1\% \\
~\cite{ACHARYA2018270} & Z-norm & Raw & CNN & supervised & Bonn & binary & mixed-subject & 88.67\% \\
~\cite{Seizure52} & Segmentation,EMD & IMFs of EMD & CNN & supervised & Bonn & binary \newline
3-class & mixed-subject & 100\% \newline
98.6\% \\
~\cite{Seizure53} & Segmentation & Raw Segments & CNN & supervised & Bonn & binary & mixed-subject & 99.1\% \\
~\cite{Seizure54} & Normalization & Raw Segments & CNN & supervised & Bonn & binary \newline
5-class & cross-subject & 97.38\% \newline
93.67\% \\
~\cite{Seizure55} & Filtering,Segmentation & Raw Segments & CNN & supervised & CHB-MIT & binary & cross-subject & SEN=86.29\% \\
~\cite{Seizure56} & Filtering,Downsampling,\newline CAR montage & Raw & CNN & supervised & private & binary & cross-subject & AUC=93.5\% \\
~\cite{Seizure57} & Segmentation,
Normalization,
Standardization & Segments & CNN & supervised & TUSZ & binary & cross-subject & 79.34\% \\
~\cite{Seizure58} & DWT & Wavelet Coefficients & CNN & supervised & Bonn & binary & mixed-subject & 100\% \\
~\cite{Seizure59} & Filtering,
Z-norm & Raw & CNN & supervised & Bonn & binary & mixed-subject & 99\% \\
~\cite{Seizure60} & Data Augmentation,\newline feature enhancement  & Enhanced Segments & CNN & supervised & CHB-MIT & binary & cross-subject & SEN=74.08\% \\
~\cite{Seizure61} & Filtering,Windowing & Raw Segments & CNN & supervised & CHB-MIT, \newline private & binary & mixed-subject & AUC=0.8 \\
~\cite{Seizure62} & Filtering,Windowing & Raw Segments & CNN & supervised & private & binary & cross-subject & AUC=0.83 \\
 ~\cite{Seizure63} & Downsampling,Z-norm,\newline Windowing,Data Augmentation & Raw Segments & CNN & supervised & private & binary & cross-subject & SEN=95.8\% \\
~\cite{Seizure64} & Z-norm,Filtering,\newline Segmentation & Raw Segments & CNN & supervised & private & binary & cross-subject & AUC=0.961 \\
~\cite{Seizure65} & Normalization,\newline Segmentation & Raw Segments & CNN & supervised & Bern-Barcelona & binary & mixed-subject & 91.5\% \\
~\cite{Seizure66} & Filtering,\newline Data Augmentation & Augmented data & CNN & supervised & Bern-Barcelona & binary & mixed-subject & 89.28\% \\
\hline
\end{tabular}
\end{table*}


\begin{table*}[ht]
\renewcommand{\arraystretch}{1.2}
\caption*{(Continued) Summary of deep learning frameworks for seizure detection}
\footnotesize
\begin{tabular}{p{0.4cm}p{2.8cm}p{2cm}p{1.5cm}p{1.9cm}p{1.9cm}p{0.8cm}p{1.8cm}p{2cm}}
\hline
\textbf{Ref.} & \textbf{Preprocessing} & \textbf{Feature} & \textbf{Backbone} & \textbf{Training} & \textbf{Dataset} & \textbf{Task} & \textbf{Partitioning} & \textbf{Accuracy} \\
\hline
~\cite{Seizure67} & Filtering,DWT,\newline Power Spectrum Band Calculation,Frequency Band Calculation & 2D Image & CNN & supervised & Bonn & binary & cross-subject & 99.99\% \\
~\cite{Seizure68} & Segmentation,Filtering & ApEn and RQA vector & CNN & supervised & Bonn & binary & mixed-subject & 99.26\% \\
~\cite{Seizure69} & Normalization,CWT & 2D Scalograms & CNN & supervised & Bonn & binary & mixed-subject & 98.78\% \\
~\cite{Seizure70} & Filtering & Raw & CNN & supervised & Bonn & binary \newline
3-class & mixed-subject & 100\% \newline
99.8\% \\
~\cite{Seizure71} & - & Raw & CNN & supervised & Bonn & 3-class & mixed-subject & 98.67\% \\
 ~\cite{Seizure72} & Segmentation,\newline Data Augmentation & Raw Segments & CNN & supervised & Bonn & binary & mixed-subject & AUC=0.92\% \\
~\cite{Seizure73} & Z-norm & Raw & CNN & supervised & Bonn & binary \newline
5-class & mixed-subject & 99.93\% \newline
94.01\% \\
~\cite{Seizure74} & Normalization & Raw & CNN & supervised & Bonn & binary \newline
3-class \newline
5-class & mixed-subject & 98.5-100\% \\
~\cite{Seizure75} & Segmentation,\newline 
Normalization & Raw Segments & CNN & supervised & Bonn & binary \newline
3-class \newline
5-class & mixed-subject & 97.63\%-99.52\% \newline
96.73\%-98.06\% \newline
93.55\% \\
~\cite{Seizure76} & Z-norm & Raw & CNN & supervised & Bonn,\newline CHB-MIT & binary & mixed-subject & 98.67\% \\
~\cite{Seizure77} & Segmentation,Baseline Removal,Resampling,
Detrending,Filtering & Raw Segments & CNN & supervised & Bonn,\newline TUSZ,\newline CHB-MIT & binary & subject-specific & 99.8\% \newline
92\% \newline
95.96\% \\
~\cite{Seizure78} & Channel selection & Raw & CNN & supervised & CHB-MIT & binary & subject-specific & 96.1\% \\
~\cite{Seizure79} & Filtering,Segmentation,\newline Spectrogram generation & 2D Spectrograms & CNN & supervised & CHB-MIT & binary & subject-specific & 77.57\% \\
~\cite{Seizure80} & Segmentation & Raw Segments & CNN & supervised & CHB-MIT & binary & mixed-subject & 96.74\% \\
~\cite{Seizure81} & Normalization,\newline Segmentation & Raw Segments & CNN & supervised & CHB-MIT & binary & cross-subject & 97\% \\
~\cite{Seizure82} & Segmentation,Filtering,\newline FFT,WT & spectral data & CNN & supervised & CHB-MIT & binary & mixed-subject & 97.25\% \\
~\cite{Seizure83} & Filtering,resampling,\newline Segmentation & Raw Segments & CNN & supervised & CHB-MIT & binary & subject-specific & 84.1\% \\
~\cite{Seizure84} & Segmentation & Raw Segments & CNN & supervised & CHB-MIT, \newline Mayo-Upenn & binary & subject-specific & AUC=0.970 \newline
AUC=0.915 \\
~\cite{Seizure85} & Downsampling,Filtering,\newline Artifact Removal & Raw Segments & CNN & supervised & private & binary & cross-subject,\newline subject-specific & 99.6\% \\
~\cite{Seizure86} & Z-norm,Filtering & Raw Segments & CNN & supervised & private & binary & cross-subject & 80\% \\
~\cite{Seizure87} & Segmentation,Filtering,\newline Data Augmentation & Raw & CNN & supervised & private & binary & cross-subject,\newline subject-specific & 96.39\% \\
~\cite{Seizure88} & Filtering,Segmentation & Raw Segments & CNN & supervised & private & binary & cross-subject & - \\
~\cite{Seizure89} & Filtering,Downsampling,\newline Segmentation & Raw Segments & CNN & supervised & private & binary & subject-specific & AUC=98.9 \\
~\cite{Seizure90} & Windowing,Normalization & Raw Segments & CNN & supervised & private & binary & cross-subject & 77\% \\
~\cite{Seizure91} & Downsampling,Windowing & Raw, Periodogram, Spectrograms, Image & CNN & supervised & Mayo-Upenn & binary & cross-subject,\newline subject-specific & 99.9\% \\
~\cite{Seizure92} & Segmentation & Raw Segments & CNN & supervised & Mayo-UPenn,\newline CHB-MIT & binary & subject-specific & AUC=0.981 \newline
AUC=0.988 \\
~\cite{Seizure93} & Time-Frequency feature extraction & Pattern Matrices & CNN & supervised & TUSZ & binary & cross-subject & AUC=0.74 \\
~\cite{Seizure94} & Segmentation & Raw Segments & CNN & supervised & TUSZ & binary & cross-subject & 80.5\% \\
~\cite{Seizure95} & 1D-AaLBP,1D-AdLBP & Histogram-based feature & CNN & supervised & Bonn,\newline CHB-MIT & binary \newline 
5-class & mixed-subject & 98.8\% - 99.65\% \newline
99.11\% \\
~\cite{Seizure96} & Filtering,Segmentation,\newline FFT & Frequency-domain features & CNN & supervised & Mayo-Upenn & binary & subject-specific & 94.74\% \\
~\cite{tudmnet} & Normalization,Differencing & Difference Matrix & CNN & supervised  & MAYO, FNUSA, private & binary & cross-subject & F2=55.93-81.54 \\
\hline
\end{tabular}
\end{table*}


\begin{table*}[ht]
\renewcommand{\arraystretch}{1.2}
\caption*{(Continued) Summary of deep learning frameworks for seizure detection}
\footnotesize
\begin{tabular}{p{0.4cm}p{2.8cm}p{2cm}p{1.5cm}p{1.9cm}p{1.9cm}p{0.8cm}p{1.8cm}p{1.5cm}}
\hline
\textbf{Ref.} & \textbf{Preprocessing} & \textbf{Feature} & \textbf{Backbone} & \textbf{Training} & \textbf{Dataset} & \textbf{Task} & \textbf{Partitioning} & \textbf{Accuracy} \\
\hline
~\cite{Seizure97} & Filtering,GPSO & Time- \& Freq-\newline domain featrues & CNN & supervised & Bonn & binary & mixed-subject & 99.65\% \\
~\cite{Seizure98} & Z-norm,FFT & Raw,features & CNN & supervised & Bonn & binary \newline
3-class & mixed-subject & 98.23\% \newline
96.33\% \\
~\cite{assali2023cnn} & Normalization,Filtering,\newline STFT & RPSD,SampEn,SI & CNN & supervised & CHB-MIT & binary & subject-specific & 94.5\% \\
~\cite{Seizure100} & Normalization,\newline Segmentation & Raw Segments & CNN & supervised & private & binary & mixed-subject & 99.61\% \\
~\cite{Seizure101} & Filtering,Segmentation & Freq-features,\newline Time-Freq Image & CNN \newline 3D-CNN & supervised & Helsinki & binary & cross-subject & 90.06\% \\
~\cite{Seizure102} & Filtering,Segmentation,\newline Artifact rejection
 & Raw Segments & CNN\newline 2D-CNN & supervised & private & binary & cross-subject & 96.39\% \\
~\cite{Seizure103} & Normalization & Raw & CNN\newline CNN-LSTM & supervised & CHB-MIT & binary & subject-specific & 91.50\% \newline
92.11\% \\
~\cite{Seizure104} & Segmentation & Raw & CNN,LSTM & supervised & CHB-MIT & binary & subject-specific & 89.21\% \\
~\cite{Seizure105} & Segmentation,\newline 
Normalization & NaN & CNN\newline LSTM \newline GRU & supervised & Bonn & binary & mixed-subject & 97.27\% \newline 96.82\% \newline 96.67\% \\
~\cite{Seizure106} & STFT & 2D Spectrograms & CNN+Attention & supervised & CHB-MIT & binary & cross-subject & 96.22\% \\
~\cite{Seizure107} & Filtering,Downsampling,\newline Segmentation & Raw Segments & CNN+Attention & supervised & private & binary & cross-subject & AUC=0.97 \\
~\cite{Seizure108} & Downsampling,\newline Segmentation & Raw Segments & CNN-BiGRU & supervised & CHB-MIT,\newline Bonn,\newline Mayo-Upenn & binary & mixed-subject & 0.985 \\
~\cite{Seizure109} & DWT & Statistical,Freq-,\newline Nonlinear features & CNN-BiGRU\newline +Attention & supervised & Freiburg & binary & mixed-subject & 98.35\% \\
~\cite{Seizure110} & Filtering,Segmentation,\newline Normalization & Raw Segments & CNN-BiLSTM & supervised & private & binary & cross-subject & AUC=0.9042 \\
~\cite{Seizure111} & Normalization,K-means SMOTE & Raw Segments & CNN-BiLSTM & supervised & Bonn & binary \newline
5-class & mixed-subject & 99.41\% \newline
84.10\% \\
~\cite{Seizure112} & Downsampling,Bipolar Reference,Segmentation & Raw Segments & CNN-BiLSTM\newline +Attention & supervised & Mayo-UPenn,\newline private & binary & cross-subject & 94.12\% \\
~\cite{dutta2024deep} & Windowing & Raw Segments & CNN-BiLSTM\newline +Attention & supervised & CHB-MIT & binary & subject-specific & 96.61\% \\
~\cite{Seizure114} & Filtering,S-transform & Adjacency matrix & CNN-GCN & supervised & CHB-MIT & binary & cross-subject & 98\% \\
 ~\cite{Seizure115} & TCP & Raw & CNN-GRU & supervised & TUSZ & binary & cross-subject & 86.57\% \\
~\cite{Seizure116} & Segmentation,WPT & Multi-view feature matrix & CNN-GRU & supervised & CHB-MIT & binary & subject-specific & SEN=94.50\% \\
~\cite{Seizure117} & Filtering,CWT & 2D Scalograms & CNN-GRU & supervised & Bonn & binary \newline
3-class \newline
5-class & mixed-subject & 100\% \newline
100\% \newline
99.4\% \\
~\cite{xu2023patient} & Windowing,Filtering,\newline Z-norm & - & CNN-GRU & supervised & CHB-MIT & binary & subject-specific & AUC=0.88 \\
~\cite{Seizure119} & Frequency Decomposition,Image generation & 2D Image & CNN-LSTM & supervised & CHB-MIT & binary & cross-subject,\newline subject-specific & SEN=96\% \\
~\cite{Seizure120} & Segmentation,PCA & LFCCs & CNN-LSTM & supervised & TUSZ,private & 6-class & cross-subject & SEN=30\% \\
~\cite{Seizure121} & - & Raw & CNN-LSTM & supervised & Bonn & binary \newline
3-class & mixed-subject & 100\% \newline
98.33\% \\
~\cite{Seizure122} & Segmentation & Raw Segments & CNN-LSTM & supervised & Bonn & binary & mixed-subject & 98.8\% \\
~\cite{Seizure123} & Normalization & Raw & CNN-LSTM & supervised & Bonn & binary \newline
5-class & mixed-subject & 99.39\% \newline
82\% \\
~\cite{Seizure124} & Filtering,Segmentation & Raw Segments & CNN-LSTM & supervised & Bonn, \newline Freiburg, \newline
CHB-MIT & binary & mixed-subject & 100\% \newline
96.17\% \newline
95.29\% \\
 ~\cite{Seizure125} & Segmentation,\newline Image generation & 2D Image & CNN-LSTM & supervised & CHB-MIT & 4-class & cross-subject & 99\% \\
~\cite{Seizure126} & Segmentation,FFT,DWT & Time-Frequency  features & CNN-LSTM & supervised & Freiburg & binary & mixed-subject & 99.27\% \\
~\cite{Seizure127} & Segmentation,\newline Format Conversion & EEG video & CNN-LSTM & supervised & private & binary & cross-subject & SEN=88\% \\
\hline
\end{tabular}
\end{table*}


\begin{table*}[ht]
\renewcommand{\arraystretch}{1.2}
\caption*{(Continued) Summary of deep learning frameworks for seizure detection}
\footnotesize
\begin{tabular}{p{0.4cm}p{2.8cm}p{2cm}p{1.5cm}p{1.9cm}p{1.9cm}p{0.8cm}p{1.8cm}p{1.5cm}}
\hline
\textbf{Ref.} & \textbf{Preprocessing} & \textbf{Feature} & \textbf{Backbone} & \textbf{Training} & \textbf{Dataset} & \textbf{Task} & \textbf{Partitioning} & \textbf{Accuracy} \\
\hline
~\cite{Seizure128} & Filtering,Segmentation,\newline CWT,STFT & 2D Spectrogram,\newline Scalogram & CNN-LSTM & supervised & Bonn \newline CHB-MIT \newline Bern-Barcelona & binary & mixed-subject & 99.94\% \newline
93.77\% \newline
95.08\% \\
~\cite{Seizure129} & Filtering,STFT & 2D Spectrograms & CNN-LSTM & supervised & CHB-MIT & binary & subject-specific & 94.5\% \\
~\cite{Seizure130} & Filtering,Difference & Raw,Differential signal & CNN-LSTM\newline +Attention & supervised & Bonn & binary \newline
5-class & mixed-subject & 98.87\% \newline
90.17\% \\
~\cite{Seizure131} & Filtering,Resampling,TCP & Segments & CNN-RNN & supervised & TUSZ & binary & cross-subject & 82.27\% \\
~\cite{Seizure132} & Segmentation & Raw Segments & CNN-RNN \newline +Attention & supervised & CHB-MIT & binary & mixed-subject & SEN=92.88\% \\
~\cite{Seizure133} & Normalization & Raw Segments & CNN-Transformer & supervised & TUSZ & various & cross-subject & AUC=0.72 \\
~\cite{Seizure134} & Channel selection, \newline Windowing & Raw Segments & CNN-Transformer & supervised & CHB-MIT & binary & cross-subject, \newline subject-specific & SEN=65.5\% \\
~\cite{sun2022continuous} & Filtering,resampling, \newline Windowing & Raw Segments & CNN-Transformer & supervised & SWEC-ETHZ, \newline private & binary & subject-specific & SEN=97.5\% \\
~\cite{Seizure136} & Filtering,Z-norm,DWT & Rhythm Signals & CNN-Transformer & supervised & CHB-MIT & binary & cross-subject & SEN=91.7\% \\
~\cite{Seizure137} & Filtering,Windowing & Raw Segments & CNN-Transformer & supervised & CHB-MIT & binary & subject-specific & AUC=0.937 \\
~\cite{Seizure138} & Filtering,Downsampling,\newline Bipolar Reference & Raw Segments & CNN-Transformer & supervised & TUSZ, \newline CHB-MIT & binary & cross-subject & 49.1\%-85.8\% \\
~\cite{Seizure139} & Filtering,Segmentation,\newline STFT & Time-Frequency  features & CNN-Transformer & supervised & CHB-MIT & binary & cross-subject & 94.75\% \\
~\cite{Seizure140} & Bipolar referencing,\newline Filtering,Z-norm & Raw Segments & CNN-Transformer & supervised & SWEC-ETHZ \newline HUP & binary & cross-subject & 91.15\% \newline
88.84\% \\
~\cite{Seizure141} & DWT & Wavelet-based features & DBN & supervised & private & binary & cross-subject & 96.87\% \\
~\cite{Seizure142} & Segmentation,GASF & GASF Image & Pre-Trained Networks, Deep ANN & supervised & Bern-Barcelona & binary & mixed-subject & AUC=0.92 \\
~\cite{Seizure143} & Min-max Normalization & Raw & DNN & supervised & Bonn & binary & mixed-subject & 97.17\% \\
~\cite{Seizure144} & Normalization & Raw & DNN & supervised & Bonn & binary & mixed-subject & 80\% \\
~\cite{Seizure145} & Filtering,Segmentation,ToC & SAE-based & DNN & supervised & Bonn & binary \newline
3-class \newline
5-class & mixed-subject & 100\% \newline
99.6\% \newline
97.2\% \\
~\cite{Seizure146} & DWT,Normalization & Nonlinear and entropy features & DNN & supervised & Bonn, \newline Bern-Barcelona, \newline CHB-MIT & binary \newline 
3-class & mixed-subject & 93.61\%(Bonn) \\
~\cite{Seizure147} & Filtering,Z-norm,DWT & Wavelet Coefficients & DWT-Net & supervised & TUSZ & binary & cross-subject & SEN=59.07\% \\
~\cite{Seizure148} & Filtering,Z-norm,\newline Network construction & Adjacency matrix & GAT & supervised & CHB-MIT & binary & subject-specific & 98.89\% \\
~\cite{Seizure149} & Filtering,Network \newline construction & Node Feature \newline matrix,Adjacency matrix & GAT+GRU & supervised & CHB-MIT, \newline private & binary & cross-subject, \newline subject-specific & 98.74\% \\
~\cite{Seizure150} & Filtering,Z-norm,\newline Network construction & Adjacency matrix,Raw & GAT\newline +Transformer & supervised & CHB-MIT & binary & cross-subject, \newline subject-specific & 98.3\% \\
~\cite{Seizure151} & Filtering,Z-norm & Node Feature \newline matrix,Adjacency matrix & GAT-BiLSTM & supervised & CHB-MIT & binary & subject-specific & 98.52\% \\
~\cite{Seizure152} & ICA & Correlation \newline matrix & GCN & supervised & Bonn, \newline
CHB-MIT & binary
3-class & mixed-subject & 99.8\% \newline
99.2\% \\
~\cite{Seizure153} & FFT,VG & Frequency-domain Network & GCN & supervised & Bonn, \newline private & binary & mixed-subject & 100\% \\
~\cite{zhao2021eeg} & Filtering,Segmentation,\newline Network construction & Raw Segments,\newline Adjacency matrix & GCN & supervised & CHB-MIT & binary & subject-specific & 99.3\% \\
~\cite{Seizure155} & Filtering,Z-norm,\newline Segmentation,Network construction & EEG Network & GCN & supervised & private & binary & mixed-subject & AUC=0.91 \\

\hline
\end{tabular}
\end{table*}


\begin{table*}[ht]
\renewcommand{\arraystretch}{1.2}
\caption*{(Continued) Summary of deep learning frameworks for seizure detection}
\footnotesize
\begin{tabular}{p{0.4cm}p{2.8cm}p{2cm}p{1.5cm}p{1.9cm}p{1.9cm}p{0.8cm}p{1.8cm}p{2cm}}
\hline
\textbf{Ref.} & \textbf{Preprocessing} & \textbf{Feature} & \textbf{Backbone} & \textbf{Training} & \textbf{Dataset} & \textbf{Task} & \textbf{Partitioning} & \textbf{Accuracy} \\
\hline
~\cite{Seizure156} & Segmentation,Network construction & Adjacency matrix & GCN & supervised & CHB-MIT & binary & subject-specific & 98.38\% \\
~\cite{Seizure157} & Filtering,Z-norm & Node Feature \newline matrix,Adjacency matrix & GCN+Attention & supervised & CHB-MIT & binary & subject-specific & 98.7\% \\
~\cite{Seizure158} & Filtering,Windowing & Raw Segments & GCN-Transformer & supervised & CHB-MIT & binary & subject-specific & AUC=0.935 \\
~\cite{rahmani2023meta} & Filtering,Segmentation,FFT & Node Feature \newline matrix,Adjacency matrix & GNN & supervised & TUSZ & binary & cross-subject & 81.77\% \\
 ~\cite{Seizure160} & Filtering,Segmentation,\newline Network construction & NaN & GNN+ Transformer & supervised & CHB-MIT & binary & subject-specific & 98.43\% \\
~\cite{Seizure161} & Segmentation & Raw Segments & GRU & supervised & Bonn & 3-class & mixed-subject & 98\% \\
~\cite{Seizure162} & DWT & Wavelet Coefficients & GRU & supervised & Bonn & binary & subject-specific & 98.5\% \\
~\cite{Seizure163} & - & Raw & GRU & supervised & Bonn & binary & mixed-subject & 97.5\% \\
~\cite{Seizure164} & Segmentation & Raw Segments & LSTM & supervised & Bonn & 3-class & mixed-subject & 100\% \\
~\cite{Seizure165} & - & Raw & LSTM & supervised & Bonn & binary & mixed-subject & 95.54\% \\
~\cite{Seizure166} & Data Augmentation,\newline Segmentation & Raw Segments & LSTM & supervised & Bonn & binary & mixed-subject & 100\% \\
~\cite{Seizure167} & Z-norm,DCT & Hurst \& ARMA features & LSTM & supervised & Bonn & binary \newline
3-class & mixed-subject & 99.17\% \newline
94.81\% \\
% ODLN ~\cite{Seizure168} & Filtering,Synchronization,
% Segmentation & Raw Segments & LSTM & supervised & CHB-MIT & binary & SEN=100\% \\
~\cite{Seizure169} & DWT & 20 Eigenvalue features & LSTM & supervised & Bonn & binary & mixed-subject & 99\% \\
~\cite{Seizure170} & Filtering,Segmentation,FFT & Freq-domain \newline features & LSTM & supervised & CHB-MIT & binary & subject-specific & 98.14\% \\
~\cite{Seizure171} & DWT,CFS & Time-Frequency  features & LSTM & supervised & TUSZ & binary \newline
4-class & cross-subject & 98.08\% \newline
95.92\% \\
~\cite{Seizure172} & Filtering,decomposition & Time- \& Freq- domain featrues & LSTM & supervised & CHB-MIT, \newline Siena, \newline Beirut, \newline Bonn & binary & mixed-subject, \newline cross-subject & 94.69\% \\
~\cite{Seizure173} & Filtering & Montage grid & RNN & supervised & CHB-MIT & binary & subject-specific & SEN=100\% \\
~\cite{Seizure174} & Segmentation & Raw Segments & RNN & supervised & CHB-MIT & binary & cross-subject & 88.7\% \\
~\cite{Seizure175} & Segmentation & Raw Segments & RNN & supervised & CHB-MIT & binary & mixed-subject & 87\%\\
~\cite{Seizure176} & - & Raw & RNN & supervised & Bonn & binary \newline
3-class \newline
5-class & mixed-subject & 99.33\% \newline
98.2\% \newline
81.33\% \\
~\cite{Seizure177} & Filtering,Z-norm,\newline Windowing & Raw Segments & RNN-Transformer & supervised & Bonn, \newline CHB-MIT & binary & subject-specific & 95.06\% \\
~\cite{Seizure178} & STFT & Spectrogram & RNN-Transformer & supervised & Bonn, \newline CHB-MIT & binary & subject-specific & 99.75\% \\
% ~\cite{Seizure179} & Spike Preprocessing & Spike Encoding & SNN & supervised & private & binary & 92.5\% \\
~\cite{Seizure180} & STFT & Spectrograms & TGCN & supervised & private & binary & cross-subject & AUC=0.928 \\
~\cite{Seizure181} & Resampling,Segmentation & Raw Segments & Transformer & supervised & TUSZ & binary & cross-subject & SEN=9.03\% \\
~\cite{Seizure182} & STFT,Bipolar Montage & Time-Frequency Graph & Transformer & supervised & TUSZ & binary & cross-subject & AUC=0.921 \\
~\cite{9353630} & Subspace Filtering,\newline ICLabel  & Raw,Subspace Filtering,ICLabel & U-net & supervised & TUSZ & binary & cross-subject & - \\
~\cite{meisel2019identifying} &	Filtering,PSD & PSD & DNN & supervised & private & binary & subject-specific	& -\\
~\cite{guo2021detecting} & Filtering &	Hypergraph-based HSO &	DNN	& supervised & private &	binary & mixed-subject & 90.70\%\\
~\cite{wagh2021domain} & Filtering,Downsampling,\newline Segmentation & 2D Topomap & 2D-CNN & self-supervised & TUSZ & binary & cross-subject & AUC=0.92 \\
~\cite{zheng2022task} & - & Raw Segments & CNN & self-supervised & CHB-MIT, \newline Mayo-UPenn, \newline private & binary & mixed-subject, \newline cross-subject & AUC=0.92-0.95 \\
%~\cite{cai2023mbrain} & Segmentation & Raw Segments & CNN-GCN & self-supervised & TUSZ, \newline private & binary & cross-subject & AUC=0.78 \\
~\cite{chen2022brainnet} & Windowing & Raw Segments & CNN-GNN & self-supervised & private & binary & cross-subject & F2=76.87\% \\

\hline
\end{tabular}
\end{table*}


\begin{table*}[ht]
\renewcommand{\arraystretch}{1.2}
\caption*{(Continued) Summary of deep learning frameworks for seizure detection}
\footnotesize
\begin{tabular}{p{0.4cm}p{2.8cm}p{2cm}p{1.5cm}p{1.9cm}p{1.9cm}p{0.8cm}p{1.8cm}p{1.5cm}}
\hline
\textbf{Ref.} & \textbf{Preprocessing} & \textbf{Feature} & \textbf{Backbone} & \textbf{Training} & \textbf{Dataset} & \textbf{Task} & \textbf{Partitioning} & \textbf{Accuracy} \\
\hline
~\cite{yuan2024ppi} & Downsampling,\newline Segmentation & Time- \& Freq- domain featrues, Raw & CNN-LSTM & self-supervised & Mayo-UPenn,\newline FNUSA, \newline private & binary & cross-subject & F1=85.6\% \newline F1=82.3\% \newline F1=87.1\% \\
~\cite{tang2021self} & Z-norm,FFT & Adjacency matrix,Frequency-domain features & GNN & self-supervised & TUSZ & binary & cross-subject & AUC=0.875 \\
~\cite{ho2023self} & FFT,graph construction & EEG Network & GNN & self-supervised & TUSZ & binary & cross-subject & F1=0.534\% \\
~\cite{lih2023epilepsynet} & Segmentation,PCC & PCC matrix & Transformer & self-supervised & Turkish & binary & cross-subject & 85\% \\
~\cite{XIAO2024105464} & Filtering,Z-norm,\newline Windowing & Raw Segments & Transformer & self-supervised & CHB-MIT & binary & cross-subject & 97.07\% \\
~\cite{li2022spp} & Filtering,Segmentation & Raw Segments &	CNN	& self-supervised & TUSZ & binary & cross-subject &	- \\
~\cite{peng2023wavelet2vec} & DWPT	& Wavelets & Transformer &	self-supervised & TUSZ & 4-class &	cross-subject & 73\%\\
~\cite{Seizure193} & Normalization, \newline Data Enhancement & AE-based & AE & semi-supervised & Bonn & binary \newline
5-class & cross-subject & 99.6\% \newline
96.4\% \\
~\cite{Seizure194} & Segmentation,Filtering,\newline Data Augmentation & Raw Segments & CNN & semi-supervised & CHB-MIT, \newline private & binary & mixed-subject & 90.58\% \\
~\cite{barry2021high} & Artifacts removal,FFT &	2D Spectrograms & CNN & semi-supervised	& private	& binary &	cross-subject	& 95.70\%\\
~\cite{Seizure195} & STFT & SSDA-based & 2D-CNN & unsupervised & CHB-MIT & binary & cross-subject & 94.37\% \\
~\cite{Seizure196} & FFT & BP-ASE-based & 2D-CNN & unsupervised & CHB-MIT & binary & cross-subject & 99.4\% \\
~\cite{abdelhameed2018epileptic} & - & AE-based & CNN & unsupervised & Bonn & binary  \newline
3-class & cross-subject & 100\% \newline
99.33\% \\
~\cite{wen2018deep} & Normalization & AE-based & CNN & unsupervised & Bonn, \newline CHB-MIT & binary & cross-subject & 100\% \newline
92\% \\
~\cite{Seizure199} & Segmentation,STFT & GAN-based & CNN & unsupervised & CHB-MIT, \newline EPILEPSIAE, \newline Freiburg & binary & subject-specific & 77.68\% \newline
75.47\% \newline
65.05\%  \\
~\cite{Zhan2020EpilepsyDetection} & FT,WT & Spectrograms & CNN & unsupervised & Freiburg & clustering & subject-specific & 97.38\% \\
~\cite{Seizure201} & Filtering,Segmentation & AE-based & CNN & unsupervised & private & 3-class & subject-specific & 98.84\% \\
~\cite{Seizure202} & Filtering,Segmentation & AE-based & CNN & unsupervised & Bonn & binary & cross-subject & 99.8\% \\
~\cite{Seizure203} & Normalization & Raw & DBN & unsupervised & private & 5-class & cross-subject & F1=0.93\% \\
~\cite{turner2014deep} & Min-max Normalization & Time-domain features & DBN & unsupervised & private & binary & cross-subject, \newline subject-specific & F1=90\% \\
~\cite{Seizure205} & Filtering,Normalization & DCAE-based & DCAE & unsupervised & Bonn, \newline Bern-Barcelona & binary & mixed-subject & 96\% \newline  93.21\% \\
~\cite{Seizure206} & Segmentation,\newline Normalization & SSAE-based & DNN & unsupervised & Bonn & binary & mixed-subject & 96\% \\
~\cite{Seizure207} & Filtering & SAE-based & DNN & unsupervised & Bonn & binary \newline
3-class \newline
5-class & mixed-subject & 100\% \\
~\cite{Seizure208} & STFT & SSDA-based & DNN & unsupervised & CHB-MIT & binary & mixed-subject & 93.92\% \\
~\cite{Seizure209} & Taguchi Method & SSAE-based & DNN & unsupervised & Bonn & binary & mixed-subject & 100\% \\
~\cite{Seizure210} & Segmentation,Z-norm & DSAE-based & DNN & unsupervised & Bonn & binary & mixed-subject & 100\% \\
~\cite{Seizure211} & Filtering,Segmentation,\newline HWPT,FD & AE-based & DNN & unsupervised & Bonn & binary & mixed-subject & 98.67\% \\
~\cite{Seizure212} & Segmentation,CWT & SAE-based & DNN & unsupervised & CHB-MIT & binary & mixed-subject & 93.92\% \\
~\cite{Seizure213} & Downsampling,Filtering,\newline Z-norm & AE-based & DNN & unsupervised & private & binary & subject-specific & SEN=100\% \\
~\cite{Seizure214} & ESD & DSAE-based & DNN & unsupervised & private & binary & mixed-subject & 100\% \\
~\cite{9096344} & FBSE-EWT & SAE-based & DNN & unsupervised & Bern-Barcelona & binary & mixed-subject & 100\% \\
~\cite{you2020unsupervised} & Filtering,Segmentation,\newline Z-norm,STFT & 2D Spectrograms & GAN & unsupervised & private & binary & subject-specific & AUC=0.9393 \\

\hline
\end{tabular}
\end{table*}

\begin{table*}[ht]
\renewcommand{\arraystretch}{1.2}
\caption{Summary of deep learning frameworks for sleep staging}
\label{tab:sleeps}
\footnotesize
\begin{tabular}{p{0.4cm}p{2.8cm}p{2cm}p{1.5cm}p{1.9cm}p{1.9cm}p{0.8cm}p{1.8cm}p{2cm}}
\hline
\textbf{Ref.} & \textbf{Preprocessing} & \textbf{Feature} & \textbf{Backbone} & \textbf{Training} & \textbf{Dataset} & \textbf{Task} & \textbf{Partitioning} & \textbf{Accuracy} \\
\hline
~\cite{Sleep1} & Filtering,Feature \newline Extraction & Time- \& Freq- features & 3D-CNN & supervised & ISRUC & 5-class & cross-subject & 82\%-83.2\% \\
~\cite{eldele2021attention} & AFR & Raw Segments & CNN & supervised & Sleep-EDF, \newline SHHS & 5-class & cross-subject & 82.9\%-86.6\% \\
~\cite{Sleep3} & Filtering,Resampling & Raw Segments & CNN-Transformer & supervised & Sleep-EDF, \newline ISRUC, \newline private & 5-class & cross-subject & 84.76\%-86.32\% \\
~\cite{Sleep4} & Resampling,Segmentation & Raw Segments & CNN & supervised & Sleep-EDF, \newline SHHS  & 5-class  & cross-subject & 85.3\% \newline 88.1\% \\
~\cite{Sleep5} & DCT & DCT Coefficients & CNN-LSTM & supervised & SleepEDF, \newline DRM-SUB, \newline ISRUC & 5-class & cross-subject & 85.47\%-87.11\% \\
~\cite{Sleep6} & Segmentation & Raw Segments & CNN-BiLSTM & supervised & Sleep-EDF, \newline MASS & 5-class  & cross-subject & 82.0\% \newline 86.2\% \\
~\cite{Sleep7} & Filtering,Spectrogram Generation & 2D Spectrogram & 2D-CNN & supervised & Sleep-EDF, \newline SHHS & 5-class & mixed-subject & 83.02\%-94.17\% \\
~\cite{Sleep8} & Filtering,Downsampling & Complex Values & CNN & supervised & UCD, \newline MIT-BIH & 5-class & cross-subject & 92\% \\
~\cite{Sleep9} & Filtering,DE Calculation & DE matrix & GCN & supervised & MASS & 5-class  & cross-subject & 88.90\% \\
~\cite{Sleep10} & Filtering,Segmentation,\newline PCC,PLV & EEG Network & CNN+Attention & supervised & Sleep-EDF & 5-class & cross-subject & 81\%-85.8\% \\
~\cite{Sleep11} & - & Raw Segments & CNN-biLSTM & supervised & Sleep-EDF, \newline MASS,\newline SHHS & 5-class  & cross-subject & 83.9\%-86.7\% \\
~\cite{Sleep12} & feature extraction & Freq- features & CNN & supervised & Sleep-EDF & 5-class & cross-subject & 81.5\%-86.6\% \\
~\cite{Sleep13} & Segmentation & Raw Segments & CNN & supervised & Sleep-EDF, \newline SHHS & 5-class & cross-subject & 79.5\%-83.3\% \\
~\cite{Sleep14} & Segmentation,Network construction & Spatial-Temporal features & GCN+Attention & supervised & MASS, \newline ISRUC & 5-class  & cross-subject & 88.1\% \newline 90.5\% \\
~\cite{ZHANG2020105089} & Resampling,Filtering,HHT & Time-Frequency Image & 2D-CNN & supervised & SVU\_UCD, \newline MIT-BIH & 5-class  & cross-subject & 88.4\% \newline 87.6\% \\
~\cite{Sleep16} & Filtering & Spatial-Temporal features & CNN-GAT & supervised & Sleep-EDF & 5-class  & cross-subject & 81.6\%-84.9\% \\
~\cite{phan2019seqsleepnet} & Downsampling,STFT & Time-Freq Image & BiRNN\newline +Attention & supervised & MASS & 5-class & cross-subject & 87.1\% \\
~\cite{Sleep18} & Segmentation,\newline Normalization & Raw Segments & CNN-BiRNN & supervised & Sleep-EDF & 5-class  & cross-subject & 80.03\%-84.26\% \\
~\cite{Sleep19} & Segmentation,Multitaper Spectral Analysis & Raw,Spectrogram & CNN & supervised & MGH & 5-class & cross-subject & 85.76\% \\
~\cite{Sleep20} & Filtering,Segmentation & Raw Segments & CNN-CRF & supervised & Sleep-EDF & 5-class & cross-subject & 86.81\% \\
~\cite{Sleep21} & Segmentation & Raw Segments & CNN-LSTM & supervised & Sleep-EDF, \newline MASS & 5-class & cross-subject & 83.1\%-87.5\% \\
~\cite{vilamala2017deep} &  Multitaper Spectral \newline Estimation,Image\newline Construction & RGB Image & 2D-CNN & supervised & Sleep-EDF & 5-class  & cross-subject & 88\% \\
~\cite{Sleep23} & Segmentation & Raw Segments & CNN-BiLSTM & supervised & Sleep-EDF & 5-class & cross-subject & 85.07\%-87.02\% \\
~\cite{Sleep24} & Segmentation & Raw Segments & BiLSTM\newline +Attention & supervised & Sleep-EDF, \newline DRM-SUB & 5-class & cross-subject & 83.78\% \newline
81.72\% \\
~\cite{Sleep25} & Filtering,Segmentation,\newline Normalization,Hilbert & Stat. features,\newline Spectrogram & CNN & supervised & Sleep-EDF & 2-class & mixed-subject & 96.94\% \\
% ~\cite{Sleep26} & Resampling,Filtering & Raw Segments & CNN & supervised & private & 5-class & 80\% \\
~\cite{Sleep27} & Downsampling,\newline Segmentation & Time- \& Freq- features & CNN-BiLSTM & supervised & MASS & 5-class & cross-subject & 87.8\% \\
~\cite{Sleep28} & Downsampling,\newline Segmentation & Raw Segments & CNN-LSTM & supervised & Sleep-EDF & 5-class & cross-subject & 83.7\% \\
~\cite{yao2023cnntransformer} & Standardization & Raw Segments & CNN-Transformer & supervised & Sleep-EDF & 5-class & cross-subject & 79.5\% \\
~\cite{phan2018automatic} & Filtering,Windowing,DFT & Spectral Coefficients & GRU+Attention & supervised & Sleep-EDF & 5-class & cross-subject & 82.5\% \\
~\cite{tsinalis2016automatic} & - & Raw Segments & CNN & supervised & Sleep-EDF & 5-class  & cross-subject & 74\% \\
~\cite{chambon2018deep} & Filtering,Downsampling,\newline Normalization & Raw Segments & CNN & supervised & MASS & 5-class  & cross-subject & 82\% \\
~\cite{Sleep33} & - & Raw Segments & CNN & supervised & SHHS & 5-class  & cross-subject & 87\% \\
~\cite{Sleep34} & Segmentation & Raw Segments & CNN & supervised & Sleep-EDF & 5-class  & cross-subject & 81\% \\

\hline
\end{tabular}
\end{table*}


\begin{table*}[ht]
\renewcommand{\arraystretch}{1.2}
\caption*{(Continued) Summary of deep learning frameworks for sleep staging}
\footnotesize
\begin{tabular}{p{0.4cm}p{2.8cm}p{2cm}p{1.5cm}p{1.9cm}p{1.9cm}p{0.8cm}p{1.8cm}p{2cm}}
\hline
\textbf{Ref.} & \textbf{Preprocessing} & \textbf{Feature} & \textbf{Backbone} & \textbf{Training} & \textbf{Dataset} & \textbf{Task} & \textbf{Partitioning} & \textbf{Accuracy} \\
\hline
\cite{alvarez2021inter} & Downsampling,\newline Normalization,\newline Segmentation & Raw segments & CNN-LSTM & supervised & SHHS, ISRUC, DRM-SUB, SVUH\_UCD, HMC, Sleep-EDF& 5-class &cross-subject & $\kappa$=0.8 \\
~\cite{Sleep35} & Downsampling,STFT & Time-Frequency Image & CNN & supervised & Sleep-EDF, \newline MASS & 5-class  & cross-subject & 82.3\% \newline 83.6\% \\
~\cite{Sleep36} & Segmentation & Raw Segments & CNN & supervised & Sleep-EDF & 5-class  & cross-subject & 92.67\% \\
~\cite{Sleep37} & Z-norm & Raw Segments & CNN+Attention & supervised & Sleep-EDF & 5-class  & mixed-subject & 82.8\%-93.7\% \\
~\cite{Sleep38} & DE Calculation & DE matrix & CNN-GCN & supervised & Sleep-EDF, \newline ISRUC & 5-class  & cross-subject & 91.0\% \newline
87.4\% \\
~\cite{Sleep39} & Windowing,STFT,PSD Calculation & Spectral \& Temporal features & LSTM & supervised & MASS & 5-class  & cross-subject & 89.4\% \\
~\cite{Sleep44} & Filtering,Segmentation,\newline Normalization & Raw Segments & CNN & supervised & ISRUC & (2-5)-class & mixed-subject & 98.93\%-99.24\% \\
~\cite{mousavi2019deep} & Filtering,Windowing & Raw Segments & CNN & supervised & Sleep-EDF & (2-6)-class  & mixed-subject & 92.95\%-98.1\% \\
%~\cite{kostas2021bendr} & Resampling,Normalization & Raw Segments & Transformers & self-supervised & Sleep-EDF & 5-class  & cross-subject & AUROC=0.72 \\
% CASSL ~\cite{Sleep41} & Resampling,Filtering,Normalization & Frequency features & CNN & self-supervised & Sleep-EDF & 5-class  & 93\% \\
~\cite{yang2023self} & Filtering,Segmentation,\newline STFT & Raw Segments, \newline 2D Spectrogram & 2D-CNN & self-supervised & Sleep-EDF, \newline SHHS, \newline MGH & 5-class  & cross-subject & 72.03\%-86.90\% \\
~\cite{ye2021cosleep} & Filtering,Hilbert \newline Transform & Raw Segments,\newline 2D Spectrogram & CNN & self-supervised & Sleep-EDF, \newline ISRUC & 5-class  & cross-subject & 71.6\% \newline
57.9\% \\
%~\cite{Sleep46} & Segmentation,FFT & Time- \& Freq- features & Transformer & self-supervised & Sleep-EDF & 8-class  & mixed-subject & 90.12\% \\
~\cite{Sleep47} & Normalization & Raw Segments & CNN-RNN & self-supervised & Sleep-EDF, \newline ISRUC & 5-class  & mixed-subject & 80.0\% \newline
71.4\% \\
~\cite{Sleep48} & Segmentation & Raw Segments & Transformer & self-supervised & Sleep-EDF & 5-class  & cross-subject & 90\% \\
~\cite{kumar2022muleeg} & Resampling,Filtering,STFT & 2D Spectrogram & CNN & self-supervised & Sleep-EDF, \newline SHHS & 5-class  & cross-subject & 78.06\% \newline
 81.21\% \\
% Neuro2vec ~\cite{Sleep50} & Normalization,Segmentation,FT & Fourier Spectrum & Transformer & self-supervised & Sleep-EDF & 5-class  & 99.34\% \\
~\cite{Sleep51} & Segmentation,Channel Selection,Normalization & Raw Segments & CNN-RNN & self-supervised & Sleep-EDF, \newline ISRUC & 5-class  & mixed-subject & 80.8\% \newline
74.4\% \\
% SeqCLR ~\cite{Sleep52} & Resampling,Filtering & Raw Segments & CNN-RNN & self-supervised & Sleep-EDF & 5-class & 85.12\% \\
~\cite{Sleep53} & Segmentation,\newline Normalization & Raw Segments & CNN-RNN & self-supervised & Sleep-EDF, \newline ISRUC & 5-class  & cross-subject & 70.1\% \newline 53.6\% \\
~\cite{Sleep54} & Normalization,\newline Segmentation & Raw Segments & CNN-Transformer & self-supervised & Sleep-EDF, \newline MASS & 5-class  & cross-subject & 83.12\%
\newline 84.23\% \\
~\cite{Sleep55} & Segmentation,\newline Augmentation & Augmented \newline Segments & CNN+Attention & self-supervised & Sleep-EDF, \newline ISRUC & 5-class  & cross-subject & 82.0\% \newline
79.9\% \\
% TF-C ~\cite{Sleep56} & Augmentation & Time- And Frequency- features & CNN & self-supervised & Sleep-EDF & 5-class  & 78.24\% \\
% TS-TCC ~\cite{Sleep57} & Enhancement & Raw Segments & CNN-Transformer & self-supervised & Sleep-EDF & 5-class & 83\% \\
% ~\cite{Sleep58} & Segmentation,Filtering,Normalization & Raw Segments & CNN & self-supervised & PC18 & 5-class & 86\% \\
~\cite{banville2019self} & Filtering,Segmentation,\newline Normalization & Raw Segments & CNN & self-supervised & Sleep-EDF, \newline MASS & 5-class & cross-subject & 76-79\% \\
% ~\cite{Sleep60} & Filtering,Downsampling,Channel Selection & Relative Position,Temporal Shuffling; EEG-based CPC & CNN & self-supervised & PC18 & 5-class & 72.3\% \\
~\cite{jiang2021self} & Normalization,Filtering,\newline Segmentation & Raw Segments & CNN & self-supervised & Sleep-EDF & 5-class  & mixed-subject & 88.16\% \\
~\cite{ko2022eeg} & Filtering,Normalization,\newline Segmentation & Raw Segments, \newline 2D Spectrogram & CNN & self-supervised & Sleep-EDF & 5-class  & cross-subject & 86.8\% \\
~\cite{Sleep63} & Segmentation,\newline Normalization & Raw Segments & CNN & self-supervised & Sleep-EDF, \newline PC18 & 5-class  & cross-subject & 72.5\% \\
~\cite{Sleep64} & Normalization,\newline Segmentation & Raw Segments & CNN & semi-supervised & Sleep-EDF,\newline private & 5-class & mixed-subject & 91\% \\
~\cite{Sleep65} & Filtering,STFT & 2D Spectrogram & CNN & semi-supervised & Sleep-EDF,\newline private & 5-class  & mixed-subject & 84\% \\
~\cite{Sleep66} & Segmentation,FFT & 2D Spectrogram & 2D-CNN & semi-supervised & Sleep-EDF & 5-class & cross-subject & 89\% \\
% \hline
% \end{tabular}
% \end{table*}

% \begin{table*}[ht]
% \renewcommand{\arraystretch}{1.2}
% \caption*{(Continued) Sleep}
% \footnotesize
% \begin{tabular}{p{1.5cm}p{3cm}p{2cm}p{1.5cm}p{1.2cm}p{1.7cm}p{1.2cm}p{1.5cm}p{2cm}}
% \hline
% \textbf{Ref.} & \textbf{Preprocessing} & \textbf{Feature} & \textbf{Backbone} & \textbf{Training} & \textbf{Dataset} & \textbf{Task} & \textbf{Partitioning} & \textbf{Accuracy} \\
% \hline
~\cite{Sleep67} & Filtering,Normalization,\newline Segmentation & Raw Segments & CNN-BiGRU & semi-supervised & Sleep-EDF, \newline DRM-SUB & 5-class & mixed-subject & 82.3\% \newline
81.6\% \\
~\cite{Sleep68} & Multi-tapered Spectrogram Generation & Time-Frequency Image & GMM & semi-supervised & Sleep-EDF & 4-class & subject-specific & 73\% \\
~\cite{Sleep69} & Filtering & Raw Segments & CNN & semi-supervised & Sleep-EDF & 5-class & mixed-subject & 80\% \\
~\cite{Sleep70} & Filtering,Downsampling & Raw Segments & CNN & unsupervised & Sleep-EDF, \newline UCD & 5-class  & cross-subject & 83.4\% \newline 77.2\% \\
~\cite{Sleep71} & Segmentation,Filtering & Complex Values & CNN & unsupervised & UCD, \newline MIT-BIH & 5-class  & cross-subject & 87\% \\
~\cite{Sleep72} & Filtering,Segmentation,\newline feature extraction & Time-Frequency domain features & AE & unsupervised & Piryatinska & 3-class & mixed-subject & 80.4\% \\
~\cite{Sleep73} & Filtering,Downsampling,\newline Segmentation & Raw Segments & DBN & unsupervised & UCD & 5-class  & cross-subject & 91.31\% \\
~\cite{Sleep74} & Morlet Calculation,\newline Normalization & SSAE-based & DNN & unsupervised & Sleep-EDF & 5-class  & cross-subject & 78\% \\
~\cite{Sleep75} & Filtering,feature extraction & Time- \& Freq- features,Raw & DBN & unsupervised & UCD & 5-class  & cross-subject & 67.4\%-72.2\% \\
\hline
\end{tabular}
\end{table*}

\begin{table*}[ht]
\renewcommand{\arraystretch}{1.2}
\caption{Summary of deep learning frameworks for depression identification}
\label{tab:mdds}
\footnotesize
\begin{tabular}{p{0.4cm}p{2.8cm}p{2cm}p{1.5cm}p{1.9cm}p{1.9cm}p{0.8cm}p{1.8cm}p{2cm}}
\hline
\textbf{Ref.} & \textbf{Preprocessing} & \textbf{Feature} & \textbf{Backbone} & \textbf{Training} & \textbf{Dataset} & \textbf{Task} & \textbf{Partitioning} & \textbf{Accuracy} \\
\hline
~\cite{MDD2} & Filtering,ICA,STFT & Connectivity Graph & GCN-LSTM & supervised & PRED+CT, \newline MODMA & binary & cross-subject & 90.38\% \newline 90.57\% \\
~\cite{MDD3} & ICA,DWT,Segmentation & Frequency-domain matrix & CNN-LSTM & supervised & HUSM & binary & mixed-subject & 99.15\% \\
~\cite{MDD4} & Filtering,ICA,Z-norm,\newline STFT & 2D Spectrogram & CNN-LSTM & supervised & HUSM & binary & mixed-subject & 99.9\% \\
~\cite{sharma2021dephnn} & ICA,FFT,Windowing & Time-Frequency features & CNN-LSTM & supervised & private & binary & mixed-subject & 99.1\% \\
~\cite{seal2021deprnet} & Filtering,ICA,Segmentation & Raw Segments & CNN & supervised & private & binary & mixed-subject & 99.37\% \\
~\cite{MDD7} & Filtering,Segmentation & Raw Segments & CNN-Transformer & supervised & HUSM,\newline
private & binary & cross-subject & 93.7\% \newline 96.2\% \\
~\cite{MDD8} & Filtering,ICA,Band \newline Filter,CSP & Raw Segments & Transformer & supervised & private & binary & mixed-subject & 92.25\% \\
~\cite{MDD9} & Z-norm,Welch & Psd features & CNN-GRU+Attention & supervised & MODMA,\newline EDRA & binary & mixed-subject & 97.56\% \newline
98.33\% \\
~\cite{MDD10} & Downsampling,Z-norm,\newline Segmentation & Raw Segments & 2D-CNN & supervised & private & 3-class & mixed-subject & 79.08\% \\
~\cite{MDD11} & Filtering,Downsampling,\newline Normalization & Raw Segments & CNN-LSTM & supervised & private & binary & cross-subject & 94.69\% \\

~\cite{MDD13} & Filtering,ICA,Wpt & Brain Network & GCN & supervised & MODMA, \newline EDRA, \newline HUSM & NaN & mixed-subject & 91.11\%-93.75\% \\
~\cite{MDD14} & Baseline Removal,\newline Detrending,Filtering,STFT & Time-Frequency Image & GCN & supervised & HUSM, \newline MODMA & binary & cross-subject & 99.19\% \newline 95.53\% \\
~\cite{MDD15} & Normalization,\newline Segmentation,Network construction & Node Feature \newline matrix,Adjacency matrix & GNN & supervised & MODMA & binary & cross-subject & 84.91\% \\
~\cite{MDD16} & Filtering,DE Calculation & Differential Entropy,Adjacency matrix & GCN & supervised & PRED+CT, \newline MODMA & binary & cross-subject & 83.17\% \newline 92.87\% \\
~\cite{MDD17} & Filtering,ICA,CAR,\newline U-NET & Multi-scale Saliency-encoded Spectrogram & CNN & supervised & HUSM & binary & cross-subject & 99.22\% \\
~\cite{MDD18} & Downsampling,Filtering,\newline Segmentation & Raw Segments & CNN-RSE & supervised & private & binary & mixed-subject & 98.48\% \\
~\cite{MDD19} & Segmentation & Raw Segments & 2D-CNN & supervised & HUSM, \newline private & 3-class & mixed-subject & 98.59\% \\
~\cite{MDD20} & Filtering,Min-max Norm,Segmentation,\newline Welch & Asymmetry \newline matrix Images & 2D-CNN & supervised & HUSM & binary & mixed-subject & 98.85\% \\
~\cite{MDD21} & Filter,Image Construction & 2D Image & CNN-LSTM & supervised & HUSM & binary & cross-subject & 99.245\% \\
~\cite{MDD22} & Denoising,Filtering,STFT & 2D Spectrogram & 2D-CNN & supervised & HUSM & binary & mixed-subject & 99.58\% \\
~\cite{MDD23} & Band-pass Filter & Frequency bands & 2D-CNN & supervised & HUSM & binary & mixed-subject & 96.97\% \\
~\cite{MDD24} & Filtering,MPWD,Network construction & Adjacency matrix Of Fdmb Network & 2D-CNN & supervised & HUSM & binary & mixed-subject & 97.27\% \\
~\cite{MDD25} & MSEC,Segmentation & Raw Segments & CNN,CNN-LSTM & supervised & HUSM & binary & mixed-subject & 98.32\% \\
~\cite{MDD26} & Filtering,PLV,Welch & Multilayer Brain Network & GCN & supervised & HUSM & binary & mixed-subject & 99.29\% \\
~\cite{MDD27} & ICA,Rereferencing,Filtering & 2D Image & 2D-CNN & supervised & HUSM & binary & mixed-subject & 99.11\% \\
~\cite{MDD28} & Filtering,Z-norm,\newline Segmentation & Connectivity \newline matrix & 2D-CNN+Attention & supervised & HUSM & binary & cross-subject & 91.06\% \\
~\cite{MDD29} & ICA,Z-norm,Band Filter & Frequency bands & CNN & supervised & HUSM & binary & cross-subject & 99.6\% \\
~\cite{MDD30} & Filtering,ASR & Raw Segments & Inception & supervised & HUSM & binary & cross-subject & 91.67\% \\
~\cite{MDD31} & Filtering,CWT,WCOH & RGB Image & 2D-CNN & supervised & HUSM & binary & mixed-subject & 98.1\% \\
~\cite{MDD32} & Filtering,Windowing,\newline SWC,PLV & P-mSWC & 2D-CNN & supervised & HUSM, \newline PRED+CT & binary & mixed-subject & 93.93\%- 99.87\% \\
~\cite{acharya2018automated} & Filtering,Z-norm & Raw Segments & CNN & supervised & private & binary & mixed-subject & 95.96\% \\

\hline
\end{tabular}
\end{table*}

\begin{table*}[ht]
\renewcommand{\arraystretch}{1.2}
\caption*{(Continued) Summary of deep learning frameworks for depression identification}
\footnotesize
\begin{tabular}{p{0.4cm}p{2.8cm}p{2cm}p{1.5cm}p{1.9cm}p{1.9cm}p{0.8cm}p{1.8cm}p{1.5cm}}
\hline
\textbf{Ref.} & \textbf{Preprocessing} & \textbf{Feature} & \textbf{Backbone} & \textbf{Training} & \textbf{Dataset} & \textbf{Task} & \textbf{Partitioning} & \textbf{Accuracy} \\
\hline
~\cite{MDD34} & Filtering,STFT & 2D Spectrogram & 2D-CNN & supervised & private & binary & mixed-subject & 96.43\% \\
~\cite{MDD35} & Filtering,ICA,Segmentation & Mixed Feature matrix & CNN & supervised & private & binary & mixed-subject & 94.13\% \\
~\cite{li2019depression} & ICA,LMS,AR,Hjorth & 2D Image & CNN & supervised & private & binary & cross-subject & 84.75\% \\
~\cite{MDD37} & Filtering,Segmentation & Raw Segments & CNN & supervised & private & binary & mixed-subject & 75.29\% \\
~\cite{MDD38} & Filtering,Segmentation,\newline PLV,PLI & Connectivity \newline matrix & 2D-CNN & supervised & private & binary & cross-subject & 80.74\% \\
~\cite{MDD39} & Filtering,PLI & Connectivity \newline matrix & 2D-CNN & supervised & private & binary & mixed-subject & 67.67\% \\
~\cite{khan2021automated} & Denoising,Segmentation,\newline PDC matrix Calculation & 3D CPC & 3D-CNN & supervised & private & binary & cross-subject & 100\% \\
~\cite{ay2019automated} & Manual Denoising, \newline Filtering& Raw Segments & CNN-LSTM & supervised & private & binary & mixed-subject & 99.12\% \\
~\cite{MDD42} & Filtering,Segmentation & Raw Segments & CNN-RNN & supervised & private & binary & mixed-subject & 99.66\% \\
~\cite{MDD43} & Filtering,Image Construction & Spatial-Temporal Image & 2D-CNN & supervised & private & binary & mixed-subject & 92.66\% \\
~\cite{MDD44} & Filtering,DWT & Wavelet features & BiLSTM & supervised & private & binary & mixed-subject & 99.66\% \\
~\cite{MDD45} & Band Filter, \newline Normalization & Raw Segments & CNN & supervised & private & binary & mixed-subject & 98.13\% \\
~\cite{MDD46} & Filtering,ICA,Hanning & 2D Frames & 2D-CNN & supervised & private & binary & cross-subject & 77.2\% \\
~\cite{MDD47} & Filtering,LMS & Raw Segments & CNN-LSTM & supervised & MODMA & binary & cross-subject & 95.1\% \\
~\cite{MDD48} & Filter,Image Construction & 2D Image & DAN & supervised & MODMA & binary & cross-subject & 77\% \\
~\cite{MDD49} & Filtering,Windowing,PLI & Time- \& Spatial- domain features & CNN-RNN & supervised & MODMA & binary & mixed-subject & 96.33\% \\
~\cite{MDD50} & Filtering,Z-norm & Time-Frequency features & 2D-CNN & supervised & PRED+CT & binary & mixed-subject & 93.33\% \\
~\cite{MDD51} & ICA,Z-norm & Raw Segments & CNN-LSTM & supervised & PRED+CT & binary & mixed-subject & 99.07\% \\
~\cite{li2019eeg} & Filtering,ICA,Power Spectrum Calculation & Topographical Activity Map,\newline Frequency bands & 2D-CNN & supervised & private & binary & cross-subject & 85.62\% \\
~\cite{sam2023depression} & Filtering,ICA & Spike Trains & SNN-LSTM & supervised & PRED+CT & 4-class & cross-subject & 98\% \\
~\cite{song2022lsdd} & Filtering,downsampling & Frequency bands & CNN-LSTM & supervised & private & binary & cross-subject & 95\% \\
~\cite{zhu2022eeg} & Filtering,feature extraction & Adjacency matrix,Node features &	GCN	& supervised & private & binary & cross-subject & 97\%\\
~\cite{wang2023depression} & Image construstion & 2D Image & 2D-CNN & supervised & MODMA & binary & mixed-subject & 74\%\\
~\cite{MDD12} & Filtering,ICA & Graph & GCN & self-supervised & MODMA, \newline EDRA & binary & cross-subject & 99.19\% \newline 98.38\% \\
~\cite{MDD1} & ICA,Filtering,\newline DE Calculation & Differential Entropy & GCN & semi-supervised & MODMA & binary & cross-subject & 92.23\% \\
~\cite{zhu2019multimodal} & ICA,Filtering & AE-based & DNN & unsupervised & private & binary & cross-subject & 83.42\% \\
~\cite{shah2019deep} & ICA,Segmentation & Spike Trains & SNN & unsupervised & private & binary & mixed-subject & 72.13\% \\
~\cite{li2023gcns} & Filtering,DWT,PCC & Adjacency matrix & GCN & unsupervised & MODMA & binary & mixed-subject & 97\%\\
\hline
\end{tabular}
\end{table*}

\begin{table*}[ht]
\renewcommand{\arraystretch}{1.2}
\caption{Summary of deep learning frameworks for schizophrenia identifiaction}
\label{tab:schis}
\footnotesize
\begin{tabular}{p{0.4cm}p{2.8cm}p{2cm}p{1.5cm}p{1.9cm}p{1.9cm}p{0.8cm}p{1.8cm}p{2cm}}
\hline
\textbf{Ref.} & \textbf{Preprocessing} & \textbf{Feature} & \textbf{Backbone} & \textbf{Training} & \textbf{Dataset} & \textbf{Task} & \textbf{Partitioning} & \textbf{Accuracy} \\
\hline
~\cite{phang2019multi} & Connectivity Measures,Complex Network construction & VAR,PDC,CN & CNN & supervised & MHRC & binary & cross-subject & 91.69\% \\
~\cite{SZ2} & Z-norm,Segmentation & Raw Segments & CNN & supervised & CeonRepod & binary & mixed-subject & 98.07\% \\
~\cite{SZ3} & Segmentation,\newline Margenau–Hill & Time-Frequency Image & CNN & supervised & MHRC, \newline CeonRepod, \newline NIMH & binary & mixed-subject & 96.35\%-99.75\% \\
~\cite{SZ4} & Connectivity Networks Construction & WOC-Based \newline features & CNN & supervised & MHRC & binary & cross-subject & 90\% \\
~\cite{SZ5} & Filtering,Segmentation,\newline Welch Method & Spectrum matrix & CNN & supervised & private & binary & cross-subject & 91.12\% \\
~\cite{SZ6} & Filtering & Raw Segments & CNN & supervised & CeonRepod & binary & mixed-subject & 98.05\% \\
~\cite{10023506} & Filtering,Segmentation,\newline ASR,ICA & Connectivity \newline features & CNN & supervised & CeonRepod & binary & mixed-subject &  99.84\% \\
~\cite{SZ8} & CWT,STFT,SPWVD & Scalogram,TFR,\newline Spectrogram & CNN & supervised & NIMH & binary & mixed-subject &  93.36\% \\
~\cite{SZ9} & Filtering,Segmentation,\newline Z-norm & Raw Segments & CNN & supervised & CeonRepod & binary & mixed-subject & 99.18\% \\
~\cite{SZ10} & Filtering,ICA & Trend Time \newline Series & CNN & supervised & CeonRepod & binary & cross-subject & 93\% \\
~\cite{SZ11} & Mspca,Filtering,Multitaper & Frequency features & CNN & supervised & CeonRepod & binary & mixed-subject & 98.76\% \\
~\cite{SZ12} & Filtering,Segmentation,\newline Connectivity Measures & FC matrix & CNN & supervised & MHRC & binary & cross-subject & 94.11\% \\
~\cite{SZ13} & Filtering,Windowing,\newline Z-norm,CWT & 2D Scalogram & CNN & supervised & CeonRepod,\newline NIMH & binary & mixed-subject & 99\%\newline 96\% \\
~\cite{SZ14} & Re-Referencing,\newline Filtering,Segmentation & Raw Segments & 2D-CNN & supervised & private & 3-class & cross-subject & 81.6\%-99.2\%\\
~\cite{SZ15} & Filtering,Segmentation,FFT & Spectral Power,\newline Variance,Mobility,\newline Complexity,Mean Spectral Amp. & 2D-CNN & supervised & MHRC, \newline CeonRepod & binary & mixed-subject & 94.08\%-98.56\% \\
~\cite{SZ16} & Segmentation,STFT & 2D Spectrogram & 2D-CNN & supervised & MHRC,\newline CeonRepod & binary & mixed-subject & 95\% \newline 97\%\\
~\cite{aslan2022deep} & CWT & 2D Scalogram & 2D-CNN & supervised & MHRC,\newline CeonRepod & binary & mixed-subject & 98\% \newline 99.5\%
 \\
~\cite{zulfikar2022empirical} & Segmentation,EMD,HHT & Hilbert Spectrum & 2D-CNN & supervised & MHRC,\newline CeonRepod & binary & mixed-subject & 96.02\% \newline 98.2\%\\
~\cite{9713847} & WT,1D-LBP,ELM-AE & EEG Image & 2D-CNN & supervised & MHRC & binary & mixed-subject & 97.73\% \\
~\cite{electronics11142265} & Z-norm & EEG Image & 2D-CNN & supervised & NIMH & binary & mixed-subject & 93.2\% \\
~\cite{SZ21} & Filtering & Image matrix & 2D-CNN & supervised & NIMH & binary & mixed-subject & 98.84\% \\
~\cite{SZ22} & Filtering,CWT & 2D Scalogram & 2D-CNN & supervised & CeonRepod & binary & mixed-subject & 99\% \\
~\cite{SZ23} & Filtering,Segmentation & Raw Segments & 2D-CNN & supervised & NIMH, \newline private & binary & cross-subject & 80\% \\
~\cite{SZ24} & Baseline Correction,\newline Filtering,Segmentation & Time-Frequency features & 2D-CNN & supervised & NIMH & binary & mixed-subject & 92\% \\
~\cite{SZ25} & Segmentation,PCC & Correlation \newline matrix & 2D-CNN & supervised & MHRC & binary & mixed-subject & 90\% \\
~\cite{SZ26} & Segmentation,Phase \newline Reconstruction & RPS Portrait & 2D-CNN & supervised & CeonRepod & binary & mixed-subject & 99.37\% \\
~\cite{SZ27} & Filtering,Interpolation & EEG Image & 2D-CNN & supervised & NIMH & binary & mixed-subject & 99.23\% \\
~\cite{SZ28} & Normalization,DSTFT & DSTFT \newline Spectrogram & 2D-CNN & supervised & MHRC & binary & cross-subject & 83\% \\
~\cite{SZ29} & LSDl & 2D Spectrogram,\newline Scalogram & 2D-CNN & supervised & MHRC & binary & mixed-subject & 98.3\% \\
~\cite{SZ30} & Segmentation,Feature Selection & Nonlinear features & 2D-CNN & supervised & CeonRepod & binary & mixed-subject & 95.85\% \\
~\cite{SZ31} & Filtering,CWT,CMI & Connectivity \newline matrix & 3D-CNN & supervised & MHRC & binary & cross-subject & 97.74\% \\
~\cite{SZ32} & Normalization,DAF & 2D Image & CNN,\newline Transformer & supervised & CeonRepod & binary & mixed-subject & 98.32\%-99.04\% \\
~\cite{SZ33} & Filtering,Segmentation,\newline Z-norm & PSD features & CNN\newline CNN-LSTM & supervised & private & binary & cross-subject & 75.9\% \newline 71.5\%
 \\
~\cite{SZ34} & Filtering,Min-Max Norm & Raw & CNN-LSTM & supervised & private & binary & cross-subject & 89.98\% \\
\hline
\end{tabular}
\end{table*}

\begin{table*}[ht]
\renewcommand{\arraystretch}{1.2}
\caption*{(Continued) Summary of deep learning frameworks for schizophrenia identification}
\footnotesize
\begin{tabular}{p{0.4cm}p{2.8cm}p{2cm}p{1.5cm}p{1.9cm}p{1.9cm}p{0.8cm}p{1.8cm}p{2cm}}
\hline
\textbf{Ref.} & \textbf{Preprocessing} & \textbf{Feature} & \textbf{Backbone} & \textbf{Training} & \textbf{Dataset} & \textbf{Task} & \textbf{Partitioning} & \textbf{Accuracy} \\
\hline
~\cite{sun2021hybrid} & Filtering,Segmentation,\newline Baseline Correction,\newline Ocular Correction & FuzzyEn RGB Image & CNN-LSTM & supervised & private & binary & mixed-subject & 99.22\% \\
~\cite{SZ36} & Filtering,Segmentation & Raw Segments & CNN-LSTM & supervised & MHRC, \newline CeonRepod & binary & cross-subject & 91\% \newline 96.1\% \\
~\cite{SZ37} & MSST & Time-Frequency Feature Image & CNN-LSTM & supervised & CeonRepod & binary & mixed-subject & 84.42\% \\
~\cite{SZ38} & Filtering,TE & 2D Image & CNN-LSTM & supervised & CeonRepod & binary & mixed-subject & 99.9\% \\
~\cite{SZ39} & Artifact Removal,\newline Filtering & Raw & CNN-LSTM & supervised & NIMH & binary & cross-subject & 98.2\% \\
~\cite{SZ40} & Segmentation,Z-norm & Raw Segments & CNN-LSTM & supervised & CeonRepod & binary & mixed-subject & 99.25\% \\
~\cite{SZ41} & Filtering,PCA,ICA & Raw,features & CNN-TCN & supervised & CeonRepod & binary & mixed-subject & 99.57\% \\
~\cite{SZ42} & Filtering, feature extraction & Frequency features & DNN & supervised & private & binary & mixed-subject & 97.5\% \\
~\cite{SZ43} & Connectivity Measures,Complex Network construction & DC,CN & DNN-DBN & Supervised & MHRC & binary & cross-subject & 95\% \\
~\cite{SZ44} & Filtering,ICA & PLI,PCI & GNN & Supervised & private & binary & cross-subject & 84.17\% \\
~\cite{SZ45} & Filtering,TVD & Time- and Nonlinear features & LSTM & Supervised & CeonRepod & binary & mixed-subject & 99\% \\
~\cite{SZ46} & Dimension Reduction & End-to-end & RNN-LSTM & Supervised & MHRC & binary & mixed-subject & 98\% \\
~\cite{SZ47} & Filtering,Normalization & Spatial Feature matrix & Transformer & Supervised & CeonRepod & binary & mixed-subject & 98.99\% \\
~\cite{alves2022eeg} & Filtering,Segmentation,\newline Connections calculation & Connection \newline matrix & 2D-CNN & supervised & private & binary & mixed-subject & 100\%\\
~\cite{SZ48} & Z-norm,Filtering & AE-based & CNN & Unsupervised & CeonRepod & binary & cross-subject & 81.81\% \\
~\cite{SZ49} & Segmentation & SAE-based & DNN & Unsupervised & CeonRepod & binary & mixed-subject & 97.95\% \\
~\cite{SZ50} & Filtering & AE-based & DNN & Unsupervised & MHRC, \newline CeonRepod, \newline NIMH & binary & mixed-subject & 95.01\%-99.99\% \\
\hline
\end{tabular}
\end{table*}




\begin{table*}[ht]
\renewcommand{\arraystretch}{1.2}
\caption{Summary of deep learning frameworks for Alzheimer's Disease Diagnosis}
\label{tab:ads}
\footnotesize
\begin{tabular}{p{0.4cm}p{2.8cm}p{2cm}p{1.5cm}p{1.9cm}p{1.9cm}p{0.8cm}p{1.8cm}p{2cm}}
\hline
\textbf{Ref.} & \textbf{Preprocessing} & \textbf{Feature} & \textbf{Backbone} & \textbf{Training} & \textbf{Dataset} & \textbf{Task} & \textbf{Partitioning} & \textbf{Accuracy} \\
\hline
~\cite{AD1} & Filtering,Segmentation,\newline Connections Calculation & Connection \newline matrix & 2D-CNN & Supervised & private & binary & mixed-subject & 100\% \\
~\cite{AD2} & Filtering,Segmentation & Raw Segments & 2D-CNN & Supervised & FSA\_Alzheimer’s & binary & mixed-subject & 97.9\% \\
~\cite{shan2022spatial} & Filtering,Segmentation,\newline Network construction & Adjacency matrix,Segments & GCN & Supervised & private & binary & mixed-subject & 92.3\% \\
~\cite{AD4} & Filtering,Segmentation & Raw Segments & 2D-CNN & Supervised & private & binary & mixed-subject & 69.03\%-85.78\% \\
~\cite{AD5} & Filtering,Segmentation,\newline CWT & Time-Frequency features & 2D-CNN & Supervised & private & binary\newline 3-class & cross-subject & 85\%\newline82\% \\
~\cite{AD6} & Filtering,FFT & 2D Spectrograms & 2D-CNN & Supervised & private & binary\newline 3-class & mixed-subject & 97.11\%\newline95.04\% \\
~\cite{AD7} & Filtering,FFT & Frequency-domain features & 2D-CNN & Supervised & private & binary& - & 93.7\% \\
~\cite{AD8} & Filtering,Segmentation,\newline ICA,CWT & RGB Image & 2D-CNN & Supervised & private & 3-class & mixed-subject & 98.9\% \\
~\cite{AD9} & Filtering,Downsampling,\newline ICA & Frequency-domain features & CNN & Supervised & Fiscon & 3-class & mixed-subject & 97.1\% \\
~\cite{AD10} & Normalization,\newline Segmentation,DWT & 2D Spectrograms & CNN & Supervised & AD-59 & 3-class & cross-subject & 98.84\% \\
% ~\cite{AD11} & Gaussian Filtering & Multimodal features & CNN-DBN & Supervised & private & AD-MCI-HD-HC & 92.5\% \\
~\cite{AD12} & Filtering,Segmentation,FT & PSD Image & 2D-CNN & Supervised & private & Binary\newline 3-class & mixed-subject & 84.62\%-92.95\% \newline83.33\% \\
~\cite{AD13} & Filtering,EMD & Time-Frequency features & CNN & Supervised & private & Binary\newline 3-class & mixed-subject & 99.3\%-99.9\%\newline94.8\% \\
~\cite{AD14} & Filtering,Segmentation,RP & Frequency-domain features & DNN & Supervised & private & binary & cross-subject & 75\% \\
~\cite{AD15} & Denoising & AE-Based & RBM & Unsupervised & private & binary & mixed-subject & 92\% \\
~\cite{AD16} & Filtering,ICA,\newline Morlet Wavelet& VAE-Based & VAE & Unsupervised & private & binary & cross-subject & 98.1\% \\
~\cite{AD17} & Filtering,Segmentation,\newline CWT & SAE-Based & MLP-NN & Unsupervised & private & binary & cross-subject & 88\% \\
\hline
\end{tabular}
\end{table*}

\begin{table*}[ht]
\renewcommand{\arraystretch}{1.2}
\caption{Summary of deep learning frameworks for Parkinson's Disease Diagnosis}
\label{tab:pds}
\footnotesize
\begin{tabular}{p{0.4cm}p{2.8cm}p{2cm}p{1.5cm}p{1.9cm}p{1.9cm}p{0.8cm}p{1.8cm}p{2cm}}
\hline
\textbf{Ref.} & \textbf{Preprocessing} & \textbf{Feature} & \textbf{Backbone} & \textbf{Training} & \textbf{Dataset} & \textbf{Task} & \textbf{Partitioning} & \textbf{Accuracy} \\
\hline
~\cite{PD1} & Segmentation,Embedding Reconstruction & Reconstructed Segments & CNN-LSTM & supervised & UNM & binary & mixed-subject & 99.22\% \\
~\cite{PD2} & Gabor Transform & 2D Spectrograms & 2D-CNN & supervised & UCSD & 3-class & mixed-subject & 92.6\%-99.46\% \\
~\cite{PD3} & SPWVD,Artifact Removal,Segmentation & TFR & 2D-CNN & supervised & UCSD, \newline private & binary & mixed-subject & 99.84\%-100\% \\
% PDRNN ~\cite{PD4} & Filtering,Artifact rejection,Segmentation & Raw Segments & LSTM & supervised & private & PD Vs. HC & 88.31\% \\
~\cite{PD5} & Denoising,TQWT,WPT & Time-Frequency features & CNN & supervised & private & 3-class & mixed-subject & 92.59\%-99.92\% \\
~\cite{PD6} & Artifact rejection,\newline Filtering,Segmentation & Raw Segments & CNN-RNN & supervised & private & binary & cross-subject & 82.89\% \\
~\cite{shaban2022resting} & CWT,Segmentation & 2D Image & CNN & supervised & UCSD & 3-class & mixed-subject & 99.6\%-99.9\% \\
~\cite{PD8} & ICA,Filtering,P-Welch & PSD Image & 2D-CNN & supervised & private & binary & mixed-subject & 99.87\% \\
~\cite{PD9} & Artifacts Removal,\newline Filtering,Segmentation & DC Image & 2D-CNN & supervised & private & binary & mixed-subject & 99.62\% \\
% ~\cite{PD10} & CWT & Time-Frequency features & 2D-CNN & supervised & UCSD & binary & 99.9\% \\
~\cite{PD11} & CWT,VMD & Time-Frequency features & 2D-CNN & supervised & private & binary & mixed-subject & 92\%-96\% \\
~\cite{PD12} & Filtering,Segmentation & Raw Segments & ANN & supervised & UCSD & binary & mixed-subject & 98\% \\
~\cite{oh2020deep} & Artifact Removal,\newline Filtering & Raw Segments & CNN & supervised & private & binary & mixed-subject & 88.25\% \\
~\cite{PD14} & Filtering,Z-norm & Raw Segments & CNN & supervised & UNM, \newline UI & binary & cross-subject & 82.8\% \\
~\cite{PD15} & Artifact rejection,\newline Filtering,Normalization,\newline Segmentation & Raw Segments & CNN-GRU & supervised & private & binary & mixed-subject & 99.2\% \\
~\cite{PD16} & Segmentation & Raw Segments & CNN-LSTM & supervised & private & binary & mixed-subject & 96.9\% \\
~\cite{PD17} & FFT & 2D Spectrograms & CNN-LSTM & supervised & private & binary & mixed-subject & 99.7\% \\
~\cite{PD18} & Artifact rejection,\newline Filtering,Segmentation & Functional connectivity matrix & GCN & supervised & private & binary & mixed-subject & 90.2\% \\
\hline
\end{tabular}
\end{table*}




\begin{table*}[ht]
\renewcommand{\arraystretch}{1.2}
\caption{Summary of deep learning frameworks for ADHD identification}
\label{tab:adhds}
\footnotesize
\begin{tabular}{p{0.4cm}p{2.8cm}p{2cm}p{1.5cm}p{1.9cm}p{1.9cm}p{0.8cm}p{1.8cm}p{1.5cm}}
\hline
\textbf{Ref.} & \textbf{Preprocessing} & \textbf{Feature} & \textbf{Backbone} & \textbf{Training} & \textbf{Dataset} & \textbf{Task} & \textbf{Partitioning} &
\textbf{Accuracy} \\
\hline
~\cite{chen2019use} & segment screening & PSD & CNN & Supervised & private & Binary & mixed-subject & 90.29\% \\
~\cite{dubreuil2020deep} & Filtering,Segmentation,\newline wavelet transform & Spectrogram & CNN & Supervised & private & Binary & cross-subject & 88\% \\
~\cite{MOGHADDARI2020105738} & Resampling,filtering,\newline ASR,windowing,\newline Freq. bands separation & Frequency bands, RGB Images & CNN & Supervised & ADHD-Child & Binary & cross-subject & 98.48\% \\
~\cite{tosun2021effects} & PSD & PSD,SE & LSTM & Supervised & private & Binary & mixed-subject & 92.15\% \\
~\cite{ADHD5} & Filtering,Segmentation,\newline ICA,segment screening & End-to-end & CNN & Supervised & private & 3-class & mixed-subject & 99.46\% \\
~\cite{ADHD6} & FIR,filtering,ICA,\newline Segmentation & Dynamic connectivity tensor (DCT) & ConvLSTM +Attention & Supervised & ADHD-Child & Binary & mixed-subject & 99.75\% \\
~\cite{nouri2024detection} & Re-referencing,filtering,\newline Baseline rejection,\newline downsampling,Segmentation & PSD & CNN & Supervised & ADHD-Child & Binary & mixed-subject & 94.52\% \\
~\cite{ADHD8} & Filtering,Segmentation,\newline CWT & Time-Frequency Image & ConvMixer,\newline ResNet50,\newline ResNet18 & Supervised & ADHD-Child & Binary & mixed-subject & 72.58\% \\
\hline
\end{tabular}
\end{table*}
\begin{figure*}
    \centering
	\large
    \setlength{\tabcolsep}{0pt}
    \renewcommand{\arraystretch}{0}
    \newcommand{\imgsize}{.125\textwidth}
    \newcommand{\imgsizehalf}{.0625\textwidth}

    \newcommand{\row}[2]{%
        & \multirow{2}{*}[#2]{\includegraphics[width=\imgsize]{images/light/image_#1.png}}%
        & \multirow{2}{*}[#2]{\includegraphics[width=\imgsize]{images/light/irradiance_#1.png}}%
        & \includegraphics[width=\imgsizehalf]{images/light/texture_#1.png}%
        & \multirow{2}{*}[#2]{\includegraphics[width=\imgsize]{images/light/zest_#1.png}}%
        & \multirow{2}{*}[#2]{\includegraphics[width=\imgsize]{images/light/zest_#1_irradiance.png}}%
        & \multirow{2}{*}[#2]{\includegraphics[width=\imgsize]{images/light/mocka_v3_a100skip_nomaskp_A5_#1.png}}%
        & \multirow{2}{*}[#2]{\includegraphics[width=\imgsize]{images/light/mocka_v3_a100skip_nomaskp_A5_#1_irradiance.png}}%
        & \multirow{2}{*}[#2]{\includegraphics[width=\imgsize]{images/light/mocka_v3_a100_512px_#1.png}}%
        & \multirow{2}{*}[#2]{\includegraphics[width=\imgsize]{images/light/mocka_v3_a100_512px_#1_irradiance.png}} \\%
        
        &&&\includegraphics[width=\imgsizehalf]{images/light/overlay_#1.png} \\
    }
    
	\resizebox{.99\linewidth}{!}{%
    \begin{tabular}{H cc S{.5ex} c S{.5ex} cc S{1ex} cc S{1ex} cc}
        & \multicolumn{2}{c}{Image \& Irradiance} & Source & \multicolumn{2}{c}{ZeST} & \multicolumn{2}{c}{ours w/o \irra} & \multicolumn{2}{c}{\textbf{ours}} \\[.7ex]

        \row{8035c3ed-image_031_G_st_fabric_065_000}{11.2ex}\\[.2ex] 
        \row{4ac74304-image_043_A_tc_wood_005}{11.2ex}\\[.2ex] 
        \row{2414415f-image_039_D_acg_tiles_066}{10ex}\\[.2ex]

        \row{e3a0ce32-image_007_C_acg_painted_plaster_007}{11ex}

        \cmidrule[1pt](lr){2-3} \cmidrule[1pt](lr){5-6} \cmidrule[1pt](lr){7-8} \cmidrule[1pt](lr){9-10}
        & \target & $\phi_\irra(\target)$ & & $\hat{\image}$ & $\phi_\irra(\hat{\image})$ & $\hat{\image}$ & $\phi_\irra(\hat{\image})$ & $\hat{\image}$ & $\phi_\irra(\hat{\image})$ \\
    \end{tabular}}
    \vspace{-2mm}
    \caption{Reliance on irradiance. We compare the irradiance of the input image \target directly against the irradiance estimated from the images $\hat{\image}$ edited by ZeST and our model, both with or without irradiance \irra. Our approach better preserves the illumination of the input image, thanks to the irradiance information.}
    \label{fig:irradiance}
    \vspace{-2mm}
\end{figure*}
\begin{figure}
    \centering
    \setlength{\tabcolsep}{1pt}
    \renewcommand{\arraystretch}{0}
    \newcommand{\imgsize}{.10\textwidth}
    \newcommand{\imgsizehalf}{.05\textwidth}

    \newcommand{\row}[2]{%
        & \includegraphics[width=\imgsizehalf]{images/dropout/texture_#1.png}%
        & \multirow{2}{*}[#2]{\includegraphics[width=\imgsize]{example-image-a}}%
        & \multirow{2}{*}[#2]{\includegraphics[width=\imgsize]{images/dropout/ours_#1_drop_E_masked.png}}%
        & \multirow{2}{*}[#2]{\includegraphics[width=\imgsize]{example-image-a}}%
        & \multirow{2}{*}[#2]{\includegraphics[width=\imgsize]{images/dropout/ours_#1_masked.png}}%
        & \multirow{2}{*}[#2]{\includegraphics[width=\imgsize]{example-image-a}}%
        & \multirow{2}{*}[#2]{\includegraphics[width=\imgsize]{images/dropout/ours_#1_drop_E.png}}%
        & \multirow{2}{*}[#2]{\includegraphics[width=\imgsize]{example-image-a}}%
        & \multirow{2}{*}[#2]{\includegraphics[width=\imgsize]{images/dropout/ours_#1.png}}\\

        & \includegraphics[width=\imgsizehalf]{images/dropout/mask_#1.png} &&&& \\
    }

	\resizebox{.99\linewidth}{!}{%
    \begin{tabular}{c cS{0.2ex} HcHc | HcHc}


        & & \multicolumn{4}{c}{target: $\target\cdot\mask$} & \multicolumn{4}{c}{target: $\target$}\\

        \cmidrule[1pt](lr){3-6} \cmidrule[1pt](lr){7-10}
        & Src. & $\xx \smallsetminus \{ \normal \}$ & $\xx \smallsetminus \{ \irra \}$ & $\xx \smallsetminus \{ \normal,\irra \}$ & $\xx$ & $\xx \smallsetminus \{ \normal \}$ & $\xx \smallsetminus \{ \irra \}$ & $\xx \smallsetminus \{ \normal,\irra \}$ & $\xx$ \\[.7ex]

        \row{e3a0ce32-image_007_B_ms_paving_stones_092__grass_001}{6.5ex}\\
        \row{4ac74304-image_043_A_acg_wood_014}{6.6ex}\\[.2ex]

         & \multicolumn{1}{r}{$\xx=$} &  & $\{\target{\cdot}\mask, \normal, \mask\}$ & & $\{\target{\cdot}\mask, \normal, \irra, \mask\}$ & & $\{\target, \normal, \mask\}$ & & $\{\target, \normal, \irra, \mask\}$\\[.7ex]
    \end{tabular}}

    \caption{Ablation study on lighting cues. When deprived of lighting cues by masking out the target image \target{} (\ie, providing $\target\!\cdot\!\mask$ as target) and removing the irradiance map \irra{}, our method produces results with flat, implausible shading (leftmost). Reintroducing either the irradiance \irra{} (second column) or the masked region (third column) restores the light effects. Providing all lighting cues (\method{}) provides the best result (rightmost).
    For clarity, we report the conditioning used below.
    }
    \label{fig:dropout}
\end{figure}

\section{Experiments}

We now compare our method against state-of-the-art inpainting and material transfer methods.

\textbf{Baselines.} For inpainting models, we consider version 2.1 of Stable Diffusion~\cite{rombach2021highresolution}, the official inpainting fine-tuning of SD-XL~\cite{podell2023sdxl}, the inpainting FLUX.1 model from Black Forest Labs~\cite{flux.1}, and Blended Latent Diffusion~\cite{avrahami2023blended}. We also compare against ZeST~\cite{cheng2024zest}, the current state-of-the-art on material transfer. We employ the original models provided by the authors in all cases. 

\textbf{Data.} We conduct our quantitative analysis on 300 pairs of synthetic renders and 50 real images. The synthetic test set includes artist-made 3D scenes~\cite{evermotion} rendered with Blender's physically-based Cycles renderer~\shortcite{blender}. We render the images and the ground truth normals and irradiance maps. Unless stated otherwise, the real images are sourced from the Materialistic evaluation set \cite{sharma2023materialistic}. All our evaluation data will be publicly released. 

\textbf{Metrics.} We evaluate synthetic data using PSNR and LPIPS. Given that real data lacks ground truth maps, we evaluate its appearance using CLIP-I~\cite{radford2021learning} by computing the cosine similarity score between the exemplar image and a crop of the generated region. Additionally, we compare its estimated irradiance to that of the original target image to assess the adherence to lighting cues.



Quantitative results are reported in \cref{tab:generation}. For this experiment, all conditionings are supplied to the methods. While inpainting methods provide competitive performance, they do not perform as well as specialized methods for material transfer. On synthetic data, FLUX.1 shows competitive performance, beating even ZeST, which specializes in material transfer. On this data, our method achieves a $+3.5\%$ and $+2.4\%$ improvement in PSNR and LPIPS, respectively. On real data, we offer an improvement of $+2.6\%$ on the CLIP-I measure over ZeST, the second best-performing method. Our method establishes a new state-of-the-art on all metrics evaluated.

We further evaluate the shading error produced by each method in \cref{tab:irradiance}. This error is determined by comparing the estimated irradiance maps of the output image $\phi_\irra(\hat{\image})$ and the reference image $\phi_\irra(\image)$. We observe that newer inpainting methods based on FLUX.1 and Blended LD preserve the illumination from the original image well. Our method, guided by the irradiance map \irra{}, understandably outperforms all compared methods in illumination preservation. 


We present qualitative results in \cref{fig:baselines}. We note that earlier inpainting methods such as Stable Diffusion based methods (SD v2.1, SD-XL inpaint) have trouble with perspective projection, often offering an orthographic view of the material pasted directly into the region (second, third, and sixth rows), greatly hindering the realism of the edits. Newer methods such as FLUX.1 and Blended LD better adhere to the scene's geometry, but either lack perspective for FLUX.1 (sixth row) or differ from the exemplar material \exemplar{} (third to seventh rows). ZeST generally provides good geometry coherence, but exhibits artifacts (second, fourth, and seventh rows). In addition to good perspective projection (sixth row) and good respect for the exemplar material \exemplar{} (fourth row), our method \method{} provides more complex lighting interactions as reflections and highlights from lights (first and third rows). In general, \method{} produces material transfers that blend well with their surroundings while preserving illumination on the applied material. 

Additional analysis on color control can be found in~\cref{fig:variations}. For these results, we convert the exemplar material \exemplar{} from RGB to HSV and change its hue. Our method respects the user-defined color well, seamlessly integrating it into the scene. 

We evaluate the importance of the irradiance map~\irra{} in~\cref{fig:irradiance}. Our approach generates better matching shading than previous work, even without the irradiance map. However, high-frequency lighting effects from the original image such as highlights (first to third rows) require the irradiance map to be preserved. 

Finally, we demonstrate our ability to control the scale of the inpainted material by adjusting the scale of the exemplar~\exemplar{}. We evaluate this effect in~\cref{fig:scale} with three zoom levels. As our method processes larger features, it scales them up in the scene accordingly. 


\subsection{Ablation study}

We quantitatively evaluate the impact of each component of our method in \cref{tab:ablation}. All ablations are trained on the entirety of our \datasetname{} dataset, and evaluated on synthetic scenes. Unfreezing the UNet $(A_2)$ gives freedom for the image and mask to be used as conditionings $(A_{4+})$, which significantly boosts performance. Further introducing the irradiance map~\irra{} $(A_6)$ helps preserve the image shading, thus improving the results. Adding normals~\normal{} $(A_5)$ improves the results slightly; we hypothesize its role is to disambiguate possible confusion between the geometry and the lighting. Overall, training the IP-Adapter encoders slightly improves the result compared to solely fine-tuning the denoising U-Net, keeping the IP-Adapter layers frozen with pretrained weights.

We explore the role of lighting conditioning in \cref{fig:dropout}, showing that our model considers cues from both the target \target{} and the irradiance map \irra{}. As expected, removing all lighting cues significantly deteriorates shading quality. We do so by masking the target region \mask{} in the target image \target{} and removing the irradiance map \irra{}, that is using $\xx = \{\target\cdot\mask, \normal, \mask\}$. The best results are obtained when both the full target image \target{} (providing local geometry cues) and the irradiance (providing lighting information) are supplied. 










    
        



    



\begin{figure}
    \centering
    \setlength{\tabcolsep}{0pt}
    \renewcommand{\arraystretch}{0}
    \newcommand{\imgsize}{.18\linewidth}
    \newcommand{\imgsizehalf}{.09\linewidth}
    \newcommand{\shift}{5.6ex}

    \newcommand{\row}[1]{%
        & \includegraphics[width=\imgsizehalf,height=\imgsizehalf]{images/scale/irradiance_#1.png}%
        & \includegraphics[width=\imgsizehalf,height=\imgsizehalf]{images/scale/image_#1.png}%
        & \multirow{2}{*}[\shift]{\includegraphics[width=\imgsize,height=\imgsize]{images/scale/texture_z0_#1.png}}%
        & \multirow{2}{*}[\shift]{\includegraphics[width=\imgsize,height=\imgsize]{images/scale/train_mocka_v7adamW_mocka_v3_a100_512px_26000_#1_z0.png}}%
        & \multirow{2}{*}[\shift]{\includegraphics[width=\imgsize,height=\imgsize]{images/scale/train_mocka_v7adamW_mocka_v3_a100_512px_26000_#1_z1.png}}%
        & \multirow{2}{*}[\shift]{\includegraphics[width=\imgsize,height=\imgsize]{images/scale/train_mocka_v7adamW_mocka_v3_a100_512px_26000_#1_z3.png}} \\%
        
        &\includegraphics[width=\imgsizehalf,height=\imgsizehalf]{images/scale/normals_#1.png} %
        & \includegraphics[width=\imgsizehalf,height=\imgsizehalf]{images/scale/overlay_#1.png} \\[0.4ex]%
    }
    
	\resizebox{.99\linewidth}{!}{%
    \begin{tabular}{c cc S{.2ex} c S{.2ex} ccc}
        &\multicolumn{2}{c}{Conditions} & Texture & $\times 1$ & $\times 2$ & $\times 8$ \\[.7ex]

        \row{22f65695-image_023_D_tc_bricks_005}
        \row{36515637-image_010_A_tc_bricks_022}
        \row{36c7a88c-image_042_A_acg_paving_stones_009}
        \row{36c7a88c-image_042_A_ms_paving_stones_018__grass_003}
    \end{tabular}}
    \vspace{-2mm}
    \caption{Impact of exemplar \exemplar{} scale. We can implicitly control the 2D material appearance in the 3D scene by providing various the exemplar material \exemplar{} at different scales. We show results using the entire material ($\times{}\!1$), half the original material size ($\times{}\!2$), and one-eighth ($\times{}\!8$) and see that the transferred material scale follows that of the input.}
    \label{fig:scale}
\end{figure}

\begin{figure}
    \centering
    \setlength{\tabcolsep}{0pt}
    \renewcommand{\arraystretch}{0}
    \newcommand{\imgsize}{.2\linewidth}
	
    \resizebox{.99\linewidth}{!}{%
    \begin{tabular}{ccccc}
        \includegraphics[width=\imgsize]{images/archviz/scene_1.png}&%
        \includegraphics[width=\imgsize]{images/archviz/scene_2.png}&%
        \includegraphics[width=\imgsize]{images/archviz/scene_5.png}&%
        \includegraphics[width=\imgsize]{images/archviz/scene_6.png}&%
        \includegraphics[width=\imgsize]{images/archviz/scene_7.png}\\ 
        
        \includegraphics[width=\imgsize]{images/archviz/scene_3.png}&%
        \includegraphics[width=\imgsize]{images/archviz/scene_4.png}&%
        \includegraphics[width=\imgsize]{images/archviz/scene_8.png}&%
        \includegraphics[width=\imgsize]{images/archviz/scene_9.png}&%
        \includegraphics[width=\imgsize]{images/archviz/scene_10.png}\\ 
    \end{tabular}}
    \vspace{-2mm}
    \caption{Samples of our synthetic evaluation dataset scenes showing their diversity in appearance and illumination.}
    \label{fig:archviz}
\end{figure}




\begin{figure}
    \centering
    \setlength{\tabcolsep}{0pt}
    \renewcommand{\arraystretch}{0}
    \newcommand{\imgsize}{.18\linewidth}
    \newcommand{\imgsizehalf}{.09\linewidth}
    \newcommand{\shift}{5.7ex}
    
    \newcommand{\row}[2]{%
        &\includegraphics[width=\imgsizehalf,height=\imgsizehalf]{images/failure/irradiance_#1.png}%
        & \includegraphics[width=\imgsizehalf,height=\imgsizehalf]{images/failure/image_#1.png}%
        & \multirow{2}{*}[\shift]{\includegraphics[width=\imgsize,height=\imgsize]{images/failure/texture_#1.png}}%
        & \multirow{2}{*}[\shift]{\includegraphics[width=\imgsize,height=\imgsize]{images/failure/ours_#1.png}}%
        & \includegraphics[width=\imgsizehalf,height=\imgsizehalf]{images/failure/irradiance_#2.png}%
        & \includegraphics[width=\imgsizehalf,height=\imgsizehalf]{images/failure/image_#2.png}%
        & \multirow{2}{*}[\shift]{\includegraphics[width=\imgsize,height=\imgsize]{images/failure/texture_#2.png}}%
        & \multirow{2}{*}[\shift]{\includegraphics[width=\imgsize,height=\imgsize]{images/failure/ours_#2.png}} \\
        
        &\includegraphics[width=\imgsizehalf,height=\imgsizehalf]{images/failure/normals_#1.png}%
        & \includegraphics[width=\imgsizehalf,height=\imgsizehalf]{images/failure/mask_#1.png}%
        &&& \includegraphics[width=\imgsizehalf,height=\imgsizehalf]{images/failure/normals_#2.png}%
        & \includegraphics[width=\imgsizehalf,height=\imgsizehalf]{images/failure/mask_#2.png} \\[.7ex]
    }
    
	\resizebox{.99\linewidth}{!}{%
    \begin{tabular}{c cccc S{.5ex} cccc}
        &\multicolumn{2}{c}{Conditions} & Texture & Output & \multicolumn{2}{c}{Conditions} & Texture & Output \\[.7ex]
        
        \row{9259cdee-image_018_A_acg_bricks_023}{2414415f-image_039_B_st_pavement_022}
        \row{a9adf536-image_014_B_th_cobblestone_square}{79597e23-image_071_A_ms_paving_stones_094__grass_003}
        
    \end{tabular}}
    \vspace{-3mm}
    \caption{Limitations. We illustrate our method's limitations with objects with detailed normals being lost during material transfer (left column), and downward facing normals (right column). We believe these limitations could be mitigated by explicitly creating these situations in the training dataset.}
    \label{fig:limitations}
\end{figure}

