This paper explores the implications of employing non-volatile memory (NVM) as primary storage for a data base management system (DBMS). We investigate the modifications necessary to be applied on top of a traditional relational DBMS to take advantage of NVM features. As a case study, we modify the storage engine (SE) of PostgreSQL enabling efficient use of NVM hardware. We detail the necessary changes and challenges such modifications entail and evaluate them using a comprehensive emulation platform. Results indicate that our modified SE reduces query execution time by up to 45\% and 13\% when compared to disk and NVM storage, with average reductions of 19\% and 4\%, respectively. Detailed analysis of these results shows that while our modified SE is able to access data more efficiently, data is not close to the processing units when needed for processing, incurring long latency misses that hinder the performance. To solve this, we develop a general purpose library that employs helper threads to prefetch data from NVM hardware via a simple API. Our library further improves query execution time for our modified SE when compared to disk and NVM storage by up to 54\% and 17\%, with average reductions of 23\% and 8\%, respectively.