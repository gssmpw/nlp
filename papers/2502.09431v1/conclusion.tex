\section{Conclusion}
\label{sec:conclusion}
\noindent In this paper, we study the implications of employing NVM in the design of DBMSs. We discuss the possible options to incorporate NVM into the memory hierarchy of a DBMS computing system and conclude that, given the characteristics of NVM, a platform with a layer of DRAM where the disk is completely or partially replaced using NVM is a compelling scenario. Such an approach retains the programmability of current systems and allows direct access to the dataset stored in NVM. With this system configuration in mind we modify the PostgreSQL storage engine in two incremental steps - SE1 and SE2 - to better exploit the features offered by PMFS using memory mapped I/O.

Our evaluation shows that storing the dataset in NVM instead of disk for an unmodified version of PostgreSQL improves query execution time 
by up to 39\%, with an average of 15\%. Modifications to take advantage of NVM hardware improve the execution time by around 
19\% on average as compared to disk storage. However, the current design of database software proves to be a hurdle in maximizing the improvement. 
When comparing our baseline and \textit{SE2} using PMFS, we achieve significant speedups of up to 13\% in read-dominant queries, 
but moderate average improvements of 4\% in query execution time.

We find that the limiting factor in achieving higher performance improvements is the fact that the data is not close to the processing 
units when it is needed for processing. This is a negative side effect of directly accessing data from NVM, rather than copying it into 
application buffers to make it accessible. This leads to long latency user-level cache misses eating up the improvement achieved by avoiding 
expensive data movement operations. 

We develop a general purpose data prefetching library to mitigate this negative side effect. 
The library provides services to bring data into caches through user-level helper threads in a parallel fashion without stalling the application. 
Our library improves query execution time when compared to \emph{disk\_base95} by up to 54\%, with an average of 23\%. When compared to \emph{pmfs\_base95}, query execution time improves by up to 17\%, with 8\% average improvements.
%