\section{Related Work}
\label{sec:RelatedWork}
Previous work on usage of NVM in context of DBMS
can be divided into three broad categories: proposals for (i)
NVM-aware DBMS designs from scratch, (ii) modification of
one or more components of an already existing in-memory
DBMS, and (iii) using NVM in a disk-oriented DBMS.

In the first category, Arulraj \textit{et al.}~\cite{arulraj2015let} propose usage of a single tier memory hierarchy, i.e., without DRAM, and compare three different storage management architectures using an NVM-only system for a custom designed lightweight DBMS. In \cite{arulraj2017build}, Arulraj \textit{et al.} discuss designing a DBMS for NVM and suggest that an NVM-enabled DBMS needs to adapt the logging protocol as well as the in-memory buffer cache in order to achieve significant performance improvements. Peloton \cite{PeletonLink} is another example of a DBMS designed from scratch for DRAM/NVM storage.

In the second category, Pelley \textit{et al.}~\cite{pelley2013storage} explore a two-level hierarchy with DRAM and NVM and study different recovery methods, using Shore-MT \cite{johnson2009shore} storage engine. Others have suggested employing NVM only for logging components of a DBMS and not for dataset storage. For example, earlier works implement NVM-Logging in Shore-MT and IBM-SolidDB~\cite{fang2011high,huang2014nvram} to reduce the impact of disk I/O on transaction throughput and response time by directly writing log records into an NVM component instead of flushing them to disk. Wang \textit{et al.}~\cite{wang2014scalable} demonstrate, by modifying Shore-MT, the use of NVM for distributed logging on multi-core and multi-socket hardware to reduce contention of centralized logging with increasing system load.

While in-memory DBMS have become quite popular, disk-based DBMS still have not lost their importance as indicated in  top ten ranking of DBMS by popularity \cite{RankingLink}. Therefore, people have investigated usage of NVM in context of disk-based DBMS. In this third category, Gao \textit{et al.}~\cite{gao2011pcmlogging} use phase changing memory (PCM) is used to hold buffered updates and transaction logs in order to improve transaction processing performance of a traditional disk-based DBMS (i.e. PostgreSQL 8.4). NVM has also been used to replace a disk-located double write buffer (DWB) in MySQL/InnoDB by a log-structured buffer implemented in NVM, resulting in higher transaction throughput \cite{son2017log}.

The work presented in this paper belongs to the third category. It focuses on usage of NVM as a replacement of disk storage in a traditional DBMS and explains necessary changes in the storage engine for such a replacement. Our contribution is an NVM-aware storage engine that is complementary to and can be applied along with PCM-logging \cite{gao2011pcmlogging} and NVM-buffering \cite{son2017log} on a traditional disk-based DBMS.

%\noindent Various proposals have been presented to integrate NVM in the memory hierarchy of a DBMS. Pelley \textit{et al.}~\cite{pelley2013storage} explore a two-level hierarchy with DRAM and NVM and study different recovery methods. On the other hand, Arulraj \textit{et al.}~\cite{arulraj2015let} use a single tier memory hierarchy, i.e., without DRAM, and compare three different storage management architectures using an NVM-only system. Others have suggested employing NVM only for logging components of a DBMS and not for dataset storage. The work reported in~\cite{fang2011high,huang2014nvram} reduces the impact of disk I/O on transaction throughput and response time by directly writing log records into an NVM component instead of flushing them to disk. Authors of~\cite{wang2014scalable} employ NVM for distributed logging on multi-core and multi-socket hardware to reduce contention of centralized logging with increasing system load.

%Arulraj \textit{et al.} \cite{arulraj2017build} discuss how to design a DBMS for NVM. They suggest that a NVM-enabled DBMS needs to adapt the logging protocol as well as the in-memory buffer cache in order to achieve significant performance improvements. Our work focuses on bypassing the in-memory buffer cache for read-dominant queries by accessing data directly from NVM storage. 



Helper threads have been used for parallel speedup of legacy applications.  Researchers have used programmer-constructed helper threads \cite{kamruzzaman2011inter} as well as compiler algorithms \cite{kim2002design} for automated extraction of helper threads to enhance the performance of applications. Use of helper-threads for loosely coupled multiprocessor systems is demonstrated in \cite{jung2006helper}, with focus on efficient thread synchronization for low overhead. Others have used special hardware to construct, spawn, optimize and manage helper threads for dynamic speculative precomputation \cite{collins2001dynamic}. Although helper-thread based prefetching is a well-studied technique, we pioneer its use in the context of NVM storage for DBMS in order to resolve the data readiness problem arising from having direct access to NVM-resident data.
