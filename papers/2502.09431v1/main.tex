%%%%%%%%%%%%%%%%%%%%%%% file template.tex %%%%%%%%%%%%%%%%%%%%%%%%%
%
% This is a general template file for the LaTeX package SVJour3
% for Springer journals.          Springer Heidelberg 2010/09/16
%
% Copy it to a new file with a new name and use it as the basis
% for your article. Delete % signs as needed.
%
% This template includes a few options for different layouts and
% content for various journals. Please consult a previous issue of
% your journal as needed.
%
%%%%%%%%%%%%%%%%%%%%%%%%%%%%%%%%%%%%%%%%%%%%%%%%%%%%%%%%%%%%%%%%%%%
%
% First comes an example EPS file -- just ignore it and
% proceed on the \documentclass line
% your LaTeX will extract the file if required
\begin{filecontents*}{example.eps}
%!PS-Adobe-3.0 EPSF-3.0
%%BoundingBox: 19 19 221 221
%%CreationDate: Mon Sep 29 1997
%%Creator: programmed by hand (JK)
%%EndComments
gsave
newpath
  20 20 moveto
  20 220 lineto
  220 220 lineto
  220 20 lineto
closepath
2 setlinewidth
gsave
  .4 setgray fill
grestore
stroke
grestore
\end{filecontents*}
%
\RequirePackage{fix-cm}
%
%\documentclass{svjour3}                     % onecolumn (standard format)
%\documentclass[smallcondensed]{svjour3}     % onecolumn (ditto)
\documentclass[smallextended]{svjour3}       % onecolumn (second format)
%\documentclass[twocolumn]{svjour3}          % twocolumn
%
\smartqed  % flush right qed marks, e.g. at end of proof
%
\usepackage{graphicx}
\usepackage{textcomp}
\usepackage[super]{nth}
\usepackage{subfigure}
\usepackage{booktabs}
\usepackage{multirow}


%
% \usepackage{mathptmx}      % use Times fonts if available on your TeX system
%
% insert here the call for the packages your document requires
%\usepackage{latexsym}
% etc.
%
% please place your own definitions here and don't use \def but
% \newcommand{}{}
%
% Insert the name of "your journal" with
% \journalname{myjournal}
%
\begin{document}

\title{On Usage of Non-Volatile Memory as Primary Storage for Database Management Systems
\thanks{The research leading to these results has received funding from the European Union\textquotesingle s 7th Framework Programme under grant agreement number 318633, the Ministry of Science and Technology of Spain under contract TIN2015-65316-P, and a HiPEAC collaboration grant awarded to Naveed Ul Mustafa.
%Grants or other notes
%about the article that should go on the front page should be
%placed here. General acknowledgments should be placed at the end of the article.
}
}
%\subtitle{Do you have a subtitle?\\ If so, write it here}

\titlerunning{Naveed Ul MUSTAFA et al. On Usage of NVM as Primary Storage for DBMS}        % if too long for running head

\author{Naveed Ul Mustafa \and Adri\`{a} Armejach  \and
        Ozcan Ozturk \and  Adri\'{a}n Cristal 
        \and \\ Osman S. Unsal
         %etc.
}

%\authorrunning{Short form of author list} % if too long for running head

\institute{N. Ul Mustafa \at
              Department of Computer Engineering, Bilkent University, Ankara, Turkey. \\
              Present Address: Department of Computer Engineering, TED University, Ankara, Turkey.\\
              %Tel.: +123-45-678910\\
              %Fax: +123-45-678910\\
              \email{naveed.ul.mustafa0083@gmail.com}           %  \\
%             \emph{Present address:} of F. Author  %  if needed
           \and
           A. Armejach \at
              Barcelona Supercomputing Center (BSC), Barcelona, Spain and Universitat Polit\`{e}cnica de Catalunya (UPC), Barcelona, Spain.
           \and O. Ozturk \at
              Department of Computer Engineering, Bilkent University, Ankara 06800, Turkey.
           \and A. Cristal \at 
              Barcelona Supercomputing Center (BSC), Barcelona, Spain.
           \and O. S. Unsal \at
              Barcelona Supercomputing Center (BSC), Barcelona, Spain.
}

\date{Received: date / Accepted: date}
% The correct dates will be entered by the editor


\maketitle

\begin{abstract}
\begin{abstract}
Retrieval-Augmented Generation (RAG) is often used with Large Language Models (LLMs) to infuse domain knowledge or user-specific information. In RAG, given a user query, a retriever extracts chunks of relevant text from a knowledge base. These chunks are sent to an LLM as part of the input prompt. Typically, any given chunk is repeatedly retrieved across user questions. However, currently, for every question, attention-layers in LLMs fully compute the key values (KVs) repeatedly for the input chunks, as state-of-the-art methods cannot reuse KV-caches when chunks appear at arbitrary locations with arbitrary contexts. Naive reuse leads to output quality degradation.  This leads to potentially redundant computations on expensive GPUs and increases latency. In this work, we propose \sys, a system for managing and reusing precomputed KVs corresponding to the text chunks (we call \textit{chunk-caches}) in RAG-based systems. We present how to identify \hl{\textit{chunk-caches} that are reusable}, how to efficiently perform a small fraction of recomputation to \textit{fix} the cache to maintain output quality, and how to efficiently store and evict \textit{chunk-caches} in the hardware for maximizing reuse while masking any overheads. With real production workloads as well as synthetic datasets, we show that \sys reduces redundant computation by \textbf{51\%} over SOTA prefix-caching and \textbf{75\%} over full recomputation.
\hl{Additionally, with continuous batching on a real production workload, we get a \textbf{1.6$\times$} speedup in throughput and a \textbf{2$\times$} reduction in end-to-end response latency over prefix-caching while maintaining quality, for both the \llama-3-8B and \llama-3-70B models. 
}
\end{abstract}





%Insert your abstract here. Include keywords, PACS and mathematical subject classification numbers as needed. 
\keywords{Non volatile memory \and Relational DBMS \and Storage engine}
% \PACS{PACS code1 \and PACS code2 \and more}
% \subclass{MSC code1 \and MSC code2 \and more}
\end{abstract}

\documentclass[../main.tex]{subfiles}
\graphicspath{{../images/}}
\makeatletter
\def\input@path{{../images/}}
\makeatother
\begin{document}
\section{Introduction}
\begin{figure}
\centering
\begin{tikzpicture}
\node[inner sep=0pt] (ws) at (0, 0) {
\includegraphics[height=.4\textwidth, trim={10cm 0 10cm 0},clip]{world_space.png}};
\node[inner sep=0pt] (cs) at (6,0) {\includegraphics[height=.4\textwidth, trim={10cm 1cm 10cm 4cm},clip]{conf_space.png}};
\end{tikzpicture}
\vspace{-5pt}
\label{fig:pbrm_intro}
\caption{\textbf{Left}: Shows world space obstacles as grey spheres. Robots start and goal configuration is colored red and green, respectively. Configurations along the computed path are colored transparent blue. \textbf{Right:} Mapped world space scenario to configuration space. Obstacle region is the grey mesh. Red spheres are collision-free regions computed by the neural SCDF. The optimized shortest path in the convex corridor is the blue curve.}
\vspace{-25pt}
\end{figure}
Motion planning is the problem of finding a collision-free trajectory that connects a given start and goal configuration. The planning takes place in the configuration space of the robot. For single body robots, like mobile robots or drones, the configuration space and the world space are usually the same. This simplifies the planning, since explicit obstacle representations are available which enables geometrical tools like separating hyperplanes, smallest distance to obstacles etc., to be used when designing motion planning algorithms. For multi-body robots like manipulators, the situation is completely different. The world space obstacles are usually mapped to non-convex regions, and to make the problem even harder, the mapping is usually not known. Forming explicit representations of the obstacle region in the configuration space is usually too expensive or intractable. Despite all of this, sampling based planners are used with great success, which mainly is due to their use of implicit representations of the obstacle region. The basic idea is to construct a graph in the configuration space that covers and connects the collision-free region. From this graph, a path can be extracted that connects a given start and goal configuration. The approach is computationally expensive, since the graph is constructed with the smallest geometrical building block available, points, which represents a collision-check. Furthermore, the extracted paths from the graph are non-smooth and jagged due to the stochastic nature of the approach. This adds an additional post-processing step to the process, where the paths are shortcutted and smoothened, before the path can be used for tracking. Clearly a lot of time is invested to form this graph and produce smooth paths. Thus, if the obstacles start to move, then all of this work is done in no use, since all points that make up this graph need to be re-verified, which is simply too time consuming to be done in real time.
\\\\
In this work, we want to address the existing drawbacks of the sampling based planners. Our main contribution is an improved motion planner where each vertex in the graph covers a collision-free region in the form of a sphere instead of a point and where the edges are formed with neighboring intersecting spheres. This representation has the advantage of instead of returning piecewise linear paths, returning a sequence of overlapping spheres, i.e. a convex corridor, that connects a given start and goal configuration, illustrated in Figure \ref{fig:pbrm_intro}. This convex corridor allows us to use convex optimization to produce smooth trajectories, instead of computationally expensive post-processing methods. The representation further allows us to estimate the coverage of the collision-free space, which gives us awareness and feedback in the offline roadmap construction phase. Finally, our representation is simple to adapt to moving obstacles, simply requery for the new radii and recheck for intersections. 
\\\\
The spherical collision-free regions are formed using a signed distance function (SDF), which is a function that returns the smallest distance from an arbitrary point to the boundary of an obstacle. As the name implies, the distance is signed, thus if the point is inside the obstacle it is negative otherwise positive. If the distance is positive, a sphere with radius equal to the distance is guaranteed to cover a collision-free region. Using an SDF in motion planning is not new, but what is novel about our approach is that we express the distance in the configuration space instead of the world space and by doing so allows us to form these convex collision-free regions. We refer to the resulting SDF as a signed configuration distance function (SCDF). Computing an SCDF analytically is non-trivial, our approach is therefore to parameterize the SCDF with a deep neural network and learn the mapping by supervised learning. Our resulting neural SCDF can compute distances for different parameter values of obstacle shapes and we also show how multiple distances can be combined, thus making our approach flexible.
\section{Related work}
Motion planning algorithms can roughly be divided into three families, grid-based, sampling based and optimization based methods. Grid-based methods (GBM) discretize the planning space from which a graph is then compiled. A standard search method is A$^\star$ \citep{a_star}, which is classified as an \textit{informed} search method, since it employs a heuristic function to speed up the search. A$^\star$ guarantees to return an optimal path at the level of discretization used. GBMs usually discretize the planning space by a regular lattice and this limits the GBMs to problems with low dimensionality due to the curse of dimensionality. Thus, GBMs are usually limited to single-body robots where the degrees of freedom (DOF) are low. To overcome the inherent scaling problem with the GBMs, stochastic methods are usually used for multi-body robots. These methods are termed as sampling-based methods (SBM) and core members within this family are the rapidly-exploring random trees (RRT) \citep{rrt} and the probabilistic roadmap (PRM) \citep{prm}. RRT grows a tree from the start configuration and explores the collision-free region in a rapid way until it is able to connect to the goal region. RRT is usually improved by bi-directional planning \citep{rrt_connect}, i.e. an additional tree is grown from the goal configuration and the trees are tested for connection after any tree has been expanded. RRT is a single-query method, thus it searches for a path from scratch each time it is queried. Contrary to this, PRM is a multi-query method, which solves for multiple queries without starting from scratch. PRM does this by creating a roadmap (graph) that covers the collision-free space as an offline step. The graph is then used to solve for multiple queries. PRMs are used in cases where the environment does not change since the extra offline step is too computationally costly and needs to be re-done if the environment is changed. In our work, we address this inherent issue by using a different roadmap representation. Our vertices in the graph cover a collision-free region in the form of spheres and we form the edges by checking for intersecting spheres. If something in the environment changes, we recompute the spheres radii and recheck the intersections, without relying on collision detection. We use a trained neural network to compute the sphere radius, therefore querying for the radius can be done fast, hence our representation enables the PRM for dynamic environments.
\\\\
In the recent decades, optimization based methods (OBM) \citep{chomp, schulman, itomp, stomp} have been introduced as an alternative to SBM for multi-body robots. Like the SBM, the OBMs scale well to higher dimensional problems and produce smoother motion. It is common to use a SDF in the optimization since it is a smooth function, thus enabling gradient-based methods. However, the standard way of expressing the SDF is in world space. The distance therefore needs to be mapped to the configuration space by the forward kinematics. This mapping makes the optimization problem a non-linear program (NLP), which is computationally expensive to solve. Recently, a different approach has been proposed. In \cite{mp_gcs} motion planning is formulated as a convex optimization problem by using the graph of convex sets framework \citep{gcs}. The underlying idea is to decompose the collision-free space into intersecting convex sets from which a convex optimization problem is formulated. In cases where an explicit representation of the obstacles in the configuration space exists, like for single-body robots, creating collision-free convex regions can be done fast \citep{iris}. For multi-body robots, this is non-trivial. Existing work does this successfully \citep{iris_nlp, iris_c} by an optimization based approach, but the methods are still too time consuming to be used in the presence of moving obstacles. Our approach is instead to use deep learning to learn an SDF expressed in the configuration space. With this, we can query for shortest distances to the collision boundary, which allows us to expand spherical regions which are collision-free. Our approach is fast and therefore enables our suggested roadmap planner to be used in dynamic environments.
\\\\
Recent research has focused on learning collision detection \citep{fk_kernel_distance, diffco, graphdistnet} by predicting the signed distance between the robot links and the surrounding obstacles in the world space. The learned SDF is used in trajectory optimization but since the distance is expressed in the world space, the problem becomes an NLP and therefore takes a long time to solve. We take a novel approach and suggest to instead express the signed distance in the configuration space. This allows us to improve the PRM at the same time as it enables convex optimization for trajectory optimization, which runs faster and is more reliable than NLP solvers. In \cite{cspf} a learned signed distance function in the configuration space is proposed similar to our approach. However, their approach is restricted to point cloud representations, while we propose to represent the obstacles as parameterized geometric shapes, e.g. spheres. Furthermore, we also show how to use our learned SCDF to improve an existing roadmap planner.
\section{Problem formulation}
A robot is located in the world space, $\W \subset \R^3 $. The unique location of the robot is given by its configuration $\q \in \C$, where $\C$ is the configuration space. The set of points covered by the robots bodies at a certain configuration is expressed as $\B(\q) \subset \W$. The robot is surrounded by $\NrObst$ obstacles $\O = \bigcup_{i=1}^{\NrObst} \O_i$, where  $\O_i \subset \W$. The representation of the obstacle in the configuration space is the set $\C\O_i = \{\q \in \C \: |\: \B(\q) \cap \O_i \neq \emptyset \}$. The obstacle space is formed as $\Co = \bigcup_{i=1}^{\NrObst} \C \O_i$. The complement is referred to as the free space, $\Cf = \C \setminus \Co$. The path planning problem is a tuple, ($\Cf$, $\qStart$, $\qGoal$), where we want to connect a query pair, consisting of a start, $\qStart$, and goal configuration, $\qGoal$, with a geometric path, $\q(s): [0, 1] \mapsto \Cf$, such that $\q(0)=\qStart$ and $\q(1)=\qGoal$, or report correctly when such a path does not exist.
\end{document}

\section{Background}
\label{sec:background}
\noindent In this section, we first describe in detail the properties of NVM technologies, highlighting the implications these might have in the design of a DBMS. We then describe currently available NVM hardware and  system software to manage NVM.


\subsection{Characteristics of NVM}
\noindent \textbf{Data access latency:} Read latencies for NVM technologies will certainly be significantly lower than those of conventional disks. However, since NVM devices are still under development, sources quote varying read latencies. For example, the read latency for STT-RAM ranges from 1 to 20ns, and PC-RAM is expected to be around 50ns ~\cite{arulraj2015let,wang2013low,perez2010non}. Nonetheless, read latency of some NVM technologies is expected to be similar to that of DRAM ~\cite{mittal2016survey,arulraj2015let,wang2013low,chang2012limits,arulraj2016write,oukid2014sofort,chatzistergiou2015rewind}, which is typically around 60ns.

PC-RAM and R-RAM are reported to have a higher write latency compared to DRAM, but STT-RAM also outperforms DRAM in this regard ~\cite{arulraj2015let,wang2013low}. However, the write latency is typically not in the critical path, since it can be tolerated by using buffers ~\cite{qureshi2009scalable}.

\noindent\textbf{Density:} NVM technologies provide higher densities than DRAM, which makes them a good candidate to be used as main memory as well as primary storage, particularly in embedded systems~\cite{huang2012register}. For example, PC-RAM provides 2 to 4 times higher density as compared to DRAM~\cite{qureshi2009scalable}. Future NVMs are expected to have higher capacity and better scalability than DRAM \cite{oukid2015instant,chakrabarti2014atlas,zhang2015study,viglas2014write}
, and it is expected to scale to lower technology nodes as opposed to DRAM.

\noindent\textbf{Endurance:} The maximum number of writes a memory cell can withstand is lower for most NVM 
technologies when compared to DRAM ~\cite{qureshi2009scalable,zhou2009durable}. Specifically, PC-RAM, R-RAM, 
and STT-RAM have projected endurances of $10^{10}$, $10^{8}$, and $10^{15}$ respectively;  as compared to 
$10^{16}$ for DRAM ~\cite{arulraj2015let}. On the other hand, NVMs exhibit higher endurance than flash 
memory technologies ~\cite{wang2013low}.

\noindent\textbf{Energy consumption:} Since NVM does not need a refresh cycle to maintain data states in memory cells like a  DRAM, 
they are more energy efficient. A main memory designed using PC-RAM technology consumes significantly lower per access write energy as compared to DRAM~\cite{zhou2009durable}. Other NVM technologies also have similar lower energy consumption per bit when compared to DRAM~\cite{arulraj2015let,perez2010non}.

In addition to the features listed above, NVM technologies also provide byte-addressability like DRAM and persistency like disks. Due to these features, NVMs are starting to appear in embedded and energy-critical devices and are expected to play a major role in future computing systems. Companies like Intel and Micron have launched the 3D XPoint memory technology, which features non-volatility \cite{3DXPoint}. Intel has also introduced new instructions to support the usage of persistent memory at the instruction set architecture (ISA) level~\cite{intel2016architecture}.

\subsection{Available NVM hardware}\label{NVDIMM}

While NVM hardware has been available in recent years, it has mainly been used to implement Solid State Disks (SSD) using the NVM Express (NVMe) interface. This technology is not suitable as a DRAM replacement due to its endurance and latency properties. Researchers have been anticipating the arrival of Dual Inline Memory Modules (DIMM) based on NVM to substitute traditional DRAM DIMMs for a long time. Recently, in April 2019, Intel has released its 3D Xpoint DIMM based on NVM technology~\cite{hirofuchi2019preliminary}.

As shown in Fig.~\ref{3DXPointInterface}, 3D Xpoint DIMMs connect to the memory bus and communicate with a processor through the integrated memory controller (iMC). Each iMC can connect to up to three DIMMs~\cite{peng2019system,yang2019empirical}. A non-standard protocol, DDR-T, is followed for communication between processors and 3D XPoint DIMM. Intel\textquotesingle s Cascade Lake processors are yet the only ones to support 3D Xpoint memory.  As each processor supports two iMCs, six DIMMs are supported in total~\cite{peng2019system,yang2019empirical,izraelevitz2019basic}.

3D XPoint DIMMs can operate in two different modes: Memory and App Direct. Memory mode uses NVM to expand main memory without providing the feature of persistency, while regular DRAM serves as a cache for NVM. In this mode, operating system and CPU see the NVM memory as a volatile extension of main memory. In App Direct mode, NVM is used as a separate persistent memory and does not use DRAM as cache. App Direct mode provides an application direct access to data residing in NVM without interference of the operating system and with byte-addressability. However, it requires an NVM-aware file system to allocate, name, and access persistent data.

\begin{figure}
\centering
\includegraphics[width=.5\textwidth]{3DXPointInterface.eps}
\caption{Overview of 3D XPoint Memory\textquotesingle s interface with CPU}
\label{3DXPointInterface}
\end{figure}

\subsection{System software for NVM}

Using NVM as primary storage necessitates modifications not only in application software but also in system software in order to 
take advantage of NVM features. A traditional file system (FS) accesses the storage through a block layer. If a disk is replaced by NVM without any modifications in
the FS, the NVM storage will still be accessed at block level granularity. Hence, we will not be able to take advantage of the byte-addressability feature of NVM. 

For this reason, researchers have developed purpose-built file systems such as NOn Volatile memory Accelerated (NOVA) \cite{xu2016nova} and persistent memory file System (PMFS) ~\cite{dulloor2014system,githubPMFS}. Both file systems expose NVM to an application by providing direct access through a memory map (mmap) interface. As this work uses PMFS, we discuss it in more detail below.

PMFS is an open-source POSIX compliant FS developed by Intel Research. It offers two key features in order to facilitate usage of NVM.  
First, PMFS does not maintain a separate address space for NVM. In other words, both main memory and NVM use the same address space. This implies that there is no need to copy data from NVM to DRAM to make it accessible to an application. A process can directly access file system protected data stored in NVM at byte level granularity.

Second, in a traditional FS stored blocks can be accessed in two ways: (i) file I/O and (ii) memory mapped I/O. PMFS implements file I/O in a similar way to a traditional FS. However, the implementation of memory mapped I/O differs. In a traditional FS, memory mapped I/O would first copy pages to DRAM~\cite{dulloor2014system} from where an application can examine those pages. PMFS avoids this copy overhead by mapping NVM pages directly into the address space of a process. Fig.~\ref{Fig1} from~\cite{dulloor2014system} compares a traditional FS with PMFS.

%\subsection{NVM DIMM}
%Dual In-line Memory Module using NVM have been long anticipated by communit of memory reserachers. Recently, Intel has annnounced and made commercially available its product named Intel's 3D XPoint memory as an NVM DIMM.

\begin{figure}
\centering
\includegraphics[width=100mm]{Fig1.eps}
\caption{Comparison of traditional FS and PMFS. ``mmap'' refers to the system call for memory mapped I/O operation. 
``mmu'' is the memory management unit responsible for address mappings}
\label{Fig1}
\end{figure}




%\input{pmfs-implications.tex}
\section{Design Choices}
\label{sec:Implications}
\noindent In this section, we discuss the possible memory hierarchy designs when including NVM in a system. We also discuss the high-level modifications necessary in a traditional disk-optimized DBMS in order to take full advantage of NVM hardware.
\subsection{Memory Hierarchy Designs for an NVM-Based DBMS}

With features of byte-addressability, low latency and high capacity, NVM has the potential to replace traditional disks as well as main memory \cite{chang2012limits}. Fig.~\ref{Fig2} shows different options that might be considered when including NVM into the system. Fig.~\ref{Fig2a} depicts a traditional approach, where the intermediate state - including logs, data buffers, and partial query state - is stored in DRAM to hide disk latencies for data that is currently in use; while the bulk of the relational data is stored in a  disk.

Given the favorable characteristics of NVM over the other technologies, an option might be to replace both DRAM and disk storage 
using NVM (Fig.~\ref{Fig2b}). However, such a drastic change would require a complete redesign of current operating systems and 
application software. In addition, NVM technology is still not mature enough in terms of endurance to be used as a DRAM replacement. 
Hence, we advocate for a platform that still has a layer of DRAM memory, like \cite{kimura2015foedus}, where the disk is completely or partially replaced using NVM, 
as shown in Fig.~\ref{Fig2c} (NVM-Disk). 

Using this approach, we can retain the programmability of current systems by still having a layer of DRAM, thereby exploiting DRAM's fast read and write access latencies for temporary data structures and application code. In addition, it allows the possibility to directly access the bulk of the database relational data by using a file system such as PMFS, taking full advantage of NVM technology, which allows the system to leverage NVM's byte-addressability and to avoid API overheads~\cite{huang2014nvram} present in current FSs. Unlike an in-memory DBMS, such a setup does not need large pools of DRAM since temporary data is orders of magnitude smaller than the actual relational database stored in NVM. We believe this is a realistic scenario for future systems integrating NVM, with room for small variations such as NVM alongside DRAM to store persistent temporary data structures, or having traditional disks to store cold data.

As explained in Section~\ref{NVDIMM}, 3D XPoint memory can operate in two different modes. Memory mode is similar to a traditional design (Fig.~\ref{Fig2a}), as the 3D XPoint DIMMS are not considered as persistent memory but as the actual DRAM address space. In this mode the DRAM DIMMs are transparently used as a cache for the 3D XPoint DIMMs. This does not change the system view from an application's point of view, and makes sense if one wants to use 3D XPoint DIMMs as if they were large-capacity DRAM DIMMs. However, the App Direct mode would fit into the NVM-Disk (Fig.~\ref{Fig2c}) category. Applications still see DRAM DIMMs as a layer of volatile memory, but can also directly access the 3D XPoint DIMMs via mmap interfaces that enable byte-addressability. The system we consider and later evaluate would be based on a setup similar to that offered by 3D XPoint\textquotesingle s  App Direct mode.


%\textcolor{blue}{Memory mode operation of 3D XPoint memory (explained in Section \ref{NVDIMM}) uses a configuration similar to  NVM-Disk (see Fig.~\ref{Fig2c}) with some differences. Although DRAM is used as a cache in memory mode, 3D Xpoint memory acts as volatile extension of DRAM and sits between storage and DRAM in the memory hierarchy. Fig.~\ref{MemOverview3DXPoint}, based on \cite{peng2019system}, shows the logical view of memory hierarchy in App Direct and Memory mode operation of 3D XPoint memory.}

\begin{figure} %[!htbp]
\centering     %%% not \center
\subfigure[Traditional design]{\label{Fig2a}\includegraphics[width=27mm]{Fig2a.eps}}
\subfigure[All-in-NVM]{\label{Fig2b}\includegraphics[width=27mm]{Fig2b.eps}}
\subfigure[NVM-Disk]{\label{Fig2c}\includegraphics[width=27mm]{Fig2c.eps}}
\caption{NVM placement in the memory hierarchy of a computing system}
\label{Fig2}
\end{figure}

%\begin{figure} %[!htbp]
%\centering     %%% not \center
%\subfigure[Memory Mode]{\label{MemMode}\includegraphics[width=35mm]{3DXPointMmeoryModeConfig.eps}}
%\hspace{5pt}
%\subfigure[Direct Access Mode]{\label{DAXMode}\includegraphics[width=35mm]{3DXPointAppDirectModeConfig.eps}}
%\caption{Overview of memory hiearchy in Memory and Direct Access Mode of 3D XPoint Memory}
%\label{MemOverview3DXPoint}
%\end{figure}

\subsection{Potential Modifications in a Traditional DBMS}~\label{modList}
Using a traditional disk-based database with NVM storage will not take full advantage of NVM's features. Some important components of the DBMS need to be modified or removed when using NVM as a primary storage. 

\noindent\textbf{Avoid the block level access:} Traditional design of a DBMS uses a disk as a primary storage. Since disks favor sequential accesses, database systems hide disk latencies by issuing fewer but larger disk accesses in the form of a data block~\cite{schindler2002track}. 

Unfortunately, block level I/O causes extra data movement. For example, if a transaction updates a single byte of a tuple, it still needs to write the whole block of data to the disk. On the other hand, block level access provides good data locality.
 
Since NVM is byte-addressable, we can read and write only the required byte(s). However, reducing the data retrieval granularity down to a byte level eliminates the advantage of data locality altogether. A good compromise is to reduce the block size in such a way that the overhead of the block I/O is reduced to an acceptable level, while at the same time the application benefits from some degree of data locality. 
 
\noindent\textbf{Remove internal buffer cache of DBMS:} DBMSs usually maintain an internal buffer cache. Whenever a tuple is to be accessed, first its disk address has to be calculated. If the corresponding block of data is not found in the internal buffer cache, then it is read from disk and stored in the internal buffer cache \cite{debrabant2013anti}. 
 
This approach is unnecessary in an NVM-based database design. If the NVM address space is made visible to a process, then there is no need to copy data blocks. It is more efficient to refer to the tuple directly by its address. However, we need an NVM-aware FS, such as PMFS, to enable direct access to the NVM address space by a process.
 



\subsection{Discussion}
NVM provides the promising features of persistency, like disk storage; and byte-addressability, like DRAM. However, NVMs
have certain limitations such as lower endurance compared to DRAM \cite{arulraj2015let} and a disparity between the read and write latencies \cite{pelley2014memory}. 
Furthermore, different NVM technologies differ from each other in term of these features \cite{arulraj2015let}.

 
A storage engine aiming to improve decision support
system (DSS) queries can be designed by taking advantage
of the common features of persistency and byte-addressability.
Since DSS queries are read dominant and perform a relatively
negligible number of write operations, the design should
not be influenced or sensitive to different endurance and write
latencies found across NVM technologies. Furthermore, NVM technologies
are projected to provide read latencies similar to DRAM \cite{mittal2016survey,arulraj2015let,wang2013low,chang2012limits}.
Therefore, reading data directly from NVM storage should be comparable in terms
of access latency to reading application data stored in DRAM.

 
Usage of NVM as primary storage can also impact
other components of a DBMS besides those mentioned
in Section \ref{modList}. For example, if internal buffers are not
employed and all updates are materialized directly into the
NVM address space then the need and criticality of the redo
log can be relaxed \cite{huang2014nvram}. However, the undo log will still be
needed to recover from a system failure. These important aspects are
out of the scope of this work and we will focus on storage engine modifications.

\begin{figure*}  %[!htbp]
\centering     %%% not \center
\subfigure[PostgreSQL storage engine]{\label{Fig3a}\includegraphics[width=38mm]{Fig4.eps}}
\subfigure[Modified storage engine - SE1]{\label{Fig3b}\includegraphics[width=38mm]{Fig5New.eps}}
\subfigure[Modified storage engine - SE2]{\label{Fig3c}\includegraphics[width=38mm]{Fig6New.eps}}
\caption{High level view of read and write memory operations in PostgreSQL (read as ``pg'' in short form) and modified SEs}
\label{Fig3}
\end{figure*}
 

\input{case-study_1.tex}
\input{methodology_1.tex}
\section{Evaluation}
We provide three sets of insights into this section, organised as \textit{findings (F*)}. We quantitatively study the effect of the adversarial and counterfactual perturbations on the performance of informal reasoners and autoformalisation methods. Then, we dive deeper into method variants. Finally, 
we analyse the nature of formalisation errors made by the models.

\subsection{Robustness Analysis}
\paragraph{\textbf{\emph{F1: Noise perturbations have a stronger effect on formalisation methods than informal \ac{LLM} reasoners.}}}
Table~\ref{tab:distraction_k4_formalisation} shows that, on average, the accuracy of both direct and \ac{CoT} informal reasoning remains between $73\%$ and $74\%$ in the face of added noise. While the autoformalisation method performs similarly to informal reasoners on the original dataset, its performance decreases between $4\%$ and $11\%$. The accuracy drops especially with logical (L) and tautological (T) distractions, whose logical language formats trick the \ac{LLM} into formalizing the noisy clauses. On the other hand, the linguistically complex and more natural sentences of encyclopedic distractions show a minor effect, suggesting that \acp{LLM} successfully avoids formalizing the more complicated sentences.

\paragraph{\textbf{\emph{F2: All \ac{LLM}-based reasoning methods suffer a drop for counterfactual perturbations.}}} % influence .}}}
Table~\ref{tab:distraction_k4_formalisation} shows that counterfactual statements cause a significant decrease in performance for both the informal reasoners and autoformalisation methods of between $12\%$ and $13\%$ on average. 
Moreover, this observation also holds for all tested models, i.e., none are robust towards counterfactual perturbations across every evaluated dimension. Even the strongest model, GPT 4o-mini, yields a performance of 63-68\%, which is relatively close to the random performance of 50\%. The high impact of counterfactual statements (the single ``not'' inserted) could be due to the inability of \acp{LLM} to overwrite prior knowledge with explicitly stated information or memorization of the answers. We study the error sources further in §\ref{subsec:errors}.  

\noindent \paragraph{\textbf{\emph{F3: Introducing multiple noise sentences has an effect only for logical distractions.}}}
We show the impact of introducing between one and four sentences for the two top-performing autoformalisation models in Figure~\ref{fig:length_distraction}. The figure shows similar trends with and without counterfactual perturbations.
As additional logical distractions are introduced, the model performance consistently decreases. Tautological (T) distractions lead to a decline in accuracy with a single disruptive sentence, yet adding more noise does not worsen the outcome. 
The tautological corpus introduces truth constants for all sentences as a persistent unseen logical construct. Given that this leads only to a decrease for a single occurrence, we can assume that a model can consistently handle the same unseen logical construct. In contrast, the logical corpus increases the chance of adding text, requiring new, previously unseen reasoning constructs for each added sentence. The impact of encyclopedic noise remains negligible, generalising F1 to $k$ sentences. Similarly, counterfactual perturbations remain much more effective for all settings, generalising F2.

\begin{table}[!t]
\small
\setlength{\modelspacing}{2pt}
\setlength{\tabcolsep}{1.7pt} % Default value: 6pt
\setlength{\belowrulesep}{4pt}
\begin{threeparttable}
    \centering
    \begin{tabular}{cc l r rrr @{\quad} rrrr}
\toprule
\multirow{2}{*}{} & \multirow{2}{*}{} & Reasoning & \multirow{2}{*}{O} & \multicolumn{3}{c}{Distraction} & \multicolumn{4}{c}{Counterfactual} \\
 & & Format & & E& L & T & $\text{O}_C$ & $\text{E}_C$& $\text{L}_C$ & $\text{T}_C$\\
\midrule
\multirow{6}{*}{\rotatebox{90}{Gemma-2}} & \multirow{3}{*}{\rotatebox{90}{9b}}
   & Informal (direct) & \textbf{0.78} & \textbf{0.80} & \textbf{0.79} & \textbf{0.77} & 0.58 & 0.52 & 0.50 & 0.59 \\
 & & Informal (CoT) & 0.72 & 0.78 & 0.73 & 0.76 & 0.61 & \textbf{0.57} & \textbf{0.60} & \textbf{0.66} \\
 & & Formal (FOL) & 0.62 & 0.58 & 0.52 & 0.53 & \textbf{0.63} & 0.52 & 0.46 & 0.46 \\[\modelspacing]
\cmidrule{2-11}
 & \multirow{3}{*}{\rotatebox{90}{27b}} 
   & Informal (direct) & 0.71 & 0.69 & \textbf{0.66} & \textbf{0.68} & 0.59 & 0.51 & 0.54 & 0.59 \\
 & & Informal (CoT) & 0.66 & 0.65 & 0.64 & 0.63 & 0.62 & 0.58 & \textbf{0.62} & \textbf{0.64} \\
 & & Formal (FOL) & \textbf{0.74} & \textbf{0.74} & 0.61 & 0.61 & \underline{\textbf{0.72}} & \underline{\textbf{0.67}} & 0.58 & 0.51 \\[\modelspacing]
\midrule
\multirow{6}{*}{\rotatebox{90}{Mistral}} & \multirow{3}{*}{\rotatebox{90}{7B}} 
   & Informal (direct) & 0.77 & \textbf{0.77} & 0.75 & \textbf{0.79} & \textbf{0.63} & \textbf{0.54} & \textbf{0.54} & \textbf{0.66} \\
 & & Informal (CoT) & \textbf{0.79} & 0.75 & \textbf{0.77} & 0.78 & 0.55 & 0.52 & \textbf{0.54} & 0.58 \\
 & & Formal (FOL) & 0.62 & 0.58 & 0.54 & 0.57 & 0.50 & \textbf{0.54} & 0.51 & 0.52 \\[\modelspacing]
\cmidrule{2-11}
 & \multirow{3}{*}{\rotatebox{90}{Small}} 
   & Informal (direct) & \textbf{0.77} & \textbf{0.76} & \textbf{0.76} & \textbf{0.75} & 0.61 & 0.51 & 0.56 & 0.59 \\
 & & Informal (CoT) & 0.72 & 0.72 & 0.72 & 0.71 & \textbf{0.62} & \textbf{0.59} & \textbf{0.62} & \textbf{0.68} \\
 & & Formal (FOL) & 0.68 & 0.59 & 0.53 & 0.64 & 0.54 & 0.55 & 0.49 & 0.51 \\[\modelspacing]
\midrule
\multirow{6}{*}{\rotatebox{90}{Llama-3.1}} & \multirow{3}{*}{\rotatebox{90}{8B}} 
   & Informal (direct) & 0.63 & 0.61 & 0.64 & 0.66 & 0.61 & \textbf{0.62} & 0.59 & 0.61 \\
 & & Informal (CoT) & 0.73 & \textbf{0.73} & \textbf{0.71} & \textbf{0.72} & \textbf{0.62} & 0.59 & \textbf{0.61} & \textbf{0.65} \\
 & & Formal (FOL) & \textbf{0.77} & 0.71 & 0.63 & 0.52 & 0.60 & 0.58 & 0.55 & 0.52 \\[\modelspacing]
\cmidrule{2-11}
 & \multirow{3}{*}{\rotatebox{90}{70B}} 
   & Informal (direct) & 0.77 & 0.74 & 0.74 & 0.73 & 0.62 & 0.53 & 0.56 & 0.64 \\
 & & Informal (CoT) & \textbf{0.78} & \textbf{0.75} & \textbf{0.76} & \textbf{0.76} & 0.64 & 0.61 & \textbf{0.66} & \underline{\textbf{0.73}} \\
 & & Formal (FOL) & 0.74 & 0.73 & 0.71 & 0.71 & \textbf{0.66} & \textbf{0.62} & 0.59 & 0.57 \\[\modelspacing]
 \midrule
\multirow{3}{*}{\rotatebox{90}{GPT}} & \multirow{3}{*}{\rotatebox{90}{4o-mini}} 
   & Informal (direct) & 0.78 & 0.77 & 0.79 & 0.79 & 0.64 & 0.61 & 0.61 & 0.63 \\
 & & Informal (CoT) & 0.80 & 0.80 & \underline{\textbf{0.81}} & \underline{\textbf{0.82}} & \textbf{0.68} & \textbf{0.63} & \underline{\textbf{0.68}} & \textbf{0.64} \\
 & & Formal (FOL) & \underline{\textbf{0.84}} & \underline{\textbf{0.82}} & 0.73 & 0.79 & 0.63 & 0.62 & 0.57 & 0.54 \\[\modelspacing]
 \midrule
\multicolumn{2}{c}{\multirow{3}{*}{\textbf{Avg}}} 
 & Informal (direct) & 0.74 & 0.73 & 0.73 & 0.73 & 0.61 & 0.55 & 0.56 & 0.62 \\
 & & Informal (CoT) & 0.74 & 0.74 & 0.73 & 0.74 & 0.62 & 0.58 & 0.62 & 0.65 \\
  & & Formal (FOL) & 0.72 & 0.68 &	0.61 & 0.62 & 0.61 & 0.59 & 0.54 & 0.52 \\
\bottomrule
\end{tabular}
\caption{Accuracies of informal and autoformalisation-based deductive reasoners. The best overall model per dataset is underlined; the best model version is marked in bold.}
\label{tab:distraction_k4_formalisation}
\end{threeparttable}
\end{table} 

\begin{figure}[!t]
    \centering
    \scriptsize
    \begin{tikzpicture}
        \begin{axis}[name=gpt,
            title={GPT-4o-mini},
            width=0.6\linewidth,
            height=0.6\linewidth,
            xlabel={\# Noise sentences},
            ylabel={Accuracy},
            xmin=-0.1, xmax=4.1,
            ymin=0.5, ymax=0.9,
            xtick={1,2,4},
            ytick={0.55, 0.6, 0.65, 0.75, 0.8, 0.85},
            title style={yshift=-0.6em},
            legend style={at={(1,-0.15)},
	           anchor=north,legend columns=-1},
            x label style={at={(axis description cs:1,-0.05)},anchor=north},
            y label style={at={(axis description cs:-0.15,0.5)},anchor=south},
            ymajorgrids=true,
            grid style=dashed,
        ]
            \addplot[color=blue, mark=square,]
                coordinates {
                (0,0.848076939582825)(1,0.823076903820038)(2,0.826923072338104)(4,0.821153819561005)
                };
            \addplot[color=red, mark=triangle,]
                coordinates {
                (0,0.848076939582825)(1,0.817307710647583)(2,0.801923096179962)(4,0.759615361690521)
                };
            \addplot[color=green, mark=diamond,] 
                coordinates {
                (0,0.848076939582825)(1,0.767307698726654)(2,0.769230782985687)(4,0.803846180438995)
                };
            \addplot[color=blue, mark=square*] 
                coordinates {
                (0,0.627777755260468)(1,0.622222244739533)(2,0.600000023841858)(4,0.633333325386047)
                };
            \addplot[color=red, mark=triangle*,] 
                coordinates {
                (0,0.627777755260468)(1,0.611111104488373)(2,0.611111104488373)(4,0.594444453716278)
                };
            \addplot[color=green, mark=diamond*,] 
                coordinates {
                (0,0.627777755260468)(1,0.572222232818604)(2,0.538888871669769)(4,0.555555582046509)
                };
                \legend{E,L,T,$\text{E}_C$, $\text{L}_C$ , $\text{T}_C$}
        \end{axis}

        \begin{axis}[name=llama, at={($(gpt.east)+(0.1cm,0)$)},anchor=west,
            title={Llama 3.1 70b},
            width=0.6\linewidth,
            height=0.6\linewidth,
            xmin=-0.1,, xmax=4.1,
            ymin=0.5, ymax=0.9,
            xtick={1,2,4},
            ytick={0.55, 0.6, 0.65, 0.75, 0.8, 0.85},
            title style={yshift=-0.6em},
            yticklabel=\empty,
            ymajorgrids=true,
            grid style=dashed,
        ]
            \addplot[color=blue, mark=square,]
                coordinates {
                (0,0.838461518287659)(1,0.817307710647583)(2,0.805769205093384)(4,0.817307710647583)
                };
            \addplot[color=red, mark=triangle,]
                coordinates {
                (0,0.838461518287659)(1,0.819230794906616)(2,0.803846180438995)(4,0.771153867244721)
                };
            \addplot[color=green, mark=diamond,]
                coordinates {
                (0,0.838461518287659)(1,0.803846180438995)(2,0.807692289352417)(4,0.805769205093384)
                };
            \addplot[color=blue, mark=square*]
                coordinates {
                (0,0.627777755260468)(1,0.622222244739533)(2,0.577777802944183)(4,0.594444453716278)
                };
            \addplot[color=red, mark=triangle*,]
                coordinates {
                (0,0.627777755260468)(1,0.583333313465118)(2,0.561111092567444)(4,0.577777802944183)
                };
            \addplot[color=green, mark=diamond*,]
                coordinates {
                (0,0.627777755260468)(1,0.627777755260468)(2,0.566666662693024)(4,0.577777802944183)
                };
        \end{axis}
    \end{tikzpicture}
    \caption{Influence of the number of noisy sentences for FOL.}
    \label{fig:length_distraction}
\end{figure}



\subsection{Impact of Method Design}
\paragraph{\textbf{\emph{F4: \ac{CoT} prompting is most impactful when both noise and counterfactual perturbations are applied.}}}
The accuracies for the individual \acp{LLM} in Table~\ref{tab:distraction_k4_formalisation} show that the impact of \ac{CoT} is negligible for noise-only datasets (first four columns). Meanwhile, the benefit from \ac{CoT} is most pronounced in the datasets that combine noise and counterfactual perturbations.
The better-performing informal prompting strategy for a model remains stable for all types of distractions. Still, the decline in performance due to counterfactuals leads to a less consistent preference for a specific prompting style.

\paragraph{\textbf{\emph{F5: The best-performing grammar differs per model and is unstable across data versions.}}}

The evaluation of different logical forms for formal \ac{LLM}-based reasoning in Table~\ref{tab:distraction_k4_logical_form} shows the preference of some models for specific syntactic formats.
Llama 3.1 70B has a considerable improvement of $12\%$ with TPTP syntax on the original set, while Llama 3.1 8B benefits from the R-FOL syntax. However, all grammars show a declining accuracy trend and increased syntax errors for noise perturbations, where the best grammar loses its advantage over the rest. 
When comparing the grammars on the counterfactual partitions, we observe that TPTP is consistently more robust than the standard first-order logic grammar. Here, GPT 4o-mini shows a reduction from $O$ to $O_C$ of $20\%$ for FOL and only $12\%$ for the TPTP grammar. Since this does not correlate with fewer syntax errors, the formalisation in TPTP prevents semantical errors for counterfactual premises. 
A positive reading of these results, especially the minor differences between FOL and R-FOL, is that autoformalisation \acp{LLM} can adapt to the grammar syntax prescribed in the prompt without further loss in performance.

\begin{table}[!t]
\small
\setlength{\modelspacing}{2pt}
\setlength{\tabcolsep}{1.7pt} % Default value: 6pt
\setlength{\belowrulesep}{4pt}
\begin{threeparttable}
    \centering
    \begin{tabular}{cc l r rrr @{\quad} rrrr}
\toprule
\multirow{2}{*}{} & \multirow{2}{*}{} & Grammar & \multirow{2}{*}{O} & \multicolumn{3}{c}{Distraction} & \multicolumn{4}{c}{Counterfactual} \\
 & & Syntax & & E& L & T & $\text{O}_C$ & $\text{E}_C$& $\text{L}_C$ & $\text{T}_C$\\
\midrule
\multirow{6}{*}{\rotatebox{90}{Llama-3.1}} & \multirow{3}{*}{\rotatebox{90}{8B}} 
   & FOL & 0.77 & \textbf{0.71} & 0.61 & \textbf{0.53} & 0.58 & \textbf{0.55} & 0.52 & \textbf{0.56} \\
 & & R-FOL & \textbf{0.78} & 0.69 & \textbf{0.62} & \textbf{0.53} & 0.58 & \textbf{0.55} & \textbf{0.54} & 0.52 \\
 & & TPTP & 0.73 & 0.67 & 0.55 & 0.51 & \textbf{0.68} & 0.54 & 0.46 & 0.51 \\[\modelspacing]
\cmidrule{2-11}
 & \multirow{3}{*}{\rotatebox{90}{70B}} 
   & FOL & 0.76 & 0.73 & 0.71 & \textbf{0.72} & 0.67 & 0.57 & 0.63 & 0.56 \\
 & & R-FOL & 0.76 & 0.73 & 0.67 & 0.71 & 0.64 & 0.57 & 0.53 & 0.64 \\
 & & TPTP & \underline{\textbf{0.88}} & \underline{\textbf{0.84}} & \underline{\textbf{0.81}} & \textbf{0.72} & \underline{\textbf{0.81}} & \underline{\textbf{0.68}} & \underline{\textbf{0.67}} & \underline{\textbf{0.68}} \\[\modelspacing]
\midrule
\multirow{3}{*}{\rotatebox{90}{GPT}} & \multirow{3}{*}{\rotatebox{90}{4o-mini}} 
   & FOL & \textbf{0.84} & \textbf{0.82} & \textbf{0.72} & \underline{\textbf{0.78}} & 0.64 & \textbf{0.63} & \textbf{0.61} & 0.51 \\
 & & R-FOL & \textbf{0.84} & 0.77 & 0.70 & \underline{\textbf{0.78}} & \textbf{0.72} & 0.56 & 0.54 & \textbf{0.63} \\
 & & TPTP & 0.83 & \textbf{0.82} & 0.71 & 0.71 & 0.69 & \textbf{0.63} & 0.57 & 0.57 \\
\bottomrule
\end{tabular}
\caption{Accuracies of different formalisation grammars for autoformalisation.}
\label{tab:distraction_k4_logical_form}
\end{threeparttable}
\end{table} 

\paragraph{\textbf{\emph{F6: Feedback does not help \acp{LLM} self-correct to mitigate robustness issues.}}}
\autoref{tab:distraction_k4_feedback} shows the results with different error recovery mechanisms. The results indicate that no feedback strategy emerges as a winner in the different datasets. 
All feedback variants reduce syntax errors for noise perturbations, but given the lack of a consistent increase in accuracy, the corrected formalisations are most likely to contain semantic errors still. 
The type of feedback message only has a minor influence on correcting syntax errors, whereas Llama 3.1 70b and GPT 4o-mini correct slightly more syntax errors with specific error messages. This finding aligns with \cite{huang2023large}, who also found that \acp{LLM} cannot consistently self-correct their reasoning after receiving relevant feedback.

\begin{table}[!ht]
\small
\setlength{\modelspacing}{2pt}
\setlength{\tabcolsep}{1.7pt} % Default value: 6pt
\setlength{\belowrulesep}{4pt}
\begin{threeparttable}
    \centering
    \begin{tabular}{cc l r rrr @{\quad} rrrr}
\toprule
\multirow{2}{*}{} & \multirow{2}{*}{} & \multirow{2}{*}{Feedback} & \multirow{2}{*}{O} & \multicolumn{3}{c}{Distraction} & \multicolumn{4}{c}{Counterfactual} \\
 & & & & E& L & T & $\text{O}_C$ & $\text{E}_C$& $\text{L}_C$ & $\text{T}_C$\\
\midrule
\multirow{8}{*}{\rotatebox{90}{Llama-3.1}} & \multirow{4}{*}{\rotatebox{90}{8B}} 
   & No recovery & 0.77 & \textbf{0.72} & 0.62 & 0.53 & 0.59 & 0.58 & 0.56 & \textbf{0.56} \\
 & & Error type & \textbf{0.79} & 0.71 & 0.63 & \textbf{0.56} & \textbf{0.66} & 0.54 & 0.52 & 0.51 \\
 & & Error message & 0.78 & 0.71 & \textbf{0.67} & 0.55 & 0.59 & 0.53 & \underline{\textbf{0.64}} & 0.49 \\
 & & Warning & 0.74 & 0.66 & 0.58 & 0.55 & 0.55 & \textbf{0.60} & 0.49 & 0.49 \\[\modelspacing]
\cmidrule{2-11}
 & \multirow{4}{*}{\rotatebox{90}{70B}} 
   & No recovery & \textbf{0.77} & \textbf{0.72} & \textbf{0.73} & 0.71 & \textbf{0.64} & 0.59 & \textbf{0.61} & 0.56 \\
 & & Error type & 0.72 & 0.70 & 0.72 & \textbf{0.73} & 0.62 & 0.56 & 0.60 & 0.58 \\
 & & Error message & 0.71 & 0.70 & \textbf{0.73} & 0.71 & \textbf{0.64} & 0.59 & 0.54 & \underline{\textbf{0.64}} \\
 & & Warning & 0.69 & \textbf{0.72} & 0.72 & 0.72 & 0.62 & \underline{\textbf{0.65}} & \textbf{0.61} & 0.63 \\[\modelspacing]
\midrule
\multirow{4}{*}{\rotatebox{90}{GPT}} & \multirow{4}{*}{\rotatebox{90}{4o-mini}} 
   & No recovery & \underline{\textbf{0.84}} & \underline{\textbf{0.82}} & 0.73 & 0.79 & 0.64 & \textbf{0.62} & 0.56 & \textbf{0.56} \\
 & & Error type & 0.83 & 0.79 & 0.74 & 0.76 & 0.67 & 0.57 & 0.56 & \textbf{0.56} \\
 & & Error message & \underline{\textbf{0.84}} & 0.78 & \underline{\textbf{0.77}} & \underline{\textbf{0.80}} & 0.62 & 0.59 & 0.56 & \textbf{0.56} \\
 & & Warning & \underline{\textbf{0.84}} & 0.75 & 0.73 & 0.76 & \underline{\textbf{0.70}} & 0.61 & \textbf{0.61} & 0.55 \\
 \bottomrule
\end{tabular}
\caption{Accuracies of error recovery strategies.}
\label{tab:distraction_k4_feedback}
\end{threeparttable}
\end{table} 

\subsection{Error Analysis}
\label{subsec:errors}
\paragraph{\textbf{\emph{F7: Autoformalisation increases syntax errors for noise perturbations.}}}
The low performance for noise perturbations correlates with more syntax errors for all models and distraction categories (cf. execution rates in Table~\ref{tab:appendix_k4_formalisation_exec}). The three worst-performing models (both Mistral models, Gemma-2 9b) generate, at best, for $37\%$  and, at worst, for only $4\%$ of the samples, a valid logical form.
Gemma-2 9b and Llama3.1 8b produce more syntax errors than the larger counterparts, suggesting that larger models are more robust towards noise perturbations. 
The accuracy of syntactically valid samples is higher than the informal reasoning methods for most distractions (Table~\ref{tab:appendix_k4_formalisation_vacc}), motivating informal reasoning as a backup strategy for formal reasoning. The error message feedback reveals two common syntax errors: 1) errors by models with an initial low execution rate exhibit issues with the template structure, including using incorrect keywords or adding conversational phrases;
2) perturbation-related errors, the most common of which is using undefined truth constants as part of tautological distractions. 

\paragraph{\textbf{\emph{F8: Autoformalisation increases semantic errors for counterfactuals.}}}
Unlike the introduced noise, counterfactual perturbations do not lead to more syntax errors. The execution rate in Table~\ref{tab:appendix_k4_formalisation_exec} is stable or improves for counterfactuals. However, we see a drop in accuracy for the counterfactual column $\text{O}_C$ in Table~\ref{tab:distraction_k4_formalisation} and can conclude that the number of logical forms with semantic errors has to increase. This suggests that the introduced negation is not correctly formalised. Looking at the warnings generated by the feedback mechanism, for GPT 4o-mini, $161$ warning messages are generated on the unperturbed data. $54$ of these were fixed with a single iteration. Not considering predicates and individuals as part of the context is the most frequent warning across all models. 
\section{Software library for NVM data prefetching}
\label{sec:library}
\noindent The data readiness problem is likely to appear in any application that directly accesses large pools of data residing in NVM. In this section, we provide details on our general purpose data prefetching library, that aims to aid programmability to easily prefetch desired data regions. The library uses POSIX threads and implements a simple API to create, control and assign jobs to threads.

\subsection{Helper Threads}

Prefetching data~\cite{annavaram2001data} reduces the cache miss rate and hence
accelerates an application\textquotesingle s execution. An in-advance knowledge of memory regions
to be accessed can be used to prefetch data into caches before it is needed. However,
the application should not be stalled while prefetching the data. This can be achieved
by using independent helper threads for data prefetching.

Synchronization between the main computation thread and helper thread(s) is 
important~\cite{jung2006helper}. Prefetching data too early before it is needed
by the computation thread can result in cache pollution. Furthermore, required
cache lines may get evicted before they are accessed. Similarly, prefetching data
too late is also not useful, rather counter-productive. It can also lead to cache
pollution and degrade performance.

In our implementation, a helper thread is a simple block of code. Given a starting
memory address and the amount of data to be prefetched, the helper thread prefetches
data into caches without interfering with the main computation thread. We employ a
job queue to build a single producer - (multiple) consumer relationship between
a computation thread and one or multiple helper threads. We employ light-weight 
compare and swap instructions for synchronization in the job queue between a 
computation thread and the different helper threads.

A computation thread places the starting address and the amount of data to be prefetched into a job unit and enqueues it into the job queue. On the other end, a helper thread picks the job item, unpacks it and then prefetches the data into caches. Data prefetching is performed without stalling the computation thread.

\subsection{Library Services}
A programmer is responsible for inserting API calls to construct helper threads. However, as SE2 already has knowledge of the size and location of the block to be read, inserting data prefetching APIs in SE2 source code is not a tedious task. Our library provides three basic services via a simple API.

\begin{enumerate}%[leftmargin=*]
 \item \textbf{Creation of helper threads:} The library supports the  creation of a user-specified number of helper threads per computation thread, that are synchronized using a job queue. A slightly different thread creation policy is also supported, as explained in detail in Section 7.3.
 \item \textbf{Assigning work to helper threads:} Work is assigned to helper threads by placing jobs into their job queue. On arrival of a job into the queue, helper threads wake up, one fetches the job from the queue and starts the data prefetching. After completing the job, the helper thread again waits for the next job\textquotesingle s arrival if the queue is empty.
 \item \textbf{Mapping threads to cores:} Our library also supports the selection of a core (in a multicore platform) on which a particular helper thread is to be executed by setting thread affinities. The affinities can also be set with respect to the computational thread, i.e., selecting the same or a different core.
\end{enumerate}

\subsection{Thread Mapping Schemes for Our Case Study}
\label{sec:mapping-schemes}


\noindent As discussed in Section~\ref{sec:evaluation}, \emph{SE2} directly accesses data located in NVM disk, without copying it into any local buffer. This direct access results in high cache miss rates when the data is needed for processing. However, due to ad-hoc data prefetching, SE2 achieves performance improvements for a few queries (i.e., Q11, Q15,  and Q19), but not for the majority of them. By performing ad-hoc data prefetching we were able to prefetch some blocks into caches, but not most of them due to the overheads it entailed performing the prefetching inside the computation thread. Furthermore, ad-hoc placement of data prefetching in the source code of an application can be tedious and difficult to maintain.

By using our data prefetching library services, data can be brought closer to processing cores before 
it is needed. When accessing a file for a read operation, \emph{SE2} creates a memory mapping of the file 
using \verb+mmap()+. Additionally, it has knowledge of the location and size of the data block to be read. Therefore, \emph{SE2} can pack this information into a job and place it into the job queue to be processed by helper threads. By using this approach, PostgreSQL can continue with its computation while the required data is prefetched into caches by a helper thread. 

Due to the way PostgreSQL is structured, it is not necessary to have more than one helper thread active to service a queued job in time before the next arrives. Therefore, we propose two mappings that employ only one helper thread and a third mapping that employs a different thread creation policy by instantiating two helper threads that work in tandem as we explain in the following subsections.

\begin{figure*} %[htbp]
\centering     %%% not \center
\subfigure[\emph{M1} - Different physical core]{\label{CaseA}\includegraphics[width=32mm]{ThreadMapping_CaseA.eps}}
\subfigure[\emph{M2} - Same physical core (HT)]{\label{CaseB}\includegraphics[width=52mm]{ThreadMapping_CaseB.eps}}
\subfigure[\emph{M3} - Thread mapping for the two helper threads scheme]{\label{CaseE}\includegraphics[width=87mm]{ThreadMapping_CaseE.eps}}
\caption{Different Helper thread mapping schemes with and without hyper-threading (HT) enabled}
\label{Thread-Mapping}
\end{figure*}


\subsubsection{Single helper thread}



When using a single helper thread there are two options for thread mapping. Mapping it to a different core than that of the computation thread, or to the same core. The latter option makes sense if the target machine supports hyper-threading - i.e., two hardware thread contexts per core. We explore these thread mappings:
\begin{enumerate}
 \item \verb+M1+ - Map helper thread to a different physical core, as shown in Fig.~\ref{CaseA}. In this case, each thread resides in a different core and hence prefetching is not done at the level of private caches but at the level of the last level cache, which is shared across cores.
 \item \verb+M2+ - Map helper thread into the same physical core while making use of hyper-threading, as shown in Fig.~\ref{CaseB}. Both threads reside within the same physical core, hence prefetching will also populate the private L1 and L2 caches present in the core.
\end{enumerate}

\subsubsection{Two helper threads}

As described for case \emph{M2}, the helper thread prefetches data into the private caches of the core where the computation thread is executing. However, in this scenario, the helper thread competes with the computation thread for hardware resources. This competition can slow down the execution of the computation thread. On the other hand, for case \emph{M1} there is no such competition for hardware resources at the expense of prefetching into the LLC, further away from the processing core.

A good compromise can be achieved using two helper threads working together. One helper thread is mapped to a different physical core than that of the computation thread and will process the jobs enqueued by the computation thread, similar to case \emph{M1}. Once this first thread finishes processing the job, it enqueues the same job into a second job queue that is processed by the second helper thread that is mapped into the same physical core as the computation thread. This thread mapping scheme which we term \verb+M3+ is shown in Fig.~\ref{CaseE}. The rationale behind this proposal is that the high penalty miss from main memory to LLC will be paid by a different core, while the helper thread residing on the same core as the computation thread will put data into the private caches and complete jobs much faster since data will be already present in the LLC.

\section{Evaluation of the data prefetching library}
\label{sec:library-evalualtion}

\begin{figure*}
\centering
\includegraphics[width=\linewidth]{kernel-exec-percent-pg-new.eps}
\caption{Percentage of kernel execution time for PostgreSQL thread}
\label{kernel-time-pg}
\end{figure*}

\noindent In this section, we evaluate our modified storage engine \emph{SE2} while using the data prefetching library. 
For the scenarios in which we employ the library, we remove the simple ad-hoc software prefetching scheme used in the previous evaluation. However, the \emph{pmfs\_se2} system to which we compare does include the same ad-hoc prefetching used in the previous evaluation (Section~\ref{sec:evaluation}). The test machine and methodology employed is the same as explained in Section~\ref{sec:methodology}, with the exception that hyper-threading (HT) is enabled for \emph{M2} and \emph{M3}.



\subsection{Performance Impact on Kernel Execution Time}

Fig. \ref{kernel-time-pg} shows the percentage of kernel execution time of the PostgreSQL thread (computation thread) for each of the evaluated queries. As explained in Section~\ref{sec:evaluation}, \emph{SE2} only redirects the buffer pointer for file read operation from the local buffer cache to an NVM disk address that is within the address space of the PostgreSQL process. Hence, there is no data movement at the kernel level. As a result, the involvement of kernel in data movement and hence the average percentage of kernel execution time is already low in \emph{pmfs\_se2} as compared to \emph{pmfs\_base95}, reducing from 10\% to 3\%.

When using \verb+M1+, \verb+M2+, and \verb+M3+ helper thread schemes, by offloading the prefetching of entire blocks of data to helper threads, we can further hide kernel execution time overheads for the PostgreSQL thread running the query. Helper threads are more effective in prefetching data than the ad-hoc scheme used in pmfs\_se2. Therefore, the average percentage of kernel execution time further reduces from 3\% for \emph{pmfs\_se2} to 0.5\% in \verb+M1+, \verb+M2+, and \verb+M3+.

There is no noticeable difference between the three thread mapping schemes in terms of percentage of kernel execution time. The reason is that all three mapping schemes place data blocks at least into the LLC, which is enough to hide kernel related events such as page faults. 

\begin{figure*}
\centering
\includegraphics[width=\linewidth]{exec-time-new.eps}
\caption{Wall-clock execution time normalized with respect to pmfs\_base95}
\label{exec-time-new}
\end{figure*}
\begin{figure*}
\centering
\includegraphics[width=\linewidth]{compute-stall-pg.eps}
\caption{Execution time breakdown into compute and stall cycles for PostgreSQL thread, normalized with respect to pmfs\_base95}
\label{compute-stall-pg}
\end{figure*}

\subsection{Query Performance Improvement}\label{QPI}

Fig.~\ref{exec-time-new} shows the wall-clock query execution time normalized with respect to \emph{pmfs\_base95} for all queries. We can observe that for queries in which \emph{pmfs\_se2} obtained better performance, the new evaluated systems with our prefetch library overall obtain better execution times, especially for the \verb+M3+ thread mapping scheme. M3 shows noticeable performance improvements for queries where the sequential scan operation represents a significant fraction of the total database operations - i.e. Q03-Q12, Q15, and Q19, as shown in Fig. \ref{query-breakdown}. On the other hand, queries where sequential scan operation consume less time (i.e. Q01, Q02, Q13, Q17, and Q20), show no performance improvement. On average, M3 obtains an 8\% performance improvement over the baseline. \verb+M1+ and \verb+M2+ show up to 13\% performance improvement (Q11), with an average of 6\% when compared to pmfs\_base95. 

Query execution time is mostly affected by two factors: cache misses and competition for hardware resources between threads mapped on the same physical core. Reducing any of these two factors should lead to better query execution times. To understand the improvements seen in Fig.~\ref{exec-time-new}, we provide insights in terms of compute and stalled core cycles and L1 cache misses for the PostgreSQL thread in Fig.~\ref{compute-stall-pg} and Fig.~\ref{L1-misses-pg}, respectively.

\begin{figure*}
\centering
\includegraphics[width=\linewidth]{L1-misses-total.eps}
\caption{L1 cache misses for PostgreSQL thread normalized with respect to pmfs base95}
\label{L1-misses-pg}
\end{figure*}

Fig.~\ref{compute-stall-pg} shows an execution cycle break down into the stall and compute cycles for all evaluated queries, but just for the PostgreSQL thread. An execution cycle is classified as `compute', if at least one instruction is committed during that cycle, or as `stalled' otherwise. Both \verb+M1+ and \verb+M2+ generate similar results in terms of wall-clock execution time. In Fig.~\ref{compute-stall-pg}, we can observe that both are able to reduce the kernel stall and compute components due to less kernel involvement in the main computation thread since the prefetching and kernel related events such as page faults are handled by the helper thread. However, the user level components are still very similar. This is because, as shown in Fig.~\ref{L1-misses-pg}, the actual number of L1 cache misses is also similar despite helper threads prefetching at different levels of the memory hierarchy. We attribute this to hardware prefetchers being much more efficient once the data is already in the LLC. M3 shows better performance than M1 and M2 as it maps the helper-thread, which handles the page faults, on a core different than that of compute thread as shown in Fig. \ref{CaseE}. Nonetheless, Fig.~\ref{L1-misses-pg} shows a large L1 cache misses reduction when compared to both \emph{pmfs\_base95} and \emph{pmfs\_se2}, proving that the prefetching library is performing well in hiding them from the main computation thread.

We can see in Fig.~\ref{compute-stall-pg} that \verb+M3+, besides being able to reduce the kernel components, also reduces the time spent in stalled user significantly, e.g., in Q03, Q05, Q06, Q12, and others. This is because of two factors: (i) prefetching into the L1 cache with our library is more timely than hardware prefetching, which might still be in-flight when the data is actually needed; and (ii) by using the two thread mapping approach we are able to reduce the overhead of the helper thread that is sharing the same physical core with the computation thread. Therefore, the first helper thread in \verb+M3+ brings the data into the LLC without interfering with the PostgreSQL thread that is running on a different physical core. While the second thread running on the same core brings the data into the private caches while incurring a lower overhead due to lower latency misses.

\begin{figure*}
\centering
\includegraphics[width=\linewidth]{results_Helper1.eps}
\caption{Execution time breakdown into compute and stall cycles for 
helper thread running on same physical core as PostgreSQL in M2 and M3. Execution time is normalized with respect to M2}
\label{compute-stall-helper}
\end{figure*}

To support this explanation, Fig.~\ref{compute-stall-helper} shows the compute-stall cycle breakdown for the helper thread that shares the physical core with the computation PostgreSQL thread for \verb+M2+ and \verb+M3+ schemes. We can observe that for the queries in which \verb+M3+ performs better (e.g., Q03, Q05, Q06, and Q12), the helper thread sharing the core is more lightweight. In particular, the helper thread in \verb+M3+ does not suffer from kernel noise since the kernel related events like page faults are sorted by the other helper thread running on a different core, leading to a load reduction in the computation core. Also, a reduction in user stalled cycles can be observed (e.g., Q11 and Q15) due to lower latency memory accesses.

We find that our library is able to help improve the performance with little programming effort. The ad-hoc prefetching scheme used in \emph{pmfs\_se2}, while simple, still required code analysis to place the prefetch instructions in different places in order to maximize the number of data blocks prefetched without stalling too much the computations. With our library the creation and mapping of threads is done only once, and the creation of jobs to be enqueued came as a natural fit in a single point of the source code, i.e., when the needed block is memory mapped.

\section{Related Work}
\subsection{Multimodal Large Language Models}
% Building on the success of large language models (LLMs) \citep{yao2024tree, glm2024chatglm, achiam2023gpt, touvron2023llama, brown2020language}, multimodal large language models (MLLMs) \citep{liu2024improved, li2023blip, zhu2023minigpt, wang2023cogvlm, liu2024visual} extend these capabilities by integrating vision and text processing, achieving remarkable performance in tasks involving images, videos, and multimodal reasoning. However, handling visual data poses computational challenges due to the redundancy and low information density of high-resolution tokens \citep{liang2022evit} and the quadratic scaling of attention mechanisms \citep{vaswani2017attention}.
% For instance, models like LLaVA \citep{liu2023improvedllava} and mini-Gemini-HD \citep{li2024mini} encode high-resolution images into thousands of tokens, while video-based models such as VideoLLaVA \citep{lin2023video} and VideoPoet \citep{kondratyuk2023videopoet} allocate even more tokens to process multiple frames. These challenges highlight the need for more efficient token representations and longer context lengths to enable scalability. Recent advancements, such as Gemini \citep{geminiteam2023gemini} and LWM \citep{liu2024world}, have focused on addressing these issues by optimizing token efficiency and extending the context length, paving the way for more scalable and effective MLLMs.

The remarkable success of large language models (LLMs) \citep{radford2019language, brown2020language} has spurred a growing trend of extending their advanced reasoning capabilities to multi-modal tasks, leading to the development of vision-language models (VLMs) \citep{huang2023languageneedaligningperception, driess2023palmeembodiedmultimodallanguage, liu2024visual, Qwen-VL}. These VLMs typically consist of a visual encoder \citep{radford2021learning} that serializes input image representations and an LLM responsible for text generation. To enable the LLM to process visual inputs, an alignment module is employed to bridge the gap between visual and textual modalities. This module can take various forms, such as a simple linear layer, an MLP projector, or a more complex query-based network. While this integration allows the LLM to gain visual perception, it also introduces significant computational challenges due to the long sequences of visual tokens.

Moreover, existing VLMs often exhibit limitations, such as visual shortcomings or hallucinations, which hinder their performance. Efforts to enhance VLM capabilities by increasing input image resolution have further exacerbated computational demands. For instance, encoding higher-resolution images results in a substantial increase in the number of visual tokens. A model like LLaVA-1.5 \citep{liu2024improved} generates 576 visual tokens for a single image, while its successor, LLaVA-NeXT \citep{liu2024llavanext}, produces up to 2880 tokens at double the resolution, far exceeding the length of typical textual prompts.
Optimizing the inference efficiency of VLMs is thus a critical task to facilitate their deployment in real-world scenarios with limited computational resources.

\subsection{Visual Token Compression}
% Visual tokens often exceed text tokens by tens to hundreds of times, with visual signals being more spatially redundant compared to information dense text \citep{marr2010vision}.
% Various methods have been proposed to address this issue. For instance, LLaMA-VID \citep{li2023llama} uses a Q-Former with context tokens, and DeCo \citep{yao2024deco} applies adaptive pooling to downsample visual tokens at the patch level.
% However, these approaches require modifying model components and additional training, increasing computational and training costs.
% ToMe~\citep{bolya2022tome} reduces tokens without training by adding a token merge module to ViTs, but this disrupts early cross-modal interactions in language models~\citep{xing2024PyramidDrop}. FastV~\citep{chen2024image} selects important visual tokens using attention scores, while SparseVLM~\citep{zhang2024sparsevlm} incorporates text guidance via cross-modal attention.
% However, these methods forgo flash-attention~\citep{dao2022flashattention, dao2023flashattention2} and primarily focus on token importance, overlooking the impact of token duplication.
% In our work, we preserve hardware acceleration compatibility, including flash attention, while considering both token importance and duplication for token reduction.

Visual tokens are often significantly more numerous than text tokens, with higher spatial redundancy and lower information density. To address this issue, various methods have been proposed for reducing visual token counts in vision language models. For instance, some approaches modify model components, such as using context tokens in Q-Former \citep{li2023llama} or applying adaptive pooling at the patch level, but these typically require additional training and increase computational costs. Other techniques, like Token Merging (ToMe) \citep{bolya2022tome} and FastV \citep{chen2024image}, focus on reducing tokens without retraining by merging tokens or selecting important ones based on attention scores. SparseVLM \cite{zhang2024sparsevlm} incorporates text guidance through cross-modal attention to refine token selection. However, these methods often overlook hardware acceleration compatibility and fail to account for token duplication alongside token importance. Furthermore, while token pruning has been extensively explored in natural language processing and computer vision to improve inference efficiency, its application to VLMs remains under-explored. Existing pruning strategies, such as those in FastV and SparseVLM, rely on text-visual attention within large language models (LLMs) to evaluate token importance, which may not align well with actual visual token relevance.


\section*{Conclusion}
This paper aims to enhance our understanding of the computational complexity of computing various Shapley value variants. We found that for various ML models --- including decision trees, regression tree ensembles, weighted automata, and linear regression --- both local and global interventional and baseline SHAP can be computed in polynomial time under HMM modeled distributions. This extends popular algorithms, such as TreeSHAP, beyond their empirical distributional scope. We also establish strict complexity gaps between the various SHAP variants (baseline, interventional, and conditional) and prove the intractability of computing SHAP for tree ensembles and neural networks in simplified scenarios. Overall, we present SHAP as a versatile framework whose complexity depends on four key factors: \begin{inparaenum}[(i)] \item model type, \item SHAP variant, \item distribution modeling approach, \item and local vs. global explanations\end{inparaenum}. We believe this perspective provides deeper insight into the computational complexity of SHAP, paving the way for future work.




%We believe that our framework provides a more intricate understanding of SHAP computation complexity across different models, distributions, and variants, paving the way for further research.

Our work opens promising directions for future research. First, expanding our computational analysis to other SHAP-related metrics, such as asymmetric SHAP~\citep{frye20} and SAGE~\citep{covert2020understanding}, would be valuable. Additionally, we aim to explore more expressive distribution classes and relaxed assumptions beyond those in Section \ref{sec:tractable} while maintaining tractable SHAP computation. Finally, when exact computation is intractable (Section \ref{sec:intractable}), investigating the approximability of SHAP metrics through approximation and parameterized complexity theory~\citep{downey2012parameterized} is an important direction.

%Our work opens several promising avenues for future research on the computational properties of explainable AI methods, with a particular focus on SHAP. First, it would be interesting to broaden the computational analysis conducted in this work to include other popular SHAP-related metrics in the literature, such as asymmetric SHAP \cite{frye20} and SAGE \cite{covert2020understanding}. Also, in the future, we aim to explore more expressive distribution classes and relaxed distributional assumptions—extending beyond those examined in Section \ref{sec:tractable} —that still yield tractable SHAP computation. Finally, when exact computation proves intractable (Section \ref{sec:intractable}), it is worthwhile to theoretically investigate the question of the approximability of computing the SHAP metrics across various configurations, through the lens of approximation and parametrized complexity theory \cite{arora2009computational}.

%This paper aims to deepen our understanding of the computational complexity involved in obtaining different Shapley value variants. We found that for a variety of ML models, including decision trees, tree ensembles for regression, weighted automata, and linear regression models — computing both local and global interventional and baseline SHAP can be done in polynomial time when distributions are modeled by HMMs. This extends the distributional scope of popular algorithms like TreeSHAP, which is limited to empirical distributions. Additionally, we demonstrate a strict complexity gap between SHAP variants, showing that interventional and baseline SHAP can be strictly easier to compute than conditional SHAP. Despite these positive results, we uncovered intractability for various SHAP variants in neural networks and tree ensembles. Finally, we provided generalized complexity relations across SHAP variants. We believe that our framework offers a deeper understanding of the complexity involved in computing SHAP across various variants, models, distributions, as well as in both local and global computations, laying the groundwork for future research.







%\section{Introduction}
%\label{intro}
%Your text comes here. Separate text sections with
%\section{Section title}
%\label{sec:1}
%Text with citations \cite{RefB} and \cite{RefJ}.
%\subsection{Subsection title}
%\label{sec:2}
%as required. Don't forget to give each section
%and subsection a unique label (see Sect.~\ref{sec:1}).
%\paragraph{Paragraph headings} Use paragraph headings as needed.
%\begin{equation}
%a^2+b^2=c^2
%\end{equation}
%
%% For one-column wide figures use
%\begin{figure}
%% Use the relevant command to insert your figure file.
%% For example, with the graphicx package use
%  \includegraphics{example.eps}
%% figure caption is below the figure
%\caption{Please write your figure caption here}
%\label{fig:1}       % Give a unique label
%\end{figure}
%%
%% For two-column wide figures use
%\begin{figure*}
%% Use the relevant command to insert your figure file.
%% For example, with the graphicx package use
%  \includegraphics[width=0.75\textwidth]{example.eps}
%% figure caption is below the figure
%\caption{Please write your figure caption here}
%\label{fig:2}       % Give a unique label
%\end{figure*}
%%
%% For tables use
%\begin{table}
%% table caption is above the table
%\caption{Please write your table caption here}
%\label{tab:1}       % Give a unique label
%% For LaTeX tables use
%\begin{tabular}{lll}
%\hline\noalign{\smallskip}
%first & second & third  \\
%\noalign{\smallskip}\hline\noalign{\smallskip}
%number & number & number \\
%number & number & number \\
%\noalign{\smallskip}\hline
%\end{tabular}
%\end{table}


%\begin{acknowledgements}
%If you'd like to thank anyone, place your comments here
%and remove the percent signs.
%\end{acknowledgements}

% BibTeX users please use one of
%\bibliographystyle{spbasic}      % basic style, author-year citations
%\bibliographystyle{spmpsci}      % mathematics and physical sciences
%\bibliographystyle{spphys}       % APS-like style for physics
%\bibliography{}   % name your BibTeX data base

% Non-BibTeX users please use
\begin{thebibliography}{}
%
% and use \bibitem to create references. Consult the Instructions
% for authors for reference list style.
%




\bibitem{abraham2013scuba}
Abraham L, Allen J, Barykin O, Borkar V, Chopra B, Gerea C, Merl D, Metzler J, Reiss D, Subramanian S, Wiener JL.: Scuba: diving into data at facebook. Proceedings of the VLDB Endowment. \textbf{6}(11), 1057--1067 (2013)

%\bibitem{baulier1998datablitz}
%Baulier J, Bohannon P, Gogate S, Joshi S, Gupta C, Khivesera A, Korth HF, McIlroy P, Miller J, Narayan PP, Nemeth M. DataBlitz: A High Performance Main-Memory Storage Manager. In: Proceedings of the 24th International Conference on Very Large Data Bases. 1998, p. 701

\bibitem{barber2011blink}
Barber R, Bendel P, Czech M, Draese O, Ho F, Hrle N, Idreos S, Kim MS, Koeth O, Lee JG, Li TT, Lohman G, Morfonios , Mueller R, Murthy K, Pandis I, Qiao L, Raman V, Szabo S, Sidle R, Stolze K.: Blink: Not Your Father\textquotesingle s Database!. In: Proceedings of International Workshop on Business Intelligence for the Real-Time Enterprise, pp. 1--22 (2011)


\bibitem{farber2012sap}
F\"{a}rber F, Cha SK, Primsch J, Bornh\"{o}vd C, Sigg S, Lehner W.: SAP HANA database: data management for modern business applications. ACM Sigmod Record.  \textbf{40}(4), 45--51 (2012)

\bibitem{lindstrom2013ibm}
Lindstr\"{o}m J, Raatikka V, Ruuth J, Soini P, Vakkila K.: IBM solidDB: In-Memory Database Optimized for Extreme Speed and Availability. IEEE Data Eng. Bull. \textbf{36}(2), 14--20 (2013)

\bibitem{PeletonLink}
Peloton Database Management System. Published at http://pelotondb.org (2019)

\bibitem{pavlo2017self}
Pavlo A, Angulo G, Arulraj J, Lin H, Lin J, Ma L, Menon P, Mowry TC, Perron M, Quah I, Santurkar S, Tomasic A, Toor S, Aken D.V, Wang Z, Wu Y, Xian R, Zhang T.: Self-Driving Database Management Systems. In: Proceedings of Biennial Conference on Innovative Data Systems Research (CIDR) (2017)

\bibitem{larson2011sql}
Larson P\r{A}, Clinciu C, Hanson EN, Oks A, Price SL, Rangarajan S, Surna A, Zhou Q.: SQL server column store indexes. In: Proceedings of the ACM SIGMOD International Conference on Management of Data, pp. 1177--1184 (2011)

%\bibitem{lehman1986study}
%Lehman TJ, Carey MJ. A study of index structures for main memory database management systems. In: Proceedings of the 12th International Conference on Very Large Data Bases. 1986, 294--303

\bibitem{zhang2016reducing}
Zhang H, Andersen DG, Pavlo A, Kaminsky M, Ma L, Shen R.: Reducing the storage overhead of main-memory OLTP databases with hybrid indexes. In: Proceedings of the International Conference on Management of Data, pp. 1567--1581 (2016)

\bibitem{ongaro2011fast}
Ongaro D, Rumble SM, Stutsman R, Ousterhout J, Rosenblum M.: Fast crash recovery in RAMCloud. In: Proceedings of the Twenty-Third ACM Symposium on Operating Systems Principles (SOSP), pp. 29--41 (2011)

%\bibitem{lehman1987recovery}
%Lehman TJ, Carey MJ. A recovery algorithm for a high-performance memory-resident database system. In: Proceedings of the 1987 ACM SIGMOD international conference on Management of data. 1987, 104--117

\bibitem{diaconu2013hekaton}
Diaconu C, Freedman C, Ismert E, Larson PA, Mittal P, Stonecipher R, Verma N, Zwilling M.: Hekaton: SQL server's memory-optimized OLTP engine. In: Proceedings of the ACM SIGMOD International Conference on Management of Data, pp. 1243--1254 (2013)

\bibitem{lee2001single}
Lee I, Yeom HY.: A single phase distributed commit protocol for main memory database systems. In: Proceedings of Parallel and Distributed Processing Symposium, (2002)

\bibitem{debrabant2014prolegomenon}
DeBrabant J, Arulraj J, Pavlo A, Stonebraker M, Zdonik S, Dulloor S.: A prolegomenon on OLTP database systems for non-volatile memory. In International Workshop on Accelerating Data Management Systems Using Modern Processor and Storage Architectures. (2014)

\bibitem{driskill2010latest}
Driskill-Smith A.: Latest advances and future prospects of STT-RAM. In Non-Volatile Memories Workshop. 11--13, (2010)

\bibitem{mandelman2002challenges}
Mandelman JA, Dennard RH, Bronner GB, DeBrosse JK, Divakaruni R, Li Y, Radens CJ.: Challenges and future directions for the scaling of dynamic random-access memory (DRAM). IBM Journal of Research and Development. \textbf{46}(2.3), 187--212 (2002)

\bibitem{melnik2010dremel}
Melnik S, Gubarev A, Long JJ, Romer G, Shivakumar S, Tolton M, Vassilakis T.: Dremel: interactive analysis of web-scale datasets. Proceedings of the VLDB Endowment.  \textbf{3}(1--2) 330--339 (2010)

\bibitem{plattner2011sanssoucidb}
Plattner H.: SanssouciDB: An In-Memory Database for Processing Enterprise Workloads. In: Proceedings of Datenbanksysteme f\"{u}r Business, Technologie und Web (BTW), pp. 2--21 (2011)

\bibitem{sikka2012efficient}
Sikka V, F\"{a}rber F, Lehner W, Cha SK, Peh T, Bornh\"{o}vd C.: Efficient transaction processing in SAP HANA database: the end of a column store myth. In: Proceedings of the 2012 ACM SIGMOD International Conference on Management of Data, pp. 731--742 (2012)

\bibitem{arulraj2017build}
Arulraj J, Pavlo A.: How to Build a Non-Volatile Memory Database Management System. In: Proceedings of the 2017 ACM International Conference on Management of Data, pp. 1753--1758 (2017)

\bibitem{qureshi2009scalable}
Qureshi MK, Srinivasan V, Rivers JA.: Scalable high performance main memory system using phase-change memory technology. In: Proceedings of the 36th annual international symposium on Computer architecture (ISCA), pp. 24--33 (2009)

\bibitem{andrei2017sap}
Andrei M, Lemke C, Radestock G, Schulze R, Thiel C, Blanco R, Meghlan A, Sharique M, Seifert S, Vishnoi S, Booss D, Peh T, Schreter I, Thesing W, Wagle  M, Willhalm T.: SAP HANA adoption of non-volatile memory. Proceedings of the VLDB Endowment. \textbf{10}(12), 1754--1765 (2017)

\bibitem{raoux2008phase}
Raoux S, Burr GW, Breitwisch MJ, Rettner CT, Chen YC, Shelby RM, Salinga M, Krebs D, Chen SH, Lung HL, Lam CH.: Phase-change random access memory: A scalable technology. IBM Journal of Research and Development. \textbf{52}(4.5), 465--479 (2008)

\bibitem{strukov2008missing}
Strukov DB, Snider GS, Stewart DR, Williams RS.: The missing memristor found. nature. \textbf{453}(7191), 80--83 (2008)


\bibitem{arulraj2015let}
Arulraj J, Pavlo A, Dulloor SR.: Let's talk about storage \& recovery methods for non-volatile memory database systems. In: Proceedings of the 2015 ACM SIGMOD International Conference on Management of Data, pp. 707--722 (2015)

\bibitem{chang2012limits}
Chang J, Ranganathan P, Mudge T, Roberts D, Shah MA, Lim KT.: A limits study of benefits from nanostore-based future data-centric system architectures. In: Proceedings of the 9th conference on Computing Frontiers (CF), pp. 33--42 (2012)

\bibitem{hagmann2002real}
Hagmann R, Skeen MD.: Real-time query optimization in a decision support system. United States patent US 6,338,055. (2002)

\bibitem{shim2002past}
Shim JP, Warkentin M, Courtney JF, Power DJ, Sharda R, Carlsson C.: Past, present, and future of decision support technology. Decision support systems. \textbf{33}(2), 111--126 (2002)

\bibitem{chaudhuri2001database}
Chaudhuri S, Dayal U, Ganti V.: Database technology for decision support systems. Computer. \textbf{34}(12), 48--55 (2001)

\bibitem{DBRanking}
DB-Engines Ranking of Relational DBMS. Published at https://db-engines.com/en/ranking (2019)

\bibitem{council2008tpc}
Transaction Processing Performance Council. TPC-H benchmark specification. Published at http://www.tpc.org/tpch/ (2008)

\bibitem{perez2010non}
Perez T, De Rose CA.: Non-volatile memory: Emerging technologies and their impacts on memory systems. Technical Report, Porto Alegre (2010)

\bibitem{wang2013low}
Wang KL, Alzate JG, Amiri PK.: Low-power non-volatile spintronic memory: STT-RAM and beyond. Journal of Physics D: Applied Physics. \textbf{46}(7), 074003 (2013)


\bibitem{mittal2016survey}
Mittal S, Vetter JS.: A survey of software techniques for using non-volatile memories for storage and main memory systems. IEEE Transactions on Parallel and Distributed Systems. \textbf{27}(5), 1537--1550 (2016)

\bibitem{arulraj2016write}
Arulraj J, Perron M, Pavlo A.: Write-behind logging. Proceedings of the VLDB Endowment. \textbf{10}(7), 337--348 (2016)

\bibitem{oukid2014sofort}
Oukid I, Booss D, Lehner W, Bumbulis P, Willhalm T.: SOFORT: A hybrid SCM-DRAM storage engine for fast data recovery. In: Proceedings of the Tenth International Workshop on Data Management on New Hardware (DaMoN), (2014)

\bibitem{chatzistergiou2015rewind}
Chatzistergiou A, Cintra M, Viglas SD.: Rewind: Recovery write-ahead system for in-memory non-volatile data-structures. Proceedings of the VLDB Endowment. \textbf{8}(5), 497--508 (2015)

\bibitem{huang2012register}
Huang Y, Liu T, Xue CJ.: Register allocation for write activity minimization on non-volatile main memory for embedded systems. Journal of Systems Architecture. \textbf{58}(1), 13--23 (2012)

\bibitem{oukid2015instant}
Oukid I, Lehner W, Kissinger T, Willhalm T, Bumbulis P.: Instant Recovery for Main Memory Databases. In: Proceedings of Biennial Conference on Innovative Data Systems Research (CIDR), (2015)

\bibitem{chakrabarti2014atlas}
Chakrabarti DR, Boehm HJ, Bhandari K.: Atlas: Leveraging locks for non-volatile memory consistency. In: Proceedings of ACM International Conference on Object Oriented Programming Systems Languages \& Applications (OOPSLA), pp. 433--452 (2014)

\bibitem{zhang2015study}
Zhang Y, Swanson S.: A study of application performance with non-volatile main memory. In: Proceedings of 31st IEEE Symposium on Mass Storage Systems and Technologies (MSST), pp. 1--10 (2015)

\bibitem{viglas2014write}
Viglas SD.: Write-limited sorts and joins for persistent memory. Proceedings of the VLDB Endowment. \textbf{7}(5), 413--424 (2014)

\bibitem{zhou2009durable}
Zhou P, Zhao B, Yang J, Zhang Y.: A durable and energy efficient main memory using phase change memory technology. In: Proceedings of the 36th annual international symposium on Computer architecture (ISCA), pp. 14--23 (2009)

\bibitem{3DXPoint}
Intel and Micron. 3D XPoint Technology. https://www.micron.com/products/advanced-solutions/3d-xpoint-technology (2019)

\bibitem{intel2016architecture}
Intel. 2016. Architecture Instruction Set Extensions Programming Reference. Intel Corporation. (2016)

\bibitem{dulloor2014system}
Dulloor SR, Kumar S, Keshavamurthy A, Lantz P, Reddy D, Sankaran R, Jackson J.: System software for persistent memory. In: Proceedings of the Ninth European Conference on Computer Systems, p. 15 (2014)

\bibitem{githubPMFS}
Intel. 2014. Linux-pmfs. https://github.com/linux-pmfs/pmfs (2014)

\bibitem{kimura2015foedus}
Kimura H.: FOEDUS: OLTP Engine for a Thousand Cores and NVRAM. In: Proceedings of the ACM SIGMOD International Conference on Management of Data, pp. 691--706 (2015)

\bibitem{huang2014nvram}
Huang J, Schwan K, Qureshi MK.: NVRAM-aware logging in transaction systems. Proceedings of the VLDB Endowment. \textbf{8}(4), 389--400 (2014)

\bibitem{schindler2002track}
Schindler J, Griffin JL, Lumb CR, Ganger GR.: Track-Aligned Extents: Matching Access Patterns to Disk Drive Characteristics. In: Proceedings of the 1st USENIX Conference on File and Storage Technologies (FAST), pp. 259--274 (2002)

\bibitem{debrabant2013anti}
DeBrabant J, Pavlo A, Tu S, Stonebraker M, Zdonik S.: Anti-caching: A new approach to database management system architecture. Proceedings of the VLDB Endowment. \textbf{6}(14), 1942--1953 (2013)

%\bibitem{garcia1992main}
%Garcia-Molina H, Salem K. Main memory database systems: An overview. IEEE Transactions on knowledge and data engineering, 1992, 4(6): 509--16

\bibitem{pelley2014memory}
Pelley S, Chen PM, Wenisch TF.: Memory persistency. In: Proceedings of ACM/IEEE 41st International Symposium on Computer Architecture (ISCA), pp. 265--276 (2014)

\bibitem{momjian2001postgresql}
Momjian B.: PostgreSQL: introduction and concepts. New York: Addison-Wesley (2001)

\bibitem{neumann2015fast}
Neumann T, M\"{u}hlbauer T, Kemper A.: Fast serializable multi-version concurrency control for main-memory database systems. In: Proceedings of the ACM SIGMOD International Conference on Management of Data, pp. 677--689 (2015)

\bibitem{PostgreSQLDocBook}
The PostgreSQL Global Development Group. PostgreSQL 9.0.22 Documentation. (2015)

\bibitem{gao2011pcmlogging}
Gao S, Xu J, He B, Choi B, Hu H.: PCMLogging: reducing transaction logging overhead with PCM. In: Proceedings of the 20th ACM international conference on Information and knowledge management, pp. 2401--2404 (2011)

\bibitem{son2017log}
Son Y, Kang H, Yeom HY, Han H.: A log-structured buffer for database systems using non-volatile memory. In: Proceedings of the Symposium on Applied Computing (SAC), pp. 880--886 (2017)

\bibitem{annavaram2001data}
Annavaram M, Patel JM, Davidson ES. Data prefetching by dependence graph precomputation. In: Procedings of 28th Annual International Symposium on Computer Architecture (ISCA), pp. 52--61 (2001)

\bibitem{jung2006helper}
Jung C, Lim D, Lee J, Solihin Y.: Helper thread prefetching for loosely-coupled multiprocessor systems. In: Proceedings of 20th International Parallel and Distributed Processing Symposium, (2006)

\bibitem{pelley2013storage}
Pelley S, Wenisch TF, Gold BT, Bridge B.: Storage management in the NVRAM era. Proceedings of the VLDB Endowment. \textbf{7}(12), 121--132 (2013)

\bibitem{johnson2009shore}
Johnson R, Pandis I, Hardavellas N, Ailamaki A, Falsafi B.: Shore-MT: a scalable storage manager for the multicore era. In: Proceedings of the 12th International Conference on Extending Database Technology: Advances in Database Technology (EDBT), pp. 24--35 (2009)

\bibitem{fang2011high}
Fang R, Hsiao HI, He B, Mohan C, Wang Y.: High performance database logging using storage class memory. In: Proceedings of IEEE 27th International Conference on Data Engineering, pp. 1221--1231 (2011)

\bibitem{wang2014scalable}
Wang T, Johnson R.: Scalable logging through emerging non-volatile memory. Proceedings of the VLDB Endowment. \textbf{7}(10), 865--876 (2014)

\bibitem{RankingLink}
DB-Engines Ranking - popularity ranking of database management systems. Published at https://db-engines.com/en/ranking (2019)

\bibitem{kamruzzaman2011inter}
Kamruzzaman M, Swanson S, Tullsen DM.: Inter-core prefetching for multicore processors using migrating helper threads. In: Proceedings of the sixteenth international conference on Architectural support for programming languages and operating systems (ASPLOS), pp. 393--404 (2011)

\bibitem{kim2002design}
Kim D, Yeung D.: Design and evaluation of compiler algorithms for pre-execution. In: Proceedings of the 10th international conference on Architectural support for programming languages and operating systems (ASPLOS), pp. 159--170  (2002)
%\vspace*{-1mm}

\bibitem{collins2001dynamic}
Collins JD, Tullsen DM, Wang H, Shen JP.: Dynamic speculative precomputation. In: Proceedings of the 34th annual ACM/IEEE international symposium on Microarchitecture (MICRO), pp. 306--317 (2001)

\bibitem{hirofuchi2019preliminary}
Hirofuchi T, Takano R.: The Preliminary Evaluation of a Hypervisor-based Virtualization Mechanism for Intel Optane DC Persistent Memory Module. arXiv preprint arXiv:1907.12014, (2019)

\bibitem{peng2019system}
Peng IB, Gokhale MB, Green EW.: System evaluation of the Intel optane byte-addressable NVM. In: Proceedings of the International Symposium on Memory Systems (MEMSYS), pp. 304--315 (2019)

\bibitem{yang2019empirical}
Yang J, Kim J, Hoseinzadeh M, Izraelevitz J, Swanson S.: An Empirical Guide to the Behavior and Use of Scalable Persistent Memory. arXiv preprint arXiv:1908.03583, (2019)

\bibitem{izraelevitz2019basic}
Izraelevitz J, Yang J, Zhang L, Kim J, Liu X, Memaripour A, Soh YJ, Wang Z, Xu Y, Dulloor SR, Zhao J.: Basic performance measurements of the intel optane DC persistent memory module. arXiv preprint arXiv:1903.05714, (2019)

\bibitem{xu2016nova}
Xu J, Swanson S.: NOVA: A Log-structured File System for Hybrid Volatile/Non-volatile Main Memories. In 14th Conference on File and Storage Technologies (FAST), pp. 323--338, (2016)

\bibitem{PMDKLib}
pmem.io: PMDK. Published at https://pmem.io/pmdk/ (2019)




%\bibitem{RefJ}
%% Format for Journal Reference
%Author, Article title, Journal, Volume, page numbers (year)
%% Format for books
%\bibitem{RefB}
%Author, Book title, page numbers. Publisher, place (year)
% etc
\end{thebibliography}

\end{document}
% end of file template.tex

