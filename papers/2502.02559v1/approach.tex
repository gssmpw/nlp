\section{SafeSPLE}
\label{sec:SafeSPLE}

\begin{figure}[ht]
\centering
\includegraphics[width=6.3in]{Figures/FeatureModel.pdf}
\caption{Partial feature model representing heuristics for a safety-case software product line with 51 features. This represents 288,202 different variants -- combinations of flights, vehicles, operators, and weather conditions. Additional variants such as terrain and its altitude are not  shown.}\label{fig:featuremodel}
\end{figure}

A Software Product Line (SPL) represents a family of software-intensive systems that share a common, managed set of features developed from a common set of core assets in a prescribed way \cite{Clements2001, weiss1999software, Pohl-PL-Eng-Book, Kang1990, SEI-PL}. Individual products are constructed by selecting a set of alternative and optional features (variabilities) and composing them on top of a set of common base features (commonalities) \cite{DonBatory-FeatureDef, pl-feature-modeling-kang, DBLP:books/daglib/0087788, Gomaa, Fantechi-Gnesi}. In the case of a SafeSPL, features representing groupings of mandatory and optional safety-case nodes are combined and configured to generate a valid and appropriate safety case in support of each unique on-entry request. 

The features are typically represented using a feature model, which describes all possible valid combinations or instances. Figure \ref{fig:featuremodel} shows part of an example feature model that we developed for an sUAS mission. It has 51 features and represents 288,202 different variants, %\robyn{Myra: pls check:} 
where each of those variants is a valid possible configuration of an sUAS. In a feature model such as this, the features are shown as rectangles, with a set of dependencies (or constraints) between them. In our example, we have features for the pilot, airspace, vehicle, and weather, among others. For some features, we use ``XOR" alternative conditions (e.g., the purpose of the flight is recreational, search and rescue, or delivery), and in others, we have ``Or" conditions that allow for one or more of the features to be selected (e.g., the airspace can be sparse or congested, and the ground below the airspace may have, or not have, human activity). We also include a \textit{cross-tree constraint} stating that we may have either Sparse or Congested airspace (but not both), which allows either of those airspace descriptions to include human activity. 
Given that the feature model describes all possible valid combinations and cross-tree constraints that can be represented by first-order logic, we can potentially use satisfiability solvers to reason about a product's validity or perform analyses on subsets of the product. There are also existing tools that can convert this representation into logic to help with reasoning~\cite{BENAVIDES2010615,fama}. 
Once the feature model is in a logical representation, we can ask questions about the valid number of products, whether individual products are valid, and/or evaluate the size of a \textit{slice} through the feature model (all products given the selection of a specific concrete feature)~\cite{fama,featureide,10.1145/3176644}. We can also use this to generate test cases for the product line \cite{5456077}.

\begin{figure}[]
\centering
\includegraphics[scale = 0.70]{Figures/parameterized-safetycase.pdf}\caption{Partial parameterized safety case. Represents the top level goal (G1) and strategies. The Context nodes (C1 and C2) have parameters which can be instantiated with a defined set of concrete values.}\label{fig:param}
\end{figure}



\iffalse
\myra{should we include below?}
\robyn{either way, depending on whether it will confuse readers. We aren't solving it here but it is still open.}
  A traditional safety case \cite{} tends to focus on providing evidence that the system is safe for use, and therefore the solution nodes (circles) represent evidence. However, for on-entry safety assessments, solutions (i.e., evidence) can also represent risks that carry negative connotations and are likely to lead to on-entry denials or conditional-admits such as a case in which the on-exit flight log of the sUAS showed significant vibration when flying in windy conditions. Determining how to balance this negative claim, including when it occurred within the timeline of otherwise successful flights, represents one of the open challenges.
\fi


Figure \ref{fig:param} shows a small fragment of a parameterized safety case. This is based on recent work on assurance recipes \cite{FirestoneC18} and safety patterns \cite{depai2016}, which provide a way to generalize common parts of a safety case. In this figure, we have a primary, top-level Goal (G1) to ensure ``safe flight in a controlled zone.'' It is parameterized by its Context (C1), which typically must be verified for a range of concrete conditions. These variables will also be propagated to the solution (or evidence) nodes. The airspace where this sUAS will fly, the model of sUAS hardware, the operator (or pilot), the planned mission, and any specific regulations placed on this airspace are listed as variables. We next show two high-level strategies (but only expand on one). Strategy S1 links the top-level goal G1 to a subgoal (G2) of flying safely within its current wind conditions. The context for this strategy (C2) has parameters related to the surface wind, gusts, temperature, visibility, and precipitation. At each step, as we move down this argumentation structure, we can utilize the sUAS's specific, concrete parameters to plug in the specific goals/strategies and/or concrete evidence for its customized safety-case instantiation. Part of this process will require the definition of equivalence classes (or choices) which represent ranges of (or discrete) values that behave in a similar way.
 
We recently surveyed stakeholders about the types of features that they consider to be most important when designing a UTM-type on-entry system \cite{gohar2024towards}. Study participants were asked to approve or reject flight-entry requests presented in the form of vignettes. Participants assessed the importance of pilot, drone, environmental, and mission concepts, as well as providing textual feedback. Responses from this initial survey of ours showed that flight characteristics and environmental conditions were seen as most important, and that pilot and drone capabilities should also be considered. Textual feedback relayed doubts about any AI usage, the importance of Human-on-the-Loop, and the need for transparency in automated decision-making. Survey feedback on the set of appropriate criteria to use for on-entry decisions tended to confirm the features in our preliminary sUAS feature model, and the results will inform our design decisions for SafeSPLE going forward. 

 
\section{Vision and Process}

Putting this all together, Figure \ref{fig:vision}  presents our vision for SafeSPLE. 
We begin with a software product line based on features such as the sUAS, flight plans, operator and weather conditions, etc.  We input this information to our parameterized safety case, which can be used to generate individualized safety cases for specific sUAS \cite{VierhauserBWXCH21} as they request entry to the UTM. We use the features of our feature model to parameterize the context and evidence. Each instance is a detailed safety case that is valid for that specific instance of the product line under the given context. Since we may not have all evidence required by the generated evidence nodes, a key part of this process will involve determining appropriate (i.e., safe) approximation for reuse across the evidence nodes \cite{AgrawalKVRCL19}.  
Moreover, since each sUAS's safety case will need only a portion of the baseline safety case's options, we are hopeful that performance will be satisfactory for computing in real time.  



 \begin{figure}[ht]
\centering
\includegraphics[width=6.3in]{Figures/safetycaseprocess.pdf}\caption{Overall vision for SafeSPLE. Step one creates a feature model for the safety case product line. Step two designs the parameterized safety case, and Step three generates an instance for a specific sUAS and its characteristics (pilot, vehicle, mission, weather, etc.)}\label{fig:vision}
\end{figure}

We now describe the steps taken during the SafeSPLE process as shown in Figure \ref{fig:vision} .

\begin{enumerate}
\item \textbf{Tracing hazards to sUAS features.} We first identify and prioritize the hazards for a SafeSPL using previous studies \cite{FAA} and our surveys \cite{gohar2024towards}. We then associate hazards with the features that have been contributing causes to those hazards' occurrences, as well as associating them with those features' variation points (Figure \ref{fig:featuremodel}). To assure  coverage  of all relevant features in the SafeSPL we also identify any features involved in mitigations. 

\item\textbf{Parameterizing safety-case variability.} 
We next build the parameterized safety case (Figure \ref{fig:param}), modeling it with a combination of feature modeling and GSN \cite{GSN} tools to visualize and reason about valid products and avoid conflicts among features.\footnote{We used the FeatureIDE\cite{featureide} and AdvoCATE\cite{advocate} tools in the this paper.} As part of this step, we create equivalence classes for parameters and map these to the features in the SafeSPL.

\item\textbf{Generating sUAS-specific safety-case instances.} The final step is to generate on-demand instances of the safety case with each element being tagged with a unique, machine-readable link to its position in the baseline safety case for traceability. We also generate a list of required evidence from the leaf nodes to allow the sUAS to provide sufficient safety information.  If entry is denied the traceability will provide partial explanations for the operator. 
\end{enumerate}












% All the stuff below is not included in the paper currently. 

%\end{itemize}
%\item \textbf{Associate Hazards with Features.}
%Creating traceability.
%We associate known hazards, drawn from previous studies of accidents \cite{}, %cite [1] & [2] (Sightings)in prior AIAA paper
%with the features that have been contributing causes to that hazard's occurrence, as well as with those features' variation points.
%\end{itemize}

\iffalse
For explainability, as well for safety, traceability will be primary.   Associations between risk analysis results and sUAS features, informed by fault tree analyses and by information from logs and incident reports, will enable the automated output of explanations (rationales) to pilots and to sUAS software developers of why specific nodes in the safety case are needed and relevant. Desirable side effects of the focus on traceability are users’ increased confidence in the safety case’s relevance, as well as increased 
\fi
%We next identify the safety case variability. 
%\begin{itemize}\textbf{Determine Safety case variability.} 


%Briefly xplain commonality, variability notions
%Myra will desribe. 
%where the logic for verifying that the claim ``Y” is true for sUAS depends on whether the drone has an A feature or a B feature. In the small example shown we use the environment to represent our variability. This will propagate throughout the safety case to any claims or goals below  %maybe whether or not it has GPS?   Expand the example to explain that  
%Importantly, the evidence needed to verify A differs from the evidence needed to verify B.  Thus, the safety case for an individual sUAS will only need to contain the left branch. %continue the example


\iffalse
%\item \textbf{Design SAfeWe will design an sUAS safety-case product line, represented as a Goal Structuring Notation feature model, showing mandatory and optional features as well as cross-tree constraints. \\
a. Define baseline assurance case and identify commonality and variability \\
2b. Identify the cross tree constraints for the assurance cases \\
2c. Design the necessary evidence that needs to be collected for the claims \\
2d. Model the assurance case in combination of tools (both feature modeling and GSN tools) to be able to visualize and reason about valid products and conflicts. \\
\end{itemize}
\fi


\iffalse
{\it Safety case evolution}
It is clear that updates to the sUASs' safety cases will be needed. The drone market evolves rapidly with new drone models released regularly.  This brings both new capabilities and some new risks.  For example, the new capability for longer travel reach takes an sUAS thus equipped farther from Visual Line of Sight.  %specific example. 
In addition, the space over which drones fly changes quickly, for example, as new housing developments and research parks expand into what was previously farmland.  

Evolution of the sUAS’ safety cases will occur both for the original safety case and for the individual  sUASs' instantiated safety cases. The goal of co-evolution of the customized safety case with its specific sUAS depends on automating the identification and updates needed to the safety case when the sUAS’s feature selections change.  
\fi

%Drawing on prior work with product-family techniques for evolving safety cases,   \cite{AgrawalKVRCL19}, updates to the original Prime safety case will propagate to the individual drone safety cases next time they are invoked (i.e., next time an individual drone requests entry or accesses its own safety case).  

%Suppose that the safety claim in drone X’s customized safety case is that drone X shall stay 100 meters from other drones.   This parametrized value of 100 was based on wind speeds of <30 mph from the north. 
%If the wind changes direction, that safety constraint may need to change in real time, with the update communicated both to drone X’s onboard anomaly detector, and to drone X’s pilot. 
\iffalse
%To evolve an individual safety case in real-time based on changing environmental factors, our software [SADE?] will re-run those sUAS safety cases in which the evidence node that has changed (e.g., “wind direction is from the North”) is no longer met (e.g., wind direction now has shifted from North to South).  To achieve this, each evidence node is tagged with the measurement used to verify it (here, wind direction).  
\fi

 \iffalse
{\it Safety case composition and reuse}
Composition of safety cases requires careful modularization of the original Prime safety case.  This is because for it to be practical to instantiate a safety case (aka, create a variant safety case) for each drone, extensive reuse of its pieces is needed.  To achieve an appropriate modularization, we adapt reuse techniques . . . 
%are there some for product lines?  Need to investigate.  Else use software architecture technique. 
%maybe in https://link.springer.com/chapter/10.1007/978-3-030-12157-0_5  ??

To achieve the required scalability to the estimated 1.81 million recreational sUAS and 858,000 commercial sUAS that the FAA anticipate by 2026 \cite{faa_forecast}, reuse of safety-case components will be necessary.   
%FAA forecasts that the recreational small drone fleet will likely (i.e., base scenario) attain approximately 1.81 million units by 2026.
To support this, the safety case is modularized, with each claim tagged with a unique link to its position in the original Prime safety case. 
\fi
\iffalse
\begin{figure}[ht]
\includegraphics[width=5in]{Figures/safety-case-instance.pdf}\caption{Creating a safety case instance}\label{fig:instance}
\end{figure}
\fi