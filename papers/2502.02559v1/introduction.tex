\section {Introduction}
In recent years, we have seen a rise in the use of small Uncrewed Aircraft Systems (sUAS), which are increasingly deployed into shared areas of the national airspace. This growth is being driven by commercial, recreational, and service applications such as package delivery, photography, remote sensing, and emergency response \cite{Erdelj2017HelpFT,FAA}. Unfortunately, there has also been an increase in reported incidents involving sUAS, often linked to hardware or software malfunctions and human errors, thus increasing the risk of serious accidents \cite{cleland2022towards}. Additionally, external elements such as radio interference and adverse weather conditions have been identified as contributing factors in many reported incidents \cite{FAA,report3}.  This is leading to a need for a clearer understanding and regulations about when sUAS can participate in shared airspace. Ultimately, for controlled airspaces, both the sUAS and its operator will need vetting prior to entry to ensure the safety of all within close proximity \cite{gohar2024towards}.

Safety cases to certify safety-critical systems are a common practice worldwide, including for space, defense, rail, nuclear, healthcare, and oil and gas systems \cite{NASA12,Knig12, Hatcliff14, Leveson23}. 
In commercial aviation, certification of a plane takes a long period of evaluation. As we transition to the more dynamic scenario of many different types of sUAS flying in shared airspace, with tens or hundreds of thousands of variations in the way each sUAS, its operator, its mission, and its operating environment can be combined \cite{10.1145/3544548.3581003}, the idea of building bespoke safety cases becomes futile. Organizations or companies deploying sUAS are less likely to have the staff or expertise to develop these documents. Additionally, many sUAS are operated by individuals (perhaps a hobbyist) wanting to perform an ad hoc flight, for example, to take photos. 

One way to begin to address these concerns is an approach being taken by the U.S. Federal Aviation Administration (FAA) and the National Aeronautics and Space Administration (NASA). They are collaborating to develop a UAS Traffic Management (UTM) system where sUAS can be dynamically certified to enter (or denied entry to) the shared airspace \cite{UTM}.  

If we can design a set of criteria that can be quickly evaluated, along with requirements for 
real-time (temporary) certification, it will allow automated decision-making for entry into a UTM.  While this vision of a dynamically controlled UTM is still in development, we argue there are approaches that may help to streamline such a system and can lead to re-use and efficiency in the eventual UTM.


In this work, we propose the use of families (software product lines) of safety cases, which represent both the common and the variable features of flight characteristics (such as the operator, weather conditions, vehicle types, etc.), that will cover all possible instances of a safety case for these features. We note that these features are unlike traditional features, which normally represent program functionality. Instead, they can be mapped to the context of a safety case, which restricts the conditions under which the safety case arguments hold.  We propose that these possible alternatives can be mapped to key points of variation in a parameterized, general safety case for controlled airspace. Further, we propose that a customized instance can be automatically generated from the product line in conjunction with a parameterized safety case for each sUAS seeking entry to that airspace. 
As part of the safety case, the operator interface will detail what evidence is needed to satisfy the safety goals, and, assuming the sUAS can produce this evidence, it will be allowed to enter. Without that evidence, entry will be denied.
Referred to as SafeSPLE (or Safety Case Software Product Line Engineering), the core of our approach lies in the use of a Safety Case Software Product Line (SafeSPL).


