\documentclass[conf]{new-aiaa}
%\documentclass[journal]{new-aiaa} for journal papers
\usepackage[utf8]{inputenc}
\usepackage{graphicx} % Required for inserting images
\usepackage{xcolor}
\usepackage{graphicx}
\usepackage{amsmath}
\usepackage[version=4]{mhchem}
\usepackage{siunitx}
\usepackage{longtable,tabularx}
\usepackage{textcomp}
\setlength\LTleft{0pt} 


\newcommand{\colorann}[3]{\textcolor{#1}{${}^{#2}[$#3$]$}}
\newcommand{\robyn}[1]{\colorann{red}{robyn}{#1}}
\newcommand{\myra}[1]{\colorann{red}{myra}{#1}}
\newcommand{\jch}[1]{\colorann{red}{jane}{#1}}

\newcommand{\mike}[1]{\colorann{blue}{mike}{#1}}
\newcommand{\usman}[1]{\colorann{blue}{usman}{#1}}

\title{A Family-Based Approach to Safety Cases for Controlled Airspaces  in Small Uncrewed Aerial Systems}


\author{Michael C. Hunter,\footnote{Graduate Research Assistant, 115 Atanasoff Hall, Iowa State University, Ames, IA, USA}  Usman Gohar,\footnote{Graduate Research Assistant, 116 Atanasoff Hall, Iowa State University, Ames, IA,  USA} Myra B. Cohen,\footnote{Professor, Computer Science, 202 Atanasoff Hall, Iowa State University, Ames, IA,  USA} Robyn. R. Lutz \footnote{Professor, Computer Science, 230 Atanasoff Hall, Iowa State University, Ames, IA, USA}}
\affil{Iowa State University, Ames, IA, USA}

\author{Jane Cleland-Huang\footnote{Professor, Computer Science and Engineering, 325B Fitzpatrick Hall, University of Notre Dame, IN USA}}
\affil{Notre Dame University, Notre Dame, IN, USA}

\begin{document}

\maketitle

\begin{abstract}
As small Uncrewed Aircraft Systems (sUAS) increasingly operate in the national airspace, safety concerns arise due to a corresponding rise in reported airspace violations and incidents, highlighting the need for a safe mechanism for sUAS entry control to manage the potential overload. This paper presents work toward our aim of establishing automated, customized safety-claim support for managing on-entry requests from sUAS to enter controlled airspace. We describe our approach, Safety Case Software Product Line Engineering (SafeSPLE), which is a novel method to extend product-family techniques to on-entry safety cases. It begins with a hazard analysis and design of a safety case feature model defining key points in variation, followed by the creation of a parameterized safety case. We use these together to automate the generation of instances for specific sUAS. Finally we use a case study to demonstrate that the SafeSPLE method can be used to facilitate creation of safety cases for specific flights.  
\end{abstract}


\section{Introduction}
\label{sec:introduction}
The business processes of organizations are experiencing ever-increasing complexity due to the large amount of data, high number of users, and high-tech devices involved \cite{martin2021pmopportunitieschallenges, beerepoot2023biggestbpmproblems}. This complexity may cause business processes to deviate from normal control flow due to unforeseen and disruptive anomalies \cite{adams2023proceddsriftdetection}. These control-flow anomalies manifest as unknown, skipped, and wrongly-ordered activities in the traces of event logs monitored from the execution of business processes \cite{ko2023adsystematicreview}. For the sake of clarity, let us consider an illustrative example of such anomalies. Figure \ref{FP_ANOMALIES} shows a so-called event log footprint, which captures the control flow relations of four activities of a hypothetical event log. In particular, this footprint captures the control-flow relations between activities \texttt{a}, \texttt{b}, \texttt{c} and \texttt{d}. These are the causal ($\rightarrow$) relation, concurrent ($\parallel$) relation, and other ($\#$) relations such as exclusivity or non-local dependency \cite{aalst2022pmhandbook}. In addition, on the right are six traces, of which five exhibit skipped, wrongly-ordered and unknown control-flow anomalies. For example, $\langle$\texttt{a b d}$\rangle$ has a skipped activity, which is \texttt{c}. Because of this skipped activity, the control-flow relation \texttt{b}$\,\#\,$\texttt{d} is violated, since \texttt{d} directly follows \texttt{b} in the anomalous trace.
\begin{figure}[!t]
\centering
\includegraphics[width=0.9\columnwidth]{images/FP_ANOMALIES.png}
\caption{An example event log footprint with six traces, of which five exhibit control-flow anomalies.}
\label{FP_ANOMALIES}
\end{figure}

\subsection{Control-flow anomaly detection}
Control-flow anomaly detection techniques aim to characterize the normal control flow from event logs and verify whether these deviations occur in new event logs \cite{ko2023adsystematicreview}. To develop control-flow anomaly detection techniques, \revision{process mining} has seen widespread adoption owing to process discovery and \revision{conformance checking}. On the one hand, process discovery is a set of algorithms that encode control-flow relations as a set of model elements and constraints according to a given modeling formalism \cite{aalst2022pmhandbook}; hereafter, we refer to the Petri net, a widespread modeling formalism. On the other hand, \revision{conformance checking} is an explainable set of algorithms that allows linking any deviations with the reference Petri net and providing the fitness measure, namely a measure of how much the Petri net fits the new event log \cite{aalst2022pmhandbook}. Many control-flow anomaly detection techniques based on \revision{conformance checking} (hereafter, \revision{conformance checking}-based techniques) use the fitness measure to determine whether an event log is anomalous \cite{bezerra2009pmad, bezerra2013adlogspais, myers2018icsadpm, pecchia2020applicationfailuresanalysispm}. 

The scientific literature also includes many \revision{conformance checking}-independent techniques for control-flow anomaly detection that combine specific types of trace encodings with machine/deep learning \cite{ko2023adsystematicreview, tavares2023pmtraceencoding}. Whereas these techniques are very effective, their explainability is challenging due to both the type of trace encoding employed and the machine/deep learning model used \cite{rawal2022trustworthyaiadvances,li2023explainablead}. Hence, in the following, we focus on the shortcomings of \revision{conformance checking}-based techniques to investigate whether it is possible to support the development of competitive control-flow anomaly detection techniques while maintaining the explainable nature of \revision{conformance checking}.
\begin{figure}[!t]
\centering
\includegraphics[width=\columnwidth]{images/HIGH_LEVEL_VIEW.png}
\caption{A high-level view of the proposed framework for combining \revision{process mining}-based feature extraction with dimensionality reduction for control-flow anomaly detection.}
\label{HIGH_LEVEL_VIEW}
\end{figure}

\subsection{Shortcomings of \revision{conformance checking}-based techniques}
Unfortunately, the detection effectiveness of \revision{conformance checking}-based techniques is affected by noisy data and low-quality Petri nets, which may be due to human errors in the modeling process or representational bias of process discovery algorithms \cite{bezerra2013adlogspais, pecchia2020applicationfailuresanalysispm, aalst2016pm}. Specifically, on the one hand, noisy data may introduce infrequent and deceptive control-flow relations that may result in inconsistent fitness measures, whereas, on the other hand, checking event logs against a low-quality Petri net could lead to an unreliable distribution of fitness measures. Nonetheless, such Petri nets can still be used as references to obtain insightful information for \revision{process mining}-based feature extraction, supporting the development of competitive and explainable \revision{conformance checking}-based techniques for control-flow anomaly detection despite the problems above. For example, a few works outline that token-based \revision{conformance checking} can be used for \revision{process mining}-based feature extraction to build tabular data and develop effective \revision{conformance checking}-based techniques for control-flow anomaly detection \cite{singh2022lapmsh, debenedictis2023dtadiiot}. However, to the best of our knowledge, the scientific literature lacks a structured proposal for \revision{process mining}-based feature extraction using the state-of-the-art \revision{conformance checking} variant, namely alignment-based \revision{conformance checking}.

\subsection{Contributions}
We propose a novel \revision{process mining}-based feature extraction approach with alignment-based \revision{conformance checking}. This variant aligns the deviating control flow with a reference Petri net; the resulting alignment can be inspected to extract additional statistics such as the number of times a given activity caused mismatches \cite{aalst2022pmhandbook}. We integrate this approach into a flexible and explainable framework for developing techniques for control-flow anomaly detection. The framework combines \revision{process mining}-based feature extraction and dimensionality reduction to handle high-dimensional feature sets, achieve detection effectiveness, and support explainability. Notably, in addition to our proposed \revision{process mining}-based feature extraction approach, the framework allows employing other approaches, enabling a fair comparison of multiple \revision{conformance checking}-based and \revision{conformance checking}-independent techniques for control-flow anomaly detection. Figure \ref{HIGH_LEVEL_VIEW} shows a high-level view of the framework. Business processes are monitored, and event logs obtained from the database of information systems. Subsequently, \revision{process mining}-based feature extraction is applied to these event logs and tabular data input to dimensionality reduction to identify control-flow anomalies. We apply several \revision{conformance checking}-based and \revision{conformance checking}-independent framework techniques to publicly available datasets, simulated data of a case study from railways, and real-world data of a case study from healthcare. We show that the framework techniques implementing our approach outperform the baseline \revision{conformance checking}-based techniques while maintaining the explainable nature of \revision{conformance checking}.

In summary, the contributions of this paper are as follows.
\begin{itemize}
    \item{
        A novel \revision{process mining}-based feature extraction approach to support the development of competitive and explainable \revision{conformance checking}-based techniques for control-flow anomaly detection.
    }
    \item{
        A flexible and explainable framework for developing techniques for control-flow anomaly detection using \revision{process mining}-based feature extraction and dimensionality reduction.
    }
    \item{
        Application to synthetic and real-world datasets of several \revision{conformance checking}-based and \revision{conformance checking}-independent framework techniques, evaluating their detection effectiveness and explainability.
    }
\end{itemize}

The rest of the paper is organized as follows.
\begin{itemize}
    \item Section \ref{sec:related_work} reviews the existing techniques for control-flow anomaly detection, categorizing them into \revision{conformance checking}-based and \revision{conformance checking}-independent techniques.
    \item Section \ref{sec:abccfe} provides the preliminaries of \revision{process mining} to establish the notation used throughout the paper, and delves into the details of the proposed \revision{process mining}-based feature extraction approach with alignment-based \revision{conformance checking}.
    \item Section \ref{sec:framework} describes the framework for developing \revision{conformance checking}-based and \revision{conformance checking}-independent techniques for control-flow anomaly detection that combine \revision{process mining}-based feature extraction and dimensionality reduction.
    \item Section \ref{sec:evaluation} presents the experiments conducted with multiple framework and baseline techniques using data from publicly available datasets and case studies.
    \item Section \ref{sec:conclusions} draws the conclusions and presents future work.
\end{itemize}
\section{Temporal Representation Alignment}
\label{sec:approach}

When training a series of short-horizon goal-reaching and instruction-following tasks, our goal is to learn a representation space such that our policy can generalize to a new (long-horizon) task that can be viewed as a sequence of known subtasks.
We propose to structure this representation space by aligning the representations of states, goals, and language in a way that is more amenable to compositional generalization.

\paragraph{Notation.}
We take the setting of a goal- and language-conditioned MDP $\cM$ with state space $\cS$, continuous action space $\cA \subseteq (0,1)^{d_{\cA}}$, initial state distribution $p_0$, dynamics $\p(s'\mid s,a)$, discount factor $\gamma$, and language task distribution $p_{\ell}$.
A policy $\pi(a\mid s)$ maps states to a distribution over actions. We inductively define the $k$-step (action-conditioned) policy visitation distribution as:
\begin{align*}
    p^{\pi}_{1}(s_{1} \mid s_1, a_{1})
    &\triangleq p(s_1 \mid s_1, a_1),\\
    p^{\pi}_{k+1}(s_{k+1} \mid s_1, a_1)
    &\triangleq \nonumber\\*
      &\mspace{-120mu} \int_{\cA}\int_{\cS} p(s_{k+1} \mid s,a) \dd p^{\pi}_{k}(s \mid s_{1},a_1) \dd
        \pi(a \mid s)\\
    p^{\pi}_{k+t}(s_{k+t} \mid s_t,a_t)
    &\triangleq p^{\pi}(s_{k} \mid s_1, a_1) . \eqmark
        \label{eq:successor_distribution}
\end{align*}
Then, the discounted state visitation distribution can be defined as the distribution over $s^{+}$\llap, the state reached after $K\sim \operatorname{Geom}(1-\gamma)$ steps:
\begin{equation}
    p^{\pi}_{\gamma}(s^{+}  \mid  s,a) \triangleq \sum_{k=0}^{\infty} \gamma^{k} p^{\pi}_{k}(s^{+} \mid s,a).
    \label{eq:discounted_state_visitation}
\end{equation}

We assume access to a dataset of expert demonstrations $\cD = \{\tau_{i},\ell_i\}_{i=1}^{K}$, where each trajectory
\begin{equation}
    \tau_{i} = \{s_{t,i},a_{t,i}\}_{t=1}^{H} \in \cS \times \cA
    \label{eq:trajectory}
\end{equation}
is gathered by an expert policy $\expert$, and is then annotated with $p_{\ell}(\ell_{i} \mid s_{1,i}, s_{H,i})$.
Our aim is to learn a policy $\pi$ that can select actions conditioned on a new language instruction $\ell$.
As in prior work~\citep{walke2023bridgedata}, we handle the continuous action space by representing both our policy and the expert policy as an isotropic Gaussian with fixed variance; we will equivalently write $\pi(a\mid s, \varphi)$ or denote the mode as $\hat{a} = \pi(s,\varphi)$ for a task $\varphi$.

\begin{rebuttal}
    \subsection{Representations for Reaching Distant Goals}
    \label{sec:reaching_goals}

    We learn a goal-conditioned policy $\pi(a\mid s,g)$ that selects actions to reach a goal $g$ from expert demonstrations with behavioral cloning.
    Suppose we directly selected actions to imitate the expert on two trajectories in $\cD$:
    
    \begin{equation}
        \mspace{-100mu}\begin{tikzcd}[remember picture,sep=small]
            s_1 \rar & s_2 \rar  & \ldots \rar & s_{H} \rar & w      \quad \\
            w \rar   & s_1' \rar & \ldots \rar & s_{H}' \rar & g\quad
        \end{tikzcd}
        \begin{tikzpicture}[remember picture,overlay] \coordinate (a) at (\tikzcdmatrixname-1-5.north east);
            \coordinate (b) at (\tikzcdmatrixname-2-5.south east);
            \coordinate (c) at (a|-b);
            \draw[decorate,line width=1.5pt,decoration={brace,raise=3pt,amplitude=5pt}]
        (a) -- node[right=1.5em] {$\tau_{i}\in \cD$} (c); \end{tikzpicture}
        \label{eq:trajectory_diagram}
    \end{equation}
    When conditioned with the composed goal $g$, we would be unable to imitate effectively
        as the composed state-goal $(s,g)$ is jointly out of the training distribution.

    What \emph{would} work for reaching $g$ is to first condition the policy on the intermediate waypoint $w$, then upon reaching $w$, condition on the goal $g$, as the state-goal pairs $(s_{i},w)$, $(w,g)$, and $(s_{i}',g)$ are all in the training distribution.
    If we condition the policy on some intermediate waypoint distribution $p(w)$ (or sufficient statistics thereof) that captures all of these cases, we can stitch together the expert behaviors to reach the goal $g$.

    Our approach is to learn a representation space that captures this ability, so that a GCBC objective used in this space can effectively imitate the expert on the composed task.
     We begin with the goal-conditioned behavioral cloning~\citep{kaelbling1993learning}
        loss $\cL_{\textsc{bc}}^{\phi,\psi,\xi}$ conditioned with waypoints $w$.
    \begin{equation}
        \cL_{\textsc{bc}}\bigl(\{s_{i},a_{i},s_{i}^{+},g_{i}\}_{i=1}^{K}\bigr) = \sum_{{i=1}}^{K} \log \pi\bigl(a_{i} \mid s_{i},\psi(g_{i})\bigr).
        \label{eq:goal_conditioned_bc}
    \end{equation}
    Enforcing the invariance needed to stitch \cref{eq:trajectory_diagram} then reduces to aligning \mbox{$\psi(g) \leftrightarrow \psi(w).$}
    The temporal alignment objective $\phi(s)\leftrightarrow \phi(s^{+})$ accomplishes this indirectly by aligning both $\psi(w)$ and $\psi(g)$ to the shared waypoint representation $\phi(w)$:

    \csuse{color indices}
    \begin{align}
        &\cL_{\textsc{nce}}\bigl(\{s_{i},s_{i}^{+}\}_{i=1}^{K};\phi,\psi\bigr) =
        \log \biggl( {\frac{e^{\phi(s^+_{\i})^{T}\psi(s_{\i})}}{\sum_{{\j=1}}^{K}
                e^{\phi(s^+_{\i})^{T}\psi(s_{\j})}}} \biggr)  \nonumber\\*
                &\mspace{100mu} +
        \sum_{{\j=1}}^{K} \log \biggl( {\frac{e^{\phi(s^+_{\i})^{T}\psi(s_{\i})}}{\sum_{{\i=1}}^{K}
                e^{\phi(s^+_{\i})^{T}\psi(s_{\j})}}} \biggr)
        \label{eq:goal_alignment}
    \end{align}

        
\end{rebuttal}
\subsection{Interfacing with Language Instructions}
\label{sec:language_instructions}

To extend the representations from \cref{sec:reaching_goals} to compositional instruction following with language tasks, we need some way to ground language into the $\psi$ (future state)
representation space.
We use a similar approach to GRIF~\citep{myers2023goal}, which uses an additional CLIP-style \citep{radford2021learning} contrastive alignment loss with an additional pretrained language encoder $\xi$:
\csuse{no color indices}
\begin{align}
    &\cL_{\textsc{nce}}\bigl(\{g_{i},\ell_{i}\}_{i=1}^{K};\psi,\xi\bigr)
    = \sum_{{i=1}}^{K} \log \biggl( {\frac{e^{\psi(g_{\i})^{T}\xi(\ell_{\i})}}{\sum_{{\j=1}}^{K}
            e^{\psi(g_{\i})^{T}\xi(\ell_{\j})}}} \biggr)  \nonumber\\*
            &\mspace{100mu} +
    \sum_{{\j=1}}^{K} \log \biggl( {\frac{e^{\psi(g_{\i})^{T}\xi(\ell_{\i})}}{\sum_{{\i=1}}^{K}
            e^{\psi(g_{\i})^{T}\xi(\ell_{\j})}}} \biggr)
    \label{eq:task_alignment}
\end{align}

\subsection{Temporal Alignment}
\label{sec:temporal_alignment}

Putting together the objectives from \cref{sec:reaching_goals,sec:language_instructions} yields the Temporal Representation Alignment (\Method) approach.
\Method{} structures the representation space of goals and language instructions to better enable compositional generalization.
We learn encoders $\phi, \psi ,$ and $\xi$ to map states, goals, and language instructions to a shared representation space.

\csuse{color indices}
\begin{align}
    \cL_{\textsc{nce}} \label{eq:NCE}
    &(\{x_{i}, y_{i}\}_{i=1}^{K};f,h) =
        \sum_{{\i=1}}^{K} \log \biggl( {\frac{e^{f(y_{\i})^{T}h(x_{\i})}}{\sum_{{\j=1}}^{K}
        e^{f(y_{\i})^{T}h(x_{\j})}}} \biggr) \nonumber\\*
      &\mspace{100mu} +
        \sum_{{\j=1}}^{K} \log \biggl( {\frac{e^{f(y_{\i})^{T}h(x_{\i})}}{\sum_{{\i=1}}^{K}
        e^{f(y_{\i})^{T}h(x_{\j})}}} \biggr) \\
    \cL_{\textsc{bc}} \label{eq:BC}
    &\bigl(\{s_{i},a_{i},s^{+}_{i},\ell_{i}\}_{i=1}^{K};\pi,\psi,\xi\bigr) = \nonumber\\*
      &\mspace{-10mu} \sum_{{i=1}}^{K} \log
        \pi\bigl(a_{i} \mid s_{i},\xi(\ell_{i})\bigr) + \log \pi\bigl(a_{i} \mid
        s_{i},\psi(s^{+}_{i})\bigr) \\
    \cL_{\textsc{tra}}
    &\label{eq:TRA} \bigl( \{s_{i},a_{i},s_{i}^{+},g_{i},\ell_{i}\}_{i=1}^{K}; \pi,\phi,\psi,\xi\bigr)
        \\
    &= \underbrace{\cL_{\textsc{bc}}\bigl(\{s_{i},a_{i},s_{i}^{+},\ell_{i}\}_{i=1}^{K};\pi,\psi,\xi\bigr)}_{\text{behavioral
    cloning}} \nonumber\\*
    &+
        \underbrace{\cL_{\textsc{nce}}\bigl(\{s_{i},s_{i}^{+}\}_{i=1}^{K};\phi,\psi\bigr)}_{\text{temporal alignment}}
        + \underbrace{\cL_{\textsc{nce}}\bigl(\{g_{i},\ell_{i}\}_{i=1}^{K};\psi,\xi\bigr)}_{\text{task alignment}} \nonumber
\end{align}Note that the NCE alignment loss uses a CLIP-style symmetric contrastive objective~\citep{radford2021learning,eysenbach2024inference} \-- we highlight the indices in the NCE alignment loss~\eqref{eq:NCE} for clarity.

Our overall objective is to minimize \cref{eq:TRA} across states, actions, future states, goals, and language tasks within the training data:
\begin{align}
    &\min_{\pi,\phi,\psi,\xi} \mathbb{E}_{\substack{(s_{1,i},a_{1,i},\ldots,s_{H,i},a_{H,i},\ell) \sim \mathcal{D} \\
    i\sim\operatorname{Unif}(1\ldots H) \\
    k\sim\operatorname{Geom}(1-\gamma)}} \\*
    &\mspace{10mu}
    \Bigl[\cL_{\text{TRA}}\bigl(\{s_{t,i},a_{t,i},s_{\min(t+k,H),i},s_{H,i},\ell\}_{i=1}^{K};\pi,\phi,\psi,\xi\bigr)\Bigr].
    \label{eq:overall_objective}
\end{align}

\begin{algorithm}
    \caption{Temporal Representation Alignment}
    \label{alg:tra}
    \begin{algorithmic}[1]
        \State \textbf{input:} dataset $\mathcal{D} = (\{s_{t,i},a_{t,i}\}_{t=1}^{H},\ell_i)_{i=1}^N$
        \State initialize networks $\Theta \triangleq (\pi,\phi,\psi,\xi)$
        \While{training}
        \State sample batch $\bigl\{(s_{t,i},a_{t,i},s_{t+k,i},\ell_i)\bigr\}_{i=1}^K\sim\mathcal{D}$ \\
        \hspace*{2ex} for $k\sim\operatorname{Geom}(1-\gamma)$
        \State $\Theta \gets \Theta - \alpha \nabla_{\Theta} \cL_{\text{TRA}}\bigl(\{s_{t,i},a_{t,i},s_{t+k,i},\ell_i\}_{i=1}^K; \Theta\bigr)$
        \EndWhile
        \smallskip
        \State \textbf{output:} \parbox[t]{\linewidth}{language-conditioned policy $\pi(a_{t} | s_{t}, \xi(\ell))$ \\
            goal-conditioned policy $\pi(a_{t} | s_{t}, \psi(g))$
        }
    \end{algorithmic}
\end{algorithm}

\subsection{Implementation}
\label{sec:implementation}

A summary of our approach is shown in \cref{alg:tra}.
In essence, TRA learns three encoders: $\phi$, which encodes states, $\psi$ which encodes future goals, and $\xi$ which encodes language instructions.
Contrastive losses are used to align state representations $\phi(s_{t})$ with future goal representations $\psi(s_{t+k})$, which are in turn aligned with equivalent language task specifications $\xi(\ell)$ when available.
We then learn a behavior cloning policy $\pi$ that can be conditioned on either the goal or language instruction through the representation $\psi(g)$ or $\xi(\ell)$, respectively.

\begin{rebuttal}
    \subsection{Temporal Alignment and Compositionality}
    \label{sec:compositionality}

    We will formalize the intuition from \cref{sec:reaching_goals} that \Method{} enables compositional generalization by considering the error on a ``compositional'' version of $\cD,$ denoted $\cD^{*}$.
    Using the notation from \cref{eq:trajectory}, we can say $\cD$ is distributed according to:
    \begin{align}
        &\cD \triangleq \cD^{H} \sim \prod_{i=1}^{K} p_0(s_{1,i}) p_{\ell}(\ell_{i} \mid s_{1,i}, s_{H,i})
            \nonumber\\*
          &\mspace{60mu} \prod_{t=1}^{H} \expert(a_{t,i} \mid s_{t,i}) \p(s_{t+1,i} \mid s_{t,i}, a_{t,i}) ,
            \label{eq:dataset_distribution}
    \end{align}
    or equivalently
    \begin{align}
            &\cD^{H} \sim \prod_{i=1}^{K} p_0(s_{1,i}) p_{\ell}(\ell_{i} \mid s_{1,i}, s_{H,i}) \nonumber\\*
            &\mspace{60mu} \prod_{t=1}^{H}
            e^{\sigma^2\|\expert(s_{t,i}) - a_{t,i}\|^2}\p(s_{t+1,i} \mid s_{t,i}, a_{t,i}) ,
            \label{eq:dataset_distribution_2}
    \end{align}
    by the isotropic Gaussian assumption.
    We will define $\cD^{*} \triangleq \cD^{H'}$ to be a longer-horizon version of $\cD$ extending the behaviors gathered under $\expert$ across a horizon $\alpha H \ge H' \ge H$ that additionally satisfies a ``time-isotropy'' property: the marginal distribution of the states is uniform across the horizon, i.e., $p_0(s_{1,i}) = p_0(s_{t,i})$ for all $t \in \{1\ldots H'\}$.

    We will relate the in-distribution imitation error $\textsc{Err}(\bullet; \cD)$ to the compositional out-of-distribution imitation error $\textsc{Err}(\bullet;\cD^{*})$.
    We define
    \begin{align}
        \textsc{Err}(\hat{\pi}; \tilde{\cD})
        &= \E_{\tilde{\cD}}\Bigl[\frac{1}{H}\sum_{t=1}^{H} \mathbb{E}_{\hat{\pi}}\left[\|\tilde{a}_{t,i} -
        \hat{\pi}(\tilde{s}_{t,i}, \tilde{s}_{H, i})\|^{2}/d_{\cA}\right]\Bigr] \nonumber\\
        &\quad \text{for} \quad \{\tilde{s}_{t,i},\tilde{a}_{t,i},\tilde{\ell}_{i}\}_{t=1}^{H} \sim
            \tilde{\cD}.
            \label{eq:imitation_error}
    \end{align}
    On the training dataset this is equivalent to the expected behavioral cloning loss from \cref{eq:BC}.

                            
    \begin{assumption}
        \label{asm:policy_factorization}
        The policy factorizes through inferred waypoints as:
\begin{align}
    &\textrm{goals: }\pi(a \mid s, g)
        = \nonumber\\*
      &\mspace{50mu} \int \pi(a\mid s, w) \p(s_{t}=w \mid s_{t+k}=g) \dd{w}
        \label{eq:goal-conditioned} \\
    &\textrm{language: } \pi(a \mid s, \ell)
        = \int \pi(a\mid s, w) \nonumber\\*
      &\mspace{20mu} \p(s_{t}=w \mid s_{t+k}=g) \p(s_{t+k}=g \mid \ell) \dd{w} \dd{g} ,
        \label{eq:language-conditioned}
        \end{align}
        where denote by $\pi(s,g)$ the MLE estimate of the action $a$.

    \end{assumption}

    \makerestatable
    \begin{theorem}
        \label{thm:compositionality}
        Suppose $\cD$ is distributed according to \cref{eq:dataset_distribution} and $\cD^{*}$ is distributed according to \cref{eq:dataset_distribution}.
        When $\gamma > 1-1/H$ and $\alpha > 1$, for optimal features $\phi$ and $\psi$ under \cref{eq:overall_objective}, we have
        \begin{gather}
            \textsc{Err}(\pi; \cD^{*}) \le \textsc{Err}(\pi; \cD) +  \frac{\alpha -1}{2 \alpha }+\Bigl(\frac{ \alpha - 2 }{2\alpha}\Bigr) \1 \{\alpha >2\}  .
            \label{eq:compositionality}
        \end{gather}
    \end{theorem}

    We can also define a notion of the language-conditioned compositional generalization error:
    \begin{equation*}
        \errl(\pi; \cD^{*}) \triangleq \E_{\cD^{*}}\Bigl[\frac{1}{H}\sum_{t=1}^{H}
            \mathbb{E}_{\pi}\bigl[\|\tilde{a}_{t,i} - \pi(\tilde{s}_{t,i}, \tilde{\ell}_{i})\|^{2}\bigr]\Bigr].
            \label{eq:language_error}
    \end{equation*}

    \makerestatable
    \begin{corollary}
        \label{thm:language}
        Under the same conditions as \cref{thm:compositionality},
        \begin{equation*}
            \errl(\pi; \cD^{*}) \le \errl(\pi; \cD) +  \frac{\alpha -1}{2 \alpha }+\Bigl(\frac{ \alpha - 2 }{2\alpha}\Bigr) \1 \{\alpha >2\}  .
            \label{eq:compositionality_language}
        \end{equation*}

    \end{corollary}

    The proofs as well as a visualization of the bound are in \cref{app:compositionality}. Policy implementation details can be found in \cref{app:tra_impl}

    
                
        
    \end{rebuttal}

\section{Case Study on SafeSPLE}
We now demonstrate via a case study one way to implement a SafeSPL and parameterized safety cases.  The first part of our process is a hazard analysis. We then build a feature model. The features are then used to parameterize our safety case. Lastly, we can generate safety-case instances as requested for any of the concrete combinations of features.  

\subsection{Hazard Analysis}
To begin the SafeSPLE process for a UAS flight, we analyze the hazards of that flight, which is an important first step before creating a safety case \cite{Knig12}. A hazard is a state or event that can potentially result in an accident \cite{ericson2015hazard}. In this work, we do not describe this part of the process in depth but rather list a few of the key hazards we identified. We utilized several sources to create our list of hazards. First, we referenced several papers describing hazard analysis or safety cases for sUAS flights \cite{denpai2016, clodenpai2017, sora}. Next, we discussed sUAS hazards with colleagues and experts who have studied sUAS and flown them. This investigation gave us an extensive list of hazards, which was too long and broad to include in this paper. We narrowed down this extensive list to focus on the following hazards. 

\begin{itemize}
    \item Too much precipitation
    \item Insufficient visibility
    \item Temperatures outside the operating specifications of the sUAS
    \item Wind gusts outside the operating specifications of the sUAS
    \item Insufficient battery for the mission
\end{itemize}

The hazards above are not intended to be fully described or defined, and we do not include prevention or recovery controls or escalation factors for any of these hazards (see \cite{denpai2016} for a more in-depth discussion of hazard analysis). The ultimate consequences of each of the above hazards are generally either loss of separation from the ground or loss of separation from other air traffic. Either of these consequences could lead to the destruction of property, injury, or death. A complete risk analysis of these consequences is likewise beyond the scope of this paper. We illustrate our family-based approach below using a subset of the identified hazards in order to show how the parameterized safety case addresses the hazards for different sUAS.       

\subsection{Feature Model}

\begin{figure}[ht]
    \centering
    \includegraphics[width=.8\textwidth]{Figures/safety-case-blow-up.pdf}
    \caption{Two parts of the feature model that we focus on for this case study.}
    \label{fig:feature_model_focus}
\end{figure}

The next step in the SafeSPLE process is to create a partial feature model that could apply to a wide variety of sUAS models and missions in controlled airspace as described in Section \ref{sec:SafeSPLE} and Figure \ref{fig:featuremodel}. Since this feature model includes information about the pilot, airspace, mission, vehicle, and weather (among other things), it allows for a wide variety of different types of parameters to be used in our parameterized safety case. In figure \ref{fig:feature_model_focus} we show the two parts of the feature model that are the focus of our safety cases here - the pilot and the weather. These parameterized safety cases are described in the next section. 


\subsection{Parameterized Safety Case}

\begin{figure}[ht]
    \centering
    \includegraphics[width=.7\textwidth]{Figures/pilot_only.pdf}
    \caption{Pilot Safety Case: A safety case based only on whether the pilot is certified and has sufficient experience.}
    \label{fig:pilot_only}
\end{figure}

The next step in our case study (based on SafeSPLE) is to create two illustrative parameterized safety cases for our controlled airspace. The first safety case, seen in Figure \ref{fig:pilot_only}, is based solely on the pilot. It checks whether the pilot is certified and has sufficient flight hours. We assume that in non-commercial airspaces, flight regulations would trust a certified pilot with sufficient reputation (i.e., no significant history of problems) to perform safety checks consistent with the lower-level details of our safety cases. In other words, the pilot is in charge of ensuring a safe flight in whatever airspace they are in. Regulators often do not exclude pilots legally allowed to be in the airspace unless there is some serious prior issue \cite{FAA_TRUST, FAA_part107}. So it is our belief that any UAS Traffic Management system will likely allow certified pilots to enter the airspace unless it has some reason not to.

As shown in Figure \ref{fig:pilot_only}, our safety case checks to see if the pilot is certified to fly their sUAS, here represented using the FAA's Part 107 certification \cite{FAA_part107}. We also check to see if the pilot has sufficient flight hours to be competent to complete this flight, which is something that our managed airspace should know.  In the future, this flight-hours check might be replaced or augmented with different checks, such as the pilot's score on a competency-reputation metric, future certifications, or temporary notices to pilots that the FAA might put out. If evidence of these checks confirms that the pilot is certified and has sufficient experience to enter the controlled airspace, the associated strategy node (S1) in the safety case is satisfied.

\begin{figure}[ht]
    \centering
    \includegraphics[width=\columnwidth]{Figures/wind_only.pdf}
    \caption{Wind Safety Case: A parameterized safety case based only on the weather and the drone's capabilities. This safety case creates the instances seen in Figures \ref{fig:instance_1} and \ref{fig:instance_2}.}
    \label{fig:wind_only}
\end{figure}

Our second safety case is relevant when the pilot lacks the evidence required to satisfy our initial safety case above. There needs to be an opportunity for newer pilots to learn and fly if such flights can be done safely. Thus, our second safety case focuses on giving such pilots the information that they will need in order to complete a safe flight. This second safety case (Figure \ref{fig:wind_only}) focuses on the weather because poor weather is a common reason for a pilot to decide that a flight will not be safe or for in-flight failures \cite{weather_hazards_for_UAV}. The weather portion of the feature model also has several parameters that can map to portions of our safety case. This sort of weather-focused safety case would normally involve far more attributes than we show in Figure \ref{fig:wind_only}, but we focus only on the weather and a small amount of information (evidence) about the battery here.

In the wind safety case from Figure \ref{fig:wind_only}, we constructed a general safety case that involves a number of parameters that are found in our feature model. These parameters are indicated using square brackets, such as [Precipitation] and [UASAllowedPrecipitation]. The data types of each parameter are left intentionally vague, as there are a number of ways for these parameters to be stored. We assume that information about each [Vehicle] is publicly available and that published sUAS specifications can be converted into the same data format and type that the feature model and safety case parameters have. If a [Vehicle] does not contain information in its specifications for certain parameters, then there is an option to assume some default values that could apply to almost all drones. 

For instance, most sUAS specifications will include information about the maximum allowed wind speed within which the manufacturer states the sUAS can operate. Likewise, most sUAS specifications include both maximum and minimum allowed temperatures in which to operate (often from -10 \textcelsius \;or 0 \textcelsius \;up to 40 \textcelsius) \cite{DEERCD20, DJI_MiniPro_4_Specs}. Fewer sUAS specifications contain specific information about visibility requirements since those depend on the type of mission being flown, especially whether it needs to be flown in a visual line of sight (VLOS) or beyond a visual line of sight (BVLOS). If the pilot does not provide visibility requirement information, we thus assume that the flight must take place VLOS and proceed accordingly. Similarly, if no information is provided about an sUAS's ability to fly in various forms of precipitation, we assume that the sUAS can only operate with no precipitation. 

Note that in the wind safety case (Figure \ref{fig:wind_only}), many of the goals share a similar structure. For instance, "The forecast precipitation is within acceptable level..." and "The forecast visibility is within acceptable levels...". The repetition of these elements is intentional and allows for greater ease of human understanding of the safety case, as well as for simpler extension of the safety case when we add additional hazards we need to mitigate. 

Some of the values of the parameters in the safety case may not be available at the time of a flight request. For instance, if a pilot is applying to complete a flight several weeks or months in the future, the forecast weather conditions will be unreliable. In such a case, the safety case might not contain concrete values until closer to the flight. The pilot could still access the parameterized safety case in order to study the safety requirements for the flight. As the time of the flight approaches, a more fully instantiated safety case could be sent to the pilot. 

The information for instantiating these parameterized safety cases will need to be pulled from a variety of sources, such as publicly available weather data and manufacturers' specifications for commercially available sUAS. However, some of the parameters' information will need to come from the pilot, including their certification status, the sUAS model they will fly, their flight plan, and any additional sUAS capabilities they have added (such as detect-and-avoid systems). 
In the event that the sUAS being flown was completely home-built, there may be no public documentation of its abilities, and all of its specifications will need to be provided (or inferred) by the pilot. 
Therefore, some of the individual safety cases will necessarily contain a fair amount of uncertainty while still serving as a guideline for the pilot. 


\subsection{Instances}
As a final step in our SafeSPLE process, we demonstrate how to create instances of our parameterized safety case. This process involves obtaining the information required for all parameters and checking if all the solution nodes of the safety case remain true. In all of the safety case diagrams in Figures \ref{fig:pilot_only}, \ref{fig:wind_only}, \ref{fig:instance_1}, \ref{fig:instance_2}, these solution nodes are the bottom nodes labeled E1-E6, and have propagated from the context. If any solution node becomes false, then we can say that the pilot should either reconsider the flight, or should implement further mitigations to reduce the risk from the relevant hazard. For instance, if the safety case shows that the current wind gusts are too high, the pilot might delay the flight until the wind calms, or the pilot might decide to make the flight with a larger and more capable UAS (if available).

\begin{figure}[ht]
    \centering
    \includegraphics[width=.95\textwidth]{Figures/Instance_1.pdf}
    \caption{Safety Case Instance 1: An instance of the wind safety case (Figure \ref{fig:wind_only}) based on a mission with a DJI Mini 4 Pro drone.}
    \label{fig:instance_1}
\end{figure}

The first instance of our parameterized safety case is shown in Figure \ref{fig:instance_1}. This mission will be performed by a DJI Mini 4 Pro, a widely available drone that currently sells for just over \$1000, depending on accessories. The Mini 4 Pro is fully charged, and the mission, as planned, should take 16 minutes, flown entirely within VLOS of the pilot. This information about battery charge and the mission plan is provided by the pilot. The wind is gusting up to 6 meters/sec, with temperatures in the mid 20s \textcelsius, unlimited visibility, and no precipitation. This weather information is provided to the safety case by a commercial or governmental weather service. 

Once we obtain the information about the make and model of the drone, we can look up the DJI's published specifications. According to DJI \cite{DJI_MiniPro_4_Specs}, the Mini 4 Pro is able to fly in wind speeds up to 10 m/s, and with a fully charged battery can fly up to 34 minutes. The Mini 4 Pro can operate in temperatures between -10 \textcelsius \;and 40 \textcelsius. Using all this information, we can instantiate the safety case seen in Figure \ref{fig:instance_1}. Note that every solution node 
(labeled E1-E6) is satisfied by the above information. There is no precipitation; visibility is unlimited; the temperatures are not too hot or cold; the wind gusts are below the max allowed for the drone; and the battery reserves are more than twice as much as needed. So in this instance of the safety case the the top-level goal is satisfied. 

\begin{figure}[ht]
    \centering
    \includegraphics[width=.95\textwidth]{Figures/Instance_2.pdf}
    \caption{Safety Case Instance 2: An instance of the wind safety case (Figure \ref{fig:wind_only}) based on a mission with a DEERC D20 drone. Note that this instance fails to fulfill our safety requirements at node E4 (marked in darker red).}
    \label{fig:instance_2}
\end{figure}
%Nice examples!

In Figure \ref{fig:instance_2} we can see a second instance of our safety case. This mission will be performed by a DEERC D20 drone, another widely available drone that currently sells for around \$50. The D20 is also fully charged, and the planned mission will only take 5 minutes of flying, entirely within VLOS. The wind is gusting up to 8 m/s, with temperatures in the mid-30s \textcelsius, 3 km visibility, and no precipitation.

According to the DEERC documentation \cite{DEERCD20}, the D20 drone is capable of about 10 minutes of flight time in temperatures between 0 \textcelsius \ and 40 \textcelsius. However, the D20 documentation does not specify the maximum speed of the winds that the drone is capable of flying in. Instead, the documentation reads, "DO NOT use this drone in adverse weather conditions such as rain, snow, fog, and wind." Therefore the safety case takes a conservative approach and assigns a default value of 3 m/s to the variable [MaxAllowedWindSpd] (3 m/s is slightly less than 7 mph). This default value could, of course, be set to 0 m/s, although this seems unrealistic for most outdoor flying. Other default values might be justified.

Plugging in all of these values, we see that while most solution nodes are satisfied, the current wind conditions (gusts up to 8 m/s) do not allow for a safe flight with the D20 (default max wind speed of 3 m/s). In Figure \ref{fig:instance_2}, this is shown at solution node E4, which is colored a darker red than the other solution nodes. The safety case is designed to serve as input to the UTM on-entry decision. At this point there are two main options for how the UAS Traffic Manager could behave. The UTM could refuse entry to this pilot until the wind speed is lower, or the UTM could send the safety case to the pilot with the recommendation that the pilot make modifications to the flight plan while leaving the ultimate flight decision up to the pilot.

Creating instances of safety cases with SafeSPL should be quick and relatively straightforward, if the information it needs is available. If information on the drone's capabilities is lacking, default values can still allow the safety case to create a reasonable instance. If information about the weather is unknown, then those portions of the safety case can be left uninstantiated until more detailed information becomes available. At the very least, we can generate a partially instantiated safety case so the pilot can see the areas where information is lacking or is based on default values. This information could allow the pilot to focus on mitigation measures in those areas if needed. 

\subsection{Connecting to Safe Entry}
\label{sec:safe_entry}

The parameterized safety cases created by SafeSPLE and described above could play an important role in a to-be-developed UTM system. When a pilot requests permission to fly in the airspace controlled by the UTM, the information needed to instantiate the safety case is either submitted by the pilot or looked up by the UTM system. Once a safety case has been created for that flight, there are at least two options for what the UTM system might do with it. 

\begin{enumerate}
    \item Closed Access: The UTM system accepts or denies requests based on whether each generated safety case "passes" or "fails". In other words, if the safety case goals are not satisfied, the UTM system  denies the flight. 
    \item Open Access: The UTM system accepts or denies the flight based solely on whether the pilot is certified or trusted. The safety case then becomes a guideline that can be provided to the pilot as something of a checklist to encourage a safer flight.
\end{enumerate}

Which action the UTM should take is an ongoing discussion with no immediate correct answer.
Currently the regulations in the US appear to generally favor approach (2), the open-access model. Regardless of which approach is taken for a specific controlled airspace, we believe the use of SafeSPLE will generate valuable on-the-fly information.  This information may offer an effective and useful checklist for decision-making. 

\section{Conclusion}
In this work, we propose a simple yet effective approach, called SMILE, for graph few-shot learning with fewer tasks. Specifically, we introduce a novel dual-level mixup strategy, including within-task and across-task mixup, for enriching the diversity of nodes within each task and the diversity of tasks. Also, we incorporate the degree-based prior information to learn expressive node embeddings. Theoretically, we prove that SMILE effectively enhances the model's generalization performance. Empirically, we conduct extensive experiments on multiple benchmarks and the results suggest that SMILE significantly outperforms other baselines, including both in-domain and cross-domain few-shot settings.
\section*{Acknowledgments}
This work was supported by HKRGC GRF (Project ID: 14306721), and Hong Kong Centre for Cerebro- Cardiovascular Health Engineering (COCHE).

\bibliography{main.bib}

\end{document}