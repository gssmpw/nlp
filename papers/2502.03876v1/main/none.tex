% \subsubsection{Basic Properties for $R(\mathcal{C})$}

% We aim to maximize the number of surface points while minimizing the presence of anomaly and outlier points. Consequently, the function \( R(\cdot) \) should satisfy the following condition:
% \begin{equation}
%   \begin{aligned}
%     R(\mathcal{C}_1) <& R(\mathcal{C}_2) \iff  0<\sum\limits_{i=1}^N c_{2i}-c_{1i}
% \end{aligned}  
% \end{equation}


% \subsubsection{Basic Properties for \(E(\cdot,\cdot)\)}
% We will discuss three categories. For the points on the surface, we aim to minimize the shortest distance from each point to the reference surface. Assuming the equation of the reference surface is \( g(x) = 0 \), for points \( x_1 \) and \( x_2 \), the error term \( E \) should satisfy the following condition:
% \begin{equation}
% \begin{aligned}
%         E(\mathcal{C}; x_1) <& E(\mathcal{C};x_2) \iff \\ \min_{g(x) = 0} \, |x-x_1| <& \min_{g(x) = 0} \, |x-x_2|
% \end{aligned}
% \end{equation}


% For discrete points, no control is applied, as the regularization term has already addressed this. For defect points, we aim for continuity, meaning the distance to the nearest point should be relatively small. This characteristic distinguishes defect points from discrete points. We have:

% \begin{equation}
% \begin{aligned}
% E(\mathcal{C}; x_1) <& E(\mathcal{C}; x_2) \iff \\
% \min_{i} |x_i-x_1| <& \min_{i} |x_i-x_2|
% \end{aligned}
% \end{equation}


% 

% \subsection{Estimate $\mathcal{C}$}
% Since the category is a binary variable, we consider relax $c_i$ as a variable range from $[0,1]$.

% Here we mainly consider solving problem $\arg\min_{\mathcal{C}} \sum\limits_{i=1,j\neq i}^N\sum\limits_{j=1,j\neq i}^N \frac{|c_i-c_j|}{|x_i-x_j|^p}$. Since $|x_i-x_j|=(x_i-x_j)^2$, we have
% \begin{equation}
%    \sum\limits_{i=1,j\neq i}^N \sum\limits_{j=1,j\neq i}^N \frac{|c_i-c_j|}{|x_i-x_j|^p} = \sum\limits_{i=1,j\neq i}^N\sum\limits_{j=1,j\neq i}^N \frac{c_i^2+c_j^2-2c_ic_j}{\|x_i-x_j\|^p}
% \end{equation}

% Consider an $n\times n$ matrix $M_x$ where the element in the i-th row and j-th column is $\frac{1}{\|x_i-x_j\|^p}$ or $0$(when $i=j$), we have

% \begin{equation}
% \sum\limits_{i=1,j\neq i}^N\sum\limits_{j=1,j\neq i}^N \frac{c_i^2+c_j^2-2c_ic_j}{\|x_i-x_j\|^p} = -2\mathbf{c}M_x\mathbf{c}^T + aqaaaaqqqqqqqqqqqqqqqqqAAAAAQQQQAAAAAA
% \end{equation}





% For boundary points, whether to classify as surface point or anomaly points depends on its neighbour properties. 距离越远的点,对当前点的贡献就越小。我们希望,如果一个点是surface point,它附近的点也是surface point。对于anomaly point,也适用。因此我们定义$\mathcal{B}(\cdot,\cdot)$为:

% \begin{equation}
%     \mathcal{B}(\mathcal{C}; x_i) = \sum\limits_{j=1,j\neq i}^N \frac{|c_i-c_j|}{\|x_i-x_j\|^p}
% \end{equation}

% 其中$p$是一个超参数。p越大,则参考的邻域越小。反之,参考的邻域越大。我们通过控制$\sum\limits_{i=1}^N\mathcal{B}(\mathcal{C}; x_i)$最小化,可以保证anomaly的边界也不错误分类到surface point.We assert that $p>2$.

% 接下来我们对$p$的取值与参考邻域大小$r$的值做一个估计。我们假设点云分布在一个近似平面上,点云的密度为$\rho$,则该无穷和可以近似为
% \begin{equation}
%     \mathcal{B}(\mathcal{C}; x_i)\le \sum\limits_{j=1,j\neq i}^\infty \frac{1}{\|x_i-x_j\|^p}
% \end{equation}

% 我们进行坐标变换,选取$x_i$为圆心,设$x_j=(x\sin\theta,x\sin\theta,0)$,则有

% \begin{equation}
%     \mathcal{B}(\mathcal{C}; x_i)\le \sum\limits_{j=1,j\neq i}^\infty \frac{1}{\|r\|^p}
% \end{equation}

% 接下来我们估计$x_j$的密度。在半径为$r$到$r+\Delta r$的圆环内,圆环的面积是$\pi r \Delta r$,因此这里的点云数量是$\pi r \Delta r \rho$。因此我们将无穷和转化为积分,有

% \begin{equation}
%     \mathcal{B}(\mathcal{C}; x_i)\le \int\limits_{r=d}^\infty \frac{\pi \rho dr}{\|r\|^{p-1}}
% \end{equation}

% 这里,$d=\min\limits_i\|x_i-x_j\|$。该积分收敛的充要条件是:$p>2$。这也是我们设置限制$p>2$的理由。

% 接下来我们考虑对于不同的$p$,能否得到一个近似的约束半径$r$。我们将该积分分成两个部分

% \begin{equation}
%      \int\limits_{r=d}^\infty \frac{\pi \rho dr}{\|r\|^{p-1}} = \int\limits_{r=d}^a \frac{\pi \rho dr}{\|r\|^{p-1}} + \int\limits_{r=a}^\infty \frac{\pi \rho dr}{\|r\|^{p-1}}
% \end{equation}

% 则对于各个部分我们有:

% \begin{equation}
%      \int\limits_{r=d}^\infty \frac{\pi \rho dr}{\|r\|^{p-1}} = \frac{a^{p-1}-d^{p-1}}{p-2} + \frac{d^{p-1}}{p-2}
% \end{equation}

% 因此,前后两项之比为$(\frac{a}{d})^{p-1}-1$。我们可以假设前一项贡献了$1/K$的积分,此时前后两项之比为$K$。带入化简,我们有

% \begin{equation}
%      \frac{a}{d}= (K+1)^{\frac{1}{p-1}}
% \end{equation}

% 当$K$很大时,我们有$K+1 \approx K$。 因此,控制邻域$a$的估计范围是

% \begin{equation}
%     a \approx dK^{\frac{1}{p-1}}
% \end{equation}

% 然而这里$d$依然是变量。下面我们来估计$d$。

% 我们考虑一个半径为$R$的圆盘,则在这个圆盘上有$\pi R^2 \rho$ 个点。在一个半径为$r$,宽度为$\Delta r$ 的圆环上,有$2\pi r \Delta r \rho$ 个点, 它们到原点的距离为$r$。因此我们可以计算数学期望
% \begin{equation}
%     E(d) = \int\limits_0^R \frac{2\pi \rho r^2 dr}{\pi R^2 \rho} 
% \end{equation}

% 化简有
% \begin{equation}
%      E(d) = \frac{2R}{3} 
% \end{equation}

% 我们希望在半径为$R$的圆盘上恰好有一个点,即满足
% \begin{equation}
%     \pi R^2 \rho = 1
% \end{equation}

% 带入反解,我们有

% \begin{equation}
%     E(d) = \frac{2}{3}\sqrt{\frac{1}{\pi \rho}}
% \end{equation}

% 因此,我们把$d$的期望带入,有

% \begin{equation}
%     a \approx  \frac{2}{3}\sqrt{\frac{1}{\pi \rho}}K^{\frac{1}{p-1}}
% \end{equation}

% 根据任务期望控制的$K$值和$a$值,我们可以求得所需$p$。

% \begin{equation}
%     lg\frac{3a\sqrt{{\pi \rho}}}{2}=lgK*{\frac{1}{p-1}}
% \end{equation}
% \begin{equation}
%     p= \frac{\ln K}{\ln\frac{3a\sqrt{{\pi \rho}}}{2}}+1
% \end{equation}

\section{Estimate on Boundary}
For boundary points, their classification as surface points or anomaly points depends on their neighboring properties. Points farther away contribute less to the current point's classification. We expect that if a point is a surface point, its nearby points are also surface points. The same applies to anomaly points. Therefore, we define $\mathcal{B}(\cdot,\cdot)$ as:

\begin{equation}
\mathcal{B}(\mathcal{C}; x_i) = \sum\limits_{j=1,j\neq i}^N \frac{|c_i-c_j|}{|x_i-x_j|^p}
\end{equation}

where $p$ is a hyperparameter. A larger $p$ results in a smaller reference neighborhood, while a smaller $p$ leads to a larger reference neighborhood. By minimizing $\sum\limits_{i=1}^N\mathcal{B}(\mathcal{C}; x_i)$, we can ensure that anomaly boundaries are not misclassified as surface points. We assert that $p>2$.

Next, we estimate the relationship between $p$ and the reference neighborhood size $r$. Assuming the point cloud is distributed on an approximate plane with density $\rho$, the infinite sum can be approximated as:

\begin{equation}
\mathcal{B}(\mathcal{C}; x_i)\le \sum\limits_{j=1,j\neq i}^\infty \frac{1}{|x_i-x_j|^p}
\end{equation}

Through coordinate transformation, selecting $x_i$ as the center and setting $x_j=(x\sin\theta,x\sin\theta,0)$, we have:

\begin{equation}
\mathcal{B}(\mathcal{C}; x_i)\le \sum\limits_{j=1,j\neq i}^\infty \frac{1}{|r|^p}
\end{equation}

To estimate the density of $x_j$, consider that in an annulus with radius from $r$ to $r+\Delta r$, the area is $\pi r \Delta r$, containing $\pi r \Delta r \rho$ points. Converting the infinite sum to an integral:

\begin{equation}
\mathcal{B}(\mathcal{C}; x_i)\le \int\limits_{r=d}^\infty \frac{\pi \rho dr}{|r|^{p-1}}
\end{equation}

where $d=\min\limits_i|x_i-x_j|$. This integral converges if and only if $p>2$, which justifies our constraint.

To determine an approximate constraint radius $r$ for different values of $p$, we split the integral:

\begin{equation}
\int\limits_{r=d}^\infty \frac{\pi \rho dr}{|r|^{p-1}} = \int\limits_{r=d}^a \frac{\pi \rho dr}{|r|^{p-1}} + \int\limits_{r=a}^\infty \frac{\pi \rho dr}{|r|^{p-1}}
\end{equation}

For each part:

\begin{equation}
\int\limits_{r=d}^\infty \frac{\pi \rho dr}{|r|^{p-1}} = \frac{a^{p-1}-d^{p-1}}{p-2} + \frac{d^{p-1}}{p-2}
\end{equation}

The ratio between the terms is $(\frac{a}{d})^{p-1}-1$. Assuming the first term contributes $1/K$ of the integral, where $K$ is the ratio:

\begin{equation}
\frac{a}{d}= (K+1)^{\frac{1}{p-1}}
\end{equation}

For large $K$, $K+1 \approx K$. Thus, the estimated neighborhood range $a$ is:

\begin{equation}
a \approx dK^{\frac{1}{p-1}}
\end{equation}

To estimate $d$, consider a disk of radius $R$ containing $\pi R^2 \rho$ points. In an annulus of radius $r$ and width $\Delta r$, there are $2\pi r \Delta r \rho$ points. The expected value is:

\begin{equation}
E(d) = \int\limits_0^R \frac{2\pi \rho r^2 dr}{\pi R^2 \rho} = \frac{2R}{3}
\end{equation}

For exactly one point in the disk:
\begin{equation}
\pi R^2 \rho = 1
\end{equation}

Therefore:

\begin{equation}
E(d) = \frac{2}{3}\sqrt{\frac{1}{\pi \rho}}
\end{equation}

Substituting the expected value of $d$:

\begin{equation}
a \approx \frac{2}{3}\sqrt{\frac{1}{\pi \rho}}K^{\frac{1}{p-1}}
\end{equation}

Given desired values for $K$ and $a$, we can solve for $p$:

\begin{equation}
lg\frac{3a\sqrt{{\pi \rho}}}{2}=lgK*{\frac{1}{p-1}}
\end{equation}
\begin{equation}
p= \frac{\ln K}{\ln\frac{3a\sqrt{{\pi \rho}}}{2}}+1
\end{equation}