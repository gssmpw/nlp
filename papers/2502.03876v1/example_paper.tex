%%%%%%%% ICML 2025 EXAMPLE LATEX SUBMISSION FILE %%%%%%%%%%%%%%%%%

\documentclass{article}
\usepackage[utf8]{inputenc}
% Recommended, but optional, packages for figures and better typesetting:
\usepackage{microtype}
\usepackage{graphicx}
\usepackage{subfigure}
\usepackage{booktabs} % for professional tables

% hyperref makes hyperlinks in the resulting PDF.
% If your build breaks (sometimes temporarily if a hyperlink spans a page)
% please comment out the following usepackage line and replace
% \usepackage{icml2025} with \usepackage[nohyperref]{icml2025} above.
\usepackage{hyperref}


% Attempt to make hyperref and algorithmic work together better:
\newcommand{\theHalgorithm}{\arabic{algorithm}}

% Use the following line for the initial blind version submitted for review:
% \usepackage{icml2025}

% If accepted, instead use the following line for the camera-ready submission:
\usepackage[accepted]{icml2025}

% For theorems and such
\usepackage{amsmath}
\usepackage{amssymb}
\usepackage{mathtools}
\usepackage{amsthm}
\usepackage{enumitem}

% if you use cleveref..
\usepackage[capitalize,noabbrev]{cleveref}

%%%%%%%%%%%%%%%%%%%%%%%%%%%%%%%%
% THEOREMS
%%%%%%%%%%%%%%%%%%%%%%%%%%%%%%%%
\theoremstyle{plain}
\newtheorem{theorem}{Theorem}[section]
\newtheorem{proposition}[theorem]{Proposition}
\newtheorem{lemma}[theorem]{Lemma}
\newtheorem{corollary}[theorem]{Corollary}
\theoremstyle{definition}
\newtheorem{definition}[theorem]{Definition}
\newtheorem{assumption}[theorem]{Assumption}
\theoremstyle{remark}
\newtheorem{remark}[theorem]{Remark}

% Todonotes is useful during development; simply uncomment the next line
%    and comment out the line below the next line to turn off comments
%\usepackage[disable,textsize=tiny]{todonotes}
\usepackage[textsize=tiny]{todonotes}


% The \icmltitle you define below is probably too long as a header.
% Therefore, a short form for the running title is supplied here:
\icmltitlerunning{Position: Untrained Machine Learning for Anomaly Detection}

\begin{document}

\twocolumn[
\icmltitle{Position: Untrained Machine Learning for Anomaly Detection}

% It is OKAY to include author information, even for blind
% submissions: the style file will automatically remove it for you
% unless you've provided the [accepted] option to the icml2025
% package.

% List of affiliations: The first argument should be a (short)
% identifier you will use later to specify author affiliations
% Academic affiliations should list Department, University, City, Region, Country
% Industry affiliations should list Company, City, Region, Country

% You can specify symbols, otherwise they are numbered in order.
% Ideally, you should not use this facility. Affiliations will be numbered
% in order of appearance and this is the preferred way.
\icmlsetsymbol{equal}{*}

\begin{icmlauthorlist}
% \icmlauthor{Anonymous Author 1}{yyy,sch}
% \icmlauthor{Anonymous Author 2}{yyy}
% \icmlauthor{Anonymous Author 3}{comp}
\icmlauthor{Juan Du}{du}
\icmlauthor{Dongheng Chen}{du}
\icmlauthor{Hao Yan}{yan}
%\icmlauthor{}{sch}
%\icmlauthor{}{sch}
\end{icmlauthorlist}

\icmlaffiliation{du}{Smart Manufacturing Thrust, Systems Hub, The Hong Kong University of Science and Technology (Guangzhou), Guangzhou, China}

\icmlaffiliation{yan}{School of Computing and Augmented Intelligence, Arizona State University, Tempe, USA}
%\icmlaffiliation{sch}{School of ZZZ, Institute of WWW, Location, Country}

%\icmlcorrespondingauthor{Firstname1 Lastname1}{first1.last1@xxx.edu}
\icmlcorrespondingauthor{Dongheng Chen}{chen\_dongheng@qq.com}
\icmlcorrespondingauthor{Juan Du}{juandu@ust.hk}
\icmlcorrespondingauthor{Hao Yan}{HaoYan@asu.edu}
% You may provide any keywords that you
% find helpful for describing your paper; these are used to populate
% the "keywords" metadata in the PDF but will not be shown in the document
\icmlkeywords{Machine Learning, ICML}

\vskip 0.3in
]

% this must go after the closing bracket ] following \twocolumn[ ...

% This command actually creates the footnote in the first column
% listing the affiliations and the copyright notice.
% The command takes one argument, which is text to display at the start of the footnote.
% The \icmlEqualContribution command is standard text for equal contribution.
% Remove it (just {}) if you do not need this facility.

\printAffiliationsAndNotice{}  % leave blank if no need to mention equal contribution
% \printAffiliationsAndNotice{\icmlEqualContribution} % otherwise use the standard text.

\begin{abstract}
% 3D point-cloud anomaly detection is a real-world question underlying industry and computer science. However, in manufacturing, epically personalized manufacturing, obtaining amounts of training data is impossible. This paper aims to propose a review for untrained point-cloud anomaly detection method. Compared with unsupervised learning, untrained method doesn't require any data even unlabeled data. On the contrary, it requires some prior knowledge about the distribution of the points. 
% We are going to discuss this question in four chapters. Chapter 1 will be the introduction to the 3D point-cloud anomaly detection and related works. Chapter 2 will define the problem in mathematics and discuss two routes for solutions, decomposing and classification. Chapter 3 will review current works for point-cloud anomaly detection, and Chapter 4 will discuss the relationship between point-cloud anomaly detection and inverse problems, and consider some potential method for solutions.

% 3D point-cloud anomaly detection is a critical challenge in both industrial applications and computer science. In the context of production, particularly in personalized manufacturing, acquiring large amounts of training data is often infeasible. This paper aims to provide a comprehensive review of untrained point-cloud anomaly detection methods. Unlike unsupervised learning, untrained methods do not rely on any data, including unlabeled data. Instead, they leverage prior knowledge about the distribution of points.

% This paper is organized into four chapters. Chapter 1 introduces the problem of 3D point-cloud anomaly detection and reviews related works. Chapter 2 provides a formal definition of the problem and explores two potential solution approaches: decomposition and classification. Chapter 3 reviews existing methods for point-cloud anomaly detection. Finally, Chapter 4 examines the relationship between point-cloud anomaly detection and inverse problems, and discusses potential methodologies for addressing this challenge.

Anomaly detection based on 3D point cloud data is an important research problem and receives more and more attention recently. Untrained anomaly detection based on only one sample is an emerging research problem motivated by real manufacturing industries such as personalized manufacturing that only one sample can be collected without any additional labels. How to accurately identify anomalies based on one 3D point cloud sample is a critical challenge in both industrial applications and the field of machine learning. This paper aims to provide a formal definition of untrained anomaly detection problem based on 3D point cloud data, discuss the differences between untrained anomaly detection and current unsupervised anomaly detection methods. Unlike unsupervised learning, untrained methods do not rely on any data, including unlabeled data. Instead, they leverage prior knowledge about the manufacturing surfaces and anomalies. Examples are used to illustrate these prior knowledge and untrained machine learning model. Afterwards, literature review on untrained anomaly detection based on 3D point cloud data is also provided, and the potential of untrained deep neural networks for anomaly detection is also discussed as outlooks.

\end{abstract}


\section{Introduction}
\label{Introduction}

Anomaly detection is a crucial research problem within the machine learning community. Recently, there has been significant research focused on using machine learning for anomaly detection. Based on the availability of labels, anomaly detection methods can be categorized into three types: unsupervised, supervised, and semi-supervised anomaly detection. Unsupervised anomaly detection is particularly popular because it does not require labeled data, which is often scarce, allowing the algorithm to independently identify anomalies without the need for predefined labels. In contrast, supervised anomaly detection relies on labeled data, which limits the algorithm to detecting only those anomalies that it has encountered during training. Semi-supervised anomaly detection combines elements of both supervised and unsupervised approaches, enabling the handling of some labeled data alongside large amounts of unlabeled data. Among these three categories, unsupervised anomaly detection is particularly valuable because it eliminates the need for costly data labeling.

Based on the requirements for training data, unsupervised anomaly detection methods can be further divided into training-based methods and untrained methods. Training-based unsupervised methods rely on a sufficient number of anomaly-free training samples to learn the intrinsic patterns of the target object's surfaces, which then allows for the detection of anomalies. This type of unsupervised anomaly detection methods have proven effective in addressing data representation issues and managing the diverse nature of anomalies. However, they are limited by their reliance on large amounts of anomaly-free training data, which can be a significant obstacle in scenarios where data samples are scarce such as personalized manufacturing and personalized medicine. These methods are equipped with sophisticated feature extractors that perform well, and they learn from anomaly-free samples without assuming the nature of potential anomalies, theoretically enabling them to detect any anomaly type. Nonetheless, in practical applications, anomalies with sparse characteristics or patterns resembling those of anomaly-free data can result in inaccurate detections.

In contrast, untrained methods do not require any anomaly-free samples for training, meaning that anomaly detection can be performed using just a single sample. Untrained unsupervised anomaly detection is becoming increasingly valuable as it addresses some of the limitations of traditional training-based unsupervised anomaly detection methods. While current untrained methods still require certain assumptions about anomalies, they offer a flexible and resource-efficient alternative, especially in situations where access to extensive anomaly-free datasets is not feasible such as personalized manufacturing and personalized medicine. By not relying on predefined feature extractors or large datasets, untrained methods can adapt to various data and anomaly types. This flexibility allows them to effectively detect anomalies that might be rare, subtle, or closely resemble normal data. Such adaptability is crucial in real-world applications, where the nature of anomalies is often unpredictable and varied. Therefore, untrained unsupervised anomaly detection is essential for advancing the capability and reliability of anomaly detection systems.

Developing untrained machine learning methods for anomaly detection presents three key challenges. First, effectively representing high-dimensional data for anomaly detection is a complex task. High-density 3D point cloud data can encompass millions of unstructured points, making it difficult to represent and creating hurdles for real-time computation. Second, having only one sample for anomaly detection makes it challenging to learn features and build an effective model. This necessitates a comprehensive utilization of prior knowledge, yet integrating this knowledge into the model remains a significant challenge. Finally, anomalies can appear locally, vary greatly, and are often sparse across objects. New anomalies can arise unexpectedly, and even anomalies of the same type can show significant differences, making detection difficult. Moreover, anomalies are typically sparse on object surfaces, constituting only a small part of the anomaly sample area, which makes the task of learning anomaly features even more difficult.

To address these challenges, several initial efforts have been made, including local geometry-based methods \cite{wang2023mvgcn}, sparse learning-based approaches \cite{tao2025pointsgrade}, and statistical methods \cite{tao2023anomaly}. A more detailed review will be provided in the following section. However, many research problems still need to be addressed. To advance the current state of untrained machine learning for anomaly detection, this paper first formally defines the anomaly detection problem via untrained machine learning. Then three potential research methodology frameworks are proposed, along with three specific examples for each framework. Finally, the paper concludes with a discussion of the conclusions and future outlook for untrained machine learning in anomaly detection.

\section{Paper Review}
In the realm of manufacturing, anomaly detection using 3D point cloud data has witnessed a significant shift towards unsupervised methods, primarily due to the scarcity of annotated data. These unsupervised approaches can be broadly categorized into training-based methods and untrained methods, each with its own characteristics and applications \cite{tao2023anomaly,cao_survey_2024,rani_advancements_2024}.
training-based unsupervised methods operate under the assumption that sufficient anomaly-free training samples are available. This enables them to learn the intrinsic patterns of normal surfaces, which are then used to detect anomalies. There are two main techniques within this category: feature embedding and reconstruction \cite{cao_complementary_2024,chu_shape-guided_2023,horwitz_back_2023,wang_multimodal_2023}.
Feature embedding-based methods involve a two-step process. First, latent features are extracted from anomaly-free training data. For example, in memory bank-based methods like Back To the Feature (BTF) \cite{horwitz_back_2023}, traditional descriptors such as FPFH are used to extract features from training point cloud patches, and these features are stored in a memory bank. The memory bank is then downsampled to represent the distribution of normal features. During inference, features that deviate significantly from this distribution are flagged as anomalies. Another approach is to use knowledge distillation (KD). Methods like 3D-ST \cite{bergmann_anomaly_2023} transfer knowledge from a pre-trained teacher network (e.g., RandLA-Net \cite{hu_learning_2022}) to a student network. The student network is trained on anomaly-free data to mimic the teacher's output, and during inference, discrepancies between the two are used to calculate anomaly scores.

Reconstruction-based methods aim to achieve anomaly detection at the point level. Autoencoder-based methods, such as EasyNet \cite{chen_easynet_2023}, use a feature encoder and decoder to reconstruct input 3D point cloud data. Trained only on anomaly-free data, they calculate anomaly scores based on the discrepancies between the input and the reconstructed data. For instance, EasyNet employs a multi-scale, multi-modality feature encoder-decoder for 3D depth map reconstruction, enabling real-time detection. However, some autoencoder-based methods like EasyNet and Cheating Depth \cite{zavrtanik_cheating_2024} require structured depth maps as input and struggle with unstructured point clouds. To address this, Li et al. \cite{li_towards_2023} proposed the self-supervised Iterative Mask Reconstruction Network (IMRNet) for unstructured point cloud reconstruction and anomaly detection. Principal component analysis (PCA)-based methods, like those by Von Enzberg and Al-Hamadi \cite{von_enzberg_multiresolution_2016} and Zhang et al. \cite{zhang_automatic_2018}, identify normal patterns through principal components from training data and use them to reconstruct test samples for anomaly detection.
Despite their effectiveness, training-based methods are limited by the requirement for extensive anomaly-free training data. They also face challenges in accurately detecting anomalies with sparse properties or similar patterns to normal data. Feature embedding-based methods are good at localizing anomalies but have difficulty in obtaining accurate anomaly boundaries, while reconstruction-based methods can get more accurate anomaly boundaries but struggle with complex surface reconstruction, leading to higher false positive rates \cite{masuda_toward_2023,zavrtanik_cheating_2024,li_towards_2023,roth_towards_2022}.


Untrained methods, on the other hand, do not require training on a large dataset. 
They rely on prior knowledge to model the normal surface or possible anomalies \cite{jovancevic_3d_2017}. Local geometry-based methods explore local geometric characteristics to detect anomalies. For example, Jovan\v{c}evi'{c} et al. \cite{jovancevic_3d_2017} used a region-growing segmentation algorithm with local normal and curvature data to segment point cloud airplane surfaces. Wei et al. \cite{wei_microhardness_2021} developed local features to compute anomaly scores, and Miao et al. \cite{miao_pipeline_2022} used FPFH and normal vector aggregation for defect detection on gas turbine blades. 
Global geometry-based methods utilize the global shapes of manufacturing parts. Statistical-based methods, such as the one proposed by Tao et al. \cite{tao2023anomaly}, make assumptions about the shape of the product, like low-rankness and smoothness, and formulate the anomaly detection problem within a probabilistic framework. CAD model-based methods compare the point cloud with CAD models through rigid registration. \cite{zhao_defect_2023} used the standard iterative closest point (ICP) algorithm for 3D printing defect detection, and various improvements have been made to the ICP algorithm, such as the octree-based registration algorithm by \cite{he_octree-based_2023} to accelerate the process. Untrained methods can handle one single sample directly, which is an advantage in scenarios with limited data.

In conclusion, unsupervised anomaly detection methods for 3D point cloud data in manufacturing have made significant progress. training-based methods are suitable for scenarios with abundant anomaly-free data, while untrained methods offer a solution for situations where data is scarce. 
\section{3D Point Cloud Anomaly Detection}
In this section, we provide a systematic mathematical formulation of untrained anomaly detection problem based on 3D point cloud data. The problem definition is as follows:

Consider a 3D point cloud sample \( \mathbf{Y} = \{ \mathbf{Y}_i \in \mathbb{R}^3 \mid i = 1, 2, \ldots, N \} \), where \( N \) denotes the total number of points. The Anomaly Detection task aims to partition \( \mathbf{Y} \) into two disjoint subsets \( \mathbf{Y}_0 \) and \( \mathbf{Y}_1 \), satisfying:
\begin{enumerate}[itemsep=1pt]
    \item \( \mathbf{Y}_0 \cup \mathbf{Y}_1 = \mathbf{Y} \)
    \item \( \mathbf{Y}_0 \cap \mathbf{Y}_1 = \emptyset \)
    \item \( \mathbf{Y}_0 \) represents the reference surface points
    \item \( \mathbf{Y}_1 \) represents the anomaly points
\end{enumerate}
Furthermore, we make the following assumptions regarding the distribution of \( \mathbf{Y}_0 \) and \( \mathbf{Y}_1 \):
\begin{enumerate}[itemsep=1pt]
    \item \( \mathbf{Y}_0 \) is sufficiently dense, which is supported by contemporary 3D scanning technologies.
    \item \( \mathbf{Y}_1 \) exhibits sparse distribution characteristics.
\end{enumerate}
The point cloud anomaly detection problem can be formulated using three complementary modeling frameworks: the Classification Framework, the Decomposition Framework, and the Local Geometry Framework.
\subsection{Classification Framework}
From a classification perspective, the problem is formulated as a binary classification task defined by a mapping function \( c: \mathbf{Y} \rightarrow \{0,1\} \), where:
\begin{equation}
    c(\mathbf{Y}_i) = \begin{cases}
        0 & \text{if } \mathbf{Y}_i \in \mathbf{Y}_0 \text{ (reference surface point)} \\
        1 & \text{if } \mathbf{Y}_i \in \mathbf{Y}_1 \text{ (anomaly point)}
    \end{cases}
\end{equation}

The classification problem is solved through a probabilistic approach. The relationships among variables \(\mathcal{C}\), \(\mathbf{Y}\), and the parameter set \(\mathbf{\Theta}\) can be characterized via the joint likelihood function \(p(\mathbf{Y}, \mathcal{C}|\mathbf{\Theta})\). Here, \(\mathbf{\Theta}\) encompasses parameters associated with the reference surface representation.

The optimal classification set \(\mathcal{C}\) is then obtained by maximizing the joint likelihood:
\begin{equation}
    \mathcal{C} = \arg\max\limits_{\mathcal{C}}p(\mathbf{Y}, \mathcal{C}|\mathbf{\Theta})
\end{equation}
The classification framework is suitable for the object surface where point relationship can be modeled. If it is hard to model the statistical relationship between points and point distributions, the following decomposition framework and local-geometry based framework can be considered. 

\subsection{Decomposition Framework}
The data decomposition frameworks assumes the point cloud \(\mathbf{Y}\) can be represented as a superposition of three components:
\begin{equation}
    \mathbf{Y} = \mathbf{X} + \mathbf{A} + \mathbf{E}
\end{equation}
where:
\begin{itemize}[itemsep=1pt]
    \item \(\mathbf{X}\) denotes the reference surface component, representing the underlying regular geometric structure.
    \item \(\mathbf{A}\) represents the anomaly component, capturing structural deviations from the reference surface.
    \item \(\mathbf{E}\) accounts for measurement noise inherent in the data acquisition process.
\end{itemize}
    
As anomaly usually sparsely exists on the surfaces, we propose corresponding solution methods. For the decomposition-based formulation, we develop a sparse learning approach to recover the anomaly component. The details of these solutions are elaborated below:

Given point cloud data with random measurement noise, we propose a decomposition-based solution. The observed point cloud \(\mathbf{Y}\) is decomposed as:
\begin{equation}
    \mathbf{Y} = \mathbf{X} + \mathbf{A} + \mathbf{E}
\end{equation}
where each component represents:
\begin{itemize}[itemsep=1pt]
    \item \(\mathbf{X} = [\mathbf{X}_1,\ldots,\mathbf{X}_N]^T \in \mathbb{R}^{N \times 3}\): the reference surface points.
    \item \(\mathbf{A} = [\mathbf{a}_1,\ldots,\mathbf{a}_N]^T \in \mathbb{R}^{N \times 3}\): the anomaly component with row-sparsity.
    \item \(\mathbf{E} = [\mathbf{e}_1,\ldots,\mathbf{e}_N]^T \in \mathbb{R}^{N \times 3}\): the measurement noise.
\end{itemize}
To recover these components, we formulate a sparse learning optimization problem:
\begin{equation}
    \min_{\mathbf{A}} J(\mathbf{A}) = L(\mathbf{A}; \mathbf{X}, \mathbf{\Theta}) + \lambda p_s(\mathbf{A})
    \label{eq:optimization}
\end{equation}
where:
\begin{itemize}[itemsep=1pt]
    \item \(L(\mathbf{A}; \mathbf{X}, \mathbf{\Theta})\) is a loss function that quantifies the fitting residuals between \(\mathbf{Y}\) and \(\mathbf{X} + \mathbf{A}\), as well as enforces smoothness constraints on the reference surface parameters \(\mathbf{\Theta}\).
    \item \(p_s(\mathbf{A})\) is a penalty term promoting row-sparsity in \(\mathbf{A}\), and non-zero rows indicate the anomaly locations.
    \item \(\lambda\) is a tuning parameter controlling the sparsity level.
    \item \(\mathbf{\Theta}\) represents the estimated parameters of the reference surface.
\end{itemize}

Decomposition framework is useful for the surfaces where anomaly points can be separated from the normal reference surfaces. In addition, decomposition framework models the object surface as a whole, thereby enabling the accurate anomaly detection.  

% \section{Proposed Solution Methods}

\subsection{Local Geometry Framework}
A classical perspective suggests that anomaly detection can be based on neighborhood information. For any vertex, we can select its k-neighborhood and compute its Point Feature Histograms (PFH). These histograms possess invariance to rigid body transformations. After obtaining the Point Feature Histograms, we can employ machine learning methods to train a classifier that distinguishes between normal and anomalous points, thereby completing the task. 

There are various local-geometry based feature learning methods\cite{Du202X}, and key ideas are utilizing local shape descriptors to extract the geometry feature from the neighbourhood. More details can be refereed to  Du et al. 2025. Due to the local modeling of the object surfaces, local-geometry based methods usually cannot detection anomaly boundary very accurately, so the authors suggest to prioritize the first two frameworks. However, when the assumptions of classification framework and decomposition framework fail, local-geometry based framework  can serve as an alternatve, which is more general in real applications.


\section{Examples}
\subsection{Example of Classification Framework}

The Bayesian network structure presented in \cite{tao2023anomaly} establishes a probabilistic framework for anomaly detection in unstructured 3D point cloud data. This review examines its key components and theoretical foundations.

\textbf{Core Components and Relationships}

The network models three fundamental relationships:

1) Point location ($x_i$) depends on point type ($c_i$):
\begin{equation}
p(x_i | C, X, D) = p(x_i | c_i)
\end{equation}

2) Local smoothness ($d_{ij}$) depends on adjacent point types:
\begin{equation}
d_{ij} = \|K_i-K_j\|_F^2
\end{equation}
where $K_i$ represents the structure-aware tensor of point $i$.

3) Point type inference relies on spatial location and neighborhood smoothness:
\begin{equation}
p(c_i | C, X, D) = p(c_i | x_i, \{d_{ij}, c_j, j \in \mathcal{N}_i\})
\end{equation}

\textbf{Joint Distribution}

The network enables factorization of the joint distribution:
\begin{equation}
p(X, D, C) = \prod_i p(x_i | c_i)p(c_i) \prod_{j \in \mathcal{N}_i} p(d_{ij} | c_i, c_j)
\end{equation}
The structure's validity is established through the Markov property, ensuring that each node's distribution depends only on its Markov blanket. This property validates the conditional independence assumptions and enables efficient probabilistic inference.

\subsection{Example of Decomposition Framework }
Recent studies have proposed various methods for anomaly detection based on decomposition. \cite{tao2025pointsgrade} considered a loss function defined as:
\begin{equation}
\min_{\mathbf{A}}J(\mathbf{A})=L(\mathbf{A};\mathbf{H},\mathbf{Y})+\lambda p_{s}(\mathbf{A})
\end{equation}
In this formulation, the loss function $L(\mathbf{A};\mathbf{H},\mathbf{Y})$ is formulated to quantify the fitting residuals between the observed point cloud $\mathbf{Y}$ and the sum of the reference point cloud $\mathbf{X}$ and the anomalous component $\mathbf{A}$. This quantification effectively measures how closely the combination of $\mathbf{X}$ and $\mathbf{A}$ can approximate $\mathbf{Y}$. Moreover, it enforces the smoothness of $\mathbf{X}$ through the graph-based smoothness metric $\mathbf{H}$.

The matrix $\mathbf{H}$ is derived from Graph Signal Processing (GSP) theories. By representing the reference point cloud $\mathbf{X}$ as a graph, $\mathbf{H}$ is constructed in such a way that $\mathbf{HX}$ can capture the smoothness of $\mathbf{X}$. Geometrically, the $i^{\text{th}}$ row of $\mathbf{HX}$, denoted as $(\mathbf{HX})_i$, can be interpreted as the difference between the $i^{\text{th}}$ point and the convex combination of its neighbors. This interpretation approximates the deviation of the $i^{\text{th}}$ point from the local plane. For a smooth $\mathbf{X}$, the value of $\mathbf{HX}$ is approximately zero. This property is integrated into the loss function to guarantee the smoothness of the reference surface.

The penalty term $p_{s}(\mathbf{A})$ is introduced with the aim of promoting the row-sparsity of $\mathbf{A}$. Given that anomalies are typically sparse, a row-sparse $A$ implies that only a small number of points are identified as anomalies. The authors selected the group LOG penalty, defined as $p_{s}(\mathbf{A})=\sum_{i}\log(\sqrt{\left\|a_{i}\right\|_{2}^{2}+\varepsilon}+\left\|a_{i}\right\|_{2})$, for this purpose, where the group LOG penalty can penalize different predictors more evenly. This characteristic enables a better recovery of the sparse anomalous component $\mathbf{A}$. 


\subsection{Example of Local-Geometry based Framework}
This example is sourced from the work presented in \cite{rusu2009fast}. The computational approach it employs unfolds as follows:

\textbf{Neighborhood Selection}
Given a point $p$ in a 3D point cloud, we first select all its neighbors enclosed within a sphere of radius $r$. Let $k$ denote the number of these neighbors, and we refer to this set as the $k$-neighborhood of $p$.

\textbf{Darboux Frame Construction and Feature Calculation}
For each pair of points $p_i$ and $p_j$ ($i \neq j$) in the $k$-neighborhood of $p$ with their estimated normals $n_i$ and $n_j$ (where $p_i$ is the point with a smaller angle between its associated normal and the line connecting the points), we define a Darboux $uvn$ frame:
\begin{align}
\mathbf{u} &= \mathbf{n}_i \\
\mathbf{v} &= (\mathbf{p}_j - \mathbf{p}_i) \times \mathbf{u} \\
\mathbf{w} &= \mathbf{u} \times \mathbf{v}\\
\alpha &= \mathbf{v} \cdot \mathbf{n}_j \\
\phi &= \frac{\mathbf{u} \cdot (\mathbf{p}_j - \mathbf{p}_i)}{\|\mathbf{p}_j - \mathbf{p}_i\|} \\
\theta &= \arctan(\mathbf{w} \cdot \mathbf{n}_j, \mathbf{u} \cdot \mathbf{n}_j)
\end{align}
Previous research sometimes included a fourth feature—the Euclidean distance from $p_i$ to $p_j$. However, recent experiments have demonstrated that its omission does not significantly impact robustness in certain cases, particularly in 2.5D datasets.

\textbf{Histogram Construction}
After computing these features for all point pairs in the $k$-neighborhood of $p$, we quantize these feature values and bin them into a histogram. The resulting histogram constitutes the Point Feature Histogram (PFH) at point $p$, characterizing the local geometric properties in its vicinity.

\textbf{Simplified Point Feature Histogram (SPFH) Calculation}
For each query point $p$, we initially compute only the relationships between itself and its neighbors, termed the Simplified Point Feature Histogram (SPFH).

\textbf{FPFH Computation from SPFH}
Subsequently, for each point $p$, we re-determine its $k$ neighbors. The Fast Point Feature Histogram (FPFH) at point $p$ is computed as:

\begin{equation}
FPFH(p) = SPF(p) + \frac{1}{k}\sum_{i=1}^{k}\frac{1}{\omega_k} \cdot SPF(p_k)
\end{equation}

\noindent
where $\omega_k$ represents the distance between query point $p$ and a neighbor point $p_k$ in a given metric space. This formulation combines the SPFH of point $p$ with a weighted sum of its neighbors' SPFHs, achieving reduced computational complexity compared to PFH while maintaining most of its discriminative power.

\section{Conclusion}
\label{sec:conclusion}

This paper systematically investigates untrained machine learning methods for anomaly detection in 3D point cloud data, addressing critical challenges in scenarios with limited training samples. The significance of untrained methods lies in their ability to operate without pre-trained models or extensive anomaly-free datasets, making them particularly valuable for emerging applications such as personalized manufacturing and personalized medicine where data scarcity is a fundamental constraint. Unlike traditional training-based unsupervised anomaly detection approaches that require substantial computational resources and labeled data, untrained methods achieve remarkable adaptability through the integration of geometric priors and sparse learning principles.

We formally defined the untrained anomaly detection problem through  mathematical formulations, establishing three complementary frameworks to address different application scenarios. The \textit{Classification Framework} provides a probabilistic approach to distinguish anomalies from normal surfaces by maximizing joint likelihood functions. The \textit{Decomposition Framework} separates point clouds into reference surface, anomaly, and noise components through sparse optimization, enabling precise anomaly localization. The \textit{Local Geometry Framework} leverages neighborhood geometric features for anomaly identification when prior knowledge about surface structure is insufficient. These frameworks collectively address key challenges including high-dimensional data representation, single-sample learning, and sparse anomaly detection.

Three concrete implementations demonstrate the practical effectiveness of these frameworks: 1) The decomposition-based approach by \cite{tao2025pointsgrade} achieves sparse anomaly recovery through graph-based smoothness constraints and group LOG penalties; 2) The Bayesian classification framework in \cite{tao2023anomaly} establishes probabilistic relationships between point locations and neighborhood smoothness; 3) The Fast Point Feature Histogram method \cite{rusu2009fast} exemplifies local geometry analysis through Darboux frame construction and feature histogram computation. These examples validate the frameworks' capabilities in handling diverse anomaly types while maintaining computational efficiency.

The positions presented in this paper advance the understanding of untrained anomaly detection mechanisms. By eliminating dependency on training data and predefined feature extractors, the proposed methodologies provide a foundation for developing robust anomaly detection systems in data-scarce environments. Future research should focus on enhancing computational efficiency for real-time applications, improving detection accuracy for sub-surface anomalies, and extending these frameworks to multi-modal data integration scenarios.


\section{Outlook}
\label{sec:outlook}

Future research in untrained anomaly detection should focus on integrating prior knowledge of defect characteristics into detection frameworks. A promising direction lies in treating anomaly detection as an inverse problem, where the goal shifts from merely identifying anomalies to jointly reconstructing normal surfaces and inferring potential defect distributions. If prior information about defect patterns (e.g., typical shapes, spatial frequencies, or material-specific manifestations) can be systematically incorporated, such inverse problem formulations could enable more precise detection while reducing false positives.

Three key directions emerge for advancing this paradigm:
\begin{enumerate}[itemsep=1pt]
    \item Developing defect pattern libraries through physical modeling and industrial data collaboration.
    \item Creating adaptive algorithms that balance prior knowledge and data-driven observations.
    \item Exploring hybrid approaches that combine inverse problem methodologies with emerging deep learning architectures.
\end{enumerate}
This perspective aligns with industrial needs for explainable detection systems and lays the foundation for next-generation quality control solutions. While challenges remain in computational efficiency and theoretical guarantees, the integration of physical knowledge with machine learning principles offers a viable path toward more robust and generalizable anomaly detection systems.







% In the unusual situation where you want a paper to appear in the
% references without citing it in the main text, use \nocite
\nocite{langley00}

\bibliography{example_paper}
\bibliographystyle{icml2025}



%%%%%%%%%%%%%%%%%%%%%%%%%%%%%%%%%%%%%%%%%%%%%%%%%%%%%%%%%%%%%%%%%%%%%%%%%%%%%%%
%%%%%%%%%%%%%%%%%%%%%%%%%%%%%%%%%%%%%%%%%%%%%%%%%%%%%%%%%%%%%%%%%%%%%%%%%%%%%%%
% APPENDIX
%%%%%%%%%%%%%%%%%%%%%%%%%%%%%%%%%%%%%%%%%%%%%%%%%%%%%%%%%%%%%%%%%%%%%%%%%%%%%%%
%%%%%%%%%%%%%%%%%%%%%%%%%%%%%%%%%%%%%%%%%%%%%%%%%%%%%%%%%%%%%%%%%%%%%%%%%%%%%%%
\newpage
\appendix
\onecolumn

%%%%%%%%%%%%%%%%%%%%%%%%%%%%%%%%%%%%%%%%%%%%%%%%%%%%%%%%%%%%%%%%%%%%%%%%%%%%%%%
%%%%%%%%%%%%%%%%%%%%%%%%%%%%%%%%%%%%%%%%%%%%%%%%%%%%%%%%%%%%%%%%%%%%%%%%%%%%%%%


\end{document}


% This document was modified from the file originally made available by
% Pat Langley and Andrea Danyluk for ICML-2K. This version was created
% by Iain Murray in 2018, and modified by Alexandre Bouchard in
% 2019 and 2021 and by Csaba Szepesvari, Gang Niu and Sivan Sabato in 2022.
% Modified again in 2023 and 2024 by Sivan Sabato and Jonathan Scarlett.
% Previous contributors include Dan Roy, Lise Getoor and Tobias
% Scheffer, which was slightly modified from the 2010 version by
% Thorsten Joachims & Johannes Fuernkranz, slightly modified from the
% 2009 version by Kiri Wagstaff and Sam Roweis's 2008 version, which is
% slightly modified from Prasad Tadepalli's 2007 version which is a
% lightly changed version of the previous year's version by Andrew
% Moore, which was in turn edited from those of Kristian Kersting and
% Codrina Lauth. Alex Smola contributed to the algorithmic style files.
