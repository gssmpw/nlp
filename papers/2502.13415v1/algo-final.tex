\section{Indifferentially Private Matching Protocol}
\label{sec:algo}

We describe the components of our private matching protocol.

\noindent \textbf{Bid Type.}
For a user~$i$ with $\mathsf{Id}_i$, its bid has a type~$\tau_i \in \{\mathsf{Buy}, \mathsf{Sell}\}$ with price $p_i$
and quantity $x_i \in \Z_{\geq 0}$.



\noindent \textbf{Bid Submission.}
Given the bid $(\mathsf{Id}_i, \tau_i, p_i, x_i)$
from user~$i$, 
suppose a user wants to achieve $(\epsilon, \delta)$-IDP.
Then,
it performs the following during bid submission.
\begin{enumerate}

\item Sample non-negative noise $N_i \in \Z$ 
according to the truncated geometric distribution $\Geom^Z(e^\epsilon)$,
for some even $Z \geq \ceil{\frac{2}{\epsilon} \ln \frac{1}{\delta}}$.


\item Create $y_i := x_i + N_i$ nodes of type $(\tau_i, p_i)$,
where the type is known to the server.

Out of the $y_i$ nodes, $x_i$ are real nodes and $N_i$ are fake nodes.
Each node contains encrypted information of whether it is real or fake.

The list of created nodes are sent to the server, where all real nodes
appear before the fake nodes.  
Observe that the server cannot distinguish between
the real and the fake nodes.
\end{enumerate}


\noindent \textbf{Cryptographic Assumptions.}  Depending on what security guarantee of the final protocol is needed, 
we can assume different properties of the ciphertext.

\begin{itemize}

\item To achieve just our notion of indifferential privacy,
the encrypted information of a node can be achieved by 
a non-malleable commitment scheme~\cite{DBLP:journals/jacm/LinP15}.
In practice, we can also use a hash function such as AES.

\textbf{Ideal Commitment Scheme.}
For ease of exposition, we describe our protocol using
an ideal commitment scheme.  A user creates a commitment of a message,
which is an opaque object whose contents the adversary cannot observe.
At a later time, the user can choose to open the commitment, which
must return the original message. In particular, a cryptographic commitment scheme has these key properties: (1) Hiding: The value stays secret until revealed.
(2) Binding: The committed value can't be changed.
(3) Unforgeability: No one can forge or tamper with the commitment.
(4) Correctness: The committed value is always correctly revealed.




\item Assuming the shuffler model, the server does not know
which nodes come from which user.  Hence, 
in this case, each node also contains encrypted information about the user ID.
However, the server still knows which nodes originate from the same user,
and the nodes from the same user are sorted in a list such that real nodes appear 
first before the fake nodes.
\end{itemize}



\noindent \textbf{Node Popularity.}
In any case, the server can create an ordering on the nodes on the buyer and the seller sides
according to the price.
For a buy node, a higher price is more popular; for a sell node, a lower price is more popular. 
On each side, the most or the least popular price is known as an \emph{extreme} price.




\noindent \textbf{Matching Graph.}
The algorithm maintains a bipartite graph
 $G = (\mcal{B}, \mcal{S}; E)$,
where there is an edge between a buy and a sell node
if the prices are compatible, i.e., the buying price
is at least the selling price.
%For succinct notation, we use $\iota$ to denote one side
%and $\overline{\iota}$ to denote the other side.
As some nodes are matched during the process,
we assume that the algorithm will immediately remove any isolated node
that no longer has a neighbor.


\noindent \textbf{Maximum Matching Problem.}
Given a bipartite graph $G = (\mcal{B}, \mcal{S}; E)$,
a matching $M \subseteq E$ is a subset of edges
such that no two edges in $M$ are incident on the same node.
The \emph{maximum matching problem} aims to find a matching
$M$ with the maximum number of edges.  We next describe a
strategy to find a maximum matching on a bipartite graph
induced by buyer and seller nodes.



\noindent \textbf{Polar Opposite.}
Suppose the current matching graph $G$ has no isolated nodes.
For an arbitrary side, %~$\iota$, 
observe that any node on this side
with the most popular price has an edge connected to
any node on the other side %~$\overline{\iota}$ 
with the least popular price.
We say that two such nodes from the two sides are \emph{polar opposite} of each other.
The following fact shows that an optimal matching can be returned by iteratively matching
polar opposites.

\begin{fact}
\label{fact:basic_matching}
Suppose a matching graph $G$ has no fake or isolated nodes,
and $u$ and $v$ are any two nodes from the different sides that
are polar opposite of each other.
Then, there exists a maximum matching in $G$ in which $u$ and $v$ are matched.
\end{fact}

\begin{proof}
Without loss of generality, suppose $u$ has the most popular price
on its side, while $v$ has the least popular price on its side.
Suppose $M$ is a maximum matching in which $u$ and $v$ are not matched
to each other.  We will modify $M$ without decreasing the matching size such that $u$ and $v$
are matched to each other in the following steps.

\begin{enumerate}

\item If $u$ is not matched to any node, we can make $u$ replace
any other node on its side, because it has the most popular price.

Hence, we may assume that $u$ is matched in $M$.

\item If $v$ is not matched in $M$, then we can replace the partner of~$u$
with $v$ to make $u$ and $v$ matched to each other.

\item Otherwise, we have $u$ is matched to some $v'$ and $v$ is matched to some $u'$.
Observe that if $u'$ is compatible with the least popular price on the other side,
then it must be compatible with $v'$.  Hence, $u$ and $u'$ can swap partners
such that $u$ and $v$ become matched.
\end{enumerate}
\end{proof}



%Moreover, it will return
%a maximum matching that gives priority to the
%most popular nodes on side~$\iota$
%and the least popular nodes on side~$\overline{\iota}$.


\noindent \textbf{Matching Procedure with Fake Nodes.}
With our assumption, for each node type,
we assume that the nodes are sorted such that
the nodes from the same user are together and its real nodes appear first.

In each step, the server picks nodes $u$ and $v$ from the two sides
that are polar opposite of each other.
The owners of nodes $u$ and $v$ will participate in the following protocol:

\begin{enumerate}

\item If the node of an owner is real,
the owner will open its commitment to reveal the node is real.

\item If the node of an owner is fake,
this implies that all real nodes of that owner are already
matched and the owner will reveal all its fake nodes.




\item If both the nodes $u$ and $v$ are opened to be real,
the nodes $u$ and $v$ are mathced and removed from the matching graph.

\item If one of the nodes is real and the other node is fake,
the real node will be considered in the next iteration if it still has
a polar opposite in the remaining graph.

\end{enumerate}

\noindent \textbf{Correctness Intuition.}  The protocol
is outlined in Algorithm~\ref{alg:matching};
note that any arbitrary unspecified choice made by the algorithm
can either be randomized or deterministic according to some additional rules.

\begin{lemma}
Algorithm~\ref{alg:matching} returns
a maximum matching between real buy and sell nodes in the matching graph.
\end{lemma}

\begin{proof}
Comparing with the case with no fake nodes,
observe that whenever a fake node is encountered,
it will be removed immediately.  Hence, the matching behavior
is exactly the same as the case with no fake nodes.
From Fact~\ref{fact:basic_matching},
a maximum matching is returned.
\end{proof}



\begin{algorithm}[H]
\caption{Matching Protocol with Fake Nodes} \label{alg:matching}
\SetAlgoLined

Matching graph $G \gets (\mcal{B}, \mcal{S}; E)$

$u \gets \msf{NULL}$

$M \gets \emptyset$

\While{$E \neq \emptyset$}{

    \If{$u = \msf{NULL}$}{
			From any side and any extreme price on that side, select the next node~$u$.
		}
		
		Select the next node~$v$ that is a polar opposite to~$u$.
		
		
		The algorithm announces that an attempt is made to match $u$ and $v$.
		
		The owners of the two nodes open their commitments (if they have not already done so) and reveal
		whether the nodes are real or fake.
		
		
		The algorithm infers that if an owner reveals a fake node,
		all the remaining nodes by the same owner are also fake.
				

    \eIf{either $u$ or $v$ is fake}{
        			
        Any fake node and its incident edges are removed from $G$.
    }{
        \If{both $u$ and $v$ are real}{
            					
            The pair $(u,v)$ is matched and added to $M$;
						remove $u$ and $v$ from $G$.
						
        }
    }
    Remove any isolated node from $G$.
		
		If exactly one of $u$ and $v$ is still unmatched and remains
		in the graph, set $u$ to be that node; otherwise,
		set $u$ to $\msf{NULL}$.
				
}

	\textbf{return} matching $M$
\end{algorithm}



\subsection{Privacy Analysis}

We show that our dark pool auction protocol satisfies IDP.


\ignore{

We first recall a result the generalizes Fact~\ref{fact:geom_dp}.
\begin{fact}
Suppose $P$ and $Q$ are two distributions on a finite sample space~$\Omega$
such that the following conditions hold.

\begin{compactitem}

\item  Denote $S_P := \{\omega \in \Omega: P(\omega) > 0 \wedge Q(\omega) = 0\}$
and $S_Q := \{\omega \in \Omega: P(\omega) = 0 \wedge Q(\omega) > 0\}$.
We have $\max \{P(S_P), Q(S_Q)\} \leq \delta$.

\item For any $\omega \in \Omega$ such that $P(\omega) \cdot Q(\omega) > 0$,
we have
$e^{-\epsilon} \leq \frac{P(\omega)}{Q(\omega)} \leq e^{\epsilon}.$

\end{compactitem}


Then, we have $\HS_{e^\epsilon}(P \| Q) \leq \delta$.

\end{fact}

}
%We outline each step in the privacy analysis.

\noindent \textbf{Neighboring Input Configurations.}
Recall that we consider neighboring input configurations
$C_0$ and $C_1$ in which exactly one user~$i$ has 
different bidding quantities.
Suppose in $C_0$, the quantity is $n_0$ and
in $C_1$, the quantity is $n_1 = n_0 + 1$.


\noindent \textbf{Adversarial View.} We decompose
the view space into $\Gamma \times \Lambda$,
where $\Gamma$ corresponds to 
the number of nodes created by user~$i$,
and $\Lambda$ includes (i) the bid submissions by all other users,
and (ii) any random bits
used to make choices in the matching procedure in Algorithm~\ref{alg:matching}.
The distributions on $\Gamma$ differ in $C_0$ and $C_1$,
but in the two scenarios, the distributions on $\Lambda$ are identical
and independent of the $\Gamma$ component.

Note that any other information observed by the adversary can be recovered
by a point in $\Gamma \times \Lambda$.



\noindent \textbf{Indifference Relation.}  
We define an almost trivial relation on $\Gamma \times \Lambda$, namely,
$(\gamma_0, \lambda_0) \sim (\gamma_1, \lambda_1)$ \emph{iff}
$\gamma_0 = \gamma_1$ and $\lambda_0 = \lambda_1$.
This means that the same number of
nodes are submitted by user~$i$ in the two scenarios;
moreover, all other users submit exactly the same bids in the two scenarios,
and the also the same random bits are used in the matching process.

However, this does not mean that the views of the adversary are the same in
both scenarios, because the number of real nodes for user~$i$ are different
in $C_0$ and $C_1$.  Nevertheless, if the views are different,
it must be because $n_0$ real nodes of user~$i$ have been matched
in both scenarios, and the algorithm tries to match the next node from user~$i$.
This is consistent with the indifference relation
described in Definition~\ref{defn:indifference}.

\begin{lemma}
Using an ideal commitment scheme,
the dark pool auction protocol is $(\epsilon, \delta)$-IDP.
\end{lemma}

\begin{proof}
Recall that from Definition~\ref{defn:idp},
our goal is to prove an upper bound for:
$\HS^{\mcal{F}}_{\epsilon}(\msf{View}(C_0) \| \msf{View}(C_1))$.

Because of the ideal commitment scheme,
during the bid submission, the server can only see
how many nodes are created by each user.
From user~$i$, the information collected at this stage 
is a point in $\Gamma$.
Suppose that in $C_0$, the distribution of the number of created nodes by user~$i$ is $P_0$;
and in $C_1$, the corresponding distribution is $P_1$.
Note that $P_0$ and $P_1$ are both distributions on $\Gamma$.
From Fact~\ref{fact:geom_dp},
we have: 
$\HS_{e^\epsilon}(P_0\| P_1) \leq \delta$.


According to the above discussion,
for both $C_0$ and $C_1$,
we have the same distribution $Q$ on $\Lambda$
that represents other users' bids and the random bits used by the server during
the matching process.

As argued above, a point in $\Gamma \times \Lambda$
is sufficient to recover the view of the adversary.
Hence,
we have:

$\HS^{\mcal{F}}_{e^\epsilon}(\msf{View}(C_0) \| \msf{View}(C_1))
= \HS_{e^\epsilon}((P_0, Q) \| (P_1, Q))$.

Moreover, observe that if $(\gamma_0, \lambda_0) = (\gamma_1, \lambda_1)$
correspond to two points in $\Gamma \times \Lambda$
under configurations $C_0$ and $C_1$, respectively,
it must be the case that during bid submission,
exactly $\gamma_0 = \gamma_1$ nodes are created by user~$i$.
Furthermore, everything is identical in both scenarios,
except that the $(n_0 + 1)$-st node in user~$i$'s list
is fake in $C_0$ and is real in $C_1$.
Therefore, the adversary either has identical views
in both scenarios (because at most
$n_0$ nodes in user~$i$'s list have been attempted to be matched), or else it must be the case that 
all real nodes of user~$i$ in $C_0$ have been fully matched;
this is consistent with the indifference relation~$\mcal{F}$.


However, note that $Q$ is independent of the $\Gamma$ component.
In general, the hockey-stick divergence (as well as other commonly used divergences)
satisfies the property that observing the same extra independent randomness
should not change the value of the divergence.
Hence, 
we have:
$\HS_{e^\epsilon}((P_0, Q) \| (P_1, Q)) = \HS_{e^\epsilon}(P_0\| P_1)$,
which we have already shown is at most~$\delta$.  This completes the proof.
\end{proof}
\ignore{
Using standard hybrid arguments~\cite{DBLP:conf/crypto/MironovPRV09}, we get the following corollary immediately.
}

\begin{corollary}
Replacing the ideal commitment scheme with a real-world commitment
scheme (such as~\cite{DBLP:journals/jacm/LinP15}),
the dark pool auction protocol is $(\epsilon, \delta)$-computational IDP.
\end{corollary}

\section{Implementation and Evaluation}


\subsection{Implementation}
We provide a comprehensive end-to-end implementation of our indifferentially private dark pool auction. We have also implemented a non-private auction to compare the total computation and communication time. Our code is available at \url{https://github.com/adyaagrawal/idp-darkpool}. 

Both the non-private and Indifferentially Private (IDP) protocols were incorporated into ABIDES~\cite{ByrdHB20},\footnote{ABIDES has also been used in simulating privacy preserving federated learning protocols such as the most recent work of~\cite{MaWAPR23}.} an open-source high-fidelity simulator tailored for AI research in financial markets such as stock exchanges. ABIDES is ideal for this purpose, as it supports simulations with tens of thousands of clients interacting with a central server for transaction processing, along with customizable pairwise network latencies to simulate real-world communication delays. We run the simulations on a personal x64-based Windows PC equipped with a single Intel Core i5-10210U processor running at 1.6 GHz and 16 GB of DDR4 memory. 

ABIDES employs the cubic network delay model, where the latency is determined by a base delay (within a specified range) and a jitter that influences the percentage of messages arriving within a designated timeframe, thereby shaping the tail of the delay distribution. Our experiments were conducted using three distinct network settings: local (client machines in NYC), global (client machines from NYC to Sydney), and very wide network (client machines across the world). For each setting, we configured the base delay according to Table~\ref{tab:network_delays} while employing the ABIDES's default parameters for jitter.

\begin{table}[h]
    \centering
    \begin{tabular}{|c|c|}
        \hline
        \textbf{Network Setting} & \textbf{Base Delay (ms)} \\
        \hline
        Local (Within NYC)              & 0.021 - 0.1                           \\
        Global (NYC to Sydney)   & 21 - 53    \\
        Very Wide Network (Across the World)        & 10 - 100                                \\
        \hline
    \end{tabular}
    \caption{Base delay for different network configurations.}
    \label{tab:network_delays}
\end{table}

In our non-private implementation, clients submit their orders without concealing their identities. These orders are organized into two doubly linked lists: one for buy orders and another for sell orders. Our implementation follows the maximum matching algorithm explained in Section~\ref{sec:algo}. Once matches are identified, they are sent back to the clients for execution.


For our IDP implementation, clients generate an array of sorted orders, with a few randomly numbered fake ones to ensure indifferential privacy.
They commit to their identity and whether the order is real or fake using the Cryptodome library in Python based on AES. The maximum matching algorithm is employed and if the match contains the order that the individual client has placed, the client opens the commitment to the server to reveal whether the order is real or fake. Then, the server requests the client to reveal the identity or tries to find a new match. Once both the parties have opened their identity and revealed it to the server, the orders are executed and the next round of matching starts. 


\begin{figure*}[t]
    \centering
    % Use a smaller width to fit within the column constraints
    \begin{subfigure}[t]{0.3\textwidth}  
        \centering
        \includegraphics[width=\textwidth]{./local.pdf}  % Path to your figure file
        \caption{\textnormal{Local setting.}}
        \label{fig:local}
    \end{subfigure}
    \hspace{1em} % Horizontal space between subfigures
    \begin{subfigure}[t]{0.3\textwidth}  
        \centering
        \includegraphics[width=\textwidth]{./global.pdf}  % Path to your figure file
        \caption{\textnormal{Global setting.}}
        \label{fig:global}
    \end{subfigure}
    \hspace{1em} % Horizontal space between subfigures
    \begin{subfigure}[t]{0.3\textwidth}  
        \centering
        \includegraphics[width=\textwidth]{./world.pdf}  % Path to your figure file
        \caption{\textnormal{Very Wide Network setting.}}
        \label{fig:world}
    \end{subfigure}
    \caption{\textbf{Comparison of Time Taken Across Different Settings.}}
    \label{fig:comparison}
\end{figure*}


\subsection{Experimentation}

\noindent \textbf{Performance comparison with non-private implementation.}
To evaluate the performance of our order matching system, we compared the total time required to match orders using both the non-private and Indifferentially Private (IDP) implementations in all three different network settings. We conducted experiments with a range of parties each generating \(2^3\) orders, starting from 4 parties (\(2^5\) orders) and extending up to 1024 parties (\(2^{13}\) orders). 


The data generation process is designed to ensure that each set of randomly generated orders consistently achieves a matching rate of 70\% to 80\%.\footnote{When the \% of matches is lower, the total running time naturally decreases, as fewer matches result in less overhead.} For instance, each client generates 8 sorted orders, with 5 to 7 being real and 1 to 3 being fake. The prices for buy orders are randomly selected between 99 and 101, while sell orders are priced between 98 and 100.







\ignore{
\begin{figure}[htbp]
    \centering
    % Use a larger width to avoid overfull hbox issues
    \begin{subfigure}[t]{0.5\textwidth}  
        \centering
        \includegraphics[width=\textwidth]{./local.pdf}  % Path to your figure file
        \caption{\textnormal{Local setting.}}
        \label{fig:local}
    \end{subfigure}
    \vspace{1em} % Vertical space between subfigures
    \begin{subfigure}[t]{0.5\textwidth}  
        \centering
        \includegraphics[width=\textwidth]{./global.pdf}  % Path to your figure file
        \caption{\textnormal{Global setting.}}
        \label{fig:global}
    \end{subfigure}
    \vspace{1em} % Vertical space between subfigures
    \begin{subfigure}[t]{0.5\textwidth}  
        \centering
        \includegraphics[width=\textwidth]{./world.pdf}  % Path to your figure file
        \caption{\textnormal{Very Wide Network setting.}}
        \label{fig:world}
    \end{subfigure}
    \caption{\textbf{Comparison of Time Taken Across Different Settings.}}
    \label{fig:comparison}
\end{figure}

}



From Figure~\ref{fig:comparison}, we can see that the total time taken using indifferential privacy auction does not deviate greatly from the non-private real world implementation. In Table~\ref{tab:comparison} we also mention the orders per second for two different scenarios. 


\noindent \textbf{Performance comparison with FHE implementation.}
Considering throughput as the number of orders per second, FHE-based solutions can only process around 0.07 orders per second, as shown in Table 4 of~\cite{MazloomDPB23} for 40 orders, using threshold fully homomorphic encryption (tFHE) implemented via the GPU FHE library of~\cite{BalchDP20}. Moreover, the actual runtime is even longer, as Table 4 does not account for the additional overhead from threshold decryption and key management, which increases with the number of clients (as highlighted in Table 2 of~\cite{AsharovBPV20}).

For context, according to~\cite{AsharovBPV20}, running a comparison on encrypted data for a large amount of orders, $2^{13}$ orders, takes $7.6$ seconds on CPU, while our system processes and matches all orders in $13.32$ seconds in total (see Table~\ref{tab:comparison}), not just performing a comparison on a single order. The FHE-based solution requires at least one comparison operation per matching attempt, adding to its computational burden. Furthermore, the FHE-based timings from~\cite{MazloomDPB23} are measured on GPUs; actual performance on CPUs would be even slower, whereas our experiments are conducted entirely on CPUs.

This illustrates that FHE solutions are an overkill, even with the benefit of GPU acceleration, while our approach offers a practical solution. Last but not least, as shown in Table 4 of~\cite{MazloomDPB23}, the latest multi-operator solution, based on secure multiparty computation, is capable of processing only 26 orders per second, even on high-performance hardware. 

\begin{table}[hbt!]
    \centering
    \begin{tabular}{|c|c|c|}
        \hline
        \textbf{Orders} & \textbf{Non-Private Time (secs)} & \textbf{IDP Time (secs)} \\
        \hline
        40             & 0.019698                        & 0.046887                 \\
        \(2^{13}\)     & 3.887790                        & 13.328184                \\
        \hline
    \end{tabular}
    \caption{Comparison of Non-Private and IDP total running times for a small (40) and large size ($2^{13}$) of orders. The throughput (orders per second) for the non-private case is $2031$ and $2107$ and for the indifferential private case is $853$ and $615$ for 40 and $2^{13}$ orders, respectively.}
    \label{tab:comparison}
\end{table}

\noindent \textbf{Scalability with larger datasets.}
In the  experiments above, we compared the run-times of our protocol with a real-world auction protocol for up to 1,024 clients, each submitting 8 orders (resulting in a total of $2^{13}$ orders). The parameters used in our benchmarks reflect realistic scenarios based on data from U.S. dark pools. We have also extended our experiments to include scenarios where both the number of orders per client and the total number of clients double incrementally, scaling up to a total of 262K orders. From Figure~\ref{fig:scalable}, we can see that the differentially private protocol scales linearly with the increased workload. 

\begin{figure}[H]
    \centering
    % Use a larger width to avoid overfull hbox issues
    \begin{subfigure}[t]{0.45\textwidth}  
        \centering
        \includegraphics[width=\textwidth]{./client.pdf}  % Path to your figure file
        \caption{\textnormal{Runtimes in seconds when number of clients increases and the orders per client is fixed to 8}}
        \label{fig:client}
    \end{subfigure}
    \hspace{1em} % Vertical space between subfigures
    \begin{subfigure}[t]{0.45\textwidth}  
        \centering
        \includegraphics[width=\textwidth]{./order.pdf}  % Path to your figure file
        \caption{\textnormal{Runtimes in seconds when the number of orders per client increases and the total number of clients is fixed to 1024}}
        \label{fig:order}
    \end{subfigure}
    \vspace{1em} % Vertical space between subfigures
    \caption{\textbf{Extended runtime tables for larger number of clients and orders}}
    \label{fig:scalable}
\end{figure}


\section{Conclusion}
In this work, we addressed the limitations of existing privacy-preserving auctions in high-frequency trading, such as dark pools, where auctioneers can be untrustworthy. Previous methods, like fully homomorphic encryption, were impractical due to their overhead. Our approach, based on the new notion of Indifferential Privacy, provides an efficient, privacy-preserving continuous double auction that enables maximum matching while minimizing risks. This makes our system a practical and secure alternative addressing both performance and security concerns in modern trading.



