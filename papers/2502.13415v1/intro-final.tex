\section{Introduction}
Dark pools are private trading venues designed for institutional investors to execute large trades anonymously, concealing details such as price and identities until after the transaction. Orders, which include the trade direction (buy or sell), volume, and price, are matched by the operator when they have opposite directions and compatible bid and ask prices. While they help prevent price swings from large orders, there are trust concerns. Operators may engage in front running, using insider knowledge of upcoming trades to execute their own trades first and profit from the price movement. Additionally, dark pool operators, often large financial institutions, may prioritize their own trades over clients', creating a conflict of interest. The lack of transparency also makes it difficult to detect manipulative practices, raising concerns about fairness. Several dark pool operators have been fined for misconduct, including misleading investors and failing to maintain proper trading practices. Notable examples include Barclays (\$70 million) and Credit Suisse (\$84.3 million) in 2016 for misrepresenting their dark pool operations, Deutsche Bank (\$3.7 million in 2017) for similar issues, ITG (\$20.3 million in 2015) for conflicts of interest, and Citigroup (\$12 million in 2018) for misleading clients about trade execution~\cite{fines}.

While dark pools aim to prevent the leakage of large orders, operators still gain privileged access to clients' hidden orders, creating potential for conflicts of interest or misuse of sensitive data. Recent research has focused on cryptographically protecting order information. These systems allow users to submit orders in encrypted form, enabling dark pool operators to compare orders without revealing their contents, only unveiling them when matches occur. This approach ensures greater security and mitigates risks associated with operator access to sensitive trade information. 

Asharov et al.~\cite{AsharovBPV20} introduced a secure dark pool model using Threshold Fully Homomorphic Encryption (FHE), which combines two cryptographic techniques: FHE, allowing computations on encrypted data without decryption, and Threshold Cryptography, where a secret (such as a decryption key) is split among multiple parties. In this approach, data is encrypted with a public key, and computations are performed on the encrypted data by an untrusted party. Decryption requires a threshold number of participants to combine their key shares. In Asharov et al.~\cite{AsharovBPV20} model, orders are encrypted under a public key, and the operator matches them directly on the encrypted data. Once a match is found, the orders are decrypted by the clients using their decryption shares, ensuring that no single entity, including the operator, can access the sensitive order details. Throughout the process, the orders remain encrypted, preventing any single party from accessing both the data and the decryption key. However, FHE is known for being computationally heavy, and adding a threshold mechanism increases the complexity. Optimizing this for real-time or large-scale applications is still an active research area. Moreover, the need for multiple parties to collaborate on decryption and sometimes computation can introduce significant communication overhead. As a result, the process takes nearly a full second to complete a single match, significantly impacting performance.

A recent development, Prime Match~\cite{polychroniadou2023prime}, introduces a solution to protect the confidentiality of periodic auctions run by a market operator. Prime Match allows users to submit orders in an encrypted form, with the operator comparing these orders through encryption and only revealing them if a match occurs. The auctions capture the trade direction (buy or sell) and the desired volume, but exclude price. Prime Match represents the first financial tool based on secure multiparty computation (MPC). In the high-stakes, highly competitive financial sector, MPC is gaining significant traction as a crucial enabler of privacy, with J.P. Morgan successfully deploying Prime Match in production.


However, in continuous double auctions without excluding the price, such as those in dark pools, where the computational complexity surpasses that of simpler periodic auctions like Prime Match, secure computation techniques fall short. They cannot support high-frequency trading within acceptable timeframes, limiting the feasibility of these methods for enhancing privacy in dark pools and the broader financial sector, where speed and efficiency are critical.

In this work, we pose the question: Can we achieve a privacy-preserving solution to dark pools with efficiency comparable to non-private dark pool protocols? We propose a system that combines differential privacy with encryption, providing a more efficient alternative to secure MPC and FHE. Differential Privacy achieves privacy by adding noise to the results of queries or computations on datasets. The level of noise is determined by a privacy parameter, which quantifies the trade-off between privacy and accuracy. By leveraging differential privacy, our dark pool approach ensures that individual orders are obfuscated while still allowing for effective matching. This method conceals the most critical aspect of dark pools—the volumes of orders—thus preserving the primary objective of dark pools, which is to hide large trades. 

In summary, integrating differential privacy with encryption offers a streamlined, efficient solution that balances privacy and computational feasibility, making it an attractive practical alternative to impractical methods like MPC and FHE. 

\subsection{Our Contributions:}

\paragraph{Problem Statement:}

The dark pool consists of $n$ agents (clients), and an operator who receives the orders from the clients. Each order takes one of two forms: (1) Buy Order: $({\sf buy}, \price, \amount)$, where $\amount$ is the quantity/volume, and $\price$ is the highest price the buyer is willing to pay per share.
(2) Sell Order: $({\sf sell}, \price, \amount)$, where $\price$ is the lowest price the seller is willing to accept per share. A buy order can be matched to a sell order if the buying
price is at least the selling price. Our objective is to design matching protocols that maximize the total number of matches while preserving user data privacy. Specifically, we aim to conceal the quantity $\amount$ of the orders during the process. Our key contributions are as follows:
\begin{enumerate}
    \item {\bf Practical Dark Pool Solution:} We propose an efficient solution for continuous double auctions (such as dark pools) that conceals both bid and ask quantities, effectively reducing reliance on trusted auctioneers while ensuring privacy guarantees.
    \item {\bf Novel Privacy Concept:} We introduce indifferential privacy, a new extension of differential privacy tailored to this context, which can be of independent interest with potential applicability beyond auctions and dark pools.
    \item {\bf Maximum Matching:} Our new notion of indifferential privacy allows us to achieve the optimal maximum matching which is impossible to achieve under conventional differential privcy~\cite{DBLP:journals/siamcomp/HsuHRRW16}. 
     \item {\bf Efficiency and Implementation: } Our system significantly outperforms previous privacy-preserving auction models, which often struggled with practicality and hindered their adoption in production. We show that our solution rivals non-private auction protocols in terms of performance, making it viable for real-world deployment. 
\end{enumerate}

\paragraph{High-Level Idea of our techniques:} To preserve privacy while matching buy and sell orders, each order is viewed as containing 
$\amount$ units, with users submitting 
$\amount+ noise$  orders to the server based on indifferential privacy. The orders are represented as nodes in a bipartite graph, with sell orders from sellers $S_i$ on the left and buy orders from buyers $B_i$ on the right. See Figure~\ref{fig:auction} for an example. The server ranks the nodes based on price $p$, where higher prices for buyers and lower prices for sellers are considered more favorable, referred to as "extreme" prices. The algorithm then constructs a bipartite graph, with edges connecting buy and sell nodes if the buying price meets or exceeds the selling price. 


\begin{figure}[hbt!]
    \centering
    \includegraphics[width=0.6\columnwidth]{./auction.png}  % Replace with the path to your figure file
    \caption{\textbf{Maximum Matching Example.} \textnormal{The figure illustrates a bipartite graph and its maximum matching where nodes are first sorted by price, followed by the user's genuine orders and then their fake orders. This arrangement ensures optimal matching, maximizing the number of successful pairings according to algorithm~\ref{alg:matching}.}}
    \label{fig:auction}
\end{figure}





In particular, the algorithm constructs and maintains a bipartite graph, where edges exist between buy and sell nodes when their prices are compatible, i.e., the buying price is at least equal to the selling price. As matches are made, isolated nodes (the gray nodes in Figure~\ref{fig:auction})—those without neighbors, such as when their price cannot be met—are promptly removed from the graph. 
In the maximum matching problem, the goal is to identify a set of edges where no two edges share a node, thereby maximizing the total number of matches. As mentioned above, nodes with the most popular price on one side of the graph are naturally connected to nodes with the least popular price on the opposite side. These pairs, referred to as polar opposite nodes, form the basis of an iterative matching strategy that guarantees an optimal matching solution.

Our approach maintains this optimal matching even when users submit a number of noise-injected fake (the red nodes in Figure~\ref{fig:auction}) orders to obscure the true amounts of their order. To achieve this, we introduce  in-differential privacy (detailed in Section~\ref{sec:IDP}) and orders are not only sorted by price but also arranged such that each user’s true orders are followed by their corresponding fake orders. 

Unlike traditional differential privacy, our new in-differential privacy concept allows for selective disclosure of information post-match, aligning with realistic scenarios. For instance, in the context of a dark pool trading environment, it becomes acceptable to reveal the actual quantity of a trade once it has been fully matched, as it no longer poses a risk to privacy.
To establish this new privacy framework, we integrate graph refinement techniques, ensuring that the protocol not only protects user data but also facilitates an efficient and optimal matching process, even under the presence of noise.

\paragraph{Implementation:} We present an end-to-end implementation of our system. While previous FHE-based solutions processed fewer than one order per second, our system dramatically outperforms them, handling between 600 and 850 orders per second (according to Table~\ref{tab:comparison}), depending on the input volume. Furthermore, we provide an analysis of the overhead introduced by our privacy-preserving mechanism compared to the non-private version. While privacy inevitably incurs some cost and does not come for free, our system's overhead remains minimal and practical, making it highly suitable for high-frequency trading environments.

\subsection{Related Work}

The works of~\cite{CartlidgeSA19,CartlidgeSA21,MazloomDPB23} leverage secure multiparty computation (MPC) with multiple operators instead of a single one. While this method is computationally faster than fully homomorphic encryption (FHE), it comes with significant communication overhead due to the necessary interactions among multiple operators. Moreover, the practicality of this approach is limited, as most current dark pool systems operate with a single operator. Importantly, MPC can only guarantee the privacy of orders if the dark pool operators do not collude, raising concerns in scenarios where collusion is feasible. Given these challenges, it is crucial to focus on solutions that emphasize single-operator architectures, which can streamline communication and enhance privacy.

The work of Massacci et al.~\cite{massacci2018futuresmex} proposes a distributed market exchange for futures assets that features a multi-step functionality, including a dark pool component. Their experiments show that the system can support up to ten traders. Notably, their model does not conceal orders; instead, it discloses an aggregated list of all pending buy and sell orders, which sets it apart from our solution. Moreover, there are existing works proposing private dark pool constructions utilizing blockchain technology~\cite{bag2019seal,galal2021publicly,ngo2021practical}, our focus diverges from this area. Furthermore, all these solutions experience slowdowns due to the reliance on computationally intensive public key cryptographic mechanisms.


The work of Hsu et al.~\cite{DBLP:journals/siamcomp/HsuHRRW16} also considered a private matching problem under the notion of \emph{joint differential privacy}, where the view of the adversary consists of the output received by all users except the user whose privacy is concerned.
Since the notion is still based on the conventional approach of using divergence on the adversarial views for neighboring inputs, their privacy notion can only lead to an almost optimal matching. In contrast, our new notion can achieve the exact optimal matching.


The authors in~\cite{PolychroniadouC24} employ differential privacy in a distinct and simplified setting of volume matching~\cite{BalchDP20,GamaCPSA22,polychroniadou2023prime}, where prices are predetermined and fixed, to obfuscate aggregated client volumes and conceal the trading activity of concentrated clients. In their auction mechanism, the obfuscated aggregate volumes are published daily, enabling buyers to make informed matching decisions based on this publicly available inventory. 