\section{Discussion and Limitations}\label{sec:discussion}

\noindent\textbf{SMC-creating instructions.} We profiled potential instructions that can invalidate the instruction cache lines available in x86 ISA, which may lead to SMC conflicts in both Intel and AMD devices. However, there might be more instructions than the profiled instructions in this work. These instructions can be discovered with a sophisticated fuzzing framework by either observing the timing values or performance counters~\cite{weber2021osiris}. It would also be interesting to analyze other machine clear events described in Ragab et al.~\cite{ragab2021rage} to determine their effect on timing measurements.

\noindent\textbf{Performance counter-based detection.} Performance counter-based detection systems are vulnerable to evasive attacks as sophisticated attackers can modify their attack code behavior to bypass detection systems~\cite{kosasihsok}. We believe that SMC-based attacks will always increment the machine clears event compared to benign applications, which is more difficult to hide compared to cache misses and stall cycle counters.

\noindent\textbf{Comparison with Previous Instruction Cache Attacks.} There are several attacks targeting the instruction cache~\cite{aciiccmez2007yet,aciiccmez2010new,yarom2016mastik} and $\mu$op cache~\cite{kim2021uc,ren2021see}. These attacks achieve higher bandwidth than the proposed SMC attacks when they are used for covert channels since accessing instructions in the L1i cache takes more time than for the SMC attacks. However, there are two advantages of the proposed SMC attacks: 1) The sibling virtual core slows down in parallel, which increases the time difference between secret activities, as shown in the RSA and SRP attack case studies. The slow-down on the sibling virtual core reaches up to 10 times, compensating for the time spent on the L1i cache accesses compared to previous work. 2) The uncertainty amount between an L1i cache hit and miss in the SMC attacks is significantly less than other attacks since they can only achieve a 1-2 cycle difference, missing important secret-specific actions. Moreover, some environments, such as AMD processors, do not provide high-resolution timers. For instance, the rdtsc instruction has only 20 cycles resolution, which cannot be used to distinguish L1i cache hit and miss for the Mastik tool. Hence, the single-trace attack with Prime+IStore was able to achieve a higher success rate compared to the Prime+Probe attack with the Mastik tool.