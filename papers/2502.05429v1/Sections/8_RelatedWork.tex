\section{Related work}\label{sec:relatedwork}

\noindent\textbf{SMC-based Attacks.} There are two previous works on the usage of SMC conflicts. The first work~\cite{aldaya2022hyperdegrade} utilizes the clflush instruction on the shared libraries by repeatedly flushing the executed instructions. The authors slow down the victim process running in the sibling core and increase the resolution of the Flush+Reload attack. This work shows that an attacker can slow down the sibling core 43 times due to a high number of machine clears and cache misses. On the other hand, our attack shows that an attacker can leak secret keys without shared libraries by creating eviction sets in the L1i cache. The second work~\cite{ragab2021rage} focuses on extending the speculative window size by creating several machine clear events including SMC. The secret is leaked through the Flush+Reload cache covert channel through the data cache, while our attack leaks the secrets from the L1i cache covert channel from an indirect branch.

\noindent\textbf{Instruction Cache Attacks.}
Aciicmez et al.~\cite{aciiccmez2007yet} showed that Prime+Probe attacks could be implemented on the L1 instruction cache to monitor the executed instructions from the victim process in the sibling core. %They also proposed that attacking L1 instruction cache can have the potential to reveal RSA cryptosystems. 
Moreover, Aciicmez et al.~\cite{aciiccmez2010new} build upon earlier work~\cite{aciiccmez2007yet} in instruction cache attacks, providing more effective ways to explore the L1I cache incorporates Vector Quantization (VQ) and Hidden Markov Models (HMM) techniques. This work recovered the DSA private key utilizing lattice methods and proposed a mitigation method by disabling multi-threading, disabling caching, cache flushing, and arranging the memory layout with dummy \texttt{nop} instructions.

\noindent\textbf{Spectre Attacks.} A large body of Spectre attacks has been explored in the security community~\cite{xiong2021survey,canella2019systematic}. These either attacks focus on manipulating branch prediction mechanisms~\cite{kocher2020spectre,chen2019sgxpectre,koruyeh2018spectre,van2020lvi} or establishing new covert channels~\cite{bhattacharyya2019smotherspectre,loughlin2021dolma,kocher2020spectre}. %We focus on various side-channels to encode secrets into microarchitectural states. 
Smotherspectre~\cite{bhattacharyya2019smotherspectre} creates a side channel to leak secrets based on the executed instructions. The covert channel makes use of the execution ports assigned to distinct instruction types. DOLMA~\cite{loughlin2021dolma} shows that data TLB structure can be leveraged to leak secrets in the transient domain if the secret is encoded into a separate page. The first examples of Spectre attacks~\cite{kocher2020spectre} leverage the data cache to leak secrets through data caches.   