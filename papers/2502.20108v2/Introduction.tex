\subsection{Motivation}

Over time, diffusion model-based approaches have proven their value in robotic policy learning tasks \cite{chi2023diffusionpolicy, yang2024diff-es, yu2024ldp}. Dating back to the advancement of diffusion models, they have gained recognition as a cornerstone in the field of generative modeling \cite{peebles2023DiT}. Conditioned diffusion models extend vanilla diffusion models by incorporating additional information during the generation process, while latent diffusion models improve computational efficiency and sample quality by operating in a compressed latent space \cite{rombach2022latentDiff}. As shown above, diffusion models have exhibited promising potential in generating high-quality data across various modalities and improving the representation of complex data structures \cite{yang2023diffrep}. \par In robotic applications, multisensory data often includes rich and heterogeneous sources, such as camera images, LiDAR point clouds, etc. Diffusion models, through their ability to condition on various modalities, can generate coherent and contextually relevant outputs. This capability is particularly advantageous for robotic state-action mapping, where accurate interpretation and synthesis of multisensory inputs are crucial for effective decision-making and action execution \cite{li2025grmg}. \par Regarding autonomous driving, where the end-to-end paradigm has evolved vigorously \cite{sun2024sparsedrive}, state-action mapping is a core principle that enables vehicles to learn effective decision-making policies directly from raw sensor inputs \cite{liao2024diffusiondrive}. This process involves mapping the current state of the vehicle and its environment to appropriate control and planning actions. \par To enrich the state-understanding capacity of end-to-end autonomous driving systems, Visual Language Models (VLMs) have exhibited an outstanding impact, significantly improving the systems' interpreting capability of complex driving scenarios \cite{tian2024drivevlm, guo2024vlm-auto}. Accordingly, it is essential to enhance the decision-making capabilities as the improvement of state understanding by proposing adaptive and context-aware actions tailored to various driving scenarios \cite{li2024hydra}. 

\begin{figure}[t]
    \centering
    \includegraphics[width=0.98\linewidth]{Figs/cover.png}
    \begin{subfigure}{0.48\textwidth}
        \caption{VLM's path proposals from  \ \  (b) Conditional sampled paths \\ the continuous frames based on  \ \ by our diffusion Transformers. \\ a consistent scenario.}
        \label{fig:vlm_path_proposals}
    \end{subfigure}
    
    \caption{We conduct the experiments with our VDT-Auto on unseen real-world driving dataset in a zero-shot way. The VLM is fine-tuned on our processed nuScenes dataset while the path proposals from the fine-tuned VLM are able to provide contextual approximation across the unseen continuous frames. Subsequently, our diffusion Transformers sample the path proposals based on the geometric and contextual conditions.}
    \label{fig:cover}
\vspace{-0.5cm}
\end{figure}

\par With these insights, we propose VDT-Auto, an end-to-end paradigm that bridges states and actions via VLM and diffusion Transformers. For state understanding, images from the surrounding cameras are encoded into bird’s-eye view (BEV) features. In addition, a front image among the surrounding images is passed to a supervised fine-tuned VLM for contextual interpretation by the description of the detection, the advice of ego vehicle's behavior and the proposal of a path. Meanwhile, the designed diffusion Transformers encode both the BEV features from the BEV backbone and the contextual embeddings from the VLM as the states to predict the optimized path, where the added noise in the diffusion process is sampled from the VLM's proposed path. Our contributions in this paper are summarized as follows: 
\begin{itemize}
    \item We introduce a novel pipeline, VDT-Auto, which employs a BEV encoder and a VLM to geometrically and contextually parse the environment. The parsed information is then used to condition the diffusion process of our diffusion Transformers to generate the optimized actions of the ego vehicle.
    \item VDT-Auto is differentiable, where we use a processed nuScenes dataset to train our BEV encoder for perception and fine-tune our VLM for conditioning the diffusion Transformers. The constructed and processed dataset will be publicly available. 
    \item In the nuScenes open-loop planning evaluation, our VDT-Auto achieved $0.52$ m on average L2 errors and $21\%$ on average collision rate. In our real-world driving dataset, VDT-Auto showed promising performance on the unseen data in a zero-shot way.
\end{itemize}

\subsection{Related Work}

\section{Related works}
Implicit Neural Representations are designed to learn continuous representations of target functions by taking advantages of the approximation power of neural networks.
%
Their inherent continuous property can beneficial in many cases like video compression~\citep{chen2021nerv,strumpler2022implicit}, 3D modeling~\citep{park2019deepsdf,atzmon2020sal,9010266,gropp2020implicit,sitzmann2019scene} and volume rendering~\citep{pumarola2021d, barron2021mip,martin2021nerf,barron2023zip}.
%
However, simply employing MLPs may result in spectral bias, where oversmoothed outputs are generated due to the inherent tendency of MLPs to prioritize learning low-frequency components first. Consequently, many studies have focused on these drawbacks and explored various methods to address this issue.
%
The most straightforward way to address this issue is by projecting the coordinates into the higher dimension~\citep{tancik2020fourier, wang2021spline}.
%
However, these methods can lead to noisy outputs if there is a mismatch in the embeddings variance.
%
To address this, \citet{landgraf2022pins} propose dividing the Random Fourier Features into multiple levels of detail, allowing the MLPs to disregard unnecessary high-frequency components. Another type of approach to mitigating the spectral bias introduced by the ReLU activation function, as proposed by \citet{sitzmann2020implicit}, \citet{ramasinghe2022beyond}, \citet{saragadam2023wire}, and \citet{shenouda2024relus}, is to modify the activation function itself by using alternatives such as the Sine function, Wavelets, or a combination of ReLU with other functions. There are also efforts to modify network structures to mitigate spectral bias~\citep{mujkanovic2024neural}. 
%
\citet{lindell2022bacon} introduce a network design that treats MLPs as filters applied to the input of the next layer, known as Multiplicative Filter Networks (MFNs). 
%
Additionally, based on the discrete nature of signals like images and videos, grid-based approaches (e.g., Grid Tangent Kernel~\citep{zhao2024grounding}, DINER~\citep{xie2023diner}, and Fourier Filter Bank~\citep{wu2023neural}) have been proposed to address spectral bias, as the grid property allows for sharp changes in features, which facilitates learning fine details.
Even though, there are some prior works trying to solve the inherent problems of Fourier features embeddings ~\citep{landgraf2022pins, yuce2022structured, hertz2021sape, saratchandran2024sampling}, limited research has addressed both the underlying causes of high-frequency noise and provides a non-heuristic solution even if these embeddings are widely employed into many downstream tasks.


