\subsection{Motivation}

Over time, diffusion model-based approaches have proven their value in robotic policy learning tasks \cite{chi2023diffusionpolicy, yang2024diff-es, yu2024ldp}. Dating back to the advancement of diffusion models, they have gained recognition as a cornerstone in the field of generative modeling \cite{peebles2023DiT}. Conditioned diffusion models extend vanilla diffusion models by incorporating additional information during the generation process, while latent diffusion models improve computational efficiency and sample quality by operating in a compressed latent space \cite{rombach2022latentDiff}. As shown above, diffusion models have exhibited promising potential in generating high-quality data across various modalities and improving the representation of complex data structures \cite{yang2023diffrep}. \par In robotic applications, multisensory data often includes rich and heterogeneous sources, such as camera images, LiDAR point clouds, etc. Diffusion models, through their ability to condition on various modalities, can generate coherent and contextually relevant outputs. This capability is particularly advantageous for robotic state-action mapping, where accurate interpretation and synthesis of multisensory inputs are crucial for effective decision-making and action execution \cite{li2025grmg}. \par Regarding autonomous driving, where the end-to-end paradigm has evolved vigorously \cite{sun2024sparsedrive}, state-action mapping is a core principle that enables vehicles to learn effective decision-making policies directly from raw sensor inputs \cite{liao2024diffusiondrive}. This process involves mapping the current state of the vehicle and its environment to appropriate control and planning actions. \par To enrich the state-understanding capacity of end-to-end autonomous driving systems, Visual Language Models (VLMs) have exhibited an outstanding impact, significantly improving the systems' interpreting capability of complex driving scenarios \cite{tian2024drivevlm, guo2024vlm-auto}. Accordingly, it is essential to enhance the decision-making capabilities as the improvement of state understanding by proposing adaptive and context-aware actions tailored to various driving scenarios \cite{li2024hydra}. 

\begin{figure}[t]
    \centering
    \includegraphics[width=0.98\linewidth]{Figs/cover.png}
    \begin{subfigure}{0.48\textwidth}
        \caption{VLM's path proposals from  \ \  (b) Conditional sampled paths \\ the continuous frames based on  \ \ by our diffusion Transformers. \\ a consistent scenario.}
        \label{fig:vlm_path_proposals}
    \end{subfigure}
    
    \caption{We conduct the experiments with our VDT-Auto on unseen real-world driving dataset in a zero-shot way. The VLM is fine-tuned on our processed nuScenes dataset while the path proposals from the fine-tuned VLM are able to provide contextual approximation across the unseen continuous frames. Subsequently, our diffusion Transformers sample the path proposals based on the geometric and contextual conditions.}
    \label{fig:cover}
\vspace{-0.5cm}
\end{figure}

\par With these insights, we propose VDT-Auto, an end-to-end paradigm that bridges states and actions via VLM and diffusion Transformers. For state understanding, images from the surrounding cameras are encoded into bird’s-eye view (BEV) features. In addition, a front image among the surrounding images is passed to a supervised fine-tuned VLM for contextual interpretation by the description of the detection, the advice of ego vehicle's behavior and the proposal of a path. Meanwhile, the designed diffusion Transformers encode both the BEV features from the BEV backbone and the contextual embeddings from the VLM as the states to predict the optimized path, where the added noise in the diffusion process is sampled from the VLM's proposed path. Our contributions in this paper are summarized as follows: 
\begin{itemize}
    \item We introduce a novel pipeline, VDT-Auto, which employs a BEV encoder and a VLM to geometrically and contextually parse the environment. The parsed information is then used to condition the diffusion process of our diffusion Transformers to generate the optimized actions of the ego vehicle.
    \item VDT-Auto is differentiable, where we use a processed nuScenes dataset to train our BEV encoder for perception and fine-tune our VLM for conditioning the diffusion Transformers. The constructed and processed dataset will be publicly available. 
    \item In the nuScenes open-loop planning evaluation, our VDT-Auto achieved $0.52$ m on average L2 errors and $21\%$ on average collision rate. In our real-world driving dataset, VDT-Auto showed promising performance on the unseen data in a zero-shot way.
\end{itemize}

\subsection{Related Work}

\subsubsection{Conditioned Diffusion Models}

By operating the data in latent space instead of pixel space, conditioned diffusion models have gained promising development \cite{rombach2022latentDiff}. MM-Diffusion \cite{ruan2023mmdi} designed for joint audio and video generation took advantage of coupled denoising autoencoders to generate aligned audio-video pairs from Gaussian noise. Extending the scalability of diffusion models, diffusion Transformers treat all inputs, including time, conditions, and noisy image patches, as tokens, leveraging the Transformer architecture to process these inputs \cite{bao2023ViTDiff}. In DiT \cite{peebles2023DiT}, William et al. emphasized the potential for diffusion models to benefit from Transformer architectures, where conditions were tokenized along with image tokens to achieve in-context conditioning. 

\subsubsection{Diffusion Models in Robotics}

Recently, a probabilistic multimodal action representation was proposed by Cheng Chi et al. \cite{chi2023diffusionpolicy}, where the robot action generation is considered as a conditional diffusion denoising process. Leveraging the diffusion policy, Ze et al. \cite{ze20243d} conditioned the diffusion policy on compact 3D representations and robot poses to generate coherent action sequences. Furthermore, GR-MG combined a progress-guided goal image generation model with a multimodal goal-conditioned policy, enabling the robot to predict actions based on both text instructions and generated goal images \cite{li2025grmg}. BESO used score-based diffusion models to learn goal-conditioned policies from large, uncurated datasets without rewards. Score-based diffusion models progressively add noise to the data and then reverse this process to generate new samples, making them suitable for capturing the multimodal nature of play data \cite{reuss2023md}. RDT-1B employed a scalable Transformer backbone combined with diffusion models to capture the complexity and multimodality of bimanual actions, leveraging diffusion models as a foundation model to effectively represent the multimodality inherent in bimanual manipulation tasks \cite{liu2024rdt-1b}. NoMaD exploited the diffusion model to handle both goal-directed navigation and task-agnostic exploration in unfamiliar environments, using goal masking to condition the policy on an optional goal image, allowing the model to dynamically switch between exploratory and goal-oriented behaviors \cite{sridhar2023nomad}. The aforementioned insights grounded the significant advancements of diffusion models in robotic tasks.

\subsubsection{VLM-based Autonomous Driving}

End-to-end autonomous driving introduces policy learning from sensor data input, resulting in a data-driven motion planning paradigm \cite{chen2024vadv2}. As part of the development of VLMs, they have shown significant promise in unifying multimodal data for specific downstream tasks, notably improving end-to-end autonomous driving systems\cite{ma2024dolphins}. DriveMM can process single images, multiview images, single videos, and multiview videos, and perform tasks such as object detection, motion prediction, and decision making, handling multiple tasks and data types in autonomous driving \cite{huang2024drivemm}. HE-Drive aims to create a human-like driving experience by generating trajectories that are both temporally consistent and comfortable. It integrates a sparse perception module, a diffusion-based motion planner, and a trajectory scorer guided by a Vision Language Model to achieve this goal \cite{wang2024hedrive}. Based on current perspectives, a differentiable end-to-end autonomous driving paradigm that directly leverages the capabilities of VLM and a multimodal action representation should be developed. 










