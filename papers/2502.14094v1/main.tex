\documentclass[conference]{IEEEtran}
%% IEEE CNS addition:
\makeatletter
\def\ps@headings{%
\def\@oddhead{\mbox{}\scriptsize\rightmark \hfil \thepage}%
\def\@evenhead{\scriptsize\thepage \hfil \leftmark\mbox{}}%
\def\@oddfoot{}%
\def\@evenfoot{}}
\makeatother
\pagestyle{empty}

\usepackage[utf8]{inputenc}
\usepackage{amsmath}
\usepackage{booktabs}
\usepackage{listings}
\usepackage{cite}
\usepackage{adjustbox}
\usepackage{xcolor}
\usepackage{color,soul}
\usepackage{multirow}
\usepackage{enumerate}
\usepackage[
  separate-uncertainty = true,
  multi-part-units = repeat
]{siunitx}
\usepackage{algorithm}
\usepackage{graphicx}
\usepackage{textcomp}
%\usepackage{hyperref}
\usepackage{caption}
\usepackage{subcaption}
\usepackage{authblk}
\usepackage{caption}
\usepackage{amsfonts}
\usepackage{algpseudocode}
\usepackage{url}
\usepackage[justification=centering]{caption}
\newcommand{\R}{\mathbb{R}}

\definecolor{colorEntityBack}{rgb}{0.01, 0.01, 0.4}
\definecolor{colorPolicyBackDarkDark}{rgb}{0.81, 0.81, 0.86}
\definecolor{ashgrey}{rgb}{0.7, 0.75, 0.71}
\definecolor{beaublue}{rgb}{0.74, 0.83, 0.9}
\definecolor{hotpink}{rgb}{1.0, 0.41, 0.71}
\definecolor{darkgreen}{rgb}{0.09, 0.45, 0.27}
\definecolor{cerulean}{rgb}{0.105, 0.672, 0.836}
\definecolor{candyapplered}{rgb}{1.0, 0.03, 0.0}
\definecolor{cadetgrey}{rgb}{0.57, 0.64, 0.69}
\definecolor{trolleygrey}{rgb}{0.5, 0.5, 0.5}

\newcommand{\og}[1]{\textcolor{red}{\textsf{Onat:~~#1}}}
\newcommand{\sean}[1]{\textcolor{blue}{\textsf{Sean:~~#1}}}
\newcommand{\changes}[1]{\textcolor{darkgreen}{#1}}
\begin{document}

\date{18 February 2024}

\title{CND-IDS: Continual Novelty Detection for Intrusion Detection Systems}

%\author{}
\author[1]{Sean Fuhrman}
\author[2]{Onat Gungor}
\author[2]{Tajana Rosing}

\affil[1]{Department of Electrical and Computer Engineering, University of California, San Diego}
\affil[2]{Department of Computer Science and Engineering, University of California, San Diego}
\affil[ ]{\textit{\{stfuhrma, ogungor, tajana\}@ucsd.edu}}
%\IEEEoverridecommandlockouts
%\IEEEpubid{\makebox[\columnwidth]{978-1-7281-6127-3/21/\$31.00~\copyright2022 IEEE \hfill} \hspace{\columnsep}\makebox[\columnwidth]{ }}

\maketitle
\pagestyle{plain}
\pagenumbering{gobble}
\newcommand{\norm}[1]{\left\lVert#1\right\rVert}
\newcommand{\Design}[0]{CND-IDS}

\begin{abstract}
%The Industrial Internet of Things (IIoT) security has become a significant challenge due to increased inter-connectivity and large scale networks. 
Intrusion detection systems (IDS) play a crucial role in IoT and network security by monitoring system data and alerting to suspicious activities. Machine learning (ML) has emerged as a promising solution for IDS, offering highly accurate intrusion detection. However, ML-IDS solutions often overlook two critical aspects needed to build reliable systems: continually changing data streams and a lack of attack labels. Streaming network traffic and associated cyber attacks are continually changing, which can degrade the performance of deployed  ML models. Labeling attack data, such as zero-day attacks, in real-world intrusion scenarios may not be feasible, making the use of ML solutions that do not rely on attack labels necessary. To address both these challenges, we propose \Design{}, a continual novelty detection IDS framework which consists of (i) a  learning-based feature extractor that continuously updates new feature representations of the system data, and (ii) a novelty detector that identifies new cyber attacks by leveraging principal component analysis (PCA) reconstruction. Our results on realistic intrusion datasets show that \Design{} achieves up to 6.1$\times$ F-score improvement, and up to 6.5$\times$ improved forward transfer over the SOTA unsupervised continual learning algorithm. Our code will be released upon acceptance. 
%\Design{} also outperforms all other state-of-the-art novelty detection algorithms consistently. Novelty detection (ND) algorithms offer a solution to detect attacks without labels.

\end{abstract}

%\begin{IEEEkeywords}
%Industrial internet of things (I-IoT), predictive maintenance, intelligent fault diagnosis, hyper-dimensional computing, adversarial attacks, resilient machine learning 
%\end{IEEEkeywords}

\section{Introduction}
\section{Introduction}


Sequential resource allocation is a fundamental problem in many domains, including healthcare, finance, and public policy \cite{considine2023optimizing,boehmer2024optimizing, yu2024fincon}. This task involves allocating limited resources over time while accounting for dynamic changes and competing demands. Deep reinforcement learning (RL) is an effective method to optimize decision-making for such challenges, offering efficient and scalable policies~\cite{yu2021reinforcement,talaat2022effective, xiong2023reinforcement,zhao2024towards}. However, deep RL policies generally provide action recommendations without human-readable reasoning and explanations. Such lack of interpretability poses a major challenge in critical domains where decisions must be transparent, justifiable, and in line with human decision-makers to ensure trust and compliance with ethical and regulatory standards.



For example, doctors may need to decide whether to prioritize intervention for Patient A or Patient B based on their current vital signs~\cite{boehmer2024optimizing}. An RL algorithm might suggest: \textit{ ``Intervene with Patient A "} with the implicit goal of maximizing the value function. However, the underlying reasoning may not be clear to the doctors, leaving them uncertain about the factors influencing the decision \cite{milani2024explainable}. For doctors, a more effective suggestion could be risk-based with specific information, e.g., \textit{``Patient A's vital signs are likely to deteriorate leading to higher potential risk compared to Patient B, so intervention with Patient A is prioritized"} \cite{gebrael2023enhancing, boatin2021wireless}.




\begin{figure*}[tbp]
    \centering
    \includegraphics[width=0.99\linewidth]{Figures/icml25_ProposedFramework.pdf}
    \caption{Overview of the \rbrl framework for joint sequential decision-making and explanation generation at time instance $t$. Starting with current state $\bs_t$,  a state-to-language descriptor generates \lang($\bs_t$), which is used to create the input prompt 
$\bp_t$. The LLM processes 
$\bp_t$
  to produce a thought 
$\pmb{\tau}_t$  and a set of candidate rules 
$\cR_t$ . An attention-based policy network selects a rule 
$\arule_t$ , which is used to derive an executable action $\aenv_t$ satisfying the budget constraint $B(\bs_t)$ for the environment 
  and a human-readable explanation $\pmb{\ell}_t^{expl}$, while also providing a rule reward $r_t^{\text{rule}}$ 
 . The environment transitions to the next state 
$\bs_{t+1}$ , returning an environment reward $r_t^{\text{env}}$ 
 . This process is repeated iteratively at subsequent time steps. 
}
    \label{fig:Proposed_framework}
\end{figure*}


Language agents \cite{sumers2024cognitive} leverage large language models (LLMs) for multi-step decision-making using reasoning techniques like chain of thought (CoT) \cite{wei2022chain} and ReAct \cite{yao2023react}. They enable natural language goal specification \cite{du2023guiding} and enhance human understanding \cite{hu2023language, srivastava2024policy}. However, LLMs struggle with complex sequential decision-making, such as resource allocation \cite{furuta2024exposing}, making RL a crucial tool for refining them into effective policy networks \cite{carta2023grounding, tantrue, wen2024reinforcing, zhai2024fine}. Yet, fine-tuning LLMs for policy learning is highly challenging due to the substantial computational costs and the complexity of token-level optimization \cite{rashid2024critical}, which remains an open challenge, particularly in sequential resource allocation.

Consequently, aiming to combine the strengths of both deep RL and language agents, we pose the following question:


\vspace{-0.1in}
\begin{tcolorbox}[colback=white!5!white,colframe=white!75!white]
\textit{%
Can we design a language agent framework that can simultaneously perform sequential resource allocation and provide human-readable explanations? }
\end{tcolorbox}
\vspace{-0.15in}






Motivated by existing work that employs predefined rules or concepts to explain RL policies \cite{Das2023State2Explanation} or guide RL exploration \cite{likmeta2020combining}, we explore the potential of using rules to prioritize individuals in resource allocation problems. In the context of language agents, rules are defined as ``structured statements" that capture prioritization among choices in a given state, aligning with the agent's goals \cite{srivastava2024policy}. 
Rules offer a flexible framework for encoding high-level decision criteria and priority logic, similar as the celebrated index policy for prioritizing arms in resource allocation problems \cite{whittle1988restless}, making them ideal for guiding resource allocation strategies while explaining the rationale behind decisions.%



Building on this, we propose a novel framework called Rule-Bottleneck Reinforcement Learning (\rbrl), which integrates the strengths of LLMs and RL to bridge the gap between decision-making and interoperability, by optimizing LLM-generated rules with RL. 
\rbrl provides a framework (as shown in Figure \ref{fig:Proposed_framework}) that simultaneously makes sequential resource allocation decisions and provides human-readable explanations. \rbrl leverages LLMs to generate candidate rules and employs RL to optimize policy selection, enabling the creation of effective decision policies while simultaneously providing human-understandable explanations. 

Our contributions are summarized as follows. \textit{First}, to avoid the computational cost and complexity of directly fine-tuning language agents, we leverage LLMs to generate a diverse set of rules, where each rule serves as a prioritization strategy for individuals in resource allocation. This approach enhances flexibility and interpretability in decision-making.
\textit{Second}, we extend the conventional state-action space by integrating the thoughts and rules generated by LLMs, creating a novel framework that enables reinforcement learning to operate on a richer, more interpretable decision structure.
\textit{Third}, we introduce an attention-based training framework that maps states to queries and rules to keys. The rule selection process is optimized by a policy network trained using the Soft Actor-Critic (SAC) algorithm \cite{haarnoja2018soft}, ensuring robust and efficient decision-making. In particular, the LLM also acts as a feedback mechanism, providing guidance during RL exploration to improve policy optimization and promote more effective learning. 
 



We evaluate our method in three environments from two real-world domains: \texttt{HeatAlerts}, where resources are allocated to mitigate extreme heat events; and \texttt{WearableDeviceAssignment}, for distributing monitoring devices to patients. 
Using cost-effective LLMs such as gpt-4o-mini \cite{openai2024gpt4omini} and Llama 3.1 8B \cite{meta2024llama3.1}, we first assess decision performance by comparing \rbrl with pure RL methods and language agent baselines. We then evaluate explanation quality through a human survey conducted under IRB approval. The results demonstrate \rbrl's effectiveness in both decision quality and interpretability.















\section{Related Work}
\section{Related Work}

\subsection{View-Dependent Control}
View-dependent representations have a long history in computer graphics.
In his pioneering work, Rademacher proposed interpolating between \textit{key viewpoints} and associated \textit{key deformations} to manipulate 3D models~\cite{rademacher1999view}.
Other researchers have extended the idea to create 3D animation systems~\cite{10.1111:j.1467-8659.2004.00772.x}, streamline the modeling process~\cite{DBLP:journals/corr/abs-2103-15472}, and integrate physical simulation\cite{koyama2013view}.
Of particular note, Rivers et al.~\cite{rivers25Dcartoonmodels} introduced \textit{2.5D Cartoon Models}, a combination of planar meshes transformed, based upon view angle, so as to appears three dimensional.
Our work draws upon these works but is, to our knowledge, the first work to attempt to use view-dependent techniques to retarget 3D motion onto 2D characters.   

\subsection{Animation from 2D Images}

% output is still 2D
Many researchers have proposed different methods for creating animations from 2D images. Hornung et al.~\cite{Hornung2007anim2Dpicmotion} presented a method to deform a character from a photograph given user-provided joint annotations.
\textit{Toonsynth}~\cite{Dvoroznak18-SIG} and \textit{Neural Puppet}~\cite{poursaeed2020neural} both present methods to create new images of hand-drawn characters from examples.
% output is 3D model
Other researchers have proposed methods of obtaining 3D geometry from 2D sketches~\cite{igarashi2006teddy, Dvoroznak20-SA} and images~\cite{ArtiSketch,weng2019photo}.
% focus on sketches specifically
A number of works have specifically focused on childlike drawings.
Lingens et al.~\cite{lingens2020towards} proposed an evolutionary algorithm for animating children's drawings. 
\textit{MagicToon}~\cite{feng2017magictoon} creates a 3D model from childlike drawings for AR applications.
Similar to our work, Smith et al.~\cite{SmithHodgins} proposed a method for animating childlike drawings using 3D skeletal motion. 
However, the resulting animations are only suitable for use in 2D applications and the type of motions it supports are limited.

Unlike these previous works, our solution can be used in 3D contexts but does not create a 3D model. We instead relying upon a view-dependent formulation of the animated character.

\section{\Design{} Framework}
\label{framework}
\begin{figure*}
    \centering
    \includegraphics[width=0.95\linewidth]{Figs/Overview.png}
    \caption{\textbf{Framework overview of VDT-Auto.} At each time step, the surrounding images are encoded by the BEV encoder to provide the geometric feature grids of the scenario. A front image from the surrounding images is analyzed by our fine-tuned VLM to provide the contextual information of the conditions. Based on the BEV feature grids and VLM output, we construct the conditional latents for our diffusion Transformers, where the BEV feature grids and VLM's detection and advice are embedded and VLM's path proposal is sampled for conditioning. In Section \Romannum{3}, we introduce our noise sampling approach in details. Finally, our diffusion Transformers denoise the VLM's path proposal, conditioning on the geometric feature grids of the scenario and the contextual information from our fine-tuned VLM.}
    \label{fig:overview}
\vspace{-0.3cm}
\end{figure*}

\subsection{BEV Encoder}

In Fig. \ref{fig:overview}, our BEV encoder is based on LSS \cite{philion2020lss, hu2021fiery}, where the surrounding camera images from the $T$ time steps are lifted into the BEV feature grids. $F^{k}_{t}\in\mathbb{R}^{(C_{f} + D_{d}) \times H \times W}$ represents the extracted features of the $k$-th camera at time $t$ from the image backbone, where $F^{k}_{t,C_{f}} \in \mathbb{R}^{C_{f} \times H \times W}$ is the contextual features and $F^{k}_{t,D_{d}} \in \mathbb{R}^{D_{d} \times H \times W}$ represents the estimated depth distribution. Then the contextual feature map in height dimension $F'^{k}_{t}$ is computed as $F^{k}_{t,C_{f}} \otimes F^{k}_{t,D_{d}}$. According to the nuScenes camera setup \cite{caesar2020nuscenes}, with the intrinsics and extrinsics of the cameras, $F'^{k}_{t}$ is then aggregated and weighted along the height dimension into the ego-centered coordinate system to obtain the BEV feature grids $G_{t} \in \mathbb{R}^{C_\text{state} \times H \times W}$ at time $t$, where $C_\text{state}$ is the number of state channels.

\subsection{VLM Module}

For our work, Qwen2-VL-7B is used to bridge the input of sensory data and the output of contextual conditions \cite{wang2024qwen2vl}. In Qwen2-VL, Multimodal Rotary Position Embedding (M-RoPE) is applied to process multimodal input by decomposing rotary embedding into temporal, height, and width components, which equips Qwen2-VL with powerful multimodal data handling capabilities. In our VDT-Auto, the supervised fine-tuning of Qwen2-VL-7B is carried out by feeding a front image of surrounding cameras and system prompts, expecting the output of the description of the detection, the structured advice of the behavior of the ego vehicle, and the proposal of a path. To achieve supervised fine-tuning, we constructed our fine-tuning dataset by extracting ground truth information from the nuScenes dataset \cite{caesar2020nuscenes}. In Section \Romannum{3}, we will introduce more details about our dataset construction and supervised fine-tuning.

\subsection{Diffusion Prerequisites}

In Fig. \ref{fig:overview}, we show our entire VDT-Auto pipeline, where the feature grids $G_{t} \in \mathbb{R}^{C_\text{state} \times H \times W}$ of the BEV encoder and the contextual output $S_t$ of Qwen2-VL-7B including the description of the detection and structured advice on the behavior of the ego vehicle are encoded as state conditions for the diffusion process. Thus, in our designed diffusion Transformers, the conditioned policy $\pi_{\theta}(A_t | G_t, S_t)$ predicts the denoised path $A_t = (a^0_t, a^1_t, \ldots,a^n_t)$ of length $n$, conditioned on both the current BEV features $G_t$ and contextual embeddings $S_t$ \cite{bao2023onemmdi, han2024emma, reuss2024mdt}. During training, the proposal of a path based on supervised fine-tuning of Qwen2-VL-7B at time $t$ pairing with current BEV features $G_t$ and contextual embeddings $S_t$ to form the training set, where our diffusion Transformers aim to maximize log-likelihood $\ell_{\text{training}}$ throughout the training set,

\vspace{-0.3cm}
\begin{equation}
    \ell_{\text{training}} = \underset{\theta}{\arg\max}{}_{(a_t^i, g_t^i, s_t^i) \in (A'_t, G_t, S_t)} \log {\pi_\theta}({a_t^i | g_t^i, s_t^i}),
\end{equation}
where $a_t^i, g_t^i, s_t^i$ are sampled from our constructed training set. We extract the noise distribution $\sigma_{\text{VLM}}$ from the path output of the supervised fine-tuned Qwen2-VL-7B to construct the noisy path dataset $A'_t$ by adding the sampled noise from the extracted noise distribution $\sigma_{\text{VLM}}$ to the ground truth path $A_{gt}$ of nuScenes.

\par In Section \Romannum{3}, we demonstrate that the noise distribution $\sigma_{\text{VLM}}$ of the path proposal from our fine-tuned Qwen2-VL-7B is treated as a normal distribution, where we examine the extracted noise using One-Sample Kolmogorov-Smirnov test for both the $x$ and $y$ coordinates of the paths \cite{Kstest}. 

\subsection{Loss Functions}

Our diffusion Transformers predict the denoised path $A_t$ conditioned on current BEV features $G_t$ and contextual embeddings $S_t$. Therefore, the loss function is defined as follows.

\vspace{-0.3cm}
\begin{equation}
\begin{split}
    \mathcal{L}_{\text{train}} = \mathcal{L}_{\text{MSE}} (\pi_{\theta}(A_t | g_t, s_t, \boldsymbol{\epsilon}), A_{gt}) + \\
    \mathcal{L}_{\text{MSE}}(\sum_{j=1}^{n} a^j, \sum_{j=1}^{n} a_{gt}^j),
\end{split}
\end{equation}
where $\pi_{\theta}$ is our trained diffusion Transformers. Under the conditions of encoded BEV features $g_t \in G_t$, contextual embeddings $s_t \in S_t$, and added noise $\boldsymbol{\epsilon} \in \sigma_{\text{VLM}}$, the first part of our loss function is the mean squared error between the path prediction $A_t$ and the ground truth path from nuScenes $A_{gt}$. Besides, the second part of our loss function is the mean squared error between the cumulative sum of the waypoints $a^j \in A_t$ and $a_{gt}^j \in A_{gt}$.



\section{Experimental Analysis}
\label{experimental}

\newpage
\section{Experiment}\label{sec-experiment}
\subsection{Experimental Setup}
We briefly introduce experimental settings to verify our proposed MoR, including Datasets \& Baselines, Implementation Details, and Evaluation Metrics. More details are in Appendix~\ref{app-expr-setting}.

\textbf{Datasets \& Baselines:} We use three TG-KBs from STaRK~\cite{wu2024stark} covering three knowledge domains, including E-commerce Products (Amazon), Academic Papers (MAG), and Biomedicine (Prime). We compare our MoR with baselines established by~\citet{wu2024stark} and categorize them into textual/structural/hybrid-based ones. More recent state-of-the-art hybird retrieval approaches fro TG-KBs such as KAR~\cite{xia2024knowledge} and MFAR$^{*}$~\cite{li2024multi} are also compared.


\textbf{Implementation Details:} 
To enhance the planning capability of our planning module, we fine-tune the Llama 3.2 (3B) on 1000 sampled queries with their corresponding ground-truth planning graphs, serving as the textual graph generator. In the absence of ground-truths, we synthesize them using LLMs. For the Prime dataset, we empirically find that directly prompting LLMs can hardly generate accurate planning graphs due to the lack of biomedical domain knowledge~\cite{Shen2024TagLLMRG}. Therefore, we adopt an alternative approach. First, we instruct LLMs to extract triplets from each query and then construct the planning graphs by merging triplets with shared entities. 
During mixed traversal, textual matching can be implemented using any lexical or semantic methods. For this study, we employ BM25 for Amazon and MAG and fine-tune a contriever to complement the biomedical knowledge for Prime.
To initialize the structural traversal, we employ textual matching to locate the top 5 nodes that are most relevant to the query as seeds. Additionally, at each layer, we incorporate the top 10 nodes retrieved via textual matching and append them to the current candidate set for the next round of traversal. Notably, due to the uncertainty of LLMs, the generated planning graphs can be invalid. In this case, we will directly conduct textual matching to retrieve candidates. For our ablations without reranker, we employ Ada-002~\cite{wu2024stark} with cosine similarity as the scorer to rank candidates for evaluating performance.

\textbf{Evaluation Metrics:}
We follow~\citet{wu2024stark} for evaluation by reporting Hit@1 (H@1), Hit@5 (H@5), Recall@20 (R@20), and mean reciprocal rank MRR to evaluate in the full spectrum. 


 

\newpage
\subsection{Overall Retrieval Performance}
We compare MoR with other baselines on three TG-KBs in Table~\ref{tab-merged}. Generally, hybrid methods, AvaTAR, KAR, MFAR$^{*}$, and our MoR, achieve better performance than purely textual or structural methods owing to their ability to integrate both structural and textual knowledge. 
Among all baselines, our proposed MoR achieves the overall best performance with a substantial margin on average, with the first ranking on MAG and the second ranking on Amazon/Prime datasets. This demonstrates the effectiveness of our proposed mixture of structural and textual knowledge retrieval. 
Textual retrieval performs better on Amazon than on MAG, suggesting that Amazon queries rely more on textual knowledge. In contrast, its weaker performance on MAG is due to MAG's lower textual richness and stronger structural signals. This disparity aligns with the distribution analysis presented by~\citet{wu2024stark} and supports our hypothesis that queries in different TG-KB datasets require varying desires for textual and structural knowledge. Meanwhile, structural retrieval methods such as conventional knowledge graph-based ones perform poorly because they are designed for graphs with minimal textual information compared to TG-KBs.
Different from Amazon and MAG, all existing methods without supervised tuning (e.g., Ada-002) exhibit significantly lower performance on Prime. This is due to the extreme domain expertise required in biology, where word-count-based, pre-trained textual similarity-based, and even more powerful LLMs are all poorly applicable here. Through fine-tuning, MFAR$^{*}$ and our proposed MoR generally achieve better performance, demonstrating the necessity of domain-specific knowledge for answering queries in knowledge-intensive domains. 




\newpage
\subsection{Ablation Study}
After verifying the superiority of MoR, we conduct ablation studies to assess its different components, including module and feature ablation.

\subsubsection{Module Ablation}


To assess the contribution of each module in MoR, namely, Text Matching-based Retrieval, Neighborhood-Fetching-based Structural Retrieval, and Reranker, we conduct a series of ablation experiments. First, we remove the Reranker, resulting in the variant MoR$_{\text{w/o R}}$. On top of that, we further separately eliminate Text Retrieval and Structural Retrieval, yielding MoR$_{\text{w/o RT}}$ and MoR$_{\text{w/o RS}}$, respectively.
As shown in Table~\ref{tab-merged}, the complete MoR framework consistently achieves the highest performance across all datasets, demonstrating the synergistic effect of the Textual Retriever, Structural Retriever, and Reranker.
After removing Reranker, MoR$_{\text{w/o R}}$ exhibits a consistent performance drop across all datasets and evaluation metrics. This underscores the importance of the Reranker in refining retrieval by filtering noisy candidates from the intermediate reasoning stage. 
Eliminating Text Retrieval, i.e., MoR$_{\text{w/o RT}}$, leads to a notable performance drop on Amazon but an unexpected improvement on MAG. This suggests that while textual knowledge benefits Amazon, it introduces misleading hard negatives that compromise the ranking method (e.g., Ada-002) for MAG. Conversely, removing Structural Retrieval, MoR$_{\text{w/o RS}}$, results in a slight performance decrease further on MAG, reinforcing the importance of structural knowledge in MAG-related queries.
%
These results underscore the Reranker's crucial role in adaptively harmonizing, balancing, and selecting knowledge from both structural and textual retrieval experts.






\begin{table}[t!]
\small
\setlength\tabcolsep{4.5pt}
\centering
\begin{tabular}{l|ccc|cccc}
\toprule
\textbf{Dataset} &\textbf{TF} & \textbf{SF} & \textbf{TI} & \textbf{H@1} & \textbf{H@5} & \textbf{R@20} & \textbf{MRR} \\ \midrule
\multirow{7}{*}{\textbf{MAG}} 
& \cmark & \xmark & \xmark & 48.96 & 73.02 & 72.44 & 59.79 \\
&      \xmark            & \cmark       &         \xmark         & 18.79 & 41.91 & 52.85 & 29.84 \\
&        \xmark          &         \xmark         & \cmark       & 18.16 & 41.53 & 52.78 & 29.31 \\
\cline{2-8}
& \cmark       & \cmark       &    \xmark              & 58.04 & 77.14 & 74.42 & 66.75 \\
& \cmark       &        \xmark          & \cmark       & \underline{58.16} & \underline{77.59} & \underline{74.96} & \underline{66.85} \\
&          \xmark        & \cmark       & \cmark       & 17.93 & 38.01 & 46.79 & 27.48 \\
\cline{2-8}
& \cmark       & \cmark       & \cmark       & \textbf{58.19} & \textbf{78.34} & \textbf{75.01} & \textbf{67.14} \\ \midrule
\multirow{7}{*}{\textbf{Amazon}}    
& \cmark       &      \xmark            &       \xmark           & \underline{51.21} & \underline{74.05} & \underline{59.79} & \underline{61.27} \\
&        \xmark          & \cmark       &      \xmark            & 8.09  & 24.48 & 25.62 & 16.94 \\
&         \xmark         &      \xmark            & \cmark       & 5.84  & 16.62 & 12.94 & 11.57 \\
\cline{2-8}
& \cmark       & \cmark       &      \xmark            & 50.91 & 73.38 & 59.58 & 61.15 \\
& \cmark       &         \xmark         & \cmark       & 51.09 & 73.56 & 59.61 & 61.14 \\
&            \xmark      & \cmark       & \cmark       & 8.09  & 24.48 & 25.62 & 16.94 \\
\cline{2-8}
& \cmark       & \cmark       & \cmark       & \textbf{52.19} & \textbf{74.65} & \textbf{59.92} & \textbf{62.24} \\ \bottomrule
\end{tabular}
\caption{Ablation study investigating the importance of three features, Textual Fingerprint (\textbf{TF}), Structural Fingerprint (\textbf{SF}), and Traversal Identifier (\textbf{TI}), of the traversal trajectories used in our Structure-aware Reranker.}
\label{tab-feature-ablation}
\vspace{-2ex}
\end{table}



\subsubsection{Feature Ablation}
The above ablation study highlights the crucial role of Structure-aware Reranker in adaptively integrating structural and textual knowledge. To further analyze the contributions of its three key features, \textbf{Textual Fingerprint (TF)}, \textbf{Structural Fingerprint (SF)}, and \textbf{Traversal Identifier (TI)} defined in Section~\ref{sec-organizing}, we conduct a feature ablation analysis and report retrieval performance across different feature configurations in Table~\ref{tab-feature-ablation}.
%Overall and individual performance
Overall, using three features together yields the best performance on both MAG and Amazon, highlighting their synergistic effect. Individually, TF contributes the most and outperforms SF and TI on both datasets. 
The reason is that based on the definition in Section~\ref{sec-organizing}, TF directly captures the relevance between the query and the retrieved nodes along the trajectory, whereas SF and TI primarily characterize the structural patterns and retrieval types, serving more as complementary factors. Therefore, equipping TF with these complementary factors (i.e., SF or TI) yields around 10\% additional gains on MAG. This is because SF and TI help the reranker selectively emphasize the relevance scores given by TF for certain nodes along the path. However, this boost is not observed on Amazon. We hypothesize that the textual knowledge needed there is predominantly derived from the final node on each path, making the structural cues provided by SF and TI less beneficial and even prone to overfitting. A deeper analysis to further justify this hypothesis is in Section~\ref{sec-further}. Overall, these findings underscore the varying importance of structural features in ranking across datasets.



\begin{table}[t!]
\small
\setlength\tabcolsep{4pt}
\centering
\begin{tabular}{l|ccc|ccc}
\toprule
\multirow{2}{*}{\textbf{Feature}} & \multicolumn{3}{c|}{\textbf{MAG}} & \multicolumn{3}{c}{\textbf{Amazon}} \\

 & H@1 & R@20 & MRR & H@1 & R@20 & MRR \\
\midrule
Last Node & 49.91 & 73.49 & 59.92 & 50.36 & 59.62 & 61.05   \\
Full Path & \textbf{58.19} & \textbf{75.01} & \textbf{67.14} & \textbf{52.19} & \textbf{59.92} & \textbf{62.24}   \\
\bottomrule
\end{tabular}
\caption{Comparing reranking performance using last node in the retrieved trajectory and the whole trajectory.}
\label{tab-Reranker-ablation}
\vspace{-2ex}
\end{table}

\begin{figure}[t!]
    \centering
    \includegraphics[width=0.49\textwidth, height = 0.22\textwidth]{figures/query-pattern-20250215.png}
    \vspace{-4.5ex}
    \caption{Imbalance number of queries and performance of different retrievers across different logical structures.}
    \label{fig-analysis}
    \vspace{-3ex}
\end{figure}





\subsection{Further Analysis}\label{sec-further}
This section understands MoR’s behavior by examining three questions, each of which enriches our insight into MoR’s functionality and offers novel perspectives inspiring future query retrieval research.

\textbf{Do structure signals affect reranking?}
To assess the impact of trajectory information on the Reranker's decision-making, we introduce a node-based Reranker that constructs trajectory features using only TF/SF/TI of the last node. In Table~\ref{tab-Reranker-ablation}, the path-based Reranker outperforms the node-based variant, especially on MAG. This highlights the critical role of trajectory features/structural knowledge in reranking. The minor performance boost on Amazon after switching to the full path trajectory indicates its textual knowledge preference over the last node rather than the whole trajectory.


\textbf{How does MoR perform on different logical structures?}
Figure~\ref{fig-analysis} shows the average performance of MoR on each query group categorized by their logical structures, where "Others" refer to queries with undefined logical structures in~\citet{wu2024stark} MoR consistently outperforms structural and textual retrievers across different logical structures. Among all queries, MoR performs the worst on "P → P" queries due to the ambiguity, although well-known, uniquely caused by repeated product entities from multi-step traversal.
The average-performing ``Others" group underscores the utility of diverse planning strategies for the same query.
Lastly, the skewed query distribution and retrieval performance across planning patterns reflect the varying nature of real-world planning needs. We hope these insights inspire research on data-centric reasoning designs and error control of planning.


\begin{figure}[t!]
    \centering
    \includegraphics[width=0.5\textwidth]{figures/heatmap-20250215.pdf}
    \vspace{-3ex}
    \caption{Saliency map visualization of query attention over three entities along the retrieved paths}
    \label{fig-map}
    \vspace{-2ex}
\end{figure}

\textbf{Does MoR indeed adaptively leverage the trajectory knowledge?} To understand how our proposed reranker prioritizes candidates in the Top-K results, we visualize the saliency map by computing the gradient of ranking scores with respect to the textual fingerprint (TF) of three nodes along the traversed path, which quantifies their importance for answering a given query. Figure~\ref{fig-map} illustrates this by analyzing trajectories for 100 ground-truth candidates across 100 queries on the Amazon and MAG datasets. Each dimension corresponds to a traversed node, with the final one representing the candidate itself. 
While the saliency score is concentrated in the last dimension for Amazon, 
MAG exhibits a more evenly distributed saliency pattern, where multiple nodes along the path contribute significantly to ranking score computation. This suggests that structural knowledge is more critical for answering queries in MAG, aligning with the previously observed lower performance of purely textual retrieval on MAG in Table~\ref{tab-merged}. Further case studies explain why the reranker attends different nodes for different queries. In Figure~\ref{fig-map}(a), the reranker favors the last two dimensions as the rich textual restriction (i.e., "Northwest Company..." and "NFL Seattle...") aids in identifying the correct node at the corresponding reasoning step, as discussed in Section~\ref{sec-reasoning}. The correct nodes, having higher similarity scores with the query, help guide the retrieval process toward the ground truth.
Conversely, in Figure~\ref{fig-map}(b),
since the first node ("University of Lausanne") helps narrow the search space and the last node ("frameless...") further filter candidates, both nodes have high saliency scores. Overall, our findings demonstrate that the reranker dynamically adapts its reliance on structural and textual knowledge depending on the dataset and query. 


\section{Conclusion}
\label{conclusion}
Software development is increasingly conceived as a collaboration activity between developers and AIs. Indeed, IDEs already implement features to enable interactive development, with AI suggesting implementations that are reused by developers.

Although multiple studies show this interaction can be successful, there is still limited understanding of how the models must be configured and used in the context of code generation tasks. This study addresses this gap, systematically investigating the impact of several key parameters, including the repeated submission of a prompt to accommodate for the non-deterministic nature of the models.

Our study reveals several key findings about the usage of ChatGPT. In particular, we discovered how creativity, although up to a limited extent, is useful to increase the range of methods whose code can be generated correctly. A major role is played by parameter top-p, which is commonly underrated, and instead has a major impact on the correctness of the results, with lower values producing better results. Finally, prompts should be submitted multiple times, with $5$ repetitions combined with a temperature of $1.2$ resulting in an effective configuration in our experiments.  

Future work concerns two main research directions. One is about replicating this experiment with other AI assistants, to validate our findings in multiple contexts. The second research direction concerns finding strategies to deal with the need to submit the same prompt multiple times to obtain a useful result, and thus developing approaches able to select or merge multiple responses automatically. 

\section*{Acknowledgments}
This work has been funded in part by NSF, with award numbers \#1826967, \#1911095, \#2003279, \#2052809, \#2100237, \#2112167, \#2112665, and in part by PRISM and CoCoSys, centers in JUMP 2.0, an SRC program sponsored by DARPA.

\newpage
\bibliographystyle{ieeetr}
\bibliography{bibfile}


\end{document}