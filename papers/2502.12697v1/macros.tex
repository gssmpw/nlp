% =============================================================================
% Packages
% =============================================================================

\usepackage[utf8]{inputenc}					% To write accents more easily

\usepackage[margin=1in]{geometry}		% To set the margins

\usepackage{amsmath,amsthm,amsfonts}		% Math features
\usepackage{dsfont}
%\usepackage{bbold}
\usepackage{mathtools}
\usepackage[usenames,dvipsnames]{xcolor}

\usepackage[inline, shortlabels]{enumitem}  %enumerate (Isa)

%%%authors

%\usepackage{authblk}



\usepackage[linesnumbered,ruled,vlined]{algorithm2e}
                                            % To write algorithms
\RestyleAlgo{boxruled}
\newcommand{\alginput}[1]{{\bf Input}: #1 \\ \vspace{3mm}}
\newcommand{\algoutput}[1]{\vspace{3mm} {\bf Output}: #1}

\usepackage{graphicx}						% To include images

\usepackage{hyperref}						% To add hyperlinks and choose their colors
\hypersetup{
	hidelinks,
    colorlinks = true,
    linkcolor = MidnightBlue,
    citecolor = MidnightBlue
}

\usepackage[noabbrev, capitalize,nameinlink]{cleveref}			% To improve cross references

\makeatletter
\newtheorem*{rep@theorem}{\rep@title}
\newcommand{\newreptheorem}[2]{%
\newenvironment{rep#1}[1]{%
 \def\rep@title{#2 \ref{##1} (restated)}%
 \begin{rep@theorem}}%
 {\end{rep@theorem}}}
\makeatother

\newtheorem{theorem}{Theorem}
\newreptheorem{theorem}{Theorem}

\newtheorem{lemma}[theorem]{Lemma}

\newtheorem{claim}[theorem]{Claim}

\newtheorem{proposition}[theorem]{Proposition}
\newtheorem{corollary}[theorem]{Corollary}
\newtheorem{observation}[theorem]{Observation}
\theoremstyle{definition}
\newtheorem{definition}[theorem]{Definition}
\newtheorem*{remark}{Remark}

\crefname{claim}{Claim}{Claims}
\crefname{observation}{Observation}{Observations}
\crefname{equation}{Eq.}{Eqs.}


% =============================================================================
% General math macros
% =============================================================================

\DeclareMathOperator{\bbE}{\mathbb{E}}
\DeclareMathOperator{\bbP}{\mathbb{P}}
\DeclareMathOperator{\bbN}{\mathbb{N}}
\DeclareMathOperator{\bbR}{\mathbb{R}}

\newcommand{\bbOne}{\mathds{1}}
\newcommand{\pa}[1]{\left( #1 \right)}

\newcommand{\floor}[1]{\left\lfloor #1 \right\rfloor}
\newcommand{\ceil}[1]{\left\lceil #1 \right\rceil}

\newcommand*{\medcap}{\mathbin{\scalebox{1.5}{\ensuremath{\cap}}}}
\newcommand{\overbar}[1]{\mkern 1.5mu\overline{\mkern-1.5mu#1\mkern-1.5mu}\mkern 1.5mu}

\newcommand{\argmin}{\textrm{argmin}}
\newcommand{\argmax}{\textrm{argmax}}

\renewcommand{\Pr}{\mathbb{P}}
\newcommand{\E}{{\rm I\kern-.3em E}}
\newcommand{\Var}{\mathrm{Var}}
\newcommand{\Cov}{\mathrm{Cov}}
\newcommand{\polylog}{\mathrm{polylog}}
\newcommand{\poly}{\mathrm{poly}}
\newcommand{\unif}{\mathrm{Uniform}}
\newcommand{\binomial}{\mathrm{Binomial}}
\newcommand{\dis}{\mathrm{dis}}





