\section{Related Works}
\label{sec:related_works}


\paragraph{Other algorithms for Leader Election in the Beeping model.}

Other works typically consider
%Almost all  other works consider
a stronger definition of Leader Election. They often enforce a {\em safety} condition, ensuring that the population never contains more than one node in a leader state. Moreover, they require nodes to {\em terminate}, i.e., to commit on their final state. Some works also require nodes to detect when a final configuration has been reached, and even the identity of the elected leader.

Importantly however, these works do not attempt to minimize the number of states available to the agents, with the notable exception of~\cite{gilbert_computational_2015}, which is discussed separately below. To the best of our knowledge, all others rely on storing integers as large as~$\Theta(\log n)$, resulting in~$\Theta(n)$ different states. This relatively large amount of memory allows for more sophisticated algorithms, and is crucial to satisfying the stronger requirements listed above.
In terms of running time, the algorithm in~\cite{GhaffariH13} converges in~$O(D + \log n \log \log n) \cdot \min \{ \log \log n, \log \frac{n}{D} \}$ rounds with high probability.
The same paper states a~$\Omega(D + \log n)$ lower bound, applicable to both deterministic and probabilistic protocols.
A subsequent work~\cite{ForsterSW14} presents a deterministic algorithm converging in~$O(D \log n)$ rounds.
Since then, more efficient algorithms (both random and deterministic) have been identified that reach the lower bound: for example, the protocol in ~\cite{DufoulonBB18} converges in~$O(D + \log n)$ rounds, and the one in~\cite{CzumajD19} runs in time proportional to the broadcasting time. A comparison of the performance of these algorithms and the protocol presented in this paper is provided in \Cref{tab:result}.

In contrast, the protocols in~\cite{gilbert_computational_2015} follow the same simplicity criteria considered here, while addressing a stronger variant of the problem than ours. Specifically, they enforce the aforementioned safety condition, and also prohibit nodes from leaving the leader state. However, unlike our setting, they allow for an error probability~$\varepsilon$, which affects both their convergence time and the number of states used by a factor of approximately~$\log \frac{1}{\varepsilon}$. Moreover, these protocols are limited to single-hop (i.e., fully connected) networks.
Our work can be seen as an extension to arbitrary graphs, using a somewhat simpler, albeit less efficient, algorithm.

\paragraph{Radio Networks and the Stone Age model.} Leader Election has also received some attention within Radio Networks~\cite{chlamtac_broadcasting_1985} and the Stone Age model~\cite{emek_stone_2013}, which are closely related to the Beeping model.
In Radio Networks, messages are usually made up of~$\Theta(\log n)$ bits. However, they are received by a node only when exactly one neighbor is emitting in a given round; when several neighbors are emitting simultaneously, the content of the message is lost, and the collision might or might not be detected depending on the setting.
The leader election problem itself is discussed in, e.g.,~\cite{chlebus_electing_2012,kowalski_leader_2009}.
In the Stone Age model, nodes are activated asynchronously, and communicate by displaying messages within an alphabet of finite size. When activated, a node~$u$ can count the number of neighbors holding the same message~$\sigma$, but only up to some threshold~$b \geq 1$.
Importantly, both models allow nodes to accurately detect the situation where a single neighbor emits a signal (unless~$b=1$), which significantly impacts algorithm design.

\paragraph{Population protocols.} 

Leader election has been studied in the Population Protocols model~\cite{AspnesR09}. In the classical definition of this model, an edge is chosen uniformly at random at each time step, after which the corresponding nodes interact and may update their states.
As in this paper, leader election in Population Protocols is typically understood as eventual, meaning that detecting termination is not required. Most works assume that the communication graph is fully connected.
On the clique, leader election requires at least $\Omega(n^2)$ expected interactions with constantly many states~\cite{DotyS18},
$\Omega(n^2/\mathrm{polylog}(n))$ expected interactions with~$O(\log \log n)$ states~\cite{AlistarhAEGR17}, and~$\Omega(n \log n)$ when the number of states is unbounded~\cite{SudoM20}. The protocol in~\cite{BerenbrinkGK20} achieves optimal space and time complexity, using $O(\log \log n)$ states and $O(n \log n)$ expected interactions.
For arbitrary graphs, a $O(\log^2 n)$ state algorithm has been recently identified~\cite{alistarh_nearoptimal_2022}, whose convergence time is~$O(\text{Broadcast time} \cdot \log n)$.
Overall, although the Beeping Model and Population Protocols share similar motivations, they exhibit significant differences that make it difficult to compare convergence times across the two settings.


%\inote{Leader Election has also been studied in the population protocol model \cite{AspnesR09}. In the classical definition of this model, a system of $n$ agents operates such that, at each time step, a randomly chosen pair interacts. It has been proven that leader election in population protocols with a constant number of states requires at least $\Omega(n^2)$ expected interactions \cite{DotyS18}. This lower bound was later generalized in \cite{AlistarhAEGR17}, where the authors demonstrated that any population protocol with  $O(\log \log n)$ states requires $\Omega(n^2/\mathrm{polylog}(n))$ expected interactions to solve leader election. Furthermore, \cite{SudoM20} establishes that for any number of states, any population protocol electing a leader requires at least $\Omega(n \log n)$ interactions, both in expectation and w.h.p. On the other hand, \cite{BerenbrinkGK20} presents a protocol that achieves optimal space and time complexity, using $O(\log \log n)$ states and $O(n \log n)$ expected interactions. All these works share the same assumption as ours: they consider the variant of eventual leader election, meaning the protocol does not require termination detection.}


%\rnote{Copying your notes: (I moved other notes in DRAFT-RELATED-WORKS.tex)
%\begin{itemize}
%    \item \cite{BerenbrinkGK20} eventual leader election in population protocol, $O( n \log n)$ expected interactions and $\Theta(\log \log n)$ states (it is optimal).

%    \item \cite{FischerJ06} self-stabilizing uniform leader election in population protocols (finite-state automata) and anonymous networks, but works in complete graphs and rings.

%    \item \cite{AlistarhG15} in population protocol, it says also `` and it has recently been shown that no population protocol using a constant number of states per node may overcome this linear bound'' (may be worth to look at it)

%    \item \cite{ChenC19} self-stabilizing leader election in population protocol (finite-state) on rings.
%\end{itemize}
%}


\begin{table}[]
\centering
\caption{Overview of existing results regarding Leader Election in the Beeping model.}
\label{tab:result}
\begin{tabular}{c|c|c|c|c|c|c}
Ref.                                  & Round Complexity                                                                                                   & Unique IDs & Knowledge & Safety & States &\begin{tabular}[c]{@{}c@{}}Termination \\ Detection\end{tabular} \\ \hline
\cite{GhaffariH13}   & \begin{tabular}[c]{@{}c@{}}$O(D+ \log n \log \log n)$\\ $\cdot \min\{\log \log n, \log \frac{n}{D}\}$\end{tabular} & yes        & $n,D$     & w.h.p. & $\Omega(\log n)$ & yes 
\\ \cite{ForsterSW14} & $O(D \log n)$ & yes & none & det. & $\Omega(\log n)$ & yes

\\
\cite{DufoulonBB18}  & $O(D + \log n)$                                                                                                    & yes        & none      & det.   & $\Omega(\log n)$ & yes                                                              \\
\cite{DufoulonBB18} & $O(D+\log n)$                                                                                                      & no         & $n$       & w.h.p. & $\Omega(\log n)$ & yes                                                              \\
\textbf{This Paper}                            & $O(D^2 \log n)$                                                                                                    & no         & none      & w.h.p. & $O(1)$ & no                                                               \\
\textbf{This Paper}                            & $O(D \log n)$                                                                                                      & no         & $D$       & w.h.p. & $O(1)$ & no\tablefootnote{In this case, assuming the additional knowledge of~$n$, the algorithm could stop after~$\Omega(D \log n)$ rounds to achieve termination detection w.h.p.}                                                              
\end{tabular}
\end{table}
%\paragraph{SIR}