{
% \begin{figure}[th]
\begin{figure*}[t]   % changed to top
\begin{center}
\centerline{\includegraphics[width=\textwidth]{Figures/Analogy.pdf}}
\mycap{Gen-AI Workflows Tuning Methods.}
{
%Top row represents gen-AI workflows and their optimization (training) steps; bottom row represents ML models.
SLM and LLM represent different language models (\eg, small and large). Code, Tool, and Data represent code blocks, tool calls, and data retrieval. Dash curved lines represent loops or control flow changes. p* represents prompt optimizations or additional information added for the downstream step.
%\jingbo{I like this analogy because it makes the concept easier to understand. But it also introduces a question -- what is the different between AutoML and this work? Why AutoML methods cannot be applied/adapted here?}
}
\Description{Four graphs with illustrated optimization methods applied.}
\label{fig-analogy}
\end{center}
\end{figure*}
}