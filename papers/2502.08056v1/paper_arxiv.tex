%%
%% This is file `sample-sigconf.tex',
%% generated with the docstrip utility.
%%
%% The original source files were:
%%
%% samples.dtx  (with options: `all,proceedings,bibtex,sigconf')
%% 
%% IMPORTANT NOTICE:
%% 
%% For the copyright see the source file.
%% 
%% Any modified versions of this file must be renamed
%% with new filenames distinct from sample-sigconf.tex.
%% 
%% For distribution of the original source see the terms
%% for copying and modification in the file samples.dtx.
%% 
%% This generated file may be distributed as long as the
%% original source files, as listed above, are part of the
%% same distribution. (The sources need not necessarily be
%% in the same archive or directory.)
%%
%%
%% Commands for TeXCount
%TC:macro \cite [option:text,text]
%TC:macro \citep [option:text,text]
%TC:macro \citet [option:text,text]
%TC:envir table 0 1
%TC:envir table* 0 1
%TC:envir tabular [ignore] word
%TC:envir displaymath 0 word
%TC:envir math 0 word
%TC:envir comment 0 0
%%
%% The first command in your LaTeX source must be the \documentclass
%% command.
%%
%% For submission and review of your manuscript please change the
%% command to \documentclass[manuscript, screen, review]{acmart}.
%%
%% When submitting camera ready or to TAPS, please change the command
%% to \documentclass[sigconf]{acmart} or whichever template is required
%% for your publication.
%%
%%
\documentclass[sigconf]{acmarxiv}

\usepackage{algorithm}
\usepackage{algorithmic}
\usepackage{pifont}
\usepackage{listings} % Import listings package
\usepackage{xcolor} % For syntax highlighting
\usepackage{caption}
\usepackage{mdframed}


\captionsetup[figure]{justification=raggedright, singlelinecheck=false}
\captionsetup[table]{justification=centering}




\DeclareMathOperator*{\argmin}{argmin}
\DeclareMathOperator*{\argmax}{argmax}

\definecolor{deepgreen}{rgb}{0,0.5,0}

\lstdefinestyle{pythonstyle}{
  language=Python,
  basicstyle=\small \ttfamily,
  keywordstyle=\color{blue},
  commentstyle=\color{gray},
  stringstyle=\color{deepgreen},
  breaklines=true,
  frame=single,
  tabsize=2,
  showstringspaces=false
}
%%
%% \BibTeX command to typese
% t BibTeX logo in the docs
\AtBeginDocument{%
  \providecommand\BibTeX{{%
    Bib\TeX}}}

%% Rights management information.  This information is sent to you
%% when you complete the rights form.  These commands have SAMPLE
%% values in them; it is your responsibility as an author to replace
%% the commands and values with those provided to you when you
%% complete the rights form.
\setcopyright{acmlicensed}
\copyrightyear{2025}
\acmYear{2025}
\acmDOI{XXXXXXX.XXXXXXX}
% %% These commands are for a PROCEEDINGS abstract or paper.
% \acmConference[Under Review for KDD '25]{Under review for International Conference on Knowledge Discovery and Data Mining 2025}{August 03--07,
%   2025}{Toronto, Canada}
\acmConference[Preprint]{Under review for International Conference on Knowledge Discovery and Data Mining 2025}{Under Review}{© 2025}
% %%
% %%  Uncomment \acmBooktitle if the title of the proceedings is different
% %%  from ``Proceedings of ...''!
% %%
% %%\acmBooktitle{Woodstock '18: ACM Symposium on Neural Gaze Detection,
% %%  June 03--05, 2018, Woodstock, NY}
% \acmISBN{978-X-XXXX-XXXX-X/2025/XX}


%%
%% Submission ID.
%% Use this when submitting an article to a sponsored event. You'll
%% receive a unique submission ID from the organizers
%% of the event, and this ID should be used as the parameter to this command.
%%\acmSubmissionID{123-A56-BU3}

%%
%% For managing citations, it is recommended to use bibliography
%% files in BibTeX format.
%%
%% You can then either use BibTeX with the ACM-Reference-Format style,
%% or BibLaTeX with the acmnumeric or acmauthoryear sytles, that include
%% support for advanced citation of software artefact from the
%% biblatex-software package, also separately available on CTAN.
%%
%% Look at the sample-*-biblatex.tex files for templates showcasing
%% the biblatex styles.
%%

%%
%% The majority of ACM publications use numbered citations and
%% references.  The command \citestyle{authoryear} switches to the
%% "author year" style.
%%
%% If you are preparing content for an event
%% sponsored by ACM SIGGRAPH, you must use the "author year" style of
%% citations and references.
%% Uncommenting
%% the next command will enable that style.
%%\citestyle{acmauthoryear}


%%
%% end of the preamble, start of the body of the document source.
\begin{document}

%%
%% The "title" command has an optional parameter,
%% allowing the author to define a "short title" to be used in page headers.
\title[Cognify: Supercharging Gen-AI Workflows With Hierarchical Autotuning]{Cognify: Supercharging Gen-AI Workflows With Hierarchical Autotuning}

%%
%% The "author" command and its associated commands are used to define
%% the authors and their affiliations.
%% Of note is the shared affiliation of the first two authors, and the
%% "authornote" and "authornotemark" commands
%% used to denote shared contribution to the research.
\author{Zijian He}
\authornote{Both authors contributed equally to this research.}
\email{zih015@ucsd.edu}
\author{Reyna Abhyankar}
\authornotemark[1]
\email{vabhyank@ucsd.edu}
\affiliation{%
  \institution{UC San Diego}
  \city{La Jolla}
  \state{CA}
  \country{USA}
}


\author{Vikranth Srivatsa}
\affiliation{%
  \institution{UC San Diego}
  \city{La Jolla}
  \state{CA}
  \country{USA}
}
\email{vsrivatsa@ucsd.edu}

\author{Yiying Zhang}
\affiliation{
  \institution{UC San Diego, GenseeAI, Inc.}
  \city{La Jolla}
  \state{CA}
  \country{USA}
}
\email{yiying@gensee.ai}



%%
%% By default, the full list of authors will be used in the page
%% headers. Often, this list is too long, and will overlap
%% other information printed in the page headers. This command allows
%% the author to define a more concise list
%% of authors' names for this purpose.
\renewcommand{\shortauthors}{He*, Abhyankar*, Srivatsa, Zhang}

\renewcommand{\ttdefault}{lmtt}

\newcommand{\x}{$\times$}
\newcommand{\eg}{e.g.}
\newcommand{\ie}{i.e.}
\newcommand{\sysname}{Cognify}
\newcommand{\search}{\texttt{AdaSeek}}
\newcommand{\fixme}[1]{{\color{red}\textbf{\fbox{FIXME} #1}}}
\newcommand{\FIXME}[1]{{\color{red}\textbf{\fbox{FIXME} #1}}}

\newcommand{\mycap}[2]{\caption{{\bf #1} \textnormal{#2}}}

%%

%% The abstract is a short summary of the work to be presented in the
%% article.

\sloppy

\begin{abstract}  
Test time scaling is currently one of the most active research areas that shows promise after training time scaling has reached its limits.
Deep-thinking (DT) models are a class of recurrent models that can perform easy-to-hard generalization by assigning more compute to harder test samples.
However, due to their inability to determine the complexity of a test sample, DT models have to use a large amount of computation for both easy and hard test samples.
Excessive test time computation is wasteful and can cause the ``overthinking'' problem where more test time computation leads to worse results.
In this paper, we introduce a test time training method for determining the optimal amount of computation needed for each sample during test time.
We also propose Conv-LiGRU, a novel recurrent architecture for efficient and robust visual reasoning. 
Extensive experiments demonstrate that Conv-LiGRU is more stable than DT, effectively mitigates the ``overthinking'' phenomenon, and achieves superior accuracy.
\end{abstract}  
%%
%% The code below is generated by the tool at http://dl.acm.org/ccs.cfm.
%% Please copy and paste the code instead of the example below.
%%
\begin{CCSXML}
<ccs2012>
   <concept>
       <concept_id>10010147.10010257.10010293.10010294</concept_id>
       <concept_desc>Computing methodologies~Neural networks</concept_desc>
       <concept_significance>500</concept_significance>
       </concept>
   <concept>
       <concept_id>10010147.10010178.10010205.10010207</concept_id>
       <concept_desc>Computing methodologies~Discrete space search</concept_desc>
       <concept_significance>500</concept_significance>
       </concept>
 </ccs2012>
\end{CCSXML}

\ccsdesc[500]{Computing methodologies~Neural networks}
\ccsdesc[500]{Computing methodologies~Discrete space search}


%%
%% Keywords. The author(s) should pick words that accurately describe
%% the work being presented. Separate the keywords with commas.
\keywords{Agentic Workflows, Gen-AI Workflows, Optimization, LLM, Bayesian Optimization, Test-time Scaling}
%% A "teaser" image appears between the author and affiliation
%% information and the body of the document, and typically spans the
%% page.

%%
%% This command processes the author and affiliation and title
%% information and builds the first part of the formatted document.

\maketitle

{
% \begin{figure}[th]
\begin{figure*}[t]   % changed to top
\begin{center}
\centerline{\includegraphics[width=\textwidth]{Figures/Analogy.pdf}}
\mycap{Gen-AI Workflows Tuning Methods.}
{
%Top row represents gen-AI workflows and their optimization (training) steps; bottom row represents ML models.
SLM and LLM represent different language models (\eg, small and large). Code, Tool, and Data represent code blocks, tool calls, and data retrieval. Dash curved lines represent loops or control flow changes. p* represents prompt optimizations or additional information added for the downstream step.
%\jingbo{I like this analogy because it makes the concept easier to understand. But it also introduces a question -- what is the different between AutoML and this work? Why AutoML methods cannot be applied/adapted here?}
}
\Description{Four graphs with illustrated optimization methods applied.}
\label{fig-analogy}
\end{center}
\end{figure*}
}
\section{Introduction}
\label{sec:introduction}
The business processes of organizations are experiencing ever-increasing complexity due to the large amount of data, high number of users, and high-tech devices involved \cite{martin2021pmopportunitieschallenges, beerepoot2023biggestbpmproblems}. This complexity may cause business processes to deviate from normal control flow due to unforeseen and disruptive anomalies \cite{adams2023proceddsriftdetection}. These control-flow anomalies manifest as unknown, skipped, and wrongly-ordered activities in the traces of event logs monitored from the execution of business processes \cite{ko2023adsystematicreview}. For the sake of clarity, let us consider an illustrative example of such anomalies. Figure \ref{FP_ANOMALIES} shows a so-called event log footprint, which captures the control flow relations of four activities of a hypothetical event log. In particular, this footprint captures the control-flow relations between activities \texttt{a}, \texttt{b}, \texttt{c} and \texttt{d}. These are the causal ($\rightarrow$) relation, concurrent ($\parallel$) relation, and other ($\#$) relations such as exclusivity or non-local dependency \cite{aalst2022pmhandbook}. In addition, on the right are six traces, of which five exhibit skipped, wrongly-ordered and unknown control-flow anomalies. For example, $\langle$\texttt{a b d}$\rangle$ has a skipped activity, which is \texttt{c}. Because of this skipped activity, the control-flow relation \texttt{b}$\,\#\,$\texttt{d} is violated, since \texttt{d} directly follows \texttt{b} in the anomalous trace.
\begin{figure}[!t]
\centering
\includegraphics[width=0.9\columnwidth]{images/FP_ANOMALIES.png}
\caption{An example event log footprint with six traces, of which five exhibit control-flow anomalies.}
\label{FP_ANOMALIES}
\end{figure}

\subsection{Control-flow anomaly detection}
Control-flow anomaly detection techniques aim to characterize the normal control flow from event logs and verify whether these deviations occur in new event logs \cite{ko2023adsystematicreview}. To develop control-flow anomaly detection techniques, \revision{process mining} has seen widespread adoption owing to process discovery and \revision{conformance checking}. On the one hand, process discovery is a set of algorithms that encode control-flow relations as a set of model elements and constraints according to a given modeling formalism \cite{aalst2022pmhandbook}; hereafter, we refer to the Petri net, a widespread modeling formalism. On the other hand, \revision{conformance checking} is an explainable set of algorithms that allows linking any deviations with the reference Petri net and providing the fitness measure, namely a measure of how much the Petri net fits the new event log \cite{aalst2022pmhandbook}. Many control-flow anomaly detection techniques based on \revision{conformance checking} (hereafter, \revision{conformance checking}-based techniques) use the fitness measure to determine whether an event log is anomalous \cite{bezerra2009pmad, bezerra2013adlogspais, myers2018icsadpm, pecchia2020applicationfailuresanalysispm}. 

The scientific literature also includes many \revision{conformance checking}-independent techniques for control-flow anomaly detection that combine specific types of trace encodings with machine/deep learning \cite{ko2023adsystematicreview, tavares2023pmtraceencoding}. Whereas these techniques are very effective, their explainability is challenging due to both the type of trace encoding employed and the machine/deep learning model used \cite{rawal2022trustworthyaiadvances,li2023explainablead}. Hence, in the following, we focus on the shortcomings of \revision{conformance checking}-based techniques to investigate whether it is possible to support the development of competitive control-flow anomaly detection techniques while maintaining the explainable nature of \revision{conformance checking}.
\begin{figure}[!t]
\centering
\includegraphics[width=\columnwidth]{images/HIGH_LEVEL_VIEW.png}
\caption{A high-level view of the proposed framework for combining \revision{process mining}-based feature extraction with dimensionality reduction for control-flow anomaly detection.}
\label{HIGH_LEVEL_VIEW}
\end{figure}

\subsection{Shortcomings of \revision{conformance checking}-based techniques}
Unfortunately, the detection effectiveness of \revision{conformance checking}-based techniques is affected by noisy data and low-quality Petri nets, which may be due to human errors in the modeling process or representational bias of process discovery algorithms \cite{bezerra2013adlogspais, pecchia2020applicationfailuresanalysispm, aalst2016pm}. Specifically, on the one hand, noisy data may introduce infrequent and deceptive control-flow relations that may result in inconsistent fitness measures, whereas, on the other hand, checking event logs against a low-quality Petri net could lead to an unreliable distribution of fitness measures. Nonetheless, such Petri nets can still be used as references to obtain insightful information for \revision{process mining}-based feature extraction, supporting the development of competitive and explainable \revision{conformance checking}-based techniques for control-flow anomaly detection despite the problems above. For example, a few works outline that token-based \revision{conformance checking} can be used for \revision{process mining}-based feature extraction to build tabular data and develop effective \revision{conformance checking}-based techniques for control-flow anomaly detection \cite{singh2022lapmsh, debenedictis2023dtadiiot}. However, to the best of our knowledge, the scientific literature lacks a structured proposal for \revision{process mining}-based feature extraction using the state-of-the-art \revision{conformance checking} variant, namely alignment-based \revision{conformance checking}.

\subsection{Contributions}
We propose a novel \revision{process mining}-based feature extraction approach with alignment-based \revision{conformance checking}. This variant aligns the deviating control flow with a reference Petri net; the resulting alignment can be inspected to extract additional statistics such as the number of times a given activity caused mismatches \cite{aalst2022pmhandbook}. We integrate this approach into a flexible and explainable framework for developing techniques for control-flow anomaly detection. The framework combines \revision{process mining}-based feature extraction and dimensionality reduction to handle high-dimensional feature sets, achieve detection effectiveness, and support explainability. Notably, in addition to our proposed \revision{process mining}-based feature extraction approach, the framework allows employing other approaches, enabling a fair comparison of multiple \revision{conformance checking}-based and \revision{conformance checking}-independent techniques for control-flow anomaly detection. Figure \ref{HIGH_LEVEL_VIEW} shows a high-level view of the framework. Business processes are monitored, and event logs obtained from the database of information systems. Subsequently, \revision{process mining}-based feature extraction is applied to these event logs and tabular data input to dimensionality reduction to identify control-flow anomalies. We apply several \revision{conformance checking}-based and \revision{conformance checking}-independent framework techniques to publicly available datasets, simulated data of a case study from railways, and real-world data of a case study from healthcare. We show that the framework techniques implementing our approach outperform the baseline \revision{conformance checking}-based techniques while maintaining the explainable nature of \revision{conformance checking}.

In summary, the contributions of this paper are as follows.
\begin{itemize}
    \item{
        A novel \revision{process mining}-based feature extraction approach to support the development of competitive and explainable \revision{conformance checking}-based techniques for control-flow anomaly detection.
    }
    \item{
        A flexible and explainable framework for developing techniques for control-flow anomaly detection using \revision{process mining}-based feature extraction and dimensionality reduction.
    }
    \item{
        Application to synthetic and real-world datasets of several \revision{conformance checking}-based and \revision{conformance checking}-independent framework techniques, evaluating their detection effectiveness and explainability.
    }
\end{itemize}

The rest of the paper is organized as follows.
\begin{itemize}
    \item Section \ref{sec:related_work} reviews the existing techniques for control-flow anomaly detection, categorizing them into \revision{conformance checking}-based and \revision{conformance checking}-independent techniques.
    \item Section \ref{sec:abccfe} provides the preliminaries of \revision{process mining} to establish the notation used throughout the paper, and delves into the details of the proposed \revision{process mining}-based feature extraction approach with alignment-based \revision{conformance checking}.
    \item Section \ref{sec:framework} describes the framework for developing \revision{conformance checking}-based and \revision{conformance checking}-independent techniques for control-flow anomaly detection that combine \revision{process mining}-based feature extraction and dimensionality reduction.
    \item Section \ref{sec:evaluation} presents the experiments conducted with multiple framework and baseline techniques using data from publicly available datasets and case studies.
    \item Section \ref{sec:conclusions} draws the conclusions and presents future work.
\end{itemize}

\section{Experiments}
\label{sec:Experiments} 

We conduct several experiments across different problem settings to assess the efficiency of our proposed method. Detailed descriptions of the experimental settings are provided in \cref{sec:apendix_experiments}.
%We conduct experiments on optimizing PINNs for convection, wave PDEs, and a reaction ODE. 
%These equations have been studied in previous works investigating difficulties in training PINNs; we use the formulations in \citet{krishnapriyan2021characterizing, wang2022when} for our experiments. 
%The coefficient settings we use for these equations are considered challenging in the literature \cite{krishnapriyan2021characterizing, wang2022when}.
%\cref{sec:problem_setup_additional} contains additional details.

%We compare the performance of Adam, \lbfgs{}, and \al{} on training PINNs for all three classes of PDEs. 
%For Adam, we tune the learning rate by a grid search on $\{10^{-5}, 10^{-4}, 10^{-3}, 10^{-2}, 10^{-1}\}$.
%For \lbfgs, we use the default learning rate $1.0$, memory size $100$, and strong Wolfe line search.
%For \al, we tune the learning rate for Adam as before, and also vary the switch from Adam to \lbfgs{} (after 1000, 11000, 31000 iterations).
%These correspond to \al{} (1k), \al{} (11k), and \al{} (31k) in our figures.
%All three methods are run for a total of 41000 iterations.

%We use multilayer perceptrons (MLPs) with tanh activations and three hidden layers. These MLPs have widths 50, 100, 200, or 400.
%We initialize these networks with the Xavier normal initialization \cite{glorot2010understanding} and all biases equal to zero.
%Each combination of PDE, optimizer, and MLP architecture is run with 5 random seeds.

%We use 10000 residual points randomly sampled from a $255 \times 100$ grid on the interior of the problem domain. 
%We use 257 equally spaced points for the initial conditions and 101 equally spaced points for each boundary condition.

%We assess the discrepancy between the PINN solution and the ground truth using $\ell_2$ relative error (L2RE), a standard metric in the PINN literature. Let $y = (y_i)_{i = 1}^n$ be the PINN prediction and $y' = (y'_i)_{i = 1}^n$ the ground truth. Define
%\begin{align*}
%    \mathrm{L2RE} = \sqrt{\frac{\sum_{i = 1}^n (y_i - y'_i)^2}{\sum_{i = 1}^n y'^2_i}} = \sqrt{\frac{\|y - y'\|_2^2}{\|y'\|_2^2}}.
%\end{align*}
%We compute the L2RE using all points in the $255 \times 100$ grid on the interior of the problem domain, along with the 257 and 101 points used for the initial and boundary conditions.

%We develop our experiments in PyTorch 2.0.0 \cite{paszke2019pytorch} with Python 3.10.12.
%Each experiment is run on a single NVIDIA Titan V GPU using CUDA 11.8.
%The code for our experiments is available at \href{https://github.com/pratikrathore8/opt_for_pinns}{https://github.com/pratikrathore8/opt\_for\_pinns}.


\subsection{2D Allen Cahn Equation}
\begin{figure*}[t]
    \centering
    \includegraphics[scale=0.38]{figs/Burgers_operator.pdf}
    \caption{1D Burgers' Equation (Operator Learning): Steady-state solutions for different initializations $u_0$ under varying viscosity $\varepsilon$: (a) $\varepsilon = 0.5$, (b) $\varepsilon = 0.1$, (c) $\varepsilon = 0.05$. The results demonstrate that all final test solutions converge to the correct steady-state solution. (d) Illustration of the evolution of a test initialization $u_0$ following homotopy dynamics. The number of residual points is $\nres = 128$.}
    \label{fig:Burgers_result}
\end{figure*}
First, we consider the following time-dependent problem:
\begin{align}
& u_t = \varepsilon^2 \Delta u - u(u^2 - 1), \quad (x, y) \in [-1, 1] \times [-1, 1] \nonumber \\
& u(x, y, 0) = - \sin(\pi x) \sin(\pi y) \label{eq.hom_2D_AC}\\
& u(-1, y, t) = u(1, y, t) = u(x, -1, t) = u(x, 1, t) = 0. \nonumber
\end{align}
We aim to find the steady-state solution for this equation with $\varepsilon = 0.05$ and define the homotopy as:
\begin{equation}
    H(u, s, \varepsilon) = (1 - s)\left(\varepsilon(s)^2 \Delta u - u(u^2 - 1)\right) + s(u - u_0),\nonumber
\end{equation}
where $s \in [0, 1]$. Specifically, when $s = 1$, the initial condition $u_0$ is automatically satisfied, and when $s = 0$, it recovers the steady-state problem. The function $\varepsilon(s)$ is given by
\begin{equation}
\varepsilon(s) = 
\left\{\begin{array}{l}
s, \quad s \in [0.05, 1], \\
0.05, \quad s \in [0, 0.05].
\end{array}\right.\label{eq:epsilon_t}
\end{equation}

Here, $\varepsilon(s)$ varies with $s$ during the first half of the evolution. Once $\varepsilon(s)$ reaches $0.05$, it remains fixed, and only $s$ continues to evolve toward $0$. As shown in \cref{fig:2D_Allen_Cahn_Equation}, the relative $L_2$ error by homotopy dynamics is $8.78 \times 10^{-3}$, compared with the result obtained by PINN, which has a $L_2$ error of $9.56 \times 10^{-1}$. This clearly demonstrates that the homotopy dynamics-based approach significantly improves accuracy.

\subsection{High Frequency Function Approximation }
We aim to approximate the following function:
$u=    \sin(50\pi x), \quad x \in [0,1].$
The homotopy is defined as $H(u,\varepsilon) = u - \sin(\frac{1}{\varepsilon}\pi x), $
where $\varepsilon \in [\frac{1}{50},\frac{1}{15}]$.

\begin{table}[htbp!]
    \caption{Comparison of the lowest loss achieved by the classical training and homotopy dynamics for different values of $\varepsilon$ in approximating $\sin\left(\frac{1}{\varepsilon} \pi x\right)$
    }
    \vskip 0.15in
    \centering
    \tiny
    \begin{tabular}{|c|c|c|c|c|} 
    \hline 
    $ $ & $\varepsilon = 1/15$ & $\varepsilon = 1/35$ & $\varepsilon = 1/50$ \\ \hline 
    Classical Loss                & 4.91e-6     & 7.21e-2     & 3.29e-1       \\ \hline 
    Homotopy Loss $L_H$                      & 1.73e-6     & 1.91e-6     & \textbf{2.82e-5}       \\ \hline
    \end{tabular}
    % On convection, \al{} provides 14.2$\times$ and 1.97$\times$ improvement over Adam or \lbfgs{} on L2RE. 
    % On reaction, \al{} provides 1.10$\times$ and 1.99$\times$ improvement over Adam or \lbfgs{} on L2RE.
    % On wave, \al{} provides 6.32$\times$ and 6.07$\times$ improvement over Adam or \lbfgs{} on L2RE.}
    \label{tab:loss_approximate}
\end{table}

As shown in \cref{fig:high_frequency_result}, due to the F-principle \cite{xu2024overview}, training is particularly challenging when approximating high-frequency functions like $\sin(50\pi x)$. The loss decreases slowly, resulting in poor approximation performance. However, training based on homotopy dynamics significantly reduces the loss, leading to a better approximation of high-frequency functions. This demonstrates that homotopy dynamics-based training can effectively facilitate convergence when approximating high-frequency data. Additionally, we compare the loss for approximating functions with different frequencies $1/\varepsilon$ using both methods. The results, presented in \cref{tab:loss_approximate}, show that the homotopy dynamics training method consistently performs well for high-frequency functions.





\subsection{Burgers Equation}
In this example, we adopt the operator learning framework to solve for the steady-state solution of the Burgers equation, given by:
\begin{align}
& u_t+\left(\frac{u^2}{2}\right)_x - \varepsilon u_{xx}=\pi \sin (\pi x) \cos (\pi x), \quad x \in[0, 1]\nonumber\\
& u(x, 0)=u_0(x),\label{eq:1D_Burgers} \\
& u(0, t)=u(1, t)=0, \nonumber 
\end{align}
with Dirichlet boundary conditions, where $u_0 \in L_{0}^2((0, 1); \mathbb{R})$ is the initial condition and $\varepsilon \in \mathbb{R}$ is the viscosity coefficient. We aim to learn the operator mapping the initial condition to the steady-state solution, $G^{\dagger}: L_{0}^2((0, 1); \mathbb{R}) \rightarrow H_{0}^r((0, 1); \mathbb{R})$, defined by $u_0 \mapsto u_{\infty}$ for any $r > 0$. As shown in Theorem 2.2 of \cite{KREISS1986161} and Theorems 2.5 and 2.7 of \cite{hao2019convergence}, for any $\varepsilon > 0$, the steady-state solution is independent of the initial condition, with a single shock occurring at $x_s = 0.5$. Here, we use DeepONet~\cite{lu2021deeponet} as the network architecture. 
The homotopy definition, similar to ~\cref{eq.hom_2D_AC}, can be found in \cref{Ap:operator}. The results can be found in \cref{fig:Burgers_result} and \cref{tab:loss_burgers}. Experimental results show that the homotopy dynamics strategy performs well in the operator learning setting as well.


\begin{table}[htbp!]
    \caption{Comparison of loss between classical training and homotopy dynamics for different values of $\varepsilon$ in the Burgers equation, along with the MSE distance to the ground truth shock location, $x_s$.}
    \vskip 0.15in
    \centering
    \tiny
    \begin{tabular}{|c|c|c|c|c|} 
    \hline  
    $ $ & $\varepsilon = 0.5$ & $\varepsilon = 0.1$ & $\varepsilon = 0.05$ \\ \hline 
    Homotopy Loss $L_H$                &  7.55e-7     & 3.40e-7     & 7.77e-7       \\ \hline 
    L2RE                      & 1.50e-3     & 7.00e-4     & 2.52e-2       \\ \hline
        MSE Distance $x_s$                      & 1.75e-8     & 9.14e-8      & 1.2e-3      \\ \hline
    \end{tabular}
    % On convection, \al{} provides 14.2$\times$ and 1.97$\times$ improvement over Adam or \lbfgs{} on L2RE. 
    % On reaction, \al{} provides 1.10$\times$ and 1.99$\times$ improvement over Adam or \lbfgs{} on L2RE.
    % On wave, \al{} provides 6.32$\times$ and 6.07$\times$ improvement over Adam or \lbfgs{} on L2RE.}
    \label{tab:loss_burgers}
\end{table}



% \begin{itemize}
%     \item Relate the curvature in the problem to the differential operator. Use this to demonstrate why the problem is ill-conditioned
%     \item Give an argument for why using Adam + L-BFGS is better than just using L-BFGS outright. The idea is that Adam lowers the errors to the point where the rest of the optimization becomes convex \ldots
%     \item Show why we need second-order methods. We would like to prove that once we are close to the optimum, second-order methods will give condition-number free linear convergence. Specialize this to the Gauss-Newton setting, with the randomized low-rank approximation.
%     % \item Show that it is not possible to get superlinear convergence under the interpolation assumption with an overparameterized neural network. This should be true b/c the Hessian at the optimum will have rank $\min(n, d)$, and when $d > n$, this guarantees that we cannot have strong convexity.
% \end{itemize}
\section{The \search\ Search Algorithm}
\label{sec:search}

%In traditional ML, structure changes and step (operator) changes are performed before model training, \ie, fixed to the training process, and weights are updated with SGD, because weights are continous, differentiable values, and there are significantly more weights than structure and operator changes. In workflow autotuning, all three types of cogs can be chosen with a unified search-based approach, because all of them are non-differentiable configurations and the number of cogs in different types are all small.
%Thus, \sysname\ only needs to navigate the search space of combination of cogs as the search space to produce its workflow optimization results.

%We propose, \textit{\textbf{\search}}, an adaptive hierarchical search algorithm that autotunes gen-AI workflows based on observed end-to-end workflow results. In each search iteration, \search\ selects a combination of cogs to apply to the workflow and executes the resulting workflow with user-provided training inputs. \search\ evaluates the final generation quality using the user-specified evaluator and measures the execution time and cost for each training input. These results are aggregated and serve as BO observations and pruning criteria.
%the optimizer can condition on and propose better configurations in later trials. The optimizer will also be informed about the violation of any user-specified metric thresholds. More details of this mechanism can be found in Appendix ~\ref{appdx:TPE}.

With our insights in Section~\ref{sec:theory}, we believe that search methods based on Bayesian Optimizer (BO) can work for all types of cogs in gen-AI workflow autotuning because of BO's efficiency in searching discrete search space.
A key challenge in designing a BO-based search is the limited search budgets that need to be used to search a high-dimensional cog space. 
For example, for 4 cogs each with 4 options and a workflow of 3 LLM steps, the search space is $4^{12}$. Suppose each search uses GPT-4o and has 1000 output tokens, the entire space needs around \$168K to go through. A user search budget of \$100 can cover only 0.06\% of the search space. A traditional BO approach cannot find good results with such small budgets.
%The entire search space grows exponentially with the number of cogs and the number of steps in a workflow. Moreover, different cogs and different combinations of cogs can have varying impacts on different workflows. 
%Without prior knowledge, it is difficult to determine the amount of budget to give to each cog.

To confront this challenge, we propose \textit{\textbf{\search}}, an adaptive hierarchical search algorithm that efficiently assigns search budget across cogs based on budget size and observed workflow evaluation results, as defined in Algorithms~\ref{alg:main} and \ref{alg:outer} and described below.
%autotunes gen-AI workflows based on observed end-to-end workflow results.
%\search\ includes a search layer partitioning method, a search budget initial assignment method, an evaluation-guided budget re-allocation mechanism, and a convergence-based early-exiting strategy. We discuss them in details below.

%\zijian{\search\ allows users to specify the optimization budget allowed in terms of the maximum number of search iterations. Based on the relationship between the complexity of the search space and the available budget, we will separate all tunable parameters into different layers each optimized by independent Bayesian optimization routines. Then we will decide the maximum budget each layer can get with a bottom-up partition strategy. Besides search space and resource partition, we also employ a novel allocation algorithm that integrates successive halving~\cite{successivehalving} and a convergence-based early exiting strategy to facilitate efficient usage of assigned budget.}


% The outermost layer searches and selects structures for a workflow; the middle layer searches and selects step options under the workflow structure selected in the outermost layer; the innermost layer searches and selects weights with the given workflow structure and steps. 

\begin{algorithm}[h]
    \caption{\search\ Algorithm}
    \label{alg:main}
      \small
\begin{algorithmic}[1]
\STATE \textbf{Global Value:} $R = \emptyset$ \COMMENT{Global result set}
%\STATE \textbf{Global Value:} $F = \emptyset$ \COMMENT{Global observation set}

%Reduct factor $\eta > 1$, explore width $W$
\STATE \textbf{Input:} User-specified Total Budget $TB$
\STATE \textbf{Input:} Cog set $C = \{c_{11},c_{12},...\}, \{c_{21},c_{22},...\}, \{c_{31},c_{32},...\}$

    \STATE
%\FOR{$i = 1,2,3$}
    %\COMMENT{$\alpha$ is a configurable value default to 1.1}
%\ENDFOR
%\STATE
%    \STATE \{$B_1,B_2,B_3$\} = LayerPartition($C$) \COMMENT{Calculate ideal layer budget}
    %\STATE \textbf{Glob}.budgets = budgets
%    \STATE opt\_layers = init\_opt\_routines() \COMMENT{A list of optimize routine each layer will use for search}
%\STATE
%    \FOR{$i \in L, \dots, 1$}
%        \IF{$i == L$}
 %           \STATE opt\_layers[L] = InnerLayerOpt
  %      \ELSE
   %         \STATE opt\_layers[i] = OuterLayerOpt
            %\STATE opt\_layers[i].next\_layer\_budgets = B[i+1]
            %\STATE opt\_layers[i].next\_layer\_routine = opt\_layers[i+1]
    %    \ENDIF
    %\ENDFOR
%\STATE opt\_layers[1].invoke($\emptyset$, B[1])
\STATE $U = 0$ \COMMENT{Used budget so far, initialize to 0}

\STATE \COMMENT{Perform search with 1 to 3 layers until budget runs out}
\FOR{$L = 1,2,3$} 
        \IF{$L=1$}
            \STATE $C_1 = C_1 \cup C_2 \cup C_3$ \COMMENT{Merge all cogs into a single layer}
        \ENDIF
        \IF{$L==2$}
            \STATE $C_1 = C_1 \cup C_2$ \COMMENT{Merge step and weight cogs}
            \STATE $C_2 = C_3$ \COMMENT{Architecture cog becomes the second layer}
        \ENDIF
        \STATE
    \FOR{$i = 1,..,L$}
    \STATE $NC_i = |C_i|$ \COMMENT{Total number of cogs in layer $L$} 
%    NO_i &= \sum_{L} \{\text{number of possible options in cog } c_{ij}\} \\
    \STATE $S_i = NC_i^\alpha$ \COMMENT{Estimated expected search size in layer $i$}
    \ENDFOR
    \STATE $E_L = \prod\limits_{i=1}^{L}S_i$ \COMMENT{Expected total search size in the current round}
    \STATE $E = TB - U > E_L$ ? $E_L$ : $(TB - U)$ \COMMENT{Consider insufficient budget} 
    \IF{$L==3$ and $(TB - U)$ > $E_L$}
         \STATE $E = TB - U$ \COMMENT{Spend all remaining budget if at 3 layer}
    \ENDIF
    %\STATE$TL = |N|$ \COMMENT{number of layers}
    \FOR{$i = 1,..,L$}
        \STATE $B_i =  \lfloor S_i \times \sqrt[L]{\frac{E}{E_L}}\rfloor$
        %$B$ = BudgetAssign($N$, $TL$, $TB$)
        \COMMENT{Assign budget proportionally to $S_i$}
    \ENDFOR
    \STATE
\STATE \texttt{LayerSearch} ($\emptyset$, $B$, $L$, $B_L$) \COMMENT{Hierarchical search from layer $L$}
\STATE
\STATE $U = U + E$
\IF{$U \geq TB$}
\STATE break \COMMENT{Stop search when using up all user budget}
\ENDIF
\ENDFOR
%\STATE
%\STATE $O$ = \texttt{SelectBestConfigs} ($R$)
%\IF{$L == 1$}
%    \STATE InnerLayerOpt($\emptyset$, B[1])
%\ELSE
%    \STATE OuterLayerOpt($\emptyset$, B[1], 1)
%\ENDIF
\STATE
\STATE \textbf{Output:} $O$ = \texttt{SelectBestConfigs} ($R$) \COMMENT{Return best optimizations}
\end{algorithmic}
\end{algorithm}

\subsection{Hierarchical Layer and Budget Partition}
\label{sec:ssp}

%We motivate \search's adaptive hierarchical search 
A non-hierarchical search has all cog options in a single-layer search space for an optimizer like BO to search, an approach taken by prior workflow optimizers~\cite{dspy-2-2024,gptswarm}.
With small budgets, a single-layer hierarchy allows BO-like search to spend the budget on dimensions that could potentially generate some improvements.
%While given enough budget, the single-layer space can be extensively searched to find global optimal, with little budget, 
However, a major issue with a single-layer search space is that a search algorithm like BO can be stuck at a local optimum even when budgets increase.
% (unless the budget is close to covering a very large space across dimensions).
To mitigate this issue, our idea is to perform a hierarchical search that works by choosing configurations in the outermost layer first, then under each chosen configuration, choosing the next layer's configurations until the innermost layer. 
With such a hierarchy, a search algorithm could force each layer to sample some values. Given enough budget, each dimension will receive some sampling points, allowing better coverage in the entire search space. However, with high dimensionality (\ie, many types of cogs) and insufficient budget, a hierarchical search may not be able to perform enough local search to find any good optimizations.

To support different user-specified budgets and to get the best of both approaches, we propose an adaptive hierarchical search approach, as shown in Algorithm~\ref{alg:main}.
\search\ starts the search by combining all cogs into one layer ($L=1$, line 9 in Algorithm~\ref{alg:main}) and estimating the expected search budget of this single layer to be the total number of cogs to the power of $\alpha$ (lines 16-19, by default $\alpha = 1.1$). This budget is then passed to the \texttt{LayerSearch} function (Algorithm~\ref{alg:outer}) to perform the actual cog search. When the user-defined budget is no larger than this estimated budget, we expect the single-layer, non-hierarchical search to work better than hierarchical search.
%as the budget for this single layer.

If the user-defined budget is larger, \search\ continues the search with two layers ($L=2$), combining step and weight cogs into the inner layer and architecture cogs as the outer layer (lines 11-14).
\search\ estimates the total search budget for this round as the product of the number of cogs in each of the two layers to the power of $\alpha$ (lines 16-20). It then distributes the estimated search budget between the two layers proportionally to each layer's complexity (lines 22-24) and calls the upper layer's \texttt{LayerSearch} function. Afterward, if there is still budget left, \search\ performs a last round of search using three layers and the remaining budget in a similar way as described above but with three separate layers (architecture as the outermost, step as the middle, and weight cogs as the innermost layer). Two or three layers work better for larger user-defined budgets, as they allow for a larger coverage of the high-dimensional search space.

Finally, \search\ combines all the search results to select the best configurations based on user-defined metrics (line 34).

%\search\ organizes cogs by having architecture cogs in the outer-most search layer, step cogs in the middle layer, and weight cogs in the inner-most layer (line 4 in Algorithm~\ref{alg:main}).
%This is because step cogs' input and output format are dependent on the workflow structure, and the effectiveness of weights (\eg, prompting) are dependent on steps (\eg, LLM model). 

% increases the number of layers until hitting the user-specified total search budget, $TB$

%Thus, the first step of \search\ is to determine the number of layers in its hierarchy and what cogs to include in a layer.
%Intuitively, structure cogs should be placed in the outer-most search layer to be determined first before exploring other cogs. This is because other cogs change node and edge values, and it is easier for 
%However, instead of a fixed number of layers in the hierarchy, we adapt the cog layering according to user-specified total search budgets, $TB$, and the complexity of each layer, using Algorithm~\ref{alg:main}.

% the following \texttt{LayerPartition} method.
%We begin by modeling the relationship between the expected number of evaluations and the number of cogs as well as the number of options in each layer:

%We first consider the identity of each cog in the search space. All structure-cogs will be placed in the outer-most search layer exclusively, which is similar to non-differentiable NAS in traditional ML training. This layer will fix the workflow graph and pass it to the following layer, allowing a stabilized search space for faster convergence.

%Since step-cogs will not create a changing search space, the partition of step-cogs and weight-cogs is conditioned on the search space complexity and the given total budget. Separating step-cogs out can benefit from a more flexible budget allocation strategy and broader exploration for local search at weight-cogs but performs poorly when the given budget is more constrained, in that case, we will optimize them jointly in the same layer.


%\small
%\begin{align*}
%    C &= \{c_{11},c_{12},...\}, \{c_{21},c_{22},...\}, \{c_{31},c_{32},...\} \\
%    NC_i &= \text{total number of cogs in layer i} \\
%    NO_i &= \sum_{j} \{\text{number of possible options in cog } c_{ij}\} \\
%    N_i &= max(NC_i^\alpha,NO_i) \\
%    N_i &= \sum_{j} \{\text{number of possible options in } C_{ij}\} \\
%    N_i &= max(|C_i|^\alpha, N_i) \\
%    B_j &= \prod\limits_{i=1}^{j}N_i, j \in \{1,2,3\}
%\end{align*}

%\normalsize
%where $L$ represents the total number of layers and can be 1, 2, or 3. 
%$C$ represents the entire cog search space, with each row $c_{i*}$ being one of the three types of cogs and lower layers having lower-numbered rows (\eg, $c_{1*}$ being weight cogs). $NC_i$ is the number of cogs in layer $i$, and $NO_i$ is the total number of options across all cogs in layer $i$. $N_i$ is our estimation of the complexity of layer $i$ based on $NC_i$ and $NO_i$ ($\alpha$ is a configurable weight to control the importance between $NC_i$ and $NO_i$; by default $\alpha = 1.1$). 
%$\alpha$ stands for a control parameter, setting the intensity of this scaling behavior w.r.t the number of cogs, we found that $\alpha = 1.2$ is empirically sufficient and efficient for optimizing real workloads. 
%$B_j$ is the expected total number of workflow evaluations for all the lower $j$ layers.
%After calculating $B_1$, $B_2$, and $B_3$, we compare the total budget $TB$ with them.
%When $TB \geq B_3$, we set the total number of layers, $TL$, to 3. When $B_2 \leq TB < B_3$, we set the total number of layers to 2 and merge the step and weight cogs into one layer. When $TB < B_1$, we put all cogs in one layer.
%We only create a separate layer for step-cogs when the given budget $TB$ is greater or equal to the total expected budget for three layers.

%\subsection{Seach Budget Partition}
%\label{sec:sbp}
%After determining cog layers, we distribute the total budget, $TB$, across the layers proportionally to each layer's expected budget $N_i$: , which is the \texttt{BudgetAssign} function.
%We follow a bottom-up partition strategy, where lower layers will try to greedily take the expected budget. This stems from two simple heuristics: (1) feedback to the upper layer is more accurate when the succeeding layer is trained with enough iterations, and (2) the effectiveness of a structure change depends on the setting of individual steps in the workflow (\eg, majority voting is more powerful when each LLM-agent is embedded with diverse few-shot examples or reasoning styles). In cases where the given resource exceeds the total expected budget, 
%We assign $TB$ across layers proportionally to their expected budget $N_i$. 
%The budget assigned at each layer $B_i$ given the total available number of evaluations $TB$ is obtained as follows:

%\small
%\begin{align}
%B_i &=  \lfloor N_i \times \sqrt[L]{\frac{TB}{B^*}}\rfloor
%    B_L &= \begin{cases}
%        min(N_L, TB) & TB < B^* \\
%        \lfloor N_L \times \sqrt[L]{\frac{TB}{B^*}}\rfloor & TB \geq B^*
%    \end{cases}
%    \\
%    B_i &= \begin{cases}
%        min(N_i, \lfloor\frac{TB}{\prod_{j=i+1}^L B_j}\rfloor) & TB < B^* \\
%        \lfloor N_i \times \sqrt[L]{\frac{TB}{B^*}}\rfloor & TB \geq B^*
%    \end{cases}
%\end{align}

%\normalsize


\subsection{Recursive Layer-Wise Search Algorithm}
%The calculation above pre-assigns cogs to layers and search budgets to each layer. 
We now introduce how \search\ performs the actual search in a recursive manner until the inner-most layer is searched, as presented in Algorithm~\ref{alg:outer} \texttt{LayerSearch}. 
Our overall goal is to ensure strong cog option coverage within each layer while quickly directing budgets to more promising cog options based on evaluation results.
%So far, we have determined the optimization layer structure and the maximum allowed search iteration each layer will get. Next, we introduce how the budget is consumed in each layer. The inner-most layer, where weight-cogs, and potentially step-cogs, reside, follows the conventional Bayesian optimization process, exhausting all budgets unless an early stop signal is sent. This signal will be triggered when the current optimizer witnesses $p$ consecutive iterations without any improvements above the threshold. All optimization layers use early stopping to avoid budget waste.
%Algorithm~\ref{alg:inner} describes the search happening at the inner-most (bottom) layer, and 
Specifically, every layer's search is under a chosen set of cog configurations from its upper layers ($C_{chosen}$) and is given a budget $b$. 
In the inner-most layer (lines 7-20), \search\ samples $b$ configurations and evaluates the workflow for each of them together with the configurations from all upper layers ($C_{chosen}$). The evaluation results are added to the feedback set $F$ as the return of this layer.

\begin{algorithm}[h]
  %\algsetup{linenosize=\tiny}
  \small
    \caption{\texttt{LayerSearch} Function}
    \label{alg:outer}
\begin{algorithmic}[1]
%\STATE \textbf{Global Config:} Reduct factor $\eta > 1$, explore width $W$
\STATE \textbf{Global Value:} $R$ \COMMENT{Global result set}
%\STATE \textbf{Global Value:} $F$ \COMMENT{Global observation set}
\STATE \textbf{Input:} $C_{chosen}$: configs chosen in upper layers
\STATE \textbf{Input:} $B$: Array storing assigned budgets to different layers
\STATE \textbf{Input:} $curr\_layer$: this layer's level
\STATE \textbf{Input:} $curr\_b$: this layer's assigned budget
%\STATE
%\FUNCTION{LayerSearch\hspace{0.4em}($C_{chosen}$, $B$, $curr\_layer$, $curr\_b$)}

    \STATE
    \STATE \COMMENT{Search for inner-most layer}
    \IF{curr\_layer == 1}
        \STATE $F = \emptyset$ \COMMENT{Init this layer's feedback set to empty}
        %\STATE $F^{\prime} = match(C_{chosen}, F)$ \COMMENT{Local feedback set}
        \FOR{$k = 0, \dots, curr\_b$}
            \STATE $\lambda$ = \texttt{TPESample} (1) \COMMENT{Sample one configuration using TPE}
            \STATE $f = $ \texttt{EvaluateWorkflow} ($C_{chosen} \cup \lambda$)
            \STATE $R = R \cup \{C_{chosen} \cup \lambda\}$ \COMMENT{Add configuration to global $R$}
            \IF{\texttt{EarlyStop} (f)}
            \STATE break
            \ENDIF
            \STATE $F = F \cup \{f\}$ \COMMENT{Add evaluate result to feedback $F$}
        \ENDFOR
        %\STATE $F = F \cup F^{\prime}$
        \STATE \textbf{Return} $F$
    \ENDIF
    \STATE
    \STATE \COMMENT{Search for non-inner-most layer}
    %\STATE $K = \lfloor \frac{b}{W} \rfloor$, 
    \STATE $b\_used = 0$, $TF = \emptyset$ \COMMENT{Init this layer's used budget and feedback set}
    \STATE $R = \lceil\frac{curr\_b}{\eta}\rceil$, $S = \lfloor\frac{curr\_b}{R}\rfloor$ \COMMENT{Set $R$ and $S$ based on $curr\_b$}
    \STATE
    \WHILE{$b\_{used}$ $\leq$ $curr\_b$}
        \STATE \COMMENT{Sample $W$ configs at a time until running out of $curr\_b$}
        \STATE $n = (curr\_b - b_{used})$ > $W$ ? $W$ : $(curr\_b - b_{used})$
        %\IF{$b - b_{used} < W$}
        %    \STATE $n = b_l - b_{used}$
        %\ELSE
         %   \STATE $n=W$
        %\ENDIF
        \STATE $b\_used$ += $n$
        %\STATE $n = \text{min}(W,\ b_l - kW)$ \COMMENT{Propose $W$ configs and meet $b_l$ constraint}
        \STATE $\Theta = $ \texttt{TPESample} ($n$) \COMMENT{Sample a chunk of $n$ configs in the layer} 
        %\STATE $F^{\prime} = match(C_{chosen}, F)$ \COMMENT{Per-chunk feedback set}
        \STATE $F = \emptyset$ \COMMENT{Init this layer's feedback set to empty}
        \STATE
        \FOR{$s = 0, 1, \dots, S$}
            \STATE $r_s = R\cdot \eta^s$
            \FOR{$\theta \in \Theta$}
                %\IF{$curr\_layer < max\_layer$}
                    \STATE $f =$ \texttt{LayerSearch} ($C_{chosen} \cup \{\theta\}$, $B$, curr\_layer$-1$, $r_s$)
                %\ELSE
                %    \STATE $f =$ InnerOpt($\gamma \cup \{\theta\}$, $r_s$)
                %\STATE $f$ = $opt\_layers[current\_layer+1](\gamma \cup \{\theta\}, r_s)$ \{Optimize the current config at the next layer with $r_s$ budget \}
                %\ENDIF
                \STATE $F = F \cup f$ \COMMENT{Add evaluate result to feedback}
                \IF{\texttt{EarlyStop} ($f$)}
                    \STATE $\Theta = \Theta - \{\theta\}$ \COMMENT{Skip converged configs}
                \ENDIF
            \ENDFOR
            \STATE $\Theta$ = Select top $\lfloor \frac{|\Theta|}{\eta}\rfloor$ configs from $F$ based on user-specified metrics
        \ENDFOR
        \STATE
        \IF{\texttt{EarlyStop} ($F$)}
            \STATE break \COMMENT{Skip remaining search if results converged}
        \ENDIF
        \STATE $TF = TF \cup F$
    \ENDWHILE
    %\STATE $F = F \cup TF$
        \STATE \textbf{Return} $TF$

%\ENDFUNCTION

%\STATE \textbf{Output:} Best metrics in all trials
\end{algorithmic}
\end{algorithm}

% consumption\_nextlayer\_bucket = WSR

% for s in 0, 1,...S do
%     w = W*\eta^{s}
%     r = R*\eta^{-s}

% total budget at next layer = b_l / W * WSR = b_l * SR

% b_l * SR <= b_l * B_l+1

% S = B_{l+1} / R



For a non-inner-most layer, \search\ samples a chunk ($W$) of points at a time using the TPE BO algorithm~\cite{bergstra2011tpe} until all this layer's pre-assigned budget is exhausted (lines 27-30). Within a chunk, \search\ uses a successive-halving-like approach to iteratively direct the search budget to more promising configurations within the chunk (the dynamically changing set, $\Theta$). In each iteration, \search\ calls the next-level search function for each sampled configuration in $\Theta$ with a budget of $r_s$ and adds the evaluation observations from lower layers to the feedback set $F$ for later TPE sampling to use (lines 35-37).
In the first iteration ($s=0$), $r_s$ is set to $R\cdot \eta^0=R$ (line 34). After the inner layers use this budget to search, \search\ filters out configurations with lower performance and only keeps the top $\lfloor \frac{|\Theta|}{\eta}\rfloor$ configurations as the new $\Theta$ to explore in the next iteration (line 42). In each next iteration, \search\ increases $r_s$ by $\eta$ times (line 34), essentially giving more search budget to the better configurations from the previous iteration.

The successive halving method effectively distributes the search budget to more promising configurations, while the chunk-based sampling approach allows for evaluation feedback to accumulate quickly so that later rounds of TPE can get more feedback (compared to no chunking and sampling all $b$ configurations at the same time). To further improve the search efficiency, we adopt an {\em early stop} approach where we stop a chunk or a layer's search when we find its latest few searches do not improve workflow results by more than a threshold, indicating convergence (lines 14,38,45).

%algorithm takes as input other cog settings from previous layers and the assigned budget at the current layer. It tiles the search loop into fixed-size blocks (line 4), each runs the SuccessiveHalving (SH) subroutine in the inner loop (line 7-15). In each SH iteration, only top-$1/\eta$-quantile configurations in $\Theta$ will continue in the next round with $\eta$ times larger budget consumption. As a result, exponentially more trials will be performed by more promising configurations. 

%On average, \textit{Outer-layer search} will create $K$ brackets, each granting approximately $WRS$ budget to the next layer. $R$ represents the smallest amount of resource allocated to any configurations in $\Theta$. 

% layer - 1: budget = 4
% K * W <= b\_current layer
% layer -1: itear 0: propose 2

%     SH:
%     2 config -> R
%     1 config -> 2R

%     iteration 1: propose 2 = W
%     SH:
%     2 config -> R
%     1 config -> 2R

% W configs; each has R resource

% W / eta configs; each has R * eta resource

% R -> least resource one config can get = B2 - smth
% R + R*eta + ... + R*eta\^s -> most promising = B2 + smth


% $L2$ is the middle layer where structure-cogs and step-cogs may be placed exclusively. We employ hyperband for its robustness in exploration and exploitation trace-off. If this layer exists, it will instruct $L1$ the number of search iterations to run in each invocation. Specifically, in each iteration at line 4, \sysname will propose $n$ configurations and run SuccessiveHalving (SH) subroutine (line 8-15). SH will optimize each proposal and use the search results from $L1$ to rank their performance. Each time only the top-performing $n \cdot \eta^{-i}$ can continue in the next round with a larger budget. With this strategy, exponentially more search budgets are allocated to more promising configs at $L2$.

% \input{algo-l2-search}

% $L3$ is the outer-most layer for structure-cogs only when $L2$ is created. For this layer, we employ plain SH without hyperband because of its predictable convergence behavior. This is mainly due to two factors: (1) structure change to the workflow is more significant thus different configurations are more likely to deviate after training with the following layers. (2) with the search space partition strategy in Sec ~\ref{sec:ssp}, we can assume the available budget at each layer is substantial when $L3$ exists. Given this prior knowledge, we can avoid grid searching control parameter $n$ as in the hyperband but adopt a more aggressive allocation scheme to bias towards better proposals and moderate search wastes.



%\subsubsection{Runtime Budget Adaptation}
%Using static estimation of the expected budget for each layer is not enough, we also adjust the assignment during the optimization based on real convergence behavior. Specifically, for layer $i$, we record the number of configurations evaluated in each optimize routine. We set the convergence indicator $C_{ij}$ of $j^{th}$ routine with this number if the search early exits before reaching the budget limit, otherwise 2\x of its assigned resource. Then we update $E_i$ with $\frac{\sum_{j}^M C_{ij}}{M}$. \sysname\ will update the budget partition according to Sec~\ref{sec:sbp} for any newly spawned optimizer routines. Besides controlling the proportion of budgets across layers, a smaller/larger $B_{l+1}$ will also guide the SH in Alg~\ref{alg:outer} to shrink/extend the budget $R$ for differentiating config performance.


\section{\sysname\ Design}
\label{sec:cognify}

We build \sysname, an extensible gen-AI workflow autotuning platform based on the \search\ algorithm. The input to \sysname\ is the user-written gen-AI workflow (we currently support LangChain \cite{langchain-repo}, DSPy \cite{khattab2024dspy}, and our own programming model), a user-provided workflow training set, a user-chosen evaluator, and a user-specified total search budget. \sysname\ currently supports three autotuning objectives: generation quality (defined by the user evaluator), total workflow execution cost, and total workflow execution latency. Users can choose one or more of these objectives and set thresholds for them or the remaining metrics (\eg, optimize cost and latency while ensuring quality to be at least 5\% better than the original workflow). 
\sysname\ uses the \search\ algorithm to search through the cog space.
When given multiple optimization objectives, \sysname\ maintains a sorted optimization queue for each objective and performs its pruning and final result selection from all the sorted queues (possibly with different weighted numbers).
To speed up the search process, we employ parallel execution, where a user-configurable number of optimizers, each taking a chunk of search load, work together in parallel. %Below, we introduce each type of cogs in more details.
\sysname\ returns multiple autotuned workflow versions based on user-specified objectives.
\sysname\ also allows users to continue the auto-tuning from a previous optimization result with more budgets so that users can gradually increase their search budget without prior knowledge of what budget is sufficient.
Appendix~\ref{sec:apdx-example} shows an example of \sysname-tuned workflow outputs. 
\sysname\ currently supports six cogs in three categories, as discussed below. 

%In \sysname, we call every workflow optimization technique a {\em cog}, including structure-changing cogs like task decomposition, step-changing cogs like model selection, and weight-changing cogs like adding few-shot examples to prompts. 
%\sysname\ places structure-changing cogs in the outermost layer, step cogs in the middle layer, and weight cogs in the innermost layer, because \fixme{TODO}.


\subsection{Architecture Cogs}
\label{sec:structure-cog}
%Changing the structure of a workflow can potentially improve its generation quality (\eg, by using multiple steps to attempt at a task in parallel or in chain) or reduce its execution cost and latency (\eg, by merging or removing steps).
\sysname\ currently supports two architecture cogs: task decomposition and task ensemble.
Task decomposition~\cite{khot2023decomposed} breaks a workflow step into multiple sub-steps and can potentially improve generation quality and lower execution costs, as decomposed tasks are easier to solve even with a small (cheaper) model.
There are numerous ways to perform task decomposition in a workflow. 
%, as all LM steps can potentially be decomposed and into different numbers of sub-steps in different ways. Throwing all options to the Bayesian Optimizer would drastically increase the search space for \sysname. 
To reduce the search space, we propose several ways to narrow down task decomposition options. Even though we present these techniques in the setting of task decomposition, they generalize to many other structure-changing tuning techniques.

%First, we narrow down a selected set of steps in a workflow to decompose. 
Intuitively, complex tasks are the hardest to solve and worth decomposition the most. We use a combination of LLM-as-a-judge \cite{vicuna_share_gpt} and static graph (program) analysis to identify complex steps. We instruct an LLM to give a rating of the complexity of each step in a workflow. We then analyze the relationship between steps in a workflow and find the number of out-edges of each step (\ie, the number of subsequent steps getting this step's output). More out-edges imply that a step is likely performing more tasks at the same time and is thus more complex. We multiply the LLM-produced rating and the number of out-edges for each step and pick the modules with scores above a learnable threshold as the target for task decomposition. We then instruct an LLM to propose a decomposition (\ie, generate the submodules and their prompts) for each candidate step. %We provide the LLM with few-shot examples for what proposed modules for a separate task could look like. We also add a refinement step that validates whether the proposition decomposition maintains the semantics of the original trajectory. Once candidate decompositions are generated, those are used for the entirety of the optimization.

{
\begin{figure*}[t!]
\begin{center}
\centerline{\includegraphics[width=0.95\textwidth]{Figures/big_grid.pdf}}
\vspace{-0.1in}
\mycap{Generation Quality vs Cost/Latency.}{Dashed lines show the Pareto frontier (upper left is better). Cost shown as model API dollar cost for every 1000 requests. Cognify selects models
from GPT-4o-mini and Llama-8B. DSPy and Trace do not support model selection and are given GPT-4o-mini for all steps. Trace results for Text-2-SQL and FinRobot have 0 quality and are not included.} 
\Description{Eight graphs with different shapes representing baselines compared to points on a Pareto frontier.}
\label{fig-biggrid}
\end{center}
\end{figure*}
}


The second structure-changing cog that \sysname\ supports is task ensembling. This cog spawns multiple parallel steps (or samplers) for a single step in the original workflow, as well as an aggregator step that returns the best output (or combination of outputs). By introducing parallel steps, \sysname\ can optimize these independently with step and weight cogs. This provides the aggregator with a diverse set of outputs to choose from. 
%The aggregator is prompted with the role of the samplers, as well as the inputs to each. It also receives a criteria by which it should make a decision. We choose to prompt it with a qualitative description of how it should resolve discrepancies between outputs. 


\subsection{Step Cogs}
We currently support two step-changing cogs: model selection for language-model (LM) steps and code rewriting for code steps.

For model selection, to reduce its search space, we identify ``important'' LM steps---steps that most critically impact the final workflow output to reduce the set \search\ performs TPE sampling on. Our approach is to test each step in isolation by freezing other steps with the cheapest model and trying different models on the step under testing. 
We then calculate the difference between the model yielding the best and worst workflow results as the importance of the step under testing. %For each model, we get the workflow output quality score using sampled user-supplied inputs and user-specific evaluator. We then calculate the difference between the highest and lowest scores as this module's importance. 
After testing all the steps, we choose the steps with the highest K\% importance as the ones for TPE to sample from.
%, where K is determined based on user-chosen stop criteria. We then initialize the Bayesian optimization to start with the state where important modules use the largest model and all other modules use the cheapest model. We set the TPE optimization bandwidth of each module to be the inverse of importance, \ie, the more important a module is the more TPE spends on optimizing.

The second step cog \sysname\ supports is code rewriting, where it automatically changes code steps to use better implementation. To rewrite a code step, \sysname\ finds the $k$ worst- and best-performing training data points and feeds their corresponding input and output pairs of this code step to an LLM. We let the LLM propose $n$ new candidate code pieces for the step at a time.
%in parallel to generate a set of $n$ candidates.
In subsequent trials, the optimizer dynamically updates the candidate set using feedback from the evaluator.


\subsection{Weight Cogs}
\sysname\ currently supports two weight-changing cogs: reasoning and few-shot examples.
First, \sysname\ supports adding reasoning capability to the user's original prompt, with two options: zero-shot Chain-of-Thought \cite{wei2022chain} (\ie, ``think step-by-step...'') and dynamic planning \cite{huang2022language} (\ie, ``break down the task into simpler sub-tasks...''). These prompts are appended to the user's prompt. In the case where the original module relies on structured output, we support a reason-then-format option that injects reasoning text into the prompt while maintaining the original output schema.

Second, \sysname\ supports dynamically adding few-shot examples to a prompt. At the end of each iteration, we choose the top-$k$-performing examples for an LM step in the training data and use their corresponding input-output pairs of the LM step as the few-shot examples to be appended to the original prompt to the LM step for later iterations' TPE sampling. As such, the set of few-shot examples is constantly evolving during the optimization process based on the workflow's evaluation results. 
%Few-shot examples are available to all modules, even intermediate steps in the workflow. We use the full trajectory of each request to generate examples for the intermediate steps. Furthermore, we automatically filter out examples that do not meet a user-specified threshold. 




\begin{table*}[t]
\centering
\fontsize{11pt}{11pt}\selectfont
\begin{tabular}{lllllllllllll}
\toprule
\multicolumn{1}{c}{\textbf{task}} & \multicolumn{2}{c}{\textbf{Mir}} & \multicolumn{2}{c}{\textbf{Lai}} & \multicolumn{2}{c}{\textbf{Ziegen.}} & \multicolumn{2}{c}{\textbf{Cao}} & \multicolumn{2}{c}{\textbf{Alva-Man.}} & \multicolumn{1}{c}{\textbf{avg.}} & \textbf{\begin{tabular}[c]{@{}l@{}}avg.\\ rank\end{tabular}} \\
\multicolumn{1}{c}{\textbf{metrics}} & \multicolumn{1}{c}{\textbf{cor.}} & \multicolumn{1}{c}{\textbf{p-v.}} & \multicolumn{1}{c}{\textbf{cor.}} & \multicolumn{1}{c}{\textbf{p-v.}} & \multicolumn{1}{c}{\textbf{cor.}} & \multicolumn{1}{c}{\textbf{p-v.}} & \multicolumn{1}{c}{\textbf{cor.}} & \multicolumn{1}{c}{\textbf{p-v.}} & \multicolumn{1}{c}{\textbf{cor.}} & \multicolumn{1}{c}{\textbf{p-v.}} &  &  \\ \midrule
\textbf{S-Bleu} & 0.50 & 0.0 & 0.47 & 0.0 & 0.59 & 0.0 & 0.58 & 0.0 & 0.68 & 0.0 & 0.57 & 5.8 \\
\textbf{R-Bleu} & -- & -- & 0.27 & 0.0 & 0.30 & 0.0 & -- & -- & -- & -- & - &  \\
\textbf{S-Meteor} & 0.49 & 0.0 & 0.48 & 0.0 & 0.61 & 0.0 & 0.57 & 0.0 & 0.64 & 0.0 & 0.56 & 6.1 \\
\textbf{R-Meteor} & -- & -- & 0.34 & 0.0 & 0.26 & 0.0 & -- & -- & -- & -- & - &  \\
\textbf{S-Bertscore} & \textbf{0.53} & 0.0 & {\ul 0.80} & 0.0 & \textbf{0.70} & 0.0 & {\ul 0.66} & 0.0 & {\ul0.78} & 0.0 & \textbf{0.69} & \textbf{1.7} \\
\textbf{R-Bertscore} & -- & -- & 0.51 & 0.0 & 0.38 & 0.0 & -- & -- & -- & -- & - &  \\
\textbf{S-Bleurt} & {\ul 0.52} & 0.0 & {\ul 0.80} & 0.0 & 0.60 & 0.0 & \textbf{0.70} & 0.0 & \textbf{0.80} & 0.0 & {\ul 0.68} & {\ul 2.3} \\
\textbf{R-Bleurt} & -- & -- & 0.59 & 0.0 & -0.05 & 0.13 & -- & -- & -- & -- & - &  \\
\textbf{S-Cosine} & 0.51 & 0.0 & 0.69 & 0.0 & {\ul 0.62} & 0.0 & 0.61 & 0.0 & 0.65 & 0.0 & 0.62 & 4.4 \\
\textbf{R-Cosine} & -- & -- & 0.40 & 0.0 & 0.29 & 0.0 & -- & -- & -- & -- & - & \\ \midrule
\textbf{QuestEval} & 0.23 & 0.0 & 0.25 & 0.0 & 0.49 & 0.0 & 0.47 & 0.0 & 0.62 & 0.0 & 0.41 & 9.0 \\
\textbf{LLaMa3} & 0.36 & 0.0 & \textbf{0.84} & 0.0 & {\ul{0.62}} & 0.0 & 0.61 & 0.0 &  0.76 & 0.0 & 0.64 & 3.6 \\
\textbf{our (3b)} & 0.49 & 0.0 & 0.73 & 0.0 & 0.54 & 0.0 & 0.53 & 0.0 & 0.7 & 0.0 & 0.60 & 5.8 \\
\textbf{our (8b)} & 0.48 & 0.0 & 0.73 & 0.0 & 0.52 & 0.0 & 0.53 & 0.0 & 0.7 & 0.0 & 0.59 & 6.3 \\  \bottomrule
\end{tabular}
\caption{Pearson correlation on human evaluation on system output. `R-': reference-based. `S-': source-based.}
\label{tab:sys}
\end{table*}



\begin{table}%[]
\centering
\fontsize{11pt}{11pt}\selectfont
\begin{tabular}{llllll}
\toprule
\multicolumn{1}{c}{\textbf{task}} & \multicolumn{1}{c}{\textbf{Lai}} & \multicolumn{1}{c}{\textbf{Zei.}} & \multicolumn{1}{c}{\textbf{Scia.}} & \textbf{} & \textbf{} \\ 
\multicolumn{1}{c}{\textbf{metrics}} & \multicolumn{1}{c}{\textbf{cor.}} & \multicolumn{1}{c}{\textbf{cor.}} & \multicolumn{1}{c}{\textbf{cor.}} & \textbf{avg.} & \textbf{\begin{tabular}[c]{@{}l@{}}avg.\\ rank\end{tabular}} \\ \midrule
\textbf{S-Bleu} & 0.40 & 0.40 & 0.19* & 0.33 & 7.67 \\
\textbf{S-Meteor} & 0.41 & 0.42 & 0.16* & 0.33 & 7.33 \\
\textbf{S-BertS.} & {\ul0.58} & 0.47 & 0.31 & 0.45 & 3.67 \\
\textbf{S-Bleurt} & 0.45 & {\ul 0.54} & {\ul 0.37} & 0.45 & {\ul 3.33} \\
\textbf{S-Cosine} & 0.56 & 0.52 & 0.3 & {\ul 0.46} & {\ul 3.33} \\ \midrule
\textbf{QuestE.} & 0.27 & 0.35 & 0.06* & 0.23 & 9.00 \\
\textbf{LlaMA3} & \textbf{0.6} & \textbf{0.67} & \textbf{0.51} & \textbf{0.59} & \textbf{1.0} \\
\textbf{Our (3b)} & 0.51 & 0.49 & 0.23* & 0.39 & 4.83 \\
\textbf{Our (8b)} & 0.52 & 0.49 & 0.22* & 0.43 & 4.83 \\ \bottomrule
\end{tabular}
\caption{Pearson correlation on human ratings on reference output. *not significant; we cannot reject the null hypothesis of zero correlation}
\label{tab:ref}
\end{table}


\begin{table*}%[]
\centering
\fontsize{11pt}{11pt}\selectfont
\begin{tabular}{lllllllll}
\toprule
\textbf{task} & \multicolumn{1}{c}{\textbf{ALL}} & \multicolumn{1}{c}{\textbf{sentiment}} & \multicolumn{1}{c}{\textbf{detoxify}} & \multicolumn{1}{c}{\textbf{catchy}} & \multicolumn{1}{c}{\textbf{polite}} & \multicolumn{1}{c}{\textbf{persuasive}} & \multicolumn{1}{c}{\textbf{formal}} & \textbf{\begin{tabular}[c]{@{}l@{}}avg. \\ rank\end{tabular}} \\
\textbf{metrics} & \multicolumn{1}{c}{\textbf{cor.}} & \multicolumn{1}{c}{\textbf{cor.}} & \multicolumn{1}{c}{\textbf{cor.}} & \multicolumn{1}{c}{\textbf{cor.}} & \multicolumn{1}{c}{\textbf{cor.}} & \multicolumn{1}{c}{\textbf{cor.}} & \multicolumn{1}{c}{\textbf{cor.}} &  \\ \midrule
\textbf{S-Bleu} & -0.17 & -0.82 & -0.45 & -0.12* & -0.1* & -0.05 & -0.21 & 8.42 \\
\textbf{R-Bleu} & - & -0.5 & -0.45 &  &  &  &  &  \\
\textbf{S-Meteor} & -0.07* & -0.55 & -0.4 & -0.01* & 0.1* & -0.16 & -0.04* & 7.67 \\
\textbf{R-Meteor} & - & -0.17* & -0.39 & - & - & - & - & - \\
\textbf{S-BertScore} & 0.11 & -0.38 & -0.07* & -0.17* & 0.28 & 0.12 & 0.25 & 6.0 \\
\textbf{R-BertScore} & - & -0.02* & -0.21* & - & - & - & - & - \\
\textbf{S-Bleurt} & 0.29 & 0.05* & 0.45 & 0.06* & 0.29 & 0.23 & 0.46 & 4.2 \\
\textbf{R-Bleurt} & - &  0.21 & 0.38 & - & - & - & - & - \\
\textbf{S-Cosine} & 0.01* & -0.5 & -0.13* & -0.19* & 0.05* & -0.05* & 0.15* & 7.42 \\
\textbf{R-Cosine} & - & -0.11* & -0.16* & - & - & - & - & - \\ \midrule
\textbf{QuestEval} & 0.21 & {\ul{0.29}} & 0.23 & 0.37 & 0.19* & 0.35 & 0.14* & 4.67 \\
\textbf{LlaMA3} & \textbf{0.82} & \textbf{0.80} & \textbf{0.72} & \textbf{0.84} & \textbf{0.84} & \textbf{0.90} & \textbf{0.88} & \textbf{1.00} \\
\textbf{Our (3b)} & 0.47 & -0.11* & 0.37 & 0.61 & 0.53 & 0.54 & 0.66 & 3.5 \\
\textbf{Our (8b)} & {\ul{0.57}} & 0.09* & {\ul 0.49} & {\ul 0.72} & {\ul 0.64} & {\ul 0.62} & {\ul 0.67} & {\ul 2.17} \\ \bottomrule
\end{tabular}
\caption{Pearson correlation on human ratings on our constructed test set. 'R-': reference-based. 'S-': source-based. *not significant; we cannot reject the null hypothesis of zero correlation}
\label{tab:con}
\end{table*}

\section{Results}
We benchmark the different metrics on the different datasets using correlation to human judgement. For content preservation, we show results split on data with system output, reference output and our constructed test set: we show that the data source for evaluation leads to different conclusions on the metrics. In addition, we examine whether the metrics can rank style transfer systems similar to humans. On style strength, we likewise show correlations between human judgment and zero-shot evaluation approaches. When applicable, we summarize results by reporting the average correlation. And the average ranking of the metric per dataset (by ranking which metric obtains the highest correlation to human judgement per dataset). 

\subsection{Content preservation}
\paragraph{How do data sources affect the conclusion on best metric?}
The conclusions about the metrics' performance change radically depending on whether we use system output data, reference output, or our constructed test set. Ideally, a good metric correlates highly with humans on any data source. Ideally, for meta-evaluation, a metric should correlate consistently across all data sources, but the following shows that the correlations indicate different things, and the conclusion on the best metric should be drawn carefully.

Looking at the metrics correlations with humans on the data source with system output (Table~\ref{tab:sys}), we see a relatively high correlation for many of the metrics on many tasks. The overall best metrics are S-BertScore and S-BLEURT (avg+avg rank). We see no notable difference in our method of using the 3B or 8B model as the backbone.

Examining the average correlations based on data with reference output (Table~\ref{tab:ref}), now the zero-shoot prompting with LlaMA3 70B is the best-performing approach ($0.59$ avg). Tied for second place are source-based cosine embedding ($0.46$ avg), BLEURT ($0.45$ avg) and BertScore ($0.45$ avg). Our method follows on a 5. place: here, the 8b version (($0.43$ avg)) shows a bit stronger results than 3b ($0.39$ avg). The fact that the conclusions change, whether looking at reference or system output, confirms the observations made by \citet{scialom-etal-2021-questeval} on simplicity transfer.   

Now consider the results on our test set (Table~\ref{tab:con}): Several metrics show low or no correlation; we even see a significantly negative correlation for some metrics on ALL (BLEU) and for specific subparts of our test set for BLEU, Meteor, BertScore, Cosine. On the other end, LlaMA3 70B is again performing best, showing strong results ($0.82$ in ALL). The runner-up is now our 8B method, with a gap to the 3B version ($0.57$ vs $0.47$ in ALL). Note our method still shows zero correlation for the sentiment task. After, ranks BLEURT ($0.29$), QuestEval ($0.21$), BertScore ($0.11$), Cosine ($0.01$).  

On our test set, we find that some metrics that correlate relatively well on the other datasets, now exhibit low correlation. Hence, with our test set, we can now support the logical reasoning with data evidence: Evaluation of content preservation for style transfer needs to take the style shift into account. This conclusion could not be drawn using the existing data sources: We hypothesise that for the data with system-based output, successful output happens to be very similar to the source sentence and vice versa, and reference-based output might not contain server mistakes as they are gold references. Thus, none of the existing data sources tests the limits of the metrics.  


\paragraph{How do reference-based metrics compare to source-based ones?} Reference-based metrics show a lower correlation than the source-based counterpart for all metrics on both datasets with ratings on references (Table~\ref{tab:sys}). As discussed previously, reference-based metrics for style transfer have the drawback that many different good solutions on a rewrite might exist and not only one similar to a reference.


\paragraph{How well can the metrics rank the performance of style transfer methods?}
We compare the metrics' ability to judge the best style transfer methods w.r.t. the human annotations: Several of the data sources contain samples from different style transfer systems. In order to use metrics to assess the quality of the style transfer system, metrics should correctly find the best-performing system. Hence, we evaluate whether the metrics for content preservation provide the same system ranking as human evaluators. We take the mean of the score for every output on each system and the mean of the human annotations; we compare the systems using the Kendall's Tau correlation. 

We find only the evaluation using the dataset Mir, Lai, and Ziegen to result in significant correlations, probably because of sparsity in a number of system tests (App.~\ref{app:dataset}). Our method (8b) is the only metric providing a perfect ranking of the style transfer system on the Lai data, and Llama3 70B the only one on the Ziegen data. Results in App.~\ref{app:results}. 


\subsection{Style strength results}
%Evaluating style strengths is a challenging task. 
Llama3 70B shows better overall results than our method. However, our method scores higher than Llama3 70B on 2 out of 6 datasets, but it also exhibits zero correlation on one task (Table~\ref{tab:styleresults}).%More work i s needed on evaluating style strengths. 
 
\begin{table}%[]
\fontsize{11pt}{11pt}\selectfont
\begin{tabular}{lccc}
\toprule
\multicolumn{1}{c}{\textbf{}} & \textbf{LlaMA3} & \textbf{Our (3b)} & \textbf{Our (8b)} \\ \midrule
\textbf{Mir} & 0.46 & 0.54 & \textbf{0.57} \\
\textbf{Lai} & \textbf{0.57} & 0.18 & 0.19 \\
\textbf{Ziegen.} & 0.25 & 0.27 & \textbf{0.32} \\
\textbf{Alva-M.} & \textbf{0.59} & 0.03* & 0.02* \\
\textbf{Scialom} & \textbf{0.62} & 0.45 & 0.44 \\
\textbf{\begin{tabular}[c]{@{}l@{}}Our Test\end{tabular}} & \textbf{0.63} & 0.46 & 0.48 \\ \bottomrule
\end{tabular}
\caption{Style strength: Pearson correlation to human ratings. *not significant; we cannot reject the null hypothesis of zero corelation}
\label{tab:styleresults}
\end{table}

\subsection{Ablation}
We conduct several runs of the methods using LLMs with variations in instructions/prompts (App.~\ref{app:method}). We observe that the lower the correlation on a task, the higher the variation between the different runs. For our method, we only observe low variance between the runs.
None of the variations leads to different conclusions of the meta-evaluation. Results in App.~\ref{app:results}.

\putsec{related}{Related Work}

\noindent \textbf{Efficient Radiance Field Rendering.}
%
The introduction of Neural Radiance Fields (NeRF)~\cite{mil:sri20} has
generated significant interest in efficient 3D scene representation and
rendering for radiance fields.
%
Over the past years, there has been a large amount of research aimed at
accelerating NeRFs through algorithmic or software
optimizations~\cite{mul:eva22,fri:yu22,che:fun23,sun:sun22}, and the
development of hardware
accelerators~\cite{lee:cho23,li:li23,son:wen23,mub:kan23,fen:liu24}.
%
The state-of-the-art method, 3D Gaussian splatting~\cite{ker:kop23}, has
further fueled interest in accelerating radiance field
rendering~\cite{rad:ste24,lee:lee24,nie:stu24,lee:rho24,ham:mel24} as it
employs rasterization primitives that can be rendered much faster than NeRFs.
%
However, previous research focused on software graphics rendering on
programmable cores or building dedicated hardware accelerators. In contrast,
\name{} investigates the potential of efficient radiance field rendering while
utilizing fixed-function units in graphics hardware.
%
To our knowledge, this is the first work that assesses the performance
implications of rendering Gaussian-based radiance fields on the hardware
graphics pipeline with software and hardware optimizations.

%%%%%%%%%%%%%%%%%%%%%%%%%%%%%%%%%%%%%%%%%%%%%%%%%%%%%%%%%%%%%%%%%%%%%%%%%%
\myparagraph{Enhancing Graphics Rendering Hardware.}
%
The performance advantage of executing graphics rendering on either
programmable shader cores or fixed-function units varies depending on the
rendering methods and hardware designs.
%
Previous studies have explored the performance implication of graphics hardware
design by developing simulation infrastructures for graphics
workloads~\cite{bar:gon06,gub:aam19,tin:sax23,arn:par13}.
%
Additionally, several studies have aimed to improve the performance of
special-purpose hardware such as ray tracing units in graphics
hardware~\cite{cho:now23,liu:cha21} and proposed hardware accelerators for
graphics applications~\cite{lu:hua17,ram:gri09}.
%
In contrast to these works, which primarily evaluate traditional graphics
workloads, our work focuses on improving the performance of volume rendering
workloads, such as Gaussian splatting, which require blending a huge number of
fragments per pixel.

%%%%%%%%%%%%%%%%%%%%%%%%%%%%%%%%%%%%%%%%%%%%%%%%%%%%%%%%%%%%%%%%%%%%%%%%%%
%
In the context of multi-sample anti-aliasing, prior work proposed reducing the
amount of redundant shading by merging fragments from adjacent triangles in a
mesh at the quad granularity~\cite{fat:bou10}.
%
While both our work and quad-fragment merging (QFM)~\cite{fat:bou10} aim to
reduce operations by merging quads, our proposed technique differs from QFM in
many aspects.
%
Our method aims to blend \emph{overlapping primitives} along the depth
direction and applies to quads from any primitive. In contrast, QFM merges quad
fragments from small (e.g., pixel-sized) triangles that \emph{share} an edge
(i.e., \emph{connected}, \emph{non-overlapping} triangles).
%
As such, QFM is not applicable to the scenes consisting of a number of
unconnected transparent triangles, such as those in 3D Gaussian splatting.
%
In addition, our method computes the \emph{exact} color for each pixel by
offloading blending operations from ROPs to shader units, whereas QFM
\emph{approximates} pixel colors by using the color from one triangle when
multiple triangles are merged into a single quad.


\section{Conclusions}

We provided deterministic distributed algorithms to efficiently simulate a round of algorithms designed for the CONGEST model on the Beeping Networks. This allowed us to improve polynomially the time complexity of several (also graph) problems on Beeping  Networks. The first simulation by the Local Broadcast algorithm is shorter by a polylogarithmic factor than the other, more general one -- yet still powerful enough to implement some algorithms, including the prominent solution to Network Decomposition~\cite{ghaffari2021improved}.
The more general one could be used for solving problems such as MIS.
We also considered efficient pipelining of messages via several layers of BN.
%We also proved that our solutions could not be substantially improved if the considered problems require content-oblivious local broadcast, by proving an almost-tight lower bound.

Two important lines of research arise from our work.
First, whether some (graph) problems do not need local broadcast to be solved deterministically, and whether their time complexity could be asymptotically below $\Delta^2$.
Second, could a lower bound on any deterministic local broadcast algorithm, better than $\Omega(\Delta\log n)$, be proved?
%our lower bound be tightened and extended to any, not necessarily content-oblivious \mam{and non-adaptive}, solutions to the Local Broadcast problem?

% \todo{Propose to develop algorithms that work in time depending on the diameter of the network}

% \todo{Discussion of noisy beeping channel.}





%%
%% The next two lines define the bibliography style to be used, and
%% the bibliography file.
\bibliographystyle{ACM-Reference-Format}
\bibliography{sysml,all-defs,references}


%%
%% If your work has an appendix, this is the place to put it.
\subsection{Lloyd-Max Algorithm}
\label{subsec:Lloyd-Max}
For a given quantization bitwidth $B$ and an operand $\bm{X}$, the Lloyd-Max algorithm finds $2^B$ quantization levels $\{\hat{x}_i\}_{i=1}^{2^B}$ such that quantizing $\bm{X}$ by rounding each scalar in $\bm{X}$ to the nearest quantization level minimizes the quantization MSE. 

The algorithm starts with an initial guess of quantization levels and then iteratively computes quantization thresholds $\{\tau_i\}_{i=1}^{2^B-1}$ and updates quantization levels $\{\hat{x}_i\}_{i=1}^{2^B}$. Specifically, at iteration $n$, thresholds are set to the midpoints of the previous iteration's levels:
\begin{align*}
    \tau_i^{(n)}=\frac{\hat{x}_i^{(n-1)}+\hat{x}_{i+1}^{(n-1)}}2 \text{ for } i=1\ldots 2^B-1
\end{align*}
Subsequently, the quantization levels are re-computed as conditional means of the data regions defined by the new thresholds:
\begin{align*}
    \hat{x}_i^{(n)}=\mathbb{E}\left[ \bm{X} \big| \bm{X}\in [\tau_{i-1}^{(n)},\tau_i^{(n)}] \right] \text{ for } i=1\ldots 2^B
\end{align*}
where to satisfy boundary conditions we have $\tau_0=-\infty$ and $\tau_{2^B}=\infty$. The algorithm iterates the above steps until convergence.

Figure \ref{fig:lm_quant} compares the quantization levels of a $7$-bit floating point (E3M3) quantizer (left) to a $7$-bit Lloyd-Max quantizer (right) when quantizing a layer of weights from the GPT3-126M model at a per-tensor granularity. As shown, the Lloyd-Max quantizer achieves substantially lower quantization MSE. Further, Table \ref{tab:FP7_vs_LM7} shows the superior perplexity achieved by Lloyd-Max quantizers for bitwidths of $7$, $6$ and $5$. The difference between the quantizers is clear at 5 bits, where per-tensor FP quantization incurs a drastic and unacceptable increase in perplexity, while Lloyd-Max quantization incurs a much smaller increase. Nevertheless, we note that even the optimal Lloyd-Max quantizer incurs a notable ($\sim 1.5$) increase in perplexity due to the coarse granularity of quantization. 

\begin{figure}[h]
  \centering
  \includegraphics[width=0.7\linewidth]{sections/figures/LM7_FP7.pdf}
  \caption{\small Quantization levels and the corresponding quantization MSE of Floating Point (left) vs Lloyd-Max (right) Quantizers for a layer of weights in the GPT3-126M model.}
  \label{fig:lm_quant}
\end{figure}

\begin{table}[h]\scriptsize
\begin{center}
\caption{\label{tab:FP7_vs_LM7} \small Comparing perplexity (lower is better) achieved by floating point quantizers and Lloyd-Max quantizers on a GPT3-126M model for the Wikitext-103 dataset.}
\begin{tabular}{c|cc|c}
\hline
 \multirow{2}{*}{\textbf{Bitwidth}} & \multicolumn{2}{|c|}{\textbf{Floating-Point Quantizer}} & \textbf{Lloyd-Max Quantizer} \\
 & Best Format & Wikitext-103 Perplexity & Wikitext-103 Perplexity \\
\hline
7 & E3M3 & 18.32 & 18.27 \\
6 & E3M2 & 19.07 & 18.51 \\
5 & E4M0 & 43.89 & 19.71 \\
\hline
\end{tabular}
\end{center}
\end{table}

\subsection{Proof of Local Optimality of LO-BCQ}
\label{subsec:lobcq_opt_proof}
For a given block $\bm{b}_j$, the quantization MSE during LO-BCQ can be empirically evaluated as $\frac{1}{L_b}\lVert \bm{b}_j- \bm{\hat{b}}_j\rVert^2_2$ where $\bm{\hat{b}}_j$ is computed from equation (\ref{eq:clustered_quantization_definition}) as $C_{f(\bm{b}_j)}(\bm{b}_j)$. Further, for a given block cluster $\mathcal{B}_i$, we compute the quantization MSE as $\frac{1}{|\mathcal{B}_{i}|}\sum_{\bm{b} \in \mathcal{B}_{i}} \frac{1}{L_b}\lVert \bm{b}- C_i^{(n)}(\bm{b})\rVert^2_2$. Therefore, at the end of iteration $n$, we evaluate the overall quantization MSE $J^{(n)}$ for a given operand $\bm{X}$ composed of $N_c$ block clusters as:
\begin{align*}
    \label{eq:mse_iter_n}
    J^{(n)} = \frac{1}{N_c} \sum_{i=1}^{N_c} \frac{1}{|\mathcal{B}_{i}^{(n)}|}\sum_{\bm{v} \in \mathcal{B}_{i}^{(n)}} \frac{1}{L_b}\lVert \bm{b}- B_i^{(n)}(\bm{b})\rVert^2_2
\end{align*}

At the end of iteration $n$, the codebooks are updated from $\mathcal{C}^{(n-1)}$ to $\mathcal{C}^{(n)}$. However, the mapping of a given vector $\bm{b}_j$ to quantizers $\mathcal{C}^{(n)}$ remains as  $f^{(n)}(\bm{b}_j)$. At the next iteration, during the vector clustering step, $f^{(n+1)}(\bm{b}_j)$ finds new mapping of $\bm{b}_j$ to updated codebooks $\mathcal{C}^{(n)}$ such that the quantization MSE over the candidate codebooks is minimized. Therefore, we obtain the following result for $\bm{b}_j$:
\begin{align*}
\frac{1}{L_b}\lVert \bm{b}_j - C_{f^{(n+1)}(\bm{b}_j)}^{(n)}(\bm{b}_j)\rVert^2_2 \le \frac{1}{L_b}\lVert \bm{b}_j - C_{f^{(n)}(\bm{b}_j)}^{(n)}(\bm{b}_j)\rVert^2_2
\end{align*}

That is, quantizing $\bm{b}_j$ at the end of the block clustering step of iteration $n+1$ results in lower quantization MSE compared to quantizing at the end of iteration $n$. Since this is true for all $\bm{b} \in \bm{X}$, we assert the following:
\begin{equation}
\begin{split}
\label{eq:mse_ineq_1}
    \tilde{J}^{(n+1)} &= \frac{1}{N_c} \sum_{i=1}^{N_c} \frac{1}{|\mathcal{B}_{i}^{(n+1)}|}\sum_{\bm{b} \in \mathcal{B}_{i}^{(n+1)}} \frac{1}{L_b}\lVert \bm{b} - C_i^{(n)}(b)\rVert^2_2 \le J^{(n)}
\end{split}
\end{equation}
where $\tilde{J}^{(n+1)}$ is the the quantization MSE after the vector clustering step at iteration $n+1$.

Next, during the codebook update step (\ref{eq:quantizers_update}) at iteration $n+1$, the per-cluster codebooks $\mathcal{C}^{(n)}$ are updated to $\mathcal{C}^{(n+1)}$ by invoking the Lloyd-Max algorithm \citep{Lloyd}. We know that for any given value distribution, the Lloyd-Max algorithm minimizes the quantization MSE. Therefore, for a given vector cluster $\mathcal{B}_i$ we obtain the following result:

\begin{equation}
    \frac{1}{|\mathcal{B}_{i}^{(n+1)}|}\sum_{\bm{b} \in \mathcal{B}_{i}^{(n+1)}} \frac{1}{L_b}\lVert \bm{b}- C_i^{(n+1)}(\bm{b})\rVert^2_2 \le \frac{1}{|\mathcal{B}_{i}^{(n+1)}|}\sum_{\bm{b} \in \mathcal{B}_{i}^{(n+1)}} \frac{1}{L_b}\lVert \bm{b}- C_i^{(n)}(\bm{b})\rVert^2_2
\end{equation}

The above equation states that quantizing the given block cluster $\mathcal{B}_i$ after updating the associated codebook from $C_i^{(n)}$ to $C_i^{(n+1)}$ results in lower quantization MSE. Since this is true for all the block clusters, we derive the following result: 
\begin{equation}
\begin{split}
\label{eq:mse_ineq_2}
     J^{(n+1)} &= \frac{1}{N_c} \sum_{i=1}^{N_c} \frac{1}{|\mathcal{B}_{i}^{(n+1)}|}\sum_{\bm{b} \in \mathcal{B}_{i}^{(n+1)}} \frac{1}{L_b}\lVert \bm{b}- C_i^{(n+1)}(\bm{b})\rVert^2_2  \le \tilde{J}^{(n+1)}   
\end{split}
\end{equation}

Following (\ref{eq:mse_ineq_1}) and (\ref{eq:mse_ineq_2}), we find that the quantization MSE is non-increasing for each iteration, that is, $J^{(1)} \ge J^{(2)} \ge J^{(3)} \ge \ldots \ge J^{(M)}$ where $M$ is the maximum number of iterations. 
%Therefore, we can say that if the algorithm converges, then it must be that it has converged to a local minimum. 
\hfill $\blacksquare$


\begin{figure}
    \begin{center}
    \includegraphics[width=0.5\textwidth]{sections//figures/mse_vs_iter.pdf}
    \end{center}
    \caption{\small NMSE vs iterations during LO-BCQ compared to other block quantization proposals}
    \label{fig:nmse_vs_iter}
\end{figure}

Figure \ref{fig:nmse_vs_iter} shows the empirical convergence of LO-BCQ across several block lengths and number of codebooks. Also, the MSE achieved by LO-BCQ is compared to baselines such as MXFP and VSQ. As shown, LO-BCQ converges to a lower MSE than the baselines. Further, we achieve better convergence for larger number of codebooks ($N_c$) and for a smaller block length ($L_b$), both of which increase the bitwidth of BCQ (see Eq \ref{eq:bitwidth_bcq}).


\subsection{Additional Accuracy Results}
%Table \ref{tab:lobcq_config} lists the various LOBCQ configurations and their corresponding bitwidths.
\begin{table}
\setlength{\tabcolsep}{4.75pt}
\begin{center}
\caption{\label{tab:lobcq_config} Various LO-BCQ configurations and their bitwidths.}
\begin{tabular}{|c||c|c|c|c||c|c||c|} 
\hline
 & \multicolumn{4}{|c||}{$L_b=8$} & \multicolumn{2}{|c||}{$L_b=4$} & $L_b=2$ \\
 \hline
 \backslashbox{$L_A$\kern-1em}{\kern-1em$N_c$} & 2 & 4 & 8 & 16 & 2 & 4 & 2 \\
 \hline
 64 & 4.25 & 4.375 & 4.5 & 4.625 & 4.375 & 4.625 & 4.625\\
 \hline
 32 & 4.375 & 4.5 & 4.625& 4.75 & 4.5 & 4.75 & 4.75 \\
 \hline
 16 & 4.625 & 4.75& 4.875 & 5 & 4.75 & 5 & 5 \\
 \hline
\end{tabular}
\end{center}
\end{table}

%\subsection{Perplexity achieved by various LO-BCQ configurations on Wikitext-103 dataset}

\begin{table} \centering
\begin{tabular}{|c||c|c|c|c||c|c||c|} 
\hline
 $L_b \rightarrow$& \multicolumn{4}{c||}{8} & \multicolumn{2}{c||}{4} & 2\\
 \hline
 \backslashbox{$L_A$\kern-1em}{\kern-1em$N_c$} & 2 & 4 & 8 & 16 & 2 & 4 & 2  \\
 %$N_c \rightarrow$ & 2 & 4 & 8 & 16 & 2 & 4 & 2 \\
 \hline
 \hline
 \multicolumn{8}{c}{GPT3-1.3B (FP32 PPL = 9.98)} \\ 
 \hline
 \hline
 64 & 10.40 & 10.23 & 10.17 & 10.15 &  10.28 & 10.18 & 10.19 \\
 \hline
 32 & 10.25 & 10.20 & 10.15 & 10.12 &  10.23 & 10.17 & 10.17 \\
 \hline
 16 & 10.22 & 10.16 & 10.10 & 10.09 &  10.21 & 10.14 & 10.16 \\
 \hline
  \hline
 \multicolumn{8}{c}{GPT3-8B (FP32 PPL = 7.38)} \\ 
 \hline
 \hline
 64 & 7.61 & 7.52 & 7.48 &  7.47 &  7.55 &  7.49 & 7.50 \\
 \hline
 32 & 7.52 & 7.50 & 7.46 &  7.45 &  7.52 &  7.48 & 7.48  \\
 \hline
 16 & 7.51 & 7.48 & 7.44 &  7.44 &  7.51 &  7.49 & 7.47  \\
 \hline
\end{tabular}
\caption{\label{tab:ppl_gpt3_abalation} Wikitext-103 perplexity across GPT3-1.3B and 8B models.}
\end{table}

\begin{table} \centering
\begin{tabular}{|c||c|c|c|c||} 
\hline
 $L_b \rightarrow$& \multicolumn{4}{c||}{8}\\
 \hline
 \backslashbox{$L_A$\kern-1em}{\kern-1em$N_c$} & 2 & 4 & 8 & 16 \\
 %$N_c \rightarrow$ & 2 & 4 & 8 & 16 & 2 & 4 & 2 \\
 \hline
 \hline
 \multicolumn{5}{|c|}{Llama2-7B (FP32 PPL = 5.06)} \\ 
 \hline
 \hline
 64 & 5.31 & 5.26 & 5.19 & 5.18  \\
 \hline
 32 & 5.23 & 5.25 & 5.18 & 5.15  \\
 \hline
 16 & 5.23 & 5.19 & 5.16 & 5.14  \\
 \hline
 \multicolumn{5}{|c|}{Nemotron4-15B (FP32 PPL = 5.87)} \\ 
 \hline
 \hline
 64  & 6.3 & 6.20 & 6.13 & 6.08  \\
 \hline
 32  & 6.24 & 6.12 & 6.07 & 6.03  \\
 \hline
 16  & 6.12 & 6.14 & 6.04 & 6.02  \\
 \hline
 \multicolumn{5}{|c|}{Nemotron4-340B (FP32 PPL = 3.48)} \\ 
 \hline
 \hline
 64 & 3.67 & 3.62 & 3.60 & 3.59 \\
 \hline
 32 & 3.63 & 3.61 & 3.59 & 3.56 \\
 \hline
 16 & 3.61 & 3.58 & 3.57 & 3.55 \\
 \hline
\end{tabular}
\caption{\label{tab:ppl_llama7B_nemo15B} Wikitext-103 perplexity compared to FP32 baseline in Llama2-7B and Nemotron4-15B, 340B models}
\end{table}

%\subsection{Perplexity achieved by various LO-BCQ configurations on MMLU dataset}


\begin{table} \centering
\begin{tabular}{|c||c|c|c|c||c|c|c|c|} 
\hline
 $L_b \rightarrow$& \multicolumn{4}{c||}{8} & \multicolumn{4}{c||}{8}\\
 \hline
 \backslashbox{$L_A$\kern-1em}{\kern-1em$N_c$} & 2 & 4 & 8 & 16 & 2 & 4 & 8 & 16  \\
 %$N_c \rightarrow$ & 2 & 4 & 8 & 16 & 2 & 4 & 2 \\
 \hline
 \hline
 \multicolumn{5}{|c|}{Llama2-7B (FP32 Accuracy = 45.8\%)} & \multicolumn{4}{|c|}{Llama2-70B (FP32 Accuracy = 69.12\%)} \\ 
 \hline
 \hline
 64 & 43.9 & 43.4 & 43.9 & 44.9 & 68.07 & 68.27 & 68.17 & 68.75 \\
 \hline
 32 & 44.5 & 43.8 & 44.9 & 44.5 & 68.37 & 68.51 & 68.35 & 68.27  \\
 \hline
 16 & 43.9 & 42.7 & 44.9 & 45 & 68.12 & 68.77 & 68.31 & 68.59  \\
 \hline
 \hline
 \multicolumn{5}{|c|}{GPT3-22B (FP32 Accuracy = 38.75\%)} & \multicolumn{4}{|c|}{Nemotron4-15B (FP32 Accuracy = 64.3\%)} \\ 
 \hline
 \hline
 64 & 36.71 & 38.85 & 38.13 & 38.92 & 63.17 & 62.36 & 63.72 & 64.09 \\
 \hline
 32 & 37.95 & 38.69 & 39.45 & 38.34 & 64.05 & 62.30 & 63.8 & 64.33  \\
 \hline
 16 & 38.88 & 38.80 & 38.31 & 38.92 & 63.22 & 63.51 & 63.93 & 64.43  \\
 \hline
\end{tabular}
\caption{\label{tab:mmlu_abalation} Accuracy on MMLU dataset across GPT3-22B, Llama2-7B, 70B and Nemotron4-15B models.}
\end{table}


%\subsection{Perplexity achieved by various LO-BCQ configurations on LM evaluation harness}

\begin{table} \centering
\begin{tabular}{|c||c|c|c|c||c|c|c|c|} 
\hline
 $L_b \rightarrow$& \multicolumn{4}{c||}{8} & \multicolumn{4}{c||}{8}\\
 \hline
 \backslashbox{$L_A$\kern-1em}{\kern-1em$N_c$} & 2 & 4 & 8 & 16 & 2 & 4 & 8 & 16  \\
 %$N_c \rightarrow$ & 2 & 4 & 8 & 16 & 2 & 4 & 2 \\
 \hline
 \hline
 \multicolumn{5}{|c|}{Race (FP32 Accuracy = 37.51\%)} & \multicolumn{4}{|c|}{Boolq (FP32 Accuracy = 64.62\%)} \\ 
 \hline
 \hline
 64 & 36.94 & 37.13 & 36.27 & 37.13 & 63.73 & 62.26 & 63.49 & 63.36 \\
 \hline
 32 & 37.03 & 36.36 & 36.08 & 37.03 & 62.54 & 63.51 & 63.49 & 63.55  \\
 \hline
 16 & 37.03 & 37.03 & 36.46 & 37.03 & 61.1 & 63.79 & 63.58 & 63.33  \\
 \hline
 \hline
 \multicolumn{5}{|c|}{Winogrande (FP32 Accuracy = 58.01\%)} & \multicolumn{4}{|c|}{Piqa (FP32 Accuracy = 74.21\%)} \\ 
 \hline
 \hline
 64 & 58.17 & 57.22 & 57.85 & 58.33 & 73.01 & 73.07 & 73.07 & 72.80 \\
 \hline
 32 & 59.12 & 58.09 & 57.85 & 58.41 & 73.01 & 73.94 & 72.74 & 73.18  \\
 \hline
 16 & 57.93 & 58.88 & 57.93 & 58.56 & 73.94 & 72.80 & 73.01 & 73.94  \\
 \hline
\end{tabular}
\caption{\label{tab:mmlu_abalation} Accuracy on LM evaluation harness tasks on GPT3-1.3B model.}
\end{table}

\begin{table} \centering
\begin{tabular}{|c||c|c|c|c||c|c|c|c|} 
\hline
 $L_b \rightarrow$& \multicolumn{4}{c||}{8} & \multicolumn{4}{c||}{8}\\
 \hline
 \backslashbox{$L_A$\kern-1em}{\kern-1em$N_c$} & 2 & 4 & 8 & 16 & 2 & 4 & 8 & 16  \\
 %$N_c \rightarrow$ & 2 & 4 & 8 & 16 & 2 & 4 & 2 \\
 \hline
 \hline
 \multicolumn{5}{|c|}{Race (FP32 Accuracy = 41.34\%)} & \multicolumn{4}{|c|}{Boolq (FP32 Accuracy = 68.32\%)} \\ 
 \hline
 \hline
 64 & 40.48 & 40.10 & 39.43 & 39.90 & 69.20 & 68.41 & 69.45 & 68.56 \\
 \hline
 32 & 39.52 & 39.52 & 40.77 & 39.62 & 68.32 & 67.43 & 68.17 & 69.30  \\
 \hline
 16 & 39.81 & 39.71 & 39.90 & 40.38 & 68.10 & 66.33 & 69.51 & 69.42  \\
 \hline
 \hline
 \multicolumn{5}{|c|}{Winogrande (FP32 Accuracy = 67.88\%)} & \multicolumn{4}{|c|}{Piqa (FP32 Accuracy = 78.78\%)} \\ 
 \hline
 \hline
 64 & 66.85 & 66.61 & 67.72 & 67.88 & 77.31 & 77.42 & 77.75 & 77.64 \\
 \hline
 32 & 67.25 & 67.72 & 67.72 & 67.00 & 77.31 & 77.04 & 77.80 & 77.37  \\
 \hline
 16 & 68.11 & 68.90 & 67.88 & 67.48 & 77.37 & 78.13 & 78.13 & 77.69  \\
 \hline
\end{tabular}
\caption{\label{tab:mmlu_abalation} Accuracy on LM evaluation harness tasks on GPT3-8B model.}
\end{table}

\begin{table} \centering
\begin{tabular}{|c||c|c|c|c||c|c|c|c|} 
\hline
 $L_b \rightarrow$& \multicolumn{4}{c||}{8} & \multicolumn{4}{c||}{8}\\
 \hline
 \backslashbox{$L_A$\kern-1em}{\kern-1em$N_c$} & 2 & 4 & 8 & 16 & 2 & 4 & 8 & 16  \\
 %$N_c \rightarrow$ & 2 & 4 & 8 & 16 & 2 & 4 & 2 \\
 \hline
 \hline
 \multicolumn{5}{|c|}{Race (FP32 Accuracy = 40.67\%)} & \multicolumn{4}{|c|}{Boolq (FP32 Accuracy = 76.54\%)} \\ 
 \hline
 \hline
 64 & 40.48 & 40.10 & 39.43 & 39.90 & 75.41 & 75.11 & 77.09 & 75.66 \\
 \hline
 32 & 39.52 & 39.52 & 40.77 & 39.62 & 76.02 & 76.02 & 75.96 & 75.35  \\
 \hline
 16 & 39.81 & 39.71 & 39.90 & 40.38 & 75.05 & 73.82 & 75.72 & 76.09  \\
 \hline
 \hline
 \multicolumn{5}{|c|}{Winogrande (FP32 Accuracy = 70.64\%)} & \multicolumn{4}{|c|}{Piqa (FP32 Accuracy = 79.16\%)} \\ 
 \hline
 \hline
 64 & 69.14 & 70.17 & 70.17 & 70.56 & 78.24 & 79.00 & 78.62 & 78.73 \\
 \hline
 32 & 70.96 & 69.69 & 71.27 & 69.30 & 78.56 & 79.49 & 79.16 & 78.89  \\
 \hline
 16 & 71.03 & 69.53 & 69.69 & 70.40 & 78.13 & 79.16 & 79.00 & 79.00  \\
 \hline
\end{tabular}
\caption{\label{tab:mmlu_abalation} Accuracy on LM evaluation harness tasks on GPT3-22B model.}
\end{table}

\begin{table} \centering
\begin{tabular}{|c||c|c|c|c||c|c|c|c|} 
\hline
 $L_b \rightarrow$& \multicolumn{4}{c||}{8} & \multicolumn{4}{c||}{8}\\
 \hline
 \backslashbox{$L_A$\kern-1em}{\kern-1em$N_c$} & 2 & 4 & 8 & 16 & 2 & 4 & 8 & 16  \\
 %$N_c \rightarrow$ & 2 & 4 & 8 & 16 & 2 & 4 & 2 \\
 \hline
 \hline
 \multicolumn{5}{|c|}{Race (FP32 Accuracy = 44.4\%)} & \multicolumn{4}{|c|}{Boolq (FP32 Accuracy = 79.29\%)} \\ 
 \hline
 \hline
 64 & 42.49 & 42.51 & 42.58 & 43.45 & 77.58 & 77.37 & 77.43 & 78.1 \\
 \hline
 32 & 43.35 & 42.49 & 43.64 & 43.73 & 77.86 & 75.32 & 77.28 & 77.86  \\
 \hline
 16 & 44.21 & 44.21 & 43.64 & 42.97 & 78.65 & 77 & 76.94 & 77.98  \\
 \hline
 \hline
 \multicolumn{5}{|c|}{Winogrande (FP32 Accuracy = 69.38\%)} & \multicolumn{4}{|c|}{Piqa (FP32 Accuracy = 78.07\%)} \\ 
 \hline
 \hline
 64 & 68.9 & 68.43 & 69.77 & 68.19 & 77.09 & 76.82 & 77.09 & 77.86 \\
 \hline
 32 & 69.38 & 68.51 & 68.82 & 68.90 & 78.07 & 76.71 & 78.07 & 77.86  \\
 \hline
 16 & 69.53 & 67.09 & 69.38 & 68.90 & 77.37 & 77.8 & 77.91 & 77.69  \\
 \hline
\end{tabular}
\caption{\label{tab:mmlu_abalation} Accuracy on LM evaluation harness tasks on Llama2-7B model.}
\end{table}

\begin{table} \centering
\begin{tabular}{|c||c|c|c|c||c|c|c|c|} 
\hline
 $L_b \rightarrow$& \multicolumn{4}{c||}{8} & \multicolumn{4}{c||}{8}\\
 \hline
 \backslashbox{$L_A$\kern-1em}{\kern-1em$N_c$} & 2 & 4 & 8 & 16 & 2 & 4 & 8 & 16  \\
 %$N_c \rightarrow$ & 2 & 4 & 8 & 16 & 2 & 4 & 2 \\
 \hline
 \hline
 \multicolumn{5}{|c|}{Race (FP32 Accuracy = 48.8\%)} & \multicolumn{4}{|c|}{Boolq (FP32 Accuracy = 85.23\%)} \\ 
 \hline
 \hline
 64 & 49.00 & 49.00 & 49.28 & 48.71 & 82.82 & 84.28 & 84.03 & 84.25 \\
 \hline
 32 & 49.57 & 48.52 & 48.33 & 49.28 & 83.85 & 84.46 & 84.31 & 84.93  \\
 \hline
 16 & 49.85 & 49.09 & 49.28 & 48.99 & 85.11 & 84.46 & 84.61 & 83.94  \\
 \hline
 \hline
 \multicolumn{5}{|c|}{Winogrande (FP32 Accuracy = 79.95\%)} & \multicolumn{4}{|c|}{Piqa (FP32 Accuracy = 81.56\%)} \\ 
 \hline
 \hline
 64 & 78.77 & 78.45 & 78.37 & 79.16 & 81.45 & 80.69 & 81.45 & 81.5 \\
 \hline
 32 & 78.45 & 79.01 & 78.69 & 80.66 & 81.56 & 80.58 & 81.18 & 81.34  \\
 \hline
 16 & 79.95 & 79.56 & 79.79 & 79.72 & 81.28 & 81.66 & 81.28 & 80.96  \\
 \hline
\end{tabular}
\caption{\label{tab:mmlu_abalation} Accuracy on LM evaluation harness tasks on Llama2-70B model.}
\end{table}

%\section{MSE Studies}
%\textcolor{red}{TODO}


\subsection{Number Formats and Quantization Method}
\label{subsec:numFormats_quantMethod}
\subsubsection{Integer Format}
An $n$-bit signed integer (INT) is typically represented with a 2s-complement format \citep{yao2022zeroquant,xiao2023smoothquant,dai2021vsq}, where the most significant bit denotes the sign.

\subsubsection{Floating Point Format}
An $n$-bit signed floating point (FP) number $x$ comprises of a 1-bit sign ($x_{\mathrm{sign}}$), $B_m$-bit mantissa ($x_{\mathrm{mant}}$) and $B_e$-bit exponent ($x_{\mathrm{exp}}$) such that $B_m+B_e=n-1$. The associated constant exponent bias ($E_{\mathrm{bias}}$) is computed as $(2^{{B_e}-1}-1)$. We denote this format as $E_{B_e}M_{B_m}$.  

\subsubsection{Quantization Scheme}
\label{subsec:quant_method}
A quantization scheme dictates how a given unquantized tensor is converted to its quantized representation. We consider FP formats for the purpose of illustration. Given an unquantized tensor $\bm{X}$ and an FP format $E_{B_e}M_{B_m}$, we first, we compute the quantization scale factor $s_X$ that maps the maximum absolute value of $\bm{X}$ to the maximum quantization level of the $E_{B_e}M_{B_m}$ format as follows:
\begin{align}
\label{eq:sf}
    s_X = \frac{\mathrm{max}(|\bm{X}|)}{\mathrm{max}(E_{B_e}M_{B_m})}
\end{align}
In the above equation, $|\cdot|$ denotes the absolute value function.

Next, we scale $\bm{X}$ by $s_X$ and quantize it to $\hat{\bm{X}}$ by rounding it to the nearest quantization level of $E_{B_e}M_{B_m}$ as:

\begin{align}
\label{eq:tensor_quant}
    \hat{\bm{X}} = \text{round-to-nearest}\left(\frac{\bm{X}}{s_X}, E_{B_e}M_{B_m}\right)
\end{align}

We perform dynamic max-scaled quantization \citep{wu2020integer}, where the scale factor $s$ for activations is dynamically computed during runtime.

\subsection{Vector Scaled Quantization}
\begin{wrapfigure}{r}{0.35\linewidth}
  \centering
  \includegraphics[width=\linewidth]{sections/figures/vsquant.jpg}
  \caption{\small Vectorwise decomposition for per-vector scaled quantization (VSQ \citep{dai2021vsq}).}
  \label{fig:vsquant}
\end{wrapfigure}
During VSQ \citep{dai2021vsq}, the operand tensors are decomposed into 1D vectors in a hardware friendly manner as shown in Figure \ref{fig:vsquant}. Since the decomposed tensors are used as operands in matrix multiplications during inference, it is beneficial to perform this decomposition along the reduction dimension of the multiplication. The vectorwise quantization is performed similar to tensorwise quantization described in Equations \ref{eq:sf} and \ref{eq:tensor_quant}, where a scale factor $s_v$ is required for each vector $\bm{v}$ that maps the maximum absolute value of that vector to the maximum quantization level. While smaller vector lengths can lead to larger accuracy gains, the associated memory and computational overheads due to the per-vector scale factors increases. To alleviate these overheads, VSQ \citep{dai2021vsq} proposed a second level quantization of the per-vector scale factors to unsigned integers, while MX \citep{rouhani2023shared} quantizes them to integer powers of 2 (denoted as $2^{INT}$).

\subsubsection{MX Format}
The MX format proposed in \citep{rouhani2023microscaling} introduces the concept of sub-block shifting. For every two scalar elements of $b$-bits each, there is a shared exponent bit. The value of this exponent bit is determined through an empirical analysis that targets minimizing quantization MSE. We note that the FP format $E_{1}M_{b}$ is strictly better than MX from an accuracy perspective since it allocates a dedicated exponent bit to each scalar as opposed to sharing it across two scalars. Therefore, we conservatively bound the accuracy of a $b+2$-bit signed MX format with that of a $E_{1}M_{b}$ format in our comparisons. For instance, we use E1M2 format as a proxy for MX4.

\begin{figure}
    \centering
    \includegraphics[width=1\linewidth]{sections//figures/BlockFormats.pdf}
    \caption{\small Comparing LO-BCQ to MX format.}
    \label{fig:block_formats}
\end{figure}

Figure \ref{fig:block_formats} compares our $4$-bit LO-BCQ block format to MX \citep{rouhani2023microscaling}. As shown, both LO-BCQ and MX decompose a given operand tensor into block arrays and each block array into blocks. Similar to MX, we find that per-block quantization ($L_b < L_A$) leads to better accuracy due to increased flexibility. While MX achieves this through per-block $1$-bit micro-scales, we associate a dedicated codebook to each block through a per-block codebook selector. Further, MX quantizes the per-block array scale-factor to E8M0 format without per-tensor scaling. In contrast during LO-BCQ, we find that per-tensor scaling combined with quantization of per-block array scale-factor to E4M3 format results in superior inference accuracy across models. 



\end{document}
\endinput
%%
%% End of file `sample-sigconf.tex'.
