
% \clearpage

\appendix
\section{Appendix}



\subsection{Bayesian Optimization}


\begin{algorithm}[H]
\caption{Bayesian Optimization}
\label{alg:bo}
\begin{algorithmic}[1]
\STATE \textbf{Input:}\\
\hspace{1em}Evaluator function $f$\\
\hspace{1em}Search Budget $B$\\
\hspace{1em}Acquisition function $a$\\
\hspace{1em}Surrogate model $S$
\STATE \textbf{Define:} History observations $D = \emptyset$
    \FOR{$t \in 0, \dots, B-1$}
        \STATE $S_t = \text{Fit}(D)$ \COMMENT{Get the surrogate model with available observations}
        \STATE $x^* = \argmin_{x\in \chi} a(S_t, D)$ \COMMENT{Sample next configuration to evaluate}
        \STATE $y^* = f(x^*)$
        \STATE $D = D \cup \{(x^*, y^*)\}$
    \ENDFOR
\STATE \textbf{Output:} $H$
\end{algorithmic}
\end{algorithm}

Assuming our goal is to minimize the objective function $f(x)$. At each iteration $i$, Bayesian optimization (BO) constructs a surrogate model $S := p(y \mid x, D)$ to approximate the objective function $f$, where the set $D = \{(x_n,y_n)\}^{i-1}_{n=0}$ contains previously observed data points. BO also leverages an acquisition function to balance exploration and exploitation, based on predictions from $S$. The optimization process is be summarized in ~Alg \ref{alg:bo}. A widely used acquisition function is \textbf{Expected Improvement (EI)} \cite{Jones1998EfficientGO}:

\begin{align*}
    \text{EI}_{y^*}(x) := \int_{-\infty}^{y^*}(y^* - y)p(y\mid x, D)dy
\end{align*}

EI intuitively indicates the expectation that the objective function $f(x)$, approximated by $S$, will be below a certain threshold $y^*$.

\subsubsection{Optimizing EI with Tree-structured Parzen Estimator (TPE)}
\label{appdx:TPE}
In this work, we choose TPE as our surrogate model for its natural support of mixed discrete and continuous search space and efficiency in model building. TPE ~\cite{NIPS2011_TPE} employs kernel density estimators (KDEs) as the probabilistic model. Instead of modeling $p(y\mid x, D)$ directly, it models $p(x\mid y, D)$ by:

\begin{align*}
    p(x\mid y, D) := \begin{cases}
      l(x) := p(x\mid D^{(l)}) & y \leq y^{\pi}\\
      g(x) := p(x\mid D^{(g)}) & y > y^{\pi}
    \end{cases}       
\end{align*}

where the entire observation set $D$ is separated into two groups and $y^{\pi}$ is the top $\pi$ quantile evaluation results in $D$. The control parameter $\pi$ is calculated based on $|D|$ in each iteration ~\cite{Watanabe2023TreestructuredPE}. The next configuration $x^*$ to evaluate is proposed with:
\begin{align*}
    x^* := \argmax_{x \in \chi} \frac{l(x)}{g(x)}
\end{align*}

Intuitively this represents the configuration that maximizes the probability of residing in top-$\pi$ observations versus in worse group. ~\cite{bergstra2011tpe} has shown this is theoretically equivalent to minimizing EI. In \sysname, we also add observations that violate metric threshold to group $D^{(g)}$ so that the surrogate model can reason about the constraints.

\subsection{Budget Constraints}
Here we show that each layer $l$ will follow the assigned budget $B_l$ in ~\ref{alg:outer}. Starting from the outer-most layer. The number of proposed configurations at this layer is $C_1 = KW \leq B_1$. The total number of budget consumed at the next layer is:

\begin{align*}
    C_2 &= \sum_{k}^K \sum_{s}^S r_s\cdot|\Theta_s|\\
    &\leq\sum_{k}^K\sum_{s}^S R\eta^s \cdot W\eta^{-s} \\
    &\leq \frac{B_1}{W} \cdot \frac{B_{2}}{R} \cdot RW
\end{align*}
which aligns with the constraint of $B_1\cdot B_{2}$. The calculation can be easily extended to three-layer situation, where multiple instances of the search at the middle layer will be executed each with an uneven budget. The total number of evaluation required is:

\begin{align*}
    c_3[i] &\leq c_{2}[i]\cdot B_3\\
    C_3 &= \sum_i c_{3}[i] \\
    &\leq B_{3} \sum_i c_2[i] \\
    &\leq B_3 \prod_{j=1}^2B_j
\end{align*}
where $c_l[i]$ represents the budget assigned to $i^{th}$ optimize instance at search layer $l$. Sum of each $c_l[i]$ at layer $i$ stands for the total number of configurations proposed at that layer, which will be the product of all preceding budgets.


\subsection{Programming Model}
\label{sec:programming-model}

\sysname\ natively supports programs written in LangChain/LangGraph and DSPy. In addition, \sysname\ comes with its own programming model: \texttt{cognify.Model} and \texttt{cognify.StructuredModel}. These two constructs serve as drop-in replacements for all model calls in a user's program. There are four necessary components to construct a \texttt{cognify.Model}:
\begin{enumerate}
    \item \textbf{System prompt}: this defines the role of the model
    \item \textbf{Inputs}: placeholder values that are filled in at runtime based on the end-user request
    \item \textbf{Output format}: either a label assigned to the output value or a schema 
    \item \textbf{LM backend}: the model that will be called (in the absence of model selection)
\end{enumerate}

To invoke the optimizer, users simply need to register their workflow entry point, data loader, and evaluator functions with Cognify. The following snippet is an example of a workflow (defined in \texttt{workflow.py}).
\\
\\
\begin{lstlisting}[style=pythonstyle]
import cognify

math_solver = cognify.Model(
    "math_solver",
    system_prompt="You are a helpful assistant. Your task is to solve a math problem by identifying key variables and formulating important equations.",
    input_variables=[cognify.Input("problem")],
    output=cognify.OutputLabel("math_model"),
    lm_config=cognify.LMConfig(model="gpt-4o")
)

# Workflow
@cognify.register_workflow
def math_solver_workflow(problem):
    answer = math_solver(inputs={
        "problem": problem
    })
    return {"prediction": answer}
\end{lstlisting}

Then, in \texttt{config.py}, users can defined their data loader and evaluator.

\begin{lstlisting}[style=pythonstyle]
# Data loader
@cognify.register_data_loader
def load_data():
    with open("data.json", "r") as f:
        data = json.load(f)

   dataset = []
   for d in data:
      input_sample = {
         'problem': d["problem"],
      }
      label = {
         'label': d["solution"],
      }
      dataset.append(
        (input_sample, label)
      )

   # split the data into train, validation, and test
   return dataset[:20], dataset[20:30], dataset[30:]

# Evaluator
@cognify.register_evaluator
def score_solution(prediction, label):
    return prediction == label
\end{lstlisting}


Invoking the optimizer is straightforward.

\begin{lstlisting}[style=pythonstyle]
$ cognify optimize workflow.py
\end{lstlisting}

\subsection{Example Workflow}
\label{sec:apdx-example}

The following is the workflow used for the data visualization.
{
\begin{figure}[h]
\begin{center}
\centerline{\includegraphics[width=\columnwidth]{Figures/datavis.pdf}}
\mycap{Data Visualization Workflow}
{
Input is natural language description of a plotting task along with a CSV file containing relevant data. Expected output is a figure. Workflow contains 4 agents.
}
\end{center}
\Description{A task graph with 5 nodes and 2 conditional edges.}
\label{fig-datavis-workflow}
\end{figure}
}

An example optimization for this workflow is as follows:
\begin{itemize}
    \item Query expansion: Few-shot example (shown below)
    \item Initial code generation: Reasoning (with planning option, \ie, ``break down the task into simpler sub-tasks...") 
    \item Debugger: Reasoning (with chain-of-thought option, \ie, ``think step-by-step...")
\end{itemize}

The example used for query expansion was selected due to strong performance during the optimization process:\\

\begin{mdframed}[backgroundcolor=gray!20, linewidth=1pt]
\#\# Example 1 \#\#\\

\textbf{Query:} Create a combination chart from the ``data.csv" dataset, which contains sales data for various mobile phone brands across different quarters of the year. The columns in the CSV file are [``Quarter", ``Samsung", ``Nokia/Microsoft", ``Apple", ``LG", ``ZTE", ``Huawei"]. For each brand, create a box plot to represent the distribution of their sales data. On each box plot, display all the individual sales data points for that brand. Additionally, calculate the average sales for each brand and draw a line connecting these average values across the box plots. Use a consistent color scheme for the same quarter across different years, varying the shades to distinguish between years. Include a legend to aid in understanding the color coding.\\

\textbf{Output:} To create a combination chart from the ``data.csv" dataset that includes box plots for sales data of various mobile phone brands and a line connecting the average sales, follow these detailed instructions:

\#\#\# Step-by-Step Instructions\\
\#\#\#\# Step 1: Import Necessary Libraries\\
You will need to import the following libraries to handle data manipulation and visualization:
\begin{lstlisting}[style=pythonstyle]
import pandas as pd
import matplotlib.pyplot as plt
import seaborn as sns
\end{lstlisting}

\#\#\#\# Step 2: Load the Data\\
Read the CSV file into a pandas DataFrame. Ensure that the file ``data.csv" is in the same directory as your script or provide the full path.

\begin{lstlisting}[style=pythonstyle]
# Load the dataset
data = pd.read_csv('data.csv')
\end{lstlisting}

\#\#\#\# Step 3: Data Preparation\\
Check the structure of the DataFrame to ensure it has been loaded correctly. We will also need to calculate the average sales for each brand.

\begin{lstlisting}[style=pythonstyle]
# Display the first few rows of the dataset
print(data.head())

# Calculate the average sales for each brand
average_sales = data.mean()
\end{lstlisting}

\#\#\#\# Step 4: Create Box Plots\\
We will create box plots for each brand using seaborn. We will also overlay the individual sales data points on the box plots.

\begin{lstlisting}[style=pythonstyle]
# Set the style of seaborn
sns.set(style="whitegrid")

# Create a figure and axis
plt.figure(figsize=(12, 6))

# Melt the DataFrame to long format for seaborn
data_melted = data.melt(id_vars='Quarter', var_name='Brand', value_name='Sales')

# Create the box plot
box_plot = sns.boxplot(x='Brand', y='Sales', hue='Quarter', data=data_melted, palette='Set2', dodge=True)

# Overlay individual data points
sns.stripplot(x='Brand', y='Sales', hue='Quarter', data=data_melted, palette='Set2', dodge=True, marker='o', alpha=0.5)
\end{lstlisting}

\#\#\#\# Step 5: Plot Average Sales Line\\
Now we will plot the average sales for each brand as a line connecting the average values.

\begin{lstlisting}[style=pythonstyle]
# Create a line plot for average sales
for i, brand in enumerate(average_sales.index):
  plt.plot([i], [average_sales[brand]], marker='o', color='black', markersize=8, label=f'Avg {brand}' if i == 0 else "")
\end{lstlisting}

\#\#\#\# Step 6: Customize the Plot\\
Add titles, labels, and a legend to make the plot more informative.

\begin{lstlisting}[style=pythonstyle]
# Add titles and labels
plt.title("Sales Distribution of Mobile Phone Brands by Quarter")
plt.xlabel("Brand")
plt.ylabel("Sales")
plt.xticks(rotation=45)

# Add a legend
plt.legend(title="Quarter", bbox_to_anchor=(1.05, 1), loc='upper left')

# Show the plot
plt.tight_layout()
plt.show()
\end{lstlisting}

\#\#\#\# Step 7: Save the Plot (Optional)\\
If you want to save the plot as an image file, you can do so by adding the following line before plt.show():

\begin{lstlisting}[style=pythonstyle]
plt.savefig("sales_distribution_chart.png", bbox_inches='tight')
\end{lstlisting}

\#\#\# Complete Code\\
Here is the complete code for your reference:

\begin{lstlisting}[style=pythonstyle]
import pandas as pd
import matplotlib.pyplot as plt
import seaborn as sns

# Load the dataset
data = pd.read_csv('data.csv')

# Display the first few rows of the dataset
print(data.head())

# Set the style of seaborn
sns.set(style="whitegrid")

# Create a figure and axis
plt.figure(figsize=(12, 6))

# Melt the DataFrame to long format for seaborn
data_melted = data.melt(id_vars='Quarter', var_name='Brand', value_name='Sales')

# Create the box plot
box_plot = sns.boxplot(x='Brand', y='Sales', hue='Quarter', data=data_melted, palette='Set2', dodge=True)

# Overlay individual data points
sns.stripplot(x='Brand', y='Sales', hue='Quarter', data=data_melted, palette='Set2', dodge=True, marker='o', alpha=0.5)

# Calculate the average sales for each brand
average_sales = data.mean()

# Create a line plot for average sales
for i, brand in enumerate(average_sales.index):
  plt.plot([i], [average_sales[brand]], marker='o', color='black', markersize=8, label=f'Avg {brand}' if i == 0 else "")

# Add titles and labels
plt.title("Sales Distribution of Mobile Phone Brands by Quarter")
plt.xlabel("Brand")
plt.ylabel("Sales")
plt.xticks(rotation=45)

# Add a legend
plt.legend(title="Quarter", bbox_to_anchor=(1.05, 1), loc='upper left')

# Show the plot
plt.tight_layout()
plt.show()

# Optional: Save the plot as an image file
plt.savefig("sales_distribution_chart.png", bbox_inches='tight')
\end{lstlisting}

\#\#\# Conclusion\\
This code will generate a combination chart with box plots for each mobile phone brand's sales data, individual sales points, and a line connecting the average sales for each brand. Adjust the color palette and other parameters as needed to fit your specific requirements.


\end{mdframed}



% \begin{lstlisting}[style=pythonstyle]
% "inputs": {
% "query": "Create a combination chart from the \"data.csv\" dataset, which contains sales data for various mobile phone brands across different quarters of the year. The columns in the CSV file are [\"Quarter\", \"Samsung\", \"Nokia/Microsoft\", \"Apple\", \"LG\", \"ZTE\", \"Huawei\"]. For each brand, create a box plot to represent the distribution of their sales data. On each box plot, display all the individual sales data points for that brand. Additionally, calculate the average sales for each brand and draw a line connecting these average values across the box plots. Use a consistent color scheme for the same quarter across different years, varying the shades to distinguish between years. Include a legend to aid in understanding the color coding."
% },
% "output": "To create a combination chart from the \"data.csv\" dataset that includes box plots for sales data of various mobile phone brands and a line connecting the average sales, follow these detailed instructions:\n\n\#\#\# Step-by-Step Instructions\n\n\#\#\#\# Step 1: Import Necessary Libraries\nYou will need to import the following libraries to handle data manipulation and visualization:\n\n```python\nimport pandas as pd\nimport matplotlib.pyplot as plt\nimport seaborn as sns\n```\n\n\#\#\#\# Step 2: Load the Data\nRead the CSV file into a pandas DataFrame. Ensure that the file \"data.csv\" is in the same directory as your script or provide the full path.\n\n```python\n# Load the dataset\ndata = pd.read_csv('data.csv')\n```\n\n\#\#\#\# Step 3: Data Preparation\nCheck the structure of the DataFrame to ensure it has been loaded correctly. We will also need to calculate the average sales for each brand.\n\n```python\n# Display the first few rows of the dataset\nprint(data.head())\n\n# Calculate the average sales for each brand\naverage_sales = data.mean()\n```\n\n\#\#\#\# Step 4: Create Box Plots\nWe will create box plots for each brand using seaborn. We will also overlay the individual sales data points on the box plots.\n\n```python\n# Set the style of seaborn\nsns.set(style=\"whitegrid\")\n\n# Create a figure and axis\nplt.figure(figsize=(12, 6))\n\n# Melt the DataFrame to long format for seaborn\ndata_melted = data.melt(id_vars='Quarter', var_name='Brand', value_name='Sales')\n\n# Create the box plot\nbox_plot = sns.boxplot(x='Brand', y='Sales', hue='Quarter', data=data_melted, palette='Set2', dodge=True)\n\n# Overlay individual data points\nsns.stripplot(x='Brand', y='Sales', hue='Quarter', data=data_melted, palette='Set2', dodge=True, marker='o', alpha=0.5)\n```\n\n\#\#\#\# Step 5: Plot Average Sales Line\nNow we will plot the average sales for each brand as a line connecting the average values.\n\n```python\n# Create a line plot for average sales\nfor i, brand in enumerate(average_sales.index):\n    plt.plot([i], [average_sales[brand]], marker='o', color='black', markersize=8, label=f'Avg {brand}' if i == 0 else \"\")\n```\n\n\#\#\#\# Step 6: Customize the Plot\nAdd titles, labels, and a legend to make the plot more informative.\n\n```python\n# Add titles and labels\nplt.title('Sales Distribution of Mobile Phone Brands by Quarter')\nplt.xlabel('Brand')\nplt.ylabel('Sales')\nplt.xticks(rotation=45)\n\n# Add a legend\nplt.legend(title='Quarter', bbox_to_anchor=(1.05, 1), loc='upper left')\n\n# Show the plot\nplt.tight_layout()\nplt.show()\n```\n\n\#\#\#\# Step 7: Save the Plot (Optional)\nIf you want to save the plot as an image file, you can do so by adding the following line before `plt.show()`:\n\n```python\nplt.savefig('sales_distribution_chart.png', bbox_inches='tight')\n```\n\n\#\#\# Complete Code\nHere is the complete code for your reference:\n\n```python\nimport pandas as pd\nimport matplotlib.pyplot as plt\nimport seaborn as sns\n\n# Load the dataset\ndata = pd.read_csv('data.csv')\n\n# Display the first few rows of the dataset\nprint(data.head())\n\n# Set the style of seaborn\nsns.set(style=\"whitegrid\")\n\n# Create a figure and axis\nplt.figure(figsize=(12, 6))\n\n# Melt the DataFrame to long format for seaborn\ndata_melted = data.melt(id_vars='Quarter', var_name='Brand', value_name='Sales')\n\n# Create the box plot\nbox_plot = sns.boxplot(x='Brand', y='Sales', hue='Quarter', data=data_melted, palette='Set2', dodge=True)\n\n# Overlay individual data points\nsns.stripplot(x='Brand', y='Sales', hue='Quarter', data=data_melted, palette='Set2', dodge=True, marker='o', alpha=0.5)\n\n# Calculate the average sales for each brand\naverage_sales = data.mean()\n\n# Create a line plot for average sales\nfor i, brand in enumerate(average_sales.index):\n    plt.plot([i], [average_sales[brand]], marker='o', color='black', markersize=8, label=f'Avg {brand}' if i == 0 else \"\")\n\n# Add titles and labels\nplt.title('Sales Distribution of Mobile Phone Brands by Quarter')\nplt.xlabel('Brand')\nplt.ylabel('Sales')\nplt.xticks(rotation=45)\n\n# Add a legend\nplt.legend(title='Quarter', bbox_to_anchor=(1.05, 1), loc='upper left')\n\n# Show the plot\nplt.tight_layout()\nplt.show()\n\n# Optional: Save the plot as an image file\nplt.savefig('sales_distribution_chart.png', bbox_inches='tight')\n```\n\n\#\#\# Conclusion\nThis code will generate a combination chart with box plots for each mobile phone brand's sales data, individual sales points, and a line connecting the average sales for each brand. Adjust the color palette and other parameters as needed to fit your specific requirements."
% \end{lstlisting}




