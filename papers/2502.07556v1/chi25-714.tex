%%
%% This is file `sample-manuscript.tex',
%% generated with the docstrip utility.
%%
%% The original source files were:
%%
%% samples.dtx  (with options: `manuscript')
%% 
%% IMPORTANT NOTICE:
%% 
%% For the copyright see the source file.
%% 
%% Any modified versions of this file must be renamed
%% with new filenames distinct from sample-manuscript.tex.
%% 
%% For distribution of the original source see the terms
%% for copying and modification in the file samples.dtx.
%% 
%% This generated file may be distributed as long as the
%% original source files, as listed above, are part of the
%% same distribution. (The sources need not necessarily be
%% in the same archive or directory.)
%%
%%
%% Commands for TeXCount
%TC:macro \cite [option:text,text]
%TC:macro \citep [option:text,text]
%TC:macro \citet [option:text,text]
%TC:envir table 0 1
%TC:envir table* 0 1
%TC:envir tabular [ignore] word
%TC:envir displaymath 0 word
%TC:envir math 0 word
%TC:envir comment 0 0
%%
%%
%% The first command in your LaTeX source must be the \documentclass
%% command.
%%
%% For submission and review of your manuscript please change the
%% command to \documentclass[manuscript, screen, review]{acmart}.
%%
%% When submitting camera ready or to TAPS, please change the command
%% to \documentclass[sigconf]{acmart} or whichever template is required
%% for your publication.
%%
%%
\documentclass[sigconf]{acmart}

%%
%% \BibTeX command to typeset BibTeX logo in the docs
\AtBeginDocument{%
  \providecommand\BibTeX{{%
    Bib\TeX}}}

%% Rights management information.  This information is sent to you
%% when you complete the rights form.  These commands have SAMPLE
%% values in them; it is your responsibility as an author to replace
%% the commands and values with those provided to you when you
%% complete the rights form.

\copyrightyear{2025}
\acmYear{2025}
\setcopyright{acmlicensed}
\acmConference[CHI '25]{CHI Conference on Human Factors in Computing Systems}{April 26-May 1, 2025}{Yokohama, Japan}
\acmBooktitle{CHI Conference on Human Factors in Computing Systems (CHI '25), April 26-May 1, 2025, Yokohama, Japan}
\acmDOI{10.1145/3706598.3713801}
\acmISBN{979-8-4007-1394-1/25/04}

%% These commands are for a PROCEEDINGS abstract or paper.
% \acmConference[Conference acronym 'XX]{Make sure to enter the correct
%   conference title from your rights confirmation emai}{June 03--05,
%   2018}{Woodstock, NY}
%%
%%  Uncomment \acmBooktitle if the title of the proceedings is different
%%  from ``Proceedings of ...''!
%%
%%\acmBooktitle{Woodstock '18: ACM Symposium on Neural Gaze Detection,
%%  June 03--05, 2018, Woodstock, NY}
% \acmISBN{978-1-4503-XXXX-X/18/06}
%%
%%  Uncomment \acmBooktitle if the title of the proceedings is different
%%  from ``Proceedings of ...''!
%%
%%\acmBooktitle{Woodstock '18: ACM Symposium on Neural Gaze Detection,
%%  June 03--05, 2018, Woodstock, NY}
% \acmPrice{15.00}
% \acmISBN{978-1-4503-XXXX-X/18/06}
% 看上面这里!!!!!



%%
%% Submission ID.
%% Use this when submitting an article to a sponsored event. You'll
%% receive a unique submission ID from the organizers
%% of the event, and this ID should be used as the parameter to this command.
\acmSubmissionID{7990}

%%
%% For managing citations, it is recommended to use bibliography
%% files in BibTeX format.
%%
%% You can then either use BibTeX with the ACM-Reference-Format style,
%% or BibLaTeX with the acmnumeric or acmauthoryear sytles, that include
%% support for advanced citation of software artefact from the
%% biblatex-software package, also separately available on CTAN.
%%
%% Look at the sample-*-biblatex.tex files for templates showcasing
%% the biblatex styles.
%%




%%
%% The majority of ACM publications use numbered citations and
%% references.  The command \citestyle{authoryear} switches to the
%% "author year" style.
%%
%% If you are preparing content for an event
%% sponsored by ACM SIGGRAPH, you must use the "author year" style of
%% citations and references.
%% Uncommenting
%% the next command will enable that style.
%%\citestyle{acmauthoryear}


\usepackage{enumitem}
% \usepackage{amsmath}
\usepackage{multirow}
\usepackage{xspace}
% \usepackage[normalem]{ulem}
\usepackage{url}
% %For the table of prompt
% \usepackage{longtable, array, xcolor}
% \usepackage{xspace}
% \usepackage{makecell}
% \usepackage{graphicx}
% \usepackage{tabularx}
% \newcommand{\tool}{Tool Name}
% \newcolumntype{Y}{>{\centering\arraybackslash}X}
% \usepackage{textcomp}

\newcommand{\tool}{\emph{SketchFlex}\xspace}

% For glossaries
\usepackage[nonumberlist,nopostdot]{glossaries}
\setglossarystyle{altlist}
\makenoidxglossaries


\newglossarystyle{twocolborderbold}{
  % Increase the line spacing
  \renewcommand{\arraystretch}{1.1}
  % Start the glossary environment set up for a two-column, long table without side vertical lines
  \renewenvironment{theglossary}%
    {
    \begin{longtable}{p{0.3\textwidth}p{0.65\textwidth}}
    \noalign{\hrule height 1.5pt} % This will add a horizontal line at the top of the table
    }
    {
    \end{longtable}
    }%
  % Header row with horizontal lines
  \renewcommand*{\glossaryheader}{%
    \textbf{\small Term} & \textbf{\small Definition} \\ \hline
    \endhead
  }%
  % Entry format: term on the left in bold and small font, definition on the right, with a regular horizontal line after
  \renewcommand*{\glossentry}[2]{%
    \textbf{\footnotesize \glstarget{##1}{\glossentryname{##1}}} & \footnotesize\glossentrydesc{##1} \\ \hline % Bold line for each entry
  }%
  % No sub-entries in this style
  \renewcommand*{\subglossentry}[3]{%
    % Ignore sub-entries
  }%
  % No blank line between groups
  \renewcommand*{\glsgroupskip}{}%

  \renewcommand*{\endfoot}{\noalign{\hrule height 1.5pt}}
}


%%
%% end of the preamble, start of the body of the document source.
\begin{document}


\newcommand{\eg}{\emph{e.g.}}
\newcommand{\ie}{\emph{i.e.}}

\newcommand{\strike}[1]{\textcolor{red}{\sout{#1}}}
\newcommand{\strikeg}[1]{\textcolor{blue}{\sout{#1}}}
\newcommand{\add}[1]{\textcolor{red}{#1}}
\newcommand{\replace}[2]{\strikeg{#1 }\add{#2}}
\newcommand{\new}[1]{\textcolor{blue}{#1}}
% \newcommand{\new}[1]{\textcolor{black}{#1}}
\newcommand{\cmt}[2]{\textcolor{blue}{#1}\add{#2}}
\newcommand{\yl}[1]{\textcolor{orange}{Ye: #1}}

%For the table of prompt
\definecolor{newgreen}{rgb}{0.0, 0.5, 0.0}
\definecolor{newblue}{rgb}{0.0, 0.0, 0.0}
\definecolor{newred}{rgb}{1.0, 0.0, 0.0}

\definecolor{blue}{rgb}{0.0, 0.0, 1.0}
\newcommand{\rev}[1]{\textcolor{blue}{#1}}

\newenvironment{tightcenter}{%
  \setlength\topsep{0pt}
  \setlength\parskip{0pt}
  \begin{center}
}{%
  \end{center}
}


\title[SketchFlex]{SketchFlex: Facilitating Spatial-Semantic Coherence in Text-to-Image Generation with Region-Based Sketches}

%%
%% The "author" command and its associated commands are used to define
%% the authors and their affiliations.
%% Of note is the shared affiliation of the first two authors, and the
%% "authornote" and "authornotemark" commands
%% used to denote shared contribution to the research.

\author{Haichuan Lin}
\affiliation{%
  \department{Thrust of Computational Media and Arts}
  \institution{The Hong Kong University of Science and Technology (Guangzhou)}
  \city{Guangzhou}
  \country{China}
}
\email{hlin386@connect.hkust-gz.edu.cn}

\author{Yilin Ye}
\authornote{Yilin Ye is the corresponding author.}
\affiliation{%
  \department{Thrust of Computational Media and Arts}
  \institution{The Hong Kong University of Science and Technology (Guangzhou)}
  \city{Guangzhou}
  \state{Guangdong}
  \country{China}
}
\affiliation{%
  \department{Academy of Interdisciplinary Studies}
  \institution{The Hong Kong University of Science and Technology}
  \city{Hong Kong SAR}
  \country{China}
}
\email{yyebd@connect.ust.hk}

\author{Jiazhi Xia}
\affiliation{%
  \department{School of Computer Science and Engineering}
  \institution{Central South University}
  \city{Changsha}
  \country{China}
}
\email{xiajiazhi@csu.edu.cn}

\author{Wei Zeng}
\affiliation{%
  \institution{The Hong Kong University of Science and Technology (Guangzhou)}
  \city{Guangzhou}
  \state{Guangdong}
  \country{China}
}
\affiliation{%
  \institution{The Hong Kong University of Science and Technology}
  \city{Hong Kong SAR}
  \country{China}
}
\email{weizeng@hkust-gz.edu.cn}

%%
%% By default, the full list of authors will be used in the page
%% headers. Often, this list is too long, and will overlap
%% other information printed in the page headers. This command allows
%% the author to define a more concise list
%% of authors' names for this purpose.
% \renewcommand{\shortauthors}{}
\renewcommand{\shortauthors}{Lin et al.}

\begin{abstract}

% Recent works to jointly reconstruct 3D human and object from a single RGB image, are mostly model-based, that fail to capture the fine details of the clothed human body and object surface. In this paper, we introduce ReCHOR, a novel, model-free, first-method to produce realistic clothed human-object reconstructions from a monocular view. This is extremely challenging due to human-object occlusions, diverse interactions and depth ambiguity, as it needs to infer both 3D spatial awareness and high resolution details. Our core idea is based on estimating neural implicit representations for human and object respectively by an attention-based neural implicit model that attends to pixel-aligned features from both the global human-object image for spatial awareness and  the local separate view of human and object images for high quality details. Additionally, the network is conditioned on semantic features from an initial estimated human-object pose prior and a generative diffusion model that inpaints occluded regions, thus enabling the retrieval of details from them.
% We also propose a synthetic dataset with rendered scenes of diverse, inter-occluded 3D human and object scans, to train our network. We evaluate our method on the synthetic and real world BEHAVE dataset. Our experiments show that our method outperforms the SOTA in achieving realistic clothed human-object reconstructions.
Recent approaches to jointly reconstruct 3D humans and objects from a single RGB image represent 3D shapes with template-based or coarse models, which fail to capture details of loose clothing on human bodies. In this paper, we introduce a novel implicit approach for jointly reconstructing realistic 3D clothed humans and objects from a monocular view. For the first time, we model both the human and the object with an implicit representation, allowing to capture more realistic details such as clothing. This task is extremely challenging due to human-object occlusions and the lack of 3D information in 2D images, often leading to poor detail reconstruction and depth ambiguity. To address these problems, we propose a novel attention-based neural implicit model that leverages image pixel alignment from both the input human-object image for a global understanding of the human-object scene and from local separate views of the human and object images to improve realism with, for example, clothing details. Additionally, the network is conditioned on semantic features derived from an estimated human-object pose prior, which provides 3D spatial information about the shared space of humans and objects. To handle human occlusion caused by objects, we use a generative diffusion model that inpaints the occluded regions, recovering otherwise lost details. For training and evaluation, we introduce a synthetic dataset featuring rendered scenes of inter-occluded 3D human scans and diverse objects. Extensive evaluation on both synthetic and real-world datasets demonstrates the superior quality of the proposed human-object reconstructions over competitive methods.
\end{abstract}

% \received{20 February 2023}
% \received[revised]{12 March 2009}
% \received[accepted]{5 June 2009}



%%
%% This command processes the author and affiliation and title
%% information and builds the first part of the formatted document.

\maketitle
\section{Introduction}\label{sec:intro}

In computational finance, Monte Carlo simulations are used extensively to estimate the expected value of financial payoffs based on the solution of stochastic differential equations (SDEs) which model the evolution of stock prices, interest rates, exchange rates and other quantities \cite{glasserman04}.  Monte Carlo methods are very general and flexible, but for high accuracy it requires generating a large number of costly SDE path approximations, which has motivated research into a number of variance reduction or, equivalently, cost reduction techniques. One such method is
Multilevel Monte Carlo (MLMC), which was proposed in \cite{GILES2008} and was adapted for various applications that are summarised in \cite{Giles_overview17} and successfully combined with other methods such as quasi-Monte Carlo methods. The main idea of MLMC is to approximate the payoff using different time stepping resolutions when numerically solving the underlying SDE and to generate an optimal number of samples on each level, such that the overall computational cost is minimised subject to the desired bound on the variance. %, such that the total computational cost is minimised. 
The computational savings come from the fact that most samples are computed on the coarser levels and hence are less expensive while only a few samples from the finest levels are required \cite{GILES2008}.


Among the directions in which the computational cost 
of MLMC methods could further be reduced, an important avenue is the use of lower precision calculations, especially for the first Monte Carlo levels where the targeted accuracy is relatively low. 
 An overview of the research on mixed precision for the standard Monte Carlo (MC) framework is provided in \cite{ChowMixedPrecisionStandardMC} but only a few references study the potential of low precision computation in the MLMC framework \cite{Rounding_error_oliver}. To the best of our knowledge, the only MLMC framework with customised precision in the literature is \cite{brugger2014mixed}, but they use a uniform precision for all operations on each Monte Carlo level instead of optimising 
 the precision of each intermediary variable to reduce as much as possible the cost of path generation.
 
An important motivation for an MLMC framework with variable precision would be performing the low precision computations on reconfigurable hardware devices such as Field Programmable Gate Arrays (FPGAs). FPGAs contain customizable logic blocks and connectors that make it easy to adapt the digital circuit architecture for a specific application, leading to a highly parallel and optimised implementation. Therefore they are successfully exploited in applications that require high speed and have high computational workload, such as signal processing \cite{woods2008fpga}, and real time applications like high frequency trading \cite{HFT1,HFT2}. That is why a number of previous works in hardware architecture design implemented the MLMC algorithm to price financial options using FPGAs as accelerators, which resulted in improved speed and power efficiency compared to full CPU architectures \cite{Schryver2013AMM}. The paper \cite{lindsey2016domain} also proposed 
a Domain Specific Language to automate the configuration of FPGAs for this specific application. However, only \cite{brugger2014mixed} proposed a heuristic to reduce the precision in calculations.

In addition, all aforementioned works considered that the random number generation (RNG) is performed in single or double precision. Yet in most cases an important portion of the workload in the overall MLMC simulation comes from the RNG and in \cite{brugger2014mixed} this limited the total computational savings.
To reduce the cost of MLMC simulations in particular those based on the Geometric Brownian Motion (GBM), \cite{approximateICDF_Oliver, NestedOliver} have proposed to use approximate random numbers that are generated by applying an approximation of the inverse CDF to uniform random numbers. In \cite{NestedOliver}, the authors proposed a way to integrate these lower precision random variables into a \textit{nested} MLMC framework and completed a numerical analysis to bound the resulting error at each MC level by a product of the time step and the error in the random number approximation. The same authors show in \cite{approximateICDF_Oliver} that using approximate random variables reduces the cost of path generation by a factor 7.


In this paper we propose a nested MLMC framework that combines the use of approximate random normal variables and lower precision calculations to reduce the computational cost of MLMC even further than \cite{brugger2014mixed,NestedOliver}. We illustrate the efficiency of our framework in Matlab, after making several assumptions on the cost of operations and size of the errors that we carefully justify. We focus on the case of GBM and use the approximate RNG methods presented in \cite{approximateICDF_Oliver} as well as a new slightly modified method that combines CDF inversion and the central limit theorem. To choose the precision of the variables in the low precision path generation, we introduce a novel method to optimise the bit-widths. This optimisation is performed before the main path generation loop is executed and is based on a linear model of the payoff error  
due to rounding when computing in low precision. The error model relies on algorithmic differentiation in a similar manner to \cite{unifying-bwoptim,bitwidth-AD,ADAPT}. The bit-width optimisation procedure can be performed off-line, so this stage can be excluded from the on-line time complexity of our framework. The user specified desired accuracy is then enforced by calculating on-line the number of samples that need to be generated.

In terms of hardware design, we suggest implementing the low precision path generation on FPGAs and the full-precision ones on a CPU or GPU. 
The FPGA offers enough flexibility to define a separate bit-width for every variable in the low precision path generation, and can be reconfigured periodically to update the bit-widths when the market parameters have changed considerably. 


The paper is organized as follows : \Cref{sec:MLMC} introduces MLMC and nested MLMC to make clear the estimator that is implemented in our framework. Then in \Cref{sec:RNG} we detail the methods that could be used to obtain approximate random normally distributed numbers very cheaply for the low precision path generation. In \Cref{sec:error_model} and \Cref{sec:costModel} we propose an error model and a cost model (resp.) that we then use to formulate the optimisation problem that is solved to obtain the optimal bit-widths of fixed point variables in \Cref{sec:optimisation}. Finally we summarise our results and future directions in \Cref{sec:conclusion}.



\section{Related Work}

\paragraph{LLMs for Agent tasks.}

Our research is related to deploying large language models (LLMs) as agents for decision-making tasks in interactive environments~\citep{liu2023agentbench,zhou2023webarena,shridhar2020alfred,toyama2021androidenv}. Earlier works, such as~\citep{yao2023webshopscalablerealworldweb}, fine-tuned models like BERT~\citep{devlin2019bertpretrainingdeepbidirectional} for decision-making in simplified environments, such as online shopping or mobile phone manipulation. With the advent of large language models~\citep{brown2020languagemodelsfewshotlearners,openai2024gpt4technicalreport}, it became feasible to perform decision-making tasks through zero-shot or few-shot in-context learning. To better assess the capabilities of LLMs as agents, several models have been developed~\citep{deng2024mind2web,xiong2024watch,hong2023cogagent,yan2023gpt}. Most approaches~\citep{zheng2024seeact,deng2024mind2web} provide the agent with observation and action history, and the language model predicts the next action via in-context learning. Additionally, some methods~\citep{zhang2023building,li2023camel,song2024trial} attempt to distill trajectories from state-of-the-art language models to train more effective policy models. In contrast, our paper introduces a novel framework that automatically learns a reward model from LLM agent navigation, using it to guide the agents in making more effective plans.

\textbf{LLM Planning.} Our paper is also related to planning with large language models. Early researchers~\citep{brown2020languagemodelsfewshotlearners} often prompted large language models to directly perform agent tasks. Later, \citet{yao2022react} proposed ReAct, which combined LLMs for action prediction with chain-of-thought prompting~\citep{wei2022chain}. Several other works~\citep{yao2023treethoughtsdeliberateproblem,hao2023reasoning,zhao2023large,qiao2024agentplanningworldknowledge} have focused on enhancing multi-step reasoning capabilities by integrating LLMs with tree search methods. Our model differs from these previous studies in several significant ways. First, rather than solely focusing on text generation tasks, our pipeline addresses multi-step action planning tasks in interactive environments, where we must consider not only historical input but also multimodal feedback from the environment. Additionally, our pipeline involves automatic learning of the reward model from the environment without relying on human-annotated data, whereas previous works rely on prompting-based frameworks that require large commercial LLMs like GPT-4~\citep{openai2024gpt4technicalreport} to learn action prediction. Furthermore, \Model supports a variety of planning algorithms beyond tree search.

\textbf{Learning from AI Feedback.} In contrast to prior work on LLM planning, our approach also draws on recent advances in learning from AI feedback~\citep{bai2022constitutional,lee2023rlaif,yuan2024self,sharma2024critical,pan2024autonomous,koh2024tree}. These studies initially prompt state-of-the-art large language models to generate text responses that adhere to predefined principles and then potentially fine-tune the LLMs with reinforcement learning. Like previous studies, we also prompt large language models to generate synthetic data. However, unlike them, we focus not on fine-tuning a better generative model but on developing a classification model that evaluates how well action trajectories fulfill the intended instructions. This approach is simpler, requires no reliance on state-of-the-art LLMs, and is more efficient. We also demonstrate that our learned reward model can integrate with various LLMs and planning algorithms, consistently improving their performance.

\textbf{Inference-Time Scaling.} ~\citet{snell2024scaling} validates the efficacy of inference-time scaling for language models. Based on inference-time scaling, various methods have been proposed, such as random sampling~\citep{wang2022self} and tree-search methods~\citep{hao2023reasoning, zhang2024accessing, guan2025rstar}. Concurrently, several works have also leveraged inference-time scaling to improve the performance of agentic tasks. ~\citet{koh2024tree} adopts a training-free approach, employing MCTS to enhance policy model performance during inference and prompting the LLM to return the reward. ~\citet{gu2024your} introduces a novel speculative reasoning approach to bypass irreversible actions by leveraging LLMs or VLMs. It also employs tree search to improve performance and prompts an LLM to output rewards. ~\citet{yu2024exact} proposes Reflective-MCTS to perform tree search and fine-tune the GPT model, leading to improvements in ~\citet{koh2024visualwebarena}. ~\citet{putta2024agent} also utilizes MCTS to enhance performance on web-based tasks such as ~\citet{yao2023webshopscalablerealworldweb} and real-world booking environments. ~\cite{lin2025qlass} utilizes the stepwise reward to give effective intermediate guidance across different agentic tasks. Our work differs from previous efforts in two key aspects: (1) Broader Application Domain. Unlike prior studies that primarily focus on tasks from a single domain, our method demonstrates strong generalizability across web agents, mathematical reasoning, and scientific discovery domains, further proving its effectiveness. (2) Flexible and Effective Reward Modeling. Instead of simply prompting an LLM as a reward model, we finetune a small scale VLM~\citep{lin2023vila} to evaluate input trajectories. %Our reward scores range continuously between 0 and 1, in contrast to existing methods that rely on discrete scoring (e.g., 0 and 1, or 0, 0.5, and 1) through direct LLM prompting.

% Concurrently, several works have also leveraged inference-time scaling to improve the performance of agentic tasks. ~\citet{pan2024autonomous} demonstrates that LLMs and VLMs, such as the GPT series, can function as evaluators or reward models to provide guidance for fine-tuning or reflection, thereby enhancing digital agents. This lays the groundwork for subsequent studies that directly prompt LLMs as reward models. ~\citet{koh2024tree} adopts a training-free approach, employing MCTS to enhance policy model performance during inference. However, it is limited to web environments~\citep{koh2024visualwebarena}. Moreover, its value function relies on prompting an LLM, which is less effective than our proposed method. We validate our approach through ablation studies, demonstrating that our fine-tuned reward model is more effective. ~\citet{gu2024your} introduces a novel speculative reasoning approach to bypass irreversible actions, such as purchasing a product, by leveraging LLMs or VLMs. It also employs tree search to improve performance, but it remains restricted to the web domain~\citep{koh2024visualwebarena, deng2024mind2web}. Additionally, it lacks reward modeling and instead prompts an LLM to output rewards. ~\citet{yu2024exact} proposes Reflective-MCTS to perform tree search and fine-tune the GPT model, leading to improvements in ~\citep{koh2024visualwebarena}. However, this work focuses solely on a single web agent task, and its reward modeling is derived from multi-agent debate, differing from our more effective and efficient reward modeling approach. ~\citet{putta2024agent} also utilizes MCTS to enhance performance, but it is limited to web-based tasks such as ~\citep{yao2023webshopscalablerealworldweb} and real-world booking environments.
\section{PRELIMINARY STUDY}\label{sec:formative study}
To understand the practice of T2I generative models, we conducted in-depth interviews with eight users about their experience of using GenAI. 
We aim to identify the advantages and limitations of existing T2I generative models and identify design goals for potential improvement beyond existing methods. 

\subsection{Study Design}
\textbf{Participants}.
We recruited eight users of GenAI aged between 25 and 33 (4 females and 4 males), including four PhD students studying Digital Media or Computer Science (P1, P2, P7, P8), one UI/UX designer (P3) and three game concept designers (P4-P6).
All users have experience using popular generative models and tools like Stable Diffusion and Midjourney, with varying degree of expertise in GenAI, including four novice users with less than 6 months' usage (P1, P2, P4, P7), and four expert users with at least one-year experience and familiar with different models (P3, P5, P6, P8). 
In terms of painting expertise, three novices (P1, P7, P8) have no experience in painting, one beginner (P2) has limited knowledge about painting, and four experts (P3-P6) are proficient in painting.

\noindent
\textbf{Procedure}.
Each interview lasted 60–90 minutes. 
Initially, participants were asked to provide a self-introduction, focusing on their experience in GenAI usage, with questions like "\emph{have you used any text-to-image models like Midjourney or Stable Diffusion?}". 
Next, we explored various scenarios in which participants used or would like to use generative models, using both live demonstrations with Stable Diffusion or Midjourney 
% and explained prompt-based and spatial. 
together with explanations of prompt-based and spatial control.
Finally, we engaged in an open discussion about the advantages and limitations of the use cases presented, as well as the participants' expectations for future developments.

\subsection{Findings}

Two authors independently used open coding to analyze and code audio-recorded user feedback, generating initial codes to capture key insights from participants' experiences and expectations. 
The first author then worked with another co-author to refine and validate the themes through iterative discussions.
Based on this analysis, we distilled three primary user requirements, which are summarized as follows:

\subsubsection{Rough sketch based spatial control is preferable by novice users.}
Users with varying levels of experience exhibit different preferences in using spatial conditioning models. 
Experienced users in both GenAI and painting (P3, P6) have integrated these models into their workflow, significantly increasing their productivity. 
P6 shared his experience: "\emph{we create a line draft or use 3D modeling to generate a DepthMap, then use ControlNet for drawing}."
% He also mentioned that such workflow is widely used not only in his department, but also other companies.
Conversely, novice users in either GenAI or painting, despite being aware of spatial control features, often find these models difficult to use or insufficiently flexible to meet their expectations.
Some novice users in GenAI (e.g., P1, P4, P7) explained that these methods involve a steep learning curve, with program settings being overly complex. 
For instance, P7 noted, "\emph{I have tried using Canny, for example, but it doesn't understand my sketch very well. You need to adjust some settings, such as the guidance coefficient.}" 
Similarly, novice users in painting described a reliance on pre-existing spatial conditions found online, which limited their ability to fully realize their creative intentions. 
"\emph{I usually search images on the internet that are similar to what I want, like a person with a specific posture. Then I use the searched image to apply ControlNet. 
But in most cases, the searched image does not fully meet my expectations. 
Sometimes I can't even find an existing image that is close to my idea}," explained P8.

\begin{figure*}[t]
    \centering
    \includegraphics[width=0.99\textwidth]{src/img/overall_framework.pdf}
    \vspace{-2mm}
    \caption{\tool mainly consists of three components: (1) sketch-aware prompt recommendation that support users in crafting effective prompts for the rough sketch; (2) object shape refinement through single object decomposition and generation; and (3) spatial adjustment and anchoring of object shapes.}
    \Description{\tool mainly consists of three components: (1) sketch-aware prompt recommendation that support users in crafting effective prompts for the rough sketch; (2) object shape refinement through single object decomposition and generation; and (3) spatial adjustment and anchoring of object shapes.}
    \label{fig:workflow}
    \vspace{-4mm}
\end{figure*}

Novice painters expressed strong interest in spatial control through rough sketches.
Participants with a professional painting background preferred to use line sketches and color blocks, as these techniques closely aligned with their established drawing workflows and design practices.
In contrast, novice painters prefer scribbles or regions, as these methods offer greater flexibility and require less precision when representing image elements.
P7 envisioned an interaction approach suited to their needs: "\emph{If I just draw a stick man, it knows it is a person. Then if I draw a few lines, it knows that it is an ocean. And a simple circle would represent the sun}."

\subsubsection{Combining prompt tuning and spatial control is challenging.}
Some participants familiar with spatial conditioning models reported that prompt tuning becomes more difficult given extra spatial conditions. 
This challenge arises from a potential misalignment between spatial conditions and prompts.
Such misalignment mostly occur when the intended image contains multiple objects. 
P5 found that the prompt must match the spatial condition between objects, and had to curate proper words for such prompt each time. 
P6 shared a similar point, "\emph{The count is important—if there are three people in your conditioned image but your prompt does not specify three people, the model might generate the wrong number. 
Relationships are also crucial; If you don’t specify connections, the generated result can seem disjointed, which is especially prominent in multi-diffusion}." 
The MultiDiffusion model~\cite{bar2023multidiffusion} is one of control methods that P6 frequently uses for region control.
Regarding the prompting difficulty, both P5 and P6 expressed a desire for a tool that could recommend prompts based on spatial conditions to reduce their time and cognitive load.
Despite these challenges were primarily raised by experienced participants, as novice users rarely engage with these techniques, it is clear that prompt tuning would be even more difficult for novice users.


\subsubsection{Iterative generation and refinement is difficult in end-to-end generation.}
All participants mentioned that refining end-to-end generated results is challenging. 
Specifically, they often want to change a part of the generated image while keeping other elements unchanged, as it is hard to generate image that all parts fulfill their requirement. 
However, even minor adjustments to the prompt or spatial condition can lead to significant changes in the entire generated image. 
P3 noted, "\emph{In a project, I used a reference image and made a slight change to the prompt, like adding `wearing glasses' to the description. The result looked similar, but it was just...different}."
P8 also shared, "\emph{I use AI-generated images to illustrate the scene layout for my project, including environment settings, object positioning, and content alignment. In most cases, it’s sufficient if the generated image fulfills two of these three requirements. However, in my experience, achieving all three requirements without iterating on specific areas is impossible}."
Some experienced GenAI users employ techniques like in-painting or use Photoshop for manual regional editing, but such methods require additional efforts. 
P3 explained, "\emph{I use in-painting and Photoshop to edit the parts I want to change. First, I break down the elements in the image using Photoshop, then in-paint or manually re-draw some of them. It’s time-consuming, and the in-painting doesn’t have the same coherent effect as generating the image from scratch}."
These feedback suggest that users require an iterative generation process that can progressively adjust specific parts to reach a satisfactory outcome.


%%%%%%%%%%%%%%%%%%%%%%%%%%%%%%%%%%%%%%%%%%%%%
%%%%%%%%%%%%%%%%%%%%%%%%%%%%%%%%%%%%%%%%%%%%%
\subsection{Design Goals}
The formative study illustrates the diverse methods available for spatially controlling T2I models and the needs for users in using these methods. 
However, the study also highlights that novice users often find it challenging to fine-tune prompts and prepare spatial conditions to match their intended outcomes with AI-generated images.
Our goal is to empower users with limited expertise in art and computer science to more freely create with GenAI. %using AI's generative and control capabilities. 
To achieve this, the design of a new interactive tool supporting this application should incorporate user-friendly interactions for more flexible, less demanding input and iteration while delivering high-quality results aligned with user intentions.

Based on our findings, we propose the following design goals for \tool to enhance flexible spatial control in image generation, specifically tailored for novice users.
\begin{itemize}
    \item \textbf{G1: Providing Flexible Spatial Control}.
    \tool aims to empower novice users to easily and intuitively achieve spatial control without requiring expertise in computer science or painting. 
    This control should allow users to input rough spatial conditions and provide tools for refining and enhancing the final generated result. Based on the study, we will implement region control as the rough sketch input preferred by novice users.
    \item \textbf{G2: Automating Prompt Tuning for Spatial Conditioned T2I Generation}.
    \tool should include an automated prompt tuning feature that considers both spatial conditions and the initial text prompt. 
     This can reduce the cognitive load of users while ensuring the generation of high-quality, cohesive images that accurately reflect the user's intent.
    \item \textbf{G3: Enhancing Flexibility in Image Adjustment by Decomposed Generation}.
    \tool should implement a decomposed generation approach, allowing users to make adjustments at the level of individual objects or elements within the image to facilitate iterative generation. 
    This will help avoid unintended changes to the overall image when making local adjustments in different generation rounds, thus iteratively improving user control and satisfaction.
\end{itemize}


\section{\tool System}\label{sec:system_description}

\begin{figure*}[t]
    \centering
    \includegraphics[width=0.85\textwidth]{src/img/inference.pdf}
    \vspace{-2mm}
    \caption{Our sketch-aware prompt recommendation first builds a semantic space through data-driven analysis of key semantic elements covering single object and cross object properties. Then the semantic space is integrated with retrieval of attributes and relationships reference from semantic dataset. Finally, these semantic guidance is combined with users' initial sketch to form a sketch-aware multi-modal prompt to the MLLM to support spatial-aware inference.}
    \Description{Our sketch-aware prompt recommendation first builds a semantic space through data-driven analysis of key semantic elements covering single object and cross-region properties. Then the semantic space is integrated with retrieval of attributes and relationships reference from semantic dataset. Finally, these semantic guidance is combined with users' initial sketch to form a sketch-aware multi-modal prompt to the MLLM to support spatial-aware inference.}
    \label{fig:inference}
    % \vspace{-3mm}
\end{figure*}

Based on the design goals, we design \tool, an interactive system that allows users to generate images controlled by inputs of rough sketches and simple prompts, while generating semantically cohesive and region-controlled images.
The overall framework of \tool is shown in Figure~\ref{fig:workflow}. 
The framework consists of three stages: 1) the user can draw a sketch, assign a corresponding regional prompt, and use automatic prompt recommendation to refine their initial prompt \textbf{(G1, G2)}; 2) the sketch with object decomposition and single-object generation \textbf{(G3)}; and 3) spatial adjustment and anchoring of object shapes   \textbf{(G1, G3)}.


\subsection{Sketch-Aware Prompt Recommendation}
\label{ssec:prompt_rec}
Crafting effective prompts for rough sketch-based image generation is a challenging task, as users must not only create prompts for each individual region but also ensure coherence across the entire image. 
We introduce a prompt recommendation method that automatically enhances the user’s initial input, to produce a spatially cohesive prompt that aligns with the overall composition.
Figure~\ref{fig:inference} shows the overall workflow of this process.


\subsubsection{Semantic Space Reasoning}
\begin{table}[thb] \small
    \centering
    \caption{Common examples in the semantic space across various T2I description datasets.}
    \vspace{-2mm}
    \begin{tabular}{|l|l|c|}
    \hline 
    \textbf{Space Item} & \textbf{Property} & \textbf{Instance} \\
    \hline 
    \multirow{3}{*}{\textbf{Single Object}} 
    & Type &   Man, Car, Dog, Tree, Window \\
    && Table, Ocean, Park, Wall\\ \cline{2-3}
    & Attribute & Wooden, Tall, Red, Large\\
    &&Fluffy, Round, Slim, Silver   \\ \cline{2-3}
    & State &   Standing, Moving, Swaying, \\
    &&Broken, Sleeping, Lying\\\hline
    \multirow{2}{*}{\textbf{Cross Object}} 
    &  Direction  &  Facing to, Aligned in, \\
    &&Diagonally placed\\\cline{2-3}
    & Relationship & Next to, Under, Parked on, \\
    &&Sitting by, Supporting\\ \hline
    \multirow{3}{*}{\textbf{Overall}}  
    & Lightning  &   Natrual daylight, indoor lighting, \\
    && Soft light, Moon light\\\cline{2-3}
    & Camera  &   Close-up shot, Wide-angle shot, \\
    && Overhead shot, Extreme long shot \\\cline{2-3}
    & Style  &   Realistic, Minimalist, Cinematic, \\
    && Abstractm, Anime, Oil painting\\
    \hline
    \end{tabular}
    \vspace{-1mm}
    \label{tab:example_space}
\end{table}



Previous prompt-tuning methods have primarily focused on text-to-image models, which offer limited support for refining individual region prompts and often struggle to ensure cohesiveness across the entire image.
To address this, the first step is to identify the types of prompts needed to generate a coherent image under rough sketch-based control.
Specifically, we define an "\emph{semantic space}" that provides intuitive guidance that helps users to easily input and adjust their prompts in the appropriate regions.
The process of constructing this semantic space involves analyzing online sources~\cite{sd,wang2022diffusiondb,civitai,xie2023prompt} and prompt-guideline literature~\cite{oppenlaender2023prompting,oppenlaender2023taxonomy,liu2022design} to identify critical elements that contribute to high-quality image generation. Additionally, examining image description datasets in computer vision—including traditional task datasets~\cite{caesar2018coco,pham2021learning,krishna2017visual} and recent datasets tailored for image generation~\cite{onoe2024docci}—helps uncover relevant dimensions for describing images.

\begin{figure*}[t]
    \centering
    \includegraphics[width=0.995\textwidth]{src/img/refinement.pdf}
    \vspace{-2mm}
    \caption{Spatial-condition sketch refinement can help novice users refine their sketch by generating more realistic and accurate sketch for each object through single object decomposition and generation, and subsequently allowing users to interactively refine the sketch by object selection and spatial adjustment.}
    \Description{Spatial-condition sketch refinement can help novice users refine their sketch by generating more realistic and accurate sketch for each object through single object decomposition and generation, and subsequently allowing users to interactively refine the sketch by object selection and spatial adjustment.}
    \label{fig:decompose}
    \vspace{-3mm}
\end{figure*}

\begin{table*}[thb] 
    \centering
    \caption{Example Statistics of objects, attributes and relationships in Visual Genome~\cite{krishna2017visual} and VAW~\cite{pham2021learning}.}
    \vspace{-2mm}
    \begin{tabular}{lccccc}
    \hline 
    \textbf{Space Item} & \textbf{1st} & \textbf{2nd} & \textbf{3rd} & \textbf{20th} & \textbf{50th}  \\
    \hline 
    Object & window (52k)  & man (52k) & shirt (39k) & trees (17k) & sidewalk (8k)  \\
    Attribute & white (311k) & black (195k) &  blue (118k) & clear (15k) & colorful (5.8k)   \\
    Relationship  & on (645k) & has (245k) &  in (219k) & sitting on (13k) & laying on (3.5k)    \\
    \hline
    \end{tabular}
    \vspace{-1mm}
    \label{tab:statsitic}
\end{table*}

We summarize these prompt aspects and conduct experiments to identify the key components essential for high-quality output. 
% Finally, we propose a semantic space that integrates these necessary elements to ensure coherence when generating images from rough sketches.
Table~\ref{tab:example_space} presents the common dimensions and example instances of the identified semantic space within the T2I datasets.
The specific elements within the space are listed below.
\begin{itemize}
    \item \textbf{Single Object Prompt} contains local prompts for each single object. 
    It contains \textit{type} of the object; \textit{attribute} of object including main attributes such as color, texture, shape; and \textit{state} indicates how the object acts, including still, standing, running, etc. The background is a special object that only has type and attribute.
    \item \textbf{Cross Object Prompt} explicitly specifies how objects in different regions interact, which is crucial for the coherence of the generated image.
    It includes \textit{direction} of objects and \textit{relationship} that multiple objects interact with each other.
    \item \textbf{Overall Prompt} does not directly affect multi-object cohesiveness but allows users to optionally specify the overall visual effect, including \textit{lighting}, \textit{style} and \textit{camera}.
    % \item \textbf{Overall Prompt} is not indispensable for a semantically cohesive image generation, but has significant impact in visual effect, include \textit{lighting}, \textit{style} and \textit{camera}. 
    % Previous works consider quality modifiers (e.g., best quality, high resolution). In our experiment, the state-of-the-art model can already generate high quality images without these modifiers.
\end{itemize}




Figure~\ref{fig:inference} (right) illustrates a completed semantic space for a sketch featuring a girl, a cat, and a background (\textit{i.e.}, areas without objects).
Once the individual prompt components are defined, the separate prompts within a single region are concatenated into one unified prompt using commas (\textit{e.g.}, "type: girl, attribute: long hair" becomes "girl, long hair"). 
% To ensure that the concatenated prompt remains within the 77-token limit imposed by CLIP, we employ the bag-of-conditions approach~\cite{omost}.}



\subsubsection{Sketch-Augmented Prompting}
While the semantic space simplifies the prompt input and adjustment process, manually entering all prompts and identifying the appropriate ones can still be labor intensive and cognitively demanding, particularly when dealing with many objects. 
To alleviate this burden, we utilize a MLLM GPT-4o, to automatically complete the semantic space based on the user's sketch and initial prompt.

Specifically, the user’s initial prompt and rough sketch are input into the MLLM, with the semantic space acting as a contextual guide within the prompt template. The model generates prompts to populate the semantic space, incorporating both the initial input prompt and the sketch. It utilizes spatial reasoning, guided by the Chain-of-Thought~\cite{wei2022chain} strategy, to account for the \textit{shape}, \textit{location}, and \textit{interaction} of the objects within the user's sketch.
However, relying solely on the MLLM to fill the semantic space can be risky, as it may overfit to certain content or produce results with reduced coherence~\cite{cao2023beautifulprompt}. 
To address this, we further enhance the process by retrieving reference attributes and relationships between objects from crowd-sourced text-image datasets based on real-world images~\cite{pham2021learning,krishna2017visual}. 
The datasets are organized into a dictionary, where object names serve as keys and their corresponding attributes or relationships are stored as values. 
During retrieval, \tool randomly samples $k=10$ examples from the values based on the given object names as keys, providing the MLLM with reference data. 
The raw datasets are sourced and publicly available from ~\cite{pham2021learning,krishna2017visual}.
Example statistics are shown in Table~\ref{tab:statsitic}. 
The rich semantics in these datasets enhance both diversity and coherence in the completed semantic space.

% Table~\ref{tab} presents example statistics of objects, attributes, and relationships within these datasets. 
The overall prompt generation process, as shown in Figure~\ref{fig:inference}, incorporates user input, semantic space, and retrieved attributes and relationships to guide generation. 
We employ few-shot learning~\cite{wang2023large} to enhance the quality and robustness of the generated prompts.


\begin{figure*}[t]
    \centering
    \includegraphics[width=0.995\textwidth]{src/img/ablation_zoom.pdf}
    \vspace{-2mm}
    \caption{Ablation study shows that prompt recommendation avoids common issues like missing objects and unrealistic relationships while sketch refinement further enhances fine-grained control.}
    \Description{Ablation study shows that prompt recommendation avoids common issues like missing objects and unrealistic relationships while sketch refinement further enhances fine-grained control.}
    \label{fig:ablation}
    \vspace{-4mm}
\end{figure*}

\subsection{Spatial-Condition Sketch Refinement}
\label{ssec:sketch_refine}
To help users refine their rough sketches into fine-grained shapes that align with their intentions and allow for iterative refinement, we propose a decompose-and-recompose approach. 
As Figure~\ref{fig:decompose} shows, the sketch is first decomposed into individual objects.
The users can then generate, select, and adjust the desired single-object images. 
Finally, these selected objects, along with their spatial conditions, are combined to generate the final result.

\subsubsection{Single Object Decomposition}
Instead of directly using a single object sketch for generation, we first classify objects into two categories: \textbf{thing} as foreground object with specific shapes, such as humans, animals, or chairs, and \textbf{stuff} as background object without a defined shape, like oceans, grass, or sky, based on ~\cite{caesar2018coco}. 
% \new{"Things" are objects with specific shapes, such as humans, animals, or chairs, while "stuff" refers to objects without a defined shape, like oceans, grass, or the sky.}
To classify an object, we compute the word embedding of its type and find the nearest match in a category list containing "things" and "stuff" in the object list in the COCO-Stuff dataset~\cite{caesar2018coco}.
During single object decomposition stage, each thing object is extracted using FAST SAM (Segment Anything)~\cite{zhao2023fast}, to make sure the single object generation maximally preserves consistency to the original sketch.
Once the single object sketch is decomposed, the generation of each individual object is carried out. 
Since the goal is to generate fine-grained object shapes but not the final image, the process can be accelerated by using low-step inference (6 steps) with the Lightning Diffusion model~\cite{luo2023lcm} and a lower resolution (512x512). 
In our experiment, generating 12 images took approximately 4 seconds on a GTX 4090.


To ensure that the generated result aligns more closely with the user's sketch, we filter the generated images based on the Intersection over Union (IoU) and CLIP score~\cite{hessel2021clipscore}, which measure spatial correspondence and semantic alignment, respectively, between the user sketch and the generated single object. 
We compute a weighted sum of the IoU and CLIP scores, then sort the images and select the top four for the user to choose from.
Once the user selects an image containing the desired object shape, the target object is automatically extracted using FAST SAM~\cite{zhao2023fast}.
Each object with refined shape will automatically replace the original rough sketch.
If users have no desired image in one generation, they can perform multi-round generation until finding the desired one.
Figure~\ref{fig:decompose} shows an example in which the rough sketch of a girl and a dog is refined to specific shapes.


\begin{figure*}[t]
    \centering
    \includegraphics[width=0.8\textwidth]{src/img/interface.pdf}
    \vspace{-3mm}
    \caption{\tool interface consists of (a) Canvas view, (b) Prompt Recommend view, (c) Sketch Refine view and (d) Result view.}
    \Description{SketchFlex interface consists of (a) Canvas view, (b) Prompt Recommend view, (c) Sketch Refine view and (d) Result view.}
    \label{fig:interface}
    \vspace{-3mm}
\end{figure*}

\subsubsection{Single Object Adjustment}
Our system provides flexible control not only over the shapes of objects but also over their size and position. As shown in Figure~\ref{fig:decompose}
, users can easily adjust the size and spatial placement of each object. 
Once adjustments are made, the corresponding single object shape mask is moved accordingly. 
All individual object shapes are then combined into an "anchor" image.
% The term "anchor" represents the idea that the generated result, like a boat floating within the rough sketch regions, may vary with each generation, just as a boat drifts on water. 
% However, with the anchor in place, the boat remains fixed to a specific location—similar to how the selected object shapes remain fixed in the generated result. 
Shape anchoring refers to the process of fixing the object shapes in the generated result.
To apply this shape anchoring, we extract the edges of the selected object using Canny edge detection and feed them into ControlNet to ensure that the final output adheres to the user's shape preferences. 
Once users have made their desired adjustments, they can generate an image with the exact fine-grained shapes they prefer.


A related issue with the original rough sketch-based generation is that a single prompt is applied to each region separately. 
To generate images that capture relationships between objects, we create a joint mask, \textit{i.e.}, the union of object masks, for two related objects, allowing the relationship prompt to influence the interaction between them. 
Equation 1 illustrates the creation of a joint mask for the relationship between region $i$ and region $j$, 
where $\oplus$ denotes the concatenation operation.

\begin{equation}
M_{ij} = [M_i \oplus M_j].
\end{equation}

However, using a joint mask alone can sometimes result in multiple objects being generated within a single object area. 
To address this, we apply negative prompts to exclude objects outside the intended region, preventing unwanted elements from appearing in the wrong areas, as shown in Equations 2-3.
In Equation 2, the cross-attention map that correlates text and image patches is updated such that the text embedding $C_i$ is amplified by a scalar $\lambda_{m_i}$ within the masked region $M_i$.
In Equation 3, the relationship embedding condition is reinforced within $M_{ij}$, while the influence of $C_i / C_j$ is reduced in the complementary areas.

\begin{equation}
A_i \leftarrow \lambda_{m_i} \cdot C_i \odot M_i,
\end{equation}

\begin{equation}
A_{(i,j)} \leftarrow \lambda_{m_{ij}} \cdot C_{ij} \odot M_{ij} - \lambda_{m_{ij}} \cdot C_i \odot (M_{ij}-M_i) - \lambda_{m_{ij}} \cdot C_j \odot (M_{ij} - M_j).
\end{equation}



Figure~\ref{fig:ablation} shows the ablation results of our system’s functions.
Without prompt recommendation and sketch refinement, the rough sketch-based generation often produces undesirable outcomes, such as missing objects, unrealistic relationships, and incorrect perspectives.
By incorporating our prompt recommendation, which refines prompts to be spatially aligned with the sketch, the results become more stable and coherent, with the generated objects closely matching the original sketch.
Finally, with prompt recommendation and sketch refinement, users can achieve fine-grained control, enabling them to generate detailed results that align with their sketches and creative intentions. For example, in the first column, the posture of the man lying on the chair with both arms spread out is more accurately captured in our result compared to the other two. 
In the third column, the girl's head is positioned slightly to the right and below the car within the mask, which aligns with our result. In contrast, in the other two results, the girl's head overlaps with the car, resulting in an incorrect spatial relationship.










\section{Interface Design}
With the constructed data flow and recommended charts, we further design the interface of \system{} to support the recommendation and efficient tracing on the data flow.
We demonstrate the interface design of \system{}.

\begin{figure*}[!htb]
    \centering
    \includegraphics[width=\textwidth]{figures/interface.png}
    \caption{The Interface of \system{}. \system{} contains a chart view showing the recommended charts under each cell (A \& B), which can be opened or collapsed. Users can select a chart of interest for detail tracing across the flow. A flow view is shown on the right, demonstrating the relationships between data tables, with the chart selected and column details (C). A pinned view is situated on the top right of the interface, showing the charts that have been selected. The interface contains a control panel for switching the variables to be traced and downloading, uploading, and re-running the results.}
    \label{fig:interface}
\end{figure*}

To facilitate exploration, we have created an interactive interface that is tightly integrated with Jupyter Notebook. This interface consists of two key components, including chart views (\autoref{fig:interface}-A \& B), a flow view (\autoref{fig:interface}-C), a pinned list (\autoref{fig:interface}-D), and a control panel (\autoref{fig:interface}-E).

\subsection{Chart View}
The chart view (\autoref{fig:interface}-A \& B) is placed under each cell, showing the recommended charts for the tables in the cell. 
The chart view is collapsed by default to avoid information overwhelming (\autoref{fig:interface}-A), and users can click the button (\autoref{fig:interface}-A2) to open the view. 
An example of the opened view can be found in \autoref{fig:interface}-B, where the color of the button is changed (\autoref{fig:interface}-B1).
Note that each cell contains multiple tables, and each table will receive a list of recommendations.
The recommended charts of all tables in a cell are ranked together and listed in the chart view, as mentioned in ~\autoref{sec:ranking}.
Therefore, we provide a filtering panel available for users to accurately specify the data table and columns they are interested in (\autoref{fig:interface}-B2). 
By default, \system{} show the charts of the latest data variable. 
The user can further specify the factors for the recommendation for the chart, such as operation types and fact types (\autoref{fig:interface}-B3). 
Upon applying these filters, the list of charts will be updated, allowing users to explore and choose the charts that align with their preferences and requirements.
Each recommended chart (\autoref{fig:interface}-B4) is accompanied by a list of column details (\autoref{fig:interface}-B5).
Users can click the ``pin'' button (\autoref{fig:interface}-B6) to trace the table changes with the same visual encodings of the pinned chart within the flow view (\autoref{fig:interface}-C). 


\subsection{Flow View}
The flow view (\autoref{fig:interface}-C) provides an overview of the whole EDA flow, with the traced chart acting as the ``sight glasses'' (\textbf{R1}). The flow graph (\autoref{fig:interface}-C3) and traced charts (\autoref{fig:interface}-C1 \& C2) are displayed side by side. Each node represents a data table at a specific line of code, maintaining consistent visual encodings across the flow. For instance, the chart and column details in \autoref{fig:interface}-C2 correspond to the table \code{df\_C3\_L1}, which is the value of the data frame \code{df} at line 1 of cell 3.

Nodes are color-coded to represent their states. Blue nodes (\autoref{fig:interface}-C4) indicate a chart different from the preceding one, and \system{} displays both by default. Light blue nodes (\autoref{fig:interface}-C7) indicate similar charts, which are closed by default but can be opened by clicking. Red nodes (\autoref{fig:interface}-C8) represent untraceable nodes, such as \code{df\_groupby\_C4\_L2}, where a \code{groupby} operation removed the "cylinder" column, making chart rendering impossible.

Link colors encode relationships between nodes and the traced one. Black links indicate a relationship, with the direction showing the order of operations. For example, the link in \autoref{fig:interface}-C5 shows that \code{df\_C3\_L1} was transformed into \code{df\_copy\_C5\_L1} after several operations. Grey links indicate no direct relationship, as in \autoref{fig:interface}-C6, where \code{df\_groupby\_C4\_L2} has no direct transformation link to \code{df\_copy\_C5\_L1}.

\begin{figure*}[!htb]
    \centering
    \includegraphics[width=0.5\textwidth]{figures/re-run.png}
    \caption{The stepped layout after re-running some of the cells. The cells before (A) and after (B) the re-running correspond to the flow on the left column (C) and the right one (D). When \system{} detects a re-running, a new column will be appended on the right, showing the re-running cell and the succeeding ones. A link will connect the nodes of different versions (E) for better understanding.}
    \label{fig:re-run}
\end{figure*}

The flow view utilizes a stepped layout to efficiently trace past tables after re-running cells (\textbf{R4}). As shown in \autoref{fig:re-run}, the analyst revises and re-runs cell four (\autoref{fig:re-run}-A), changing its index to six (\autoref{fig:re-run}-B). Most tools overwrite the original result, losing the previous state. In contrast, \system{} uses a stepped layout to preserve the original state, creating a new column on the right (\autoref{fig:re-run}-D) to display the new results alongside the old ones (\autoref{fig:re-run}-C). The links between nodes indicate their relationships.
This layout offers two advantages (\autoref{fig:re-run}-E). First, side-by-side visualization allows for easy comparison between different code versions. Second, the stepped layout avoids cyclic links within the same column, making the flow clearer than with the overwriting approach.

\subsection{Pinned View and Control Panel}
The pinned charts are listed on the right (\autoref{fig:interface}-D), where users can switch the tables and chart configurations for tracing by clicking on the card in the list.
In this case, users click the first chart for tracing (\autoref{fig:interface}-D1).

The control panel is situated on the top (\autoref{fig:interface}-E).
Users can change the traced node with a toggle list.
Users can also download the exploration log, upload a previous log, and regenerate the charts and flows using the control panel.

\section{Evaluation}

% \saidur{Working on it}




\begin{table*}[!t]
% \small
\centering
\caption{Summary of Results for EMBER Domain-IL Experiments.}
\vspace{-0.2cm}
\label{tab:ember_DIL}

\begin{tabular}{p{1.1cm}|l|c|c|c|c|c|c|c} 

% \toprule 

\multirow{2}{*}{\textbf{Group}} & \multirow{2}{*}{\textbf{Method}} & \multicolumn{7}{c}{\textbf{Budget}} \\ \cline{3-9}

&  & 1K & 10K & 50K & 100K & 200K & 300K & 400K \\ \midrule

\multirow{3}{*}{Baselines} 
& Joint  & \multicolumn{7}{c}{96.4$\pm$0.3} \\ 
& None   & \multicolumn{7}{c}{93.1$\pm$0.1} \\ 
& GRS    & 93.6$\pm$0.3 & 94.1$\pm$1.3 & 95.3$\pm$0.2 & 95.3$\pm$0.7 & 95.9$\pm$0.1 & 95.8$\pm$0.6 & 96.0$\pm$0.3 \\ 
\midrule

\multirow{4}{*}{\parbox{0.7cm}{Prior \\ Work}} 
& ER~\cite{er}     & 80.6$\pm$0.1 & 73.5$\pm$0.5 & 70.5$\pm$0.3 & 69.9$\pm$0.1 & 70.0$\pm$0.1 & 70.0$\pm$0.1 & 70.0$\pm$0.1 \\ 
& AGEM~\cite{agem}   & 80.5$\pm$0.1 & 73.6$\pm$0.2 & 70.4$\pm$0.3 & 70.0$\pm$0.1 & 70.0$\pm$0.2 & 70.0$\pm$0.1 & 70.0$\pm$0.1 \\ 
& GR~\cite{gr}     & \multicolumn{7}{c}{93.1$\pm$0.2} \\ 
& RtF~\cite{rtf}    & \multicolumn{7}{c}{93.2$\pm$0.2} \\ 
& BI-R~\cite{BIR}   & \multicolumn{7}{c}{93.4$\pm$0.1} \\ 
\midrule

\multirow{4}{*}{\system}      
& \system-R         & \textbf{93.7$\pm$0.1} & \textbf{94.7$\pm$0.1} & \textbf{95.4$\pm$0.1} & \textbf{95.3$\pm$0.6} & \textbf{96.0$\pm$0.1} & \textbf{96.1$\pm$0.1} & \textbf{96.1$\pm$0.1} \\ 
& \system-U         & \textbf{93.6$\pm$0.2} & 94.0$\pm$0.2 & 95.1$\pm$0.1 & \textbf{95.3$\pm$0.1} & 95.5$\pm$0.1 & 95.7$\pm$0.1 & 95.8$\pm$0.1 \\  \cline{2-9}
& MADAR$^{\theta}$-R & \textbf{93.6$\pm$0.1} & \textbf{94.4$\pm$0.3} & \textbf{95.3$\pm$0.2} & \textbf{95.8$\pm$0.1} & \textbf{96.1$\pm$0.1} & \textbf{96.1$\pm$0.1} & \textbf{96.1$\pm$0.1} \\ 
& MADAR$^{\theta}$-U & 93.5$\pm$0.2 & 94.1$\pm$0.2 & 94.9$\pm$0.1 & 95.2$\pm$0.2 & 95.6$\pm$0.1 & 95.7$\pm$0.1 & 95.7$\pm$0.1 \\ 

\bottomrule

\end{tabular}
\vspace{-0.2cm}
\end{table*}









\begin{figure}[!t]
    \centering
    \begin{subfigure}{0.485\linewidth}
        \centering
        \includegraphics[width=1.0\linewidth]{figures_TIFS/EMBER_IFS_DIL_RATIO.pdf}
        \label{fig:EMBER_DIL_IFS_R}
        \vspace{-0.4cm}
        \caption{MADAR Ratio}
    \end{subfigure}
    \hfill
    \begin{subfigure}{0.485\linewidth}
        \centering
        \includegraphics[width=1.0\linewidth]{figures_TIFS/EMBER_IFS_DIL_UNIFORM.pdf}
        \label{fig:EMBER_DIL_IFS_U}
        \vspace{-0.4cm}
        \caption{MADAR Uniform}
    \end{subfigure}
    \vfill
    \begin{subfigure}{0.485\linewidth}
        \centering
        \includegraphics[width=1.0\linewidth]{figures_TIFS/EMBER_AWS_DIL_RATIO.pdf}
        \label{fig:EMBER_DIL_AWS_R}
        \vspace{-0.4cm}
        \caption{MADAR$^\theta$ Ratio}
    \end{subfigure}
    \hfill
    \begin{subfigure}{0.485\linewidth}
        \centering
        \includegraphics[width=1.0\linewidth]{figures_TIFS/EMBER_AWS_DIL_UNIFORM.pdf}
        \label{fig:EMBER_DIL_AWS_U}
        \vspace{-0.4cm}
        \caption{MADAR$^\theta$ Uniform}
    \end{subfigure}

    \caption{EMBER Domain-IL: Comparison of the MADAR-R, MADAR-U, MADAR$^\theta$-R, and MADAR$^\theta$-U with Joint baseline.}
    \label{fig:ember_DIL}
    \vspace{-0.3cm}
\end{figure}





\begin{table*}[!t]
\centering
\caption{Summary of Results for AZ Domain-IL Experiments.}
\vspace{-0.3cm}
\label{tab:az_DIL}
\begin{tabular}{p{1.1cm}|l|c|c|c|c|c|c|c} 

% \toprule 

\multirow{2}{*}{\textbf{Group}} & \multirow{2}{*}{\textbf{Method}} & \multicolumn{7}{c}{\textbf{Budget}} \\ \cline{3-9}

&  & 1K & 10K & 50K & 100K & 200K & 300K & 400K \\ \midrule

\multirow{3}{*}{Baselines} 
& Joint  & \multicolumn{7}{c}{97.3$\pm$0.1} \\ 
& None   & \multicolumn{7}{c}{94.4$\pm$0.1} \\ 
& GRS    & 95.3$\pm$0.1 & 96.4$\pm$0.1 & 96.9$\pm$0.1 & 97.1$\pm$0.1 & 97.1$\pm$0.1 & 97.2$\pm$0.1 & 97.2$\pm$0.1 \\ 
\midrule

\multirow{4}{*}{\parbox{0.7cm}{Prior \\ Work}} 
& ER~\cite{er}     & 40.4$\pm$0.1 & 40.1$\pm$0.1 & 41.1$\pm$0.2 & 42.6$\pm$0.1 & 44.0$\pm$0.1 & 45.9$\pm$0.1 & 48.6$\pm$1.1 \\ 
& AGEM~\cite{agem}   & 45.4$\pm$0.1 & 47.4$\pm$0.2 & 49.2$\pm$0.2 & 53.7$\pm$0.6 & 54.2$\pm$0.3 & 54.8$\pm$0.4 & 56.7$\pm$0.3 \\ 
& GR~\cite{gr}     & \multicolumn{7}{c}{93.3$\pm$0.4} \\ 
& RtF~\cite{rtf}     & \multicolumn{7}{c}{93.4$\pm$0.2} \\ 
& BI-R~\cite{BIR}     & \multicolumn{7}{c}{93.5$\pm$0.1} \\ 
\midrule

\multirow{4}{*}{\system}      
& \system-R         & \textbf{95.8$\pm$0.1} & \textbf{96.6$\pm$0.1} & \textbf{96.9$\pm$0.1} & \textbf{97.0$\pm$0.1} & \textbf{97.0$\pm$0.1} & \textbf{97.0$\pm$0.1} & \textbf{97.0$\pm$0.1} \\ 
& \system-U         & \textbf{95.7$\pm$0.1} & 95.5$\pm$0.1 & 95.2$\pm$0.2 & 95.2$\pm$0.1 & 95.4$\pm$0.1 & 95.8$\pm$0.2 & 96.3$\pm$0.2 \\ \cline{2-9}
& MADAR$^{\theta}$-R & \textbf{95.8$\pm$0.2} & \textbf{96.6$\pm$0.1} & \textbf{96.9$\pm$0.1} & \textbf{96.9$\pm$0.1} & \textbf{97.1$\pm$0.1} & \textbf{97.1$\pm$0.1} & \textbf{97.2$\pm$0.1} \\ 
& MADAR$^{\theta}$-U & 95.6$\pm$0.1 & 96.1$\pm$0.1 & 96.6$\pm$0.1 & 96.8$\pm$0.1 & \textbf{97.0$\pm$0.1} & \textbf{97.1$\pm$0.1} & \textbf{97.1$\pm$0.1} \\ 

\bottomrule

\end{tabular}
\vspace{-0.3cm}
\end{table*}



\begin{figure}[!t]
    \centering
    \begin{subfigure}{0.485\linewidth}
        \centering
        \includegraphics[width=1.0\linewidth]{figures_TIFS/AZ_IFS_DIL_RATIO.pdf}
        \label{fig:AZ_DIL_IFS_R}
        \vspace{-0.4cm}
        \caption{MADAR Ratio}
    \end{subfigure}
    \hfill
    \begin{subfigure}{0.485\linewidth}
        \centering
        \includegraphics[width=1.0\linewidth]{figures_TIFS/AZ_IFS_DIL_UNIFORM.pdf}
        \label{fig:AZ_DIL_IFS_U}
        \vspace{-0.4cm}
        \caption{MADAR Uniform}
    \end{subfigure}
    \hfill
    \begin{subfigure}{0.485\linewidth}
        \centering
        \includegraphics[width=1.0\linewidth]{figures_TIFS/AZ_AWS_DIL_RATIO.pdf}
        \label{fig:AZ_DIL_AWS_R}
        \vspace{-0.4cm}
        \caption{MADAR$^\theta$ Ratio}
    \end{subfigure}
    \hfill
    \begin{subfigure}{0.485\linewidth}
        \centering
        \includegraphics[width=1.0\linewidth]{figures_TIFS/AZ_AWS_DIL_UNIFORM.pdf}
        \label{fig:AZ_DIL_AWS_U}
        \vspace{-0.4cm}
        \caption{MADAR$^\theta$ Uniform}
    \end{subfigure}

    \caption{AZ Domain-IL: Comparison of the MADAR-R, MADAR-U, MADAR$^\theta$-R, and MADAR$^\theta$-U with Joint baseline.}
    \label{fig:az_DIL}
    \vspace{-0.3cm}
\end{figure}






% \subsection{Experimental Setup, Datasets, and Baselines}


We present the results of our \system\ framework and MADAR$^\theta$ in the Domain-IL, Class-IL, and Task-IL scenarios using the EMBER and AZ datasets discussed in Section~\ref{sec:dataset}. To denote our techniques, we use the following abbreviations: {\bf \system-R} for \system-Ratio, {\bf \system-U} for \system-Uniform, {\bf MADAR$^\theta$-R} for MADAR$^\theta$-Ratio, and {\bf MADAR$^\theta$-U} for MADAR$^\theta$-Uniform.

For all three scenarios, we compare \system\ against widely studied replay-based continual learning (CL) techniques, including experience replay (ER)\cite{er}, average gradient episodic memory (AGEM)\cite{agem}, deep generative replay (GR)\cite{gr}, Replay-through-Feedback (RtF)\cite{rtf}, and Brain-inspired Replay (BI-R)\cite{BIR}. Additionally, we evaluate \system\ against iCaRL\cite{icarl}, a replay-based method specifically designed for Class-IL. For the Class-IL and Task-IL scenarios, we additionally compare \system\ with Task-specific Attention Modules in Lifelong Learning (TAMiL)\cite{tamil}. Furthermore, we benchmark MADAR against MalCL\cite{malcl}, a method specifically designed for Class-IL. Notably, most recent work focuses primarily on Class-IL and Task-IL scenarios, limiting direct comparisons in the Domain-IL scenario. In our results tables, the best-performing methods and those within the error margin of the top results are highlighted. 

%Finally, we built upon the codebase provided by \cite{continual-learning-malware} for implementation and evaluation.


% In this study, we utilize large-scale malware datasets, including the EMBER dataset~\cite{ember}, a widely recognized benchmark for Windows malware classification, and two Android malware datasets derived from AndroZoo~\cite{AndroZoo}, which were specifically curated for this research. Our approach is evaluated against two primary baselines:

% \begin{smitemize}
%     \item \textbf{None}: A baseline where the model is trained sequentially on each new task without employing any continual learning (CL) techniques, serving as an informal lower bound.
%     \item \textbf{Joint}: A baseline where the model is trained on both new and previously seen data at each step, representing an informal upper bound. While resource-intensive, the \textbf{Joint} baseline consistently achieves robust performance.
% \end{smitemize}

% Additionally, we introduce a third baseline: \textbf{Global Reservoir Sampling (GRS)}. This method is based on reservoir sampling~\cite{vitter1985random} and builds upon prior work by \cite{continual-learning-malware}. GRS provides an unbiased representation of class distributions and serves as a strong benchmark for comparing our diversity-aware approach.




% We now present the results of our \system framework for both \system and MADAR$^\theta$ in the Domain-IL, Class-IL, and Task-IL scenarios for EMBER and AZ datasets. We use the following four abbreviations to denote our techniques---{\bf \system-R} for \system-Ratio, {\bf ~\system-U} for \system-Uniform, {\bf MADAR$^\theta$-R} for MADAR$^\theta$-Ratio, and {\bf ~MADAR$^\theta$-U} for MADAR$^\theta$-Uniform.  For all three scenarios, we compare \system\ with the most widely studied replay-based CL techniques: experience replay (ER)~\cite{er}, average gradient episodic memory (AGEM)~\cite{agem}, deep generative replay (GR)~\cite{gr}, Replay-through-Feedback (RtF)~\cite{rtf}, and Brain-inspired Replay (BI-R)~\cite{BIR}. In addition, we compare \system\ with iCaRL~\cite{icarl}, a replay-based technique specifically designed for Class-IL. Furthermore, we compare \system with Task-specific Attention Modules in Lifelong learning (TAMiL)~\cite{bhat2023task} which is designed for Class-IL and Task-IL scenarios. In addition, we also compare MADAR with MalCL~\cite{malcl} specifically designed for Class-IL. We observe that recent works mostly focus on Class-IL and Task-IL scenarios which limits what we can compare with in the Domain-IL scenario. The results of the best-performing method, as well as those within the error range of the best results, are highlighted in the results tables. We built upon the code of the prior work by \cite{continual-learning-malware}.

% In this study, we use large-scale Windows and Android malware datasets: EMBER~\cite{ember}, a Windows malware dataset from prior work, recognized as a standard benchmark for malware classification, and two new Android malware datasets derived from AndroZoo~\cite{AndroZoo}, specifically assembled for this research.

% We adopt two baselines for comparison: {\em None} and {\em Joint}.  {\em None} sequentially trains the model on each new task without any CL techniques, serving as an informal minimum baseline. By contrast, {\em Joint} uses all new and prior data for training at each step, acting as an informal maximum baseline. Despite its resource demands, {\em Joint} ensures strong performance throughout the dataset. We also introduce an additional baseline -- Global Reservoir Sampling (GRS) built upon {\em reservoir sampling}~\cite{vitter1985random} and \cite{continual-learning-malware}. GRS provides an unbiased sampling of the underlying class distributions and serves as a strong point of comparison for our diversity-aware approach.

% In this study, we utilize large-scale malware datasets, including the EMBER dataset~\cite{ember}, a widely used benchmark for Windows malware classification, and two Android malware datasets derived from AndroZoo~\cite{AndroZoo}, specifically assembled for this research. We compare our approach against two baselines: {\em None}, where the model is trained sequentially on each new task without any CL techniques, serving as an informal lower bound; and {\em Joint}, which trains on both new and previous data at each step, representing an informal upper bound. Although resource-intensive, {\em Joint} ensures consistently strong results. Additionally, we introduce another baseline -- Global Reservoir Sampling (GRS), an approach based on {\em reservoir sampling}~\cite{vitter1985random} and \cite{continual-learning-malware}, which provides an unbiased representation of class distributions and serves as a strong point of comparison for our diversity-aware approach.


\subsection{Domain-IL}
\label{domainilexps}

%% #of training samples --> 674994
%As shown in Table~\ref{tab:combined_DIL}, a



In EMBER, we have 12 tasks, each representing the monthly data distribution spanning January--December 2018. Our results, detailed in Table~\ref{tab:ember_DIL}, provide a comprehensive view of each method's performance, reported as the average accuracy over all tasks $\mathbf{\overline{AP}}$. Additionally, Figure~\ref{fig:ember_DIL} illustrates the progression of average accuracy over time compared to the \textit{Joint} baseline. 

The informal lower and upper performance bounds for this configuration are approximated by the \textit{None} and \textit{Joint} methods, achieving $\mathbf{\overline{AP}}$ scores of 93.1\% and 96.4\%, respectively. Meanwhile, \textit{GRS} serves as a strong baseline, providing unbiased sampling without incorporating sample diversity awareness.

% In EMBER, we have 12 tasks, each representing the monthly data distribution spanning January--December 2018. Our results, detailed in Table~\ref{tab:ember_DIL}, present a nuanced view of each method's performance, reported as the average accuracy over all tasks $\mathbf{\overline{AP}}$. In addition, Figure~\ref{fig:ember_DIL} represents the progression of average accuracy as the task progresses compared with {joint} baseline. The informal lower and upper performance bounds for this configuration can be approximated by the {\em None} and {\em Joint} methods, which get $\mathbf{\overline{AP}}$ of 93.1\% and 96.4\%, respectively. Meanwhile, {\em GRS} represents a strong baseline for unbiased sampling without awareness of sample diversity.

At a lower budget of 1K, \system-R, \system-U, and MADAR$^\theta$-R exhibit competitive performance, all achieving $\mathbf{\overline{AP}}$ of over $93.6$\%, significantly outperforming prior approaches. This highlights their ability to effectively utilize limited resources. In particular, \system-R achieves the highest accuracy at this budget, with $\mathbf{\overline{AP}}$ of $93.7\%$.

As the memory budget increases, the performance of all \system\ and MADAR$^\theta$ variants improves steadily. At a budget of 200K, \system-R and MADAR$^\theta$-R achieve an impressive $\mathbf{\overline{AP}}$ of $96.0\%$ and $96.1\%$, respectively, closely approaching the $96.4\%$ achieved by the \textit{Joint} baseline, which utilizes over 670K samples. Uniform strategies, including \system-U and MADAR$^\theta$-U, are only slightly behind, with $\mathbf{\overline{AP}}$ values of $95.5\%$ and $95.6\%$, respectively.

% At lower budget of 1K, GRS, \system-R, and \system-U exhibit competitive performance, all significantly better than prior work with $\mathbf{\overline{AP}}$ above $93.6$\%, indicating their effective utilization of limited resources. ER and AGEM performed far below even the \emph{None} baseline, while GR could only match it. For higher budgets, GRS and \system\ methods all show excellent performance. At a 200K budget, \system-R yields $\mathbf{\overline{AP}}$ of $96.0$\%, close to the $96.4$\% reached by the Joint baseline that used over 670K samples. GRS is competitive, while Uniform strategies are only slightly behind.




\begin{table*}[!t]
\centering
\caption{Summary of Results for EMBER Class-IL Experiments.}
\vspace{-0.3cm}
\label{tab:ember_CIL}
\begin{tabular}{p{1.1cm}|l|c|c|c|c|c|c|c} 

% \toprule 

\multirow{2}{*}{\textbf{Group}} & \multirow{2}{*}{\textbf{Method}} & \multicolumn{7}{c}{\textbf{Budget}} \\ \cline{3-9}

&  & 100 & 500 & 1K & 5K & 10K & 15K & 20K \\ \midrule

\multirow{3}{*}{Baselines} 
& Joint  & \multicolumn{7}{c}{86.5$\pm$0.4} \\ 
& None   & \multicolumn{7}{c}{26.5$\pm$0.2} \\ 
& GRS    & 51.9$\pm$0.4 & 70.3$\pm$0.5 & 75.4$\pm$0.7 & 82.0$\pm$0.2 & 83.5$\pm$0.1 & 84.3$\pm$0.3 & 84.6$\pm$0.2 \\ \midrule

\multirow{6}{*}{\parbox{0.7cm}{Prior \\ Work}} 
& TAMiL~\cite{tamil}  & 32.2$\pm$0.3 & 33.1$\pm$0.2 & 35.3$\pm$0.2 & 36.7$\pm$0.1 & 38.2$\pm$0.3 & 37.2$\pm$0.2 & 38.8$\pm$0.2 \\ 
& iCaRL~\cite{icarl}  & 53.9$\pm$0.7 & 58.7$\pm$0.7 & 60.0$\pm$1.0 & 63.9$\pm$1.2 & 64.6$\pm$0.8 & 65.5$\pm$1.0 & 66.8$\pm$1.1 \\ 
& ER~\cite{er}     & 27.5$\pm$0.1 & 27.8$\pm$0.1 & 28.0$\pm$0.1 & 27.9$\pm$0.1 & 28.0$\pm$0.1 & 28.0$\pm$0.1 & 28.2$\pm$0.1 \\ 
& AGEM~\cite{agem}   & 27.3$\pm$0.1 & 27.4$\pm$0.1 & 27.7$\pm$0.1 & 28.5$\pm$0.1 & 28.2$\pm$0.1 & 28.3$\pm$0.1 & 28.2$\pm$0.1 \\ 
& GR~\cite{gr}     & \multicolumn{7}{c}{26.8$\pm$0.2} \\ 
& RtF~\cite{rtf}   & \multicolumn{7}{c}{26.5$\pm$0.1} \\ 
& BI-R~\cite{BIR}   & \multicolumn{7}{c}{26.9$\pm$0.1} \\ 
& MalCL~\cite{malcl}   & \multicolumn{7}{c}{54.5$\pm$0.3} \\ 
\midrule

\multirow{4}{*}{\system} 
& \system-R & \textbf{68.0$\pm$0.4} & 73.6$\pm$0.2 & 76.0$\pm$0.3 & 81.5$\pm$0.2 & 83.2$\pm$0.2 & 83.8$\pm$0.2 & 84.0$\pm$0.2 \\ 
& \system-U & 66.4$\pm$0.4 & \textbf{76.5$\pm$0.2} & \textbf{79.4$\pm$0.4} & \textbf{83.8$\pm$0.2} & \textbf{84.8$\pm$0.1} & \textbf{85.5$\pm$0.1} & \textbf{85.8$\pm$0.3} \\ \cline{2-9}
& MADAR$^{\theta}$-R & {\bf 67.9$\pm$0.3} & 72.7$\pm$0.5 & 72.7$\pm$0.5 & 81.7$\pm$0.2 & 83.2$\pm$0.1 & 83.9$\pm$0.1 & 84.5$\pm$0.2 \\ 
& MADAR$^{\theta}$-U & 67.5$\pm$0.3 & {\bf 76.4$\pm$0.4} & {\bf 78.5$\pm$0.4} & {\bf 84.1$\pm$0.1} & {\bf 85.3$\pm$0.1} & {\bf 85.8$\pm$0.2} & {\bf 86.2$\pm$0.2} \\ 

\bottomrule

\end{tabular}
\vspace{-0.2cm}
\end{table*}



\begin{figure}[!t]
    \centering
    \begin{subfigure}{0.485\linewidth}
        \centering
        \includegraphics[width=1.0\linewidth]{figures_TIFS/EMBER_CIL_IFS_RATIO.pdf}
        \label{fig:EMBER_CIL_IFS_R}
        \vspace{-0.4cm}
        \caption{MADAR Ratio}
    \end{subfigure}
    \hfill
    \begin{subfigure}{0.485\linewidth}
        \centering
        \includegraphics[width=1.0\linewidth]{figures_TIFS/EMBER_CIL_IFS_UNIFORM.pdf}
        \label{fig:EMBER_CIL_IFS_U}
        \vspace{-0.4cm}
        \caption{MADAR Uniform}
    \end{subfigure}
    \vfill
    \begin{subfigure}{0.485\linewidth}
        \centering
        \includegraphics[width=1.0\linewidth]{figures_TIFS/EMBER_CIL_AWS_RATIO.pdf}
        \label{fig:EMBER_CIL_AWS_R}
        \vspace{-0.4cm}
        \caption{MADAR$^\theta$ Ratio}
    \end{subfigure}
    \hfill
    \begin{subfigure}{0.485\linewidth}
        \centering
        \includegraphics[width=1.0\linewidth]{figures_TIFS/EMBER_CIL_AWS_UNIFORM.pdf}
        \label{fig:EMBER_CIL_AWS_U}
        \vspace{-0.4cm}
        \caption{MADAR$^\theta$ Uniform}
    \end{subfigure}

    \caption{EMBER Class-IL: Comparison of the MADAR-R, MADAR-U, MADAR$^\theta$-R, and MADAR$^\theta$-U with Joint baseline.}
    \label{fig:ember_CIL}
    \vspace{-0.3cm}
\end{figure}


For the experiments with AZ-Domain, we consider 9 tasks, each representing a yearly data distribution from 2008 to 2016. The performance of each method is presented in Table~\ref{tab:az_DIL} as $\mathbf{\overline{AP}}$ and compared to two baselines: \textit{None}, which achieves $94.4\%$, and \textit{Joint}, which reaches $97.3\%$. Additionally, Figure~\ref{fig:az_DIL} illustrates the progression of average accuracy across tasks, highlighting the comparison with the \textit{Joint} baseline.

Similar to the results observed with EMBER, our MADAR techniques consistently outperform prior methods such as ER, AGEM, GR, RtF, and BI-R across all budget levels. For lower budgets, such as 1K, \system-R achieves $\mathbf{\overline{AP}}$ of $95.8\%$ and coming within 1.5\% of the \textit{Joint} baseline.

At higher budgets, ranging from 100K to 400K, \system-R continues to demonstrate high $\mathbf{\overline{AP}}$ scores of up to $97.0\%$, closely matching GRS and only marginally below the \textit{Joint} baseline, which requires significantly more training samples (680K). Notably, MADAR$^\theta$-R exhibits comparable performance, reaching a peak $\mathbf{\overline{AP}}$ of $97.2\%$ at the highest budget level, further underscoring the efficacy of our diversity-aware approach.



% For the experiments with AZ-Domain, we have 9 tasks, each representing a year from 2008 to 2016. The performance of each method is shown in Table~\ref{tab:az_DIL} as $\mathbf{\overline{AP}}$ and compared with two baselines: {\em None} at $94.4\pm0.1$ and {\em Joint} at $97.3\pm0.1$. Additionally, Figure~\ref{fig:az_DIL} illustrates the progression of average accuracy as tasks progress, compared to the \textit{Joint} baseline. 

% As with EMBER, we find that our MADAR techniques greatly surpass previous methods like ER, AGEM, GR, RtF, and BI-R for every budget level. For lower budgets like 1K, \system-R slightly outperforms GRS and is within 1.5\% of {\em Joint}. For higher budgets (100K-400K), \system-R perform well -- in line with GRS and just slightly below {\em Joint}, which requires 680K training samples. 


% In summary, our results empirically depict the effectiveness of MADAR's diversity-aware sample selection in maximizing the efficiency and effectiveness of a malware classifier in Domain-IL. \system-R is either better or on par with GRS and significantly better than prior work.

In summary, these results empirically demonstrate the effectiveness of MADAR's diversity-aware sample selection in enhancing the efficiency and accuracy of malware classification in Domain-IL scenarios. \system-R and MADAR$^\theta$-R, in particular, consistently either yield on-par or outperform GRS while delivering significant improvements over prior methods.












\begin{table*}[!t]
\centering
\caption{Summary of Results for AZ Class-IL Experiments.}
\vspace{-0.3cm}
\label{tab:az_CIL}
\begin{tabular}{p{1.1cm}|l|c|c|c|c|c|c|c} 

% \toprule 

\multirow{2}{*}{\textbf{Group}} & \multirow{2}{*}{\textbf{Method}} & \multicolumn{7}{c}{\textbf{Budget}} \\ \cline{3-9}

&  & 100 & 500 & 1K & 5K & 10K & 15K & 20K \\ \midrule

\multirow{3}{*}{Baselines} 
& Joint  & \multicolumn{7}{c}{94.2$\pm$0.1} \\ 
& None   & \multicolumn{7}{c}{26.4$\pm$0.2} \\ 
& GRS    & 43.8$\pm$0.7 & 62.9$\pm$0.8 & 70.2$\pm$0.4 & 83.0$\pm$0.3 & 86.4$\pm$0.2 & 88.2$\pm$0.2 & 89.1$\pm$0.2 \\ \midrule

\multirow{6}{*}{\parbox{0.7cm}{Prior \\ Work}} 
& TAMiL~\cite{tamil}  & 53.4$\pm$0.3 & 55.2$\pm$0.3 & 57.6$\pm$0.3 & 60.8$\pm$0.2 & 63.5$\pm$0.1 & 65.3$\pm$0.5 & 67.7$\pm$0.3 \\ 
& iCaRL~\cite{icarl}  & 43.6$\pm$1.2 & 54.9$\pm$1.0 & 61.7$\pm$0.7 & 77.2$\pm$0.4 & 81.5$\pm$0.6 & 83.4$\pm$0.5 & 84.6$\pm$0.5 \\ 
& ER~\cite{er}     & 50.8$\pm$0.7 & 58.3$\pm$0.6 & 58.9$\pm$0.2 & 59.2$\pm$0.8 & 62.9$\pm$0.7 & 63.1$\pm$0.5 & 64.2$\pm$0.4 \\ 
& AGEM~\cite{agem}   & 27.3$\pm$0.7 & 28.0$\pm$1.4 & 27.1$\pm$0.3 & 28.0$\pm$0.6 & 28.2$\pm$1.0 & 29.8$\pm$2.6 & 28.0$\pm$0.8 \\ 
& GR~\cite{gr}     & \multicolumn{7}{c}{22.7$\pm$0.3} \\ 
& RtF~\cite{rtf}    & \multicolumn{7}{c}{22.9$\pm$0.3} \\ 
& BI-R~\cite{BIR}   & \multicolumn{7}{c}{23.4$\pm$0.2} \\ 
& MalCL~\cite{malcl}   & \multicolumn{7}{c}{59.8$\pm$0.4} \\ 
\midrule

\multirow{4}{*}{\system} 
& \system-R & \textbf{59.4$\pm$0.6} & 67.8$\pm$0.9 & 71.9$\pm$0.5 & 82.9$\pm$0.2 & 86.3$\pm$0.1 & 88.2$\pm$0.2 & 89.1$\pm$0.1 \\ 
& \system-U & 57.3$\pm$0.5 & \textbf{70.4$\pm$0.4} & \textbf{76.2$\pm$0.2} & \textbf{86.8$\pm$0.1} & \textbf{89.8$\pm$0.1} & \textbf{91.0$\pm$0.1} & \textbf{91.5$\pm$0.1} \\ \cline{2-9}
& MADAR$^{\theta}$-R & {\bf 58.8$\pm$0.3} & 66.2$\pm$0.7 & 71.0$\pm$0.7 & 81.2$\pm$0.3 & 85.1$\pm$0.2 & 86.9$\pm$0.2 & 88.1$\pm$0.1 \\ 
& MADAR$^{\theta}$-U & 58.5$\pm$0.7 & {\bf 70.1$\pm$0.2} & {\bf 74.7$\pm$0.2} & {\bf 85.5$\pm$0.1} & {\bf 88.7$\pm$0.1} & {\bf 90.3$\pm$0.2} & {\bf 90.7$\pm$0.1} \\ 

\bottomrule

\end{tabular}
\vspace{-0.2cm}
\end{table*}








\begin{figure}[!t]
    \centering
    \begin{subfigure}{0.485\linewidth}
        \centering
        \includegraphics[width=1.0\linewidth]{figures_TIFS/AZ_CIL_IFS_RATIO.pdf}
        \label{fig:AZ_CIL_IFS_R}
        \vspace{-0.4cm}
        \caption{MADAR Ratio}
    \end{subfigure}
    \hfill
    \begin{subfigure}{0.485\linewidth}
        \centering
        \includegraphics[width=1.0\linewidth]{figures_TIFS/AZ_CIL_IFS_UNIFORM.pdf}
        \label{fig:AZ_CIL_IFS_U}
        \vspace{-0.4cm}
        \caption{MADAR Uniform}
    \end{subfigure}
    \vfill
    \begin{subfigure}{0.485\linewidth}
        \centering
        \includegraphics[width=1.0\linewidth]{figures_TIFS/AZ_CIL_AWS_RATIO.pdf}
        \label{fig:AZ_CIL_AWS_R}
        \vspace{-0.4cm}
        \caption{MADAR$^\theta$ Ratio}
    \end{subfigure}
    \hfill
    \begin{subfigure}{0.485\linewidth}
        \centering
        \includegraphics[width=1.0\linewidth]{figures_TIFS/AZ_CIL_AWS_UNIFORM.pdf}
        \label{fig:AZ_CIL_AWS_U}
        \vspace{-0.4cm}
        \caption{MADAR$^\theta$ Uniform}
    \end{subfigure}

    \caption{AZ Class-IL: Comparison of the MADAR-R, MADAR-U, MADAR$^\theta$-R, and MADAR$^\theta$-U with Joint baseline.}
    \label{fig:az_CIL}
    \vspace{-0.3cm}
\end{figure}





\subsection{Class-IL}
\label{classilexps}



In this set of experiments with EMBER, we consider 11 tasks, starting with 50 classes (representing distinct malware families) in the initial task, and incrementally adding five new classes in each subsequent task. Table~\ref{tab:ember_CIL} presents the performance of each method, measured by average accuracy $\mathbf{\overline{AP}}$. The \textit{None} and \textit{Joint} baselines achieve $\mathbf{\overline{AP}}$ values of $26.5\%$ and $86.5\%$, respectively, providing informal lower and upper bounds. Figure~\ref{fig:ember_CIL} illustrates the progression of average accuracy across tasks, showing how the \system\ and MADAR$^\theta$ methods compare to the \textit{Joint} baseline.

At a very low budget of just 100 samples, \system-R achieves a notable $\mathbf{\overline{AP}}$ of $68.0\%$, outperforming GRS and prior methods by a significant margin. As the budget increases, \system-U emerges as the top performer, achieving $\mathbf{\overline{AP}}$ values of $76.5\%$ and $79.4\%$ at 1K and 10K budgets, respectively, surpassing all other methods, including GRS. 

%For example, at a 10K budget, \system-U outperforms GRS, which achieves $83.5\%$, with an $\mathbf{\overline{AP}}$ of $84.8\%$.

At higher budgets, \system-U and MADAR$^\theta$-U continue to excel, with MADAR$^\theta$-U achieving the best results overall. At a 20K budget, MADAR$^\theta$-U reaches an $\mathbf{\overline{AP}}$ of $86.2\%$, nearly equaling the \textit{Joint} baseline, which uses over {\bf 150 times} more training samples. These results clearly demonstrate the effectiveness of MADAR's diversity-aware sample selection and the effectiveness of \system-U and MADAR$^\theta$-U in leveraging limited resources.

In contrast, prior methods such as ER, AGEM, GR, RtF, and BI-R fail to exceed 30\% $\mathbf{\overline{AP}}$, while more advanced techniques like TAMiL and MalCL achieve only $38.2\%$ and $54.8\%$, respectively. Even iCaRL, designed specifically for Class-IL, achieves only $64.6\%$, further highlighting the significant advantage of our approaches in the malware domain.


% In this set of experiments with EMBER, we have 11 tasks, where the initial task starts with 50 classes---one for each of 50 malware families---and five classes are added in each subsequent task. The performance of these methods, detailed in Table~\ref{tab:az_CIL}, is measured by average accuracy $\mathbf{\overline{AP}}$ with {\em None} and {\em Joint} training baselines at an $\mathbf{\overline{AP}}$ of $26.5\pm0.2$ and $86.5\pm0.4$, respectively. Additionally, Figure~\ref{fig:ember_CIL} illustrates the progression of average accuracy across tasks, highlighting the comparison with the \textit{Joint} baseline. 

% For a very low budget of 100 samples, \system methods greatly outperform GRS, with \system-R getting 16\% higher $\mathbf{\overline{AP}}$. For more reasonable budgets, however, the uniform variant \system-U offers the best performance. For example, with a 10K budget, \system-U yields at least 84.8\% $\mathbf{\overline{AP}}$, which is better than GRS at 83.5\% $\mathbf{\overline{AP}}$. They also fare far better than all prior works, with ER, AGEM, GR, RtF, and BI-R below 30\%, TAMiL at 38.2\%, MalCL at 54.8\% and iCaRL at only 64.6\%. These poor results for the prior methods are in line with other findings in the malware domain~\cite{continual-learning-malware}. For a budget of 20K, \system-U reaches $85.8\pm0.3$, nearly as good as the Joint baseline that uses a maximum budget over 150 times larger.



In the Class-IL setting of AZ-Class, we consider 11 tasks. The summary results of all experiments are provided in Table~\ref{tab:az_CIL}, with comparisons against the \textit{None} and \textit{Joint} baselines, which achieve $\mathbf{\overline{AP}}$ scores of $26.4\%$ and $94.2\%$, respectively. Figure~\ref{fig:az_CIL} illustrates the progression of average accuracy across tasks, showing how each method performs relative to the \textit{Joint} baseline.

As shown in Table~\ref{tab:az_CIL}, among the prior methods, iCaRL performs best across most budget configurations, outperforming techniques such as MalCL, TAMiL, ER, AGEM, GR, RtF, and BI-R. Therefore, we focus on comparing MADAR's performance with iCaRL. At a low budget of 100 samples, iCaRL and GRS achieve less than $44\%$ $\mathbf{\overline{AP}}$, while all MADAR methods surpass $57\%$. In particular, \system-R and MADAR$^\theta$-R achieve $\mathbf{\overline{AP}}$ scores of $59.4\%$ and $58.8\%$, respectively, highlighting their efficiency at low-resource levels.

As the budget increases, all methods improve, but \system-U consistently delivers the best results. At a budget of 1K, \system-U achieves the highest $\mathbf{\overline{AP}}$ at $70.4\%$, followed closely by MADAR$^\theta$-U at $70.1\%$. This trend continues as budgets increase, with \system-U outperforming all other methods, achieving $\mathbf{\overline{AP}}$ scores of $89.8\%$ at 10K and $91.5\%$ at 20K. Compared to GRS, which achieves $90.1\%$ at 20K, and iCaRL, which trails at $84.6\%$, \system-U demonstrates clear superiority. MADAR$^\theta$-U also performs GRS reaching $90.7\%$ at 20K.



% We have 11 tasks for the Class-IL setting of AZ-Class. The summary results of all the experiments are shown in Table~\ref{tab:az_CIL} and benchmarked against {\em None} and {\em Joint} with $\mathbf{\overline{AP}}$ of $26.4\pm0.2$ and $94.2\pm0.1$, respectively. Figure~\ref{fig:az_CIL} illustrates the progression of average accuracy across tasks, highlighting the comparison with the \textit{Joint} baseline. 


% As we can from Table~\ref{tab:az_CIL} that, among TAMiL, iCaRL, ER, AGEM, GR, RtF, and BI-R, iCaRL outperforms in most of the budget configurations. Therefore, we discuss the results of MADAR in comparison with iCaRL. For a low budget of 100, iCaRL and GRS get less than 44\%, while all MADAR methods achieve over 57\%. As budgets increase, all methods improve, with \system-U offering the best results at every budget from 1K to 20K. At 20K, it reaches $91.5\pm0.1\%$, which is 1.4\% higher than GRS and 6.9\% higher than iCaRL.



In summary, our experiments demonstrate the effectiveness of \system's diversity-aware replay techniques in the Class-IL setting for both EMBER and AZ datasets. While GRS shows significant improvement with larger budgets, \system's uniform variants consistently outperform it across all budget levels. These results underscore \system's ability to mitigate catastrophic forgetting and enhance malware classification performance, even in resource-constrained environments.

% In summary, our experiments clearly demonstrate the effectiveness of \system's diversity-aware replay techniques in Class-IL for both EMBER and AZ datasets. Additionally, while GRS shows significant improvement with an increased budget, the uniform variants of \system  are more effective at every budget level. \system  significantly improves performance in malware classification by mitigating catastrophic forgetting, and they do so using fewer resources.












\begin{table*}[!t]
\centering
\caption{Summary of Results for EMBER Task-IL Experiments.}
\vspace{-0.3cm}
\label{tab:ember_TIL}
\begin{tabular}{p{1.1cm}|l|c|c|c|c|c|c|c} 

% \toprule 

\multirow{2}{*}{\textbf{Group}} & \multirow{2}{*}{\textbf{Method}} & \multicolumn{7}{c}{\textbf{Budget}} \\ \cline{3-9}

&  & 100 & 500 & 1K & 5K & 10K & 15K & 20K \\ \midrule

\multirow{3}{*}{Baselines} 
& Joint  & \multicolumn{7}{c}{97.0$\pm$0.3} \\ 
& None   & \multicolumn{7}{c}{74.6$\pm$0.7} \\ 
& GRS    & 86.9$\pm$0.3 & 87.4$\pm$0.3 & 93.6$\pm$0.3 & 94.4$\pm$0.2 & 94.7$\pm$0.3 & 94.9$\pm$0.1 & 95.0$\pm$0.1 \\ \midrule

\multirow{6}{*}{\parbox{0.7cm}{Prior \\ Work}} 
& TAMiL~\cite{tamil}  & 72.8$\pm$0.1 & 81.5$\pm$0.3 & 86.9$\pm$0.2 & 88.1$\pm$0.3 & 90.3$\pm$0.1 & 93.2$\pm$0.3 & 94.2$\pm$0.7 \\ 
& ER~\cite{er}     & 67.4$\pm$0.3 & 84.9$\pm$0.2 & 89.5$\pm$0.5 & 93.9$\pm$0.2 & 94.8$\pm$0.2 & 95.2$\pm$0.1 & 95.4$\pm$0.1 \\ 
& AGEM~\cite{agem}   & 79.6$\pm$0.2 & 81.7$\pm$0.2 & 83.8$\pm$0.4 & 84.9$\pm$0.2 & 86.1$\pm$0.2 & 88.9$\pm$0.2 & 89.3$\pm$0.1 \\ 
& GR~\cite{gr}     & \multicolumn{7}{c}{79.8$\pm$0.3} \\ 
& RtF~\cite{rtf}    & \multicolumn{7}{c}{77.8$\pm$0.2} \\ 
& BI-R~\cite{BIR}   & \multicolumn{7}{c}{87.2$\pm$0.3} \\ \midrule

\multirow{4}{*}{\system} 
& \system-R & 92.1$\pm$0.2 & 92.3$\pm$0.9 & 93.8$\pm$0.2 & 94.2$\pm$0.1 & 94.8$\pm$0.2 & {\bf 95.7$\pm$0.2} & {\bf 95.6$\pm$0.1} \\ 
& \system-U & {\bf 93.4$\pm$0.2} & {\bf 93.7$\pm$0.3} & {\bf 93.9$\pm$0.3} & {\bf 94.8$\pm$0.2} & {\bf 95.6$\pm$0.1} & {\bf 95.7$\pm$0.1} & {\bf 95.8$\pm$0.2} \\ \cline{2-9}
& MADAR$^{\theta}$-R & {\bf 93.1$\pm$0.2} & {\bf 93.3$\pm$0.1} & {\bf 93.6$\pm$0.1} & 94.3$\pm$0.1 & 94.6$\pm$0.2 & 94.8$\pm$0.2 & 94.7$\pm$0.3 \\ 
& MADAR$^{\theta}$-U & {\bf 93.2$\pm$0.1} & 93.1$\pm$0.2 & {\bf 93.8$\pm$0.2} & {\bf 94.4$\pm$0.1} & {\bf 94.8$\pm$0.1} & {\bf 95.3$\pm$0.2} & {\bf 95.5$\pm$0.3} \\ 

\bottomrule

\end{tabular}
\vspace{-0.3cm}
\end{table*}



\begin{figure}[!t]
    \centering
    \begin{subfigure}{0.485\linewidth}
        \centering
        \includegraphics[width=1.0\linewidth]{figures_TIFS/EMBER_TIL_IFS_RATIO.pdf}
        \label{fig:EMBER_TIL_IFS_R}
        \vspace{-0.4cm}
        \caption{MADAR Ratio}
    \end{subfigure}
    \hfill
    \begin{subfigure}{0.485\linewidth}
        \centering
        \includegraphics[width=1.0\linewidth]{figures_TIFS/EMBER_TIL_IFS_UNIFORM.pdf}
        \label{fig:EMBER_TIL_IFS_U}
        \vspace{-0.4cm}
        \caption{MADAR Uniform}
    \end{subfigure}
    \vfill
    \begin{subfigure}{0.485\linewidth}
        \centering
        \includegraphics[width=1.0\linewidth]{figures_TIFS/EMBER_TIL_AWS_RATIO.pdf}
        \label{fig:EMBER_TIL_AWS_R}
        \vspace{-0.4cm}
        \caption{MADAR$^\theta$ Ratio}
    \end{subfigure}
    \hfill
    \begin{subfigure}{0.485\linewidth}
        \centering
        \includegraphics[width=1.0\linewidth]{figures_TIFS/EMBER_TIL_AWS_UNIFORM.pdf}
        \label{fig:EMBER_TIL_AWS_U}
        \vspace{-0.4cm}
        \caption{MADAR$^\theta$ Uniform}
    \end{subfigure}

    \caption{EMBER Task-IL: Comparison of the MADAR-R, MADAR-U, MADAR$^\theta$-R, and MADAR$^\theta$-U with Joint baseline.}
    \label{fig:ember_TIL}
    \vspace{-0.3cm}
\end{figure}

























\subsection{Task-IL}
\label{taskilexps-ember}


In this set of experiments with EMBER, we consider 20 tasks, with 5 new classes added in each task. The summarized results are shown in Table~\ref{tab:ember_TIL}, where performance is reported as the average accuracy over all tasks ($\mathbf{\overline{AP}}$). It is worth noting that Task-IL is considered the easiest scenario in continual learning~\cite{van2022three, BIR}. The \textit{None} and \textit{Joint} methods serve as informal lower and upper bounds, achieving $\mathbf{\overline{AP}}$ scores of $74.6\%$ and $97\%$, respectively. Figure~\ref{fig:ember_TIL} visualizes the progression of average accuracy across tasks, highlighting comparisons with the \textit{Joint} baseline.

As shown in Table~\ref{tab:ember_TIL}, ER consistently outperforms TAMiL, A-GEM, GR, RtF, and BI-R across all budget configurations and even surpasses GRS in some cases. However, \system\ variants significantly outperform all prior methods, particularly under lower budget constraints (100–1K). For example, \system-U achieves the highest $\mathbf{\overline{AP}}$ of $93.4\%$ and $93.7\%$ at budgets of 100 and 1K, respectively, outperforming GRS and all other approaches. Similarly, MADAR$^\theta$-U performs competitively, with $\mathbf{\overline{AP}}$ of $93.2\%$ at a 100 budget and $93.8\%$ at 1K.

As the budget increases, the performance gap among \system, ER, and GRS narrows; however, \system\ variants continue to either outperform or match other techniques. Notably, the \system-U variant of MADAR achieves the best overall performance at a budget of 20K, attaining a $\mathbf{\overline{AP}}$ of $95.8\%$, which closely approaches the \textit{Joint} baseline. Similarly, \system-R yields $\mathbf{\overline{AP}}$ of $95.6\%$ at 20K.



% In this set of experiments with EMBER, we have 20 tasks with 5 new classes in each task. Table~\ref{tab:ember_TIL} shows a summarized view of this set of experiments, where the performances are presented as the average accuracy over all tasks ($\mathbf{\overline{AP}}$). Note that Task-IL is considered the easiest scenario of continual learning~\cite{van2022three, BIR}. The {\em None} and {\em Joint} methods, which are the informal lower and upper bounds of this configuration, attain $\overline{AP}$ of $74.6\%$ and $\overline{AP}$ of $97.03\%$, respectively. Figure~\ref{fig:ember_TIL} illustrates the progression of average accuracy across tasks, showing how each method performs relative to the \textit{Joint} baseline.

% As we can see from Table~\ref{tab:combined_TIL}, ER outperforms TAMiL, A-GEM, GR, RtF, and BI-R in all budget configurations and outperforms GRS for few configurations. \system, on the other hand, outperforms all the prior methods significantly in lower budget constraints ($100$–$1K$). For instance, \system-U reaches $\mathbf{\overline{AP}}$ of 93.9\% with only 1K replay samples, compared with 93.6\% for GRS. The performance gap among MADAR, ER, and GRS gets closer as the budget increases; however, \system  variants continue to either outperform or perform on par with other techniques. In particular, the \system-U variant of MADAR outperforms all the other techniques and attains $\mathbf{\overline{AP}}$ of 95.8\% with a 20K replay budget, which is close to joint level performance.


Task-IL for AZ consists of 20 tasks, each with 5 non-overlapping classes. The results are summarized in Table~\ref{tab:az_TIL} and benchmarked against the \textit{None} and \textit{Joint} baselines, which achieve $\mathbf{\overline{AP}}$ values of $74.5\%$ and $98.8\%$, respectively. Figure~\ref{fig:az_TIL} illustrates the progression of average accuracy across tasks, showing how each method performs relative to the \textit{Joint} baseline.

As seen in Table~\ref{tab:az_TIL}, ER consistently outperforms TAMiL, AGEM, GR, RtF, BI-R, and GRS across most budget configurations, making it a strong baseline for comparison. At a low budget of 100 samples, \system-U achieves $\mathbf{\overline{AP}}$ of $88.1\%$, which is 4.5\% higher than ER's performance. Similarly, MADAR$^\theta$-U demonstrates competitive performance, achieving $87.9\%$ at the same budget.

As the budget increases, \system-U continues to deliver the best performance, reaching $\mathbf{\overline{AP}}$ scores of $94.5\%$ at a 1K budget and $98.1\%$ at a 10K budget, outperforming all other methods, including ER and GRS. At the highest budget of 20K, \system-U achieves an $\mathbf{\overline{AP}}$ of $98.7\%$, surpassing ER by 1.2\% and nearly matching the \textit{Joint} baseline. Notably, MADAR$^\theta$-U also performs well, achieving $98.1\%$. In contrast, \system-R and MADAR$^\theta$-R perform slightly lower but remain competitive, with $\mathbf{\overline{AP}}$ values of $97.9\%$ and $96.9\%$ at a 20K budget, respectively. These results indicate that ratio-based methods, while effective, are slightly less robust than uniform sampling in this scenario.

In summary, \system-U and MADAR$^\theta$-U consistently demonstrate better performance across most of the budget levels, particularly excelling at low-resource settings and achieving near-optimal results at higher budgets. These findings underscore the effectiveness of \system\ framework in Task-IL scenarios and their ability to approach joint-level performance with significantly fewer resources.


% Task-IL for AZ contains 20 tasks, each with 5 non-overlapping classes. Our results are shown in Table~\ref{tab:az_TIL}, compared against the {\em None} and {\em Joint} benchmarks, with $\mathbf{\overline{AP}}$ of 74.5\% and 98.8\%, respectively. Figure~\ref{fig:az_TIL} illustrates the progression of average accuracy across tasks, showing how each method performs relative to the \textit{Joint} baseline. As with EMBER, ER outperforms TAMiL, AGEM, GR, RtF, BI-R, and GRS for most budgets, so we use it for comparison. For a low budget of 100, \system-U achieves an $\overline{AP}$ of 88.1\%, 4.5\% higher than that of ER. For a higher budget of 20K, \system-U attains an $\overline{AP}$ of 98.7\%, which is 1.2\% higher than that of ER and very close to the joint level performance of 98.8\%.


% Overall, mirroring the success seen with the EMBER dataset, our proposed techniques also surpass previous work in Task-IL in the context of the AZ-Class dataset. Additionally, while ER and GRS shows significant improvement with an increased budget, the uniform variant of IFS of MADAR is more effective at every budget level.








\begin{table*}[!t]
\centering
\caption{Summary of Results for AZ Task-IL Experiments.}
\vspace{-0.3cm}
\label{tab:az_TIL}
\begin{tabular}{p{1.1cm}|l|c|c|c|c|c|c|c} 

% \toprule 

\multirow{2}{*}{\textbf{Group}} & \multirow{2}{*}{\textbf{Method}} & \multicolumn{7}{c}{\textbf{Budget}} \\ \cline{3-9}

&  & 100 & 500 & 1K & 5K & 10K & 15K & 20K \\ \midrule

\multirow{3}{*}{Baselines} 
& Joint  & \multicolumn{7}{c}{98.8$\pm$0.2} \\ 
& None   & \multicolumn{7}{c}{74.5$\pm$0.2} \\ 
& GRS    & 85.2$\pm$0.1 & 89.2$\pm$0.2 & 90.8$\pm$0.1 & 91.6$\pm$0.2 & 93.5$\pm$0.1 & 93.9$\pm$0.1 & 95.2$\pm$0.1 \\ \midrule

\multirow{6}{*}{\parbox{0.7cm}{Prior \\ Work}} 
& TAMiL  & 80.5$\pm$0.4 & 85.3$\pm$0.6 & 91.5$\pm$0.2 & 92.1$\pm$0.1 & 93.5$\pm$0.1 & 94.0$\pm$0.2 & 94.8$\pm$0.2 \\ 
& ER     & 83.6$\pm$0.2 & 90.2$\pm$0.1 & 92.3$\pm$0.3 & 95.6$\pm$0.1 & 96.2$\pm$0.1 & 96.8$\pm$0.2 & 97.5$\pm$0.2 \\ 
& AGEM   & 76.7$\pm$0.5 & 82.8$\pm$0.2 & 85.3$\pm$0.1 & 85.6$\pm$0.2 & 86.7$\pm$0.2 & 88.9$\pm$0.2 & 91.3$\pm$0.3 \\ 
& GR     & \multicolumn{7}{c}{75.6$\pm$0.2} \\ 
& RtF    & \multicolumn{7}{c}{74.2$\pm$0.3} \\ 
& BI-R   & \multicolumn{7}{c}{85.4$\pm$0.2} \\ \midrule

\multirow{4}{*}{\system} 
& \system-R & 86.0$\pm$0.3 & 90.3$\pm$0.2 & 92.4$\pm$0.1 & 95.8$\pm$0.2 & 96.7$\pm$0.1 & 97.1$\pm$0.1 & 97.9$\pm$0.2 \\ 
& \system-U & {\bf 88.1$\pm$0.3} & {\bf 92.9$\pm$0.2} & {\bf 94.5$\pm$0.3} & {\bf 97.2$\pm$0.2} & {\bf 98.1$\pm$0.1} & {\bf 98.5$\pm$0.1} & {\bf 98.7$\pm$0.1} \\ \cline{2-9}
& MADAR$^{\theta}$-R & 87.3$\pm$0.3 & {\bf 90.6$\pm$0.2} & 93.2$\pm$0.2 & 95.7$\pm$0.2 & 95.9$\pm$0.1 & 96.6$\pm$0.1 & 96.9$\pm$0.1 \\ 
& MADAR$^{\theta}$-U & {\bf 87.9$\pm$0.2} & {\bf 90.8$\pm$0.2} & {\bf 93.6$\pm$0.1} & {\bf 96.2$\pm$0.3} & {\bf 97.2$\pm$0.2} & {\bf 97.5$\pm$0.2} & {\bf 98.1$\pm$0.1} \\ 

\bottomrule

\end{tabular}
\vspace{-0.3cm}
\end{table*}



\begin{figure}[!t]
    \centering
    \begin{subfigure}{0.45\linewidth}
        \centering
        \includegraphics[width=1.0\linewidth]{figures_TIFS/AZ_TIL_IFS_RATIO.pdf}
        \label{fig:AZ_TIL_IFS_R}
        \vspace{-0.4cm}
        \caption{MADAR Ratio}
    \end{subfigure}
    \hfill
    \begin{subfigure}{0.45\linewidth}
        \centering
        \includegraphics[width=1.0\linewidth]{figures_TIFS/AZ_TIL_IFS_UNIFORM.pdf}
        \label{fig:AZ_TIL_IFS_U}
        \vspace{-0.4cm}
        \caption{MADAR Uniform}
    \end{subfigure}
    \vfill
    \begin{subfigure}{0.45\linewidth}
        \centering
        \includegraphics[width=1.0\linewidth]{figures_TIFS/AZ_TIL_AWS_RATIO.pdf}
        \label{fig:AZ_TIL_AWS_R}
        \vspace{-0.4cm}
        \caption{MADAR$^\theta$ Ratio}
    \end{subfigure}
    \hfill
    \begin{subfigure}{0.45\linewidth}
        \centering
        \includegraphics[width=1.0\linewidth]{figures_TIFS/AZ_TIL_AWS_UNIFORM.pdf}
        \label{fig:AZ_TIL_AWS_U}
        \vspace{-0.4cm}
        \caption{MADAR$^\theta$ Uniform}
    \end{subfigure}

    \caption{AZ Task-IL: Comparison of the MADAR-R, MADAR-U, MADAR$^\theta$-R, and MADAR$^\theta$-U with Joint baseline.}
    \label{fig:az_TIL}
    \vspace{-0.3cm}
\end{figure}


\subsection{Analysis and Discussion}\label{diss}


Our results demonstrate that MADAR yields markedly better performances compared to previous methods for both the EMBER and AZ datasets across all CL settings. This clearly indicates that diversity-aware replay is effective in preserving the stability of a CL-based system for malware classification, while prior CL techniques largely fail to achieve acceptable performance.


\paragraphX{\bf MADAR in low-budget settings.} In Domain-IL, MADAR achieves competitive performance even with a 1K budget, surpassing prior work by over 3 percentage points in EMBER and AZ. At higher budgets, ratio-based selection (\system-R and MADAR$^{\theta}$-R) achieves near Joint baseline performance (96.4\% in EMBER and 97.3\% in AZ) while using significantly fewer resources. This demonstrates MADAR’s efficiency in leveraging limited samples to achieve robust classification.


\paragraphX{\bf MADAR is both effective and scalable.} Traditional CL methods, including ER and AGEM, experience significant performance degradation as tasks increase. In contrast, MADAR maintains high accuracy across 20 Task-IL tasks, with \system-U achieving 95.8\% in EMBER and 98.7\% in AZ at a 20K budget, nearly matching the {\em Joint} baseline.




\paragraphX{\bf Ratio vs. Uniform Budgeting.} A consistent trend across our experiments is that ratio-based selection performs best in Domain-IL, whereas uniform-based selection is superior in Class-IL and Task-IL. MADAR$^{\theta}$-U reaches 91.5\% in AZ at 20K, significantly outperforming iCaRL and TAMiL. Furthermore, in EMBER, \system-U achieves near {\em Joint} baseline performance at just a 5K budget, underscoring the effectiveness of uniform selection in class-incremental settings. Intuitively, this makes sense because ratio budgeting for binary classification in the Domain-IL setting naturally captures the contributions of each family to the overall malware distribution. Additionally, since there are many small families in the Domain-IL datasets, uniformly sampling from them consumes budget while offering little improvement in malware coverage. In contrast, our Class-IL and Task-IL experiments perform classification across families, which is better supported by Uniform budgeting to maintain class balance and ensure coverage over all families. Moreover, in most settings we can leverage efficient representations using MADAR$^\theta$ to scale the approach regardless of feature dimension without significant loss of performance.



\paragraphX{\bf GRS remains a strong baseline at high budgets.} While MADAR consistently outperforms GRS in low-resource settings, GRS performs comparably at higher budgets, particularly in Domain-IL. This suggests that diversity-aware replay is most impactful when the number of available samples per class is limited, whereas uniform selection provides sufficient representation at larger budgets.















\if 0
Our results demonstrate that MADAR yields markedly better performances compared to previous methods for both the EMBER and AZ datasets across all CL settings. This clearly indicates that diversity-aware replay is effective in preserving the stability of a CL-based system for malware classification, while prior CL techniques largely fail to achieve acceptable performance.


In the Domain-IL scenario, MADAR consistently achieves better performance than all other methods, particularly at lower budgets. For example, MADAR's uniform and ratio variants surpass other methods with $\mathbf{\overline{AP}}$ values exceeding $93.6\%$ in EMBER and $95.7\%$ in AZ at a 1K budget. As the memory budget increases, the ratio-based variants (\system-R and MADAR$^\theta$-R) excel, approaching the \textit{Joint} baselines of $96.4\%$ for EMBER and $97.3\%$ for AZ. Notably, these results are achieved with significantly fewer replay samples compared to the \textit{Joint} baseline, highlighting MADAR's efficiency in leveraging limited resources.


In the Class-IL scenario, MADAR achieves remarkable improvements over prior methods, including iCaRL and TAMiL, on both EMBER and AZ datasets. For EMBER, \system-U achieves near \textit{Joint} baseline performance with a budget as low as 5K, outperforming iCaRL  method with fewer resources. Similarly, in AZ, MADAR$^\theta$-U reaches an impressive $\mathbf{\overline{AP}}$ of $91.5\%$ at a 20K budget, significantly surpassing prior techniques. Across both datasets, uniform variants (\system-U and MADAR$^\theta$-U) consistently outperform other methods, demonstrating their effectiveness in managing resources and adapting to evolving class distributions.


In the Task-IL scenario, MADAR outperforms prior methods by a significant margin for both the EMBER and AZ datasets, confirming that diversity-aware replay is effective for this scenario. For EMBER, \system-U achieves $\mathbf{\overline{AP}}$ values of $95.8\%$ at a 20K budget, effectively matching \textit{Joint} performance with a fraction of the resources. For AZ, MADAR$^\theta$-U attains $98.7\%$ at 20K, further underscoring the efficacy of diversity-aware techniques in resource-constrained settings.These findings highlight that the MADAR framework, particularly the uniform variant, not only matches but often exceeds the effectiveness of existing techniques, confirming its robustness across various budget levels in Task-IL.


The Ratio variants worked better for Domain-IL experiments, while Uniform variants worked well in Class-IL and Task-IL. Intuitively, this makes sense because ratio budgeting for binary classification in the Domain-IL setting naturally captures the contributions of each family to the overall malware distribution. Additionally, since there are many small families in the Domain-IL datasets, uniformly sampling from them consumes budget while offering little improvement in malware coverage. In contrast, our Class-IL and Task-IL experiments perform classification across families, which is better supported by Uniform budgeting to maintain class balance and ensure coverage over all families. Moreover, in most settings we can leverage efficient representations using MADAR$^\theta$ to scale the approach regardless of feature dimension without significant loss of performance.


Our results show that GRS performs very well, in some cases closer to the performances of MADAR. Indeed, uniform random sampling should be expected to be a strong baseline, since it provides an unbiased estimate of the true underlying distribution. MADAR is particularly effective in Class-IL and Task-IL, and for lower budgets in Domain-IL, while GRS generally performs as well as MADAR in higher-budget Domain-IL settings. We hypothesize that MADAR's diversity-aware approach is more important when the number of samples per class is limited. In our Domain-IL experiments, larger budgets enable a sufficient representation of the distributions of both classes with uniform selection, making MADAR useful only at smaller budget sizes. 
\fi 

















\section{Conclusion and future directions} \label{sec:conclusion}

In this paper we proposed a nested MLMC framework that offers important computational savings by performing most calculations in low precision and exploiting approximate random normal variables for the low precision path calculations. The low precision calculations could be performed in fixed precision on an FPGA for greater efficiency, and we suggested a procedure to optimise the bit-widths of every variable at each Monte Carlo level. This is an important improvement over previous mixed precision MLMC frameworks which held the lower precision fixed \cite{Rounding_error_oliver} or defined uniform bit-width at every level heuristically \cite{brugger2014mixed}. Our numerical results suggest that for the first levels our procedure reduces the cost at these levels by a factor 5 or 7. Hence the overall savings are significant since most paths are calculated on the first levels. Our approach would be even more efficient for the Milstein scheme because its higher order strong convergence leads to a greater proportion of the computational costs being on the coarsest levels.

The next stage of the research project will be to implement the RNG methods and the nested framework on FPGAs to determine the hardware requirements and confirm the extent of the computational savings. It would also be good to compare the performance benefits to using half-precision floating point arithmetic on GPUs or CPUs for the low-accuracy computations.




\begin{acks}
The authors wish to thank the anonymous reviewers for their valuable comments. 
This paper is partially supported by National Natural Science Foundation of China (NO. U23A20313, 62372471), the Guangzhou Basic and Applied Basic Research Foundation (No. 2024A04J6462), and The Science Foundation for Distinguished Young Scholars of Hunan Province (NO. 2023JJ10080).
\end{acks}

\balance
%%% -*-BibTeX-*-
%%% Do NOT edit. File created by BibTeX with style
%%% ACM-Reference-Format-Journals [18-Jan-2012].

\bibliographystyle{ACM-Reference-Format} 
\bibliography{chi25-714}
\appendix

\section{Appendix: Prompt}
\label{sec:appendix}
``Here is a sketch of an image. 
$\{input\_color\_mask\}$, while the rest of the white space is the background. 
I need you to infer details of the image based on the given sketch.
The details should include the possible background likely to be present with the $\{input\_color\_mask\}$, the attribute of each object (like wearing, texture, color etc.), the state (including action, posture, etc.) of each object, the direction of each object and the relationships between objects.

You should first analyze the mask carefully, considering the size, location, and relative position of each object mask. Ensure that specific actions are analyzed based on the mask, and infer each aspect with a reasoning process before providing the final output.
The final output format should be: $\{format\_example\}$, and you should refer to the example: $\{few\_shot\}$. You are going to complete the "" in each item, you need to complete them in multiple short phrases based on your above reasoning.

The state and relationship should be as detailed as possible while ensuring they align with the mask, formatted as: objectA action/spatial relation objectB, with both objectA and objectB included.
You should properly refer to some examples of attributes of object $\{attributes\}$ and relationships $\{relationships\}$.
Do not include words like `or', `possibly' in your final output, there should no ambiguity in your output.
Make sure all aspects of given mask is filled.''

\end{document}
\endinput
