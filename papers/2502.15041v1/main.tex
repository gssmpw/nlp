%%
%% This is a skeleton file demonstrating the use of IEEEtran.cls
%% (requires IEEEtran.cls version 1.8b or later) with an IEEE Computer
%% Society conference paper.
%%
%% Support sites:
%% http://www.michaelshell.org/tex/ieeetran/
%% http://www.ctan.org/pkg/ieeetran
%% and
%% http://www.ieee.org/

% *** Authors should verify (and, if needed, correct) their LaTeX system  ***
% *** with the testflow diagnostic prior to trusting their LaTeX platform ***
% *** with production work. The IEEE's font choices and paper sizes can   ***
% *** trigger bugs that do not appear when using other class files.       ***
% The testflow support page is at:
% http://www.michaelshell.org/tex/testflow/



\documentclass[conference,compsoc]{IEEEtran}
% Some/most Computer Society conferences require the compsoc mode option,
% but others may want the standard conference format.
%
% If IEEEtran.cls has not been installed into the LaTeX system files,
% manually specify the path to it like:
% \documentclass[conference,compsoc]{../sty/IEEEtran}


% Some very useful LaTeX packages include:
% (uncomment the ones you want to load)


% *** MISC UTILITY PACKAGES ***
%
%\usepackage{ifpdf}
% Heiko Oberdiek's ifpdf.sty is very useful if you need conditional
% compilation based on whether the output is pdf or dvi.
% usage:
% \ifpdf
%   % pdf code
% \else
%   % dvi code
% \fi
% The latest version of ifpdf.sty can be obtained from:
% http://www.ctan.org/pkg/ifpdf
% Also, note that IEEEtran.cls V1.7 and later provides a builtin
% \ifCLASSINFOpdf conditional that works the same way.
% When switching from latex to pdflatex and vice-versa, the compiler may
% have to be run twice to clear warning/error messages.


% *** CITATION PACKAGES ***
%
\ifCLASSOPTIONcompsoc
  % IEEE Computer Society needs nocompress option
  % requires cite.sty v4.0 or later (November 2003)
  \usepackage[nocompress]{cite}
\else
  % normal IEEE
  \usepackage{cite}
\fi
% cite.sty was written by Donald Arseneau
% V1.6 and later of IEEEtran pre-defines the format of the cite.sty package
% \cite{} output to follow that of the IEEE. Loading the cite package will
% result in citation numbers being automatically sorted and properly
% "compressed/ranged". e.g., [1], [9], [2], [7], [5], [6] without using
% cite.sty will become [1], [2], [5]--[7], [9] using cite.sty. cite.sty's
% \cite will automatically add leading space, if needed. Use cite.sty's
% noadjust option (cite.sty V3.8 and later) if you want to turn this off
% such as if a citation ever needs to be enclosed in parenthesis.
% cite.sty is already installed on most LaTeX systems. Be sure and use
% version 5.0 (2009-03-20) and later if using hyperref.sty.
% The latest version can be obtained at:
% http://www.ctan.org/pkg/cite
% The documentation is contained in the cite.sty file itself.
%
% Note that some packages require special options to format as the Computer
% Society requires. In particular, Computer Society  papers do not use
% compressed citation ranges as is done in typical IEEE papers
% (e.g., [1]-[4]). Instead, they list every citation separately in order
% (e.g., [1], [2], [3], [4]). To get the latter we need to load the cite
% package with the nocompress option which is supported by cite.sty v4.0
% and later.


% *** GRAPHICS RELATED PACKAGES ***
%
\ifCLASSINFOpdf
  % \usepackage[pdftex]{graphicx}
  % declare the path(s) where your graphic files are
  % \graphicspath{{../pdf/}{../jpeg/}}
  % and their extensions so you won't have to specify these with
  % every instance of \includegraphics
  % \DeclareGraphicsExtensions{.pdf,.jpeg,.png}
\else
  % or other class option (dvipsone, dvipdf, if not using dvips). graphicx
  % will default to the driver specified in the system graphics.cfg if no
  % driver is specified.
  % \usepackage[dvips]{graphicx}
  % declare the path(s) where your graphic files are
  % \graphicspath{{../eps/}}
  % and their extensions so you won't have to specify these with
  % every instance of \includegraphics
  % \DeclareGraphicsExtensions{.eps}
\fi
% graphicx was written by David Carlisle and Sebastian Rahtz. It is
% required if you want graphics, photos, etc. graphicx.sty is already
% installed on most LaTeX systems. The latest version and documentation
% can be obtained at:
% http://www.ctan.org/pkg/graphicx
% Another good source of documentation is "Using Imported Graphics in
% LaTeX2e" by Keith Reckdahl which can be found at:
% http://www.ctan.org/pkg/epslatex
%
% latex, and pdflatex in dvi mode, support graphics in encapsulated
% postscript (.eps) format. pdflatex in pdf mode supports graphics
% in .pdf, .jpeg, .png and .mps (metapost) formats. Users should ensure
% that all non-photo figures use a vector format (.eps, .pdf, .mps) and
% not a bitmapped formats (.jpeg, .png). The IEEE frowns on bitmapped formats
% which can result in "jaggedy"/blurry rendering of lines and letters as
% well as large increases in file sizes.
%
% You can find documentation about the pdfTeX application at:
% http://www.tug.org/applications/pdftex



% *** MATH PACKAGES ***
%
\usepackage{amsmath}
% A popular package from the American Mathematical Society that provides
% many useful and powerful commands for dealing with mathematics.
%
% Note that the amsmath package sets \interdisplaylinepenalty to 10000
% thus preventing page breaks from occurring within multiline equations. Use:
%\interdisplaylinepenalty=2500
% after loading amsmath to restore such page breaks as IEEEtran.cls normally
% does. amsmath.sty is already installed on most LaTeX systems. The latest
% version and documentation can be obtained at:
% http://www.ctan.org/pkg/amsmath


% *** SPECIALIZED LIST PACKAGES ***
%
%\usepackage{algorithmic}
% algorithmic.sty was written by Peter Williams and Rogerio Brito.
% This package provides an algorithmic environment fo describing algorithms.
% You can use the algorithmic environment in-text or within a figure
% environment to provide for a floating algorithm. Do NOT use the algorithm
% floating environment provided by algorithm.sty (by the same authors) or
% algorithm2e.sty (by Christophe Fiorio) as the IEEE does not use dedicated
% algorithm float types and packages that provide these will not provide
% correct IEEE style captions. The latest version and documentation of
% algorithmic.sty can be obtained at:
% http://www.ctan.org/pkg/algorithms
% Also of interest may be the (relatively newer and more customizable)
% algorithmicx.sty package by Szasz Janos:
% http://www.ctan.org/pkg/algorithmicx


% *** ALIGNMENT PACKAGES ***
%
% \usepackage{array}
% Frank Mittelbach's and David Carlisle's array.sty patches and improves
% the standard LaTeX2e array and tabular environments to provide better
% appearance and additional user controls. As the default LaTeX2e table
% generation code is lacking to the point of almost being broken with
% respect to the quality of the end results, all users are strongly
% advised to use an enhanced (at the very least that provided by array.sty)
% set of table tools. array.sty is already installed on most systems. The
% latest version and documentation can be obtained at:
% http://www.ctan.org/pkg/array


% IEEEtran contains the IEEEeqnarray family of commands that can be used to
% generate multiline equations as well as matrices, tables, etc., of high
% quality.


% *** SUBFIGURE PACKAGES ***
% \ifCLASSOPTIONcompsoc
%  \usepackage[caption=false,font=footnotesize,labelfont=sf,textfont=sf]{subfig}
% \else
%  \usepackage[caption=false,font=footnotesize]{subfig}
% \fi
% subfig.sty, written by Steven Douglas Cochran, is the modern replacement
% for subfigure.sty, the latter of which is no longer maintained and is
% incompatible with some LaTeX packages including fixltx2e. However,
% subfig.sty requires and automatically loads Axel Sommerfeldt's caption.sty
% which will override IEEEtran.cls' handling of captions and this will result
% in non-IEEE style figure/table captions. To prevent this problem, be sure
% and invoke subfig.sty's "caption=false" package option (available since
% subfig.sty version 1.3, 2005/06/28) as this is will preserve IEEEtran.cls
% handling of captions.
% Note that the Computer Society format requires a sans serif font rather
% than the serif font used in traditional IEEE formatting and thus the need
% to invoke different subfig.sty package options depending on whether
% compsoc mode has been enabled.
%
% The latest version and documentation of subfig.sty can be obtained at:
% http://www.ctan.org/pkg/subfig


% *** FLOAT PACKAGES ***
%
%\usepackage{fixltx2e}
% fixltx2e, the successor to the earlier fix2col.sty, was written by
% Frank Mittelbach and David Carlisle. This package corrects a few problems
% in the LaTeX2e kernel, the most notable of which is that in current
% LaTeX2e releases, the ordering of single and double column floats is not
% guaranteed to be preserved. Thus, an unpatched LaTeX2e can allow a
% single column figure to be placed prior to an earlier double column
% figure.
% Be aware that LaTeX2e kernels dated 2015 and later have fixltx2e.sty's
% corrections already built into the system in which case a warning will
% be issued if an attempt is made to load fixltx2e.sty as it is no longer
% needed.
% The latest version and documentation can be found at:
% http://www.ctan.org/pkg/fixltx2e


%\usepackage{stfloats}
% stfloats.sty was written by Sigitas Tolusis. This package gives LaTeX2e
% the ability to do double column floats at the bottom of the page as well
% as the top. (e.g., "\begin{figure*}[!b]" is not normally possible in
% LaTeX2e). It also provides a command:
%\fnbelowfloat
% to enable the placement of footnotes below bottom floats (the standard
% LaTeX2e kernel puts them above bottom floats). This is an invasive package
% which rewrites many portions of the LaTeX2e float routines. It may not work
% with other packages that modify the LaTeX2e float routines. The latest
% version and documentation can be obtained at:
% http://www.ctan.org/pkg/stfloats
% Do not use the stfloats baselinefloat ability as the IEEE does not allow
% \baselineskip to stretch. Authors submitting work to the IEEE should note
% that the IEEE rarely uses double column equations and that authors should try
% to avoid such use. Do not be tempted to use the cuted.sty or midfloat.sty
% packages (also by Sigitas Tolusis) as the IEEE does not format its papers in
% such ways.
% Do not attempt to use stfloats with fixltx2e as they are incompatible.
% Instead, use Morten Hogholm'a dblfloatfix which combines the features
% of both fixltx2e and stfloats:
%
% \usepackage{dblfloatfix}
% The latest version can be found at:
% http://www.ctan.org/pkg/dblfloatfix


% *** PDF, URL AND HYPERLINK PACKAGES ***
%
\usepackage{url}
% url.sty was written by Donald Arseneau. It provides better support for
% handling and breaking URLs. url.sty is already installed on most LaTeX
% systems. The latest version and documentation can be obtained at:
% http://www.ctan.org/pkg/url
% Basically, \url{my_url_here}.




% *** Do not adjust lengths that control margins, column widths, etc. ***
% *** Do not use packages that alter fonts (such as pslatex).         ***
% There should be no need to do such things with IEEEtran.cls V1.6 and later.
% (Unless specifically asked to do so by the journal or conference you plan
% to submit to, of course. )

\usepackage{multirow}
\usepackage{xcolor}
\usepackage{stfloats}

\usepackage{graphicx}
\usepackage{booktabs}
\usepackage{color, colortbl}
\usepackage{subcaption}
\usepackage{algorithm}
\usepackage{algpseudocode}
\usepackage{diagbox}
\usepackage{seqsplit}
\usepackage{soul}
\usepackage[colorlinks=true,urlcolor=black]{hyperref}



% Define colorblind safe colors
\definecolor{red}{HTML}{de2d26}
\definecolor{green}{HTML}{31a354}

\newcommand{\xo}[1]{{\color{red}{XO: {#1}}}}
\newcommand{\dc}[1]{{\color{blue}{DC: {#1}}}}
\newcommand{\gl}[1]{{\color{orange}{GL: {#1}}}}


\AtBeginDocument{%
  \providecommand\BibTeX{{%
    Bib\TeX}}}

% correct bad hyphenation here
\hyphenation{op-tical net-works semi-conduc-tor}



\begin{document}
%
% paper title
% Titles are generally capitalized except for words such as a, an, and, as,
% at, but, by, for, in, nor, of, on, or, the, to and up, which are usually
% not capitalized unless they are the first or last word of the title.
% Linebreaks \\ can be used within to get better formatting as desired.
% Do not put math or special symbols in the title.
\title{Benchmarking Android Malware Detection: Rethinking the Role of Traditional and Deep Learning Models}

% author names and affiliations
% use a multiple column layout for up to three different
% affiliations

\author{\IEEEauthorblockN{Guojun Liu}
  \IEEEauthorblockA{
    University of South Florida\\
    Tampa, Florida\\
    guojunl@usf.edu}
\and
\IEEEauthorblockN{Doina Caragea}
\IEEEauthorblockA{
  Kansas State University\\
  Manhattan, Kansas\\
  dcaragea@ksu.edu}
\and
\IEEEauthorblockN{Xinming Ou}
  \IEEEauthorblockA{
    University of South Florida\\
    Tampa, Florida\\
    xou@usf.edu}
\and
\IEEEauthorblockN{Sankardas Roy}
\IEEEauthorblockA{
  Bowling Green State University\\
  Bowling Green, Ohio\\
  sanroy@bgsu.edu}
}


% use for special paper notices
%\IEEEspecialpapernotice{(Invited Paper)}


% make the title area
\maketitle

% As a general rule, do not put math, special symbols or citations
% in the abstract
\begin{abstract}  
Test time scaling is currently one of the most active research areas that shows promise after training time scaling has reached its limits.
Deep-thinking (DT) models are a class of recurrent models that can perform easy-to-hard generalization by assigning more compute to harder test samples.
However, due to their inability to determine the complexity of a test sample, DT models have to use a large amount of computation for both easy and hard test samples.
Excessive test time computation is wasteful and can cause the ``overthinking'' problem where more test time computation leads to worse results.
In this paper, we introduce a test time training method for determining the optimal amount of computation needed for each sample during test time.
We also propose Conv-LiGRU, a novel recurrent architecture for efficient and robust visual reasoning. 
Extensive experiments demonstrate that Conv-LiGRU is more stable than DT, effectively mitigates the ``overthinking'' phenomenon, and achieves superior accuracy.
\end{abstract}  

% no keywords
\begin{IEEEkeywords}
Android malware detection, Machine learning, Deep learning, Dataset, Benchmark
\end{IEEEkeywords}

% For peer review papers, you can put extra information on the cover
% page as needed:
% \ifCLASSOPTIONpeerreview
% \begin{center} \bfseries EDICS Category: 3-BBND \end{center}
% \fi
%
% For peerreview papers, this IEEEtran command inserts a page break and
% creates the second title. It will be ignored for other modes.
\IEEEpeerreviewmaketitle


\section{Introduction}
\label{sec:introduction}
The business processes of organizations are experiencing ever-increasing complexity due to the large amount of data, high number of users, and high-tech devices involved \cite{martin2021pmopportunitieschallenges, beerepoot2023biggestbpmproblems}. This complexity may cause business processes to deviate from normal control flow due to unforeseen and disruptive anomalies \cite{adams2023proceddsriftdetection}. These control-flow anomalies manifest as unknown, skipped, and wrongly-ordered activities in the traces of event logs monitored from the execution of business processes \cite{ko2023adsystematicreview}. For the sake of clarity, let us consider an illustrative example of such anomalies. Figure \ref{FP_ANOMALIES} shows a so-called event log footprint, which captures the control flow relations of four activities of a hypothetical event log. In particular, this footprint captures the control-flow relations between activities \texttt{a}, \texttt{b}, \texttt{c} and \texttt{d}. These are the causal ($\rightarrow$) relation, concurrent ($\parallel$) relation, and other ($\#$) relations such as exclusivity or non-local dependency \cite{aalst2022pmhandbook}. In addition, on the right are six traces, of which five exhibit skipped, wrongly-ordered and unknown control-flow anomalies. For example, $\langle$\texttt{a b d}$\rangle$ has a skipped activity, which is \texttt{c}. Because of this skipped activity, the control-flow relation \texttt{b}$\,\#\,$\texttt{d} is violated, since \texttt{d} directly follows \texttt{b} in the anomalous trace.
\begin{figure}[!t]
\centering
\includegraphics[width=0.9\columnwidth]{images/FP_ANOMALIES.png}
\caption{An example event log footprint with six traces, of which five exhibit control-flow anomalies.}
\label{FP_ANOMALIES}
\end{figure}

\subsection{Control-flow anomaly detection}
Control-flow anomaly detection techniques aim to characterize the normal control flow from event logs and verify whether these deviations occur in new event logs \cite{ko2023adsystematicreview}. To develop control-flow anomaly detection techniques, \revision{process mining} has seen widespread adoption owing to process discovery and \revision{conformance checking}. On the one hand, process discovery is a set of algorithms that encode control-flow relations as a set of model elements and constraints according to a given modeling formalism \cite{aalst2022pmhandbook}; hereafter, we refer to the Petri net, a widespread modeling formalism. On the other hand, \revision{conformance checking} is an explainable set of algorithms that allows linking any deviations with the reference Petri net and providing the fitness measure, namely a measure of how much the Petri net fits the new event log \cite{aalst2022pmhandbook}. Many control-flow anomaly detection techniques based on \revision{conformance checking} (hereafter, \revision{conformance checking}-based techniques) use the fitness measure to determine whether an event log is anomalous \cite{bezerra2009pmad, bezerra2013adlogspais, myers2018icsadpm, pecchia2020applicationfailuresanalysispm}. 

The scientific literature also includes many \revision{conformance checking}-independent techniques for control-flow anomaly detection that combine specific types of trace encodings with machine/deep learning \cite{ko2023adsystematicreview, tavares2023pmtraceencoding}. Whereas these techniques are very effective, their explainability is challenging due to both the type of trace encoding employed and the machine/deep learning model used \cite{rawal2022trustworthyaiadvances,li2023explainablead}. Hence, in the following, we focus on the shortcomings of \revision{conformance checking}-based techniques to investigate whether it is possible to support the development of competitive control-flow anomaly detection techniques while maintaining the explainable nature of \revision{conformance checking}.
\begin{figure}[!t]
\centering
\includegraphics[width=\columnwidth]{images/HIGH_LEVEL_VIEW.png}
\caption{A high-level view of the proposed framework for combining \revision{process mining}-based feature extraction with dimensionality reduction for control-flow anomaly detection.}
\label{HIGH_LEVEL_VIEW}
\end{figure}

\subsection{Shortcomings of \revision{conformance checking}-based techniques}
Unfortunately, the detection effectiveness of \revision{conformance checking}-based techniques is affected by noisy data and low-quality Petri nets, which may be due to human errors in the modeling process or representational bias of process discovery algorithms \cite{bezerra2013adlogspais, pecchia2020applicationfailuresanalysispm, aalst2016pm}. Specifically, on the one hand, noisy data may introduce infrequent and deceptive control-flow relations that may result in inconsistent fitness measures, whereas, on the other hand, checking event logs against a low-quality Petri net could lead to an unreliable distribution of fitness measures. Nonetheless, such Petri nets can still be used as references to obtain insightful information for \revision{process mining}-based feature extraction, supporting the development of competitive and explainable \revision{conformance checking}-based techniques for control-flow anomaly detection despite the problems above. For example, a few works outline that token-based \revision{conformance checking} can be used for \revision{process mining}-based feature extraction to build tabular data and develop effective \revision{conformance checking}-based techniques for control-flow anomaly detection \cite{singh2022lapmsh, debenedictis2023dtadiiot}. However, to the best of our knowledge, the scientific literature lacks a structured proposal for \revision{process mining}-based feature extraction using the state-of-the-art \revision{conformance checking} variant, namely alignment-based \revision{conformance checking}.

\subsection{Contributions}
We propose a novel \revision{process mining}-based feature extraction approach with alignment-based \revision{conformance checking}. This variant aligns the deviating control flow with a reference Petri net; the resulting alignment can be inspected to extract additional statistics such as the number of times a given activity caused mismatches \cite{aalst2022pmhandbook}. We integrate this approach into a flexible and explainable framework for developing techniques for control-flow anomaly detection. The framework combines \revision{process mining}-based feature extraction and dimensionality reduction to handle high-dimensional feature sets, achieve detection effectiveness, and support explainability. Notably, in addition to our proposed \revision{process mining}-based feature extraction approach, the framework allows employing other approaches, enabling a fair comparison of multiple \revision{conformance checking}-based and \revision{conformance checking}-independent techniques for control-flow anomaly detection. Figure \ref{HIGH_LEVEL_VIEW} shows a high-level view of the framework. Business processes are monitored, and event logs obtained from the database of information systems. Subsequently, \revision{process mining}-based feature extraction is applied to these event logs and tabular data input to dimensionality reduction to identify control-flow anomalies. We apply several \revision{conformance checking}-based and \revision{conformance checking}-independent framework techniques to publicly available datasets, simulated data of a case study from railways, and real-world data of a case study from healthcare. We show that the framework techniques implementing our approach outperform the baseline \revision{conformance checking}-based techniques while maintaining the explainable nature of \revision{conformance checking}.

In summary, the contributions of this paper are as follows.
\begin{itemize}
    \item{
        A novel \revision{process mining}-based feature extraction approach to support the development of competitive and explainable \revision{conformance checking}-based techniques for control-flow anomaly detection.
    }
    \item{
        A flexible and explainable framework for developing techniques for control-flow anomaly detection using \revision{process mining}-based feature extraction and dimensionality reduction.
    }
    \item{
        Application to synthetic and real-world datasets of several \revision{conformance checking}-based and \revision{conformance checking}-independent framework techniques, evaluating their detection effectiveness and explainability.
    }
\end{itemize}

The rest of the paper is organized as follows.
\begin{itemize}
    \item Section \ref{sec:related_work} reviews the existing techniques for control-flow anomaly detection, categorizing them into \revision{conformance checking}-based and \revision{conformance checking}-independent techniques.
    \item Section \ref{sec:abccfe} provides the preliminaries of \revision{process mining} to establish the notation used throughout the paper, and delves into the details of the proposed \revision{process mining}-based feature extraction approach with alignment-based \revision{conformance checking}.
    \item Section \ref{sec:framework} describes the framework for developing \revision{conformance checking}-based and \revision{conformance checking}-independent techniques for control-flow anomaly detection that combine \revision{process mining}-based feature extraction and dimensionality reduction.
    \item Section \ref{sec:evaluation} presents the experiments conducted with multiple framework and baseline techniques using data from publicly available datasets and case studies.
    \item Section \ref{sec:conclusions} draws the conclusions and presents future work.
\end{itemize}

\putsec{related}{Related Work}

\noindent \textbf{Efficient Radiance Field Rendering.}
%
The introduction of Neural Radiance Fields (NeRF)~\cite{mil:sri20} has
generated significant interest in efficient 3D scene representation and
rendering for radiance fields.
%
Over the past years, there has been a large amount of research aimed at
accelerating NeRFs through algorithmic or software
optimizations~\cite{mul:eva22,fri:yu22,che:fun23,sun:sun22}, and the
development of hardware
accelerators~\cite{lee:cho23,li:li23,son:wen23,mub:kan23,fen:liu24}.
%
The state-of-the-art method, 3D Gaussian splatting~\cite{ker:kop23}, has
further fueled interest in accelerating radiance field
rendering~\cite{rad:ste24,lee:lee24,nie:stu24,lee:rho24,ham:mel24} as it
employs rasterization primitives that can be rendered much faster than NeRFs.
%
However, previous research focused on software graphics rendering on
programmable cores or building dedicated hardware accelerators. In contrast,
\name{} investigates the potential of efficient radiance field rendering while
utilizing fixed-function units in graphics hardware.
%
To our knowledge, this is the first work that assesses the performance
implications of rendering Gaussian-based radiance fields on the hardware
graphics pipeline with software and hardware optimizations.

%%%%%%%%%%%%%%%%%%%%%%%%%%%%%%%%%%%%%%%%%%%%%%%%%%%%%%%%%%%%%%%%%%%%%%%%%%
\myparagraph{Enhancing Graphics Rendering Hardware.}
%
The performance advantage of executing graphics rendering on either
programmable shader cores or fixed-function units varies depending on the
rendering methods and hardware designs.
%
Previous studies have explored the performance implication of graphics hardware
design by developing simulation infrastructures for graphics
workloads~\cite{bar:gon06,gub:aam19,tin:sax23,arn:par13}.
%
Additionally, several studies have aimed to improve the performance of
special-purpose hardware such as ray tracing units in graphics
hardware~\cite{cho:now23,liu:cha21} and proposed hardware accelerators for
graphics applications~\cite{lu:hua17,ram:gri09}.
%
In contrast to these works, which primarily evaluate traditional graphics
workloads, our work focuses on improving the performance of volume rendering
workloads, such as Gaussian splatting, which require blending a huge number of
fragments per pixel.

%%%%%%%%%%%%%%%%%%%%%%%%%%%%%%%%%%%%%%%%%%%%%%%%%%%%%%%%%%%%%%%%%%%%%%%%%%
%
In the context of multi-sample anti-aliasing, prior work proposed reducing the
amount of redundant shading by merging fragments from adjacent triangles in a
mesh at the quad granularity~\cite{fat:bou10}.
%
While both our work and quad-fragment merging (QFM)~\cite{fat:bou10} aim to
reduce operations by merging quads, our proposed technique differs from QFM in
many aspects.
%
Our method aims to blend \emph{overlapping primitives} along the depth
direction and applies to quads from any primitive. In contrast, QFM merges quad
fragments from small (e.g., pixel-sized) triangles that \emph{share} an edge
(i.e., \emph{connected}, \emph{non-overlapping} triangles).
%
As such, QFM is not applicable to the scenes consisting of a number of
unconnected transparent triangles, such as those in 3D Gaussian splatting.
%
In addition, our method computes the \emph{exact} color for each pixel by
offloading blending operations from ROPs to shader units, whereas QFM
\emph{approximates} pixel colors by using the color from one triangle when
multiple triangles are merged into a single quad.



\section{Research Methodology}~\label{sec:Methodology}

In this section, we discuss the process of conducting our systematic review, e.g., our search strategy for data extraction of relevant studies, based on the guidelines of Kitchenham et al.~\cite{kitchenham2022segress} to conduct SLRs and Petersen et al.~\cite{PETERSEN20151} to conduct systematic mapping studies (SMSs) in Software Engineering. In this systematic review, we divide our work into a four-stage procedure, including planning, conducting, building a taxonomy, and reporting the review, illustrated in Fig.~\ref{fig:search}. The four stages are as follows: (1) the \emph{planning} stage involved identifying research questions (RQs) and specifying the detailed research plan for the study; (2) the \emph{conducting} stage involved analyzing and synthesizing the existing primary studies to answer the research questions; (3) the \emph{taxonomy} stage was introduced to optimize the data extraction results and consolidate a taxonomy schema for REDAST methodology; (4) the \emph{reporting} stage involved the reviewing, concluding and reporting the final result of our study.

\begin{figure}[!t]
    \centering
    \includegraphics[width=1\linewidth]{fig/methodology/searching-process.drawio.pdf}
    \caption{Systematic Literature Review Process}
    \label{fig:search}
\end{figure}

\subsection{Research Questions}
In this study, we developed five research questions (RQs) to identify the input and output, analyze technologies, evaluate metrics, identify challenges, and identify potential opportunities. 

\textbf{RQ1. What are the input configurations, formats, and notations used in the requirements in requirements-driven
automated software testing?} In requirements-driven testing, the input is some form of requirements specification -- which can vary significantly. RQ1 maps the input for REDAST and reports on the comparison among different formats for requirements specification.

\textbf{RQ2. What are the frameworks, tools, processing methods, and transformation techniques used in requirements-driven automated software testing studies?} RQ2 explores the technical solutions from requirements to generated artifacts, e.g., rule-based transformation applying natural language processing (NLP) pipelines and deep learning (DL) techniques, where we additionally discuss the potential intermediate representation and additional input for the transformation process.

\textbf{RQ3. What are the test formats and coverage criteria used in the requirements-driven automated software
testing process?} RQ3 focuses on identifying the formulation of generated artifacts (i.e., the final output). We map the adopted test formats and analyze their characteristics in the REDAST process.

\textbf{RQ4. How do existing studies evaluate the generated test artifacts in the requirements-driven automated software testing process?} RQ4 identifies the evaluation datasets, metrics, and case study methodologies in the selected papers. This aims to understand how researchers assess the effectiveness, accuracy, and practical applicability of the generated test artifacts.

\textbf{RQ5. What are the limitations and challenges of existing requirements-driven automated software testing methods in the current era?} RQ5 addresses the limitations and challenges of existing studies while exploring future directions in the current era of technology development. %It particularly highlights the potential benefits of advanced LLMs and examines their capacity to meet the high expectations placed on these cutting-edge language modeling technologies. %\textcolor{blue}{CA: Do we really need to focus on LLMs? TBD.} \textcolor{orange}{FW: About LLMs, I removed the direct emphase in RQ5 but kept the discussion in RQ5 and the solution section. I think that would be more appropriate.}

\subsection{Searching Strategy}

The overview of the search process is exhibited in Fig. \ref{fig:papers}, which includes all the details of our search steps.
\begin{table}[!ht]
\caption{List of Search Terms}
\label{table:search_term}
\begin{tabularx}{\textwidth}{lX}
\hline
\textbf{Terms Group} & \textbf{Terms} \\ \hline
Test Group & test* \\
Requirement Group & requirement* OR use case* OR user stor* OR specification* \\
Software Group & software* OR system* \\
Method Group & generat* OR deriv* OR map* OR creat* OR extract* OR design* OR priorit* OR construct* OR transform* \\ \hline
\end{tabularx}
\end{table}

\begin{figure}
    \centering
    \includegraphics[width=1\linewidth]{fig/methodology/search-papers.drawio.pdf}
    \caption{Study Search Process}
    \label{fig:papers}
\end{figure}

\subsubsection{Search String Formulation}
Our research questions (RQs) guided the identification of the main search terms. We designed our search string with generic keywords to avoid missing out on any related papers, where four groups of search terms are included, namely ``test group'', ``requirement group'', ``software group'', and ``method group''. In order to capture all the expressions of the search terms, we use wildcards to match the appendix of the word, e.g., ``test*'' can capture ``testing'', ``tests'' and so on. The search terms are listed in Table~\ref{table:search_term}, decided after iterative discussion and refinement among all the authors. As a result, we finally formed the search string as follows:


\hangindent=1.5em
 \textbf{ON ABSTRACT} ((``test*'') \textbf{AND} (``requirement*'' \textbf{OR} ``use case*'' \textbf{OR} ``user stor*'' \textbf{OR} ``specifications'') \textbf{AND} (``software*'' \textbf{OR} ``system*'') \textbf{AND} (``generat*'' \textbf{OR} ``deriv*'' \textbf{OR} ``map*'' \textbf{OR} ``creat*'' \textbf{OR} ``extract*'' \textbf{OR} ``design*'' \textbf{OR} ``priorit*'' \textbf{OR} ``construct*'' \textbf{OR} ``transform*''))

The search process was conducted in September 2024, and therefore, the search results reflect studies available up to that date. We conducted the search process on six online databases: IEEE Xplore, ACM Digital Library, Wiley, Scopus, Web of Science, and Science Direct. However, some databases were incompatible with our default search string in the following situations: (1) unsupported for searching within abstract, such as Scopus, and (2) limited search terms, such as ScienceDirect. Here, for (1) situation, we searched within the title, keyword, and abstract, and for (2) situation, we separately executed the search and removed the duplicate papers in the merging process. 

\subsubsection{Automated Searching and Duplicate Removal}
We used advanced search to execute our search string within our selected databases, following our designed selection criteria in Table \ref{table:selection}. The first search returned 27,333 papers. Specifically for the duplicate removal, we used a Python script to remove (1) overlapped search results among multiple databases and (2) conference or workshop papers, also found with the same title and authors in the other journals. After duplicate removal, we obtained 21,652 papers for further filtering.

\begin{table*}[]
\caption{Selection Criteria}
\label{table:selection}
\begin{tabularx}{\textwidth}{lX}
\hline
\textbf{Criterion ID} & \textbf{Criterion Description} \\ \hline
S01          & Papers written in English. \\
S02-1        & Papers in the subjects of "Computer Science" or "Software Engineering". \\
S02-2        & Papers published on software testing-related issues. \\
S03          & Papers published from 1991 to the present. \\ 
S04          & Papers with accessible full text. \\ \hline
\end{tabularx}
\end{table*}

\begin{table*}[]
\small
\caption{Inclusion and Exclusion Criteria}
\label{table:criteria}
\begin{tabularx}{\textwidth}{lX}
\hline
\textbf{ID}  & \textbf{Description} \\ \hline
\multicolumn{2}{l}{\textbf{Inclusion Criteria}} \\ \hline
I01 & Papers about requirements-driven automated system testing or acceptance testing generation, or studies that generate system-testing-related artifacts. \\
I02 & Peer-reviewed studies that have been used in academia with references from literature. \\ \hline
\multicolumn{2}{l}{\textbf{Exclusion Criteria}} \\ \hline
E01 & Studies that only support automated code generation, but not test-artifact generation. \\
E02 & Studies that do not use requirements-related information as an input. \\
E03 & Papers with fewer than 5 pages (1-4 pages). \\
E04 & Non-primary studies (secondary or tertiary studies). \\
E05 & Vision papers and grey literature (unpublished work), books (chapters), posters, discussions, opinions, keynotes, magazine articles, experience, and comparison papers. \\ \hline
\end{tabularx}
\end{table*}

\subsubsection{Filtering Process}

In this step, we filtered a total of 21,652 papers using the inclusion and exclusion criteria outlined in Table \ref{table:criteria}. This process was primarily carried out by the first and second authors. Our criteria are structured at different levels, facilitating a multi-step filtering process. This approach involves applying various criteria in three distinct phases. We employed a cross-verification method involving (1) the first and second authors and (2) the other authors. Initially, the filtering was conducted separately by the first and second authors. After cross-verifying their results, the results were then reviewed and discussed further by the other authors for final decision-making. We widely adopted this verification strategy within the filtering stages. During the filtering process, we managed our paper list using a BibTeX file and categorized the papers with color-coding through BibTeX management software\footnote{\url{https://bibdesk.sourceforge.io/}}, i.e., “red” for irrelevant papers, “yellow” for potentially relevant papers, and “blue” for relevant papers. This color-coding system facilitated the organization and review of papers according to their relevance.

The screening process is shown below,
\begin{itemize}
    \item \textbf{1st-round Filtering} was based on the title and abstract, using the criteria I01 and E01. At this stage, the number of papers was reduced from 21,652 to 9,071.
    \item \textbf{2nd-round Filtering}. We attempted to include requirements-related papers based on E02 on the title and abstract level, which resulted from 9,071 to 4,071 papers. We excluded all the papers that did not focus on requirements-related information as an input or only mentioned the term ``requirements'' but did not refer to the requirements specification.
    \item \textbf{3rd-round Filtering}. We selectively reviewed the content of papers identified as potentially relevant to requirements-driven automated test generation. This process resulted in 162 papers for further analysis.
\end{itemize}
Note that, especially for third-round filtering, we aimed to include as many relevant papers as possible, even borderline cases, according to our criteria. The results were then discussed iteratively among all the authors to reach a consensus.

\subsubsection{Snowballing}

Snowballing is necessary for identifying papers that may have been missed during the automated search. Following the guidelines by Wohlin~\cite{wohlin2014guidelines}, we conducted both forward and backward snowballing. As a result, we identified 24 additional papers through this process.

\subsubsection{Data Extraction}

Based on the formulated research questions (RQs), we designed 38 data extraction questions\footnote{\url{https://drive.google.com/file/d/1yjy-59Juu9L3WHaOPu-XQo-j-HHGTbx_/view?usp=sharing}} and created a Google Form to collect the required information from the relevant papers. The questions included 30 short-answer questions, six checkbox questions, and two selection questions. The data extraction was organized into five sections: (1) basic information: fundamental details such as title, author, venue, etc.; (2) open information: insights on motivation, limitations, challenges, etc.; (3) requirements: requirements format, notation, and related aspects; (4) methodology: details, including immediate representation and technique support; (5) test-related information: test format(s), coverage, and related elements. Similar to the filtering process, the first and second authors conducted the data extraction and then forwarded the results to the other authors to initiate the review meeting.

\subsubsection{Quality Assessment}

During the data extraction process, we encountered papers with insufficient information. To address this, we conducted a quality assessment in parallel to ensure the relevance of the papers to our objectives. This approach, also adopted in previous secondary studies~\cite{shamsujjoha2021developing, naveed2024model}, involved designing a set of assessment questions based on guidelines by Kitchenham et al.~\cite{kitchenham2022segress}. The quality assessment questions in our study are shown below:
\begin{itemize}
    \item \textbf{QA1}. Does this study clearly state \emph{how} requirements drive automated test generation?
    \item \textbf{QA2}. Does this study clearly state the \emph{aim} of REDAST?
    \item \textbf{QA3}. Does this study enable \emph{automation} in test generation?
    \item \textbf{QA4}. Does this study demonstrate the usability of the method from the perspective of methodology explanation, discussion, case examples, and experiments?
\end{itemize}
QA4 originates from an open perspective in the review process, where we focused on evaluation, discussion, and explanation. Our review also examined the study’s overall structure, including the methodology description, case studies, experiments, and analyses. The detailed results of the quality assessment are provided in the Appendix. Following this assessment, the final data extraction was based on 156 papers.

% \begin{table}[]
% \begin{tabular}{ll}
% \hline
% QA ID & QA Questions                                             \\ \hline
% Q01   & Does this study clearly state its aims?                  \\
% Q02   & Does this study clearly describe its methodology?        \\
% Q03   & Does this study involve automated test generation?       \\
% Q04   & Does this study include a promising evaluation?          \\
% Q05   & Does this study demonstrate the usability of the method? \\ \hline
% \end{tabular}%
% \caption{Questions for Quality Assessment}
% \label{table:qa}
% \end{table}

% automated quality assessment

% \textcolor{blue}{CA: Our search strategy focused on identifying requirements types first. We covered several sources, e.g., ~\cite{Pohl:11,wagner2019status} to identify different formats and notations of specifying requirements. However, this came out to be a long list, e.g., free-form NL requirements, semi-formal UML models, free-from textual use case models, UML class diagrams, UML activity diagrams, and so on. In this paper, we attempted to primarily focus on requirements-related aspects and not design-level information. Hence, we generalised our search string to include generic keywords, e.g., requirement*, use case*, and user stor*. We did so to avoid missing out on any papers, bringing too restrictive in our search strategy, and not creating a too-generic search string with all the aforementioned formats to avoid getting results beyond our review's scope.}


%% Use \subsection commands to start a subsection.



%\subsection{Study Selection}

% In this step, we further looked into the content of searched papers using our search strategy and applied our inclusion and exclusion criteria. Our filtering strategy aimed to pinpoint studies focused on requirements-driven system-level testing. Recognizing the presence of irrelevant papers in our search results, we established detailed selection criteria for preliminary inclusion and exclusion, as shown in Table \ref{table: criteria}. Specifically, we further developed the taxonomy schema to exclude two types of studies that did not meet the requirements for system-level testing: (1) studies supporting specification-driven test generation, such as UML-driven test generation, rather than requirements-driven testing, and (2) studies focusing on code-based test generation, such as requirement-driven code generation for unit testing.





\section{Traditional Machine Learning Models Considered}
\label{sec:alternative_ml}

Widely explored ML models in Android malware detection
include Naive Bayes (NB)~\cite{Yerima:AINA13, Sharma:CANS14}, support
vector machines (SVM)~\cite{Gascon:AISec13, Arp:NDSS14}, k-nearest
neighbors (k-NN)~\cite{Wu:AsiaJCIS12, Sharma:CANS14}, and random
forest (RF)~\cite{Mariconti:NDSS2017}.  We consider these four widely
used models as potential baselines to compare with the proposed
DL models.  Additionally, Gradient
Boosted Decision Trees (GBDTs) are powerful classification tools, with
CatBoost~\cite{CatBoost:NEURIPS2018} being a notable variant of
GBDT-based ensemble techniques. 
Having been introduced in late 2018, CatBoost has demonstrated strong
performance in classification and regression tasks across various
domains, including cybersecurity~\cite{CatBoost:NEURIPS2018}. Thus we also include
it as one of the ML models in our benchmarking. % Given
% their effectiveness, we consider these five widely used ML models as
% additional baselines comparisons for DL models DetectBERT, Enc+MLP, CapsGNN and ExcelFormer.

\textit{NB:} This is a simple but effective classifier. Furthermore,
it is the fastest classifier we evaluated. It can get the
classification result quickly even with half a million apps.

    \textit{SVM:} Widely used classifier in Android malware
classification. It requires more memory and computation time during
the training stage, which poses a significant challenge for
large datasets. We set the kernel to be linear, and use a grid
search with values 0.001, 0.1, 1.0, and 10 to decide the optimal
regularization parameter to train the SVM model.

    \textit{k-NN:} Another simple and robust traditional
classifier. The value of the hyper-parameter \(k\) has a direct impact
on the performance. We use a grid search with values between 3 and 15
to determine the optimal \(k\) value to train the k-NN model.

    \textit{RF:} A popular ensemble-type model based on decision
trees. It has shown promising performance in Android malware
detection~\cite{Mariconti:NDSS2017} and has the capability to be
applied to market-scale datasets~\cite{Gong:EuroSys20}. We perform a
grid search to find the optimal number of trees in the forest. The
estimator value is between 100 and 300.

    \textit{CatBoost:} Another ensemble-based model built on decision
trees. While it is relatively new compared to the other ML
models, recent studies across various domains have demonstrated its
effectiveness in classification
tasks~\cite{CatBoost:NEURIPS2018}. % Although it has not been widely
% applied to Android malware detection, its strong performance makes it
% a promising candidate for this domain.
CatBoost is sensitive to
hyperparameters, and our experimental results are based on the
following setup: 1000 iterations, a learning rate of 0.1, a depth of
10, and an L2 regularization parameter (l2\_leaf\_reg) of 5.

%%% Local Variables:
%%% mode: latex
%%% TeX-master: "../main"
%%% End:


\section{Additional Benchmarking for DetectBERT}

\textbf{DetectBERT} is a recent approach proposed by Sun et al.~\cite{Sun:esem24}, which utilizes a pre-trained BERT-like model combined with Correlated Multiple Instance Learning (c-MIL) to process Smali code from APKs. This approach enhances class-level features and aggregates them into an app-level representation, improving the effectiveness of Android malware detection. 

\subsection{Benchmarking for DetectBERT}
DetectBERT was evaluated against two state-of-the-art Android malware
detection models, Drebin~\cite{Arp:NDSS14} and
DexRAY~\cite{Nadia:MALHat21}. Drebin utilizes SVM on static feature
vectors, while DexRAY is a DL framework that transforms low-level
bytecode into images and processes them using a CNN for malware
detection. The results indicate that DetectBERT slightly outperforms
both baseline models.

\subsection{Dataset for DetectBERT}
\textbf{DetectBERT Benchmark Dataset.} This large-scale dataset originates from the DexRay~\cite{Nadia:MALHat21} study and comprises 96,994 benign apps and 61,809 malware apps. The labeling process is based on VirusTotal reports: apps that are not flagged by any antivirus engines are considered benign, while those detected as malicious by more than two antivirus engines are labeled as malware.  

\subsection{Additional Benchmarking}

% DetectBERT
\begin{table}[t]
    \centering
    \begin{tabular}{lcccc}
        \toprule[0.8pt]
        \textbf{Model} & \textbf{Accuracy} & \textbf{Precision} & \textbf{Recall} & \textbf{F1 Score} \\
        \hline
        Drebin & 0.97 & 0.97 & 0.94 & 0.96 \\
        DexRay & 0.97 & 0.97 & 0.95 & 0.96 \\
        \textbf{DetectBERT} & \textbf{0.97} & \textbf{0.98} & \textbf{0.95} & \textbf{0.97} \\

        \midrule\midrule
        \rowcolor{gray!20} NB & 0.85 & 0.87 & 0.85 & 0.86 \\
        \rowcolor{gray!20} RF & 0.97 & 0.97 & 0.97 & 0.97 \\
        \rowcolor{gray!20} KNN & 0.97 & 0.97 & 0.97 & 0.97 \\
        \rowcolor{gray!20} SVM & 0.96 & 0.96 & 0.96 & 0.96 \\
        \rowcolor{gray!20} CatBoost & 0.98 & 0.98 & 0.98 & 0.98 \\
        \bottomrule[0.8pt]
    \end{tabular}
    \vspace{10pt}
    \caption{Performance comparison of various models on DetectBERT benchmark dataset. The unshaded are from the original paper, while the  shaded results are from our benchmarking.}
    \label{tab:detectbert_comparison}
    \vspace{-.2in}
  \end{table}

Sun et al.~\cite{Sun:esem24} open sourced their DetectBERT model for
Android malware detection, which enables us to use the exact same
dataset, data splits, and evaluation metrics for training and testing
various models, ensuring fair comparisons.

One of the challenges in applying the DetectBERT model is generating
the feature vector for each smali class in an APK file from a
large-scale dataset. To address this, we explored the performance of
more cost-effective approaches on this dataset.  Since DREBIN is one
of the baseline models used in the original research, we generated DREBIN-like static features for
each app, including requested permissions, component names, intents,
suspicious API calls, and network addresses. We then calculated the
mutual information of the unique static features in this dataset and
ranked them accordingly. By testing different feature vector sizes
based on the top N features, we found that a feature vector consisting
of 2,919 static features per app yielded optimal
results.  Table~\ref{tab:detectbert_comparison} presents the
performance of various ML models against DetectBERT. The shaded rows represent the models we tested with
these features. Except for NB, all other
models (KNN, SVM, RF, and CatBoost) achieved performance comparable to
DetectBERT. CatBoost slightly outperforms DetectBERT.
These ML models also require significantly less compute
time than DetectBERT.

In the paper the authors also conducted experiments
on data sets that adhere to temporal consistency. However
the published dataset does not include the train/test data split
for this experiment, and as a result we were not able
to compare additional ML models' results against the results
published in the paper.

% performance due to the unavailability of the
% original training and testing set lists from the open-source data. If
% we were to create our own splits based on release year, the resulting
% dataset might differ from the original work, compromising the fairness
% of the comparison. 

%%% Local Variables:
%%% mode: latex
%%% TeX-master: "../main"
%%% End:


\section{Additional Benchmarking for Enc+MLP}

\textbf{Enc + MLP} is % one of the most recent deep active learning
% schemes
proposed by Chen et al.~\cite{Chen:USENIX23} to address the
concept drift~\cite{Jordaney:USENIX17} problem in Android malware
detection. This scheme integrates contrastive learning with active
learning, structuring the model into two subnetworks. The first
subnetwork, a hierarchical contrastive encoder
(\textbf{\textit{Enc}}), employs contrastive learning techniques to
encode input embeddings, ensuring that embeddings of the same malware
family are closer to each other. The second subnetwork is an
\textbf{MLP} classifier that utilizes these embeddings for malware
classification.

The model is initially trained on a labeled dataset to establish the
hierarchical contrastive classifier. Following this, an active
learning process begins, where the trained classifier predicts labels
for a batch of test samples. A pseudo-loss sample selector in the
hierarchical contrastive classifier identifies the most uncertain apps
within a predefined labeling budget and adds them to the training set
for the next iteration. The model is then retrained using a warm
start, and this process repeats continuously, refining the model with
each subsequent batch of test data. This approach improves upon
existing active learning baselines for Android malware detection while
minimizing the need for manual labeling, as demonstrated through
evaluations on two publicly available datasets.


\subsection{Benchmarking for \textit{Enc+MLP}}

\textit{Enc+MLP} uses multiple baseline models and explores
uncertainty sampling for both binary and multiclass classifiers.  The
binary classifiers include a fully connected neural network (Binary
MLP), a linear Support Vector Machine (Binary SVM), and Gradient
Boosted Decision Trees (GBDT).  The multiclass classifiers include an
MLP and an SVM, and the authors also experimented with a combined
classifier, ``Multiclass MLP + Binary SVM.''  Uncertainty is measured as
one minus the maximum prediction score across all classes.  All
baseline models employ a cold-start approach (retraining from scratch
on the updated training data) for active learning.

\subsection{Datasets for \textit{Enc+MLP}}

\textit{Enc+MLP} was evaluated on two datasets: APIGraph and AndroZoo.\\


\begin{table*}[t]
  \captionsetup{skip=4pt}
    \centering
    \begin{minipage}{0.49\textwidth} 
        \centering
        \begin{tabular}{c|c|c|c}
        \bottomrule
        \textbf{Year} & \multicolumn{1}{c|}{\begin{tabular}[c|]{@{}c@{}}\textbf{Malicious}\\ \textbf{Apps}\end{tabular}} & \multicolumn{1}{c|}{\begin{tabular}[c]{@{}c@{}}\textbf{Benign} \\ \textbf{Apps}\end{tabular}} & \textbf{Total} \\ 
        \hline
        2012 & 3,061  & 27,472  & 30,533  \\ 
        2013 & 4,854  & 43,714  & 48,568  \\ 
        2014 & 5,809  & 52,676  & 58,485  \\ 
        2015 & 5,508  & 51,944  & 57,452  \\ 
        2016 & 5,324  & 50,712  & 56,036  \\ 
        2017 & 2,465  & 24,847  & 27,312  \\ 
        2018 & 3,783  & 38,146  & 41,929  \\ 
        \bottomrule
    \end{tabular}
    \caption{APIGraph Dataset}
    \label{tab:apigraph}
    \end{minipage}
    \hfill
    \begin{minipage}{0.49\textwidth} % Adjust width as needed
        \centering
        \begin{tabular}{c|c|c|c}
        \bottomrule
        \textbf{Year} & \multicolumn{1}{c|}{\begin{tabular}[c|]{@{}c@{}}\textbf{Malicious}\\ \textbf{Apps}\end{tabular}} & \multicolumn{1}{c|}{\begin{tabular}[c]{@{}c@{}}\textbf{Benign} \\ \textbf{Apps}\end{tabular}} & \textbf{Total} \\ 
        \hline
        2019 & 4,542  & 40,947  & 45,489  \\ 
        2020 & 3,982  & 34,921  & 38,904  \\ 
        2021 & 1,676  & 13,985  & 15,662  \\ 
        \bottomrule
    \end{tabular}
    \caption{AndroZoo Dataset}
    \label{tab:androzoo}
  \end{minipage}
  \vspace{-.2in}
\end{table*}


\textbf{APIGraph Dataset.} Collected by Chen et
al.~\cite{Chen:USENIX23} using the app hash list provided by
APIGraph~\cite{Zhang:CCS20}, this dataset spans seven years of Android
apps from 2012 to 2018. The apps are ordered based on their appearance
timestamps in VirusTotal~\cite{VirusTotal}, addressing both spatial
and temporal biases~\cite{Pendlebury:USENIXSecurity19}. The dataset is
evenly distributed across the years, with each year consisting of
approximately 90\% benign apps and 10\% malicious apps. The malicious
apps are sourced from VirusTotal~\cite{VirusTotal},
VirusShare~\cite{VirusShare}, and the AMD dataset~\cite{AMD:dimva17},
while the benign apps are obtained from
AndroZoo~\cite{Allix:MSR16}. Table~\ref{tab:apigraph} provides a
detailed breakdown of this dataset.

\textbf{AndroZoo Dataset.} Collected by Chen et
al.~\cite{Chen:USENIX23} from AndroZoo~\cite{Allix:MSR16}, this dataset includes
Android apps from 2019 to 2021.The apps are also ordered based on their appearance
timestamps. Malicious apps were randomly selected
based on VirusTotal reports, where at least 15 antivirus engines
flagged them as malware. Benign apps were randomly chosen from those
with 0 positive detections by any antivirus engine. The dataset
maintains a similar malware/benign app ratio, with 90\%
benign apps and 10\% malicious apps each
year. Table~\ref{tab:androzoo} provides a detailed breakdown of the
collected data for each year.


\subsection{Additional Benchmarking}

 % APIGraph & AndroZoo

\begin{table*}[htbp]
  \centering
    \begin{tabular}{c|c|c|ccc|ccc}
      \toprule[0.5pt]
      \multirow{3}{*}{\begin{tabular}[c]{@{}c@{}}\textbf{Monthly}\\ \textbf{Sample}\\ \textbf{Budget}\end{tabular}} & \multirow{3}{*}{\begin{tabular}[c]{@{}c@{}}\textbf{Model}\\ \textbf{Architecture}\end{tabular}} & \multirow{3}{*}{\begin{tabular}[c]{@{}c@{}}\textbf{Sample}\\ \textbf{Selector}\end{tabular}} & \multicolumn{3}{c|}{\multirow{2}{*}{\begin{tabular}[c]{@{}c@{}}\textbf{APIGraph Dataset}\\ Average Performance (\%)\end{tabular}}} & \multicolumn{3}{c}{\multirow{2}{*}{\begin{tabular}[c]{@{}c@{}}\textbf{AndroZoo Dataset}\\ Average Performance (\%)\end{tabular}}}\\                   & & & \multicolumn{3}{c|}{} & \multicolumn{3}{c}{} \\
      & & & FNR & FPR & F1 & FNR & FPR & F1 \\
   
        \hline\hline

        \multirow{10}{*}{50}                                                                               & Binary MLP & Uncertainty & 23.77 & 0.52 & 83.84 & 53.12 & 0.46 & 59.50 \\
        \cline{2-9}                                                                                       & Multiclass MLP & Uncertainty & 16.10 & 4.64 & 73.77 & 49.86 & 28.52 & 28.65 \\
        \cline{2-9}                                                                                       & \begin{tabular}[c]{@{}c@{}}Multiclass MLP\\ + Binary SVM\end{tabular} & Uncertainty & 38.40 & 1.01 & 71.38 & 73.13 & 2.87 & 34.04 \\
        \cline{2-9}
        & \multirow{2}{*}{Binary SVM} & Uncertainty & \textbf{16.92} & \textbf{0.61} & \textbf{87.72} & \textbf{48.77} & \textbf{0.29} & \textbf{63.42}  \\
        &  & CADE OOD & 36.11 & 12.9 & 71.70 & 62.01 & 0.55 & 50.26 \\
        \cline{2-9}                                                                                       & multiclass SVM & Uncertainty & 35.79 & 0.17 & 87.43 & 65.77 & 0.09 & 46.91 \\
        \cline{2-9}
        & multiclass GBDT & Uncertainty & 31.75 & 0.54 & 77.92 & 50.35 & 0.47 & 61.06 \\
        \cline{2-9}
        & Enc + MLP & Pseudo Loss & \textbf{15.15} & \textbf{0.64} & \textbf{89.39} & \textbf{27.65} & \textbf{0.53} & \textbf{79.92}  \\
        % & \multirow{2}{*}{Enc + MLP} & \multirow{2}{*}{Pseudo Loss} & \textbf{15.15} & \textbf{0.64} & \textbf{89.39} & \textbf{27.65} & \textbf{0.53} & \textbf{79.92}  \\
        % &  &  & \textcolor{green}{($\downarrow$ 1.77)} & \textcolor{green}{($\downarrow$ 0.09)} & \textcolor{green}{($\uparrow$ 1.51)} & \textcolor{green}{($\downarrow$ 21.12)} & \textcolor{red}{($\uparrow$ 0.24)} & \textcolor{green}{($\uparrow$ 16.50)} \\
        \cline{2-9}                                                                                       & \cellcolor{gray!20} RF & \cellcolor{gray!20} Uncertainty & \cellcolor{gray!20} 17.99 & \cellcolor{gray!20} 0.20 & \cellcolor{gray!20} 88.95 & \cellcolor{gray!20} 50.28 & \cellcolor{gray!20} 0.10 & \cellcolor{gray!20} 61.87 \\
        \cline{2-9}                                                                                       &  \cellcolor{gray!20} CatBoost & \cellcolor{gray!20} Uncertainty & \cellcolor{gray!20} \textbf{14.70} & \cellcolor{gray!20} \textbf{0.33} & \cellcolor{gray!20} \textbf{90.38} & \cellcolor{gray!20} \textbf{25.72} & \cellcolor{gray!20} \textbf{0.14} & \cellcolor{gray!20} \textbf{83.81} \\
        \hline\hline

        \multirow{10}{*}{100}                                                                              & Binary MLP & Uncertainty & 20.64 & 0.49 & 86.03 & 46.39 & 0.30 & 65.26 \\
        \cline{2-9}                                                                                       & Multiclass MLP & Uncertainty & 14.77 & 6.44 & 69.91 & 35.34 & 32.64 & 33.72 \\
        \cline{2-9}                                                                                       & \begin{tabular}[c]{@{}c@{}}Multiclass MLP\\ + Binary SVM\end{tabular} & Uncertainty & 30.45 & 1.76 & 74.11 & 73.47 & 3.88 & 31.69 \\
        \cline{2-9}
        & \multirow{2}{*}{Binary SVM} & Uncertainty & \textbf{15.41} & \textbf{0.68} & \textbf{88.38} & \textbf{43.07} & \textbf{0.32} & \textbf{68.33}  \\
        &  & CADE OOD & 23.48 & 0.96 & 82.22 & 58.78 & 0.70 & 52.47 \\
        \cline{2-9}                                                                                       & multiclass SVM & Uncertainty & 23.86 & 0.17 & 82.18 & 54.29 & 0.12 & 58.26 \\
        \cline{2-9}
        & multiclass GBDT & Uncertainty & 27.76 & 0.67 & 80.15 & 48.59 & 0.76 & 62.58 \\
        \cline{2-9}
        & Enc + MLP & Pseudo Loss & \textbf{13.69} & \textbf{0.44} & \textbf{90.42} & \textbf{27.35} & \textbf{0.41} & \textbf{80.07}  \\     
        % & \multirow{2}{*}{Enc + MLP} & \multirow{2}{*}{Pseudo Loss} & \textbf{13.69} & \textbf{0.44} & \textbf{90.42} & \textbf{27.35} & \textbf{0.41} & \textbf{80.07}  \\
        % &  &  & \textcolor{green}{($\downarrow$ 1.72)} & \textcolor{green}{($\downarrow$ 0.24)} & \textcolor{green}{($\uparrow$ 2.04)} & \textcolor{green}{($\downarrow$ 15.72)} & \textcolor{red}{($\uparrow$ 0.09)} & \textcolor{green}{($\uparrow$ 11.74)} \\
        \cline{2-9}                                                                                       & \cellcolor{gray!20} RF & \cellcolor{gray!20} Uncertainty & \cellcolor{gray!20} 15.38 & \cellcolor{gray!20} 0.20 & \cellcolor{gray!20} 90.53 & \cellcolor{gray!20} 48.59 & \cellcolor{gray!20} 0.09 & \cellcolor{gray!20} 63.64 \\
        \cline{2-9}                                                                                       &  \cellcolor{gray!20} CatBoost & \cellcolor{gray!20} Uncertainty & \cellcolor{gray!20} \textbf{11.77} & \cellcolor{gray!20} \textbf{0.26} & \cellcolor{gray!20} \textbf{92.39} & \cellcolor{gray!20} \textbf{26.63} & \cellcolor{gray!20} \textbf{0.13} & \cellcolor{gray!20} \textbf{82.12} \\
        \hline\hline

        \multirow{10}{*}{200}                                                                              & Binary MLP & Uncertainty & 19.71 & 0.39 & 86.97 & 42.57 & 0.34 & 68.47 \\
        \cline{2-9}                                                                                       & Multiclass MLP & Uncertainty & 14.56 & 4.26 & 75.65 & 39.78 & 34.76 & 28.59 \\
        \cline{2-9}                                                                                       & \begin{tabular}[c]{@{}c@{}}Multiclass MLP\\ + Binary SVM\end{tabular} & Uncertainty & 29.46 & 1.98 & 74.09 & 70.32 & 0.93 & 39.51 \\
        \cline{2-9}
        & \multirow{2}{*}{Binary SVM} & Uncertainty & \textbf{14.07} & \textbf{0.86} & \textbf{88.47} & \textbf{40.31} & \textbf{0.37} & \textbf{70.24}  \\
        &  & CADE OOD & 21.68 & 0.67 & 84.50 & 51.32 & 0.78 & 59.11 \\
        \cline{2-9}                                                                                       & multiclass SVM & Uncertainty & 21.19 & 0.21 & 86.90 & 44.77 & 0.13 & 66.55 \\
        \cline{2-9}
        & multiclass GBDT & Uncertainty & 24.71 & 0.56 & 82.71 & 42.97 & 0.80 & 67.28 \\
        \cline{2-9}
        & Enc + MLP & Pseudo Loss & \textbf{9.42} & \textbf{0.48} & \textbf{92.72} & \textbf{27.67} & \textbf{0.39} & \textbf{80.51}  \\
        % & \multirow{2}{*}{Enc + MLP} & \multirow{2}{*}{Pseudo Loss} & \textbf{9.42} & \textbf{0.48} & \textbf{92.72} & \textbf{27.67} & \textbf{0.39} & \textbf{80.51}  \\
        % &  &  & \textcolor{green}{($\downarrow$ 4.65)} & \textcolor{green}{($\downarrow$ 0.38)} & \textcolor{green}{($\uparrow$ 4.25)} & \textcolor{green}{($\downarrow$ 12.64)} & \textcolor{red}{($\uparrow$ 0.02)} & \textcolor{green}{($\uparrow$ 10.27)} \\
        \cline{2-9}                                                                                       & \cellcolor{gray!20} RF & \cellcolor{gray!20} Uncertainty & \cellcolor{gray!20} 13.44 & \cellcolor{gray!20} 0.19 & \cellcolor{gray!20} 91.76 & \cellcolor{gray!20} 49.45 & \cellcolor{gray!20} 0.90 & \cellcolor{gray!20} 62.74 \\
        \cline{2-9}                                                                                       &  \cellcolor{gray!20} CatBoost & \cellcolor{gray!20} Uncertainty & \cellcolor{gray!20} \textbf{10.40} & \cellcolor{gray!20} \textbf{0.24} & \cellcolor{gray!20} \textbf{93.27} & \cellcolor{gray!20} \textbf{22.19} & \cellcolor{gray!20} \textbf{0.13} & \cellcolor{gray!20} \textbf{86.06} \\
        \hline\hline

        \multirow{10}{*}{400}                                                                              & Binary MLP & Uncertainty & \textbf{16.04} & \textbf{0.40} & \textbf{89.25} & 36.25 & 0.34 & 73.70 \\
        \cline{2-9}                                                                                       & Multiclass MLP & Uncertainty & 15.07 & 4.15 & 75.94 & 34.48 & 24.44 & 38.34 \\
        \cline{2-9}                                                                                       & \begin{tabular}[c]{@{}c@{}}Multiclass MLP\\ + Binary SVM\end{tabular} & Uncertainty & 28.85 & 1.68 & 75.69 & 73.94 & 1.92 & 33.74 \\
        \cline{2-9}
        & \multirow{2}{*}{Binary SVM} & Uncertainty & 12.86 & 0.90 & 89.02 & 34.73 & 0.43 & 74.12  \\
        &  & CADE OOD & 20.61 & 0.59 & 85.52 & 49.98 & 0.94 & 59.53 \\
        \cline{2-9}                                                                                       & multiclass SVM & Uncertainty & 17.87 & 0.24 & 88.88 & 40.99 & 0.14 & 69.61 \\
        \cline{2-9}
        & multiclass GBDT & Uncertainty & 20.16 & 0.46 & 86.24 & \textbf{33.62} & \textbf{0.38} & \textbf{76.82} \\
        \cline{2-9}
        & Enc + MLP & Pseudo Loss & \textbf{7.84} & \textbf{0.50} & \textbf{93.50} & \textbf{21.49} & \textbf{0.31} & \textbf{85.81}  \\
        % & \multirow{2}{*}{Enc + MLP} & \multirow{2}{*}{Pseudo Loss} & \textbf{7.84} & \textbf{0.50} & \textbf{93.50} & \textbf{21.49} & \textbf{0.31} & \textbf{85.81}  \\
        % &  &  & \textcolor{green}{($\downarrow$ 8.20)} & \textcolor{red}{($\uparrow$ 0.10)} & \textcolor{green}{($\uparrow$ 4.25)} & \textcolor{green}{($\downarrow$ 12.13)} & \textcolor{green}{($\downarrow$ 0.07)} & \textcolor{green}{($\uparrow$ 8.99)} \\
        \cline{2-9}                                                                                       & \cellcolor{gray!20} RF & \cellcolor{gray!20} Uncertainty & \cellcolor{gray!20} 11.06 & \cellcolor{gray!20} 0.20 & \cellcolor{gray!20} 93.13 & \cellcolor{gray!20} 47.79 & \cellcolor{gray!20} 0.90 & \cellcolor{gray!20} 64.29 \\
        \cline{2-9}                                                                                       &  \cellcolor{gray!20} CatBoost & \cellcolor{gray!20} Uncertainty & \cellcolor{gray!20} \textbf{9.41} & \cellcolor{gray!20} \textbf{0.23} & \cellcolor{gray!20} \textbf{93.87} & \cellcolor{gray!20} 26.60 & \cellcolor{gray!20} 0.10 & \cellcolor{gray!20} 82.16 \\
        \hline
    \end{tabular}
    \vspace{10pt}
    \caption{Performance comparison of different models on APIGraph and AndroZoo datasets.  The unshaded results are from the original paper, while the  shaded results are obtained as part of our benchmarking.}
    \label{tab:chen_dataset_comparison}
\end{table*}


Enc + MLP was evaluated on the two datasets and compared with multiple baseline
models. % This is one of the most recent studies applying ML to Android
% malware detection.
The results showed that it outperforms all
baseline models, with a particularly significant improvement on the
AndroZoo dataset. The authors have made all research
artifacts publicly available, including the datasets used in the
experiments and the model-related code. This accessibility allows us
to integrate additional testing models into their framework
seamlessly, enabling the evaluation of more ML models using the
same input and evaluation metrics.

We selected additional ML models from the ones
mentioned in Section~\ref{sec:alternative_ml}: NB, KNN, RF, and CatBoost
(excluding SVM, as it was already included as a baseline in the
original work), to expand the set of baseline models. Table~\ref{tab:chen_dataset_comparison} presents
the performance of all tested models, with shaded rows indicating
models not included in the original work. If an additional model did
not outperform the original baseline models, we omitted its results
from the table. As a result, Table~\ref{tab:chen_dataset_comparison} only includes RF and CatBoost as
additional baseline models. The results show that RF achieves
performance comparable to the ‘Enc+MLP’ model on the APIGraph dataset
but performs poorly on the AndroZoo dataset. However, CatBoost
outperforms ‘Enc+MLP’ on both datasets across different labeling
budgets, except for the AndroZoo dataset when the labeling budget is
set to 400.


%%% Local Variables:
%%% mode: latex
%%% TeX-master: "../main"
%%% End:


\section{Benchmarking for CapsGNN}

\textbf{Capsule Graph Neural Networks.}  The Capsule Graph Neural
Network (CapsGNN) approach proposed by
Zhang~et~al.~\cite{Zhang:ICLR19} represents an attractive approach to
explore in the context of Android malware detection, since it captures
information at graph level and enhances graph embeddings obtained with
Graph Neural Networks (GNN)~\cite{Kipf:LCLR17} by adopting the Capsule
Neural Network (CapsNet)~\cite{Sabour:arXiv17} ideas. CapsNet helps
capture different aspects of the graph, corresponding to different
properties, in the final embeddings.

The Capsule Neural Networks (CapsNet) architecture was introduced by
Sabour~et~al.~\cite{Sabour:arXiv17} for image feature extraction,
although the core concept of a capsule in neural networks was created
by Hinton et al.~\cite{Hinton:ICANN2011}. The CapsNet's design is
based on CNNs, but in addition to detecting features as a CNN does,
CapsNet also aims to encode instantiation parameters (such as
position, orientation, texture) as part of the detected
features. Thus, rather than representing the features using scalar
values as in CNNs, the CapsNet uses capsules, which represent a group
of neurons, to encode features as vectors.  The length of a capsule
determines the probability that the corresponding feature is
present. The direction of the capsules reflects the instantitation
parameters of the feature.

Rather than using a pooling layer to transmit the feature information
between layers, CapsNet uses a dynamic routing mechanism to find the
best connections between the current and next layer capsules based on
agreement. The key innovation of routing by agreement is the
capability to capture the spatial relationships between parts and
their whole. In our case, the CapsNet is meant to encode precise
semantic information from the inter-procedural control flow
graphs (ICFGs) of an app.

As GCN, CapsGNN takes as input a graph data structure, while also
employing capsules (vectors) instead of scalar values to represent
different instantiation parameters of the features. Thus, CapsGNN is a
promising DL candidate for Android malware detection, as it can work
directly with the ICFG and capture different aspects of the
graphs. Figure~\ref{fig:capsgnn_view} illustrates the CappsGNN
architecture implemented in our Android malware detection
system. CapsGNN extracts a primary capsule for each node in an input
graph. As opposed to GCN, which extracts node embeddings from the last
layer of the network, CapsGNN extracts node embeddings from different
layers and represents them as primary capsules (more precisely, the
primary capsules are obtained by summing up node embeddings from
different layers). As graphs in other application domains, the ICFGs
generated from Android apps are diverse, with graph sizes varying
widely. As mentioned above, the number of primary capsules depends on
the number of nodes in a graph, as each capsule corresponds to a
node. Furthermore, the number of graph and class capsules depends on
the number of primary capsules. Thus, the node capsules need to be
scaled to ensure that the graph capsules from different graphs have
the same dimension and are comparable, despite differences in graph
sizes. The CapsGNN framework designed by
Zhang~et~al.~\cite{Zhang:ICLR19}, which we use in this work,
implements an attention module between primary capsules and graph
capsules. The attention module ensures that: 1) the model identifies
the most relevant parts of the graph, and 2) the node capsules are
scaled to the same dimension, so that they subsequently lead to
comparable graph and class embeddings.

 \begin{figure}[t]
   \centering
   \includegraphics[width=\linewidth]{figures/capsgnn.pdf}
   \caption{CapsGNN Architecture in Android Malware Detection. \(\textbf{N}\) is the number of nodes in one graph, \(\textbf{C}\) is the number of node attribute channels, \(\textit{d}\) is the node embedding dimension.}
   \label{fig:capsgnn_view}
   \vspace{-.2in}
\end{figure}


\subsection{Benchmarking for CapsGNN}

CapsGNN is a DL model designed to capture complex patterns within an
app. We include five widely used ML models, as mentioned in
Section~\ref{sec:alternative_ml}, as potential baselines for
comparison with the CapsGNN model.

In addition to traditional ML models, we also experimented with
ExcelFormer, a state-of-the-art neural network proposed by Chen et
al.~\cite{excelformer:kdd24}, which outperforms Gradient Boosted
Decision Trees (GBDTs) and existing DL models for tabular data. It
addresses key challenges in deep tabular learning through three key
innovations: (1) a semi-permeable attention module that reduces the
influence of less informative features, (2) data augmentation
techniques specifically designed for tabular data, and (3) an
attentive feedforward network that enhances model fitting
capability. These design choices make ExcelFormer a highly effective
solution for diverse tabular datasets. Extensive and rigorous
experiments on real-world datasets demonstrate that ExcelFormer
consistently outperforms previous models across various tabular
prediction tasks. Given that an APK's structural representations can
be effectively converted into tabular data, we selected ExcelFormer to
assess its potential for improving Android malware detection
performance.

The Android system is an event-based system, and the event-driven
control flow can involve various method calls based on the app's
components' life cycles. Independent method-level control flow graphs
(CFGs) or API call sequences could not capture the true invocation
order of API calls. In contrast, a ICFG provides a more accurate
representation of the actual execution sequence of
these. operations. Therefore, we utilize ICFGs as the input
representation for our CapsGNN model.

Traditional ML models usually need a predefined feature set to
represent an application, and the performance of the classifier highly
depends on the features. Section~\ref{sec:related} briefly explains
some prior works' approaches to feature selection. The most common
features used in prior works are API calls, app permissions, and a few
other pieces of information from raw app code. The feature sources are
similar but different approaches collect and organize the features in
different ways. Among prior works based on traidtional ML,
DREBIN~\cite{Arp:NDSS14} has shown high detection preformance on
datasets with realistic size and malware to benign app ratio. Recent
work~\cite{Daoudi:TOPS22} shows that DREBIN's features are still
robust for apps as late as 2019. DREBIN extracts static features such
as components, intents, API calls, permissions, and network addresses
totaling 500,000 features. Roy et al.~\cite{Roy:ACSAC15} achieved
similar performance as DREBIN by choosing only 471 static features
from the original DREBIN's feature set. To balance the computational
cost and performance on large-scale datasets, we use the same features
as those in Roy~et~al.'s work~\cite{Roy:ACSAC15}, which consist of 471
features including permissions, intent actions, discriminative APIs,
obfuscation signatures, and native code signatures. To support future
research, the complete feature list will release as open-source with
other related code following the acceptance of this work.

\subsection{Dataset for CapsGNN}

\begin{figure*}[t]
     \centering
     \begin{subfigure}[b]{0.38\textwidth}
         \centering
         \includegraphics[width=\textwidth]{figures/window_detail.png}
         \caption{Breakdown of a 6-batch window}
         \label{fig:window_detail}
     \end{subfigure}
     \hfill
     \begin{subfigure}[b]{0.49\textwidth}
         \centering
         \includegraphics[width=\textwidth]{figures/sliding_window.pdf}
         \caption{6-batch sliding windows  (with 4 training batches)}
         \label{fig:sliding_window_overview}
     \end{subfigure}
    \hfill
     \begin{subfigure}[b]{0.49\textwidth}
         \centering
         \includegraphics[width=\textwidth]{figures/batch_5.pdf}
         \caption{7-batch sliding windows  (with 5 training batches)}
         \label{fig:5_batches}
     \end{subfigure}
     \hfill
     \begin{subfigure}[b]{0.49\textwidth}
         \centering
         \includegraphics[width=\textwidth]{figures/batch_6.pdf}
         \caption{8-batch sliding windows  (with 6 training batches)}
         \label{fig:6_batches}
     \end{subfigure}
        \caption{Sliding Windows}
        \label{fig:Sliding_windows}
 \end{figure*}


The dataset plays a crucial role in ML-based malware detection
research. To ensure meaningful evaluation, the dataset size needs to
reflect the scale of Android apps found in real-world markets. A key
challenge in existing datasets is sampling
bias~\cite{Daniel:USENIXSecurity22}, which can arise when benign and
malicious apps are collected from different sources. To mitigate this
issue, data should be gathered from a single market, ensuring
consistency in distribution and characteristics.  Furthermore, the
organization of training and testing subsets should align with
real-world deployment scenarios. Specifically, training should be
performed on earlier data, while testing should be conducted on later
data—a methodology supported by prior
research~\cite{Pendlebury:USENIXSecurity19}. Additionally, the testing
subset must accurately represent the real-world ratio of benign to
malicious apps. Some industry
experts~\cite{Pendlebury:USENIXSecurity19} suggest that the proportion
of malware in app markets is relatively low, approximately 6\%.  Since
class imbalance can significantly impact classifier performance,
particularly in terms of precision and recall, it is crucial to
construct a dataset that mirrors these real-world distributions.  By
systematically addressing these concerns, we aim to create a more
representative and reliable dataset for Android malware detection
research.

To assess the impact of app evolution on model performance, we adopt a
\emph{sliding window} approach for constructing training, validation,
and test datasets, as illustrated in
Figure~\ref{fig:Sliding_windows}. Specifically, we first partition the
data into batches based on app release dates, with each batch
containing 5,000 apps—comprising 300 malicious apps and 4,700 benign
apps within the same time frame.

We subsequently form windows of consecutive batches, with the first
few batches used for training, the batch before the last for
validation and the last batch for testing, as shown in
Figure~\ref{fig:window_detail} for a 6-batch window. In each
subsequent experiment, we slide the window forward by one batch to
form a new dataset for training and evaluation.  Our objective is to
maximize data availability for both training and testing. A larger
training set contributes to more robust models, while an extensive
test set ensures reliable performance estimation. Each batch is
designed to include 5,000 apps, which we believe is sufficient to
provide accurate performance evaluations. To determine the optimal
number of training batches for model stability, we experiment with
windows containing 4, 5, and 6 training batches, as shown in
Figure~\ref{fig:sliding_window_overview}, Figure~\ref{fig:5_batches}
and Figure~\ref{fig:6_batches}, respectively.

As illustrated in Figure\ref{fig:sliding_window_overview}, our dataset
configuration results in 22 overlapping 6-batch windows, with each
window consisting of 28,200 benign apps and 1,800 malicious
apps—except for the final window, where the availability of malicious
apps from late 2021 was limited. Our data partitioning strategy
ensures that all experiments are conducted on subsets of consistent
size and with a fixed malicious-to-benign ratio. Additionally, this
approach enables the observation of temporal trends in Android malware
detection.

\begin{figure}[tbp]
  \centering
  \includegraphics[width=.95\linewidth]{figures/model_performance.pdf}
  \caption{Top-performing models among all tested models}
  \label{fig:top_ml_models}
  \vspace{-.2in}
\end{figure}

We will publicly release this dataset upon the publication of the paper.

\subsection{Benchmarking Results}

Figure~\ref{fig:top_ml_models} illustrates the top-performing ML models when using a window with four training batches. It can be observed that all models exhibit very similar performance. RF, CatBoost, and KNN show nearly identical performance, often overlapping on the graph, while SVM performs slightly lower than the other three models across most windows. 

\begin{figure*}[tbp]
    \centering
    \begin{subfigure}{0.47\textwidth}
        \centering
        \includegraphics[width=\textwidth]{figures/diff_train_rf.pdf}
        \caption{Random forest performance over the sliding windows. The training datasize in a window is varied: 4-batches, 5-batches, and 6-batches.}
        \label{fig:diff_train_rf}
    \end{subfigure}
    \hspace{3mm}
    \begin{subfigure}{0.47\textwidth}
        \centering
        \includegraphics[width=\textwidth]{figures/diff_train_capsgnn.pdf}
        \caption{CapsGNN performance over the sliding windows. The training datasize in a window is varied: 4-batches, 5-batches, and 6-batches.}
        \label{fig:diff_train_capsgnn}
    \end{subfigure}
    \caption{Main caption for both figures}
    \label{fig:main_figure}
\end{figure*}


% \begin{table}[htbp]
%   \centering
%   \begin{tabular}{l*{2}c}
%     \toprule[1pt]
%     $_{Size}$ \space \textbackslash \space $^{Model}$
%     & \begin{tabular}{@{}c@{}}RandomForest \\ (seconds) \end{tabular}
%     &  \begin{tabular}{@{}c@{}}CapsGNN \\ (hours) \end{tabular} \\
%     \midrule
%     4 batches & 55 & 13 \\
%     5 batches & 75 & 17 \\
%     6 batches & 89 & 20 \\
%     \bottomrule[1pt]
%   \end{tabular}
%   \vspace{10pt}
%   \caption{Model Running Time with Different Training Data Size}
%   \label{tab:model_run_time}
% \end{table}


% \begin{figure}
%   \centering
%   \includegraphics[width=.9\linewidth]{figures/all_model_best_performance.pdf}
%   \caption{RF, CatBoost, ExcelFormer and CapsGNN Performance Comparison: Training data consists of 6-batches per window.}
%   \label{fig:evaluation_result}
%   \vspace{-4mm}
% \end{figure}

\begin{figure*}[t]
    \centering
    \begin{minipage}{0.5\textwidth}
      \centering
        \includegraphics[width=.9\linewidth]{figures/all_model_best_performance.pdf}
        \caption{RF, CatBoost, ExcelFormer and CapsGNN Performance Comparison: Training data consists of 6-batches per window.}
        \label{fig:evaluation_result}
    \end{minipage}%
    \hfill
    \begin{minipage}{0.45\textwidth}
      \centering
        \begin{tabular}{l*{2}c}
         \toprule[1pt]
         $_{Size}$ \space \textbackslash \space $^{Model}$
         & \begin{tabular}{@{}c@{}}RandomForest \\ (seconds) \end{tabular}
         &  \begin{tabular}{@{}c@{}}CapsGNN \\ (hours) \end{tabular} \\
         \midrule
         4 batches & 55 & 13 \\
         5 batches & 75 & 17 \\
         6 batches & 89 & 20 \\
         \bottomrule[1pt]
         \end{tabular}
         \vspace{10pt}
        \captionof{table}{Model Running Time with Different Training Data Size}
        \label{tab:model_run_time}
      \end{minipage}
      \vspace{-.2in}
\end{figure*}


\textbf{How much data is needed for model training?} 
To understand how many training batches need to be included in an
experiment window to obtain robust models, we explored windows with
different number of training batches as illustrated in
Figure~\ref{fig:sliding_window_overview}, Figure~\ref{fig:5_batches}
and Figure~\ref{fig:6_batches} , which show windows with 4, 5 and 6
training batches, respectively.

As can be seen in Figure~\ref{fig:diff_train_rf} and
Figure~\ref{fig:diff_train_capsgnn} for RF and CapsGNN, a larger
training dataset does not always result in increased model
performance.

For the traditional ML model, once the size of the training data
reaches a certain level, the performance of the models does not
improve significantly even when we further increase the size of the
training data. In particular, the models trained on 4 batches of data
have similar performance as the models trained on 6 batches of
data. This can be seen for most windows in
Figure~\ref{fig:diff_train_rf}.  For CapsGNN, as
Figure~\ref{fig:diff_train_capsgnn} shows, the performance of the
model improves when training data size in a window is increased from 4
batches to 6 batches (this is expected given that the CapsGNN model
has a large number of parameters that need to be learned). However,
the time and resources required to train the model also prohibitively
increase (see Table~\ref{tab:model_run_time} for time requirements of
the two models), which made it impossible for us to experiment with
even larger training data for CapsGNN.  In contrast, the RF model
needs less than two minutes to train regardless of the size of the
training subset. Given these results, the remaining experiments are
performed using a window size of 8 batches (with 6 batches being used
for training).

The results in Figure~\ref{fig:evaluation_result} show that the
CapsGNN model has similar performance to the other models for windows
from the start of 2018 until the middle of 2019 (i.e., from window 3
through window 14). However, CapsGNN does outperform baseline models
in most test batches past late 2019 (in particular, windows 15, 16,
17, and 18).  We observe that CapsGNN model has more robust
performance than baseline models especially when data evolves (or
undergoes significant changes) past late 2019.  This is not a
surprising result given that the CapsGNN's input includes much richer
program semantics information in the form of ICFG graphs, together
with features not present in the ICFG graph itself, including
permissions, intent actions and obfuscated/native code
signatures. Moreover, CapsGNN dynamically identifies predictive
features in the input as part of the learning process, making it
easier to keep up with changes in the data. As opposed to that, the
input of the RF model includes only manually designed static features
that do not change with changes in the data.  It is surprising to see
that, despite the strictly richer CapsGNN's input and also the
significantly more computational resources required by CapsGNN,
CapsGNN fails to produce significant performance improvements over the
RF model when the data is relatively stable.

%%% Local Variables:
%%% mode: latex
%%% TeX-master: "../main"
%%% End:


\section{Conclusion}
In this work, we propose a simple yet effective approach, called SMILE, for graph few-shot learning with fewer tasks. Specifically, we introduce a novel dual-level mixup strategy, including within-task and across-task mixup, for enriching the diversity of nodes within each task and the diversity of tasks. Also, we incorporate the degree-based prior information to learn expressive node embeddings. Theoretically, we prove that SMILE effectively enhances the model's generalization performance. Empirically, we conduct extensive experiments on multiple benchmarks and the results suggest that SMILE significantly outperforms other baselines, including both in-domain and cross-domain few-shot settings.


% references section

% can use a bibliography generated by BibTeX as a .bbl file
% BibTeX documentation can be easily obtained at:
% http://mirror.ctan.org/biblio/bibtex/contrib/doc/
% The IEEEtran BibTeX style support page is at:
% http://www.michaelshell.org/tex/ieeetran/bibtex/
%\bibliographystyle{IEEEtran}
% argument is your BibTeX string definitions and bibliography database(s)
%\bibliography{IEEEabrv,../bib/paper}
%
% <OR> manually copy in the resultant .bbl file
% set second argument of \begin to the number of references
% (used to reserve space for the reference number labels box)

\bibliographystyle{IEEEtran}
\bibliography{android}

% conference papers do not normally have an appendix
\subsection{Lloyd-Max Algorithm}
\label{subsec:Lloyd-Max}
For a given quantization bitwidth $B$ and an operand $\bm{X}$, the Lloyd-Max algorithm finds $2^B$ quantization levels $\{\hat{x}_i\}_{i=1}^{2^B}$ such that quantizing $\bm{X}$ by rounding each scalar in $\bm{X}$ to the nearest quantization level minimizes the quantization MSE. 

The algorithm starts with an initial guess of quantization levels and then iteratively computes quantization thresholds $\{\tau_i\}_{i=1}^{2^B-1}$ and updates quantization levels $\{\hat{x}_i\}_{i=1}^{2^B}$. Specifically, at iteration $n$, thresholds are set to the midpoints of the previous iteration's levels:
\begin{align*}
    \tau_i^{(n)}=\frac{\hat{x}_i^{(n-1)}+\hat{x}_{i+1}^{(n-1)}}2 \text{ for } i=1\ldots 2^B-1
\end{align*}
Subsequently, the quantization levels are re-computed as conditional means of the data regions defined by the new thresholds:
\begin{align*}
    \hat{x}_i^{(n)}=\mathbb{E}\left[ \bm{X} \big| \bm{X}\in [\tau_{i-1}^{(n)},\tau_i^{(n)}] \right] \text{ for } i=1\ldots 2^B
\end{align*}
where to satisfy boundary conditions we have $\tau_0=-\infty$ and $\tau_{2^B}=\infty$. The algorithm iterates the above steps until convergence.

Figure \ref{fig:lm_quant} compares the quantization levels of a $7$-bit floating point (E3M3) quantizer (left) to a $7$-bit Lloyd-Max quantizer (right) when quantizing a layer of weights from the GPT3-126M model at a per-tensor granularity. As shown, the Lloyd-Max quantizer achieves substantially lower quantization MSE. Further, Table \ref{tab:FP7_vs_LM7} shows the superior perplexity achieved by Lloyd-Max quantizers for bitwidths of $7$, $6$ and $5$. The difference between the quantizers is clear at 5 bits, where per-tensor FP quantization incurs a drastic and unacceptable increase in perplexity, while Lloyd-Max quantization incurs a much smaller increase. Nevertheless, we note that even the optimal Lloyd-Max quantizer incurs a notable ($\sim 1.5$) increase in perplexity due to the coarse granularity of quantization. 

\begin{figure}[h]
  \centering
  \includegraphics[width=0.7\linewidth]{sections/figures/LM7_FP7.pdf}
  \caption{\small Quantization levels and the corresponding quantization MSE of Floating Point (left) vs Lloyd-Max (right) Quantizers for a layer of weights in the GPT3-126M model.}
  \label{fig:lm_quant}
\end{figure}

\begin{table}[h]\scriptsize
\begin{center}
\caption{\label{tab:FP7_vs_LM7} \small Comparing perplexity (lower is better) achieved by floating point quantizers and Lloyd-Max quantizers on a GPT3-126M model for the Wikitext-103 dataset.}
\begin{tabular}{c|cc|c}
\hline
 \multirow{2}{*}{\textbf{Bitwidth}} & \multicolumn{2}{|c|}{\textbf{Floating-Point Quantizer}} & \textbf{Lloyd-Max Quantizer} \\
 & Best Format & Wikitext-103 Perplexity & Wikitext-103 Perplexity \\
\hline
7 & E3M3 & 18.32 & 18.27 \\
6 & E3M2 & 19.07 & 18.51 \\
5 & E4M0 & 43.89 & 19.71 \\
\hline
\end{tabular}
\end{center}
\end{table}

\subsection{Proof of Local Optimality of LO-BCQ}
\label{subsec:lobcq_opt_proof}
For a given block $\bm{b}_j$, the quantization MSE during LO-BCQ can be empirically evaluated as $\frac{1}{L_b}\lVert \bm{b}_j- \bm{\hat{b}}_j\rVert^2_2$ where $\bm{\hat{b}}_j$ is computed from equation (\ref{eq:clustered_quantization_definition}) as $C_{f(\bm{b}_j)}(\bm{b}_j)$. Further, for a given block cluster $\mathcal{B}_i$, we compute the quantization MSE as $\frac{1}{|\mathcal{B}_{i}|}\sum_{\bm{b} \in \mathcal{B}_{i}} \frac{1}{L_b}\lVert \bm{b}- C_i^{(n)}(\bm{b})\rVert^2_2$. Therefore, at the end of iteration $n$, we evaluate the overall quantization MSE $J^{(n)}$ for a given operand $\bm{X}$ composed of $N_c$ block clusters as:
\begin{align*}
    \label{eq:mse_iter_n}
    J^{(n)} = \frac{1}{N_c} \sum_{i=1}^{N_c} \frac{1}{|\mathcal{B}_{i}^{(n)}|}\sum_{\bm{v} \in \mathcal{B}_{i}^{(n)}} \frac{1}{L_b}\lVert \bm{b}- B_i^{(n)}(\bm{b})\rVert^2_2
\end{align*}

At the end of iteration $n$, the codebooks are updated from $\mathcal{C}^{(n-1)}$ to $\mathcal{C}^{(n)}$. However, the mapping of a given vector $\bm{b}_j$ to quantizers $\mathcal{C}^{(n)}$ remains as  $f^{(n)}(\bm{b}_j)$. At the next iteration, during the vector clustering step, $f^{(n+1)}(\bm{b}_j)$ finds new mapping of $\bm{b}_j$ to updated codebooks $\mathcal{C}^{(n)}$ such that the quantization MSE over the candidate codebooks is minimized. Therefore, we obtain the following result for $\bm{b}_j$:
\begin{align*}
\frac{1}{L_b}\lVert \bm{b}_j - C_{f^{(n+1)}(\bm{b}_j)}^{(n)}(\bm{b}_j)\rVert^2_2 \le \frac{1}{L_b}\lVert \bm{b}_j - C_{f^{(n)}(\bm{b}_j)}^{(n)}(\bm{b}_j)\rVert^2_2
\end{align*}

That is, quantizing $\bm{b}_j$ at the end of the block clustering step of iteration $n+1$ results in lower quantization MSE compared to quantizing at the end of iteration $n$. Since this is true for all $\bm{b} \in \bm{X}$, we assert the following:
\begin{equation}
\begin{split}
\label{eq:mse_ineq_1}
    \tilde{J}^{(n+1)} &= \frac{1}{N_c} \sum_{i=1}^{N_c} \frac{1}{|\mathcal{B}_{i}^{(n+1)}|}\sum_{\bm{b} \in \mathcal{B}_{i}^{(n+1)}} \frac{1}{L_b}\lVert \bm{b} - C_i^{(n)}(b)\rVert^2_2 \le J^{(n)}
\end{split}
\end{equation}
where $\tilde{J}^{(n+1)}$ is the the quantization MSE after the vector clustering step at iteration $n+1$.

Next, during the codebook update step (\ref{eq:quantizers_update}) at iteration $n+1$, the per-cluster codebooks $\mathcal{C}^{(n)}$ are updated to $\mathcal{C}^{(n+1)}$ by invoking the Lloyd-Max algorithm \citep{Lloyd}. We know that for any given value distribution, the Lloyd-Max algorithm minimizes the quantization MSE. Therefore, for a given vector cluster $\mathcal{B}_i$ we obtain the following result:

\begin{equation}
    \frac{1}{|\mathcal{B}_{i}^{(n+1)}|}\sum_{\bm{b} \in \mathcal{B}_{i}^{(n+1)}} \frac{1}{L_b}\lVert \bm{b}- C_i^{(n+1)}(\bm{b})\rVert^2_2 \le \frac{1}{|\mathcal{B}_{i}^{(n+1)}|}\sum_{\bm{b} \in \mathcal{B}_{i}^{(n+1)}} \frac{1}{L_b}\lVert \bm{b}- C_i^{(n)}(\bm{b})\rVert^2_2
\end{equation}

The above equation states that quantizing the given block cluster $\mathcal{B}_i$ after updating the associated codebook from $C_i^{(n)}$ to $C_i^{(n+1)}$ results in lower quantization MSE. Since this is true for all the block clusters, we derive the following result: 
\begin{equation}
\begin{split}
\label{eq:mse_ineq_2}
     J^{(n+1)} &= \frac{1}{N_c} \sum_{i=1}^{N_c} \frac{1}{|\mathcal{B}_{i}^{(n+1)}|}\sum_{\bm{b} \in \mathcal{B}_{i}^{(n+1)}} \frac{1}{L_b}\lVert \bm{b}- C_i^{(n+1)}(\bm{b})\rVert^2_2  \le \tilde{J}^{(n+1)}   
\end{split}
\end{equation}

Following (\ref{eq:mse_ineq_1}) and (\ref{eq:mse_ineq_2}), we find that the quantization MSE is non-increasing for each iteration, that is, $J^{(1)} \ge J^{(2)} \ge J^{(3)} \ge \ldots \ge J^{(M)}$ where $M$ is the maximum number of iterations. 
%Therefore, we can say that if the algorithm converges, then it must be that it has converged to a local minimum. 
\hfill $\blacksquare$


\begin{figure}
    \begin{center}
    \includegraphics[width=0.5\textwidth]{sections//figures/mse_vs_iter.pdf}
    \end{center}
    \caption{\small NMSE vs iterations during LO-BCQ compared to other block quantization proposals}
    \label{fig:nmse_vs_iter}
\end{figure}

Figure \ref{fig:nmse_vs_iter} shows the empirical convergence of LO-BCQ across several block lengths and number of codebooks. Also, the MSE achieved by LO-BCQ is compared to baselines such as MXFP and VSQ. As shown, LO-BCQ converges to a lower MSE than the baselines. Further, we achieve better convergence for larger number of codebooks ($N_c$) and for a smaller block length ($L_b$), both of which increase the bitwidth of BCQ (see Eq \ref{eq:bitwidth_bcq}).


\subsection{Additional Accuracy Results}
%Table \ref{tab:lobcq_config} lists the various LOBCQ configurations and their corresponding bitwidths.
\begin{table}
\setlength{\tabcolsep}{4.75pt}
\begin{center}
\caption{\label{tab:lobcq_config} Various LO-BCQ configurations and their bitwidths.}
\begin{tabular}{|c||c|c|c|c||c|c||c|} 
\hline
 & \multicolumn{4}{|c||}{$L_b=8$} & \multicolumn{2}{|c||}{$L_b=4$} & $L_b=2$ \\
 \hline
 \backslashbox{$L_A$\kern-1em}{\kern-1em$N_c$} & 2 & 4 & 8 & 16 & 2 & 4 & 2 \\
 \hline
 64 & 4.25 & 4.375 & 4.5 & 4.625 & 4.375 & 4.625 & 4.625\\
 \hline
 32 & 4.375 & 4.5 & 4.625& 4.75 & 4.5 & 4.75 & 4.75 \\
 \hline
 16 & 4.625 & 4.75& 4.875 & 5 & 4.75 & 5 & 5 \\
 \hline
\end{tabular}
\end{center}
\end{table}

%\subsection{Perplexity achieved by various LO-BCQ configurations on Wikitext-103 dataset}

\begin{table} \centering
\begin{tabular}{|c||c|c|c|c||c|c||c|} 
\hline
 $L_b \rightarrow$& \multicolumn{4}{c||}{8} & \multicolumn{2}{c||}{4} & 2\\
 \hline
 \backslashbox{$L_A$\kern-1em}{\kern-1em$N_c$} & 2 & 4 & 8 & 16 & 2 & 4 & 2  \\
 %$N_c \rightarrow$ & 2 & 4 & 8 & 16 & 2 & 4 & 2 \\
 \hline
 \hline
 \multicolumn{8}{c}{GPT3-1.3B (FP32 PPL = 9.98)} \\ 
 \hline
 \hline
 64 & 10.40 & 10.23 & 10.17 & 10.15 &  10.28 & 10.18 & 10.19 \\
 \hline
 32 & 10.25 & 10.20 & 10.15 & 10.12 &  10.23 & 10.17 & 10.17 \\
 \hline
 16 & 10.22 & 10.16 & 10.10 & 10.09 &  10.21 & 10.14 & 10.16 \\
 \hline
  \hline
 \multicolumn{8}{c}{GPT3-8B (FP32 PPL = 7.38)} \\ 
 \hline
 \hline
 64 & 7.61 & 7.52 & 7.48 &  7.47 &  7.55 &  7.49 & 7.50 \\
 \hline
 32 & 7.52 & 7.50 & 7.46 &  7.45 &  7.52 &  7.48 & 7.48  \\
 \hline
 16 & 7.51 & 7.48 & 7.44 &  7.44 &  7.51 &  7.49 & 7.47  \\
 \hline
\end{tabular}
\caption{\label{tab:ppl_gpt3_abalation} Wikitext-103 perplexity across GPT3-1.3B and 8B models.}
\end{table}

\begin{table} \centering
\begin{tabular}{|c||c|c|c|c||} 
\hline
 $L_b \rightarrow$& \multicolumn{4}{c||}{8}\\
 \hline
 \backslashbox{$L_A$\kern-1em}{\kern-1em$N_c$} & 2 & 4 & 8 & 16 \\
 %$N_c \rightarrow$ & 2 & 4 & 8 & 16 & 2 & 4 & 2 \\
 \hline
 \hline
 \multicolumn{5}{|c|}{Llama2-7B (FP32 PPL = 5.06)} \\ 
 \hline
 \hline
 64 & 5.31 & 5.26 & 5.19 & 5.18  \\
 \hline
 32 & 5.23 & 5.25 & 5.18 & 5.15  \\
 \hline
 16 & 5.23 & 5.19 & 5.16 & 5.14  \\
 \hline
 \multicolumn{5}{|c|}{Nemotron4-15B (FP32 PPL = 5.87)} \\ 
 \hline
 \hline
 64  & 6.3 & 6.20 & 6.13 & 6.08  \\
 \hline
 32  & 6.24 & 6.12 & 6.07 & 6.03  \\
 \hline
 16  & 6.12 & 6.14 & 6.04 & 6.02  \\
 \hline
 \multicolumn{5}{|c|}{Nemotron4-340B (FP32 PPL = 3.48)} \\ 
 \hline
 \hline
 64 & 3.67 & 3.62 & 3.60 & 3.59 \\
 \hline
 32 & 3.63 & 3.61 & 3.59 & 3.56 \\
 \hline
 16 & 3.61 & 3.58 & 3.57 & 3.55 \\
 \hline
\end{tabular}
\caption{\label{tab:ppl_llama7B_nemo15B} Wikitext-103 perplexity compared to FP32 baseline in Llama2-7B and Nemotron4-15B, 340B models}
\end{table}

%\subsection{Perplexity achieved by various LO-BCQ configurations on MMLU dataset}


\begin{table} \centering
\begin{tabular}{|c||c|c|c|c||c|c|c|c|} 
\hline
 $L_b \rightarrow$& \multicolumn{4}{c||}{8} & \multicolumn{4}{c||}{8}\\
 \hline
 \backslashbox{$L_A$\kern-1em}{\kern-1em$N_c$} & 2 & 4 & 8 & 16 & 2 & 4 & 8 & 16  \\
 %$N_c \rightarrow$ & 2 & 4 & 8 & 16 & 2 & 4 & 2 \\
 \hline
 \hline
 \multicolumn{5}{|c|}{Llama2-7B (FP32 Accuracy = 45.8\%)} & \multicolumn{4}{|c|}{Llama2-70B (FP32 Accuracy = 69.12\%)} \\ 
 \hline
 \hline
 64 & 43.9 & 43.4 & 43.9 & 44.9 & 68.07 & 68.27 & 68.17 & 68.75 \\
 \hline
 32 & 44.5 & 43.8 & 44.9 & 44.5 & 68.37 & 68.51 & 68.35 & 68.27  \\
 \hline
 16 & 43.9 & 42.7 & 44.9 & 45 & 68.12 & 68.77 & 68.31 & 68.59  \\
 \hline
 \hline
 \multicolumn{5}{|c|}{GPT3-22B (FP32 Accuracy = 38.75\%)} & \multicolumn{4}{|c|}{Nemotron4-15B (FP32 Accuracy = 64.3\%)} \\ 
 \hline
 \hline
 64 & 36.71 & 38.85 & 38.13 & 38.92 & 63.17 & 62.36 & 63.72 & 64.09 \\
 \hline
 32 & 37.95 & 38.69 & 39.45 & 38.34 & 64.05 & 62.30 & 63.8 & 64.33  \\
 \hline
 16 & 38.88 & 38.80 & 38.31 & 38.92 & 63.22 & 63.51 & 63.93 & 64.43  \\
 \hline
\end{tabular}
\caption{\label{tab:mmlu_abalation} Accuracy on MMLU dataset across GPT3-22B, Llama2-7B, 70B and Nemotron4-15B models.}
\end{table}


%\subsection{Perplexity achieved by various LO-BCQ configurations on LM evaluation harness}

\begin{table} \centering
\begin{tabular}{|c||c|c|c|c||c|c|c|c|} 
\hline
 $L_b \rightarrow$& \multicolumn{4}{c||}{8} & \multicolumn{4}{c||}{8}\\
 \hline
 \backslashbox{$L_A$\kern-1em}{\kern-1em$N_c$} & 2 & 4 & 8 & 16 & 2 & 4 & 8 & 16  \\
 %$N_c \rightarrow$ & 2 & 4 & 8 & 16 & 2 & 4 & 2 \\
 \hline
 \hline
 \multicolumn{5}{|c|}{Race (FP32 Accuracy = 37.51\%)} & \multicolumn{4}{|c|}{Boolq (FP32 Accuracy = 64.62\%)} \\ 
 \hline
 \hline
 64 & 36.94 & 37.13 & 36.27 & 37.13 & 63.73 & 62.26 & 63.49 & 63.36 \\
 \hline
 32 & 37.03 & 36.36 & 36.08 & 37.03 & 62.54 & 63.51 & 63.49 & 63.55  \\
 \hline
 16 & 37.03 & 37.03 & 36.46 & 37.03 & 61.1 & 63.79 & 63.58 & 63.33  \\
 \hline
 \hline
 \multicolumn{5}{|c|}{Winogrande (FP32 Accuracy = 58.01\%)} & \multicolumn{4}{|c|}{Piqa (FP32 Accuracy = 74.21\%)} \\ 
 \hline
 \hline
 64 & 58.17 & 57.22 & 57.85 & 58.33 & 73.01 & 73.07 & 73.07 & 72.80 \\
 \hline
 32 & 59.12 & 58.09 & 57.85 & 58.41 & 73.01 & 73.94 & 72.74 & 73.18  \\
 \hline
 16 & 57.93 & 58.88 & 57.93 & 58.56 & 73.94 & 72.80 & 73.01 & 73.94  \\
 \hline
\end{tabular}
\caption{\label{tab:mmlu_abalation} Accuracy on LM evaluation harness tasks on GPT3-1.3B model.}
\end{table}

\begin{table} \centering
\begin{tabular}{|c||c|c|c|c||c|c|c|c|} 
\hline
 $L_b \rightarrow$& \multicolumn{4}{c||}{8} & \multicolumn{4}{c||}{8}\\
 \hline
 \backslashbox{$L_A$\kern-1em}{\kern-1em$N_c$} & 2 & 4 & 8 & 16 & 2 & 4 & 8 & 16  \\
 %$N_c \rightarrow$ & 2 & 4 & 8 & 16 & 2 & 4 & 2 \\
 \hline
 \hline
 \multicolumn{5}{|c|}{Race (FP32 Accuracy = 41.34\%)} & \multicolumn{4}{|c|}{Boolq (FP32 Accuracy = 68.32\%)} \\ 
 \hline
 \hline
 64 & 40.48 & 40.10 & 39.43 & 39.90 & 69.20 & 68.41 & 69.45 & 68.56 \\
 \hline
 32 & 39.52 & 39.52 & 40.77 & 39.62 & 68.32 & 67.43 & 68.17 & 69.30  \\
 \hline
 16 & 39.81 & 39.71 & 39.90 & 40.38 & 68.10 & 66.33 & 69.51 & 69.42  \\
 \hline
 \hline
 \multicolumn{5}{|c|}{Winogrande (FP32 Accuracy = 67.88\%)} & \multicolumn{4}{|c|}{Piqa (FP32 Accuracy = 78.78\%)} \\ 
 \hline
 \hline
 64 & 66.85 & 66.61 & 67.72 & 67.88 & 77.31 & 77.42 & 77.75 & 77.64 \\
 \hline
 32 & 67.25 & 67.72 & 67.72 & 67.00 & 77.31 & 77.04 & 77.80 & 77.37  \\
 \hline
 16 & 68.11 & 68.90 & 67.88 & 67.48 & 77.37 & 78.13 & 78.13 & 77.69  \\
 \hline
\end{tabular}
\caption{\label{tab:mmlu_abalation} Accuracy on LM evaluation harness tasks on GPT3-8B model.}
\end{table}

\begin{table} \centering
\begin{tabular}{|c||c|c|c|c||c|c|c|c|} 
\hline
 $L_b \rightarrow$& \multicolumn{4}{c||}{8} & \multicolumn{4}{c||}{8}\\
 \hline
 \backslashbox{$L_A$\kern-1em}{\kern-1em$N_c$} & 2 & 4 & 8 & 16 & 2 & 4 & 8 & 16  \\
 %$N_c \rightarrow$ & 2 & 4 & 8 & 16 & 2 & 4 & 2 \\
 \hline
 \hline
 \multicolumn{5}{|c|}{Race (FP32 Accuracy = 40.67\%)} & \multicolumn{4}{|c|}{Boolq (FP32 Accuracy = 76.54\%)} \\ 
 \hline
 \hline
 64 & 40.48 & 40.10 & 39.43 & 39.90 & 75.41 & 75.11 & 77.09 & 75.66 \\
 \hline
 32 & 39.52 & 39.52 & 40.77 & 39.62 & 76.02 & 76.02 & 75.96 & 75.35  \\
 \hline
 16 & 39.81 & 39.71 & 39.90 & 40.38 & 75.05 & 73.82 & 75.72 & 76.09  \\
 \hline
 \hline
 \multicolumn{5}{|c|}{Winogrande (FP32 Accuracy = 70.64\%)} & \multicolumn{4}{|c|}{Piqa (FP32 Accuracy = 79.16\%)} \\ 
 \hline
 \hline
 64 & 69.14 & 70.17 & 70.17 & 70.56 & 78.24 & 79.00 & 78.62 & 78.73 \\
 \hline
 32 & 70.96 & 69.69 & 71.27 & 69.30 & 78.56 & 79.49 & 79.16 & 78.89  \\
 \hline
 16 & 71.03 & 69.53 & 69.69 & 70.40 & 78.13 & 79.16 & 79.00 & 79.00  \\
 \hline
\end{tabular}
\caption{\label{tab:mmlu_abalation} Accuracy on LM evaluation harness tasks on GPT3-22B model.}
\end{table}

\begin{table} \centering
\begin{tabular}{|c||c|c|c|c||c|c|c|c|} 
\hline
 $L_b \rightarrow$& \multicolumn{4}{c||}{8} & \multicolumn{4}{c||}{8}\\
 \hline
 \backslashbox{$L_A$\kern-1em}{\kern-1em$N_c$} & 2 & 4 & 8 & 16 & 2 & 4 & 8 & 16  \\
 %$N_c \rightarrow$ & 2 & 4 & 8 & 16 & 2 & 4 & 2 \\
 \hline
 \hline
 \multicolumn{5}{|c|}{Race (FP32 Accuracy = 44.4\%)} & \multicolumn{4}{|c|}{Boolq (FP32 Accuracy = 79.29\%)} \\ 
 \hline
 \hline
 64 & 42.49 & 42.51 & 42.58 & 43.45 & 77.58 & 77.37 & 77.43 & 78.1 \\
 \hline
 32 & 43.35 & 42.49 & 43.64 & 43.73 & 77.86 & 75.32 & 77.28 & 77.86  \\
 \hline
 16 & 44.21 & 44.21 & 43.64 & 42.97 & 78.65 & 77 & 76.94 & 77.98  \\
 \hline
 \hline
 \multicolumn{5}{|c|}{Winogrande (FP32 Accuracy = 69.38\%)} & \multicolumn{4}{|c|}{Piqa (FP32 Accuracy = 78.07\%)} \\ 
 \hline
 \hline
 64 & 68.9 & 68.43 & 69.77 & 68.19 & 77.09 & 76.82 & 77.09 & 77.86 \\
 \hline
 32 & 69.38 & 68.51 & 68.82 & 68.90 & 78.07 & 76.71 & 78.07 & 77.86  \\
 \hline
 16 & 69.53 & 67.09 & 69.38 & 68.90 & 77.37 & 77.8 & 77.91 & 77.69  \\
 \hline
\end{tabular}
\caption{\label{tab:mmlu_abalation} Accuracy on LM evaluation harness tasks on Llama2-7B model.}
\end{table}

\begin{table} \centering
\begin{tabular}{|c||c|c|c|c||c|c|c|c|} 
\hline
 $L_b \rightarrow$& \multicolumn{4}{c||}{8} & \multicolumn{4}{c||}{8}\\
 \hline
 \backslashbox{$L_A$\kern-1em}{\kern-1em$N_c$} & 2 & 4 & 8 & 16 & 2 & 4 & 8 & 16  \\
 %$N_c \rightarrow$ & 2 & 4 & 8 & 16 & 2 & 4 & 2 \\
 \hline
 \hline
 \multicolumn{5}{|c|}{Race (FP32 Accuracy = 48.8\%)} & \multicolumn{4}{|c|}{Boolq (FP32 Accuracy = 85.23\%)} \\ 
 \hline
 \hline
 64 & 49.00 & 49.00 & 49.28 & 48.71 & 82.82 & 84.28 & 84.03 & 84.25 \\
 \hline
 32 & 49.57 & 48.52 & 48.33 & 49.28 & 83.85 & 84.46 & 84.31 & 84.93  \\
 \hline
 16 & 49.85 & 49.09 & 49.28 & 48.99 & 85.11 & 84.46 & 84.61 & 83.94  \\
 \hline
 \hline
 \multicolumn{5}{|c|}{Winogrande (FP32 Accuracy = 79.95\%)} & \multicolumn{4}{|c|}{Piqa (FP32 Accuracy = 81.56\%)} \\ 
 \hline
 \hline
 64 & 78.77 & 78.45 & 78.37 & 79.16 & 81.45 & 80.69 & 81.45 & 81.5 \\
 \hline
 32 & 78.45 & 79.01 & 78.69 & 80.66 & 81.56 & 80.58 & 81.18 & 81.34  \\
 \hline
 16 & 79.95 & 79.56 & 79.79 & 79.72 & 81.28 & 81.66 & 81.28 & 80.96  \\
 \hline
\end{tabular}
\caption{\label{tab:mmlu_abalation} Accuracy on LM evaluation harness tasks on Llama2-70B model.}
\end{table}

%\section{MSE Studies}
%\textcolor{red}{TODO}


\subsection{Number Formats and Quantization Method}
\label{subsec:numFormats_quantMethod}
\subsubsection{Integer Format}
An $n$-bit signed integer (INT) is typically represented with a 2s-complement format \citep{yao2022zeroquant,xiao2023smoothquant,dai2021vsq}, where the most significant bit denotes the sign.

\subsubsection{Floating Point Format}
An $n$-bit signed floating point (FP) number $x$ comprises of a 1-bit sign ($x_{\mathrm{sign}}$), $B_m$-bit mantissa ($x_{\mathrm{mant}}$) and $B_e$-bit exponent ($x_{\mathrm{exp}}$) such that $B_m+B_e=n-1$. The associated constant exponent bias ($E_{\mathrm{bias}}$) is computed as $(2^{{B_e}-1}-1)$. We denote this format as $E_{B_e}M_{B_m}$.  

\subsubsection{Quantization Scheme}
\label{subsec:quant_method}
A quantization scheme dictates how a given unquantized tensor is converted to its quantized representation. We consider FP formats for the purpose of illustration. Given an unquantized tensor $\bm{X}$ and an FP format $E_{B_e}M_{B_m}$, we first, we compute the quantization scale factor $s_X$ that maps the maximum absolute value of $\bm{X}$ to the maximum quantization level of the $E_{B_e}M_{B_m}$ format as follows:
\begin{align}
\label{eq:sf}
    s_X = \frac{\mathrm{max}(|\bm{X}|)}{\mathrm{max}(E_{B_e}M_{B_m})}
\end{align}
In the above equation, $|\cdot|$ denotes the absolute value function.

Next, we scale $\bm{X}$ by $s_X$ and quantize it to $\hat{\bm{X}}$ by rounding it to the nearest quantization level of $E_{B_e}M_{B_m}$ as:

\begin{align}
\label{eq:tensor_quant}
    \hat{\bm{X}} = \text{round-to-nearest}\left(\frac{\bm{X}}{s_X}, E_{B_e}M_{B_m}\right)
\end{align}

We perform dynamic max-scaled quantization \citep{wu2020integer}, where the scale factor $s$ for activations is dynamically computed during runtime.

\subsection{Vector Scaled Quantization}
\begin{wrapfigure}{r}{0.35\linewidth}
  \centering
  \includegraphics[width=\linewidth]{sections/figures/vsquant.jpg}
  \caption{\small Vectorwise decomposition for per-vector scaled quantization (VSQ \citep{dai2021vsq}).}
  \label{fig:vsquant}
\end{wrapfigure}
During VSQ \citep{dai2021vsq}, the operand tensors are decomposed into 1D vectors in a hardware friendly manner as shown in Figure \ref{fig:vsquant}. Since the decomposed tensors are used as operands in matrix multiplications during inference, it is beneficial to perform this decomposition along the reduction dimension of the multiplication. The vectorwise quantization is performed similar to tensorwise quantization described in Equations \ref{eq:sf} and \ref{eq:tensor_quant}, where a scale factor $s_v$ is required for each vector $\bm{v}$ that maps the maximum absolute value of that vector to the maximum quantization level. While smaller vector lengths can lead to larger accuracy gains, the associated memory and computational overheads due to the per-vector scale factors increases. To alleviate these overheads, VSQ \citep{dai2021vsq} proposed a second level quantization of the per-vector scale factors to unsigned integers, while MX \citep{rouhani2023shared} quantizes them to integer powers of 2 (denoted as $2^{INT}$).

\subsubsection{MX Format}
The MX format proposed in \citep{rouhani2023microscaling} introduces the concept of sub-block shifting. For every two scalar elements of $b$-bits each, there is a shared exponent bit. The value of this exponent bit is determined through an empirical analysis that targets minimizing quantization MSE. We note that the FP format $E_{1}M_{b}$ is strictly better than MX from an accuracy perspective since it allocates a dedicated exponent bit to each scalar as opposed to sharing it across two scalars. Therefore, we conservatively bound the accuracy of a $b+2$-bit signed MX format with that of a $E_{1}M_{b}$ format in our comparisons. For instance, we use E1M2 format as a proxy for MX4.

\begin{figure}
    \centering
    \includegraphics[width=1\linewidth]{sections//figures/BlockFormats.pdf}
    \caption{\small Comparing LO-BCQ to MX format.}
    \label{fig:block_formats}
\end{figure}

Figure \ref{fig:block_formats} compares our $4$-bit LO-BCQ block format to MX \citep{rouhani2023microscaling}. As shown, both LO-BCQ and MX decompose a given operand tensor into block arrays and each block array into blocks. Similar to MX, we find that per-block quantization ($L_b < L_A$) leads to better accuracy due to increased flexibility. While MX achieves this through per-block $1$-bit micro-scales, we associate a dedicated codebook to each block through a per-block codebook selector. Further, MX quantizes the per-block array scale-factor to E8M0 format without per-tensor scaling. In contrast during LO-BCQ, we find that per-tensor scaling combined with quantization of per-block array scale-factor to E4M3 format results in superior inference accuracy across models. 
            


% THAT'S all folks
\end{document}
