\begin{abstract}

  Android malware detection has been extensively studied using both traditional machine learning (ML) and deep learning (DL) approaches. While many state-of-the-art detection models, particularly those based on DL, claim superior performance, they often rely on limited comparisons, lacking comprehensive benchmarking against traditional ML models across diverse datasets. This raises concerns about the robustness of DL-based approaches' performance and the potential oversight of simpler, more efficient ML models. In this paper, we conduct a systematic evaluation of Android malware detection models across four datasets: three recently published, publicly available datasets and a large-scale dataset we systematically collected. We implement a range of traditional ML models, including Random Forests (RF) and CatBoost, alongside advanced DL models such as Capsule Graph Neural Networks (CapsGNN), BERT-based models, and ExcelFormer-based models. Our results reveal that while advanced DL models can achieve strong performance, they are often compared against an insufficient number of traditional ML baselines. In many cases, simpler and more computationally efficient ML models achieve comparable or even superior performance. These findings highlight the need for rigorous benchmarking in Android malware detection research. We encourage future studies to conduct more comprehensive benchmarking comparisons between traditional and advanced models to ensure a more accurate assessment of detection capabilities. To facilitate further research, we provide access to our dataset, including app IDs, hash values, and labels.   

\end{abstract}

%%% Local Variables:
%%% mode: latex
%%% TeX-master: "main"
%%% End:
