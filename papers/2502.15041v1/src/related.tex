\section{Related Work}
\label{sec:related}

Machine learning-based Android malware detection systems have been
researched for a decade.  Crowdroid~\cite{Iker:SPSM11} is an earlier
ML framework for Android malware detection.
DroidAPIMiner~\cite{Aafer:SecureComm13} performs static
analysis to extract lightweight features at API level, then uses
different classification models such as k-NN, SVM and Decision Trees
(DT) to detect malware. DREBIN~\cite{Arp:NDSS14} uses similar static analysis techniques
as DroidAPIMiner to generate a feature set that contains components,
permissions, filtered intents, API calls and network addresses, then
uses SVM as a malware classifier. 
Roy et al.~\cite{Roy:ACSAC15} performed experimentation studies to investigate
a number of research questions related to using ML in Android malware
detection.
MaMaDroid~\cite{Mariconti:NDSS2017} abstracts apps' API calls from 340
packages and 11 families of malware, then
applies a Markov chain to form the sequence of abstracted API calls extracted
from call graphs as a feature vector, and finally performs  classification
using Random Forest, k-NN and SVM algorithms. APIChecker~\cite{Gong:EuroSys20}
uses dynamic analysis to collect API features, applies nine machine learning algorithms for Android malware detection and evaluates their performance on the T-Market data with about 500K apps. The results show that RF yields the best overall performance among the selected models.
DroidAPIMiner~\cite{Gascon:AISec13},
DREBIN~\cite{Arp:NDSS14}, DroidMat~\cite{Wu:AsiaJCIS12}, and
MaMaDroid~\cite{Mariconti:NDSS2017} also compared their performance with
other traditional ML classifiers and anti-virus scanners. MaMaDroid uses four different ML algorithms and the detection results obtained with RF have better performance in general.

Recent years have seen an increase in deep learning related Android malware detection
research.  The deep neural network models used include deep belief
networks (DBN)~\cite{Yuan:SIGCOMM14, Su:TrustCom16}, convolutional
neural networks (CNN)~\cite{McLaughlin:CODASPY17, Karbab:DFRWS18,Nadia:MALHat21, Arslan:CCPE22}, pre-trained BERT models~\cite{Sun:esem24},
recurrent neural networks (mainly long short-term memory networks, or
LSTM)~\cite{Vinayakumar:JIFS18, Xiao:MTA19}, and multilayer perceptrons (MLP) with continuous learning schemes~\cite{Chen:USENIX23}, among others.
Droid-Sec~\cite{Yuan:SIGCOMM14} trained and tested a DBN model based on 500 apps (250 malware). Experimental results
showed that the DL model has significant accuracy improvement as
compared to traditional ML models, such as SVM, NB, Logistic
Regression, and MLP.
Deep4MalDroid~\cite{Hou:WIW16} evaluated a Stacked AutoEncoders (SAEs) model based on 3000
apps (50\% of which are malware). Results showed that the proposed
DL model has higher accuracy than the selected ML
models, specifically, SVM, ANN, NB, and DT.
DeepFlow~\cite{Zhu:ISCC17} proposed a DBN model, and
conducted experiments based on 3000 benign apps and 8000 malicious
apps. The comparison between DL and traditional ML models showed that
DBN and SVM outperform other selected ML models in balancing the
recall and precision.
DexRay~\cite{Nadia:MALHat21} converts the raw bytecode of DEX
  files in the APK as one stream vector, then converts the vector to a
  grey-scale image which is provided as input to a CNN
  architecture. DexRay evaluated the performance of the CNN model based on 61,809
  malicious and 134,134 benign apps, then compared its detection
  results with DREBIN's results and other image-based malware detection
  approaches. DexRay yields almost the same performance metrics as DREBIN; however, its performance can be significantly affected if obfuscated apps are absent in the training data but present in the test data. DetectBERT~\cite{Sun:esem24} leverages a pre-trained BERT-like model with correlated Multiple Instance Learning (c-MIL) to process Smali code from APKs, enriching class-level features and aggregating them into an app-level representation for effective Android malware detection. It employs the DexRay dataset as its benchmark baseline for evaluation.
DeepRefiner~\cite{Xu:EuroS&P18} is a two-layer
detection system, with the first detection layer based on MLP, and the
second layer based on an LSTM network. The experimental data include 47,525
benign apps and 62,915 malicious apps. The detection performance,
based on accuracy, TPR and FPR, is approximately 10\% higher than that
of a fine-turned SVM model.  DANdroid~\cite{Millar:CODASPY20} uses the
DREBIN dataset to evaluate its proposed DL discriminative adversarial network,
and the performance is very close to that of DREBIN.
DL-Droid~\cite{Alzaylaee:Computer&security20} generates 420 dynamic
stateful features such as application attributes, actions and
permissions by running apps on real devices, then feeding them into a
multilayer perceptron classifier. It uses weighted F measure (w-FM) as
main evaluation metric and compares it against that of seven selected ML
algorithms. DL-Droid reached the highest w-FM value of 98.82\% (the best
ML model RF reached w-FM value of
97.1\%). AMD-CNN~\cite{Arslan:CCPE22} is a visualization-based Android
malware detection tool. It extracts the features from Android manifest
file and converts them to RGC images, then feeds the images to a CNN
network. It was evaluated based on 960 benign and 960 malicious
apps. \cite{Molina:computersecurity23} identified five factors that
influence the training and performance of ML-based Android malware
detectors. The authors conducted a fair comparison of eight detectors
on a mixed-market dataset, accounting for these factors. The detectors
include seven traditional ML models (across four types) and one DBN
with two hidden layers of 150 neurons each. Chen et
al.~\cite{Chen:USENIX23} proposed a continuous learning approach using
MLP to enhance the adaptability of ML-based Android malware detectors
to concept drift~\cite{Jordaney:USENIX17}. Their work includes a
comprehensive comparison with major baseline models, with both input
data and model-related code available as open source. The evaluation
was conducted using two established datasets,
APIGRAPH~\cite{Zhang:CCS20} and AndroZoo~\cite{Allix:MSR16}.

Our study complements these prior works by extensive benchmarking the proposed DL models against
additional standard ML models that were not examined by the original researchers. Our results lead
to reassessment of DL's advantage against ML in Android malware detection, and highlights the need
for extensive benchmarking for this problem domain.

% In contrast to previous works, our study evaluates multiple established datasets,  including a large-scale dataset that reflects real-world benign-to-malicious app ratios. By assessing both deep learning (DL) and traditional machine learning (ML) models using the same evaluation metrics, we aim to reassess the role of DL and ML in Android malware detection and highlight the critical need for extensive benchmarking in this field.



%%% Local Variables:
%%% mode: latex
%%% TeX-master: "../main"
%%% End:
