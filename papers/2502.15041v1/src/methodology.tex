\section{Benchmarking Methodology}
\label{sec:benchmark_method}

% More advanced detection models have been developed over time,
% particularly DL-based Android malware detection models.
Benchmarking
across different models is a standard approach for evaluating and
comparing their performance, as well as identifying the strengths and
weaknesses of various approaches. However, determining the sufficient
number of models to compare with is not easy, as further benchmarking
using more models could always be beneficial. For this reason, it is
utterly important that researchers make the dataset upon which their
models are evaluated available. This will enable other researchers
to evaluate additional models that had not been considered in the original
researcher. More extensive
evaluations can provide deeper insights and help understand the true
strengths of the various models for Android malware detection.

One of the key challenges for such extensive evaluation is that not all works publish the datasets used in their study.  Fortunately, we
have identified recent studies, such as Chen et al.’s
Enc+MLP~\cite{Chen:USENIX23} and Sun et al.’s
DetectBERT~\cite{Sun:esem24}, that have made their datasets publicly
available. This allowed us conduct more extensive benchmarking of the models proposed
in these two works.  Additionally, we collected a Google Play-only dataset and used two DL approaches,
Capsule Graph Neural Networks (CapsGNN) and ExcelFormer, which have not been explored before for Android
malware detection.
Each of the three works employs different datasets to evaluate their models,
and their approaches to app representation also vary. 
Our primary objective is not to assert the
superiority of any specific model but to demonstrate how comprehensive
benchmarking enhances the understanding of model performance. To this end we adopt
the following benchmarking strategy.

\begin{enumerate}
\item We keep the app representation (e.g., feature vectors) used in the original research.
  When the provided datasets include feature vectors used in learning,
  we simply use that information. When the
original feature vectors are not publicly available, we generate
them from the App apk files based on the information provided in the paper. % If the provided datasets only include app IDs,
  % we use the ID to retrieve the app's APK file and prepare the representation based
  % on the information provided in the paper.

%   In our extensive benchmarking, we
% respect each original research’s choice of app representation and do
% not alter them when evaluating additional ML models. 

\item We use the exact same datasets as used in the original research,
  % Rather, we keep the same dataset and app
  % representation as in the original work,
  and benchmark additional
  ML models that were not examined in the original research.
\end{enumerate}


Since each work uses different datasets,
the second point indicates that when interpreting the
results presented in this paper, comparisons shall only be made within
each work. Comparison of performance results across different works is
not meaningful.
While it may be tempting to use the same dataset to evaluate all the
models proposed in all these works, the research community has not
agreed upon a ``gold standard'' dataset for Android malware detection.
In fact to understand how choices in datasets construction
impact models' measured performance is a non-trivial research question
in its own right and beyond the scope of this paper.



% However, our objective is not to alter
% their methodologies but to demonstrate the significance of extensive
% benchmarking by introducing additional baseline models beyond those
% originally selected.


%In the following section, we explore additional models that were not
%included in the two prior studies.
In the following sections, we first discuss 
traditional machine learning models considered and then explore additional models
that were not included in the two prior studies.



% In this paper, we
% highlight the need for more extensive benchmarking in current Android
% malware detection.

%%% Local Variables:
%%% mode: latex
%%% TeX-master: "../main"
%%% End:
