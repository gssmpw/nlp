\section{Introduction}
 The Android ecosystem maintains the leading position as the largest
 and most active system among the worldwide mobile computing
 environment by far. Among its ecosystem components, the app markets
 which consist of Google Play, as well as manufacturer and third-party app stores,
 play a crucial role in releasing and distributing the apps
 to end users. According to one recent {\it 42matters}'s statistics~\cite{42matters:googleplay2025}, there
 were 40,325 apps on average released monthly just through
 Google Play Store between November 2024 and Juanuary 2025.
 The app market’s rapid growth makes it crucial to have an automated and effective app vetting
 system that scales to the app market sizes.
 An important component of a vetting
 system is the application of machine learning to triage the
 apps in the vetting process.

 Traditional machine learning approaches (shorthanded ML hereafter) such as Naive Bayes (NB), k-Nearest
Neighbors (k-NN), Support Vector Machines (SVM), Random Forest (RF),
Artificial Neural Networks (ANN), Gradient Boosting, etc. have been
applied to train models on a training dataset, which consists of apps
whose maliciousness is known. Then the trained models are applied to
classify a test dataset, where apps' maliciousness is unknown.  Many
traditional ML models have shown outstanding
performance~\cite{Iker:SPSM11, Aafer:SecureComm13, Gascon:AISec13,
WiSec:Saurabh13, Arp:NDSS14, Yang:ESORICS14, Periravian:ICTAI13,
Chen:ASIACCS16, Yan:MobiSys19, Gong:EuroSys20} on datasets from small
to market-scale.

 More recently, deep learning approaches (shorthanded DL hereafter) have been studied on Android
 malware detection~\cite{Yuan:SIGCOMM14, Su:TrustCom16, Hou:WIW16,
   McLaughlin:CODASPY17, Karbab:DFRWS18, Xu:EuroS&P18, Xu:ICFEM18,
   Xiao:MTA19, Kim:TIFS18, Wang:JAIHC19, Oak:AISec19,
   Alzaylaee:Computer&security20, Nadia:MALHat21, Chen:USENIX23, Sun:esem24}. One potential advantage of DL over
   ML is that it does not require an extensive feature selection and engineering process, although careful consideration still
   needs to be applied as to how to represent app
   data provided as input to the DL models.
 
   Despite the proposed DL approaches for this problem, there exist limited
   comparisons~\cite{Yuan:SIGCOMM14, Wang:JAIHC19,Shar:MOBILESoft20, Nadia:MALHat21, Sun:esem24}
   between ML and the proposed DL models.
 More often than not researchers do not publish the
dataset used to compare performances of the various models, making it
impossible for other researchers to further benchmark
the model against additional alternatives. It is encouraging that we
are seeing some recent works where researchers publish the datasets used
in their study. Upon looking through the literature, we identified two
works where 1) the researchers compared novel DL models against
ML models when applied to Android malware detection,
2) the researchers published the datasets used in their experiments,
and 3) the datasets are of substantial sizes.  
This enabled us to conduct additional experiments to benchmark the
proposed DL models against additional ML models that were not
examined in the original research.
In addition, we conducted benchmarking of two other DL models
 against the ML models, on a large dataset that we curated.
Our experiments show
 mixed results regarding whether DL models can outperform 
ML models for Android malware detection.
The results call for more extensive benchmarking of new deep learning models
against traditional machine learning models for
Android malware detection.



%  \item
%    Whereas the results of any such comparison cannot be easily extrapolated,
% we have taken care when selecting the compared models to ensure  that they are among
%  the most promising in traditional ML and modern DL used for Android malware
%  detection, based on existing literature.
%  Our experimentation shows mixed results for the DL approach --
%  it does not produce significantly better
%  results than the traditional ML approach, except in later years
%  when some major changes in app patterns were observed in the data.
%  These findings call for a more careful consideration of future DL models,
%  and more such systematic comparisons with
%  traditional ML models, when they are applied to Android malware detection.



%%% Local Variables:
%%% mode: latex
%%% TeX-master: "../main"
%%% End:
