\section{Traditional Machine Learning Models Considered}
\label{sec:alternative_ml}

Widely explored ML models in Android malware detection
include Naive Bayes (NB)~\cite{Yerima:AINA13, Sharma:CANS14}, support
vector machines (SVM)~\cite{Gascon:AISec13, Arp:NDSS14}, k-nearest
neighbors (k-NN)~\cite{Wu:AsiaJCIS12, Sharma:CANS14}, and random
forest (RF)~\cite{Mariconti:NDSS2017}.  We consider these four widely
used models as potential baselines to compare with the proposed
DL models.  Additionally, Gradient
Boosted Decision Trees (GBDTs) are powerful classification tools, with
CatBoost~\cite{CatBoost:NEURIPS2018} being a notable variant of
GBDT-based ensemble techniques. 
Having been introduced in late 2018, CatBoost has demonstrated strong
performance in classification and regression tasks across various
domains, including cybersecurity~\cite{CatBoost:NEURIPS2018}. Thus we also include
it as one of the ML models in our benchmarking. % Given
% their effectiveness, we consider these five widely used ML models as
% additional baselines comparisons for DL models DetectBERT, Enc+MLP, CapsGNN and ExcelFormer.

\textit{NB:} This is a simple but effective classifier. Furthermore,
it is the fastest classifier we evaluated. It can get the
classification result quickly even with half a million apps.

    \textit{SVM:} Widely used classifier in Android malware
classification. It requires more memory and computation time during
the training stage, which poses a significant challenge for
large datasets. We set the kernel to be linear, and use a grid
search with values 0.001, 0.1, 1.0, and 10 to decide the optimal
regularization parameter to train the SVM model.

    \textit{k-NN:} Another simple and robust traditional
classifier. The value of the hyper-parameter \(k\) has a direct impact
on the performance. We use a grid search with values between 3 and 15
to determine the optimal \(k\) value to train the k-NN model.

    \textit{RF:} A popular ensemble-type model based on decision
trees. It has shown promising performance in Android malware
detection~\cite{Mariconti:NDSS2017} and has the capability to be
applied to market-scale datasets~\cite{Gong:EuroSys20}. We perform a
grid search to find the optimal number of trees in the forest. The
estimator value is between 100 and 300.

    \textit{CatBoost:} Another ensemble-based model built on decision
trees. While it is relatively new compared to the other ML
models, recent studies across various domains have demonstrated its
effectiveness in classification
tasks~\cite{CatBoost:NEURIPS2018}. % Although it has not been widely
% applied to Android malware detection, its strong performance makes it
% a promising candidate for this domain.
CatBoost is sensitive to
hyperparameters, and our experimental results are based on the
following setup: 1000 iterations, a learning rate of 0.1, a depth of
10, and an L2 regularization parameter (l2\_leaf\_reg) of 5.

%%% Local Variables:
%%% mode: latex
%%% TeX-master: "../main"
%%% End:
