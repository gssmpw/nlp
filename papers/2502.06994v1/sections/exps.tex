\section{Experiments}
\label{Section: Experiments}






% Colors: 
% green: *5
% red: <10 *5, >=10 *3
\begin{table*}[htbp]
\begin{center}
\begin{small}
\vspace{-1.2em}
\caption{\textbf{Out-of-Sync Recovery Evaluation on \textit{Caller} and \textit{Callee}.} The influence of increased task complexity introduced by dependency tracing on agents' \textit{out-of-sync} recovery performance: $\Delta_\textit{complexity}=\Delta_\textit{(Callee-Caller)}$.}
\label{tab:table 2 (caller vs callee)}
\vspace{0.2em}  % Reduces space below the caption
    \begin{tabular}{>{\centering\arraybackslash}m{2.4cm}|>{\centering\arraybackslash}m{1.8cm}||>{\centering\arraybackslash}m{0.9cm}>{\centering\arraybackslash}m{0.9cm}>{\centering\arraybackslash}m{0.9cm}|>{\centering\arraybackslash}m{0.9cm}>{\centering\arraybackslash}m{0.9cm}>{\centering\arraybackslash}m{0.9cm}|>{\centering\arraybackslash}m{0.9cm}>{\centering\arraybackslash}m{0.9cm}>{\centering\arraybackslash}m{0.9cm}}
    
    \toprule
    \multirow{2}{*}{\centering \textbf{Agent}} & \multirow{2}{*}{\centering \textbf{Recovery}} & \multicolumn{3}{c|}{\textbf{Caller (\%)}} & \multicolumn{3}{c|}{\textbf{Callee (\%)}} & \multicolumn{3}{c}{\textbf{$\Delta_\textit{complexity}$ (\%)}} \\
    
    \cmidrule{3-11}
    &  & $LA_{file}$ & $LA_{func}$ & $SR$ & $LA_{file}$ & $LA_{func}$ & $SR$ & $LA_{file}$ & $LA_{func}$ & $SR$ \\
    
    \midrule
    
    \multirow{3}{*}{\centering \textbf{Llama-3.1-8B}} 
    & \textbf{Independent} & 13.33 & 8.00 & 1.33 & 4.00 & 1.33 & 0.67 & \cellcolor{basecolor_red!46.65} -9.33 & \cellcolor{basecolor_red!33.35} -6.67 & \cellcolor{basecolor_red!3.30} -0.66 \\
    & \textbf{Collaborative} & 32.00 & 26.00 & 2.00 & 22.00 & 16.67 & 0.67 & \cellcolor{basecolor_red!50.0} -10.00 & \cellcolor{basecolor_red!46.65} -9.33 & \cellcolor{basecolor_red!6.65} -1.33 \\
    & \textbf{$\Delta_{\textit{\text{collaborator}}}$} & \cellcolor{basecolor_green!93.35} +18.67 & \cellcolor{basecolor_green!90.0} +18.00 & \cellcolor{basecolor_green!3.35} +0.67 & \cellcolor{basecolor_green!90.0} +18.00 & \cellcolor{basecolor_green!76.7} +15.34 & \cellcolor{basecolor_green!0.00} +0.00 & \cellcolor{basecolor_red!3.35} -0.67 & \cellcolor{basecolor_red!13.3} -2.66 & \cellcolor{basecolor_red!3.35} -0.67 \\
    
    \midrule
    
    \multirow{3}{*}{\centering \textbf{Llama-3.1-70B}} 
     & \textbf{Independent} & 8.67 & 5.33 & 4.00 & 8.00 & 4.67 & 1.33 & \cellcolor{basecolor_red!3.35} -0.67 & \cellcolor{basecolor_red!3.3} -0.66 & \cellcolor{basecolor_red!13.35} -2.67 \\
    & \textbf{Collaborative} & 12.00 & 6.00 & 3.33 & 12.67 & 5.33 & 3.33 & \cellcolor{basecolor_green!3.35} +0.67 & \cellcolor{basecolor_red!3.35} -0.67 & \cellcolor{basecolor_green!0.00} +0.00 \\
    & \textbf{$\Delta_{\textit{\text{collaborator}}}$} & \cellcolor{basecolor_green!16.65} +3.33 & \cellcolor{basecolor_green!6.65} +1.33 & \cellcolor{basecolor_red!3.35} -0.67 & \cellcolor{basecolor_green!23.35} +4.67 & \cellcolor{basecolor_green!3.3} +0.66 & \cellcolor{basecolor_green!10.0} +2.00 & \cellcolor{basecolor_green!6.7} +1.34 & \cellcolor{basecolor_red!3.35} -0.67 & \cellcolor{basecolor_green!13.35} +2.67 \\
    
    \midrule
    
    \multirow{3}{*}{\centering \textbf{GPT-4o mini}} 
    & \textbf{Independent} & 13.29 & 9.30 & 5.32 & 7.97 & 6.64 & 2.66 & \cellcolor{basecolor_red!26.60} -5.32
 & \cellcolor{basecolor_red!13.30} -2.66 & \cellcolor{basecolor_red!13.30} -2.66 \\
    & \textbf{Collaborative} & 15.28 & 11.96 & 7.97 & 9.30 & 4.65 & 2.66 & \cellcolor{basecolor_red!29.9} -5.98 & \cellcolor{basecolor_red!36.55} -7.31 & \cellcolor{basecolor_red!26.55} -5.31 \\
    & \textbf{$\Delta_{\textit{\text{collaborator}}}$} & \cellcolor{basecolor_green!9.95} +1.99 & \cellcolor{basecolor_green!13.3} +2.66 & \cellcolor{basecolor_green!13.25} +2.65 & \cellcolor{basecolor_green!6.65} +1.33 & \cellcolor{basecolor_red!9.95} -1.99 & \cellcolor{basecolor_green!0.0} +0.00 & \cellcolor{basecolor_red!3.30} -0.66 & \cellcolor{basecolor_red!23.25} -4.65 & \cellcolor{basecolor_red!13.25} -2.65 \\
    
    \midrule

    \multirow{3}{*}{\centering \textbf{DeepSeek}} 
    & \textbf{Independent} & 58.00 & 47.33 & 8.67 & 37.33 & 22.67 & 6.00 & \cellcolor{basecolor_red!62.01} -20.67 &  \cellcolor{basecolor_red!73.98} -24.66  &  \cellcolor{basecolor_red!13.35} -2.67  \\
    & \textbf{Collaborative} & 52.00 & 47.33 & 8.67 & 42.00 & 27.33 & 6.67 & \cellcolor{basecolor_red!30.0} -10.00 & \cellcolor{basecolor_red!60.0} -20.00 &  \cellcolor{basecolor_red!10.0} -2.00  \\
    & \textbf{$\Delta_{\textit{\text{collaborator}}}$} & \cellcolor{basecolor_red!30.0} -6.00 & \cellcolor{basecolor_green!0.0} +0.00 & \cellcolor{basecolor_green!0.0} +0.00 & \cellcolor{basecolor_green!23.35} +4.67 & \cellcolor{basecolor_green!23.3} +4.66 & \cellcolor{basecolor_green!3.35} +0.67 & \cellcolor{basecolor_green!53.35} +10.67 & \cellcolor{basecolor_green!23.30} +4.66 & \cellcolor{basecolor_green!3.35} +0.67 \\
    
    \midrule

    \multirow{3}{*}{\centering \textbf{GPT-4o}} 
    & \textbf{Independent} & 14.67 & 11.33 & 6.67 & 14.00 & 7.33 & 1.33 & \cellcolor{basecolor_red!3.35} -0.67 & \cellcolor{basecolor_red!20.0} -4.00 & \cellcolor{basecolor_red!26.70} -5.34 \\
    & \textbf{Collaborative} & 39.33 & 35.33 & 10.00 & 38.67 & 34.00 & 6.00 & \cellcolor{basecolor_red!3.30} -0.66 & \cellcolor{basecolor_red!6.65} -1.33 & \cellcolor{basecolor_red!20.00} -4.00 \\
    & \textbf{$\Delta_{\textit{\text{collaborator}}}$} & \cellcolor{basecolor_green!123.3} +24.66 & \cellcolor{basecolor_green!120.0} +24.00 & \cellcolor{basecolor_green!16.65} +3.33 & \cellcolor{basecolor_green!123.35} +24.67 & \cellcolor{basecolor_green!133.35} +26.67 & \cellcolor{basecolor_green!23.35} +4.67 & \cellcolor{basecolor_green!0.05} +0.01 & \cellcolor{basecolor_green!13.35} +2.67 & \cellcolor{basecolor_green!6.70} +1.34 \\
    
    \midrule

    \multirow{3}{*}{\centering \textbf{Llama-3.3-70B}} 
    & \textbf{Independent} &  80.67  &  60.00  &  18.67  &  47.33  &  34.67  &  14.00  &  \cellcolor{basecolor_red!100.02} -33.34  &  \cellcolor{basecolor_red!75.99} -25.33  &  \cellcolor{basecolor_red!23.35} -4.67  \\
    & \textbf{Collaborative} &  77.33  &  64.67  &  22.00  &  56.00  &  42.67  &  16.00  &  \cellcolor{basecolor_red!63.99} -21.33  &  \cellcolor{basecolor_red!66.00} -22.00  &  \cellcolor{basecolor_red!30.00} -6.00  \\
    & \textbf{$\Delta_{\textit{\text{collaborator}}}$} &  \cellcolor{basecolor_red!16.7} -3.34  &  \cellcolor{basecolor_green!23.35} +4.67  &  \cellcolor{basecolor_green!16.65} +3.33  &  \cellcolor{basecolor_green!43.35} +8.67  &  \cellcolor{basecolor_green!40.0} +8.00  &  \cellcolor{basecolor_green!10.0} +2.00 &  \cellcolor{basecolor_green!60.05} +12.01  &  \cellcolor{basecolor_green!16.65} +3.33  &  \cellcolor{basecolor_red!6.65} -1.33  \\
    
    \midrule

    \multirow{3}{*}{\centering \textbf{Claude-3.5-Sonnet}} 
    & \textbf{Independent} & 50.83 & 47.51 & 25.41 & 77.35 & 65.19 & 30.94 & \cellcolor{basecolor_green!132.6} +26.52 & \cellcolor{basecolor_green!88.4} +17.68 & \cellcolor{basecolor_green!27.65} +5.53 \\
    & \textbf{Collaborative} & 43.09 & 38.67 & 28.73 & 79.56 & 65.19 & 38.67 & \cellcolor{basecolor_green!182.35} +36.47 & \cellcolor{basecolor_green!132.60} +26.52 & \cellcolor{basecolor_green!49.70} +9.94 \\
    & \textbf{$\Delta_{\textit{\text{collaborator}}}$} & \cellcolor{basecolor_red!38.7} -7.74 & \cellcolor{basecolor_red!44.2} -8.84 & \cellcolor{basecolor_green!16.6} +3.32 & \cellcolor{basecolor_green!11.05} +2.21 & \cellcolor{basecolor_green!0.0} +0.00 & \cellcolor{basecolor_green!38.65} +7.73 & \cellcolor{basecolor_green!49.75} +9.95 & \cellcolor{basecolor_green!44.2} 
    +8.84 & \cellcolor{basecolor_green!22.05} +4.41 \\
    
    \bottomrule
    
    \end{tabular}
\end{small}
\end{center}
\vspace{-1.5em}
\end{table*}






\subsection{Setup}
\label{Section: Experiment Setup}

% Through independent or collaborative recovery (\Sref{Section: Recovery Settings}), we evaluate different LLM agents' \textit{out-of-sync} recovery capabilities based on the following experimental setups: 

\textbf{Recovery Protocol.}
For baselines, each agent is allowed up to $30$ turns to achieve $B_n=S_n$, which is then extended to $50$ turns to assess agents' temporal resource awareness and exploitation.
Financial resources are mapped similarly to each resource-aware recovery task.
Provided with different action options---\textcolor{fig2_env}{\textbf{interacting with \textit{Env}}}, \textcolor{fig2_code}{\textbf{proposing a solution}}, or \textcolor{fig2_ask}{\textbf{proactively seeking collaborator assistance}} (\Sref{Section: Recovery Settings})---both independent and collaborative agents take each of their moves autonomously.
% based on individual state beliefs ($B_t$, $t \in [2,n]$) and recovery planning. 

\textbf{\textit{Env} Space.}
We employ \textit{OpenHands} \cite{OpenHands} to empower agents to autonomously explore and inspect the codebase environment by executing various commands.
This exploration enables them to develop a comprehensive understanding of the codebase for \textit{out-of-sync} recovery.
% Our \textit{Docker} environment ensures standardized repository access and command execution, along with controlled validation of recovery solutions.

\textbf{Agents.} Our experiments assess the \textit{out-of-sync} recovery capabilities of seven LLMs, including four open-source (\textit{Meta-Llama-3.1-8B, Meta-Llama-3.1-70B,  Meta-Llama-3.3-70B, and DeepSeek-V2.5}) and three close-source (\textit{Claude-3.5-Sonnet, GPT-4o mini, and GPT-4o}) LLMs \cite{meta_llama_3.1_8B, meta_llama_3.1_70B, meta_llama_3.3_70B, deepseek_v2.5, claude_3.5_sonnet, gpt_4o, gpt_4o_mini} in two recovery settings (\Sref{Section: Recovery Settings}), respectively. 







\begin{figure}[H]
\begin{center}
\begin{small}
\includegraphics[width=1\linewidth]{sections/figs/figure5.png}
\vspace{-2em}
    \caption{\textbf{Influence of Collaborator Assistance.} We quantify collaborator influence on agent \textit{out-of-sync} recovery performance as $\Delta_{\textit{\text{collaborator}}}$ to unveil its \textcolor{fig5_green}{\textbf{+positive}} or \textcolor{fig5_red}{\textbf{-negative}} impact on certain aspects of agents' recovery performance.}
    \vspace{-1.2em}  % Reduces space below the caption
    \label{fig:figure 5 (recovery performance and human influence)}
\end{small}
\end{center}
\end{figure}









%%%%%%%%%%%%%%%%%%%%
% Performance Gap
%%%%%%%%%%%%%%%%%%%%
\subsection{Significant Ability Gaps Among Agents Powered by Different LLMs}
\label{Section: Agent Recovery Capabilities}
% Remain consistent in: 
% (1) Independent & Collaborative
% (2) Caller & Callee

% Obervation
Our experiments on \textit{SyncBench} (\tref{tab:table 2 (caller vs callee)}-\ref{tab:table 1 (performance summary)}) reveal substantial capability gaps among seven LLM agents.

% Independent
\textbf{Baselines for Out-of-Sync Recovery.}
LLM agents' independent \textit{out-of-sync} recoveries demonstrate significant variations in their baseline capabilities, ranging from \textit{Claude-3.5-Sonnet} ($SR=28.18\%$) to \textit{Llama-3.1} agents ($SR \leq 2.67\%$).
Their localization capabilities also vary remarkably, regardless of pinpointing the exact \textit{out-of-sync} function ($LA_{func} \in [4.67, 56.35]\%$) or less precisely localizing responsible \textit{Python} files ($LA_{file} \in [8.33, 64.09]\%$). Likewise, our evaluation on agents' technical capabilities (\tref{tab:table 1 + CSR (overall performance summary)}) also exhibits substantial gaps among LLMs.
% \xw{let's use the words amount different LLMs, since the agent implementation didn't change, but we only change the underlying LLMs}
% ($CSR_{file} \in [3.81, 43.97]\%$, $CSR_{func} \in [7.07, 53.40]\%$)

% Collaborative
\textbf{Persistent Gaps Despite Varying Recovery Conditions.}
\tref{tab:table 1 (performance summary)} and \tref{tab:table 1 + CSR (overall performance summary)} show similar performance disparities for collaborative agents mirroring their independent recoveries, despite the generally positive influence of collaborative assistance.
These performance gaps remain significant for tasks of different complexity (\tref{tab:table 2 (caller vs callee)} \& \ref{tab:table 2 + CSR (caller vs callee)}).
% Analyzing \textit{Caller}
% ($SR \in [1.33, 28.73]\%$, $LA_{file} \in [8.67, 43.09]\%$, $LA_{func} \in [5.33, 47.51]\%$, $CSR_{file} \in [6.25, 66.67]\%$, $CSR_{func} \in [7.69, 75.05]\%$)
% and \textit{Callee}
% ($SR \in [0.67, 38.67]\%$, $LA_{file} \in [4.00, 79.56]\%$, $LA_{func} \in [1.33, 65.19]\%$, $CSR_{file} \in [3.05, 48.55]\%$, $CSR_{func} \in [4.02, 62.48]\%$)
% separately, agents' capability gaps widen in complex scenarios involving dependency tracing (\tref{tab:table 2 (caller vs callee)} \& \ref{tab:table 2 + CSR (caller vs callee)}).
% While \textit{Claude-3.5-Sonnet} maintains robust performance with positive gains ($+\Delta_\textit{complexity}$), other agents' performance degrades by over $23.07\%$.
% Agents' technical performance also degrades significantly (\tref{tab:table 2 + CSR (caller vs callee)}).
Agents' persistent performance variances across diverse task scenarios highlight their underlying ability gaps in identifying and resolving \textit{out-of-sync} to maintain effective collaborations.


% in terms of both their localization and recovery capabilities.
% ($LA_{file}$$\in$$[12.29, 66.67]\%$, $LA_{func}$$\in$$[5.67, 53.67]\%$, $SR$$\in$$[1.33, 33.70]\%$)
% \tref{tab:table 1 + CSR (overall performance summary)} extends these significant ability gaps to their technical capabilities ($CSR_{file}$$\in$$[4.93, 54.95]\%$, $CSR_{func}$$\in$$[6.24, 64.90]\%$).
%%%%% Independent + Collaborative %%%%%
% Assessing two recovery channels collectively widens their performance gaps even further ($SR \in [0.33, 33.70]\%$, $LA_{file} \in [8.33, 64.09]\%$, $LA_{func} \in [4.67, 56.35]\%$, $CSR_{file} \in [3.81, 54.95]\%$, $CSR_{func} \in [6.24, 64.90]\%$).




% Overall
% Viewing collectively on all \textit{out-of-sync} tasks (\tref{tab:table 1 (performance summary)}), in addition to their overall SR scores that range from \textit{Llama-3.1-8B}'s independent recovery ($0.33\%$) to \textit{Claude-3.5-Sonnet}'s collaborative performance ($33.70\%$); LLM agents' localization capabilities also vary remarkably, regardless of pinpointing the exact \textit{out-of-sync} code ($LA_{func}$: from \textit{Llama-3.1-8B}'s $4.67\%$ to \textit{Claude-3.5-Sonnet}'s $56.35\%$) or their less precise localization of \textit{Python} files containing the responsible functions ($LA_{file}$: from \textit{Llama-3.1-70B}'s $8.33\%$ to \textit{Claude-3.5-Sonnet}'s $64.09\%$).
% Caller & Callee
% With \textit{Claude-3.5-Sonnet} notably excelling while \textit{Llama-3.1} markedly underperformed, LLM agents' individual consistency in their \textit{out-of-sync} localization and recovery exhibits unambiguous ability gaps in maintaining collaborative synchronization.
% to evaluate agents' recovery capabilities across multiple dimensions: their \textit{out-of-sync} recovery performance, their collaboration initiatives, the influence of collaborator assistance, and their resource awareness and utilization behaviors.
% including (1) $LA$: agents' state mismatche localization, (2) $mSR$: agents' re-synchronization capabilities, (3) the influence of collaborator assistance ($\Delta_{\textit{\text{collaborator}}}$), and (4) agents' resource awareness and utilization.



\subsection{In Achieving Recovery Success: Technical, Reasoning, and Collaborative Competences}
% \subsection{Multifaceted Abilities for Effective Recoveries \xw{this is too abstract -- what is multifaceted ability? be concrete!}}
\label{Section: Multifaceted Abilities for Effective Recoveries}


Conditioned on localization success, $CSR$ (\eqref{Eq: Evaluation Metric 3 (CSR)}) is significantly influenced by how much time left for technical recovery after accurate localizations, which are largely determined by agents' abilities to efficiently identify root causes of $B_k \neq S_k$.
Comparing \tref{tab:table 2 (caller vs callee)}-\ref{tab:table 1 (performance summary)} with \tref{tab:table 1 + CSR (overall performance summary)}-\ref{tab:table 2 + CSR (caller vs callee)}, low-performing agents can also showcase strong technical problem-solving capacities (\textit{e.g.,} $CSR$: \textit{Llama-3.1-70B} $>$ \textit{GPT-4o}), despite their notably underperformed localization and recovery abilities (\textit{e.g.,} \textit{Llama-3.1-70B}: $LA_{file} \leq 12.33\%$, $LA_{func} \leq 5.67\%$, $SR \leq 3.33\%$) and remarkably low willingness to collaborate (e.g., \textit{Llama-3.1-70B}: $ASR=1.37\%$).
This observation further substantiates that successful \textit{out-of-sync} recoveries hinge on not only agents' technical problem-solving proficiency, but their efficient cause analysis and effective collaboration capabilities.







%%%%%%%%%%%%%%%%%%%%
% Human Influence
%%%%%%%%%%%%%%%%%%%%
\subsection{Collaborative Assistance Improves Performance---
But Agents Seldom Seek Help
% \xw{again! this is way too vague! just say something like "Interact with collaborator improves performance -- But agent don't always asks for collaborator helps"}
}
\label{Section: Beneficial Collaborator Influence Limited By Disadvantageous Agent Initiative}



\textbf{Positive Collaborator Influence.}
As shown in \fref{fig:figure 5 (recovery performance and human influence)}, collaborator assistance generally improves recovery performance ($SR: +\Delta_{\textit{\text{collaborator}}}$$\in$$[0.33, 5.52]\%$), with the magnitude varying dependent on agents' technical capabilities and willingness to collaborate.
% Compared with collaborator influence on their localization ($LA_{file}: \Delta_{\textit{\text{collaborator}}} \in [-2.76, +24.67]\%$, $LA_{func}: \Delta_{\textit{\text{collaborator}}} \in [-4.42, +25.34]\%$), agents benefit more in resolving \textit{out-of-sync} (\tref{tab:table 1 (performance summary)} \& \ref{tab:table 1 + CSR (overall performance summary)}).
Agents with stronger independent recovery capabilities and collaboration willingness (\textit{Claude-3.5-Sonnet}: $SR=28.18\%$ and $ASR=4.86\%$), together with conditioned technical proficiency (\tref{tab:table 1 + CSR (overall performance summary)}), obtain higher performance gains ($\Delta_{\textit{\text{collaborator}}}=+5.52\%$) than other agents ($SR \leq 4.00\%$ and $ASR \leq 2.98\%$). 



\textbf{Performance Upper Bound: Solving task with Oracle Information.}
% Ceiling test
To establish the theoretical upper bound of collaborator influence and identify agents' collaboration capability gaps, we additionally conduct a single-turn experiment by providing agents with oracle information that is used to simulate the collaborator (\textit{i.e.,} the know-everything agent). With \textit{GPT-4o mini} as the agent tackling \textit{out-of-sync} and \textit{GPT-4o} as the know-everything collaborator (\Sref{Section: LLM-simulated Collaborators}), we configure each single-turn task with collaborator’s exhaustive task-specific natural language instructions on how to accomplish recovery success.
%, meanwhile replacing all resource constraints with the single-attempt recovery setting.
% Its evaluation follows our standard \textit{parsing-based execution validation} protocol (\Sref{Section: Evaluation Metrics}), maintaining consistency with multi-turn experiments while operating within the one-turn constraint.
Furnished with full recovery instructions ($ASR=100\%$), the high upper bound ($SR=86.33\%$) lends further evidence to LLM agents' significantly untapped potential in effective collaboration.

% To establish theoretical performance bounds and identify capability gaps, we additionally conduct single-turn experiments by providing agents with optimal collaboration conditions for recovery. With \textit{GPT-4o mini} performing agent \textit{out-of-sync} tasks and \textit{GPT-4o} simulating know-everything collaborator, we configure each single-turn task with collaborator’s complete task-specific natural language instruction on how to successfully recover from $B_2$ to $B_n$, meanwhile replacing all resource constraints with the single-attempt setting. Evaluation follows our standard \textit{parsing-based execution validation} protocol (\Sref{Section: Evaluation Metrics}), maintaining consistency with multi-turn experiments (\Sref{Section: Multi-Turn Experiments}) while operating within the one-turn constraint.
% The effectiveness of collaborative software development hinges on how well collaborators can detect and recover from \textit{out-of-sync} states.
% Our analysis reveals fundamental patterns in how collaborator assistance affects agents' \textit{out-of-sync} recoveries (\fref{fig:figure 5 (recovery performance and human influence)}).



\textbf{LLM Agents' Low Willingness to Collaborate.}
% \xw{use plain words to convey the message clearer: smth like not all LLM likes collaboration}
Despite strong technical capacities (\tref{tab:table 1 + CSR (overall performance summary)}), LLM agents show limited collaboration willingness (\fref{fig:figure 7 (Time Allocation)}).
Nevertheless, \textit{Claude-3.5-Sonnet} with the highest performance ($SR=33.70\%$) and collaboration willingness ($ASR=4.86\%$) derives the most benefit from collaboration ($\Delta_{\textit{\text{collaborator}}}=+5.52\%$).
It is followed by \textit{GPT-4o} who obtains notable improvements in both $LA$ ($\Delta_{\textit{\text{collaborator}}}=+25.34\%$) and $SR$ ($\Delta_{\textit{\text{collaborator}}}=+4.00\%$) through proactive assistance seeking.
The lowest $ASR=1.21\%$ presented by \textit{DeepSeek} ($SR: \Delta_{\textit{\text{collaborator}}}=+0.34\%$) contrastingly substantiates the significance of proactive collaboration initiative.






\begin{figure}[H]
    \centering
    \vspace{-0.5em}  % Reduces space above the caption
    \includegraphics[width=1\linewidth]{sections/figs/figure6.png}
    \vspace{-1.8em}
    \caption{\textbf{Question Quality.} Agents from left to right on the \textit{X}-axis according to their $ASR$ from low to high.}
    \vspace{-1.5em}  % Reduces space below the caption
    \label{fig:figure 6 (Time Allocation) (Question Quality)}
\end{figure}






%%%%%%%%%%%%%%%%%%%%
% Communication Patterns
%%%%%%%%%%%%%%%%%%%%
\subsection{Quality and Strategy of Communication Are Crucial for Recovery Success
% \xw{Quality and Strategy of Communication is Crucial for Recovery Success -- your first sentence is actually great use it as title}
}
% Question Quality and Communication Strategy Collectively Determine Collaboration Quality
\label{Section: Collaborative Effectiveness} 

The quality and strategy of communication prove crucial for recovery success, with several key patterns emerging:

\textbf{Agents with More High-Quality Questions Achieve Better Performance.} 
% High-quality queries correlate more strongly with recovery success than query volume. 
Depending on whether the question asked by the agent can lead to recovery success (\Sref{Appendix:C.4 (Effective Assistance Seeking)}), we rate the quality of each query as low (\textit{resulting in recovery failure}) or high (\textit{resulting in recovery success}), which can be further classified into two general categories: \textit{\textcolor{fig7_blue}{localization queries}} closely related to localizing \textit{out-of-sync} causes and \textit{\textcolor{fig7_salmon}{solution queries}} seeking guidance on \textit{out-of-sync} resolution. 
% error resolution and re-synchronization details
Despite no significant correlation between query volume and recovery success, agents with a larger proportion of high-quality questions
% in both categories
achieve higher performance (\fref{fig:figure 6 (Time Allocation) (Question Quality)}).
%(\textit{\textcolor{fig7_blue}{localization}} and \textit{\textcolor{fig7_salmon}{solution}})
% Arranged based on $ASR$ from low to high along the X axis, higher $ASR$ scores show no directly correlation with larger proportions of high-quality questions in either categories.



\begin{figure*}[!t]
\begin{center}
\begin{small}
\vspace{-0.5em}
    \includegraphics[width=1\linewidth]{sections/figs/figure7.png}
    \vspace{-1.8em}
    \caption{\textbf{Time Allocation.} Agents' performance are ranked from low to high according to their \textit{independent} $SR$ scores, based on which they are positioned on the \textit{X}-axis from left to right. The \textit{Y}-axis depicts each agent's time allocation.}
    \vspace{-1.5em}  % Reduces space below the caption
    \label{fig:figure 7 (Time Allocation)}
\end{small}
\end{center}
\end{figure*}












\textbf{Strategic Early Exploration Facilitates Recovery Success.}
%%%%%%%%%%%%%%%%%%%%%%%%%%%%%%%%%%%%%%%%
% First half asking turns proportion (success, failure)
% GPT-4o mini: 100 --- 81.38
% GPT-4o: 91.67 --- 79.10
% Llama-3.1-8B: 100 --- 97.93
% Llama-3.1-70B: 100 --- 97.41
% Deepseek: 85.71 --- 65.69
% Claude: 93.62 --- 55.76
%%%%%%%%%%%%%%%%%%%%%%%%%%%%%%%%%%%%%%%%
We compute each agent's communication timing distribution respectively for its success and failure cases (\fref{fig:figure C1 (first-half action distribution)}).
Results reveal that the proportion of assistance seeking in agents' first half of recovery time is substantially larger in success cases ($85.71\%-100.00\%$) than in recovery failures ($55.76\%-97.93\%$).
As top-performing agents exhibit distinct communication strategies with front-load queries, random or back-loaded assistance-seeking demonstrates less effective improvements on agents' performance. Compared with solution proposal timing that shows trivial differences between success and failure cases (averagely $2.79$ \textit{turns delayed} in successful recoveries), collaborative agents benefit markedly more from advancing their assistance seeking (averagely $10.50$ \textit{turns ahead} in successful recoveries).






%%%%%%%%%%%%%%%%%%%%%%%%%%%%%%
% Task Complexity
%%%%%%%%%%%%%%%%%%%%%%%%%%%%%%
\subsection{More Challenging Tasks Decrease Performance While Better Manifest Collaboration Benefits}
\label{Section: Effects of Task Complexity}
% (1) Caller & callee: Easy tasks have in general higher performance among all agents.
% (2) Some repos and tasks that are not solved in ceiling tests, are successfully resolved by principle experiments. 
% (3) Human influence on recovery efficiency (less turns to recover)

% Our in-depth analysis reveals critical factors affecting agents' \textit{out-of-sync} recovery effectiveness.



% \textbf{More Challenging Tasks Can Lead To Decreased Performance But More Evident Collaboration Benefits.}
We observe a large negative influence of increased task complexity on agents' recoveries.
%
%%%%%% Caller & Callee %%%%%%
%
\textit{Callee}'s additional dependency tracing allows it to serve more challenging \textit{out-of-sync} tasks (\Sref{Section: Benchmark Construction}).
Comparing agents' performance between \textit{Caller} and \textit{Callee} (\tref{tab:table 2 (caller vs callee)} \& \ref{tab:table 2 + CSR (caller vs callee)}), \textit{Claude-3.5-Sonnet}'s performance gains ($+\Delta_\textit{complexity}$) demonstrate its superior technical capabilities in resolving complicated \textit{out-of-sync} tasks.
Nevertheless, \textit{Callee}, presenting higher task complexity, in general undermines agents' performance ($-\Delta_\textit{complexity}$).

%%%%%% Repository complexity %%%%%%
Leveraging dissimilar complexity levels of 21 source repositories (\fref{fig:figure C8 (repo-wise performance)}), our repository-wise evaluation reflects consonant patterns between task complexity and recovery success.
% \textit{11-whisper} ($SR$: \textit{Independent} $33.33\%$, \textit{Collaborative} $22.22\%$) proposes the least recovery difficulty, followed by \textit{18-optuna} ($SR$: \textit{Independent} $20.83\%$, \textit{Collaborative} $21.43\%$). The lowest performance delivered on \textit{13-sphinx} ($SR$: \textit{Independent} $0.88\%$, \textit{Collaborative} $4.70\%$) serves more challenging tasks.
While the repository \texttt{11-whisper} proposes the least recovery difficulty ($SR$: \textit{Independent} $33.33\%$, \textit{Collaborative} $22.22\%$),
the lowest performance delivered on the repository \texttt{13-sphinx} ($SR$: \textit{Independent} $0.88\%$, \textit{Collaborative} $4.70\%$) serves more challenging tasks.

%%%%%% Collaborator influence manifests in more complex tasks %%%%%%
Although repository complexity manifests negative correlations with recovery success, \textbf{\textit{the effectiveness of collaborator assistance increases on more challenging tasks}}, comparing to its trivial or negative influence on agents' recoveries in simpler \textit{out-of-sync} scenarios (\textit{e.g.,} \texttt{11-whisper} with $\Delta_{\textit{\text{collaborator}}}=-11.11\%$, in contrast to \texttt{13-sphinx} with $\Delta_{\textit{\text{collaborator}}}=+3.82\%$).
This holds in \textit{Callee} where agents notably gain more benefits from collaborator assistance (\tref{tab:table 2 (caller vs callee)}), despite higher task complexity.

% Our repo-wise comparison elaborates clear performance patterns given the similar trends observed across all agents equipped with different recovery methods. This pattern persists in varying evaluation matrices across repositories with higher variance in recovery performance for complex repositories. 



% \textbf{LLM Agents Can Succeed In Upper-Bound Failures.}
% Comparison with performance upper bound (\Sref{Section: Beneficial Collaborator Influence Limited By Disadvantageous Agent Initiative}) reveals interesting patterns in agents' tackling \textit{out-of-sync}. Despite \textit{GPT-4o mini}'s remarkable upper-bound performance ($SR=86.33\%$), there are cases where it fails in upper-bound tests while agents succeed in multi-turn experiments (\Sref{Appendix:C.6 (Can Be Better Than Ceiling)}). This discloses underlying advantages of multi-turn interactive recovery over single-turn instruction in terms of autonomous action planning and task-oriented adaptations in collaborative systems.
% \xw{i think this paragraph is not informative - consider remove}

% recovery planning and adaptive assistance-seeking.






%%%%%%%%%%%%%%%%%%%%
% Resource Awareness
%%%%%%%%%%%%%%%%%%%%
\subsection{Agents' Significant Lack of Resource Awareness}
\label{Section: Resource Awareness}

% \ref{tab:table C2 (Time Awareness)}
% \ref{tab:table C3 (Budget Awareness)}
% \ref{tab:table C4 (Cost Awareness)}
% \ref{tab:table C1 (Appendix: Recovery Efficiency)

We systematically vary resource constraints (Tables \ref{tab:table C2 (Time Awareness)}-\ref{tab:table C1 (Appendix: Recovery Efficiency)}) to investigate LLM-based agents' resource awareness in two key dimensions (\Sref{Section: Resource Awareness Module}):
(1) \textit{Time Resources}, through comparing the standard $30$-turn recovery with extended $50$-turn performance;
(2) \textit{Financial Resources}, through varying initial budgets from $\$1000$ (insufficient for $30$-turn costs) to $\$3000$ (adequate for any $30$-turn action taking patterns), and halving or doubling the cost of collaborator assistance.
Our experiments reveal critical limitations in agents' resource awareness and adaptive resource utilization capabilities.

\textbf{More Recovery Time Can Not Guarantee Performance Gains.}
Extended time produces divergent effects (\fref{fig:figure C3 (time awareness - allocation)} \& \tref{tab:table C2 (Time Awareness)}):
diminishing returns on \textit{Llama-3.1-8B} ($SR$: \textit{Independent} $-0.33\%$, \textit{Collaborative} $-1.00\%$),
while notable improvements on \textit{Llama-3.1-70B} ($SR$: \textit{Independent} $+3.67\%$, \textit{Collaborative} $+4.67\%$).
This observation suggests that extending the recovery time limit alone is insufficient for improved performance, while LLM agents' competences in effective time utilization and technical expertise (\tref{tab:table 1 + CSR (overall performance summary)}) factor underlyingly into recovery success.

\textbf{Agents' Low Sensitivity to Financial Resources.}
% LLM agents show minimal sensitivity to economic resources. 
Tripling the initial budget yields trivial changes in action planning ($ASR$ improvements: $[+0.22,+1.06]\%$, \fref{fig:figure C6 (budget awareness - allocation)}) and recovery performance ($SR$ variations: $[-0.66,+1.67]\%$, \tref{tab:table C3 (Budget Awareness)}). Similarly, halving or doubling the action cost of assistance seeking contributes to negligible differences in LLM agents' willingness to collaborate ($ASR$ variations: $[-0.11,+0.23]\%$ for halved cost,  $[-0.46,+0.04]\%$ for doubled cost, \fref{fig:figure C7 (cost awareness - allocation)}) and recovery performance ($SR$ variations: $[-2.00,-1.00]\%$ for halved cost, $[-1.00,+1.00]\%$ for doubled cost, \tref{tab:table C4 (Cost Awareness)}).
These findings highlight the fundamental deficiencies of existing LLM agents in effectively recognizing resource constraints and adaptively leveraging available resources \cite{agent_alignment}.
% \xw{This highlights the weakness of existing LLM agents in recognizing resource constraints and use them well \cite{agent_alignment}.}

% These findings unfold current LLM agents' lack of meaningful resource awareness, with performance primarily determined by their underlying capabilities without strategic resource management. This limitation suggests a significant opportunity for improving agents' resource-aware decision making in collaborative settings.

% Out-of-sync recovery success depends heavily on agents' ability to strategically deploy available resources. 
% In pursuit of resource-efficient collaboration in real-world scenarios, our experiments shed light on agents' awareness of time, budget, and cost, through its recovery journey.



% \textbf{Strategic Action Planning.}
% The sequence and timing of recovery actions significantly impact recovery progress. Successful recoveries prioritize repository exploration in early turns (\fref{fig:figure 7 (Time Allocation)}) that helps establish essential contextual understanding before attempting solutions.

% \textbf{Resource Utilization.} 
% Endowing agents with the awareness of resource constraints, the ability of strategic resource deployment is highlighted for real-world collaborations. 
% We focus on the resources of \textit{Time Management} and \textit{Cost Efficiency} to unveil implications towards resource-efficient \textit{out-of-sync} recovery. 

% \textbf{(1) \textit{Time Management}.} 
%%%%%%%%%%%%%%%%%%%%%%%%%%%%%%%%%%%%%%%%
% First half asking turns proportion (success---failure)
% GPT-4o mini: 100 --- 81.38
% GPT-4o: 91.67 --- 79.10
% Llama-3.1-8B: 100 --- 97.93
% Llama-3.1-70B: 100 --- 97.41
% Deepseek: 85.71 --- 65.69
% Claude: 93.62 --- 55.76
%%%%%%%%%%%%%%%%%%%%%%%%%%%%%%%%%%%%%%%%
% Comparing agents' performance between $30$-turn and $50$-turn recovery settings (\tref{tab:table C2 (Time Awareness)}), high-performing agents typically allocate $75\%-95\%$ of their time to repository exploration and context understanding through \textit{Env} interaction and proactive assistance-seeking. Increasing the maximum time limit from $30$ to $50$ shows diminishing returns on \textit{Llama-3.1-8B} ($\Delta SR: Independent-0.33\%, Collaborative-1.00\%$) while significant improvements on \textit{Llama-3.1-70B} ($\Delta SR: Independent+3.67\%, Collaborative+4.67\%$). The fact that \textit{Llama-3.1-70B} strategically allocates early turns (the first half of time) for exploring ($Independent: 88.68\%$ \textit{Env} interaction; $Collaborative: 100.00\%$ assistance asking and $94.64\%$ \textit{Env} interaction) while postponing their solution proposal to later phases, critically emphasizes early-stage exploration over augmented total available time in effective out-of-sync recovery. \textit{Claude-3.5-Sonnet} ($SR=33.70\%$) also supports this observation as it allocates $93.62\%$ assistance-seeking turns in its first half of recovery time among its success cases. 
% The consistant impairing effects of increased time availability on positive collaborator assistance further highlights the significance of strategic action planning through extended recovery journey.


% \textbf{(2) \textit{Cost Efficiency}.} 
% Our budget-cost awareness reveals varying levels of resource sensitivity among agents. Providing the identical basic cost settings (both \textit{solution proposal} and \textit{proactive assistance-seeking} at a cost of $\$100$), we first alter the initial budget between $\$1000$ and $\$3000$, where an initial budgets of $\$3000$ guarantees limitless action taking over all 30 turns (\tref{tab:table 3 (Budget Awareness)}). Surprisingly, initial budget variations exhibit less impact on performance than expected ($+\Delta SR<1.67\%$), suggesting the low sensitivity of agents in budget management. The adjustment of collaborator assistance costs ($\$50$, $\$100$, $\$200$) produces similar performance changes among agents (\tref{tab:table C4 (Cost Awareness)}). Both halved and doubled assistance costs contribute to trivial differences in terms of time allocation for proactive queries ($<0.26\%$) as well as overall recovery performance ($<2.00\%$), unfolding agents' low sensitivity of cost and budget, along with their lacks of strategic balance calculations for resource estimation and utilization planning throughout their recovery journeys.











