\section{Related Work}

\subsection{Theory of Mind in Collaborative Systems}

The concept of ToM---the ability to model and reason about others' mental states---has emerged as the foundation of collaborative systems. 
LLMs' ToM capabilities have been evaluated and applied to various domains through benchmark construction \cite{benchmark_tombench, benchmark_opentom, benchmark_Hi_ToM, benchmark_fantom} and experimental investigation \cite{LLM_ToM_human_like_reasoning, ToM_multimodal_video_LLM, ToM_nature, spontaneous_ToM, improve_LLM_ToM, ToM_in_HRI}.
% revealing their abilities in exhibiting sophisticated mental state reasoning in situated contexts.
% Exploring ToM abilities of human and LLMs, from false belief understanding to indirect query interpretation \cite{ToM_nature}, also provides important implications for collaborative systems.
% While OpenToM \cite{opentom} unfolds the limited ToM reasoning capabilities of LLMs in tracing mental states, empirical studies \cite{spontaneous_ToM} provide insights into the spontaneous emergence of ToM in LLMs. 
We apply ToM to software engineering agents for their maintaining of a shared understanding of codebase states, which is particularly of critical essence in asynchronous collaboration environments where \textit{out-of-sync} situations frequently arise due to temporal gaps between contributions.





\subsection{Collaborative Software Engineering Systems}


The evolution of software engineering tools and practices relies heavily on the premise of synchronized collaboration. Modern version control systems \cite{version_control_git, version_control_PredictingMC, version_control_NanoVC}, although implement sophisticated mechanisms for detecting and resolving conflicts arising from divergent codebase states, primarily address syntactic conflicts rather than semantic understanding divergence. 
Recent work showcases that LLM reasoning can be effectively advanced by diverse means, such as human feedback \cite{human-sim-mint, image_retrieval}, \textit{Env} interaction \cite{OpenHands, CodeAct, CodeAgent}, multi-agent cooperation \cite{AgentCoder}, etc.
Pressing closer to real-world repository-level programming, advances in LLMs inspire their agentic engagements in software engineering \cite{LLM_in_software_engineering}, in addition to benchmarks built upon \github repositories \cite{SWE-bench, R2E, benchmark_CoReQA}.
However, real-world dynamic environments require adaptations for collaborative systems based on the presumption of relatively static environments. Agents' lack of automatic synchronization and resource awareness therefore arise as latent obstacles that significantly impede collaborative intelligence and resource efficiency that we aim to provide insights into.







\section{Conclusion and Discussion}
\label{Section: Conclusion and Discussion}

In this paper, we investigate the \textit{out-of-sync} challenge in collaborative software engineering by introducing our \textit{out-of-sync} recovery framework, \textbf{\textit{SyncMind}} \Sref{Section: framework}, and evaluation benchmark, \textbf{\textit{SyncBench}} \Sref{Section: Benchmark}.
% , with our construction method generalizable to diverse \textit{Python} repositories and scalable to accommodate custom needs.
Experiments reveal that successful \textit{out-of-sync} recoveries require not only
% sophisticated ToM capabilities to model actual world states, 
technical proficiency (\Sref{Section: Agent Recovery Capabilities}-\ref{Section: Multifaceted Abilities for Effective Recoveries}), but also effective collaboration (\Sref{Section: Beneficial Collaborator Influence Limited By Disadvantageous Agent Initiative}-\ref{Section: Collaborative Effectiveness}) and adaptive resource management (\Sref{Section: Resource Awareness}) abilities.
Based on our evaluation of multiple aspects of their recovery performance (\Sref{Section: Evaluation Metrics}), results unveil existing LLM agents' limited collaboration willingness and resource awareness, providing insights for future development of collaborative systems with stronger collaboration initiative and cooperation competences, along with more adaptive resource utilization capacities.
%
% Although generally benefitting from collaborator assistance, results unveil existing LLM agents' limited proactive collaboration initiative and resource awareness, particularly in real-world collaborative scenarios where belief state divergence is challenging but essential to resolve and collaborators' domain knowledge is crucial for \textit{out-of-sync} recovery success.
%
% These insights highlight the need to develop collaborative systems with stronger collaboration willingness and cooperation competences, along with more adaptive resource utilization capacities.
%
% Our evaluation benchmark and framework provide the foundation for evaluating and improving collaborative software engineering with resource efficiency, ultimately working toward bridging the gap between agents' technical capabilities and their collaborative effectiveness.
%
Detailed discussions of our findings and limitations are presented in \textit{Appendix}~\ref{Appendix A: Discussions and Limitations}.
