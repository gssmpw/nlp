\section{Related Work}
\label{sec: relatedwork}
\subsection{Traditional methods}

Traditional methods often fall short in capturing the dynamic and diverse interactions. This shortcoming has led to a growing interest in leveraging artificial intelligence (AI) to enhance drug-gene research. Identifying disease-associated long non-coding RNAs (lncRNAs) has been the focus of various models, highlighting the role of non-coding genetic elements in diseases. GAERF ____, introduced by Wu \textit{et al.}, employs graph autoencoders (GAE) combined with random forest techniques for detecting disease-related lncRNAs. In contrast, LPLNS ____ utilizes label propagation with linear neighborhood similarity to predict novel lncRNA-disease associations, thereby deepening our understanding of lncRNA roles in disease. lncRNA prediction methods add an extra layer of genetic interaction data, which, when integrated with drug-gene and miRNA predictions, offers a comprehensive view of genetic influences on diseases. Moreover, predicting disease-associated miRNAs sheds light on how these small RNA molecules regulate genes, serving as a vital complement to drug-gene interaction prediction. SMALF ____, created by Liu \textit{et al.}, integrates latent features from the miRNA-disease association matrix with original features, utilizing XGBoost ____ to predict unknown miRNA-disease associations. Likewise, CNNMDA ____, introduced by Xuan \textit{et al.}, employs network representation learning in conjunction with convolutional neural networks (CNN).
 
\subsection{GNN-based methods}

Recent progress in AI, especially with graph neural networks (GNNs), has demonstrated significant potential in uncovering drug-gene associations. GNNs excel at capturing patterns from graph-structured data, making them highly effective in modeling drug-gene relationships. These models facilitate the identification of potential interactions, which is valuable for drug repurposing and discovering new therapeutic targets. For example, SGCLDGA ____ integrates GNNs with contrastive learning, using graph convolutional network (GCN)  ____ to derive vector representations of drugs and genes from a bipartite graph. It employs singular value decomposition (SVD) to refine the graph and create multiple views. Further, it optimizes these representations with a contrastive loss function to effectively differentiate positive and negative samples. In addition, multiple GNN-based methods have been developed for predicting miRNA-disease associations. HGCNMDA ____ combines GNNs with node2vec and GCN to jointly learn features of miRNAs and diseases, offering a comprehensive perspective by integrating diverse biological networks. LAGCN ____ employs graph convolution along with attention mechanisms to learn and integrate embeddings for drugs and diseases from heterogeneous networks, effectively concentrating on crucial features to enhance prediction accuracy. Lastly, NIMCGCN ____ applies GCNs to uncover hidden features of miRNAs and diseases from similarity networks, thereby improving the discovery of novel miRNA-disease associations by leveraging similarities in known associations. These miRNA-disease prediction models contribute an additional layer of insight into the genetic underpinnings of diseases, which can aid in developing more targeted therapies when combined with drug-gene interaction predictions.

\subsection{Meta-path based methods}

Meta-path-based methods are extensively applied in drug-disease association prediction due to their effectiveness in revealing latent relationships through intermediate nodes within heterogeneous networks. A meta-path, defined as a sequence of relations linking different node types, captures essential semantic information, thereby enhancing predictive accuracy. MGP-DDA, in studying drug-disease associations, constructs a tripartite network involving drugs, gene ontology (go) functions, and diseases, employing three specific meta-paths to capture their interrelations ____. These meta-paths generate features based on path instance counts, which are then used for drug-disease association prediction, showing robust performance by retaining richer semantic information from intermediate nodes. Similarly, HSGCLRDA integrates meta-path aggregation in a drug-disease-protein heterogeneous network ____. By constructing both global and local feature graphs through meta-paths, HSGCLRDA can capture detailed structural and contextual information about the interactions between drugs, diseases, and proteins. This meta-path-based aggregation is combined with contrastive learning to optimize feature representations, leading to improved prediction accuracy. Both methods highlight the strength of meta-path-based techniques in enhancing prediction tasks by retaining and utilizing rich semantic information from heterogeneous networks. These approaches not only enable the discovery of novel drug-disease associations but also contribute to the broader understanding of the underlying biological mechanisms.