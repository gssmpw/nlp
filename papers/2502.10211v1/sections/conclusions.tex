\section{Conclusions}
\label{sec:conclusions}
In this paper, we took on the detection of control-flow anomalies in the business processes of organizations. To address the shortcomings of existing \revision{conformance checking}-based techniques and support the development of competitive techniques for control-flow anomaly detection while maintaining the explainable nature of \revision{conformance checking}, we proposed a novel \revision{process mining}-based feature extraction approach with alignment-based \revision{conformance checking}. This \revision{conformance checking} variant aligns the deviating control flow with a reference process model; the resulting alignment can be inspected to extract additional statistics such as the number of times a given activity caused mismatches. Secondly, we integrated this approach into a flexible and explainable framework for developing techniques for control-flow anomaly detection. The framework combines \revision{process mining}-based feature extraction and dimensionality reduction, which can handle high-dimensional feature sets, achieve detection effectiveness, and support explainability. In addition to our proposed \revision{process mining}-based feature extraction approach, the framework allows employing other approaches, enabling flexible development of multiple \revision{conformance checking}-based and \revision{conformance checking}-independent techniques for control-flow anomaly detection. Using a set of synthetic and real-world datasets, we experimented with several \revision{conformance checking}-based and \revision{conformance checking}-independent framework techniques, and compared the results with baseline \revision{conformance checking}-based techniques. The results showed that the \revision{conformance checking}-based framework techniques using our approach outperform the baseline \revision{conformance checking}-based techniques while maintaining the explainable nature of \revision{conformance checking}. In particular, the best-performing framework techniques achieved 97.3\%, 88.9\%, 90.5\% and 88.5\% $F1$ for the PDC 2020, PDC 2021, ERTMS and COVAS datasets, whereas the baseline \revision{conformance checking}-based techniques could achieve up to 36.1\%, 56.6\%, 78.8\%, and 79.1\%. We also provide an explanation of why existing \revision{conformance checking}-based techniques may be ineffective. Finally, the results indicated that detection effectiveness is not solely dependent on the specific framework techniques used, as no one-size-fits-all \revision{process mining}-based feature extraction approach is suitable for all the datasets. In fact, analysis of variance outlined that both the dataset and \revision{process mining}-based feature extraction approach show statistically significant differences and high importance in $F1$ values.

Future work will extend the capabilities of the framework by considering additional event log attributes including information related to other perspectives as well as the control-flow one \cite{aalst2022pmhandbook}. Such information can be used in \revision{process mining}-based feature extraction to enrich the tabular data, address other types of anomalies, and result in improved detection effectiveness. The extension will also include object-centric process mining, which allows for focusing on the activities of the different entities involved in a process instead of flattening the event data and treating all objects uniformly \cite{aalst2023ocpm}.



