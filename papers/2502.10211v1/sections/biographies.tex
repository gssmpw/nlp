\begin{IEEEbiography}[{\includegraphics[height=1.25in]{images/authors/VITALE.jpg}}]{Francesco Vitale} is a PhD student in Information Technology and Electrical Engineering at the Department of Electrical Engineering and Information Technology of the University of Naples Federico II, where he has earned his M.Sc. degree in Computer Engineering in 2021. He has been a Visiting researcher at the Chair of Process and Data Science (PADS) - RWTH Aachen University in 2023. His research activities mainly address Industry 4.0, Anomaly Detection, and Process Mining.
\end{IEEEbiography}

\begin{IEEEbiography}[{\includegraphics[height=1.25in]{images/authors/PEGORARO.png}}]{Marco Pegoraro} is a Scientific Assistant (PhD
candidate) at the Chair of Process and Data Science (PADS) - RWTH Aachen University. He joined the chair in 2018. In 2016, he completed his M.Sc. degree in Computer Science at the University of Padua, Italy. His specialization is process mining, and he is particularly active in the fields of process mining in healthcare and analysis of uncertain, probabilistic, and non-deterministic data.
\end{IEEEbiography}

\begin{IEEEbiography}[{\includegraphics[height=1.25in]{images/authors/AALST.png}}]{Wil M. P. van der Aalst} is a Full professor at RWTH Aachen University, leading the Process and Data Science (PADS) group. He is also the Chief Scientist at Celonis, part-time affiliated with the Fraunhofer FIT, and a member of the Board of Governors of Tilburg University. He also has unpaid professorship positions at Queensland University of Technology (since 2003) and the Technische Universiteit Eindhoven (TU/e). Currently, he is also a distinguished fellow of Fondazione Bruno Kessler (FBK) in Trento, deputy CEO of the Internet of Production (IoP) Cluster of Excellence, and co-director of the RWTH Center for Artificial Intelligence.

\end{IEEEbiography}
\begin{IEEEbiography}[{\includegraphics[height=1.25in]{images/authors/MAZZOCCA.png}}]{Nicola Mazzocca} is a Full Professor of Computer Systems at the Department of Electrical Engineering and Information Technology of the University of Naples Federico II. Since 1994, he has held numerous university courses and has been involved in several professional training activities on different topics, including high-performance systems, distributed and embedded systems, security, and reliability. His research activities concern computer architecture, distributed systems, high-performance computing, and safety-critical applications. He is author of more than 250 publications in international journals, books, and conference proceedings.
\end{IEEEbiography}
\vfill