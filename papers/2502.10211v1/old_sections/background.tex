\section{Preliminaries}
\label{sec:background}

Following is a review of techniques for implementing control-flow anomaly detection's Phase 1) and 2), focusing on ML/DL, PM, and the combination of PM and ML. Moreover, key concepts for control-flow anomaly detection are formally defined, such as event logs, sublogs, fitness, and alignments. Finally, we describe our feature extraction proposal based on alignment-based CC.

\subsection{Machine Learning and Deep Learning}
Arguably, ML provides the most commonly applied techniques for control-flow anomaly detection, implementing several learning paradigms, which can be characterized as supervised, semi-supervised, or unsupervised \cite{witten2017dmml}. Among these, the unsupervised paradigm has gained large interest due to the scarcity of labeled datasets and zero-day anomalies, employing several approaches, such as distance-based methods, e.g., kNN and clustering, and dimensionality reduction-based methods, e.g., the Principal Component Analysis (PCA) and the Autoencoder (AE) \cite{zoppi2021unsupervisedad,sakurada2014adeutoencoders}.

Recently, DL by means of RNNs and their variants, such as LSTMs and GRUs, has been used for control-flow anomaly detection, leading to satisfactory performance \cite{nolle2022binet, lahann2022lstmadpi}. However, the complexity, costly training, and explainability of DL techniques' detection results represent major challenges \cite{wang2022dlspatiotemporaldm,chen2022efficientrnntraining}. 

It is worth noting that due to ML being rooted in statistics \cite{witten2017dmml}, these approaches require the encoding of event logs into numerical data by feature extraction, such as counting the frequencies of activities in event logs, using specific embeddings such as word2vec, or computing N-grams \cite{ko2023adsystematicreview}.

