Join order optimization is critical in achieving good query performance. Despite decades of research and practice, modern query optimizers could still generate inferior join plans that are orders of magnitude slower than optimal. Existing research on robust query processing often lacks theoretical guarantees on join-order robustness while sacrificing query performance. In this paper, we rediscover the recent \PT technique from a robustness point of view. We introduce two new algorithms, \TreeStruct and \SafeSubJoin, and then propose \RPT (\rpt) that is provably robust against arbitrary join orders of an acyclic query. We integrated \RPT with \duckdb, a state-of-the-art analytical database, and evaluated against all the queries in \tpch, \job, and \tpcds benchmarks. Our experimental results show that \rpt improves join-order robustness by orders of magnitude compared to the baseline. With \rpt, the largest ratio between the maximum and minimum execution time out of random join orders for a single acyclic query is only $1.6\times$ (the ratio is close to 1 for most evaluated queries). Meanwhile, applying \rpt also improves the end-to-end query performance by $\approx$$1.5\times$ (per-query geometric mean). We hope that this work sheds light on solving the practical join ordering problem.
