\section{Additional Robustness Results (Bushy)}

In Appendix C, we present the distribution of execution time with random bushy plans for each query of \RPT, compared to our baseline methods: vanilla DuckDB, Bloom Join, and \PT. These results are shown in \Cref{fig:tpch-bushy} (\tpch) and \Cref{fig:job-bushy} (\job), \Cref{fig:tpcds-bushy-a} (\tpcds query 1-52), \Cref{fig:tpcds-bushy-b} (\tpcds query 53-99), \Cref{fig:dsb-bushy-a} (\dsb query 1-52) and \Cref{fig:dsb-bushy-b} (\dsb query 53-99).

The conclusions are consistent with the left-deep results, but we observe a deterioration in robustness. As discussed in the paper, this can be attributed to incorrect probe-build side selection during the hash join.

\begin{figure*}[t!]
    \centering
    \includegraphics[width=\linewidth]{./pic/job-bushy-full-version.pdf}
    \caption{The distribution of execution time with random bushy plans for each query in \job \textnormal{-- Normalized by the execution time of default \duckdb. The figure is log-scaled. The box denotes 25- to 75-percentile (with the orange line as the median), while the horizontal lines denote min and max (excluding outliers).}}
    \label{fig:job-bushy}
\end{figure*}

\begin{figure*}[t!]
    \centering
    \includegraphics[width=\linewidth]{./pic/tpcds-bushy-full-version-a.pdf}
    \caption{The distribution of execution time with random bushy plans for each query (1-52) in \tpcds \textnormal{-- Normalized by the execution time of default \duckdb. The figure is log-scaled. The box denotes 25- to 75-percentile (with the orange line as the median), while the horizontal lines denote min and max (excluding outliers). `*' indicates timeouts. Cyclic queries are in red.}}
    \label{fig:tpcds-bushy-a}
\end{figure*}

\begin{figure*}[t!]
    \centering
    \includegraphics[width=\linewidth]{./pic/tpcds-bushy-full-version-b.pdf}
    \caption{The distribution of execution time with random bushy plans for each query (53-99) in \tpcds \textnormal{-- Normalized by the execution time of default \duckdb. The figure is log-scaled. The box denotes 25- to 75-percentile (with the orange line as the median), while the horizontal lines denote min and max (excluding outliers). `*' indicates timeouts. Cyclic queries are in red.}}
    \label{fig:tpcds-bushy-b}
\end{figure*}

\begin{figure*}[t!]
    \centering
    \includegraphics[width=\linewidth]{./pic/dsb-bushy-full-version-a.pdf}
    \caption{The distribution of execution time with random bushy plans for each query (1 - 52) in \dsb \textnormal{-- Normalized by the execution time of default \duckdb. The figure is log-scaled. The box denotes 25- to 75-percentile (with the orange line as the median), while the horizontal lines denote min and max (excluding outliers). `*' indicates timeouts. Cyclic queries are in red.}}
    \label{fig:dsb-bushy-a}
\end{figure*}

\begin{figure*}[t!]
    \centering
    \includegraphics[width=\linewidth]{./pic/dsb-bushy-full-version-b.pdf}
    \caption{The distribution of execution time with random bushy plans for each query (53 - 99) in \dsb \textnormal{-- Normalized by the execution time of default \duckdb. The figure is log-scaled. The box denotes 25- to 75-percentile (with the orange line as the median), while the horizontal lines denote min and max (excluding outliers). `*' indicates timeouts. Cyclic queries are in red.}}
    \label{fig:dsb-bushy-b}
\end{figure*}