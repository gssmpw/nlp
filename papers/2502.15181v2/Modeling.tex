\section{Toward Join-Order Robustness}
\label{sec:modeling}

This section introduces new algorithms with analyses to make \PT robust for acyclic queries. In \cref{sec:modeling:transfer}, we propose the \TreeStruct algorithm in the transfer phase (i.e., the counterpart of the semi-join phase in \Yann) that not only guarantees a full reduction but also minimizes the \BF construction time. \cref{sec:modeling:join} discusses approaches to guarantee that the join order selected in the join phase is ``safe'' (i.e., there is no intermediate result blowup).

\subsection{Generating a Robust Transfer Schedule}
\label{sec:modeling:transfer}

The transfer phase in the original \PT algorithm~\cite{yang2023PT} adopts \StoL, a simple heuristic-based algorithm to build the transfer graph. As described in \cref{sec:prelim:pt}, \StoL assigns the direction for each edge in the (undirected) join graph from the smaller table to the larger table to form a DAG. \PT then generates a \emph{transfer schedule} (i.e., the forward and backward passes of \BFs) by following the edges in this DAG. The \StoL algorithm, however, does not guarantee a full reduction for acyclic queries. As shown in \cref{fig:small2large-example}, for example, consider the natural join $R(A,B) \Join S(A,C) \Join T(B, D)$ where $|R| < |S| < |T|$. In this case, \StoL will generate a transfer graph that leads to a forward pass of $S \Semijoin_b R$ and $T \Semijoin_b R$ followed by a backward pass of $R \Semijoin_b S$ and $R \Semijoin_b T$. This transfer schedule fails to ``connect'' $S$ and $T$: if $S$ has a predicate, this filter information can never reach $T$ via the transfer of \BFs (and vice versa), leading to an incomplete reduction.

\begin{figure}[t!]
    \center
    \includegraphics[width=\linewidth]{pic/rst.pdf}
    \caption{An example of the \StoL algorithm in the original \PT}
    \label{fig:small2large-example}
\end{figure}

Although \StoL cannot pre-filter all the non-result tuples, pushing larger tables toward the end of the transfer schedule is insightful because a smaller table is likely a more selective filter. The new \TreeStruct will preserve this strategy while guaranteeing a full reduction. Before diving into our new algorithm, let us define the concepts of a \emph{join tree} and \emph{acyclicity} precisely.

Without loss of generality, we only consider natural joins with a connected join graph in this section\footnote{For equality predicates such as $R.A = S.B$, we treat $A$ and $B$ as the same attribute in this context. If the join graph has multiple components, we can generalize the concept of join tree to join forest}. For a natural join query $q$, its \emph{join graph} $G_q$ is an undirected graph where the vertices are the relations in $q$. If two relations have attributes in common, they are connected by an edge in $G_q$. A \emph{join tree} $T_q$ is a spanning tree of $G_q$ such that for every attribute $A$, the relations containing $A$ induce a \emph{connected} subgraph $T^A_q$ of $T_q$. The join tree is then used to define the \emph{acyclicity} of a query:

\begin{definition}[$\alpha$-acyclicity \cite{yannakakis1981YA}]\label{def:acyclic}
A natural join query $q$ is {\em acyclic} if and only if there exists a \emph{join tree} of $q$.
\end{definition}

Acyclicity is crucial for \YannAlg to achieve the $O(N + OUT)$ complexity because it guarantees a non-decreasing intermediate result in the join phase. If the subgraph for an attribute $A$ is \emph{not} connected, a tuple may survive in the first join involving $A$ but later get eliminated by the second join using $A$. This breaks the above non-decreasing property. An acyclic natural join satisfies the following lemma:

\begin{lemma}[\cite{Maier1983}]\label{thm:mst}
Let $q$ be an acyclic natural join query. For each edge $(R,S)$ in the join graph $G_q$, where $R$ and $S$ are the vertices (i.e., relations), define the weight of the edge $w(R,S)$ as the number of shared attributes between $R$ and $S$: $w(R,S)=\left|\operatorname{attr}(R)\cap\operatorname{attr}(S)\right|$. 
Then, a subgraph of $G_q$ is a join tree of $q$ if and only if it is a maximum spanning tree for $G_q$.
\end{lemma}

The intuition behind the lemma is that for a spanning tree $T$ of $G_q$, $T$'s total weight equals the summation of the edge count of each attribute-induced subgraph. $T$ is a join tree means that every attribute-induced subgraph $T^A$ is connected. This is equivalent to saying that every $T^A$ is a subtree, and it is impossible for any $T^A$ to have more edges (otherwise $T$ will not be a tree). $T$ must be a maximum spanning tree (MST). Note that the weights defined on the edges are not considered heuristics for join costs. They are used to transform the problem of finding a join tree into the problem of finding an MST in the join graph.

\SetAlgoNlRelativeSize{0}
\begin{algorithm}[t!]
    \KwIn{join graph $G_q$}
    \KwOut{tree $T$}
    $T \gets \varnothing$; $\mathcal{R} \gets$ all relations; $\mathcal{R}' \gets \{R_{max}\}$\;

    \While{$\mathcal R' \neq \mathcal R$}{
    Find an edge $e = \{R,S\} \in E(G_q)$ with the largest weight
    such that $R\in\mathcal R\setminus \mathcal R',S\in\mathcal R'$. Choose the edge with the largest $R$ to break ties\;\label{step:choose}
    
    Add $e$ to $T$ with direction from $R$ to $S$\;
    
    $\mathcal R'\gets \mathcal R'\cup\{R\}$\;\label{step:expand_r}
    }

    return $T$\;
    \caption{\TreeStruct}
    \label{alg:LargestRoot}
\end{algorithm}

We now know that for an acyclic query, a join tree guarantees a full (semi-join) reduction of the query, and we can find a join tree by constructing a maximum spanning tree on its weighted join graph. We next introduce our \TreeStruct algorithm. As shown in \cref{alg:LargestRoot}, we use Prim's algorithm to construct a maximum spanning tree $T$ on the join graph $G_q$. The edges in $T$ point from leaf to root, indicating a forward pass schedule. Because the algorithm starts with the largest relation $R_{max}$ in $\mathcal{R}'$, $R_{max}$ is the root of $T$ (hence the name \TreeStruct). And because of \cref{thm:mst}, $T$ is a join tree if query $q$ is acyclic, guaranteeing a full reduction in the transfer phase. 

Placing the largest relation at the root of the join tree is important, especially for queries following a star schema. It is more efficient to filter the much larger fact table using the dimension tables first before building a \BF on the fact table. In addition, \TreeStruct pushes larger relations toward the root by including them early in $T$ in the tie-breaking strategy in \cref{step:choose}. This allows larger relations to get filtered first by probing other \BFs before building their own, thus minimizing the total \BF construction time in the transfer phase. Notice that \cref{step:choose} in \TreeStruct does not specify a tie-breaking policy for choosing $S \in \mathcal{R}'$. In reality, most edges have weight 1 because relations typically join on only one attribute. Although the choice of $S$ does not compromise the theoretical guarantee of the algorithm producing an MST, it could affect the shape of the join tree. In general, a flatter tree allows more parallelism in building the \BFs, while a deeper tree might allow filtering irrelevant tuples out earlier. Both the tie-breaking policies for $R$ and $S$ do not affect the strong theoretical guarantee (i.e., a full reduction) of \TreeStruct.

Unlike \YannAlg, \TreeStruct also applies to cyclic queries. The algorithm's output is still a spanning tree with the largest relation at the root, but it is not a join tree. In this case, the transfer schedule generated by \TreeStruct does not guarantee a fully reduced instance for the subsequent join phase. Still, it transfers any predicate to all relations at least once and is effective empirically, as we will show in the experiments in \cref{sec:eval}.


%-----------------------------------------------------------------------------------------------------------------------


\subsection{Choosing a Safe Join Order}
\label{sec:modeling:join}

Once the transfer phase generates a fully reduced instance of the database, the algorithm enters the join phase to produce the final output. According to \YannAlg, the join order is derived from the join tree used in the semi-join phase by performing the joins bottom up. Although such a (almost fixed) join order guarantees the asymptotic complexity of \YannAlg (i.e.,$O(N + OUT)$), it prevents the optimizer from exploring more join orders that potentially have smaller costs. Ideally, we want to leverage the cost models in the optimizer to search for cheaper plans, but we want to constrain the optimizer to only consider join orders with intermediate results always upper bounded by the output size. Such a ``safe'' join order provides a (theoretical) robustness guarantee: its runtime cost is at most a constant factor away from the optimal. In other words, even ill-behaved data distributions will not cause the runtime to deviate more than some bounded quantity.

\begin{definition}[\cite{safesubjoin}]\label{def:safe}
Let $q$ be an acyclic natural join query. A subjoin $q'$ of $q$ is {\em safe} if for every fully reduced instance $I$, we have $q'(I) = \pi_{\operatorname{attr}(q')}(q(I))$.
\end{definition}

The above definition ensures that if a subjoin $q'$ is safe, then the output of $q'$ is a projection of the final output, and thus $|q'(I)| \leq |q(I)|$. If every subjoin of a join order is safe, then the cumulative intermediate result size is within a constant factor of $|q(I)|$ (i.e., the optimal). It is straightforward to see that subjoins that involve Cartesian products can be unsafe. But unsafe subjoins are not restricted to Cartesian products. Consider the natural join $q = R(A,B,C) \Join S(A,B) \Join T(B,C)$. Let $I$ be the fully reduced instance:
$R = \{(1,1,1), (2,1,2), \dots, (n,1,n)\}$, $S = \{(1,1),(2,1), \dots, (n,1)\}$, and $T = \{(1,1),(1,2), \dots, (1,n)\}$.
Then subjoin $q' = S(A,B) \Join T(B,C)$ is unsafe because $|q'(I)| = n^2$, while $|q(I)| = n$. Therefore, any query plan that joins $S$ with $T$ first -- even on a fully reduced instance -- will create a quadratic blowup on the intermediate result.

One approach to avoid unsafe join orders is to \emph{identify the class of acyclic queries} for which any join order that does not involve Cartesian products is safe.

\begin{definition}[$\gamma$-acyclicity~\cite{acyclic_degrees}]\label{def:gamma}
A natural join query $q$ is $\gamma$-acyclic if and only if there is no $\gamma$-cycle in $q$. This is equivalent to (1) $q$ is $\alpha$-acyclic, and (2) we cannot find three relations $R, S, T$ with attributes $x, y, z$ that form a $\gamma$-cycle of size 3: $R(x, y), S(y, z), T(x, y, z)$.
\end{definition}

\begin{lemma}[\cite{acyclic_degrees}]
Every connected join expression\footnote{No Cartesian products, binary joins only.} $\theta$ of $q$ is monotone (i.e., no tuple gets removed while executing any binary join in $\theta$) if and only if $q$ is $\gamma$-acyclic.
\end{lemma}

\begin{theorem}\label{thm:gamma_safe}
Every subjoin (without Cartesian products) for natural join query $q$ is safe if and only if $q$ is $\gamma$-acyclic.
\end{theorem}


\begin{proof}
    It is sufficient to show that every subjoin is safe if and only if every connected join expression is monotone.

    Consider any connected join expression $\theta'$ for subjoin $q'$, $\theta'_1$ for subjoin $q'_1$, and $\theta'_2$ for subjoin $q'_2$,
    where $\theta'=\theta'_1\Join \theta'_2$.
    Because every subjoin without Cartesian products is safe, we have
    $q'(I)=\pi_{\operatorname{attr}(q')}(q(I))$, 
    $q'_1(I)=\pi_{\operatorname{attr}(q'_1)}(q(I))$, and
    $q'_2(I)=\pi_{\operatorname{attr}(q'_2)}(q(I))$.
    Because $\operatorname{attr}(q'_i) \subseteq \operatorname{attr}(q')$ for $i = 1, 2$,
    we have $|\pi_{\operatorname{attr}(q')}(q(I))| \ge |\pi_{\operatorname{attr}(q_i')}(q(I))|$.
    Therefore, $\theta'$ is monotone.

    For the other direction,
    consider any connected join expression $\theta'$ of a subjoin $q'$.
    Extend $\theta'$ to a complete join expression $\theta$ of $q$.
    Because $\theta'$ is part of $\theta$ and every connected join expression of $q$ is monotone,
    we have $q'(I)=\pi_{\operatorname{attr}(q')}(q(I))$ for any fully reduced instance $I$.
    Therefore, $q'$ is safe.
\end{proof}

\cref{thm:gamma_safe} gives a strong \textbf{robustness guarantee}: if a query is $\gamma$-acyclic, we can fully trust the optimizer for join ordering on a fully-reduced instance (i.e., the join phase) because it can never pick an unsafe join order. $\gamma$-acyclic queries are a subset of $\alpha$-acyclic (i.e., acyclic) queries according to \cref{def:gamma}. To quickly check for $\gamma$-acyclicity in practice, it is sufficient (not necessary) to show that no two relations in the join graph are directly connected by more than one edge (i.e., no composite-key joins).


For queries that are acyclic but not $\gamma$-acyclic, we must \emph{supervise the optimizer} to check whether a given subjoin is safe. 
A safe subjoin can be characterized by the following lemma:

\begin{lemma}[\cite{safesubjoin}]\label{lemma:safe}
Let $q$ be an acyclic natural join query. A subjoin $q'$ of $q$ is safe if and only if there exists some join tree of $q$ such that the relations in $q'$ are connected.
\end{lemma}

For the example natural join $q = R(A,B,C) \Join S(A,B) \Join T(B,C)$, there is only one join tree for $q$: $S - R - T$. Hence, both $R \Join S$ and $R \Join T$ are safe subjoins, but $S \Join T$ is not. Using \cref{lemma:safe}, we developed the \SafeSubJoin algorithm to detect whether a subjoin $q'$ is safe. As shown in \cref{alg:SafeSubJoin}, \SafeSubJoin first computes a maximum spanning tree $T'$ for $q'$ using the \TreeStruct algorithm. It then continues to run another instance of \TreeStruct by modifying the initialization step as $T \gets T'$; $\mathcal{R} \gets$ all relations in $q$; $\mathcal{R}' \gets$ all relations in $q'$. \SafeSubJoin returns true if the resulting spanning tree $T$ is a maximum spanning tree of $q$ (i.e., a join tree of $q$).

\SetAlgoNlRelativeSize{0}
\begin{algorithm}[t!]
    \KwIn{natural join $q$, subjoin $q'$}
    \KwOut{True or False}
    $T' \gets$ \TreeStruct($G_{q'}$)\;
    $T \gets$ \TreeStruct($G_{q}$) with the initialization step as:
    $T \gets T'$; $\mathcal{R} \gets$ all relations in $q$; $\mathcal{R}' \gets$ all relations in $q'$\;
    \eIf{$T$ is a maximum spanning tree of $q$}{
        return True\;
    }{
        return False\;
    }    
    \caption{\SafeSubJoin}
    \label{alg:SafeSubJoin}
\end{algorithm}