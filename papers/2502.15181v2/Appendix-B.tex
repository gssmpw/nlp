\section{Additional Robustness Results (Left Deep)}

In Appendix B, we present the distribution of execution times with random left-deep plans for each query of \RPT, compared to our baseline methods: DuckDB, Bloom Join, and \PT. These results are shown in \Cref{fig:tpch-left} (\tpch), \Cref{fig:job-left} (\job), \Cref{fig:tpcds-left-a} (\tpcds query 1-52), \Cref{fig:tpcds-left-b} (\tpcds query 53-99), \Cref{fig:dsb-left-a} (\dsb query 1-52) and \Cref{fig:dsb-left-b} (\dsb query 53-99).

For most acyclic queries, both \RPT and \PT outperform vanilla DuckDB and Bloom Join in terms of robustness. However, for specific acyclic queries (e.g., JOB 32a and 32b, TPC-DS 54 and 83, DSB 54 and 83), \PT is also not robust. This is because the \StoL transfer algorithm used by \PT lacks a theoretical guarantee.

Even for cyclic queries, \RPT can improve robustness to some extent. However, due to the absence of the theoretical guarantee for cyclic queries, \RPT fails to constrain their maximum execution time.

\begin{figure*}[t!]
    \centering
    \begin{subfigure}{0.44\linewidth}
        \includegraphics[width=\linewidth]{./pic/tpch-left-deep-full-version.pdf}
        \caption{Only left deep}
        \label{fig:tpch-left}
    \end{subfigure}
    \begin{subfigure}{0.44\linewidth}
        \includegraphics[width=\linewidth]{./pic/tpch-bushy-full-version.pdf}
        \caption{Bushy}
        \label{fig:tpch-bushy}
    \end{subfigure}
    \centering
    \caption{The distribution of execution time with random left deep plans for each query in \tpch \textnormal{-- Normalized by the execution time of default \duckdb. The figure is log-scaled. The box denotes 25- to 75-percentile (with the orange line as the median), while the horizontal lines denote min and max (excluding outliers). `*' indicates timeouts. Cyclic queries are in red.}}
\end{figure*}

\begin{figure*}[t!]
    \centering
    \includegraphics[width=0.9\linewidth]{./pic/job-left-deep-full-version.pdf}
    \caption{The distribution of execution time with random left deep plans for each template in \job \textnormal{-- Normalized by the execution time of default \duckdb. The figure is log-scaled. The box denotes 25- to 75-percentile (with the orange line as the median), while the horizontal lines denote min and max (excluding outliers).}}
    \label{fig:job-left}
\end{figure*}

\begin{figure*}[t!]
    \centering
    \includegraphics[width=\linewidth]{./pic/tpcds-left-deep-full-version-a.pdf}
    \caption{The distribution of execution time with random left deep plans for each query (1 - 52) in \tpcds \textnormal{-- Normalized by the execution time of default \duckdb. The figure is log-scaled. The box denotes 25- to 75-percentile (with the orange line as the median), while the horizontal lines denote min and max (excluding outliers). `*' indicates timeouts. Cyclic queries are in red.}}
    \label{fig:tpcds-left-a}
\end{figure*}

\begin{figure*}[t!]
    \centering
    \includegraphics[width=\linewidth]{./pic/tpcds-left-deep-full-version-b.pdf}
    \caption{The distribution of execution time with random left deep plans for each query (53 - 99) in \tpcds \textnormal{-- Normalized by the execution time of default \duckdb. The figure is log-scaled. The box denotes 25- to 75-percentile (with the orange line as the median), while the horizontal lines denote min and max (excluding outliers). `*' indicates timeouts. Cyclic queries are in red.}}
    \label{fig:tpcds-left-b}
\end{figure*}

\begin{figure*}[t!]
    \centering
    \includegraphics[width=\linewidth]{./pic/dsb-left-deep-full-version-a.pdf}
    \caption{The distribution of execution time with random left deep plans for each query (1 - 52) in \dsb \textnormal{-- Normalized by the execution time of default \duckdb. The figure is log-scaled. The box denotes 25- to 75-percentile (with the orange line as the median), while the horizontal lines denote min and max (excluding outliers). `*' indicates timeouts. Cyclic queries are in red.}}
    \label{fig:dsb-left-a}
\end{figure*}

\begin{figure*}[t!]
    \centering
    \includegraphics[width=\linewidth]{./pic/dsb-left-deep-full-version-b.pdf}
    \caption{The distribution of execution time with random left deep plans for each query (53 - 99) in \dsb \textnormal{-- Normalized by the execution time of default \duckdb. The figure is log-scaled. The box denotes 25- to 75-percentile (with the orange line as the median), while the horizontal lines denote min and max (excluding outliers). `*' indicates timeouts. Cyclic queries are in red.}}
    \label{fig:dsb-left-b}
\end{figure*}