\section{Background Material}
The proposed method, which selects which cells of a background grid to retain, is related to the computer graphics problem of rasterization~\cite{GPUGemsRaster}. 
Consequently, we first introduce background information about rasterization and anti-aliasing, after which fundamentals about tools employed in this work are introduced, including homology, persistent homology, and winding numbers.

\paragraph{Rasterization.}\label{sec:rasterization}
 \emph{Rasterization} is the process of converting an arbitrary geometry into a grid-based representation.
In traditional computer graphics, the background grid is screen pixels, and the objects are triangles embedded in floating point $\mathbb{R}^2$. 
The problem is to select which color and intensity to display in each pixel. 
\emph{Aliasing}~\cite{crow1977aliasing,mitchell1990antialiasing} is a significant problem in rasterization: 
pixel values are sensitive to the offset, rotation, and size of the objects, as well as which locations within a pixel are sampled.
Consider a non axis-aligned edge shared by a blue and a red triangle.
For a given pixel, if we choose red or blue we get increased contrast but also stair-step patterns called ``jaggies.''
If we choose a purple mixture the image appears smoother, but shading can produce Moir\'e patterns.
% "jaggies" is an old term, 1970's or earlier, and there is no clear origin nor citation for it
Both patterns are glaring to human eyes. 
For small triangles, the pixel topology may not match the triangle topology; see \cref{fig:TopologicalErrorRaster}.
Such topological errors may be visually insignificant, but they can lead to significant errors in simulation results.
% Herein we are concerned with topological aliasing rather than jaggies.  Skip saying that, because resolving pinches looks a bit like smoothing jaggies.

\begin{figure}[!htb]
\centering
\includegraphics[width=0.47\columnwidth]{RasterTriangles.png}
\hspace{0.02\columnwidth}
\includegraphics[width=0.47\columnwidth]{RasterPixels.png}
\caption{Rasterization of triangles into pixels for computer graphics.  Note the pinches from the two left cyan triangles, the archipelago from the lower right pink triangle, and the multitude of additional topological errors in the lower right. Image courtesy Wikipedia \url{https://en.wikipedia.org/wiki/Rasterisation}}\label{fig:TopologicalErrorRaster}
%\url{{https://en.wikipedia.org/wiki/Rasterisation#/media/File:Top-left_triangle_rasterization_rule.gif}}. \label{fig:TopologicalErrorRaster}}
\end{figure}

\paragraph{Topological Anti-aliasing.}\label{sec:antialiasing}
%For volume-fraction meshing, deciding whether to retain a cell is sensitive to the offset and rotation of the objects (as in graphics), and grid size (analogous to pixel density in graphics).
%{\Bd 
As in graphics, volume-fraction meshing is sensitive to the offset and rotation of objects, as well as the grid size (analogous to pixel density in graphics).
%}
An axis-aligned gap is closed or open depending on its size and position relative to the grid; see \cref{fig:parallel_axis_aligned_gap}. A gap smaller than half the grid size is always closed. A gap larger than the grid size is always open. Between these, shifting the grid left will cause the mesh to alternate between closed and open.

In \cref{fig:diagonal,fig:jaggies} we see the aliasing effects of rotations, where the feature is not aligned with the grid.
In \cref{fig:diagonal} the gap is a fixed width, but about the size of the grid cells, leading to the gap being inconsistently resolved as open or closed, with separate sides connected by pinch points.
In \cref{fig:jaggies} we see a similar inconsistency, but exacerbated because the gap width varies.
The boundary lines meet at a sharp angle, so fewer cells are retained as the apex is approached, leading to a chain of small disjoint mesh islands which we call an \emph{archipelago}.

\begin{figure}[b]
\centering
\subfloat
{
\begin{minipage}{0.28\columnwidth}
\centering
\includegraphics[width=\columnwidth]{parallel-gap-small.pdf}
\footnotesize gap$<$grid$/2$ \\ closed
\label{fig:parallel_gap_small}
\end{minipage}
}
\hspace{3pt}
\subfloat
{
\begin{minipage}{0.28\columnwidth}
\centering
\includegraphics[width=\columnwidth]{parallel-gap-big.pdf}
\footnotesize gap$>$grid \\ open
\label{fig:parallel_gap_big}
\end{minipage}
}
\hspace{3pt}
\subfloat
{
\begin{minipage}{0.28\columnwidth}
\centering
\includegraphics[width=\columnwidth]{parallel-gap-medium.pdf}
\footnotesize gap $\in \lbrack\frac{1}{2},1\rbrack$ grid \\ depends
\label{fig:parallel_gap_medium}
\end{minipage}
}
\caption{Small grid-aligned gaps are closed, large gaps are open, and intermediate gaps depend on their offset.}\label{fig:parallel_axis_aligned_gap}
\end{figure}

\begin{figure}[t]
\centering
\subfloat[raw geometry]
{
\includegraphics[trim=36cm 11cm 33cm 10cm,width=0.44\columnwidth,clip]{diagonal-raw.png}
}
\hspace{8pt}
\subfloat[filled cells]
{
\includegraphics[trim=36cm 11cm 33cm 10cm,width=0.44\columnwidth,clip]{diagonal-filled.png}
\label{fig:filled_cells}
}
\caption{This unaligned gap is resolved inconsistently.}
%{\Rd add picture of this separated by anti-aliasing to anti-aliasing or results section. Ensure the two pictures are consistent.}
\label{fig:diagonal}
\end{figure}


\begin{figure}[t]
\centering
\includegraphics[trim=13cm 20cm 14cm 10cm,width=0.98\columnwidth,clip]{jaggies.png}
\caption{Rotational aliasing may cause stair-step patterns, and archipelagos of isolated islands near where two lines meet at a sharp angle. 
Bold-outlined cells are filled, thin are open.}
%{\Rd add picture of islands joined by edges from subgrid sampling to anti-aliasing or results section. Show also templates, final quad mesh. Ensure the two pictures are consistent.}
\label{fig:jaggies}
\end{figure}

To address these undesirable local topological features, a topological anti-aliasing method is defined and coupled with introducing/removing various templated cells, as described in \cref{sec:methods}.
%We introduce techniques to resolve undesirable local topological features. 
The first undesirable feature is pinches, where exactly two grid cells meet at a vertex with no shared edge, or exactly two 3D grid cells meet at an edge with no shared faces. 
These must be removed because the mesh is required to be locally connected face-to-face or disjoint.
(The complement is also connected face-to-face or disjoint.)
%
All other ways in which a mesh can be non-manifold do not occur, because we form the mesh from the union of some cells of a structured grid.
%
Pinches are either separated into different components by removing small elements, or thickened into meeting face-to-face by adding small elements using data from the anti-aliasing framework.
These small elements come from templates that split background grid cells and perform swaps.
%
The second undesirable feature is archipelagos.
%
Our anti-aliasing technique joins some islands with template elements around connecting edges and quads, and removes any small islands that remain.
%
For comparison, Sculpt resolves pinches by adding or removing entire grid cells,
and resolves small components by removing them~\cite{owen2017hexahedral}.

For simplicity we only discuss domains with a single material and only discuss the retained cells.
In principle our anti-aliasing could be extended and applied to multi-material volume-fraction 
meshing~\cite{ZHANG2010405,owen2017hexahedral}.




\paragraph{Homology.}
The topology of a mesh should contain the significant features of a domain for its intended computational analysis.
%An important aspect of domain topology is the number of connected components, the number of holes or loops, and the number of voids in a shape.
Herein, we shall study mesh topology using a cellular complex: nodes are zero-cells, edges are one-cells, faces are two-cells, and volumes are three-cells. 
Specifically, we will make use of simplicial and cubical complexes, in which two-cells are triangles and quadrilaterals, and three-cells are tetrahedra and hexahedra, respectively.
Homology \cite{Hatcher:2001} is a mathematical tool that distinguishes cell complexes using certain algebraic quotient groups. 
The \emph{Betti numbers} $B_{i}$ count the rank of these groups.
Specifically, $B_{0}$ equals the number of connected components, $B_{1}$ is the number of holes, and $B_2$ is the number of cavities or voids.
For planar domains $B_2$ will always be zero.

Persistent homology \cite{Edelsbrunner:2002,Otter:2017} describes homology changes as objects are added and connections are made.
A \emph{filtration} has a ``persistence parameter'' which defines when a cell enters the complex. 
A filtration is monotonic, so no cell may ever leave the complex after entering. 
However, the homology has both additions and removals because adding a cell could, e.g., create a new connected component, or combine two components into one. 
The parameter value at which a group generator is created is called its ``birth,'' while the value it disappears is called its ``death.''
Birth and death coordinates are plotted in a persistence diagram such as \cref{fig:persistence_diagram}.
This not only counts Betti numbers, but tracks individual components and holes.
%For this paper, the persistence parameter will be the volume fraction. 
%However, to compute the persistent homology of an object, typically the domain must be broken into simplicial complexes---beginning with vertices (0-cells), then edges (1-cells), continuing with faces (2-cells), etc.~\cite{moon2019statistical}.
% {\Rd I got the ideas for most of this paragraph from Scott's 2019 paper.}
%When operating on meshes, persistent homology typically assumes input of a simplicial mesh, whose zero-cells are vertices, one-cells are edges, and two-cells are (triangular) faces.


%Taking inspiration from Sculpt, 
%Sculpt's use of volume fraction serves as inspiration for what kinds of information can be kept in/out. 
This work studies the persistent homology of cubical background grids using volume fractions as the persistence parameter.
(In other work the signed distance to a domain boundary was the parameter~\cite{moon2019statistical}.) 
Alternatively, zigzag persistence, which does not require a monotonic filtration 
%because simplices may enter or leave the complex
~\cite{carlsson2010zigzag,dey2024computing}, could be used, but doing so would increase complexity and computational expense.

\paragraph{Winding Numbers.}
Volume fractions may be estimated by sampling points and  counting the fraction of them inside the geometry. 
However, for ``ugly'' geometry, what is ``inside'' may be poorly defined.
The generalized winding number \cite{barill2018fast,Jacobson:2013} overcomes this obstacle; see \cref{fig:jacobson_winding}.
It gives answers identical to ray shooting for watertight domains, and gives answers that humans find both reasonable and intuitive for other domains.
\begin{figure}[!b]
\centering
\includegraphics[width=0.40\columnwidth]{winding-continuous.pdf}
\includegraphics[width=0.58\columnwidth]{winding-disjoint.pdf}
\caption{Winding number point and field values, courtesy Jacobson et al.~\cite{Jacobson:2013} Figures 4 and 6. Used with permission of the Association for Computing Machinery, conveyed through Copyright Clearance Center, Inc.}\label{fig:jacobson_winding}
\end{figure}
In its traditional form, the winding number at a point with respect to a closed curve describes the net number of times the curve encircles the point in the counter-clockwise direction, with negative numbers indicating clockwise encirclement.
The winding number is the integral of the angle of the ray from the point to the curve as it is traversed. 
The generalized winding number extends this definition to sets of open curves.
It yields a continuous value where 0 indicates outside and 1 is inside.
For geometry with gaps, the winding number near a gap is typically between 0 and 1. 
In extreme cases, such as overlapping domain boundaries, invalid geometries may give values beyond $[0,1]$.
In 3D, the winding number integrates the solid angles seen from a point, e.g., for a volume defined by a triangle soup. 
Which normal direction is outward-facing determines the sign of the solid-angle contribution.

% A cell's volume fraction is the mean of the winding numbers of its samples.


%However, the volume fraction is not robust in the presence of faulty data (e.g. where ``interior'' and ``exterior'' are ill-defined). 
%For this purpose, a generalized winding number is used to determine the volume fraction for the background cells. 
%As a result, a generalized winding number \cite{barill2018fast,Jacobson:2013} has been introduced to specify whether a point is in the interior or exterior of a geometry with ``reasonably consistent orientation'' \cite{Jacobson:2013}.
%Given an input of ``reasonably consistent orientation,'' \cite{Jacobson:2013} the winding number uses this orientation to determine if a sample point is in or out. 
%For an embedded and closed geometry, the generalized winding number will equal the traditional winding number, and will be either zero or one---zero being out and one being in.
%For more ill-defined geometries, this number will typically vary continuously between zero or one in a similar manner to determine whether the point is interior or not.
%The filtration of a fixed background grid investigates homological changes as a result of the volume fraction.
%The approximation of the original geometry using the background grid changes as a result.

% Moved to methodology as future work
%Although not shown in this work, to further inform how ``interior'' or ``exterior'' a cell is, we intend to use quadtree type refinement based on the winding number data. The volume fraction for the child cells will be computed using multiple sample points. Given a user defined threshold, the difference in volume fractions will determine if a parent cell is to be refined.

%From this point, the persistent homology is used to inform how the topology changes based on variation in the volume fraction given a fixed mesh size. 

%Persistent homology was previously used for engineering  purposes to predict fluid flow through porous rock~\cite{moon2019statistical}. 
%There, the rock and its voids were represented using voxels, and operated on a cubical cell complex. 
%Similarly, we will compute the dual of our output mesh, which in turn will be used to compute the persistent homology. (
%In contrast to the present work's use of a generalized volume fraction, the persistence parameter for porous rock was signed distance to a rock-void interface.

%\begin{figure}
%\centering
%\includegraphics[width=0.40\columnwidth]{copied-figures/winding-continuous.pdf}
%\includegraphics[width=0.58\columnwidth]{copied-figures/winding-disjoint.pdf}
%\caption{Winding number point and field values, courtesy Figures 4 and 6 from Jacobson et al.~\cite{Jacobson:2013}.\label{fig:jacobson_winding}}
%\end{figure}

