\subsection{Topological Effects of Grid Refinement.}
To conclude the results section, we demonstrate that topological pathologies may arise when naively using persistent homology to find an appropriate grid size to achieve a desired topology.
When features are isolated or globally the same scale, grid refinement has intuitive and predictable topological effects. 
However, general inputs may \emph{not} demonstrate monotonic filtration behavior with grid refinement. 
Counterexamples show non-monotonic filtration behavior by grid size. Discretization by grid cells and their alignment with input features strongly effects topological behavior. Thus algorithm parameters of when to refine the background grid may have unexpected effects on mesh topology. 

\paragraph{Convergence.}
For many inputs, as the grid is refined, topological features of the input are resolved and the output mesh topology becomes stable. 
However, for some inputs, the topology never converges and no filtration is possible.
For some inputs it may be possible to define a filtration, with simplices only appearing, never disappearing. 
If simplices appear and disappear, zig-zag persistence could computationally predict topology. 


In \cref{fig:diagonal,fig:jaggies} a feature is inconsistently resolved due to aliasing effects of unaligned grids. 
For the constant-width gap in \cref{fig:diagonal}, refining or coarsening the grid makes the gap resolved consistently as open or closed.
However, for the variable-sized gap in \cref{fig:jaggies}, global uniform refinement may merely move where the problem occurs. 
The example is a wedge of material bounded by two lines meeting at a small angle $\alpha$ at an apex. 
In locations where the grid size is about the same as the local width, whether a cell is included or excluded can change every few grid cells, leading to many separate connected grid components. 
For any small grid size, there will be some portion of the wedge where the lines are about that  size apart, specifically in the range $[\frac{1}{2},1] \cot \alpha$ squares away from the apex.
The geometry is undesirable because the islands move location. 
The grid topology may be constant over refinement, but that topology differs from the topology of the underlying object and is undesirable.
%\marginnote{scott: removed ``islands'' and reworded. Is it clear now?}
%\marginnote{I do not understand what is being said after the comma in the second-to-last sentence of this paragraph. LaTeX is placing this comment on the wrong side of the page as well...}


\paragraph{Non-convergence.}
In each of the examples in \cref{fig:two_triangles} the output mesh topology does not converge under refinement. That is, there is no grid size below which the output mesh topology does not change.
The background grid is uniform, but we only draw some of the relevant cells at each level of refinement. Blue (closure) cells are mostly material and thus included in the output mesh. Red (gap) cells are unfilled and excluded.
Under refinement, the meshes alternate between one and two connected components ad infinitum.
The grid squares containing the corner alternate between filled and open, because of the corner’s relative position inside its square.
The descriptions of the geometries are finite, just two triangular blocks. It is simple, plausible, and without sharp angles. The only unusual feature is the blocks touch at a single pinch point.


The upper and lower examples in \cref{fig:two_triangles} have alternate sizes of when they are open and closed. If an input has both of these pinch features between two material blocks, then exactly one of the pinches will be closed, giving a mesh with the homology of a disk. It is converged in the sense that the homology does not change under refinement, but the local connectivity does change. Hence, even if we were to use zigzag homology it would not distinguish between this case and a single solid block of material.

\begin{figure}[!htb]
\centering
\includegraphics[width=0.98\columnwidth]{two_triangles.pdf}
\includegraphics[width=0.98\columnwidth]{two_trianglesB.pdf}
\caption{In these counterexamples to convergence, the grid topology alternates between one and two connected components ad infinitum under refinement. The alternations of the upper and lower examples are opposite.}\label{fig:two_triangles}
\end{figure}
 
The analytic description of the geometries in \cref{fig:two_triangles} is two triangular blocks of material with slopes $-3$ and $1$ meeting at a corner. In the upper example, the corner's coordinate is $(\frac{2}{3},0),$ and in the lower it is  $(\frac{1}{3},0).$ 
If we start with a unit grid with a vertex at the origin, then under refinement the grid square containing the corner alternates between having the corner $2/3$ of the way along the bottom edge (blue), and $1/3$ of the way (red). Such blue squares have volume fraction $10/18$ and the red squares $7/18.$ This construction is not tight. The slopes may be different. The corner may lie at some other coordinate, and a grid vertex will never lie on it if its x-coordinate is not $k/2^m$ for some $\{k,m\} \in \mathbb{Z}.$ Thus more complicated sequences than alternating may be constructed.
