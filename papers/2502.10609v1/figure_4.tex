\begin{figure*}[!htb]
\centering
\subfloat[Vol. Frac.: 50\%; $B_0 =  4; B_1 =  2$]{\includegraphics[trim=22.5cm 0cm 25cm 0cm, width=0.6\columnwidth,clip]{vol_0.5.png}}\qquad
\subfloat[Vol. Frac.: 15\%; $B_0 =  1; B_1 = 7$]{\includegraphics[trim=22.5cm 0cm 25cm 0cm, width=0.6\columnwidth,clip]{vol_0.15.png}}\qquad
\subfloat[Vol. Frac.: 1\%; $B_0 = 1; B_1 =  12$]{\includegraphics[trim=22.5cm 0cm 25cm 0cm,width=0.6\columnwidth,clip]{vol_0.01.png}}
%\newline
%\subfloat[Vol. Frac.: 0]{
%\includegraphics[trim=0cm 0cm 0cm 0cm, width=0.6\columnwidth, clip]{sections/latex_figures/120x120_Chesapeake/vol_0.0.png}}\qquad
%\subfloat[Vol. Frac.: 0]{
%\includegraphics[trim=5cm 5cm 5cm 5cm, width=0.6\columnwidth, clip]{sections/latex_figures/120x120_Chesapeake/vol_0.0_.35.pdf}}

\caption{Chesapeake Bay meshes and their Betti numbers change as the volume-fraction threshold is lowered. 
%Left, cells with volume fraction $\ge 0.5$ in the bay. Center, threshold $\ge 0.15.$ Right, all cells. 
The model contains deliberate errors: gaps, overlaps, and offsets. 
As a result of these errors, the winding numbers of sampled points may be positive in regions that are clearly inland, as in the lower-left region of the rightmost figure.
Also note how rasterization affects the thin bays and peninsulas, where here we have skipped anti-aliasing.}
\label{fig:volfrac_chesapeake}
\end{figure*}

\begin{figure}[!htb]
\includegraphics[width=\columnwidth,trim = 0cm 0cm 0cm 1cm, clip]{persistence_diagram.png}
\caption{The persistence diagram of the Chesapeake Bay grid, with parameter 1 minus the volume fraction.}\label{fig:persistence_diagram}
\end{figure}
