\section{Introduction}
Getting geometry that is suitable for mesh generation is often more difficult and time consuming than creating a mesh from that geometry.
``Ugly'' geometry is ubiquitous ``in the wild.'' 
Industrial and commercial CAD data are often hand-designed ``blueprints'' to guide assembly, and do not represent the as-built part. 
Data from imaging and segmentation may have topological inconsistencies.
%{\Bd 
Even in cases with valid geometry and topology, analysts must carefully review, modify, and defeature models based on the intended purpose of the mesh because typical techniques  generate meshes whose topology and geometry match the input models.
%}
%Thus the analyst must carefully review and modify the input based on the intended purpose of the mesh.
Common issues in ``ugly'' geometry are gaps and overlaps,  features smaller than the desired mesh size, topological complexities such as small holes, and small angles and thin regions that would produce poor-quality elements.

However, the mesh is a discrete approximation to the geometry. Why require a higher fidelity in the input than is aspired to in the output?  Indeed, the community is developing tools to mesh ugly geometry, robustly producing meshes that are topologically correct and have high-quality elements despite topological and geometrical defects and small features in the input.

Sculpt~\cite{owen2012parallel,owen2015evaluation,sculpt} is one such tool, achieving a hexahedral mesh of reasonable quality, but reconstructing an approximation of the input geometry and topology.
Inexact reconstruction is a \emph{benefit} in the case of gaps, overlaps, and small features. 
The Sculpt algorithm starts with a background grid overlaying the input geometry. 
The fraction of each grid cell that lies inside the geometry of an input material is its \emph{volume fraction}.
Cells with volume fractions above a threshold (e.g., one-half) are retained; the rest are discarded. 
Heuristics remove undesirable  topology such as pinch points and components consisting of only a few cells. Retained cells are then snapped to the geometry, and mesh quality is achieved through pillowing~\cite{mitchel1995pillowing,10.1007/978-3-540-87921-3_28,ZHANG2010405}, smoothing~\cite{Knupp:2012:TMOP}, and other changes to mesh topology and node positions.
%
Similarly, Morph~\cite{morph_noble_2,morph_staten_1} is a parallel tet mesher using a background grid that snaps nodes to geometry based on dimension, proximity, and how other nodes are snapped.
When no suitable node snap is found, Morph adds new nodes at the intersections with the geometry to produce nodes on the geometry boundary. 
In both Sculpt and Morph, the size of the background grid indirectly determines the geometric fidelity of the output to the input.
TetWild~\cite{10.1145/3386569.3392385,10.1145/3197517.3201353} uses a Delaunay triangulation rather than a background grid. Triangles representing the geometry are incrementally inserted and the mesh is refined. Edge length and geometric proximity parameters control the mesh resolution and geometric fidelity.


Though these tools always produce meshes with valid topology, there is no a priori knowledge of what the homology of the output will be, nor how it will compare to the input topology or the desired topology.
For some downstream operations, small holes and features play a critical role in final results, while for others, these same holes and features are extraneous, may lead to overly dense meshes, and must be (manually) removed.
In sum, a single model may require multiple representations depending on its intended purpose, each with varying mesh sizes and topological needs.
However, tools with control over how to select the appropriate topology of the mesh for its intended purpose are in short supply, meaning that much of this work is deferred to time-consuming manual manipulation by engineers.

%{\Rd Motivation
%\begin{itemize}
%	\item Mesh generation process fraught with challenges, particularly when dealing with messy geometry
%	\item Industrial and commercial data often with geometric gaps and overlaps that the analyst must carefully review to prescribe the right behavior for the right analysis
%	\item Frequently this behavior even changes between analyses
%	\item Engineer is aware of the relevant element size, features needed, number of parts for an analysis, but must clean up messy data to get desired input
%	\item Features and parts often take the form of loops and components, which can be described using homology
%	\item The purpose of this work is explore how to extract a valid quadrilateral mesh given a desired number of elements and an a priori known topology using concepts from persistent homology
%	\item This deviates from traditional paradigm, where one assumes a valid geometry and generates a mesh to match that topology
%	\item Instead, here we assume that the geometry likely may be invalid in one or more places, but work to generate a meaningful mesh given the analyst's expert knowledge of the desired homology
%	\item We develop a planar two-dimensional prototype of this technique in Rhinoceros and demonstrate that it can effectively be used to help determine topology.
%	\item We demonstrate the efficacy of the tool on both clean and messy datasets by using a generalized winding number \cite{Jacobson:2013} and the concept of volume fraction, similar to Sculpt, to create quadrilateral meshes from messy data.
%	\item After generating topologically appropriate mesh, this would then be used as input to extract a mesh with higher geometric fidelity for use in analysis 
%\end{itemize}
%}
%
%\begin{enumerate}
%	\item We have created a version of sculpt in 2D
%	\begin{itemize}
%		\item Creates a background grid bounding the object to be sculpted
%		\item Using winding numbers, checks if a user defined number of points in each background grid cell are in or out of the object being sculpted
%		\item Computes the volume fraction of each background grid cell from winding number output
%		\item Includes background grid faces in the final mesh if the volume fraction is above or equal to a user defined cut off
%		\item Finally creates a sculpted mesh of the original object
%	\end{itemize}
%	\item The dual mesh of the sculpted mesh is created
%	\item The dual mesh is input into aleph to compute persistent homology
%	\item In theory, by iterating and changing background mesh refinement we can use the persistent homology to tell us important points where the mesh changes
%	\item Counter examples
%	\begin{itemize}
%		\item The homology does not always behave monotonically
%		\item Example of oscillations between in and out of face
%		\item Sharp angles
%		\item Offsetting and location/orientation of bounding box
%		\item Other examples
%	\end{itemize}
%\end{enumerate}


Herein, we explore how to robustly predict and achieve the desired mesh topology for algorithms based on background grids and volume fractions (including Sculpt) 
%{\Rd
 through the use of persistent homology and generalized winding numbers.
% }
%We use volume fraction techniques, specifically Sculpt and variations.


To accomplish this goal, \briancode{}---a prototype mesher now available on GitHub \cite{tusqh_github:2025}---was developed in Rhinoceros 3D.
It is a testbed for research and demonstrates that our techniques are effective.
\briancode{} mimics the initial steps of Sculpt, using a background grid and volume fractions to decide which grid cells to retain.
As with TetWild, it uses generalized winding numbers~\cite{barill2018fast,Jacobson:2013}
to define the ``interior'' for valid and invalid geometries.
We explore the topological structure of potential meshes under different volume-fraction thresholds using persistent homology~\cite{Edelsbrunner:2002,Otter:2017}.
Local connectivity decisions based on sub-sampling volume-fractions mitigate the effects of the arbitrary orientation and offset of the background grid.
This enables the analyst to measure and select the desired mesh topology, which then informs which volume-fraction threshold to select. 
The user may also adjust the volume-fraction threshold on a sliding scale and visualize the choice of meshes. 
These meshes can serve as input for subsequent steps to improve geometric fidelity, such as Sculpt's snapping, pillowing, and smoothing.
Concluding theoretical results demonstrate that obvious applications of grid-based volume-fraction methods cannot guarantee consistent topological output.



% terminology
% jaggies -> aliasing 
% topological antialiasing
% pinch antialiasing, shrinking creates a straight, thickening creates an isthmus
% archipelago antialiasing (disconnected components), either bridging between them, or sinking them.

