\section{Conclusion}

In this work, a mesh generation framework is developed to facilitate the creation of meshes on potentially dirty geometry based on user-specified needs through the use of persistent homology.
The framework is built on a volume-fraction based meshing method, similar to Sculpt, in which mesh elements are considered ``in'' or ``out'' of the given geometry depending on whether they have volume fraction above a cutoff range; the most desirable mesh can then be selected based on the  appropriate homological structure induced by this volume fraction.
%The work additionally explores topological anti-aliasing methods to mitigate the undesirable artifacts of selecting a given background grid size with a given shift and orientation.
%We demonstrate both computationally and theoretically that the proposed anti-aliasing method, called subgrid sampling, as well as the use of templates to mitigate these effects, produces meshes with better topological behavior meshes without these methods.
%Results are presented in both two and three dimensions.
These volume-fraction based algorithms, including Sculpt, behave quite differently than boundary-fitted algorithms, such as Delaunay Refinement, when the local mesh size is decreased. 
 For boundary-fitted algorithms, reducing the mesh size to the local feature size or less allows the mesh to have good quality and recover the input topology exactly. 
 For the family of volume-fraction codes, the mesh quality is good regardless of the local feature size, but in some cases the topology does not converge as the mesh size decreases by subdividing the background grid. 
 However, for a fixed grid, we may predict the resulting mesh topology, measure how that changes as we vary the volume-fraction threshold using persistent homology, and select an appropriate topology for the mesh's intended purpose. 
When the background grid is about the same size as gaps and thin-region features, the alignment and orientation of the background grid with respect to the features can create aliasing artifacts in the mesh.
There may be spurious pinches and archipelagos.
A gap or thin region may be resolved inconsistently in different locations.
Subgrid sampling provides guidance on whether to connect, separate, or remove components.
These connections and separations can be achieved using templates of small elements. 
We have demonstrated theoretically and experimentally that subgrid sampling can mitigate the effects of aliasing, making connections more consistent.
The software, \briancode{,} demonstrates the potential of the meshing framework in both two and three dimensions, and is planned for open source release.
As a counterpoint we have theoretical analysis showing that for any volume-fraction threshold we choose, there are cases where aliasing artifacts will still arise.


There are a number of avenues for future work to build on and improve this framework.
First, it may be valuable to have adaptive, non-uniform background grids and volume fraction thresholds to mesh more interesting geometries at reduced expense.
However, doing this may require the use of zigzag persistence to accurately capture mesh topology, particularly when refinement does not converge on separation or closure of local mesh features.
Further work should also better utilize fitting algorithms, such as those present in Sculpt, to more accurately fit to the geometric input data that \briancode{} approximates.
To do this more flexibly, additional research should generalize these methods to unstructured cubical complexes.
Finally, it is anticipated that for this framework to be usable in practice, it would need to be redeveloped using C++.
%\begin{itemize}
%\item Adaptive, non-uniform volume fraction thresholds.
%\item Adaptive, non-uniform background grids, e.g., quadtrees.  
%\item Zigzag persistence and anti-aliasing for grid refinement.
%%
%{\Rd
%\item Translation of prototype to C++ ?
%\item Building a user friendly pipeline to connect the whole process?
%}
%\end{itemize}
%

%Accomplishments
%{\Rd
%\begin{itemize}
%	\item made 2D prototype of meshing technique to create mesh based on desired topology of analyst
%	\item Persistence and winding numbers allow it to operate in the presence of messy data
%	\item Geometry would then be better fit, after manner similar to Sculpt
%	\item future work
%	\begin{enumerate}
%		\item Extend to three dimensions
%		\item Integrate with sculpt geometry features for valuable output
%		\item leverage more powerful persistent homology toolsets, such as Aleph
%	\end{enumerate}
%\end{itemize}
%}
%
%\appendix[Proofs for anti-aliasing]
%might be only in arxiv version if it makes us go over 12 pages