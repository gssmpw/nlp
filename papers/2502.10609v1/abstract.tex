\begin{abstract} \small\baselineskip=9pt 
%% \marginpar{\Rd I feel like this is a little misleading. It is my understanding that it could be robust, but that it is not currently incorporated into Sculpt.}
%% This contribution is removed from the Sculpt3d accomplishments. We can discuss the nuances of its status later.
%% this is implemented in Sculpt
%% Sculpt meshing is now robust for non-watertight geometry. 
%% This is achieved by using the generalized winding number, rather than ray-shooting, for inside/outside queries. 
%%
%For meshing algorithms based on volume fractions (e.g., Sculpt), subdividing the background grid often improves the topological and geometrical fidelity of the output mesh to the input geometry. Unfortunately, in some simple 2D examples, topology does not converge under refinement.
%% The inclusion of subcells in the output mesh does not behave monotonically under refinement.
%%
%Fortunately, for a fixed grid size, the mesh topology changes predictably as the volume-fraction threshold is varied.
%The threshold is a persistence parameter for a monotonic filtration of mesh cells.
%Thus we can use persistent homology to efficiently predict or control mesh topology.
%We may resolve small local topological features, such as pinches, by adding or removing templates of small elements, so that elements meet face-to-face or are disjoint. 
%Deciding when to disconnect elements is based on volume fractions and subgrid sampling to mitigate the aliasing effects of grid alignment.
%These decisions are included in our persistent homology predictions.
%We demonstrate this on geographical, mechanical, and graphics models through our volume-fraction code \briancode\ in 2D and 3D.
%\briancode\ is robust for non-watertight geometry by using the generalized winding number for inside-out queries.
%%
%{\Rd
%{\Bd 
This work develops a framework to create meshes with user-specified homology from potentially dirty geometry by coupling background grids, persistent homology, and a generalization of volume fractions.
%}
For a mesh with fixed grid size, the topology of the output mesh changes predictably and monotonically as its volume-fraction threshold decreases.
Topological anti-aliasing methods are introduced to resolve pinch points and disconnected regions that are artifacts of user choice of grid size and orientation, making the output meshes suitable for downstream processes including analysis.
The methodology is demonstrated on geographical, mechanical, and graphics models in 2D and 3D using a custom-made software called \briancode{.}
The work demonstrates that the proposed framework is viable for generating meshes on topologically invalid geometries and for automatic defeaturing of small geometric artifacts.
Finally, the work shows that although subdividing the background grid frequently improves the topological and geometrical fidelity of the output mesh, there are simple 2D examples for which the topology does not converge under refinement for volume-fraction codes.
%}
\end{abstract}
