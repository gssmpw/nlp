\section{Appendix: Anti-Aliasing Proof}

The following result is proved in this appendix.

\begin{theorem}
Given a rectangular lattice in $\mathbb{R}^2$ with characteristic length $\ell$ overlaying two parallel half-spaces separated by a length of $L$, topological rasterization may occur when $\ell(\sqrt{2}-1)<L\leq\ell$.
For the subgrid sampling scheme proposed in this text, topological rasterization may only occur when $\ell(\sqrt{2}-1)<L<\frac{\sqrt{2}\ell}{2}.$
Finally, topological rasterization due to changes in orientation cannot be resolved for $L$ such that $\ell(\sqrt{2}-1)<L<\frac{\ell}{2}.$
\end{theorem}

Our analysis proceeds in the context of \cref{fig:parallel_axis_aligned_gap,fig:diagonal}.
Assume that two connected components of a 2D domain are parallel to each other and separated by a gap of length $L$.
For the sake of clarity and simplicity, assume both components of the domain are half-spaces, and that as a result, $L$ is the only length of consequence for the geometry to be meshed.
Call the half-spaces $\Omega$ and the complement $\Omega^c$.
Without loss of generality, assume that centerline of $\Omega$ and $\Omega^c$ is the $y$-axis.
Assume a rectangular lattice (``background grid'') tiling the Euclidean plane, rotated by an arbitrary angle and shifted by an arbitrary translation.
Let us assume that lengths of the lattice are equal in both principal directions (e.g. the $x$ and the $y$ direction if no lattice rotation)---else a rotation by 90 degrees will shift biases introduced by anisotropy.
Call this length $\ell$.

By definition of the centroid of a domain, a cell of the lattice, $C$, will intersect $\Omega^c$ maximally if its centroid intersects the centerline of $\Omega^c$.
This is the limiting case for potential rasterization.
% Scott: I don't know what this means. If it is unimportant, let's leave it off: %(and axis of symmetry by 180 degrees) 
As a result, we investigate rotations of the lattice translated such that a cell $C$ intersects $\Omega^c$ on its centerline.
There are three scenarios describing the domain $C\cap \Omega^c$.
\begin{enumerate}
	\item The boundary of $\Omega$ is parallel to $C$, in which case $C \cap \Omega^c$ is a rectangle. % do we need this case? special case of a parallelogram.
	\item The boundary of $\Omega$ intersects $C$ at edges on opposite sides of $C,$ in which case $C\cap \Omega^c$ is a parallelogram. %do not share a common vertex
	\item The boundary of $\Omega$ intersects $C$ at edges that share a common vertex, in which case $C \cap \Omega^c$ is a hexagon.
\end{enumerate}
The first of these cases is a special case of the second, but is particularly important for the analyses and so it has been called out particularly.

Let a circle of radius $r$ be defined as the set of points traced by all potential rotations of $C$ about the centroid of $C$.
By definition
\begin{equation}
	r=\frac{\sqrt{2}\ell}{2}.
\end{equation}
Define the function $p(\theta)$ as the position of a vertex of cell $C$ on this circle at specified radian, $\theta$
\begin{equation}
	p(\theta) = r \big(\cos(\theta),\sin(\theta)\big).
\end{equation}
Due to symmetry of the geometry, it is sufficient to analyze $\theta \in [\frac{5\pi}{4},\frac{3\pi}{2}]$.
Consequently, all following results assume $\theta$ in this range.
This configuration setup is depicted in \cref{fig:proof_pic}.

The equations of the line passing through points $p(\theta)$ and $p(\theta + \frac{\pi}{2})$ can be solved to yield
\begin{equation}
	y(x;\theta) = \frac{r - x \cos(\theta) + x \sin(\theta)}{\cos(\theta) + \sin(\theta)},
\end{equation}
\begin{equation}
	x(y;\theta) = \frac{r - y \cos(\theta) - y \sin(\theta)}{\cos(\theta) - \sin(\theta)}.
\end{equation}
Note that the function $x(y;\theta)$ is ill-defined at $\theta=\frac{5\pi}{4}$.

\begin{figure}
\centering
	\subfloat[Cell configurations]{
		\includegraphics[width=1\columnwidth]{Slide2.png}
			\label{fig:proof_pic_setup}
	}
	\qquad
	\subfloat[Area configuration 1]{
		\includegraphics[width=1\columnwidth]{Slide3.png}
			\label{fig:proof_pic_parallelogram}
	}
	\qquad
	\subfloat[Area configuration 2]{
		\includegraphics[width=1\columnwidth]{Slide4.png}
			\label{fig:proof_pic_triangle}
	}
	\caption{Configurations are shown for maximal intersection of a square with arbitrary orientation centered between two half-spaces.
	Measurements are also depicted which define how to arrive at computed areas.}
	\label{fig:proof_pic}
\end{figure}

Using the notation from \cref{fig:proof_pic_parallelogram}, the area of $C$ intersecting one half-space of $\Omega$ in the case where $C\cap\Omega^c$ is a parallelogram is 
\begin{align*}
	A_1 &= \frac{1}{2}h_\Delta b_\Delta + \ell_{||} h_{||}\\
		&= \frac{1}{2}\big(r (\cos(\theta) - \sin(\theta))\big)\frac{-2r}{\cos(\theta)+\sin(\theta)} + \\
		& \qquad\big(-\frac{L}{2}-r\cos(\theta)\big)\frac{-2r}{\cos(\theta)+\sin(\theta)}\\
		&= \frac{r \big(L + r \cos(\theta)  + r\sin(\theta)\big)}{\cos(\theta) + \sin(\theta)} 
\end{align*}
Because $\theta \in [\frac{5\pi}{4},\frac{3\pi}{2}]$, the denominator is always positive.
Similarly, using the notation from \cref{fig:proof_pic_triangle}, the area of $C$ intersecting one half-space of $\Omega$ in the case where $C\cap\Omega^c$ is a hexagon is
\begin{align*}
	A_2 &= \frac{1}{2}h_\Delta b_\Delta \\
	     &= -\big(\frac{L}{2} + r\sin(\theta)\big)^2\sec(2\theta)
\end{align*}
Note that the transition from the first case to the second case occurs when $\ell_{||} h_{||} = 0,$ or in other words when $\frac{L}{2} = -r \cos(\theta) = -\frac{\sqrt{2}\ell}{2} \cos(\theta).$
When $\frac{L}{2}$ is less than this term, the area of intersection will be $A_1$, while when $\frac{L}{2}$ is greater than the term, the area of intersection will be $A_2$.
 
 Taking the derivative of $A_1$ with respect to $\theta$ and setting equal to zero, it is found that $A_1$ is maximal when $L=0$ (i.e. no gap between half-spaces) and when $\theta = \frac{5\pi}{4}$.
 Furthermore, taking the derivative of $A_2$ with respect to $\theta$ and setting equal to zero, critical points of $A_2$ are found to be at $\cos(\theta) = 0, \sec(2\theta) = 0, \frac{L}{2} = -r \sin(\theta), $ and $L = -r \csc(\theta)$.
 The first only occurs when $\theta = \frac{3\pi}{2}$ and the second is never zero and is bounded away from $\theta=\frac{5\pi}{4}$ because $A_2$ can never be the correct area of choice when $\theta=\frac{5\pi}{4}$.
 The third results in $A_2$ equaling zero, and is a minimum area.
 Finally, the fourth introduces a critical point at $\theta = -\csc^{-1}\big(\frac{L}{r}\big)$ provided $\frac{L}{r} = \frac{\sqrt{2}L}{\ell} \geq 1$.
Particularly, $\theta = \frac{3\pi}{2}$ is a local minimum when $L \leq \frac{\sqrt{2}\ell}{2}$ but a local maximum when $L > \frac{\sqrt{2}\ell}{2}$, and $\theta = -\csc^{-1}\big(\frac{L}{r}\big)$ is a local minimum when $L \geq \frac{\sqrt{2}\ell}{2}$.

The above mentioned formulas can be used to show that $A_1\Big|_{\theta = \frac{5\pi}{4}} \geq A_2\Big|_{\theta = \frac{3\pi}{2}}$ for $\frac{L}{2} \leq (2-\sqrt{2})r$ and $A_1\Big|_{\theta = \frac{5\pi}{4}} \leq A_2\Big|_{\theta = \frac{3\pi}{2}}$ for $\frac{L}{2} \geq (2-\sqrt{2})r$.
Particularly, when $\frac{L}{2} < \ell(\sqrt{2}-1)$, the area of $C\cap\Omega$ oriented at $\theta = \frac{5\pi}{4}$ is larger than that when $\theta = \frac{3\pi}{2}$, while the converse is true when $\frac{L}{2} > \ell(\sqrt{2}-1)$.

Now, assume that a persistence parameter, $\phi$ defined by the volume fraction of a square cell, and assume that any cell, $\hat{C},$ with $\phi(\hat{C}) \geq \frac{1}{2}$ is present in the cell complex.
When $L = \ell$ and $\theta = \frac{5\pi}{4}$, depending on the translational positioning of $L$, either all cells $C$ will contain at least 50\% volume fraction (if cell boundaries are aligned to the centerline of $\Omega$) or a vertical line of cells will have volume fraction less that 50\% to potentially 0\%.
For this same orientation, $\theta$, this potential for rasterization will persist until a length $L = \frac{\ell}{2}$, at which point all cells must have at least 50\% volume fraction and will all be contained in the cell complex.
Nonetheless, for any $\theta > \frac{5\pi}{4}$, cells $C$ with centroid on the centerline of $\Omega$  will have volume fraction less than 50\% and thus will not be in the complex.
However, when $L = \ell(\sqrt{2}-1), A_2\Big|_{\theta = \frac{3\pi}{2}} = \frac{\ell^2}{4}$ so all cells $C$ will be contained in the cell complex for all $\theta \in [\frac{5\pi}{4},\frac{3\pi}{2}]$.
Thus, for a gap of length $L$ and a lattice of length $\ell$, the topology of the domain in the lattice cannot be definitively resolved for $\ell(\sqrt{2}-1) < L \leq \ell$.

This above assessment was based on the fact that the maximal area of $C\cap\Omega^c$ would be found when the centroid of $C$ intersected the centerline of $\Omega$.
Similarly, the minimal area of $C\cap\Omega^c$ will be found when the intersection of the centerline of $\Omega$ and $C$ only contains points on the boundary of $C$.
For a fixed $L,$ the maximal intersection will be when $C\cap\Omega^c$ applies when given an orientation of $\theta = \frac{5\pi}{4}$, and the minimal for an orientation of $\theta = \frac{3\pi}{2}$.

Now, consider the proposed topological anti-aliasing technique, where volume fractions for a lattice are defined not only for 2-cells, but also for 1-cells and 0-cells by shifting the lattice by $\frac{\ell}{2}$ in the vertical or horizontal direction for 1-cells and the vertical and horizontal directions for 0-cells.
Assume that $L = \frac{\sqrt{2}\ell}{2} + \epsilon$ for some $\epsilon>0$ whose limit approaches zero.
Assume that $C$ intersects the centroid of $\Omega$.
Then the volume fraction of $C$ is less than 50\% by necessity.
Furthermore, because $C$ intersects the centerline of $\Omega$, a translation of $C$ in one of its principle directions that is also in the positive $y$ direction must also intersect the centerline.
Consequently, it, too, must have volume fraction of less than 50\%.
Proceeding inductively ensures separation of the two half-spaces in the upward direction from $C$, and a similar result can be shown for translations in the downward direction.
By so doing, it is seen that the anti-aliasing scheme correctly separates the two half-spaces.

Finally, for $L = \frac{\sqrt{2}\ell}{2}$, if $C$ intersects the centerline of $\Omega$ and has volume fraction equal to 50\%,  then because of the length of $L$ relative to $\ell$ the orientation of the lattice must be rotated by $\frac{\pi}{4}$ relative to that of the half-spaces.
A translation of $C$ by $\frac{\ell}{2}$ will yield a cell whose centroid lies exactly on the centerline of $\Omega$, and, by earlier analysis, has volume fraction that is less than 50\%.
Call this cell $D$.
Translations of $D$ in both principle directions by $\frac{\ell}{2}$ will yield cells that remain on the centerline of $\Omega$, and thus continue to have volume fractions less than 50\%.
Consequently, the anti-aliasing scheme proposed also separates the two half-spaces.
Rearranging the defined equalities yields the advertised result.



