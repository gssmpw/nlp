Multi-agent path planning is a critical challenge in robotics, requiring agents to navigate complex environments while avoiding collisions and optimizing travel efficiency. This work addresses the limitations of existing approaches by combining Gaussian belief propagation with path integration and introducing a novel tracking factor to ensure strict adherence to global paths. The proposed method is tested with two different global path-planning approaches: rapidly exploring random trees and a structured planner, which leverages predefined lane structures to improve coordination. A simulation environment was developed to validate the proposed method across diverse scenarios, each posing unique challenges in navigation and communication. Simulation results demonstrate that the tracking factor reduces path deviation by 28\% in single-agent and 16\% in multi-agent scenarios, highlighting its effectiveness in improving multi-agent coordination, especially when combined with structured global planning.

% Multi-agent path planning is a key challenge in robotics, requiring agents to navigate complex environments, avoid collisions, and optimize travel efficiency. This work tackles the limitations of existing approaches by combining Gaussian belief propagation with two global path-planning methods: rapidly exploring random trees and structured planner, improving navigation efficiency and collision avoidance. 
% The introduction of a novel tracking factor ensures strict adherence to the global paths. A simulation environment has been developed to validate the proposed method in diverse scenarios, each presenting unique challenges in navigation and communication. Simulation tests show that the path deviation is reduced by 28\% in single-agent and 16\% in multi-agent scenarios.


% This work addresses limitations in multi-agent path planning systems, particularly in complex environments. It enhances Gaussian Belief Propagation by integrating a tracking factor that ensures strict adherence to global paths. The proposed approach is evaluated using both Rapidly-exploring Random Trees and a structured planning method, demonstrating improved navigation efficiency and collision avoidance. The tracking factor reduces path deviation by up to 28\% in single-agent scenarios and 16\% in multi-agent scenarios. While the random nature of the Rapidly-exploring Random Trees leads to an increase in inter-robot collisions, the structured planner eliminates collisions entirely, highlighting the benefits of path tracking combined with structured global planning for multi-agent coordination.

% This work addresses the limitations of current multi-agent path planning systems, especially in complex environments. The research focuses on enhancing Gaussian Belief Propagation with global path planning capabilities. We introduce a novel approach that integrates a tracking factor and global path planning to improve navigation and collision avoidance. The tracking factor ensures adherence to planned paths, while global path planning optimizes overall route efficiency. Despite challenges such as increased collisions due to the random nature of the Optimal Rapidly Exploring Random Trees algorithm and the introduction of the tracking factor, our approach demonstrates significant potential. The tracking factor effectively decreases the path deviation, proven both in solo scenarios by up to 28\% and multi-agent scenarios by up to 16\%, even though it has an adverse effect on interrobot collisions. However, in both cases, the global planning integration performs as expected.