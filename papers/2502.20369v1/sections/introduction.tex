\section{INTRODUCTION}

The evolution of automation has played a fundamental role in advancing both industry and society since the Industrial Revolution. As technology continues to progress, autonomous systems are reaching new levels of complexity and functionality. These advancements are particularly evident in sectors such as logistics, warehouse management, and autonomous transportation, where multi-agent systems are increasingly employed to improve operational efficiency. For instance, autonomous vehicles demonstrate the potential to enhance traffic flow through coordination and communication between vehicles. In constrained environments, such as warehouses or bounded road networks, multi-agent systems must operate efficiently, avoiding deadlocks and collisions. Enhancing the capabilities of these systems will enable future technologies to perform more effectively, conserving both time and resources. 

Effective multi-agent path planning is central to the success of these systems, which optimizes the routes taken by individual agents to achieve their objectives while avoiding collisions. This necessitates sophisticated approaches to coordination, communication, and collision avoidance, ensuring that agents can navigate their environments without interfering with each other. 

Current systems often encounter difficulties in collision avoidance and efficient navigation in such settings. In particular, the field of \ac{GBP} collaborative planning lacks the ability to ensure strict adherence to the global path, which is crucial for optimizing multi-agent coordination over longer routes. While global path planning provides a high-level framework for efficient navigation, maintaining adherence to these paths in dynamic, multi-agent environments introduces significant challenges.

This work addresses the limitations of existing multi-agent path planning systems, particularly in complex and constrained environments. A novel approach that integrates a tracking factor is proposed, to enhance the overall efficiency of multi-agent navigation. The tracking factor ensures precise adherence to pre-determined global paths, allowing the system to optimize navigation efficiency and collision avoidance in real time.
\begin{figure}[t]
    \centering
    \includegraphics[width=0.98\linewidth]{img/front-page-image.pdf}
    \caption{Illustration of the Complex environment with multiple crossings and bends. Larger spheres represent agents, while the chains denote variables in agents' factor graphs and the solid lines indicate agents' traveled path.}
    \label{fig:abstract}
\end{figure}
\medbreak
\noindent The key contributions of this study are as follows: 
\begin{enumerate}
    \item Global path planning has been integrated into the \ac{GBP} Planner~\cite{GBPPlanner}, enhancing multi-agent efficiency and coordination in complex environments. 
    \item A novel tracking factor is introduced to ensure precise path adherence, improving navigation accuracy and reducing deviations in multi-agent planning. 
    \item A multi-agent \ac{GBP} simulation framework\footnote{All source code for this work is made publicly available on \href{https://github.com/au-Master-Thesis/magics}{GitHub} under the MIT license.} has been developed to validate and refine the proposed \ac{GBP} algorithms across diverse scenarios, offering a foundation for further research and optimization in multi-agent systems.
\end{enumerate}
\newpage
