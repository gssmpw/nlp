\section{CONCLUSIONS}
In this paper, we introduced a novel tracking factor within a \ac{GBP} framework to enhance multi-agent path planning. By incorporating this tracking factor, robots are able to adhere more strictly to global paths, significantly improving navigation efficiency and collision avoidance. Our approach was tested with two global planners: RRT* and \ac{SP}. While the random nature of RRT* resulted in increased inter-robot collisions due to uncoordinated path intersections, the structured planner combined with the tracking factor completely eliminated collisions, highlighting the advantages of path tracking combined with structured global planning. The tracking factor proved especially effective in reducing path deviation and improving multi-agent coordination by ensuring that robots remained aligned with the desired paths even in complex environments. In multi-agent scenarios, this approach demonstrated a 16\% improvement in path deviation, further emphasizing its potential to promote smoother and safer navigation.

% Future work will focus on applying the tracking factor in real-world scenarios, addressing the challenges of tuning the factor for dynamic environments, and exploring automated strategies for balancing path adherence with adaptive navigation.
% In the next development phase, the system will be implemented and tested on physical robots in various real-world scenarios to validate its performance beyond simulated environments and address any unforeseen challenges. Additionally, further research will focus on exploring automated or adaptive methods for tuning factor certainties in real time, particularly for balancing the tracking factor with other constraints. This may involve the application of rule-based or learning-based techniques to optimize the balance between path adherence and collision avoidance dynamically, based on the robot's perceived view of its surroundings at any given moment.

%As the immediate next step, we intend to implement and test the system on physical robots in various real-world scenarios to validate its performance outside of simulated environments and address any unforeseen challenges. Another promising avenue to explore is to research an automated or adaptive method for tuning factor certainties online, particularly for balancing the tracking factor with other constraints. This could involve rule or learning-based techniques to optimize the balance between path adherence and collision avoidance dynamically, given a robot's perceived view of its surrounding environment, at the current moment.


% \vspace{5em}
% to extend our method to problems with unreliable commu-
% nication and task uncertainty, which are common features
% of the target applications. Further, we plan to implement
% the CAPAM-TD models on a more realistic multi-robot
% simulation environment, thereof transitioning to deployment
% and testing over physical testbed

% By addressing these areas, future work can build upon the foundations laid in this study to create more robust, efficient, and adaptable multi-agent path planning systems.

% \kristoffer{Do we need something about future work?}
