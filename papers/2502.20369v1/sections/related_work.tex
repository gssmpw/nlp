\section{RELATED WORKS}
% The challenge of multi-agent path planning in complex environments has driven significant research, leading to the development of distributed approaches that emphasize computational efficiency and robustness.

% \Ac{NMPC} has been effectively applied in distributed multi-agent systems to address the challenges of asynchronous communication and message delays. The use of \ac{NMPC} ensures that robots maintain collision-free trajectories even when network conditions are less than ideal \cite{ferranti2022distributed, tordesillas2020mader}. By adjusting the prediction horizon based on real-time communication delays, \ac{NMPC} provides a robust solution for environments where synchronization between agents is challenging. 

% While \ac{NMPC} focuses on maintaining reliable trajectories through precise adjustments in response to communication delays, the \ac{GNN} approach offers a different strategy by optimizing the communication itself. \ac{GNN}s facilitate local information sharing among agents by learning which data is most critical for decision-making, effectively compressing and reducing the amount of information that needs to be exchanged \cite{li2020graph, niu2021multi}. This selective data transmission allows \ac{GNN}s to operate efficiently in environments where communication bandwidth may be limited or unreliable. \ac{GNN}s aggregate and process environmental data, enabling more informed and cooperative decision-making at the local level, making them particularly suitable for complex environments where direct communication is necessary. However, \ac{GNN}s primarily focus on local decision-making, often neglecting the broader global context that is critical for optimal navigation efficiency.

% A very promising method is the \ac{GBP} approach on factor graphs \cite{GBPPlanner, RobotWeb}. This technique allows agents to iteratively exchange marginal distributions of future states, achieving robust performance in high-density environments. The resilience of \ac{GBP} to communication failures and its capacity to manage asynchronous message passing make it a strong candidate for real-world deployments. However, \ac{GBP}, like many other distributed methods, focuses primarily on local collision avoidance and does not inherently incorporate global path planning, which can lead to suboptimal navigation over longer trajectories.

% Global path planning is crucial for optimizing navigation efficiency in multi-robot systems by managing the flow of the fleet and predicting efficient paths, thereby minimizing travel time and energy consumption \cite{le2023cameta, s23125615, 10368056}. While \ac{GBP} provides a solid foundation for local decision-making, the lack of a global planning component can cause the system to overlook the broader context of the environment. This limitation is significant because local methods, including those based on decentralized reinforcement learning, often prioritize immediate collision avoidance without considering the global context, leading to suboptimal results in terms of overall efficiency.




% New related works:
% Multi-agent path planning is a critical area in robotics, where agents must coordinate and navigate through complex environments while avoiding collisions and optimizing travel efficiency. 
Traditionally, path planning has been approached using centralized and distributed methodologies, each with strengths and limitations. Centralized methods, in which a single controller coordinates all robots’ paths, have shown high performance in structured environments. These approaches often rely on global information and can efficiently handle large-scale coordination. However, as the number of robots grows or as environments become more dynamic and complex, centralized methods face scalability issues due to their computational demands. For example, methods like priority-based planning \cite{soria2021predictive, okumura2022priority, machines10090773, le2023cameta} have demonstrated strong performance but require detailed, global knowledge of the environment and agent locations, which limits their applicability in more dynamic and unpredictable settings.

To address the limitations of centralized planning, distributed methods have gained significant attention. One widely-used distributed multi-robot planner is the \ac{ORCA} algorithm \cite{NH-OCRA, snape2010smooth, claes2012collision}, which allows robots to modify their velocities based on the positions of neighboring agents. While \ac{ORCA} has been successful in many real-world applications, it often produces inefficient and jerky trajectories in dense environments due to its reliance on instantaneous velocity adjustments without long-term planning. Other distributed methods, like \cite{alonso2018cooperative}, rely on short-term trajectory predictions but fail to account for the entire look-ahead time window, making them less effective in highly dynamic environments.

% More recent works have begun to explore the use of graph-based methods, which allow for more flexible and scalable communication between agents. For instance, approaches using the alternating direction method of multipliers \cite{van2016online} and distributed model predictive control \cite{luis2020online} have demonstrated the ability to maintain long-term trajectory optimization through iterative communication. However, these methods often assume ideal communication conditions and struggle when facing unreliable or intermittent communications. \Ac{NMPC} has been effectively applied in distributed multi-agent systems to address the challenges of asynchronous communication and message delays. The use of \ac{NMPC} ensures that robots maintain collision-free trajectories even when network conditions are less than ideal \cite{ferranti2022distributed, tordesillas2020mader}. By adjusting the prediction horizon based on real-time communication delays, \ac{NMPC} provides a robust solution for environments where synchronization between agents is challenging. 

More recent works have explored the use of graph-based methods, which enable more flexible communication between agents. Approaches utilizing the alternating direction method of multipliers (ADMM) \cite{van2016online} and distributed model predictive control (MPC) \cite{luis2020online} have shown the ability to maintain long-term trajectory optimization through iterative communication. However, these methods often rely on ideal communication conditions and can struggle with intermittent or unreliable networks. Building on the ADMM framework, a distributed trajectory optimization approach has been proposed to address asynchronous communication and message delays \cite{ferranti2022distributed, tordesillas2020mader}. This method adjusts safety margins and leverages predictive strategies to ensure collision-free trajectories, even in the face of packet loss. Although it enhances robustness to communication faults, the approach introduces conservatism, potentially leading to less efficient paths and longer completion times.


In recent years, learning-based methods have also been explored for multi-robot coordination. \Ac{GNN} \cite{li2020graph, niu2021multi} have been applied to optimize local communication and decision-making, particularly in grid-world environments. While \ac{GNN}s can handle local information sharing efficiently, their application to more complex environments has been limited. Additionally, methods such as reinforcement learning-based systems \cite{s23125615, 10368056, paul2023efficient} have explored the optimization of task allocation and path planning in multi-agent systems, but they often face challenges in scalability and robustness, particularly when communication failures occur.

Another promising approach is the integration of differentiable decentralized planners like D2CoPlan \cite{sharma2023d2coplan}, which allows for the efficient management of multi-robot coverage by incorporating both local and global objectives. Similarly, scalable systems that leverage large language models for multi-robot collaboration \cite{chen2024scalable} have begun to explore how decentralized systems can balance between efficiency and scalability. However, these methods still remain in early development stages.

A promising development in the field of multi-agent path planning is the use of \ac{GBP} for multi-agent path planning. \ac{GBP} allows for distributed computation over factor graphs, where each robot can communicate and optimize its path with local neighbors through message passing. The Robot Web project \cite{RobotWeb} first demonstrated the potential of \ac{GBP} for localization in large multi-robot systems, showing that robots could iteratively share and update their beliefs about positions. This concept was later extended with a \ac{GBP}-based planner \cite{GBPPlanner}, a purely distributed technique formulated using a generic factor graph to handle both dynamics and collision constraints over a forward time window. However, while a \ac{GBP}-based planner provides a strong framework for decentralized communication, like many other distributed methods, it focuses primarily on local collision avoidance and does not inherently incorporate global path planning, which can lead to suboptimal navigation over longer trajectories.

To address these limitations, our work integrates global path planning into the \ac{GBP} framework. Our approach introduces a tracking factor that ensures robots adhere to global paths, improving both navigation robustness towards communication failures and collision avoidance in multi-agent systems. This method is particularly effective in complex environments, where the integration of global and local planning components is essential for scalable, real-world deployments.