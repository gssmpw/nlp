\section{METHODOLOGY}
% In this section, we detail the methodology used to integrate global path planning with Gaussian Belief Propagation, thereby addressing the limitations of existing multi-agent path planning systems. 
% While \ac{GBP} has proven effective for local planning and collision avoidance in densely populated environments \cite{GBPPlanner, RobotWeb}, its lack of global planning limits its applicability in more complex scenarios. Our approach introduces a tracking factor that ensures robots follow pre-determined global paths, optimizing long-range navigation and enhancing coordination in multi-agent systems.

\ac{GBP} relies on factor graphs to model the multi-agent path planning problem, where the states of robots are represented as variables, and the constraints or dependencies between these states are captured as factors \cite{dellaert_factor_2021}. These factors impose soft constraints on one or more variables, guiding the agents' movements to satisfy all imposed constraints as effectively as possible. The key mechanism within \ac{GBP} is message passing; a probabilistic inference technique where each node, representing a variable or factor, iteratively exchanges information with neighboring nodes. These messages encode the node's beliefs about the variables' values, which are updated based on the information received. Over successive iterations, this process converges to a joint distribution that optimally satisfies the graph's constraints, determining the most likely future position and velocity in the plane for each robot \cite{ortiz_visual_2021}.

The factors introduced in the \ac{GBP} planner include four essential factors: (i) the pose factor $f_p$ represents the robot's estimated position and orientation; (ii) the dynamics factor $f_d$, captures the relationship between consecutive poses, ensuring consistency with the robot's motion model; (iii) the obstacle factor $f_o$, imposes constraints to avoid collisions based on pixel sampling from a signed distance field image representation of the environment; (iv) and the inter-robot factor $f_i$, which accounts for the relative positioning and interaction between multiple robots in a collaborative setting. These factors are parameterized by the scalar value $d_i$ which specifies the distance, from which the factor is active.

% Global planning has been made as an extension to the original \ac{GBP} planner software developed by \cite{GBPPlanner}. 
The original algorithm in \cite{GBPPlanner} works well on a local level but lacks a global overview of how to get from $A$ to $B$, which results in many local optima\footnote[1]{Video comparison of the three methods: \href{https://www.youtube.com/watch?v=Uzz57A4Tk5E}{video}}. To solve this, a global pathfinding algorithm has to be leveraged. The proposed method is algorithm-agnostic; any path-finding algorithm that outputs a series of waypoints avoiding obstacles can be used. 
This global plan can be introduced into the factor graph using two different approaches.

% MAYBE:
\subsection{Approach 1: waypoint tracking}
The \ac{WT} approach serves as the baseline in this work, leveraging a global planner to generate a sequence of waypoints that act as intermediate navigation goals for the local planning algorithm. This approach automates the process of global path planning, which was previously done manually in \cite{GBPPlanner}. \ac{WT} remains a foundational method for integrating global pathfinding into multi-agent planning, using the same core variables and factors from the original \ac{GBP} planner, but now with automated waypoint generation that dynamically adapts to the environment. The process consists of the following steps: 
\begin{enumerate} 
    \item A global planner computes a path from the start position $A$ to the goal position $B$, accounting for environmental constraints and obstacles. 
    \item The path is segmented into waypoints, which act as intermediate targets for the local planner.
    \item The existing local planning algorithm follows these waypoints without any modification to its core operation. 
\end{enumerate}
The approach is designed to guide the robot towards the waypoints along the given path, but it does not strictly enforce adherence to the exact obstacle-free line between waypoints. This flexibility allows the robots to deviate from the globally planned path, enabling more creative maneuvering around each other and permitting corner-cutting. While this can provide some advantages in dynamic environments, it may also result in an increase in intersecting paths among robots, particularly in densely populated areas. This makes it an effective baseline for evaluating more accurate approaches, such as path tracking, which prioritizes stricter adherence to global paths.


% Despite its inherent flexibility, the \ac{WT} approach serves as a solid foundation for comparison because it ensures robots navigate through an obstacle-free path generated by the global planner. However, since adherence to the global path is not strictly enforced, the \ac{WT} approach can result in higher path deviations in densely populated areas or more complex environments. 

\subsection{Approach 2: path tracking}
\begin{figure}[!b]
    \centering
    \vspace{-4mm}
    \subfloat[The tracking factor pulls the variable towards and along the path, with a green area near waypoint $w_1$ indicating corner tracking.]{%
        \includegraphics[width=0.65\linewidth]{img/tracking-factor-explain-1.pdf}%
        \label{fig:tracking-factor-explain-1}%
    }
    \hfill
    \subfloat[Tracking factors for robot R moving from $w_0$ to $w_1$. ]{%
        \includegraphics[width=0.33\linewidth]{img/tracking-factor-explain-2.pdf}%
        \label{fig:tracking-factor-explain-2}%
    }
    \caption{Illustration of the tracking factor concept. The tracking factor measurement is shown in orange, the state variables is shown in blue, and the gray line is the global path connecting two waypoints.}
    \label{fig:tracking-design}
\end{figure}

The \ac{PT} approach builds on the foundation of global planning by introducing a tracking factor, $f_t$, to enforce stricter adherence to the planned path. This tracking factor is integrated into the factor graph structure used in \ac{GBP}, ensuring that robots follow the global path more closely. The $f_t$ attaches to each variable within the robot's prediction horizon, with the exception of the first and last variables, which are anchored to ensure stability at the starting and goal positions.

Fig. \ref{fig:tracking-design} demonstrates how the tracking factor operates. It measures the perpendicular distance from the robot's current position to the planned path and applies a pulling force that not only reduces the deviation from the path but also nudges the robot slightly forward, particularly around corners. This forward pull helps to guide the robot smoothly along turns, ensuring more precise navigation when compared to the more flexible \ac{WT} approach.

\subsection{Tracking factor}
% The design of the tracking factor, $f_t$, is inspired by the inter-robot factors $f_i$, but with an opposite purpose. While the inter-robot factors ensure that robots maintain a safe distance from each other by imposing a repulsive force, the tracking factor pulls the robot toward a desired path, enforcing adherence to the planned trajectory. 
The tracking factor $f_t$ exerts a pulling force that guides the variable toward the desired path, ensuring adherence to the planned path. 
This is achieved through two key components: the measurement function and the Jacobian, which guide the robot along the global path while respecting its dynamics and other constraints.

\subsubsection{Measurement Function}
The measurement function evaluates the deviation of the variable's current position, \( \mathbf{x}_\mathrm{pos} = [x\ y]^\top \), from a desired path $\textbf{GP} = \{\textbf{p}_1, \dots, \textbf{p}_n\}$, where $\textbf{p}_i = [x_i, y_i]$. Each line segment along the path is defined as \( \mathbf{l}_i = \mathbf{p}_{i+1} - \mathbf{p}_i \), where the index \( i \) identifies the current segment the system is tracking. The output of the measurement function is a scalar value between 0 and 1, which dictates the strength of the pulling effect that guides the variable along the trajectory.

The core concept of the tracking factor is to project the current position of the variable \( \mathbf{x}_\mathrm{pos} \) onto the line segment \( \mathbf{l}_i \). The distance between this projection and the actual position serves as the basis for the measurement. The projection function computes the projection of the point \( \mathbf{x} \) onto the line segment between \( \mathbf{p}_i\) and \( \mathbf{p}_{i+1} \), and is defined as follows:
\begin{equation}
    \small
\mathcal{P}_i = \mathbf{p}_i + \frac{(\mathbf{x} - \mathbf{p}_i) \cdot (\mathbf{p}_{i+1} - \mathbf{p}_i)}{\|\mathbf{p}_{i+1} - \mathbf{p}_i\|^2} (\mathbf{p}_{i+1} - \mathbf{p}_i).
\end{equation}
As the variable approaches the waypoint $\textbf{p}_{i+1}$, the transition to the next line segment occurs when the variable enters a predefined radius around the waypoint, controlled by the configurable threshold \( r_{\text{switch}} \). At this point, the index \( i \) is incremented, i.e., \( i = i+1 \).

Once \( i \) is updated, a conditional criterion, denoted by \( q \), determines whether the variable is transitioning between two line segments:
\begin{equation}
    q = \|\mathcal{P}_i - \mathbf{p}_i\| < r_{\text{switch}} \land \|\mathcal{P}_{i-1} - \mathbf{p}_i\| < r_{\text{switch}}\label{eq.condition},
\end{equation}
\noindent where \( \mathcal{P}_i \) is the projection onto the current line segment, and \( \mathcal{P}_{i-1} \) is the projection onto the previous segment.
Based on this condition, the measurement point \( \mathbf{x}_\mathrm{meas} \) is defined as:
\begin{equation}
    \small
    \mathbf{x}_\mathrm{meas} =
    \begin{cases}
        \mathbf{x_\mathrm{pos}} + \frac{1}{2}((\mathcal{P}_i - \mathbf{x}_\mathrm{pos}) - (\mathcal{P}_{i-1} - \mathbf{x}_\mathrm{pos}))\, &\mathrm{if}\ q \\
        \mathcal{P}_i + \mathbf{d} \cdot \frac{||\mathbf{x}_\mathrm{vel}||}{s_v}\, &\mathrm{else}
    \end{cases} \label{eq.meas-point},
\end{equation}
\noindent where \( \mathbf{d} = \frac{\mathbf{l}_i}{\|\mathbf{l}_i\|} \) is the normalized direction vector of the current line segment, and the term \( \mathbf{d} \cdot \frac{\|\mathbf{x}_\mathrm{vel}\|}{s_v} \) ensures that the system maintains movement along the segment, not just perpendicular correction. The parameter \( s_v\), chosen heuristically, balances forward momentum and prevents overshooting. The condition \( q \) serves two key purposes. First, it facilitates a smooth transition between the path segments \( \textbf{p}_{i-1}\) and \( \textbf{p}_i \), enabling the system to consider both segments in subsequent iterations. Second, it pulls the system toward the corner formed by the intersection of the two segments, effectively reducing corner-cutting behavior. This is illustrated in Fig.~\ref{fig:tracking-factor-explain-1}.

Finally, the measurement function \( h(\mathbf{x}) \), over the state vector \( \mathbf{x} \in \mathbb{R}^{4\times 1} = [\mathbf{x}_{\text{pos}}, \mathbf{x}_{\text{vel}}]^\top \), computes a scaled and clamped distance between the current position \( \mathbf{x}_\mathrm{pos} \) and the measurement point \( \mathbf{x}_\mathrm{meas} \):
\begin{equation}
   h(\mathbf{x}) = \min\left(1, \frac{\|\mathbf{x}_{\text{pos}} - \mathbf{x}_{\text{meas}}\|}{d_a}\right)
    % \mathbf{x} = [x_{\text{pos}}, y_{\text{pos}}, \dot{x}_{\text{pos}}, \dot{y}_{\text{pos}}],
\end{equation}
The raw distance is normalized by the parameter \( d_a \), which governs how quickly the attraction force reaches its maximum. The result is clamped to a maximum of 1 to ensure stability in the factor graph inference and prevent excessively large attraction forces.

\subsubsection{Jacobian}
The Jacobian is used to compute the next position of the variable, imposed by the constraint represented by the tracking factor. The most recent measurement, \( h \), scales the Euclidean difference between \( \mathbf{x}_\mathrm{meas} \) and \( \mathbf{x}_\mathrm{pos} \), ensuring the velocity of the system is accounted for. The Jacobian is defined as follows:
\begin{equation}
    \mathbf{J} = \left[ \frac{1}{h(\mathbf{x})}(x_\mathrm{meas} - x_\mathrm{pos})\ \frac{1}{h(\mathbf{x})}(y_\mathrm{meas} - y_\mathrm{pos})\ 0\ 0 \right] \label{eq.jacobian}.
\end{equation}
The Jacobian is padded with zeros to incorporate the velocity component of the linearization point.


\begin{figure}
    \centering
    \includegraphics[width=0.7\linewidth]{img/solo-gp-deviation.pdf}
    \vspace{-3mm}
    \caption{The path deviation visualized between waypoint tracking (blue), path tracking (green), and desired path (black).}
    \label{fig:tracking}
\end{figure}
By leveraging this tracking factor, the \ac{PT} approach should significantly reduce deviations from the planned trajectory, resulting in more accurate path-following behavior. This is especially beneficial in scenarios where high precision is critical, such as in densely populated environments or when robots must closely coordinate to avoid collisions.
While the structured nature of \ac{PT} enhances multi-agent coordination and is ideal for environments demanding precise path tracking and collision avoidance, it does impose stricter trajectory-following constraints. This can limit the robot's ability to make creative maneuvers or adjust to unforeseen obstacles in real-time making it less adaptable in highly dynamic settings where frequent adjustments are required.