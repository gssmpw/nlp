\section{RESULTS}
The results presented in Table \ref{table:tracking} for the solo modality offer an isolated assessment of the impact of the tracking factor. First, a single robot navigates through the complex environment using the RRT* global planner. A notable reduction in \ac{PPD} is observed at approximately 28\%. This outcome highlights the potential of the tracking factor when evaluated in isolation.
\begin{table}[b]
    \centering
    \caption{Navigation results in the complex environment.}
    \begin{tabular}{ccc|cc|cc}
        \toprule
        \multirow{2}{*}{\textbf{Modality}} & \multirow{2}{*}{\textbf{Method}} & \multirow{2}{*}{\textbf{Planner}} & \multicolumn{2}{c|}{\textbf{PPD [m]}} & \multicolumn{2}{c}{\textbf{Collisions}} \\
        & & & \textbf{Mean} & \textbf{Std.} & \textbf{Int.} & \textbf{Ext.} \\
        \midrule
        \multirow{2}{*}{\textbf{Solo}}
        & WT & RRT* & 1.03 & 0.11 & 0 & 0 \\
        & PT & RRT* & 0.74 & 0.08 & 0 & 0 \\
        \midrule
        \multirow{4}{*}{\textbf{Collaborative}}
        & WT & RRT* & 1.00 & 0.23 & 2.8 & 0 \\
        & PT & RRT* & 0.86 & 0.25 & 11.9 & 0 \\
        & WT & SP & 0.64 & 0.11 & 0 & 0 \\
        & PT & SP & 0.54 & 0.10 & 0 & 0 \\
        \bottomrule
    \end{tabular}
    \label{table:tracking}
\end{table}

The results for the collaborative modality in the complex environment are also summarized in Table \ref{table:tracking}. When using the \ac{SP}, both the \ac{WT} and \ac{PT} methods successfully avoid collisions, between robots and with the environment. In contrast, when utilizing the sample-based planner RRT*, as in the solo modality, a significant increase in inter-robot collisions occurs, with a modest \ac{PPD} reduction of only 14\% between \ac{PT} and \ac{WT}. This disparity is likely attributed to the random nature of RRT*, which generates independent paths for each robot without considering the paths of others. As a result, many globally planned paths overlap, leading to chaotic and conflicting trajectories that place excessive demands on the local \ac{GBP} planner, ultimately making collisions unavoidable. In contrast, the structured planner, which assigns dedicated lanes for each direction, effectively mitigates these issues, resulting in zero collisions.

When comparing \ac{WT} to \ac{PT} in the collaborative \ac{SP} scenario, \ac{PT} demonstrates an almost 16\% reduction in \ac{PPD} on average, with a corresponding decrease in variance. These improvements are promising, particularly considering that \ac{PT} outperforms the RRT* approach by 2\% while maintaining zero collisions.

It is important to note that these results are obtained under ideal communication conditions. As illustrated in Fig. \ref{fig:sjtw-collisions}, the tracking factor does influence the number of inter-robot collisions under conditions of increasing communication failure. However, the increase is relatively minor, with significant deviations from \ac{WT} results only occurring when the communication failure rate exceeds 60\%. Thus, the marginal increase in collisions is outweighed by the improvements in path deviation, especially in scenarios where collisions occur regardless. 
An increase in obstacle collision can also be observed when the communication failure rate exceeds 50\%. At this failure rate, few iterations of \ac{GBP} include external information from other robots, which in turn also means that the internal iterations are weighted disproportionately. In this scenario, each robot has optimized for internal costs, which can cause suboptimal external collision avoidance. The weighting between $\sigma_{f_i}$ and $\sigma_{f_o}$ determines whether it is more likely to become an inter robot or environment collision.

% This is due to too few external iterations where the perception of the world becomes too skewed and on the external updates are going through, a lot of internal changes to the variables have happened so a large cost pushes the variable towards a collision, whether it is an inter robot collision or an environment collision depends on the weighting between $\sigma_{f_i}$ and $\sigma_{f_o}$.

The degradation of \ac{PPD} under these conditions, as shown in Fig. \ref{fig:sjtw-ppe}, further illustrates the trade-offs between accuracy and collision avoidance. As the communication failure rate increases, the robots become less coordinated, requiring more agile maneuvers to avoid collisions. This reduces path efficiency, as the robots deviate from optimal trajectories and struggle to maintain smooth, collision-free navigation.

% A significant qualitative outcome of this research is the increased complexity introduced by the addition of the tracking factor, $f_t$, to the optimization problem. This complexity manifests as a higher likelihood of the system becoming trapped in local minima that deviate from the global optimum. The tracking factor, in particular, must maintain a delicate balance with the dynamic factor, as it cannot be assigned a greater level of certainty. If the certainty of the tracking factor exceeds that of the dynamic factor, the robot might get stuck at sharper corners along the path. Additionally, while the tracking factor pulls the robot toward the path, the inter-robot factor may sometimes require deviation from this path to avoid collisions, a situation where balance is crucial as collision avoidance takes precedence. The introduction of additional factors complicates the manual tuning process of factor certainties, and without improvements to this tuning process, extending the factor graph with more constraints will prove challenging.


\begin{figure}[b]
    \centering
    \includegraphics[width=1\linewidth]{img/collisions-structured-junction-twoway.pdf}
    \caption{Results from the junction environment. There is a slight increase in inter-robot collisions when using path tracking instead of waypoint. The number of collisions with the environment is the same for WT and PT. Average over 5 runs for each failure rate with 1200 robots.}
    \vspace{-5 pt}
    \label{fig:sjtw-collisions}
\end{figure}

A significant qualitative outcome of this research is, the introduction of the tracking factor $f_t$ offers significant benefits for path adherence, but it requires a delicate tuning process. If the balance between the tracking and dynamics factors is not carefully maintained, there is a risk of the system becoming trapped in local minima. Proper tuning is crucial to prevent the robot from getting stuck at sharp corners or over-correcting due to excessive reliance on the tracking factor. Furthermore, while the tracking factor helps guide the robot along the path, it must work in harmony with the inter-robot factor to avoid collisions. Effective tuning ensures these factors complement each other, optimizing both path adherence and collision avoidance.

The global planning element does not degrade local cooperative collision avoidance, as waypoint tracking automates what was previously manual. However, the random nature of the RRT* algorithm can cause suboptimal crossing paths. Manually placed waypoints can be risk-averse, reducing path crosses. The tracking factor exacerbates this issue by pulling actors towards planned paths without much consideration of others, making collision avoidance difficult. Balancing tracking and inter-robot factors helps, but collisions remain more frequent than with waypoint tracking alone. Solving this is crucial for effective path tracking with tracking factors. Still, we believe in a structured environment with roads and rules, the tracking factor would ensure more predictable and safer navigation. 

\begin{figure}
    \centering
    \includegraphics[width=1\linewidth]{img/perpendicular-path-deviation-structured-junction-twoway.pdf}
    \caption{Perpendicular path deviation as a function of the failure probability. Lower is better.}
    \vspace{-5 pt}
    \label{fig:sjtw-ppe}
\end{figure}