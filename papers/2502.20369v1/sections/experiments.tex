\

\section{EXPERIMENTS}
The performance of the system is assessed based on key metrics that capture aspects such as navigation efficiency, safety, and path accuracy. These experiments are conducted in different environments, each presenting unique challenges to test the robustness of the approaches.

\subsection{Metric}
Several metrics are used to measure efficiency, safety, and accuracy in evaluating our navigation strategies. The following metrics provide a comprehensive assessment of the system's performance.
\begin{itemize}
    % \item \textit{Distance traveled:} The cumulative distance covered by the robot until it reaches its destination. Effective trajectories aim to minimize this measure.
    \item \textit{Inter Robot Collisions:} Number of collisions between robots. The physical size of each robot is represented by a bounding circle. A collision between two robots happens when their circles intersect. That is a collision is only registered if robots \(R_a\) and \(R_b\) intersect at timestep \(t_n\) but not at \(t_{n-1}\). 
    
    \item \textit{Environment Robot Collisions:}  Number of collisions between robots and the environment. Similar to \textit{Inter Robot Collisions}, bounding circles are used for the robots, while the environment obstacle is equipped with a collider of the same geometric layout.
    % The same detection registration used for \textit{Inter Robot Collisions} is used here.
    
    \item \textit{Root mean squared error of \ac{PPD}:} At each sampled position, the distance between it and the closest projection onto each line segment of the planned path is measured and accumulated, using the \ac{RMSE}, defined as follows:
    % A visual depiction of this is shown in Figure 21.
    \begin{equation}
        \text{RMSE} = \sqrt{\frac{1}{n} \sum_{j=1}^{n} \left( \min_{i} \left\lbrace \|P_j - p(P_j, L_i)\|^2 \right\rbrace \right)^2},
    \end{equation}
    where \(L_i \in \{L_1, L_2, ..., L_m\}\) is the set of line segments making up the planned path, \(P_j \in \{P_1, P_2, ..., P_n\}\) is the set of sampled positions, and \(\| P_j - p(P_j, L_i) \|^2\) is the squared distance between the sampled position \(P_j\) and the projection of \(P_j\) onto the line segment \(L_i\). This is measured to test the effect of the proposed tracking factor, as some applications require that robots follow a dictated path with little deviation. 
\end{itemize}
\subsection{Scenarios}
\label{sec:scenarios}
We introduce three scenarios to test the proposed strategies, each presenting unique challenges in navigation and communication, thereby assessing the robots' robustness in varied conditions. If not specified otherwise, the following parameters are used for each scenario: Robot radius $r_R = 2$m, communication radius $r_\text{comms} = 20$m, communication failure probability $\gamma = 0\%$, target speed $v_t = 5$m/s, time horizon $t_{K-1} = 5$s, number of internal iterations per timestep $T_I = 10$, and external $T_E = 10$. The standard deviation for each factor is $\sigma_{f_d} = 0.1$, $\sigma_{f_p} = 10^{-15}$, $\sigma_{f_i} = 0.005$, $\sigma_{f_o} = 0.005$ and $\sigma_{f_t} = 0.15$. 
% \footnote{For reproduceability, all parameters are available in the \href{https://github.com/au-Master-Thesis/magics}{code repository}. Under \texttt{config/scenarios/<scenario-name>/config.toml}.}.

% \href{https://github.com/AU-Master-Thesis/magics/tree/main/config/scenarios/Collaborative\%20Complex}



% \href{https://github.com/AU-Master-Thesis/magics/tree/main/config/scenarios/Structured%20Junction%20Twoway}{
\textbf{The junction environment} consists of a single two-way junction designed to evaluate robot coordination in high-throughput conditions. Each robot is randomly assigned a spawning point on one of the four sides of the junction, with a target destination chosen from the remaining three sides. Consequently, all robots must navigate through the junction, requiring effective coordination to avoid collisions and ensure smooth passage, as shown in Fig. \ref{fig:junction}. To increase robot density in the center, each robot is assigned a smaller radius of $r_R = 1m$.

%% structured junction twoway:
    % "sigma-pose-fixed": 1e-15,
    % "sigma-factor-dynamics": 0.1,
    % "sigma-factor-interrobot": 0.005,
    % "sigma-factor-obstacle": 0.005,
    % "sigma-factor-tracking": 0.15,
    % "lookahead-multiple": 3,
    %     "planning-horizon": 5.0,
    % "target-speed": 5.0,
    % "radius": {
    %   "min": 1.0,
    %   "max": 1.0
    % },
    % "communication": {
    %   "radius": 20.0,
    %   "failure-rate": 0.0
    % },
    % "inter-robot-safety-distance-multiplier": 2.5

% \href{https://github.com/AU-Master-Thesis/magics/tree/main/config/scenarios/Structured%20Junction%20Twoway}{
\textbf{Communications failure} is tested within the junction environment and tests the robustness of each presented strategy in terms of inter robot collisions, environment robot collisions, and RMSE of \ac{PPD} by selecting the degree of communication failure $\gamma$ from $[0\%, 10\%, \dots, 70\%]$. A communication failure is defined as shutting down all communication for a single unit at a given timestep $t_n$. 

\textbf{The complex environment} is a building designed to have a maze-like structure with hallways, junctions and lane merging, creating a challenging navigation environment. What looks like dead ends in Fig. \ref{fig:complex_env} are spawning and goal locations. Each individual robot gets a task to traverse the complex environment to one of the goal locations. 

Two global planners are tested in this environment. The first is the asymptotically optimal RRT* path planning algorithm \cite{karaman2011sampling}, which is employed to automatically generate global paths. However, the inherent randomness of the RRT* algorithm can result in unstructured paths, increasing the likelihood of robot intersections as they navigate the environment. The second global planner is a \ac{SP} that incorporates lane structures to guide the robots along more organized and predictable paths. The environment needs to be predefined as a directed graph, where edges represent lanes going in one direction. This approach is designed to reduce intersections by encouraging robots to follow predefined lanes, thus promoting smoother and more efficient navigation through the environment. To find a valid path through the environment, the A* algorithm can be used to determine the shortest route. For both the solo and collaborative modalities, these parameters are different from the default; $v_t =7$ m/s, $\sigma_{f_i} = \sigma_{f_o} = 10^{-3}$.

% planning-horizon = 5.0
% target-speed = 5.0
% inter-robot-safety-distance-multiplier = 3.0

% internal = 10
% external = 10

%% collaborative complex:
    % "sigma-pose-fixed": 1e-15,
    % "sigma-factor-dynamics": 0.1,
    % "sigma-factor-interrobot": 0.005,
    % "sigma-factor-obstacle": 0.005,
    % "sigma-factor-tracking": 0.15,
    % "lookahead-multiple": 3,
    %     "planning-horizon": 5.0,
    % "target-speed": 5.0,
    % "radius": {
    %   "min": 1.0,
    %   "max": 1.0
    % },
    % "communication": {
    %   "radius": 20.0,
    %   "failure-rate": 0.0
    % },
    % "inter-robot-safety-distance-multiplier": 3.0

%% solo gp:
    % "sigma-pose-fixed": 1e-15,
    % "sigma-factor-dynamics": 0.1,
    % "sigma-factor-interrobot": 0.01,
    % "sigma-factor-obstacle": 0.01,
    % "sigma-factor-tracking": 0.15,
    % "lookahead-multiple": 3,
    % "inter-robot-safety-distance-multiplier": 3.2
    %    "planning-horizon": 5.0,
    % "target-speed": 7.0,
    % "radius": {
    %   "min": 2.0,
    %   "max": 2.0
    % },

    %% collaborative gp:
    %     "sigma-pose-fixed": 1e-15,
    % "sigma-factor-dynamics": 0.1,
    % "sigma-factor-interrobot": 0.01,
    % "sigma-factor-obstacle": 0.01,
    % "sigma-factor-tracking": 0.15,

    %     "planning-horizon": 5.0,
    % "target-speed": 7.0,
    % "radius": {
    %   "min": 2.0,
    %   "max": 2.0
    % },
    % "communication": {
    %   "radius": 20.0,
    %   "failure-rate": 0.0
    % },
    % "inter-robot-safety-distance-multiplier": 4.0
% \tcbset{%
%     colback=ctpBase,
%     boxrule=0mm,
%     left=1mm,
%     top=1mm,
%     right=1mm,
%     bottom=1mm,
%     boxsep=0mm,
%     before skip=0mm,
%     after skip=0mm,
%     arc=3mm,
%     opacityframe=0,
%     interior hidden
% }


\begin{figure}
    \centering
    \subfloat[Junction environment]{%
        % \begin{tcolorbox}[
        %     width=0.4\linewidth,
        %     colback=white,
        %     boxrule=2pt,
        %     arc=6pt,
        %     outer arc=0pt,
        %     enlarge left by=10pt,
        %     enlarge right by=10pt,
        %     enlarge top by=10pt,
        %     enlarge bottom by=10pt
        % ]
        % \includegraphics[width=\textwidth]{img/structured-junction-twoway-example-screenshot.png}
        % \end{tcolorbox}%
        % \label{fig:junction}%
    
        \begin{tcolorbox}[width=0.4\linewidth]
            % \includegraphics[width=\textwidth]{img/junction-experiment-preview.pdf}
            \includegraphics[width=\textwidth]{img/structured-junction-twoway-example-screenshot.png}
        \end{tcolorbox}%
        \label{fig:junction}%
    }
    \hfill
    \subfloat[Complex environment]{%
        \begin{tcolorbox}[
            width=0.55\linewidth,
            % colback=red,
            boxrule=0pt,
            % enlarge left by=10pt,
            % enlarge right by=10pt,
            % enlarge top by=10pt,
            % enlarge bottom by=-10pt
            % sidebyside,
            % sidebyside gap=5mm,
            boxsep=2.5pt
        ]
            \includegraphics[width=\textwidth]{img/collaborative-complex.pdf}
        \end{tcolorbox}%
        \label{fig:complex_env}%
    }
    \caption{Illustration of the two environments used in the experiments: (a) junction environment, (b) complex environment.}
    \label{fig:tracking-factor-explain-3}
\vspace{-3mm}
\end{figure}



% \begin{figure}[b]
%     \centering
%     \vspace{-4mm}
%     \subfloat[Tracking Along Path]{%
%         \includegraphics[width=0.65\linewidth]{img/tracking-factor-explain-1.pdf}%
%         \label{fig:tracking-factor-explain-1}%
%     }
%     \hfill
%     \subfloat[Exemplification]{%
%         \includegraphics[width=0.33\linewidth]{img/tracking-factor-explain-2.pdf}%
%         \label{fig:tracking-factor-explain-2}%
%     }
%     \caption{Illustration of the tracking factor concept. a) shows how the tracking factor pulls the variable towards and along the path, with a green area near waypoint $w_1$ indicating corner tracking. b) illustrates tracking factors for robot R moving from $w_0$ to $w_1$. The tracking factor measurement is shown in orange.}
%     \label{fig:tracking-design}
% \end{figure}