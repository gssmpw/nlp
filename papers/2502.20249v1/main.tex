% CVPR 2025 Paper Template; see https://github.com/cvpr-org/author-kit

\documentclass[10pt,twocolumn,letterpaper]{article}

%%%%%%%%% PAPER TYPE  - PLEASE UPDATE FOR FINAL VERSION
%\usepackage{cvpr}              % To produce the CAMERA-READY version
% \usepackage[review]{cvpr}      % To produce the REVIEW version
\usepackage[pagenumbers]{cvpr} % To force page numbers, e.g. for an arXiv version
%\usepackage{comment}
\usepackage{arydshln}
% Import additional packages in the preamble file, before hyperref
%
% --- inline annotations
%
\newcommand{\red}[1]{{\color{red}#1}}
\newcommand{\todo}[1]{{\color{red}#1}}
\newcommand{\TODO}[1]{\textbf{\color{red}[TODO: #1]}}
% --- disable by uncommenting  
% \renewcommand{\TODO}[1]{}
% \renewcommand{\todo}[1]{#1}



\newcommand{\VLM}{LVLM\xspace} 
\newcommand{\ours}{PeKit\xspace}
\newcommand{\yollava}{Yo’LLaVA\xspace}

\newcommand{\thisismy}{This-Is-My-Img\xspace}
\newcommand{\myparagraph}[1]{\noindent\textbf{#1}}
\newcommand{\vdoro}[1]{{\color[rgb]{0.4, 0.18, 0.78} {[V] #1}}}
% --- disable by uncommenting  
% \renewcommand{\TODO}[1]{}
% \renewcommand{\todo}[1]{#1}
\usepackage{slashbox}
% Vectors
\newcommand{\bB}{\mathcal{B}}
\newcommand{\bw}{\mathbf{w}}
\newcommand{\bs}{\mathbf{s}}
\newcommand{\bo}{\mathbf{o}}
\newcommand{\bn}{\mathbf{n}}
\newcommand{\bc}{\mathbf{c}}
\newcommand{\bp}{\mathbf{p}}
\newcommand{\bS}{\mathbf{S}}
\newcommand{\bk}{\mathbf{k}}
\newcommand{\bmu}{\boldsymbol{\mu}}
\newcommand{\bx}{\mathbf{x}}
\newcommand{\bg}{\mathbf{g}}
\newcommand{\be}{\mathbf{e}}
\newcommand{\bX}{\mathbf{X}}
\newcommand{\by}{\mathbf{y}}
\newcommand{\bv}{\mathbf{v}}
\newcommand{\bz}{\mathbf{z}}
\newcommand{\bq}{\mathbf{q}}
\newcommand{\bff}{\mathbf{f}}
\newcommand{\bu}{\mathbf{u}}
\newcommand{\bh}{\mathbf{h}}
\newcommand{\bb}{\mathbf{b}}

\newcommand{\rone}{\textcolor{green}{R1}}
\newcommand{\rtwo}{\textcolor{orange}{R2}}
\newcommand{\rthree}{\textcolor{red}{R3}}
\usepackage{amsmath}
%\usepackage{arydshln}
\DeclareMathOperator{\similarity}{sim}
\DeclareMathOperator{\AvgPool}{AvgPool}

\newcommand{\argmax}{\mathop{\mathrm{argmax}}}     



% It is strongly recommended to use hyperref, especially for the review version.
% hyperref with option pagebackref eases the reviewers' job.
% Please disable hyperref *only* if you encounter grave issues, 
% e.g. with the file validation for the camera-ready version.
%
% If you comment hyperref and then uncomment it, you should delete *.aux before re-running LaTeX.
% (Or just hit 'q' on the first LaTeX run, let it finish, and you should be clear).
\definecolor{cvprblue}{rgb}{0.21,0.49,0.74}
\usepackage[pagebackref,breaklinks,colorlinks,allcolors=cvprblue]{hyperref}

%%%%%%%%% PAPER ID  - PLEASE UPDATE
\def\paperID{7487} % *** Enter the Paper ID here
\def\confName{CVPR}
\def\confYear{2025}

%%%%%%%%% Custom latex
\section{Summary of Mathematical Notations}
\label{sec:notations}

We summarize the main mathematical notations used in the main paper in Table \ref{table:notations}.

\begin{table*}[h]
	\centering
	\caption{Summary of main mathematical notations.}
	%\resizebox{1\columnwidth}{!}{
		\begin{tabular}{c|l}
			\toprule
			Notation  & Description  \\
			\hline
            $ \mathcal{G} $ & a document graph \\
			$ \mathcal{D} $ & a corpus of documents, $ \mathcal{D}=\{d_i\}_{i=1}^{N} $ \\
			$ N $ & number of documents in the corpus, $ N=|\mathcal{D}| $ \\
			$ d_i $ & document $ i $ containing a sequence of words, $ d_i=\{w_{i,v}\}_{v=1}^{|d_i|}\subset\mathcal{V} $ \\
			$ \mathcal{V} $ & vocabulary \\
			$ |d_i| $ & number of words in document $ i $ \\
			$ \mathcal{E} $ & a set of graph edges connecting documents, $ \mathcal{E}=\{e_{ij}\} $ \\
			$ \mathcal{N}(i) $ & the neighbor set of document $ i $ \\
            $ \Bbb H^{n,K} $ & Hyperboloid model with dimension $ n $ and curvature $ -1/K $ \\
			$ \mathcal{T}_{\textbf{x}}\Bbb H^{n,K} $ & tangent (Euclidean) space around hyperbolic vector $ x\in\Bbb H^{n,K} $ \\
			$ \exp_{\textbf{x}}^K(\textbf{v}) $ & exponential map, projecting tangent vector $ \textbf{v} $ to hyperbolic space \\
			$ \log_{\textbf{x}}^K(\textbf{y}) $ & logarithmic map, projecting hyperbolic vector $ \textbf{y} $ to $ \textbf{x} $'s tangent space \\
			$ d_{\mathcal{L}}^K(\textbf{x},\textbf{y}) $ & hyperbolic distance between hyperbolic vectors $ \textbf{x} $ and $ \textbf{y} $ \\
			$ \text{PT}_{\textbf{x}\rightarrow\textbf{y}}^K(\textbf{v}) $ & parallel transport, transporting $ \textbf{v} $ from $ \textbf{x} $'s tangent space to $ \textbf{y} $'s \\
			$ H $ & length of a path on topic tree \\
            $ \sigma(t,i) $ & similarity between topic $ t $ and document $ i $ \\
            $ \bm{\pi}_i $ & path distribution of document $ i $ over topic tree \\
            $ \textbf{z}_{t,p} $ & hyperbolic ancestral hidden state of topic $ t $ \\
            $ \textbf{z}_{t,s} $ & hyperbolic fraternal hidden state of topic $ t $ \\
			$ \textbf{z}_t $ & hyperbolic hidden state of topic $ t $ \\
            $ \sigma(h,i) $ & similarity between topic $ t $ and document $ i $ \\
            $ \textbf{z}_h $ & hyperbolic hidden state of level $ h $ \\
			$ \bm{\delta}_i $ & level distribution of document $ i $ over topic tree \\
            $ \bm{\theta}_i $ & topic distribution of document $ i $ over topic tree \\
            $ \textbf{e}_i $ & hierarchical tree embedding of document $ i $ \\
            $ T $ & number of topics on topic tree \\
            $ \textbf{g}_i $ & hierarchical graph embedding of document $ i $ \\
			$ \textbf{U} $ & a matrix of word embeddings, $ \textbf{U}\in\Bbb R^{|\mathcal{V}|\times(n+1)} $ \\
			$ \bm{\beta} $ & topic-word distribution $ \bm{\beta}\in\Bbb R^{T\times |\mathcal{V}|} $ \\
			\bottomrule
		\end{tabular}
	%}
	%\vspace{-0.2cm}
	\label{table:notations}
\end{table*}

%%%%%%%%% AUTHORS - PLEASE UPDATE
\author{Pierre Vuillecard\\
IDIAP, EPFL\\
{\tt\small pierre.vuillecard@idiap.ch}
% For a paper whose authors are all at the same institution,
% omit the following lines up until the closing ``}''.
% Additional authors and addresses can be added with ``\and'',
% just like the second author.
% To save space, use either the email address or home page, not both
\and
Jean-Marc Odobez\\
IDIAP, EPFL\\
% First line of institution2 address\\
{\tt\small odobez@idiap.ch}
}

\begin{document}

%%%%%%%% TITLE - PLEASE UPDATE
\title{ \papertitle }

\twocolumn[{%
\renewcommand\twocolumn[1][]{#1}%
\maketitle


\vspace{-1.2cm}
\begin{center}
    \centering
    \captionsetup{type=figure}
    \includegraphics[width=1.\textwidth]{images/Main_figure_gaze.pdf}
    \vspace{-6mm}
    \captionof{figure}{
    \textbf{Significance of ST-WSGE.} Our self-training based weakly-supervised framework for robust 3D gaze estimation in real-world conditions (e.g., varying appearance, extreme poses, resolution, and occlusion).  
     All predictions used our image and video agnostic Gaze Transformer (\model) model.
     Top row: importance of the training diversity using ST-WSGE and GazeFollow (GF) for generalization compared to standard supervised methods. 
     Bottom row: influence of temporal context between image and video inference.  Circles in  images represent unit disks where 3D gaze vectors are projected onto the image plane (x, y in yellow) and a top-down view (x, z in blue). Images from VideoAttentionTarget, GFIE, and MPIIFaceGaze datasets.
}
    \label{fig:intro}
\end{center}%
}]

%%%%%%%%% ABSTRACT

\begin{abstract}
Accurate 3D gaze estimation in unconstrained real-world environments remains a significant challenge due to variations in appearance, head pose, occlusion, and the limited availability of in-the-wild 3D gaze datasets. To address these challenges, we introduce a novel Self-Training Weakly-Supervised Gaze Estimation framework (\framework). This two-stage learning framework leverages diverse 2D gaze datasets, such as gaze-following data, which offer rich variations in appearances, natural scenes, and gaze distributions, and proposes an approach to generate 3D pseudo-labels and enhance model generalization.
Furthermore, traditional modality-specific models, designed separately for images or videos, limit the effective use of available training data. To overcome this, we propose the Gaze Transformer (\model), a modality-agnostic architecture capable of simultaneously learning static and dynamic gaze information from both image and video datasets. By combining 3D video datasets with 2D gaze target labels from gaze following tasks, our approach achieves the following key contributions:
(i) Significant state-of-the-art improvements in within-domain and cross-domain generalization on unconstrained benchmarks like Gaze360 and GFIE, with notable cross-modal gains in video gaze estimation;
(ii) Superior cross-domain performance on datasets such as MPIIFaceGaze and Gaze360 compared to frontal face methods.
Code and pre-trained models will be released to the community.
\end{abstract}
\vspace{-0.5cm}




%\maketitle

%%%%%%%%% BODY TEXT
\section{Introduction}

Despite the remarkable capabilities of large language models (LLMs)~\cite{DBLP:conf/emnlp/QinZ0CYY23,DBLP:journals/corr/abs-2307-09288}, they often inevitably exhibit hallucinations due to incorrect or outdated knowledge embedded in their parameters~\cite{DBLP:journals/corr/abs-2309-01219, DBLP:journals/corr/abs-2302-12813, DBLP:journals/csur/JiLFYSXIBMF23}.
Given the significant time and expense required to retrain LLMs, there has been growing interest in \emph{model editing} (a.k.a., \emph{knowledge editing})~\cite{DBLP:conf/iclr/SinitsinPPPB20, DBLP:journals/corr/abs-2012-00363, DBLP:conf/acl/DaiDHSCW22, DBLP:conf/icml/MitchellLBMF22, DBLP:conf/nips/MengBAB22, DBLP:conf/iclr/MengSABB23, DBLP:conf/emnlp/YaoWT0LDC023, DBLP:conf/emnlp/ZhongWMPC23, DBLP:conf/icml/MaL0G24, DBLP:journals/corr/abs-2401-04700}, 
which aims to update the knowledge of LLMs cost-effectively.
Some existing methods of model editing achieve this by modifying model parameters, which can be generally divided into two categories~\cite{DBLP:journals/corr/abs-2308-07269, DBLP:conf/emnlp/YaoWT0LDC023}.
Specifically, one type is based on \emph{Meta-Learning}~\cite{DBLP:conf/emnlp/CaoAT21, DBLP:conf/acl/DaiDHSCW22}, while the other is based on \emph{Locate-then-Edit}~\cite{DBLP:conf/acl/DaiDHSCW22, DBLP:conf/nips/MengBAB22, DBLP:conf/iclr/MengSABB23}. This paper primarily focuses on the latter.

\begin{figure}[t]
  \centering
  \includegraphics[width=0.48\textwidth]{figures/demonstration.pdf}
  \vspace{-4mm}
  \caption{(a) Comparison of regular model editing and EAC. EAC compresses the editing information into the dimensions where the editing anchors are located. Here, we utilize the gradients generated during training and the magnitude of the updated knowledge vector to identify anchors. (b) Comparison of general downstream task performance before editing, after regular editing, and after constrained editing by EAC.}
  \vspace{-3mm}
  \label{demo}
\end{figure}

\emph{Sequential} model editing~\cite{DBLP:conf/emnlp/YaoWT0LDC023} can expedite the continual learning of LLMs where a series of consecutive edits are conducted.
This is very important in real-world scenarios because new knowledge continually appears, requiring the model to retain previous knowledge while conducting new edits. 
Some studies have experimentally revealed that in sequential editing, existing methods lead to a decrease in the general abilities of the model across downstream tasks~\cite{DBLP:journals/corr/abs-2401-04700, DBLP:conf/acl/GuptaRA24, DBLP:conf/acl/Yang0MLYC24, DBLP:conf/acl/HuC00024}. 
Besides, \citet{ma2024perturbation} have performed a theoretical analysis to elucidate the bottleneck of the general abilities during sequential editing.
However, previous work has not introduced an effective method that maintains editing performance while preserving general abilities in sequential editing.
This impacts model scalability and presents major challenges for continuous learning in LLMs.

In this paper, a statistical analysis is first conducted to help understand how the model is affected during sequential editing using two popular editing methods, including ROME~\cite{DBLP:conf/nips/MengBAB22} and MEMIT~\cite{DBLP:conf/iclr/MengSABB23}.
Matrix norms, particularly the L1 norm, have been shown to be effective indicators of matrix properties such as sparsity, stability, and conditioning, as evidenced by several theoretical works~\cite{kahan2013tutorial}. In our analysis of matrix norms, we observe significant deviations in the parameter matrix after sequential editing.
Besides, the semantic differences between the facts before and after editing are also visualized, and we find that the differences become larger as the deviation of the parameter matrix after editing increases.
Therefore, we assume that each edit during sequential editing not only updates the editing fact as expected but also unintentionally introduces non-trivial noise that can cause the edited model to deviate from its original semantics space.
Furthermore, the accumulation of non-trivial noise can amplify the negative impact on the general abilities of LLMs.

Inspired by these findings, a framework termed \textbf{E}diting \textbf{A}nchor \textbf{C}ompression (EAC) is proposed to constrain the deviation of the parameter matrix during sequential editing by reducing the norm of the update matrix at each step. 
As shown in Figure~\ref{demo}, EAC first selects a subset of dimension with a high product of gradient and magnitude values, namely editing anchors, that are considered crucial for encoding the new relation through a weighted gradient saliency map.
Retraining is then performed on the dimensions where these important editing anchors are located, effectively compressing the editing information.
By compressing information only in certain dimensions and leaving other dimensions unmodified, the deviation of the parameter matrix after editing is constrained. 
To further regulate changes in the L1 norm of the edited matrix to constrain the deviation, we incorporate a scored elastic net ~\cite{zou2005regularization} into the retraining process, optimizing the previously selected editing anchors.

To validate the effectiveness of the proposed EAC, experiments of applying EAC to \textbf{two popular editing methods} including ROME and MEMIT are conducted.
In addition, \textbf{three LLMs of varying sizes} including GPT2-XL~\cite{radford2019language}, LLaMA-3 (8B)~\cite{llama3} and LLaMA-2 (13B)~\cite{DBLP:journals/corr/abs-2307-09288} and \textbf{four representative tasks} including 
natural language inference~\cite{DBLP:conf/mlcw/DaganGM05}, 
summarization~\cite{gliwa-etal-2019-samsum},
open-domain question-answering~\cite{DBLP:journals/tacl/KwiatkowskiPRCP19},  
and sentiment analysis~\cite{DBLP:conf/emnlp/SocherPWCMNP13} are selected to extensively demonstrate the impact of model editing on the general abilities of LLMs. 
Experimental results demonstrate that in sequential editing, EAC can effectively preserve over 70\% of the general abilities of the model across downstream tasks and better retain the edited knowledge.

In summary, our contributions to this paper are three-fold:
(1) This paper statistically elucidates how deviations in the parameter matrix after editing are responsible for the decreased general abilities of the model across downstream tasks after sequential editing.
(2) A framework termed EAC is proposed, which ultimately aims to constrain the deviation of the parameter matrix after editing by compressing the editing information into editing anchors. 
(3) It is discovered that on models like GPT2-XL and LLaMA-3 (8B), EAC significantly preserves over 70\% of the general abilities across downstream tasks and retains the edited knowledge better.


\section{Related Work}
Our work draws on and contributes to research in mobility aids and the built environment, online image-based survey for urban assessment, personalized routing applications and accessibility maps.

\subsection{Mobility Aids and the Built Environment}
People who use mobility aids (\textit{e.g.,} canes, walkers, mobility scooters, manual wheelchairs and motorized wheelchairs) face significant challenges navigating their communities.
Studies have repeatedly found that sidewalk conditions can significantly impede mobility among these users~\cite{bigonnesse_role_2018,fomiatti_experience_2014,f_bromley_city_2007,rosenberg_outdoor_2013, harris_physical_2015,korotchenko_power_2014}. 
In a review of the physical environment's role in mobility, \citet{bigonnesse_role_2018} summarized factors affecting mobility aid users, including uneven or narrow sidewalks (\textit{e.g.,}~\cite{fomiatti_experience_2014,f_bromley_city_2007}), rough pavements (\textit{e.g.,}~\cite{fomiatti_experience_2014,f_bromley_city_2007}), absent or poorly designed curb ramps (\textit{e.g.,}~\cite{rosenberg_outdoor_2013, f_bromley_city_2007, korotchenko_power_2014}), lack of crosswalks (\textit{e.g.,}~\cite{harris_physical_2015}), and various temporary obstacles (\textit{e.g.,}~\cite{harris_physical_2015}).

Though most research on mobility disability and the built environment has focused on wheelchair users~\cite{bigonnesse_role_2018}, mobility challenges are not experienced uniformly across different user populations~\cite{prescott_factors_2020, bigonnesse_role_2018}. 
For example, crutch users could overcome a specific physical barrier (such as two stairs down to a street), whereas motorized wheelchair users could not (without a ramp)~\cite{bigonnesse_role_2018}. 
Such variability demonstrates how person-environment interaction can differ based on mobility aids and environmental factors~\cite{sakakibara_rasch_2018,smith_review_2016}.
Further, mobility aids such as canes, crutches, or walkers are more commonly used than wheelchairs in the U.S.~\cite{taylor_americans_2014, firestine_travel_2024}: in 2022, approximately 4.7 million adults used a cane, crutches, or a walker, compared to 1.7 million who used a wheelchair~\cite{firestine_travel_2024}.
This underscores the importance of considering a diverse range of mobility aid users in urban accessibility research.
For example, \citet{prescott_factors_2020} explored the daily path areas of users of manual wheelchairs, motorized wheelchairs, scooters, walkers, canes, and crutches and found that the type of mobility device had a strong association with users' daily path area size.
Our study aims to further advance knowledge of how different mobility aid users perceive sidewalk barriers, with a more inclusive understanding of urban accessibility.

\begin{figure*}
    \centering
    \includegraphics[width=1\linewidth]{figures/figure-tutorial.png}
    \caption{Survey Part 2.1 showed all 52 images and asked participants to rate their passability based on their lived experience and use of their mobility aid. Above is the interactive tutorial we showed at the beginning of this part.}
    \Description{This figure shows a screenshot from the online survey. In survey part 2.1, participants were presented with 52 images and were asked to rate their passibility based on their lived experience and use of their mobility aid. The screenshot shows the interactive tutorial shown before this section.}
    \label{fig:survey-part2-instructions}
\end{figure*}

\subsection{Online Image-Based Survey for Urban Assessment}
Sidewalk barriers hinder individuals with mobility impairments not just by preventing particular travel paths but also by reducing confidence in self-navigating and decreasing one's willingness to travel to areas that might be physically challenging or unsafe~\cite{vasudevan_exploration_2016,clarke_mobility_2008}.
Prior work in this area traditionally uses three main study methods: in-person interviews (\textit{e.g}.~\cite{rosenberg_outdoor_2013,castrodale_mobilizing_2018}), GPS-based activity studies (\textit{e.g.,}~\cite{prescott_exploration_2021, prescott_factors_2020,rosenberg_outdoor_2013}), and online-questionnaires (\textit{e.g.,}~\cite{carlson_wheelchair_2002}). 
In-person interviews, while providing detailed and nuanced information, are limited by small sample sizes~\cite{rosenberg_outdoor_2013}. GPS-based activity studies involve tracking mobility aids user activity over a period of time, offering insights into movement patterns and activity space; however, these studies are constrained by geographical location~\cite{prescott_exploration_2021}. In contrast, online questionnaires can reach much larger populations and cover broader geographical regions, but they often yield high-level information that lacks the depth and nuance of the other approaches~\cite{carlson_wheelchair_2002}.
Our study aims to strike a balance between these approaches, capturing nuanced perspectives of mobility aid users about the built environment while maintaining a sufficiently large enough sample size for robust statistical analysis. 
Building on~\citet{bigonnesse_role_2018}'s work, we explore not only the types of factors considered to be barriers, but the \textit{intensity} of these barriers and their differential impacts.

Visual assessment of environmental features has long been employed by researchers across diverse fields, including human well-being~\cite{humpel_environmental_2002}, ecosystem sustainability~\cite{gobster_shared_2007}, and public policy~\cite{dobbie_public_2013}. 
These studies examine the relationship between images and the reactions they provoke in respondents or compare differences in reactions between groups.
Over the past decade, online visual preference surveys have gained popularity (\textit{e.g.,}~\cite{evans-cowley_streetseen_2014, salesses_collaborative_2013, goodspeed_research_2017}), where respondents are asked to make pairwise comparisons between randomly selected images.
Using this approach has two advantages: it adheres to the law of comparative judgment~\cite{thurstone_law_2017} by allowing respondents to make direct comparisons, and it prevents inter-rater inconsistency possible with scale ratings~\cite{goodspeed_research_2017}.
Additionally, online surveys generally offer advantages of increased sample sizes, reduced costs, and greater flexibility~\cite{wherrett_issues_1999}.
For people with disabilities, online surveys can be particularly beneficial. They help reach hidden or difficult-to-access populations~\cite{cook_challenges_2007,wright_researching_2005} and are believed to encourage more honest answers to sensitive questions~\cite{eckhardt_research_2007} by providing a higher level of anonymity and confidentiality~\cite{cook_challenges_2007, wright_researching_2005}.

\begin{figure*}
    \centering
    \includegraphics[width=1\linewidth]{figures/figure-comaprison-screenshot.png}
    \caption{In survey Part 2.2, participants were asked to perform a series of pairwise comparisons based on their 2.1 responses.}
    \Description{This figure shows a screenshot from the online survey. In Survey Part 2.2, participants were asked to perform a series of pairwise comparisons based on their 2.1 responses.}
    \label{fig:survey-part2b-pairwise}
\end{figure*}

\subsection{Personalized Routing Applications and Accessibility Maps}
Navigation challenges faced by mobility aid users can be mitigated through the provision of routes and directions that guide them to destinations safely, accurately, and efficiently~\cite{kasemsuppakorn_understanding_2015}. However, current commercial routing applications (\textit{e.g.}, \textit{Google Maps}) do not provide sufficient guidance for mobility aid users.
To address this gap, significant research has focused on routing systems for this population over the past two decades~\cite{barczyszyn_collaborative_2018, karimanzira_application_2006, matthews_modelling_2003, kasemsuppakorn_understanding_2015, volkel_routecheckr_2008, holone_people_2008, wheeler_personalized_2020, gharebaghi_user-specific_2021, ding_design_2007}.
One early, well-known prototype system is \textit{MAGUS}~\cite{matthews_modelling_2003}, which computes optimal routes for wheelchair users based on shortest distance, minimum barriers, fewest slopes, and limits on road crossings and challenging surfaces.
\textit{U-Access}~\cite{sobek_u-access_2006} provides the shortest route for people with three accessibility levels: unaided mobility, aided mobility (using crutch, cane, or walker), and wheelchair users.
However, U-Access only considers distance and ignores other
important factors for mobility aid users~\cite{barczyszyn_collaborative_2018}.
A series of projects by Kasemsuppakorn \textit{et al}.~\cite{kasemsuppakorn_personalised_2009, kasemsuppakorn_understanding_2015} attempted to create personalized routes for wheelchair users using fuzzy logic and \textit{Analytic Hierarchy Process} (AHP).

While influential, many personalized routing prototypes face limited adoption due to a scarcity of accessibility data for the built environment. 
Geo-crowdsourcing~\cite{karimi_personalized_2014}, a.k.a. volunteered geographic information (VGI)~\cite{goodchild_citizens_2007}, has emerged as an effective solution~\cite{karimi_personalized_2014, wheeler_personalized_2020}.
In this approach, users annotate maps with specific criteria or share personal experiences of locations, typically using web applications based on Google Maps or \textit{OpenStreetMap} (OSM)~\cite{karimi_personalized_2014}.
Examples include \textit{Wheelmap}~\cite{mobasheri_wheelmap_2017}, \textit{CAP4Access}~\cite{cap4access_cap4access_2014}, \textit{AXS Map}~\cite{axs_map_axs_2012}, and \textit{Project Sidewalk}~\cite{saha_project_2019}.
Recent research demonstrated the potential of using crowdsourced geodata for personalized routing~\cite{goldberg_interactive_2016, bolten_accessmap_2019,menkens_easywheel_2011, neis_measuring_2015}.
For example, \textit{EasyWheel}~\cite{menkens_easywheel_2011}, a mobile social navigation system based on OSM, provides wheelchair users with optimized routing, accessibility information for points of interest, and a social community for reporting barriers. 
\textit{AccessMap}~\cite{bolten_accessmap_2019} offers routing information tailored to users of canes, manual wheelchairs, or powered wheelchairs, calculating routes based on OSM data that includes slope, curbs, stairs and landmarks. 
Our work builds on the above by gathering perceptions of sidewalk obstacles from different mobility aid users to create generalizable profiles based on mobility aid type. We envision that these profiles can provide starting points in tools like Google Maps for personalized routing but can be further customized by the end user to specify additional needs (\textit{e.g.}, ability to navigate hills, \textit{etc.})

Beyond routing applications, our study data can contribute to modeling and visualizing higher-level abstractions of accessibility. 
Similar to \textit{AccessScore}~\cite{li_interactively_2018}, data from our survey can provide personalizable and interactive visual analytics of city-wide accessibility. By identifying both differences between mobility groups and common barriers within groups, we can develop analytical tools to prioritize barriers and assess the impact of their mitigation or removal, potentially benefiting the broadest range of mobility group users. Incorporating perceptions of passibility into urban planning processes provides a new dimension for urban planners' toolkits, which are often narrowly focused on compliance with ADA standards.




\section{Proposed approach}
\label{sec:approach}

For the task of review comment generation, Knowledge-Based Systems (KBS) draw on codified rules and expert knowledge to deliver feedback that is consistent with established coding standards and best practices. Static analyzers, a prominent example of KBS, systematically follow predefined guidelines to detect code issues, offering reliable and structured feedback. While KBS achieve high precision, they are limited in scope, covering only a subset of possible issues encountered during code changes. In contrast, Learning-Based Systems (LBS) harness the adaptive potential of language models, which, by training on historical data, can recognize intricate patterns and generate contextually relevant review comments. This adaptability allows LBS to cover a broader range of issues present in the dataset, though often at the expense of precision. In this work, we conjecture that by combining these two strategies, it is possible to achieve the best of both approaches, namely, broader issue coverage coupled with improved precision.

\subsection{Overview}

Figure~\ref{fig:combination} illustrates our approach, outlining three strategies to combine Knowledge-Based Systems (KBS) and Learning-Based Systems (LBS) to enhance code review automation. 

\begin{figure}[htbp!]
  \centering
  \includegraphics[width=1\linewidth]{figures/combination.png}
  \caption{Different strategies to combine learning and knowledge-based systems}
  \label{fig:combination} 
\end{figure}

These strategies leverage KBS insights at different stages of the LBS pipeline, specifically during data preparation, inference, and final output. The three strategies are as follows:

\begin{itemize} 
    \item \emph{Data-Augmented Training (DAT)}: In this strategy, we enhance the training dataset by augmenting a real-world dataset with synthetic data generated from both KBS and LBS. This enriched dataset is then used to fine-tune a language model, enabling the LBS to incorporate both data-driven patterns and rule-based knowledge. This combination helps the model gain a more comprehensive understanding of code review patterns, improving robustness in varied review scenarios.
    \item \emph{Retrieval-Augmented Generation (RAG)}: Here, KBS insights are integrated directly into the LBS inference process. Through RAG, relevant information is dynamically retrieved from KBS (i.e., static analysis results) and injected into the prompts during generation. By incorporating the results of the KBS into the instruction, the LBS aligns its responses with established coding standards and practices, providing feedback grounded in structured, rule-based knowledge.
    \item \emph{Naive Concatenation of Outputs (NCO)}: This strategy merges the feedback generated by KBS and LBS after inference, combining their outputs to produce a unified code review. By consolidating KBS’s rule-based precision with LBS’s contextual depth, NCO offers a comprehensive review that covers a broader range of potential issues.
\end{itemize}

These strategies allow the LBS to benefit from the structured, rule-based insights of KBS, enhancing its ability to generate accurate, contextually appropriate, and standards-compliant code review comments.


\subsection{Baseline Model Preparation and Static Analyzers Selection}

While our approach is applicable to a wide range of LLMs and static analysis tools, we propose a specific configuration to illustrate the three strategies and establish the baselines for validating our conjecture. To set up the baseline systems, we first defined the LBS. We fine-tuned a large language model on an extensive code review dataset \cite{li2022automating}, referred to as \(\mathcal{D}_{\mathcal{M}_i}\), which pairs code changes with detailed reviews.

The selected model for fine-tuning is \emph{CodeLlama-7b}, trained for comment generation (i.e., generating review comments from code changes) with the following hyperparameter settings. The training was conducted on four \emph{NVIDIA RTX 3090} GPUs, using a batch size of $4$ per device. To boost efficiency, we applied gradient accumulation with a step size of $4$, updating the optimizer only after multiple batches. We used 4-bit quantization to improve memory and computational efficiency. Additionally, we employed Quantized Low-Rank Adaptation (QLoRA) \cite{hu2021lora}, a Parameter-Efficient Fine-Tuning (PEFT) technique, with $r = 16$, $\alpha = 32$, and $dropout = 0.05$. This method decomposes weight updates into low-rank matrices, reducing the parameters needed for fine-tuning and optimizing training efficiency \cite{hu2021lora}.

This resulted in a model, denoted as \(\mathcal{M}_i\), capable of generating detailed human-like code reviews.
Since \(\mathcal{M}_i\) represents the LBS component and was trained using the data from \(\mathcal{D}_{\mathcal{M}_i}\), we used the test set to generate reviews by both static analyzers and the fine-tuned model \(\mathcal{M}_i\).

To focus on a relevant subset of available static analyzers, we filtered the test set to include only Java code samples, producing a subset of $27,267$ entries, denoted \(\mathcal{D}_o\). Each entry in \(\mathcal{D}_o\) is a tuple \( (f,c) \), where \( f \) represents the source code file and \( c \) denotes the code change. Here, \( c \) is input to the LBS, while \( f \) serves as input to the KBS.
We limited our selection of static analyzers to tools that process Java source code directly. Although this decision excludes tools designed for Java bytecode analysis, it allows for a broader range of issue types. Specifically, we selected two well-established static analyzers: PMD \cite{pmd} and Checkstyle \cite{checkstyle}, both of which are designed to identify potential issues directly in source code.

PMD is a static code analysis tool that identifies issues in code by applying a set of rules aimed at detecting common problems, which are categorized into eight groups: best practices, coding style, design, documentation, error-prone, multi-threading, performance, and security \cite{lenarduzzi2023critical}. By analyzing source code against these rules, PMD generates detailed reports highlighting areas for improvement and enables users to create custom rules for specific analyses \cite{oskouei2018comparing}.

Checkstyle is another static code analysis tool for Java that offers predefined style configurations for standard checks, including Google Java Style and Sun Java Style. Its rules cover various aspects such as annotations, class design, coding, and naming conventions. Checkstyle also supports custom configuration files tailored to user needs~\cite{hovemeyer2004finding, balachandran2013reducing, oskouei2018comparing, lenarduzzi2023critical}.




\begin{figure}[!htbp]
\begin{subfigure}{1\linewidth}
  \centering
  \includegraphics[width=1\linewidth]{figures/approaches/approach-FTr.pdf}
  \caption{Data augmented training}
  \label{fig:approach1}
\end{subfigure}
\\ \vspace{10pt}
\begin{subfigure}{1\linewidth}
  \centering
  \includegraphics[width=1\linewidth]{figures/approaches/approach-RAG.pdf}
  \caption{Retrieval augmented generation}
  \label{fig:approach2}
\end{subfigure}
\\ \vspace{10pt}
\begin{subfigure}{1\linewidth}
  \centering
  \includegraphics[width=1\linewidth]{figures/approaches/approach-Concat.pdf}
  \caption{Naive concatenation of outputs}
  \label{fig:approach3}
\end{subfigure}
\caption{Proposed strategies to combine LBS and KBS}
    %\\The forward arrows represent the forward pass of the model, the circled shapes represent data (inputs, outputs, and ground truth), and the double arrows represent the losses.}
\label{fig:approaches}
\end{figure}





\subsection{Data Augmented Training}

As shown in Figure~\ref{fig:approach1}, this strategy involves retraining the LBS using an augmented dataset \({Da}\), which includes review comments generated by both, static analyzers and the fine-tuned model \(\mathcal{M}_i\). Through this retraining process, the LBS learns from both data sources, producing a more refined model referred to as \(\mathcal{M}_{FT}\).


A simple approach to augmenting the dataset would have been to apply static analysis to the code in \(\mathcal{D}_{\mathcal{M}_i}\) and add or concatenate the generated comments with the existing ones. However, this method does not guarantee data quality within the augmented dataset and fails to account for the insights inferred by the LBS \(\mathcal{M}_i\). 
Therefore, we employ an ensemble learning approach where the two distinct sources—the LBS and the KBS—serve as \emph{experts} to generate data for fine-tuning a model. The underlying rationale is that both KBS and LBS reviews are inherently synthetic. By combining their outputs, we achieve a more balanced and consistent fine-tuning process.

To produce the augmented dataset \(\mathcal{D}_a\), we designed a two-step process (i.e., \emph{data generation} and \emph{data filtering}), as depicted in Figure~\ref{fig:approach}.


\begin{figure}[htbp!]
  \centering
  \includegraphics[width=\linewidth]{figures/dataset.png}
  \caption{Dataset augmentation pipeline}
  \label{fig:approach} 
\end{figure}


In the Data Generation phase, we used the original dataset \(\mathcal{D}_o\) as input. For each code change \( c \), we employed our fine-tuned model \(\mathcal{M}_i\) to generate four context-aware, human-like reviews. Simultaneously, static analyzer rules were applied to each source code \( f \) to produce structured and precise feedback. Each static analyzer generated a report containing several reviews, including the start and end sections of code where each issue was identified. Since our approach focuses on code changes, we extracted the code section highlighted in each review, adding a few context lines before and after each extracted segment.
We then merged the reviews generated by both the static analyzers and \(\mathcal{M}_i\) into a single, unified dataset. Each data point in the dataset consists of tuples in the form \( (f, c, r, t) \), where \( f \) represents the source code file, \( c \) is the code change, \( r \) denotes the review comment, and \( t \) indicates the method used to generate the review (either KBS or LBS). 
%Note that for the same code \( (f,c) \) in \(\mathcal{D}_o\), we may have multiple tuples—i.e., samples—each corresponding to a generated comment.

After data generation, we applied a systematic Data Filtering step to evaluate and refine the merged dataset, ensuring that only the most relevant and meaningful reviews were retained for each source code. While the fine-tuned model \(\mathcal{M}_i\) can generate context-aware, human-like reviews, its output may sometimes include irrelevant or less meaningful feedback, particularly when handling complex or ambiguous code changes. Similarly, static analyzers, although reliable, may produce output overloaded with false positives \cite{johnson2013don, aniche2020effectiveness}, making it difficult to separate significant concerns from noise.
Therefore, filtering both \(\mathcal{M}_i\)'s and the static analyzers' reviews was essential to maintain a dataset of high-quality, meaningful feedback. This filtering process involved rating each review based on its relevance to the corresponding code. The ratings provided a quantitative measure of the review quality generated by both static analyzers and \(\mathcal{M}_i\).


\begin{figure}[hbt!]
  \centering
  \includegraphics[width=1\linewidth]{figures/llama3_rating.png}
  \caption{Judgment of review comments using \emph{Llama3-70B}}
  \label{fig:llama3_rating}
\end{figure}


To ensure a fair and scalable rating system, we leveraged large language models, which have demonstrated remarkable performance in similar assessment tasks \cite{zheng2024judging, huang2024empirical, weyssow2024codeultrafeedback}. By using these models, we achieved a more accurate and consistent evaluation of each review, enhancing the dataset's quality and making it a valuable resource for fine-tuning. As shown in Figure~\ref{fig:llama3_rating}, we used \emph{Llama3-70b}, inputting the code and its corresponding reviews. We instructed the model to rate each review on a 10-point scale. A threshold rating of 8 was set, with only reviews surpassing this threshold retained in the final dataset.
After filtering for relevance and quality, we ensured that each comment exceeding the threshold was treated separately. For a source code \(f\) with \(n\) reviews \((r_1...r_n)\), we generated \(n\) distinct data points: \(<f,r_1>\),  \(<f,r_2>\),..., \(<f,r_n>\). Additionally, for each comment, we extracted and included the specific segment of code change related to the issue being addressed, the dataset was then structered as \(<c_1,r_1>\),  \(<c_2,r_2>\),..., \(<c_n,r_n>\).

To prevent overrepresentation of specific rules, we randomly discarded reviews associated with rules that have an excessively high number of reviews. Furthermore, to maintain a balanced dataset, we randomly discarded a subset of learning-based reviews, ensuring an equal distribution between knowledge-based and learning-based reviews.
 
The final dataset \(\mathcal{D}_a\) consists of $78,776$ samples, ensuring an equal representation of reviews generated by both KBS and LBS methodologies, as shown in Figure~\ref{fig:chart}. It also ensures a balanced distribution across all KBS rules. 
\begin{figure}[!htbp]
  \centering
  \vspace{-2em}
  \includegraphics[width=0.6\linewidth]{figures/chart.png}
  \caption{Distribution of LBS and KBS Reviews in Our Dataset}
  \label{fig:chart} 
\end{figure}


To obtain the \(\mathcal{M}_{FT}\) model, we partitioned \(\mathcal{D}_a\) dataset into three subsets: 80\% of the samples were assigned to the training set, while the remaining 20\% was equally divided, with 10\% designated for validation and 10\% for testing. Each subset maintained a balanced mix of LBS and KBS reviews. We then fine-tuned the CodeLlama-7b model on this dataset with QLoRA to optimize memory efficiency \cite{hu2021lora}. 



\subsection{Retrieval Augmented Generation}

Retrieval-Augmented Generation (RAG) is a technique designed to enhance the generative capabilities of language models by incorporating external knowledge into their prompts during the inference phase \cite{jiang2023active}. In the context of code review, this strategy can be used to embed KBS-generated feedback directly into the prompts of a language-based system, as shown in Figure~\ref{fig:approach2}.



In our approach, the fine-tuned model \(\mathcal{M}_i\) takes as input the code changes from the \(\mathcal{D}_a\) dataset, along with outputs from PMD and Checkstyle. Incorporating KBS knowledge into the prompt guides the model to produce more relevant and precise reviews. This combination ensures that the generated review comments are both comprehensive and contextually informed. As a result, the reviews generated align closely with established coding standards and best practices, thereby enhancing their overall quality. The augmented prompt is illustrated in Figure~\ref{fig:rag_prompt}.

\begin{figure}[htbp!]
  \centering
  \includegraphics[width=0.9\linewidth]{figures/rag_prompt.png}
  \caption{Prompt used to generate review comments using RAG}
  \label{fig:rag_prompt}
\end{figure}



\subsection{Naive Concatenation of Outputs}

The Naive Concatenation of Outputs strategy serves as a baseline approach in which review comments generated separately by the LBS and KBS for the same code are combined to form a single review. As illustrated in Figure~\ref{fig:approach3}, the review comment generated by \(\mathcal{M}_i\) is directly concatenated with the output from the static analyzer (either PMD or Checkstyle).
This approach is straightforward, requiring minimal adjustments to the inference pipeline while ensuring that the final review delivers comprehensive feedback from both systems.















\section{Experiments}
\seclabel{experiments}
Our experiments are designed to test a) the extent to which open loop execution is an issue for precise mobile manipulation tasks, b) how effective are blind proprioceptive correction techniques, c) do object detectors and point trackers perform reliably enough in wrist camera images for reliable control, d) is occlusion by the end-effector an issue and how effectively can it be mitigated through the use of video in-painting models, and e) how does our proposed \name methodology compare to large-scale imitation learning? 


\subsection{Tasks and Experimental Setup}
We work with the Stretch RE2 robot. Stretch RE2 is a commodity mobile manipulator with a 5DOF arm mounted on top of a non-holomonic base. We upgrade the robot to use the Dex Wrist 3, which has an eye-in-hand RGB-D camera (Intel D405). 
We consider 3 task families for a total
of 6 different tasks: a) holding a knob to pull open a cabinet or drawer, b) holding a
handle to pull open a cabinet, and c) pushing on objects (light buttons, books
in a book shelf, and light switches). Our focus is on generalization. {\it
Therefore, we exclusively test on previously unseen instances, not used during
development in any way.} 
\figref{tasks} shows the instances that we test on. 

All tasks involve some precise manipulation, followed by execution of a motion
primitive. {\bf For the pushing tasks}, the precise motion is to get the
end-effector exactly at the indicated point and the motion primitive is to push
in the direction perpendicular to the surface and retract the end-effector 
upon contact. The robot is positioned such
that the target position is within the field of view of the wrist camera. A user
selects the point of pushing via a mouse click on the wrist camera image. The
goal is to push at the indicated location. Success is determined by whether the
push results in the desired outcome (light turns on / off or book gets pushed in). 
The original rubber gripper bends upon contact, we use a rigid known tool
that sticks out a bit. We take the geometry of the tool into account while servoing.

{\bf For the opening articulated object tasks}, the precise manipulation is grasping the
knob / handle, while the motion primitive is the whole-body motion that opens
the cupboard. Computing and executing this full body motion is difficult. We
adopt the modular approach to opening articulated objects (MOSART) from Gupta \etal~\cite{gupta2024opening} and invoke it
after the gripper has been placed around the knob / handle. The whole tasks 
starts out with the robot about 1.5m way from the target object, with the 
target object in view
from robot's head mounted camera. We use MOSART to compute articulation
parameters and convey the robot to a pre-grasp
location with the target handle in view of the wrist camera. At this point,
\name (or baseline) is used to center the gripper around the knob / handle, 
before resuming MOSART: extending the gripper till contact, close the gripper, and play rest of the predicted motion plan. Success is 
determined by whether the cabinet opens by more than $60^\circ$
or the drawer is pulled out by more than $24cm$, similar to the criteria used in \cite{gupta2024opening}.


For the precise manipulation part, all baselines consume the current and
previous RGB-D images from the wrist camera and output full body motor
commands.

% % Please add the following required packages to your document preamble:
% % \usepackage{graphicx}
% \begin{table*}[!ht]
% \centering
% \caption{}
% \label{tab:my-table}
% \resizebox{\textwidth}{!}{%
% \begin{tabular}{lcccccc}
% \toprule
%  & \multicolumn{2}{c}{ours} & \multicolumn{2}{c}{Gurobi} & \multicolumn{2}{c}{MOSEK} \\
%  & \multicolumn{1}{l}{time (s)} & \multicolumn{1}{l}{optimality gap (\%)} & \multicolumn{1}{l}{time (s)} & \multicolumn{1}{l}{optimality gap (\%)} & \multicolumn{1}{l}{time (s)} & \multicolumn{1}{l}{optimality gap (\%)} \\ \hline
% \begin{tabular}[c]{@{}l@{}}Linear Regression\\ Synthetic \\ (n=16000, p=16000)\end{tabular} & 57 & 0.0 & 3351 & - & 2148 & - \\ \hline
% \begin{tabular}[c]{@{}l@{}}Linear Regression\\ Cancer Drug Response\\ (n=822, p=2300)\end{tabular} & 47 & 0.0 & 1800 & 0.31 & 212 & 0.0 \\ \hline
% \begin{tabular}[c]{@{}l@{}}Logistic Regression\\ Synthetic\\ (n=16000, p=16000)\end{tabular} & 271 & 0.0 & N/A & N/A & 1800 & - \\ \hline
% \begin{tabular}[c]{@{}l@{}}Logistic Regression\\ Dorothea\\ (n=1150, p=91598)\end{tabular} & 62 & 0.0 & N/A & N/A & 600 & 0.0 \\
% \bottomrule
% \end{tabular}%
% }
% \end{table*}

% Please add the following required packages to your document preamble:
% \usepackage{multirow}
% \usepackage{graphicx}
\begin{table*}[]
\centering
\caption{Certifying optimality on large-scale and real-world datasets.}
\vspace{2mm}
\label{tab:my-table}
\resizebox{\textwidth}{!}{%
\begin{tabular}{llcccccc}
\toprule
 &  & \multicolumn{2}{c}{ours} & \multicolumn{2}{c}{Gurobi} & \multicolumn{2}{c}{MOSEK} \\
 &  & time (s) & opt. gap (\%) & time (s) & opt. gap (\%) & time (s) & opt. gap (\%) \\ \hline
\multirow{2}{*}{Linear Regression} & \begin{tabular}[c]{@{}l@{}}synthetic ($k=10, M=2$)\\ (n=16k, p=16k, seed=0)\end{tabular} & 79 & 0.0 & 1800 & - & 1915 & - \\ \cline{2-8}
 & \begin{tabular}[c]{@{}l@{}}Cancer Drug Response ($k=5, M=5$)\\ (n=822, p=2300)\end{tabular} & 41 & 0.0 & 1800 & 0.89 & 188 & 0.0 \\ \hline
\multirow{2}{*}{Logistic Regression} & \begin{tabular}[c]{@{}l@{}}Synthetic ($k=10, M=2$)\\ (n=16k, p=16k, seed=0)\end{tabular} & 626 & 0.0 & N/A & N/A & 2446 & - \\ \cline{2-8}
 & \begin{tabular}[c]{@{}l@{}}DOROTHEA ($k=15, M=2$)\\ (n=1150, p=91598)\end{tabular} & 91 & 0.0 & N/A & N/A & 634 & 0.0 \\
 \bottomrule
\end{tabular}%
}
% \vspace{-3mm}
\end{table*}

\begin{figure*}
\insertW{1.0}{figures/figure_6_cropped_brighten.pdf}
\caption{{\bf Comparison of \name with the open loop (eye-in-hand) baseline} for opening a cabinet with a knob. Slight errors in getting to the target cause the end-effector to slip off, leading to failure for the baseline, where as our method is able to successfully complete the task.}
\figlabel{rollout}
\end{figure*}

\begin{table}
\setlength{\tabcolsep}{8pt}
  \centering
  \resizebox{\linewidth}{!}{
  \begin{tabular}{lcccg}
  \toprule
                              & \multicolumn{2}{c}{\bf Knobs} & \bf Handle & \bf \multirow{2}{*}{\bf Total} \\
                              \cmidrule(lr){2-3} \cmidrule(lr){4-4}
                              & \bf Cabinets & \bf Drawer & \bf Cabinets & \\
  \midrule
  RUM~\cite{etukuru2024robot}  & 0/3    & 1/4         & 1/3         & 2/10 \\
  \name (Ours) & 2/3    & 2/4         & 3/3     &  7/10 \\
  \bottomrule
  \end{tabular}}
  \caption{Comparison of \name \vs RUM~\cite{etukuru2024robot}, a recent large-scale end-to-end imitation learning method trained on 1200 demos for opening cabinets and 525 demos for opening drawers across 40 different environments. Our evaluation spans objects from three environments across two buildings.}
  \tablelabel{rum}
\end{table}

\subsection{Baselines}
We compare against three other methods for the precise manipulation part of
these tasks. 
\subsubsection{Open Loop (Eye-in-Hand)} To assess the precision requirements of
the tasks and to set it in context with the manipulation capabilities of the
robot platform, this baseline uses open loop execution starting from estimates
for the 3D target position from the first wrist camera image.
\subsubsection{MOSART~\cite{gupta2024opening}}
The recent modular system for opening cabinets and drawers~\cite{gupta2024opening}
reports impressive performance with open-loop control (using the head camera from 1.5m away), combined with proprioception-based feedback to 
compensate for errors in perception and control when interacting with handles. 
We test if such correction is also sufficient for interacting with knobs. Note 
that such correction is not possible for the smaller buttons and pliable books.

\subsubsection{\name (no inpainting)} To understand how much of an issue
occlusion due to the end-effector is during manipulation, we ablate the use of
inpainting. %

\subsubsection{Robot Utility Models (RUM)~\cite{etukuru2024robot}}
For the opening articulated object tasks, we also compare to Robot Utility Models (RUM), 
a closed-loop imitation learning method recently proposed by Etukuru et al. \cite{etukuru2024robot}.
RUM is trained on a substantial dataset comprising expert demonstrations, including 
1,200 instances of cabinet opening and 525 of drawer opening, gathered from roughly 
40 different environments.
This dataset stands as the most extensive imitation 
learning dataset for articulated object manipulation to date, establishing RUM as a 
strong baseline for our evaluation.

Similar to our method, we use MOSART to compute articulation
parameters and convey the robot to a pre-grasp location
with the target handle in view of the wrist camera.
One of the assumptions of RUM is a good view of the handle.
To benefit RUM, we try out three different heights of the wrist camera,
and \textit{report the best result for RUM.}

\begin{figure*}
\insertW{1.0}{figures/figure_9_cropped_brighten.pdf}
\caption{{\bf \name \vs open loop (eye-in-hand) baseline for pushing on user-clicked points}. Slight errors in getting to the target cause failure, where as \name successfully turns the lights off. Note the quality of CoTracker's track ({\color{blue} blue dot}).}
\figlabel{rollout_v2}
\end{figure*}

\begin{figure*}
\insertW{1.0}{figures/figure_5_v2_cropped_brighten.pdf}
\caption{{\bf Comparison of \name with and without inpainting}. Erroneous detection without inpainting causes execution to fail, where as with inpainting the target is correctly detected leading to a successful grasp and a successful execution.}
\figlabel{rollouts2}
\end{figure*}


\subsection{Results}
\tableref{results} presents results from our experiments. 
Our training-free approach \name successfully 
solves over 85\% of task instances that we test on.
As noted, all these
tests were conducted on unseen object instances in unseen
environments that were not used for development in any way. We discuss our key
experimental findings below.

\subsubsection{Closing the loop is necessary for these precise tasks} 
While the proprioception-based strategies proposed in MOSART~\cite{gupta2024opening}
work out for handles, they are inadequate for targets like knobs and just
don't work for tasks like pushing buttons. Using estimates from the wrist
camera is better, but open loop execution still fails for knobs and pushing
buttons. 

\subsubsection{Vision models work reasonably well even on wrist camera images}
Inpainting works well on wrist camera images (see \figref{occlusion} and \figref{inpainting}).
Closing the loop using feedback from vision detectors and point trackers on
wrist camera images also work well, particularly when we use in-painted images.
See some examples detections and point tracks in \figref{rollout} and \figref{rollout_v2}. 
Detic~\cite{zhou2022detecting} was able to reliably detect the knobs and
handles and CoTracker~\cite{karaev2023cotracker} was able to successfully track
the point of interaction letting us solve 24/28 task instances.

\subsubsection{Erroneous detections without inpainting hamper performance on 
handles and our end-effector out-painting strategy effectively mitigates it} 
As shown in \figref{rollouts2}, presence of the end-effector caused the object
detector to miss fire leading to failed execution. Our out painting approach
mitigates this issue leading to a higher success rate than the 
approach without out-painting. Interestingly, CoTracker~\cite{karaev2023cotracker} is quite robust
to occlusion (possibly because it tracks multiple points) and doesn't benefit
from in-painting. 


\subsubsection{Closed-loop imitation learning struggles on novel objects}
As presented in \tableref{rum}, \name significantly outperforms RUM in a paired evaluation on unseen objects across three novel environments. A common failure mode of RUM is its inability to grasp the object's handle, even when it approaches it closely.
Another failure mode we observe is RUM misidentifying keyholes or cabinet edges as handles, also resulting in failed grasp attempts.
These result demonstrate that a modular approach that leverages the broad generalization capabilities of vision foundation models is able to generalize much better than an end-to-end imitation learning approach trained on 1000+ demonstrations, which must learn all aspects of the task from scratch.



\section{Conclusion}

In this paper, we propose a sample weight averaging strategy to address variance inflation of previous independence-based sample reweighting algorithms. 
We prove its validity and benefits with theoretical analyses. 
Extensive experiments across synthetic and multiple real-world datasets demonstrate its superiority in mitigating variance inflation and improving covariate-shift generalization.  


\clearpage
{
    \small
    \bibliographystyle{ieeenat_fullname}
    \bibliography{reference}
}

% WARNING: do not forget to delete the supplementary pages from your submission 

%%%%%%%%% Merge with supplemental materials %%%%%%%%%%


\twocolumn[{%
 \centering
 \LARGE \papertitle \\[1em]
 \large Supplementary Material\\[1em]
}]

%\begin{center}
%\textbf{\large Supplemental Materials: Modality Agnostic 3D Gaze Estimation in the Wild}
%\end{center}

%%%%%%%%% Merge with supplemental materials %%%%%%%%%%
%%%%%%%%% Prefix a "S" to all equations, figures, tables and reset the counter %%%%%%%%%%
\renewcommand*{\thesection}{\Alph{section}}
\setcounter{section}{0}
\setcounter{equation}{0}
\setcounter{figure}{0}
\setcounter{table}{0}
%\setcounter{page}{1}
\makeatletter
\renewcommand{\theequation}{S\arabic{equation}}
\renewcommand{\thefigure}{S\arabic{figure}}
\renewcommand{\thetable}{S\arabic{table}}

%\renewcommand{\bibnumfmt}[1]{[S#1]}
%\renewcommand{\citenumfont}[1]{S#1}
%%%%%%%%%% Prefix a "S" to all equations, figures, tables and reset the counter %%%%%%%%%%



\mypartitle{What is expected?} The supplementary material consists of datasets details, experiments details, and extended experiments analysis mentioned in the main paper. In addition, videos of qualitative examples of our method on VideoAttentionTarget further demonstrate the robustness in challenging real-world scenarios.  

\section{Datasets Details}

\subsection{Datasets}
\label{sec:supp_dataset}
\mypartitle{ Gaze360 (G360).} \cite{Gaze360} is video 3D gaze datasets. It is collected in both indoor and outdoor environments in unconstrained setting, which contains 3D gaze of 238 subjects with a wide-range head pose and gaze direction. G360 is recorded at 8FPS. In all of our experiments, we always used the same training set as \cite{Gaze360} with 126928 samples. For the test set, we followed the split of \cite{Gaze360} where G360 Full corresponds to "All 360°" (the entire test set) with 25969 samples, G360 180 corresponds to "Front 180°" (gaze within 90°) with 20322 samples, and G360 40 to "Front Facing" (gaze within 20°) with 3995 samples. In addition to those splits, we consider G360 Back (gaze above 90°) \cite{chen2020360} with 5647 samples and finally G360 Face (all detected faces) with 16031 samples, which is used in many constrained gaze studies \cite{zhang2017s,chen2018appearance,Rt-gene,cheng2020coarse,cheng2022gaze,abdelrahman2023l2cs,yan2023gaze,catruna2024crossgaze}. When we refer to G360 Face 180 (15895 samples), it corresponds to the detected face with a gaze within 90°, a subset of G360 180, the same for G360 Face 40 with 3687 samples. We used the validation set described in \cite{Gaze360} with 17038 samples.

\mypartitle{GFIE.} \cite{GFIE} is a video 3D gaze dataset collected indoors with 71799 frames from 61 subjects (27 male and 34 female). It is an unconstrained dataset with a wide range of head poses. It was collected for gaze following task; using a complex calibrated laser setup, they can infer the 3D gaze from the eye to the visual target direction. They recorded people doing various indoor activities at 30 fps. We follow the data splits described in \cite{GFIE}, 59217 for training, 6281 for validation, and 6281 for testing.

\mypartitle{MPSGaze (MPS).} \cite{MPS} is a modified 3D gaze datasets that has been automatically generated using ETH-Xgaze \cite{ETH} eyes. They apply a blending technique on people from the Widerface~\cite{yang2016wider} dataset to put eyes with a known 3D gaze from ETH on heads with similar head poses. This dataset is diverse, with more than 10k identities and challenging poses, appearances, and lighting conditions. However, the blending process reduces the quality of the visual appearance, and it contains only near frontal head poses and no back view. We used the same training and test split with 24282 samples in training and 6277 samples in testing. No validation is defined in this work. 

\mypartitle{EYEDIAP (EDIAP).} \cite{Eyediap} is a 3D gaze video dataset. It includes videos from 16 subjects (30 fps), using either screen targets (CS, DS subset EDIAP) or 3D floating balls ( FT subset EDIAP-FT) as gaze targets. It is a constrained setup with mainly frontal head poses. 
%Compared to EDIAP with screen target, EDIAP-FT contains a wider gaze and head pose distribution. 
Following \cite{wang2022contrastive,Cheng2021Survey}, we used the evaluation set under screen target session (CS, DS, namely EDIAP) with 16674 samples from 14 subjects. 
%In addition, we consider the floating target session (FT, namely EDIAP-FT) with 8346 samples.

\mypartitle{MPIIFaceGaze (MPII).} \cite{MPIIGaze} is a 3D gaze image dataset collected from 15 subjects in a screen-based gaze target setup, resulting in a constrained dataset with mostly frontal head pose. We follow the standard evaluation protocol \cite{MPIIGaze,wang2022contrastive,Cheng2021Survey}, which selects 3000 images from each subject to form an evaluation set for a total of 45000 samples. 

\mypartitle{GazeFollow (GF).} \cite{recasens2015they} is a 2D gaze image dataset annotated on in the wild dataset for the gaze the following task. The 2D target label corresponds to where a given person is looking at in the image. It is a diverse dataset that includes various head poses, appearances, scenes, and lighting conditions. Overall, it has around 130K annotated person-target instances in 122K images.

\subsection{Video Processing}
As mentioned in the main section, for video clip input, our approach predicts the 3D gaze from an 8-frame video clip. However, video datasets have different frame rates, which can impact the gaze prediction. In this work, since G360 has a lower frame rate, we resample EYEDIAP and GFIE to match G360's frame rate of 8 fps. 

\subsection{Gaze Representation} 
Working with different 3D gaze datasets requires a unified way to define and represent the 3D gaze vector. Usually, in constrained gaze estimation, studies use data normalization to map the input image to a normalized space where a virtual camera is used to warp the face patch out of the original input image according to the 3D head pose \cite{ETH}. Thus, the gaze is expressed in this virtual camera coordinate defined by the 3D head pose. \\
%
However, in unconstrained settings, it is not possible to get access to a robust and reliable 3D head pose; thus, we follow the gaze representation of Gaze360 \cite{Gaze360} in the ``Eye coordinate system". The practical interpretation of the eye coordinate system is that the positive x-axis points to the left, the positive y-axis points up, and the positive z-axis points away from the camera, \ie [-1,0,0] is a gaze looking to the right or [0,0,-1] straight into the camera from the camera's point of view, irrespective of subjects position in the world. The origin of the gaze vector is the middle of the eyes, except for MPS and MPII, where the gaze origin is the average of 3D eyes and mouth landmarks resulting in an origin located at the middle of the nose, and for GF, we used the center of the head bounding box as the origin.

\begin{figure}
    \centering
    \includegraphics[width=0.99\linewidth]{images/head_crop_plot.png}
    \vspace{-4mm}
    \caption{Input head crop using different scales. In our work, a scale of -0.1 is used and proved to be effective in both constrained and frontal face setting \cref{sec:crop_size} }
    \label{fig:head_crop_input}
\end{figure}

\section{Experiments Details}

\mypartitle{Metric.}
We follow the test split described in the state-of-the-art method and explained in \cref{sec:supp_dataset}. As a metric, we use the standard angular error in degrees between the predicted and ground truth gaze prediction \cite{Gaze360,ETH,Eyediap,MPIIGaze}. Previous methods reporting video evaluation used a 7-frame video clip and predict the middle frame gaze direction. Since our approach outputs eight gaze directions from an 8-frame video clip, for a fair comparison, we use the 4th gaze prediction of an 8-frame video clip to compute the evaluation metric. 

\mypartitle{Training.}
We used the same setup in all the experiments to be as fair as possible. All the models are trained for a minimum of 20 epochs. We used an early stopping on the validation set with a patience of 10 epochs. We use the AdamW optimizer \cite{AdamW} with a learning rate of 1e-4 and a cosine annealing schedule with a 5 epochs linear warmup (from 2e-5 to 1e-4). For evaluation, we report the performance of the best model defined by the best angular error on the validation set.

\mypartitle{Data augmentation.}
Data augmentation is crucial for robust gaze estimation in the wild. In this work, we used standard data augmentation techniques. First, we applied jittering during the head crop to introduce slight variations in scale and aspect ratio, which reduces the model's sensitivity to noisy or imprecise head bounding boxes. Next, color jittering was applied by adjusting brightness, contrast, and saturation, making the model more resilient to diverse lighting conditions commonly encountered in real-world scenarios. Since gaze labels, such as those in the GF 2D dataset, may exhibit bias toward one side, we applied horizontal flipping to the images while appropriately adjusting the gaze direction, ensuring more balanced training data in the yaw gaze direction. These augmentations collectively improved the model’s ability to handle variations in data and enhance its generalization to unseen environments.

\section{Additional Experiments}

\subsection{Effect of Head Crop Size}
\label{sec:crop_size}
As mentioned by Chen \etal~\cite{chen2020360}, the input head crop scale impacts the 3D gaze estimation. We find that the effect on the prediction depends on the head orientation. \cref{fig:head_crop_input} illustrates the different inputs with different head crop scales. As shown in \cref{fig:supp_head_crop_effect}, a smaller head crop tighter to the face improves 3D gaze estimation on frontal head poses, while a larger head crop improves gaze on the non-frontal head pose. Indeed, as shown in \cref{fig:head_crop_input}, a tighter crop increases the eye resolution in the image and a larger crop provides more context about the head orientation and upper body orientation, which gives a strong prior for the gaze direction when eyes are not visible.
In the context of gaze estimation in the wild, a scale of -10\% is part of the Pareto front as illustrated in \cref{fig:supp_head_crop_effect} and is also the best on the G360 Full image as shown in \cref{fig:supp_head_crop_angular}. Therefore, it is a reasonable trade-off between frontal and back view performance. We use it for all our experiments.

\begin{figure}[htbp]
    \centering
    \begin{subfigure}{\linewidth}
        \centering
        \includegraphics[width=0.99\textwidth]{images/Head_crop_angular_error.png}
        \caption{Effect of head bounding box scale as input on the 3D gaze angular error on G360 Full test set. A scale ratio of 0.1 corresponds to a 10\% bounding box scale.}
        \label{fig:supp_head_crop_angular}
    \end{subfigure}
    
    
    \begin{subfigure}{\linewidth}
        \centering
        \includegraphics[width=0.8\textwidth]{images/Head_crop_effect.png}
        \caption{Effect of head bounding box scale on the angular error with respect to G360 Back and G360 40 test subset.}
        \label{fig:supp_head_crop_effect}
    \end{subfigure}
    
    \caption{\textbf{Effect of head crop size.}}
    \label{fig:head_crop}
\end{figure}

\subsection{Constrain Gaze Evaluation}

% Please add the following required packages to your document preamble:
% \usepackage{graphicx}
\begin{table}[t]
\centering
\resizebox{\columnwidth}{!}{%
\begin{tabular}{llrrrrrrr}
\toprule
 &  & \multicolumn{1}{c}{MPII} & \multicolumn{1}{l}{} & \multicolumn{2}{c}{EDIAP}  \\ \cline{3-3} \cline{5-6} 
Method & Training Dataset & Img &  & Img & Vid  \\ 
\hline
PureGaze~\cite{cheng2022puregaze} (Res18)        & G360I Face & 9.3 &   & 9.2 & -  \\
Liu \etal~\cite{liu2021generalizing} (Res18)      & G360I Face & 7.7 &   & 9.0 & -  \\
Liu \etal~\cite{liu2021generalizing} (Res50)     & G360I Face & 8.3 &   & 7.5 & - \\
RAT~\cite{bao2022generalizing} (Res18)     & G360I Face & 7.6 &   & \textcolor{gray}{7.1}$^*$ & -  \\
RAT~\cite{bao2022generalizing} (Res50)     & G360I Face & 7.7 &   & \textcolor{gray}{7.1}$^*$ & -  \\
CDG~\cite{wang2022contrastive} (Res50)           & G360I Face & 7.0 &   & \textbf{7.3} & -  \\
\midrule
 Supervised (\model) & G360I\&V             & 7.43 &   & 8.88 & 8.28  \\
 \framework (\model) & G360I\&V+GF          & \textbf{6.43} &   & 8.87 & 8.19  \\
 %\model & G360I\&V+GF+MPS       \\
\bottomrule
\end{tabular}%
}
\caption{\textbf{Comparision with state-of-the-art on constrained domain generalization benchmarks}. All these methods \cite{cheng2022puregaze,liu2021generalizing,wang2022contrastive,ververas20223dgazenet,bao2022generalizing} use a face crop as input and are trained on the detected face subset of Gaze360. Our method is trained and tested on head crop which makes it more general but more challenging for frontal gaze estimation. $^*$ In \cite{bao2022generalizing} they used only 6400 sample for EDIAP but we follow \cite{Cheng2021Survey,cheng2022puregaze,wang2022contrastive} with 16674 samples. }
\label{tab:generalization_constrained}
\vspace{-3mm}
\end{table}
\vspace{-1mm}


The objective of this work is to improve unconstrained gaze estimation in the wild. As seen in \cref{sec:crop_size}, compared to a tight face crop a larger crop improves gaze in challenging head pose. Therefore, a larger crop is more suited to our objective.  In contrast, some methods specialize in frontal gaze estimation and rely on tight face crops, which provide better resolution for the eye regions. While this is not a fully fair comparison, we compare our approach to these constrained methods for generalization on constraint benchmarks. Note that for the constrained methods, models are trained and tested only on a subset of detected faces (G360 Face), while in our approach the model is trained on G360 Full. \\
As shown in \cref{tab:generalization_constrained}, on MPII, the supervised GaT lags behind the best method by 6\%. On EDIAP, GaT is 21\% behind the best method in image evaluation but narrows the gap to 13\% when evaluated on videos. Then, when using our \framework learning framework including GF labels, we observe an important improvement on MPII with state-of-the-art angular error of 6.43 compared to 7 from CDG. On EDIAP the improvement is marginal. Compared to EDIAP, MPII has more diversity in lighting conditions and environment. GF doesn't contain a lot of frontal gaze direction but has a broad diversity of environments. Therefore, the improvement on MPII should come from the additional diversity that GF brings but this is not useful for EDIAP prediction. 
While constrained methods excel in frontal settings, they fail in unconstrained scenarios. Our approach, which achieves state-of-the-art performance in unconstrained environments (G360, GFIE) while remaining competitive in constrained settings (MPII, EDIAP), proves to be a versatile and robust solution for gaze estimation in the wild.

\subsection{Qualitative Analysis}

\begin{figure}[t]
    \centering
    \includegraphics[width=0.8\linewidth]{images/plot_image_video_gaze360.png}
    \caption{ \textbf{Image vs video predictions, where does it help?}. GaT trained on G360I\&V and tested on G360 Full image and video. The difference between image and video angular error with respect to the ground truth gaze directions from the camera ([0,0,-1]). The mean and standard deviation are displayed for each 10° bin. Positive values indicate better performance in video prediction compared to image prediction. }
    \label{fig:graph_img_vs_video}
\end{figure}

\mypartitle{When does temporal context contribute most effectively?}
As seen in the main paper, video prediction consistently outperforms image prediction. To understand the significance of temporal context in gaze estimation, we examined cases with large angular errors between image and video predictions. Several key observations emerged. As illustrated in \cref{fig:temporal_pros} in the first two rows, temporal context proves valuable during blinks, as it allows the model to interpolate gaze direction when the eyes are closed. If the head pose is not informative, temporal context helps disambiguate between blinking and looking down since the eyes are not visible, as shown in row 1. Additionally, when individuals are viewed entirely from behind (rows 6-7), video inferences provide a more consistent gaze direction in relation to time. Thus, there is less jittering and it might improve the prediction accuracy. In rows 4-5, the head and eye motion can be used in video prediction to improve the gaze direction. Finally, it can help in case of occlusion, as seen in row 3.\\
Furthermore, we explore the impact of image- and video-based prediction with respect to gaze direction. Indeed, we expected more improvement when people are from the back since additional head motion cues can be useful for gaze estimation. In the results, video prediction on G360 Back clearly improves image prediction. In addition, in \cref{fig:graph_img_vs_video}, we plot the difference between image and video prediction angular error for different gaze directions. If we look at the trend, video prediction seems to be better, especially for gaze over 150°, but given the standard deviation, it might not be a statistically significant observation. A more detailed analysis by considering only cases where there is a head motion can better highlight the impact of video prediction.\\
\mypartitle{What are the limitations of temporal context for gaze?}
We investigate prediction made on the VideoAttentionTarget~\cite{Chong_2020_CVPR} (VAT) videos using our \framework framework and \model model. VAT is a challenging dataset with real-world scenarios, various appearances, and diverse gaze distribution, making it well-suited for assessing our approach. Our qualitative analysis reveals two limitations of video-based inference compared to image-based inference using our model. The first limitation arises in cases of rapid head rotation, as illustrated in \cref{fig:temporal_cons}, temporal context may be misused, leading to predictions that do not align with the actual gaze. It might be because no rapid head motion is present in the G360 training sets. 
The second aspect involves cases of ``gaze recentering", where the gaze direction returns to its initial position following a shift. This behavior can occur very rapidly, within just 3-4 frames. Due to the smoothing effect in the temporal modeling, the predicted gaze may not exhibit the same amplitude as the actual movement. Indeed, this behavior is not present in the G360 dataset, and the use of videos sampled at 8 frames per second may limit the ability to capture fine-grained gaze dynamics. However, such behavior is better captured during image-based inference. This highlights a trade-off: while video-based inference provides smoother and more robust predictions, image-based inference offers greater accuracy but can result in jittery outputs. To mitigate the lack of natural gaze behavior we apply our \framework framework using 2D gaze video data from VAT. Unfortunately, since current benchmarks don't contain natural gaze behavior, the results don't show quantitative improvement. Further research to evaluate this aspect is needed.   \\
\mypartitle{In which scenarios does \framework with GazeFollow labels provide the most benefit?}
We demonstrated the advantages of \framework with GazeFollow labels across various benchmarks, both within- and cross-datasets. But in which scenarios does it outperform supervised methods trained solely on G360? To address this question, we analyze predictions made in real-world scenarios using the VideoAttentionTarget (VAT) dataset~\cite{Chong_2020_CVPR}. Our findings reveal that \framework achieves the most notable improvements in cases of extreme head poses, particularly when the head is facing downward, as shown in \cref{fig:sup_vs_stwsge}. It is also more robust to appearance diversity like hair partially occluding the face or varying skin tones. It also helps in difficult lighting conditions and low-resolution inputs.
Additionally, we include a video (provided in the supplementary materials) displaying predictions on VAT with an explanation, enabling a direct comparison between the two methods and a clearer visualization of our approach's performance on real-world data.

\begin{figure*}[t]
    \centering
    \includegraphics[width=1\linewidth]{images/presentation1.pdf}
    \caption{ \textbf{Illustration of image against video prediction.} Comparison between single-image (frame 0) and video predictions (frame -3 to 4). We use our \framework learning framework with \model trained on G360 and GF. All examples are from G360 test set. Rows 1-2 illustrate eye blinks, Row 3 shows an example of occlusion, Rows 4-5 demonstrate frontal head/eyes motion, and Rows 6-7 depict back view prediction. In the last row, the first two frames are not part of the test subset. Arrows in \textcolor{red}{red} represent image predictions, and arrows in \textcolor{magenta}{magenta} are video predictions. The angular error between groundtruth and prediction is displayed in red at the top right corner. The circles in the images represent unit disks where 3D gaze vectors are projected onto the image plane (x,y in yellow) and a top view (x,z in blue)}
    \label{fig:temporal_pros}
\end{figure*}

\begin{figure*}[t]
    \centering
    \includegraphics[width=1\linewidth]{images/presentation2_over.pdf}
    \caption{\textbf{Illustration of image and video prediction in case of rapid head motion.} We use our \framework learning framework with \model trained on G360 and GF. All examples are from VideoAttentionTarget~\cite{Chong_2020_CVPR} (VAT). Arrows in \textcolor{red}{red} represent image predictions, and arrows in \textcolor{magenta}{magenta} are video predictions. The circles in the images represent unit disks where 3D gaze vectors are projected onto the image plane (x,y in yellow) and a top view (x,z in blue). Note that since VAT has a frame per second (fps) of 24 and G360 has a fps of 8, we show the temporal context used for video inference corresponding to 8 fps. }
    \label{fig:temporal_cons}
\end{figure*}

\begin{figure*}[t]
    \centering
    \includegraphics[width=1\linewidth]{images/presentation3.pdf}
    \caption{\textbf{Illustration of supervised against \framework learning framework with GazeFollow label.} We use in both experiments our \model model. All examples are from VideoAttentionTarget~\cite{Chong_2020_CVPR} (VAT). Arrows in \textcolor{blue}{blue} represent image predictions with supervised \model trained on G360, and arrows in \textcolor{red}{red} are image predictions with \framework \model trained on G360 and GF. The circles in the images represent unit disks where 3D gaze vectors are projected onto the image plane (x,y in yellow) and a top view (x,z in blue). }
    \label{fig:sup_vs_stwsge}
\end{figure*}



\end{document}

