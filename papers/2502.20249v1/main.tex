% CVPR 2025 Paper Template; see https://github.com/cvpr-org/author-kit

\documentclass[10pt,twocolumn,letterpaper]{article}

%%%%%%%%% PAPER TYPE  - PLEASE UPDATE FOR FINAL VERSION
%\usepackage{cvpr}              % To produce the CAMERA-READY version
% \usepackage[review]{cvpr}      % To produce the REVIEW version
\usepackage[pagenumbers]{cvpr} % To force page numbers, e.g. for an arXiv version
%\usepackage{comment}
\usepackage{arydshln}
% Import additional packages in the preamble file, before hyperref
\newcommand{\CG}{\mathcal{G}\xspace}
\newcommand{\CV}{\mathcal{V}\xspace}
\newcommand{\CE}{\mathcal{E}\xspace}
\newcommand{\CA}{\mathcal{A}\xspace}
\newcommand{\CF}{\mathcal{F}\xspace}
\newcommand{\CR}{\mathcal{R}\xspace}
\newcommand{\CB}{\mathcal{B}\xspace}
\newcommand{\CX}{\mathcal{X}\xspace}
\newcommand{\CK}{\mathcal{K}\xspace}
\newcommand{\CM}{\mathcal{M}\xspace}
\newcommand{\CC}{\mathcal{C}\xspace}
\newcommand{\CL}{\mathcal{L}\xspace}
\newcommand{\CI}{\mathcal{I}\xspace}
\newcommand{\CQ}{\mathcal{Q}\xspace}
\newcommand{\CO}{\mathcal{O}\xspace}
\newcommand{\CP}{\mathcal{P}\xspace}
\newcommand{\CS}{\mathcal{S}\xspace}
\newcommand{\CT}{\mathcal{T}\xspace}
\newcommand{\CJ}{\mathcal{J}\xspace}
\usepackage[para]{footmisc}
\usepackage{subfig}
% \usepackage{subcaption}
% \usepackage{array}
% \usepackage{colortbl}



% It is strongly recommended to use hyperref, especially for the review version.
% hyperref with option pagebackref eases the reviewers' job.
% Please disable hyperref *only* if you encounter grave issues, 
% e.g. with the file validation for the camera-ready version.
%
% If you comment hyperref and then uncomment it, you should delete *.aux before re-running LaTeX.
% (Or just hit 'q' on the first LaTeX run, let it finish, and you should be clear).
\definecolor{cvprblue}{rgb}{0.21,0.49,0.74}
\usepackage[pagebackref,breaklinks,colorlinks,allcolors=cvprblue]{hyperref}

%%%%%%%%% PAPER ID  - PLEASE UPDATE
\def\paperID{7487} % *** Enter the Paper ID here
\def\confName{CVPR}
\def\confYear{2025}

%%%%%%%%% Custom latex
\newcommand{\p}{\textsc{p}}

\newcommand{\data}{\mathcal{D}}
\newcommand{\extdata}{\mathcal{D}_e}
\newcommand{\shots}{S}
\newcommand{\classes}{C}
\newcommand{\numbershadowmodels}{M}

\newcommand{\hpofunction}{$\texttt{HPO}$}
\newcommand{\trainfunction}{$\textsc{train}$}

\newcommand{\hyperparamsdata}{\eta_{{}_{\data}}}
\newcommand{\hyperparamsextdata}{\eta_{{}_{\extdata}}}

\newcommand{\model}{\mathcal{M}} 
\newcommand{\modeltar}{\mathcal{M}_{\mathcal{T}}} 
\newcommand{\modelshadow}{\mathcal{M}_{\mathcal{S}}} 
\newcommand{\return}{\textbf{return }}
\newcommand{\grid}{\mathbb{M}}

\newcommand{\tpr}{\textsc{tpr}}
\newcommand{\fpr}{\textsc{fpr}}


\newcommand{\lira}{\mathrm{LiRA}} 

\newcommand{\datashadow}{\mathcal{D}_{\mathrm{shadow}}}
\newcommand{\architecture}{\mathcal{A}}
\newcommand{\prob}{\mathbb{P}}
\newcommand{\normal}{\mathcal{N}}
\newcommand{\attack}{\texttt{KNOWN}}
\newcommand{\bb}{\texttt{BLACK-BOX}}
\newcommand{\logits}{\textsc{logits}}


%%%%%%%%% AUTHORS - PLEASE UPDATE
\author{Pierre Vuillecard\\
IDIAP, EPFL\\
{\tt\small pierre.vuillecard@idiap.ch}
% For a paper whose authors are all at the same institution,
% omit the following lines up until the closing ``}''.
% Additional authors and addresses can be added with ``\and'',
% just like the second author.
% To save space, use either the email address or home page, not both
\and
Jean-Marc Odobez\\
IDIAP, EPFL\\
% First line of institution2 address\\
{\tt\small odobez@idiap.ch}
}

\begin{document}

%%%%%%%% TITLE - PLEASE UPDATE
\title{ \papertitle }

\twocolumn[{%
\renewcommand\twocolumn[1][]{#1}%
\maketitle


\vspace{-1.2cm}
\begin{center}
    \centering
    \captionsetup{type=figure}
    \includegraphics[width=1.\textwidth]{images/Main_figure_gaze.pdf}
    \vspace{-6mm}
    \captionof{figure}{
    \textbf{Significance of ST-WSGE.} Our self-training based weakly-supervised framework for robust 3D gaze estimation in real-world conditions (e.g., varying appearance, extreme poses, resolution, and occlusion).  
     All predictions used our image and video agnostic Gaze Transformer (\model) model.
     Top row: importance of the training diversity using ST-WSGE and GazeFollow (GF) for generalization compared to standard supervised methods. 
     Bottom row: influence of temporal context between image and video inference.  Circles in  images represent unit disks where 3D gaze vectors are projected onto the image plane (x, y in yellow) and a top-down view (x, z in blue). Images from VideoAttentionTarget, GFIE, and MPIIFaceGaze datasets.
}
    \label{fig:intro}
\end{center}%
}]

%%%%%%%%% ABSTRACT

\begin{abstract}
Accurate 3D gaze estimation in unconstrained real-world environments remains a significant challenge due to variations in appearance, head pose, occlusion, and the limited availability of in-the-wild 3D gaze datasets. To address these challenges, we introduce a novel Self-Training Weakly-Supervised Gaze Estimation framework (\framework). This two-stage learning framework leverages diverse 2D gaze datasets, such as gaze-following data, which offer rich variations in appearances, natural scenes, and gaze distributions, and proposes an approach to generate 3D pseudo-labels and enhance model generalization.
Furthermore, traditional modality-specific models, designed separately for images or videos, limit the effective use of available training data. To overcome this, we propose the Gaze Transformer (\model), a modality-agnostic architecture capable of simultaneously learning static and dynamic gaze information from both image and video datasets. By combining 3D video datasets with 2D gaze target labels from gaze following tasks, our approach achieves the following key contributions:
(i) Significant state-of-the-art improvements in within-domain and cross-domain generalization on unconstrained benchmarks like Gaze360 and GFIE, with notable cross-modal gains in video gaze estimation;
(ii) Superior cross-domain performance on datasets such as MPIIFaceGaze and Gaze360 compared to frontal face methods.
Code and pre-trained models will be released to the community.
\end{abstract}
\vspace{-0.5cm}




%\maketitle

%%%%%%%%% BODY TEXT
\section{Introduction}

% State of the world (robots for creative activites)
The term ``robot,'' originally signifying `forced labor,' has long been associated with labor and work. Robots have demonstrated their utility in various automated productive and social contexts, where the primary goals are improving productivity, safety, and fostering social interactions with humans~\cite{simoes2022designing, weidemann2021role, honig2018understanding}. However, an increasing number of cases feature using of robots in creative settings. Unlike productive contexts, where the focus is on efficiency and task completion~\cite{arents2022smart}, or social contexts, where communication and trust are prioritized~\cite{nam2020trust, saunderson2019robots}, creative environments prioritize artistic innovation and expression~\cite{hsueh2024counts}. This shift fundamentally alters the dynamics of human-robot interaction, redefining the roles and expectations for both humans and robots.

For instance, robots’ social behaviors are leveraged to support the generation and expression of creative ideas~\cite{hu2021exploring, sandoval2022human, alves2020creativity}, and programmable robotic movements and trajectories are employed to inspire artistic activities such as sketching~\cite{lin2020your}. These studies often engage participants from creative fields who possess limited prior experience with robotics, and are typically conducted in short-term, experimental settings. Consequently, the findings from these studies remain constrained since much can be learned from professional practitioners' experiences to inform system design such as digital fabrication~\cite{hirsch2023nothing}. There is a notable gap in research examining the long-term, active, and practical experience of integrating robotic systems into the creative processes. As a result, the deeper insights into how robots facilitate and shape creative processes, beyond simply augmenting human creativity, remain underexplored. In this study, we aim to better understand the impacts of robots on creative processes and outcomes.

As early as Leonardo da Vinci's 16th century ``Automaton,'' artists have explored the creative affordances of robotic systems~\cite{shanken2002cybernetics, pagliarini2009development, jeon2017robotic}. The artistic creation process typically encompasses various stages, including the exploration of materials and techniques, ongoing experimentation and iteration, and the continual refinement of the artists' insights into their creative subjects~\cite{lewis2023art, sturdee2022state}. Therefore, investigating the artistic process involving robots offers an opportunity to gain deeper insights into robots' creative potential. Robotic art, in particular, provides a compelling case for this exploration.

We define robotic art as artworks that utilize robotic or automated machines to create artistic experiences and tangible artifacts. One example is robotic installation art, in which robots are programmed to follow specific rules that embody the artist’s expression (\autoref{fig:teaser} (a)). Another example is responsive art, in which robots react to their environment, with behaviors that change over time or in response to spectators (\autoref{fig:teaser} (b)). Additionally, there are robotic creators, which possess a degree of agency, allowing them to collaborate with human artists and produce works that extend beyond mere replication of human-created art (\autoref{fig:teaser} (c) and (d)). As such, robotic art becomes a rich case for exploring human-machine interactions in creative contexts. Gaining a deeper understanding of how robots facilitate artistic expression can provide insights for designing computing systems to support creative activities~\cite{gomez2021robot}.

% Therefore, we did...
We draw on semi-structured, in-depth interviews with renowned professional robotic artists to investigate the use of robots in artistic practice. Specifically, our goal is to understand how artistic exploration of robotic systems challenges conventional assumptions about the functions of robots, such as their roles in automating repetitive tasks or serving human needs. We also explore the implications of robots in the artistic process and examine how creativity may emerge within robotic art. To address these interrelated inquiries, our study focuses on the practice of robotic art, posing the research question: \textit{How do robotic artists utilize robots in their artistic practice?} We approach this inquiry through the perspectives and experiences of robotic artists, who creatively design, modify, and repurpose robotic systems for artistic expression and exploration.

% The key findings are...
Our findings highlight the social, material, and temporal dimensions of artists' practices that shape their creativity and artistic outcomes. The creation of robotic art is largely a social process, as artists receive both explicit and implicit feedback through the audience's reactions and reception of their work. Simultaneously, the embodiment and malfunctions inherent to robotic systems drive artistic experimentation. The temporal processes of creation and exhibition, beyond just the final product, further enhance the creative value. Our empirical analysis presents how creativity emerges through the interplay of social, material, and temporal interactions among artists, robots, audiences, and the environment.

% The contributions of this work are...
We make two main contributions to HCI in this study. 
First, we elucidate the interactive mechanisms among key actors---human creators, machines, audiences, and environments---within the practice of robotic art, a topic that remains underexplored in HCI. Our findings reveal the significance of sociality (e.g., interactions between artists and audiences), materiality (e.g., the embodiment and malfunctions of robots), and temporality (e.g., the processes of creation and exhibition) in shaping creative values. We propose that these three facets are central to the creative process and facilitate the emergence of creativity in robotic art.
Second, drawing from the findings, we offer implications for \textit{socially informed}, \textit{material-attentive}, and \textit{process-oriented} creation with computing systems. We suggest leveraging these three aspects to enhance creativity and the creative experience. Specifically, we discuss the value of incorporating implicit audience feedback, designing with technical malfunctions, and focusing on the post-creation process to foster alternative creative experiences with machines~\cite{alter2010designing, juarez2022glitch}.



%\subsection{End-User Programming}

%\kenneth{The way I like to think about Related Work is that this section should (sometimes subtly, not explicitly, but effectively!) answer some underlying questions that reviewers might want to ask. So, here we go:}\steven{sounds good!}

\subsection{Ways of Optimizing Prompts for LLMs}
%\subsection{Prompt Engineering and How Good Humans Are at It}
Prompts are the primary means by which users interact with, utilize, and instruct LLMs. 
Since the emergence of these models, researchers and developers have invested significant effort into understanding how to craft better prompts for more effective use. 

\paragraph{Automatic Prompt Optimization.}
Much of the prior work has focused on automatically optimizing prompts. 
A common theme across these studies is the use of gold-standard labels to guide the optimization process.
For example, \citet{pryzant2023automatic} introduced an automatic prompt optimization method inspired by gradient descent; 
\citet{manas2024improving} presented an approach that begins with a user prompt and iteratively generates revised prompts to maximize consistency between the generated image and prompt, without human intervention; 
\citet{wan2024teach} explored two types of prompt optimization, instruction and exemplar, and suggested that combining both can yield optimal results; 
\citet{sun2023autohint} combined zero-shot and few-shot learning to optimize prompts automatically; %eliminating the need for manual prompt engineering; 
and \citet{levi2024intent} improved prompt optimization through synthetic data generation and iterative refinement, focusing on aligning prompts with user intent by creating challenging boundary cases for iterative prompt refinement.
While these studies were interesting and relevant, they generally assumed the availability of gold-standard labels and did not address situations where labels are absent or where standards are constantly evolving.

\paragraph{User-Driven Prompt Optimization.}
In addition to automatic prompt optimization, some research has focused on human capabilities in optimizing prompts. 
\citet{zhou2023revisiting} found that manual prompting often outperforms automated methods in various scenarios; 
\citet{10.1145/3544548.3581388} discovered that people tend to design prompts opportunistically rather than systematically, which often leads to lower success rates. 
To the best of our knowledge, the most relevant prior work is by \citet{wang2024end}, who developed an iterative refinement system that enables users to prompt LLMs to build a personalized classifier for social media content. 
Their study explored three user strategies for improving prompts and measured their effectiveness. 
While conceptually related to our work, their focus was not on how users evolve their understanding and expectations when interacting with LLMs. 
Instead, participants labeled ground truth at the beginning of the study, prior to using the system.



%--------------------- dead kitten --------------
\begin{comment}
 





The most relevant prior work is by \citet{wang2024end}, who developed an iterative refinement system allowing users to prompt LLMs to build a personalized classifier for social media content.
While their work is closely related to ours in concept, their study did not focus on how users evolve their understanding and expectations while working with LLMs. 
Instead, participants labeled ground truth at the outset before using the system.


\kenneth{The key question for our paper is this: Did prior work try to measure users' prompt engineering performance *over multiple iterations*? (What do we know about human performance in prompt engineering?) I think you can maybe find some papers, especially papers for automatic prompt optimization like DSPy, measuring users' individual prompt's output accuracy (or MSE) or performance (e.g., BLEU in generation task), but it might be hard to find papers capture and measure *multiple iterations* from the same user for the same prompt.--This is the main argument for our paper: we did something that was hard and thus has not been done.}

\kenneth{Take a look at this survey paper:~\cite{chen2023unleashing}}



\steven{iterative tool involve human}
PromptIDE is an interactive tool that helps the experts to iteratively refine tools by providing various prompts, visualizing their performance on small validation datasets, and iterative optimizing them based on quantitative feedback~\cite{strobelt2022interactive}. \steven{gold label exists}

PromptAID is a visual analytics system that helps non-experts iteratively improve prompts through exploration, perturbation, testing, and refinement. It supports prompts through keyword adjustment, paraphrasing, and adding few-shot examples. \steven{has test dataset, it is a complex system}

\steven{automate prompting}
\citet{pryzant2023automatic} introduces an automatic prompt optimization prompt inspired by gradient descent. \steven{this fell into software designing, involve gold labels}

The study starts from a user prompt and iteratively generates revised prompts with the goal of maximizing a consistency score between the generated image and prompt without a human in the loop\cite{manas2024improving}\steven{without human involvement in the loop, gold labels}

\citet{zhou2023revisiting} found that manual prompting often performed better than automated methods in various steps. 

\cite{wan2024teach} explores the distinction between two types of prompt optimization: instruction optimizer and exemplar. This study suggested combining both approaches could lead to optimal results.

\cite{sun2023autohint} combines zero-shot and few-shot learning to optimize prompts automatically, without manual efforts in prompt engineering.

\cite{levi2024intent} improve prompt engineering optimization by synthetic data generation and iterative refinement, focusing on aligning prompts with user intent by generating challenging boundary cases and using these to refine the prompt iteratively.





\paragraph{Prompt Engineering Tools.}
\kenneth{After making the first point, we can have a follow-up paragraph to say that many tools were created to help people do prompt engineering (list a few and name their focuses), but again, they did not focus on measuring how good humans are in prompt engineering--- Of course, there could be an argument that suggests: no matter how good you are, you will always need some tool. It is true---for example, ChainForge basically create a easy-to-use UI that make things easier, which is not really about accuracy---But for annotation tasks, performance is still critical and it is always good to know how well human did, almost like many AI leaderboard has various "human" performance for comparison.}
PromptMaker, a platform for rapidly prototyping new ML models using prompt-based programming, was difficult to evaluate their prompts systematically~\cite{10.1145/3491101.3503564}.

\cite{arawjo2024chainforge}  is an Open-source visual toolkit for prompt engineering and on-demand hypothesis testing of text-generation LLMs.

 promptfoo is test-driven LLM development, not trial-and-error, producing matrix views that let you quickly evaluate outputs across many prompts~\cite{webster2023promptfoo}.

\cite{madaan2024self} introduces a method that LLM iterative improve their output by using their own feedback, without external supervision. 

\saniya{austin etal points:
1. used only COPRO, evaluation criteria utilized a custom LLM-as-a-judge metric. The paper showed that their automated prompt optimizer worked better tha DSPy }
   
\end{comment}


\subsection{Tools for Prompt Engineering}
With the advances in LLMs, numerous tools have been developed to assist with prompt engineering. 
Most of these tools follow a software-engineering paradigm, where testing (such as unit tests or integration tests) is a central concept, and thus often assume the existence of gold-standard labels.
For example, PromptIDE is an interactive tool that helps experts iteratively refine prompts by providing various inputs, visualizing their performance on small validation datasets, and optimizing them based on quantitative feedback~\cite{strobelt2022interactive}; 
PromptAid is a visual analytics system for interactively creating, refining, testing, and iterating prompts while tracking accuracy changes~\cite{mishra2023promptaid};
%It allows users to adjust prompts through keyword modifications, paraphrasing, and adding few-shot examples; 
ChainForge is an open-source visual toolkit for prompt engineering and on-demand hypothesis testing of text-generation LLMs~\cite{arawjo2024chainforge};
and, promptfoo applies a test-driven approach to LLM development, producing matrix views that enable quick evaluation of outputs across multiple prompts~\cite{webster2023promptfoo}.
While these tools are inspiring and valuable, the scenarios we focus on do not rely on the constant availability of gold labels.

%\cite{mishra2023promptaid}


\begin{comment}






\kenneth{In here, we want to answer this questions: Why do we need to built \system? Can't we just use some existing tools??? The underlying answer could be: all the tools, including the one we mentioned in previous subsection, were not really aiming for ``general users'' and only thing general users can reliably use is probably chat interface come with ChatGPT etc.}

\citet{10.1145/3544548.3581388} mentioned that people tended to design prompts opportunistically, not systematically, which resulted in less success. \system provides a systematic process for composing and refining prompts, allowing non-expert users to adapt to the prompt creation process effortlessly.

\saniya{Amy Zhang points:
\newline 1. Accuracy didnot improve; reported improvements in recall
\newline 2. Observed that humans are pretty bad at being consistent
\newline 3. Quoted  Miles Turpin, Julian Michael, Ethan Perez, and Samuel Bowman. 2024. Language models don't always say what they think: unfaithful explanations
in chain-of-thought prompting. Advances in Neural Information Processing Systems 36 (2024).
Han Wang, Ming Shan Hee, Md Rabiul Awal, Kenny Tsu Wei Choo, and Roy Ka-Wei Lee. 2023. Evaluating GPT-3 Generated Explanations for
Hateful Content Moderation. arXiv:2305.17680 [cs.CL] for not using LLM prompt explanations
\newline 4. They had a bigger training set of around 700 examples: paper excerpt: "This process resulted in a balanced dataset of 800 comments. We randomly divided our dataset into a training dataset and a test dataset of 100 examples for each participant. The training dataset was used to help participants create their classiiers, whereas the test dataset was labeled by participants and used to evaluate their created classiiers."
}
    
\end{comment}

\subsection{Human-LLM Collaborative Data Annotation}
%Another relevant area of research involves using LLMs for data annotation. 
Beyond simply treating LLMs as automatic labelers---common in countless NLP projects~\cite{tan2024large}---a growing body of work explores how to combine human and LLM efforts to achieve better annotation outcomes, such as improved accuracy or speed.
Even as LLMs outperform humans in many labeling tasks, human-AI collaboration often produces better results than either alone~\cite{vaccaro2024combinations}.
For example, \citet{kim2024meganno+} introduced a human-LLM collaborative annotation system where LLMs handle bulk annotation tasks, while humans selectively verify and refine the annotations. 
%\steven{However, this system was limited to deployment within Jupyter Notebook, lacking an end-to-end solution. This design imposed significant barriers, as it required users to possess technical expertise for system setup before using the tool, limiting accessibility and scalability in non-technical domains.}
\citet{goel2023llms} proposed an approach that combines LLMs with human expertise to efficiently generate ground truth labels for medical text annotation.
Additionally, \citet{10.1145/3613904.3642834} demonstrated how aggregating crowd workers' labels with GPT-4's output can achieve higher labeling accuracy than either source alone.
These studies generally aim to split the workflow of data labeling between humans and LLMs in a smart way, making the task more effective or efficient. 

In contrast, our work does not focus on dividing or combining the workload, but on how humans can teach LLMs---through prompt refinement---to better label the specific type of data.
Few prior studies have emphasized iterative prompt refinement in human-LLM collaborative data annotation.
For example, \citet{liu2024harnessing} developed a workflow for video content analysis, refining prompts to improve LLM-generated annotations and align them with human judgment.
Additionally, \citet{zhang2023llmaaa} proposed LLMAAA, which uses LLMs as annotators in a feedback loop to label data efficiently.
Their study shows that poorly designed prompts result in subpar performance, especially in complex tasks. %while incorporating demonstrations and aligning label descriptions with natural language significantly enhances accuracy and reliability.
Our work advances this relatively understudied area of human-LLM collaborative annotation research.

%----------------------------- dead kitten --------------------------------

\begin{comment}








\steven{\citet{vaccaro2024combinations} emphaized that designing innovative processes for integrating humans and AI is as critical as developing advanced AI technologies. This aligns with the need for LLM-powered systems that iteratively guide AI outputs to meet user-specific standards, prioritizing effective collaboration between users and AI systems.}

\steven{\citet{liyanage2024gpt} found that GPT-4, using few-shot, zero-shot, and Chain-of-Thoughts (CoT) prompting techniques, could not outperform models fine-tuned on human-labeled data. Among these, the few-shot approach exhibited the highest degree of similarity to human annotations. However, in scenarios where gold labels are unavailable, fine-tuning is not applicable, and alternative methods must be explored.}

\steven{\citet{liu2024harnessing} developed a workflow for video content analysis, iteratively crafting prompts to enhance LLMs' ability to generate structured annotations and comprehensive explanations that aligned with human judgment. }

\steven{\citet{zamfirescu2023herding} found that while prompts can effectively address most UX goals, they struggle with nuanced, edge-case, or spontaneous interactions. The study highlights that the effectiveness of each instruction in the prompt is highly sensitive to its phrasing and location. Additionally, highly prescriptive prompts, though reliable, limited the spontaneity and flexibility of GPT responses.
In our system, users are only required to provide task information—such as task descriptions, rules, and examples—to construct instructions, allowing for greater flexibility in accommodating diverse task requirements..}

\steven{\citet{guyre2024prompt} illustrates how prompt engineering can empower non-experts to design tailored conversational agents by iteratively refining prompts and infusing domain-specific knowledge. Their study emphasizes democratizing chatbot development, allowing users to align AI behavior with their specific goals and values.}

\steven{\citet{zhang2023llmaaa} proposes LLMAAA that leverages LLMs as Active Annotators in a feedback loop to efficiently annotate data. The study highlights that poorly designed prompts lead to suboptimal performance by LLM annotators, particularly in complex or domain-specific tasks. However, incorporating demonstrations and aligning label descriptions with natural language significantly enhances annotation accuracy and reliability.}

%\kenneth{Here, we then answer this question: Did people create ANYTHING to support LLM-powered data annotation? There are two parts of the answer to this: 1) Many or even most papers, including our CHI paper last year, focus on the labeling performance of LLMs, for example, as compared to crowdsourcing. They did not focus on the UI aspect of it. 2) Some prompt chaining tools, like ChainForge, can support workflow like this, but (a) hey do not focus on data annotation in particular so some functions are missing, like data resampling, and (b) more importantly, they do not aim to support general users. Most of them expect you to know some programming, e.g., ChainForge clearly say it's a visual programming tool. They're not really aiming for generic users.}


\cite{kim2024meganno+} introduced a human-LLM collaborative annotation system that allows LLM to handle bulk annotation tasks while humans verify selectively to refine annotation. 

\cite{goel2023llms} introduced an approach that combines LLM wth human expertise to create an efficient method for generating
ground truth labels for medical text annotation.


\cite{shankar2024validates} introduced a tool, EvalGen, to address the challenge of validating LLM. 
EvalGen helps users design evaluation criteria for LLM outputs and align that evaluation with human preferences through a mixed-initiative system.
A key finding is the concept of criteria drift, where users modify their evaluation standards while grading outputs. 


\cite{brade2023promptify} Promptify utilizes an LLM-powered suggestion engine to help users quickly explore and craft diverse prompts for text-to-image generation tasks.

    
\end{comment}


%\subsection{Survey Study in Data Annotation}
%\steven{
We conducted a survey study to investigate how individuals interact with LLMs and utilize gold-standard labels in the data annotation process. 
The participants primarily represent roles in research, machine learning engineering, and software development. \\
\textbf{Workflows: }Participants described diverse workflows for integrating LLMs into data annotation process, highlighting a common iterative and human-in-the-loop approach. \textbf{Most workflows begin with manual annotation of a small subset of data to establish a baseline.} Participants then employ prompt engineering, iteratively refining LLM prompts by evaluating their performance against the manually annotated subset. \\
Once refined, the prompts are used to label larger datasets, with participants using tools or manual checks to review the LLM's annotations and identify any invalid labels. The process is typically concluded with a thorough manual verification of the dataset. \\
One participant mentioned they manually tabulate data points along with their descriptions. \\
\textbf{Initialize Prompting: }Most participants use their pre-defined prompts to initialized the annotation on their known tasks. 
For new tasks, one participant mentioned that they initialize the annotation process with LLMs by starting with a clear problem definition and iteratively refining a classification-based approach. For less familiar tasks, some participants may seek suggestions from the LLM to guide the initial setup.
\textbf{Revising Prompt: } Participants use a small dataset to finetune the prompt. They address issues by adding rules or context examples to tackle failure cases. When inconsistencies or error arise, they revisit and recheck the manually tagged dataset to improve performance. Some participants also engage the LLM by asking questions about data points and their descriptions, retraining to against inconsistencies to minimize hallucinations and enhance annotation reliability.
}

\subsection{Gold-Standard Labels in Annotation Tasks}\label{sec:related-work-gold-label}
Decades of research have shown that gold-standard labels play a critical role in quality control for data annotation pipelines~\cite{han2020crowd,gadiraju2015training,le2010ensuring,doroudi2016toward,hettiachchi2021challenge}.
Embedding items with known labels into the data annotation process allows requesters to reliably capture quality signals, 
such as workers' level of expertise~\cite{abraham2016many, abassi2019worker, yang2018improving} %\kenneth{TODO: Add refs about using gold labels to decide workers' expertise level}\steven{added}
or attentiveness to tasks~\cite{hettiachchi2021challenge, oleson2011programmatic}. %\kenneth{TODO: Add refs about using gold labels to do attention checks for workers}\steven{added}
These insights enable requesters to take appropriate actions, such as 
retraining annotators~\cite{le2010ensuring, doroudi2016toward,hettiachchi2021challenge}, %\kenneth{TODO: Add refs about retraining workers}\steven{added}
removing low-performing workers~\cite{10.1145/3613904.3642834, snow2008cheap,downs2010your,le2010ensuring}, %\kenneth{TODO: Add refs about removing or blocking low-performing workers}\steven{added}
or identifying potential issues in the annotation interfaces~\cite{toomim2011utility,10.1145/3613904.3642834, rahmanian2014user, komarov2013crowdsourcing}. %\kenneth{TODO: Add refs for crowd worker interfaces. At least cite: Toomim, M., Kriplean, T., Pörtner, C., \& Landay, J. (2011, May). Utility of human-computer interactions: Toward a science of preference measurement. In Proceedings of the SIGCHI Conference on Human Factors in Computing Systems (pp. 2275-2284).}\steven{added}
Gold labels are also beneficial for requesters during post-annotation data processing. 
They can be used to weight labels from different workers in label aggregation~\cite{abassi2017gold,abassi2019worker}, %\kenneth{TODO: Add label aggregation methods that use gold labels particularly to weight different workers}\steven{added}
improve label aggregation strategies~\cite{khattak2011quality, snow2008cheap},  %\kenneth{TODO: Add label aggregation methods that learn whatever from gold labels}\steven{added}
or 
exclude unreliable workers' outputs entirely~\cite{abassi2019worker}. %\kenneth{TODO: Cite ref using gold labels to remove workers from label aggregation}\steven{added}
Beyond requesters, gold labels are also beneficial for data labelers like crowd workers. 
Gold labels can be used to train workers~\cite{doroudi2016toward, le2010ensuring, gadiraju2015training,han2020crowd}, %\kenneth{TODO: Cite ref that uses gold labels for worker training}\steven{added}
provide real-time feedback to help them recalibrate their understanding of the task~\cite{le2010ensuring,hettiachchi2021challenge}, %\kenneth{TODO: Cite the visible gold paper from Amazon}\steven{added}
or remind them to pay more attention~\cite{ hettiachchi2021challenge,oleson2011programmatic}. %\kenneth{TODO: Cite attention check papers}\steven{amazon paper also warn workers in real time}

While gold labels are useful for quality control, as stated in the Introduction (Section~\ref{sec:intro}), %\kenneth{TODO: Update references}\steven{done}
they are not always available in real-world scenarios due to constraints such as data privacy or the cost of gathering gold labels~\cite{liu2019deep, yang2019evaluating, oikarinen2021detecting, slote2024unlocking}.
To address these challenges, researchers have developed methods to generate (approximations of) quality signals without gold labels. 
In the realm of LLM-powered data annotation, for instance, CoPrompter evaluates how well an LLM's output aligns with user-specified requirements as a feedback mechanism~\cite{joshi2024coprompter}. %\kenneth{TODO: Cite: Joshi, I., Shahid, S., Venneti, S., Vasu, M., Zheng, Y., Li, Y., ... \& Chan, G. Y. Y. (2024). CoPrompter: User-Centric Evaluation of LLM Instruction Alignment for Improved Prompt Engineering. arXiv preprint arXiv:2411.06099.}\steven{added}
Other studies also leverage the stability~\cite{chiang2023can} %\kenneth{TODO: Add ref}\steven{added}
%chiang2023can found LLM evaluation are stable over different formatting
or confidence~\cite{gligoric2024can} %\kenneth{TODO: Add ref}\steven{added}
%gligoric2024can introduce CONFIDENCEDRIVEN INFERENCE: a method that combines LLM annotations and LLM confidence indicators to strategically select which human annotations should be collected
of LLM outputs to infer quality signals.
%Our research investigates how effectively humans can iteratively refine prompts to guide LLMs in labeling data when gold-standard labels are unavailable, providing alternative quality signals.
Our research examines how effectively humans can refine prompts to guide LLMs in labeling data without gold-standard labels, providing insights into human prompting capabilities in the absence of reliable guidance signals.










%------------- dead kitten -------------
\begin{comment}




\kenneth{------------------------KENNETH IS WORKING HERE----------------------}



Gold-standard labels are widely used for quality control and crowd worker training~\cite{doroudi2016toward, gadiraju2015training,le2010ensuring,hettiachchi2021challenge}. For example, \citet{hettiachchi2021challenge} demonstrated that incorporating visible gold questions -- where annotators receive periodic feedback based on pre-labeled gold-standard examples -- could improve their work quality. 
Similarly, \citet{doroudi2016toward} found that providing expert examples was the most effective method of training for crowd workers and can help workers avoid specific types of incorrect responses. 
Additionally, \citet{le2010ensuring} employed dynamic learning systems that leveraged gold-standard labels to deliver real-time feedback and improve worker outcomes.
These studies, however, predominantly address the annotators' perspective -- workers who adhere to predefined guidelines and follow established standards.
While annotators are crucial components of the task pipeline, our study shifts focus to the requesters' perspective, those responsible for task design and pipeline management.
For requesters, gold-standard labels serve as signals to assess worker performance and refine training processes, thereby improving the overall quality of the entire pipeline.
Critically, the aforementioned studies assume the availability of gold-standard labels, typically under controlled experimental settings. 
In real-world scenarios, this assumption often does not hold due to constraints such as data privacy, security concerns, or the absence of labeled data~\cite{liu2019deep, yang2019evaluating, oikarinen2021detecting, slote2024unlocking}. 
To address this gap, our research explores settings where predefined gold-standard labels are unavailable. 
We designed a novel framework for requesters to iteratively develop and evolve their labeling standards through interactions with LLMs. 
By bridging the divide between controlled experiments and real-world challenges, our work highlights the potential of adaptive, LLM-driven approaches for dynamic task management without reliance on predefined gold-standard labels.

\steven{\citet{hettiachchi2021challenge} demonstrated that incorporating visible gold questions -- where annotators receive periodic feedback based on pre-labeled gold-standard examples -- could improve their work quality. 
Their study leveraged gold-standard labels to train crowd workers to align with pre-defined standards, effectively guiding annotators thorugh examples and feedback. 
While this approach focues on improving labeling quality at the annotator level, our work shifts the focus to requester and researcher perspective. Instead solely training labelers to meet pre-existing standards, we emphasize the broader implications of designing system in the entire labeling process, particularly in context involving dynamic or subjective tasks. \citet{gadiraju2015training} showed that training workers with gold labels can enhance accuracy and decrease response times. \citet{han2020crowd} used gold standard labels to guide crowd workers in revising incorrect judgments to align with predefined standards. 
}

\steven{
\citet{doroudi2016toward} found that providing expert examples was the most effective method of training for crowd workers. In our study, however, each participant was treated as an individual researcher rather than a crowd worker. While this finding underscores the value of providing gold labels to improve language model performance, it does not directly highlight their significance for researchers. Furthermore, \citet{doroudi2016toward} observed that gold standard labels help workers avoid specific types of incorrect responses. 
In contrast, our task is subjective, with participants’ standards potentially shifting across iterations. Introducing pre-set gold standard labels could enforce a uniform standard across each participant, which might not align with the iterative and subjective nature of our study
}

\steven{\citet{gadiraju2015training} showed that training workers with gold labels can enhance accuracy and decrease response times. [They were still focusing on crowd worker level.] }

\steven{\citet{han2020crowd} used gold standard labels for quality control and to guide crowd workers in revising incorrect judgments to align with predefined standards.}

\steven{\citet{le2010ensuring} employed gold standard labels within a dynamic learning environment that provided real-time feedback to train workers. However, the selection of specific examples for training could influence worker responses, potentially introducing bias in their judgments. [This is why we implemented a random sample in our system]}


\steven{\citet{liu2019deep} developed a HITL system that kept model upgrading with progressively collected data without having a pre-labeled data. [\textbf{they used 30 samples per iteration.} -add to justification for 10 and 50 instances.]}

\steven{\citet{wall2019using} found that end-users could build models without using expert patterns that have comparable performance to those who built by expert. This approach was required more effort and more mental demand than those who received guidance.}

\kenneth{TODO: Add references to every part of this paragraph.}
Decades of research have established that gold-standard labels are highly effective for quality control in data annotation~\cite{han2020crowd,gadiraju2015training,le2010ensuring,doroudi2016toward,hettiachchi2021challenge}. 
Embedding items with known labels into the annotation process enables requesters to monitor annotator or data quality and take actions such as retraining annotators, removing them from the pipeline, or reducing their weight in label aggregation. 
Beyond requesters, gold labels also allow for real-time feedback to workers, helping them recalibrate their understanding of the task or focus more carefully.
While gold labels are widely recognized as useful for quality control, most research assumes their availability.
However, as discussed in our Introduction (Section X), this assumption does not necessarily hold in real-world scenarios due to constraints such as data privacy or the cost of gathering gold labels~\cite{liu2019deep, yang2019evaluating, oikarinen2021detecting, slote2024unlocking}. 
To address these challenges, researchers have developed systems to provide proxy quality signals without gold labels. 
For instance, CoPrompter evaluates how well an LLM's output aligns with user-specified requirements as a feedback mechanism. 
Other studies leverage the stability or confidence of LLM outputs to infer quality signals.
Our research investigates how effectively humans can refine prompts to guide LLMs when gold-standard labels are unavailable.
    
\end{comment}

%\subsection{Explanations in AI-Assisted Tools}


%\subsection{Variables in System}
%There are lots of variables in a system could impact user's performance. 
\citet{kulesza2012tell} suggested that the more users understand the underlying system, the more effectively they can control it. 

\steven{\citet{lee2024clarify} introduces a system that allows non-expert users to train and correct models by directly interact with model using natural languages. In each iteration, the system will use similarity score between user description and image and display images above a threshold. The system will also provide 0-1 score indicating how well description separates the error cases from the correct prediction. Basically using metrics to guide user.
It does not mentioned about the sample size selection.}

\steven{[Data Instance:] In active learning, the goal is to minimize the amount of interaction required by users by querying the most important information~\cite{bernard2018vial}. [This can be used to justify why we increase to 50, to ensure the diversity. We cannot deploy algorithms to find most representative data sample because of the technical limitation of Google App Script]}

\steven{[Data Instance:] \citet{vermetten2022analyzing} investigated how the number of sample size affects the reliability of algorithm comparisons in iterative optimization. The study found that small sample sizes lead to high variability in performance estimates and larger sample sizes could decrease the impact of outliers. The performance could loss due to small samples and increasing sample size consistently improves reliability. }

\steven{\citet{purohit2018ranking} suggested capping the maximum number of annotation tasks assigned per unit of time to manage workload effectively to mitigate annotator burnout.}

\steven{\citet{pandey2022modeling} mentioned annotator can develop a mental representation of a concept by seeing a sufficient number of examples.}

\steven{\citet{wang2016human} limited users to verify the top-50 in each round, where users did binary classification on whether image was match or not.}

\steven{[explanation]\citet{kulesza2015principles} presents a system that explains the reason behind each prediction for users to better understand the system's logic to tailor the system toward their needs. In the system, users will modify feature weights within the model. n our LLM-powered system, users need to use natural language to guide the system. However, this can be more challenging because large models are less responsive to prompt variations compared to smaller models~\cite{zhuo2024prosa}.}

\steven{The more users understand the underlying system, the more effectively they can control it~\cite{kulesza2012tell}.}

\steven{\citet{teso2023leveraging} discusses a general framework for incorporating explanations into interactive machine learning. Users can get a better understanding of the machine's logic by observing the machine's explanations. [In LLM system, the explanation is the supporting argument for selecting a label.] Once understanding the bugs and limitation, users could modify the algorithm to correct flaws~\cite{kulesza2015principles}. [In our case, user cannot directly modify LLMs but only provide natural language to guide them. Also, subjective tasks does not have universal correct answers, where users need to provide their own standards to steer LLMs. ] }

\steven{[Task Difficulty:] 
A task being too difficult can frustrate users~\cite{zheng2022virtual}, particularly when exceeding their skill level, and a task being to easy can lead to boredom~\cite{zhang2021personalized}.
  These study focused on the impacts of difficulty on users' performance on a pre-defined task. However, in our study, our work prioritizes the dynamics of human-LLM interaction, emphasizing how effective humans could guide LLMs to align with their standard. In this context, the difficulty level of the task itself is less critical, as our primary objective is to assess the effectiveness of human guidance, regardless of the inherent complexity of the task.}


\steven{[task type:]\citet{cayir2016study} found the complexity and definition of a task significantly influence user performance. }

\steven{[task type:] \citet{hettiachchi2022survey} discusses different task assignment methods, including the modeling of worker performance and the impact of task heterogeneity on assignment strategies.
\citet{zhen2021crowdsourcing} provides a detailed exploration of task assignment challenges, task types, and their effects on worker performance and task outcomes. 
}
% \steven{ending of related word}We wanted to design a system to bridge the gap of xxxx: a graphical interface implemented on Google Sheet add-on, generalizing to single-class data annotation tasks, without requiring extensive knowledge of programming and system configuration. By combining the widespread familiarity and advanced features of Google Sheets with large-scale data annotation and iteration tracking, we aimed to make it easier for people to experiment with and benefit from LLMs.
\section{Proposed approach}
\label{sec:approach}

For the task of review comment generation, Knowledge-Based Systems (KBS) draw on codified rules and expert knowledge to deliver feedback that is consistent with established coding standards and best practices. Static analyzers, a prominent example of KBS, systematically follow predefined guidelines to detect code issues, offering reliable and structured feedback. While KBS achieve high precision, they are limited in scope, covering only a subset of possible issues encountered during code changes. In contrast, Learning-Based Systems (LBS) harness the adaptive potential of language models, which, by training on historical data, can recognize intricate patterns and generate contextually relevant review comments. This adaptability allows LBS to cover a broader range of issues present in the dataset, though often at the expense of precision. In this work, we conjecture that by combining these two strategies, it is possible to achieve the best of both approaches, namely, broader issue coverage coupled with improved precision.

\subsection{Overview}

Figure~\ref{fig:combination} illustrates our approach, outlining three strategies to combine Knowledge-Based Systems (KBS) and Learning-Based Systems (LBS) to enhance code review automation. 

\begin{figure}[htbp!]
  \centering
  \includegraphics[width=1\linewidth]{figures/combination.png}
  \caption{Different strategies to combine learning and knowledge-based systems}
  \label{fig:combination} 
\end{figure}

These strategies leverage KBS insights at different stages of the LBS pipeline, specifically during data preparation, inference, and final output. The three strategies are as follows:

\begin{itemize} 
    \item \emph{Data-Augmented Training (DAT)}: In this strategy, we enhance the training dataset by augmenting a real-world dataset with synthetic data generated from both KBS and LBS. This enriched dataset is then used to fine-tune a language model, enabling the LBS to incorporate both data-driven patterns and rule-based knowledge. This combination helps the model gain a more comprehensive understanding of code review patterns, improving robustness in varied review scenarios.
    \item \emph{Retrieval-Augmented Generation (RAG)}: Here, KBS insights are integrated directly into the LBS inference process. Through RAG, relevant information is dynamically retrieved from KBS (i.e., static analysis results) and injected into the prompts during generation. By incorporating the results of the KBS into the instruction, the LBS aligns its responses with established coding standards and practices, providing feedback grounded in structured, rule-based knowledge.
    \item \emph{Naive Concatenation of Outputs (NCO)}: This strategy merges the feedback generated by KBS and LBS after inference, combining their outputs to produce a unified code review. By consolidating KBS’s rule-based precision with LBS’s contextual depth, NCO offers a comprehensive review that covers a broader range of potential issues.
\end{itemize}

These strategies allow the LBS to benefit from the structured, rule-based insights of KBS, enhancing its ability to generate accurate, contextually appropriate, and standards-compliant code review comments.


\subsection{Baseline Model Preparation and Static Analyzers Selection}

While our approach is applicable to a wide range of LLMs and static analysis tools, we propose a specific configuration to illustrate the three strategies and establish the baselines for validating our conjecture. To set up the baseline systems, we first defined the LBS. We fine-tuned a large language model on an extensive code review dataset \cite{li2022automating}, referred to as \(\mathcal{D}_{\mathcal{M}_i}\), which pairs code changes with detailed reviews.

The selected model for fine-tuning is \emph{CodeLlama-7b}, trained for comment generation (i.e., generating review comments from code changes) with the following hyperparameter settings. The training was conducted on four \emph{NVIDIA RTX 3090} GPUs, using a batch size of $4$ per device. To boost efficiency, we applied gradient accumulation with a step size of $4$, updating the optimizer only after multiple batches. We used 4-bit quantization to improve memory and computational efficiency. Additionally, we employed Quantized Low-Rank Adaptation (QLoRA) \cite{hu2021lora}, a Parameter-Efficient Fine-Tuning (PEFT) technique, with $r = 16$, $\alpha = 32$, and $dropout = 0.05$. This method decomposes weight updates into low-rank matrices, reducing the parameters needed for fine-tuning and optimizing training efficiency \cite{hu2021lora}.

This resulted in a model, denoted as \(\mathcal{M}_i\), capable of generating detailed human-like code reviews.
Since \(\mathcal{M}_i\) represents the LBS component and was trained using the data from \(\mathcal{D}_{\mathcal{M}_i}\), we used the test set to generate reviews by both static analyzers and the fine-tuned model \(\mathcal{M}_i\).

To focus on a relevant subset of available static analyzers, we filtered the test set to include only Java code samples, producing a subset of $27,267$ entries, denoted \(\mathcal{D}_o\). Each entry in \(\mathcal{D}_o\) is a tuple \( (f,c) \), where \( f \) represents the source code file and \( c \) denotes the code change. Here, \( c \) is input to the LBS, while \( f \) serves as input to the KBS.
We limited our selection of static analyzers to tools that process Java source code directly. Although this decision excludes tools designed for Java bytecode analysis, it allows for a broader range of issue types. Specifically, we selected two well-established static analyzers: PMD \cite{pmd} and Checkstyle \cite{checkstyle}, both of which are designed to identify potential issues directly in source code.

PMD is a static code analysis tool that identifies issues in code by applying a set of rules aimed at detecting common problems, which are categorized into eight groups: best practices, coding style, design, documentation, error-prone, multi-threading, performance, and security \cite{lenarduzzi2023critical}. By analyzing source code against these rules, PMD generates detailed reports highlighting areas for improvement and enables users to create custom rules for specific analyses \cite{oskouei2018comparing}.

Checkstyle is another static code analysis tool for Java that offers predefined style configurations for standard checks, including Google Java Style and Sun Java Style. Its rules cover various aspects such as annotations, class design, coding, and naming conventions. Checkstyle also supports custom configuration files tailored to user needs~\cite{hovemeyer2004finding, balachandran2013reducing, oskouei2018comparing, lenarduzzi2023critical}.




\begin{figure}[!htbp]
\begin{subfigure}{1\linewidth}
  \centering
  \includegraphics[width=1\linewidth]{figures/approaches/approach-FTr.pdf}
  \caption{Data augmented training}
  \label{fig:approach1}
\end{subfigure}
\\ \vspace{10pt}
\begin{subfigure}{1\linewidth}
  \centering
  \includegraphics[width=1\linewidth]{figures/approaches/approach-RAG.pdf}
  \caption{Retrieval augmented generation}
  \label{fig:approach2}
\end{subfigure}
\\ \vspace{10pt}
\begin{subfigure}{1\linewidth}
  \centering
  \includegraphics[width=1\linewidth]{figures/approaches/approach-Concat.pdf}
  \caption{Naive concatenation of outputs}
  \label{fig:approach3}
\end{subfigure}
\caption{Proposed strategies to combine LBS and KBS}
    %\\The forward arrows represent the forward pass of the model, the circled shapes represent data (inputs, outputs, and ground truth), and the double arrows represent the losses.}
\label{fig:approaches}
\end{figure}





\subsection{Data Augmented Training}

As shown in Figure~\ref{fig:approach1}, this strategy involves retraining the LBS using an augmented dataset \({Da}\), which includes review comments generated by both, static analyzers and the fine-tuned model \(\mathcal{M}_i\). Through this retraining process, the LBS learns from both data sources, producing a more refined model referred to as \(\mathcal{M}_{FT}\).


A simple approach to augmenting the dataset would have been to apply static analysis to the code in \(\mathcal{D}_{\mathcal{M}_i}\) and add or concatenate the generated comments with the existing ones. However, this method does not guarantee data quality within the augmented dataset and fails to account for the insights inferred by the LBS \(\mathcal{M}_i\). 
Therefore, we employ an ensemble learning approach where the two distinct sources—the LBS and the KBS—serve as \emph{experts} to generate data for fine-tuning a model. The underlying rationale is that both KBS and LBS reviews are inherently synthetic. By combining their outputs, we achieve a more balanced and consistent fine-tuning process.

To produce the augmented dataset \(\mathcal{D}_a\), we designed a two-step process (i.e., \emph{data generation} and \emph{data filtering}), as depicted in Figure~\ref{fig:approach}.


\begin{figure}[htbp!]
  \centering
  \includegraphics[width=\linewidth]{figures/dataset.png}
  \caption{Dataset augmentation pipeline}
  \label{fig:approach} 
\end{figure}


In the Data Generation phase, we used the original dataset \(\mathcal{D}_o\) as input. For each code change \( c \), we employed our fine-tuned model \(\mathcal{M}_i\) to generate four context-aware, human-like reviews. Simultaneously, static analyzer rules were applied to each source code \( f \) to produce structured and precise feedback. Each static analyzer generated a report containing several reviews, including the start and end sections of code where each issue was identified. Since our approach focuses on code changes, we extracted the code section highlighted in each review, adding a few context lines before and after each extracted segment.
We then merged the reviews generated by both the static analyzers and \(\mathcal{M}_i\) into a single, unified dataset. Each data point in the dataset consists of tuples in the form \( (f, c, r, t) \), where \( f \) represents the source code file, \( c \) is the code change, \( r \) denotes the review comment, and \( t \) indicates the method used to generate the review (either KBS or LBS). 
%Note that for the same code \( (f,c) \) in \(\mathcal{D}_o\), we may have multiple tuples—i.e., samples—each corresponding to a generated comment.

After data generation, we applied a systematic Data Filtering step to evaluate and refine the merged dataset, ensuring that only the most relevant and meaningful reviews were retained for each source code. While the fine-tuned model \(\mathcal{M}_i\) can generate context-aware, human-like reviews, its output may sometimes include irrelevant or less meaningful feedback, particularly when handling complex or ambiguous code changes. Similarly, static analyzers, although reliable, may produce output overloaded with false positives \cite{johnson2013don, aniche2020effectiveness}, making it difficult to separate significant concerns from noise.
Therefore, filtering both \(\mathcal{M}_i\)'s and the static analyzers' reviews was essential to maintain a dataset of high-quality, meaningful feedback. This filtering process involved rating each review based on its relevance to the corresponding code. The ratings provided a quantitative measure of the review quality generated by both static analyzers and \(\mathcal{M}_i\).


\begin{figure}[hbt!]
  \centering
  \includegraphics[width=1\linewidth]{figures/llama3_rating.png}
  \caption{Judgment of review comments using \emph{Llama3-70B}}
  \label{fig:llama3_rating}
\end{figure}


To ensure a fair and scalable rating system, we leveraged large language models, which have demonstrated remarkable performance in similar assessment tasks \cite{zheng2024judging, huang2024empirical, weyssow2024codeultrafeedback}. By using these models, we achieved a more accurate and consistent evaluation of each review, enhancing the dataset's quality and making it a valuable resource for fine-tuning. As shown in Figure~\ref{fig:llama3_rating}, we used \emph{Llama3-70b}, inputting the code and its corresponding reviews. We instructed the model to rate each review on a 10-point scale. A threshold rating of 8 was set, with only reviews surpassing this threshold retained in the final dataset.
After filtering for relevance and quality, we ensured that each comment exceeding the threshold was treated separately. For a source code \(f\) with \(n\) reviews \((r_1...r_n)\), we generated \(n\) distinct data points: \(<f,r_1>\),  \(<f,r_2>\),..., \(<f,r_n>\). Additionally, for each comment, we extracted and included the specific segment of code change related to the issue being addressed, the dataset was then structered as \(<c_1,r_1>\),  \(<c_2,r_2>\),..., \(<c_n,r_n>\).

To prevent overrepresentation of specific rules, we randomly discarded reviews associated with rules that have an excessively high number of reviews. Furthermore, to maintain a balanced dataset, we randomly discarded a subset of learning-based reviews, ensuring an equal distribution between knowledge-based and learning-based reviews.
 
The final dataset \(\mathcal{D}_a\) consists of $78,776$ samples, ensuring an equal representation of reviews generated by both KBS and LBS methodologies, as shown in Figure~\ref{fig:chart}. It also ensures a balanced distribution across all KBS rules. 
\begin{figure}[!htbp]
  \centering
  \vspace{-2em}
  \includegraphics[width=0.6\linewidth]{figures/chart.png}
  \caption{Distribution of LBS and KBS Reviews in Our Dataset}
  \label{fig:chart} 
\end{figure}


To obtain the \(\mathcal{M}_{FT}\) model, we partitioned \(\mathcal{D}_a\) dataset into three subsets: 80\% of the samples were assigned to the training set, while the remaining 20\% was equally divided, with 10\% designated for validation and 10\% for testing. Each subset maintained a balanced mix of LBS and KBS reviews. We then fine-tuned the CodeLlama-7b model on this dataset with QLoRA to optimize memory efficiency \cite{hu2021lora}. 



\subsection{Retrieval Augmented Generation}

Retrieval-Augmented Generation (RAG) is a technique designed to enhance the generative capabilities of language models by incorporating external knowledge into their prompts during the inference phase \cite{jiang2023active}. In the context of code review, this strategy can be used to embed KBS-generated feedback directly into the prompts of a language-based system, as shown in Figure~\ref{fig:approach2}.



In our approach, the fine-tuned model \(\mathcal{M}_i\) takes as input the code changes from the \(\mathcal{D}_a\) dataset, along with outputs from PMD and Checkstyle. Incorporating KBS knowledge into the prompt guides the model to produce more relevant and precise reviews. This combination ensures that the generated review comments are both comprehensive and contextually informed. As a result, the reviews generated align closely with established coding standards and best practices, thereby enhancing their overall quality. The augmented prompt is illustrated in Figure~\ref{fig:rag_prompt}.

\begin{figure}[htbp!]
  \centering
  \includegraphics[width=0.9\linewidth]{figures/rag_prompt.png}
  \caption{Prompt used to generate review comments using RAG}
  \label{fig:rag_prompt}
\end{figure}



\subsection{Naive Concatenation of Outputs}

The Naive Concatenation of Outputs strategy serves as a baseline approach in which review comments generated separately by the LBS and KBS for the same code are combined to form a single review. As illustrated in Figure~\ref{fig:approach3}, the review comment generated by \(\mathcal{M}_i\) is directly concatenated with the output from the static analyzer (either PMD or Checkstyle).
This approach is straightforward, requiring minimal adjustments to the inference pipeline while ensuring that the final review delivers comprehensive feedback from both systems.












\section{Empirical Evaluation}
\begin{table*}[!ht]
    \centering
    \resizebox{0.88\textwidth}{!}{    
    \begin{tabular}{r|cccccc|cccccc}
        \toprule 
        & \multicolumn{6}{c}{\textbf{LLaVA-1.5-7B}} & \multicolumn{6}{c}{\textbf{LLaVA-1.5-13B}} \\ 
        \cmidrule(lr){2-7}\cmidrule(lr){8-13}
        & \multicolumn{3}{c}{\textbf{MM-SafetyBench}} & \multicolumn{3}{c|}{\textbf{MOSSBench}} & \multicolumn{3}{c}{\textbf{MM-SafetyBench}} & \multicolumn{3}{c}{\textbf{MOSSBench}} \\
        \textbf{Method} & \textbf{DSR}$\uparrow$ & \textbf{RR}$\uparrow$ & \textbf{Avg}$\uparrow$ & \textbf{DSR}$\uparrow$ & \textbf{RR}$\uparrow$ & \textbf{Avg}$\uparrow$ & \textbf{DSR}$\uparrow$ & \textbf{RR}$\uparrow$ & \textbf{Avg}$\uparrow$ & \textbf{DSR}$\uparrow$ & \textbf{RR}$\uparrow$ & \textbf{Avg}$\uparrow$\\
        \midrule
        w/o Defense          & 0.06  & 0.98  & 0.52  & 0.14  & 0.97  & 0.55  & 0.10  & 0.97  & 0.53  & 0.30  & 0.96  & 0.63  \\
        \midrule
        \multicolumn{13}{c}{Baseline} \\
        \midrule
        Responsible          & 0.12  & 0.96  & 0.54  & 0.32  & 0.96  & 0.64  & 0.18  & 0.96  & 0.57  & 0.47  & 0.92  & 0.70  \\
        Policy               & 0.08  & 0.96  & 0.52  & 0.18  & 0.98  & 0.58  & 0.12  & 0.97  & 0.55  & 0.34  & 0.97  & 0.65  \\
        Demonstration        & 0.15  & 0.97  & 0.56  & 0.37  & 0.95  & 0.66  & 0.25  & 0.96  & 0.60  & 0.52  & 0.92  & \textbf{0.72}  \\
        SFT                  & 0.20  & 0.95  & 0.58  & 0.50  & 0.88  & 0.69  & 0.13  & 0.98  & 0.55  & 0.49  & 0.88  & 0.68 \\
        SafeDecoding         & 0.08  & 0.97  & 0.53  & 0.31  & 0.94  & 0.62  & 0.12  & 0.96  & 0.54  & 0.42  & 0.93  & 0.68  \\
        Caption              & 0.09  & 0.98  & 0.53  & 0.21  & 0.98  & 0.60  & 0.12  & 0.97  & 0.55  & 0.27  & 0.94  & 0.60  \\
        Caption (w/o image)  & 0.16  & 0.95  & 0.55  & 0.34  & 0.94  & 0.64  & 0.22  & 0.93  & 0.57  & 0.45  & 0.89  & 0.67 \\
        Intention            & 0.07  & 0.98  & 0.53  & 0.20  & 0.99  & 0.59  & 0.11  & 0.96  & 0.54  & 0.26  & 0.97  & 0.61  \\
        \midrule
        % \multicolumn{13}{c}{} \\
        % \midrule
        \midrule
        \multicolumn{13}{c}{SR++} \\
        \midrule        
        Responsible-Demonstration & 0.18 & 0.95 & 0.57 & 0.40 & 0.94 & 0.67 & 0.29 & 0.96 & 0.62 & 0.58 & 0.85 & \textbf{0.72} \\
        Responsible-Policy & 0.12 & 0.96 & 0.54 & 0.27 & 0.97 & 0.62 & 0.18 & 0.96 & 0.57 & 0.46 & 0.94 & 0.70 \\
        Policy-Demonstration & 0.13 & 0.96 & 0.55 & 0.37 & 0.97 & 0.67 & 0.20 & 0.96 & 0.58 &0.51 & 0.93 & \textbf{0.72}\\
        Responsible-Policy-Demonstration & 0.15 & 0.96 & 0.55 & 0.38 & 0.95 & 0.66 & 0.25 & 0.97 & 0.61 & 0.53 & 0.88 & 0.70\\
        \midrule
        \multicolumn{13}{c}{SR+MO} \\
        \midrule     
        Responsible-SFT & 0.56 & 0.93 & \textbf{0.75} & 0.61 & 0.72 & 0.67 & 0.35 & 0.96 & 0.65 & 0.74 & 0.62 & 0.68 \\
        Responsible-SafeDecoding & 0.30 & 0.96 & 0.63 & 0.54 & 0.87 & \underline{0.70} & 0.23 & 0.96 & 0.59 & 0.63 & 0.79 & 0.71\\
        Demonstration-SFT & 0.60 & 0.90 & \textbf{0.75} & 0.65 & 0.77 & \textbf{0.71} & 0.56 & 0.92 & \textbf{0.74} & 0.67 & 0.70 & 0.68\\
        Demonstration-SafeDecoding & 0.38 & 0.96 & \underline{0.67} & 0.55 & 0.87 & \textbf{0.71} & 0.40 & 0.96 & \underline{0.68} & 0.62 & 0.78 & 0.70\\
        \midrule
        \multicolumn{13}{c}{QR++} \\
        \midrule   
        Caption-Intention & 0.09 & 0.97 & 0.53 & 0.20 & 0.98 & 0.59 & 0.14 & 0.95 & 0.55 & 0.26 & 0.96 & 0.61\\
        % Caption-Intention (w/o image) & 0.18 & 0.96 & 0.57 & 0.32 & 0.95 & 0.64 & 0.25 & 0.92 & 0.59 & 0.45 & 0.92 & 0.68\\
        \midrule
        % \multicolumn{13}{c}{} \\
        % \midrule
        \midrule
        \multicolumn{13}{c}{QR\textbar{}SR} \\
        \midrule   
        Caption-Responsible & 0.34 & 0.96 & 0.65 & 0.53 & 0.79 & 0.66 & 0.33 & 0.96 & 0.65 & 0.50 & 0.82 & 0.66\\
        Intention-Responsible & 0.36 & 0.97 & \underline{0.67} & 0.51 & 0.86 & 0.68 & 0.27 & 0.96 & 0.61 & 0.49 & 0.90 & 0.70\\
        Caption-Responsible (w/o image) & 0.96 & 0.25 & 0.60 & 0.93 & 0.16 & 0.55 & 0.60 & 0.80 & \underline{0.70} & 0.72 & 0.72 & \textbf{0.72}\\
        % Responsible-Intention (w/o image) & 0.99 & 0.06 & 0.52 & 0.95 & 0.17 & 0.56 & 0.61 & 0.81 & 0.71 & 0.68 & 0.77 & 0.72\\
        \midrule
        \multicolumn{13}{c}{QR\textbar{}MO} \\
        \midrule
        Caption-SafeDecoding & 0.20 & 0.96 & 0.58 & 0.39 & 0.88 & 0.64 & 0.33 & 0.94 & 0.63 & 0.40 & 0.90 & 0.65 \\
        Intention-SFT & 0.28 & 0.97 & 0.62 & 0.43 & 0.78 & 0.61 & 0.25 & 0.96 & 0.60 & 0.50 & 0.88 & 0.69\\
        Caption-SafeDecoding (w/o image) & 0.24 & 0.95 & 0.60 & 0.41 & 0.89 & 0.65 & 0.36 & 0.85 & 0.61 & 0.56 & 0.84 & 0.70\\
        \bottomrule
    \end{tabular}}
    \caption{Comparison results of ensemble strategies with the corresponding individual defenses. \textbf{Bold} indicates the best overall performance, while \underline{underlined} highlights the top three methods.} % and the full score is 100\%
    \label{tab:en_inter_results}
\end{table*}


\subsection{Experimental Setup}
We empirically evaluate various defense methods and their ensemble strategies on LLaVA-1.5-7B and LLaVA-1.5-13B~\cite{liu2024visual} to validate their effectiveness in standard settings. Using MM-SafetyBench and MOSSBench datasets, we assess safety and helpfulness by measuring defense success rate (DSR) on harmful queries and response rate (RR) on benign queries. We evaluate 28 defense methods, including system reminders, optimization techniques, query refactoring, and noise injection, as well as inter- and intra-mechanism ensembles. Detailed descriptions of defense methods and experimental setups are provided in Appendix~\ref{sec:defense strategies} and~\ref{sec:experiment_detail}. 
For a broader evaluation, we add more experiments in Appendix~\ref{sec:utility}, ~\ref{sec:diverse_attacks} and~\ref{sec:time}, including evaluation with the MM-Vet dataset for testing the quality of model's response on general queries, tests on JailbreakV-28K for more diverse and complex attack scenarios, and a comparison of inference time for different defense methods.

\subsection{Individual Defense Results}

Table~\ref{tab:indi_results} shows results of individual defense methods across four categories. Most methods, except for noise injection, effectively improve model safety across different models and datasets, as evidenced by increased defense success rates. This aligns with our analysis in Figure~\ref{fig:analysis results} where system reminder, model optimization and query refactoring lead to an overall increase in refusal probabilities. 

\paragraph{Safety shift defenses compromise helpfulness.} System reminder and model optimization methods generally reduce response rates on the benign subset while increasing defense success rates on the harmful subset. This confirms that safety shift tend to compromise helpfulness. This is more pronounced in MOSSBench than MM-SafetyBench due to the more apparent harmfulness and concealed harmlessness in MOSSBench queries.

\paragraph{Harmfulness discrimination defenses mitigate over-defense.} Query refactoring methods, except for Caption (w/o image), generally achieve the highest response rates on the benign subset, particularly for MOSSBench with misleadingly benign queries. This validates that harmfulness discrimination improves the model's ability to distinguish between truly harmful and benign queries. Notably, the removal of images in the Caption (w/o image) significantly reduces response rates for both harmful and benign queries, highlighting the crucial role images play in jailbreaking LVLMs.
% \paragraph{Image matters.} The removal of images in the Caption (w/o image) and Intention (w/o image) defenses leads to significant improvements in DSR compared to their image-included counterparts, underscoring the crucial role that images play in jailbreaking LVLMs.

\paragraph{Multimodal defense is challenging.}
However, all individual defense methods still exhibit limited defense success rates. While larger-scale LVLMs (i.e., LLaVA-1.5-13B) tend to achieve slightly higher success rates, they are also more susceptible to over-defense. This underscores the inherent challenges of jailbreak defense for LVLMs, especially when relying on individual defense methods. 

\subsection{Ensemble Defense Results}
Table~\ref{tab:en_inter_results} provides the empirical evaluation of both inter-mechanism and intra-mechanism ensemble strategies, leading to the following insights:

\paragraph{Ensembles improve safety.} Compared to individual methods, most ensemble strategies effectively enhance safety across both datasets and model sizes, showing increased defense success rates, especially in \textit{SR+MO} and \textit{QR\textbar{}SR} methods.

\paragraph{Inter-mechanism ensembles amplify.} Our evaluation shows most \textit{SR++} and \textit{SR+MO} ensembles improve defense success rates while reducing responses rates, whereas the \textit{QR++} ensemble better maintain responses rates. This confirms that inter-mechanism ensembles can amplify a single defense mechanism. Specifically, safety shift ensembles would further enhance model safety at the expense of helpfulness, while harmfulness discrimination ensemble better preserves helpfulness. Among inter-mechanism ensembles, those combining different types of specific methods (e.g., SR+MO) show a more pronounced amplification effect than those combining the same type (e.g., SR++). 
Notably, the Demonstration-SFT method excels in defense strength, utility, and response rate. Its success comes from combining two strong safety shift defenses, Demonstration and SFT, which complement each other and boost overall performance.

\paragraph{Intra-mechanism ensembles complement.} Compared to inter-mechanism ensembles, most \textit{QR\textbar{}SR} and \textit{QR\textbar{}MO} methods—except those without input images—can simultaneously maintain decent defense success rates and stable response rates,
compared to the undefended model and individual defense methods. This demonstrates that intra-mechanism ensemble can complement each other to achieve a more balanced trade-off. Additionally, the removal of input images offering a most conservative ensemble for multimodal defense while still maintaining certain helpfulness.
% In contrast, the defenses in intra-mechanism ensemble complement each other, strengthening safety while maintaining a stable level of helpfulness.
% In contrast, intra-mechanism ensembles combine the strengths of both mechanisms to achieve a more balanced trade-off. Specifically, \textit{QR\textbar{}SR} and \textit{QR\textbar{}MO} increase the refusal probability for harmful queries, while maintaining or even decreasing the refusal probability for benign queries, thereby improving the model's ability to distinguish between benign and harmful queries. This makes them a better choice for general scenarios where balancing safety and helpfulness is essential. 


\subsection{How Do Fine-tuning Affect Model Safety?}
We examine how different fine-tuning methods impact the safety of LVLMs by training LLaVA-1.5-7B using DPO and SFT with two datasets: SPA-VL~\cite{zhang2024spa} and VLGuard~\cite{zong2024safety}. SPA-VL focuses on safety discussions, while VLGuard emphasizes query rejection. We also test the effect of adding 5000 general instruction-following data from LLaVA.  

Table~\ref{tab:training_dataset_results} shows that DPO with SPA-VL and LLaVA provides a slight safety boost without significantly changing response behavior. In contrast, SFT has a stronger impact, but its effectiveness depends on the dataset. SPA-VL improves safety while maintaining helpfulness, though it may miss some harmful cases. VLGuard, however, makes the model overly defensive, rejecting too many queries. Adding LLaVA data helps balance safety and helpfulness, reducing excessive refusals.  


\begin{table}[ht]
    \centering
    \resizebox{0.49\textwidth}{!}{
    \begin{tabular}{r|cccccc}
        \toprule 
        & \multicolumn{3}{c}{\textbf{MM-SafetyBench}} & \multicolumn{3}{c}{\textbf{MOSSBench}} \\
        \textbf{Method} & \textbf{DSR}$\uparrow$ & \textbf{RR}$\uparrow$ & \textbf{Avg}$\uparrow$ & \textbf{DSR}$\uparrow$ & \textbf{RR}$\uparrow$ & \textbf{Avg}$\uparrow$ \\
        \midrule
        w/o Defense          & 0.06  & 0.98  & 0.52  & 0.14  & 0.97  & 0.55 \\
        \midrule
        \multicolumn{7}{c}{DPO} \\
        \midrule
        \multicolumn{1}{l|}{SPA-VL + LLaVA}          & 0.06  & 0.97  & 0.52  & 0.28  & 0.97  & 0.63  \\
        \midrule
        \multicolumn{7}{c}{SFT} \\
        \midrule
        \multicolumn{1}{l|}{SPA-VL}          & 0.24  & 0.96  & 0.60  & 0.58  & 0.78  & 0.68  \\
        + LLaVA     & 0.20  & 0.95  & 0.58  & 0.50  & 0.88  & 0.69  \\
        \midrule
        \multicolumn{1}{l|}{VLGuard}          & 1.00  & 0.09  & 0.55  & 0.90  & 0.21  & 0.55  \\
        + LLaVA     & 0.97  & 0.43  & 0.70  & 0.76  & 0.58  & 0.67  \\
        \bottomrule
    \end{tabular}}
    \caption{Comparison of varying fine-tuning settings.} % and the full score is 100\%
    \label{tab:training_dataset_results}
\end{table}

% In this work, we propose WildLong, a novel framework for synthesizing diverse, scalable, and realistic instruction-response datasets designed for long-context tasks. Our approach addresses key challenges in dataset creation by leveraging meta-information extraction from real-world user queries, graph-based modeling of co-occurrence relationships, and adaptive instruction-response generation.
% WildLong is built on the principles of diversity, scalability, and realism, enabling it to support complex reasoning tasks such as cross-document comparison, and aggregation, which are essential for real-world applications. By integrating meta-information into the data generation process and systematically exploring new combinations through graph-based modeling, WildLong generates diverse datasets that reflect the complexity of extended contexts.
% Experimental results demonstrate that WildLong significantly improves long-context task performance, surpassing other open-source long-context-optimized models across multiple benchmarks. Importantly, this improvement is achieved without requiring supplementary short-context instruction tuning, highlighting the robustness and generalizability of our approach.
% The success of WildLong highlights the potential of structured, meta-information-driven data synthesis to enhance the capabilities of LLMs for complex, real-world tasks. By addressing the critical gaps in long-context dataset diversity and quality, WildLong sets a new standard for long-context instruction tuning and paves the way for further advancements in equipping LLMs to tackle the challenges of extended-context reasoning.
% We propose WildLong, a framework for synthesizing diverse, scalable, and realistic instruction-response datasets for long-context tasks. By leveraging meta-information extraction, graph-based modeling, and adaptive instruction generation, WildLong generates long-context instruction-tuning data with real-world complexity.
% Experiments show improved long-context task performance while retaining short-context performance without additional short-context fine-tuning, demonstrating its robustness and generalizability. We hope WildLong provides insights into generalizing instruction tuning and inspires further advancements in long-context reasoning for LLMs.
We propose WildLong, a framework for synthesizing diverse, scalable, and realistic instruction-response datasets for long-context tasks. 
It integrates meta-information extraction to ensure realistic complexity, graph-based modeling for systematic instruction expansion, and adaptive instruction generation for enhanced contextual relevance.
Our fine-tuned models consistently outperform baselines and maintain short-context performance without mixing short-context data. Notably, our finetuned Llama-3.1-8B model surpasses most open-source long-context models on Longbench-Chat and demonstrates competitive performances with even larger models across benchmarks.
WildLong enables the synthesis of instruction-tuning data that produces robust models capable of handling diverse long-context tasks. Extending beyond synthetic QA and summarization, it bridges the gap to more complex, realistic challenges, advancing the effectiveness of long-context LLMs.
We hope WildLong provides insights into generalizing synthetic data and inspires further progress in long-context reasoning for LLMs.

\clearpage
{
    \small
    \bibliographystyle{ieeenat_fullname}
    \bibliography{reference}
}

% WARNING: do not forget to delete the supplementary pages from your submission 

%%%%%%%%% Merge with supplemental materials %%%%%%%%%%


\twocolumn[{%
 \centering
 \LARGE \papertitle \\[1em]
 \large Supplementary Material\\[1em]
}]

%\begin{center}
%\textbf{\large Supplemental Materials: Modality Agnostic 3D Gaze Estimation in the Wild}
%\end{center}

%%%%%%%%% Merge with supplemental materials %%%%%%%%%%
%%%%%%%%% Prefix a "S" to all equations, figures, tables and reset the counter %%%%%%%%%%
\renewcommand*{\thesection}{\Alph{section}}
\setcounter{section}{0}
\setcounter{equation}{0}
\setcounter{figure}{0}
\setcounter{table}{0}
%\setcounter{page}{1}
\makeatletter
\renewcommand{\theequation}{S\arabic{equation}}
\renewcommand{\thefigure}{S\arabic{figure}}
\renewcommand{\thetable}{S\arabic{table}}

%\renewcommand{\bibnumfmt}[1]{[S#1]}
%\renewcommand{\citenumfont}[1]{S#1}
%%%%%%%%%% Prefix a "S" to all equations, figures, tables and reset the counter %%%%%%%%%%



\mypartitle{What is expected?} The supplementary material consists of datasets details, experiments details, and extended experiments analysis mentioned in the main paper. In addition, videos of qualitative examples of our method on VideoAttentionTarget further demonstrate the robustness in challenging real-world scenarios.  

\section{Datasets Details}

\subsection{Datasets}
\label{sec:supp_dataset}
\mypartitle{ Gaze360 (G360).} \cite{Gaze360} is video 3D gaze datasets. It is collected in both indoor and outdoor environments in unconstrained setting, which contains 3D gaze of 238 subjects with a wide-range head pose and gaze direction. G360 is recorded at 8FPS. In all of our experiments, we always used the same training set as \cite{Gaze360} with 126928 samples. For the test set, we followed the split of \cite{Gaze360} where G360 Full corresponds to "All 360°" (the entire test set) with 25969 samples, G360 180 corresponds to "Front 180°" (gaze within 90°) with 20322 samples, and G360 40 to "Front Facing" (gaze within 20°) with 3995 samples. In addition to those splits, we consider G360 Back (gaze above 90°) \cite{chen2020360} with 5647 samples and finally G360 Face (all detected faces) with 16031 samples, which is used in many constrained gaze studies \cite{zhang2017s,chen2018appearance,Rt-gene,cheng2020coarse,cheng2022gaze,abdelrahman2023l2cs,yan2023gaze,catruna2024crossgaze}. When we refer to G360 Face 180 (15895 samples), it corresponds to the detected face with a gaze within 90°, a subset of G360 180, the same for G360 Face 40 with 3687 samples. We used the validation set described in \cite{Gaze360} with 17038 samples.

\mypartitle{GFIE.} \cite{GFIE} is a video 3D gaze dataset collected indoors with 71799 frames from 61 subjects (27 male and 34 female). It is an unconstrained dataset with a wide range of head poses. It was collected for gaze following task; using a complex calibrated laser setup, they can infer the 3D gaze from the eye to the visual target direction. They recorded people doing various indoor activities at 30 fps. We follow the data splits described in \cite{GFIE}, 59217 for training, 6281 for validation, and 6281 for testing.

\mypartitle{MPSGaze (MPS).} \cite{MPS} is a modified 3D gaze datasets that has been automatically generated using ETH-Xgaze \cite{ETH} eyes. They apply a blending technique on people from the Widerface~\cite{yang2016wider} dataset to put eyes with a known 3D gaze from ETH on heads with similar head poses. This dataset is diverse, with more than 10k identities and challenging poses, appearances, and lighting conditions. However, the blending process reduces the quality of the visual appearance, and it contains only near frontal head poses and no back view. We used the same training and test split with 24282 samples in training and 6277 samples in testing. No validation is defined in this work. 

\mypartitle{EYEDIAP (EDIAP).} \cite{Eyediap} is a 3D gaze video dataset. It includes videos from 16 subjects (30 fps), using either screen targets (CS, DS subset EDIAP) or 3D floating balls ( FT subset EDIAP-FT) as gaze targets. It is a constrained setup with mainly frontal head poses. 
%Compared to EDIAP with screen target, EDIAP-FT contains a wider gaze and head pose distribution. 
Following \cite{wang2022contrastive,Cheng2021Survey}, we used the evaluation set under screen target session (CS, DS, namely EDIAP) with 16674 samples from 14 subjects. 
%In addition, we consider the floating target session (FT, namely EDIAP-FT) with 8346 samples.

\mypartitle{MPIIFaceGaze (MPII).} \cite{MPIIGaze} is a 3D gaze image dataset collected from 15 subjects in a screen-based gaze target setup, resulting in a constrained dataset with mostly frontal head pose. We follow the standard evaluation protocol \cite{MPIIGaze,wang2022contrastive,Cheng2021Survey}, which selects 3000 images from each subject to form an evaluation set for a total of 45000 samples. 

\mypartitle{GazeFollow (GF).} \cite{recasens2015they} is a 2D gaze image dataset annotated on in the wild dataset for the gaze the following task. The 2D target label corresponds to where a given person is looking at in the image. It is a diverse dataset that includes various head poses, appearances, scenes, and lighting conditions. Overall, it has around 130K annotated person-target instances in 122K images.

\subsection{Video Processing}
As mentioned in the main section, for video clip input, our approach predicts the 3D gaze from an 8-frame video clip. However, video datasets have different frame rates, which can impact the gaze prediction. In this work, since G360 has a lower frame rate, we resample EYEDIAP and GFIE to match G360's frame rate of 8 fps. 

\subsection{Gaze Representation} 
Working with different 3D gaze datasets requires a unified way to define and represent the 3D gaze vector. Usually, in constrained gaze estimation, studies use data normalization to map the input image to a normalized space where a virtual camera is used to warp the face patch out of the original input image according to the 3D head pose \cite{ETH}. Thus, the gaze is expressed in this virtual camera coordinate defined by the 3D head pose. \\
%
However, in unconstrained settings, it is not possible to get access to a robust and reliable 3D head pose; thus, we follow the gaze representation of Gaze360 \cite{Gaze360} in the ``Eye coordinate system". The practical interpretation of the eye coordinate system is that the positive x-axis points to the left, the positive y-axis points up, and the positive z-axis points away from the camera, \ie [-1,0,0] is a gaze looking to the right or [0,0,-1] straight into the camera from the camera's point of view, irrespective of subjects position in the world. The origin of the gaze vector is the middle of the eyes, except for MPS and MPII, where the gaze origin is the average of 3D eyes and mouth landmarks resulting in an origin located at the middle of the nose, and for GF, we used the center of the head bounding box as the origin.

\begin{figure}
    \centering
    \includegraphics[width=0.99\linewidth]{images/head_crop_plot.png}
    \vspace{-4mm}
    \caption{Input head crop using different scales. In our work, a scale of -0.1 is used and proved to be effective in both constrained and frontal face setting \cref{sec:crop_size} }
    \label{fig:head_crop_input}
\end{figure}

\section{Experiments Details}

\mypartitle{Metric.}
We follow the test split described in the state-of-the-art method and explained in \cref{sec:supp_dataset}. As a metric, we use the standard angular error in degrees between the predicted and ground truth gaze prediction \cite{Gaze360,ETH,Eyediap,MPIIGaze}. Previous methods reporting video evaluation used a 7-frame video clip and predict the middle frame gaze direction. Since our approach outputs eight gaze directions from an 8-frame video clip, for a fair comparison, we use the 4th gaze prediction of an 8-frame video clip to compute the evaluation metric. 

\mypartitle{Training.}
We used the same setup in all the experiments to be as fair as possible. All the models are trained for a minimum of 20 epochs. We used an early stopping on the validation set with a patience of 10 epochs. We use the AdamW optimizer \cite{AdamW} with a learning rate of 1e-4 and a cosine annealing schedule with a 5 epochs linear warmup (from 2e-5 to 1e-4). For evaluation, we report the performance of the best model defined by the best angular error on the validation set.

\mypartitle{Data augmentation.}
Data augmentation is crucial for robust gaze estimation in the wild. In this work, we used standard data augmentation techniques. First, we applied jittering during the head crop to introduce slight variations in scale and aspect ratio, which reduces the model's sensitivity to noisy or imprecise head bounding boxes. Next, color jittering was applied by adjusting brightness, contrast, and saturation, making the model more resilient to diverse lighting conditions commonly encountered in real-world scenarios. Since gaze labels, such as those in the GF 2D dataset, may exhibit bias toward one side, we applied horizontal flipping to the images while appropriately adjusting the gaze direction, ensuring more balanced training data in the yaw gaze direction. These augmentations collectively improved the model’s ability to handle variations in data and enhance its generalization to unseen environments.

\section{Additional Experiments}

\subsection{Effect of Head Crop Size}
\label{sec:crop_size}
As mentioned by Chen \etal~\cite{chen2020360}, the input head crop scale impacts the 3D gaze estimation. We find that the effect on the prediction depends on the head orientation. \cref{fig:head_crop_input} illustrates the different inputs with different head crop scales. As shown in \cref{fig:supp_head_crop_effect}, a smaller head crop tighter to the face improves 3D gaze estimation on frontal head poses, while a larger head crop improves gaze on the non-frontal head pose. Indeed, as shown in \cref{fig:head_crop_input}, a tighter crop increases the eye resolution in the image and a larger crop provides more context about the head orientation and upper body orientation, which gives a strong prior for the gaze direction when eyes are not visible.
In the context of gaze estimation in the wild, a scale of -10\% is part of the Pareto front as illustrated in \cref{fig:supp_head_crop_effect} and is also the best on the G360 Full image as shown in \cref{fig:supp_head_crop_angular}. Therefore, it is a reasonable trade-off between frontal and back view performance. We use it for all our experiments.

\begin{figure}[htbp]
    \centering
    \begin{subfigure}{\linewidth}
        \centering
        \includegraphics[width=0.99\textwidth]{images/Head_crop_angular_error.png}
        \caption{Effect of head bounding box scale as input on the 3D gaze angular error on G360 Full test set. A scale ratio of 0.1 corresponds to a 10\% bounding box scale.}
        \label{fig:supp_head_crop_angular}
    \end{subfigure}
    
    
    \begin{subfigure}{\linewidth}
        \centering
        \includegraphics[width=0.8\textwidth]{images/Head_crop_effect.png}
        \caption{Effect of head bounding box scale on the angular error with respect to G360 Back and G360 40 test subset.}
        \label{fig:supp_head_crop_effect}
    \end{subfigure}
    
    \caption{\textbf{Effect of head crop size.}}
    \label{fig:head_crop}
\end{figure}

\subsection{Constrain Gaze Evaluation}

% Please add the following required packages to your document preamble:
% \usepackage{graphicx}
\begin{table}[t]
\centering
\resizebox{\columnwidth}{!}{%
\begin{tabular}{llrrrrrrr}
\toprule
 &  & \multicolumn{1}{c}{MPII} & \multicolumn{1}{l}{} & \multicolumn{2}{c}{EDIAP}  \\ \cline{3-3} \cline{5-6} 
Method & Training Dataset & Img &  & Img & Vid  \\ 
\hline
PureGaze~\cite{cheng2022puregaze} (Res18)        & G360I Face & 9.3 &   & 9.2 & -  \\
Liu \etal~\cite{liu2021generalizing} (Res18)      & G360I Face & 7.7 &   & 9.0 & -  \\
Liu \etal~\cite{liu2021generalizing} (Res50)     & G360I Face & 8.3 &   & 7.5 & - \\
RAT~\cite{bao2022generalizing} (Res18)     & G360I Face & 7.6 &   & \textcolor{gray}{7.1}$^*$ & -  \\
RAT~\cite{bao2022generalizing} (Res50)     & G360I Face & 7.7 &   & \textcolor{gray}{7.1}$^*$ & -  \\
CDG~\cite{wang2022contrastive} (Res50)           & G360I Face & 7.0 &   & \textbf{7.3} & -  \\
\midrule
 Supervised (\model) & G360I\&V             & 7.43 &   & 8.88 & 8.28  \\
 \framework (\model) & G360I\&V+GF          & \textbf{6.43} &   & 8.87 & 8.19  \\
 %\model & G360I\&V+GF+MPS       \\
\bottomrule
\end{tabular}%
}
\caption{\textbf{Comparision with state-of-the-art on constrained domain generalization benchmarks}. All these methods \cite{cheng2022puregaze,liu2021generalizing,wang2022contrastive,ververas20223dgazenet,bao2022generalizing} use a face crop as input and are trained on the detected face subset of Gaze360. Our method is trained and tested on head crop which makes it more general but more challenging for frontal gaze estimation. $^*$ In \cite{bao2022generalizing} they used only 6400 sample for EDIAP but we follow \cite{Cheng2021Survey,cheng2022puregaze,wang2022contrastive} with 16674 samples. }
\label{tab:generalization_constrained}
\vspace{-3mm}
\end{table}
\vspace{-1mm}


The objective of this work is to improve unconstrained gaze estimation in the wild. As seen in \cref{sec:crop_size}, compared to a tight face crop a larger crop improves gaze in challenging head pose. Therefore, a larger crop is more suited to our objective.  In contrast, some methods specialize in frontal gaze estimation and rely on tight face crops, which provide better resolution for the eye regions. While this is not a fully fair comparison, we compare our approach to these constrained methods for generalization on constraint benchmarks. Note that for the constrained methods, models are trained and tested only on a subset of detected faces (G360 Face), while in our approach the model is trained on G360 Full. \\
As shown in \cref{tab:generalization_constrained}, on MPII, the supervised GaT lags behind the best method by 6\%. On EDIAP, GaT is 21\% behind the best method in image evaluation but narrows the gap to 13\% when evaluated on videos. Then, when using our \framework learning framework including GF labels, we observe an important improvement on MPII with state-of-the-art angular error of 6.43 compared to 7 from CDG. On EDIAP the improvement is marginal. Compared to EDIAP, MPII has more diversity in lighting conditions and environment. GF doesn't contain a lot of frontal gaze direction but has a broad diversity of environments. Therefore, the improvement on MPII should come from the additional diversity that GF brings but this is not useful for EDIAP prediction. 
While constrained methods excel in frontal settings, they fail in unconstrained scenarios. Our approach, which achieves state-of-the-art performance in unconstrained environments (G360, GFIE) while remaining competitive in constrained settings (MPII, EDIAP), proves to be a versatile and robust solution for gaze estimation in the wild.

\subsection{Qualitative Analysis}

\begin{figure}[t]
    \centering
    \includegraphics[width=0.8\linewidth]{images/plot_image_video_gaze360.png}
    \caption{ \textbf{Image vs video predictions, where does it help?}. GaT trained on G360I\&V and tested on G360 Full image and video. The difference between image and video angular error with respect to the ground truth gaze directions from the camera ([0,0,-1]). The mean and standard deviation are displayed for each 10° bin. Positive values indicate better performance in video prediction compared to image prediction. }
    \label{fig:graph_img_vs_video}
\end{figure}

\mypartitle{When does temporal context contribute most effectively?}
As seen in the main paper, video prediction consistently outperforms image prediction. To understand the significance of temporal context in gaze estimation, we examined cases with large angular errors between image and video predictions. Several key observations emerged. As illustrated in \cref{fig:temporal_pros} in the first two rows, temporal context proves valuable during blinks, as it allows the model to interpolate gaze direction when the eyes are closed. If the head pose is not informative, temporal context helps disambiguate between blinking and looking down since the eyes are not visible, as shown in row 1. Additionally, when individuals are viewed entirely from behind (rows 6-7), video inferences provide a more consistent gaze direction in relation to time. Thus, there is less jittering and it might improve the prediction accuracy. In rows 4-5, the head and eye motion can be used in video prediction to improve the gaze direction. Finally, it can help in case of occlusion, as seen in row 3.\\
Furthermore, we explore the impact of image- and video-based prediction with respect to gaze direction. Indeed, we expected more improvement when people are from the back since additional head motion cues can be useful for gaze estimation. In the results, video prediction on G360 Back clearly improves image prediction. In addition, in \cref{fig:graph_img_vs_video}, we plot the difference between image and video prediction angular error for different gaze directions. If we look at the trend, video prediction seems to be better, especially for gaze over 150°, but given the standard deviation, it might not be a statistically significant observation. A more detailed analysis by considering only cases where there is a head motion can better highlight the impact of video prediction.\\
\mypartitle{What are the limitations of temporal context for gaze?}
We investigate prediction made on the VideoAttentionTarget~\cite{Chong_2020_CVPR} (VAT) videos using our \framework framework and \model model. VAT is a challenging dataset with real-world scenarios, various appearances, and diverse gaze distribution, making it well-suited for assessing our approach. Our qualitative analysis reveals two limitations of video-based inference compared to image-based inference using our model. The first limitation arises in cases of rapid head rotation, as illustrated in \cref{fig:temporal_cons}, temporal context may be misused, leading to predictions that do not align with the actual gaze. It might be because no rapid head motion is present in the G360 training sets. 
The second aspect involves cases of ``gaze recentering", where the gaze direction returns to its initial position following a shift. This behavior can occur very rapidly, within just 3-4 frames. Due to the smoothing effect in the temporal modeling, the predicted gaze may not exhibit the same amplitude as the actual movement. Indeed, this behavior is not present in the G360 dataset, and the use of videos sampled at 8 frames per second may limit the ability to capture fine-grained gaze dynamics. However, such behavior is better captured during image-based inference. This highlights a trade-off: while video-based inference provides smoother and more robust predictions, image-based inference offers greater accuracy but can result in jittery outputs. To mitigate the lack of natural gaze behavior we apply our \framework framework using 2D gaze video data from VAT. Unfortunately, since current benchmarks don't contain natural gaze behavior, the results don't show quantitative improvement. Further research to evaluate this aspect is needed.   \\
\mypartitle{In which scenarios does \framework with GazeFollow labels provide the most benefit?}
We demonstrated the advantages of \framework with GazeFollow labels across various benchmarks, both within- and cross-datasets. But in which scenarios does it outperform supervised methods trained solely on G360? To address this question, we analyze predictions made in real-world scenarios using the VideoAttentionTarget (VAT) dataset~\cite{Chong_2020_CVPR}. Our findings reveal that \framework achieves the most notable improvements in cases of extreme head poses, particularly when the head is facing downward, as shown in \cref{fig:sup_vs_stwsge}. It is also more robust to appearance diversity like hair partially occluding the face or varying skin tones. It also helps in difficult lighting conditions and low-resolution inputs.
Additionally, we include a video (provided in the supplementary materials) displaying predictions on VAT with an explanation, enabling a direct comparison between the two methods and a clearer visualization of our approach's performance on real-world data.

\begin{figure*}[t]
    \centering
    \includegraphics[width=1\linewidth]{images/presentation1.pdf}
    \caption{ \textbf{Illustration of image against video prediction.} Comparison between single-image (frame 0) and video predictions (frame -3 to 4). We use our \framework learning framework with \model trained on G360 and GF. All examples are from G360 test set. Rows 1-2 illustrate eye blinks, Row 3 shows an example of occlusion, Rows 4-5 demonstrate frontal head/eyes motion, and Rows 6-7 depict back view prediction. In the last row, the first two frames are not part of the test subset. Arrows in \textcolor{red}{red} represent image predictions, and arrows in \textcolor{magenta}{magenta} are video predictions. The angular error between groundtruth and prediction is displayed in red at the top right corner. The circles in the images represent unit disks where 3D gaze vectors are projected onto the image plane (x,y in yellow) and a top view (x,z in blue)}
    \label{fig:temporal_pros}
\end{figure*}

\begin{figure*}[t]
    \centering
    \includegraphics[width=1\linewidth]{images/presentation2_over.pdf}
    \caption{\textbf{Illustration of image and video prediction in case of rapid head motion.} We use our \framework learning framework with \model trained on G360 and GF. All examples are from VideoAttentionTarget~\cite{Chong_2020_CVPR} (VAT). Arrows in \textcolor{red}{red} represent image predictions, and arrows in \textcolor{magenta}{magenta} are video predictions. The circles in the images represent unit disks where 3D gaze vectors are projected onto the image plane (x,y in yellow) and a top view (x,z in blue). Note that since VAT has a frame per second (fps) of 24 and G360 has a fps of 8, we show the temporal context used for video inference corresponding to 8 fps. }
    \label{fig:temporal_cons}
\end{figure*}

\begin{figure*}[t]
    \centering
    \includegraphics[width=1\linewidth]{images/presentation3.pdf}
    \caption{\textbf{Illustration of supervised against \framework learning framework with GazeFollow label.} We use in both experiments our \model model. All examples are from VideoAttentionTarget~\cite{Chong_2020_CVPR} (VAT). Arrows in \textcolor{blue}{blue} represent image predictions with supervised \model trained on G360, and arrows in \textcolor{red}{red} are image predictions with \framework \model trained on G360 and GF. The circles in the images represent unit disks where 3D gaze vectors are projected onto the image plane (x,y in yellow) and a top view (x,z in blue). }
    \label{fig:sup_vs_stwsge}
\end{figure*}



\end{document}

