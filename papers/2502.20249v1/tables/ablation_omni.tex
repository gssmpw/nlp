\begin{table}[t]
\centering
\resizebox{\columnwidth}{!}{%
\begin{tabular}{lrrrrrrrrrrr}
\toprule
 &  \multicolumn{2}{c}{G360 Full} & \multicolumn{1}{l}{} & \multicolumn{2}{c}{G360 180} & \multicolumn{1}{l}{} & \multicolumn{2}{c}{G360 40} & \multicolumn{1}{l}{} & \multicolumn{2}{c}{GFIE}  \\
 \cline{2-3} \cline{5-6} \cline{8-9} \cline{11-12}  
Training Data & Img & Vid &  & Img & Vid &  & Img & Vid &  & Img & Vid  \\ 
\midrule
G360V+GF    & 13.5 & 12.1 &   & 11.6 & 10.2 &   & 8.3 & 7.7 &   & 15.7 & 17.9 \\
G360I\&V+GF    & 13.2 & 12.2 &   & 11.3 & 10.3 &   & 8.6 & 7.7 &   & 15.9 & 15.5  \\
\midrule
GFIEV+GF  & 22.8 & 24.2 &   & 22.4 & 23.9 &   & 29.9 & 31.8 &   & 13.4 & 13.0  \\
GFIEI\&V+GF  & 21.5 & 21.1 &   & 20.6 & 20.3 &   & 26.6 & 26.7 &   & 13.0 & 12.7 \\
\bottomrule
\end{tabular}%
}
\vspace{-2mm}
\caption{ \textbf{Impact of the training datasets modalities on cross-modal generalization.} We experiment with \model model, \framework framework, and different training dataset modalities. }
\label{tab:ablation_cross_modal}
\vspace{-2mm}
\end{table}
