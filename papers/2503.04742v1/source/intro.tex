\section{Introduction}

In 2023, the European Union finally announced its long-awaited Artificial Intelligence Act (AI Act). The AI act consists in a series of regulations on AI that not only impacts European AI, but also ultimately influences other countries and regions in the world in what is known as the Brussels' effect\cite{bradford2020brussels, bradford2024brussels, siegmann2022brussels, gunst2021brussels}. 
One of the most puzzling features of the AI Act is that, due to intensive lobbying from large AI models companies\cite{perrigo2023exclusive}, the Act ended up putting more restrictions on some specialised AI models than it does on those claiming general capabilities. To capture the absurdity of the situation, we propose the following analogy: can we imagine if in medical practice, a general practitioner had more freedom to operate on a patient's eye than an ophthalmologist? The answer is obviously negative. The less specialised a medical doctor, the more restrictions different medical legislation or codes of ethics\cite{american1871code, codesante} would impose on what they can do.

In the 2020s, AGI, in which the ``G'' stands for \emph{general}, has increasingly become an overused term to include several \emph{desirable} properties in AI entities.
However, it is unclear whether \emph{generality} ought to be regarded as \emph{desirable},
especially in terms of auditability, reusability and security, 
but also in terms of industrial value.
In this paper, we review social science and statistical arguments \emph{for} generality, 
and highlight their limits in the context of non-human entities.
We then instead make a case \emph{against} generality, 
by leveraging arguments from adversarial machine learning, %~\cite{el2022impossible}
from complex system engineering %~\cite{DBLP:conf/isss/GajskiG02,DBLP:journals/tem/EfatmaneshnikSQ20} 
and from social sciences. 
We then stress the value of \emph{specification}, which requires specialised systems.

The rest of this paper is organised as follows. In Section~\ref{sec:context}, we lay down the broader context of our work and review useful notions such as \emph{generality}, \emph{task} and what we call ``\emph{friends of specialisation}'' such as decentralisation or separation of powers. In Section~\ref{sec:against}, we propose three arguments against specialisation from social sciences, economics and statistics, and discuss their limitations in the context of non-human entities. 
This brings us to Section~\ref{sec:prospecialisation}, in which we propose our arguments for making non-human entities specialised. 
Section~\ref{sec:specification} complements our argument by making a case for \emph{specification}.
The section also discusses the limits of specification, 
and the need of \emph{specified governance} to address these limits.
Finally, \ref{sec:conclusion} concludes this paper.
