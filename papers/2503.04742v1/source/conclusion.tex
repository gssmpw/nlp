\section{Conclusion}
\label{sec:conclusion}

In this paper, we showed the limits of the arguments against specialisation, when applied to non-human information-processing entities.
We highlighted the industrial, democratic and security values of specialisation,
especially when it is accompanied with careful specification.
In particular, we articulated how the state of knowledge in adversarial machine learning,
complex system engineering, economics, and the sociology of work, occupations and organisations all point to 
the numerous issues of generality.
Finally, after acknowledging the limits of specifications in general,
we emphasised on the importance of specifying the governance of under-specified tasks,
especially when these tasks are complex 
or when they do not lend themselves to consensual specifications across populations and time.

We hope that the improved understanding of the value of specialisation will help
researchers, developers, managers, organisations, regulators and politicians
better orient the construction of a more prosperous,
more secure and more sovereign information space.

