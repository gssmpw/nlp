\begin{abstract}
    With the rise of large multi-modal AI models, fuelled by recent interest in large language models (LLMs), the notion of artificial general intelligence (AGI) went from being restricted to a fringe community, to dominate mainstream large AI development programs. 
    In contrast, in this paper, we make a \emph{case for specialisation}, by reviewing the pitfalls of generality and stressing the industrial value of specialised 
    systems.
    
    Our contribution is threefold. First, we review the most widely accepted arguments \emph{against} specialisation, and discuss how their relevance in the context of human labour is actually an argument \emph{for} specialisation in the case of non human agents, be they algorithms or human organisations. Second, we propose four arguments \emph{in favor of} specialisation, ranging from machine learning robustness, to computer security, social sciences and cultural evolution.  
    Third, we finally make a case for \emph{specification}, discuss how the machine learning approach to AI has so far failed to catch up with good practices from safety-engineering and formal verification of software, and discuss how some emerging good practices in machine learning help reduce this gap.
    In particular, we justify the need for \emph{specified governance} for hard-to-specify systems.
\end{abstract}