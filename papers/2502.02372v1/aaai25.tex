%File: formatting-instructions-latex-2025.tex
%release 2025.0
\documentclass[letterpaper]{article} % DO NOT CHANGE THIS
\usepackage{aaai25}  % DO NOT CHANGE THIS
\usepackage{times}  % DO NOT CHANGE THIS
\usepackage{helvet}  % DO NOT CHANGE THIS
\usepackage{courier}  % DO NOT CHANGE THIS
\usepackage[hyphens]{url}  % DO NOT CHANGE THIS
\usepackage{graphicx} % DO NOT CHANGE THIS
\urlstyle{rm} % DO NOT CHANGE THIS
\def\UrlFont{\rm}  % DO NOT CHANGE THIS
\usepackage{natbib}  % DO NOT CHANGE THIS AND DO NOT ADD ANY OPTIONS TO IT
\usepackage{caption} % DO NOT CHANGE THIS AND DO NOT ADD ANY OPTIONS TO IT
\frenchspacing  % DO NOT CHANGE THIS
\setlength{\pdfpagewidth}{8.5in}  % DO NOT CHANGE THIS
\setlength{\pdfpageheight}{11in}  % DO NOT CHANGE THIS
%
% These are recommended to typeset algorithms but not required. See the subsubsection on algorithms. Remove them if you don't have algorithms in your paper.
\usepackage{algorithm}
\usepackage{algorithmic}
\usepackage{amsmath}
\usepackage{cleveref} 
\usepackage{booktabs} 
% \usepackage{subcaption}
\usepackage{cleveref}





%
% These are are recommended to typeset listings but not required. See the subsubsection on listing. Remove this block if you don't have listings in your paper.
\usepackage{newfloat}
\usepackage{listings}
\DeclareCaptionStyle{ruled}{labelfont=normalfont,labelsep=colon,strut=off} % DO NOT CHANGE THIS
\lstset{%
	basicstyle={\footnotesize\ttfamily},% footnotesize acceptable for monospace
	numbers=left,numberstyle=\footnotesize,xleftmargin=2em,% show line numbers, remove this entire line if you don't want the numbers.
	aboveskip=0pt,belowskip=0pt,%
	showstringspaces=false,tabsize=2,breaklines=true}
\floatstyle{ruled}
\newfloat{listing}{tb}{lst}{}
\floatname{listing}{Listing}
%
% Keep the \pdfinfo as shown here. There's no need
% for you to add the /Title and /Author tags.
\pdfinfo{
/TemplateVersion (2025.1)
}

% DISALLOWED PACKAGES
% \usepackage{authblk} -- This package is specifically forbidden
% \usepackage{balance} -- This package is specifically forbidden
% \usepackage{color (if used in text)
% \usepackage{CJK} -- This package is specifically forbidden
% \usepackage{float} -- This package is specifically forbidden
% \usepackage{flushend} -- This package is specifically forbidden
% \usepackage{fontenc} -- This package is specifically forbidden
% \usepackage{fullpage} -- This package is specifically forbidden
% \usepackage{geometry} -- This package is specifically forbidden
% \usepackage{grffile} -- This package is specifically forbidden
% \usepackage{hyperref} -- This package is specifically forbidden
% \usepackage{navigator} -- This package is specifically forbidden
% (or any other package that embeds links such as navigator or hyperref)
% \indentfirst} -- This package is specifically forbidden
% \layout} -- This package is specifically forbidden
% \multicol} -- This package is specifically forbidden
% \nameref} -- This package is specifically forbidden
% \usepackage{savetrees} -- This package is specifically forbidden
% \usepackage{setspace} -- This package is specifically forbidden
% \usepackage{stfloats} -- This package is specifically forbidden
% \usepackage{tabu} -- This package is specifically forbidden
% \usepackage{titlesec} -- This package is specifically forbidden
% \usepackage{tocbibind} -- This package is specifically forbidden
% \usepackage{ulem} -- This package is specifically forbidden
% \usepackage{wrapfig} -- This package is specifically forbidden
% DISALLOWED COMMANDS
% \nocopyright -- Your paper will not be published if you use this command
% \addtolength -- This command may not be used
% \balance -- This command may not be used
% \baselinestretch -- Your paper will not be published if you use this command
% \clearpage -- No page breaks of any kind may be used for the final version of your paper
% \columnsep -- This command may not be used
% \newpage -- No page breaks of any kind may be used for the final version of your paper
% \pagebreak -- No page breaks of any kind may be used for the final version of your paperr
% \pagestyle -- This command may not be used
% \tiny -- This is not an acceptable font size.
% \vspace{- -- No negative value may be used in proximity of a caption, figure, table, section, subsection, subsubsection, or reference
% \vskip{- -- No negative value may be used to alter spacing above or below a caption, figure, table, section, subsection, subsubsection, or reference

\setcounter{secnumdepth}{2} %May be changed to 1 or 2 if section numbers are desired.

% The file aaai25.sty is the style file for AAAI Press
% proceedings, working notes, and technical reports.
%

% Title

% Your title must be in mixed case, not sentence case.
% That means all verbs (including short verbs like be, is, using,and go),
% nouns, adverbs, adjectives should be capitalized, including both words in hyphenated terms, while
% articles, conjunctions, and prepositions are lower case unless they
% directly follow a colon or long dash
\title{MaintaAvatar: A Maintainable Avatar Based on Neural Radiance Fields by Continual Learning}
\author{
    %Authors
    % All authors must be in the same font size and format.
    Shengbo Gu\textsuperscript{\rm 1,3},
    Yu-Kun Qiu\textsuperscript{\rm 1,3},
    Yu-Ming Tang\textsuperscript{\rm 1,3},
    Ancong Wu\textsuperscript{\rm 1,3}\thanks{Corresponding Authors.},
    {Wei-Shi Zheng\textsuperscript{\rm 1,2,3}}\footnotemark[1]
}
\affiliations{
    %Afiliations
    \textsuperscript{\rm 1}School of Computer Science and Engineering, Sun Yat-sen University, China\\
    \textsuperscript{\rm 2}Peng Cheng Laboratory, Shenzhen, China\\
    \textsuperscript{\rm 3}Key Laboratory of Machine Intelligence and Advanced Computing,
 Ministry of Education, China\\
    % If you have multiple authors and multiple affiliations
    % use superscripts in text and roman font to identify them.
    % For example,

    % Sunil Issar\textsuperscript{\rm 2}, 
    % J. Scott Penberthy\textsuperscript{\rm 3}, 
    % George Ferguson\textsuperscript{\rm 4},
    % Hans Guesgen\textsuperscript{\rm 5}
    % Note that the comma should be placed after the superscript

    % 1101 Pennsylvania Ave, NW Suite 300\\
    % Washington, DC 20004 USA\\
    % email address must be in roman text type, not monospace or sans serif
    \{gushb3,qiuyk,tangym9\}@mail2.sysu.edu.cn, wuanc@mail.sysu.edu.cn, wszheng@ieee.org 
%
% See more examples next
}

%Example, Single Author, ->> remove \iffalse,\fi and place them surrounding AAAI title to use it
\iffalse
\title{My Publication Title --- Single Author}
\author {
    Author Name
}
\affiliations{
    Affiliation\\
    Affiliation Line 2\\
    name@example.com
}
\fi

\iffalse
%Example, Multiple Authors, ->> remove \iffalse,\fi and place them surrounding AAAI title to use it
\title{My Publication Title --- Multiple Authors}
\author {
    % Authors
    % First Author Name\textsuperscript{\rm 1,\rm 2},
    % Second Author Name\textsuperscript{\rm 2},
    % Third Author Name\textsuperscript{\rm 1}
}
\affiliations {
    % Affiliations
    \textsuperscript{\rm 1}Affiliation 1\\
    \textsuperscript{\rm 2}Affiliation 2\\
    firstAuthor@affiliation1.com, secondAuthor@affilation2.com, thirdAuthor@affiliation1.com
}
\fi


% REMOVE THIS: bibentry
% This is only needed to show inline citations in the guidelines document. You should not need it and can safely delete it.
\usepackage{bibentry}
% END REMOVE bibentry

\begin{document}
\maketitle
\begin{abstract}
The generation of a virtual digital avatar is a crucial research topic in the field of computer vision. Many existing works utilize Neural Radiance Fields (NeRF) to address this issue and have achieved impressive results. However, previous works assume the images of the training person are available and fixed while the appearances and poses of a subject could constantly change and increase in real-world scenarios. 
How to update the human avatar but also maintain the ability to render the old appearance of the person is a practical challenge.
One trivial solution is to combine the existing virtual avatar models based on NeRF with continual learning methods.
However, there are some critical issues in this approach: learning new appearances and poses can cause the model to forget past information, which in turn leads to a degradation in the rendering quality of past appearances, especially color bleeding issues, and incorrect human body poses.
In this work, we propose a maintainable avatar (MaintaAvatar) based on neural radiance fields by continual learning, which resolves the issues by utilizing a Global-Local Joint Storage Module and a Pose Distillation Module. Overall, our model requires only limited data collection to quickly fine-tune the model while avoiding catastrophic forgetting, thus achieving a maintainable virtual avatar.
The experimental results validate the effectiveness of our MaintaAvatar model.
\end{abstract}

% Uncomment the following to link to your code, datasets, an extended version or similar.
%
% \begin{links}
%     \link{Code}{https://aaai.org/example/code}
%     \link{Datasets}{https://aaai.org/example/datasets}
%     \link{Extended version}{https://aaai.org/example/extended-version}
% \end{links}

\section{Introduction}



Free-viewpoint rendering of 3D scenes has attracted significant academic and industrial attention, with effective solutions such as neural radiance fields (NeRF) and 3D Gaussian Splatting (3DGS) demonstrating impressive performance in this field.~\cite{nerf,barron2021mip,martin2018lookingood,martin2021nerf,wang2021ibrnet,zhang2020nerf++,kerbl20233d,qian20243dgs,li2024animatable,zhou2024feature,sun2024recent}.
Among them, human-specific methods \cite{neuralbody,liu2020neural,humannerf,sun2022human,hu2024gauhuman,pan2023transhuman,zhao2022humannerf,li2024animatable} model dynamic human bodies by establishing a deformable radiance field, achieving the driving and free-viewpoint rendering of dynamic human bodies.

In this paper, we focus on creating a maintainable avatar that models a certain character with ever-changing poses and appearances (see Figure \ref{fig:shouye}).
% In the real world, the poses and appearances of a person could be updated frequently. 8.13修改如下:
In the real world, the poses and appearances of a person could be updated frequently, and consequently, the corresponding human avatar should also be updatable.
Employing conventional static human-nerf methods to track real-world trends will lead to a high demand for computation, training time, and extra storage room, let alone the need for abundant training samples from new styles and poses. 
We raise such a question:
\textit{Is it possible to maintain a dynamic human avatar using only a few shots from the updated character, with the ability to trace back to any point of time? }
\begin{figure}
  \centering
  \includegraphics[width=0.5\textwidth]{shouyetu.pdf} % 插入PDF图片,调整宽度
  % \vspace{-7mm}
  % \caption{In reality, a person's pose and appearance are constantly updating, and our model is capable of continuously learning new data in sequence. Then, our model is capable of rendering any viewpoint, human pose, as well as any appearance that it has seen before.} 8.13修改为:
  \caption{In reality, a person's pose and appearance constantly update. Our maintainable avatar is designed to continuously learn from sequential data, enabling it to render any previously encountered viewpoint, pose, or appearance.}
  \label{fig:shouye}
\end{figure}

The main challenge of such a task lies in the old avatar re-rendering with old poses and appearances, which is related to the well-known catastrophic forgetting \cite{french1999catastrophic,yang2023continual} of deep neural networks.
Specifically, if the model directly learns from new data, it tends to forget the information about past poses and appearances, resulting in poor rendering outcomes.
Further, we identify two major issues that employ the popular `replay' strategy that stores or generates (fake) old inputs for future task training~\cite{meil,cl-nerf,clnerf,shin2017continual} into a human avatar:
(1) The appearances between different tasks can interfere with each other, resulting in color bleeding and affecting the geometric rendering quality of each task. (2) During the process of continual learning of novel appearances and poses, the model may render incorrect poses in past tasks, consequently affecting the driving of the virtual avatar.
 


To address these problems, we propose two key solutions. First, the Global-Local Joint Storage Module stores the distinctions between global and local information of different appearances separately in global embeddings and Tri-planes. This allows the model to better differentiate the variations among different appearances. Second, the Pose Distillation Module extracts the pose information from past tasks as supervision signals for novel tasks, enabling the model to better retain the pose information of past tasks. 
% Additionally, considering that in reality, the datasets of a subject are usually limited in size, we construct a pre-trained model throughƒ training on various other human bodies.
% Based on the prior knowledge from the retrained model, our model can quickly adapt to new human bodies.
Additionally, we create a pretrained model trained on multiple human bodies as the initialization for training, which can adjust the surface of the avatar model with only a few images, and quickly adapt to new human bodies.
We summarize our main contributions as follows:
\begin{itemize}
    % \item 
    % To the best of our knowledge, we are the first to explore maintainable virtual avatars to handle sequential data.
    % % Our model MaintaAvatar is the first work specially designed for maintainable virtual avatars. 
    % We analyze the two critical issues in the continual learning of maintainable virtual avatars.
    % In order to prevent color bleeding between different appearances, we introduce the Global-Local Joint Storage Module that can precisely model the differences between various appearances.
    % \item 
    % Besides, to alleviate the information loss in human pose, we introduce the Pose Distillation Module, which is capable of preserving the correct pose information of past appearances.
    % \item
    % Finally, we establish a sustainably maintainable free-viewpoint personalized virtual avatar capable of generating novel appearances and novel poses. Only a few images (5-15 images) of a person are required for our model to rapidly create a representation of the virtual avatar in the novel appearance. Experimental results in two datasets demonstrate the effectiveness of our model.
    \item 
    We study the problem of modeling a maintainable virtual avatar, which only needs a few shots to quickly update the character's appearance and also maintain the ability to render the old appearance. This maintainable virtual avatar can adapt in real-time to updates in the appearance of real-world individuals.
    \item 
    We analyze the two critical issues in the continual learning of maintainable virtual avatars. First,
    in order to prevent color bleeding between different appearances, we introduce the Global-Local Joint Storage Module that can precisely model the differences between various appearances.
    \item 
    Besides, to alleviate the information loss in the human pose, we introduce the Pose Distillation Module, which is capable of preserving the correct pose information of past appearances.

    
\end{itemize}
Experimental results in two datasets demonstrate the effectiveness of our model, achieving state-of-the-art performance.
\section{Related Work}
\label{sec:related}
%-------------------------------------------------------------------------
\noindent\textbf{Neural Radiance Fields.}
NeRF \cite{nerf} is a widely acclaimed 3D technology that primarily focuses on synthesizing novel views. It represents a 3D environment as an implicit neural radiance field. Given a camera pose, it can generate images from any viewpoint using ray tracing. Specifically, NeRF \cite{nerf} utilizes a Multilayer Perceptron (MLP) \cite{multilayer} to map spatial coordinates and view directions to their corresponding colors and opacity. NeRF has sparked significant interest and extensive research across a range of domains, including applications in autonomous driving \cite{hu2024pc,chen2024s,cheng2023uc,feldmann2024nerfmentation}, controllable human avatars \cite{humannerf,personnerf,neuralbody,sun2022human,jiang2022neuman}, large urban scenes \cite{wang2023neural,xu2023grid,turki2023suds}, and text-driven 3D scene editing \cite{bao2023sine} and so on.

\noindent\textbf{Human-specific Neural Representation.}
Previous work \cite{humannerf,personnerf,zhao2022humannerf,hu2024gauhuman,hu2024gaussianavatar,neuralbody} expressed the human body as an implicit neural field, learning dynamic human figures by transforming bodies of different poses into a T-pose in a standard space. 
\cite{personnerf} learns human bodies with multiple appearances from unstructured data, addressing sparse observations in multi-outfit datasets by fusing appearance features via a shared network. However, it cannot create a virtual avatar that continually updates appearances while retaining the ability to render old ones. \cite{actorsnerf,mps,sherf,pan2023transhuman,kwon2023neural,zhao2022humannerf} propose a human body NeRF model that can quickly generalize to unseen human bodies in a few-shot setting without the need for continual learning. However, their performance is poor. 
\cite{liu2024humangaussian,kolotouros2024dreamhuman} generate virtual avatars through text guidance, and \cite{zhang2023text,mendiratta2023avatarstudio} further provide the capability of controllable editing, but they cannot create virtual avatars corresponding to real-world characters.
Recent works have leveraged Gaussian Splatting \cite{wu20234d,chen2023text,kerbl20233d,tang2023dreamgaussian} for human representation, achieving impressive results. \cite{hu2024gauhuman,qian20243dgs,hu2024gaussianavatar} further use it for fast training and real-time rendering.

\noindent\textbf{NeRF for Continual Learning.}
NeRF for continual learning is a new research hotspot. \cite{clnerf,meil,instant} propose a replay-based algorithm to establish continual learning NeRF, which can continuously learn new perspectives or dynamic scenes and achieve impressive results. \cite{cl-nerf} propose a lightweight expert adaptor to adapt to scene changes and a knowledge distillation learning objective to retain the invariant model. \cite{ILnerf} focusing on real-world scenarios with unknown camera poses. However, they cannot handle dynamic human bodies as they lack body modeling.

In contrast, MaintaAvatar is the first to propose maintaining dynamic virtual characters, enabling sequential learning of diverse poses, viewpoints, and appearances.
\label{sec:intro}

\begin{figure*}[t]
      \centering
      \includegraphics[width=1\linewidth]{yanshiwengao.pdf}
      % \vspace{1pt}
      \caption{$\textbf  {MaintaAvatar Pipeline.}$ In this paper, we propose a continual learning strategy pipeline primarily based on the replay method. 
      %We establish a small storage space to store the camera parameters and human body poses trained for each task. 
      During the training of \textit{Task T}, we replicate and freeze the network $ {\Theta}_{T-1}$ from the past \textit{Task T-1}. Given the camera parameters from \textit{Task T-1}, the network $\Theta_{ T-1}$ can generate corresponding patches and the residual human body pose of one randomly selected past appearance, which are utilized to supervise the training of \textit{Task T}. Simultaneously, $\Theta_T$ 
      is trained using images from the new appearance. In addition to image supervision, we incorporate a Pose Distillation Module to enhance the memory of past pose information, thereby improving rendering quality. Ultimately, our model is capable of continuously learning the novel appearance without forgetting past appearances.}
      \label{fig:continuelearing}
\end{figure*}

\section{MaintaAvatar}
\label{sec:network}
Our approach, MaintaAvatar, maintains a virtual avatar that can continuously learn novel poses and novel appearances without the need for retraining, saving both training and data collection costs. The pipeline is shown in Figure \ref{fig:continuelearing}. We introduce a deformation field to model the dynamic avatar (Section \ref{seq:deformation}) and employ a strategy based on generative replay to enable continual learning of the model, avoiding catastrophic forgetting (Section \ref{continual learning strategy}).   Building on this, we propose the Global-Local Joint Storage Module to model global and local appearance variations (Section \ref{sub:global-local}). 
Furthermore, we propose a Pose Distillation Module to mitigate the rendering of incorrect poses during the continual learning process (Section \ref{sub:pose}). 

\subsection{Preliminaries}

 
\noindent\textbf{Deformable NeRF Based on SMPL Model.}
\label{seq:deformation}
Following PersonNeRF \cite{personnerf}, we model NeRF \cite{nerf} as a deformable neural radiance field. Based on the observed SMPL model parameters, we can warp the canonical volume $\text{F}_c$ to the observed volume $\text{F}_o$ \cite{humannerf,personnerf}. 
% PersonNeRF\cite{personnerf},HumanNeRF\cite{humannerf} also empoly the strategy on human representations. But they can't continuously maintain a virtual avatar and would lead to catastrophic forgetting. In contrast, our model leverages the generative replay algorithm to maintain a virtual avatar, continuously learning new appearances while preserving past ones.

Specifically, similar to PersonNeRF \cite{personnerf}, with SMPL model, we establish a rigid deformation field. By employing inverse linear skinning, we transform the vertices from the canonical volume $\text{F}_c$ into their corresponding positions in the observed volume $\text{F}_o$: 
\begin{equation}
\text{F}_\mathit{o} (\mathbf{x},\mathbf{p}) = \text{F}_c (\text{T}_{\mathit{skel}}(\mathbf{x},\mathbf{p}) ).
\end{equation}
where $\mathbf{p} = (\mathit J,\Omega)$ represents body pose, including joint positions $\mathit J$ and local joint rotations $\Omega$. $\mathbf{x}$ represents the coordinates of the sampling points. $\text{T}_{skel}$ is defined as rigid body motion, which maps an observed position to 
the canonical space:
\begin{equation}
\text{T}_{\mathit{skel}}(\mathbf{x},\mathbf{p}) = \sum^K_{i=1} \textit{w}^i_o(\mathbf{x})(\textit{R}_i \mathbf{x} + \textbf{t}_ \mathit{i}).
\end{equation}
where ($\textit{R}_i$,$\textbf{t}_ \mathit i$) is the mapping of the $i$-th bone from the observation volume to the canonical volume, which can be computed through body pose $\mathbf p$. $\textit{w}^i_o(\mathbf x)$ and $\textit{w}^i_c(\mathbf x)$ are the corresponding weight in the observed volume and the canonical volume \cite{humannerf,personnerf}, respectively. We derive the formula for computing $\textit{w}^i_o(\mathbf x)$ as:
\begin{equation}
\textit{w}^i_o(\mathbf x) = \dfrac{\textit{w}^i_c(\mathbf x)(\textit{R}_i \mathbf x + \textbf{t}_ \mathit i)}{\sum^K_{k=1} \textit{w}^k_c(\mathbf x)(\textit{R}_k \mathbf x + \textbf{t}_ \mathit k)}.
\end{equation}
To address the issue of inaccuracies in the body pose parameters from the dataset, 
some work utilizes a plug-and-play module $MLP_p$ \cite{humannerf,personnerf,hu2024gauhuman}. This module predicts the residual between the current body pose parameters and the true body pose parameters, correcting the body pose. As follow: 

\begin{equation}
{\Delta_\mathrm \Omega} (\mathbf p) = \text{MLP}_{\theta_{\rm pose}} (\rm \Omega).
\end{equation}
Then the pose $\mathbf p$ could be updated to:
\begin{equation}
\textit{P}_{\rm{pose}}(\mathbf{p}) = (\mathit{J},  {\Delta_ \mathrm \Omega} (\mathbf{p}) \otimes  \rm \Omega).
\end{equation}
\begin{figure*}[t]
  \centering
  \includegraphics[width=0.9\textwidth]{crop_7.31.pdf} % 插入PDF图片,调整宽度
  \caption{$\textbf{Pipeline for MaintaAvatar Network}$ $\Theta$ $\textbf{Structure.}$ For any given human body pose, we utilize skeletal motion based on the SMPL model to transform the body from the observation space to the canonical space. Meanwhile, we employ a network $MLP_p$ to predict the residual $\Delta_\Omega(\mathbf{p})$ between the current pose parameters and the true pose parameters. Subsequently, our Global-Local Joint Storage Module generates Tri-plane-based local embedding and global embedding for each appearance. These embedding, along with coordinate embedding, are fed into $MLP_o$ to predict color and opacity.}
  \label{fig:network}

\end{figure*}
\noindent\textbf{Continual Learning for NeRF.}
\label{continual learning strategy}
The primary approach for continual learning for NeRF is to integrate a continual learning strategy with the NeRF network, parameterized by ${\Theta}$. And $\Theta_{ T}$ represents the network parameter of the current task, and $\Theta_{ T-1}$ represents the network structure of the past task. At every time step t of the continual NeRF: 

1. Freeze and copy the model $ {\Theta}_{T-1}$ from $\textit{Task T-1}$.

2. Given camera parameters, and pose parameters from the current task, current model $ {\Theta}_{T}$ firstly transforms the human body from the canonical volume $\text{F}_c$ to the observed volume $\text{F}_o$ and then utilizes an MLP network to render the color corresponding to each ray $\widehat C(\mathbf{r})$.

3. Utilizing the saved human body pose and camera parameters from past tasks, the network $ {\Theta}_{T-1}$ executes the process for step 2, and obtains the color supervision signal $\widetilde C(\mathbf{\bar r})$ of the past tasks.

4. The supervision signals $\widetilde C(\mathbf{\bar r})$ generated by model $\Theta_{ T-1}$ and the images obtained from the current task are used to supervise the training of model $\Theta_{ T}$.
 




% \subsection{Architecture}
% \label{architecture}
% Our overall architecture consists of ray generation and the network $\Theta$. We first formulate how to generate past rays for the past model $\Theta_{ T-1}$ to obtain supervision signals, and then we show how our network renders dynamic human bodies.

% \noindent\textbf{Generate past rays.}
% \label{generate past rays}
% To obtain the supervision signal from the past model $ {\Theta}_{T-1}$, we first need to generate past rays. To do this, firstly, we save all the camera parameters and human body pose parameters from past tasks, which take up a very small amount of storage space. 
% Secondly, we generate past rays on a patch-by-patch basis:
% \begin{equation}
% M(\theta_{camera},\mathbf p)\longrightarrow \mathcal{\bar R} .
% \end{equation}
% where $M$ represents the computational formula for generating rays from camera parameters and human body poses, $\theta_{camera}$ represents the camera parameters, $\mathbf p$ represents the human body pose, and $\mathcal{\bar R}$ represents the patch rays generated on a patch-by-patch basis.


% \noindent\textbf{Network $ {\Theta}$.}
% \label{network}
% Our network architecture is shown in the \cref{fig:network}. Based on continual learning NeRF (\cref{continual learning strategy}), we decompose appearance information into geometry and color. Existing work, such as PersonNeRF\cite{personnerf}, does not take into account that the different appearances of a human body can be decomposed into different geometric shapes and colors. To make the model's representational capacity more comprehensive, we introduce a geometry embedding aimed at representing the human's geometric characteristics. More specifically, we drew inspiration from the findings in \cite{inv}, which observed that MLPs inherently segregate object geometric properties from their color attributes. In this model, the early layers of MLPs store geometric information, while the later layers store color information. Based on this insight, we incorporate the geometry embedding, at the first layer of $MLP_o$ and the color embedding at the n-th layer (we set n=6 in MaintaAvatar), as shown in \cref{fig:network}.
% \begin{equation}
% {MLP_o}:(\gamma(x),\ell^g,\ell^c)\longrightarrow(c, \sigma).
% \end{equation}
% where $MLP_o$ is shown in \cref{fig:network} (a). $\gamma(x)$, $\ell^g$, $\ell^c$ respectively represent the coordinate embedding, geometry embedding, and color embedding.


\subsection{Global-Local Joint Storage Module}
\label{sub:global-local}
Most current researchs model dynamic scene variations using global geometry and color embeddings. However, our experiments show that this approach causes mutual interference between the novel and past appearances, leading to color blending (Figure \ref{fig:global}). 
Unlike the continual simple changes in dynamic scenes \cite{clnerf}, changes in human appearance usually involve drastic variations in color and geometry (Figure \ref{fig:shouye}). It is challenging to represent these significant spatial variations through global embeddings alone in continual learning, making global embeddings inadequate for accurately representing human appearance variations.

To address this challenge, we propose the Global-Local Joint Storage Module to model both global and local appearance variations. Global embeddings can represent the overall changes in human appearance, while local embeddings is used to represent the fine-grained variations based on these global changes. As shown in the Figure \ref{fig:network},
we randomly generate the global geometry embedding $\ell^g$ and color embedding $\ell^c$ to model global variations in appearance. On this basis, we introduce local embedding $\ell^t$ to model local variations superimposed on the global changes. 
To capture the local information of each appearance, we randomly generate a trainable condition embedding $\ell^a$ for each appearance, which is fed into the generator $\mathcal{G}$ to generate a Tri-plane. Then, for each sampling point on the ray, the local embeddings can be obtained by querying its position coordinates $x$ on the corresponding appearance Tri-plane.
\begin{equation}
\ell^t = q(\mathcal{G} (\ell^a) ,x)  
\end{equation}
Finally, we use the following formula to obtain the opacity and color of each sampling point:
\begin{equation}
{MLP_o}:(\gamma(x),\ell^g,\ell^c,\ell^t                             
             )\longrightarrow(c, \sigma).
\end{equation}
where $MLP_o$ is shown in Figure \ref{fig:network}. $\gamma(x)$, $\ell^g$, $\ell^c$, $\ell^t$ respectively represent the coordinate embedding, global geometry embedding, global color embedding and local embedding.


\begin{table*}[t]

\centering
\begin{tabular}{l|ccc|ccc}
\toprule
& \multicolumn{3}{c|}{\textbf{novel view}} & \multicolumn{3}{c}{\textbf{novel pose}} \\
\cmidrule{2-4} \cmidrule{5-7} 
\textbf{Method}& PSNR$\uparrow$ & SSIM$\uparrow$ & LPIPS$\downarrow$ & PSNR$\uparrow$ & SSIM$\uparrow$ & LPIPS$\downarrow$ \\
\midrule
Joint & 29.507 & 0.9652 & 34.579 &29.626&0.9702 &27.329\\
\midrule
$\text{CLNeRF}$ &12.3&0.927&152 &-&-&-                             \\ 
$\text{MEIL-NeRF}$ &25.810&0.9429&71.282 &25.902&0.9465&69.822                             \\ 
$\text{PersonNeRF}$ & 28.199 & 0.9626 & 44.354 &29.008&0.9665&36.724\\
$\text{PersonNeRF}_{CL}$ & 28.605 & 0.9632 & 38.826& 28.678 & 0.9665 & 32.889 \\
\midrule
 Ours & \textbf{29.495} & \textbf{0.9663} & \textbf{36.497} & \textbf{29.630} & \textbf{0.9702} & \textbf{30.166}\\

\bottomrule
\end{tabular}
\caption{The overall free-viewpoint and novel pose rendering experimental results on ZJU-MoCap  \cite{neuralbody}, which represent the evaluation metrics across all previously trained tasks using the weights trained on the final task. The best results are marked in bold.}
\label{tab:table1}
\end{table*}

\subsection{Pose Distillation Module}

\label{sub:pose}
Human pose is an important topic in the field of human body rendering. Inaccuracies in poses can lead to a significant degradation in rendering quality, resulting in unrealistic effects. $MLP_p$ is a plug-and-play method to correct pose errors in datasets, and is suitable for most human-specific neural representation. 
However, we found that current continual learning models for the human body may introduce incorrect poses from past tasks because $MLP_p$ overfits to learning new poses. Therefore, maintaining the accuracy of poses from past tasks during continual learning is a challenge to address. In this section, we propose a Pose Distillation Module. As shown in the Figure \ref{fig:continuelearing}. Similar to multi-layer distillation loss \cite{sun2019patient}, we distill the outputs of $MLP_p$ that perform pose correction.
\begin{equation}
\mathcal{L}_{\mathrm {POSE}} = \parallel \widehat \Delta_\Omega(\mathbf{p}) -   \widetilde \Delta_\Omega(\mathbf{p}) \parallel_2^2.
\label{eq:POSE}
\end{equation}
where $\widehat \Delta_\Omega(\mathbf{p})$  represents the output of the $MLP_p$ for the novel model and $ \widetilde \Delta_\Omega(\mathbf{p})$  represents the output of the $MLP_p$ for the previously frozen model, as shown in the Figure \ref{fig:network}. we calculate the loss using the L2 norm. In practice, our training process is divided into two phases. The first phase focuses on optimizing rendering quality, during which we do not activate the pose distillation loss function, as doing so could result in blurred rendering outcomes. However, in the second phase of training, which begins at $t_0$ as indicated in Equation \ref{eq:lambda}, we shift our focus to pose correction. In this phase, we activate the pose distillation loss function while freezing all components of the network except for the network $MLP_p$ and the color embedding $\ell^c$. This strategic shift allows the model to better capture and memorize the pose information from previous appearances. Additionally, the continual learning in color embedding will prompt the model to find a balance between the colors of the novel and past appearances.





\subsection{Optimization}
\subsubsection{Loss Function.}
The overall loss function is composed as follows:
\begin{equation}
\mathcal{L} = \lambda_1 \mathcal{L}_{\mathrm {CR}} + \lambda_p \mathcal{L}_{\mathrm {CL}} + \lambda_\beta \mathcal{L}_{\mathrm {POSE}}.
\end{equation}
% where $\mathcal{L}_{\mathrm {LPIPS}}$ are referenced in \cref{sec:Model Optimize}, $\mathcal{L}_{\mathrm {MSE}}$, $\mathcal{L}_{\mathrm {POSE}}$ are respectively defined by \cref{eq:MSE},\cref{eq:POSE}.
$\mathcal{L}_{\mathrm {POSE}}$ is defined by Equation \ref{eq:POSE}. $\mathcal{L}_{\mathrm {CR}}$ represents the loss function for learning novel tasks, while $\mathcal{L}_{\mathrm {CL}}$ represents the loss function for memorizing past tasks. $\mathcal{L}_{\mathrm {CR}}$ and $\mathcal{L}_{\mathrm {CL}}$ are respectively defined by the following formulas:
\begin{equation}
\mathcal{L}_{\mathrm {CR}} = \sum_{\mathbf{r} \in \mathcal{R}}[\parallel  \widehat C(\mathbf{r}) -   C(\mathbf{r}) \parallel_2^2]   +  \lambda_2 LPIPS(\widehat C(\mathbf{r}), C(\mathbf{r})).
\label{eq:current loss}
\end{equation}

\begin{equation}
\mathcal{L}_{\mathrm {CL}} = \sum_{\mathbf{\bar r} \in \mathcal{\bar R}}[\parallel  \widehat C(\mathbf{\bar r}) -   \widetilde C(\mathbf{\bar r}) \parallel_2^2]   +  \lambda_2 LPIPS(\widehat C(\mathbf{\bar r}),\widetilde C(\mathbf{\bar r})).
\label{eq:incre loss}
\end{equation}
where $\mathbf{r}$ refer to the current rays of the set of rays $\mathcal{R}$ in each batch, while $\mathbf{\mathrm{\bar r}}$ refer to the past rays of the set of rays $\mathcal{\bar R} 
 $ in each batch. $\mathit C$, $\widehat{C}$ and $\widetilde C$  are the ground truth RGB colors, current network predicted RGB colors, and past network predicted RGB colors as supervisory signals. 
 
 We set $\lambda_2$ to 0.2. $\lambda_1$, $\lambda_p$ refer to weights for controlling the trade-off of current rays and past rays \cite{meil}. In which, $\lambda_1$ = 0.2 and $\lambda_p$ is defined by the following formula:

\begin{equation}
\label{eq:lambda}
\lambda_{p} = 
\begin{cases} 
\sin\left(-\frac{\pi}{2} + \frac{\pi(t-t_{\text{init}})}{t_{\text{max}}-t_{0}-t_{\text{init}}}\right) + 1 & \text{if } t < t_{\text{max}} - t_0 \\
1 & \text{otherwise}.
\end{cases}
\end{equation}
where $t_\text{init}$, $t_\text{max}$ respectively represent the initial and final iteration of the current task. $t_{0}$ represents the iteration to activate the Pose Distillation Module. For $\lambda_\beta$, if $t < t_{\text{max}} - t_0$, we set it to 0; otherwise, we set it to 800. It is worth noting that if $t<t_{max}-t_{0}$, we will freeze all networks except for network $MLP_p$ and the color embedding $\ell^c$.
% \begin{equation}
% \lambda_{\beta} = 
% \begin{cases} 
% 0 & \text{if } t < t_{\text{max}} - t_0 \\
% 800 & \text{otherwise}.
% \end{cases}
% \end{equation}





\section{Experiments}
% We describe the datasets used in the experiments in \cref{subsec:datasets}, followed by our experimental setup and comparison methods in \cref{subsec:setting}. Subsequently, we present the experimental results in \cref{subsec:Results} and conduct ablation studies in \cref{subsec:Ablations}.
\subsection{Datasets}
\label{subsec:datasets}
Our model is evaluated on ZJU-MoCap \cite{neuralbody} \cite{neuralbody} and THuman2.0 dataset \cite{yu2021function4d}. 

\noindent\textbf{ZJU-MoCap \cite{neuralbody}.}
For ZJU-MoCap \cite{neuralbody}, we select subjects (377, 392, 393, 394) for our dataset, as they all feature the same individual in different sets of clothing. This dataset includes one camera assigned for training and the other 22 cameras for evaluation. For each task, we choose only five images with different viewpoints (ensuring a wide distribution of viewpoints as much as possible.) and different poses for training.

\noindent\textbf{THuman2.0 \cite{yu2021function4d}.}
For Thuman2.0 \cite{yu2021function4d}, we select subjects (262, 220, 207, 125) as the dataset. Thuman2.0 \cite{yu2021function4d} provides SMPL but does not offer rendered images or corresponding camera parameters. We use PyTorch3D to obtain rendered images from different viewpoints. To fully utilize the limited dataset, we render images from four viewpoints (0, 90, 180, 270 degrees) for each pose as the training set and render images from nine viewpoints (0, 40, 80, ..., 280, 320 degrees) for evaluation.


\begin{table*}[t]
\centering
% \resizebox{\textwidth}{!}{%
\setlength{\tabcolsep}{3pt}
\begin{tabular}{lcccccccccccc}
\toprule
& \multicolumn{3}{c}{\textbf{subject 262}} & \multicolumn{3}{c}{\textbf{subject 220}} & \multicolumn{3}{c}{\textbf{subject 207}} & \multicolumn{3}{c}{\textbf{subject 125}} \\%{\textbf{subject 207}}  & \multicolumn{3}{c}{\textbf{subject 125}}\\
\cmidrule(lr){2-4} \cmidrule(lr){5-7} \cmidrule(lr){8-10} \cmidrule(lr){11-13} 
\textbf{Method}& PSNR$\uparrow$ & SSIM$\uparrow$ & LPIPS$\downarrow$ & PSNR$\uparrow$ & SSIM$\uparrow$ & LPIPS$\downarrow$ & PSNR$\uparrow$ & SSIM$\uparrow$ & LPIPS$\downarrow$ & PSNR$\uparrow$ & SSIM$\uparrow$ & LPIPS$\downarrow$ \\
\midrule
$\text{CLNeRF}$ &12.9&0.94&236&12.9&0.938&235&14.2&0.913&123&10.8&0.907&224\\ 
$\text{MEIL-NeRF}$ &18.907&0.9344&78.161&17.142&0.9210&97.789&18.411&0.9409&72.456&15.484&0.9215&96.626\\ 
$\text{PersonNeRF}$  &20.094&0.9298&82.099&20.485&0.9338&78.517&21.560&0.9493&59.248&18.930&0.9390&71.145 \\
$\text{PersonNeRF}_{CL}$ & 20.905 & 0.9415 & 63.568 & 20.130 & 0.9365 & 72.227& 22.951 & 0.9624 & 45.814 & 19.448 & 0.9449 & 65.084 \\
 Our method & \textbf{22.316} & \textbf{0.9491} & \textbf{56.023} & \textbf{21.642} & \textbf{0.9449} & \textbf{64.173} & \textbf{23.998} & \textbf{0.9664} & \textbf{41.271} & \textbf{20.209} & \textbf{0.9488} & \textbf{60.884} \\
        
\bottomrule
\end{tabular}% 
% }
\caption{The overall free-viewpoint experimental results on Thuman2.0 \cite{yu2021function4d}. The best results are displayed in boldfaced font.}
\label{tabthuman2.0}
\end{table*} 



\subsection{Implementation Details}
\label{subsec:setting}
The random seed is set to 42. The network $MLP_o$ and the $MLP_p$ have 8 and 4 layers respectively. The global color embedding $\ell^c$ (length 48), global geometry embedding $\ell^g$ (length 16), and condition embedding $\ell^a$ (length 16) are all optimized during training. The Tri-plane has dimensions of 3*512*512*8. We set the learning rates for the $MLP_o$ and both the $\ell^c$ and $\ell^g$ to $5\times10^{-4}$, and the rest to $5\times10^{-5}$. Adam is adopted as the optimizer. 
For the current task, we sample 6 patches of 32×32 size, whereas for past tasks, we sampled one patch of 64×64 size. 128 points are sampled from each ray. In the ZJU-MoCap dataset \cite{neuralbody}, each task is trained for 12,000 iterations, with the pose distillation loss (Equation \ref{eq:POSE}) inactive for the first 10,000 iterations and activated for the final 2,000. In contrast, in the Thuman2.0 dataset \cite{yu2021function4d}, tasks undergo 80,000 iterations, divided into two phases: the pose distillation loss remains inactive for the initial 70,000 iterations and becomes active for the last 10,000. To enable the model to quickly fine-tune to new human bodies, as well as to cope with a small amount of data, we constructed a pretrained model of other similarly sized human bodies as initialization to train the new human.

% \begin{table}[t]
% \centering
% \caption{Experimental results of continual learning for different human body poses with the same human appearance.}
% \label{newpose}
% % \resizebox{\textwidth}{!}{%
% \begin{tabular}{lccc}
% \toprule
  
% \textbf{Method}& PSNR$\uparrow$ & SSIM$\uparrow$ & LPIPS$\downarrow$   \\
% \midrule
% Model-0 &29.41717&0.96472&33.24391\\
% $\text{PersonNeRF}_{CL}$ & 29.27588 & 0.96587 & 28.95529  \\
%  Our method & \textbf{29.58165} & \textbf{0.96749} & \textbf{28.85697} \\
        
% \bottomrule
% \end{tabular}% 
% % }
% \end{table}



\subsection{Comparison with Other Methods}
\label{subsec:Results}

Up to now, there has been no ongoing research regarding the continual learning of multi-appearance human bodies. Therefore, we compare our model with $\mathrm{CLNeRF}$, $\mathrm {MEIL\text{-}NeRF}$, $\mathrm {PersonNeRF}$ and $\mathrm {PersonNeRF}_{CL}$, as shown in Table \ref{tab:table1}. Specifically, we respectively augment $\mathrm{CLNeRF}$ \cite{clnerf}, $\mathrm {MEIL\text{-}NeRF}$ \cite{meil} and $\mathrm {PersonNeRF}$ \cite{personnerf} with a pre-trained model as well as global geometry and color embeddings. Additionally, we apply the continual learning strategy of $\mathrm {MEIL\text{-}NeRF}$ to $\mathrm {PersonNeRF}$ as the $\mathrm {PersonNeRF}_{CL}$.
As a reference of performance upper bound, we report the results of training on the dataset of all tasks together denoted as ``Joint''. The evaluation metrics we use are PSNR \cite{hore2010image}, SSIM \cite{hore2010image}, and LPIPS. 

\noindent\textbf{Results on ZJU-MoCap \cite{neuralbody}.}
We first compare our method with others and ``Joint'' on ZJU-MoCap \cite{neuralbody}. The free-viewpoint and novel pose rendering results are shown in Table \ref{tab:table1}.
In all metrics, our method approaches the performance of ``Joint'' but does not fully reach it, while still outperforming all other methods.
% However, we still surpass ``Joint'' in terms of PSNR and SSIM. This is because, after introducing the Global-Local Joint Storage Module, the 'Joint' method finds it more challenging to identify the optimal parameters. This further demonstrates that the Global-Local Joint Storage Module is more suitable for continual learning.


% \vspace{-0.5cm}
The visualization results of free-viewpoint rendering on ZJU-MoCap \cite{neuralbody} are illustrated in Figure \ref{fig:global}. We use the model of final task ($Task3$). The $\mathrm {PersonNeRF}_{CL}$'s rendering of past tasks demonstrates issues with poor rendering quality and incorrect pose. Specifically, there are problems with missing details in color rendering and overall rendering inaccuracies in poses of the head, arms, and so on. Our proposed Global-Local Joint Storage Module and Pose Distillation Module effectively address these issues.

% In addition, we also verified that our model is capable of continually learning different human body poses with the same human appearance, as shown in \cref{newpose}. We use the subject 394 of ZJU-MoCap \cite{neuralbody} dataset, Model-0 represents the initial model before continual learning, having been trained on only one task. 

% The experiment validated that our model MaintaAvatar can continually learn novel poses to enhance the pose generalization of the virtual avatar.



% \begin{figure}[t]

%     \centering
%     \includegraphics[width=0.98\linewidth]{折线图5.pdf}
    
%     \caption{The experimental results of the model for rendering task 0 after training up to tasks 0-3. For instance, the "$Task$ $2$" indicates the rendering performance on $Task$ $0$ after the model has been trained on $Task$ $2$. $Task$ $0-3$ are defined as four different appearances, corresponding to subjects (394, 393, 377, 392) respectively.}
%     \label{fig:task0}
    
% \end{figure}


\begin{figure}[t]
    % \vspace{-4mm}
    \centering
    \includegraphics[width=1.0\linewidth]{zjurender.pdf}
    \caption{The visualization comparison results for the free-viewpoint rendering of our method and the $\mathrm {PersonNeRF}_{CL}$ on ZJU-MoCap \cite{neuralbody} for the past tasks. Our method demonstrates superior rendering quality, especially in terms of color and human pose.}
    % \vspace{-5mm}
    \label{fig:global}
\end{figure}

\noindent\textbf{Results on Thuman2.0 \cite{yu2021function4d}.}
We further evaluated our model on Thuman2.0 \cite{yu2021function4d}. The results in Table \ref{tabthuman2.0} show that our model significantly outperforms the other models. At the same time, Figure \ref{fig:G-L消融Thu} demonstrates that our model with Global-Local Joint Storage Module can learn new appearances more quickly by better understanding the differences between different appearances.

\begin{figure}[t]
  % \vspace{-4mm}
  \centering
  % 左侧图片
    \includegraphics[width=1\linewidth]{GLthu.pdf}
    \caption{
    The Global-Local Joint Storage Module can help the model learn new appearances more quickly.}
    \label{fig:G-L消融Thu}
\end{figure}

% \documentclass{article}
% \usepackage{graphicx}
% \usepackage{booktabs}
% \usepackage{float} % For the [H] specifier

% \begin{document}

\begin{table}[t]
\centering
\setlength{\tabcolsep}{3pt} % 调整列间距,默认是6pt
\fontsize{10}{12}\selectfont
\begin{tabular}{l|ccc|ccc}
\toprule
& \multicolumn{3}{c|}{Novel View} & \multicolumn{3}{c}{Novel Pose} \\
\cmidrule(lr){2-4} \cmidrule(lr){5-7}
Method & PSNR & SSIM & LPIPS & PSNR & SSIM & LPIPS \\
\midrule
w/o pose & 29.21 & 0.9652 & 37.52 & 29.59 & 0.9698 & 30.83 \\
w/o G-L  & 28.89 & 0.9646 & 36.60 & 29.15 & 0.9686 & 30.99 \\
Full & \textbf{29.50} & \textbf{0.9663} & \textbf{36.50} & \textbf{29.63} & \textbf{0.9702} & \textbf{30.17} \\
\bottomrule
\end{tabular}
% }
\caption{Ablations in ZJU-MoCap: ``w/o G-L'' omits the Global-Local Joint Storage Module, ``w/o pose'' omits the Pose Distillation Module.}
\label{tab:ablations}
\end{table}

\begin{figure}[t]
  % \vspace{-4mm}
  \centering
  % 左侧图片
    \includegraphics[width=\linewidth]{GLxiaoron.pdf}
    \caption{
    The rendering result for the past task (yellow apperance) after training on the current task (green apperance). The Global-Local Joint Storage Module effectively prevents color bleeding between different appearances.}
    \label{fig:G-L消融}
\end{figure}

% \begin{figure}[t]
%     \centering
%     % 第一张图片

%         \includegraphics[width=0.7\linewidth]{pose消融实验3.pdf}
%         \caption{The pose distillation loss function improves the quality of novel view synthesis. The Pose Distillation Module is referred to as ``pose''.}
%         \label{fig:image1}
% \end{figure}
    
%     % \vspace{4mm} % 调整两张图之间的间距
% \begin{figure}[t]    
%     % 第二张图片
   
%         \includegraphics[width=0.7\linewidth]{超参数5.pdf}
%         \caption{The hyperparameter $\lambda_p$ enables the model to find balance between current appearance and past appearance.}
%         \label{fig:lambdap}
   
% \end{figure}
\begin{figure}
    \centering
    % 第一张图片
    % \begin{subfigure}
        \includegraphics[width=0.65\linewidth]{pose3.pdf}
        \caption{The pose distillation loss function improves the quality of novel view synthesis. The Pose Distillation Module is referred to as ``pose''.}
        \label{fig:image1}
    % \end{subfigure}
    
    % \vspace{4mm} % 调整两张图之间的间距
    \end{figure}
    \begin{figure}
    \centering
    % 第二张图片
    % \begin{subfigure}
        \includegraphics[width=0.65\linewidth]{chaocanshu5.pdf}
        \caption{The hyperparameter $\lambda_p$ enables the model to find balance between current appearance and past appearance.}
        \label{fig:lambdap}
    % \end{subfigure}
\end{figure}

\subsection{Ablation Studies}
\label{subsec:Ablations}
% \noindent\textbf{Effectiveness of anti-forgetting.} \Cref{fig:task0} shows the free-viewpoint rendering experimental results of the model for rendering $Task$ $0$ after training up to $Task$ $0-3$. These results can measure the degree of forgetting of past appearances by the model. "Ours w/o pose" denotes the results of removing the Pose Distillation Module from our model, and "Ours w/o head" denotes the results of without using the head representation sharing strategy. We compare our model with $\mathrm {PersonNeRF}_{CL}$. Experiments demonstrate that both the Global-Local Joint Storage Module and the Pose Distillation Module can enhance the model's memory capability. Notably, as training on novel tasks progresses, the performance of our method on $Task$ 0 is consistently improved under three metrics. This is because we compensate for the limitation of the few-shot training set by training with more images.

\noindent\textbf{Global-Local Joint Storage Module.}
``w/o G-L'' denotes the results without using the Global-Local Joint Storage Module. ``full model'' denotes our full model. Table \ref{tab:ablations} and Figure \ref{fig:G-L消融} demonstrate the rendering improvements brought by the Global-Local Joint Storage Module. We use the old model with green clothing to learn yellow clothing. Without using this module, the generated yellow appearance is influenced by the dark green appearance. Figure \ref{fig:G-L消融Thu} demonstrates the training speed improvements.



% \begin{figure}
%   \vspace{-4mm}
%   \centering
%   % 左侧图片
%     \includegraphics[width=0.8\linewidth]{pose消融实验3.pdf}
%     \caption{The pose distillation loss function improves the quality of novel view synthesis. The Pose Distillation Module is referred to as "pose". }
%     \label{fig:image1}
%     \vspace{-4mm}
% \end{figure}


% \begin{figure}
%     \vspace{-4mm}
%     \hspace{0.6cm}
%     \includegraphics[width=0.8\linewidth]{超参数5.pdf}
%     \caption{The hyperparameter $\lambda_p$ enables the model to find balance between current appearance and past appearance.}
%     \label{fig:lambdap}
% \end{figure}

\noindent\textbf{Pose Distillation Module.} ``w/o pose'' denotes the results of removing the Pose Distillation Module from our model. The ablation study regarding pose distillation is shown in Figure \ref{fig:image1} and Table \ref{tab:ablations}. We present the visualization experimental results on Thuman2.0 \cite{yu2021function4d}. The Pose Distillation Module is capable of rendering a human figure with the correct pose more effectively.




% \vspace{-1mm}
\noindent\textbf{$\lambda_p$ Hyperparameter.}
The hyperparameter $\lambda_p$ controls the trade-off between current and past tasks, as demonstrated in the ablation experiment shown in Figure \ref{fig:lambdap}. $\lambda_p$ is defined as the Equation \ref{eq:lambda}. Without $\lambda_p$, the model struggles to balance the current task and past tasks, especially in cases where there is a significant color difference between current and past appearances. 






% \begin{figure}[t]
%   \vspace{-4mm}
%   \centering
%   % 左侧图片
%     \includegraphics[width=\linewidth]{crop_pretrain1.pdf}
%     \caption{
%     We conducted a comparison of the novel view synthesis capabilities across multiple few-shot methods for NeRF.}
%     \label{fig:pretrained}
% \end{figure}




% \begin{figure}[t]
%   \vspace{-4mm}
%   \centering
%   % 左侧图片
%   \begin{minipage}[t]{0.45\linewidth}
%     \includegraphics[width=\linewidth]{G-L消融.pdf}
%     \caption{
%     The head representation sharing strategy improves the quality of head rendering, handling details such as the eyes, mouth, and nose with greater finesse.}
%     \label{fig:head}
%   \end{minipage}
%   \hfill % 添加一些空间或者使用 \hspace{5mm} 控制空间大小
%   % 右侧图片
%   \begin{minipage}[t]{0.52\linewidth}
  
%     \includegraphics[width=\linewidth]{crop_pretrain1.pdf}
%     \caption{
% We conducted a comparison of the novel view synthesis capabilities across multiple methods. }
%     \label{fig:pretrained}
%   \end{minipage}
%   \vspace{-5mm}
% \end{figure}
% \begin{figure}
    
% \end{figure}
% \noindent\textbf{Pretrained model.} \Cref{fig:pretrained} shows a comparison experiment among several methods addressing the few-shot challenge.
% "+DietNeRF" and "+FreeNeRF" represent the integration of their models with HumanNeRF\cite{humannerf}. HumanNeRF\cite{humannerf} serves as the baseline for PersonNeRF\cite{personnerf} and does not employ any specific solutions for the few-shot issue. PersonNeRF\cite{personnerf} aims to enhance the generalizability of novel views by rendering them concurrently during training and regularizing against the novel view images. DietNeRF\cite{dietnerf} tackles the few-shot problem by leveraging CLIP \cite{radford2021learning } to bridge the semantic gap between training views and novel views. FreeNeRF\cite{freenerf} modifies the encoding strategy to prompt the model to first learn low-frequency information. 
% While DietNeRF\cite{dietnerf} and FreeNeRF\cite{freenerf} perform commendably in static scenes, their effectiveness falters with dynamic human figures due to the limited dataset of only five images. 
% Differently,

% Our approach outperforms others by delivering the sharpest and most accurate visual results. At the same time, our model requires only one-tenth the training duration of other models.
% It is because we employ pre-trained models obtained from several other human figures, which allows our model to rapidly modify surface details at a speed ten times faster.

% This enables a swift fine-tuning to new human bodies, thus avoiding the issue of blurriness. 

\section{Limitations and Conclusion}
\noindent\textbf{Limitations.} Our method shows performance drops with significant clothing shape changes and struggles with pose generalization in complex poses due to limited exposure in the few-shot dataset.

\noindent\textbf{Conclusion.} MaintaAvatar is the first work to propose a maintainable virtual avatar to address the issue of continual changes in human appearance. We introduce two main components: a Global-Local Joint Storage Module to prevent color bleeding between different appearances and a Pose Distillation Module to solve the pose forgetting problem in continual learning. MaintaAvatar requires only a minimal set of training images to fine-tune to a novel appearance within a fixed, shorter training time, without forgetting the past appearances. 
% \bibliographystyle{splncs04}

\section*
{Acknowledgments}
This work was supported partially by NSFC (92470202, U21A20471), Guangdong NSF Project (No. 2023B1515040025).

\bibliography{aaai25}

\end{document}
