\section{Related Work}
\label{sec:related}
%-------------------------------------------------------------------------
\noindent\textbf{Neural Radiance Fields.}
NeRF \cite{nerf} is a widely acclaimed 3D technology that primarily focuses on synthesizing novel views. It represents a 3D environment as an implicit neural radiance field. Given a camera pose, it can generate images from any viewpoint using ray tracing. Specifically, NeRF \cite{nerf} utilizes a Multilayer Perceptron (MLP) \cite{multilayer} to map spatial coordinates and view directions to their corresponding colors and opacity. NeRF has sparked significant interest and extensive research across a range of domains, including applications in autonomous driving \cite{hu2024pc,chen2024s,cheng2023uc,feldmann2024nerfmentation}, controllable human avatars \cite{humannerf,personnerf,neuralbody,sun2022human,jiang2022neuman}, large urban scenes \cite{wang2023neural,xu2023grid,turki2023suds}, and text-driven 3D scene editing \cite{bao2023sine} and so on.

\noindent\textbf{Human-specific Neural Representation.}
Previous work \cite{humannerf,personnerf,zhao2022humannerf,hu2024gauhuman,hu2024gaussianavatar,neuralbody} expressed the human body as an implicit neural field, learning dynamic human figures by transforming bodies of different poses into a T-pose in a standard space. 
\cite{personnerf} learns human bodies with multiple appearances from unstructured data, addressing sparse observations in multi-outfit datasets by fusing appearance features via a shared network. However, it cannot create a virtual avatar that continually updates appearances while retaining the ability to render old ones. \cite{actorsnerf,mps,sherf,pan2023transhuman,kwon2023neural,zhao2022humannerf} propose a human body NeRF model that can quickly generalize to unseen human bodies in a few-shot setting without the need for continual learning. However, their performance is poor. 
\cite{liu2024humangaussian,kolotouros2024dreamhuman} generate virtual avatars through text guidance, and \cite{zhang2023text,mendiratta2023avatarstudio} further provide the capability of controllable editing, but they cannot create virtual avatars corresponding to real-world characters.
Recent works have leveraged Gaussian Splatting \cite{wu20234d,chen2023text,kerbl20233d,tang2023dreamgaussian} for human representation, achieving impressive results. \cite{hu2024gauhuman,qian20243dgs,hu2024gaussianavatar} further use it for fast training and real-time rendering.

\noindent\textbf{NeRF for Continual Learning.}
NeRF for continual learning is a new research hotspot. \cite{clnerf,meil,instant} propose a replay-based algorithm to establish continual learning NeRF, which can continuously learn new perspectives or dynamic scenes and achieve impressive results. \cite{cl-nerf} propose a lightweight expert adaptor to adapt to scene changes and a knowledge distillation learning objective to retain the invariant model. \cite{ILnerf} focusing on real-world scenarios with unknown camera poses. However, they cannot handle dynamic human bodies as they lack body modeling.

In contrast, MaintaAvatar is the first to propose maintaining dynamic virtual characters, enabling sequential learning of diverse poses, viewpoints, and appearances.
\label{sec:intro}

\begin{figure*}[t]
      \centering
      \includegraphics[width=1\linewidth]{yanshiwengao.pdf}
      % \vspace{1pt}
      \caption{$\textbf  {MaintaAvatar Pipeline.}$ In this paper, we propose a continual learning strategy pipeline primarily based on the replay method. 
      %We establish a small storage space to store the camera parameters and human body poses trained for each task. 
      During the training of \textit{Task T}, we replicate and freeze the network $ {\Theta}_{T-1}$ from the past \textit{Task T-1}. Given the camera parameters from \textit{Task T-1}, the network $\Theta_{ T-1}$ can generate corresponding patches and the residual human body pose of one randomly selected past appearance, which are utilized to supervise the training of \textit{Task T}. Simultaneously, $\Theta_T$ 
      is trained using images from the new appearance. In addition to image supervision, we incorporate a Pose Distillation Module to enhance the memory of past pose information, thereby improving rendering quality. Ultimately, our model is capable of continuously learning the novel appearance without forgetting past appearances.}
      \label{fig:continuelearing}
\end{figure*}