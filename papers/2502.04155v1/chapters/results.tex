\section{Case studies}
\label{sec:results}
For our case study, we consider Boston and Cambridge, Massachusetts, USA, as a unified system. Together, these cities have a combined population of approximately 750,000 and cover an area of 55 square miles~\cite{area_bos,area_cam}. 
In this study, we analyze the travel behavior of around 30,000 citizens per hour, representing approximately~$4\%$ of the population in motion at any given time. 
%
To structure the study, the region is divided into eight zones, each centered around key city landmarks: Massachusetts Institute of Technology, Harvard University, Massachusetts General Hospital, Logan Airport, Boston City Hall, Boston Common, the Prudential Center, and Fenway Park. 
These landmarks were selected to capture a mix of essential services (airport, hospital, universities, government buildings) and leisure destinations (parks, shopping centers, and entertainment venues).
%
It is important to note that this model does not capture all travel movements in the city.
Specifically, residential and office areas are not explicitly included in the zone definitions.
As commuting between home and work accounts for a significant portion of daily travel, and such trips typically follow consistent patterns less influenced by cost, we exclude these travelers from our analysis. 
Instead, our focus is on more dynamic travel behavior, where mode choice is more sensitive to pricing and availability.
%
\noindent Without loss of generality, we categorize travelers into three distinct populations, based on trip intent: employees, students, and leisure travelers. 
Note that a traveler's classification depends on the purpose of their trip (e.g., a working professional commuting to the office is classified as an employee, but the same individual visiting a park on the weekend is considered a leisure traveler).
The distribution of travelers originating from each zone is determined by the nature of its defining landmark.
Hospitals, for instance, generate a high number of employee trips, whereas universities primarily contribute to student travel.
In particular, the data on trip patterns is sourced from local survey.
We consider four modes of transportation: bus, \gls{abk:amod}, shared bikes, and walking. 

\begin{figure}[h]
    \centering
    \includegraphics[width=0.9\linewidth]{img/pixel_perfect_fig2.pdf}
    \caption{a) shows the mode-share proportion of citizens leaving zones~$1$ and~$2$. b) After doubling the number of buses in each zone from~$15$ to~$30$, the proportion of citizens traveling via bus also doubles. A mode-shift from walking to bus can be observed at the second iteration. c) Line graph showing the average travel time, municipality revenue from bus and bike tax, and CO$_2$ emissions produced during each timestep. When doubling the number of buses, the average travel time decreases and CO$_2$ emissions increase. d) Line graph showing the revenue and costs incurred by the bus company. When doubling the number of buses, costs marginally increase, but revenue significantly increases.}
    \label{fig:double_num_buses}
\end{figure}

\noindent Using the \gls{abk:GUI}, users can configure the transportation system from the perspective of the municipality and three private mobility providers (the bus company, the \gls{abk:amod} operator, and the bike-sharing service). 
When acting as the municipality, users set tax rates on bikes and cars as a percentage of company revenue. 
As private mobility \glspl{abk:msp}, users control fleet allocation across zones and determine pricing strategies. 
Bus fares are per trip, whereas \gls{abk:amod} and bike-sharing fares are distance-based.
%
The system reaches an equilibrium by minimizing travel costs while respecting capacity constraints. 
To assess the quality of the equilibrium, we report several metrics of interest, including average travel time for a citizen across all routes, modes, and populations, environmental impact in form of CO$_2$ emissions for the various modes, and costs and revenues for \glspl{abk:msp}.
The \gls{abk:GUI} enables users to iteratively modify the city's configuration and compare the resulting equilibria. 
In the following, we provide a detailed overview of the \gls{abk:GUI}'s functionality and its role in facilitating system analysis and showcasing the complexity of the problem.

\subsection{GUI Working Principle}
We conduct case studies and analyze the results based on the visualization provided by the \gls{abk:GUI}. 
First, we present a nominal case to introduce the core functionalities of the \gls{abk:GUI}. 
Then, we examine how changes in the number of buses and the pricing of \gls{abk:amod} services affect equilibrium outcomes by adjusting the corresponding parameters and comparing the visualized results.
%

\begin{figure}[h]
    \centering
    \includegraphics[width=0.7\linewidth]{img/picture_perfect_donuts_iteration_2_fig_3.pdf}
    \caption{a) shows the mode-share proportion of citizens leaving zones~$5$ and~$6$. b) After doubling the price of \gls{abk:amod}, the proportion of citizens traveling via \gls{abk:amod} decreases to~$0\%$ due to citizens choosing a cheaper mode, which is primarily bus.}
    \label{fig:double_amod_price}
\end{figure}

\paragraph*{Nominal Case}
To establish a baseline scenario, we place~$15$ buses,~$90$ AMoD vehicles, and~$60$ bikes in each of Boston's~$8$ zones. 
These values are derived from the total daily availability of buses~\cite{num_buses}, taxis~\cite{num_taxis} and Ubers~\cite{num_ubers}, and Blue Bikes~\cite{num_bikes} in Boston. 
% source buses: http://roster.transithistory.org/
% source bikes: https://www.boston.gov/departments/transportation/bluebikes
% source taxi: https://www.welcomepickups.com/boston/taxi/
% source uber: https://www.americaninno.com/boston/uber-driver-data-boston-has-10000-uber-drivers/
We impose a 20\% revenue tax on cars and bikes, while bus fares remain fixed at 2 USD per ride~\cite{busfares}.
\gls{abk:amod} and bike-sharing fares are distance-based, with \gls{abk:amod} services starting at 1 USD per mile, and bike-sharing at 0.20 USD per mile.
% bus fare source: https://www.mbta.com/fares/bus-fares
% amod numbers from finding uber price and dividing by distance
%
In line with existing studies~\cite{lanzetti2023interplay}, we define the value of time of employees, students, and leisure travelers as~$35$ USD/h,~$15$ USD/h, and~$7$ USD/h, respectively. 
%
Using these parameters, the \gls{abk:GUI} generates the output reported in \cref{fig:gui_overview}.
%
\begin{figure*}[t]
    \includegraphics[width=\linewidth]{img/gui_overview.pdf}
    \caption{Overview of the \gls{abk:GUI}. In the leftmost column is the drop down menu for city selection, the buttons to run and re-run the simulation, an interaction menu to set parameters, and a table enumerating the zones in the city. In the center column, the user is presented with graphs showing performance metrics across simulation iterations. In the rightmost column, a ``System Overview" visualizing the proportion of all citizens \textit{leaving} zone~$i$ traveling by mode~$m$. Finally, a map allows the user to locate the studied zones geographically.}
    \label{fig:gui_overview}
\end{figure*}
%
Based on the nominal case shown in \cref{fig:gui_overview}, we make two adjustments to observe how the system reacts. 
First, we double the number of buses per zone from~$15$ to~$30$. 
Second, we double the price of \gls{abk:amod} from~$1$ USD per mile to~$2$ USD per mile. In each case, we hold all the other variables constant from the nominal case.

\paragraph*{Doubling the number of buses}
In this iteration, we double the number of buses per zone from~$15$ to~$30$, holding all other parameters constant. 
Focusing on zones~$1$ and~$2$, the original bus capacity was~$750$ passengers per zone, which accounted for~$44\%$ mode-share (see~\cref{fig:double_num_buses}a).
After increasing the number of buses, the capacity doubles to 1,500 passengers per zone, and bus mode-share doubles (see~\cref{fig:double_num_buses}b).
This shift results in several notable effects.
First, since walking is the slowest mode, replacing it with bus transit significantly shortens travel duration for affected travelers.
Second, a higher number of buses leads to greater fuel consumption and emissions (see~\cref{fig:double_num_buses}c).
Finally, more passengers generate additional fare-based income for the bus service (see~\cref{fig:double_num_buses}d).


% \begin{figure}[t!]
% \centerline{\includesvg[width=0.75\columnwidth]{CCTA25/img/pixel_perfect_fig2.svg}}
% \caption{Example of using SVG on Overleaf}
% \label{fig: example}
% \end{figure}

% \includesvg[width=0.5\textwidth]{CCTA25/img/pixel_perfect_fig2}

% \begin{figure}[h]
%     \centering
%     {\includesvg[width=7cm]{CCTA25/img/pixel_perfect_fig2.svg}}
%     \caption{test fig2}
%     \label{fig:iter1_zones12}
% \end{figure}

% \begin{figure}[h]
%     \centering
%     \includegraphics[width=7cm]{CCTA25/img/iter1_zones12.png}
%     \caption{first iteration: this should be one fig with the second iteration pie charts?}
%     \label{fig:iter1_zones12}
% \end{figure}

% \begin{figure}[h]
%     \centering
%     \includegraphics[width=7cm]{CCTA25/img/iter2_zones12.png}
%     \caption{second iteration: this should be one fig with the first iteration pie charts?}
%     \label{fig:iter2_zones12}
% \end{figure}

% \begin{figure}[h]
%     \centering
%     \includegraphics[width=7cm]{CCTA25/img/iter2_graphs_mun_bus.png}
%     \caption{this should be one fig with the pie charts? and i need the legend}
%     \label{fig:fig2_graph}
% \end{figure}

\paragraph*{Doubling the price of \gls{abk:amod}}
In this iteration, we double the price of \gls{abk:amod} from~$1$ USD per mile to~$2$ USD per mile. 
In the first iteration, \gls{abk:amod} was fully saturated due to its significantly higher speed compared to other modes (see~\cref{fig:double_amod_price}a).
When fares across modes are similar, travel time becomes the dominant factor in decision-making.
However, with the price increase to 2 USD per mile, the cost of \gls{abk:amod} rises, making it less attractive to travelers.
As a result, many citizens switch to alternative modes, primarily buses.
This shift is particularly evident in zones 5 and 6, as shown in~\cref{fig:double_amod_price}b, where a substantial portion of former \gls{abk:amod} users opt for bus transit instead.
% \begin{figure}[h]
%     \centering
%     \includegraphics[width=7cm]{CCTA25/img/iter1_zones56.png}
%     \caption{first iteration: this should be one fig with the second iteration pie charts?}
%     \label{fig:iter1_zones12}
% \end{figure}

% \begin{figure}[h]
%     \centering
%     \includegraphics[width=7cm]{CCTA25/img/iter2_zones56.png}
%     \caption{second iteration: this should be one fig with the first iteration pie charts?}
%     \label{fig:iter2_zones12}
% \end{figure}

\subsection{Remarks}
The demand patterns used in these case studies are approximate or rounded estimates.
Going forward, incorporating more granular and accurate data would allow for a more precise representation of the city's mobility dynamics.
In this analysis, we modified one variable a a time to isolate its impact and demonstrate how the model responds to specific parameter changes.
However, simultaneously altering multiple variables can introduce unexpected network-wide effects, underscoring the complexity and interdependencies inherent in urban mobility systems.


\subsection{\gls{abk:GUI} capabilities}
In this paper, we have shown two iterations for one region with eight zones.
The \gls{abk:GUI} currently supports 3 realities: Lugano, Switzerland ($8$ zones); 
Boston/Cambridge, MA, USA ($8$ zones); and Kyiv, Ukraine ($12$ zones). 
Importantly the \gls{abk:GUI} is designed with a strong emphasis on user engagement, allowing users to contribute and model their own cities.
The user must provide coordinates of the zones, the number of citizens per population at each zone, demand patterns along each route for each population, and information about value of time for each population.

%%%%%%%%%%%%%%%%%%%%%%%%%%%%%%%%%%%%%%%%%%%%%%%%%%%%%%%%%%%%%%%%%%%%%%%%%%%% NOT USING

% \begin{figure}[h]
%     \centering
%     \includegraphics[width=7cm]{CCTA25/img/iter_1_graphs.png}
%     \caption{Plots showing metrics for equilibrium for the first iteration. The x-axis of each plot is the iteration. We judge the quality of the equilibrium based on these metrics. For each \glspl{abk:msp}, to make a profit, we want revenue to exceed cost.}
%     \label{fig:iter1_graphs}
% \end{figure}

% \begin{figure}[h]
%     \centering
%     \includegraphics[width=7cm]{CCTA25/img/iter_1_pie.png}
%     \caption{Pie charts showing the proportions of travelers leaving zone~$i$ via each mode. By noting the modes that travelers choose in this simulation, the ``player" can modify their strategy in the next iteration.}
%     \label{fig:iter1_pie}
% \end{figure}

% Notice in zones~$2,3,4,7 \text{ and } 8$ that more than~$50\%$ of the citizens originating from that zone are walking (see~\cref{fig:iter1_pie}). Walking is the slowest mode, and since there are many people walking, the city has a high average travel time (see ``Municipality" in~\cref{fig:iter1_graphs}). Additionally, in zones~$1,2,3,4,7 \text{ and } 8$, all fare-based modes (bus, \gls{abk:amod}, bike) are at maximum capacity, which suggests that there is more demand for fare-based modes than we are currently providing (see~\cref{fig:iter1_pie}). In~\cref{fig:iter1_graphs}), we see for all \gls{abk:msp} that revenue exceeds costs, which is further evidence that we do not have too many of one vehicle type. 

% \begin{figure}[h]
%     \centering
%     \includegraphics[width=7cm]{CCTA25/img/iter2_graphs.png}
%     \caption{CAPTION}
%     \label{fig:iter2_graphs}
% \end{figure}

% \begin{figure}[h]
%     \centering
%     \includegraphics[width=7cm]{CCTA25/img/iter2_pie.png}
%     \caption{CAPTION}
%     \label{fig:iter2_pie}
% \end{figure}

% For example, if we now place~$10,000$ buses in each zone (while keeping all the other values constant), the bus capacity far exceeds the number of citizens originating at each zone. Since the bus cost exceeds bus revenue, the bus is no longer profitable (see~\cref{fig:iter2_graphs}. Even though there are~$500,000$ bus seats available from each zone, not all the citizens choose to ride the bus (see~\cref{fig:iter2_pie}), which means that at these prices and distances, \gls{abk:amod}

% \subsection{Kyiv, Ukraine}
% Kyiv has twelve zones, corresponding to Shevchenko University, Parliament, Feofania Hospital, Boryspil Airport, Adriiyvskyi Descent, Golden Gate, Dynamo Stadium, Maidan (Independence Square), Kyiv Bridge, St. Sophia's, Opera House, and Holosiivsky Park.

% \paragraph*{Price and value of time}
% \textbf{nominal price of a ride}
% Based on \todo{cite}, we represent the value of time of employees, students, and leisure travelers as 131 UAH/h, 48 UAH/h, and 20 UAH/h, respectively. 