\section{Modeling Interactions in Multi-Modal Mobility Systems using Population Games}
\label{sec:model}
% Note for myself later (for the process): method for writing the section, read through the code and made sure all the variables in the code are accounted for in the model
In this section, we present a model characterizing the interactions between stakeholders of a multi-modal mobility ecosystem.
Specifically, our model builds on~\cite{belgioioso2021semi}, which focuses on solving generalized Nash Equilibrium problems in monotone aggregate games.

\subsection{Modeling of the Populations and Zones}
We model a city by means of~$N$ zones, indexed by~$i \in \{1, \dots, N\}$. 
Each citizen in zone~$i$ seeks to travel to a different zone~$j$ where~$j \in  \{1, \dots, N\} \setminus \{i\}$ (i.e., we do not allow any citizen to stay in their original zone).
To travel between zones, citizens can choose between~$M$ modes of transportation.
Rather than modeling the behavior of each individual separately, we group travelers into~$K$ distinct populations (e.g., students, employees, and leisure travelers).
For population~$k$ in zone $i$, we denote~$d_{ijk}$ as the travel demand for population~$k$, representing the number of individuals from population~$k$ who wish to travel from zone~$i$ to zone~$j$.
By definition,~$d_{ijk} \in \mathbb{R}_{\geq 0}$ for all $i,j \in \{1, \dots, N\}$ and $k \in \{1, \dots, K\}$.
\subsection{Modeling of the MSPs}
In this model, we focus on the capacity of \glspl{abk:msp} as they depart from the origin zone.
Each provider~$m$ has a maximum capacity~$C_i^m$ for trips originating in zone~$i$.
Importantly, note that~$C_i^m$  represents the total number of available ``seats'', and not the number of actual vehicles.
This distinction is important for modes of transport where a single vehicle can carry multiple passengers per trip (e.g., buses).
The value of~$C_i^m$ may depend on various factors, such as the municipal budget for purchasing vehicles, or the cost of licenses for \gls{abk:amod} systems.
In the developed \gls{abk:GUI}, however,~$C_i^m$ is determined by the number of vehicles allocated to each zone by the player.
In this model, we do not distinguish capacity for various destinations that originate from the same origin.
Instead, we consider the maximum number of passengers that mode~$m$ can transport from the origin zone as a whole.
Finally, we associate the index~$m=0$ to walking, and set~$C_i^0 = +\infty $ for all~$i \in N$, ensuring that walking is always a feasible mode of transportation (the slowest one). 
\subsection{Cost of Traveling}
The cost~$c_{ijk}^m$ for an individual from population~$k$ traveling from zone~$i$ to zone~$j$ via mode~$m$ is determined by combining the monetary fare for the selected mode with the travel time~$t_{ij}^m$.
The travel time is converted into monetary terms leveraging the average value of time for population~$k$, reflecting the opportunity cost of travel.
For each origin-destination (OD) pair~$i,j$ and mode~$m$, the travel time~$t_{ij}^m \in \mathbb{R}_{\geq 0}$ represents the time required to travel from zone~$i$ to zone~$j$.
If~$i=j$, then~$t_{ij}^m=0$ for all modes.
Notably,~$t_{ij}^m$ and~$t_{ji}^m$ are not necessarily equal; for instance, during morning rush hour, commuting into a city may take longer than leaving it.

\noindent Let~$p_{ij}^m$ denote the monetary fare for traveling from zone~$i$ to zone~$j$ using mode~$m$, and let~$w_k$ represent the average value of time for population~$k$ (e.g., calculated based on the average wage of a citizen of population~$k$ inn a given city).
The cost for an individual is then given by
\begin{equation}\label{cost fxn}
    c_{ijk}^m = p_{ij}^m + w_k \cdot t_{ij}^m.
\end{equation}
Notably, the costs experienced by individuals are independent of the number of other travelers (e.g., the cost of taking a bus is the same whether it is shared with one person or fifty).
For the special case of walking ($m=0$), we define the fare as~$p_{ij}^0=0$, ensuring that walking is alwas a feasible option.
However, note that the total cost of walking is not necessarily zero, as it is more time-intensive than other modes of travel.

\subsection{Equilibrium}
We represent the decisions of each population using the variables~$\{x_{ijk}^m\}$, where~$x_{ijk}^m \in [0,1]$ denotes the proportion of population~$k$ along route~$ij$ traveling via mode~$m$.
The total number of citizens in population~$k$ is denoted by~$P_k$, and~$P_{i,j,k}$ represents the number of citizens from population~$k$ traveling along route~$ij$. 
For a configuration to be \emph{feasible}, it must meet the transportation demand while adhering to system's capacity constraints.
%
\begin{definition}[Feasible configuration] \label{def: feasibility}
A configuration~$\{x_{ijk}^{m}\}$ is \emph{feasible} if it satisfies the following conditions:
\begin{enumerate}
\item $x_{ijk}^m\geq 0$ for all~$i,j,k,m$;
\item No citizen of population~$k$ may remain in their original zone, i.e.,
\begin{equation*}
    \sum_{m}x_{iik}^m=0, \text{ for all }i,k;
\end{equation*}
\item All citizens of population~$k$ traveling along route~$ij$ must reach their destination, i.e.,
\begin{equation*}
    \sum_{m}x_{ijk}^m=1, \text{ for all }i,j,k;
\end{equation*}
\item The total demand across all origin-destination pairs and modes must equal the population size, i.e.,
\begin{equation*}
    \sum_{i,j,m} d_{ijk}x_{ijk}^m=P_k \text{ for all }k;
\end{equation*}
\item The system's capacity limits must not be exceeded, i.e., 
\begin{equation*}
    \sum_{j,k} d_{ijk} x_{ijk}^m \leq C_{i}^m \text{ for all }i,m;
\end{equation*}
\end{enumerate}
\end{definition}

From this definition, we have established the minimum requirement for an \emph{equilibrium}. Now, we establish the requirement for an \emph{optimal} solution. A \emph{Nash equilibrium} is attained when no agent can lower their travel cost by taking another mode of transportation. Formally: 

\begin{definition}[Nash Equilibrium]
Let~$\{x_{ijk}^{m}\}$ be a feasible configuration.
We say that~$\{x_{ijk}^m\}$ is a \emph{Nash equilibrium} of the game if for all~$i,j,k,m$ with~$x_{ijk}^{m}>0$ every other mode~$m'\in\{0,\ldots,M\}$ either (i) leads to higher cost, i.e., 
\begin{equation*}
c_{ijk}^m\left(\sum_{j,k}  P_{i,j, k} x_{ijk}^m\right)
    \leq 
c_{ijk}^{m'}\left(\sum_{j,k}  P_{i,j,k} x_{ijk}^{m'}\right),
\end{equation*}
or (ii) it is saturated, i.e., 
\begin{equation*}
    \sum_{j,k}  P_{i,j,k} x_{ijk}^m = C_{i}^m.
\end{equation*}
\end{definition}

In other words, a Nash equilibrium occurs when no subset of individuals can reduce their individual travel cost by switching modes, assuming that all others maintain their choices.
Furthermore, the system-wide travel cost at equilibrium is locally optimal, meaning that no alternative configuration reduces the cost without violating the feasibility constraints.

\subsection{Analysis of the Game}
In this section, we show that the game can be analyzed as a convex optimization problem, guaranteeing \emph{existence} of equilibria. Our result takes the following form:

\begin{theorem}[Equilibria of the game]\label{thm:equilibria}
Let~$\{x_{ijk}^{m}\}$ be a feasible configuration resulting from the convex optimization problem
\begin{subequations}\label{eq:optimization}
\begin{align}
    \min_{x_{ijk}^m}\: &
    \sum_{i,j,k,m} c_{ijk}^m  P_{i,j,k} x_{ijk}^m 
    \\
    \mathrm{s.t. }\: &
    \sum_{j,k} d_{ijk} x_{ijk}^m \leq C_{i}^m \text{ for all }i,m
    \\
    & \sum_{m}x_{ijk}^m=1, \text{ for all }i,j,k, 
    \\
    & \sum_{i,j,m} d_{ijk}x_{ijk}^m=P_k \text{ for all }k
    \\
    &x_{ijk}^m\geq 0 \text{ for all }i,j,k,m.
\end{align}
\end{subequations}
Then,~$\{x_{ij}^{k,m}\}$ is an equilibrium. 
In particular, an equilibrium \emph{always} exists.
\end{theorem}

\begin{remark}[Convexity and Linearity]
    The optimization problem~\eqref{eq:optimization} is indeed convex. 
    All constraints are clearly linear in the decision variables. 
    To show convexity of the objective function, observe that the cost of traveling for each citizen is independent. We aim to minimize the cost of traveling for all agents, but because costs are independent, minimizing the total cost is the same as minimizing the individual cost functions. This makes the objective function also linear in the decision variables. Since the optimization problem is linear, it is also convex. 
    In particular, the optimization problem has~$N^2MK$ decision variables and~$\mathcal{O}(N(N-1)MK)$ constraints (since~$x_{iik}^m=0$ for all~$i\in\{1,\ldots,N\}$).
\end{remark}

\cref{thm:equilibria} states that we can solve for equilibria of the game using an off-the-shelf convex optimizer. In the case of the \gls{abk:GUI}, since our problem is linear, we used the \texttt{linprog} function from SciPy's optimization module.


\subsection{Discussion}
It is worth noting a few key points. First, we abstract away from the topology of the road and public transit network and disregard personal vehicles. By doing so, we can assume that all travelers have instant access to their mode of choice and there is no congestion from other vehicles (public or private). Second, the model cannot account for repeated operation of the system within the given time frame. The \gls{abk:GUI} asks the user to set parameters for a one hour timeframe, and it models all trips within that hour simultaneously. 