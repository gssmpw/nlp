\section{Introduction}
\label{sec:introduction}
In past decades, the continued rise in urbanization has had a deep impact on cities worldwide.
Currently, approximately 55\% of the world's population lives in urban areas, a figure projected to reach 70\% by 2050~\cite{un2020}.
Such population shift has increased travel demaned in metropolitan environments, exacerbating existing externalities such as congestion and pollution~\cite{czepkiewicz2018urbanites}.
As a result, cities must take complex decisions, expanding and reshaping their transportation infrastructure to accommodate new evolving travel needs, all while ensuring accessibility, equity, fairness, sustainability, and performance~\cite{ranchordas2020smart}.
%
In this context, the emergence of \glspl{abk:msp} such as ride-hailing operators, micromobility platforms, and, soon, \gls{abk:amod} systems~\cite{zardinilanzettiAR2021}, provides opportunities and adds complexity.
Notably, such services have gained a substantial role in urben transportation.
While they provide additional, point-to-point mobility options, they are profit-driven, and rely on public infrastructure, raising concerns about their long-term impact, regulations, and social equity~\cite{berger2018drivers,rogers2015social}.
This dynamic highlights the crucial need for effective regulatory strategies to balance public and private interests in urban mobility.
Furthermore, sustainability is a pressing priority.
Critically, cities account for approximately 78\% of the global energy consumption and over 60\% of greenhouse gas emissions~\cite{un2020bis}. Transportation alone is responsible for around 30\% of these emissions in the USA, highlighting the urgency of implementing policies that mitigate environmental impact. 
Cities and governments have recognized this challenge, and have set ambitious targets, such as New York City's plan to increase sustainable transportation modes from 68\% to 80\%, and the European Union's commitment to a 90\% emissions reduction by 2050~\cite{euplan2020,OneNYC}.

\noindent Such interconnected challenges illustrate the complexity of designing and regularing urban mobility systems, and emphasize the need for structured decision-making tools.
In this paper, we present a user-friendly \gls{abk:GUI}\footnote{Available at \href{https://bit.ly/mobility-game}{bit.ly/mobility-game}.} that leverages a simple game-theoretic framework to model multi-modal mobility systems, extending our previous efforts tackling this problem~\cite{zardinilanzetti2021,zardinilanzetti2023,zardini2023camod}.
The \gls{abk:GUI} allows users to make decisions from the perspective of various mobility system stakeholders and assess the impact of such decisions at the system level.
Furthermore, we present numerical examples for the city of Boston to illustrate its features, and provide guidelines for readers interested in replicating the results.

\paragraph{Related Literature}
Our work lies at the interface of game-theoretic modeling of transportation systems and policy making for future mobility.
Game theory has been used to formulate and solve various mobility-related problems, ranging from defining pricing strategies for \glspl{abk:msp}~\cite{mingbao2010pricing, gong2014analysis, chen2016management, kuiteing2017network, bimpikis2019spatial, yang2019subsidy, seo24tcns}, to the analysis of interactions between \glspl{abk:msp} and users~\cite{lei2018evolutionary, dandl2019autonomous, turan2021competition}, and the interactions between authorities and \glspl{abk:msp}~\cite{hernandez2018game, di2019unified, mo2021dynamic, balac2019modeling, lanzetti2023interplay}.
Further efforts have focused on competition between \glspl{abk:msp} and public transit, highlighting the potential disruptive effects of emerging mobility systems~\cite{krichene2017stackelberg, lazar2020optimal}.
Another key research direction studies the role of game theory in policy making, specifically in managing congestion and designing regulatory interventions, such as tolling strategies~\cite{bianco2016game}, incentive mechanisms~\cite{swamy2012effectiveness, paccagnan2021optimal, lazar2020optimal}, and Stackelberg games for traffic network management~\cite{krichene2017stackelberg}.

In addition to game-theoretic approaches, policy research on emerging mobility systems has focused on mitigating externalities and achieving socially efficient solutions.
In this context, studies have proposed mechanisms such as congestion pricing, labor subsidies, and land-use regulations to reduce transportation-related emissions~\cite{fullerton2002can, iwata2016can, zhang2016optimal, maljkovic2023hierarchical}.
Furthermore, economic analyses of ride-hailing and carpooling models have been proposed~\cite{zoepf2018economics,ostrovsky2019carpooling}.
However, while these studies provide valuable insights into specific aspects of mobility policy, they often focus on isolated stakeholders, or ignore the interactive dynamics between different players in the mobility ecosystem.
%
Our work aims to bridge this gap by developing a structured, game-theoretic approach accounting for multi-level decision-making in transportation systems.
\paragraph{Statement of Contribution}
In this paper, we build on~\cite{belgioioso2021semi, zardini2021game, zardini2022} to develop a game-theoretic formulation of interactions in multi-modal mobility systems.
Specifically, we present a framework to formulate interactions problems and solve them to find equilibria.
Furthermore, we present a \gls{abk:GUI} to be used as an educational tool to visualize the complexity of the problem, as well as decision-making tool for stakeholders of the mobility ecosystem. 
Finally, we showcase the framework via a case study of Boston/Cambridge, Massachusetts, USA. 
We conduct a sensitivity analysis with real-world scenarios to highlight related complexities.
\paragraph{Organization of the Manuscript}
The remainder of this paper is organized as follows. 
We specify the problem setting and models for interactions in mobility systems in \cref{sec:model}.
In \cref{sec:results} we detail case studies and present numerical results.
Finally, we draw conclusions and present future research interests in \cref{sec:conclusion}.