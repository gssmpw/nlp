\section{Conclusion}\label{sec:conclusion}
In this paper we have studied the impact of transportation policies and mobility service design leveraging a simple, game-theoretic framework integrated into an interactive \gls{abk:GUI}.
Specifically, we modeled urban mobility dynamics in Bostom/Cambridge.
Through a series of case studies, we examined how changes in fleet allocation, pricing, and taxation influence equilibrium travel behavior, modal choices, and efficiency.
By iterating on these properties, we highlighted the complex interdependencies within mobility systems, reinforcing the need for data-driven decision-making in urban transportation planning. 
Importantly, the \gls{abk:GUI} serves as both as a research tool and an educational platform, allowing users to configure cities, simulate transportation scenarios, and analyze network-wide effects.
Importantly, it provides an engaging way for students and policymakers to explore the challenges of urban mobility, gaining hands-on experience in balancing efficiency, sustainability, and financial factors.

Future work will focus on improving the model capabilities, incorporating more granular, real-world data, and refining modeling techniques. 
%
\section{Acknowledgments}
We want to thank Christian Hartnik, Nicolas Lanzetti, Saverio Bolognani, and Giuseppe Belgioioso, who contributed to an earlier version of the GUI and project.