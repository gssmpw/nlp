\section{Details on VIMPC cost designs} \label{app:hw_details}
In both simulations and experiments, MPPI variants using DCBF to ensure safety utilized a state-dependent cost function:
\begin{equation}
    q(x_k^m) = (x_k^m - x_{g})^\intercal Q (x_k^m - x_{g}),
\end{equation}
whereas the standard MPPI employed the cost,
\begin{equation}
    q(x_k^m) = (x_k^m - x_{g})^\intercal Q (x_k^m - x_{g}) + \mathbf{1}(x_k^m),
\end{equation}
where \( Q = \diag(q_{v_x}, q_{v_y}, q_{\dot{\psi}}, q_{\omega_F}, q_{\omega_R}, q_{e_\psi}, q_{e_y}, q_s) \) represents the cost weights, and \( x_g = \text{diag}(v_g, 0, \dots, 0) \) specifies the target velocity. The collision cost function is defined as:
\begin{equation}\label{JC}
    \mathbf{1}(x_k^m) :=
    \begin{cases}
      0, & \text{if } x_k^m \text{ is within the track}, \\
      C_\text{obs}, & \text{otherwise}.
    \end{cases}
\end{equation}
Unlike standard MPPI, Shield-MPPI and NS-MPPI incorporate a DCBF constraint violation penalty but do not include an explicit collision cost. 