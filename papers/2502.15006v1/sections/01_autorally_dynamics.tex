\subsection{AutoRally Dynamics Modelling}\label{AppendixA} We model the AutoRally using a rear-wheel drive, single-track bicycle model \cite{CSSMPC}.
The vehicle system is represented in curvilinear coordinates using the centerline of the track as the reference curve. Using a curvilinear coordinate system provides a more intuitive interpretation of the vehicle's position and heading relative to the track, as compared to Cartesian coordinates.

To this end, we consider a nonlinear, continuous-time system,
\begin{align}\label{eqn:VehicleDynamicsContinuousSystem}
    \dot{x} = F(x, u),
\end{align}
where the system state of the AutoRally is given by,
\begin{align}
	x &= \begin{bmatrix}
		v_x, v_y, \dot{\psi}, \omega_F, \omega_R, e_\psi,  e_y, s
	\end{bmatrix}^\intercal,
	\label{eqn:autorally_state_space}
\end{align}
 and where $v_x$ is the longitudinal velocity, $v_y$ is the lateral velocity,  and $\dot{\psi}$ is the yaw rate. The front and rear wheel angular velocities are denoted by $\omega_F$ and $\omega_R$.
The yaw error and the lateral distance error from the centerline of the track are denoted by $e_\psi$ and $e_y$, respectively (see \Cref{fig:wheel_velocities}). 
The variable $s$ is the curvilinear position along the track centerline.
The control input to the system \eqref{eqn:VehicleDynamicsContinuousSystem} is,
\begin{align}
    u &= \begin{bmatrix}
		\delta, T
	\end{bmatrix}^\intercal,
\end{align}
where $\delta$ is the steering angle and $T$ denotes the values for throttle input (if positive) or braking input (if negative). The dynamics of a single-track dynamic bicycle model \cite{CSSMPC} used to model \eqref{eqn:VehicleDynamicsContinuousSystem} are given by,
\begin{subequations}\label{eqn:VehicleDynamics}
\begin{align} 
\dot{v}_{x} &= \frac{f_{F x}\cos\delta - f_{F y}\sin\delta + f_{R x}}{m} + v_{y} \dot{\psi}, \\ 
\dot{v}_{y}&= \frac{f_{F x}\sin\delta + f_{F y}\cos\delta + f_{R y}}{m} - v_x \dot{\psi}, \\ 
\ddot{\psi} &=\frac{\left(f_{F y}\cos\delta + f_{F x}\sin\delta\right) \ell_{F} - f_{R y}\ell_{R}}{I_z},\\
\dot{\omega}_F &= -\frac{r_F}{I_{\omega F}}f_{Fx},\\
\dot{\omega}_R &= \Theta(\omega_R, T), \label{eqn:rear_wheel_acceleration}\\
\dot{e}_\psi &= \dot{\psi} - \frac{v_{x} \cos{e_{\psi}} - v_{y} \sin{e_{\psi}}}{1 - \rho(s) e_{y}} \rho(s), \\
\dot{e}_y &= v_{x} \sin{e_{\psi}} + v_{y} \cos{e_{\psi}},\\
\dot{s} &= \frac{v_{x} \cos{e_{\psi}} - v_{y} \sin{e_{\psi}}}{1 - \rho(s) e_{y}},
\end{align}
\end{subequations}
where $m$ and $I_z$ are the vehicle's mass and moment of inertia about the vertical axis, respectively. The radius of the front wheel is $r_F$, and the moment of inertia of the front wheel about the front axle is $I_{\omega F}$. The curvature of the track centerline at position $s$ is $\rho(s)$.
\begin{figure}[!h]
    \centering
    \includegraphics[scale=0.18]{figs/dynamic_bicycle_map_coordinates.png}
    \caption{Schematic of the dynamic bicycle model.}
    \label{fig:wheel_velocities}
\end{figure}
The front and rear tire frictional forces are denoted by $f_{Fx}, f_{Fy}, f_{Rx}, f_{Ry}$, where the subscripts $F, R$ indicate front and rear tires and $x, y$ indicate longitudinal and lateral directions. These frictional forces are computed using the ellipse model in \cite{velenis2010steady},
\begin{subequations}\label{egs:fric_from_normals}
\begin{align}
    f_{ix} &= f_{iz}\mu_{ix}, \\
    f_{iy} &= f_{iz}\mu_{iy} + \Phi_i(\delta, \alpha_i), \label{f_iy}
\end{align}
\end{subequations}
where $i = F, R$ indicates front or rear wheels, $f_{iz}$ is the normal force acting on the tire. The tire friction coefficients $\mu_{ix}, \mu_{iy}$ are dependent on wheel slip angles and the intrinsic tire parameters. 

Inspired by \cite{acosta2018tire}, we use a neural network to model the residual error $\Phi_i(\delta, \alpha_i)$, where $\delta$ is steering angle, $\alpha_i = \arctan(v_{iy}/v_{ix})$ is wheel slip angle. $\Phi_i$ is trained using experimental data collected from the AutoRally. The rear wheel angular acceleration $\dot{\omega}_R$ is given by \eqref{eqn:rear_wheel_acceleration}. Modelling $\dot{\omega}_R$ is challenging, because it is determined by a variety of mechanical and physical conditions, including nonlinear tire behavior, dynamic load transfer, road surface irregularities and the complex interaction between these components. To this end, we use a data-driven approach and utilize a neural network $\Theta(\cdot, \cdot)$ trained on experimental data to model the rear wheel angular acceleration $\dot{\omega}_R$.

The nonlinear, continuous-time system \eqref{eqn:VehicleDynamicsContinuousSystem} is discretized and converted to a discrete-time system using Euler integration,
\begin{align}\label{eqn:VehicleDynamicsDiscrete}
    x_{k+1} = f(x_k, u_k) = x_k + F(x_k, u_k)\Delta t,
\end{align}
where $\Delta t = t_{k+1} - t_k$ is the discretization time interval.
We employ an Unscented Kalman Filter (UKF) on the discrete-time system \eqref{eqn:VehicleDynamicsDiscrete} to identify the vehicle and tire parameters using the approach in \cite{changxi}. The resulting parameters of the vehicle model include the vehicle mass $m = 22~\mathrm{kg}$, moment of inertia $I_z = 1.1 ~\mathrm{kg\cdot m^2}$, $l_F = 0.34~\mathrm{m}$, $l_R = 0.23~\mathrm{m}$, $I_{\omega F} = 0.10 ~\mathrm{kg\cdot m^2}$ and front wheel radius $r_F = 0.095~\mathrm{m}$. The parameters in Pacejka's Magic Formula used by the tire friction elipse model \cite{velenis2010steady} are computed as $\tire_\rB = 4.1$, $\tire_\rC=0.95$, $\tire_\rD=1.1$.