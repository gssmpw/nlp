\section{NCBF Training Details}

The definition of the avoid set is given by \eqref{eqn:AvoidSet}, 
\begin{align}
    \mathcal{A} = \{x|h(x) > 0 \}, \nonumber
\end{align}
where the avoidance heuristic $h(x)$ is used to define the avoid set. 
Furthermore, the definition of the policy value function $V^{h,\pi}(x_0)$ given by \eqref{eqn:ValueFunctionDefinition}, and the ensuing training losses \eqref{eqn:DPcost}, \eqref{eqn:DPcostModified} are all dependent on $h(x)$. 
A suitable selection of the avoidance heuristic $h(x)$ must ensure that its corresponding avoidance set $\mathcal{A}$ encompasses all system states that overlap with any obstacles.
Additionally, the heuristic should improve the quality of information captured by the loss function \eqref{eqn:DPcostModified}, thereby making it more effective for neural network training. 
To this end, a reasonable choice of the avoidance heuristic utilizes coordinates in \eqref{eqn:autorally_state_space},
\begin{align}
    h_0(x) = w_I^2 - e_y^2,
\end{align}
where $w_I = 1.5~\mathrm{m}$ is half of the track width, and $e_y$ is the lateral deviation of the vehicle CoM to the track centerline. The visualization of $h_0(x)$ is demonstrated by the orange curve in \Cref{fig:AvoidanceHeuristicVisualization}.

\begin{figure}[!h]
    \centering
    \centerline{\includegraphics[scale=0.23]{figs/h_track_comparison.png}}
    \caption{Avoidance heuristic visualization. The orange curve shows the original avoidance heuristic $h_0$, while the blue curve demonstrates the modified avoidance heuristic $h$ used for training the NCBF.}
    \label{fig:AvoidanceHeuristicVisualization}
\end{figure}
However, the neural network approximation $V_\theta^{h_0, \pi}(x)$ of the policy value function $V^{h_0, \pi}$ given by \eqref{eqn:ValueFunctionDefinition} may not accurately distinguish the unsafe states in the avoid set $\mathcal{A}_0 = \{x|h_0(x) > 0 \}$ from the rest of the state space, due to the fact that trained neural networks can suffer from insufficient accuracy and precision \cite{jiang2023neural, de2018high}. This is demonstrated by the orange line and its error margins in Fig.\ref{fig:closeup_plot}. To alleviate this problem, we introduce discontinuity to $h_0(x)$ for enhanced tolerance of policy value function modelling errors while maintaining the same avoid set \eqref{eqn:AvoidSet}, such that,
\begin{align}
    \mathcal{A} = \{x|h(x) > 0 \} = \{x|h_0(x) > 0 \}.
\end{align}
where our choice of the avoidance heuristic is given by,
\begin{align}
h(x) = 
\begin{cases} 
h_0(x) - 0.3, & \text{if } e_y < w_I, \\
h_0(x) + 0.2, & \text{if } w_I \leq e_y \leq w_O,\\
2.8, & \text{if } w_O \leq e_y,
\end{cases}
\end{align}
where $w_O$ is the``crash width''. If $w_O < e_y$, the vehicle crashes and needs to be reset. If $ w_I \leq e_y \leq w_O$, the AutoRally collides with the soft drainage pipes but can still continue driving. 
As shown by the blue curve in Fig. \ref{fig:AvoidanceHeuristicVisualization}, $h(x)$ is designed to reduce the error of the avoid set of the resulting NCBF by introducing a discontinuity around zero. This is further demonstrated in Fig. \ref{fig:closeup_plot}.
\begin{figure}[!h]
    \centering
    \centerline{\includegraphics[scale=0.25]{figs/closeup_plot.png}}
    \caption{Improving accuracy and precision of NCBF modeled avoid set boundary by using the modified heuristic $h(x)$. When using the original avoidance heuristic $h_0(x)$ to supervise the NCBF training process, the resulting modeled avoid set boundary can be anywhere within $e_y \in (a, b)$ due to model errors. Instead, using $h(x)$ to supervise the training process results in a modeled avoid set with an accurate boundary shown by the dashed blue line, despite model errors.}
    \label{fig:closeup_plot}
\end{figure}

\begin{figure}[!h]
    \centering
    \centerline{\includegraphics[scale=0.25]{figs/NCBFTraining.png}}
    \caption{NCBF Training. The left figure shows  collected system states (yellow dots) in AutoRally simulations. The right figure visualizes the resulting Neural CBF. The NCBF $B(x)$ takes the 8-dimensional state $x$ as input. 
    The red color indicates an unsafe region where $B(x) > 0$, while the blue color indicates a safe region where $B(x) \leq 0$.}
    \label{fig:NCBFTraining}
\end{figure}

\begin{figure}[!h]
    \centering
    \centerline{\includegraphics[scale = 0.34]{figs/NCBFExampleAdjusted.png}}
    \caption{A simulation example of the value change of the learned Neural CBF $B(x)$ compared to the avoid set heuristic $h(x)$. The orange trajectory produced by the standard MPPI results in a collision with the track boundary. The square purple dot shows the point where $B(x)$ shifts from negative to positive values, corresponding to the purple vertical line in the curve plot above, which shows the value change of the heuristics $B(x)$, $h(x)$ along the MPPI trajectory. The dots along the track centerline are observations collected starting at the square red dot, used to augment the training data.}
    \label{fig:NCBFexample}
\end{figure}
 The orange dots in the left plot in Fig. \ref{fig:NCBFTraining} represent the vehicle states collected using the training data. The dots outside of the track show that the autonomous car went off the track at several turns. These data are used to train a neural DCBF, leading to a visualization as demonstrated by the plot on the right in Fig. \ref{fig:NCBFTraining}, where the NCBF $B(x)$ is mapped onto a two-dimensional Cartesian plane using the mean values of the other state dimensions derived from the training dataset. The red color shows the area where $B(x) > 0$ , and as a result, the neural CBF slows down the Autorally when it approaches sharp turns to ensure safety.

From Fig. \ref{fig:NCBFexample}, we see that the learned NCBF $B(x)$ has a smaller safe set (negative level set) than the avoid set heuristic $h(x)$, as it turns positive earlier than $h(x)$ when the vehicle approaches the track boundary. When used as a safety filter, $B(x)$ can slow down the autonomous vehicle much earlier, making its negative level set control-invariant. 