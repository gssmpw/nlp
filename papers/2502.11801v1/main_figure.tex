\begin{figure*}[tb]
	\centering
	\includegraphics[width=1.0\textwidth]{main_v7.png}
 % \vspace{-2mm}
    \caption{\textbf{Overview of 3D Gaussian Inpainting with Depth-Guided Cross-View Consistency.} Given a 3D Gaussian Splatting model $G_{1:N}$ pretrained on multi-view images $I_{1:K}$ at camera poses $\xi_{1:K}$, our goal is to perform 3D inpainting based on the object masks $M_{1:K}$ (e.g., provided by SAM). With the rendered depth maps $D_{1:K}$, the stage of Inferring Depth-Guide Inpainting Mask is able to refine the inpainting masks to preserve visible backgrounds across camera views. The stage of Inpainting-guided 3DGS Refinement then utilizes such masks to jointly update the new Gaussians $G'_{1:N'}$ for both novel-view rendering and inpainting purposes. 
    % \tc{I think maybe in the middle, the removed gaussian G' can leave the blur hull, and move the $G^{Ref}$ to the right $P^{Ref}$ position, change the $P^{Ref}$ also a bear scene with $G^{Ref}$, otherwise it's hard to know what is the $P^{Ref}$. Another idea is probably visulaize the point cloud instead of image? not sure whether this can better understand $P^{Ref}$ and $G^{Ref}$. So there will be three gaussion, one is original one, second is the one with blur hull, third one is final one with $P^{Ref}$}
    }
    \vspace{-1mm}
	\label{fig:main}
\end{figure*}

