\section{Introduction}\label{sec:intro}


The rise of rapidly mutating virus strains~\citep{hadj2022covid}, exemplified by those related to SARS-CoV-2~\citep{yuki2020covid}, along with drug-resistant cancer cells~\citep{MANS2017}, has heightened the urgency and interest in accelerating the development of effective treatments. 
However, traditional De Novo drug discovery processes are prohibitively expensive, often costing from hundreds of millions to billions of dollars~\citep{dickson2009cost}, due to their extensive and resource-demanding nature. 
Despite considerable efforts, the success rate of drug discovery remains low, with many candidates failing early in the development process.
This has led to an increased focus on drug repurposing, which leverages existing FDA-approved drugs for new therapeutic uses rather than creating new drugs from scratch. While drug repurposing has seen some success~\citep{pushpakom2019drug}, its effectiveness is often constrained because drugs are typically developed with a high specificity for a particular disease.




Drug optimization seeks to overcome the limitations inherent in both De Novo drug discovery and drug repurposing by enhancing an existing FDA-approved drug.  
Drug optimization is becoming an increasingly vital field, yet it remains relatively underexplored compared to drug discovery and repurposing efforts.
DrugImprover~\citep{liu2023drugimprover} has started to clearly define the drug optimization problem by using Tanimoto similarity. 
Additionally, the DrugImprover framework contributes to the drug optimization domain in three key aspects: a detailed workflow for drug optimization, an Advantage-alignment Policy Optimization (APO) reinforcement learning (RL) algorithm to enhance the multi-objective generative model for drug optimization, and an extensive dataset featuring 1 million ligands and their OEDOCK scores for five proteins related to human cancer cells and 24 high-affinity binding sites on the SARS-CoV-2 protein 3CLPro (PDB ID: 7BQY). The DrugImprover framework features a pretrained generative model that is subsequently refined using the APO reinforcement learning algorithm to ensure the molecules produced align with new objectives. Although DrugImprover has demonstrated encouraging outcomes, its effectiveness is limited due to its dependence on the less complex LSTM network architecture, which might lead to limited scalability and capacity, contextual understanding.

On the other hand, generative pretrained language models have shown exceptional performance in diverse areas, from natural language understanding and generation exemplified by ChatGPT~\citep{wu2023brief, ouyang2022training}, to text-to-video conversion as seen in SORA~\citep{liu2024sora}, and in programming and code generation through platforms like PG-TD~\citep{zhang2023planning} and Copilot~\citep{nguyen2022empirical}. Nevertheless, in the field of drug discovery, tools such as DrugGPT~\citep{li2023druggpt}, ChatDrug~\citep{liu2024conversational} and ChemGPT~\citep{frey2023neural}, despite making some initial strides, have not yet achieved performance on par with their counterparts in other domains. The field of drug discovery remains in expectation of breakthroughs similar to the ones achieved by LMs in other domains.


Several challenges have hindered the impact of Language Models (LMs) on drug design: Firstly, the molecules created need to satisfy multiple criteria such as solubility and synthesizability, and must also secure a high docking score against a specific target site. However, existing drug discovery LMs generally only undergo pretraining with molecules and do not focus on enhancing multiple attributes simultaneously. Secondly, given that drugs sharing similar chemical structures tend to exhibit comparable biological or chemical effects~\citep{bender2004molecular}, it is crucial to optimize the drug while preserving the beneficial chemical structure of the original molecule. Lastly, the current methodologies in drug discovery LMs primarily focus on maximizing likelihood during the decoding phase, rather than customizing the optimization process to meet specific goals.


\fix{
REINVENT 4~\citep{he2021molecular, he2022transformer, loeffler2024reinvent} achieves the state of art performance in LM-based
drug optimization by using the Transformer model and conducting experiments with randomly selected ligand pairs that maintain constrained Tanimoto Similarity.
While this method successfully produces ligands with high Tanimoto Similarity, it encounters difficulties in consistently improving the properties of the original drug and tends to simply maximize the likelihood of molecule generation.
}







In this work, we propose \algname, a novel scaffold-based GPT with three-stage optimization process for drug optimization. This optimization process is designed to enhance existing drugs to rapidly evolving virus variants and cancer cells, overcoming the limitation of earlier drug optimization efforts.
The contributions of \algname are summarized as follows:

$\bullet$ A framework for drug optimization using \fix{GPT} that involves a three-step optimization process. \fix{The underlying motivation is that each stage is complementary to each other, enhances the performance of the preceding one. Furthermore, we have conducted ablation studies to illustrate the performance gains achieved at each step.}

$\bullet$ A scaffolds-based \fix{GPT}  with a novel two-phase incremental training specifically designed for drug optimization. The motivation for incremental training is to conduct local optimization before embarking on global optimization by breaking down the whole training corpus into knowledge pieces in an incremental order. We also conducted an ablation study in table 3 to demonstrate the effectiveness of the incremental training.



$\bullet$ A novel token-level decoding optimization strategy, $\TOPN$, utilizes pretrained \fix{GPT} to enable controlled, reward-guided generation that aligns with targeted objectives.

$\bullet$ Through extensive experiments and ablation studies on real-world viral and cancer-related benchmarks, we demonstrate that \algname outperforms the competing baselines
and reliably improves upon existing molecules/drugs in terms of desired targeted objectives while preserving original scaffold, resulting in superior drug candidates.

















