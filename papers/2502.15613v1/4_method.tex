

\section{Methodology}
\subsection{Problem Statement}

We formalize the challenge of gripper-agnostic manipulation via diffusion policies through three core components:

\begin{itemize}
    \item \textbf{Gripper Configuration}: Let $\mathbb{G} \subset \mathbb{R}^{d_g}$ denote the space of two-finger gripper parameters encoding morphology, maximum width $w^{\max}$, and tool-center-point (TCP) $z^{\text{}}$, as shown in Fig. \ref{fig:grippers}. 
    \item \textbf{Observation Domain}: $\mathcal{O} = \mathcal{S}_{\text{sce}} \times \mathcal{S}_{\text{rob}}$ where $\mathcal{S}_{\text{sce}}$ represents scene observations (3D point clouds) and $\mathcal{S}_{\text{rob}} = SE(3) \times [0,g]$ the robot state (end-effector pose $\mathbf{x}_{\text{ee}} \in SE(3)$, gripper width $g$)\footnote{This paper forces on two-finger grippers problems in SE(3) space. SE(6)-space and three-finger grippers implementation is future work.}. $\mathbf{x}_{\text{ee}}$ is gripper-agnostic and reading from the robot. $g$ is specific for grippers.
    \item \textbf{Action Space}: $\mathcal{A} \subset \mathbb{R}^{d_a}$ containing end-effector displacements $\Delta\mathbf{x}_{\text{ee}}$ and gripper commands. 
\end{itemize}

The policy $\pi_\theta$ is trained on demonstrations $\mathcal{D} = \{\tau^{(i)}\}_{i=1}^N$ collected with a reference gripper $\mathbb{G}_0 \in \mathbb{G}$. Each trajectory $\tau = \{(\mathbf{o}_t, \mathbf{a}_t)\}_{t=0}^T$ satisfies: $\mathbf{o}_t = (\mathcal{S}^0_{\text{sce}}, \mathbf{x}_{\text{ee}}^0, g_t^0)$ and $\mathbf{a}_t \sim \pi_{\text{expert}}(\cdot|\mathbf{o}_t)$,
where superscript $0$ indicates $\mathbb{G}_0$ parameters.

During deployment with novel gripper $\mathbb{G}_i \neq \mathbb{G}_0$, the observation-action distribution shifts due to (1) visual/kinematic differences $\mathcal{O}_i \neq \mathcal{O}_0$, and (2) policy mismatch $p_{\theta}(\mathcal{A}|\mathcal{O}_i) \neq p_{\theta}(\mathcal{A}|\mathcal{O}_0)$. This manifests as trajectory divergence:
\begin{equation}
\|\tau_{1:T}^{\mathbb{G}_i} - \tau_{1:T}^{\mathbb{G}_0}\|_{\mathcal{W}} > \delta_{\text{tol}}\text{,}
\end{equation}
where $\mathcal{W}$ is the task-specific metric space and $\delta_{\text{tol}}$ the success threshold, e.g., objects cannot be grasped with shorter grippers, and collisions can result from using longer grippers.

\def\gfigurew{1.95cm}
\def\gfigureh{1.3cm}
\def\graphw{8.5cm}
\def\gfigurehnew{2cm}
\def\gfigurewnew{2cm}
\def\gfigurewnewg{1.7cm}
\def\xshift{-0.05cm}
\tikzstyle{rec-mid} = [minimum height=6cm, minimum width=\graphw, rounded corners=0.4cm, draw=black, dashed]
\tikzstyle{rec-split} = [rectangle split, 
        rectangle split horizontal,
        rectangle split parts=3,
        rectangle split every empty part={},
        rectangle split empty part width=0.3*\graphw,
        text width=0.3*\graphw,text centered,
        draw=none,]
        
\begin{figure}[!t]
  \centering
  \captionsetup{singlelinecheck=off}
    \begin{tikzpicture}
        \node(border)[minimum height=4.9cm, minimum width=\graphw, draw=none, dashed] at (0,0){};
 
        \node(gripper1)[rec-mid,rec-split,minimum height=2.43cm, rounded corners=0.1cm,anchor=south, draw=none] at (border.south){g4 \nodepart{two} g5 \nodepart{three} g6};
        
        \node(gripper2)[rec-mid,rec-split,minimum height=2.43cm, rounded corners=0.1cm, draw=none, anchor=north] at (border.north){g1 \nodepart{two} g2 \nodepart{three} g3};
        
        
        \draw[rounded corners=0.1cm, black, dashed,line width=0.8pt] (gripper1.south west) rectangle (gripper2.north east);
        \draw[black, dashed,line width=0.8pt] (gripper2.south west) -- (gripper2.south east);
        \draw[dashed,draw=black,line width=0.8pt] (gripper2.two split north) -- (gripper1.two split south) 
        (gripper2.one split north) -- (gripper1.one split south);
        
        \node[inner sep=1pt,anchor=south](img-flexible)at($(gripper2.one south)-(0,0)$) {\includegraphics[width=\gfigurewnew]{figure/rm-flexible-adjust-rm.jpg}};
        \draw[stealth-stealth, draw=red,line width=0.8pt] ($(img-flexible.south)+(0,0.1)$) -- (img-flexible.north)node[above](flexible-arrow){};
        \draw[stealth-stealth, draw=blue,line width=0.8pt] ($(flexible-arrow.north)+(-0.4,-0.5)$) -- ($(flexible-arrow.north)+(0.4,-0.5)$)node(flexiblew)[right=0.1cm,blue]{\footnotesize width};
        \node[blue] at (flexiblew.south){\footnotesize 8};
        \node(flexibleh)[anchor=center,text centered,align=center, red] at ($(flexible-arrow.center)+(0,-0.05)$){\footnotesize height 20};

        
        \node[inner sep=1pt,anchor=south](img-long)at($(gripper2.two south)+(0,0)$){\includegraphics[width=\gfigurewnew]{figure/rm-long.jpg}};
        \draw[stealth-stealth, draw=red,line width=0.8pt] ($(img-long.south)+(0,0.1)$) -- (img-long.north)node[above](long-arrow){};
        \draw[stealth-stealth, draw=blue,line width=0.8pt] ($(long-arrow.north)+(-0.5,-0.5)$) -- ($(long-arrow.north)+(0.5,-0.5)$)node(longw)[right,blue]{\footnotesize width};
        \node(longh)[anchor=west,text centered,align=center, red] at ($(img-long.south)+(0,0.8)$){\footnotesize height};
        \node[blue] at (longw.south){\footnotesize 11.5};
        \node[red] at (longh.south){\footnotesize 23.5};
        
        \node[inner sep=1pt,anchor=south](img-short1)at($(gripper2.three south)+(0,0)$){\includegraphics[width=\gfigurewnew]{figure/rm-short1.jpg}};
        \draw[stealth-stealth, draw=red,line width=0.8pt] ($(img-short1.south)+(0,0.1)$) -- (img-short1.north)node[above](short1-arrow){};
        \draw[stealth-stealth, draw=blue,line width=0.8pt] ($(short1-arrow.north)+(-0.4,-0.5)$) -- ($(short1-arrow.north)+(0.4,-0.5)$)node(short1w)[right=0.1cm,blue]{\footnotesize width};
        \node[blue] at (short1w.south){\footnotesize 8.5};
        \node(short1h)[anchor=center,text centered,align=center, red] at ($(short1-arrow.center)+(0,0.0)$){\footnotesize height 17.5};

        
        \node[inner sep=1pt,anchor=south](img-short2)at($(gripper1.one south)-(0,0)$){\includegraphics[width=\gfigurewnew]{figure/rm-short2.jpg}};
        \draw[stealth-stealth, draw=red,line width=0.8pt] ($(img-short2.south)+(0,0.1)$) -- (img-short2.north) node[above](short2-arrow){};
        \draw[stealth-stealth, draw=blue,line width=0.8pt] ($(short2-arrow.north)+(-0.5,-0.5)$) -- ($(short2-arrow.north)+(0.5,-0.5)$)node(short2w)[right,blue]{\footnotesize width};
        \node[blue] at (short2w.south){\footnotesize 10};
        \node(short2h)[anchor=center,text centered,align=center, red] at ($(short2-arrow.center)+(0,0.0)$){\footnotesize height 18};

        
        \node[inner sep=1pt,anchor=south](img-short3)at($(gripper1.two south)+(0,0)$){\includegraphics[width=\gfigurewnew]{figure/rm-short3.jpg}};
        \draw[stealth-stealth, draw=red,line width=0.8pt] ($(img-short3.south)+(0,0.1)$) -- (img-short3.north)node[above](short3-arrow){};
        \draw[stealth-stealth, draw=blue,line width=0.8pt] ($(short3-arrow.north)+(-0.35,-0.5)$) -- ($(short3-arrow.north)+(0.35,-0.5)$)node(short3w)[right,blue]{\footnotesize width};
        \node[blue] at (short3w.south){\footnotesize 7.5};
        \node(short3h)[anchor=center,text centered,align=center, red] at ($(short3-arrow.center)+(0,0.0)$){\footnotesize height 16};

        
        \node[inner sep=1pt,anchor=south](img-robotiq)at($(gripper1.three south)+(0,0)$){\includegraphics[width=\gfigurewnewg]{figure/rm-robotiq.jpg}};
        \draw[stealth-stealth, draw=red,line width=0.8pt] ($(img-robotiq.south)+(0,0)$) -- (img-robotiq.north)node[above](robotiq-arrow){};
        \draw[stealth-stealth, draw=blue,line width=0.8pt] ($(robotiq-arrow.north)+(-0.4,-0.5)$) -- ($(robotiq-arrow.north)+(0.4,-0.5)$) node(robotiqw)[right=0.05cm,blue]{\footnotesize width};
        \node[blue] at (robotiqw.south){\footnotesize 8.5};
        \node(robotiqh2)[anchor=center,text centered,align=center, red] at ($(robotiq-arrow.center)+(0,0.0)$){\footnotesize height 16};
        

        
        %%%%%%%%%%%%%%%%%%%%%%%%%%%%% gripper note
        \node(t1)[color=red,anchor=north west] at (gripper2.north west){$\mathbb{G}_0$};
        \draw[line width=1pt,draw=none,fill=cyan!30!white] (t1.center) circle (7pt);
        \node[color=black] at (t1.center){$\mathbb{G}_0$};
        
        \node(t2)[color=red,anchor=west] at (gripper2.one split|-t1.west){$\mathbb{G}_1$};
        \draw[line width=1pt,draw=none,fill=cyan!30!white] (t2.center) circle (7pt);
        \node[color=black] at (t2.center){$\mathbb{G}_1$};
        
        \node(t3)[color=red,anchor=west] at (gripper2.two split|-t2.west){$\mathbb{G}_2$};
        \draw[line width=1pt,draw=none,fill=cyan!30!white] (t3.center) circle (7pt);
        \node[color=black] at (t3.center){$\mathbb{G}_2$};
        
        \node(t4)[color=red,anchor=north west] at (gripper1.north west){$\mathbb{G}_3$};
        \draw[line width=1pt,draw=none,fill=cyan!30!white] (t4.center) circle (7pt);
        \node[color=black] at (t4.center){$\mathbb{G}_3$};
        
        \node(t5)[color=red,anchor=west] at (gripper1.one split|-t4.west){$\mathbb{G}_4$};
        \draw[line width=1pt,draw=none,fill=cyan!30!white] (t5.center) circle (7pt);
        \node[color=black] at (t5.center){$\mathbb{G}_4$};
        
        \node(t6)[color=red,anchor=west] at (gripper1.two split|-t5.west){$\mathbb{G}_5$};
        \draw[line width=1pt,draw=none,fill=cyan!30!white] (t6.center) circle (7pt);
        \node[color=black] at (t6.center){$\mathbb{G}_5$};

        
    \end{tikzpicture}
    \caption{Gripper variants of different morphologies. (Unit: cm)}
    \label{fig:grippers}
    \vspace{-1.2em}
\end{figure} 


To mitigate these issues, 
we develop policy $\pi_\theta^*$ that maintains task performance under gripper variation, combining (1) {gripper-invariant knowledge learning} and (2) {morphology-aware trajectory optimization} to achieve $\pi_\theta^*$ without policy retraining. The overview framework is shown in Fig. \ref{fig:framework}.

\def\policywidth{5cm}
\def\shorthorizon{0.42*\policywidth}
\def\heightgap{0.6cm}
\def\figurew{1.95cm}
\def\figureh{1.65cm}
\def\gfigurew{1.95cm}
\def\gfigureh{1.3cm}
\def\figureww{1.5cm}

\tikzstyle{background-color}=[fill=yellow!5!white]
\tikzstyle{rec-mid} = [minimum height=6cm, minimum width=6cm, rounded corners=0.4cm, draw=black, dashed]
\tikzstyle{rec-split} = [rectangle split, 
        rectangle split horizontal,
        rectangle split parts=3,
        rectangle split every empty part={},
        rectangle split empty part width=1.75cm,
        text width=1.75cm,text centered,
        draw=none,]
        
\tikzstyle{token} = [rectangle, rounded corners=2pt, text centered, draw=black,line width=1pt, minimum height=0.6cm, minimum width=0.88cm]

\tikzstyle{seq-split} = [rectangle split, 
        rectangle split horizontal,
        rectangle split parts=3,
        rectangle split every empty part={},
        rectangle split empty part width=0.3*\shorthorizon,
        text width=0.3*\shorthorizon,text centered,
        draw=none,]

\tikzstyle{long-sequence} = [minimum width=\policywidth,minimum height=0.7cm,draw=none]
\tikzstyle{short-sequence} = [minimum width=\shorthorizon,minimum height=0.65cm,draw=none]

\tikzstyle{short-sequence-two} = [minimum width=0.67*\shorthorizon,minimum height=0.65cm,draw=none]
\tikzstyle{seq-split-two} = [rectangle split, 
        rectangle split horizontal,
        rectangle split parts=2,
        rectangle split every empty part={},
        rectangle split empty part width=0.3*\shorthorizon,
        text width=0.3*\shorthorizon,text centered,
        draw=none,]

\begin{figure*}[t]
  \centering
  \captionsetup{singlelinecheck=off}
    \begin{tikzpicture}
        \node(border)[minimum height=6cm, minimum width=17.6cm, draw=none, dashed] at (0,0){};
        
        %%%%%%%%%%%%%%%%%%%%%%%%%%%%%%%%%%%%%%%%%%%%%%%%%%%%%%%%
        \node(obs)[rec-mid, draw=none, fill=cyan!5, anchor=north west] at (border.north west){};
        \node(state)[rec-mid,rec-split,minimum height=1.8cm,rounded corners=0.2cm, anchor=north west,fill=cyan!5, 
        ] at (obs.north west){ robot pose\\$\mathcal{S}_{\text{rob}}$ \nodepart{two} pointcloud\\$\mathcal{S}_{\text{sce}}$ \nodepart{three} kinect \\$\mathcal{G}^*_{\text{prob}}$};
        \draw[densely dashed,draw=gray,line width=1pt] (state.two split north) -- (state.two split south)
        (state.one split north) -- (state.one split south);
        
        %%%%%%%%%%% figure
        \node[inner sep=0pt,anchor=center](pc) at (state.center){\includegraphics[width=\figurew,height=\figureh]{figure/pc_rs-small.jpg}};
        \node[inner sep=0pt,anchor=east](pose)at($(pc.west)+(-0.05,0)$){\includegraphics[width=\figurew,height=\figureh]{figure/rgb_rs-small.png}};
        \node[inner sep=0pt,anchor=west](gmap)at($(pc.east)+(0.05,0)$){\includegraphics[width=\figurew,height=\figureh]{figure/certainty_map.png}};
        \node[white,anchor=north west] at (pose.north west){$\mathcal{S}^{ }_{\text{rob}}$};
        \node[black,anchor=north west] at (pc.north west){$\mathcal{S}^{ }_{\text{sce}}$};
        \node[white,anchor=north west] at (gmap.north west){$\mathcal{G}^*_{\text{prob}}$};
        
        %%%%%%%%%% gripper
        \node(gripper1)[rec-mid,rec-split,minimum height=1.3cm, rounded corners=0.1cm,anchor=south west, draw=none,rectangle split part fill={white!10,white!10,white!10}] at (obs.south west){gripper4 \nodepart{two} gripper5 \nodepart{three} gripper6};
        
        
        \node(gripper2)[rec-mid,rec-split,minimum height=1.3cm, rounded corners=0.1cm, draw=none, anchor=south,rectangle split part fill={white!10,white!10,white!10},] at (gripper1.north){gripper1 \nodepart{two} gripper2 \nodepart{three} gripper3};
        
        
        %%%%%%%%%%%%% gripper images
        \node[inner sep=1pt,anchor=east](img-flexible)at($(gripper2.one split)+(-0.2,0)$){\includegraphics[height=\gfigureh]{figure/rm-flexible1.jpg}};
        
        \node[inner sep=1pt,anchor=east](img-long)at($(gripper2.two split)+(-0.2,0)$){\includegraphics[height=\gfigureh]{figure/rm-long1.jpg}};
        
        \node[inner sep=1pt,anchor=east](img-short1)at($(gripper2.two east)+(-0.2,0)$){\includegraphics[height=\gfigureh]{figure/rm-short1-1.jpg}};
        
        \node[inner sep=1pt,anchor=east](img-short2)at($(gripper1.one split)+(-0.2,0)$){\includegraphics[height=\gfigureh]{figure/rm-short2-1.jpg}};
        
        \node[inner sep=1pt,anchor=east](img-short3)at($(gripper1.two split)+(-0.2,0)$){\includegraphics[height=\gfigureh]{figure/rm-short3-1.jpg}};
        
        \node[inner sep=1pt,anchor=east](img-robotiq)at($(gripper1.two east)+(-0.2,0)$){\includegraphics[height=\gfigureh]{figure/rm-robotiq1.jpg}};
        
        \draw[dashed,draw=black,line width=1pt] (gripper2.two split north) -- (gripper2.two split south)
        (gripper2.one split north) -- (gripper2.one split south);
        \draw[dashed,draw=black,line width=1pt] (gripper2.south west) -- (gripper2.south east);
        \draw[dashed,draw=black,line width=1pt] (gripper1.two split north) -- (gripper1.two split south) 
        (gripper1.one split north) -- (gripper1.one split south);
        \draw[dashed,draw=black,line width=1pt] (gripper2.north west) rectangle (gripper1.south east);
        %%%%%%%%%%%%%%%%%%%%%%%%%%%%%%%%%%%%%%%%%%%%%%%
        \node(policy)[rec-mid, draw=none, minimum width=\policywidth, background-color, anchor=west] at ($(obs.east)+(0.5,0)$){};
        
        
        %%%% observation sequence
        \node(obs-seq)[long-sequence, anchor=north] at ($(policy.north)-(0,0.4)$){};
        \draw[draw=black,-stealth,line width=1pt] ($(obs-seq.north west)+(0.1,0)$) -- ($(obs-seq.north east)-(0.1,0)$) 
        ($(obs-seq.south west)+(0.1,0)$) -- ($(obs-seq.south east)-(0.1,0)$);
        \node(obs-seq-short)[short-sequence-two, anchor=west,fill=color-obs-seq,seq-split-two] at ($(obs-seq.west)+(0.1,0)$){\normalsize $o_{t-1}$ \nodepart{two} $o_{t}$};
        \draw[densely dashed,draw=gray]  (obs-seq-short.one split north) -- (obs-seq-short.one split south);
        
        \node(obs-t)[long-sequence, minimum height=0.65cm, anchor=north east, rounded corners=0.2cm, draw=gray,line width=1pt, minimum width=0.75*\policywidth,text centered] at ($(state.south east)-(0,0.5)$){(\textcolor{red}{$\mathcal{S}'_{\text{rob}}$},$\mathcal{S}^{ }_{\text{sce}}$, $\mathcal{G}^*_{\text{prob}})\rightarrow o_t$};
        
        
        \node(gmap)[black] at (gripper2.mid|-obs-t.west) {$\mathbb{M}$};
        \draw[line width=1pt,draw=color-gmap] (gmap.center) circle (7pt);
        \draw[line width=0.8pt,-stealth,draw=black] (state.one south) -- (gmap.north);
        \draw[line width=0.8pt,-stealth,draw=color-gmap] (gripper2.one north) -- (gmap.south) node[below=0.3cm,color=color-gmap,right]{\footnotesize Gripper mapping};
        \draw[line width=0.8pt,-stealth,draw=black] (gmap.east) -- (obs-t.west) node[midway,above,red]{$\mathcal{S}'_{\text{rob}}$};
        \draw[line width=0.8pt,-stealth,draw=black] (state.two south) -- (state.two south|- obs-t.north);
        \draw[line width=0.8pt,-stealth,draw=black] (state.three south) -- (state.three south|- obs-t.north);
        \draw[line width=0.8pt,-stealth,draw=black] (obs-t.east) -- ++ (0.3,0) -- ++ (0,2.55) -- ++(1.64,0) -- (obs-seq-short.two north)node(obs-arrow){};
        \node[black,anchor=west] at ($(obs-arrow.east|-obs-seq.north)+(0,0.15)$){observation sequence};
        
        
        %%%% trained policy
        \node(policy-seq)[long-sequence, anchor=north, rounded corners=0.2cm, draw=gray,line width=1pt, minimum width=0.9*\policywidth,text centered] at ($(obs-seq.south)-(0,\heightgap)$){Trained policy $\epsilon_{\theta}(\mathcal{O},\mathcal{A},k)$};
        
        \draw [f/.tip = Fast Triangle, line width=1ex, {Triangle Cap[reversed] }-{Triangle Cap[] . ff}, YellowGreen!80] ($(obs-seq-short.south east)+(-0.5,0)$) -- ++(0,-\heightgap) node(arrow1){}; %($(obs-seq-short.south east)+(-0.5,0)$ |-policy-seq.north);
        
       
        %%%% action sequence original
        \node(act-seq)[long-sequence, anchor=north] at ($(policy-seq.south)-(0,\heightgap)$){};
        \draw[draw=black,-stealth,line width=1pt] ($(act-seq.north west)+(0.1,0)$) -- ($(act-seq.north east)-(0.1,0)$) 
        ($(act-seq.south west)+(0.1,0)$) -- ($(act-seq.south east)-(0.1,0)$);
        \node(act-seq-short)[short-sequence, anchor=west,fill=cyan!30!white,seq-split] at ($(obs-seq-short.one split |- act-seq.west)$){\normalsize $a_{t}$ \nodepart{two} $a_{t+1}$ \nodepart{three} $a_{t+2}$};
        \draw[densely dashed,draw=gray] (act-seq-short.two split north) -- (act-seq-short.two split south) 
        (act-seq-short.one split north) -- (act-seq-short.one split south);
        
        \draw [f/.tip = Fast Triangle, line width=1ex, {Triangle Cap[reversed] }-{Triangle Cap[] . ff}, cyan!50!white] (arrow1.south|-policy-seq.south) -- ++(0,-\heightgap) node[midway,right] {trajectory inference};
        
        %%%%%%%%%%% obs sequence 2
        \node(obs-seq-short2)[short-sequence-two, anchor=west,fill=color-obs-seq,seq-split-two] at ($(act-seq-short.two split|-obs-seq-short.west)$){\normalsize $o_{t+2}$ \nodepart{two} $o_{t+3}$};
        \draw[densely dashed,draw=gray]  (obs-seq-short2.one split north) -- (obs-seq-short2.one split south);
        
        
        %%%% action sequence optimization
        \node(act-seq2)[long-sequence, anchor=south] at ($(policy.south)+(0,0.3)$){};
        \draw[draw=black,-stealth,line width=1pt] ($(act-seq2.north west)+(0.1,0)$) -- ($(act-seq2.north east)-(0.1,0)$) 
        ($(act-seq2.south west)+(0.1,0)$) -- ($(act-seq2.south east)-(0.1,0)$);
        \node(act-seq2-short)[short-sequence, anchor=west,fill=cyan!30!white,seq-split] at ($(act-seq-short.west|-act-seq2.west)$){\normalsize \textcolor{red}{$a^*_{t}$} \nodepart{two} \textcolor{red}{$a^*_{t+1}$} \nodepart{three} \textcolor{red}{$a^*_{t+2}$}};
        \draw[densely dashed,draw=red] (act-seq2-short.two split north) -- (act-seq2-short.two split south) 
        (act-seq2-short.one split north) -- (act-seq2-short.one split south);
        
        %%%%%%%%%%%%%%%%%%%%%%%%%%%%%%%%%%%%%%%%%%%%%%%
        \node(projection)[rec-mid, draw=none, minimum width=5.5cm, background-color, anchor=east] at (border.east){};
        \node[anchor=north,red] at (projection.north) {Online optimization};
        
        \node(a1)[token,draw=Blue!90,fill=cyan!30!white, anchor=west] at (obs-seq.east-|projection.west){$a_t$};
        
        \node(a21)[token,draw=Blue!90,fill=cyan!30!white, anchor=north] at ($(a1.south)+(0,-0.4)$) {$a_{t}$};
        \node(a22)[token,draw=Blue!90,fill=cyan!30!white, anchor=west] at ($(a21.east)+(0.4,0)$) {$a_{t+1}$};
        \draw[black,line width=1pt] (a21.east)--(a22.west) node[midway,above]{+};
        
        \node(a31)[token,draw=Blue!90,fill=cyan!30!white, anchor=north] at ($(a21.south)+(0,-0.4)$) {$a_{t}$};
        \node(a32)[token,draw=Blue!90,fill=cyan!30!white, anchor=west] at ($(a31.east)+(0.4,0)$) {$a_{t+1}$};
        \node(a33)[token,draw=Blue!90,fill=cyan!30!white, anchor=west] at ($(a32.east)+(0.4,0)$) {$a_{t+2}$};
        \draw[black,line width=1pt] (a31.east)--(a32.west) node[midway,above]{+};
        \draw[black,line width=1pt] (a32.east)--(a33.west) node[midway,above]{+};
        
        %%% original action line,  cyan!50!white
        \draw[line width=0.8pt,-stealth,draw=Blue!90](act-seq-short.east)--++(1.6,0)--++(0,2.65)--(a1.west);
        \draw[line width=0.8pt,stealth-,draw=Blue!90](a21.west)--++(-0.36,0);
        \draw[line width=0.8pt,stealth-,draw=Blue!90](a31.west)--++(-0.36,0);
        
        \node(v3)[token,draw=BrickRed!70,fill=color-opti, anchor=west] at ($(a33.east)+(0.85,0)$) {$\nu^*_{t+2}$};
        \node(v2)[token,draw=BrickRed!70,fill=color-opti, anchor=south] at ($(v3.north)+(0,0.4)$) {$\nu^*_{t+1}$};
        \node(v1)[token,draw=BrickRed!70,fill=color-opti, anchor=south] at ($(v2.north)+(0,0.4)$) {$\nu^*_{t}$};
        
        \draw[line width=1pt,blue,dashed, arrows = {-Stealth[color=red]}] (a1.east) -- (v1.west) node[midway,below]{$\mathcal{S}'_{\text{rob}}+a_t+\nu \geq \epsilon_{\text{safe}}$};
        \draw[line width=1pt,blue,dashed, arrows = {-Stealth[color=red]}] (a22.east) -- (v2.west);
        \draw[line width=1pt,blue,dashed, arrows = {-Stealth[color=red]}] (a33.east) -- (v3.west);
        
        \draw[line width=0.8pt,draw=gray,dashed] ($(v3.south west)+(-0.05,-0.05)$) rectangle ($(v1.north east)+(0.05,0.05)$) node(rec1){};
        
        \node(img-frame)[rec-mid,rounded corners=0.2cm,minimum height=2.5cm,minimum width=5.5cm,anchor=south, text centered,line width=1pt,fill=white] at (projection.south){};
        
        \node(rob-image2)[inner sep=1pt,anchor=north] at ($(img-frame.north)+(0,-0.05)$){\includegraphics[height=\gfigureh]{figure/rgb_robot2.png}};
        
        \node(rob-image2-pc)[inner sep=1pt,anchor=north] at ($(rob-image2.south)$){\includegraphics[width=\figureww]{figure/pc_robot2.jpg}};
        
        \node(rob-image1)[inner sep=1pt,anchor=east] at ($(rob-image2.west)+(-0.3,0)$){\includegraphics[height=\gfigureh]{figure/rgb_robot1.png}};
        \node(rob-image1-pc)[inner sep=1pt,anchor=center] at ($(rob-image1.south|-rob-image2-pc.west)$){\includegraphics[width=\figureww]{figure/pc_robot1.jpg}};
        
        \node(rob-image3)[inner sep=1pt,anchor=west] at ($(rob-image2.east)+(0.3,0)$){\includegraphics[height=\gfigureh]{figure/rgb_robot3.png}};
        \node(rob-image3-pc)[inner sep=1pt,anchor=center] at ($(rob-image3.south|-rob-image2-pc.east)$){\includegraphics[width=\figureww]{figure/pc_robot3.jpg}};
        
        \node[red,anchor=north] at ($(rob-image1-pc.south)+(0,0.14)$) {$a^*_{t}$};
        \node[red,anchor=north] at ($(rob-image2-pc.south)+(0,0.14)$) {$a^*_{t+1}$};
        \node[red,anchor=north] at ($(rob-image3-pc.south)+(0,0.14)$) {$a^*_{t+2}$};
        
        \draw[draw=red,-stealth,line width=1pt] (act-seq2-short.east) -- (img-frame.west|-act-seq2-short.east) node[midway,above=-0.05cm]{\footnotesize execute};
        %%%%%%%%%%%%%%%%%%%%%%%%
        \node(bigplus)[black] at ($(act-seq-short.south)+(0,-0.65)$) {$\bigoplus$};
        \draw[cyan!50!white,-stealth,line width=1pt] (act-seq-short.south) -- ($(bigplus.north)+(0,-0.1)$);
        \draw[BrickRed!70,-stealth,line width=1pt] ($(v3.south)+(0,-0.05)$)--++(0,-0.15)-- ++(-5.05,0) -- ++(0,-1.05) -- ($(bigplus.east)+(-0.1,0)$);
        \draw[red,-stealth,line width=1pt] ($(bigplus.south)+(0,0.1)$) -- (act-seq2-short.north);
        \node[black, BrickRed!70, anchor=west]at($(bigplus.north west)+(0.5,0)$){Safety projection};
        
        
        %%%%%%%%%%%% subfigure caption
        \node(atext)[anchor=north,black] at (obs.south) {(a) Multi-model observation};
        \node(btext)[black,anchor=center] at (policy.south east|-atext.east) {(b) Trajectory inference with safety projection};
        
    \end{tikzpicture}
    \caption{Overview of policy. (a) The multi-modal observation consists of robot pose $\mathcal{S}'_{\text{rob}}$, scene point clouds $\mathcal{S}_{\text{sce}}$, and grasping probability map $\mathcal{G}^*_{\text{prob}}$. Gripper morphological variations are encoded into $\mathcal{S}'_{\text{rob}}$ via gripper mapping. (b) Safety-Constrained trajectory projection via online optimization process, enforcing the executive trajectory to satisfy task and safety constraints.}
    \label{fig:framework}
    %\vspace{-1.4em}
\end{figure*} 
\def\figurew{2cm}
\def\figureh{1.7cm}
\def\heightgap{0.3cm}
\def\heightgapp{0.5cm}
\def\widthgap{0.05cm}
\begin{figure}[htbp]
  \centering
  \captionsetup{singlelinecheck=off}
    \begin{tikzpicture}
        \node(border)[minimum height=4.9cm, minimum width=8.5cm, draw=none, dashed] at (0,0){};
        \node(box1)[minimum height=4.9cm, minimum width=4.25cm, draw=none, dashed,anchor=west] at (border.west){};
        \node(box2)[minimum height=4.9cm, minimum width=4.25cm, draw=none, dashed,anchor=east] at (border.east){};
        
        %%%% realsense
        \node[inner sep=0pt,anchor=north west](rs-ori) at ($(box1.north west)+(\widthgap,-\heightgap)$){\includegraphics[width=\figurew,height=\figureh]{figure/gmap-rgb0.169.png}};
        \node[inner sep=0pt,anchor=north east](rs-depth) at ($(box1.north east)+(-\widthgap,-\heightgap)$){\includegraphics[width=\figurew,height=\figureh]{figure/gmap-depth0.169.png}};
        \node[inner sep=0pt,anchor=north west](rs-gmap1) at ($(box2.north west)+(\widthgap,-\heightgap)$){\includegraphics[width=\figurew,height=\figureh]{figure/gmap-certainty_map0.169.png}};
        \node[inner sep=0pt,anchor=north east](rs-gmap2) at ($(box2.north east)+(-\widthgap,-\heightgap)$){\includegraphics[width=\figurew,height=\figureh]{figure/gmap-certainty_map_star0.169.png}};
        
        \node(ori)[black,anchor=south]at($(rs-ori.north)+(0,-0.1)$){Original img};
        \node[black]at(ori.west-|rs-depth.north){Depth img};
        \node[black]at(ori.west-|rs-gmap1.north){$\mathcal{G}_{\text{prob}}$};
        \node[black]at(ori.west-|rs-gmap2.north){$\mathcal{G}^*_{\text{prob}}$};
        \node[black,anchor=north]at(box2.west|-rs-gmap1.south){Case1 - \textcolor{red}{gripper:} {flexible}, \textcolor{cyan}{height:} {0.18m}, \textcolor{Salmon}{object:} {block}};
        
        %%%% robotiq
        \node[inner sep=0pt,anchor=south west](iq-ori) at ($(box1.south west)+(\widthgap,\heightgapp)$){\includegraphics[width=\figurew,height=\figureh]{figure/gmap-robotiqrgblow.png}};
        \node[inner sep=0pt,anchor=south east](iq-depth) at ($(box1.south east)+(-\widthgap,\heightgapp)$){\includegraphics[width=\figurew,height=\figureh]{figure/gmap-robotiqdepthlow.png}};
        \node[inner sep=0pt,anchor=south west](iq-gmap1) at ($(box2.south west)+(\widthgap,\heightgapp)$){\includegraphics[width=\figurew,height=\figureh]{figure/gmap-robotiqcertainty_maplow.png}};
        \node[inner sep=0pt,anchor=south east](iq-gmap2) at ($(box2.south east)+(-\widthgap,\heightgapp)$){\includegraphics[width=\figurew,height=\figureh]{figure/gmap-robotiqcertainty_map_starlow.png}};
        \node[black,anchor=north]at(box2.west|-iq-gmap1.south){Case2 - \textcolor{red}{gripper:} {Robotiq-2f}, \textcolor{cyan}{height:} {0.08m}, \textcolor{Salmon}{object:} {block}};
        
        %%%% robotiq banana
        \node[inner sep=0pt,anchor=north](iqb-ori) at ($(iq-ori.south)+(0,-\heightgapp-1)$){\includegraphics[width=\figurew,height=\figureh]{figure/gmap-rgblow_banana.png}};
        \node[inner sep=0pt,anchor=center](iqb-depth) at (iq-depth.south|-iqb-ori.west){\includegraphics[width=\figurew,height=\figureh]{figure/gmap-depthlow_banana.png}};
        \node[inner sep=0pt,anchor=center](iqb-gmap1) at (iq-gmap1.south|-iqb-depth.west){\includegraphics[width=\figurew,height=\figureh]{figure/gmap-certainty_maplow_banana.png}};
        \node[inner sep=0pt,anchor=center](iqb-gmap2) at (iq-gmap2.south|-iqb-gmap1.west){\includegraphics[width=\figurew,height=\figureh]{figure/gmap-certainty_map_starlow_banana.png}};
        \node[black,anchor=north]at(iqb-gmap1.south west){Case3 - \textcolor{red}{gripper:} {Robotiq-2f}, \textcolor{cyan}{height:} {0.14m}, \textcolor{Salmon}{object:} {banana}};
  
        
    \end{tikzpicture}
    \caption{The gripper-agnostic grasping knowledge. Dynamic changes in robot pose and gripper variants cause changes in visual observations, including RGB-D images and object grasp probabilities, $\mathcal{G}^*_{\text{prob}}$ provides stable visual information.}
    \label{fig:gmap}
    \vspace{-1.7em}
\end{figure} 

\subsection{Learning Gripper-agnostic Grasping Knowledge}
Visuomotor policies, like Diffusion policy and 3D Diffusion Policy~\cite{chi2023diffusion,Ze2024DP3}, depend on visual observations to generate robot trajectories. However, swapping out different grippers during the online execution can alter visual observations (both RGB and point cloud inputs), as shown in Fig.\ref{fig:grippers}. Such alterations lead to out-of-distribution trajectory generations, reducing the task success rate\cite{liu2023towards}. To mitigate this limitation, we introduce a gripper-invariant \textit{grasping probability map} $\mathcal{G}_{\text{prob}}$ as an additional observation component, which captures object-centric grasp affordances  that are independent of end-effector geometry, thereby guiding the policy to focus on relevant object features rather than gripper-specific visual patterns. By decoupling object-related cues from the gripper's appearance, $\mathcal{G}_{\text{prob}}$ enhances the policy's robustness to variations in gripper morphology, maintaining stable task performance across different grippers. 


We adopt the Generative Grasping CNN (GG-CNN) \cite{morrison2020learning} for $\mathcal{G}_{\text{prob}}$ synthesis from depth images. GG-CNN is pre-trained on the Cornell Grasping Dataset~\cite{lenz2015deep}, which contains 885 RGB-D images with annotated parallel-jaw grasps across 240 objects. However, real-world pick-and-place manipulations introduce two key challenges: (1) the hand-eye camera moving with the robot, causing scale variations in object pixels, and (2) lighting changes disturb depth sensor readings. These factors degrade GG-CNN's output stability, i.e., $\mathcal{G}_{\text{prob}}$, and destabilize policy training and inference performance. To address this issue, our solution involves: (1) threshold filtering: discard pixels with $\mathcal{G}_{\text{prob}} < 0.7$, (2) centroid computation: $\mathds{O} = \frac{1}{N}\sum_{i=1}^N (u_i,v_i)$ for remaining pixels, and (3) region masking: generate $\mathcal{G}^*_{\text{prob}}$ through circular masking ($r=30$ pixels) about $\mathds{O}$. The map $\mathcal{G}^*_{\text{prob}}$ satisfies:
\begin{equation}
    \mathcal{G}^*_{\text{prob}}(u,v) = \begin{cases}
        1 & \text{, if } \|(u,v) - \mathds{O}\|_2 \leq 30 \\
        0 & \text{, otherwise}
    \end{cases}
\end{equation}


This spatial filtering maintains grasp affordance information while eliminating outlier predictions caused by sensor noise, as shown in Fig.\ref{fig:gmap}. The policy observation consists of:
\begin{equation}
    \label{eq:obs}
    \mathcal{O}^* = \mathcal{G}^*_{\text{prob}}\times \mathcal{S}_{\text{sce}}\times \mathcal{S}_{\text{rot}},
\end{equation}

\subsection{Optimizing Trajectory Generation} 
\label{sec:optimization}
The training scheme of our policy $\pi_{\theta}$ is consistent with that of Diffusion Policy (DP)~\cite{chi2023diffusion}, i.e. DDPM, with observations $\mathcal{O}^*$ and MSE training loss. During inference, $\pi_{\theta}$ employs Denoising Diffusion Implicit Models (DDIM)\cite{song2021denoising} sampling with 10 denoising iterations. However, our policy introduces a strategy to enforce the generative trajectory to fit different grippers, comprising gripper-geometry aware mapping and safety-constrained trajectory  projection. 


\textbf{Gripper mapping:} Gripper morphological variations induce the end-effector's pose discrepancies during identical object manipulation, primarily along: (1) vertical axis ($z$): tool-center-point offset, and (2) gripper state ($g$): grasping width differences. The discrepancies causes inconsistent actions predicted by $\pi_\theta$ across grippers.

Let $\mathbb{G}_{(0)}$ denote the reference gripper used during training, and $\mathbb{G}_{(i)}$ represent a novel gripper of category $i$. We define differentiable mapping functions $\mathds{M}_h(\cdot)$ and $\mathds{M}_g(\cdot)$ that project $\mathbb{G}_{(i)}$ parameters to the $\mathbb{G}_{(0)}$ basis: 
\begin{equation}
\label{eq:gripper_mapping}
\begin{aligned}
    z'_{(i)} &= \mathds{M}_h(z_{(i)}) = z_{(i)} + \Delta h_{(i)}, \\
    g'_{(i)} &= \mathds{M}_g(g_{(i)}) = \alpha_{(i)} g_{(i)},
\end{aligned}
\end{equation}
where $z_{(i)}$ is the measured height of end-effector equipped with $\mathbb{G}_{(i)}$, $\Delta h_{(i)} = z_{(0)} - z_{(i)}$ is the height offset from reference gripper, $g_{(i)} \in [g^{\min}, g^{\max}]$ is the real-time grasping width, $\alpha_{(i)} = {g^{\max}_{(0)}}/{g^{\max}_{(i)}}$ scales widths. The mapping parameters $\{\Delta h_{(i)}, \alpha_{(i)}\}$ are obtained through offline calibration with mechanical measurement of gripper dimensions. 

This transformation preserves the policy's internal representation while adapting to physical gripper properties, enabling zero-shot generalization to novel end-effectors. Current implementation focuses on translational pose adaptation, and the rotational compensation remains future work. During policy execution, the transformed pose $\mathcal{S}'_{\text{rob}} = (x,y,z'_{(i)},g'_{(i)})$ is fed to $\pi_\theta$ instead of $\mathcal{S}_{\text{rob}}$ in \eqref{eq:obs}, maintaining observation-space consistency across grippers.


\textbf{Safety-Constrained Trajectory Projection} 
While gripper mapping aligns geometric parameters, visual perception differences from gripper morphology can still induce unsafe trajectory variations. To guarantee constraint satisfaction, we integrate a projection layer into the DDIM denoising process~\cite{song2021denoising}. The modified reverse diffusion step becomes:
%
\begin{equation}\label{eq:modified_ddim}
  \mathbf{a}^{k-1}_{t} = \text{Proj}_{\mathcal{C}}\left(\mu_{k}(\mathbf{a}^k_{t}, \epsilon_{\theta}(\mathbf{a}^k_{t},\mathbf{o}_{t},k))\right),
\end{equation}
where $\text{Proj}_{\mathcal{C}}(\cdot)$ enforces safety constraints $\mathcal{C}$ through the following two steps.

\noindent\paragraph{\textbf{Constraint-aware denoising}}
For efficiency, projection activates only in the final denoising steps ($k \leq 2$). At each step $k$, we solve a quadratic program problem:

\begin{equation}\label{eq:safety_optimization}
\begin{aligned}
    \nu^{k*}_t &= \underset{\nu^k_t}{\arg\min} \|\nu^k_t\|^2_2 \\
    \text{s.t. } & \mathcal{S}'_{\text{rob}}(z)_t + \Phi\left(\mathbf{a}^{k}_t\right) + \nu^k_t \geq \epsilon_{\text{safe}}
\end{aligned}
\end{equation}
where $\Phi(\cdot)$ maps latent actions to Cartesian displacement, which is denormalization in our case, $\epsilon_{\text{safe}} = 0.01$ m (safety margin), and $\nu^k_t$ is the minimal corrective offset.


\noindent\paragraph{\textbf{Temporal consistency enforcement}} 
To maintain safety over the policy's $T_a$-step action horizon ($j \in [0,T_a-1]$), we extend \eqref{eq:safety_optimization} with cumulative constraints:
%
\begin{equation}\label{eq:sequence_constraint}
\mathcal{S}'_{\text{rob}}(z)_{t} + \sum_{r=0}^{j}\Phi(\mathbf{a}^{k}_{t+r}) + \nu^k_{t+j} \geq \epsilon_{\text{safe}}
\end{equation}

The projected actions $\mathbf{a}^{k*}_t = \Phi^{-}[\Phi(\mathbf{a}^{k}_t) + \nu^{k*}_t]$ guarantee:
\begin{equation}
\mathbb{P}\left(\bigcap_{j=0}^{T_a} \{\mathcal{S}'_{\text{rob}}(z)_{t+j} \geq \epsilon_{\text{safe}}\}\right) = 1,
\end{equation} 
indicating the cumulative trajectory is always safe, with an example of $\mathcal{S}'_{\text{rob}}(z)_{t+1}=\mathcal{S}'_{\text{rob}}(z)_t+\Phi(\mathbf{a}^{k*}_t)$.


\begin{algorithm}[!t]
    \caption{Gripper-Aware Trajectory Generation}\label{alg:safe_diffusion}
    \setstretch{1}
    \begin{algorithmic}[1]
    
        \Require Novel gripper $\mathbb{G}_i$, Observation $\mathbf{o}_t$, Safety margin $\epsilon_{\text{safe}}$, trained noise predicted $\epsilon_\theta$
        \Ensure Safe trajectory $\tau = \{\mathbf{a}_{t:t+T_a-1}\}$
        
        \noindent \textcolor{gray}{// \textbf{Online Inference:}}
        \State $\mathcal{S}'_{\text{rob}} \leftarrow \mathds{M}_h(z_{(i)}) = z_{(i)} + \Delta h_{(i)}$ \textcolor{gray}{// \textbf{Gripper Mapping:}}
        \State $g'_{(i)} \leftarrow \mathds{M}_g(g_{(i)}) = \alpha_{(i)} g_{(i)}$
        \State $\mathbf{\tilde{o}}_t \leftarrow (\mathcal{G}^*_{\text{prob}}, \mathcal{S}^{ }_{\text{sce}}, \mathcal{S}'_{\text{rob}})$
    
        \State $\mathbf{a}^{K}_t \sim \mathcal{N}(0, \mathbf{I})$ \textcolor{gray}{// \textbf{Diffusion Process:}}
        \Repeat
            \State $k \leftarrow K-1$, and $K \leftarrow K-1$
            \State $\mathbf{a}^{k}_t \leftarrow \mathcal{N}\big(\mu_{k}(\mathbf{a}^{k+1}_{t}, \epsilon_{\theta}(\mathbf{a}^{k+1}_{t},\mathbf{\tilde{o}}_t,k+1)), 0\big)$
            \State \textbf{if} $k \leq 1$: \textcolor{gray}{// \textbf{Safety Projection:}}
            \State\quad\ \textbf{for} $j \leftarrow 0$ \textbf{to} $T_a-1$ \textbf{do}
            \State\quad\quad\ \ $\nu^{k*}_{t+j} \leftarrow \arg\min\limits_{\nu} \|\nu^k_{t+j}\|^2_2$
            \State\quad\quad\ \ $\text{s.t. } \mathcal{S}'_{\text{rob}}(z)_{t}+\sum\limits_{r=0}^{j}\Phi(\mathbf{a}^{k}_{t+r}) + \nu^k_{t+j} \geq \epsilon_{\text{safe}}$
            \State\quad\ $\mathbf{a}^{k}_{t:t+T_a-1} \leftarrow \Phi^{-}[\Phi(\mathbf{a}^{k}_{t:t+T_a-1}) + \nu^{k*}_{t:t+T_a-1}]$
            %\State\quad$a^{k}_{t+j} \leftarrow a^{k*}_{t+j}$
        \Until{$K=0$}
        \State $\tau \leftarrow \text{Decode}(\mathbf{a}^{0}_{t:t+T_a-1})$
    \end{algorithmic}
\end{algorithm} 
