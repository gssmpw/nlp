%%%%%%%%%%%%%%%%%%%%%%%%%%%%%%%%%%%%%%%%%%%%%%%%%%%%%%%%%%%%%%%%%%%%%%%%%%%%%%%%
%2345678901234567890123456789012345678901234567890123456789012345678901234567890
%        1         2         3         4         5         6         7         8

\documentclass[letterpaper, 10 pt, conference]{ieeeconf}  % Comment this line out if you need a4paper

%\documentclass[a4paper, 10pt, conference]{ieeeconf}      % Use this line for a4 paper

\IEEEoverridecommandlockouts                              % This command is only needed if
                                                          % you want to use the \thanks command

\overrideIEEEmargins                                      % Needed to meet printer requirements.
\pdfminorversion=4

\usepackage{tabu}
\usepackage{graphicx}
\usepackage{newclude}
\usepackage{color}
\usepackage[dvipsnames, svgnames, x11names]{xcolor}
\usepackage[colorlinks,linkcolor=blue,breaklinks=true]{hyperref}
\usepackage[para,online,flushleft]{threeparttable}
\usepackage{etoolbox}
%\usepackage{subfigure}
\usepackage{stfloats}
\usepackage{subcaption}
\usepackage{amsmath}
\usepackage{ulem}
\usepackage{tikz}
\usepackage{cite}
\usepackage{url}
\usepackage{verbatim}
\usepackage{booktabs}
\usepackage{pifont}
\usepackage{algorithm}
\usepackage{algpseudocode}
\usepackage{dsfont}
\usepackage{amsfonts} 
\usepackage{setspace}
\usepackage[skip=1.5pt]{caption}
% \usepackage[ruled,vlined]{algorithm2e}
\usetikzlibrary{positioning}
\usetikzlibrary{matrix}
\usetikzlibrary{shapes.multipart} 
\usetikzlibrary{arrows.meta}
\usetikzlibrary{calc}
\makeatletter
\patchcmd{\@makecaption}
  {\scshape}
  {}
  {}
  {}
\makeatletter
\patchcmd{\@makecaption}
  {\\}
  {.\ }
  {}
  {}
\makeatother
\def\tablename{Table}

\title{\LARGE \bf
Pick-and-place Manipulation Across Grippers Without Retraining:
\\ A Learning-optimization Diffusion Policy Approach
}


\author{Xiangtong Yao$^{1*}$, Yirui Zhou$^{1*}$, Yuan Meng$^{1}$, Liangyu Dong$^{1}$, Lin Hong$^{2}$,\\Zitao Zhang$^{1}$, Zhenshan Bing$^{1,3\dagger}$, Kai Huang$^{4}$, Fuchun Sun$^{5}$, Alois Knoll$^{1}$
        % <-this % stops a space  
\thanks{$^1$ Technical University of Munich, Munich, Germany}
\thanks{$^2$ Hong Kong University of Science and Technology, China}
\thanks{$^3$ Nanjing University, Nanjing, China}
\thanks{$^4$ Sun Yat-sen University, Guangzhou, China}
\thanks{$^5$ Tsinghua University, Beijing, China}
\thanks{$^*$ Equal contribution}
\thanks{$^{\dagger}$Corresponding author: Zhenshan Bing {\tt\small zhenshan.bing@tum.de}}}

\providecommand{\keywords}[1]{\textbf{\textit{Keywords---}} #1}


\let\oldtwocolumn\twocolumn
\renewcommand\twocolumn[1][]{%
    \oldtwocolumn[{#1}{
    \vspace{-5pt}
    \begin{center}
           \includegraphics[width=\textwidth]{figure/figure1.pdf}
           \captionof{figure}{(a) Different issues of the trajectory generation of the vanilla diffusion policy across grippers without the policy retaining. (b) Our policy aims to transfer pick-and-place skills across grippers without policy retraining while ensuring task completion, involving (i) offline training the policy $\pi_\theta$ with data collected via a base gripper, (ii) gripper-aware mapping, and (iii) online optimization adaption to enforce the generated trajectories to fit different unseen grippers.} 
           \label{fig:first-figure}
    \end{center}
    }]
}


\begin{document}


\maketitle

%%%%%%%%%%%%%%%%%%%%%%%%%%%%%%%%%%%%%%%%%%%%%%%%%%%%%%%%%%%%%%%%%%%%%%%%%%%%%%%%
\begin{abstract}
%%% improved version
Current robotic pick-and-place policies typically require consistent gripper configurations across training and inference. This constraint imposes high retraining or fine-tuning costs, especially for imitation learning-based approaches, when adapting to new end-effectors. To mitigate this issue, we present a diffusion-based policy with a hybrid learning-optimization framework, enabling zero-shot adaptation to novel grippers without additional data collection for retraining policy. During training, the policy learns manipulation primitives from demonstrations collected using a base gripper. At inference, a diffusion-based optimization strategy dynamically enforces kinematic and safety constraints, ensuring that generated trajectories align with the physical properties of unseen grippers. This is achieved through a constrained denoising procedure that adapts trajectories to gripper-specific parameters (e.g., tool-center-point offsets, jaw widths) while preserving collision avoidance and task feasibility. We validate our method on a Franka Panda robot across six gripper configurations, including 3D-printed fingertips, flexible silicone gripper, and Robotiq 2F-85 gripper. Our approach achieves a 93.3\% average task success rate across grippers (vs. 23.3-26.7\% for diffusion policy baselines), supporting tool-center-point variations of 16–23.5 cm and jaw widths of 7.5–11.5 cm. The results demonstrate that constrained diffusion enables robust cross-gripper manipulation while maintaining the sample efficiency of imitation learning, eliminating the need for gripper-specific retraining. Video and code are available at \href{https://github.com/yaoxt3/GADP}{https://github.com/yaoxt3/GADP}.


\end{abstract}


\definecolor{color-obs-seq}{RGB}{219, 232, 213}
\definecolor{color-act-seq}{RGB}{172, 204, 255}
\definecolor{color-opti}{RGB}{248,206,204} 
%\definecolor{color-gmap}{RGB}{204,255,204} 
\definecolor{color-gmap}{RGB}{0,204,153} 

% -------------------------------------------
% -- Color style: Honkai Star Rail Firefly --
% -------------------------------------------
\definecolor{ffblue}{RGB}{097, 108, 140}
\definecolor{ffdarkgreen}{RGB}{086, 140, 135}
\definecolor{fflightgreen}{RGB}{178, 213, 155}
\definecolor{ffyellow}{RGB}{242, 222, 121}
\definecolor{ffred}{RGB}{217, 095, 024}
\definecolor{ffred_pv}{RGB}{202, 074, 046}
\definecolor{fforange_pv}{RGB}{232, 141, 047}
\definecolor{ffgreen_pv}{RGB}{059, 165, 149}
\definecolor{ffgreendark_pv}{RGB}{032, 117, 106}
% -------------------------------------------
% -- Color style: Honkai Star Rail Firefly --
% -------------------------------------------
\definecolor{nature_tab_gray1}{HTML}{D8D6C2}
\definecolor{nature_tab_gray2}{HTML}{ECEADF}

\definecolor{graspw}{RGB}{0, 128, 0}
\definecolor{dp}{RGB}{64, 224, 208}
\definecolor{dp3}{RGB}{63, 63, 255}
\definecolor{ours}{RGB}{148, 0, 211}
\definecolor{blockcolor}{RGB}{238, 130, 238}
\definecolor{safeline}{RGB}{255, 0, 0}
\definecolor{bananacolor}{RGB}{255, 165, 0}

%\definecolor{ffdarkgreen}{RGB}{086, 140, 135}

\include*{1_introduction}
\include*{2_related_work}
\include*{3_background}
\include*{4_method}
\include*{5_experiments}



\section{CONCLUSIONS AND OUTLOOK}

This paper proposes a diffusion-based policy for transferring pick-and-place knowledge across different grippers. This knowledge transition does not require retraining or fine-turning the policy with the new gripper's configuration. Instead, it only needs to introduce the configuration in the policy inference phase and make the generated trajectories satisfy safety constraints, ensuring the successful completion of pick-and-place tasks. To enhance the flexibility of integrating different grippers into our method, the gripper's configurations can be described with free-form language instructions during the policy inference phase.
%%%%%%%%%%%%%%%%%%%%%%%%%%%%%%%%%%%%%%%%%%%%%%%%%%%%%%%%%%%%%%%%%%%%%%%%%%%%%%%%

\normalem
\bibliographystyle{./IEEEtran} % use IEEEtran.bst style
\bibliography{./IEEEexample}


\end{document}
