
\documentclass[10pt,journal,compsoc]{IEEEtran}

\ifCLASSOPTIONcompsoc
  % IEEE Computer Society needs nocompress option
  % requires cite.sty v4.0 or later (November 2003)
  \usepackage[nocompress]{cite}
\else
  % normal IEEE
  \usepackage{cite}
\fi

% *** GRAPHICS RELATED PACKAGES ***
%
\ifCLASSINFOpdf
   \usepackage[pdftex]{graphicx}
  % declare the path(s) where your graphic files are
   \graphicspath{{../pdf/}{../jpeg/}}
  % and their extensions so you won't have to specify these with
  % every instance of \includegraphics
   \DeclareGraphicsExtensions{.pdf,.jpeg,.png,.jpg}
\else
  % or other class option (dvipsone, dvipdf, if not using dvips). graphicx
  % will default to the driver specified in the system graphics.cfg if no
  % driver is specified.
   \usepackage[dvips]{graphicx}
  % declare the path(s) where your graphic files are
   \graphicspath{{../eps/}}
  % and their extensions so you won't have to specify these with
  % every instance of \includegraphics
   \DeclareGraphicsExtensions{.eps}
\fi




\usepackage{hyperref}
\usepackage{amsmath}
\usepackage{algorithm}
\usepackage{algorithmic}
\usepackage{amsfonts}
\usepackage{wrapfig}


% *** SUBFIGURE PACKAGES ***
% \ifCLASSOPTIONcompsoc
%   \usepackage[caption=false,font=footnotesize,labelfont=sf,textfont=sf]{subfig}
% \else
%   \usepackage[caption=false,font=footnotesize]{subfig}
% \fi
\usepackage{subcaption}
\usepackage{stfloats}

\usepackage{amssymb}

\usepackage{cleveref}
\usepackage{xcolor}
\usepackage{adjustbox}
\usepackage{xspace}


\usepackage{multirow}
\usepackage{xcolor}
\definecolor{myorange}{RGB}{236,166,128} % Define a custom orange color
\definecolor{myblue}{RGB}{39,131,177} % Define a custom orange color
\usepackage{tikz} % For drawing lines
\newcommand{\name}{\textit{WonderHuman}\xspace}
\newcommand{\TODO}[1]{{\color{red}\textbf{TODO: #1}}}
\newcommand{\ZL}[1]{{\color{blue}{#1}}}



% \def\ie{{\em i.e.}}
% \def\eg{{\em e.g.}}
% \def\etal{{\em et al.}}
\def\up{{\bf$\uparrow $} }
\def\down{{\bf$\downarrow $} }
% \newcommand{\figref}[1]{Fig. \ref{#1}}
% \newcommand{\tabref}[1]{Tab. \ref{#1}}
% \newcommand{\equref}[1]{(\ref{#1})}
% \newcommand{\secref}[1]{Section \ref{#1}}
% \newcommand{\algref}[1]{Alg. \ref{#1}}
% \newcommand{\myPara}[1]{\vspace{.05in}\noindent\textbf{#1}}
% \newcommand{\todo}[1]{\textcolor{red}{\bf [#1]}}
% \newcommand{\bl}[1]{\textbf{#1}}
% \newcommand{\mc}[1]{\mathcal{#1}}
% \newcommand{\mb}[1]{\mathbb{#1}}
% \newcommand{\tabincell}[2]{\begin{tabular}{@{}#1@{}}#2\end{tabular}}
% \newcommand{\bul}[1]{\underline{\textbf{#1}}}
% %\newcommand{\bm}[1]{\mbox{\boldmath{$#1$}}}
% \newcommand{\br}[1]{\bm{\mathrm{#1}}}

% \hyphenation{op-tical net-works semi-conduc-tor}
\usepackage{ragged2e}
% \usepackage{booktabs}
% \usepackage{color}
% \usepackage{multirow}
% \usepackage{bm}
% \usepackage{bbm}
% \usepackage{balance}
% \renewcommand{\algorithmicrequire}{ \textbf{Input:}}      %Use Input in the format of Algorithm
% \renewcommand{\algorithmicensure}{ \textbf{Output:}}     %UseOutput in the format of Algorithm

\newcommand{\ZY}[1]{{\color{red}#1}}


\begin{document}

\title{
WonderHuman: Hallucinating Unseen Parts in Dynamic 3D Human Reconstruction
}


\author{Zilong~Wang,~Zhiyang~Dou,~Yuan~Liu,~Cheng~Lin,~Xiao~Dong,~Yunhui~Guo,~Chenxu~Zhang,\\~Xin~Li,~Wenping~Wang,~Xiaohu~Guo% <-this % stops a space
\IEEEcompsocitemizethanks{\IEEEcompsocthanksitem Z. Wang, Y. Guo, C. Zhang and X. Guo are with the Department of Computer Science, The University of Texas at Dallas, Richardson, Texas.
\IEEEcompsocthanksitem Z. Dou and C. Lin are with the Computer Graphics Group, The University of Hong Kong, Pokfulam, Hong Kong.
\IEEEcompsocthanksitem Y. Liu is with the School of Engineering, The Hong Kong University of Science and Technology, Clear Water Bay, Hong Kong.
\IEEEcompsocthanksitem X. Dong is with the Department of Computer Science, BNU-HKBU United International College, Zhuhai, China.
\IEEEcompsocthanksitem X. Li and W. Wang are with the Department of Computer Science \& Engineering, Texas A\&M University, College Station, Texas.

\IEEEcompsocthanksitem Corresponding Author: X. Guo, Email: xguo@utdallas.edu
}
}

% \markboth{Submission to IEEE Transactions on Visualization and Computer Graphics}%
% {Z. Wang \MakeLowercase{\textit{et al.}}:}

\IEEEtitleabstractindextext{%
\begin{abstract}
\justifying In this paper, we present \textit{WonderHuman} to reconstruct dynamic human avatars from a monocular video for high-fidelity novel view synthesis. 
Previous dynamic human avatar reconstruction methods typically require the input video to have full coverage of the observed human body. However, in daily practice, one typically has access to limited viewpoints, such as monocular front-view videos, making it a cumbersome task for previous methods to reconstruct the unseen parts of the human avatar. To tackle the issue, we present \textit{WonderHuman}, which leverages 2D generative diffusion model priors to achieve high-quality, photorealistic reconstructions of dynamic human avatars from monocular videos, including accurate rendering of unseen body parts. Our approach introduces a Dual-Space Optimization technique, applying Score Distillation Sampling (SDS) in both canonical and observation spaces to ensure visual consistency and enhance realism in dynamic human reconstruction. Additionally, we present a View Selection strategy and Pose Feature Injection to enforce the consistency between SDS predictions and observed data, ensuring pose-dependent effects and higher fidelity in the reconstructed avatar. In the experiments, our method achieves SOTA performance in producing photorealistic renderings from the given monocular video, particularly for those challenging unseen parts. The project page and source code can be found at \url{https://wyiguanw.github.io/WonderHuman/}.%The code will be made publicly accessible upon publication.

\end{abstract}

\begin{IEEEkeywords}
Monocular Video, 3D Gaussian Splatting, Human Unseen Part Reconstruction, Diffusion, Score Distillation Sampling.
\end{IEEEkeywords}}

\maketitle


\IEEEdisplaynontitleabstractindextext

\IEEEpeerreviewmaketitle


\section{Introduction}
\label{section:introduction}

% redirection is unique and important in VR
Virtual Reality (VR) systems enable users to embody virtual avatars by mirroring their physical movements and aligning their perspective with virtual avatars' in real time. 
As the head-mounted displays (HMDs) block direct visual access to the physical world, users primarily rely on visual feedback from the virtual environment and integrate it with proprioceptive cues to control the avatar’s movements and interact within the VR space.
Since human perception is heavily influenced by visual input~\cite{gibson1933adaptation}, 
VR systems have the unique capability to control users' perception of the virtual environment and avatars by manipulating the visual information presented to them.
Leveraging this, various redirection techniques have been proposed to enable novel VR interactions, 
such as redirecting users' walking paths~\cite{razzaque2005redirected, suma2012impossible, steinicke2009estimation},
modifying reaching movements~\cite{gonzalez2022model, azmandian2016haptic, cheng2017sparse, feick2021visuo},
and conveying haptic information through visual feedback to create pseudo-haptic effects~\cite{samad2019pseudo, dominjon2005influence, lecuyer2009simulating}.
Such redirection techniques enable these interactions by manipulating the alignment between users' physical movements and their virtual avatar's actions.

% % what is hand/arm redirection, motivation of study arm-offset
% \change{\yj{i don't understand the purpose of this paragraph}
% These illusion-based techniques provide users with unique experiences in virtual environments that differ from the physical world yet maintain an immersive experience. 
% A key example is hand redirection, which shifts the virtual hand’s position away from the real hand as the user moves to enhance ergonomics during interaction~\cite{feuchtner2018ownershift, wentzel2020improving} and improve interaction performance~\cite{montano2017erg, poupyrev1996go}. 
% To increase the realism of virtual movements and strengthen the user’s sense of embodiment, hand redirection techniques often incorporate a complete virtual arm or full body alongside the redirected virtual hand, using inverse kinematics~\cite{hartfill2021analysis, ponton2024stretch} or adjustments to the virtual arm's movement as well~\cite{li2022modeling, feick2024impact}.
% }

% noticeability, motivation of predicting a probability, not a classification
However, these redirection techniques are most effective when the manipulation remains undetected~\cite{gonzalez2017model, li2022modeling}. 
If the redirection becomes too large, the user may not mitigate the conflict between the visual sensory input (redirected virtual movement) and their proprioception (actual physical movement), potentially leading to a loss of embodiment with the virtual avatar and making it difficult for the user to accurately control virtual movements to complete interaction tasks~\cite{li2022modeling, wentzel2020improving, feuchtner2018ownershift}. 
While proprioception is not absolute, users only have a general sense of their physical movements and the likelihood that they notice the redirection is probabilistic. 
This probability of detecting the redirection is referred to as \textbf{noticeability}~\cite{li2022modeling, zenner2024beyond, zenner2023detectability} and is typically estimated based on the frequency with which users detect the manipulation across multiple trials.

% version B
% Prior research has explored factors influencing the noticeability of redirected motion, including the redirection's magnitude~\cite{wentzel2020improving, poupyrev1996go}, direction~\cite{li2022modeling, feuchtner2018ownershift}, and the visual characteristics of the virtual avatar~\cite{ogawa2020effect, feick2024impact}.
% While these factors focus on the avatars, the surrounding virtual environment can also influence the users' behavior and in turn affect the noticeability of redirection.
% One such prominent external influence is through the visual channel - the users' visual attention is constantly distracted by complex visual effects and events in practical VR scenarios.
% Although some prior studies have explored how to leverage user blindness caused by visual distractions to redirect users' virtual hand~\cite{zenner2023detectability}, there remains a gap in understanding how to quantify the noticeability of redirection under visual distractions.

% visual stimuli and gaze behavior
Prior research has explored factors influencing the noticeability of redirected motion, including the redirection's magnitude~\cite{wentzel2020improving, poupyrev1996go}, direction~\cite{li2022modeling, feuchtner2018ownershift}, and the visual characteristics of the virtual avatar~\cite{ogawa2020effect, feick2024impact}.
While these factors focus on the avatars, the surrounding virtual environment can also influence the users' behavior and in turn affect the noticeability of redirection.
This, however, remains underexplored.
One such prominent external influence is through the visual channel - the users' visual attention is constantly distracted by complex visual effects and events in practical VR scenarios.
We thus want to investigate how \textbf{visual stimuli in the virtual environment} affect the noticeability of redirection.
With this, we hope to complement existing works that focus on avatars by incorporating environmental visual influences to enable more accurate control over the noticeability of redirected motions in practical VR scenarios.
% However, in realistic VR applications, the virtual environment often contains complex visual effects beyond the virtual avatar itself. 
% We argue that these visual effects can \textbf{distract users’ visual attention and thus affect the noticeability of redirection offsets}, while current research has yet taken into account.
% For instance, in a VR boxing scenario, a user’s visual attention is likely focused on their opponent rather than on their virtual body, leading to a lower noticeability of redirection offsets on their virtual movements. 
% Conversely, when reaching for an object in the center of their field of view, the user’s attention is more concentrated on the virtual hand’s movement and position to ensure successful interaction, resulting in a higher noticeability of offsets.

Since each visual event is a complex choreography of many underlying factors (type of visual effect, location, duration, etc.), it is extremely difficult to quantify or parameterize visual stimuli.
Furthermore, individuals respond differently to even the same visual events.
Prior neuroscience studies revealed that factors like age, gender, and personality can influence how quickly someone reacts to visual events~\cite{gillon2024responses, gale1997human}. 
Therefore, aiming to model visual stimuli in a way that is generalizable and applicable to different stimuli and users, we propose to use users' \textbf{gaze behavior} as an indicator of how they respond to visual stimuli.
In this paper, we used various gaze behaviors, including gaze location, saccades~\cite{krejtz2018eye}, fixations~\cite{perkhofer2019using}, and the Index of Pupil Activity (IPA)~\cite{duchowski2018index}.
These behaviors indicate both where users are looking and their cognitive activity, as looking at something does not necessarily mean they are attending to it.
Our goal is to investigate how these gaze behaviors stimulated by various visual stimuli relate to the noticeability of redirection.
With this, we contribute a model that allows designers and content creators to adjust the redirection in real-time responding to dynamic visual events in VR.

To achieve this, we conducted user studies to collect users' noticeability of redirection under various visual stimuli.
To simulate realistic VR scenarios, we adopted a dual-task design in which the participants performed redirected movements while monitoring the visual stimuli.
Specifically, participants' primary task was to report if they noticed an offset between the avatar's movement and their own, while their secondary task was to monitor and report the visual stimuli.
As realistic virtual environments often contain complex visual effects, we started with simple and controlled visual stimulus to manage the influencing factors.

% first user study, confirmation study
% collect data under no visual stimuli, different basic visual stimuli
We first conducted a confirmation study (N=16) to test whether applying visual stimuli (opacity-based) actually affects their noticeability of redirection. 
The results showed that participants were significantly less likely to detect the redirection when visual stimuli was presented $(F_{(1,15)}=5.90,~p=0.03)$.
Furthermore, by analyzing the collected gaze data, results revealed a correlation between the proposed gaze behaviors and the noticeability results $(r=-0.43)$, confirming that the gaze behaviors could be leveraged to compute the noticeability.

% data collection study
We then conducted a data collection study to obtain more accurate noticeability results through repeated measurements to better model the relationship between visual stimuli-triggered gaze behaviors and noticeability of redirection.
With the collected data, we analyzed various numerical features from the gaze behaviors to identify the most effective ones. 
We tested combinations of these features to determine the most effective one for predicting noticeability under visual stimuli.
Using the selected features, our regression model achieved a mean squared error (MSE) of 0.011 through leave-one-user-out cross-validation. 
Furthermore, we developed both a binary and a three-class classification model to categorize noticeability, which achieved an accuracy of 91.74\% and 85.62\%, respectively.

% evaluation study
To evaluate the generalizability of the regression model, we conducted an evaluation study (N=24) to test whether the model could accurately predict noticeability with new visual stimuli (color- and scale-based animations).
Specifically, we evaluated whether the model's predictions aligned with participants' responses under these unseen stimuli.
The results showed that our model accurately estimated the noticeability, achieving mean squared errors (MSE) of 0.014 and 0.012 for the color- and scale-based visual stimili, respectively, compared to participants' responses.
Since the tested visual stimuli data were not included in the training, the results suggested that the extracted gaze behavior features capture a generalizable pattern and can effectively indicate the corresponding impact on the noticeability of redirection.

% application
Based on our model, we implemented an adaptive redirection technique and demonstrated it through two applications: adaptive VR action game and opportunistic rendering.
We conducted a proof-of-concept user study (N=8) to compare our adaptive redirection technique with a static redirection, evaluating the usability and benefits of our adaptive redirection technique.
The results indicated that participants experienced less physical demand and stronger sense of embodiment and agency when using the adaptive redirection technique. 
These results demonstrated the effectiveness and usability of our model.

In summary, we make the following contributions.
% 
\begin{itemize}
    \item 
    We propose to use users' gaze behavior as a medium to quantify how visual stimuli influences the noticebility of redirection. 
    Through two user studies, we confirm that visual stimuli significantly influences noticeability and identify key gaze behavior features that are closely related to this impact.
    \item 
    We build a regression model that takes the user's gaze behavioral data as input, then computes the noticeability of redirection.
    Through an evaluation study, we verify that our model can estimate the noticeability with new participants under unseen visual stimuli.
    These findings suggest that the extracted gaze behavior features effectively capture the influence of visual stimuli on noticeability and can generalize across different users and visual stimuli.
    \item 
    We develop an adaptive redirection technique based on our regression model and implement two applications with it.
    With a proof-of-concept study, we demonstrate the effectiveness and potential usability of our regression model on real-world use cases.

\end{itemize}

% \delete{
% Virtual Reality (VR) allows the user to embody a virtual avatar by mirroring their physical movements through the avatar.
% As the user's visual access to the physical world is blocked in tasks involving motion control, they heavily rely on the visual representation of the avatar's motions to guide their proprioception.
% Similar to real-world experiences, the user is able to resolve conflicts between different sensory inputs (e.g., vision and motor control) through multisensory integration, which is essential for mitigating the sensory noise that commonly arises.
% However, it also enables unique manipulations in VR, as the system can intentionally modify the avatar's movements in relation to the user's motions to achieve specific functional outcomes,
% for example, 
% % the manipulations on the avatar's movements can 
% enabling novel interaction techniques of redirected walking~\cite{razzaque2005redirected}, redirected reaching~\cite{gonzalez2022model}, and pseudo haptics~\cite{samad2019pseudo}.
% With small adjustments to the avatar's movements, the user can maintain their sense of embodiment, due to their ability to resolve the perceptual differences.
% % However, a large mismatch between the user and avatar's movements can result in the user losing their sense of embodiment, due to an inability to resolve the perceptual differences.
% }

% \delete{
% However, multisensory integration can break when the manipulation is so intense that the user is aware of the existence of the motion offset and no longer maintains the sense of embodiment.
% Prior research studied the intensity threshold of the offset applied on the avatar's hand, beyond which the embodiment will break~\cite{li2022modeling}. 
% Studies also investigated the user's sensitivity to the offsets over time~\cite{kohm2022sensitivity}.
% Based on the findings, we argue that one crucial factor that affects to what extent the user notices the offset (i.e., \textit{noticeability}) that remains under-explored is whether the user directs their visual attention towards or away from the virtual avatar.
% Related work (e.g., Mise-unseen~\cite{marwecki2019mise}) has showcased applications where adjustments in the environment can be made in an unnoticeable manner when they happen in the area out of the user's visual field.
% We hypothesize that directing the user's visual attention away from the avatar's body, while still partially keeping the avatar within the user's field-of-view, can reduce the noticeability of the offset.
% Therefore, we conduct two user studies and implement a regression model to systematically investigate this effect.
% }

% \delete{
% In the first user study (N = 16), we test whether drawing the user's visual attention away from their body impacts the possibility of them noticing an offset that we apply to their arm motion in VR.
% We adopt a dual-task design to enable the alteration of the user's visual attention and a yes/no paradigm to measure the noticeability of motion offset. 
% The primary task for the user is to perform an arm motion and report when they perceive an offset between the avatar's virtual arm and their real arm.
% In the secondary task, we randomly render a visual animation of a ball turning from transparent to red and becoming transparent again and ask them to monitor and report when it appears.
% We control the strength of the visual stimuli by changing the duration and location of the animation.
% % By changing the time duration and location of the visual animation, we control the strengths of attraction to the users.
% As a result, we found significant differences in the noticeability of the offsets $(F_{(1,15)}=5.90,~p=0.03)$ between conditions with and without visual stimuli.
% Based on further analysis, we also identified the behavioral patterns of the user's gaze (including pupil dilation, fixations, and saccades) to be correlated with the noticeability results $(r=-0.43)$ and they may potentially serve as indicators of noticeability.
% }

% \delete{
% To further investigate how visual attention influences the noticeability, we conduct a data collection study (N = 12) and build a regression model based on the data.
% The regression model is able to calculate the noticeability of the offset applied on the user's arm under various visual stimuli based on their gaze behaviors.
% Our leave-one-out cross-validation results show that the proposed method was able to achieve a mean-squared error (MSE) of 0.012 in the probability regression task.
% }

% \delete{
% To verify the feasibility and extendability of the regression model, we conduct an evaluation study where we test new visual animations based on adjustments on scale and color and invite 24 new participants to attend the study.
% Results show that the proposed method can accurately estimate the noticeability with an MSE of 0.014 and 0.012 in the conditions of the color- and scale-based visual effects.
% Since these animations were not included in the dataset that the regression model was built on, the study demonstrates that the gaze behavioral features we extracted from the data capture a generalizable pattern of the user's visual attention and can indicate the corresponding impact on the noticeability of the offset.
% }

% \delete{
% Finally, we demonstrate applications that can benefit from the noticeability prediction model, including adaptive motion offsets and opportunistic rendering, considering the user's visual attention. 
% We conclude with discussions of our work's limitations and future research directions.
% }

% \delete{
% In summary, we make the following contributions.
% }
% % 
% \begin{itemize}
%     \item 
%     \delete{
%     We quantify the effects of the user's visual attention directed away by stimuli on their noticeability of an offset applied to the avatar's arm motion with respect to the user's physical arm. 
%     Through two user studies, we identified gaze behavioral features that are indicative of the changes in noticeability.
%     }
%     \item 
%     \delete{We build a regression model that takes the user's gaze behavioral data and the offset applied to the arm motion as input, then computes the probability of the user noticing the offset.
%     Through an evaluation study, we verified that the model needs no information about the source attracting the user's visual attention and can be generalizable in different scenarios.
%     }
%     \item 
%     \delete{We demonstrate two applications that potentially benefit from the regression model, including adaptive motion offsets and opportunistic rendering.
%     }

% \end{itemize}

\begin{comment}
However, users will lose the sense of embodiment to the virtual avatars if they notice the offset between the virtual and physical movements.
To address this, researchers have been exploring the noticing threshold of offsets with various magnitudes and proposing various redirection techniques that maintain the sense of embodiment~\cite{}.

However, when users embody virtual avatars to explore virtual environments, they encounter various visual effects and content that can attract their attention~\cite{}.
During this, the user may notice an offset when he observes the virtual movement carefully while ignoring it when the virtual contents attract his attention from the movements.
Therefore, static offset thresholds are not appropriate in dynamic scenarios.

Past research has proposed dynamic mapping techniques that adapted to users' state, such as hand moving speed~\cite{frees2007prism} or ergonomically comfortable poses~\cite{montano2017erg}, but not considering the influence of virtual content.
More specifically, PRISM~\cite{frees2007prism} proposed adjusting the C/D ratio with a non-linear mapping according to users' hand moving speed, but it might not be optimal for various virtual scenarios.
While Erg-O~\cite{montano2017erg} redirected users' virtual hands according to the virtual target's relative position to reduce physical fatigue, neglecting the change of virtual environments. 

Therefore, how to design redirection techniques in various scenarios with different visual attractions remains unknown.
To address this, we investigate how visual attention affects the noticing probability of movement offsets.
Based on our experiments, we implement a computational model that automatically computes the noticing probability of offsets under certain visual attractions.
VR application designers and developers can easily leverage our model to design redirection techniques maintaining the sense of embodiment adapt to the user's visual attention.
We implement a dynamic redirection technique with our model and demonstrate that it effectively reduces the target reaching time without reducing the sense of embodiment compared to static redirection techniques.

% Need to be refined
This paper offers the following contributions.
\begin{itemize}
    \item We investigate how visual attractions affect the noticing probability of redirection offsets.
    \item We construct a computational model to predict the noticing probability of an offset with a given visual background.
    \item We implement a dynamic redirection technique adapting to the visual background. We evaluate the technique and develop three applications to demonstrate the benefits. 
\end{itemize}



First, we conducted a controlled experiment to understand how users perceived the movement offset while subjected to various distractions.
Since hand redirection is one of the most frequently used redirections in VR interactions, we focused on the dynamic arm movements and manually added angular offsets to the' elbow joint~\cite{li2022modeling, gonzalez2022model, zenner2019estimating}. 
We employed flashing spheres in the user's field of view as distractions to attract users' visual attention.
Participants were instructed to report the appearing location of the spheres while simultaneously performing the arm movements and reporting if they perceived an offset during the movement. 
(\zhipeng{Add the results of data collection. Analyze the influence of the distance between the gaze map and the offset.}
We measured the visual attraction's magnitude with the gaze distribution on it.
Results showed that stronger distractions made it harder for users to notice the offset.)
\zhipeng{Need to rewrite. Not sure to use gaze distribution or a metric obtained from the visual content.}
Secondly, we constructed a computational model to predict the noticing probability of offsets with given visual content.
We analyzed the data from the user studies to measure the influence of visual attractions on the noticing probability of offsets.
We built a statistical model to predict the offset's noticing probability with a given visual content.
Based on the model, we implement a dynamic redirection technique to adjust the redirection offset adapted to the user's current field of view.
We evaluated the technique in a target selection task compared to no hand redirection and static hand redirection.
\zhipeng{Add the results of the evaluation.}
Results showed that the dynamic hand redirection technique significantly reduced the target selection time with similar accuracy and a comparable sense of embodiment.
Finally, we implemented three applications to demonstrate the potential benefits of the visual attention adapted dynamic redirection technique.
\end{comment}

% This one modifies arm length, not redirection
% \citeauthor{mcintosh2020iteratively} proposed an adaptation method to iteratively change the virtual avatar arm's length based on the primary tasks' performance~\cite{mcintosh2020iteratively}.



% \zhipeng{TO ADD: what is redirection}
% Redirection enables novel interactions in Virtual Reality, including redirected walking, haptic redirection, and pseudo haptics by introducing an offset to users' movement.
% \zhipeng{TO ADD: extend this sentence}
% The price of this is that users' immersiveness and embodiment in VR can be compromised when they notice the offset and perceive the virtual movement not as theirs~\cite{}.
% \zhipeng{TO ADD: extend this sentence, elaborate how the virtual environment attracts users' attention}
% Meanwhile, the visual content in the virtual environment is abundant and consistently captures users' attention, making it harder to notice the offset~\cite{}.
% While previous studies explored the noticing threshold of the offsets and optimized the redirection techniques to maintain the sense of embodiment~\cite{}, the influence of visual content on the probability of perceiving offsets remains unknown.  
% Therefore, we propose to investigate how users perceive the redirection offset when they are facing various visual attractions.


% We conducted a user study to understand how users notice the shift with visual attractions.
% We used a color-changing ball to attract the user's attention while instructing users to perform different poses with their arms and observe it meanwhile.
% \zhipeng{(Which one should be the primary task? Observe the ball should be the primary one, but if the primary task is too simple, users might allocate more attention on the secondary task and this makes the secondary task primary.)}
% \zhipeng{(We need a good and reasonable dual-task design in which users care about both their pose and the visual content, at least in the evaluation study. And we need to be able to control the visual content's magnitude and saliency maybe?)}
% We controlled the shift magnitude and direction, the user's pose, the ball's size, and the color range.
% We set the ball's color-changing interval as the independent factor.
% We collect the user's response to each shift and the color-changing times.
% Based on the collected data, we constructed a statistical model to describe the influence of visual attraction on the noticing probability.
% \zhipeng{(Are we actually controlling the attention allocation? How do we measure the attracting effect? We need uniform metrics, otherwise it is also hard for others to use our knowledge.)}
% \zhipeng{(Try to use eye gaze? The eye gaze distribution in the last five seconds to decide the attention allocation? Basically constructing a model with eye gaze distribution and noticing probability. But the user's head is moving, so the eye gaze distribution is not aligned well with the current view.)}

% \zhipeng{Saliency and EMD}
% \zhipeng{Gaze is more than just a point: Rethinking visual attention
% analysis using peripheral vision-based gaze mapping}

% Evaluation study(ideal case): based on the visual content, adjusting the redirection magnitude dynamically.

% \zhipeng{(The risk is our model's effect is trivial.)}

% Applications:
% Playing Lego while watching demo videos, we can accelerate the reaching process of bricks, and forbid the redirection during the manipulation.

% Beat saber again: but not make a lot of sense? Difficult game has complicated visual effects, while allows larger shift, but do not need large shift with high difficulty



\section{Related work}
\mvspace{-2mm}
\paragraph{Human-AI complementarity.}

Many empirical studies of human-AI collaboration focus on AI-assisted human decision-making for legal, ethical, or safety reasons~\citep{bo2021toward, boskemper2022measuring, bondi2022role, schemmer2022meta}.
However, a recent meta-analysis by \citet{vaccaro2024combinations} finds that, on average, human–AI teams perform worse than the better of the two agents alone. 
In response, a growing body of work seeks to evaluate and enhance complementarity in human–AI systems \citep{bansal2021does, bansal2019updates, bansal2021most, wilder2021learning, rastogi2023taxonomy, mozannar2024effective}.
The present work differs from much of this prior work by approaching human-AI complementarity from the perspective of information value and use, including asking whether the human and AI decisions provide additional information that is not used by the other.
\mvspace{-2mm}
\paragraph{Evaluation of human decision-making with machine learning.}
Our work contributes methods for evaluating the decisions of human-AI teams~\citep{kleinberg2015prediction, kleinberg2018human, lakkaraju2017selective, mullainathan2022diagnosing,  rambachan2024identifying, guo2024decision, ben2024does, shreekumar2025x}.
\citet{kleinberg2015prediction} proposed that evaluations of human-AI collaboration should be based on the information that is available at the time of decisions.
% \jessica{can omit:} A significant portion of this literature addresses \textit{performative prediction}~\citep{perdomo2020performative}, where predictions or decisions affect future outcomes. 
% Because counterfactual decisions’ outcomes remain unobserved, researchers typically rely on worst-case analyses to bound the potential performance \citep{rambachan2024identifying, ben2024does}. 
% Though these issues arise in many canonical human-AI collaboration tasks, we focus on standard ``prediction policy problems'' where the payoff can be translated into policy gains~\citep{kleinberg2015prediction}.
According to this view, our work defines Bayesian best-attainable-performance benchmarks similar to several prior works~\citep{guo2024decision, wu2023rational,agrawal2020scaling, fudenberg2022measuring}. 
Closest to our work, \citet{guo2024decision} model the expected performance of a rational Bayesian agent faced with deciding between the human and AI recommendations as the theoretical upper bound on the expected performance of any human-AI team.
This benchmark provides a basis for identifying exploitable information within a decision problem.

\mvspace{-3mm}
\paragraph{Human information in machine learning.}

Some approaches focus on automating the decision pipeline by explicitly incorporating human expertise in developing machine learning models, such as by learning to defer~\citep{mozannar2024show, madras2018predict, raghu2019algorithmic, keswani2022designing, keswani2021towards, okati2021differentiable}.
\citet{corvelo2023human} propose multicalibration over human and AI model confidence information to guarantee the existence of an optimal monotonic decision rule.
\citet{alur2023auditing} propose a hypothesis testing framework to evaluate the added value of human expertise over AI forecasts.
Our work shares the motivation of incorporating human expertise, but targets a slightly broader scope by quantifying the information value for all available signals and agent decisions in a human–AI decision pipeline.




\section{Preliminaries }
\label{sec:Preliminaries}

% Before we introduce our method, we first overview the important basics of 3D dynamic human modeling with Gaussian splatting. Then, we discuss the diffusion-based 3d generation techniques, and how they can be applied to human modeling.
% \ZY{I stopp here. TBC.}
% \subsection{Dynamic human modeling with Gaussian splatting}
\subsection{3D Gaussian Splatting}
3D Gaussian splatting~\cite{kerbl3Dgaussians} is an explicit scene representation that allows high-quality real-time rendering. The given scene is represented by a set of static 3D Gaussians, which are parameterized as follows: Gaussian center $x\in {\mathbb{R}^3}$, color $c\in {\mathbb{R}^3}$, opacity $\alpha\in {\mathbb{R}}$, spatial rotation in the form of quaternion $q\in {\mathbb{R}^4}$, and scaling factor $s\in {\mathbb{R}^3}$. Given these properties, the rendering process is represented as:
\begin{equation}
  I = Splatting(x, c, s, \alpha, q, r),
  \label{eq:splattingGA}
\end{equation}
where $I$ is the rendered image, $r$ is a set of query rays crossing the scene, and $Splatting(\cdot)$ is a differentiable rendering process. We refer readers to Kerbl et al.'s paper~\cite{kerbl3Dgaussians} for the details of Gaussian splatting. 



% \ZY{I would suggest move this part to the method part.}
% GaissianAvatar is a dynamic human generation model based on Gaussian splitting. Given a sequence of RGB images, this method utilizes fitted SMPLs and sampled points on its surface to obtain a pose-dependent feature map by a pose encoder. The pose-dependent features and a geometry feature are fed in a Gaussian decoder, which is employed to establish a functional mapping from the underlying geometry of the human form to diverse attributes of 3D Gaussians on the canonical surfaces. The parameter prediction process is articulated as follows:
% \begin{equation}
%   (\Delta x,c,s)=G_{\theta}(S+P),
%   \label{eq:gaussiandecoder}
% \end{equation}
%  where $G_{\theta}$ represents the Gaussian decoder, and $(S+P)$ is the multiplication of geometry feature S and pose feature P. Instead of optimizing all attributes of Gaussian, this decoder predicts 3D positional offset $\Delta{x} \in {\mathbb{R}^3}$, color $c\in\mathbb{R}^3$, and 3D scaling factor $ s\in\mathbb{R}^3$. To enhance geometry reconstruction accuracy, the opacity $\alpha$ and 3D rotation $q$ are set to fixed values of $1$ and $(1,0,0,0)$ respectively.
 
%  To render the canonical avatar in observation space, we seamlessly combine the Linear Blend Skinning function with the Gaussian Splatting~\cite{kerbl3Dgaussians} rendering process: 
% \begin{equation}
%   I_{\theta}=Splatting(x_o,Q,d),
%   \label{eq:splatting}
% \end{equation}
% \begin{equation}
%   x_o = T_{lbs}(x_c,p,w),
%   \label{eq:LBS}
% \end{equation}
% where $I_{\theta}$ represents the final rendered image, and the canonical Gaussian position $x_c$ is the sum of the initial position $x$ and the predicted offset $\Delta x$. The LBS function $T_{lbs}$ applies the SMPL skeleton pose $p$ and blending weights $w$ to deform $x_c$ into observation space as $x_o$. $Q$ denotes the remaining attributes of the Gaussians. With the rendering process, they can now reposition these canonical 3D Gaussians into the observation space.



\subsection{Score Distillation Sampling}
Score Distillation Sampling (SDS)~\cite{poole2022dreamfusion} builds a bridge between diffusion models and 3D representations. In SDS, the noised input is denoised in one time-step, and the difference between added noise and predicted noise is considered SDS loss, expressed as:

% \begin{equation}
%   \mathcal{L}_{SDS}(I_{\Phi}) \triangleq E_{t,\epsilon}[w(t)(\epsilon_{\phi}(z_t,y,t)-\epsilon)\frac{\partial I_{\Phi}}{\partial\Phi}],
%   \label{eq:SDSObserv}
% \end{equation}
\begin{equation}
    \mathcal{L}_{\text{SDS}}(I_{\Phi}) \triangleq \mathbb{E}_{t,\epsilon} \left[ w(t) \left( \epsilon_{\phi}(z_t, y, t) - \epsilon \right) \frac{\partial I_{\Phi}}{\partial \Phi} \right],
  \label{eq:SDSObservGA}
\end{equation}
where the input $I_{\Phi}$ represents a rendered image from a 3D representation, such as 3D Gaussians, with optimizable parameters $\Phi$. $\epsilon_{\phi}$ corresponds to the predicted noise of diffusion networks, which is produced by incorporating the noise image $z_t$ as input and conditioning it with a text or image $y$ at timestep $t$. The noise image $z_t$ is derived by introducing noise $\epsilon$ into $I_{\Phi}$ at timestep $t$. The loss is weighted by the diffusion scheduler $w(t)$. 
% \vspace{-3mm}
\section{The E-3DGS Method}\label{sec:Method} 
Our aim is to learn a 3D representation of a static scene using only a color event stream, where each pixel observes changes in brightness corresponding to one of the red, green, or blue channels according to a Bayer pattern, with known camera intrinsics $K_t~\in~\mathbb{R}^{3 \times 3}$, and noisy initial poses~$P_t~\in~\mathbb{R}^{3 \times 4}$, at reasonably high-frequency time steps indexed by $t$. 
Following 3DGS~\cite{3dgs}, we represent our scene by anisotropic 3D Gaussians. Our methodology comprises a technique to initialize Gaussians in the absence of a Structure from Motion (SfM) point cloud, adaptive event frame supervision of 3DGS, and a pose refinement module. 
An overview of our method is provided in Fig.~\ref{fig:methodology}.


Our E-3DGS method is not restricted to scenes of a certain size and can handle unbounded environments. It does not rely on any assumptions regarding the background color, type of camera motion, or speed. Thus, it ensures robust performance across a wide range of scenarios. 

\subsection{Event Stream Supervision} 

There are two main categories of approaches to learning 3D scene representations from event streams. 
Some apply the loss to single events~\cite{robust_enerf} based on Eq.~\eqref{eq:egm}. Others use the sum of events~\( E_{\x}(t_1,t_2) \) from Eq.~\eqref{eq:egm_sum}. We choose the second approach, as rasterization in 3DGS is well suited to efficiently render entire images rather than individual pixels. 

To optimize our Gaussian scene representation using event data, we can make a logical equivalence between the observed event stream and the scene renderings. 
To do so, we replace the true logarithmic intensities~\( L_{\x} \) in Eq.~\eqref{eq:egm_sum} with the rendered logarithmic intensities~\(\hat{L}_{\x} \) from our scene, and the times $\tau$ with the camera poses $P_t$ that were used to render the scene at the respective time steps. 
Following the approach used in~\cite{eventnerf}, the log difference is then point-wise multiplied with a Bayer filter $F$ to obtain the respective color channel. We can finally calculate the error between the logarithmic change from our model and the actual change observed from the event stream, and define the following per-pixel loss: 
\begin{equation}
    \begin{split}
    &\mathcal{L}_{\x}\left(t_1, t_2\right) = \\
    &\left\| 
    F \odot (\hat{L}_{\x}(P_{t_2}) - \hat{L}_{\x}(P_{t_1})) 
    - F \odot E_{\x}\left(t_1, t_2\right)\right\|_1, 
    \end{split}
    \label{eq:L_recon_per_pixel} 
\end{equation}
where ``$\odot$'' denotes pixelwise multiplication. 


\subsection{Frustum-Based Initialization}
\label{sec:frustum_init}

In the original 3DGS \cite{3dgs}, the Gaussians are initialized using a point cloud obtained from applying SfM on the input images. 
The authors also experimented with initializing the Gaussians at random locations within a cube. While this worked for them with a slight performance drop, it requires an assumption about the extent of the scene. 

Applying SfM directly to event streams is more challenging than RGB inputs \cite{Kim2016} and exploring this aspect is not the primary focus of this paper. 
In the absence of an SfM point cloud, we use the randomly initialized Gaussians and extend this approach to unbounded scenes. 
To this end, we initialize a specified number of Gaussians (on the order of \qty{d4}{}) in the frustum of each camera. 
This gives two benefits: 1) All the initialized Gaussians are within the observable area, and 2) We only need one loose assumption about the scene, which is the maximum depth $z_\mathrm{far}$. 


\subsection{Adaptive Event Window}\label{subsec:adaptive_window}

Rudnev et al.~\cite{eventnerf} demonstrated in EventNeRF that using a fixed event window duration results in suboptimal reconstruction. They find that larger windows are essential for capturing low-frequency color and structure, and smaller ones are essential for optimization of finer high-frequency details. While they randomly sampled the event window duration, a drawback is that it does not consider the camera speed and event rate, thus the sampled windows may contain too many or too few events.  
As our dataset features variable camera speeds, we improve upon this by sampling the number of events rather than the window duration.  
To achieve this, for each time step we randomly sample a target number of events from within the range $[N_\mathrm{min}, N_\mathrm{max}]$. 
Given a time step~$t$, we search for a previous time step~$t_s$ such that the number of events in the event frame $E(t_s, t)$ is approximately equal to the desired number. 

When determining $N_\mathrm{max}$, we find that for values where details and low-frequency structure are optimal, 3DGS tends to get unstable and sometimes prunes away Gaussians in homogeneous areas.
While this can be mitigated by choosing a much larger $N_\mathrm{max}$, this again deteriorates the details. 
Therefore, we propose a strategy to incorporate both, small and large windows. For each $t$, we choose two earlier time steps~$t_{s_1}$ and~$t_{s_2}$. The ranges for sampling the event counts for both are empirically chosen to be $[\frac{N_\mathrm{max}}{10}, N_\mathrm{max}]$ and $[\frac{N_{max}}{300}, \frac{N_\mathrm{max}}{30}]$. We then render frames from our model at times $t$, $t_{s_1}$ and $t_{s_2}$, and use two concurrent losses for the event windows $E_{\x}\left(t_{s_1}, t\right)$ and $E_{\x}\left(t_{s_2}, t\right)$. 

\subsection{As-Isotropic-As-Possible Regularization} 
\label{ssec:IsotropicReg} 

In 3DGS, Gaussians are unconstrained in the direction perpendicular to the image plane. 
This lack of constraint can result in elongated and overfitted Gaussians. 
And while they may appear correct from the training views, they introduce significant artifacts when rendered from novel views by manifesting as floaters and distortions of object surfaces. 
We also observe that the lack of multi-view consistency and tendency to overfit destabilize the pose refinement. 

To mitigate these issues, we draw inspiration from Gaussian Splatting SLAM~\cite{3dgsslam} and SplaTAM~\cite{splatam}, and apply isotropic regularization:
\begin{equation}
    \mathcal{L}_{\text{iso}} = \frac{1}{|\mathcal{G}|} \sum_{g \in \mathcal{G}} \left\| S_g - \bar{S}_g \right\|_1
    \label{eq:L_iso} \mathrm{\,,}
\end{equation}
where~$\mathcal{G}$ is the set of Gaussians visible in the image. Eq.~\eqref{eq:L_iso} imposes a soft constraint on the Gaussians to be as isotropic as possible.
We find that it helps to improve pose refinement, minimizes floaters and enhances generalizability. 

\subsection{Pose Refinement} 
\label{sec:pose_refinement}

To obtain the most accurate results, we allow the poses to be refined during optimization
by modeling the refined pose as $P'_t = P^e_t P_t$, where  $P^e_t$ is an error correction transform. 
Instead of directly optimizing~$P^e_t$ as a~$3 \times 3$ matrix, following Hempel et al.~\cite{6d_rotation} we represent it as $[r_1\,\, r_2\,\, T]$, where $r_1$ and $r_2$ represent two rotation vectors of the rotation matrix~$R = [r_1\,\, r_2\,\, r_3]$, while $T$ is the translation.
We can then obtain the~$P^e_t$ matrix from the representation using Gram-Schmidt orthogonalization (see details in Supplement~\ref{sec:supp_pose_refinement}), hence ensuring that during optimization, our error correction transform always represents a valid transformation matrix. 
$P^e_t$ is initialized to be the identity transform. Since the loss function from Eq.~\eqref{eq:L_recon_per_pixel} depends on the camera pose as well, it allows us to use the same loss to backpropagate and obtain gradients for pose refinement. 

As our goal is to refine the estimated noisy poses rather than perform SLAM, this training signal is sufficient for our needs. Moreover, we observe that poses tend to diverge with 3DGS due to the periodic opacity reset.
To combat this, we impose a soft constraint with an additional pose regularization, that encourages the matrices~$P^e_t$ to stay close to the identity matrix $I$:
\begin{equation}
    \mathcal{L}_{\text{pose}} = \| P^e_{t_{s_1}} - I \|_2 + \| P^e_{t_{s_2}} - I \|_2 + \| P^e_{t} - I \|_2
    \label{eq:L_pose} \mathrm{\,,}
\end{equation}
with all terms weighted equally. 


\subsection{Optimization}
\label{ssec:Optimization} 

Eq.~\eqref{eq:L_recon_per_pixel} defines the reconstruction loss per pixel for a single event frame. However, naively averaging these per-pixel losses over whole images leads to problems. For small event windows, most pixels have no events, which are not very informative but will then make up the majority of the loss. 
To address this, we compute separate averages of the losses for pixels with events~$\mathcal{X}_\text{evs}$ and pixels without events~$\mathcal{X}_\text{noevs}$. 
These averages are then scaled by the hyperparameter~$\alpha=0.3$ to obtain the complete weighted reconstruction loss:
\begin{equation}
    \begin{split}
        \mathcal{L}_{\text{recon}}\left(t_s, t\right) = \,\,&
        \frac{\alpha}{|\mathcal{X}_{\text{noevs}}|} \cdot 
        \left(\sum_{\x\in \mathcal{X}_{\text{noevs}}} \mathcal{L}_{\x}\left(t_s, t\right)\right) + \\
        + \,\,& \,\, \frac{1 - \alpha}{|\mathcal{X}_{\text{evs}}|} \,\,\, \cdot 
        \left(\sum_{\x\in \mathcal{X}_{\text{evs}}} \mathcal{L}_{\x}\left(t_s, t\right)\right). 
    \end{split}
    \label{eq:L_recon}
\end{equation}
To obtain the final loss, we take a weighted sum of the reconstruction losses for the two event windows from Sec.~\ref{subsec:adaptive_window} along with the isotropic and pose regularization: 
\begin{equation}
    \begin{split}
        \mathcal{L} =\,\,\,\, & 
        \lambda_1 \mathcal{L}_{\text{recon}}\left(t_{s_1}, t\right) \,\,+  \,\,
        \lambda_2 \mathcal{L}_{\text{recon}}\left(t_{s_2}, t\right)  \\&
        +\,\, \lambda_\text{iso} \mathcal{L}_{\text{iso}} \,\,+ \,\,
        \lambda_\text{pose} \mathcal{L}_{\text{pose}}
    \end{split}
    \label{eq:loss} \mathrm{\,,}
\end{equation}
where $\lambda_1$, $\lambda_2$ and $\lambda_{\text{iso}}$ are hyper-parameters. In our experiments, we use  $\lambda_1=\lambda_2=0.65$, and $\lambda_{\text{iso}}$ is set to $10$ initially and reduced to $1$ after $\qty{d4}{}$ iterations. 






\section{Experimental Evaluation}\label{section:experiments}
We already achieved our primary objective of deriving time-series-specific subsampling guarantees for DP-SGD adapted to forecasting.
Our main goal for this section is to investigate the trade-offs we discovered in discussing these guarantees.
In addition, we train common probabilistic forecasting architectures on standard datasets to verify the feasibility of training deep differentially private forecasting models while retaining meaningful utility.
The full experimental setup  is described in~\cref{appendix:experimental_setup}.
%An implementation will be made available upon publication.

\subsection{Trade-Offs in Structured Subsampling}

\begin{figure}
    \vskip 0.2in
    \centering
        \includegraphics[width=0.99\linewidth]{figures/experiments/eval_pld_deterministic_vs_random_top_level/daily_20_32_main.pdf}
        \vskip -0.3cm
        \caption{Top-level deterministic iteration (\cref{theorem:deterministic_top_level_wr}) vs top-level WOR sampling (\cref{theorem:wor_top_level_wr}) for $\numinstances=1$.
        Sampling is more private despite requiring more compositions.}
        \label{fig:deterministic_vs_random_top_level_daily_main}
    \vskip -0.2in
\end{figure}




For the following experiments, we assume that we have $N=320$ sequences, batch size $\batchsize = 32$, and noise scale $\sigma = 1$.
We further assume $L=10  (L_F + L_C) + L_F - 1$, so that 
the chance of bottom-level sampling a subsequence containing any specific element is 
$r=0.1$ when choosing $\numinstances = 1$ as the number of subsequences.
In~\cref{appendix:extra_experiments_eval_pld}, we repeat all experiments with a wider range of parameters.
All results are consistent with the ones shown here.

\textbf{Number of Subsequences $\bm{\numinstances}$.}
Let us begin with a trade-off inherent to bi-level subsampling:
We can achieve the same batch size $\batchsize$ with different $\numinstances$, each
leading to different top- and bottom-level amplification.
We claim that $\numinstances = 1$ (i.e., maximum bottom-level amplification) is preferable.
For a fair comparison, we compare our provably tight guarantee for $\numinstances=1$ (\cref{theorem:wor_top_level_wr})
with optimistic lower bounds for $\numinstances > 1$ (\cref{theorem:wor_top_wr_bottom_upper})
instead of our sound upper bounds (\cref{theorem:wor_top_level_wr_general}), i.e.,
we make the competitors stronger.
As shown in~\cref{fig:monotonicity_daily_main}, $\numinstances = 1$ only has smaller $\delta(\epsilon)$ for $\epsilon \geq 10^{-1}$ when considering a single training step.
However, after $100$-fold composition, $\numinstances = 1$ achieves smaller $\delta(\epsilon)$ even in $[10^{-3}, 10^{-1}]$ (see~\cref{fig:monotonicity_composed_daily_main}).
Our explanation is that $\numinstances > 1$ results in larger $\delta(\epsilon)$ for large $\epsilon$, i.e., is more likely to have a large privacy loss.
Because the privacy loss of a composed mechanism is the sum of component privacy losses~\cite{sommer2018privacy}, this is problematic when performing multiple training steps.
We shall thus later use $\numinstances=1$ for training.

%Intuitively, $\delta(\epsilon)$ can be interpreted as the probability that the log-likelihood ratio of $M_x$ and $M_{x'}$ (``privacy loss'') exceeds $\epsilon$.\footnote{For the formal relation between privay loss and privacy profiles, see~\cref{lemma:profile_from_pld} taken from~\cite{balle2018improving}}


\textbf{Step- vs Epoch-Level Accounting.}
Next, we show the benefit of top-level sampling sequences (\cref{theorem:wor_top_level_wr}) instead of deterministically iterating over them (\cref{theorem:deterministic_top_level_wr}), even though we risk privacy leakage at every training step.
For our parameterization and $\numinstances=1$, top-level sampling with replacement requires $10$ compositions per epoch.
As shown in~\cref{fig:deterministic_vs_random_top_level_daily_main}, the resultant epoch-level profile is nevertheless smaller, and remains so after $10$ epochs.
This is consistent with any work on DP-SGD (e.g., \cite{abadi2016deep}) that uses subsampling instead of deterministic iteration.

\textbf{Epoch Privacy vs Length.} In~\cref{appendix:extra_experiments_epoch_length} we additionally explore the fact that, if we wanted to use deterministic top-level iteration, 
the number of subsequences 
$\numinstances$ would affect epoch length.
As expected, we observe that composing many private mechanisms ($\numinstances=1$) is preferable to composing few much less private mechanisms ($\numinstances > 1$) 
when considering a fixed number of training steps.

\begin{figure}
    \vskip 0.2in
    \centering
        \includegraphics[width=0.99\linewidth]{figures/experiments/eval_pld_label_noise/daily_30_32_main.pdf}
        \vskip -0.3cm
        \caption{Varying label noise $\sigma_F$ for top-level WOR and bottom-level WR  (\cref{theorem:data_augmentation_general}) with $\sigma_C = 0, \numinstances=1$.
        Increasing $\sigma_F$ is equivalent to decreasing forecast length.
        }
        \label{fig:label_noise_daily_main}
    \vskip -0.2in
\end{figure}

\textbf{Amplification by Label Perturbation.}
Finally, because the way in which adding Gaussian noise to the context and/or forecast window 
amplifies privacy (\cref{theorem:data_augmentation_general}) 
may be somewhat opaque, let us consider top-level sampling without replacement, bottom-level sampling with replacement,
$\numinstances=1$, $\sigma_C=0$, and varying label noise standard deviations $\sigma_F$. 
As shown in~\cref{fig:label_noise_daily_main}, increasing $\sigma_F$ has the same effect as letting the forecast length $L_C$ go to zero, i.e., eliminates the risk of leaking private information if it appears in the forecast window.
Of course, this data augmentation 
will have an effect on model utility, which we investigate shortly.

\begin{figure*}
\centering
\vskip 0.2in
    \begin{subfigure}{0.49\textwidth}
        \includegraphics[]{figures/experiments/eval_pld_monotonicity_composed/daily_20_32_1_main.pdf}
        \caption{Training step $1$}\label{fig:monotonicity_daily_main}
    \end{subfigure}
    \hfill
    \begin{subfigure}{0.49\textwidth}
        \includegraphics[]{figures/experiments/eval_pld_monotonicity_composed/daily_20_32_100_main.pdf}
        \caption{Training step $100$}\label{fig:monotonicity_composed_daily_main}
    \end{subfigure}\caption{
    Top-level WOR and bottom-level WR sampling under varying number of subsequences.
    Under composition, even optimistic lower bounds (\cref{theorem:wor_top_wr_bottom_upper}) 
    indicate worse privacy for $\numinstances > 1$ than 
    our tight upper bound for $\numinstances=1$ (\cref{theorem:wor_top_level_wr}).}
    \label{fig:monotonicity_daily_main_container}
\vskip -0.2in
\end{figure*}


\subsection{Application to Probabilistic Forecasting}
While the contribution of our work lies in formally analyzing the privacy of DP-SGD adapted to forecasting, 
training models with this algorithm can serve as a sanity-check to verify that the guarantees are sufficiently strong to retain meaningful utility under non-trivial privacy budgets.


\begin{table}[b]
\vskip -0.38cm
\caption{Average CRPS on \texttt{traffic} for $\delta=10^{-7}$. Seasonal, AutoETS, and models with $\epsilon=\infty$ are without noise.}
\label{table:1_event_training_traffic_main}
\vskip 0.18cm
\begin{center}
\begin{small}
\begin{sc}
\begin{tabular}{lcccc}
\toprule
Model & $\epsilon = 0.5$ & $\epsilon = 1$ & $\epsilon = 2$ &  $\epsilon = \infty$ \\
\midrule
SimpleFF & $0.207$ & $0.195$ & $0.193$ & $0.136$ \\ 
DeepAR & $\mathbf{0.157}$ & $\mathbf{0.145}$ & $\mathbf{0.142}$ & $\mathbf{0.124}$ \\
iTransf. & $0.211$ & $0.193$ & $0.188$ & $0.135$ \\
DLinear & $0.204$ & $0.192$ & $0.188$ & $0.140$ \\
\midrule
Seasonal   & $0.251$ & $0.251$ & $0.251$ & $0.251$\\
AutoETS   & $0.407$ & $0.407$ & $0.407$ & $0.407$\\
\bottomrule
\end{tabular}
\end{sc}
\end{small}
\end{center}
\vskip -0.1in
\end{table}

\textbf{Datasets, Models, and Metrics.}
We consider three standard benchmarks: \texttt{traffic}, \texttt{electricity}, and \texttt{solar\_10\_minutes} as used in~\cite{Lai2018modeling}.
We further consider four common architectures: 
A two-layer feed-forward neural network (``SimpleFeedForward''), a recurrent neural network (``DeepAR''~\cite{salinas2020deepar}),
an encoder-only transformer (``iTransformer''~\cite{liu2024itransformer}), and a refined feed-forward network proposed to compete with attention-based models (``DLinear''~\cite{zeng2023transformers}).
We let these architectures parameterize elementwise $t$-distributions to obtain probabilistic forecasts.
We measure the quality of these probabilistic forecasts using continuous ranked probability scores (CRPS), which we approximate via mean weighted quantile losses (details in~\cref{appendix:metrics}).
As a reference for what constitutes ``meaningful utility'', we compare against seasonal na\"{i}ve forecasting and exponential smoothing (``AutoETS'') without introducing any noise.
All experiments are repeated with $5$ random seeds.


\textbf{Event-Level Privacy.} \cref{table:1_event_training_traffic_main} shows CRPS of all models on the \texttt{traffic} test set 
when setting $\delta=10^{-7}$, and training on the training set until reaching a pre-specified $\epsilon$
with $1$-event-level privacy. For the other datasets and standard deviations, see~\cref{appendix:privacy_utility_tradeoff_event_level_privacy}.
The column $\epsilon=\infty$ indicates non-DP training.
As can be seen, models can retain much of their utility and outperform the baselines, even for $\epsilon \leq 1$ which is generally considered a small privacy budget~\cite{ponomareva2023dp}.
For instance, the average CRPS of DeepAR on the traffic dataset is $0.124$ with non-DP training and $0.157$ for $\epsilon=0.5$.
Note that, since all models are trained using  our tight privacy analysis,
which specific model performs best  on which specific dataset is orthogonal to our contribution. 

\textbf{Other results.}
In~\cref{appendix:privacy_utility_tradeoff_user_level_privacy} we additionally train with $w$-event and $w$-user privacy.
In~\cref{appendix:privacy_utility_tradeoff_label_privacy}, we demonstrate that label perturbations can offer an improved privacy--utility trade-off. 
All results confirm that our guarantees for DP-SGD adapted to forecasting are strong enough to enable provably private training while retaining utility.

% !TeX root = ../main.tex

\section{Discussion}
In this chapter, we presented a novel contact-free volumetric haptic feedback device. This device uses a symmetric electromagnet combined with a dipole magnet model and a simple control law to deliver dynamically adjustable forces onto a handheld tool, such as a stylus. The tool only requires an embedded permanent magnet, allowing it to be completely untethered. Despite being contact-free, the force is grounded via the electromagnet, enabling the user to feel relatively large forces.

While our proposed method offers many advantages, it also has some drawbacks. Heat generation limits the number of interactions possible within a certain time frame. Specifically, when operating at full power, continuous interaction is limited to 5 seconds.

It is also important to note that the interaction between magnets involves both forces and torques. In this work, we focused on controlling the three force components via the 3 degrees of freedom (DoFs) of the electromagnet, allowing the torque values to adapt accordingly. However, the same procedure can be applied to control a specific torque map, leaving the force values unconstrained, or to manage a combination of force and torque. In future work, we aim to explore the dynamic capabilities of our approach, including advanced control schemes to continuously shape the force map.

Finally, our current control method relies on knowing the location of the tool, achieved through external cameras for optical tracking. However, this setup is cumbersome, expensive, and requires line-of-sight. In the next chapter, we will address this limitation by tracking the permanent magnet embedded in the tool using Hall sensors and a gradient-based method.

\appendices


\section{Training Details}
\label{sec:details}
This section provides more details about the implementation and training of our method.

\subsection{Stage I}
\subsubsection{Canonical Initialization}
We unwrap the T-pose body onto a UV map, where each pixel stores a 3D position vector. The positional UV map, with a resolution of $(512 \times 512 \times 3)$, is used to initialize Gaussians in the canonical space, ensuring proper alignment with the body’s structure. Additionally, a downsampled $(128 \times 128 \times 3)$ version of the positional UV map serves as input to the Gaussian decoder, aiding in reconstructing and refining the 3D representation. 
Furthermore, we use blend
\subsubsection{Training}
The training objectives in this stage focus on image losses and optimizations about Gaussian parameters.We set weights for each objective as $\lambda_{rgb} = 0.8$, $\lambda_{n} = 0.8$, $\lambda_{ssim} = 0.2$, $\lambda_{lpips} = 0.2$, $\lambda_{\Delta x} = 0.85$, $\lambda_{s} = 0.03$, $\lambda_{S} = 1$. 

\subsubsection{Pose Optimization} Our method leverages pose optimization from GaussianAvatar~\cite{hu2023gaussianavatar} for the In-the-wild dataset as a correction for fitted SMPL~\cite{SMPL:2015} pose parameters. We have omitted this functionality for the ZJU-Mocap dataset, as their ground truth pose is accurate. However, GausianAvatar keeps optimizing pose parameters for ZJU-Mocap dataset, which leads to inaccurate poses, especially for invisible parts. Please check the accompanying video results for more details. 
\subsection{Stage II} 
\subsubsection{Dual-space Optimization}
In this stage, we apply Dual-space optimization on top of visible appearance reconstruction to predict the invisible appearance. During training, each epoch is divided into three parts: 50\% for given view training and 50\% for Dual-space optimization. In Dual-space optimization, the weight of canonical optimization is treated as a hyperparameter, defaulting to 50\%. the The fine-tuning losses are added upon $\mathcal{L}_{StageI}$. We set $\lambda_{p} = 0.5 $ and  $\lambda_{SDS} = 0.3 $ initially. 

\subsubsection{Progressive Training}
We design a progressive training strategy in this stage, gradually diminishing the weight of SDS loss. This strategy is employed to enhance further the effectiveness and efficiency of the visible appearance reconstruction. Based on this strategy, the $\lambda_{SDS}$ is reduced gradually by following:
\begin{equation}
\lambda_{SDS}(t) = \lambda_{SDS,0} \cdot \frac{1}{2^{\lfloor \frac{t - t_{\text{0}}}{k} \rfloor}}
  \label{eq:prog.train}
\end{equation}
where $t$ and $t_{0}$ are the current epoch and starting epoch respectively, $k$ is the interval step of changing the weight. We set $t_{0} = 100$ and $k = 100$.

\subsection{Resolution}
The video resolution for the ZJU-Mocap (revised)\cite{liu2023zero1to3} and Monocap datasets is consistently maintained at $1024\times1024$ pixels, while MVHumanNet\cite{xiong2024mvhumannet} has a resolution of $2048\times1500$ pixels. For videos collected from the internet, the resolution ranges from 720p to 1080p. However, in Stage II, Zero123 only accepts $256\times256$ as input. Therefore, for SDS loss calculation, we crop the ground truth images based on their masks and resize them to $256\times256$.

 


% \subsection{Potential Social Impact} The creation and manipulation of highly realistic avatars raise ethical questions regarding privacy, consent, and misrepresentation. There is a risk of misuse, such as creating deceptive content or impersonating individuals without their permission. The widespread availability of realistic avatars could complicate identity verification processes in online environments. It may become more challenging to distinguish between real individuals and avatar representations, potentially undermining trust and security.
\begin{figure}[!t]
  \centering
  \includegraphics[width=1\linewidth]{Figures/Appendix_2.2.pdf}
    % \vspace{-1mm}
  \caption{Qualitative comparison results with SIFU~\cite{Zhang2024SIFU} and SITH~\cite{ho2024sith}.}
    % \vspace{-3mm}
  \label{fig:SIFU}
\end{figure}
\section{More Experiments}



\subsection{Comparison with Image-based Methods \label{resultImagebased}}

In this section, we compare our method with SIFU~\cite{Zhang2024SIFU}, SITH~\cite{ho2024sith}, and ELICIT~\cite{huang2022elicit}, all of which are single-image reconstruction techniques designed to synthesize unseen parts of human avatars.

SIFU proposes an approach to reconstruct clothed human avatars from single images. Qualitatively, as shown in Fig.~\ref{fig:SIFU}, this method can reconstruct decent geometry but fails to synthesize the texture of unseen parts of humans. SITH, similar to SIFU, is a method for single-image reconstruction. SITH can predict the texture of unseen parts of humans, but their generated textures contain unrealistic artifacts.
\begin{figure}[!t]
  \centering
  \includegraphics[width=1\linewidth]{Figures/Appendix_1.1.pdf}
    % \vspace{-2mm}
  \caption{Qualitative comparison results with ELICIT~\cite{huang2022elicit} on novel poses.}
    % \vspace{-2mm}
  \label{fig:ELICIT}
\end{figure}

\begin{table}[!t]
\centering
\renewcommand{\arraystretch}{1.5}

\begin{tabular}{c|c|ccc}
\hline
Dataset                    & Method & PSNR$\uparrow$&SSIM$\uparrow$&LPIPS$\downarrow$  \\ \hline
\multirow{3}{*}{MVHumanNet}   & SIFU   &19.29&0.9486&0.0706\\
                            & SITH   &19.68&0.9462&0.0699\\
                        &\textbf{Ours} & \textbf{20.98} & \textbf{0.9517} & \textbf{0.0553} \\ \hline  
\multirow{3}{*}{Monocap}   & SIFU   &18.96&0.9406&0.0659\\
                            & SITH   &19.06&0.9428&0.0.0673\\
                        &\textbf{Ours} & \textbf{21.16} & \textbf{0.9532} & \textbf{0.0549} \\ \hline  
\multirow{2}{*}{ZJU-Mocap(revised)} & ELICIT &19.23&0.9456&0.0.0689\\
                           &\textbf{Ours} & \textbf{20.82} & \textbf{0.9552} & \textbf{0.0569}  \\ \hline
\end{tabular}
\caption{Quantitative evaluation on MVHumanNet, ZJU-Mocap(revised), and Monocap datasets.}
\vspace{-3mm}
  \label{tab:exp}
\end{table}
ELICIT is a generative model that takes one image and a motion sequence as input to generate an animatable avatar. Qualitative results are shown in Fig.~\ref{fig:ELICIT}. For a fair comparison, since our method takes an image sequence as input, we are comparing the quality by synthesizing a novel pose that is not included in our inputs. Even though ELICIT can predict the unseen parts of humans, it shows blurred edges and floating artifacts while applying motions. Because only one image is used as input for ELICIT, the texture cannot be adapted to novel poses dynamically. In contrast, our method associates texture to different body parts across frames and can predict the correct texture for unseen parts robustly.

In Tab.~\ref{tab:exp}, we present the quantitative evaluation results. SIFU and SITH were tested on the Monocap dataset, while ELICIT was evaluated on the ZJU-Mocap(revised) dataset. The results demonstrate that our method consistently achieves superior performance compared to the state-of-the-art approaches, underscoring its efficacy and robustness.



% }



% \ifCLASSOPTIONcompsoc

% %  \section*{Acknowledgments}
% \else


%   \section*{Acknowledgment}
% \fi
% % The authors would like to thank the anonymous reviewers for their constructive comments and suggestions. Zhang, Li, and Guo were partially supported by a grant from National Science Foundation (2007661) and research gifts from Samsung Research America. Zeng was partially supported by NSFC (No.62072382), and Fundamental Research Funds for the Central Universities, China (No.20720190003). The opinions expressed are solely those of the authors, and do not necessarily represent those of the funding agencies.

% \ifCLASSOPTIONcaptionsoff
%   \newpage
% \fi



% \balance
\bibliographystyle{IEEEtran}
\bibliography{TVCG}
\vspace{-15mm}
\begin{IEEEbiography}[{\includegraphics[width=1in,height=1.25in,trim=70 70 80 70,clip,keepaspectratio]{Authors/zilong.png}}]{Zilong~Wang} is a Ph.D. candidate at the University of Texas at Dallas, supervised by Prof. Xiaohu Guo. He received his B.S. degree in software engineering in 2020 from Northwest University(China) and M.S. degree in software engineering in 2022 from the University of Texas at Dallas. 
His research interests include human reconstruction and animation, computer graphics, computer vision, and deep learning.
\end{IEEEbiography}
\vspace{-15mm}
\begin{IEEEbiography}[{\includegraphics[width=1in,height=1.25in,trim=70 40 80 70,clip,keepaspectratio]{Authors/zhiyang.png}}]{Zhiyang~(Frank)~Dou} is a Ph.D. candidate in the Computer Graphics Group at The University of Hong Kong, under the supervision of Prof. Wenping Wang and Prof. Taku Komura. %Zhiyang earned his B.Eng. degree with honors from Shandong University, where he was advised by Prof. Shiqing Xin.
Zhiyang’s research focuses on shape recovery and generation, character animation, geometric modeling, and the analysis of human behavior, emphasizing the intersection of artificial intelligence, computer graphics and computer vision.
\end{IEEEbiography}
\vspace{-15mm}
\begin{IEEEbiography}[{\includegraphics[width=1in,height=1.25in,trim=40 40 40 40,clip,keepaspectratio]{Authors/liuyuan.png}}]{Yuan~Liu} is an assistant professor at HKUST. He received his PhD degree in the University of Hong Kong in 2024. His research mainly concentrates on 3D vision and graphics. I currently work on topics about 3D AIGC including neural rendering, neural representations, and 3D generative models.
\end{IEEEbiography}
\vspace{-15mm}
\begin{IEEEbiography}[{\includegraphics[width=1in,height=1.25in,clip,keepaspectratio]{Authors/lincheng.jpg}}]{Cheng~Lin} received his Ph.D. from The University of Hong Kong (HKU), advised by Prof. Wenping Wang. He visited the Visual Computing Group at Technical University of Munich (TUM), advised by Prof. Matthias Nießner. Before that, he completed his B.E. degree at Shandong University. His research interests include geometric modeling, 3D vision, shape analysis, and computer graphics.
\end{IEEEbiography}
\vspace{-15mm}
\begin{IEEEbiography}[{\includegraphics[width=1in,height=1.25in,clip,keepaspectratio]{Authors/xiaodong.jpg}}]{Xiao~Dong} is an assistant Professor in the Department of Computer Science, BNU-HKBU United International College. She received the BS and PhD degrees in computer science and technology from Xiamen University, in 2013 and 2022, respectively. Her research interests include computer graphics, computer vision and deep learning.
\end{IEEEbiography}
\vspace{-15mm}
\begin{IEEEbiography}[{\includegraphics[width=1in,height=1.25in,clip,keepaspectratio]{Authors/yunhui.jpg}}]{Yunhui~Guo} is an assistant professor in the Department of Computer Science at the University of Texas at Dallas. Previously, he was a postdoctoral researcher at UC Berkeley/ICSI. He earned his PhD in Computer Science from the University of California, San Diego. His research interests include machine learning and computer vision, with a focus on developing intelligent agents that can continuously learn, dynamically adapt to evolving environments without forgetting previously acquired knowledge, and repurpose existing knowledge to handle novel scenarios.
\end{IEEEbiography}
\vspace{-15mm}
\begin{IEEEbiography}[{\includegraphics[width=1in,height=1.25in,clip,keepaspectratio]{Authors/chenxu.jpg}}]{Chenxu~Zhang} is a Research Scientist at the Intelligent Creation Lab, ByteDance. He completed his Ph.D. degree in Computer Science from the University of Texas at Dallas in 2023. He received his B.S. degree in Software Engineering in 2015 and M.S. degree in Computer Science in 2018, both from Beihang University. His research interests include computer graphics, computer vision, and deep learning.
\end{IEEEbiography}
\vspace{-15mm}
\begin{IEEEbiography}[{\includegraphics[width=1in,height=1.25in,clip,keepaspectratio]{Authors/lixin.png}}]{Xin~Li} (Senior Member, IEEE) received the B.E. degree in computer science from the University of Science and Technology of China in 2003 and the M.S. and Ph.D. degrees in computer science from the State University of New York at Stony Brook in 2005 and 2008, respectively. He is currently a Professor with the Section of Visual Computing and Creative Media, School of Performance, Visualization, and Fine Arts, Texas A\&M University. His
research interests include geometric and visual data computing, processing, and understanding, computer vision, and virtual reality.
\end{IEEEbiography}
\vspace{-10mm}
\begin{IEEEbiography}[{\includegraphics[width=1in,height=1.25in,clip,keepaspectratio]{Authors/wenping.png}}]{Wenping~Wang} (Fellow, IEEE) received the
Ph.D. degree in computer science from the University of Alberta. He is a Professor of computer science at Texas A\&M University. His research interests include computer graphics, visualization, computer vision, robotics, medical image processing, and geometric computing. He has been an journal associate editor of ACM Transactions on Graphics, IEEE Transactions on Visualization and Computer Graphics, Computer Aided Geometric Design, and Computer Graphics Forum (CGF). He has chaired a number of international conferences, including Pacific Graphics, ACM Symposium on Physical and Solid Modeling (SPM), SIGGRAPH and SIGGRAPH Asia. Prof. Wang received the John Gregory Memorial Award for his contributions to geometric modeling.
\end{IEEEbiography}
\vspace{-15mm}
\begin{IEEEbiography}[{\includegraphics[width=1in,height=1.25in,clip,keepaspectratio]{Authors/xiaohu.jpg}}]{Xiaohu~Guo} is a Full Professor of Computer Science at the University of Texas at Dallas. He received his Ph.D degree in Computer Science from Stony Brook University, and a B.S degree in Computer Science from the University of Science and Technology of China. His research interests include computer graphics, computer vision, medical imaging, with an emphasis on geometric modeling and processing, as well as body and face modeling problems. He received the prestigious NSF CAREER Award in 2012 and SIGGRAPH 2023 Best Paper Award. He has been serving on the journal editorial boards of IEEE TVCG, TMM, TCSVT, GMOD, CAVW, and on the executive committee of Solid Modeling Association.
\end{IEEEbiography}



\end{document}


