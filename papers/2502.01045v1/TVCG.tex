
\documentclass[10pt,journal,compsoc]{IEEEtran}

\ifCLASSOPTIONcompsoc
  % IEEE Computer Society needs nocompress option
  % requires cite.sty v4.0 or later (November 2003)
  \usepackage[nocompress]{cite}
\else
  % normal IEEE
  \usepackage{cite}
\fi

% *** GRAPHICS RELATED PACKAGES ***
%
\ifCLASSINFOpdf
   \usepackage[pdftex]{graphicx}
  % declare the path(s) where your graphic files are
   \graphicspath{{../pdf/}{../jpeg/}}
  % and their extensions so you won't have to specify these with
  % every instance of \includegraphics
   \DeclareGraphicsExtensions{.pdf,.jpeg,.png,.jpg}
\else
  % or other class option (dvipsone, dvipdf, if not using dvips). graphicx
  % will default to the driver specified in the system graphics.cfg if no
  % driver is specified.
   \usepackage[dvips]{graphicx}
  % declare the path(s) where your graphic files are
   \graphicspath{{../eps/}}
  % and their extensions so you won't have to specify these with
  % every instance of \includegraphics
   \DeclareGraphicsExtensions{.eps}
\fi




\usepackage{hyperref}
\usepackage{amsmath}
\usepackage{algorithm}
\usepackage{algorithmic}
\usepackage{amsfonts}
\usepackage{wrapfig}


% *** SUBFIGURE PACKAGES ***
% \ifCLASSOPTIONcompsoc
%   \usepackage[caption=false,font=footnotesize,labelfont=sf,textfont=sf]{subfig}
% \else
%   \usepackage[caption=false,font=footnotesize]{subfig}
% \fi
\usepackage{subcaption}
\usepackage{stfloats}

\usepackage{amssymb}

\usepackage{cleveref}
\usepackage{xcolor}
\usepackage{adjustbox}
\usepackage{xspace}


\usepackage{multirow}
\usepackage{xcolor}
\definecolor{myorange}{RGB}{236,166,128} % Define a custom orange color
\definecolor{myblue}{RGB}{39,131,177} % Define a custom orange color
\usepackage{tikz} % For drawing lines
\newcommand{\name}{\textit{WonderHuman}\xspace}
\newcommand{\TODO}[1]{{\color{red}\textbf{TODO: #1}}}
\newcommand{\ZL}[1]{{\color{blue}{#1}}}



% \def\ie{{\em i.e.}}
% \def\eg{{\em e.g.}}
% \def\etal{{\em et al.}}
\def\up{{\bf$\uparrow $} }
\def\down{{\bf$\downarrow $} }
% \newcommand{\figref}[1]{Fig. \ref{#1}}
% \newcommand{\tabref}[1]{Tab. \ref{#1}}
% \newcommand{\equref}[1]{(\ref{#1})}
% \newcommand{\secref}[1]{Section \ref{#1}}
% \newcommand{\algref}[1]{Alg. \ref{#1}}
% \newcommand{\myPara}[1]{\vspace{.05in}\noindent\textbf{#1}}
% \newcommand{\todo}[1]{\textcolor{red}{\bf [#1]}}
% \newcommand{\bl}[1]{\textbf{#1}}
% \newcommand{\mc}[1]{\mathcal{#1}}
% \newcommand{\mb}[1]{\mathbb{#1}}
% \newcommand{\tabincell}[2]{\begin{tabular}{@{}#1@{}}#2\end{tabular}}
% \newcommand{\bul}[1]{\underline{\textbf{#1}}}
% %\newcommand{\bm}[1]{\mbox{\boldmath{$#1$}}}
% \newcommand{\br}[1]{\bm{\mathrm{#1}}}

% \hyphenation{op-tical net-works semi-conduc-tor}
\usepackage{ragged2e}
% \usepackage{booktabs}
% \usepackage{color}
% \usepackage{multirow}
% \usepackage{bm}
% \usepackage{bbm}
% \usepackage{balance}
% \renewcommand{\algorithmicrequire}{ \textbf{Input:}}      %Use Input in the format of Algorithm
% \renewcommand{\algorithmicensure}{ \textbf{Output:}}     %UseOutput in the format of Algorithm

\newcommand{\ZY}[1]{{\color{red}#1}}


\begin{document}

\title{
WonderHuman: Hallucinating Unseen Parts in Dynamic 3D Human Reconstruction
}


\author{Zilong~Wang,~Zhiyang~Dou,~Yuan~Liu,~Cheng~Lin,~Xiao~Dong,~Yunhui~Guo,~Chenxu~Zhang,\\~Xin~Li,~Wenping~Wang,~Xiaohu~Guo% <-this % stops a space
\IEEEcompsocitemizethanks{\IEEEcompsocthanksitem Z. Wang, Y. Guo, C. Zhang and X. Guo are with the Department of Computer Science, The University of Texas at Dallas, Richardson, Texas.
\IEEEcompsocthanksitem Z. Dou and C. Lin are with the Computer Graphics Group, The University of Hong Kong, Pokfulam, Hong Kong.
\IEEEcompsocthanksitem Y. Liu is with the School of Engineering, The Hong Kong University of Science and Technology, Clear Water Bay, Hong Kong.
\IEEEcompsocthanksitem X. Dong is with the Department of Computer Science, BNU-HKBU United International College, Zhuhai, China.
\IEEEcompsocthanksitem X. Li and W. Wang are with the Department of Computer Science \& Engineering, Texas A\&M University, College Station, Texas.

\IEEEcompsocthanksitem Corresponding Author: X. Guo, Email: xguo@utdallas.edu
}
}

% \markboth{Submission to IEEE Transactions on Visualization and Computer Graphics}%
% {Z. Wang \MakeLowercase{\textit{et al.}}:}

\IEEEtitleabstractindextext{%
\begin{abstract}
\justifying In this paper, we present \textit{WonderHuman} to reconstruct dynamic human avatars from a monocular video for high-fidelity novel view synthesis. 
Previous dynamic human avatar reconstruction methods typically require the input video to have full coverage of the observed human body. However, in daily practice, one typically has access to limited viewpoints, such as monocular front-view videos, making it a cumbersome task for previous methods to reconstruct the unseen parts of the human avatar. To tackle the issue, we present \textit{WonderHuman}, which leverages 2D generative diffusion model priors to achieve high-quality, photorealistic reconstructions of dynamic human avatars from monocular videos, including accurate rendering of unseen body parts. Our approach introduces a Dual-Space Optimization technique, applying Score Distillation Sampling (SDS) in both canonical and observation spaces to ensure visual consistency and enhance realism in dynamic human reconstruction. Additionally, we present a View Selection strategy and Pose Feature Injection to enforce the consistency between SDS predictions and observed data, ensuring pose-dependent effects and higher fidelity in the reconstructed avatar. In the experiments, our method achieves SOTA performance in producing photorealistic renderings from the given monocular video, particularly for those challenging unseen parts. The project page and source code can be found at \url{https://wyiguanw.github.io/WonderHuman/}.%The code will be made publicly accessible upon publication.

\end{abstract}

\begin{IEEEkeywords}
Monocular Video, 3D Gaussian Splatting, Human Unseen Part Reconstruction, Diffusion, Score Distillation Sampling.
\end{IEEEkeywords}}

\maketitle


\IEEEdisplaynontitleabstractindextext

\IEEEpeerreviewmaketitle


\begin{figure}[ht]
    \centering
    \includegraphics[width=0.8\linewidth]{graphs/greater_than_naive.pdf}
    \vspace{0.5cm}
    \includegraphics[width=0.8\linewidth]{graphs/p1_bottom.png}
    \vspace{-5pt}
    \caption{\textcolor{positional}{Positional} vs.\ \textcolor{nonpositional}{non-positional} circuits. In a \textcolor{nonpositional}{non-positional} circuit, the same edges must be included at all positions. A \textcolor{positional}{positional} circuit can distinguish between the same edge at different positions. This specificity yields better trade-offs between circuit size and faithfulness. It can also increase both precision and recall.}
    \label{fig:p1}
    \vspace{-5pt}
\end{figure}

\section{Introduction}

\looseness=-1
A primary goal of interpretability research is to characterize the internal mechanisms in language models (LMs) and other NLP models. 
A core approach in this area is \textbf{circuit discovery}---identifying the minimal subgraph within the model's computation graph that performs a specific task \citep{olah2021framework,olah-mech}.
Typically, the nodes of a circuit represent model components (e.g., attention heads, neurons, or layers).
While manual circuit discovery methods can yield position-specific insights \citep{wanginterpretability,goldowskydill2023localizingmodelbehaviorpath}, \emph{automatic methods often overlook positional information}, treating components as uniformly relevant across all input token positions \citep{conmytowards,syed2023attribution}. 
For instance, if an attention head is included in a circuit, it is assumed to contribute equally to the computation for every position in the input sequence.
The assumption that circuits are position-invariant ignores the fact that different positions often require distinct computations.
By ignoring positions, current methods limit their ability to capture mechanisms that operate across positions, such as interactions between attention heads across positions.

In this study, we start by demonstrating that positional agnosticism is a significant limitation (\S\ref{sec:motivating}). Then, to address these limitations, we introduce a new approach: position-aware edge attribution patching (PEAP; \S\ref{sec:full_circ_discovery}; Figure~\ref{fig:p1}). Current approaches  assume that if an edge is in a circuit, then the same edge will be in the circuit at all positions, thus leading to low precision. It is also assumed that an edge's importance should be aggregated across positions before deciding whether it should be included in the circuit; this can lead to cancellation effects, and thus low recall. PEAP instead allows us to compute the importance of cross-positional edges, and separately evaluates edge importance at each position. We show that this leads to smaller and more accurate circuits; see Figure~\ref{fig:p1}.

Incorporating positional information into circuit discovery is straightforward when inputs have the same length and structure across examples.

However, realistic datasets are not nearly this templatic.
How, then, can we incorporate positional information into automatic circuit discovery?
To address this challenge, we propose \textbf{schemas} (\S\ref{sec:schema}). 
Schemas assign semantic labels to spans of tokens, enabling information aggregation across examples even when the spans differ in length.

For example, in the input ``The \textcolor{positional}{war} lasted from 1453 to 14\underline{\hspace{1em}},'' the span ``\textcolor{positional}{war}'' could be labeled as ``\emph{Subject}''.
This enables handling spans with varying lengths: the phrase ``\textcolor{positional}{Black Plague}'' in another example can be treated as a single positional span with the same role as ``\textcolor{positional}{war}''.
In experiments with two LMs and three tasks, we find that circuits discovered using schemas achieve a better trade-off between circuit size and faithfulness to the model's behavior than position-agnostic circuits.
Importantly, position-aware circuits offer a more precise representation of the underlying mechanisms, providing a more concise foundation for mechanistic explanations.

We also present a fully automated pipeline for schema generation and application (\S\ref{sec:schema-generation}) using large language models (LLMs). 
We evaluate the quality of the generated schemas and their utility in discovering position-aware circuits (\S\ref{sec:schema-eval}).
Notably, circuits derived using automatically generated and applied schemas achieve comparable faithfulness scores to circuits discovered with human-designed and manually applied schemas.

We summarize our contributions as follows:
\begin{itemize}[noitemsep,leftmargin=*,topsep=1pt,parsep=1pt]
    \item Introduce a position-aware circuit discovery method, which obtains better faithfulness than position-agnostic discovery.  
    \item Introduce dataset schemas,  facilitating positional circuit discovery in more naturalistic settings. 
    \item Develop an automated schema generation and application pipeline with LLMs, yielding schemas that are comparable to manually-annotated ones.
\end{itemize}

\section{Related work}
\mvspace{-2mm}
\paragraph{Human-AI complementarity.}

Many empirical studies of human-AI collaboration focus on AI-assisted human decision-making for legal, ethical, or safety reasons~\citep{bo2021toward, boskemper2022measuring, bondi2022role, schemmer2022meta}.
However, a recent meta-analysis by \citet{vaccaro2024combinations} finds that, on average, human–AI teams perform worse than the better of the two agents alone. 
In response, a growing body of work seeks to evaluate and enhance complementarity in human–AI systems \citep{bansal2021does, bansal2019updates, bansal2021most, wilder2021learning, rastogi2023taxonomy, mozannar2024effective}.
The present work differs from much of this prior work by approaching human-AI complementarity from the perspective of information value and use, including asking whether the human and AI decisions provide additional information that is not used by the other.
\mvspace{-2mm}
\paragraph{Evaluation of human decision-making with machine learning.}
Our work contributes methods for evaluating the decisions of human-AI teams~\citep{kleinberg2015prediction, kleinberg2018human, lakkaraju2017selective, mullainathan2022diagnosing,  rambachan2024identifying, guo2024decision, ben2024does, shreekumar2025x}.
\citet{kleinberg2015prediction} proposed that evaluations of human-AI collaboration should be based on the information that is available at the time of decisions.
% \jessica{can omit:} A significant portion of this literature addresses \textit{performative prediction}~\citep{perdomo2020performative}, where predictions or decisions affect future outcomes. 
% Because counterfactual decisions’ outcomes remain unobserved, researchers typically rely on worst-case analyses to bound the potential performance \citep{rambachan2024identifying, ben2024does}. 
% Though these issues arise in many canonical human-AI collaboration tasks, we focus on standard ``prediction policy problems'' where the payoff can be translated into policy gains~\citep{kleinberg2015prediction}.
According to this view, our work defines Bayesian best-attainable-performance benchmarks similar to several prior works~\citep{guo2024decision, wu2023rational,agrawal2020scaling, fudenberg2022measuring}. 
Closest to our work, \citet{guo2024decision} model the expected performance of a rational Bayesian agent faced with deciding between the human and AI recommendations as the theoretical upper bound on the expected performance of any human-AI team.
This benchmark provides a basis for identifying exploitable information within a decision problem.

\mvspace{-3mm}
\paragraph{Human information in machine learning.}

Some approaches focus on automating the decision pipeline by explicitly incorporating human expertise in developing machine learning models, such as by learning to defer~\citep{mozannar2024show, madras2018predict, raghu2019algorithmic, keswani2022designing, keswani2021towards, okati2021differentiable}.
\citet{corvelo2023human} propose multicalibration over human and AI model confidence information to guarantee the existence of an optimal monotonic decision rule.
\citet{alur2023auditing} propose a hypothesis testing framework to evaluate the added value of human expertise over AI forecasts.
Our work shares the motivation of incorporating human expertise, but targets a slightly broader scope by quantifying the information value for all available signals and agent decisions in a human–AI decision pipeline.


%\section{Preliminaries}\
\subsection{Preliminary: Hausdorff Distance-based Penetration Depth}
\label{sec:preliminary}

%In this section, we introduce the foundational concept of Hausdorff distance-based penetration depth algorithms, which are crucial for understanding our proposed ray tracing-based penetration depth methods.
%and ray tracing core technology, 

%\subsection{Hausdorff distance-based penetration depth}

The Hausdorff distance, as defined in Eq.~\ref{equation:hausdorff_definition}, plays a pivotal role in our approach.
Consider $A$ and $B$ as sets of vertices forming each \revision{object}, and let $d(\cdot, \cdot)$ denote the Euclidean distance between any two vertices.

%equation
\begin{equation}
    H(A,B) = \max \left( \max_{a \in A} \min_{b \in B} d(a,b) ,
    \max_{b \in B} \min_{a \in A} d(b,a) \right)
    \label{equation:hausdorff_definition}
\end{equation}
%

To compute penetration depth using the Hausdorff distance, the process involves several steps.
First, the overlapping volume $V$ between \revision{objects} $A$ and $B$ is computed.
Next, the surfaces of the overlapping volume, $\partial A$ and $\partial B$, contained within each \revision{object}, are extracted.
The final step involves computing the Hausdorff distance $H(\partial A, \partial B)$ between these surfaces.
The resulting distance $H(\partial A, \partial B)$ represents the penetration depth between the two objects.

%In this work, we introduce methods that utilize RT core technology for efficiently computing each step of the Hausdorff distance-based penetration depth calculation.

%\paragraph{BVH-Based Hausdorff Distance Computation:}
The brute-force computation of Hausdorff distance has a time complexity of $O(nm)$, where $n$ and $m$ represent the number of vertices in the two objects, as it requires evaluating all vertex pairs.
To reduce this computational burden, Bounding Volume Hierarchy (BVH) is employed, offering rapid localization of target polygons for distance assessment.
%
Tang et al.~\cite{SIG09HIST} constructed BVHs for two objects, $A$ and $B$.
Their approach begins by computing the Hausdorff distance from $A$ to $B$ (denoted as $h(A,B) = \max_{a \in A} \min_{b \in B} d(a,b)$) through a depth-first traversal of $BVH_A$.
Leveraging the property $h(A', B) \leq h(A, B) \leq h(A, B')$, where $A' \subseteq A$, this step determines the upper bound of the Hausdorff distance, $\overline{h}(A,B)$.
Subsequently, the lower bound $\underline{h}(A,B)$ is determined using $h(B,A)$.
This process yields an approximate Hausdorff distance bound satisfying $\overline{h}(A, B) - \underline{h}(A, B) \leq \epsilon$.
%
Building upon this method, Zheng et al.~\cite{zheng2022economic} implemented a four-point strategy, sampling four points on triangles (three vertices and one center point) to enable more efficient BVH traversal.
This approach is based on the observation that computing the distance between a triangle and a point is computationally less expensive than computing the distance between two triangles.

\revision{These prior methods, which focused on reducing the search space, achieved significant performance improvements for Hausdorff distance computation.
However, they are not well-suited for parallel processing on GPUs, as they require synchronization for updating and sharing the upper and lower bounds.}

\revision{Departing from previous methods, our approach exploits parallel processing on GPUs while leveraging the intrinsic capabilities of RT cores, with a unique emphasis on the ray-triangle intersection test, which is significantly accelerated by the RT core.
In alignment with Tang et al.~\cite{SIG09HIST}, our method approximates the Hausdorff distance.
However, we place greater emphasis on a ray sampling strategy designed to balance accuracy and performance.
Additionally, while previous methods focused on Hausdorff distance computation and presented penetration depth as an application, we accelerate the entire penetration depth computation process on the GPU.}

%Our approach, leveraging the intrinsic capabilities of the RT-core, also employs BVH, but with a unique emphasis on the ray-triangle intersection test, significantly accelerated by the RT-core.
%This strategy represents a departure from previous BVH-based Hausdorff methods, which predominantly focused on reducing search space by computing upper and lower bounds that requires synchronization process for updating the bounds while it is not appropriate for parallel processing.

%improving traversal efficiency during distance computation.

%This method has achieved a performance improvement of up to 20 times compared to Tang et al.'s technique~\cite{SIG09HIST}.

%This strategy involves sampling four points on triangles (three vertices and one center point) to facilitate BVH traversal.


%General algorithms for calculating the Hausdorff distance, when implemented without any additional culling strategies, exhibit a time complexity of at least $O(n^2)$ for triangular meshes~\cite{}. In the huge scenario, there is too much time cost.
%Tang et al. ~\cite{SIG09HIST} proposed the BVH-based framework for reducing the time complexity.
%They build the BVH(Bounding volume hierarchy) for model $A$ and $B$ and traverse the BVH tree of model $A$ in depth-first order. At the first traverse sequence, this algorithm gets sorted by the upper bound of the Hausdorff distance to Model $B$. Since $h(A', B) \leq h(A, B) \leq h(A, B')$ when $A' \subseteq A$ and $B' \subseteq B$, they use this upper bound to compute lower bound when reduction process. In the reduction process, they find the suitable primitive triangles to reduce the Hausdorff distance bound $\overline{h}(A, B) - \underline{h}(A, B) \leq \epsilon$. Finally, they achieve the approximated Hausdorff distance $h(A, B)$ in error bound $\epsilon$.

%Tang's method, Zheng's method
%\subsubsection{Acceleration of Hausdorff distance}

%\subsection{Ray-Tracing Core}
%\DS{Will move to realted work section}

%The Ray-Tracing Core (RT-core) is a specialized hardware component developed by NVIDIA, specifically designed for accelerating ray tracing-based rendering.
%This technology is a key feature of NVIDIA's RTX platform GPUs, such as the GeForce RTX series.
%At the heart of ray tracing is the fundamental operation of intersection checking between rays and objects.
%The RTX platform enhances this process by providing RT-core-based functionalities for efficient ray-bounding box and ray-triangle intersection tests.

%To exploit this powerful hardware for penetration depth computation, we designed RT-core based algorithms for two core tasks: firstly, to extract the overlapping region between two objects, and secondly, to calculate the Hausdorff distance for penetration depth computation.

%The ray tracing core(RT-core) is a specialized hardware on the RTX platform GPU (e.g., GeForce RTX) that was developed by NVIDIA for acceleration of ray tracing-based rendering.
%The basic operation of ray tracing is an intersection test between the ray-bounding box or ray-surface, RT core has a powerful performance up to two to four times higher than CUDA-based computation performance in these operations.

%In this work, to compute the penetration depth, we change the two main algorithms to ray-intersection-based operation to use these powerful RTX platforms.

 \section{Method}
\label{sec:method}











Given a set $\{x_{1_i},c_i\}_{i=1}^m$ of input samples and their corresponding conditioning states, our goal is to construct a flow-matching model that samples from $q(x_1|c)$ that start from our conditional prior distribution (CPD). 

\subsection{Flow Matching from Conditional Prior Distribution}
\label{sec:conditional_fm_joint}

We generalize the framework of  Sec.~\ref{sec:flow_matching} to a construction that uses an arbitrary conditional joint distribution of $\rho(x_0, x_1, c)$ which satisfy the marginal constraints:
\begin{equation*}
\label{eq:conditional_marginal}
    \int \rho(x_0, x_1, c)dx_0 = q(x_1, c),  \int \rho(x_0, x_1, c)dx_1dc = p(x_0)
\end{equation*}
Then, building on flow matching, we propose to modify the conditional probability path so that at $t=0$, we define:
\begin{equation}
    \rho_0(x_0|x_1, c) = p(x_0|x_1, c) 
\end{equation}
where $p(x_0|x_1, c)$ is the conditional distribution $\frac{\rho(x_0, x_1, c)}{q(x_1, c)}$. 
Using this construction, we satisfy the boundary condition of Eq.~\ref{eq:boundary_conditions}: 
\begin{align}
    \rho_0(x_0) &= \int\rho_0(x_0|x_1, c)q(x_1, c)dx_1dc  \\
                &=  \int p(x_0|x_1, c)dx_1dc = p(x_0)
\end{align}




The conditional probability path $\rho_t(x|x_1, c)$ does not need to be explicitly formulated. Instead, only its corresponding conditional vector field $u_t(x|x_1, c)$ needs to be defined such that points $x_0$ drawn from the conditional prior distribution $\rho_0(x_0|x_1, c) $, reach $x_1$ at $t=1$, i.e., reach distribution $\rho_1(x|x_1, c) = \delta(x - x_1)$.  We thus purpose the \emph{Conditional Generation Joint FM} $\gL_{\rm cgjfm}(\theta)$ objective:
\begin{equation}\label{eq:conditionl_joint_cfm}
    \mathbb{E}_{t\sim \mathcal{U}(0,1), q(x_0,x_1,c)} \|v_\theta(t, x, c) - u_t(x | x_1, c)\|^2
\end{equation}
where $x = \psi_t(x_0|x_1,c)$.
Training only involves sampling from $q(x_0,x_1,c)$ and does not require explicitly defining the densities $q(x_0,x_1,c)$ and $\rho_t(x|x_1,c)$.
We note that this objective is reduced to the CGFM objective Eq.~\ref{eq_conditional_generative_fm_objective} when $q(x_0,x_1,c) = q(x_1, c)p(x_0)$.

\subsection{Conditional Prior Distribution}
\label{sec:prior_distribution}

We now describe our choice of a condition-specific prior distribution. 
When choosing a conditional prior distribution we want to adhere to the following design principles:
(i) \emph{Easy to sample}: can be efficiently sampled from.
(ii) Well represents the target conditional modes. 
We design a condition-specific prior distribution based on a parametric \emph{Mixture Model} where each mode of the mixture is correlated to a specific conditional distribution $p(x_1|c)$. 
Specifically, we choose the prior distribution to be the following, \emph{easy to sample}, \emph{Gaussian Mixture Model} (GMM):
\begin{equation}\label{eq:gmm_formula}
    p_0 = \mathrm{GMM}(\gN(\mu_i, \Sigma_i)_{i=1}^n, \pi)
\end{equation}

where $\pi\in\R^n$ is a probability vector associated with the number of conditions $n$ (could be $\infty$) and $\mu_i, \Sigma_i$ are parameters determined by the conditional distribution $q(x_1|c_i)$ statistics, \emph{i.e.} 
 \begin{equation}\label{eq:gmm_parameters}
     \mu_i = \E[x_1|c_i], \quad \Sigma_i = \mathrm{cov}[x_1|c_i]
 \end{equation}
To sample from the marginal distribution $p(x_0|x_1, c_i)$, we sample from the cluster $\gN(\mu_i, \Sigma_i)$ associated with the condition $c_i$.

\noindent \textbf{Obtaining a Lower Global Truncation Error.} \quad 
Our CPD fits a GMM to the data distribution in a favorable setting, where the association between samples and clusters is given. 
\begin{equation}\label{eq:wasserstein_definition}
    d_1 \left(X, Y \right) \coloneqq \sup_{h \in \mathrm{Lip_1}} \mathbb{E}[h(X) - h(Y)] .
\end{equation}

In this process, we fit a dedicated Gaussian distribution to data points with the same condition. If the latter are close to being unimodal, this approximation is expected to be tight, in terms of the average distances between samples from the condition data mode and the fitted Gaussian. 
Tab.~\ref{tab:wasserstein_table} provides the average distances between pairs of samples from the prior and data distributions (i.e. the \emph{transport cost}) of CondOT~\cite{lipman2022flow}, BatchOT~\cite{pooladian2023multisample} and our CPD over the ImageNet-64~\cite{deng2009imagenet} and MS-COCO~\cite{lin2014microsoft} datasets. 
As expected, BatchOT which minimizes this exact measure within mini-batches, obtains better scores than the naïve pairing used in CondOT, while our CPD, which approximates the data using a GMM exploits the conditioning available in these datasets, and obtains considerably lower average distances.

As noted in \cite{pooladian2023multisample}, lower transport cost is generally associated with straighter flow trajectories, more efficient sampling and lower training time. We want to substantiate this claim from the viewpoint of cumulative errors in numerical integration.
Sampling from flow-based models consists of solving a time-dependent ODE of the form $\dot{x}_t =u_t(x_t)$, where $u_t$ is the velocity field. This equation is solved by the following integral $x_t = \int_{0}^t u_s(x_s)ds$, where the initial condition $x_0 $ is sampled from the prior distribution. Numerical integration over discrete time steps accumulate an error at each step $n$ which is known as the \emph{local truncation error $\tau_n$}, which accumulates into what is known as the \emph{global truncation error $e_n$}.  This error is bounded by ~\cite{suli2003introduction}
\begin{equation}
    |e_n| \leq \frac{max_j\tau_j}{hL}\big(e^{L(t_n-t_0)} - 1\big)
\end{equation}\label{eq:truncation_error_bound} 
where $h$ is the step size and $L$ is the Lipschitz constant of the velocity $u_t$. 
Accordingly, the distance between the endpoints of a path $\Delta = |x_1  - x_0|$  is given by $|\int_0^1 u_s(x_s)ds|$ which can be interpreted as the magnitude of the average velocity along the path $x_t$. Hence, the longer the path $\Delta$ is, the larger the integrated flow vector field $u_t$ is.
For example, if we scale a path uniformly by a factor $C>1$, i.e., $x_t \rightarrow C(x_t)$, we get,  $\frac{d}{dt}C(x_t) = C(u_t)$ in which case the Lipschitz constant $L$ is also multiplied by $C$.

By shortening the distance between the prior and and data distribution, as our CPD does, we lower the integration errors which permits the use of coarser integration steps, which in turn yield smaller global errors. Thus, our construction allows for fewer integration steps during sampling.

\subsubsection{Construction}


Next, we explain how we construct $p_0$ (Eq.~\ref{eq:gmm_formula}) for both the discrete case (e.g., class conditional generation) and continuous case (e.g., text conditional generation). 

\noindent \textbf{Discrete Condition.} \quad
In the setup of discrete conditional generation, we are given data $\{x_{1_i}, c_i\}_{i=1}^m$ where there are a finite set of conditions $c_i$.
We approximate the statistics of Eq.~\ref{eq:gmm_parameters} using the training data statistics. That is, we compute the mean and covariance matrix of each class (potentially in some latent represntation of a pretrained auto-encoder).  Since the classes at inference time are the same as in training, we use the same statistics at inference. 

\noindent \textbf{Continuous Condition.} \quad
While in the discrete case we can directly approximate the statistics in Eq.~\ref{eq:gmm_parameters} from the training data, in the continuous case (\emph{e.g.} text-conditional) we need to find those statistics also for conditions that were not seen during training. To this end, we first consider a joint representation space for training samples $\{x_{1_i}, c_i\}_{i=1}^m$, which represents the semantic distances between the conditions $c_i$ and the samples $x_{1_i}$. In the setting where $c_i$ is text, we choose a pretrained CLIP embedding. 
$c_i$ is then mapped to this representation space, and then mapped to the 
data space (which could be a latent representation of an auto-encoder), using a learned mapper $\gP_\theta$. 
Specifically, $\gP_\theta$ is trained to minimize the objective:
\begin{equation}
    \gL_{\rm prior}(\theta) = \mathbb{E}_{q(x_1,c)} \|\gP_\theta(E(c)) - x_1\|^2_2.
\end{equation}
where $E$ is the pre-trained mapping to the joint condition-sample space (e.g. CLIP). $\gP_\theta$ can be seen as approximating $\E[x_1|c]$, which is used as the mean for the condition specific Gaussian.  
At inference, where new conditions (e.g., texts) may appear, we first encode the condition $c_i$ to the joint representation space (e.g., CLIP) followed by $\gP_\theta$. This mapping provides us with the center $\mu_i$ of each Gaussian. %
We also define $\Sigma_i = \sigma_i^2\mathrm{I}$ where $\sigma_i$ is a hyper-parameter, ablated in Sec.~\ref{sec:results_quantitative} 

\subsection{Training and Inference}

Given the prior $p_0$ (either using the data statistics or by training $\gP_\theta$), for each condition $c$, we have its associated Gaussian parameters $\mu_c$ and $\Sigma_c$. The map $\psi_t(x|x_1,c)$ must be defined in order to minimize Eq.~\ref{eq:conditionl_joint_cfm} above. This corresponds to the interpolating maps between this Gaussian at $t=0$ and a small Gaussian around $x_1$ at $t=1$, defined by:
\begin{align}
    \psi_{t}(x|x_1,c) &= \sigma_t(x_1,c)x + \mu_t(x_1,c), \\ 
    \sigma_t(x_1,c) &= t (\sigma_{\min} \mathrm{I}) + (1-t)\Sigma_{c}^{1/2}, \quad \text{and} \\
    \mu_t(x_1,c) &= t x_1 + (1-t) \mu_c.
\end{align}
This results in the following target flow vector field 
\begin{equation*}
    u_t(\psi_{t}(x|x_1,c)) = \frac{d}{dt}\psi_t (x|x_1,c)  =   \big(\sigma_{\min}  \mathrm{I} - \Sigma_c^{1/2}\big)x +  x_1 - \mu_c.
\end{equation*}

During inference we are given a condition $c$ and want to sample from $q(x_1|c)$. Similarly to the training, we sample $x_0\sim p(x_0|c)$ and solve the ODE 
\begin{equation}
    \frac{d}{dt} \psi_t(x) = v_\theta \left(t, \psi_t(x), c \right), \quad \psi_0(x) = x_0
\end{equation}
Training and implementation details are in the appendix.








\begin{table}
        \renewcommand{\arraystretch}{0.8}
        \begin{threeparttable}
        \begin{tabular}{@{}lcc@{}}
        \toprule
        Parameter & Floating Allegro Hand & Bimanual Robot Arms \\
        \midrule
        % \makecell{Initial object translational \\ perturbation (cm) }& [$\pm1.5$, $\pm1.5$, 0] & [$\pm 5$, $\pm 5$, 0]\\
        % \makecell{Initial object rotational \\ perturbation (rad) }& [0, 0, $\pm 0.3$] & [0, 0, $\pm 0.3$]\\
        Init. obj. trans.  pert. (cm) & [$\pm1.5$, $\pm1.5$, 0] & [$\pm 5$, $\pm 5$, 0]\\
        Init. obj. rot. pert. (rad) & [0, 0, $\pm 0.3$] & [0, 0, $\pm 0.3$]\\
        Object side length (cm) & [5.8, 6.2]  & [28, 32] \\
        Object mass (kg) & [0.1, 0.3]  & [0.25, 0.75]  \\
        Friction coefficients & [0.7, 1.3] &  [0.2, 0.4]  \\
        Task horizon (s) & 25 & 50 / 260  (Panda / iiwa) \\
        \bottomrule
        \end{tabular}
        \end{threeparttable}
        \caption{Ranges of different physical parameters $\theta$. The initial object pose is only perturbed in yaw, x, and y to ensure the object sits stably on the table. }
        \label{tab:domain_randomization}
        \vspace{0.5em}
\end{table}

\begin{figure*}[t]
\centering
\includegraphics[width=1.0\textwidth]{figures/trajopt_unittest.png}
	\caption{\textbf{Trajectory optimization is crucial for generating dynamically feasible trajectories}. (Top) Before trajectory optimization, the kinematically retargeted demos easily lose contact and drive the object out of reach with different physical parameters or slight deviations in object states. (Bottom) Trajectory optimization encourages robots to establish contact with and maintain good manipulability of the object. The tricolor axis indicates the object orientation.}
	\label{fig:trajopt_unittest}
\end{figure*}

\section{Trajectory Optimization Experiments}

While kinematic retargeting of demonstrations might suffice to generate data for simpler manipulation tasks such as pick and place, it often falls short for the more challenging contact-rich tasks requiring frequent contact mode switches and fine-grained actions. In this section, we demonstrate that trajectory optimization is crucial for generating diverse, dynamically feasible contact-rich trajectories on three high-dimensional dexterous manipulation systems: a floating Allegro hand, bimanual iiwa arms, and bimanual Panda arms.

Our data generation framework is agnostic to the choice of the trajectory optimizer. We implement 
% a contact-implicit model predictive controller based on smoothed contact dynamics \cite{suh2024dexterous} and
the cross-entropy method (CEM) \cite{de2005tutorial} to solve \eqref{eq:predictive_control} over a distribution of physical parameters and initial conditions, as specified in Table \ref{tab:domain_randomization}. 
%\russtcomment{Right... the SQP discussion tricked me, but I guess that's only for the retargeting. I thought you had replaced this. In this case, your approach is almost doing RL, but on a policy parameterized as a trajectory... right? why is that better than doing PPO on a small neural net policy, and generating data from that? If you stick with CEM, than this will be your burden of proof, i think?}

\underline{\textbf{Task}} Manipulating the object to a target pose on the table (Fig. \ref{fig:policy_rollouts}). The object is initially placed randomly on the table with an arbitrary face upward. Task success is defined as the object reaching within 3 cm and 0.2 rad of the target pose for the Allegro hand, and within 10 cm and 0.2 rad for the bimanual robot arms.  This task requires long-horizon reasoning of complex multi-contact interactions between the robot and the object. The necessary frequent contact mode switches and high-dimensional action space pose great challenges for traditional model-based planners, while the precise contact interactions require fine-grained control actions. 

\begin{table}
\centering
        \renewcommand{\arraystretch}{0.8}
        \begin{threeparttable}
        \begin{tabular}{@{}lcccc@{}}
        \toprule
        Perturbation & Allegro Hand & iiwa Arms & Panda Arms \\
        \midrule
        Original demo &4 / 24 & 5 / 24 & 6 / 24\\
        Object size & 2 / 24 & 1 / 24 & 4 / 24\\
        % Object mass & 1 / 24& 1 / 24 & \\
        % Friction coefficients & 3 / 24 & 2 / 24 & \\
        Initial object translation & 1 / 24 & 3 / 24 & 2 / 24\\
        Initial object orientation & 2 / 24 & 3 / 24& 3 / 24\\
        \midrule
        Trajectory optimization & 2164 / 3000 & 2252 / 3000 & 2462 / 3000 \\
        \bottomrule
        \end{tabular}
        \end{threeparttable}
        \caption{Success rates of replaying kinematically retargeted trajectories of the 24 original human demos, and trajectory optimization under random perturbations in physical parameters and object initial conditions. }
        \label{tab:kin_success_rate}
\end{table}

% \begin{table}
% \centering
%         \renewcommand{\arraystretch}{0.8}
%         \begin{threeparttable}
%         \begin{tabular}{@{}ccc@{}}
%         \toprule
%         Allegro Hand & iiwa Arms & Panda Arms \\
%         \midrule
%         0.721 & 0.65 & 0.803\\
%         % Task Horizon (s) & 25 & 280 & 50 \\
%         \bottomrule
%         \end{tabular}
%         % \begin{tablenotes}
%         % \itme{*} 
%         % \end{tablenotes}
%         \end{threeparttable}
%         \caption{Success rates of trajectory optimization under random perturbations in physical parameters and object initial conditions. }
%         \label{tab:trajopt_success_rate}
%         \vspace{0.5em}
% \end{table}



% \begin{figure*}[t]
% \centering
% \includegraphics[width=0.7\textwidth]{figures/aug_traj_den.png}
% 	\caption{\textbf{Distribution of object trajectories generated from a single demonstration}. The original demonstration (orange) is locally perturbed and augmented to about 100 dynamically feasible contact-rich trajectories (blue) for each system. The density map represents the object pose distribution of the generated trajectories in the specific 2-dimensional slices.}
%     \label{fig:aug_data_distribution}
% \end{figure*}

% \begin{figure*}[t]
% \centering
% \includegraphics[width=0.9\textwidth]{figures/aug_traj_snapshots.png}
% 	\caption{\textbf{Snapshots of trajectories generated from a single demonstration}. The original demonstration (orange) is locally perturbed and augmented to about 100 dynamically feasible contact-rich trajectories (blue) for each system. The density map represents the object pose distribution of the generated trajectories in the specific 2-dimensional slices.}
%     \label{fig:aug_data_distribution}
% \end{figure*}
\begin{figure*}[t]
\centering
\includegraphics[width=1.0\textwidth]{figures/density_snapshots_aug_traj.png}
	\caption{\textbf{Distribution and snapshots of trajectories generated from a single demonstration.} (a) The original demonstration (orange) is locally perturbed and augmented to about 100 dynamically feasible contact-rich trajectories (blue) for each system. The density map represents the object pose distribution of the generated trajectories in the specific 2-dimensional slices. (b) Snapshots of 30 dynamically feasible trajectories under random physical parameters and object initial poses for bimanual iiwa arms are visualized.}
    \label{fig:aug_data_distribution}
\end{figure*}


\underline{\textbf{Dynamic Feasibility}}
While kinematic motion retargeting can generate visually plausible robot and object trajectories, these trajectories often lack dynamical consistency due to the differences in physical parameters and embodiment between the human demonstrator and the target robot. To illustrate this, we replay the kinematically retargeted trajectories of the original 24 human demos and record the success rates for each system in Table \ref{tab:kin_success_rate}. Furthermore, we randomly sample object sizes and perturbations of initial object poses according to Table \ref{tab:domain_randomization} and roll out the nominal kinematically retargeted trajectories. Some trajectories still succeed under certain perturbations thanks to caging grasps or other strategies that encourage robustness during the human demonstration. For all the systems, the successful rollouts are relatively short, manipulating the object to the goal pose within only 1 or 2 rotations. 
% Notably, the successful trajectories for the iiwa and Panda arms vary significantly, despite being generated from the same initial set of demonstrations.

The low success rate of purely kinematically retargeted trajectories highlights the importance of trajectory optimization for locally refining the demos for the particular embodiments and physical parameters. Before trajectory optimization, the floating Allegro hand lightly touches the cube and easily loses contact when rotating it clockwise (demonstrated in Fig. \ref{fig:trajopt_unittest}a). After trajectory optimization, the hand increases the contact area, establishing a stable grip for rotation. In Fig. \ref{fig:trajopt_unittest}b, similar behavior that encourages contact can be observed for the bimanual iiwa arms: the demo trajectory tries to rotate the box clockwise only using a single arm, while trajectory optimization encourages the other arm to help hold the box and reorient the box more stably. These refinements that encourage contact are particularly helpful when the object is heavier or smaller, or when the friction coefficients are lower than expected. In addition, replaying the kinematically retargeted trajectory often fails when the object pose deviates slightly from the demonstration, driving the object out of reach (visualized in Fig. \ref{fig:trajopt_unittest}c). In contrast, trajectory optimization 
%stabilizes the system in a vicinity around the demonstration, ensuring higher success rates even when the object is perturbed
accounts for the system’s true dynamics and can adjust the robot’s actions accordingly. The success rates of trajectory optimization under random perturbations in physical parameters and object initial conditions for each system are recorded in Table \ref{tab:kin_success_rate}.

\begin{figure*}[t]
    \centering
    \includegraphics[width=0.9\linewidth]{figures/policy_rollouts.png}
    \caption{\textbf{Policy rollouts for different embodiments.} The object manipulation task requires the robots to frequently make and break contact with the object. It also requires precise control of the robot since small deviations in positions can result in missing contact interactions and lead to task failure. } 
    \label{fig:policy_rollouts}
\end{figure*}

\underline{\textbf{Cross-Embodiment Generalization}} We demonstrate that a single set of human demonstrations can be effectively repurposed to generate dynamically consistent, contact-rich trajectories across different robotic embodiments with varying task horizons. Specifically, human demonstrations involving two index fingers manipulating a small cube are retargeted to fixed-base bimanual Kuka LBR iiwa and Franka Emika Panda arms manipulating a larger box (visualized in Fig. \ref{fig:kinematic_retargeting}). This approach addresses key challenges in data collection for contact-rich tasks: directly teleoperating two real robot arms to flip a large box would be both physically demanding and cost-prohibitive due to hardware latency, limited feedback, and the embodiment gap--differences in kinematic structure, degrees of freedom, and workspace between human and robotic arms. In contrast, performing the same task on a smaller scale using human fingers is more intuitive, reduces physical effort, and enables faster, more consistent demonstration collection.

The iiwa and Panda arms differ in contact geometry, velocity limits, and joint constraints, all of which are explicitly modeled within the trajectory optimization framework described in \eqref{eq:predictive_control}. For safe hardware deployment, we enforce conservative velocity limits on the iiwa arms, while only applying soft velocity regularization on the Panda arms in simulation to allow for more aggressive motions.


\underline{\textbf{Data Diversity}} 
Trajectory optimization efficiently augments a single demonstration to a wide distribution of trajectories with locally perturbed physical parameters and initial conditions as visualized in Fig. \ref{fig:aug_data_distribution}. The diverse states in the generated dataset cover a larger training distribution and encourage smoother learned policies, as will be discussed in the next section.
\begin{figure*}[t]
    \centering
    \includegraphics[width=0.9\linewidth]{figures/policy_failure.png}
    \caption{\textbf{Failure cases of baselines.} (a) The baseline policy trained on the original 24 demonstrations for the floating Allegro hand frequently misses contact or gets stuck on the cube. (b-c) The baseline policies for the bimanual robot arms often exhibit jittery motion, resulting in loss of contact, the box being kicked out of reach, or the robot arms running into and getting stuck on the box surface. } 
    \label{fig:policy_failure}
\end{figure*}

\begin{figure}
\centering
\includegraphics[width=0.42\textwidth]{figures/success_rate.png}
	\caption{Success rates of policy evaluation in simulation and hardware. }
	\label{fig:success_rate}
    \vspace*{-0.4cm}
\end{figure}

% \begin{figure}
% \centering
% \includegraphics[width=0.48\textwidth]{figures/jitteriness.png}
% 	\caption{\textbf{Joint angles of bimanual iiwa arms over time. } Each line represents the trajectory of a different joint of the iiwa arms. The policy trained on augmented datasets (b) demonstrates significantly smoother motion compared to the baseline policy (a). }
% 	\label{fig:jitteriness}
% \end{figure}

\begin{figure*}[t]
    \centering
    \includegraphics[width=0.9\linewidth]{figures/hardware_rollout.png}
    \caption{\textbf{Policy rollouts on hardware.} The fixed-base bimanual iiwa arms perform a sequence of coordinated rolling, pitching, and yawing actions to reorient the box to the goal pose. } 
    \label{fig:hardware_rollout}
\end{figure*}

\section{Behavior Cloning Experiments}
We illustrate our framework's capability to efficiently produce diverse, high-quality contact-rich datasets for training behavior cloning policies across multiple robotic platforms, including the floating Allegro hand and the bimanual Panda arms in simulation as well as bimanual iiwa arms on hardware. We show that policies trained on the generated data generalize to a wide distribution of physical parameters and initial conditions, and are much more robust and performant than the ones trained only on the original demonstrations. 
\subsection{Policy Evaluation in Simulation}
\label{subsec:policy_eval_sim}
From only 24 human demonstrations, our data generation pipeline can efficiently generate thousands of dynamically feasible contact-rich trajectories using trajectory optimization. We train state-based diffusion policies \cite{chi2023diffusion} on the 24 original demo trajectories, as well as 500 and 1000 generated trajectories. While our method is compatible with any Behavior Cloning algorithm, we adopt diffusion policies due to its recent success in contact-rich tasks \cite{chi2024universal, zhu2024should, li2024planning}. Fig. \ref{fig:policy_rollouts} visualizes the policy rollouts. We evaluate the performance by conducting 48 policy rollouts for each embodiment in simulation and record the success rates in Fig. \ref{fig:success_rate}. The success criteria are the same as specified in the trajectory optimization experiments.
%For policy evaluation, we visualize the initial states for all evaluation episodes, typical failure cases of baseline policies, and final object pose errors in Fig. \ref{fig:policy_eval}.  
% and validate that the generated data help improve the policy's robustness and generalizability.

\subsubsection{Floating Allegro Hand} 
While the human demonstrator completes the task in approximately 5 seconds on average in the virtual reality environment, the demonstration trajectories are temporally scaled by a factor of 2.5 to ensure smoother, dynamically feasible motions on the floating Allegro hand, which is subject to velocity limits. We define the task horizon as 25 seconds to allow the policy sufficient time to recover from missed contacts and other errors during the execution. The task complexity arises from the 22-dimensional action space of the Allegro hand and the long-horizon nature of the task, which requires a sequence of coordinated rolling, pitching, and yawing actions to reorient the cube to an upright position. These factors together present significant challenges for traditional model-based planners without guidance.

The baseline behavior cloning policy trained on the original set of 24 demonstrations achieves a success rate of $10 / 48 = 21\%$ and exhibits significant jittery behavior when encountering out-of-distribution states. The workspace, characterized by diverse object orientations and translations, is sufficiently large that minor deviations during policy rollouts often drive the trajectory out of the demonstrated distribution. Common failure modes include the Allegro hand repeatedly missing contact with the cube or becoming stuck on its surface while attempting reorientation (visualized in Fig. \ref{fig:policy_failure}a), which often result in the object being trapped in intermediate orientations. In contrast, policies trained on the expanded dataset generated by our pipeline demonstrate a higher likelihood of re-establishing contact with the object after initial misses, resulting in significantly improved success rates up to $39 / 48 = 81\%$.

\begin{figure*}[t]
    \centering
    \includegraphics[width=0.9\linewidth]{figures/hardware_eval.png}
    \caption{\textbf{Policy failure and recovery on hardware.} The baseline policy frequently (a) gets stuck on the box surface when small deviations from the demonstration trajectories occur, and (b) struggles to recover from out-of-distribution states, where the object is never intentionally lifted for accomplishing the task in the generated dataset. Policies trained on augmented datasets (c) sometimes fail due to unmodeled collision geometry, but (d) can recover from undesired sliding by employing firmer grasps found by trajectory optimization. } 
    \label{fig:hardware_eval}
\end{figure*}
\subsubsection{Bimanual Robot Arms}
The baseline policy trained on the original set of 24 human demonstrations achieves a success rate of $27 / 48 = 56\%$ on the bimanual iiwa system. We hypothesize that the restrictive velocity limits encourage more quasi-static behavior, leading to longer trajectories with a higher density of state-action pairs in the training data. In contrast, the baseline policy yields a success rate of $14/48=29\%$ on the bimanual Panda system, likely due to the more dynamic nature of the learned behavior under its looser velocity constraints. Both baseline policies exhibit remarkably jittery motion, frequently kicking the box out of reach, losing contact, or running into and getting stuck on the box surface during reorientation (visualized in Fig. \ref{fig:policy_failure}b and c). Policies trained on the augmented dataset, however, generate significantly smoother trajectories and are capable of re-establishing contact with the object after initial misses, resulting in as high as $44 / 48 = 92\%$ success rates for bimanual iiwa arms and $42 / 48 = 87.5\%$ for bimanual Panda arms. Additionally, the learned policies capture multimodal behaviors observed in the original human demonstrations, such as rotating the box either clockwise or counterclockwise for similar object poses. 


\subsection{Policy Evaluation on Hardware}
We zero-shot deploy the trained policies on hardware for bimanual iiwa arms to flip a 30 cm cubic box on a table (Fig. \ref{fig:hardware_rollout}). An OptiTrack motion capture system is employed to estimate the object pose. The baseline behavior cloning policy only achieves $6/23=26\%$ success rate, with most successful rollouts being relatively short-horizon, involving only 1 or 2 rotations. Common failure modes of the baseline policy include: 1) deviation from the demonstration trajectory, causing the arms to collide with the box surface (Fig. \ref{fig:hardware_eval}a), and 2) significant box sliding during rolling, resulting in the policy encountering out-of-distribution states and failing to recover (Fig. \ref{fig:hardware_eval}b). In contrast, as shown in Fig. \ref{fig:success_rate}b, the policy trained on 500 generated trajectories achieves $17 / 23 = 74\%$ success rate, while the policy trained on 1000 generated trajectories achieves $16/23=70\%$ success rate. Despite occasional box sliding during rolling, these policies demonstrate an improved ability to stabilize the box by using one arm to hold the opposite side more firmly to prevent further sliding (Fig \ref{fig:hardware_eval}d). However, as visualized in Fig \ref{fig:hardware_eval}c, both policies trained on the augmented datasets exhibit failure modes originating from unmodeled collision geometries on iiwa arms, which lead to significant undesired yaw motions of the box during pitch actions.\looseness=-1
\section{Discussion and Conclusion}
\subsection{Limitation}
Since our method depends on human body fitting and foreground segmentation, artifacts may occur due to inaccuracies in these videos within the processes. Despite incorporating pose optimization to correct poses, the reconstruction of hand parts and body shape may still exhibit artifacts in certain cases, as shown in the videos. While our approach generally yields more realistic results, similar to many existing methods~\cite{instant_nvr,weng2022humannerf,hu2023gaussianavatar}, it still faces challenges in accurately modeling loose attire, such as dresses, underscoring areas for potential improvement in future iterations.






\subsection{Conclusion}

In this paper, we introduce \name, a novel approach for high-quality dynamic human reconstruction from monocular videos. By leveraging 2D diffusion model priors, \name effectively reconstructs and infers the unseen parts of 3D human avatars. We introduce Dual-Space Optimization, which applies Score Distillation Sampling (SDS) in both canonical and observation spaces, ensuring visual consistency and realism across various poses. Furthermore, View Selection and Pose Feature Injection strategies resolve conflicts between SDS predictions and observed data, enhancing overall avatar fidelity. Extensive experiments on benchmarks demonstrate that \name outperforms state-of-the-art methods, particularly in rendering the unseen parts of the human body.


% In this paper, we introduced Wonder Human, among the pioneering methods utilizing the explicit representation of 3DGS for the efficient reconstruction of human avatars from partial-view videos. Our method achieves photorealistic rendering and reconstruction of invisible appearance. Experiments demonstrate that our approach surpasses state-of-the-art methods in terms of rendering quality for invisible region synthesis. Furthermore, we proposed dual space fine-tuning and appearance optimization, including view selection and pose feature injection, both of which proved effective in enhancing rendering quality. We anticipate that our innovative approach will stimulate further research in high-quality animatable human avatar synthesis from partial-view videos.


\appendices


\section{Training Details}
\label{sec:details}
This section provides more details about the implementation and training of our method.

\subsection{Stage I}
\subsubsection{Canonical Initialization}
We unwrap the T-pose body onto a UV map, where each pixel stores a 3D position vector. The positional UV map, with a resolution of $(512 \times 512 \times 3)$, is used to initialize Gaussians in the canonical space, ensuring proper alignment with the body’s structure. Additionally, a downsampled $(128 \times 128 \times 3)$ version of the positional UV map serves as input to the Gaussian decoder, aiding in reconstructing and refining the 3D representation. 
Furthermore, we use blend
\subsubsection{Training}
The training objectives in this stage focus on image losses and optimizations about Gaussian parameters.We set weights for each objective as $\lambda_{rgb} = 0.8$, $\lambda_{n} = 0.8$, $\lambda_{ssim} = 0.2$, $\lambda_{lpips} = 0.2$, $\lambda_{\Delta x} = 0.85$, $\lambda_{s} = 0.03$, $\lambda_{S} = 1$. 

\subsubsection{Pose Optimization} Our method leverages pose optimization from GaussianAvatar~\cite{hu2023gaussianavatar} for the In-the-wild dataset as a correction for fitted SMPL~\cite{SMPL:2015} pose parameters. We have omitted this functionality for the ZJU-Mocap dataset, as their ground truth pose is accurate. However, GausianAvatar keeps optimizing pose parameters for ZJU-Mocap dataset, which leads to inaccurate poses, especially for invisible parts. Please check the accompanying video results for more details. 
\subsection{Stage II} 
\subsubsection{Dual-space Optimization}
In this stage, we apply Dual-space optimization on top of visible appearance reconstruction to predict the invisible appearance. During training, each epoch is divided into three parts: 50\% for given view training and 50\% for Dual-space optimization. In Dual-space optimization, the weight of canonical optimization is treated as a hyperparameter, defaulting to 50\%. the The fine-tuning losses are added upon $\mathcal{L}_{StageI}$. We set $\lambda_{p} = 0.5 $ and  $\lambda_{SDS} = 0.3 $ initially. 

\subsubsection{Progressive Training}
We design a progressive training strategy in this stage, gradually diminishing the weight of SDS loss. This strategy is employed to enhance further the effectiveness and efficiency of the visible appearance reconstruction. Based on this strategy, the $\lambda_{SDS}$ is reduced gradually by following:
\begin{equation}
\lambda_{SDS}(t) = \lambda_{SDS,0} \cdot \frac{1}{2^{\lfloor \frac{t - t_{\text{0}}}{k} \rfloor}}
  \label{eq:prog.train}
\end{equation}
where $t$ and $t_{0}$ are the current epoch and starting epoch respectively, $k$ is the interval step of changing the weight. We set $t_{0} = 100$ and $k = 100$.

\subsection{Resolution}
The video resolution for the ZJU-Mocap (revised)\cite{liu2023zero1to3} and Monocap datasets is consistently maintained at $1024\times1024$ pixels, while MVHumanNet\cite{xiong2024mvhumannet} has a resolution of $2048\times1500$ pixels. For videos collected from the internet, the resolution ranges from 720p to 1080p. However, in Stage II, Zero123 only accepts $256\times256$ as input. Therefore, for SDS loss calculation, we crop the ground truth images based on their masks and resize them to $256\times256$.

 


% \subsection{Potential Social Impact} The creation and manipulation of highly realistic avatars raise ethical questions regarding privacy, consent, and misrepresentation. There is a risk of misuse, such as creating deceptive content or impersonating individuals without their permission. The widespread availability of realistic avatars could complicate identity verification processes in online environments. It may become more challenging to distinguish between real individuals and avatar representations, potentially undermining trust and security.
\begin{figure}[!t]
  \centering
  \includegraphics[width=1\linewidth]{Figures/Appendix_2.2.pdf}
    % \vspace{-1mm}
  \caption{Qualitative comparison results with SIFU~\cite{Zhang2024SIFU} and SITH~\cite{ho2024sith}.}
    % \vspace{-3mm}
  \label{fig:SIFU}
\end{figure}
\section{More Experiments}



\subsection{Comparison with Image-based Methods \label{resultImagebased}}

In this section, we compare our method with SIFU~\cite{Zhang2024SIFU}, SITH~\cite{ho2024sith}, and ELICIT~\cite{huang2022elicit}, all of which are single-image reconstruction techniques designed to synthesize unseen parts of human avatars.

SIFU proposes an approach to reconstruct clothed human avatars from single images. Qualitatively, as shown in Fig.~\ref{fig:SIFU}, this method can reconstruct decent geometry but fails to synthesize the texture of unseen parts of humans. SITH, similar to SIFU, is a method for single-image reconstruction. SITH can predict the texture of unseen parts of humans, but their generated textures contain unrealistic artifacts.
\begin{figure}[!t]
  \centering
  \includegraphics[width=1\linewidth]{Figures/Appendix_1.1.pdf}
    % \vspace{-2mm}
  \caption{Qualitative comparison results with ELICIT~\cite{huang2022elicit} on novel poses.}
    % \vspace{-2mm}
  \label{fig:ELICIT}
\end{figure}

\begin{table}[!t]
\centering
\renewcommand{\arraystretch}{1.5}

\begin{tabular}{c|c|ccc}
\hline
Dataset                    & Method & PSNR$\uparrow$&SSIM$\uparrow$&LPIPS$\downarrow$  \\ \hline
\multirow{3}{*}{MVHumanNet}   & SIFU   &19.29&0.9486&0.0706\\
                            & SITH   &19.68&0.9462&0.0699\\
                        &\textbf{Ours} & \textbf{20.98} & \textbf{0.9517} & \textbf{0.0553} \\ \hline  
\multirow{3}{*}{Monocap}   & SIFU   &18.96&0.9406&0.0659\\
                            & SITH   &19.06&0.9428&0.0.0673\\
                        &\textbf{Ours} & \textbf{21.16} & \textbf{0.9532} & \textbf{0.0549} \\ \hline  
\multirow{2}{*}{ZJU-Mocap(revised)} & ELICIT &19.23&0.9456&0.0.0689\\
                           &\textbf{Ours} & \textbf{20.82} & \textbf{0.9552} & \textbf{0.0569}  \\ \hline
\end{tabular}
\caption{Quantitative evaluation on MVHumanNet, ZJU-Mocap(revised), and Monocap datasets.}
\vspace{-3mm}
  \label{tab:exp}
\end{table}
ELICIT is a generative model that takes one image and a motion sequence as input to generate an animatable avatar. Qualitative results are shown in Fig.~\ref{fig:ELICIT}. For a fair comparison, since our method takes an image sequence as input, we are comparing the quality by synthesizing a novel pose that is not included in our inputs. Even though ELICIT can predict the unseen parts of humans, it shows blurred edges and floating artifacts while applying motions. Because only one image is used as input for ELICIT, the texture cannot be adapted to novel poses dynamically. In contrast, our method associates texture to different body parts across frames and can predict the correct texture for unseen parts robustly.

In Tab.~\ref{tab:exp}, we present the quantitative evaluation results. SIFU and SITH were tested on the Monocap dataset, while ELICIT was evaluated on the ZJU-Mocap(revised) dataset. The results demonstrate that our method consistently achieves superior performance compared to the state-of-the-art approaches, underscoring its efficacy and robustness.



% }



% \ifCLASSOPTIONcompsoc

% %  \section*{Acknowledgments}
% \else


%   \section*{Acknowledgment}
% \fi
% % The authors would like to thank the anonymous reviewers for their constructive comments and suggestions. Zhang, Li, and Guo were partially supported by a grant from National Science Foundation (2007661) and research gifts from Samsung Research America. Zeng was partially supported by NSFC (No.62072382), and Fundamental Research Funds for the Central Universities, China (No.20720190003). The opinions expressed are solely those of the authors, and do not necessarily represent those of the funding agencies.

% \ifCLASSOPTIONcaptionsoff
%   \newpage
% \fi



% \balance
\bibliographystyle{IEEEtran}
\bibliography{TVCG}
\vspace{-15mm}
\begin{IEEEbiography}[{\includegraphics[width=1in,height=1.25in,trim=70 70 80 70,clip,keepaspectratio]{Authors/zilong.png}}]{Zilong~Wang} is a Ph.D. candidate at the University of Texas at Dallas, supervised by Prof. Xiaohu Guo. He received his B.S. degree in software engineering in 2020 from Northwest University(China) and M.S. degree in software engineering in 2022 from the University of Texas at Dallas. 
His research interests include human reconstruction and animation, computer graphics, computer vision, and deep learning.
\end{IEEEbiography}
\vspace{-15mm}
\begin{IEEEbiography}[{\includegraphics[width=1in,height=1.25in,trim=70 40 80 70,clip,keepaspectratio]{Authors/zhiyang.png}}]{Zhiyang~(Frank)~Dou} is a Ph.D. candidate in the Computer Graphics Group at The University of Hong Kong, under the supervision of Prof. Wenping Wang and Prof. Taku Komura. %Zhiyang earned his B.Eng. degree with honors from Shandong University, where he was advised by Prof. Shiqing Xin.
Zhiyang’s research focuses on shape recovery and generation, character animation, geometric modeling, and the analysis of human behavior, emphasizing the intersection of artificial intelligence, computer graphics and computer vision.
\end{IEEEbiography}
\vspace{-15mm}
\begin{IEEEbiography}[{\includegraphics[width=1in,height=1.25in,trim=40 40 40 40,clip,keepaspectratio]{Authors/liuyuan.png}}]{Yuan~Liu} is an assistant professor at HKUST. He received his PhD degree in the University of Hong Kong in 2024. His research mainly concentrates on 3D vision and graphics. I currently work on topics about 3D AIGC including neural rendering, neural representations, and 3D generative models.
\end{IEEEbiography}
\vspace{-15mm}
\begin{IEEEbiography}[{\includegraphics[width=1in,height=1.25in,clip,keepaspectratio]{Authors/lincheng.jpg}}]{Cheng~Lin} received his Ph.D. from The University of Hong Kong (HKU), advised by Prof. Wenping Wang. He visited the Visual Computing Group at Technical University of Munich (TUM), advised by Prof. Matthias Nießner. Before that, he completed his B.E. degree at Shandong University. His research interests include geometric modeling, 3D vision, shape analysis, and computer graphics.
\end{IEEEbiography}
\vspace{-15mm}
\begin{IEEEbiography}[{\includegraphics[width=1in,height=1.25in,clip,keepaspectratio]{Authors/xiaodong.jpg}}]{Xiao~Dong} is an assistant Professor in the Department of Computer Science, BNU-HKBU United International College. She received the BS and PhD degrees in computer science and technology from Xiamen University, in 2013 and 2022, respectively. Her research interests include computer graphics, computer vision and deep learning.
\end{IEEEbiography}
\vspace{-15mm}
\begin{IEEEbiography}[{\includegraphics[width=1in,height=1.25in,clip,keepaspectratio]{Authors/yunhui.jpg}}]{Yunhui~Guo} is an assistant professor in the Department of Computer Science at the University of Texas at Dallas. Previously, he was a postdoctoral researcher at UC Berkeley/ICSI. He earned his PhD in Computer Science from the University of California, San Diego. His research interests include machine learning and computer vision, with a focus on developing intelligent agents that can continuously learn, dynamically adapt to evolving environments without forgetting previously acquired knowledge, and repurpose existing knowledge to handle novel scenarios.
\end{IEEEbiography}
\vspace{-15mm}
\begin{IEEEbiography}[{\includegraphics[width=1in,height=1.25in,clip,keepaspectratio]{Authors/chenxu.jpg}}]{Chenxu~Zhang} is a Research Scientist at the Intelligent Creation Lab, ByteDance. He completed his Ph.D. degree in Computer Science from the University of Texas at Dallas in 2023. He received his B.S. degree in Software Engineering in 2015 and M.S. degree in Computer Science in 2018, both from Beihang University. His research interests include computer graphics, computer vision, and deep learning.
\end{IEEEbiography}
\vspace{-15mm}
\begin{IEEEbiography}[{\includegraphics[width=1in,height=1.25in,clip,keepaspectratio]{Authors/lixin.png}}]{Xin~Li} (Senior Member, IEEE) received the B.E. degree in computer science from the University of Science and Technology of China in 2003 and the M.S. and Ph.D. degrees in computer science from the State University of New York at Stony Brook in 2005 and 2008, respectively. He is currently a Professor with the Section of Visual Computing and Creative Media, School of Performance, Visualization, and Fine Arts, Texas A\&M University. His
research interests include geometric and visual data computing, processing, and understanding, computer vision, and virtual reality.
\end{IEEEbiography}
\vspace{-10mm}
\begin{IEEEbiography}[{\includegraphics[width=1in,height=1.25in,clip,keepaspectratio]{Authors/wenping.png}}]{Wenping~Wang} (Fellow, IEEE) received the
Ph.D. degree in computer science from the University of Alberta. He is a Professor of computer science at Texas A\&M University. His research interests include computer graphics, visualization, computer vision, robotics, medical image processing, and geometric computing. He has been an journal associate editor of ACM Transactions on Graphics, IEEE Transactions on Visualization and Computer Graphics, Computer Aided Geometric Design, and Computer Graphics Forum (CGF). He has chaired a number of international conferences, including Pacific Graphics, ACM Symposium on Physical and Solid Modeling (SPM), SIGGRAPH and SIGGRAPH Asia. Prof. Wang received the John Gregory Memorial Award for his contributions to geometric modeling.
\end{IEEEbiography}
\vspace{-15mm}
\begin{IEEEbiography}[{\includegraphics[width=1in,height=1.25in,clip,keepaspectratio]{Authors/xiaohu.jpg}}]{Xiaohu~Guo} is a Full Professor of Computer Science at the University of Texas at Dallas. He received his Ph.D degree in Computer Science from Stony Brook University, and a B.S degree in Computer Science from the University of Science and Technology of China. His research interests include computer graphics, computer vision, medical imaging, with an emphasis on geometric modeling and processing, as well as body and face modeling problems. He received the prestigious NSF CAREER Award in 2012 and SIGGRAPH 2023 Best Paper Award. He has been serving on the journal editorial boards of IEEE TVCG, TMM, TCSVT, GMOD, CAVW, and on the executive committee of Solid Modeling Association.
\end{IEEEbiography}



\end{document}


