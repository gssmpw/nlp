% This is samplepaper.tex, a sample chapter demonstrating the
% LLNCS macro package for Springer Computer Science proceedings;
% Version 2.21 of 2022/01/12
%
\documentclass[runningheads]{llncs}
%
\usepackage[T1]{fontenc}
% T1 fonts will be used to generate the final print and online PDFs,
% so please use T1 fonts in your manuscript whenever possible.
% Other font encondings may result in incorrect characters.
%
\usepackage{graphicx}
% Used for displaying a sample figure. If possible, figure files should
% be included in EPS format.
%
% If you use the hyperref package, please uncomment the following two lines
% to display URLs in blue roman font according to Springer's eBook style:
%\usepackage{color}
%\renewcommand\UrlFont{\color{blue}\rmfamily}
%\urlstyle{rm}
%

\usepackage{csquotes}
\usepackage{tabularx}
\usepackage{listings}
\usepackage[binary-units]{siunitx}
\usepackage{xspace}
\usepackage{microtype}
\usepackage{xspace}
\usepackage{flushend}
\usepackage{paralist}
\usepackage{enumitem}
\usepackage{siunitx}
\usepackage{makecell}
\usepackage{pgf}
\usepackage{soul}
\usepackage{amsmath}
\usepackage{subcaption}
\usepackage{algorithm}
\usepackage{algpseudocode}
\usepackage{xcolor}
\usepackage{hyperref}
\usepackage{pgf}
\usepackage{array}
\usepackage{pifont}
\usepackage{multirow}
\usepackage{footnote}
\usepackage{graphicx}
\usepackage{subcaption}
\usepackage{booktabs}
\usepackage{varwidth}

\usepackage{tikz} 
\usetikzlibrary{shapes.geometric, positioning} 

\usepackage[most]{tcolorbox}   
\usepackage{xcolor}            
\usepackage{lipsum}            
\usepackage{microtype}         

\definecolor{darkblue}{rgb}{0.0, 0.0, 0.55} % RGB values for dark blue
\definecolor{darkgreen}{rgb}{0.0, 0.55, 0.0} % RGB values for dark blue

\lstset{
    language=Python,
    basicstyle=\ttfamily\scriptsize,  % Smaller font to help prevent overflow
    columns=fullflexible,
    showstringspaces=false,
    breaklines=true,          % Allow line wrapping
    breakatwhitespace=false,   % Prefer to break at whitespace
    keywordstyle=\color{darkblue}\bfseries, % Keywords in blue and bold
    stringstyle=\color{darkgreen},         % Strings in green
    commentstyle=\color{gray},         % Comments in gray
    numbers=left,                      % Line numbers on the left
    numberstyle=\tiny\color{gray},     % Line numbers in tiny gray font
    rulecolor=\color{black},           % Color of the frame border
}
\sisetup{
    output-decimal-marker = .,
    group-minimum-digits=6, % Set minimum number of digits for a group
    group-separator = {,}, % Set comma as the group separator
    table-number-alignment = right
}

\newif\ifcomments

\commentstrue

\ifcomments
\providecommand{\AS}[1]{\textbf{\textcolor{blue}{AS: #1}}}
\providecommand{\CD}[1]{\textbf{\textcolor{red}{CD: #1}}}
\else
\providecommand{\CC}[1]{}
\providecommand{\AS}[1]{}
\fi


%
\setlength\unitlength{1mm}
\newcommand{\twodots}{\mathinner {\ldotp \ldotp}}
% bb font symbols
\newcommand{\Rho}{\mathrm{P}}
\newcommand{\Tau}{\mathrm{T}}

\newfont{\bbb}{msbm10 scaled 700}
\newcommand{\CCC}{\mbox{\bbb C}}

\newfont{\bb}{msbm10 scaled 1100}
\newcommand{\CC}{\mbox{\bb C}}
\newcommand{\PP}{\mbox{\bb P}}
\newcommand{\RR}{\mbox{\bb R}}
\newcommand{\QQ}{\mbox{\bb Q}}
\newcommand{\ZZ}{\mbox{\bb Z}}
\newcommand{\FF}{\mbox{\bb F}}
\newcommand{\GG}{\mbox{\bb G}}
\newcommand{\EE}{\mbox{\bb E}}
\newcommand{\NN}{\mbox{\bb N}}
\newcommand{\KK}{\mbox{\bb K}}
\newcommand{\HH}{\mbox{\bb H}}
\newcommand{\SSS}{\mbox{\bb S}}
\newcommand{\UU}{\mbox{\bb U}}
\newcommand{\VV}{\mbox{\bb V}}


\newcommand{\yy}{\mathbbm{y}}
\newcommand{\xx}{\mathbbm{x}}
\newcommand{\zz}{\mathbbm{z}}
\newcommand{\sss}{\mathbbm{s}}
\newcommand{\rr}{\mathbbm{r}}
\newcommand{\pp}{\mathbbm{p}}
\newcommand{\qq}{\mathbbm{q}}
\newcommand{\ww}{\mathbbm{w}}
\newcommand{\hh}{\mathbbm{h}}
\newcommand{\vvv}{\mathbbm{v}}

% Vectors

\newcommand{\av}{{\bf a}}
\newcommand{\bv}{{\bf b}}
\newcommand{\cv}{{\bf c}}
\newcommand{\dv}{{\bf d}}
\newcommand{\ev}{{\bf e}}
\newcommand{\fv}{{\bf f}}
\newcommand{\gv}{{\bf g}}
\newcommand{\hv}{{\bf h}}
\newcommand{\iv}{{\bf i}}
\newcommand{\jv}{{\bf j}}
\newcommand{\kv}{{\bf k}}
\newcommand{\lv}{{\bf l}}
\newcommand{\mv}{{\bf m}}
\newcommand{\nv}{{\bf n}}
\newcommand{\ov}{{\bf o}}
\newcommand{\pv}{{\bf p}}
\newcommand{\qv}{{\bf q}}
\newcommand{\rv}{{\bf r}}
\newcommand{\sv}{{\bf s}}
\newcommand{\tv}{{\bf t}}
\newcommand{\uv}{{\bf u}}
\newcommand{\wv}{{\bf w}}
\newcommand{\vv}{{\bf v}}
\newcommand{\xv}{{\bf x}}
\newcommand{\yv}{{\bf y}}
\newcommand{\zv}{{\bf z}}
\newcommand{\zerov}{{\bf 0}}
\newcommand{\onev}{{\bf 1}}

% Matrices

\newcommand{\Am}{{\bf A}}
\newcommand{\Bm}{{\bf B}}
\newcommand{\Cm}{{\bf C}}
\newcommand{\Dm}{{\bf D}}
\newcommand{\Em}{{\bf E}}
\newcommand{\Fm}{{\bf F}}
\newcommand{\Gm}{{\bf G}}
\newcommand{\Hm}{{\bf H}}
\newcommand{\Id}{{\bf I}}
\newcommand{\Jm}{{\bf J}}
\newcommand{\Km}{{\bf K}}
\newcommand{\Lm}{{\bf L}}
\newcommand{\Mm}{{\bf M}}
\newcommand{\Nm}{{\bf N}}
\newcommand{\Om}{{\bf O}}
\newcommand{\Pm}{{\bf P}}
\newcommand{\Qm}{{\bf Q}}
\newcommand{\Rm}{{\bf R}}
\newcommand{\Sm}{{\bf S}}
\newcommand{\Tm}{{\bf T}}
\newcommand{\Um}{{\bf U}}
\newcommand{\Wm}{{\bf W}}
\newcommand{\Vm}{{\bf V}}
\newcommand{\Xm}{{\bf X}}
\newcommand{\Ym}{{\bf Y}}
\newcommand{\Zm}{{\bf Z}}

% Calligraphic

\newcommand{\Ac}{{\cal A}}
\newcommand{\Bc}{{\cal B}}
\newcommand{\Cc}{{\cal C}}
\newcommand{\Dc}{{\cal D}}
\newcommand{\Ec}{{\cal E}}
\newcommand{\Fc}{{\cal F}}
\newcommand{\Gc}{{\cal G}}
\newcommand{\Hc}{{\cal H}}
\newcommand{\Ic}{{\cal I}}
\newcommand{\Jc}{{\cal J}}
\newcommand{\Kc}{{\cal K}}
\newcommand{\Lc}{{\cal L}}
\newcommand{\Mc}{{\cal M}}
\newcommand{\Nc}{{\cal N}}
\newcommand{\nc}{{\cal n}}
\newcommand{\Oc}{{\cal O}}
\newcommand{\Pc}{{\cal P}}
\newcommand{\Qc}{{\cal Q}}
\newcommand{\Rc}{{\cal R}}
\newcommand{\Sc}{{\cal S}}
\newcommand{\Tc}{{\cal T}}
\newcommand{\Uc}{{\cal U}}
\newcommand{\Wc}{{\cal W}}
\newcommand{\Vc}{{\cal V}}
\newcommand{\Xc}{{\cal X}}
\newcommand{\Yc}{{\cal Y}}
\newcommand{\Zc}{{\cal Z}}

% Bold greek letters

\newcommand{\alphav}{\hbox{\boldmath$\alpha$}}
\newcommand{\betav}{\hbox{\boldmath$\beta$}}
\newcommand{\gammav}{\hbox{\boldmath$\gamma$}}
\newcommand{\deltav}{\hbox{\boldmath$\delta$}}
\newcommand{\etav}{\hbox{\boldmath$\eta$}}
\newcommand{\lambdav}{\hbox{\boldmath$\lambda$}}
\newcommand{\epsilonv}{\hbox{\boldmath$\epsilon$}}
\newcommand{\nuv}{\hbox{\boldmath$\nu$}}
\newcommand{\muv}{\hbox{\boldmath$\mu$}}
\newcommand{\zetav}{\hbox{\boldmath$\zeta$}}
\newcommand{\phiv}{\hbox{\boldmath$\phi$}}
\newcommand{\psiv}{\hbox{\boldmath$\psi$}}
\newcommand{\thetav}{\hbox{\boldmath$\theta$}}
\newcommand{\tauv}{\hbox{\boldmath$\tau$}}
\newcommand{\omegav}{\hbox{\boldmath$\omega$}}
\newcommand{\xiv}{\hbox{\boldmath$\xi$}}
\newcommand{\sigmav}{\hbox{\boldmath$\sigma$}}
\newcommand{\piv}{\hbox{\boldmath$\pi$}}
\newcommand{\rhov}{\hbox{\boldmath$\rho$}}
\newcommand{\upsilonv}{\hbox{\boldmath$\upsilon$}}

\newcommand{\Gammam}{\hbox{\boldmath$\Gamma$}}
\newcommand{\Lambdam}{\hbox{\boldmath$\Lambda$}}
\newcommand{\Deltam}{\hbox{\boldmath$\Delta$}}
\newcommand{\Sigmam}{\hbox{\boldmath$\Sigma$}}
\newcommand{\Phim}{\hbox{\boldmath$\Phi$}}
\newcommand{\Pim}{\hbox{\boldmath$\Pi$}}
\newcommand{\Psim}{\hbox{\boldmath$\Psi$}}
\newcommand{\Thetam}{\hbox{\boldmath$\Theta$}}
\newcommand{\Omegam}{\hbox{\boldmath$\Omega$}}
\newcommand{\Xim}{\hbox{\boldmath$\Xi$}}


% Sans Serif small case

\newcommand{\Gsf}{{\sf G}}

\newcommand{\asf}{{\sf a}}
\newcommand{\bsf}{{\sf b}}
\newcommand{\csf}{{\sf c}}
\newcommand{\dsf}{{\sf d}}
\newcommand{\esf}{{\sf e}}
\newcommand{\fsf}{{\sf f}}
\newcommand{\gsf}{{\sf g}}
\newcommand{\hsf}{{\sf h}}
\newcommand{\isf}{{\sf i}}
\newcommand{\jsf}{{\sf j}}
\newcommand{\ksf}{{\sf k}}
\newcommand{\lsf}{{\sf l}}
\newcommand{\msf}{{\sf m}}
\newcommand{\nsf}{{\sf n}}
\newcommand{\osf}{{\sf o}}
\newcommand{\psf}{{\sf p}}
\newcommand{\qsf}{{\sf q}}
\newcommand{\rsf}{{\sf r}}
\newcommand{\ssf}{{\sf s}}
\newcommand{\tsf}{{\sf t}}
\newcommand{\usf}{{\sf u}}
\newcommand{\wsf}{{\sf w}}
\newcommand{\vsf}{{\sf v}}
\newcommand{\xsf}{{\sf x}}
\newcommand{\ysf}{{\sf y}}
\newcommand{\zsf}{{\sf z}}


% mixed symbols

\newcommand{\sinc}{{\hbox{sinc}}}
\newcommand{\diag}{{\hbox{diag}}}
\renewcommand{\det}{{\hbox{det}}}
\newcommand{\trace}{{\hbox{tr}}}
\newcommand{\sign}{{\hbox{sign}}}
\renewcommand{\arg}{{\hbox{arg}}}
\newcommand{\var}{{\hbox{var}}}
\newcommand{\cov}{{\hbox{cov}}}
\newcommand{\Ei}{{\rm E}_{\rm i}}
\renewcommand{\Re}{{\rm Re}}
\renewcommand{\Im}{{\rm Im}}
\newcommand{\eqdef}{\stackrel{\Delta}{=}}
\newcommand{\defines}{{\,\,\stackrel{\scriptscriptstyle \bigtriangleup}{=}\,\,}}
\newcommand{\<}{\left\langle}
\renewcommand{\>}{\right\rangle}
\newcommand{\herm}{{\sf H}}
\newcommand{\trasp}{{\sf T}}
\newcommand{\transp}{{\sf T}}
\renewcommand{\vec}{{\rm vec}}
\newcommand{\Psf}{{\sf P}}
\newcommand{\SINR}{{\sf SINR}}
\newcommand{\SNR}{{\sf SNR}}
\newcommand{\MMSE}{{\sf MMSE}}
\newcommand{\REF}{{\RED [REF]}}

% Markov chain
\usepackage{stmaryrd} % for \mkv 
\newcommand{\mkv}{-\!\!\!\!\minuso\!\!\!\!-}

% Colors

\newcommand{\RED}{\color[rgb]{1.00,0.10,0.10}}
\newcommand{\BLUE}{\color[rgb]{0,0,0.90}}
\newcommand{\GREEN}{\color[rgb]{0,0.80,0.20}}

%%%%%%%%%%%%%%%%%%%%%%%%%%%%%%%%%%%%%%%%%%
\usepackage{hyperref}
\hypersetup{
    bookmarks=true,         % show bookmarks bar?
    unicode=false,          % non-Latin characters in AcrobatÕs bookmarks
    pdftoolbar=true,        % show AcrobatÕs toolbar?
    pdfmenubar=true,        % show AcrobatÕs menu?
    pdffitwindow=false,     % window fit to page when opened
    pdfstartview={FitH},    % fits the width of the page to the window
%    pdftitle={My title},    % title
%    pdfauthor={Author},     % author
%    pdfsubject={Subject},   % subject of the document
%    pdfcreator={Creator},   % creator of the document
%    pdfproducer={Producer}, % producer of the document
%    pdfkeywords={keyword1} {key2} {key3}, % list of keywords
    pdfnewwindow=true,      % links in new window
    colorlinks=true,       % false: boxed links; true: colored links
    linkcolor=red,          % color of internal links (change box color with linkbordercolor)
    citecolor=green,        % color of links to bibliography
    filecolor=blue,      % color of file links
    urlcolor=blue           % color of external links
}
%%%%%%%%%%%%%%%%%%%%%%%%%%%%%%%%%%%%%%%%%%%



% cleveref wants to be loaded very late
\usepackage[capitalise,noabbrev]{cleveref}
\crefname{line}{line}{lines}

\sisetup{detect-all} % always use the surrounding font-family, face, etc
\sisetup{separate-uncertainty = true} % print 1.234±0.005 instead of 1.234(5)

\begin{document}
%
% \title{Neuro-Symbolic Uncertainty Measures for Code Generation: A Semantic and Symbolic Approach} 
%\title{Information-Theoretic Measures for Assessing Code Generation Quality} 
\title{Assessing Correctness in LLM-Based Code Generation via Uncertainty Estimation}
%
\titlerunning{Assessing Correctness in LLM-Based Code Generation}
% If the paper title is too long for the running head, you can set
% an abbreviated paper title here
%
% \author{Arindam Sharma~\email{arindam.sharma@bristol.ac.uk} \and Cristina David~\email{cristina.david@bristol.ac.uk}} 
% % \author{Anonymous} 
% %
% \authorrunning{A. Sharma et al.}
% % \authorrunning{Anon et al.}
% % First names are abbreviated in the running head.
% % If there are more than two authors, 'et al.' is used.
% %
% \institute{University of Bristol, Bristol, UK}
\author{Arindam Sharma \and Cristina David}
\authorrunning{A. Sharma et al.}
\institute{%
  University of Bristol \\
  \email{\{arindam.sharma\}\{cristina.david\}@bristol.ac.uk}\\[0.5ex]
}
% \institute{Anonymous}
%
\maketitle              % typeset the header of the contribution
%
\keywords{Code generation  \and Correctness \and Large Language Models  \and Entropy.}

\begin{abstract}
In this work, we explore uncertainty estimation as a proxy for correctness in LLM-generated code. 
To this end, we adapt two state-of-the-art techniques from natural language generation---one based on entropy and another on mutual information---to the domain of code generation. 
Given the distinct semantic properties of code, we introduce modifications, including a semantic equivalence check based on symbolic execution. 
Our findings indicate a strong correlation between the uncertainty computed through these techniques and correctness, highlighting the potential of uncertainty estimation for quality assessment. 
Additionally, we propose a simplified version of the entropy-based method that assumes a uniform distribution over the LLM's responses, demonstrating comparable effectiveness. 
Using these techniques, we develop an abstention policy that prevents the model from making predictions when uncertainty is high, reducing incorrect outputs to near zero. 
Our evaluation on the \livecodebench dataset~\cite{livecodebench} shows that our approach significantly outperforms a baseline relying solely on LLM-reported log-probabilities. 
\end{abstract}

\section{Introduction}
\label{sec:intro}
\section{Introduction}
\label{sec:introduction}
The business processes of organizations are experiencing ever-increasing complexity due to the large amount of data, high number of users, and high-tech devices involved \cite{martin2021pmopportunitieschallenges, beerepoot2023biggestbpmproblems}. This complexity may cause business processes to deviate from normal control flow due to unforeseen and disruptive anomalies \cite{adams2023proceddsriftdetection}. These control-flow anomalies manifest as unknown, skipped, and wrongly-ordered activities in the traces of event logs monitored from the execution of business processes \cite{ko2023adsystematicreview}. For the sake of clarity, let us consider an illustrative example of such anomalies. Figure \ref{FP_ANOMALIES} shows a so-called event log footprint, which captures the control flow relations of four activities of a hypothetical event log. In particular, this footprint captures the control-flow relations between activities \texttt{a}, \texttt{b}, \texttt{c} and \texttt{d}. These are the causal ($\rightarrow$) relation, concurrent ($\parallel$) relation, and other ($\#$) relations such as exclusivity or non-local dependency \cite{aalst2022pmhandbook}. In addition, on the right are six traces, of which five exhibit skipped, wrongly-ordered and unknown control-flow anomalies. For example, $\langle$\texttt{a b d}$\rangle$ has a skipped activity, which is \texttt{c}. Because of this skipped activity, the control-flow relation \texttt{b}$\,\#\,$\texttt{d} is violated, since \texttt{d} directly follows \texttt{b} in the anomalous trace.
\begin{figure}[!t]
\centering
\includegraphics[width=0.9\columnwidth]{images/FP_ANOMALIES.png}
\caption{An example event log footprint with six traces, of which five exhibit control-flow anomalies.}
\label{FP_ANOMALIES}
\end{figure}

\subsection{Control-flow anomaly detection}
Control-flow anomaly detection techniques aim to characterize the normal control flow from event logs and verify whether these deviations occur in new event logs \cite{ko2023adsystematicreview}. To develop control-flow anomaly detection techniques, \revision{process mining} has seen widespread adoption owing to process discovery and \revision{conformance checking}. On the one hand, process discovery is a set of algorithms that encode control-flow relations as a set of model elements and constraints according to a given modeling formalism \cite{aalst2022pmhandbook}; hereafter, we refer to the Petri net, a widespread modeling formalism. On the other hand, \revision{conformance checking} is an explainable set of algorithms that allows linking any deviations with the reference Petri net and providing the fitness measure, namely a measure of how much the Petri net fits the new event log \cite{aalst2022pmhandbook}. Many control-flow anomaly detection techniques based on \revision{conformance checking} (hereafter, \revision{conformance checking}-based techniques) use the fitness measure to determine whether an event log is anomalous \cite{bezerra2009pmad, bezerra2013adlogspais, myers2018icsadpm, pecchia2020applicationfailuresanalysispm}. 

The scientific literature also includes many \revision{conformance checking}-independent techniques for control-flow anomaly detection that combine specific types of trace encodings with machine/deep learning \cite{ko2023adsystematicreview, tavares2023pmtraceencoding}. Whereas these techniques are very effective, their explainability is challenging due to both the type of trace encoding employed and the machine/deep learning model used \cite{rawal2022trustworthyaiadvances,li2023explainablead}. Hence, in the following, we focus on the shortcomings of \revision{conformance checking}-based techniques to investigate whether it is possible to support the development of competitive control-flow anomaly detection techniques while maintaining the explainable nature of \revision{conformance checking}.
\begin{figure}[!t]
\centering
\includegraphics[width=\columnwidth]{images/HIGH_LEVEL_VIEW.png}
\caption{A high-level view of the proposed framework for combining \revision{process mining}-based feature extraction with dimensionality reduction for control-flow anomaly detection.}
\label{HIGH_LEVEL_VIEW}
\end{figure}

\subsection{Shortcomings of \revision{conformance checking}-based techniques}
Unfortunately, the detection effectiveness of \revision{conformance checking}-based techniques is affected by noisy data and low-quality Petri nets, which may be due to human errors in the modeling process or representational bias of process discovery algorithms \cite{bezerra2013adlogspais, pecchia2020applicationfailuresanalysispm, aalst2016pm}. Specifically, on the one hand, noisy data may introduce infrequent and deceptive control-flow relations that may result in inconsistent fitness measures, whereas, on the other hand, checking event logs against a low-quality Petri net could lead to an unreliable distribution of fitness measures. Nonetheless, such Petri nets can still be used as references to obtain insightful information for \revision{process mining}-based feature extraction, supporting the development of competitive and explainable \revision{conformance checking}-based techniques for control-flow anomaly detection despite the problems above. For example, a few works outline that token-based \revision{conformance checking} can be used for \revision{process mining}-based feature extraction to build tabular data and develop effective \revision{conformance checking}-based techniques for control-flow anomaly detection \cite{singh2022lapmsh, debenedictis2023dtadiiot}. However, to the best of our knowledge, the scientific literature lacks a structured proposal for \revision{process mining}-based feature extraction using the state-of-the-art \revision{conformance checking} variant, namely alignment-based \revision{conformance checking}.

\subsection{Contributions}
We propose a novel \revision{process mining}-based feature extraction approach with alignment-based \revision{conformance checking}. This variant aligns the deviating control flow with a reference Petri net; the resulting alignment can be inspected to extract additional statistics such as the number of times a given activity caused mismatches \cite{aalst2022pmhandbook}. We integrate this approach into a flexible and explainable framework for developing techniques for control-flow anomaly detection. The framework combines \revision{process mining}-based feature extraction and dimensionality reduction to handle high-dimensional feature sets, achieve detection effectiveness, and support explainability. Notably, in addition to our proposed \revision{process mining}-based feature extraction approach, the framework allows employing other approaches, enabling a fair comparison of multiple \revision{conformance checking}-based and \revision{conformance checking}-independent techniques for control-flow anomaly detection. Figure \ref{HIGH_LEVEL_VIEW} shows a high-level view of the framework. Business processes are monitored, and event logs obtained from the database of information systems. Subsequently, \revision{process mining}-based feature extraction is applied to these event logs and tabular data input to dimensionality reduction to identify control-flow anomalies. We apply several \revision{conformance checking}-based and \revision{conformance checking}-independent framework techniques to publicly available datasets, simulated data of a case study from railways, and real-world data of a case study from healthcare. We show that the framework techniques implementing our approach outperform the baseline \revision{conformance checking}-based techniques while maintaining the explainable nature of \revision{conformance checking}.

In summary, the contributions of this paper are as follows.
\begin{itemize}
    \item{
        A novel \revision{process mining}-based feature extraction approach to support the development of competitive and explainable \revision{conformance checking}-based techniques for control-flow anomaly detection.
    }
    \item{
        A flexible and explainable framework for developing techniques for control-flow anomaly detection using \revision{process mining}-based feature extraction and dimensionality reduction.
    }
    \item{
        Application to synthetic and real-world datasets of several \revision{conformance checking}-based and \revision{conformance checking}-independent framework techniques, evaluating their detection effectiveness and explainability.
    }
\end{itemize}

The rest of the paper is organized as follows.
\begin{itemize}
    \item Section \ref{sec:related_work} reviews the existing techniques for control-flow anomaly detection, categorizing them into \revision{conformance checking}-based and \revision{conformance checking}-independent techniques.
    \item Section \ref{sec:abccfe} provides the preliminaries of \revision{process mining} to establish the notation used throughout the paper, and delves into the details of the proposed \revision{process mining}-based feature extraction approach with alignment-based \revision{conformance checking}.
    \item Section \ref{sec:framework} describes the framework for developing \revision{conformance checking}-based and \revision{conformance checking}-independent techniques for control-flow anomaly detection that combine \revision{process mining}-based feature extraction and dimensionality reduction.
    \item Section \ref{sec:evaluation} presents the experiments conducted with multiple framework and baseline techniques using data from publicly available datasets and case studies.
    \item Section \ref{sec:conclusions} draws the conclusions and presents future work.
\end{itemize}

\section{Motivating Example}
\label{sec:motivating}
To illustrate the relationship between uncertainty and correctness for code generation, we refer to the problem described by the prompt in Figure~\ref{fig:sampleproblem}. 

\begin{figure}[htbp]
    \centering
    \begin{tcolorbox}[
        enhanced,
        width=0.85\textwidth,
        colback=white,
        colframe=blue!50!black,
        boxrule=1pt,
        arc=5pt,              
        title=LLM Prompt,
        fonttitle=\bfseries,
        attach boxed title to top center={yshift=-2mm},
        varwidth boxed title=0.7\linewidth
      ]
    \textbf{User Prompt:}

    \vspace{0.5em}
    You are given a \texttt{0-indexed} array of strings \texttt{details}. Each element of \texttt{details} provides information about a given passenger compressed into a string of length 15. The format is as follows:
    \begin{itemize}
        \item The first 10 characters: the passenger's phone number
        \item The 11th character: the passenger's gender
        \item The 12th and 13th characters: the passenger's age
        \item The 14th and 15th characters: the seat allotted
    \end{itemize}
    Return the number of passengers whose age is \textbf{strictly greater} than 60.

    % \vspace{0.5em}
    % \textbf{Example:}
    % \begin{verbatim}
    % Input: details = ["7869194042M58A", "0741234567F75B", "6280984567F13C"]
    % Output: 1
    % Explanation: Only one passenger is older than 60.
    % \end{verbatim}

    \vspace{0.5em}
    \textbf{Task:} 
    Please provide a function in your preferred programming language that takes \texttt{details} and returns how many passengers are older than 60.
    \end{tcolorbox}
    \caption{An example LLM prompt for the ``number-of-senior-citizens'' coding problem.}
    \label{fig:sampleproblem}
\end{figure}

\begin{figure}[ht!]
    \centering
    
    \begin{subfigure}[t]{0.3\textwidth}
        
        \lstinputlisting{code/good_llm_snippet1.py}
        \vspace{2.5em} 
        \caption{Snippet 1}
        \label{lst:good1}
    \end{subfigure}
    \hfill
    \begin{subfigure}[t]{0.3\textwidth}
        
        \lstinputlisting{code/good_llm_snippet2.py}
        \vspace{3.4em}
        \caption{Snippet 2}
        \label{lst:good2}
    \end{subfigure}
    \hfill
    \begin{subfigure}[t]{0.3\textwidth}
        
        \lstinputlisting{code/good_llm_snippet3.py}
        \vspace{2.5em}
        \caption{Snippet 3}
        \label{lst:good3}
    \end{subfigure}

    \caption{Three code snippets from the \gptturbo, all correct, semantically equivalent.}
    \label{fig:good-llm-snippets}
\end{figure}

%Figure~\ref{fig:good-llm-snippets} shows the top 3 responses from the well-tuned model \ie \gptturbo. Closer observation shows that the three responses shown in Listing~\ref{lst:good1}, Listing~\ref{lst:good2} and Listing~\ref{lst:good3} of Figure~\ref{fig:good-llm-snippets} are semantically equivalent thereby indicating that the model has high confidence in its response. It also turns out to be the case that these responses all pass the testsuite for this problem. 

\begin{figure}[ht!]
    \centering
    
    \begin{subfigure}[t]{0.3\textwidth}
       
        \vspace{0.5em} 
        \lstinputlisting{code/bad_llm_snippet1.py}
        \caption{Snippet 1}
        \label{lst:bad1}
    \end{subfigure}
    \hfill
    \begin{subfigure}[t]{0.3\textwidth}
        
        \vspace{0.5em}
        \lstinputlisting{code/bad_llm_snippet2.py}
        \caption{Snippet 2}
        \label{lst:bad2}
    \end{subfigure}
    \hfill
    \begin{subfigure}[t]{0.3\textwidth}
        
        \vspace{0.5em}
        \lstinputlisting{code/bad_llm_snippet3.py}
        \caption{Snippet 3}
        \label{lst:bad3}
    \end{subfigure}

    \caption{Three code snippets from the \salesforce/\codegenmonoC, all incorrect, and semantically distinct.}
    \label{fig:bad-llm-snippets}
\end{figure}

%However, for a contemporary, worse-performing model from \salesforce (\codegenmonoC), the 3 responses shown in Listing~\ref{lst:bad1}, Listing~\ref{lst:bad2} and Listing~\ref{lst:bad3} of Figure~\ref{fig:bad-llm-snippets}, are all semantically different and hence fall in their own respective clusters. Unsurprisingly, all 3 of these responses are incorrect and do not pass any testcases of the testsuite. 

Figure~\ref{fig:good-llm-snippets} displays the top three responses generated by our first model \gptturbo, all of which are semantically equivalent and correct. 
In contrast, Figure~\ref{fig:bad-llm-snippets} presents the first three responses generated by \salesforce (\codegenmonoC), all of which are incorrect.

The log-probabilities reported by the LLMs are as follows: for the better performing \gptturbo, the three snippets have log-probabilities of \GPTsnipLogProbA, \GPTsnipLogProbB, and \GPTsnipLogProbC, respectively. 
For \codegenmonoC, the log-probabilities are \SFsnipLogProbA, \SFsnipLogProbB, and \SFsnipLogProbC. 
However, these log-probabilities alone are insufficient as proxies for correctness. 
In particular, there is no clear way to infer from the log-probabilities that all the snippets generated by \gptturbo are correct, while all those generated by \codegenmonoC are incorrect.

%At first glance, the differences in responses might suggest that both LLMs exhibit low confidence in their responses. However, a closer examination of the snippets in Listings~\ref{lst:good1}, \ref{lst:good2}, and \ref{lst:good3} in Figure~\ref{fig:good-llm-snippets} reveals that, although syntactically distinct, they are semantically equivalent. In contrast, the responses in Figure~\ref{fig:bad-llm-snippets} are both syntactically and semantically distinct.

The techniques outlined in Section~\ref{sec:symex} and Section~\ref{sec:mi} effectively identify the semantic equivalence of the snippets in Figure~\ref{fig:good-llm-snippets} and incorporate this information during the computation of entropy and mutual information, respectively. 
As shown later in the paper, both methods ultimately conclude that \gptturbo exhibits high confidence in its responses, while the model from \salesforce demonstrates significant uncertainty. 
Furthermore, we will show that uncertainty negatively correlates with correctness.
%When using uncertainty as a proxy for correctness, the results suggest that \gptturbo's responses are likely accurate, whereas \salesforce's are likely flawed. This conclusion is corroborated by the test suite results: all three responses from \gptturbo pass successfully, whereas none of \salesforce's responses do.


%\section{Background}
%\label{sec:background}
%\section{Background}\label{sec:backgrnd}

\subsection{Cold Start Latency and Mitigation Techniques}

Traditional FaaS platforms mitigate cold starts through snapshotting, lightweight virtualization, and warm-state management. Snapshot-based methods like \textbf{REAP} and \textbf{Catalyzer} reduce initialization time by preloading or restoring container states but require significant memory and I/O resources, limiting scalability~\cite{dong_catalyzer_2020, ustiugov_benchmarking_2021}. Lightweight virtualization solutions, such as \textbf{Firecracker} microVMs, achieve fast startup times with strong isolation but depend on robust infrastructure, making them less adaptable to fluctuating workloads~\cite{agache_firecracker_2020}. Warm-state management techniques like \textbf{Faa\$T}~\cite{romero_faa_2021} and \textbf{Kraken}~\cite{vivek_kraken_2021} keep frequently invoked containers ready, balancing readiness and cost efficiency under predictable workloads but incurring overhead when demand is erratic~\cite{romero_faa_2021, vivek_kraken_2021}. While these methods perform well in resource-rich cloud environments, their resource intensity challenges applicability in edge settings.

\subsubsection{Edge FaaS Perspective}

In edge environments, cold start mitigation emphasizes lightweight designs, resource sharing, and hybrid task distribution. Lightweight execution environments like unikernels~\cite{edward_sock_2018} and \textbf{Firecracker}~\cite{agache_firecracker_2020}, as used by \textbf{TinyFaaS}~\cite{pfandzelter_tinyfaas_2020}, minimize resource usage and initialization delays but require careful orchestration to avoid resource contention. Function co-location, demonstrated by \textbf{Photons}~\cite{v_dukic_photons_2020}, reduces redundant initializations by sharing runtime resources among related functions, though this complicates isolation in multi-tenant setups~\cite{v_dukic_photons_2020}. Hybrid offloading frameworks like \textbf{GeoFaaS}~\cite{malekabbasi_geofaas_2024} balance edge-cloud workloads by offloading latency-tolerant tasks to the cloud and reserving edge resources for real-time operations, requiring reliable connectivity and efficient task management. These edge-specific strategies address cold starts effectively but introduce challenges in scalability and orchestration.

\subsection{Predictive Scaling and Caching Techniques}

Efficient resource allocation is vital for maintaining low latency and high availability in serverless platforms. Predictive scaling and caching techniques dynamically provision resources and reduce cold start latency by leveraging workload prediction and state retention.
Traditional FaaS platforms use predictive scaling and caching to optimize resources, employing techniques (OFC, FaasCache) to reduce cold starts. However, these methods rely on centralized orchestration and workload predictability, limiting their effectiveness in dynamic, resource-constrained edge environments.



\subsubsection{Edge FaaS Perspective}

Edge FaaS platforms adapt predictive scaling and caching techniques to constrain resources and heterogeneous environments. \textbf{EDGE-Cache}~\cite{kim_delay-aware_2022} uses traffic profiling to selectively retain high-priority functions, reducing memory overhead while maintaining readiness for frequent requests. Hybrid frameworks like \textbf{GeoFaaS}~\cite{malekabbasi_geofaas_2024} implement distributed caching to balance resources between edge and cloud nodes, enabling low-latency processing for critical tasks while offloading less critical workloads. Machine learning methods, such as clustering-based workload predictors~\cite{gao_machine_2020} and GRU-based models~\cite{guo_applying_2018}, enhance resource provisioning in edge systems by efficiently forecasting workload spikes. These innovations effectively address cold start challenges in edge environments, though their dependency on accurate predictions and robust orchestration poses scalability challenges.

\subsection{Decentralized Orchestration, Function Placement, and Scheduling}

Efficient orchestration in serverless platforms involves workload distribution, resource optimization, and performance assurance. While traditional FaaS platforms rely on centralized control, edge environments require decentralized and adaptive strategies to address unique challenges such as resource constraints and heterogeneous hardware.



\subsubsection{Edge FaaS Perspective}

Edge FaaS platforms adopt decentralized and adaptive orchestration frameworks to meet the demands of resource-constrained environments. Systems like \textbf{Wukong} distribute scheduling across edge nodes, enhancing data locality and scalability while reducing network latency. Lightweight frameworks such as \textbf{OpenWhisk Lite}~\cite{kravchenko_kpavelopenwhisk-light_2024} optimize resource allocation by decentralizing scheduling policies, minimizing cold starts and latency in edge setups~\cite{benjamin_wukong_2020}. Hybrid solutions like \textbf{OpenFaaS}~\cite{noauthor_openfaasfaas_2024} and \textbf{EdgeMatrix}~\cite{shen_edgematrix_2023} combine edge-cloud orchestration to balance resource utilization, retaining latency-sensitive functions at the edge while offloading non-critical workloads to the cloud. While these approaches improve flexibility, they face challenges in maintaining coordination and ensuring consistent performance across distributed nodes.



%\section{Our Techniques for Assessing Uncertainty in Code Generation}
%\label{sec:tech}
\section{Semantic Equivalence Based Program Clustering}
\label{sec:symexclustering}

The NLG techniques proposed by Kuhn~\etal~\cite{kuhnsemantic} and Abbasi~\etal~\cite{abbasi2024believe} rely on semantic clustering, where semantically equivalent programs are grouped together. 
Achieving this requires an effective method for assessing program equivalence. Kuhn~\etal employ the DeBERTa-large model~\cite{he2020deberta} for this task, while Abbasi~\etal determine equivalence using an F1 score based on token inclusion~\cite{DBLP:journals/corr/JoshiCWZ17}.

In the domain of code generation, program equivalence has a precise definition: two programs are considered equivalent if they produce identical behavior for all possible inputs. 
Consequently, a domain-specific equivalence check is required.
In this paper, we base the semantic equivalence check on \emph{symbolic execution}, where, instead of executing a program with concrete inputs, \emph{symbolic variables} are used to represent inputs, generating constraints that describe the program's behavior across all possible input values~\cite{symex_klee}.

The particular flavor of symbolic execution we use in this work is inspired by the lightweight \emph{peer architecture} described in Bruni~\etal~\cite{Bruni2011APA}. 
Unlike traditional approaches that require building a standalone symbolic interpreter, this architecture embeds the symbolic execution engine as a lightweight library operating alongside the target program. 
Their design is based on the insight that languages that provide the ability to dynamically dispatch primitive operations (\eg Python) allow symbolic values to behave as native values and be tracked at runtime.

Symbolic execution typically traverses the program's control flow graph, maintaining a symbolic state consisting of \emph{path constraints} (\ie logical conditions that must be satisfied for a given execution path to be feasible) and \emph{symbolic expressions} (\ie representations of program variables as functions of the symbolic inputs). 

\paragraph{Equivalence check.} Given two code snippets \(s^{(1)}\) and \(s^{(2)}\), we check semantic equivalence between them by comparing their symbolic traces. 
One such symbolic trace, \eg \(T(s^{(1)})\), consists of the corresponding path constraint and symbolic expressions denoting all the variables encountered on the corresponding execution path. 
Intuitively, for two code snippets to be semantically equivalent, all their corresponding symbolic traces must align. 
Specifically, for each path constraint, the traces produced by both snippets must be identical meaning that there is no concrete counterexample input for which the execution of the two snippets diverges.

%\CD{I'm not sure whether it's worth formalising this a bit.}

%Semantic equivalence between two code snippets \(s^{(i)}\) and \(s^{(j)}\) is determined by comparing their symbolic traces, \(T(s^{(i)})\) and \(T(s^{(j)})\), respectively:
%\begin{multline}
%    T(s^{(i)}) \equiv T(s^{(j)}) \iff \text{Path Constraints and Symbolic Expressions of } \\
%    s^{(i)} \text{ and } s^{(j)} \text{ are identical.}
%\end{multline}

Since program equivalence is undecidable in general, we perform a bounded equivalence check. 
This approach verifies that no counterexample input exists when exploring traces up to a given depth.

%Exact equivalence is often \emph{undecidable} due to the complexity of symbolic traces. 
%Instead, we employ \emph{subsumption}, where one trace subsumes another if all behaviours of the latter are captured by the former. 
%This allows us to approximate equivalence effectively.

%By using this lightweight symbolic execution approach, our clustering methodology emphasizes functional semantics, avoiding overfitting to syntactic similarities. 
%This methodology strikes a balance between precision and efficiency, leveraging the extensibility and simplicity of the peer architecture to scale across diverse programming scenarios.


\begin{algorithm}[ht!]
    \caption{Clustering with Symbolic Execution}
    \label{alg:clustering}
    \begin{algorithmic}[1]
    \Require Set of generated code snippets $\{s^{(1)}, \ldots, s^{(M)}\}$
    \Ensure Clusters of semantically equivalent snippets $C = \{c_1, c_2, \ldots, c_k\}$
    
    \State Initialize an empty cluster set $C \gets \emptyset$, and an equivalence map $E \gets \emptyset$ \label{alg:clustering:init}
    
    \For{each snippet $s^{(i)}$}
        \If{$s^{(i)}$ is invalid}
            \State $E[s^{(i)}] \gets \{\,s^{(i)}\}$ 
            \Comment{Assign invalid snippet to its own equivalence class}
        \EndIf
    \EndFor
    
    \For{each pair of valid snippets $(s^{(i)}, s^{(j)})$} \label{alg:clustering:pairwise}
        \State Perform symbolic execution on $s^{(i)}$ and $s^{(j)}$ to extract traces $T(s^{(i)})$ and $T(s^{(j)})$ \label{alg:clustering:trace}
        \If{$T(s^{(i)}) \equiv T(s^{(j)})$} \label{alg:clustering:check}
            \State $E[s^{(i)}] \gets E[s^{(i)}] \cup \{\,s^{(j)}\}$
            \State $E[s^{(j)}] \gets E[s^{(j)}] \cup \{\,s^{(i)}\}$ \label{alg:clustering:update}
        \EndIf
        \State \Comment{Enforce transitivity of equivalences}
        \If{$s^{(i)} \sim s^{(j)}$ and $s^{(j)} \sim s^{(k)}$ for some $s^{(k)}$}
            \State $E[s^{(i)}] \gets E[s^{(i)}] \cup \{\,s^{(k)}\}$
            \State $E[s^{(k)}] \gets E[s^{(k)}] \cup \{\,s^{(i)}\}$
        \EndIf
    \EndFor
    
    \State Identify equivalence classes in $E$ to form final clusters $C$ \label{alg:clustering:extract}
    
    \State \Return $C$ \label{alg:clustering:return}
    \end{algorithmic}
    \end{algorithm}

%\CD{we need to modify the alg so that it's obvious that we are talking about sets of traces at line 8.}

\paragraph{Semantic clustering.}
Algorithm~\ref{alg:clustering} illustrates how to cluster code snippets based on their functional semantics, with an additional check for invalid snippets. 
We first create empty structures for storing the final clusters ($C$) and an equivalence map ($E$) to track relationships (line~\ref{alg:clustering:init}). 

Next, in the \emph{invalid snippet handling phase}, each code snippet $s^{(i)}$ is examined and if it is detected to be invalid, it is immediately placed in its own equivalence class in $E$ and is thus isolated from further consideration. 

In the \emph{pairwise comparison phase} (line~\ref{alg:clustering:pairwise}), each pair of \emph{valid} snippets $(s^{(i)}, s^{(j)})$ is symbolically executed to produce traces $T(s^{(i)})$ and $T(s^{(j)})$ (line~\ref{alg:clustering:trace}). 
If the traces are equivalent (line~\ref{alg:clustering:check}), indicating identical functional behavior, both snippets are added to each other's equivalence classes (line~\ref{alg:clustering:update}). 
In reality, $T(s^{(i)})$ and $T(s^{(j)})$ actually denote sets of traces, and the equivalence check involves comparing individual traces from each set that share the same path constraint.
For brevity, in Algorithm~\ref{alg:clustering}, we represent this as $T(s^{(i)}) \equiv T(s^{(j)})$.
To maintain consistency, transitivity is enforced: if $s^{(i)}$ is equivalent to $s^{(j)}$, and $s^{(j)}$ is equivalent to $s^{(k)}$, then $s^{(i)}$ must also be in the same equivalence class as $s^{(k)}$. 

Finally, the equivalence map $E$ is processed to derive the clusters themselves (line~\ref{alg:clustering:extract}), and the resulting set of clusters is returned (line~\ref{alg:clustering:return}). 
% By isolating invalid snippets in single-item clusters, the algorithm cleanly separates non-functional or syntactically invalid code from semantically consistent groups. 
% This ensures that the final clustering reflects the functional semantics of valid snippets while transparently segregating invalid code. 

\section{Estimating the Probability Distribution of LLM Responses}
\label{sec:probcomp}

Both NLG techniques we adapt for code generation, at certain points, query the LLM, sample responses along with the log-probabilities of their tokens, and apply a softmax-style normalization to interpret them as a valid probability distribution.
However, since the LLM responses in this setting are programs, they are longer than the natural language responses used in the original studies---while the evaluation for Kuhn~\etal and Abbasi~\etal considered question-answer datasets typically involving one word answers, the programs produced in this work are around 200 tokens per response. 

The probability of a response is represented as the joint probability of its tokens, meaning that it decreases exponentially with length, often leading to \emph{numerical underflow}. 
This ultimately compromises the effectiveness of the technique. 
For instance, if the probabilities for all response programs underflow, then softmax returns NaN, which then propagates through the computation.

To address the issue of exponentially decaying probabilities, we propose two methods for approximating the probability distribution of LLM responses, as outlined below.

\paragraph{Length normalization.}

%The techniques we propose require computing the probability of responses generated by language models, where the probability of a response is represented as the joint probability of its tokens. 
%However, for longer responses, this probability decreases exponentially with length, which adversely impacts our estimation of uncertainty.

%In NLG tasks such as those addressed by prior works~\cite{kuhnsemantic,abbasi2024believe}, this issue is less pronounced because the goal in their problem domains (\eg TriviaQA) is to exactly match short reference answers. 
%In contrast, for code generation, the outputs are often longer. 
%While the evaluation for Kuhn~\etal~\cite{kuhnsemantic} and Abbasi~\etal~\cite{abbasi2024believe} considered question-answer datasets while typically involve one word answers, the program snippets produced in this work were typically around 200 tokens per response. 
% \CD{For our experiments, the average number of tokens in the results generated by the LLM is ...}. 
%Notably, while it is true that accuracy tends to decrease with length, existing research demonstrates that LLMs can generate high-quality code snippets, even up to 100 lines~\cite{codetranslation2}. %Consequently, the drastic reduction in probability with length disproportionately affects these scenarios. 

One solution is to use \emph{length normalization}~\cite{DBLP:conf/wmt/MurrayC18,DBLP:conf/aclnmt/KoehnK17}, more precisely length-normalizing the log probability of a program, a technique  used by other existing works, \eg to compute length-normalized predictive entropy~\cite{DBLP:conf/iclr/MalininG21}. %This also allows compar uncertainties of sequences of different length

More concretely, to compute a length-normalized probability from log probabilities, we begin by calculating the sum of the log probabilities. 
Let \(\ell_1, \ell_2, \dots, \ell_n\) denote the log probabilities associated with each token in the sequence that forms the response.
The log probability of the response is given by:
$S = \sum_{i=1}^{n} \ell_i$.
%
Next, the log probability is normalized by the sequence length \(L\) and the normalizing factor \(\gamma\). 
The normalized log probability is computed as:
$\ell_{\text{norm}} = \frac{S}{L \cdot \gamma}$.
%
Finally, the normalized probability \(P\) of the response is obtained by exponentiating the normalized log probability:
$P = e^{\ell_{\text{norm}}}$.

Intuitively, when using length-normalization in the context of uncertainty computation, probabilities remain comparable across responses of different lengths, whereas uncertainty is linked to the semantic differences between responses. %In our experimental evaluation, we compute uncertainty measures both with length-normalization and without.

\paragraph{Uniform distribution of LLM-generated responses.}
Intuitively, when multiple LLM responses are semantically equivalent, it indicates a higher degree of certainty in the (semantics of the) generated output.
To test this intuition, we propose disregarding the log-probabilities reported by the LLM and instead assuming a uniform distribution over the sampled responses. 
Specifically, if we sample $n$ responses, each response is assigned an equal probability of $1/n$.

Our experimental results demonstrated that using the semantic equivalence-based approach (described in Section~\ref{sec:symex}) in conjunction with this distribution reveals a negative correlation between the LLM's uncertainty and correctness---see Section~\ref{sec:results-discussion}. 
Furthermore, applying an uncertainty threshold derived from this technique to filter LLM responses—allowing only those above a specified correctness score (measured as the percentage of passed unit tests)—leads to high accuracy---see Section~\ref{sec:usability}.


%This approach ensures that the probabilities are appropriately scaled with respect to the sequence length.


%%For illustration, consider below as an example response produced by \gptturbo for the prompt used in Figure~\ref{fig:sampleproblem} from \S\ref{sec:motivating}:
%% % ``\texttt{Write a Python function that counts how many people older than 60 appear in a data list.}''

%% \begin{lstlisting}[language=Python]
%%     def candidate1(details):
%%      count = 0
%%         for detail in details:
%%             age = int(detail[11:13])
%%             if age > 60:
%%                 count += 1
%%         return count
%% \end{lstlisting}

%% For illustration, let us consider a code snippet generated by the LLM of $N=20$ tokens. ,  each with an individual token probability of $0.7$. 

%% \CD{Can we actually find out the number of tokens and log probabilities?}
%% When working with language models, each token in a generated completion has an associated log probability.  
%% 

%% Then if we simply sum the log probabilities of each token in a generated response as shown earlier:
%% \[
%%    \sum_{i=1}^{20} \log(0.7)
%%    \;=\;
%%    20 \,\log(0.7)
%%    \;\approx\;
%%    -7.1335,
%% \]
%% and exponentiate this sum yields:
%% \[
%%    \exp(-7.1335)
%%    \;\approx\;
%%    0.0008.
%% \]


%% Moreover, The probabilities of individual responses tend to decay exponentially with length, which can lead to disproportionately low values for valid but lengthy outputs.

%% If we merely sum these log probabilities over $N$ tokens,



%% Hence, although $0.7$ per token is fairly high, the \emph{total} probability from multiplying all 20 token probabilities becomes quite small ($0.0008$). 
%% A shorter completion, having fewer tokens, might end up with a larger total probability even if its average token confidence is slightly lower. 
%% This demonstrates how \emph{lengthier responses} can be unfairly penalized if we only sum or multiply all token probabilities, hence motivating \textbf{length-based normalisation}.


%% Length-normalization also helps when the responses generated for a query have different lengths.




% To address this, we perform a length-based normalisation to ensure that probabilities remain comparable across responses of different lengths. Specifically, for a generated snippet \(s\), its normalized probability is defined as:
% \begin{equation}
%     \tilde{p}(s \mid x) = \frac{p(s \mid x)}{|s|^\alpha},
% \end{equation}
% where \(|s|\) is the length of the snippet \(s\), and \(\alpha\) is a hyperparameter that controls the degree of normalisation. This adjustment prevents the probabilities of longer responses from dominating or vanishing entirely, ensuring a fair representation in subsequent entropy computations.

%% When working with language models, each token in a generated completion has an associated log probability.  
%% If we merely sum these log probabilities over $N$ tokens,
%% \[
%%    \text{sum\_logprob} \;=\; \sum_{i=1}^{N} \log \bigl(p(\mathrm{token}_i)\bigr),
%% \]
%% longer completions tend to accumulate more negative values simply due to having more tokens. 
%% This can make them appear less likely, even if each token is reasonably probable.

%% To address this, we perform a length-based normalisation to ensure that probabilities remain comparable across responses of different lengths. Specifically, for a generated snippet \(s\), its normalized probability is defined as:
%% \begin{equation}
%%     \tilde{p}(s \mid x) = \frac{p(s \mid x)}{|s|^\alpha},
%% \end{equation}
%% where \(|s|\) is the length of the snippet \(s\), and \(\alpha\) is a hyperparameter that controls the degree of normalisation. 
%% This adjustment prevents the probabilities of longer responses from dominating or vanishing entirely, ensuring a fair representation in subsequent entropy computations.

% To mitigate this effect, we use \emph{length-based normalisation}.
% Instead of summing the log probabilities, we compute the \textbf{average} log probability per token:
% \[
%    \text{avg\_logprob} \;=\; \frac{1}{N} \; \sum_{i=1}^{N} \log \bigl(p(\mathrm{token}_i)\bigr).
% \]
% We then exponentiate this average to obtain a \textbf{length-normalized probability}:
% \[
%    \text{length\_normalized\_prob} \;=\;
%    \exp\!\bigl(\text{avg\_logprob}\bigr).
% \]
% Although this value is not a ``true'' probability for the entire sequence, it is 
% a fairer score for comparing completions of different lengths, 
% because a longer completion is not automatically penalized by virtue of having more tokens.


%% Hence, instead of relying on the total product of probabilities, we use the 
%% \textbf{average log probability} per token:
%% \[
%%    \text{avg\_logprob} 
%%    \;=\;
%%    \frac{1}{N} 
%%    \sum_{i=1}^{N} \log\bigl(p(\mathrm{token}_i)\bigr)
%%    \;=\;
%%    \log(0.7),
%% \]
%% for $N=20$ tokens in this simplified scenario. 
%% Exponentiating that average log probability gives:
%% \[
%%    \exp\!\bigl(\text{avg\_logprob}\bigr)
%%    \;=\;
%%    \exp(\log(0.7))
%%    \;=\;
%%    0.7.
%% \]

%% Thus, while the raw product \(\prod_{i=1}^{20} p(\mathrm{token}_i)\) is about $0.0008$, the \emph{length-normalized} probability is $0.7$, reflecting the fairer notion that each token has a $70\%$ likelihood on average. This avoids unfairly penalizing longer responses merely due to having more tokens multiplied together. 
%% Length-based normalisation is therefore crucial for comparing or ranking completions of different lengths.

% A different completion with more lines of code and docstrings might sum to a lower total log probability (due to more tokens), but its average log probability could be similar (say, $0.95$). When using length-based normalisation, these two scores ($0.97$ vs.\ $0.95$) are directly comparable as per-token likelihoods, rather than an apples-to-oranges comparison of total log probabilities.


% If you want to interpret these normalized scores as a \emph{distribution} over completions, 
% you can renormalize across all candidates so that they sum to 1:
% \[
%   \hat{p}_i \;=\; 
%   \frac{\exp\!\bigl(\text{avg\_logprob}_i\bigr)}{\sum_{j=1}^{k} \exp\!\bigl(\text{avg\_logprob}_j\bigr)},
% \]
% where $k$ is the total number of candidate completions. 
% This transforms the length-normalized scores into valid probabilities for comparing or sampling.

\section{Semantic Uncertainty via Symbolic Clustering}
\label{sec:symex}
This section presents our adaptation of the semantic entropy-based approach by Kuhn~\etal~\cite{kuhnsemantic} for code generation. 
We follow the main steps from the original work while diverging in two key aspects: the way we estimate the distribution of generated LLM responses and the clustering methodology.

%This section presents our methodology for assessing semantic uncertainty in code generation by leveraging clustering based on symbolic execution. 

\subsubsection{Generation}
The first step involves sampling $M$ code snippets (using the same hyperparameters as Kuhn~\etal~\cite{kuhnsemantic}), $\{s^{(1)}, \ldots, s^{(M)}\}$, from the LLM's output distribution $p(s \mid x)$ for a given prompt $x$. 
%Given a code generation prompt, the model generates $M$ samples, $\{s^{(1)}, \ldots, s^{(M)}\}$, from its predictive distribution $p(s \mid x)$.
% Sampling is carried out using multinomial techniques, with hyperparameters such as temperature and nucleus sampling selected based on those used by Kuhn~\etal~\cite{kuhnsemantic}.
The probabilities of the collected samples are processed using a softmax-style normalization function, ensuring that the resulting values can be interpreted as a valid probability distribution.

As explained in Section~\ref{sec:probcomp}, this process can lead to numerical underflows. 
To mitigate this, we approximate the probability distribution of the LLM responses using either length-normalization or a uniform distribution.
%
Following this approximation, let \(\tilde{p}(s \mid x)\) denote the probability of a snippet \(s\) according to the adjusted distribution.

%softmax normalized
%Then, we have: 
 %\[
 %\tilde{p}(s \mid x) = \frac{p(s \mid x)}{|s|^\alpha},
 %\]
 %where \(|s|\) is the length of the snippet \(s\), and \(\alpha\) is a hyperparameter controlling the degree of normalization. 

%\CD{fix softmax normalization}

%This is similar to the approach taken by Kuhn~\etal~\cite{kuhnsemantic} and Abbasi~\etal~\cite{abbasi2024believe} and is a necessary step to prevent the probability computations from becoming invalid within the various formulae.
% \CD{Do we use these optional techniques?}

\subsubsection{Clustering via Symbolic Execution}
The second step works by grouping the aforementioned snippets into clusters based on semantic equivalence. %, determined through symbolic execution traces.
%To determine semantic equivalence, we employ symbolic execution, a program analysis technique that computes execution paths and constraints for a given snippet. 
%Two snippets $s^{(i)}$ and $s^{(j)}$ are deemed functionally equivalent if their symbolic execution traces are identical or exhibit subsumption.
%\CD{We need details on the semantic equivalence. This is, in principle, undecidable, so we need to explain a bit more on how this works.}
%
This process, as shown in Algorithm~\ref{alg:clustering} from Section~\ref{sec:symexclustering}, is based on symbolic execution. %ensures that clustering is driven by functional, rather than syntactic or lexical, similarities, aligning with the stricter requirements of code quality evaluation.
%It is important to note that two syntactically different responses can still end up in the same cluster if they are semantically equivalent.
% This is due to our adapted symbolic execution based clustering algorithm . 

\subsubsection{Entropy Estimation}
The final step computes uncertainty as the semantic entropy over clusters, reflecting the diversity of functional behaviors.

%Semantic entropy quantifies the uncertainty in functional behaviour by measuring the probability distribution over clusters.
%
First, the probability associated with a cluster \(c\) is calculated as:
\begin{equation*}
    \tilde{p}(c \mid x) = \sum_{s \in c} \tilde{p}(s \mid x),
    %p(c \mid x) = \sum_{s \in c} p(s \mid x),    
\end{equation*}
where \(s \in c\) indicates that the snippet \(s\) belongs to the cluster \(c\).
%
Then, the entropy \(H(C \mid x)\) over the set of clusters \(C\) is defined as:
\begin{equation*}
    H(C \mid x) = -\sum_{c \in C} \log \tilde{p}(c \mid x),
    %H(C \mid x) = -\sum_{c \in C} \log p(c \mid x),    
\end{equation*}
where \(C\) denotes all semantic clusters obtained from Algorithm~\ref{alg:clustering} in Section~\ref{sec:symexclustering}. 
%
%This formulation captures both the diversity and confidence of the model's outputs while accounting for the length-based normalization, offering a self-contained metric independent of external validation.
A higher entropy indicates greater semantic diversity and hence higher uncertainty in the functional behavior captured by the clusters. 
Conversely, a lower entropy suggests that the model's outputs are concentrated around a few semantically equivalent behaviors, reflecting higher confidence.

\paragraph{Motivating example revisited for \textnormal{\gptturbo}.} To illustrate our uncertainty computation, we'll go back to the motivating example from Section~\ref{sec:motivating}.

As discussed there, Figure~\ref{fig:good-llm-snippets} (Listings~\ref{lst:good1}, \ref{lst:good2}, and \ref{lst:good3}) contains code snippets generated by \gptturbo. % that are semantically equivalent. %, all three snippets exhibit the same behaviour.
%while Figure~\ref{fig:bad-llm-snippets} (Listings~\ref{lst:bad1}, \ref{lst:bad2}, and \ref{lst:bad3} from \salesforce/\codegenmonoC) contains code snippets that are \textbf{semantically distinct}, none of the snippets share the same functional behaviour.
We denote these snippets by $s^{(1)}, s^{(2)}, s^{(3)}$, and, according to Algorithm~\ref{alg:clustering} from Section~\ref{sec:symexclustering}, they are all grouped in the same functional cluster $c_1$ as they are semantically equivalent.
%Now for the \gptturbo case, we have three generated snippets (Listings~\ref{lst:good1}, \ref{lst:good2}, and \ref{lst:good3} from Figure~\ref{fig:good-llm-snippets}), denoted $s^{(1)}, s^{(2)}, s^{(3)}$, all found by using the Algorithm~\ref{alg:clustering} from \S\ref{sec:symexclustering}  to be in one functional cluster $c_1$.
In other words, $C = \{ c_1 \}$ with $c_1 = \{ s^{(1)}, s^{(2)}, s^{(3)} \}$.

%Given the very small probabilities reported by the LLM as shown in Section~\ref{sec:motivating} (which would cause underflows in our implementation), here we use length normalized log-probabilties according to the normalization formula in Section~\ref{sec:probcomp}.
Following softmax normalization, we obtain the following probabilities for the three snippets: %, where we use $\tilde{p}$ to denote the probability based on length-normalization: 
%For these snippets, following are the normalized probabilities based on the token-level \texttt{logprobs} data obtained from \gptturbo:
\(\tilde{p}(s^{(1)} \mid x) = \GPTsnipNormProbA,\; \tilde{p}(s^{(2)} \mid x) = \GPTsnipNormProbB,\;
\tilde{p}(s^{(3)} \mid x) = \GPTsnipNormProbC.\)
Since all responses belong to the single cluster $c_1$, its cluster probability is:
\[
   \tilde{p}(c_1 \mid x)
   \;=\;
   \sum_{s \in c_1} \tilde{p}(s \mid x)
   \;=\;
   \GPTsnipNormProbA \;+\; \GPTsnipNormProbB \;+\; \GPTsnipNormProbC
   \;=\;
   1.0
\]
Thus the distribution over clusters is \(\tilde{p}(c_1 \mid x) = 1,\) and the entropy of clusters is:
\[
   H(C \mid x)
   \;=\;
   -\sum_{c \in C} \log \tilde{p}(c \mid x)
   \;=\;
   -\log(1.0)
   \;=\;
   0.
\]
A \emph{zero} semantic entropy indicates high confidence in the model's response for this prompt.

\paragraph{Motivating example revisited for \textnormal{\salesforce}.} Let's now consider the three snippets $s^{(1)}, s^{(2)}, s^{(3)}$ from Figure~\ref{fig:bad-llm-snippets} generated by the \salesforce/\codegenmonoC model. 
Their respective probabilities are:
$\tilde{p}(s^{(1)}\!\mid x) = \SFsnipNormProbA,\;
 \tilde{p}(s^{(2)}\!\mid x) = \SFsnipNormProbB,\;
 \tilde{p}(s^{(3)}\!\mid x) = \SFsnipNormProbC$.
These snippets get categorized in three distinct semantic clusters
$C = \{ c_1, c_2, c_3 \}$, with $c_1 = \{ s^{(1)} \}$,
$c_2 = \{ s^{(2)} \}$, and $c_3 = \{ s^{(3)} \}.$
 Because each snippet resides in its own cluster, the cluster probabilities are:
   $\tilde{p}(c_1 \mid x) = \SFsnipNormProbA, \tilde{p}(c_2 \mid x) = \SFsnipNormProbB, \tilde{p}(c_3 \mid x) = \SFsnipNormProbC$.
%\[
%   \tilde{p}(c_1 \mid x) = \SFsnipNormProbA,\quad
%   \tilde{p}(c_2 \mid x) = \SFsnipNormProbB,\quad
%   \tilde{p}(c_3 \mid x) = \SFsnipNormProbC.
%\]
%
The entropy then is:
\[
\begin{aligned}
   H(C \mid x)
   &=
   -\!\sum_{c \in \{c_1,c_2,c_3\}}
   \log \tilde{p}(c \mid x)
   \\
   &=
   -\,\Bigl(
      \log(\SFsnipNormProbA)\;+\;\log(\SFsnipNormProbB)\;+\;\log(\SFsnipNormProbC) 
   \Bigr) \approx \SFSE.
\end{aligned}
\]
%Numerically, this is approximately \SFSE.
%A \emph{higher entropy} in this example indicates more disagreement or diversity in the model's functional outputs: the \salesforce/\codegenmonoC model produced three \textbf{distinctly incorrect} solutions, each forming its own cluster.


Intuitively, the uncertainty estimate reflects the strength of our belief in the LLM's prediction. 
Based on this, an \textit{abstention policy} can be implemented, whereby the system abstains from making a prediction if the entropy exceeds a predefined \emph{uncertainty threshold}. 
This approach minimizes the likelihood of committing to incorrect or suboptimal solutions. The abstention threshold is empirically determined by analyzing the entropy distribution. 
The methodology for computing this threshold will be detailed in Section~\ref{sec:eval}.

%Thus, these two scenarios exemplify how \emph{semantic clustering} and the corresponding \emph{entropy measure} can capture both the diversity (or uniformity) of model generations \emph{and} the model's confidence in those generations' functional behaviour.


\section{Mutual Information Estimation via Symbolic Clustering}
\label{sec:mi}
% \AS{TODO:Redo this section, their is a sizeable gap between the theory of the paper and its practical implementation. Make sure that these gaps are explained in this section.
% Algorithm 2 and 3 of the paper work quite differently, so ensure that the explanation matches the implementation, where needed.}
This section presents an adaptation of the mutual information-based approach for quantifying epistemic uncertainty by Abbasi~\etal~\cite{abbasi2024believe} to the domain of code generation.
We follow the steps from the original work: iterative prompting for generating LLM responses, clustering, and mutual information estimation. 
However, similar to Section~\ref{sec:symex}, we diverge with respect to the methodology for clustering responses and the way we estimate the distribution of generated LLM responses.

\subsubsection{Iterative Prompting for Code Generation}
Iterative prompting is used for generating multiple responses from the LLM and consequently in constructing a pseudo joint distribution of outputs. 

% Beginning with the original prompt \(F_0(x) = x\) for \(i = 1, 2, \ldots, n\) we get a \emph{family of prompts} where the \(i\)-th response prompt in the family would be:
% \[
%   F_i(x, s_1, \ldots, s_i) = \text{``Original prompt: } x \text{. Previous responses: } s_1, \ldots, s_i \text{."}
% \]
% where $s_i$ is the \(i\)-th response from the LLM. 

More precisely, the LLM is sampled to produce $n$ responses while also getting their respective probabilities, $\mu(X_j)$ for \(j = 1, 2, \ldots, n\).
These responses are first used to construct iterative prompts by appending the response to the original prompt and asking the LLM to produce more responses. 
This step then makes use of softmax-style normalization to obtain values that can then be treated as probabilities which are used in the subsequent steps.  

As explained in Section~\ref{sec:probcomp} and Section~\ref{sec:symex}, this process can lead to numerical underflows. 
To mitigate this, we use length-normalization.
As opposed to the approach in Section~\ref{sec:symex}, here we did not use the uniform distribution approximation, as the actual LLM-reported probabilities are needed to distinguish between aleatoric and epistemic uncertainties.
%

Following length normalization, we compute conditional probabilities,  $\mu(X_m|X_n)$ for \(m,n = 1, 2, \ldots, n\), by looking at the response probabilities received from the LLM when subjected to the aforementioned iterative prompts.


% \CD{Are these $X_i$ rather than $s_i$ or that's just for clusters? Also, where do we get the probabilities $\mu(X_i)$ and $\mu(X_j \mid X_i)$ that are used below?}

\subsubsection{Clustering via Symbolic Execution}
To handle functional diversity, the generated program snippets are clustered based on their semantic equivalence using Algorithm~\ref{alg:clustering}. 
%symbolic execution. 
%Symbolic execution analyses each program to compute execution paths and constraints. 
%Two programs \(s^{(i)}\) and \(s^{(j)}\) are considered functionally equivalent if their symbolic execution traces \(T(s^{(i)})\) and \(T(s^{(j)})\) satisfy:
%\[
%T(s^{(i)}) \equiv T(s^{(j)}) \quad \text{or} \quad T(s^{(i)}) \subseteq T(s^{(j)}).
%\]

% Let \(C = \{c_1, c_2, \ldots, c_k\}\) denote the clusters formed, where each cluster \(c_j\) groups semantically equivalent programs. 
%The clustering procedure ensures that semantically redundant responses are grouped together, focusing on functional equivalence rather than syntactic similarity.
%The algorithm used is same as Algorithm~\ref{alg:clustering} from the previous section.

\subsubsection{Mutual Information Estimation}
% Once clustering is complete, mutual information is computed over the resulting clusters to quantify epistemic uncertainty. 
% The pseudo joint distribution is defined as:
% \[
% \tilde{p}(s_1, \ldots, s_n \mid x) = p(s_1 \mid F_0(x)) \prod_{i=2}^n p(s_i \mid F_{i-1}(x, s_1, \ldots, s_{i-1})).
% \]
% The marginal distribution is then:
% \[
% \tilde{p}^\otimes(s_1, \ldots, s_n) = \prod_{i=1}^n p(s_i \mid F_0(x)).
% \]

% The mutual information is then computed as:
% \[
% I(\tilde{p}) = D_{KL}(\tilde{p} \| \tilde{p}^\otimes).
% \]

Once clustering is complete, mutual information is computed over the resulting clusters to quantify epistemic uncertainty. 

The aggregated probabilities are defined as:
\[
\mu_1'(X_i) = \sum_{j \in D(i)} \mu(X_j), \quad
\mu_2'(X_t \mid X_i) = \sum_{j \in D(t)} \mu(X_j \mid X_i),
\]
where \( X_i \) and \( X_t \) are clusters, \( \mu(X_j) \) represents the probability of the output \( X_j \), and \( \mu(X_j \mid X_i) \) is the conditional probability of \( X_j \) given \( X_i \). The set \( D(i) \) contains all outputs assigned to the cluster \( X_i \).

The normalized empirical distributions are:
\[
\hat{\mu}_1(X_i) = \frac{\mu_1'(X_i)}{Z}, \quad \text{where} \quad Z = \sum_{j \in S} \mu_1'(X_j),
\]
\[
\hat{\mu}_2(X_t \mid X_i) = \frac{\mu_2'(X_t \mid X_i)}{Z_i}, \quad \text{where} \quad Z_i = \sum_{j \in S} \mu_2'(X_j \mid X_i).
\]
Here, \( \hat{\mu}_1(X_i) \) is the normalized marginal distribution for cluster \( X_i \), and \( \hat{\mu}_2(X_t \mid X_i) \) is the normalized conditional distribution for \( X_t \) given \( X_i \). The terms \( Z \) and \( Z_i \) are normalization constants to ensure that the distributions sum to 1.

The joint and pseudo-joint distributions are defined as:
\[
\hat{\mu}(X_i, X_t) = \hat{\mu}_1(X_i) \hat{\mu}_2(X_t \mid X_i), \quad
\hat{\mu}^\otimes(X_i, X_t) = \hat{\mu}_1(X_i) \sum_{j \in S} \hat{\mu}_1(X_j) \hat{\mu}_2(X_t \mid X_j).
\]
The joint distribution \( \hat{\mu}(X_i, X_t) \) combines the marginal and conditional distributions, while the pseudo-joint distribution \( \hat{\mu}^\otimes(X_i, X_t) \) assumes independence between clusters.

Finally, the mutual information is computed as:
\[
\hat{I}(\gamma_1, \gamma_2) = \sum_{i, t \in S} \hat{\mu}(X_i, X_t) \ln \left( \frac{\hat{\mu}(X_i, X_t) + \gamma_1}{\hat{\mu}^\otimes(X_i, X_t) + \gamma_2} \right).
\]
Here, \( \gamma_1 \) and \( \gamma_2 \) are small stabilization parameters to prevent division by zero, and \( S \) is the set of clusters.

% \subsubsection{Finite-Sample Estimation with Clusters}
% To estimate mutual information from the finite sample of clusters \(C = \{c_1, \ldots, c_k\}\), the probability of a cluster \(c\) is computed as:
% \[
% p(c \mid x) = \sum_{s \in c} p(s \mid x).
% \]
% The empirical mutual information is then:
% \[
% \hat{I}_k = \sum_{c \in C} \hat{p}(c \mid x) \ln \left( \frac{\hat{p}(c \mid x)}{\prod_{i=1}^n \hat{p}(c_i \mid x)} \right),
% \]
% where \(\hat{p}(c \mid x)\) is the observed cluster probability. 
% The empirical mutual information is then:
% \[
% \hat{I}_k(\gamma_1, \gamma_2) = \sum_{i, t \in S} p(X_i, X_t) \ln \left( \frac{p(X_i, X_t) + \gamma_1}{p^\otimes(X_i, X_t) + \gamma_2} \right),
% \]
% where $\hat{I}_k(\gamma_1, \gamma_2)$ is the estimated mutual information, $p(X_i, X_t)$ represents the joint empirical distribution over clusters $X_i$ and $X_t$, and $p^\otimes(X_i, X_t)$ denotes the product of their marginal probabilities for cluster pairs. 
% The stabilization parameters $\gamma_1$ and $\gamma_2$ are included to handle cases where $p(X_i, X_t)$ or $p^\otimes(X_i, X_t)$ might be zero, preventing undefined logarithmic terms. 
% This formulation allows for a robust estimation of mutual information by quantifying the dependencies between cluster distributions while addressing numerical instabilities.

% Entropy regularization is then applied for stability:
% \[
% \hat{I}_k(\gamma) = \sum_{c \in C} \hat{p}(c \mid x) \ln \left( \frac{\hat{p}(c \mid x) + \gamma}{\prod_{i=1}^n (\hat{p}(c_i \mid x) + \gamma)} \right).
% \]

This mutual information score serves as a proxy for epistemic uncertainty. 
High \(\hat{I}\) values signal significant uncertainty.
% \[
% a_\lambda(x) =
% \begin{cases} 
% 1 & \text{if } \hat{I}_k \geq \lambda, \\
% 0 & \text{otherwise.}
% \end{cases}
% \]

\paragraph{Motivating example revisited for \textnormal{\gptturbo}.}
We now illustrate the MI computation for our motivating example from Section~\ref{sec:motivating}. 
We use 3 samples with an iteration prompt length of 2.
All responses from \gptturbo fall in the same cluster and hence following the earlier formula for MI we get:

\begin{align*}
    \hat{I}(\gamma_1, \gamma_2) = \hat{\mu}(X_1, X_1) \ln \left( \frac{\hat{\mu}(X_1, X_1) + \gamma_1}{\hat{\mu}^\otimes(X_1, X_1) + \gamma_2} \right) 
    = 1.0 \ln \left( \frac{1.0 + \gamma_1}{1.0 + \gamma_2} \right).
\end{align*}

Since $\gamma_1$ and $\gamma_2$ are zero as per Abbasi~\etal~\cite{abbasi2024believe}:
%\begin{align*}
   $\ln \left( \frac{1.0 + \gamma_1}{1.0 + \gamma_2} \right) = \ln(1.0) = 0$.
%\end{align*}
Therefore, $\hat{I} = 1.0 \cdot 0 = 0$.
%\begin{align*}
%    \hat{I} &= 1.0 \cdot 0 = 0.
%\end{align*}
%
% EXPLANATION: Zero MI is expected as the paper says this "For the S.E. and M.I. methods, the responses for a large number of queries can be clustered into a single group, and therefore the semantic entropy and mutual information scores are zero."
%
A \emph{zero} MI indicates very low uncertainty in the model's responses, suggesting a high likelihood of correctness as we will show later in the paper.

\paragraph{Motivating example revisited for \textnormal{\salesforce}.}
For \codegenmonoC, all three responses fall in their own separate clusters \ie  $R_1 \in X_1$, $R_2 \in X_2$, $R_3 \in X_3$ and hence,
%
%\begin{align*}
    $P(X_1) = 0.9995, P(X_2) = 0.0004, P(X_3) = 0.0001$.
%\end{align*}

For brevity, we omit the computation of the marginals which was carried out using the same formula used by Abbasi~\etal~\cite{abbasi2024believe} which for this example looks like:

\[
\hat{I}(\gamma_1, \gamma_2)
= \sum_{i=1}^3 \sum_{j=1}^3
  \hat{\mu}(X_i, X_j)
  \ln\!\Biggl(\frac{\hat{\mu}(X_i, X_j) + \gamma_1}
                   {\hat{\mu}^\otimes(X_i, X_j) + \gamma_2}\Biggr).
\]

Computing the terms individually then looks like:

\[
\begin{aligned}
    (X_1, X_1) &: 0.800 \ln \left( \frac{0.800}{0.9990} \right) = -0.1788, & (X_1, X_2) &: 0.120 \ln \left( \frac{0.120}{0.0004} \right) = 0.6845, \\
    (X_1, X_3) &: 0.040 \ln \left( \frac{0.040}{0.0001} \right) = 0.2397, & (X_2, X_1) &: 0.050 \ln \left( \frac{0.050}{0.0004} \right) = 0.5050, \\
    (X_2, X_2) &: 0.010 \ln \left( \frac{0.010}{0.00000016} \right) = 0.1104, & (X_2, X_3) &: 0.020 \ln \left( \frac{0.020}{0.00000004} \right) = 0.2624, \\
    (X_3, X_1) &: 0.020 \ln \left( \frac{0.020}{0.0001} \right) = 0.1609, & (X_3, X_2) &: 0.020 \ln \left( \frac{0.020}{0.00000004} \right) = 0.2624, \\
    (X_3, X_3) &: 0.010 \ln \left( \frac{0.010}{0.00000001} \right) = 0.1382.
\end{aligned}
\]

Then total MI then is:
%\begin{align*}
    $\hat{I} = -0.1788 + 0.6845 + 0.2397 + 0.5050 + 0.1104 + 0.2624 + 0.1609 + 0.2624 + 0.1382 = 2.1847$.
%\end{align*}

%\begin{align*}
%    \hat{I} &= -0.1788 + 0.6845 + 0.2397 + 0.5050 + 0.1104 \\
%    &\quad + 0.2624 + 0.1609 + 0.2624 + 0.1382 = 2.1847.
%\end{align*}


While lower uncertainty is intuitively desirable, as discussed in Section~\ref{sec:symex}, the uncertainty estimate is assessed against an abstention threshold to derive meaningful conclusions (see Section~\ref{sec:usability}).

%This shows \emph{moderately} high uncertainty, due to all responses being in their own respective clusters. 

%By integrating symbolic execution-based clustering with iterative prompting and mutual information estimation, this methodology captures both functional diversity and epistemic uncertainty in program generation. 





% \section{Methodology}
% \label{sec:method}
% \section{Research Methodology}~\label{sec:Methodology}

In this section, we discuss the process of conducting our systematic review, e.g., our search strategy for data extraction of relevant studies, based on the guidelines of Kitchenham et al.~\cite{kitchenham2022segress} to conduct SLRs and Petersen et al.~\cite{PETERSEN20151} to conduct systematic mapping studies (SMSs) in Software Engineering. In this systematic review, we divide our work into a four-stage procedure, including planning, conducting, building a taxonomy, and reporting the review, illustrated in Fig.~\ref{fig:search}. The four stages are as follows: (1) the \emph{planning} stage involved identifying research questions (RQs) and specifying the detailed research plan for the study; (2) the \emph{conducting} stage involved analyzing and synthesizing the existing primary studies to answer the research questions; (3) the \emph{taxonomy} stage was introduced to optimize the data extraction results and consolidate a taxonomy schema for REDAST methodology; (4) the \emph{reporting} stage involved the reviewing, concluding and reporting the final result of our study.

\begin{figure}[!t]
    \centering
    \includegraphics[width=1\linewidth]{fig/methodology/searching-process.drawio.pdf}
    \caption{Systematic Literature Review Process}
    \label{fig:search}
\end{figure}

\subsection{Research Questions}
In this study, we developed five research questions (RQs) to identify the input and output, analyze technologies, evaluate metrics, identify challenges, and identify potential opportunities. 

\textbf{RQ1. What are the input configurations, formats, and notations used in the requirements in requirements-driven
automated software testing?} In requirements-driven testing, the input is some form of requirements specification -- which can vary significantly. RQ1 maps the input for REDAST and reports on the comparison among different formats for requirements specification.

\textbf{RQ2. What are the frameworks, tools, processing methods, and transformation techniques used in requirements-driven automated software testing studies?} RQ2 explores the technical solutions from requirements to generated artifacts, e.g., rule-based transformation applying natural language processing (NLP) pipelines and deep learning (DL) techniques, where we additionally discuss the potential intermediate representation and additional input for the transformation process.

\textbf{RQ3. What are the test formats and coverage criteria used in the requirements-driven automated software
testing process?} RQ3 focuses on identifying the formulation of generated artifacts (i.e., the final output). We map the adopted test formats and analyze their characteristics in the REDAST process.

\textbf{RQ4. How do existing studies evaluate the generated test artifacts in the requirements-driven automated software testing process?} RQ4 identifies the evaluation datasets, metrics, and case study methodologies in the selected papers. This aims to understand how researchers assess the effectiveness, accuracy, and practical applicability of the generated test artifacts.

\textbf{RQ5. What are the limitations and challenges of existing requirements-driven automated software testing methods in the current era?} RQ5 addresses the limitations and challenges of existing studies while exploring future directions in the current era of technology development. %It particularly highlights the potential benefits of advanced LLMs and examines their capacity to meet the high expectations placed on these cutting-edge language modeling technologies. %\textcolor{blue}{CA: Do we really need to focus on LLMs? TBD.} \textcolor{orange}{FW: About LLMs, I removed the direct emphase in RQ5 but kept the discussion in RQ5 and the solution section. I think that would be more appropriate.}

\subsection{Searching Strategy}

The overview of the search process is exhibited in Fig. \ref{fig:papers}, which includes all the details of our search steps.
\begin{table}[!ht]
\caption{List of Search Terms}
\label{table:search_term}
\begin{tabularx}{\textwidth}{lX}
\hline
\textbf{Terms Group} & \textbf{Terms} \\ \hline
Test Group & test* \\
Requirement Group & requirement* OR use case* OR user stor* OR specification* \\
Software Group & software* OR system* \\
Method Group & generat* OR deriv* OR map* OR creat* OR extract* OR design* OR priorit* OR construct* OR transform* \\ \hline
\end{tabularx}
\end{table}

\begin{figure}
    \centering
    \includegraphics[width=1\linewidth]{fig/methodology/search-papers.drawio.pdf}
    \caption{Study Search Process}
    \label{fig:papers}
\end{figure}

\subsubsection{Search String Formulation}
Our research questions (RQs) guided the identification of the main search terms. We designed our search string with generic keywords to avoid missing out on any related papers, where four groups of search terms are included, namely ``test group'', ``requirement group'', ``software group'', and ``method group''. In order to capture all the expressions of the search terms, we use wildcards to match the appendix of the word, e.g., ``test*'' can capture ``testing'', ``tests'' and so on. The search terms are listed in Table~\ref{table:search_term}, decided after iterative discussion and refinement among all the authors. As a result, we finally formed the search string as follows:


\hangindent=1.5em
 \textbf{ON ABSTRACT} ((``test*'') \textbf{AND} (``requirement*'' \textbf{OR} ``use case*'' \textbf{OR} ``user stor*'' \textbf{OR} ``specifications'') \textbf{AND} (``software*'' \textbf{OR} ``system*'') \textbf{AND} (``generat*'' \textbf{OR} ``deriv*'' \textbf{OR} ``map*'' \textbf{OR} ``creat*'' \textbf{OR} ``extract*'' \textbf{OR} ``design*'' \textbf{OR} ``priorit*'' \textbf{OR} ``construct*'' \textbf{OR} ``transform*''))

The search process was conducted in September 2024, and therefore, the search results reflect studies available up to that date. We conducted the search process on six online databases: IEEE Xplore, ACM Digital Library, Wiley, Scopus, Web of Science, and Science Direct. However, some databases were incompatible with our default search string in the following situations: (1) unsupported for searching within abstract, such as Scopus, and (2) limited search terms, such as ScienceDirect. Here, for (1) situation, we searched within the title, keyword, and abstract, and for (2) situation, we separately executed the search and removed the duplicate papers in the merging process. 

\subsubsection{Automated Searching and Duplicate Removal}
We used advanced search to execute our search string within our selected databases, following our designed selection criteria in Table \ref{table:selection}. The first search returned 27,333 papers. Specifically for the duplicate removal, we used a Python script to remove (1) overlapped search results among multiple databases and (2) conference or workshop papers, also found with the same title and authors in the other journals. After duplicate removal, we obtained 21,652 papers for further filtering.

\begin{table*}[]
\caption{Selection Criteria}
\label{table:selection}
\begin{tabularx}{\textwidth}{lX}
\hline
\textbf{Criterion ID} & \textbf{Criterion Description} \\ \hline
S01          & Papers written in English. \\
S02-1        & Papers in the subjects of "Computer Science" or "Software Engineering". \\
S02-2        & Papers published on software testing-related issues. \\
S03          & Papers published from 1991 to the present. \\ 
S04          & Papers with accessible full text. \\ \hline
\end{tabularx}
\end{table*}

\begin{table*}[]
\small
\caption{Inclusion and Exclusion Criteria}
\label{table:criteria}
\begin{tabularx}{\textwidth}{lX}
\hline
\textbf{ID}  & \textbf{Description} \\ \hline
\multicolumn{2}{l}{\textbf{Inclusion Criteria}} \\ \hline
I01 & Papers about requirements-driven automated system testing or acceptance testing generation, or studies that generate system-testing-related artifacts. \\
I02 & Peer-reviewed studies that have been used in academia with references from literature. \\ \hline
\multicolumn{2}{l}{\textbf{Exclusion Criteria}} \\ \hline
E01 & Studies that only support automated code generation, but not test-artifact generation. \\
E02 & Studies that do not use requirements-related information as an input. \\
E03 & Papers with fewer than 5 pages (1-4 pages). \\
E04 & Non-primary studies (secondary or tertiary studies). \\
E05 & Vision papers and grey literature (unpublished work), books (chapters), posters, discussions, opinions, keynotes, magazine articles, experience, and comparison papers. \\ \hline
\end{tabularx}
\end{table*}

\subsubsection{Filtering Process}

In this step, we filtered a total of 21,652 papers using the inclusion and exclusion criteria outlined in Table \ref{table:criteria}. This process was primarily carried out by the first and second authors. Our criteria are structured at different levels, facilitating a multi-step filtering process. This approach involves applying various criteria in three distinct phases. We employed a cross-verification method involving (1) the first and second authors and (2) the other authors. Initially, the filtering was conducted separately by the first and second authors. After cross-verifying their results, the results were then reviewed and discussed further by the other authors for final decision-making. We widely adopted this verification strategy within the filtering stages. During the filtering process, we managed our paper list using a BibTeX file and categorized the papers with color-coding through BibTeX management software\footnote{\url{https://bibdesk.sourceforge.io/}}, i.e., “red” for irrelevant papers, “yellow” for potentially relevant papers, and “blue” for relevant papers. This color-coding system facilitated the organization and review of papers according to their relevance.

The screening process is shown below,
\begin{itemize}
    \item \textbf{1st-round Filtering} was based on the title and abstract, using the criteria I01 and E01. At this stage, the number of papers was reduced from 21,652 to 9,071.
    \item \textbf{2nd-round Filtering}. We attempted to include requirements-related papers based on E02 on the title and abstract level, which resulted from 9,071 to 4,071 papers. We excluded all the papers that did not focus on requirements-related information as an input or only mentioned the term ``requirements'' but did not refer to the requirements specification.
    \item \textbf{3rd-round Filtering}. We selectively reviewed the content of papers identified as potentially relevant to requirements-driven automated test generation. This process resulted in 162 papers for further analysis.
\end{itemize}
Note that, especially for third-round filtering, we aimed to include as many relevant papers as possible, even borderline cases, according to our criteria. The results were then discussed iteratively among all the authors to reach a consensus.

\subsubsection{Snowballing}

Snowballing is necessary for identifying papers that may have been missed during the automated search. Following the guidelines by Wohlin~\cite{wohlin2014guidelines}, we conducted both forward and backward snowballing. As a result, we identified 24 additional papers through this process.

\subsubsection{Data Extraction}

Based on the formulated research questions (RQs), we designed 38 data extraction questions\footnote{\url{https://drive.google.com/file/d/1yjy-59Juu9L3WHaOPu-XQo-j-HHGTbx_/view?usp=sharing}} and created a Google Form to collect the required information from the relevant papers. The questions included 30 short-answer questions, six checkbox questions, and two selection questions. The data extraction was organized into five sections: (1) basic information: fundamental details such as title, author, venue, etc.; (2) open information: insights on motivation, limitations, challenges, etc.; (3) requirements: requirements format, notation, and related aspects; (4) methodology: details, including immediate representation and technique support; (5) test-related information: test format(s), coverage, and related elements. Similar to the filtering process, the first and second authors conducted the data extraction and then forwarded the results to the other authors to initiate the review meeting.

\subsubsection{Quality Assessment}

During the data extraction process, we encountered papers with insufficient information. To address this, we conducted a quality assessment in parallel to ensure the relevance of the papers to our objectives. This approach, also adopted in previous secondary studies~\cite{shamsujjoha2021developing, naveed2024model}, involved designing a set of assessment questions based on guidelines by Kitchenham et al.~\cite{kitchenham2022segress}. The quality assessment questions in our study are shown below:
\begin{itemize}
    \item \textbf{QA1}. Does this study clearly state \emph{how} requirements drive automated test generation?
    \item \textbf{QA2}. Does this study clearly state the \emph{aim} of REDAST?
    \item \textbf{QA3}. Does this study enable \emph{automation} in test generation?
    \item \textbf{QA4}. Does this study demonstrate the usability of the method from the perspective of methodology explanation, discussion, case examples, and experiments?
\end{itemize}
QA4 originates from an open perspective in the review process, where we focused on evaluation, discussion, and explanation. Our review also examined the study’s overall structure, including the methodology description, case studies, experiments, and analyses. The detailed results of the quality assessment are provided in the Appendix. Following this assessment, the final data extraction was based on 156 papers.

% \begin{table}[]
% \begin{tabular}{ll}
% \hline
% QA ID & QA Questions                                             \\ \hline
% Q01   & Does this study clearly state its aims?                  \\
% Q02   & Does this study clearly describe its methodology?        \\
% Q03   & Does this study involve automated test generation?       \\
% Q04   & Does this study include a promising evaluation?          \\
% Q05   & Does this study demonstrate the usability of the method? \\ \hline
% \end{tabular}%
% \caption{Questions for Quality Assessment}
% \label{table:qa}
% \end{table}

% automated quality assessment

% \textcolor{blue}{CA: Our search strategy focused on identifying requirements types first. We covered several sources, e.g., ~\cite{Pohl:11,wagner2019status} to identify different formats and notations of specifying requirements. However, this came out to be a long list, e.g., free-form NL requirements, semi-formal UML models, free-from textual use case models, UML class diagrams, UML activity diagrams, and so on. In this paper, we attempted to primarily focus on requirements-related aspects and not design-level information. Hence, we generalised our search string to include generic keywords, e.g., requirement*, use case*, and user stor*. We did so to avoid missing out on any papers, bringing too restrictive in our search strategy, and not creating a too-generic search string with all the aforementioned formats to avoid getting results beyond our review's scope.}


%% Use \subsection commands to start a subsection.



%\subsection{Study Selection}

% In this step, we further looked into the content of searched papers using our search strategy and applied our inclusion and exclusion criteria. Our filtering strategy aimed to pinpoint studies focused on requirements-driven system-level testing. Recognizing the presence of irrelevant papers in our search results, we established detailed selection criteria for preliminary inclusion and exclusion, as shown in Table \ref{table: criteria}. Specifically, we further developed the taxonomy schema to exclude two types of studies that did not meet the requirements for system-level testing: (1) studies supporting specification-driven test generation, such as UML-driven test generation, rather than requirements-driven testing, and (2) studies focusing on code-based test generation, such as requirement-driven code generation for unit testing.





% evaluation_modified.tex is the updated/latest version while evaluation.tex is the CAV submission 
\section{Evaluation}
\label{sec:eval}
% \definecolor{darkgreen}{rgb}{0.0, 0.5, 0.0}
\definecolor{violet}{rgb}{0.56, 0.0, 1.0}
\section{Evaluation}
We apply our methodology to derive counterfactual policies for various MDPs, addressing three main research questions: (1) how does our policy's performance compare to the Gumbel-max SCM approach; (2) how do the counterfactual stability and monotonicity assumptions impact the probability bounds; and (3) how fast is our approach compared with the Gumbel-max SCM method?

\begin{figure*}
    \centering
    %
    \resizebox{0.6\textwidth}{!}{
        \begin{tikzpicture}[scale=1.0, every node/.style={scale=1.0}]
            \draw[thick, black] (-3, -0.25) rectangle (10, 0.25);
            %
            \draw[black, line width=1pt] (-2.5, 0.0) -- (-2,0.0);
            \fill[black] (-2.25,0.0) circle (2pt); %
            \node[right] at (-2,0.0) {\small Observed Path};
            
            %
            \draw[blue, line width=1pt] (1.0,0.0) -- (1.5,0.0);
            \node[draw=blue, circle, minimum size=4pt, inner sep=0pt] at (1.25,0.0) {}; %
            \node[right] at (1.5,0.0) {\small Interval CFMDP Policy};
            
            %
            \draw[red, line width=1pt] (5.5,0) -- (6,0);
            \node[red] at (5.75,0) {$\boldsymbol{\times}$}; %
            \node[right] at (6,0) {\small Gumbel-max SCM Policy};
        \end{tikzpicture}
    }\\
    %
    \subfigure[\footnotesize Lowest cumulative reward: Interval CFMDP ($312$), Gumbel-max SCM ($312$)]{%
        \resizebox{0.76\columnwidth}{!}{
             \begin{tikzpicture}
                \begin{axis}[
                    xlabel={$t$},
                    ylabel={Mean reward at time step $t$},
                    title={Optimal Path},
                    grid=both,
                    width=20cm, height=8.5cm,
                    every axis/.style={font=\Huge},
                    %
                ]
                \addplot[
                    color=black, %
                    mark=*, %
                    line width=2pt,
                    mark size=3pt,
                    error bars/.cd,
                    y dir=both, %
                    y explicit, %
                    error bar style={line width=1pt,solid},
                    error mark options={line width=1pt,mark size=4pt,rotate=90}
                ]
                coordinates {
                    (0, 0.0)  +- (0, 0.0)
                    (1, 0.0)  +- (0, 0.0) 
                    (2, 1.0)  +- (0, 0.0) 
                    (3, 1.0)  +- (0, 0.0)
                    (4, 2.0)  +- (0, 0.0)
                    (5, 3.0) +- (0, 0.0)
                    (6, 5.0) +- (0, 0.0)
                    (7, 100.0) +- (0, 0.0)
                    (8, 100.0) +- (0, 0.0)
                    (9, 100.0) +- (0, 0.0)
                };
                %
                \addplot[
                    color=blue, %
                    mark=o, %
                    line width=2pt,
                    mark size=3pt,
                    error bars/.cd,
                    y dir=both, %
                    y explicit, %
                    error bar style={line width=1pt,solid},
                    error mark options={line width=1pt,mark size=4pt,rotate=90}
                ]
                 coordinates {
                    (0, 0.0)  +- (0, 0.0)
                    (1, 0.0)  +- (0, 0.0) 
                    (2, 1.0)  +- (0, 0.0) 
                    (3, 1.0)  +- (0, 0.0)
                    (4, 2.0)  +- (0, 0.0)
                    (5, 3.0) +- (0, 0.0)
                    (6, 5.0) +- (0, 0.0)
                    (7, 100.0) +- (0, 0.0)
                    (8, 100.0) +- (0, 0.0)
                    (9, 100.0) +- (0, 0.0)
                };
                %
                \addplot[
                    color=red, %
                    mark=x, %
                    line width=2pt,
                    mark size=6pt,
                    error bars/.cd,
                    y dir=both, %
                    y explicit, %
                    error bar style={line width=1pt,solid},
                    error mark options={line width=1pt,mark size=4pt,rotate=90}
                ]
                coordinates {
                    (0, 0.0)  +- (0, 0.0)
                    (1, 0.0)  +- (0, 0.0) 
                    (2, 1.0)  +- (0, 0.0) 
                    (3, 1.0)  +- (0, 0.0)
                    (4, 2.0)  +- (0, 0.0)
                    (5, 3.0) +- (0, 0.0)
                    (6, 5.0) +- (0, 0.0)
                    (7, 100.0) +- (0, 0.0)
                    (8, 100.0) +- (0, 0.0)
                    (9, 100.0) +- (0, 0.0)
                };
                \end{axis}
            \end{tikzpicture}
         }
    }
    \hspace{1cm}
    \subfigure[\footnotesize Lowest cumulative reward: Interval CFMDP ($19$), Gumbel-max SCM ($-88$)]{%
         \resizebox{0.76\columnwidth}{!}{
            \begin{tikzpicture}
                \begin{axis}[
                    xlabel={$t$},
                    ylabel={Mean reward at time step $t$},
                    title={Slightly Suboptimal Path},
                    grid=both,
                    width=20cm, height=8.5cm,
                    every axis/.style={font=\Huge},
                    %
                ]
                \addplot[
                    color=black, %
                    mark=*, %
                    line width=2pt,
                    mark size=3pt,
                    error bars/.cd,
                    y dir=both, %
                    y explicit, %
                    error bar style={line width=1pt,solid},
                    error mark options={line width=1pt,mark size=4pt,rotate=90}
                ]
              coordinates {
                    (0, 0.0)  +- (0, 0.0)
                    (1, 1.0)  +- (0, 0.0) 
                    (2, 1.0)  +- (0, 0.0) 
                    (3, 1.0)  +- (0, 0.0)
                    (4, 2.0)  +- (0, 0.0)
                    (5, 3.0) +- (0, 0.0)
                    (6, 3.0) +- (0, 0.0)
                    (7, 2.0) +- (0, 0.0)
                    (8, 2.0) +- (0, 0.0)
                    (9, 4.0) +- (0, 0.0)
                };
                %
                \addplot[
                    color=blue, %
                    mark=o, %
                    line width=2pt,
                    mark size=3pt,
                    error bars/.cd,
                    y dir=both, %
                    y explicit, %
                    error bar style={line width=1pt,solid},
                    error mark options={line width=1pt,mark size=4pt,rotate=90}
                ]
              coordinates {
                    (0, 0.0)  +- (0, 0.0)
                    (1, 1.0)  +- (0, 0.0) 
                    (2, 1.0)  +- (0, 0.0) 
                    (3, 1.0)  +- (0, 0.0)
                    (4, 2.0)  +- (0, 0.0)
                    (5, 3.0) +- (0, 0.0)
                    (6, 3.0) +- (0, 0.0)
                    (7, 2.0) +- (0, 0.0)
                    (8, 2.0) +- (0, 0.0)
                    (9, 4.0) +- (0, 0.0)
                };
                %
                \addplot[
                    color=red, %
                    mark=x, %
                    line width=2pt,
                    mark size=6pt,
                    error bars/.cd,
                    y dir=both, %
                    y explicit, %
                    error bar style={line width=1pt,solid},
                    error mark options={line width=1pt,mark size=4pt,rotate=90}
                ]
                coordinates {
                    (0, 0.0)  +- (0, 0.0)
                    (1, 1.0)  +- (0, 0.0) 
                    (2, 1.0)  +- (0, 0.0) 
                    (3, 1.0)  +- (0, 0.0)
                    (4, 2.0)  += (0, 0.0)
                    (5, 3.0)  += (0, 0.0)
                    (6, 3.17847) += (0, 0.62606746) -= (0, 0.62606746)
                    (7, 2.5832885) += (0, 1.04598233) -= (0, 1.04598233)
                    (8, 5.978909) += (0, 17.60137623) -= (0, 17.60137623)
                    (9, 5.297059) += (0, 27.09227512) -= (0, 27.09227512)
                };
                \end{axis}
            \end{tikzpicture}
         }
    }\\[-1.5pt]
    \subfigure[\footnotesize Lowest cumulative reward: Interval CFMDP ($14$), Gumbel-max SCM ($-598$)]{%
         \resizebox{0.76\columnwidth}{!}{
             \begin{tikzpicture}
                \begin{axis}[
                    xlabel={$t$},
                    ylabel={Mean reward at time step $t$},
                    title={Almost Catastrophic Path},
                    grid=both,
                    width=20cm, height=8.5cm,
                    every axis/.style={font=\Huge},
                    %
                ]
                \addplot[
                    color=black, %
                    mark=*, %
                    line width=2pt,
                    mark size=3pt,
                    error bars/.cd,
                    y dir=both, %
                    y explicit, %
                    error bar style={line width=1pt,solid},
                    error mark options={line width=1pt,mark size=4pt,rotate=90}
                ]
                coordinates {
                    (0, 0.0)  +- (0, 0.0)
                    (1, 1.0)  +- (0, 0.0) 
                    (2, 2.0)  +- (0, 0.0) 
                    (3, 1.0)  +- (0, 0.0)
                    (4, 0.0)  +- (0, 0.0)
                    (5, 1.0) +- (0, 0.0)
                    (6, 2.0) +- (0, 0.0)
                    (7, 2.0) +- (0, 0.0)
                    (8, 3.0) +- (0, 0.0)
                    (9, 2.0) +- (0, 0.0)
                };
                %
                \addplot[
                    color=blue, %
                    mark=o, %
                    line width=2pt,
                    mark size=3pt,
                    error bars/.cd,
                    y dir=both, %
                    y explicit, %
                    error bar style={line width=1pt,solid},
                    error mark options={line width=1pt,mark size=4pt,rotate=90}
                ]
                coordinates {
                    (0, 0.0)  +- (0, 0.0)
                    (1, 1.0)  +- (0, 0.0) 
                    (2, 2.0)  +- (0, 0.0) 
                    (3, 1.0)  +- (0, 0.0)
                    (4, 0.0)  +- (0, 0.0)
                    (5, 1.0) +- (0, 0.0)
                    (6, 2.0) +- (0, 0.0)
                    (7, 2.0) +- (0, 0.0)
                    (8, 3.0) +- (0, 0.0)
                    (9, 2.0) +- (0, 0.0)
                };
                %
                \addplot[
                    color=red, %
                    mark=x, %
                    line width=2pt,
                    mark size=6pt,
                    error bars/.cd,
                    y dir=both, %
                    y explicit, %
                    error bar style={line width=1pt,solid},
                    error mark options={line width=1pt,mark size=4pt,rotate=90}
                ]
                coordinates {
                    (0, 0.0)  +- (0, 0.0)
                    (1, 0.7065655)  +- (0, 0.4553358) 
                    (2, 1.341673)  +- (0, 0.67091621) 
                    (3, 1.122926)  +- (0, 0.61281824)
                    (4, -1.1821935)  +- (0, 13.82444042)
                    (5, -0.952399)  +- (0, 15.35195457)
                    (6, -0.72672) +- (0, 20.33508414)
                    (7, -0.268983) +- (0, 22.77861454)
                    (8, -0.1310835) +- (0, 26.31013314)
                    (9, 0.65806) +- (0, 28.50670214)
                };
                %
            %
            %
            %
            %
            %
            %
            %
            %
            %
            %
            %
            %
            %
            %
            %
            %
            %
            %
                \end{axis}
            \end{tikzpicture}
         }
    }
    \hspace{1cm}
    \subfigure[\footnotesize Lowest cumulative reward: Interval CFMDP ($-698$), Gumbel-max SCM ($-698$)]{%
         \resizebox{0.76\columnwidth}{!}{
            \begin{tikzpicture}
                \begin{axis}[
                    xlabel={$t$},
                    ylabel={Mean reward at time step $t$},
                    title={Catastrophic Path},
                    grid=both,
                    width=20cm, height=8.5cm,
                    every axis/.style={font=\Huge},
                    %
                ]
                \addplot[
                    color=black, %
                    mark=*, %
                    line width=2pt,
                    mark size=3pt,
                    error bars/.cd,
                    y dir=both, %
                    y explicit, %
                    error bar style={line width=1pt,solid},
                    error mark options={line width=1pt,mark size=4pt,rotate=90}
                ]
                coordinates {
                    (0, 1.0)  +- (0, 0.0)
                    (1, 2.0)  +- (0, 0.0) 
                    (2, -100.0)  +- (0, 0.0) 
                    (3, -100.0)  +- (0, 0.0)
                    (4, -100.0)  +- (0, 0.0)
                    (5, -100.0) +- (0, 0.0)
                    (6, -100.0) +- (0, 0.0)
                    (7, -100.0) +- (0, 0.0)
                    (8, -100.0) +- (0, 0.0)
                    (9, -100.0) +- (0, 0.0)
                };
                %
                \addplot[
                    color=blue, %
                    mark=o, %
                    line width=2pt,
                    mark size=3pt,
                    error bars/.cd,
                    y dir=both, %
                    y explicit, %
                    error bar style={line width=1pt,solid},
                    error mark options={line width=1pt,mark size=4pt,rotate=90}
                ]
                coordinates {
                    (0, 0.0)  +- (0, 0.0)
                    (1, 0.504814)  +- (0, 0.49997682) 
                    (2, 0.8439835)  +- (0, 0.76831917) 
                    (3, -8.2709165)  +- (0, 28.93656754)
                    (4, -9.981082)  +- (0, 31.66825363)
                    (5, -12.1776325) +- (0, 34.53463233)
                    (6, -13.556076) +- (0, 38.62845372)
                    (7, -14.574418) +- (0, 42.49603359)
                    (8, -15.1757075) +- (0, 46.41913968)
                    (9, -15.3900395) +- (0, 50.33563368)
                };
                %
                \addplot[
                    color=red, %
                    mark=x, %
                    line width=2pt,
                    mark size=6pt,
                    error bars/.cd,
                    y dir=both, %
                    y explicit, %
                    error bar style={line width=1pt,solid},
                    error mark options={line width=1pt,mark size=4pt,rotate=90}
                ]
                coordinates {
                    (0, 0.0)  +- (0, 0.0)
                    (1, 0.701873)  +- (0, 0.45743556) 
                    (2, 1.1227805)  +- (0, 0.73433129) 
                    (3, -8.7503255)  +- (0, 30.30257976)
                    (4, -10.722092)  +- (0, 33.17618589)
                    (5, -13.10721)  +- (0, 36.0648089)
                    (6, -13.7631645) +- (0, 40.56553451)
                    (7, -13.909043) +- (0, 45.23829402)
                    (8, -13.472517) +- (0, 49.96270296)
                    (9, -12.8278835) +- (0, 54.38618735)
                };
                %
            %
            %
            %
            %
            %
            %
            %
            %
            %
            %
            %
            %
            %
            %
            %
            %
            %
            %
                \end{axis}
            \end{tikzpicture}
         }
    }
    \caption{Average instant reward of CF paths induced by policies on GridWorld $p=0.4$.}
    \label{fig: reward p=0.4}
\end{figure*}

\subsection{Experimental Setup}
To compare policy performance, we measure the average rewards of counterfactual paths induced by our policy and the Gumbel-max policy by uniformly sampling $200$ counterfactual MDPs from the ICFMDP and generating $10,000$ counterfactual paths over each sampled CFMDP. \jl{Since the interval CFMDP depends on the observed path, we select $4$  paths of varying optimality to evaluate how the observed path impacts the performance of both policies: an optimal path, a slightly suboptimal path that could reach the optimal reward with a few changes, a catastrophic path that enters a catastrophic, terminal state with low reward, and an almost catastrophic path that was close to entering a catastrophic state.} When measuring the average probability bound widths and execution time needed to generate the ICFMDPs, we averaged over $20$ randomly generated observed paths
\footnote{Further training details are provided in Appendix \ref{app: training details}, and the code is provided at \href{https://github.com/ddv-lab/robust-cf-inference-in-MDPs}{https://github.com/ddv-lab/robust-cf-inference-in-MDPs}
%
%
.}.

\subsection{GridWorld}
\jl{The GridWorld MDP is a $4 \times 4$ grid where an agent must navigate from the top-left corner to the goal state in the bottom-right corner, avoiding a dangerous terminal state in the centre. At each time step, the agent can move up, down, left, or right, but there is a small probability (controlled by hyper-parameter $p$) of moving in an unintended direction. As the agent nears the goal, the reward for each state increases, culminating in a reward of $+100$ for reaching the goal. Entering the dangerous state results in a penalty of $-100$. We use two versions of GridWorld: a less stochastic version with $p=0.9$ (i.e., $90$\% chance of moving in the chosen direction) and a more stochastic version with $p=0.4$.}

\paragraph{GridWorld ($p=0.9$)}
When $p=0.9$, the counterfactual probability bounds are typically narrow (see Table \ref{tab:nonzero_probs} for average measurements). Consequently, as shown in Figure \ref{fig: reward p=0.9}, both policies are nearly identical and perform similarly well across the optimal, slightly suboptimal, and catastrophic paths.
%
However, for the almost catastrophic path, the interval CFMDP path is more conservative and follows the observed path more closely (as this is where the probability bounds are narrowest), which typically requires one additional step to reach the goal state than the Gumbel-max SCM policy.
%

\paragraph{GridWorld ($p=0.4$)}
\jl{When $p=0.4$, the GridWorld environment becomes more uncertain, increasing the risk of entering the dangerous state even if correct actions are chosen. Thus, as shown in Figure \ref{fig: reward p=0.4}, the interval CFMDP policy adopts a more conservative approach, avoiding deviation from the observed policy if it cannot guarantee higher counterfactual rewards (see the slightly suboptimal and almost catastrophic paths), whereas the Gumbel-max SCM is inconsistent: it can yield higher rewards, but also much lower rewards, reflected in the wide error bars.} For the catastrophic path, both policies must deviate from the observed path to achieve a higher reward and, in this case, perform similarly.
%
%
%
%
\subsection{Sepsis}
The Sepsis MDP \citep{oberst2019counterfactual} simulates trajectories of Sepsis patients. Each state consists of four vital signs (heart rate, blood pressure, oxygen concentration, and glucose levels), categorised as low, normal, or high.
and three treatments that can be toggled on/off at each time step (8 actions in total). Unlike \citet{oberst2019counterfactual}, we scale rewards based on the number of out-of-range vital signs, between $-1000$ (patient dies) and $1000$ (patient discharged). \jl{Like the GridWorld $p=0.4$ experiment, the Sepsis MDP is highly uncertain, as many states are equally likely to lead to optimal and poor outcomes. Thus, as shown in Figure \ref{fig: reward sepsis}, both policies follow the observed optimal and almost catastrophic paths to guarantee rewards are no worse than the observation.} However, improving the catastrophic path requires deviating from the observation. Here, the Gumbel-max SCM policy, on average, performs better than the interval CFMDP policy. But, since both policies have lower bounds clipped at $-1000$, neither policy reliably improves over the observation. In contrast, for the slightly suboptimal path, the interval CFMDP policy performs significantly better, shown by its higher lower bounds. 
Moreover, in these two cases, the worst-case counterfactual path generated by the interval CFMDP policy is better than that of the Gumbel-max SCM policy,
indicating its greater robustness.
%
\begin{figure*}
    \centering
     \resizebox{0.6\textwidth}{!}{
        \begin{tikzpicture}[scale=1.0, every node/.style={scale=1.0}]
            \draw[thick, black] (-3, -0.25) rectangle (10, 0.25);
            %
            \draw[black, line width=1pt] (-2.5, 0.0) -- (-2,0.0);
            \fill[black] (-2.25,0.0) circle (2pt); %
            \node[right] at (-2,0.0) {\small Observed Path};
            
            %
            \draw[blue, line width=1pt] (1.0,0.0) -- (1.5,0.0);
            \node[draw=blue, circle, minimum size=4pt, inner sep=0pt] at (1.25,0.0) {}; %
            \node[right] at (1.5,0.0) {\small Interval CFMDP Policy};
            
            %
            \draw[red, line width=1pt] (5.5,0) -- (6,0);
            \node[red] at (5.75,0) {$\boldsymbol{\times}$}; %
            \node[right] at (6,0) {\small Gumbel-max SCM Policy};
        \end{tikzpicture}
    }\\
    \subfigure[\footnotesize Lowest cumulative reward: Interval CFMDP ($8000$), Gumbel-max SCM ($8000$)]{%
         \resizebox{0.76\columnwidth}{!}{
             \begin{tikzpicture}
                \begin{axis}[
                    xlabel={$t$},
                    ylabel={Mean reward at time step $t$},
                    title={Optimal Path},
                    grid=both,
                    width=20cm, height=8.5cm,
                    every axis/.style={font=\Huge},
                    %
                ]
                \addplot[
                    color=black, %
                    mark=*, %
                    line width=2pt,
                    mark size=3pt,
                ]
                coordinates {
                    (0, -50.0)
                    (1, 50.0)
                    (2, 1000.0)
                    (3, 1000.0)
                    (4, 1000.0)
                    (5, 1000.0)
                    (6, 1000.0)
                    (7, 1000.0)
                    (8, 1000.0)
                    (9, 1000.0)
                };
                %
                \addplot[
                    color=blue, %
                    mark=o, %
                    line width=2pt,
                    mark size=3pt,
                    error bars/.cd,
                    y dir=both, %
                    y explicit, %
                    error bar style={line width=1pt,solid},
                    error mark options={line width=1pt,mark size=4pt,rotate=90}
                ]
                coordinates {
                    (0, -50.0)  +- (0, 0.0)
                    (1, 50.0)  +- (0, 0.0) 
                    (2, 1000.0)  +- (0, 0.0) 
                    (3, 1000.0)  +- (0, 0.0)
                    (4, 1000.0)  +- (0, 0.0)
                    (5, 1000.0) +- (0, 0.0)
                    (6, 1000.0) +- (0, 0.0)
                    (7, 1000.0) +- (0, 0.0)
                    (8, 1000.0) +- (0, 0.0)
                    (9, 1000.0) +- (0, 0.0)
                };
                %
                \addplot[
                    color=red, %
                    mark=x, %
                    line width=2pt,
                    mark size=6pt,
                    error bars/.cd,
                    y dir=both, %
                    y explicit, %
                    error bar style={line width=1pt,solid},
                    error mark options={line width=1pt,mark size=4pt,rotate=90}
                ]
                coordinates {
                    (0, -50.0)  +- (0, 0.0)
                    (1, 50.0)  +- (0, 0.0) 
                    (2, 1000.0)  +- (0, 0.0) 
                    (3, 1000.0)  +- (0, 0.0)
                    (4, 1000.0)  +- (0, 0.0)
                    (5, 1000.0) +- (0, 0.0)
                    (6, 1000.0) +- (0, 0.0)
                    (7, 1000.0) +- (0, 0.0)
                    (8, 1000.0) +- (0, 0.0)
                    (9, 1000.0) +- (0, 0.0)
                };
                %
                \end{axis}
            \end{tikzpicture}
         }
    }
    \hspace{1cm}
    \subfigure[\footnotesize Lowest cumulative reward: Interval CFMDP ($-5980$), Gumbel-max SCM ($-8000$)]{%
         \resizebox{0.76\columnwidth}{!}{
            \begin{tikzpicture}
                \begin{axis}[
                    xlabel={$t$},
                    ylabel={Mean reward at time step $t$},
                    title={Slightly Suboptimal Path},
                    grid=both,
                    width=20cm, height=8.5cm,
                    every axis/.style={font=\Huge},
                    %
                ]
               \addplot[
                    color=black, %
                    mark=*, %
                    line width=2pt,
                    mark size=3pt,
                ]
                coordinates {
                    (0, -50.0)
                    (1, 50.0)
                    (2, -50.0)
                    (3, -50.0)
                    (4, -1000.0)
                    (5, -1000.0)
                    (6, -1000.0)
                    (7, -1000.0)
                    (8, -1000.0)
                    (9, -1000.0)
                };
                %
                \addplot[
                    color=blue, %
                    mark=o, %
                    line width=2pt,
                    mark size=3pt,
                    error bars/.cd,
                    y dir=both, %
                    y explicit, %
                    error bar style={line width=1pt,solid},
                    error mark options={line width=1pt,mark size=4pt,rotate=90}
                ]
                coordinates {
                    (0, -50.0)  +- (0, 0.0)
                    (1, 50.0)  +- (0, 0.0) 
                    (2, -50.0)  +- (0, 0.0) 
                    (3, 20.0631)  +- (0, 49.97539413)
                    (4, 71.206585)  +- (0, 226.02033693)
                    (5, 151.60797) +- (0, 359.23292559)
                    (6, 200.40593) +- (0, 408.86185176)
                    (7, 257.77948) +- (0, 466.10372804)
                    (8, 299.237465) +- (0, 501.82579506)
                    (9, 338.9129) +- (0, 532.06124996)
                };
                %
                \addplot[
                    color=red, %
                    mark=x, %
                    line width=2pt,
                    mark size=6pt,
                    error bars/.cd,
                    y dir=both, %
                    y explicit, %
                    error bar style={line width=1pt,solid},
                    error mark options={line width=1pt,mark size=4pt,rotate=90}
                ]
                coordinates {
                    (0, -50.0)  +- (0, 0.0)
                    (1, 20.00736)  +- (0, 49.99786741) 
                    (2, -12.282865)  +- (0, 267.598755) 
                    (3, -47.125995)  +- (0, 378.41755832)
                    (4, -15.381965)  +- (0, 461.77616558)
                    (5, 41.15459) +- (0, 521.53189262)
                    (6, 87.01595) +- (0, 564.22243126 )
                    (7, 132.62376) +- (0, 607.31338037)
                    (8, 170.168145) +- (0, 641.48013693)
                    (9, 201.813135) +- (0, 667.29441777)
                };
                %
                %
                %
                %
                %
                %
                %
                %
                %
                %
                %
                %
                %
                %
                %
                %
                %
                %
                %
                \end{axis}
            \end{tikzpicture}
         }
    }\\[-1.5pt]
    \subfigure[\footnotesize Lowest cumulative reward: Interval CFMDP ($100$), Gumbel-max SCM ($100$)]{%
         \resizebox{0.76\columnwidth}{!}{
             \begin{tikzpicture}
                \begin{axis}[
                    xlabel={$t$},
                    ylabel={Mean reward at time step $t$},
                    title={Almost Catastrophic Path},
                    grid=both,
                    every axis/.style={font=\Huge},
                    width=20cm, height=8.5cm,
                    %
                ]
               \addplot[
                    color=black, %
                    mark=*, %
                    line width=2pt,
                    mark size=3pt,
                ]
                coordinates {
                    (0, -50.0)
                    (1, 50.0)
                    (2, 50.0)
                    (3, 50.0)
                    (4, -50.0)
                    (5, 50.0)
                    (6, -50.0)
                    (7, 50.0)
                    (8, -50.0)
                    (9, 50.0)
                };
                %
                %
                \addplot[
                    color=blue, %
                    mark=o, %
                    line width=2pt,
                    mark size=3pt,
                    error bars/.cd,
                    y dir=both, %
                    y explicit, %
                    error bar style={line width=1pt,solid},
                    error mark options={line width=1pt,mark size=4pt,rotate=90}
                ]
                coordinates {
                    (0, -50.0)  +- (0, 0.0)
                    (1, 50.0)  +- (0, 0.0) 
                    (2, 50.0)  +- (0, 0.0) 
                    (3, 50.0)  +- (0, 0.0)
                    (4, -50.0)  +- (0, 0.0)
                    (5, 50.0) +- (0, 0.0)
                    (6, -50.0) +- (0, 0.0)
                    (7, 50.0) +- (0, 0.0)
                    (8, -50.0) +- (0, 0.0)
                    (9, 50.0) +- (0, 0.0)
                };
                %
                \addplot[
                    color=red, %
                    mark=x, %
                    line width=2pt,
                    mark size=6pt,
                    error bars/.cd,
                    y dir=both, %
                    y explicit, %
                    error bar style={line width=1pt,solid},
                    error mark options={line width=1pt,mark size=4pt,rotate=90}
                ]
                coordinates {
                    (0, -50.0)  +- (0, 0.0)
                    (1, 50.0)  +- (0, 0.0) 
                    (2, 50.0)  +- (0, 0.0) 
                    (3, 50.0)  +- (0, 0.0)
                    (4, -50.0)  +- (0, 0.0)
                    (5, 50.0) +- (0, 0.0)
                    (6, -50.0) +- (0, 0.0)
                    (7, 50.0) +- (0, 0.0)
                    (8, -50.0) +- (0, 0.0)
                    (9, 50.0) +- (0, 0.0)
                };
                %
                %
                %
                %
                %
                %
                %
                %
                %
                %
                %
                %
                %
                %
                %
                %
                %
                %
                %
                \end{axis}
            \end{tikzpicture}
         }
    }
    \hspace{1cm}
    \subfigure[\footnotesize Lowest cumulative reward: Interval CFMDP ($-7150$), Gumbel-max SCM ($-9050$)]{%
         \resizebox{0.76\columnwidth}{!}{
            \begin{tikzpicture}
                \begin{axis}[
                    xlabel={$t$},
                    ylabel={Mean reward at time step $t$},
                    title={Catastrophic Path},
                    grid=both,
                    width=20cm, height=8.5cm,
                    every axis/.style={font=\Huge},
                    %
                ]
               \addplot[
                    color=black, %
                    mark=*, %
                    line width=2pt,
                    mark size=3pt,
                ]
                coordinates {
                    (0, -50.0)
                    (1, -50.0)
                    (2, -1000.0)
                    (3, -1000.0)
                    (4, -1000.0)
                    (5, -1000.0)
                    (6, -1000.0)
                    (7, -1000.0)
                    (8, -1000.0)
                    (9, -1000.0)
                };
                %
                %
                \addplot[
                    color=blue, %
                    mark=o, %
                    line width=2pt,
                    mark size=3pt,
                    error bars/.cd,
                    y dir=both, %
                    y explicit, %
                    error bar style={line width=1pt,solid},
                    error mark options={line width=1pt,mark size=4pt,rotate=90}
                ]
                coordinates {
                    (0, -50.0)  +- (0, 0.0)
                    (1, -50.0)  +- (0, 0.0) 
                    (2, -50.0)  +- (0, 0.0) 
                    (3, -841.440725)  += (0, 354.24605512) -= (0, 158.559275)
                    (4, -884.98225)  += (0, 315.37519669) -= (0, 115.01775)
                    (5, -894.330425) += (0, 304.88572805) -= (0, 105.669575)
                    (6, -896.696175) += (0, 301.19954514) -= (0, 103.303825)
                    (7, -897.4635) += (0, 299.61791279) -= (0, 102.5365)
                    (8, -897.77595) += (0, 298.80392585) -= (0, 102.22405)
                    (9, -897.942975) += (0, 298.32920557) -= (0, 102.057025)
                };
                %
                \addplot[
                    color=red, %
                    mark=x, %
                    line width=2pt,
                    mark size=6pt,
                    error bars/.cd,
                    y dir=both, %
                    y explicit, %
                    error bar style={line width=1pt,solid},
                    error mark options={line width=1pt,mark size=4pt,rotate=90}
                ]
            coordinates {
                    (0, -50.0)  +- (0, 0.0)
                    (1, -360.675265)  +- (0, 479.39812699) 
                    (2, -432.27629)  +- (0, 510.38620897) 
                    (3, -467.029545)  += (0, 526.36009628) -= (0, 526.36009628)
                    (4, -439.17429)  += (0, 583.96638919) -= (0, 560.82571)
                    (5, -418.82704) += (0, 618.43027478) -= (0, 581.17296)
                    (6, -397.464895) += (0, 652.67322574) -= (0, 602.535105)
                    (7, -378.49052) += (0, 682.85407033) -= (0, 621.50948)
                    (8, -362.654195) += (0, 707.01412023) -= (0, 637.345805)
                    (9, -347.737935) += (0, 729.29076479) -= (0, 652.262065)
                };
                %
                %
                %
                %
                %
                %
                %
                %
                %
                %
                %
                %
                %
                %
                %
                %
                %
                %
                %
                \end{axis}
            \end{tikzpicture}
         }
    }
    \caption{Average instant reward of CF paths induced by policies on Sepsis.}
    \label{fig: reward sepsis}
\end{figure*}

%
%
%
\subsection{Interval CFMDP Bounds}
%
%
Table \ref{tab:nonzero_probs} presents the mean counterfactual probability bound widths (excluding transitions where the upper bound is $0$) for each MDP, averaged over 20 observed paths. We compare the bounds under counterfactual stability (CS) and monotonicity (M) assumptions, CS alone, and no assumptions. This shows that the assumptions marginally reduce the bound widths, indicating the assumptions tighten the bounds without excluding too many causal models, as intended.
\renewcommand{\arraystretch}{1}

\begin{table}
\centering
\caption{Mean width of counterfactual probability bounds}
\resizebox{0.8\columnwidth}{!}{%
\begin{tabular}{|c|c|c|c|}
\hline
\multirow{2}{*}{\textbf{Environment}} & \multicolumn{3}{c|}{\textbf{Assumptions}} \\ \cline{2-4}
 & \textbf{CS + M} & \textbf{CS} & \textbf{None\tablefootnote{\jl{Equivalent to \citet{li2024probabilities}'s bounds (see Section \ref{sec: equivalence with Li}).}}} \\ \hline
\textbf{GridWorld} ($p=0.9$) & 0.0817 & 0.0977 & 0.100 \\ \hline
\textbf{GridWorld} ($p=0.4$) & 0.552  & 0.638  & 0.646 \\ \hline
\textbf{Sepsis} & 0.138 & 0.140 & 0.140 \\ \hline
\end{tabular}
}
\label{tab:nonzero_probs}
\end{table}


\subsection{Execution Times}
Table \ref{tab: times} compares the average time needed to generate the interval CFMDP vs.\ the Gumbel-max SCM CFMDP for 20 observations.
The GridWorld algorithms were run single-threaded, while the Sepsis experiments were run in parallel.
Generating the interval CFMDP is significantly faster as it uses exact analytical bounds, whereas the Gumbel-max CFMDP requires sampling from the Gumbel distribution to estimate counterfactual transition probabilities. \jl{Since constructing the counterfactual MDP models is the main bottleneck in both approaches, ours is more efficient overall and suitable for larger MDPs.}
\begin{table}
\centering
\caption{Mean execution time to generate CFMDPs}
\resizebox{0.99\columnwidth}{!}{%
\begin{tabular}{|c|c|c|}
\hline
\multirow{2}{*}{\textbf{Environment}} & \multicolumn{2}{c|}{\textbf{Mean Execution Time (s)}} \\ \cline{2-3} 
                                      & \textbf{Interval CFMDP} & \textbf{Gumbel-max CFMDP} \\ \hline
\textbf{GridWorld ($p=0.9$) }                  & 0.261                   & 56.1                      \\ \hline
\textbf{GridWorld ($p=0.4$)  }                 & 0.336                   & 54.5                      \\ \hline
\textbf{Sepsis}                                 & 688                     & 2940                      \\ \hline
\end{tabular}%
}
\label{tab: times}
\end{table}


We first outline the LLM and dataset selection process in Sections~\ref{sec:techchallenge} and~\ref{sec:dataset} respectively, followed by our setup in Section~\ref{sec:setup}.
We then report our results and a subsequent discussion in Sections~\ref{sec:results} and~\ref{sec:results-discussion} respectively, before concluding with a note on the usability of our techniques in Section~\ref{sec:usability}.

\subsection{LLM Selection}
\label{sec:techchallenge}
%This work faced several technical challenges arising from the limitations of available large language models (LLMs) for code generation. 
%The issues primarily stem from two areas: the quality of open-source models and the restricted capabilities of certain proprietary models.

%\textbf{Open-Source Model Quality:} Open-source models available on HuggingFace~\cite{huggingface} exhibit significant quality issues in code generation tasks. 
%            Initial experiments with general-purpose LLMs, such as \gemini, on a code hallucinations focused dataset~\cite{codehalu} revealed a negligible pass rate. 


We selected the open-source model \codegenmonoC from \salesforce~\cite{salesforcecodegen} because it was specifically pre-trained for \python, the language used in our dataset.
%
%We tested models of different sizes from \salesforce~\cite{salesforcecodegen}: \codegenmonoA, \codegenmonoB and \codegenmonoC given that they are tailored for code generation for the specific language \ie \python           
 %           Despite these models being tailored for code generation for the specific language \ie \python, their performance remained inadequate, with the best-performing model (\codegenmonoC) achieving only a \bestSalesforcePassrate top pass rate on our benchmark. 
 %           We also applied \llama to our benchmark and it achieved a comparable correctness percentage to \salesforce/\codegenmonoC.
 %           This highlights the current gap in the quality of open-source models for advanced code-generation tasks.
%
%\textbf{API Limitations of Proprietary Models:}
%
We also picked a proprietary model, namely \gptturbo~\cite{gpt35turboinstruct} from \openai~\cite{openai}.
%, demonstrate stronger performance in code generation, they pose challenges for tasks requiring fine-grained uncertainty quantification. 
            Unfortunately, the latest models from \openai don't expose the \texttt{logprobs} functionality through their API, which is needed for our computations. 
%            The only exception was \gptturbo~\cite{gpt35turboinstruct}, which supports \texttt{logprobs} and was therefore selected for our evaluation in this work. These API limitations restrict the range of models that can be used for tasks involving detailed uncertainty analysis.

%These challenges underscore the trade-offs between model quality and technical compatibility when evaluating uncertainty in code generation. 
%Improvements in both open-source and proprietary LLM capabilities are necessary to advance this area of research.



\subsection{Dataset Selection}\label{sec:dataset}
To evaluate the performance of our techniques, we use \livecodebench~\cite{livecodebench}, a contamination-free benchmark for assessing LLMs on code-related tasks
containing \totalProbsComb~problems, divided into \emph{Easy} (\totalProbs problems), \emph{Medium} (\totalProbsMedium problems) and \emph{Hard} (\totalProbsHard problems). 
\livecodebench is designed to address key challenges in code evaluation by incorporating diverse problems from programming competition platforms such as LeetCode, AtCoder, and CodeForces. 
Notably, \livecodebench's contamination-free design ensures that the selected problems have not been seen during the training of most modern LLMs, thereby eliminating data leakage concerns. 
%This guarantees that the benchmark accurately measures a model's generalisation ability rather than its capacity to memorize training data, making it a reliable benchmark for evaluating novel techniques in code generation.

%In our evaluation, we use the entirety of \livecodebench~dataset \ie \totalProbsComb~problems, divided into \emph{Easy, Medium} and \emph{Hard}.

To start with, we focused on the subset of \emph{Easy} problems in \livecodebench, where the two selected LLMs exhibited average testcase passing rates of
\SFSolutionsPassRate~for \codegenmonoC~and \GPTSolutionsPassRateSmall~ for \gptturbo.
Due to limited compute resources and the fact that \codegenmonoC already struggled on the \emph{Easy} problems, 
we then extended our evaluation to the \emph{Medium} and \emph{Hard} classes of problems only for \gptturbo. %, in order for us to get a better understanding of the applicability of our proposed techniques.
For \emph{Medium} problems, \gptturbo achieved a passing rate of \GPTSolutionsPassRateMedium, while for \emph{Hard} problems, the passing rate was \GPTSolutionsPassRateHard.
%reasons described in \S\ref{sec:techchallenge}.
%These problems are curated from high-quality competition datasets and are suitable for testing the functional correctness of generated solutions. 

Each problem is accompanied by a natural language description and Input/Output test cases. We use the natural language description in the query provided to the LLM (as shown in Figure~\ref{fig:sampleproblem}), and the test cases for evaluating our techniques.


%\begin{itemize}[leftmargin=*]
%    \item \textbf{Natural Language Problem Description:} A clear and detailed task description, written in natural language.
%    \item \textbf{Input/Output Test Cases:} A set of test cases used for verifying the correctness of the generated code.
%\end{itemize}

%The decision to use the \emph{Easy} subset stems from the observation that the performance of the LLMs under evaluation was insufficient on harder problems. 
%Higher-difficulty problems often require complex reasoning and sophisticated algorithms, which are challenging for the current generation of LLMs. 
%These limitations could result in poor-quality outputs, undermining our ability to effectively evaluate semantic uncertainty and functional clustering. 

%By selecting problems where the models exhibit a reasonable baseline performance, we ensure a meaningful evaluation of our proposed techniques. 
%The \emph{Easy} subset provides a controlled evaluation setting by avoiding extreme algorithmic complexity while maintaining sufficient diversity in problem types. 
%This allows us to focus on analysing semantic uncertainty and functional clustering in code generation without the confounding effects of low model performance.

%\livecodebench's contamination-free design ensures that the selected problems have not been seen during the training of most modern LLMs, making it a reliable benchmark for the purpose of this work.

\subsection{Experimental Setup}\label{sec:setup}

%In this study, we evaluate the performance of our techniques on a benchmark of \totalProbs problems.
Our experiments were conducted on a machine running \texttt{Ubuntu 20.04.5 LTS (Focal Fossa)} with one \texttt{NVIDIA A100 GPU (80GB)}.
 
%The evaluation involves querying a large language model (LLM) for each problem and applying both SE-based and MI-based approaches to compute uncertainty scores. 
%
%Additionally, in order to investigate the impact of our decision of using \textit{length-normalised response probablities}, we carry out an ablation study by using both the \textit{unnormalised, raw response probablities} alongside our \textit{length-normalised one}. 
%
%Finally, we assess the correctness of the generated solutions based on their performance on test cases provided by the benchmark.

For the semantic equivalence approach in Section~\ref{sec:symex}, we ask the LLM to generate \numSamples responses for each problem along with their respective \texttt{log-probabilities}.
%, as returned by the model's API. Using these \texttt{logprobs}, we compute the probability of each response. 
On top of that, for the MI-based approach described in Section~\ref{sec:mi}, we perform \numIterations iterations of prompting for each of the \numSamples generated responses. The first iteration involves querying the model with the original prompt, while the second iteration uses a concatenated prompt, combining the original prompt and the response from the first iteration. 

In order to perform symbolic execution for finding clusters, we use \crosshair~\cite{crosshair} with a per condition timeout of \CrosshairPerConditiontimeout and the same per path timeout of \CrosshairPerPathtimeout. 
In addition to this, we also impose an overall timeout of \Crosshairtotaltimeout for each pair of programs that is being checked for equivalence.
If a counterexample showing the difference in behavior is not found within this timeout, we assume that the programs belong in the same cluster. 
% \CD{clarify this}

%The SE score is then calculated based on the distribution of probabilities over the generated responses, as described in Section~\ref{sec:symex}.

We evaluate the following techniques:
\begin{itemize}[leftmargin=*]
\item \textbf{\SESymbolicRaw:} The semantic equivalence approach described in Section~\ref{sec:symex}.
%  For our SE-based approach, we query the LLM to generate \numSamples responses for each problem along with their respective \texttt{logprobs}, as returned by the model's API. 
 %                       Using these \texttt{logprobs}, we compute the probability of each response. 
    
  %                      For the \crosshair~\cite{crosshair} runs that are used to compute the clusters symbolically, we use a per condition timeout of \CrosshairPerConditiontimeout and an identical per path timeout of \CrosshairPerPathtimeout. 
   %                     Pairs of programs that hit the timeout are considered to be in the same cluster. 

    %                    The SE score is then calculated based on the distribution of probabilities over the generated responses, as described in Section~\ref{sec:symex}.
\item \textbf{\SESymbolic:} The semantic equivalence approach, but with length normalization (see Section~\ref{sec:probcomp}).
\item \textbf{\SESymbolicUnif:} The semantic equivalence approach, but with uniform distribution (see Section~\ref{sec:probcomp}).  
  %\emph{length-normaised response probablities} \ie while using \textit{unnormalised, raw response probablities}.  

    \item \textbf{\MISymbolicRaw:} The MI approach described in Section~\ref{sec:mi}.
                        %The response probabilities are computed using the same mechanism as the SE-based approach. 
                        %The \crosshair specific timeout remains the same as the SE-based approach. 
                        
                        %The MI score is then calculated as outlined in Section~\ref{sec:mi}.
    \item \textbf{\MISymbolic:} The MI approach, but with length normalization. %This is our adapted MI-based approach, but without the \textit{length-normaised response probablities} \ie while using \textit{unnormalised, raw response probablities}.
    \item \textbf{\SEOriginal:} The original implementation from Kuhn et al.~\cite{kuhnsemantic}. This incorporates length normalization, as omitting it led to probability underflows, resulting in NaN values.
    \item \textbf{\MIOriginal:} The original implementation from Abbasi et al.~\cite{abbasi2024believe}, with length normalization.
    \item \textbf{\LLMProbability:} Baseline technique where the model's response probablities are used as a proxy for correctness. 
\end{itemize}

As mentioned in Section~\ref{sec:mi}, for the MI approach, we did not explore the uniform distribution variant, as the LLM-reported probabilities are central to the technique being used to distinguish between aleatoric and epistemic uncertainties. Additionally, we prioritized experimentation with the better-performing technique---the semantic equivalence-based one.

For each problem in the benchmark, we compute:
\begin{itemize}[leftmargin=*]
\item The corresponding \emph{uncertainty score} for \SESymbolicRaw, \SESymbolic, \SESymbolicUnif, \SEOriginal, \MISymbolicRaw, \MISymbolic and \MIOriginal.
  \item The \emph{probablity of the top-ranked response} for \LLMProbability. 
  \item The \emph{correctness score}, which is calculated as follows: the top-ranked response generated by the LLM (as determined by the API's ranking) is executed against the benchmark's test cases. The percentage of test cases successfully passed is recorded as the correctness score. A generous timeout of \CorrectnessTimeout is applied to each test case. If the candidate solution exceeds this timeout, the test case is considered ``failed''.
    %\item The Pearson correlation factor and the resultant p-value for each of the collected scores and the correctness values for each problem. 
\end{itemize}


\subsection{Results}\label{sec:results}

In Table ~\ref{tab:correlation_results}, we compute the Pearson correlation between the uncertainty scores and the correctness scores for \SESymbolic, \SESymbolicUnif, \SEOriginal, \MISymbolic and \MIOriginal, respectively, and the correlation between the probablity of the top-ranked response and the correctness score for \LLMProbability. We report the Pearson correlation coefficient and p-value, with statistically significant results highlighted in bold.
%
We excluded the results for \SESymbolicRaw and \MISymbolicRaw from Table ~\ref{tab:correlation_results}, as probabilities underflowed (discussed in Section~\ref{sec:probcomp}) causing them to yield NaN values.

Table~\ref{tab:average_lines_tokens} shows an overview of the size of the responses received from the two models considered in this work: \gptturbo and \codegenmonoC. % (from \salesforce).
%As per the earlier discussed in \S\ref{sec:probcomp}, the length of the response from the models affects the performance of the techniques used.

%As such, the responses are \emph{small} with regards to Lines of Code (LOC), with \gptturbo producing, on average, \GPTSolutionsLines~lines long solutions and \salesforce/\codegenmonoC producing \SFSolutionsLines~lines on average. 

%However, the average token length is \emph{large} enough for both models (\GPTSolutionsToken~for \gptturbo and \SFSolutionsToken~for \salesforce/\codegenmonoC) such that it can significantly impact the effectiveness of the techniques used, as we will see later in the section. 

% \begin{table*}[ht!]
%     \centering
%     \caption{Correlation Results for Different Techniques}
%     \label{tab:correlation_results}

%     \begin{tabular}{l r r}
%         \toprule
%         \textbf{Technique} & \multicolumn{2}{c}{\textbf{Model (Pearson Correlation, p-value)}} \\
%         \cmidrule(r){2-3}
%          & \textbf{\gptturbo} & \textbf{\codegenmonoC} \\
%         \midrule
%         \SEOriginal & \SENLGPearsonGPT ~(\SENLGPearsonPValueGPT) & \SENLGPearsonSF ~(\SENLGPearsonPValueSF) \\

%         \SESymbolic & \SESymbolicPearsonGPT ~(\SESymbolicPearsonPValueGPT) & \SESymbolicPearsonSF ~(\SESymbolicPearsonPValueSF) \\
%         \SESymbolicRaw & NaN (NaN) & NaN (NaN) \\
%         \midrule
%         \MIOriginal & - & - \\
  
%         \MISymbolic & \MISymbolicPearsonGPT ~(\MISymbolicPearsonPValueGPT) & \MISymbolicPearsonSF ~(\MISymbolicPearsonPValueSF) \\
%         \MISymbolicRaw & - & - \\
%         \midrule
%         \LLMProbability & \LLMProbabilityPearsonGPT ~(\LLMProbabilityPearsonPValueGPT) & \LLMProbabilityPearsonSF ~(\LLMProbabilityPearsonPValueSF) \\
%         \bottomrule
%     \end{tabular}

% \end{table*}

\begin{table*}[ht!]
    \centering
    \small
    \caption{Correlation Results---Pearson coefficient(p-value)---for Different Techniques on Different Difficulty Problems (\gptturbo: Easy/Medium/Hard; \codegenmonoC: Easy). Statistically significant results are highlighted in bold.}
    \label{tab:correlation_results}
    \begin{tabularx}{\linewidth}{Xrrrr}
        \toprule
        \multirow{2}{*}{\textbf{Technique}} 
        & \multicolumn{3}{c}{\textbf{\gptturbo}} & \textbf{\codegenmonoC} \\
        \cmidrule(lr){2-4} \cmidrule(lr){5-5}
        & \textbf{Easy} & \textbf{Medium} & \textbf{Hard} & \textbf{Easy} \\
        \midrule

        \SEOriginal
          & \SENLGPearsonGPTUpdated (\SENLGPearsonPValueGPTUpdated)  
          & \SENLGPearsonGPTMediumUpdated (\SENLGPearsonPValueGPTMediumUpdated)                                         
          & \SENLGPearsonGPTHardUpdated (\SENLGPearsonPValueGPTHardUpdated)                                        
          & \SENLGPearsonSFUpdated (\SENLGPearsonPValueSFUpdated)    
          \\

        \SESymbolic
          & \textbf{\SESymbolicPearsonGPTUpdated (\SESymbolicPearsonPValueGPTUpdated)}
          & \textbf{\SESymbolicPearsonGPTMediumUpdated (\SESymbolicPearsonPValueGPTMediumUpdated)}
          & \textbf{\SESymbolicPearsonGPTHardUpdated (\SESymbolicPearsonPValueGPTHardUpdated)}
          & \textbf{\SESymbolicPearsonSFUpdated (\SESymbolicPearsonPValueSFUpdated)}
          \\

        \SESymbolicUnif
         & \textbf{\SESymbolicUnifPearsonGPTUpdated (\SESymbolicUnifPearsonPValueGPTUpdated)}
         & \textbf{\SESymbolicUnifPearsonGPTMediumUpdated (\SESymbolicUnifPearsonPValueGPTMediumUpdated)}
         & \textbf{\SESymbolicUnifPearsonGPTHardUpdated (\SESymbolicUnifPearsonPValueGPTHardUpdated)}
         & \textbf{\SESymbolicUnifPearsonSFUpdated (\SESymbolicUnifPearsonPValueSFUpdated)}
         \\

        \midrule

        \MIOriginal
         & \MINLGPearsonGPTUpdated (\MINLGPearsonPValueGPTUpdated)  
         & \MINLGPearsonGPTMediumUpdated (\MINLGPearsonPValueGPTMediumUpdated)                                         
         & \MINLGPearsonGPTHardUpdated (\MINLGPearsonPValueGPTHardUpdated)                                        
         & \MINLGPearsonSFUpdated (\MINLGPearsonPValueSFUpdated)    
         \\

        \MISymbolic
          & \textbf{\MISymbolicPearsonGPTUpdated (\MISymbolicPearsonPValueGPTUpdated)}
          & \textbf{\MISymbolicPearsonGPTMediumUpdated (\MISymbolicPearsonPValueGPTMediumUpdated)}
          & \textbf{\MISymbolicPearsonGPTHardUpdated (\MISymbolicPearsonPValueGPTHardUpdated)}
          & \textbf{\MISymbolicPearsonSFUpdated (\MISymbolicPearsonPValueSFUpdated)}
          \\

        \midrule

        \LLMProbability
          & \LLMProbabilityPearsonGPTUpdated (\LLMProbabilityPearsonPValueGPTUpdated)
          & \LLMProbabilityPearsonGPTMediumUpdated (\LLMProbabilityPearsonPValueGPTMediumUpdated)
          & \LLMProbabilityPearsonGPTHardUpdated (\LLMProbabilityPearsonPValueGPTHardUpdated)
          & \LLMProbabilityPearsonSFUpdated (\LLMProbabilityPearsonPValueSFUpdated)
          \\

        \bottomrule
    \end{tabularx}
\end{table*}

% \begin{table*}[ht!]
%     \centering
%     \caption{Average Number of Lines and Tokens for Solutions Generated by \gptturbo and \salesforce/\codegenmonoC}
%     \label{tab:average_lines_tokens}

%     \begin{tabular}{l r r}
%         \toprule
%         \textbf{Metric} & \multicolumn{2}{c}{\textbf{Model}} \\
%         \cmidrule(r){2-3}
%          & \textbf{\gptturbo} & \textbf{\codegenmonoC} \\
%         \midrule
%         Average Lines of Code & \GPTSolutionsLines & \SFSolutionsLines \\
%         Average Tokens & \GPTSolutionsToken & \SFSolutionsToken \\
%         \bottomrule
%     \end{tabular}
% \end{table*}

\begin{table*}[ht!]
    \centering
    \caption{Average Number of Lines and Tokens for Solutions Generated by \gptturbo and \salesforce/\codegenmonoC for Different Classes of Problems}
    \label{tab:average_lines_tokens}

    \begin{tabular}{l r r r r}
        \toprule
        \multirow{2}{*}{\textbf{Metric}} 
          & \multicolumn{3}{c}{\gptturbo}
          & \multicolumn{1}{c}{\codegenmonoC} \\
        \cmidrule(r){2-4}\cmidrule(r){5-5}
          & \textbf{Easy}
          & \textbf{Medium}
          & \textbf{Hard}
          & \textbf{Easy} \\

        \midrule
        Average Lines of Code
          & \GPTSolutionsLines
          & \GPTSolutionsLinesMedium
          & \GPTSolutionsLinesHard
          & \SFSolutionsLines \\
        Average Tokens
          & \GPTSolutionsToken
          & \GPTSolutionsTokenMedium
          & \GPTSolutionsTokenHard
          & \SFSolutionsToken \\
        \bottomrule
    \end{tabular}
\end{table*}

%Table~\ref{tab:correlation_results} summarises the results obtained for the various techniques considered in this experimental evaluation. 
\subsection{Discussion of Results}\label{sec:results-discussion}


\paragraph{RQ1: Is there a correlation between the uncertainty computed by the proposed techniques and correctness for code generation?}

%The experiments presented in Figure~\ref{tab:correlation_results}, we observe

The results in Table~\ref{tab:correlation_results} allow us to make the following observations:

$\bullet$ There is a \emph{\textbf{strong negative correlation between correctness and the uncertainty}} computed by \SESymbolic and \SESymbolicUnif and a \emph{\textbf{weak negative correlation}} for \MISymbolic.

$\bullet$ \emph{\textbf{Semantic equivalence among the LLM responses is a stronger indicator of correctness than the actual reported log-probabilities}}---while the probabilities assigned by the LLM carry minimal signal (see RQ4), semantic clustering provides significant insight. This is suggested by the fact that \SESymbolicUnif, which assumes uniform probabilities over LLM reponses, exhibits a correlation similar to that of \SESymbolic with length-normalization.
%
This finding is important as \SESymbolicUnif to be applied even when the LLM does not expose log-probabilities, such as in the latest GPT models.

An open question emerging from this is whether we can prove a direct correlation between the number of semantic clusters and correctness. As our current focus is on the relationship between uncertainty and correctness, we leave this investigation for future work.

$\bullet$ \emph{\textbf{The correlation between uncertainty and correctness remains stable across different problem complexities and sizes}} (i.e. easy, medium, hard). Our experiments for \SESymbolic, \SESymbolicUnif and \MISymbolic show no significant degradation with increased size and difficulty for \gptturbo.

$\bullet$ \emph{\textbf{There is little variation in the correlation between uncertainty and correctness across the two LLMs of different sizes}}. The results for \SESymbolic, \SESymbolicUnif, and \MISymbolic exhibit slight degradation on \livecodebench-Easy when transitioning from \gptturbo to \codegenmonoC.  However, the correlation remains weak in both cases.

%In contrast to this, our symbolic execution based adaptation that makes use of \textit{length-normaised response probablities} performed better than its contemporaries.
%For \gptturbo, \SESymbolic obtained a moderate negative correlation score of \SESymbolicPearsonGPT~with a statistically significant p-value of \SESymbolicPearsonPValueGPT. 
%The correlation is expected to be negative here as we are working with uncertainty scores hence a higher uncertainty score should be reflective of lower correctness.
%The story for \salesforce/\codegenmonoC was also the same. 


\paragraph{RQ2: How do the techniques based on semantic equivalence compare against those based on mutual information?}


We observed a stronger correlation between uncertainty and correctness in techniques based on semantic equivalence compared to those relying on mutual information. We hypothesize that this is due to the unambiguous nature of the problems in our dataset, reducing the need to account for aleatoric uncertainty. We leave the exploration of ambiguity in the problem formulation as future work.

\paragraph{RQ3: Are the techniques designed for natural language directly applicable to code generation?}
In all our experiments, \SEOriginal and \MIOriginal yield results that are not statistically significant, highlighting the necessity of adapting these techniques to account for the unique characteristics of code.
%The story for \SEOriginal is similar, but from a different perspective. 
%For almost all of the responses, the natural language based clustering mechanism puts all of the model's responses into a single cluster.
%This meant that the semantic-based clustering mechanism did not get a chance to contribute towards the uncertainty scores.
%Consequently, for both \gptturbo and \salesforce/\codegenmonoC, the results are \emph{not} statistically significant with the respective p-value scores of \SENLGPearsonPValueGPT~and \SENLGPearsonPValueSF, which are both larger than the threshold of \pvalue.



%\paragraph{\textbf{RQ4: Is length normalization useful for code generation?}}
%For \SESymbolicRaw, there \textit{raw response probabilities} underflowed to zero for all responses. 
%This is in accordance with the observation from Kuhn~\etal~\cite{kuhnsemantic} when responses have a large number of tokens, which is the case for our generation queries for both models, as shown in Table~\ref{tab:average_lines_tokens}.
%This, in turn, results in all uncertainty computations to decay to zero, which results in an NaN during the correlation computation.

\paragraph{RQ4: Are the probabilities computed by the LLM sufficient as a proxy for correctness?}
In all our experiments, except for \gptturbo on \livecodebench-Easy, the results are not statistically significant. This suggests that there is no meaningful correlation between correctness and the log-probabilities reported by the LLM.

%For \gptturbo while the \LLMProbability metric shows a \emph{small}, statistically significant (with~\LLMProbabilityPearsonPValueGPT~as the p-value) positive\footnote{The correlation here is positive since a high probability of the top-ranked response from the model should indicate a higher level of correctness.} correlation of~\LLMProbabilityPearsonGPT, for \salesforce/\codegenmonoC the results are \emph{not} statistically significant given the p-value \LLMProbabilityPearsonPValueSF~which is greater than the standard \pvalue. This shows the lack of effectiveness of simply using the model's probablities directly to decide the uncertainty in the model's outputs.


\subsection{Using the Uncertainty Estimation for Correctness}
\label{sec:usability}
We envision uncertainty scores being utilized as part of an abstention policy, similar to the approach described by Abbasi~\etal~\cite{abbasi2024believe}.
In this policy, an uncertainty threshold is set such that LLM responses with scores above the threshold are assumed to be good according to some metric, while those below the threshold are considered bad.
For the latter, the LLM may abstain from presenting these responses to the user.

The primary metric of interest in this work is functional correctness. LLM responses are classified as correct if their correctness score exceeds a predefined threshold and incorrect otherwise. To determine the \emph{uncertainty threshold} for the abstention policy, we select the value that maximizes correctness accuracy.

For our experimental evaluation, we considered the best performing techniques for uncertainty assessment, \SESymbolic and \SESymbolicUnif, and \LLMProbability as the baseline.
We then considered our more successful model~\gptturbo and the \livecodebench dataset, using a correctness score threshold of 90\%. 
%
As discussed in Section~\ref{sec:dataset}, \gptturbo's responses exhibit a low unit test passing rate, leading to an imbalanced dataset where incorrect responses significantly outnumber correct ones. This imbalance risks trivializing the classification task, as an ``all-incorrect'' model would dominate. To address this, we followed standard practice and applied random downsampling, removing a portion of incorrect responses to prevent the model from defaulting to a trivial solution. We split the dataset 50/50 into training and validation.
To mitigate overfitting and reduce bias,  we employed 2-fold cross-validation, alternating between training and validation for each fold.

%The metric we are conserned with in this work is functional correctness. Thus, we label LLM responses as \emph{correct}  if their correctness score exceeds the correctness threhold, or \emph{incorrect} otherwise. Then, we obtain the \emph{uncertainty threshold} to be used by the abstention policy by picking the one that maximizes accuracy. 
%In the experiments summarized in Table~\ref{tab:abstention_metrics}, we used the responses provided by \gptturbo to the \livecodebench dataset, with a correctness threshold of 90\%. To reduce overfitting and prevent biases, we used a 2-fold cross-validation, rotating the role of each fold between training and validation. as explained in Section~\ref{sec:dataset}, these responses exhibit relatively low unit test passing rate resulting in an imbalanced dataset with many more incorrect responses, causing a trivial ``all incorrect'' solution to dominate. Following common practice, we downsampled by at random dropping a certain percentage of points that are causing the trivial ``all-incorrect'' response.


%As it is the practice, if extreme class imbalance (\eg many near-zero correctness cases) causes a trivial ``all incorrect'' solution to dominate, we down-sample or partially discard excessively low-correctness samples to ensure a more balanced dataset (and thus avoid collapsing to a trivial threshold).

%Finally, we fix this threshold and measure performance on the validation set, whereby any response with an uncertainty \emph{below} the learned threshold is deemed sufficiently reliable, and any response above it is \emph{abstained} from the user.

%Concretely, we first combine each response's \emph{correctness score}, a measure of how well the solution aligns with ground truth and \emph{uncertainty score} into a single dataset.
%fraction exceeds a specified cutoff (\eg 0.9) 
% Next, we split the dataset into a training set and a validation set, ensuring we do not overfit our threshold to a single subset.

%Next, we carry out training while employing a k-fold cross validation, rotating the role of each fold between training and validation to ensure robust threshold selection.
%We evaluate the accuracy of these predictions against the ground-truth labels and select the threshold that maximizes this accuracy.


%Table~\ref{tab:abstention_metrics} summarizes results for \SESymbolic, \SESymbolicUnif and \LLMProbability for \gptturbo on the whole \livecodebench dataset, with a correctness threshold of 90\%. We used a cross-validation with $k=2$ folds. 


% More precisely, on the training set, we \emph{sort} responses by ascending uncertainty, then \emph{sweep} through possible threshold values; for each candidate threshold, all responses with uncertainty \emph{below} the threshold are predicted as \texttt{correct}, and the rest as \texttt{incorrect}.

%Table~\ref{tab:abstention_metrics}~summarises the performance of using learned abstention threshold for \SESymbolic, \SESymbolicUnif and \LLMProbability for \gptturbo. 

The evaluation results for the uncertainty threshold are presented in Table~\ref{tab:abstention_metrics}. 
Both \SESymbolic and \SESymbolicUnif achieved high accuracy scores of \SENormAcc~and \SEUnifAcc, respectively, whereas \LLMProbability~performed significantly worse, with an accuracy of only~\LLMProbabilityAcc.

A key advantage of our techniques is their extremely low False Positive (FP) rate, with \SENormFP~for both \SESymbolic and \SESymbolicUnif. 
This indicates that our methods are more conservative in accepting LLM-generated responses, a particularly valuable property for code generation. 
Given the potential safety risks associated with incorrect code, especially when LLM performance is inconsistent, this cautious approach enhances the reliability and safety of LLM-assisted coding.


\begin{table*}[ht!]
    \centering
    \caption{Abstention Metrics for \SESymbolic, \SESymbolicUnif and \LLMProbability with \gptturbo}
    \label{tab:abstention_metrics}

    \begin{tabular}{l r r r}
        \toprule
        \textbf{Technique}
          & \multicolumn{1}{c}{\textbf{Accuracy}}
          & \multicolumn{1}{c}{\textbf{False Positives}}
          & \multicolumn{1}{c}{\textbf{False Negatives}} \\
        \midrule
        \SESymbolic
          & \SENormAcc
          & \SENormFP
          & \SENormFN \\
        \SESymbolicUnif
          & \SEUnifAcc
          & \SEUnifFP
          & \SEUnifFN \\
        \LLMProbability
          & \LLMProbabilityAcc
          & \LLMProbabilityFP
          & \LLMProbabilityFN \\
        \bottomrule
    \end{tabular}
\end{table*}

% In practice, one may also tune the threshold for desired precision/coverage trade-offs or use calibration techniques (e.g.\ isotonic regression)
% so that the ``uncertainty score'' better reflects the actual probability of correctness.


%To implement this, we designed a pipeline that processes uncertainty scores and correctness fractions to derive an optimal abstention threshold. 

%The threshold training phase uses a split of the dataset into training and testing subsets, typically with 80\% of the data for training and 20\% for testing. 
%During training, the pipeline iterates over unique uncertainty score thresholds to identify the threshold that maximizes the accuracy metric \ie the number of correct abstentions made for a given correctness criterion (90\% in our case). 
% This metric accounts for both correct predictions, where uncertainty scores are below the threshold and correctness is above a specified value, and correct abstentions, where uncertainty scores are above the threshold and correctness is below this value. 
% The redefinition ensures that the abstention policy rewards both effective predictions and abstentions, promoting balanced decision-making.

%In the evaluation phase, the learned threshold is applied to the test set, where accuracy and abstention rates are calculated. 
% Accuracy is computed as the ratio of correctly handled samples (whether predicted or abstained) to the total number of samples, while the abstention rate reflects the proportion of samples where predictions were withheld due to high uncertainty. 

%We ran this pipeline on the better performing \gptturbo model and achieved a test-set accuracy of \SEAcc~for the SE scores and \MIAcc~for the MI scores. 

% \begin{figure}[ht]
%     \centering
%     % Subfigure 1
%     \begin{subfigure}[b]{0.45\linewidth}
%         \centering
%         \includegraphics[width=\linewidth]{./figures/average_se_bin_090.png}
%         \caption{Correctness Bin with SE scores: 0-90\% and 90-100\%}
%     \end{subfigure}
%     \hfill
%     % Subfigure 2
%     \begin{subfigure}[b]{0.45\linewidth}
%         \centering
%         \includegraphics[width=\linewidth]{./figures/average_mi_bin_090.png}
%         \caption{Correctness Bin with MI scores: 0-90\% and 90-100\%}
%     \end{subfigure}
    
%     \caption{Average SE/MI Scores for \gptturbo responses, showing an abstention threshold value for achieving 90\% correctness.}
%     \label{fig:mise_scores_bins}
% \end{figure}

% \begin{figure}[ht]
%     \centering
%     % Subfigure 1
%     \begin{subfigure}[b]{0.45\linewidth}
%         \centering
%         \includegraphics[width=\linewidth]{./figures/average_mi_bin_060.png}
%         \caption{Correctness Bin: 0-60\% and 60-100\%}
%     \end{subfigure}
%     \hfill
%     % Subfigure 2
%     \begin{subfigure}[b]{0.45\linewidth}
%         \centering
%         \includegraphics[width=\linewidth]{./figures/average_mi_bin_070.png}
%         \caption{Correctness Bin: 0-70\% and 70-100\%}
%     \end{subfigure}
    
%     % Subfigure 3
%     \begin{subfigure}[b]{0.45\linewidth}
%         \centering
%         \includegraphics[width=\linewidth]{./figures/average_mi_bin_080.png}
%         \caption{Correctness Bin: 0-80\% and 80-100\%}
%     \end{subfigure}
%     \hfill
%     % Subfigure 4
%     \begin{subfigure}[b]{0.45\linewidth}
%         \centering
%         \includegraphics[width=\linewidth]{./figures/average_mi_bin_090.png}
%         \caption{Correctness Bin: 0-90\% and 90-100\%}
%     \end{subfigure}
    
%     \caption{Average MI Scores for Different Correctness Bins. Each plot shows the average MI scores computed for two bins.}
%     \label{fig:mi_scores_bins}
% \end{figure}


\section{Related Work}
\label{sec:related}
\putsec{related}{Related Work}

\noindent \textbf{Efficient Radiance Field Rendering.}
%
The introduction of Neural Radiance Fields (NeRF)~\cite{mil:sri20} has
generated significant interest in efficient 3D scene representation and
rendering for radiance fields.
%
Over the past years, there has been a large amount of research aimed at
accelerating NeRFs through algorithmic or software
optimizations~\cite{mul:eva22,fri:yu22,che:fun23,sun:sun22}, and the
development of hardware
accelerators~\cite{lee:cho23,li:li23,son:wen23,mub:kan23,fen:liu24}.
%
The state-of-the-art method, 3D Gaussian splatting~\cite{ker:kop23}, has
further fueled interest in accelerating radiance field
rendering~\cite{rad:ste24,lee:lee24,nie:stu24,lee:rho24,ham:mel24} as it
employs rasterization primitives that can be rendered much faster than NeRFs.
%
However, previous research focused on software graphics rendering on
programmable cores or building dedicated hardware accelerators. In contrast,
\name{} investigates the potential of efficient radiance field rendering while
utilizing fixed-function units in graphics hardware.
%
To our knowledge, this is the first work that assesses the performance
implications of rendering Gaussian-based radiance fields on the hardware
graphics pipeline with software and hardware optimizations.

%%%%%%%%%%%%%%%%%%%%%%%%%%%%%%%%%%%%%%%%%%%%%%%%%%%%%%%%%%%%%%%%%%%%%%%%%%
\myparagraph{Enhancing Graphics Rendering Hardware.}
%
The performance advantage of executing graphics rendering on either
programmable shader cores or fixed-function units varies depending on the
rendering methods and hardware designs.
%
Previous studies have explored the performance implication of graphics hardware
design by developing simulation infrastructures for graphics
workloads~\cite{bar:gon06,gub:aam19,tin:sax23,arn:par13}.
%
Additionally, several studies have aimed to improve the performance of
special-purpose hardware such as ray tracing units in graphics
hardware~\cite{cho:now23,liu:cha21} and proposed hardware accelerators for
graphics applications~\cite{lu:hua17,ram:gri09}.
%
In contrast to these works, which primarily evaluate traditional graphics
workloads, our work focuses on improving the performance of volume rendering
workloads, such as Gaussian splatting, which require blending a huge number of
fragments per pixel.

%%%%%%%%%%%%%%%%%%%%%%%%%%%%%%%%%%%%%%%%%%%%%%%%%%%%%%%%%%%%%%%%%%%%%%%%%%
%
In the context of multi-sample anti-aliasing, prior work proposed reducing the
amount of redundant shading by merging fragments from adjacent triangles in a
mesh at the quad granularity~\cite{fat:bou10}.
%
While both our work and quad-fragment merging (QFM)~\cite{fat:bou10} aim to
reduce operations by merging quads, our proposed technique differs from QFM in
many aspects.
%
Our method aims to blend \emph{overlapping primitives} along the depth
direction and applies to quads from any primitive. In contrast, QFM merges quad
fragments from small (e.g., pixel-sized) triangles that \emph{share} an edge
(i.e., \emph{connected}, \emph{non-overlapping} triangles).
%
As such, QFM is not applicable to the scenes consisting of a number of
unconnected transparent triangles, such as those in 3D Gaussian splatting.
%
In addition, our method computes the \emph{exact} color for each pixel by
offloading blending operations from ROPs to shader units, whereas QFM
\emph{approximates} pixel colors by using the color from one triangle when
multiple triangles are merged into a single quad.



\section{Conclusion}
\label{sec:conclusion}
\section{Conclusion}
In this work, we propose a simple yet effective approach, called SMILE, for graph few-shot learning with fewer tasks. Specifically, we introduce a novel dual-level mixup strategy, including within-task and across-task mixup, for enriching the diversity of nodes within each task and the diversity of tasks. Also, we incorporate the degree-based prior information to learn expressive node embeddings. Theoretically, we prove that SMILE effectively enhances the model's generalization performance. Empirically, we conduct extensive experiments on multiple benchmarks and the results suggest that SMILE significantly outperforms other baselines, including both in-domain and cross-domain few-shot settings.

%
% ---- Bibliography ----
%
% BibTeX users should specify bibliography style 'splncs04'.
% References will then be sorted and formatted in the correct style.
%
\bibliographystyle{splncs04}
\bibliography{ref}

\end{document}
