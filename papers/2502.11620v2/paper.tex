% This is samplepaper.tex, a sample chapter demonstrating the
% LLNCS macro package for Springer Computer Science proceedings;
% Version 2.21 of 2022/01/12
%
\documentclass[runningheads]{llncs}
%
\usepackage[T1]{fontenc}
% T1 fonts will be used to generate the final print and online PDFs,
% so please use T1 fonts in your manuscript whenever possible.
% Other font encondings may result in incorrect characters.
%
\usepackage{graphicx}
% Used for displaying a sample figure. If possible, figure files should
% be included in EPS format.
%
% If you use the hyperref package, please uncomment the following two lines
% to display URLs in blue roman font according to Springer's eBook style:
%\usepackage{color}
%\renewcommand\UrlFont{\color{blue}\rmfamily}
%\urlstyle{rm}
%

\usepackage{csquotes}
\usepackage{tabularx}
\usepackage{listings}
\usepackage[binary-units]{siunitx}
\usepackage{xspace}
\usepackage{microtype}
\usepackage{xspace}
\usepackage{flushend}
\usepackage{paralist}
\usepackage{enumitem}
\usepackage{siunitx}
\usepackage{makecell}
\usepackage{pgf}
\usepackage{soul}
\usepackage{amsmath}
\usepackage{subcaption}
\usepackage{algorithm}
\usepackage{algpseudocode}
\usepackage{xcolor}
\usepackage{hyperref}
\usepackage{pgf}
\usepackage{array}
\usepackage{pifont}
\usepackage{multirow}
\usepackage{footnote}
\usepackage{graphicx}
\usepackage{subcaption}
\usepackage{booktabs}
\usepackage{varwidth}

\usepackage{tikz} 
\usetikzlibrary{shapes.geometric, positioning} 

\usepackage[most]{tcolorbox}   
\usepackage{xcolor}            
\usepackage{lipsum}            
\usepackage{microtype}         

\definecolor{darkblue}{rgb}{0.0, 0.0, 0.55} % RGB values for dark blue
\definecolor{darkgreen}{rgb}{0.0, 0.55, 0.0} % RGB values for dark blue

\lstset{
    language=Python,
    basicstyle=\ttfamily\scriptsize,  % Smaller font to help prevent overflow
    columns=fullflexible,
    showstringspaces=false,
    breaklines=true,          % Allow line wrapping
    breakatwhitespace=false,   % Prefer to break at whitespace
    keywordstyle=\color{darkblue}\bfseries, % Keywords in blue and bold
    stringstyle=\color{darkgreen},         % Strings in green
    commentstyle=\color{gray},         % Comments in gray
    numbers=left,                      % Line numbers on the left
    numberstyle=\tiny\color{gray},     % Line numbers in tiny gray font
    rulecolor=\color{black},           % Color of the frame border
}
\sisetup{
    output-decimal-marker = .,
    group-minimum-digits=6, % Set minimum number of digits for a group
    group-separator = {,}, % Set comma as the group separator
    table-number-alignment = right
}

\newif\ifcomments

\commentstrue

\ifcomments
\providecommand{\AS}[1]{\textbf{\textcolor{blue}{AS: #1}}}
\providecommand{\CD}[1]{\textbf{\textcolor{red}{CD: #1}}}
\else
\providecommand{\CC}[1]{}
\providecommand{\AS}[1]{}
\fi

\newcommand{\thought}[1]{{\color[rgb]{0.2,0.39,0.66}(#1)}}
\newcommand{\todo}[1]{{\color[rgb]{1.0,0.0,0.0}(#1)}}
\newcommand{\hsh}[1]{{\color{green!50!black} Henrik: #1}}
\newcommand{\st}[1]{{\color{red!50!black} Sebastian: #1}}

\newcommand{\ulm}[1]{_{\scaleto{\mathrm{#1}}{3pt}}}
\newcommand\at[2]{\left.#1\right|_{#2}}











\newtheorem{assumption}{Assumption}

\DeclareMathOperator*{\argmax}{arg\,max}
\DeclareMathOperator*{\argmin}{arg\,min}

\newcommand{\swname}[1]{\texttt{#1}}
\newcommand{\ie}{i\/.\/e\/.,\/~}
\newcommand{\eg}{e\/.\/g\/.,\/~}
\newcommand{\cf}{cf\/.\/~}

\newcommand{\fig}{Fig\/.\/~}
\newcommand{\defn}{Def\/.\/~}
\newcommand{\sect}{Sec\/.\/~}
\newcommand{\tabl}{Tab\/.\/~}
\newcommand{\algo}{Algorithm~}
\newcommand{\theo}{Theorem~}

\newcommand{\bnnl}{3 hidden layers}
\newcommand{\bnnn}{50 neurons}
\newcommand{\bnna}{tanh activations}

\newcommand{\capt}[1]{\mdseries{\emph{#1}}}

\newcommand{\videolink}{at \url{https://youtu.be/_d7AqTRjz6g}}
\newcommand{\codelink}{\url{https://github.com/wheelbot/mini-wheelbot}}

\newcommand{\fakepar}[1]{\vspace{0mm}\noindent\textbf{#1.}}

\newcommand{\needref}{\textcolor{red}{[REF]}}

\newcommand{\plotfontsize}{9pt}


% cleveref wants to be loaded very late
\usepackage[capitalise,noabbrev]{cleveref}
\crefname{line}{line}{lines}

\sisetup{detect-all} % always use the surrounding font-family, face, etc
\sisetup{separate-uncertainty = true} % print 1.234±0.005 instead of 1.234(5)

\begin{document}
%
% \title{Neuro-Symbolic Uncertainty Measures for Code Generation: A Semantic and Symbolic Approach} 
%\title{Information-Theoretic Measures for Assessing Code Generation Quality} 
\title{Assessing Correctness in LLM-Based Code Generation via Uncertainty Estimation}
%
\titlerunning{Assessing Correctness in LLM-Based Code Generation}
% If the paper title is too long for the running head, you can set
% an abbreviated paper title here
%
% \author{Arindam Sharma~\email{arindam.sharma@bristol.ac.uk} \and Cristina David~\email{cristina.david@bristol.ac.uk}} 
% % \author{Anonymous} 
% %
% \authorrunning{A. Sharma et al.}
% % \authorrunning{Anon et al.}
% % First names are abbreviated in the running head.
% % If there are more than two authors, 'et al.' is used.
% %
% \institute{University of Bristol, Bristol, UK}
\author{Arindam Sharma \and Cristina David}
\authorrunning{A. Sharma et al.}
\institute{%
  University of Bristol \\
  \email{\{arindam.sharma\}\{cristina.david\}@bristol.ac.uk}\\[0.5ex]
}
% \institute{Anonymous}
%
\maketitle              % typeset the header of the contribution
%
\keywords{Code generation  \and Correctness \and Large Language Models  \and Entropy.}

\begin{abstract}
In this work, we explore uncertainty estimation as a proxy for correctness in LLM-generated code. 
To this end, we adapt two state-of-the-art techniques from natural language generation---one based on entropy and another on mutual information---to the domain of code generation. 
Given the distinct semantic properties of code, we introduce modifications, including a semantic equivalence check based on symbolic execution. 
Our findings indicate a strong correlation between the uncertainty computed through these techniques and correctness, highlighting the potential of uncertainty estimation for quality assessment. 
Additionally, we propose a simplified version of the entropy-based method that assumes a uniform distribution over the LLM's responses, demonstrating comparable effectiveness. 
Using these techniques, we develop an abstention policy that prevents the model from making predictions when uncertainty is high, reducing incorrect outputs to near zero. 
Our evaluation on the \livecodebench dataset~\cite{livecodebench} shows that our approach significantly outperforms a baseline relying solely on LLM-reported log-probabilities. 
\end{abstract}

\section{Introduction}
\label{sec:intro}
\documentclass[../main.tex]{subfiles}
\graphicspath{{../images/}}
\makeatletter
\def\input@path{{../images/}}
\makeatother
\begin{document}
\section{Introduction}
\begin{figure}
\centering
\begin{tikzpicture}
\node[inner sep=0pt] (ws) at (0, 0) {
\includegraphics[height=.4\textwidth, trim={10cm 0 10cm 0},clip]{world_space.png}};
\node[inner sep=0pt] (cs) at (6,0) {\includegraphics[height=.4\textwidth, trim={10cm 1cm 10cm 4cm},clip]{conf_space.png}};
\end{tikzpicture}
\vspace{-5pt}
\label{fig:pbrm_intro}
\caption{\textbf{Left}: Shows world space obstacles as grey spheres. Robots start and goal configuration is colored red and green, respectively. Configurations along the computed path are colored transparent blue. \textbf{Right:} Mapped world space scenario to configuration space. Obstacle region is the grey mesh. Red spheres are collision-free regions computed by the neural SCDF. The optimized shortest path in the convex corridor is the blue curve.}
\vspace{-25pt}
\end{figure}
Motion planning is the problem of finding a collision-free trajectory that connects a given start and goal configuration. The planning takes place in the configuration space of the robot. For single body robots, like mobile robots or drones, the configuration space and the world space are usually the same. This simplifies the planning, since explicit obstacle representations are available which enables geometrical tools like separating hyperplanes, smallest distance to obstacles etc., to be used when designing motion planning algorithms. For multi-body robots like manipulators, the situation is completely different. The world space obstacles are usually mapped to non-convex regions, and to make the problem even harder, the mapping is usually not known. Forming explicit representations of the obstacle region in the configuration space is usually too expensive or intractable. Despite all of this, sampling based planners are used with great success, which mainly is due to their use of implicit representations of the obstacle region. The basic idea is to construct a graph in the configuration space that covers and connects the collision-free region. From this graph, a path can be extracted that connects a given start and goal configuration. The approach is computationally expensive, since the graph is constructed with the smallest geometrical building block available, points, which represents a collision-check. Furthermore, the extracted paths from the graph are non-smooth and jagged due to the stochastic nature of the approach. This adds an additional post-processing step to the process, where the paths are shortcutted and smoothened, before the path can be used for tracking. Clearly a lot of time is invested to form this graph and produce smooth paths. Thus, if the obstacles start to move, then all of this work is done in no use, since all points that make up this graph need to be re-verified, which is simply too time consuming to be done in real time.
\\\\
In this work, we want to address the existing drawbacks of the sampling based planners. Our main contribution is an improved motion planner where each vertex in the graph covers a collision-free region in the form of a sphere instead of a point and where the edges are formed with neighboring intersecting spheres. This representation has the advantage of instead of returning piecewise linear paths, returning a sequence of overlapping spheres, i.e. a convex corridor, that connects a given start and goal configuration, illustrated in Figure \ref{fig:pbrm_intro}. This convex corridor allows us to use convex optimization to produce smooth trajectories, instead of computationally expensive post-processing methods. The representation further allows us to estimate the coverage of the collision-free space, which gives us awareness and feedback in the offline roadmap construction phase. Finally, our representation is simple to adapt to moving obstacles, simply requery for the new radii and recheck for intersections. 
\\\\
The spherical collision-free regions are formed using a signed distance function (SDF), which is a function that returns the smallest distance from an arbitrary point to the boundary of an obstacle. As the name implies, the distance is signed, thus if the point is inside the obstacle it is negative otherwise positive. If the distance is positive, a sphere with radius equal to the distance is guaranteed to cover a collision-free region. Using an SDF in motion planning is not new, but what is novel about our approach is that we express the distance in the configuration space instead of the world space and by doing so allows us to form these convex collision-free regions. We refer to the resulting SDF as a signed configuration distance function (SCDF). Computing an SCDF analytically is non-trivial, our approach is therefore to parameterize the SCDF with a deep neural network and learn the mapping by supervised learning. Our resulting neural SCDF can compute distances for different parameter values of obstacle shapes and we also show how multiple distances can be combined, thus making our approach flexible.
\section{Related work}
Motion planning algorithms can roughly be divided into three families, grid-based, sampling based and optimization based methods. Grid-based methods (GBM) discretize the planning space from which a graph is then compiled. A standard search method is A$^\star$ \citep{a_star}, which is classified as an \textit{informed} search method, since it employs a heuristic function to speed up the search. A$^\star$ guarantees to return an optimal path at the level of discretization used. GBMs usually discretize the planning space by a regular lattice and this limits the GBMs to problems with low dimensionality due to the curse of dimensionality. Thus, GBMs are usually limited to single-body robots where the degrees of freedom (DOF) are low. To overcome the inherent scaling problem with the GBMs, stochastic methods are usually used for multi-body robots. These methods are termed as sampling-based methods (SBM) and core members within this family are the rapidly-exploring random trees (RRT) \citep{rrt} and the probabilistic roadmap (PRM) \citep{prm}. RRT grows a tree from the start configuration and explores the collision-free region in a rapid way until it is able to connect to the goal region. RRT is usually improved by bi-directional planning \citep{rrt_connect}, i.e. an additional tree is grown from the goal configuration and the trees are tested for connection after any tree has been expanded. RRT is a single-query method, thus it searches for a path from scratch each time it is queried. Contrary to this, PRM is a multi-query method, which solves for multiple queries without starting from scratch. PRM does this by creating a roadmap (graph) that covers the collision-free space as an offline step. The graph is then used to solve for multiple queries. PRMs are used in cases where the environment does not change since the extra offline step is too computationally costly and needs to be re-done if the environment is changed. In our work, we address this inherent issue by using a different roadmap representation. Our vertices in the graph cover a collision-free region in the form of spheres and we form the edges by checking for intersecting spheres. If something in the environment changes, we recompute the spheres radii and recheck the intersections, without relying on collision detection. We use a trained neural network to compute the sphere radius, therefore querying for the radius can be done fast, hence our representation enables the PRM for dynamic environments.
\\\\
In the recent decades, optimization based methods (OBM) \citep{chomp, schulman, itomp, stomp} have been introduced as an alternative to SBM for multi-body robots. Like the SBM, the OBMs scale well to higher dimensional problems and produce smoother motion. It is common to use a SDF in the optimization since it is a smooth function, thus enabling gradient-based methods. However, the standard way of expressing the SDF is in world space. The distance therefore needs to be mapped to the configuration space by the forward kinematics. This mapping makes the optimization problem a non-linear program (NLP), which is computationally expensive to solve. Recently, a different approach has been proposed. In \cite{mp_gcs} motion planning is formulated as a convex optimization problem by using the graph of convex sets framework \citep{gcs}. The underlying idea is to decompose the collision-free space into intersecting convex sets from which a convex optimization problem is formulated. In cases where an explicit representation of the obstacles in the configuration space exists, like for single-body robots, creating collision-free convex regions can be done fast \citep{iris}. For multi-body robots, this is non-trivial. Existing work does this successfully \citep{iris_nlp, iris_c} by an optimization based approach, but the methods are still too time consuming to be used in the presence of moving obstacles. Our approach is instead to use deep learning to learn an SDF expressed in the configuration space. With this, we can query for shortest distances to the collision boundary, which allows us to expand spherical regions which are collision-free. Our approach is fast and therefore enables our suggested roadmap planner to be used in dynamic environments.
\\\\
Recent research has focused on learning collision detection \citep{fk_kernel_distance, diffco, graphdistnet} by predicting the signed distance between the robot links and the surrounding obstacles in the world space. The learned SDF is used in trajectory optimization but since the distance is expressed in the world space, the problem becomes an NLP and therefore takes a long time to solve. We take a novel approach and suggest to instead express the signed distance in the configuration space. This allows us to improve the PRM at the same time as it enables convex optimization for trajectory optimization, which runs faster and is more reliable than NLP solvers. In \cite{cspf} a learned signed distance function in the configuration space is proposed similar to our approach. However, their approach is restricted to point cloud representations, while we propose to represent the obstacles as parameterized geometric shapes, e.g. spheres. Furthermore, we also show how to use our learned SCDF to improve an existing roadmap planner.
\section{Problem formulation}
A robot is located in the world space, $\W \subset \R^3 $. The unique location of the robot is given by its configuration $\q \in \C$, where $\C$ is the configuration space. The set of points covered by the robots bodies at a certain configuration is expressed as $\B(\q) \subset \W$. The robot is surrounded by $\NrObst$ obstacles $\O = \bigcup_{i=1}^{\NrObst} \O_i$, where  $\O_i \subset \W$. The representation of the obstacle in the configuration space is the set $\C\O_i = \{\q \in \C \: |\: \B(\q) \cap \O_i \neq \emptyset \}$. The obstacle space is formed as $\Co = \bigcup_{i=1}^{\NrObst} \C \O_i$. The complement is referred to as the free space, $\Cf = \C \setminus \Co$. The path planning problem is a tuple, ($\Cf$, $\qStart$, $\qGoal$), where we want to connect a query pair, consisting of a start, $\qStart$, and goal configuration, $\qGoal$, with a geometric path, $\q(s): [0, 1] \mapsto \Cf$, such that $\q(0)=\qStart$ and $\q(1)=\qGoal$, or report correctly when such a path does not exist.
\end{document}


\section{Motivating Example}
\label{sec:motivating}
To illustrate the relationship between uncertainty and correctness for code generation, we refer to the problem described by the prompt in Figure~\ref{fig:sampleproblem}. 

\begin{figure}[htbp]
    \centering
    \begin{tcolorbox}[
        enhanced,
        width=0.85\textwidth,
        colback=white,
        colframe=blue!50!black,
        boxrule=1pt,
        arc=5pt,              
        title=LLM Prompt,
        fonttitle=\bfseries,
        attach boxed title to top center={yshift=-2mm},
        varwidth boxed title=0.7\linewidth
      ]
    \textbf{User Prompt:}

    \vspace{0.5em}
    You are given a \texttt{0-indexed} array of strings \texttt{details}. Each element of \texttt{details} provides information about a given passenger compressed into a string of length 15. The format is as follows:
    \begin{itemize}
        \item The first 10 characters: the passenger's phone number
        \item The 11th character: the passenger's gender
        \item The 12th and 13th characters: the passenger's age
        \item The 14th and 15th characters: the seat allotted
    \end{itemize}
    Return the number of passengers whose age is \textbf{strictly greater} than 60.

    % \vspace{0.5em}
    % \textbf{Example:}
    % \begin{verbatim}
    % Input: details = ["7869194042M58A", "0741234567F75B", "6280984567F13C"]
    % Output: 1
    % Explanation: Only one passenger is older than 60.
    % \end{verbatim}

    \vspace{0.5em}
    \textbf{Task:} 
    Please provide a function in your preferred programming language that takes \texttt{details} and returns how many passengers are older than 60.
    \end{tcolorbox}
    \caption{An example LLM prompt for the ``number-of-senior-citizens'' coding problem.}
    \label{fig:sampleproblem}
\end{figure}

\begin{figure}[ht!]
    \centering
    
    \begin{subfigure}[t]{0.3\textwidth}
        
        \lstinputlisting{code/good_llm_snippet1.py}
        \vspace{2.5em} 
        \caption{Snippet 1}
        \label{lst:good1}
    \end{subfigure}
    \hfill
    \begin{subfigure}[t]{0.3\textwidth}
        
        \lstinputlisting{code/good_llm_snippet2.py}
        \vspace{3.4em}
        \caption{Snippet 2}
        \label{lst:good2}
    \end{subfigure}
    \hfill
    \begin{subfigure}[t]{0.3\textwidth}
        
        \lstinputlisting{code/good_llm_snippet3.py}
        \vspace{2.5em}
        \caption{Snippet 3}
        \label{lst:good3}
    \end{subfigure}

    \caption{Three code snippets from the \gptturbo, all correct, semantically equivalent.}
    \label{fig:good-llm-snippets}
\end{figure}

%Figure~\ref{fig:good-llm-snippets} shows the top 3 responses from the well-tuned model \ie \gptturbo. Closer observation shows that the three responses shown in Listing~\ref{lst:good1}, Listing~\ref{lst:good2} and Listing~\ref{lst:good3} of Figure~\ref{fig:good-llm-snippets} are semantically equivalent thereby indicating that the model has high confidence in its response. It also turns out to be the case that these responses all pass the testsuite for this problem. 

\begin{figure}[ht!]
    \centering
    
    \begin{subfigure}[t]{0.3\textwidth}
       
        \vspace{0.5em} 
        \lstinputlisting{code/bad_llm_snippet1.py}
        \caption{Snippet 1}
        \label{lst:bad1}
    \end{subfigure}
    \hfill
    \begin{subfigure}[t]{0.3\textwidth}
        
        \vspace{0.5em}
        \lstinputlisting{code/bad_llm_snippet2.py}
        \caption{Snippet 2}
        \label{lst:bad2}
    \end{subfigure}
    \hfill
    \begin{subfigure}[t]{0.3\textwidth}
        
        \vspace{0.5em}
        \lstinputlisting{code/bad_llm_snippet3.py}
        \caption{Snippet 3}
        \label{lst:bad3}
    \end{subfigure}

    \caption{Three code snippets from the \salesforce/\codegenmonoC, all incorrect, and semantically distinct.}
    \label{fig:bad-llm-snippets}
\end{figure}

%However, for a contemporary, worse-performing model from \salesforce (\codegenmonoC), the 3 responses shown in Listing~\ref{lst:bad1}, Listing~\ref{lst:bad2} and Listing~\ref{lst:bad3} of Figure~\ref{fig:bad-llm-snippets}, are all semantically different and hence fall in their own respective clusters. Unsurprisingly, all 3 of these responses are incorrect and do not pass any testcases of the testsuite. 

Figure~\ref{fig:good-llm-snippets} displays the top three responses generated by our first model \gptturbo, all of which are semantically equivalent and correct. 
In contrast, Figure~\ref{fig:bad-llm-snippets} presents the first three responses generated by \salesforce (\codegenmonoC), all of which are incorrect.

The log-probabilities reported by the LLMs are as follows: for the better performing \gptturbo, the three snippets have log-probabilities of \GPTsnipLogProbA, \GPTsnipLogProbB, and \GPTsnipLogProbC, respectively. 
For \codegenmonoC, the log-probabilities are \SFsnipLogProbA, \SFsnipLogProbB, and \SFsnipLogProbC. 
However, these log-probabilities alone are insufficient as proxies for correctness. 
In particular, there is no clear way to infer from the log-probabilities that all the snippets generated by \gptturbo are correct, while all those generated by \codegenmonoC are incorrect.

%At first glance, the differences in responses might suggest that both LLMs exhibit low confidence in their responses. However, a closer examination of the snippets in Listings~\ref{lst:good1}, \ref{lst:good2}, and \ref{lst:good3} in Figure~\ref{fig:good-llm-snippets} reveals that, although syntactically distinct, they are semantically equivalent. In contrast, the responses in Figure~\ref{fig:bad-llm-snippets} are both syntactically and semantically distinct.

The techniques outlined in Section~\ref{sec:symex} and Section~\ref{sec:mi} effectively identify the semantic equivalence of the snippets in Figure~\ref{fig:good-llm-snippets} and incorporate this information during the computation of entropy and mutual information, respectively. 
As shown later in the paper, both methods ultimately conclude that \gptturbo exhibits high confidence in its responses, while the model from \salesforce demonstrates significant uncertainty. 
Furthermore, we will show that uncertainty negatively correlates with correctness.
%When using uncertainty as a proxy for correctness, the results suggest that \gptturbo's responses are likely accurate, whereas \salesforce's are likely flawed. This conclusion is corroborated by the test suite results: all three responses from \gptturbo pass successfully, whereas none of \salesforce's responses do.


%\section{Background}
%\label{sec:background}
%\section{Basic Background: Supervised Learning and the PAC Model}
\label{sec:background}

At this point almost everyone has heard of machine learning (ML). Anyone likely to stumble upon this article will have also heard of its most influential special case, supervised learning, and those theoretically inclined will also be familiar with the PAC model. Nonetheless, I will set the stage by  recapping the basics.

\subsection{Basics of Supervised Learning}%Let's set the stage in any case

\emph{Supervised Learning} is the task of ``coming up'' with a function $f: \X \to \Y$ to ``explain'' or ``fit'' a sequence of input/output examples   $(x_1,y_1), \ldots, (x_n,y_n)$, with $x_i \in \X$ and $y_i \in \Y$.  Here $\X$ is a \emph{data domain} consisting of \emph{datapoints} $x \in \X$, $\Y$ is a \emph{label set} consisting of \emph{labels} $y \in \Y$, and the sequence $(x_1,y_1),\ldots,(x_n,y_n)$ is the \emph{training data} consisting of \emph{labeled examples (a.k.a. samples)}~$(x_i,y_i)$.  I~will refer to the chosen function $f$ as a \emph{predictor}, and to $n$ as the \emph{sample size}. A \emph{learning algorithm} takes as input training data, and outputs (some representation of) a predictor $f \in \Y^\X$.\footnote{Note that this describes the usual \emph{batch}, a.k.a.~\emph{offline}, setting of supervised learning. I do not discuss other paradigms such as online or active learning in this article.} 



Success in supervised learning is defined as \emph{generalization} to  future examples: For a typical \emph{test example}  $(x_{\tst},y_{\tst})$, the predicted label $y'_{\tst}=f(x_{\tst})$ should ``equal'' $y_{\tst}$, perhaps approximately. We usually assume the test example is drawn from the same  ``source'' as the training data  --- commonly, i.i.d.~from the same distribution. The quality of the prediction is quantified by $\ell(y'_{\tst},y_{\tst})$, where $\ell:~\Y~\times~\Y \to \RR_{\geq 0}$ is a \emph{loss function} chosen as part of the problem definition. Common loss functions include the 0-1 loss $\ell_{0-1}(y',y) = [y' \neq y]$ for \emph{classification} problems,\footnote{The notation $[P]$ denotes $1$ when predicate $P$ is true, and denotes $0$ when $P$ is false.} as well as the absolute loss $|y'-y|$ or squared loss $(y'-y)^2$ for \emph{regression problems} featuring $\Y  \sse \RR$.

Nontrivial generalization properties are typically only possible if one assumes something about the data.\footnote{The need for such an assumption is formalized by the  \emph{no free lunch theorems} of supervised learning \cite{wolpert_connection_1992,wolpert_lack_1996,schaffer_conservation_1994}.} The Bayesian approach to  machine learning, common in many applications, assumes some parametric form for the distribution generating the data, and postulates a prior on the parameters. This is not the approach I will take in this article. Instead, I will focus on the frequentist --- and some would say ``worst-case'' or ``adversarial'' ---  approach that is common in the computational learning theory community, embodied by the PAC model. Here we assume that the (training and test) data can be explained, perhaps approximately, by a function in some ``simple enough to learn'' class of functions $\H \sse \Y^\X$, often called the \emph{hypotheses}. Equivalently, we  seek a predictor which explains the unseen data roughly  as well as the best hypothesis $h^* \in \H$, whether or not we assume that $h^*$ itself provides a perfect explanation.



 \paragraph{Common Algorithmic Templates.} Perhaps the best known general-purpose supervised learning algorithm is \emph{empirical risk minimization (ERM)}, which chooses as its predictor a hypothesis $f \in \H$ minimizing $\frac{1}{n} \sum_{i=1}^n \ell(f(x_i),y_i)$ --- a quantity called the \emph{training error}, \emph{empirical error}, or \emph{empirical risk} of $f$. %\footnote{When multiple hypotheses minimize the empirical risk, we assume ERM breaks ties arbitrarily.}
A common template for generalizing ERM involves adding a \emph{regularization term} $\psi(f)$ to the  objective function, typically chosen to measure some notion of ``hypothesis complexity.'' An algorithm instantiating this template is known as a \emph{structural risk minimizer (SRM)}, and chooses as its predictor the hypothesis $f \in \H$ minimizing the \emph{structural risk} $\frac{1}{n} \sum_{i=1}^n \ell(f(x_i),y_i) + \psi(f)$. Other well-known algorithms, such as gradient descent and its variations,  can frequently be interpreted as approximate implementations of ERM or SRM.


\paragraph{Proper vs Improper Learning.} A learning algorithm is said to be \emph{proper} if its predictor $f$ is always chosen from the hypothesis class, i.e., $f \in \H$, otherwise it is said to be \emph{improper}. ERM  is an example of a proper learning algorithm, as are SRM algorithms of the form described above.  In the \emph{proper regime} of learning, algorithms are required to be proper. This article will be concerned with the more flexible \emph{improper regime} (a.k.a \emph{representation-independent learning}), where no such constraint is placed on the learner. In other words, all we care about is predictive power at test time, rather than any insights derived from the functional form or representation of the predictor~itself.


\subsection{The PAC Model}
A standard mathematical setup for evaluation of supervised learning algorithms, at least in the theoretical computer science community, is Valiant's \emph{Probably Approximately Correct (PAC) model} of learning (see e.g.~\cite{kearns_introduction_1994,mohri_foundations_2018}). Here, we assume there is an unknown distribution $\D$ on $\X \times \Y$ from which training and test data are  drawn.  Specifically, the labeled datapoints of the training set  $(x_1,y_1), \ldots, (x_n,y_n)$, as well as the test data  $(x_\tst,y_\tst)$, are i.i.d.~from $\D$. Often it is assumed that $\D$ lies in some class of distributions of interest. The \emph{true expected loss}, or simply \emph{loss}, of a predictor $f: \X \to \Y$ is the expected loss it incurs on draws from $\D$, written $L_\D(f) = \Ex_{(x,y) \sim \D} \ell(f(x),y)$.


There are two main ``settings'' in PAC learning. The  \emph{realizable setting} only requires that the data be perfectly explained by some hypothesis in $\H$. More generally, the \emph{agnostic setting} makes no assumption relating the data to the hypotheses, but shifts the goalposts as necessary to allow nontrivial guarantees: the expected loss at test time is evaluated only ``relative'' to that of the best hypothesis $h^* \in \H$. There are other settings which make more nuanced assumptions, such as $\D$ being of a particular parametric form or its support living in some (unknown) lower-dimensional space, etc. I will mostly discuss the realizable and agnostic settings in this article, those being the simplest and most studied from a theoretical perspective. %TODO:We will briefly discuss other settings in Section ??

The PAC model demands high probability guarantees of learners, in the worst case over distributions of interest. Consider first the realizable setting, where $\D$ is such that $\min_{h \in \H} L_{\D}(h) = 0$. A PAC learner has \emph{error} $\epsilon=\epsilon(n)$ and \emph{confidence} $\delta=\delta(n)$ if, when training data consists of $n$ i.i.d~samples from a realizable distribution $\D$, it produces a predictor $f$  satisfying $L_\D(f) \leq \epsilon$ with probability at least $1-\delta$. In the agnostic setting, where $\D$ can be arbitrary, we require $L_\D(f) - \min_{h \in \H} L_\D(h) \leq \epsilon$ with probability $1-\delta$.

In both the realizable and agnostic settings, we look for PAC learners with small $\epsilon$ and $\delta$ as a function of the sample size $n$. An equivalent perspective looks at the sample complexity $m(\epsilon,\delta)$, which is the minimum sample size which guarantees error  at most $\epsilon$ with probability at least $1-\delta$. We say a problem is \emph{PAC learnable} if its PAC sample complexity is finite whenever $\epsilon,\delta > 0$.

For most PAC learning problems, learnability and sample complexity are characterized in terms of a  ``dimension'' of the hypothesis class. Most prominently this is the \emph{VC dimension} for binary classification, the \emph{fat shattering dimension} for agnostic regression, and the \emph{DS dimension} for multiclass classification (see \cite{anthony_neural_1999,daniely_optimal_2014,brukhim_characterization_2022}). Treatment of these is beyond the scope of this article. The unfamiliar reader need not worry, however,  as dimensions will feature only tangentially in our~discussion.




%\paragraph{Learning settings: Realizable, Agnostic, etc.} In learning theory, evaluating a supervised learning algorithm requires specifying a data model and an objective. We will leave the details of the data model flexible for now, to allow for both the PAC model and the adversarial transductive model. Nonetheless we will describe two variations, which we call ``settings'', which cut across different models. The  \emph{realizable setting}  requires only that the data be perfectly explained by some hypothesis $h \in \H$ --- i.e., there exists a hypothesis which is guaranteed to suffer a loss of $0$ on training and test data. The performance of the learning algorithm is its expected loss at test time for some ``worst case'' realizable instance. More generally, the \emph{agnostic setting} makes no assumption relating the data to the hypotheses, but shifts the goalposts as necessary to allow nontrivial guarantees: the expected loss at test time is evaluated only ``relative'' to that of the best hypothesis $h^* \in \H$, again for some ``worst case'' instance. There are other settings which make more nuanced assumptions about the data, such as it is drawn from a distribution of a particular parametric form, or that it lives in some (unknown) lower-dimensional space, etc. We will mostly discuss the realizable and agnostic settings, those being the simplest and most studied from a theoretical perspective.




%%% Local Variables:
%%% mode: latex
%%% TeX-master: "learning_matching"
%%% End:


%\section{Our Techniques for Assessing Uncertainty in Code Generation}
%\label{sec:tech}
\section{Techniques}
\label{sec:techniques}

Let us recall the assumptions that are used for nearly optimal computationally efficient robust covariance estimation in the Gaussian case (see section 5.2 of \cite{DiakonikolasKK016} or chapter 4 of \cite{DK_book} for the description of the algorithm). First, we need very strong concentration assumptions. In particular, $O(1)$-sub-Gaussianity is not enough, while $O(1)$-Hanson-Wright property is enough for the analysis to work (with $\tilde{O}(d^2/\e^2)$ samples). 
Second, we need the fourth moment of $\Sigma^{-1/2}x$ to coincide with the fourth moment of the standard Gaussian distribution. 

For elliptical distributions we do not have any assumptions on the moments (covariance or even mean might not even exist), and we have to rely only the elliptical structure. Fortunately, this structure allows us to reduce the problem of scatter matrix estimation to a problem of robust covariance estimation of some specific $O(1)$-sub-Gaussian distribution. The properties of this distribution depend on the spectrum of $\Sigma$, and, as we show, the closer $\Sigma$ is to $\Id$, the better (for the robust covariance estimation) these properties are. This allows us to estimate the covariance in several steps that we describe below. 

\paragraph{Spatial Sign.} The distribution reverenced above is the \emph{spatial sign} of an elliptical distribution. It is a projection of an elliptical vector with location zero\footnote{We can always reduce the problem to this case by considering $\paren{x_{i}-x_{\lfloor n/2\rfloor+i}}/\sqrt{2}$. The resulting samples also come from (an $O(\e)$-corruption of) an elliptical distribution with the same scatter matrix.} onto some (arbitrary) sphere centered at $0$. In this paper we use the sphere of radius $\sqrt{d}$, and denote the projection of vector $x\in\R^d$ onto this sphere by $\spsign(x)$. 

If $x$ is an elliptical vector with location $0$ and scatter matrix $\Sigma$, $\spsign(x)$ depends \emph{only} on $\Sigma$, and is the same for \emph{all} elliptical distributions with the same scatter matrix. Indeed, since  $x = \xi A U$ (as in Definition \ref{def:elliptical}), the projection onto the sphere satisfies
\[
 \spsign(x) = \frac{x}{\tfrac{1}{\sqrt{d}}\norm{x}} = \frac{\xi A U}{\tfrac{1}{\sqrt{d}}\norm{\xi A U}} = \frac{A U}{\tfrac{1}{\sqrt{d}}\norm{A U}}\,.
\]
So when we study the properties of $\spsign(x)$, we can simply assume that $x$ comes from our favorite elliptical distribution: $\cN(0,\Sigma)$.

The spatial sign was extensively studied in prior works on elliptical distributions, because $\Sigma' := \Cov_{x\sim \cN(0,\Sigma)} \spsign(x)$ has the same eigenvectors as $\Sigma$, and hence $\Sigma'$ is very useful for the principal (elliptical) component analysis.
However, the eigenvalues of $\Sigma'$ differ from the eigenvalues of $\Sigma$, so even in the classical setting (without corruptions), if we use the empirical covariance of the spatial sign to estimate the eigenvalues of $\Sigma$ (or $\Sigma$ itself), the error of the estimator is not vanishing, even if we take infinitely many samples. 
Fortunately, in robust statistics we anyway have a term that does not depend on the number of samples (it should be at least $\Omega(\e)$ in the Gaussian case). So if we prove that $\Sigma'$ is $\tilde{O}(\e)$-close to $\Sigma$ (in relative Frobenius or at least relative spectral norm), then good robust estimators of $\Sigma'$ are also good robust estimators of $\Sigma$. Since $\Tr(\Sigma') = d$, we fix the scale of $\Sigma$ so that $\Tr(\Sigma) = d$.

Let us discuss how to bound $\norm{{\Sigma}^{-1/2}\, \Sigma'\, {\Sigma}^{-1/2} - \Id}$ (then we also obtain the bound on relative Frobenius norm by multiplying the spectral norm bound by $\sqrt{d}$). Since the eigenvectors of $\Sigma$ and $\Sigma'$ are the same, we can work in the basis where both matrices are diagonal. It follows that
\[
\norm{{\Sigma}^{-1/2}\, \Sigma'\, {\Sigma}^{-1/2} - \Id} = \max_{i\in[d]}\Abs{\frac{\lambda'_i}{\lambda_{i}} - 1} = \max_{i\in[d]}\Abs{\E_{x\sim \cN(0,\Sigma)}{\frac{x_{j}^2/\lambda_j}{\tfrac{1}{d}\norm{x}^2} - 1}}
\]
where $\lambda_i = \Sigma_{ii}$, and $\lambda'_i = \Sigma'_{ii}$.
The Hanson-Wright inequality implies that if $\effrank(\Sigma) \gtrsim \log(d)$, then with high probability for all of the $n = \poly(d)$ samples $x_1,\ldots, x_n\simiid \cN(0,\Sigma)$, 
\[
\norm{x_i}^2 = d\cdot \Paren{1 \pm O\Paren{\sqrt{\frac{\log d}{\effrank(\Sigma)}}}}\,.
\]

Assuming that it holds with probability $1$\footnote{In the formal argument, we analyze not exactly the spatial sign, but some function that with high probability coincides with it on all of the samples, and for this function this statement is true.}, we get a bound
$\norm{{\Sigma}^{-1/2}\, \Sigma'\, {\Sigma}^{-1/2} - \Id} \le O\Paren{\sqrt{\frac{\log d}{\effrank(\Sigma)}}}$. While this bound can be useful for (non-robust) eigenvector estimation\footnote{It was used, in particular, in \cite{ECA}, and it is enough for consistent estimation of the eigenvectors of $\Sigma$.}, for our purposes it is too bad: Even if the effective rank is as large as possible ($\effrank(\Sigma) \ge \Omega(d)$), we only get error $O(\log(d))$ in relative Frobenius norm.

To get a better bound, we get rid of the denominator by expanding it as a series: 
\[
\E\frac{x_{j}^2/\lambda_j}{\tfrac{1}{d}\norm{x}^2} = 
\sum_{k=0}^{\infty} \E{\frac{x_j^2}{\lambda_j} \Paren{1 - \tfrac{1}{d}\norm{x}^2}^k}
=\sum_{k=0}^{\infty} \E{g_j^2 \Paren{1 - \tfrac{1}{d}\sum_{i=1}^{d}\lambda_i g_i^2}^k}
\,,
\]
where $g = \Sigma^{-1/2}x \sim \cN(0,\Id)$. The term that corresponds to $k=0$ is $1$. The term that corresponds to $k=1$ is
\[
\E g_j^2\paren{1 - \tfrac{1}{d}\sum_{i=1}^{d}\lambda_i g_i^2} = 
1 - \tfrac{1}{d}\sum_{i\neq j} \lambda_i - \frac{3\lambda_j}{d} =
1 - \tfrac{1}{d}\sum_{i=1}^d \lambda_i - \frac{2\lambda_j}{d}
= \frac{2\lambda_j}{d} \le O\Paren{\frac{1}{\effrank(\Sigma)}}\,,
\]
where we used $\sum_{i=1}^d \lambda_i = \Tr(\Sigma) = d$. Similarly, the term that corresponds to $k = 2$ is also $O\Paren{\frac{1}{\effrank(\Sigma)}}$, and the other terms are much smaller. Hence we get a bound
\[
\norm{{\Sigma}^{-1/2}\, \Sigma'\, {\Sigma}^{-1/2} - \Id} \le O\Paren{\frac{1}{\effrank(\Sigma)}}\,.
\]
In the case $\effrank(\Sigma)\ge\Omega(d)$, we get $O(1/d)$ error in relative spectral norm, and $O(1/\sqrt{d})$ error in relative Frobenius norm. This is the reason why we require the lower bounds on $\e$ in \cref{thm:main}: We want to make these errors smaller than the robust estimation error $\tilde{O}(\e)$. However, if the effective rank is small, this error is still too large: For example, the error bound in relative Frobenius norm is $\Omega(1)$ if $\effrank(\Sigma) \le \sqrt{d}$. 

Furthermore, if $\effrank(\Sigma) \le o(d)$, the $\paren{d^2\times d^2}$-dimensional covariance of the $d^2$-dimensional random variable $\Paren{\Sigma'}^{-1/2}\spsign(x)\spsign(x)^\top\Paren{\Sigma'}^{-1/2}$ might not even be bounded by $O(1)$ (in spectral norm), which makes robust estimation of $\Sigma'$ with dimension-independent error in  relative Frobenius norm challenging, even if we did not aim to achieve nearly optimal error.

In order to fix these issues, first we estimate $\Sigma'$ up to some small constant error in relative spectral norm. For this, we split the sample into several (more precisely, 3) sub-samples, and for each new estimation we use fresh samples.

\paragraph{First Estimation.} It is not difficult to see that as long as $\effrank(\Sigma)\gtrsim \log(d)$, then $\spsign(x)$ is $O(1)$-sub-Gaussian. By recent result \cite{sos-subgaussian}, the bound on its fourth moment can be certified via a degree-$4$ sum-of-squares proof\footnote{This fact can be also easily verified directly without referring to \cite{sos-subgaussian}.}. Hence we can use the sum-of-squares algorithm from \cite{KS17} that estimates $\Sigma'$ in relative spectral norm up to error $O(\sqrt{\e})$ (this algorithm works if we use $n\gtrsim d^2\log^2(d)/\e^2$ samples). Since $\Sigma'$ is close to $\Sigma$, with high probability we get an estimator $\hat{\Sigma}_1$ such that 
\[
\norm{\hat{\Sigma}_1^{-1/2} \Sigma \hat{\Sigma}_1^{-1/2} - \Id}  \le O\paren{\sqrt{{\e}} + \frac{1}{\effrank(\Sigma)}}\,.
\]
In particular, $0.9 \cdot \Id \preceq \hat{\Sigma}_1^{-1}\Sigma \preceq 1.1\cdot \Id$. 
Hence if we multiply the next subsample by $\hat{\Sigma}_1^{-1}$, we get ($O(\e)$-corruption of) samples from an elliptical distribution with new scatter matrix $\tilde{\Sigma} = \rho \hat{\Sigma}_1^{-1} \Sigma$, where $\rho = d/\Tr(\hat{\Sigma}_1^{-1}\Sigma)$. 

So we can assume that we work with (corrupted) samples from an elliptical  distribution $\cD$ with scatter matrix $\Sigma$ such that $0.9 \cdot \Id \preceq \Sigma \preceq 1.1\cdot \Id$. If we fix the scale $\Tr(\Sigma) = d$, then
$\Sigma' = \Cov_{x\sim \cD}\spsign(x)$ is $O(1/d)$-close to $\Sigma$ in relative spectral norm (since $\effrank(\tilde{\Sigma}) \ge \Omega(d)$).

Now we can try to estimate $\Sigma'$ up to error $\tilde{O}(\e)$. As was previously mentioned, the algorithm for the Gaussian distribution from \cite{DiakonikolasKK016} requires strong assumptions: Hanson-Wright concentration, and the Gaussian fourth moment. 
Fortunately, as long as $\effrank(\Sigma)\ge \Omega(d)$, $\spsign(x)$ satisfies the $O(1)$-Hanson-Wright property. Indeed, it is not hard to see that with overwhelming probability $\Paren{\Sigma'}^{-1/2}\spsign(x) = \Paren{\Sigma'}^{-1/2}\spsign(\Sigma^{1/2} g)$ (where $g\sim \cN(0,\Id)$) coincides with some $O(1)$-Lipschitz function of $g$ (concretely, a composition of linear transformations with with bounded spectral norm, and a function that projects onto the sphere only the points that are close to it, and is linear otherwise). It is known that\footnote{See, for example, \cite{log-sobolev-are-hanson-wright}.}  any $O(1)$-Lipschitz function of a standard Gaussian vector satisfies the $O(1)$-Hanson-Wright property. Therefore, we do not have to worry about the concentration, and can focus on dealing with the fourth moment.

Recall that the covariance filtering algorithm with nearly optimal error uses the $(d^2\times d^2)$-dimensional covariance of the distribution in the isotropic position: $T=\E\Paren{\Cov(y)^{-1/2} yy^\top\Cov(y)^{-1/2} - \Id_d}^{\otimes 2}$. If $y\sim \cN(0,\Sigma)$, then $Q = 2\Id_{d^2}$. However, in our case $y = \spsign(x)$, and the situation is more complicated. This covariance is not only different from $2\Id_{d^2}$, it is also unknown to us, since it depends on $\Sigma$.
We study this dependence and show that the entries $T_{ijij}$ and $T_{iijj}$ are $O\Paren{\Norm{\Sigma - \Id_d}/d + \tilde{O}(1/d^2)}$-close to the entries of  $S := \E\Paren{gg^\top / \norm{g}^2 -\Id_d}^{\otimes 2}$ (and the other entries are zero for both of them). While at the first glance $T$ and $S$ seem to be very close, it is in fact not true: their values (as quadratic forms on unit vectors in $\R^{d^2}$) can differ by $O(\Norm{\Sigma - \Id_d})$. Even if we again use the algorithm from \cite{KS17} and guarantee that $\norm{\Sigma - \Id_d}\le O(\sqrt{\e})$, this bound is only $O(\sqrt{\e})$, while we need it to be $\tilde{O}(\e)$. 
Hence we have to somehow estimate $\Sigma'$ up to error $\tilde{O}(\e)$ in \emph{spectral} norm, before estimating it in  Frobenius norm. 

\paragraph{Second Estimation.}
Let us first describe why the difference between the values of $T$ and $S$ on unit $V\in\R^{d^2}$ can be $O(\Norm{\Sigma - \Id_d})$. The reason is the terms $V_{ii}V_{jj} \Paren{T_{iijj} - S_{iijj}}$ for $i\neq j$. Since $\sum_{i=1}^d V_{ii}$ can be as large as $\sqrt{d}$, the sum of $V_{ii}V_{jj} \Paren{T_{iijj} - S_{iijj}}$ can be as large as $d \cdot \normi{S-T} = O(\Norm{\Sigma - \Id_d})$. 

However, if we only consider unit $d^2$-dimensional vectors of the form $V = uu^\top$, this issue disappears. Indeed, $\sum_{i=1}^d V_{ii}$ is now bounded by $1$, and $T$ is $O(1/d)$-close to $S$ on such vectors. Furthermore, they are both $O(1/d)$-close (as quadratic forms on unit $d^2$-dimensional vectors of the form $uu^\top$) to $2\Id_{d^2}$.

In fact, if we use the covariance filtering algorithm only for such vectors, we get the desired spectral norm bound. However, the optimization problem involved is computationally hard, since we need to optimize over the set $\Set{u^{\otimes 4} \suchthat \norm{u} = 1}$.
Fortunately, we can work with the (canonical) \emph{sum-of-squares relaxation} of this set, that is, with the set of degree-$4$ pseudo-expectations of $v^{\otimes 4}$ that satisfy the constraint $\norm{u}^2 =  1$. 

In order to show that this approach works, we introduce a generalized notion of stability that we use not only for estimation in spectral norm, but also in Frobenius form at the final step. 

\begin{definition}[Generalized Stability]\label{def:stability}
    Let $m\in \N$, $\cV \subseteq \R^m$, $\cP \subseteq \R^{m\times m}$ such that $\cV \otimes \cV \subseteq \cP$, $\mu\in \R^m$ and $Q\in \R^{m\times m}$. Let $\e, \delta, r > 0$ such that $\delta \ge \e$. 
    
    A finite multiset $M$ of points from $\R^m$ is \emph{$\Paren{\e,\delta,r,\cV,\cP}$-stable with respect to $\mu$ and $Q$} if for every $v\in \cV$, every $P\in \cP$, and every $M'\subseteq M$ with $\Card{M'}\ge \paren{1-\e}\Card{M}$, the following three conditions hold:
    \begin{enumerate}
        \item $\Abs{\tfrac{1}{\card{M'}} \sum_{x\in M'} \iprod{v, x-\mu}}\le \delta$,
        \item $\Abs{\tfrac{1}{\card{M'}} \sum_{x\in M'} \iprod{P, \paren{x-\mu}\paren{x-\mu}^\top} - \iprod{P, Q}}\le \delta^2/\e$, and
        \item $\Abs{\iprod{P, Q}} \le r^2$.
    \end{enumerate}
\end{definition}

Note that we use it with $m = d^2$ and apply it to the set $S$ of samples $y_1y_1^\top,\ldots, y_ny_n^\top$, where $y = \spsign(x)$.
For estimation in spectral norm, $\cV$ is $\Set{uu^\top \suchthat \norm{v} = 1}$, $\cP$ is the set of $\pE u^{\otimes 4}$ described above, $Q = 2\Id_{d^2}$, and $r = \sqrt{2}$ (and, as in the standard filtering algorithm, $\delta = O(\e\log(1/\e)$). 

As previously mentioned, if an isotropic distribution $\cY$ satisfies the Hanson-Wright property, then it can be shown that the set of iid samples drawn from this distribution is $(\e,O(\e\log(1/\e)), O(1), \cB, \Conv(\cB\otimes \cB))$-stable with respect tp $\mu = \Id_d$ and $Q = \E_{y\sim \cY}\Paren{yy^\top - \Id_d}^{\otimes 2}$ with high probability, where $\cB = \{V\in \R^{d^2} : \normf{V} = 1\}$. Since our $\cV$ is a subset of $\cB$, and our $\cP$ is a subset of $\Conv(\cB\otimes \cB)$, we also get the stability for $\cV$ and $\cP$.

The filtering algorithm works in a similar way to the Gaussian case: It assigns weights to the samples, in the case of the covariance estimation transforms them\footnote{For the covariance filtering the samples are transformed via linear transformation $y_iy_i^\top \mapsto C^{-1/2}y_iy_i^\top C^{-1/2}$, where $C$ is the current candidate for the covariance estimator.}, and checks whether the value $\lambda := \max_{P\in\cP}\Abs{\iprod{P, Q-\hat{Q}^{(t)}}}$  is too large
(where $\hat{Q}^{(t)}$ is the weighted empirical $m\times m$-dimensional covariance of the (transformed) samples). 
In particular, it requires $Q$ (but, of course, not $\mu$) to be known.
If $\lambda$ is large, it reassigns the weights, so that the weights of the samples that made it large decrease. In the end $\lambda$ should be small, and if it is small, then the stability guarantees that the current weighted sample mean is close to $\mu$.
Note that since we can optimize linear functions over $\cP$ efficiently, the algorithm runs in polynomial time.

This notion of stability, however, is not enough for the filtering algorithm to work. Apart from the condition that  the set of samples is stable at each iteration\footnote{The transformed set needs to be stable for certain $\delta' > \delta$. This condition also appears in the Gaussian setting, and for our case can be shown in a similar way.} of the algorithm, we need some additional conditions that $\cP$ is in certain sense not much larger than $\cV\otimes \cV$. Concretely, for all $A,B\in \R^{d^2}$ and $P\in \cP$, we need to show that
\[
\Iprod{A\otimes B, P}^2 \le \Paren{\sup_{V\in \cV}\Iprod{A, V}^2}\cdot\Paren{\sup_{V\in \cV}\Iprod{B, V}^2}\,.
\]
and $\Iprod{A\otimes A, P} \ge 0$.
For $P = v^{\otimes 4}$, it is possible to show that these inequalities can be certified via degree-$4$ sum-of-squares proofs, so $\cP$ satisfies the desired properties. 

Thus, we show that the filtering algorithm for spectral covariance estimation finds $\hat{\Sigma}_2$ such that 
$\norm{\hat{\Sigma}_2^{-1/2} \Sigma \hat{\Sigma}_2^{-1/2} - \Id} \le O\Paren{\e\log(1/\e)}$. After we multiply fresh samples by $\hat{\Sigma}_2^{-1/2}$, we can assume that we work with a sample from an elliptical distribution with scatter matrix $\Sigma$ such that $\Norm{\Sigma - \Id} \le O(\e\log(1/\e))$.

\paragraph{Final Estimation.}
To estimate $\Sigma$ in Frobenius norm, we again use the stability-based filtering. We use it with $Q = S$ (recall that $S := \E\Paren{gg^\top / \norm{g}^2 -\Id_d}^{\otimes 2}$).
Since $\Sigma$ is very close to $\Id_d$, the situation is simpler than before: we show the stability of the \emph{non-transformed} samples (with the same $\delta = O(\e\log(1/\e)$), and then simply apply filtering without transformations (in other words, this filtering algorithm is oblivious to the matrix structure in $\R^{d^2}$).  This algorithm finds $\hat{\Sigma}_3$ such that $\normf{\hat{\Sigma}_3^{-1/2} \Sigma \hat{\Sigma}_3^{-1/2} - \Id} \le O\Paren{\e\log(1/\e)}$. 

Now let us use the notation $\Sigma$ for the original scatter matrix with $\Tr(\Sigma) = d$, 
$\Sigma_1 = \rho_1\hat{\Sigma}_1^{-1/2}\Sigma\hat{\Sigma}_1^{-1/2}$, 
$\Sigma_2 = \rho_2\hat{\Sigma}_2^{-1/2}\Sigma_1\hat{\Sigma}_2^{-1/2}$, where $\rho_1$ and $\rho_2$ are chosen so that $\Tr(\Sigma_1) = \Tr(\Sigma_2) = d$. Then $\hat{\Sigma} := \hat{\Sigma}_1 \hat{\Sigma}_2 \hat{\Sigma}_3$
satisfies
\[
\normf{\hat{\Sigma}^{-1/2} \Paren{\rho_1\rho_2\Sigma} \hat{\Sigma}^{-1/2}  - \Id}\le O(\e\log(1/\e))\,.
\]
So $\hat{\Sigma}$ is the desired estimator of the scatter matrix $\rho_1\rho_2\Sigma$.



%\subsection{Robust Covariance Estimation under Moment Assumptions}



% \section{Methodology}
% \label{sec:method}
% 
\section{\label{sec:method}Methodology}

Each SIEM system uses its own RDL to define threat detection rules, and each RDL has its own schema.
For example, the Splunk SIEM uses the SPL to define its threat detection rules.
The task of understanding threat detection rules and recommending relevant MITRE ATT\&CK techniques (or sub-techniques) requires complex reasoning skills.
In the case of LLMs, this can be achieved with a technique called prompt chaining in which each task is divided into multiple sub-tasks in order to understand the complex reasoning behind the task.
Therefore, we employ a multi-phase architecture based on prompt chaining that leverages the power of LLMs to take a SIEM rule defined in any RDL and map it to relevant MITRE ATT\&CK techniques using the power of LLMs.
Our approach is based on the following intuitions:
\begin{itemize}[nosep,leftmargin=*]
    \item \textit{LLMs' implicit knowledge}: LLMs possess deep understanding of diverse RDLs. This enables them to interpret any rule, regardless of the RDL it is defined in, and convert it into comprehensible natural language text.
    \item \textit{LLMs' similarity comparison capability}: LLMs are adept at analyzing and comparing textual descriptions. 
    They can intelligently assess the similarity between two textual inputs to establish a meaningful connection.
\end{itemize}

\methodName has two main phases: (1) the rule to text translation phase, and (2) the MITRE ATT\&CK techniques recommendation phase.
These two phases in the pipeline include six key steps to determine relevant TTPs, as illustrated in Figure~\ref{fig:r2t}.

Although LLMs excel at translating SIEM rules into natural language, they often lack critical domain-specific contextual information related to IoCs in the rules.
To overcome this limitation, the \textit{rule to text translation} phase includes three steps: IoC extraction, contextual information retrieval, and natural language translation.

The workflow begins with the extraction of IoCs from the rules (for example, processes, log source, event codes, and file names) that the rule searches for in the logs (step (1)).In the next sstep a web search agent performs the task of obtaining additional contextual information about the IoCs discovered ((step 2)).
By incorporating this additional domain-specific information, the pipeline enhances the language translation, resulting in a more accurate and meaningful interpretation of SIEM rules.
The rule itself and the IoCs' contextual additional information from the previous stage are then used to translate the rule from RDL to natural language (step (3)).

The \textit{MITRE ATT\&CK techniques} recommendation phase of the pipeline includes the following three steps.
The rule in processed in data source identification step in which the probable origin of the data is identified (step (4)).
The description of the rule is then used to determine probable MITRE ATT\&CK techniques based on the implicit knowledge of the LLM (step (5)).
Finally, using chain-of-thought~\cite{wei2022chain} prompting, the most relevant techniques are extracted from the list of probable techniques (step (6)).
Each step of our method is further described in detail below.


% [bb=0 0 1440 900,width=1.43\linewidth,height=0.9\textwidth]
\begin{figure*}[htbp]
   \includegraphics[width=\textwidth]{Images/stages.jpg}
    
   \caption{An illustration of the different steps in \methodName.}
   \label{fig:stages}
\end{figure*} 

\subsection{IoC Extraction}
The context associated with a SIEM detection rule is crucial for its accurate interpretation and effective application. 
Obtaining this contextual understanding requires comprehensive analysis of the embedded IoCs in the SIEM rule.
In the first step, \methodName systematically identifies and extracts all IoCs, identifying the types of IoCs and their corresponding values that form the foundational elements of the detection rules. 
Leveraging the LLM's inherent understanding of rule structures and IoCs, we employ a zero-shot prompting approach for this task. 
Zero-shot prompting enables the direct extraction of IoCs from the rules without requiring extensive pre-training on specific datasets.

\noindent The result of this stage is a dictionary structure, where:
\begin{itemize}[nosep,leftmargin=*]
    \item Keys represent types of IoC, such as processes, files, IP addresses, and log sources.
    \item Values are lists containing specific IoC details, such as process names, file names, IP addresses, and log source identifiers.
\end{itemize}

In the example depicted in Figure~\ref{fig:stages}(a), the pipeline processes a rule for which relevant MITRE ATT\&CK techniques need to be recommended. 
The IoC extractor LLM produces a dictionary structure as output, organizing the IoCs in a structured format to support subsequent stages in the analysis pipeline. 



\subsection{Contextual Information Retrieval}
In this step, an LLM agent is employed to retrieve relevant information pertaining to the IoCs extracted from the rule.
A REACT agent~\cite{react} was used in this case to generate both reasoning traces and task-specific actions in an interleaved manner.
REACT agents interact with external tools to retrieve additional information that leads to more factual and reliable responses.
The LLM agent conducts a systematic search across web resources to gather additional contextual information for each IoC value present in the rule. 
This step addresses LLMS' lack of up-to-date knowledge or specialized domain expertise (which is critical to understanding the role and significance of the IoCs in the rule), without the need for retraining or fine-tuning.
Figure~\ref{fig:stages}(b) presents an example in which the rule includes the process name \texttt{soaphound.exe} as an IoC.
As can be seen, the web search results indicate that \texttt{soaphound.exe} is being used for active directory (AD) enumeration, which is important for the understanding of the attack. 

\subsection{Natural Language Translation}

The translation of detection rules into natural language textual descriptions fulfills three key objectives:
\begin{enumerate}[nosep,leftmargin=*]
    \item \textbf{Ensures that \methodName is format-agnostic}: It converts rules defined in various RDL formats into a generic, unstructured text format, ensuring compatibility with different SIEM systems, regardless of the specific rule format.
    \item \textbf{Provides contextual explanation}: It includes all relevant contextual information to produce a concise and comprehensible explanation of the rule.
    \item \textbf{Enhances the comprehension for LLMs}: It enables LLMs to more effectively compare the translated rule with descriptions in the MITRE ATT\&CK framework by providing a unified textual representation.
\end{enumerate}
To achieve these objectives, a zero-shot prompting technique is employed. 
The input to the LLM comprises two components:
\begin{itemize}
    \item \textbf{Syntactical information}: The rule itself, providing the structural and operational details.
    \item \textbf{Contextual information}: Details of the IoCs extracted from the rule, providing semantic insights into the rule's intent and function.
\end{itemize}
The LLM utilizes these inputs to generate a natural language textual description of the rule. 
This transformation not only ensures a more interpretable representation but also facilitates further steps of analysis and comparison, particularly in aligning the rule with MITRE ATT\&CK techniques and sub-techniques.



\subsection{Data Source or Mitigation Identification}
Identifying the most relevant data component or mitigation associated with the rule description in this step is critical for filtering out irrelevant MITRE ATT\&CK techniques (or sub-techniques) in subsequent steps of the pipeline.
In the MITRE ATT\&CK framework, data sources represent various categories of information that can be gathered from sensors or logs. 
These data sources include \textit{data components}, which are specific attributes or properties within a data source that are directly relevant to detecting a particular technique or sub-technique~. 
For example, in the context of the rule described in Figure~\ref{fig:stages}(a), the term \texttt{Endpoint.Processes} indicates that the activity is happening on an endpoint. 
Presence of the terms such as, \texttt{soaphound.exe}, \texttt{--buildcache}, \texttt{--certdump} and etc. indicate that the rule searches for command line execution of an executable named \texttt{soaphound.exe} with specific parameters. 
Therefore, the appropriate data source in this example is \textit{Command}, with the corresponding data component being \textit{Command Execution}.
Additionally, \textit{mitigations} are defined as categories of technologies or strategies that can prevent or reduce the impact of specific techniques or sub-techniques. 
The MITRE ATT\&CK framework explicitly establishes relationships between data components, mitigations, and techniques (or sub-techniques), enabling a systematic approach for identifying relevant elements.

To identify the most relevant data component or mitigation associated with a given rule description, we utilize agentic retrieval augmented generation (RAG), which incorporates an AI Agent-based implementation of the RAG framework.
Data from the MITRE ATT\&CK framework, specifically related to data components and mitigations, is stored in a vector database (e.g., ChromaDB). 
The process begins with the rule description from the previous stage, which serves as the input to the AI Agent. 
The LLM-powered agent automatically generates a search query tailored to retrieve relevant information from the RAG database.

For each query, the system retrieves the five most similar documents from the database, each containing contextual information about data components or mitigations. 
These documents are then utilized by the LLM agent to contextualize the rule description. 
By comparing the content of these retrieved documents with the rule description, the LLM agent determines and outputs the most relevant data component or mitigation along with a chain-of-thought as to why the data component or mitigation is related to the rule.


\subsection{Probable Technique Recommendation}

In this step, an LM agent is utilized to propose probable MITRE ATT\&CK techniques (and sub-techniques) that may be relevant to the description of the provided rule.
We used a REACT agent in this step as well to utilize both implicit and explicit knowledge during reasoning.
For explicit knowledge, the agent searches the MITRE ATT\&CK framework to obtain the list of probable techniques (and sub-techniques).
The natural language description of the rule from the previous step serves as input to the LLM agent.
The output of this stage consists of a list of JSON objects, each containing the MITRE technique ID, technique name, and technique description as seen in Figure~\ref{fig:stages}(c).

Throughout our experiments, we observed that as the number of recommendations ($k$) increases, both the framework's average recall and precision initially improve, however beyond a certain threshold of $k$, the %average 
precision begins to decline.
Based on these observations(please refer Table~\ref{tab:results3}), we selected a $k$-value of 11 to ensure a high recall.



\subsection{Relevant Technique Extraction}
In this step, \methodName refines the set of probable MITRE ATT\&CK techniques identified in the previous stage by eliminating irrelevant entries.
This step in the pipeline serves two primary purposes: (1) to enhance precision while maintaining recall achieved in previous step, and (2) to provide a clear rationale for the selection of the labels, ensuring transparency and interpretability of the mapping process.
This refinement process is grounded in the assumption that LLMs are effective for text similarity matching tasks.

The process comprises two key steps:
\begin{itemize}
    \item \textit{Rule-technique comparison}: The description of each technique in the set of probable techniques is compared with the rule description. 
    A chain-of-thought technique is then applied to elucidate the reasoning behind the association of each technique with the rule.
    \item \textit{Confidence calculation}: The generated chain-of-thought rationale for each technique (or sub-technique) is compared with the rule description to compute a relevance (or confidence) score, as done in prior work~\cite{freitas2024ai}.
    % \item \textbf{Reasoning}: \new{Add here the reasoning that it provides - explaining in NLP why it was selected...}
\end{itemize}

Techniques with higher confidence scores are deemed more relevant to the rule. 
Conversely, techniques with scores falling below a predefined threshold are excluded.
The techniques retained after this filtering step represent the most relevant techniques corresponding to the given rule's description. 


The chain-of-thought (CoT) rationale generated during the comparison of each rule to its probable technique is also provided as an output in this step.
This rationale offers a detailed natural language explanation, articulating why a particular technique is relevant to the given rule. 
Such explanations are highly valuable for security analysts, as they provide clear and transparent reasoning behind the mapping, enabling analysts to better understand and validate the association between the rule and the technique.
Other classification models proposed in previous works within this domain also suffer from the limitation of being black-box models, which lack the ability to provide clear reasoning or explanations. 
Unlike \methodName, these models fail to generate transparent, CoT rationales that explain why a particular rule is mapped to a specific technique, making them less interpretable and less useful for security analysts.

% evaluation_modified.tex is the updated/latest version while evaluation.tex is the CAV submission 
\section{Evaluation}
\label{sec:eval}
% \section{Evaluation}
We provide three sets of insights into this section, organised as \textit{findings (F*)}. We quantitatively study the effect of the adversarial and counterfactual perturbations on the performance of informal reasoners and autoformalisation methods. Then, we dive deeper into method variants. Finally, 
we analyse the nature of formalisation errors made by the models.

\subsection{Robustness Analysis}
\paragraph{\textbf{\emph{F1: Noise perturbations have a stronger effect on formalisation methods than informal \ac{LLM} reasoners.}}}
Table~\ref{tab:distraction_k4_formalisation} shows that, on average, the accuracy of both direct and \ac{CoT} informal reasoning remains between $73\%$ and $74\%$ in the face of added noise. While the autoformalisation method performs similarly to informal reasoners on the original dataset, its performance decreases between $4\%$ and $11\%$. The accuracy drops especially with logical (L) and tautological (T) distractions, whose logical language formats trick the \ac{LLM} into formalizing the noisy clauses. On the other hand, the linguistically complex and more natural sentences of encyclopedic distractions show a minor effect, suggesting that \acp{LLM} successfully avoids formalizing the more complicated sentences.

\paragraph{\textbf{\emph{F2: All \ac{LLM}-based reasoning methods suffer a drop for counterfactual perturbations.}}} % influence .}}}
Table~\ref{tab:distraction_k4_formalisation} shows that counterfactual statements cause a significant decrease in performance for both the informal reasoners and autoformalisation methods of between $12\%$ and $13\%$ on average. 
Moreover, this observation also holds for all tested models, i.e., none are robust towards counterfactual perturbations across every evaluated dimension. Even the strongest model, GPT 4o-mini, yields a performance of 63-68\%, which is relatively close to the random performance of 50\%. The high impact of counterfactual statements (the single ``not'' inserted) could be due to the inability of \acp{LLM} to overwrite prior knowledge with explicitly stated information or memorization of the answers. We study the error sources further in §\ref{subsec:errors}.  

\noindent \paragraph{\textbf{\emph{F3: Introducing multiple noise sentences has an effect only for logical distractions.}}}
We show the impact of introducing between one and four sentences for the two top-performing autoformalisation models in Figure~\ref{fig:length_distraction}. The figure shows similar trends with and without counterfactual perturbations.
As additional logical distractions are introduced, the model performance consistently decreases. Tautological (T) distractions lead to a decline in accuracy with a single disruptive sentence, yet adding more noise does not worsen the outcome. 
The tautological corpus introduces truth constants for all sentences as a persistent unseen logical construct. Given that this leads only to a decrease for a single occurrence, we can assume that a model can consistently handle the same unseen logical construct. In contrast, the logical corpus increases the chance of adding text, requiring new, previously unseen reasoning constructs for each added sentence. The impact of encyclopedic noise remains negligible, generalising F1 to $k$ sentences. Similarly, counterfactual perturbations remain much more effective for all settings, generalising F2.

\begin{table}[!t]
\small
\setlength{\modelspacing}{2pt}
\setlength{\tabcolsep}{1.7pt} % Default value: 6pt
\setlength{\belowrulesep}{4pt}
\begin{threeparttable}
    \centering
    \begin{tabular}{cc l r rrr @{\quad} rrrr}
\toprule
\multirow{2}{*}{} & \multirow{2}{*}{} & Reasoning & \multirow{2}{*}{O} & \multicolumn{3}{c}{Distraction} & \multicolumn{4}{c}{Counterfactual} \\
 & & Format & & E& L & T & $\text{O}_C$ & $\text{E}_C$& $\text{L}_C$ & $\text{T}_C$\\
\midrule
\multirow{6}{*}{\rotatebox{90}{Gemma-2}} & \multirow{3}{*}{\rotatebox{90}{9b}}
   & Informal (direct) & \textbf{0.78} & \textbf{0.80} & \textbf{0.79} & \textbf{0.77} & 0.58 & 0.52 & 0.50 & 0.59 \\
 & & Informal (CoT) & 0.72 & 0.78 & 0.73 & 0.76 & 0.61 & \textbf{0.57} & \textbf{0.60} & \textbf{0.66} \\
 & & Formal (FOL) & 0.62 & 0.58 & 0.52 & 0.53 & \textbf{0.63} & 0.52 & 0.46 & 0.46 \\[\modelspacing]
\cmidrule{2-11}
 & \multirow{3}{*}{\rotatebox{90}{27b}} 
   & Informal (direct) & 0.71 & 0.69 & \textbf{0.66} & \textbf{0.68} & 0.59 & 0.51 & 0.54 & 0.59 \\
 & & Informal (CoT) & 0.66 & 0.65 & 0.64 & 0.63 & 0.62 & 0.58 & \textbf{0.62} & \textbf{0.64} \\
 & & Formal (FOL) & \textbf{0.74} & \textbf{0.74} & 0.61 & 0.61 & \underline{\textbf{0.72}} & \underline{\textbf{0.67}} & 0.58 & 0.51 \\[\modelspacing]
\midrule
\multirow{6}{*}{\rotatebox{90}{Mistral}} & \multirow{3}{*}{\rotatebox{90}{7B}} 
   & Informal (direct) & 0.77 & \textbf{0.77} & 0.75 & \textbf{0.79} & \textbf{0.63} & \textbf{0.54} & \textbf{0.54} & \textbf{0.66} \\
 & & Informal (CoT) & \textbf{0.79} & 0.75 & \textbf{0.77} & 0.78 & 0.55 & 0.52 & \textbf{0.54} & 0.58 \\
 & & Formal (FOL) & 0.62 & 0.58 & 0.54 & 0.57 & 0.50 & \textbf{0.54} & 0.51 & 0.52 \\[\modelspacing]
\cmidrule{2-11}
 & \multirow{3}{*}{\rotatebox{90}{Small}} 
   & Informal (direct) & \textbf{0.77} & \textbf{0.76} & \textbf{0.76} & \textbf{0.75} & 0.61 & 0.51 & 0.56 & 0.59 \\
 & & Informal (CoT) & 0.72 & 0.72 & 0.72 & 0.71 & \textbf{0.62} & \textbf{0.59} & \textbf{0.62} & \textbf{0.68} \\
 & & Formal (FOL) & 0.68 & 0.59 & 0.53 & 0.64 & 0.54 & 0.55 & 0.49 & 0.51 \\[\modelspacing]
\midrule
\multirow{6}{*}{\rotatebox{90}{Llama-3.1}} & \multirow{3}{*}{\rotatebox{90}{8B}} 
   & Informal (direct) & 0.63 & 0.61 & 0.64 & 0.66 & 0.61 & \textbf{0.62} & 0.59 & 0.61 \\
 & & Informal (CoT) & 0.73 & \textbf{0.73} & \textbf{0.71} & \textbf{0.72} & \textbf{0.62} & 0.59 & \textbf{0.61} & \textbf{0.65} \\
 & & Formal (FOL) & \textbf{0.77} & 0.71 & 0.63 & 0.52 & 0.60 & 0.58 & 0.55 & 0.52 \\[\modelspacing]
\cmidrule{2-11}
 & \multirow{3}{*}{\rotatebox{90}{70B}} 
   & Informal (direct) & 0.77 & 0.74 & 0.74 & 0.73 & 0.62 & 0.53 & 0.56 & 0.64 \\
 & & Informal (CoT) & \textbf{0.78} & \textbf{0.75} & \textbf{0.76} & \textbf{0.76} & 0.64 & 0.61 & \textbf{0.66} & \underline{\textbf{0.73}} \\
 & & Formal (FOL) & 0.74 & 0.73 & 0.71 & 0.71 & \textbf{0.66} & \textbf{0.62} & 0.59 & 0.57 \\[\modelspacing]
 \midrule
\multirow{3}{*}{\rotatebox{90}{GPT}} & \multirow{3}{*}{\rotatebox{90}{4o-mini}} 
   & Informal (direct) & 0.78 & 0.77 & 0.79 & 0.79 & 0.64 & 0.61 & 0.61 & 0.63 \\
 & & Informal (CoT) & 0.80 & 0.80 & \underline{\textbf{0.81}} & \underline{\textbf{0.82}} & \textbf{0.68} & \textbf{0.63} & \underline{\textbf{0.68}} & \textbf{0.64} \\
 & & Formal (FOL) & \underline{\textbf{0.84}} & \underline{\textbf{0.82}} & 0.73 & 0.79 & 0.63 & 0.62 & 0.57 & 0.54 \\[\modelspacing]
 \midrule
\multicolumn{2}{c}{\multirow{3}{*}{\textbf{Avg}}} 
 & Informal (direct) & 0.74 & 0.73 & 0.73 & 0.73 & 0.61 & 0.55 & 0.56 & 0.62 \\
 & & Informal (CoT) & 0.74 & 0.74 & 0.73 & 0.74 & 0.62 & 0.58 & 0.62 & 0.65 \\
  & & Formal (FOL) & 0.72 & 0.68 &	0.61 & 0.62 & 0.61 & 0.59 & 0.54 & 0.52 \\
\bottomrule
\end{tabular}
\caption{Accuracies of informal and autoformalisation-based deductive reasoners. The best overall model per dataset is underlined; the best model version is marked in bold.}
\label{tab:distraction_k4_formalisation}
\end{threeparttable}
\end{table} 

\begin{figure}[!t]
    \centering
    \scriptsize
    \begin{tikzpicture}
        \begin{axis}[name=gpt,
            title={GPT-4o-mini},
            width=0.6\linewidth,
            height=0.6\linewidth,
            xlabel={\# Noise sentences},
            ylabel={Accuracy},
            xmin=-0.1, xmax=4.1,
            ymin=0.5, ymax=0.9,
            xtick={1,2,4},
            ytick={0.55, 0.6, 0.65, 0.75, 0.8, 0.85},
            title style={yshift=-0.6em},
            legend style={at={(1,-0.15)},
	           anchor=north,legend columns=-1},
            x label style={at={(axis description cs:1,-0.05)},anchor=north},
            y label style={at={(axis description cs:-0.15,0.5)},anchor=south},
            ymajorgrids=true,
            grid style=dashed,
        ]
            \addplot[color=blue, mark=square,]
                coordinates {
                (0,0.848076939582825)(1,0.823076903820038)(2,0.826923072338104)(4,0.821153819561005)
                };
            \addplot[color=red, mark=triangle,]
                coordinates {
                (0,0.848076939582825)(1,0.817307710647583)(2,0.801923096179962)(4,0.759615361690521)
                };
            \addplot[color=green, mark=diamond,] 
                coordinates {
                (0,0.848076939582825)(1,0.767307698726654)(2,0.769230782985687)(4,0.803846180438995)
                };
            \addplot[color=blue, mark=square*] 
                coordinates {
                (0,0.627777755260468)(1,0.622222244739533)(2,0.600000023841858)(4,0.633333325386047)
                };
            \addplot[color=red, mark=triangle*,] 
                coordinates {
                (0,0.627777755260468)(1,0.611111104488373)(2,0.611111104488373)(4,0.594444453716278)
                };
            \addplot[color=green, mark=diamond*,] 
                coordinates {
                (0,0.627777755260468)(1,0.572222232818604)(2,0.538888871669769)(4,0.555555582046509)
                };
                \legend{E,L,T,$\text{E}_C$, $\text{L}_C$ , $\text{T}_C$}
        \end{axis}

        \begin{axis}[name=llama, at={($(gpt.east)+(0.1cm,0)$)},anchor=west,
            title={Llama 3.1 70b},
            width=0.6\linewidth,
            height=0.6\linewidth,
            xmin=-0.1,, xmax=4.1,
            ymin=0.5, ymax=0.9,
            xtick={1,2,4},
            ytick={0.55, 0.6, 0.65, 0.75, 0.8, 0.85},
            title style={yshift=-0.6em},
            yticklabel=\empty,
            ymajorgrids=true,
            grid style=dashed,
        ]
            \addplot[color=blue, mark=square,]
                coordinates {
                (0,0.838461518287659)(1,0.817307710647583)(2,0.805769205093384)(4,0.817307710647583)
                };
            \addplot[color=red, mark=triangle,]
                coordinates {
                (0,0.838461518287659)(1,0.819230794906616)(2,0.803846180438995)(4,0.771153867244721)
                };
            \addplot[color=green, mark=diamond,]
                coordinates {
                (0,0.838461518287659)(1,0.803846180438995)(2,0.807692289352417)(4,0.805769205093384)
                };
            \addplot[color=blue, mark=square*]
                coordinates {
                (0,0.627777755260468)(1,0.622222244739533)(2,0.577777802944183)(4,0.594444453716278)
                };
            \addplot[color=red, mark=triangle*,]
                coordinates {
                (0,0.627777755260468)(1,0.583333313465118)(2,0.561111092567444)(4,0.577777802944183)
                };
            \addplot[color=green, mark=diamond*,]
                coordinates {
                (0,0.627777755260468)(1,0.627777755260468)(2,0.566666662693024)(4,0.577777802944183)
                };
        \end{axis}
    \end{tikzpicture}
    \caption{Influence of the number of noisy sentences for FOL.}
    \label{fig:length_distraction}
\end{figure}



\subsection{Impact of Method Design}
\paragraph{\textbf{\emph{F4: \ac{CoT} prompting is most impactful when both noise and counterfactual perturbations are applied.}}}
The accuracies for the individual \acp{LLM} in Table~\ref{tab:distraction_k4_formalisation} show that the impact of \ac{CoT} is negligible for noise-only datasets (first four columns). Meanwhile, the benefit from \ac{CoT} is most pronounced in the datasets that combine noise and counterfactual perturbations.
The better-performing informal prompting strategy for a model remains stable for all types of distractions. Still, the decline in performance due to counterfactuals leads to a less consistent preference for a specific prompting style.

\paragraph{\textbf{\emph{F5: The best-performing grammar differs per model and is unstable across data versions.}}}

The evaluation of different logical forms for formal \ac{LLM}-based reasoning in Table~\ref{tab:distraction_k4_logical_form} shows the preference of some models for specific syntactic formats.
Llama 3.1 70B has a considerable improvement of $12\%$ with TPTP syntax on the original set, while Llama 3.1 8B benefits from the R-FOL syntax. However, all grammars show a declining accuracy trend and increased syntax errors for noise perturbations, where the best grammar loses its advantage over the rest. 
When comparing the grammars on the counterfactual partitions, we observe that TPTP is consistently more robust than the standard first-order logic grammar. Here, GPT 4o-mini shows a reduction from $O$ to $O_C$ of $20\%$ for FOL and only $12\%$ for the TPTP grammar. Since this does not correlate with fewer syntax errors, the formalisation in TPTP prevents semantical errors for counterfactual premises. 
A positive reading of these results, especially the minor differences between FOL and R-FOL, is that autoformalisation \acp{LLM} can adapt to the grammar syntax prescribed in the prompt without further loss in performance.

\begin{table}[!t]
\small
\setlength{\modelspacing}{2pt}
\setlength{\tabcolsep}{1.7pt} % Default value: 6pt
\setlength{\belowrulesep}{4pt}
\begin{threeparttable}
    \centering
    \begin{tabular}{cc l r rrr @{\quad} rrrr}
\toprule
\multirow{2}{*}{} & \multirow{2}{*}{} & Grammar & \multirow{2}{*}{O} & \multicolumn{3}{c}{Distraction} & \multicolumn{4}{c}{Counterfactual} \\
 & & Syntax & & E& L & T & $\text{O}_C$ & $\text{E}_C$& $\text{L}_C$ & $\text{T}_C$\\
\midrule
\multirow{6}{*}{\rotatebox{90}{Llama-3.1}} & \multirow{3}{*}{\rotatebox{90}{8B}} 
   & FOL & 0.77 & \textbf{0.71} & 0.61 & \textbf{0.53} & 0.58 & \textbf{0.55} & 0.52 & \textbf{0.56} \\
 & & R-FOL & \textbf{0.78} & 0.69 & \textbf{0.62} & \textbf{0.53} & 0.58 & \textbf{0.55} & \textbf{0.54} & 0.52 \\
 & & TPTP & 0.73 & 0.67 & 0.55 & 0.51 & \textbf{0.68} & 0.54 & 0.46 & 0.51 \\[\modelspacing]
\cmidrule{2-11}
 & \multirow{3}{*}{\rotatebox{90}{70B}} 
   & FOL & 0.76 & 0.73 & 0.71 & \textbf{0.72} & 0.67 & 0.57 & 0.63 & 0.56 \\
 & & R-FOL & 0.76 & 0.73 & 0.67 & 0.71 & 0.64 & 0.57 & 0.53 & 0.64 \\
 & & TPTP & \underline{\textbf{0.88}} & \underline{\textbf{0.84}} & \underline{\textbf{0.81}} & \textbf{0.72} & \underline{\textbf{0.81}} & \underline{\textbf{0.68}} & \underline{\textbf{0.67}} & \underline{\textbf{0.68}} \\[\modelspacing]
\midrule
\multirow{3}{*}{\rotatebox{90}{GPT}} & \multirow{3}{*}{\rotatebox{90}{4o-mini}} 
   & FOL & \textbf{0.84} & \textbf{0.82} & \textbf{0.72} & \underline{\textbf{0.78}} & 0.64 & \textbf{0.63} & \textbf{0.61} & 0.51 \\
 & & R-FOL & \textbf{0.84} & 0.77 & 0.70 & \underline{\textbf{0.78}} & \textbf{0.72} & 0.56 & 0.54 & \textbf{0.63} \\
 & & TPTP & 0.83 & \textbf{0.82} & 0.71 & 0.71 & 0.69 & \textbf{0.63} & 0.57 & 0.57 \\
\bottomrule
\end{tabular}
\caption{Accuracies of different formalisation grammars for autoformalisation.}
\label{tab:distraction_k4_logical_form}
\end{threeparttable}
\end{table} 

\paragraph{\textbf{\emph{F6: Feedback does not help \acp{LLM} self-correct to mitigate robustness issues.}}}
\autoref{tab:distraction_k4_feedback} shows the results with different error recovery mechanisms. The results indicate that no feedback strategy emerges as a winner in the different datasets. 
All feedback variants reduce syntax errors for noise perturbations, but given the lack of a consistent increase in accuracy, the corrected formalisations are most likely to contain semantic errors still. 
The type of feedback message only has a minor influence on correcting syntax errors, whereas Llama 3.1 70b and GPT 4o-mini correct slightly more syntax errors with specific error messages. This finding aligns with \cite{huang2023large}, who also found that \acp{LLM} cannot consistently self-correct their reasoning after receiving relevant feedback.

\begin{table}[!ht]
\small
\setlength{\modelspacing}{2pt}
\setlength{\tabcolsep}{1.7pt} % Default value: 6pt
\setlength{\belowrulesep}{4pt}
\begin{threeparttable}
    \centering
    \begin{tabular}{cc l r rrr @{\quad} rrrr}
\toprule
\multirow{2}{*}{} & \multirow{2}{*}{} & \multirow{2}{*}{Feedback} & \multirow{2}{*}{O} & \multicolumn{3}{c}{Distraction} & \multicolumn{4}{c}{Counterfactual} \\
 & & & & E& L & T & $\text{O}_C$ & $\text{E}_C$& $\text{L}_C$ & $\text{T}_C$\\
\midrule
\multirow{8}{*}{\rotatebox{90}{Llama-3.1}} & \multirow{4}{*}{\rotatebox{90}{8B}} 
   & No recovery & 0.77 & \textbf{0.72} & 0.62 & 0.53 & 0.59 & 0.58 & 0.56 & \textbf{0.56} \\
 & & Error type & \textbf{0.79} & 0.71 & 0.63 & \textbf{0.56} & \textbf{0.66} & 0.54 & 0.52 & 0.51 \\
 & & Error message & 0.78 & 0.71 & \textbf{0.67} & 0.55 & 0.59 & 0.53 & \underline{\textbf{0.64}} & 0.49 \\
 & & Warning & 0.74 & 0.66 & 0.58 & 0.55 & 0.55 & \textbf{0.60} & 0.49 & 0.49 \\[\modelspacing]
\cmidrule{2-11}
 & \multirow{4}{*}{\rotatebox{90}{70B}} 
   & No recovery & \textbf{0.77} & \textbf{0.72} & \textbf{0.73} & 0.71 & \textbf{0.64} & 0.59 & \textbf{0.61} & 0.56 \\
 & & Error type & 0.72 & 0.70 & 0.72 & \textbf{0.73} & 0.62 & 0.56 & 0.60 & 0.58 \\
 & & Error message & 0.71 & 0.70 & \textbf{0.73} & 0.71 & \textbf{0.64} & 0.59 & 0.54 & \underline{\textbf{0.64}} \\
 & & Warning & 0.69 & \textbf{0.72} & 0.72 & 0.72 & 0.62 & \underline{\textbf{0.65}} & \textbf{0.61} & 0.63 \\[\modelspacing]
\midrule
\multirow{4}{*}{\rotatebox{90}{GPT}} & \multirow{4}{*}{\rotatebox{90}{4o-mini}} 
   & No recovery & \underline{\textbf{0.84}} & \underline{\textbf{0.82}} & 0.73 & 0.79 & 0.64 & \textbf{0.62} & 0.56 & \textbf{0.56} \\
 & & Error type & 0.83 & 0.79 & 0.74 & 0.76 & 0.67 & 0.57 & 0.56 & \textbf{0.56} \\
 & & Error message & \underline{\textbf{0.84}} & 0.78 & \underline{\textbf{0.77}} & \underline{\textbf{0.80}} & 0.62 & 0.59 & 0.56 & \textbf{0.56} \\
 & & Warning & \underline{\textbf{0.84}} & 0.75 & 0.73 & 0.76 & \underline{\textbf{0.70}} & 0.61 & \textbf{0.61} & 0.55 \\
 \bottomrule
\end{tabular}
\caption{Accuracies of error recovery strategies.}
\label{tab:distraction_k4_feedback}
\end{threeparttable}
\end{table} 

\subsection{Error Analysis}
\label{subsec:errors}
\paragraph{\textbf{\emph{F7: Autoformalisation increases syntax errors for noise perturbations.}}}
The low performance for noise perturbations correlates with more syntax errors for all models and distraction categories (cf. execution rates in Table~\ref{tab:appendix_k4_formalisation_exec}). The three worst-performing models (both Mistral models, Gemma-2 9b) generate, at best, for $37\%$  and, at worst, for only $4\%$ of the samples, a valid logical form.
Gemma-2 9b and Llama3.1 8b produce more syntax errors than the larger counterparts, suggesting that larger models are more robust towards noise perturbations. 
The accuracy of syntactically valid samples is higher than the informal reasoning methods for most distractions (Table~\ref{tab:appendix_k4_formalisation_vacc}), motivating informal reasoning as a backup strategy for formal reasoning. The error message feedback reveals two common syntax errors: 1) errors by models with an initial low execution rate exhibit issues with the template structure, including using incorrect keywords or adding conversational phrases;
2) perturbation-related errors, the most common of which is using undefined truth constants as part of tautological distractions. 

\paragraph{\textbf{\emph{F8: Autoformalisation increases semantic errors for counterfactuals.}}}
Unlike the introduced noise, counterfactual perturbations do not lead to more syntax errors. The execution rate in Table~\ref{tab:appendix_k4_formalisation_exec} is stable or improves for counterfactuals. However, we see a drop in accuracy for the counterfactual column $\text{O}_C$ in Table~\ref{tab:distraction_k4_formalisation} and can conclude that the number of logical forms with semantic errors has to increase. This suggests that the introduced negation is not correctly formalised. Looking at the warnings generated by the feedback mechanism, for GPT 4o-mini, $161$ warning messages are generated on the unperturbed data. $54$ of these were fixed with a single iteration. Not considering predicates and individuals as part of the context is the most frequent warning across all models. 

We first outline the LLM and dataset selection process in Sections~\ref{sec:techchallenge} and~\ref{sec:dataset} respectively, followed by our setup in Section~\ref{sec:setup}.
We then report our results and a subsequent discussion in Sections~\ref{sec:results} and~\ref{sec:results-discussion} respectively, before concluding with a note on the usability of our techniques in Section~\ref{sec:usability}.

\subsection{LLM Selection}
\label{sec:techchallenge}
%This work faced several technical challenges arising from the limitations of available large language models (LLMs) for code generation. 
%The issues primarily stem from two areas: the quality of open-source models and the restricted capabilities of certain proprietary models.

%\textbf{Open-Source Model Quality:} Open-source models available on HuggingFace~\cite{huggingface} exhibit significant quality issues in code generation tasks. 
%            Initial experiments with general-purpose LLMs, such as \gemini, on a code hallucinations focused dataset~\cite{codehalu} revealed a negligible pass rate. 


We selected the open-source model \codegenmonoC from \salesforce~\cite{salesforcecodegen} because it was specifically pre-trained for \python, the language used in our dataset.
%
%We tested models of different sizes from \salesforce~\cite{salesforcecodegen}: \codegenmonoA, \codegenmonoB and \codegenmonoC given that they are tailored for code generation for the specific language \ie \python           
 %           Despite these models being tailored for code generation for the specific language \ie \python, their performance remained inadequate, with the best-performing model (\codegenmonoC) achieving only a \bestSalesforcePassrate top pass rate on our benchmark. 
 %           We also applied \llama to our benchmark and it achieved a comparable correctness percentage to \salesforce/\codegenmonoC.
 %           This highlights the current gap in the quality of open-source models for advanced code-generation tasks.
%
%\textbf{API Limitations of Proprietary Models:}
%
We also picked a proprietary model, namely \gptturbo~\cite{gpt35turboinstruct} from \openai~\cite{openai}.
%, demonstrate stronger performance in code generation, they pose challenges for tasks requiring fine-grained uncertainty quantification. 
            Unfortunately, the latest models from \openai don't expose the \texttt{logprobs} functionality through their API, which is needed for our computations. 
%            The only exception was \gptturbo~\cite{gpt35turboinstruct}, which supports \texttt{logprobs} and was therefore selected for our evaluation in this work. These API limitations restrict the range of models that can be used for tasks involving detailed uncertainty analysis.

%These challenges underscore the trade-offs between model quality and technical compatibility when evaluating uncertainty in code generation. 
%Improvements in both open-source and proprietary LLM capabilities are necessary to advance this area of research.



\subsection{Dataset Selection}\label{sec:dataset}
To evaluate the performance of our techniques, we use \livecodebench~\cite{livecodebench}, a contamination-free benchmark for assessing LLMs on code-related tasks
containing \totalProbsComb~problems, divided into \emph{Easy} (\totalProbs problems), \emph{Medium} (\totalProbsMedium problems) and \emph{Hard} (\totalProbsHard problems). 
\livecodebench is designed to address key challenges in code evaluation by incorporating diverse problems from programming competition platforms such as LeetCode, AtCoder, and CodeForces. 
Notably, \livecodebench's contamination-free design ensures that the selected problems have not been seen during the training of most modern LLMs, thereby eliminating data leakage concerns. 
%This guarantees that the benchmark accurately measures a model's generalisation ability rather than its capacity to memorize training data, making it a reliable benchmark for evaluating novel techniques in code generation.

%In our evaluation, we use the entirety of \livecodebench~dataset \ie \totalProbsComb~problems, divided into \emph{Easy, Medium} and \emph{Hard}.

To start with, we focused on the subset of \emph{Easy} problems in \livecodebench, where the two selected LLMs exhibited average testcase passing rates of
\SFSolutionsPassRate~for \codegenmonoC~and \GPTSolutionsPassRateSmall~ for \gptturbo.
Due to limited compute resources and the fact that \codegenmonoC already struggled on the \emph{Easy} problems, 
we then extended our evaluation to the \emph{Medium} and \emph{Hard} classes of problems only for \gptturbo. %, in order for us to get a better understanding of the applicability of our proposed techniques.
For \emph{Medium} problems, \gptturbo achieved a passing rate of \GPTSolutionsPassRateMedium, while for \emph{Hard} problems, the passing rate was \GPTSolutionsPassRateHard.
%reasons described in \S\ref{sec:techchallenge}.
%These problems are curated from high-quality competition datasets and are suitable for testing the functional correctness of generated solutions. 

Each problem is accompanied by a natural language description and Input/Output test cases. We use the natural language description in the query provided to the LLM (as shown in Figure~\ref{fig:sampleproblem}), and the test cases for evaluating our techniques.


%\begin{itemize}[leftmargin=*]
%    \item \textbf{Natural Language Problem Description:} A clear and detailed task description, written in natural language.
%    \item \textbf{Input/Output Test Cases:} A set of test cases used for verifying the correctness of the generated code.
%\end{itemize}

%The decision to use the \emph{Easy} subset stems from the observation that the performance of the LLMs under evaluation was insufficient on harder problems. 
%Higher-difficulty problems often require complex reasoning and sophisticated algorithms, which are challenging for the current generation of LLMs. 
%These limitations could result in poor-quality outputs, undermining our ability to effectively evaluate semantic uncertainty and functional clustering. 

%By selecting problems where the models exhibit a reasonable baseline performance, we ensure a meaningful evaluation of our proposed techniques. 
%The \emph{Easy} subset provides a controlled evaluation setting by avoiding extreme algorithmic complexity while maintaining sufficient diversity in problem types. 
%This allows us to focus on analysing semantic uncertainty and functional clustering in code generation without the confounding effects of low model performance.

%\livecodebench's contamination-free design ensures that the selected problems have not been seen during the training of most modern LLMs, making it a reliable benchmark for the purpose of this work.

\subsection{Experimental Setup}\label{sec:setup}

%In this study, we evaluate the performance of our techniques on a benchmark of \totalProbs problems.
Our experiments were conducted on a machine running \texttt{Ubuntu 20.04.5 LTS (Focal Fossa)} with one \texttt{NVIDIA A100 GPU (80GB)}.
 
%The evaluation involves querying a large language model (LLM) for each problem and applying both SE-based and MI-based approaches to compute uncertainty scores. 
%
%Additionally, in order to investigate the impact of our decision of using \textit{length-normalised response probablities}, we carry out an ablation study by using both the \textit{unnormalised, raw response probablities} alongside our \textit{length-normalised one}. 
%
%Finally, we assess the correctness of the generated solutions based on their performance on test cases provided by the benchmark.

For the semantic equivalence approach in Section~\ref{sec:symex}, we ask the LLM to generate \numSamples responses for each problem along with their respective \texttt{log-probabilities}.
%, as returned by the model's API. Using these \texttt{logprobs}, we compute the probability of each response. 
On top of that, for the MI-based approach described in Section~\ref{sec:mi}, we perform \numIterations iterations of prompting for each of the \numSamples generated responses. The first iteration involves querying the model with the original prompt, while the second iteration uses a concatenated prompt, combining the original prompt and the response from the first iteration. 

In order to perform symbolic execution for finding clusters, we use \crosshair~\cite{crosshair} with a per condition timeout of \CrosshairPerConditiontimeout and the same per path timeout of \CrosshairPerPathtimeout. 
In addition to this, we also impose an overall timeout of \Crosshairtotaltimeout for each pair of programs that is being checked for equivalence.
If a counterexample showing the difference in behavior is not found within this timeout, we assume that the programs belong in the same cluster. 
% \CD{clarify this}

%The SE score is then calculated based on the distribution of probabilities over the generated responses, as described in Section~\ref{sec:symex}.

We evaluate the following techniques:
\begin{itemize}[leftmargin=*]
\item \textbf{\SESymbolicRaw:} The semantic equivalence approach described in Section~\ref{sec:symex}.
%  For our SE-based approach, we query the LLM to generate \numSamples responses for each problem along with their respective \texttt{logprobs}, as returned by the model's API. 
 %                       Using these \texttt{logprobs}, we compute the probability of each response. 
    
  %                      For the \crosshair~\cite{crosshair} runs that are used to compute the clusters symbolically, we use a per condition timeout of \CrosshairPerConditiontimeout and an identical per path timeout of \CrosshairPerPathtimeout. 
   %                     Pairs of programs that hit the timeout are considered to be in the same cluster. 

    %                    The SE score is then calculated based on the distribution of probabilities over the generated responses, as described in Section~\ref{sec:symex}.
\item \textbf{\SESymbolic:} The semantic equivalence approach, but with length normalization (see Section~\ref{sec:probcomp}).
\item \textbf{\SESymbolicUnif:} The semantic equivalence approach, but with uniform distribution (see Section~\ref{sec:probcomp}).  
  %\emph{length-normaised response probablities} \ie while using \textit{unnormalised, raw response probablities}.  

    \item \textbf{\MISymbolicRaw:} The MI approach described in Section~\ref{sec:mi}.
                        %The response probabilities are computed using the same mechanism as the SE-based approach. 
                        %The \crosshair specific timeout remains the same as the SE-based approach. 
                        
                        %The MI score is then calculated as outlined in Section~\ref{sec:mi}.
    \item \textbf{\MISymbolic:} The MI approach, but with length normalization. %This is our adapted MI-based approach, but without the \textit{length-normaised response probablities} \ie while using \textit{unnormalised, raw response probablities}.
    \item \textbf{\SEOriginal:} The original implementation from Kuhn et al.~\cite{kuhnsemantic}. This incorporates length normalization, as omitting it led to probability underflows, resulting in NaN values.
    \item \textbf{\MIOriginal:} The original implementation from Abbasi et al.~\cite{abbasi2024believe}, with length normalization.
    \item \textbf{\LLMProbability:} Baseline technique where the model's response probablities are used as a proxy for correctness. 
\end{itemize}

As mentioned in Section~\ref{sec:mi}, for the MI approach, we did not explore the uniform distribution variant, as the LLM-reported probabilities are central to the technique being used to distinguish between aleatoric and epistemic uncertainties. Additionally, we prioritized experimentation with the better-performing technique---the semantic equivalence-based one.

For each problem in the benchmark, we compute:
\begin{itemize}[leftmargin=*]
\item The corresponding \emph{uncertainty score} for \SESymbolicRaw, \SESymbolic, \SESymbolicUnif, \SEOriginal, \MISymbolicRaw, \MISymbolic and \MIOriginal.
  \item The \emph{probablity of the top-ranked response} for \LLMProbability. 
  \item The \emph{correctness score}, which is calculated as follows: the top-ranked response generated by the LLM (as determined by the API's ranking) is executed against the benchmark's test cases. The percentage of test cases successfully passed is recorded as the correctness score. A generous timeout of \CorrectnessTimeout is applied to each test case. If the candidate solution exceeds this timeout, the test case is considered ``failed''.
    %\item The Pearson correlation factor and the resultant p-value for each of the collected scores and the correctness values for each problem. 
\end{itemize}


\subsection{Results}\label{sec:results}

In Table ~\ref{tab:correlation_results}, we compute the Pearson correlation between the uncertainty scores and the correctness scores for \SESymbolic, \SESymbolicUnif, \SEOriginal, \MISymbolic and \MIOriginal, respectively, and the correlation between the probablity of the top-ranked response and the correctness score for \LLMProbability. We report the Pearson correlation coefficient and p-value, with statistically significant results highlighted in bold.
%
We excluded the results for \SESymbolicRaw and \MISymbolicRaw from Table ~\ref{tab:correlation_results}, as probabilities underflowed (discussed in Section~\ref{sec:probcomp}) causing them to yield NaN values.

Table~\ref{tab:average_lines_tokens} shows an overview of the size of the responses received from the two models considered in this work: \gptturbo and \codegenmonoC. % (from \salesforce).
%As per the earlier discussed in \S\ref{sec:probcomp}, the length of the response from the models affects the performance of the techniques used.

%As such, the responses are \emph{small} with regards to Lines of Code (LOC), with \gptturbo producing, on average, \GPTSolutionsLines~lines long solutions and \salesforce/\codegenmonoC producing \SFSolutionsLines~lines on average. 

%However, the average token length is \emph{large} enough for both models (\GPTSolutionsToken~for \gptturbo and \SFSolutionsToken~for \salesforce/\codegenmonoC) such that it can significantly impact the effectiveness of the techniques used, as we will see later in the section. 

% \begin{table*}[ht!]
%     \centering
%     \caption{Correlation Results for Different Techniques}
%     \label{tab:correlation_results}

%     \begin{tabular}{l r r}
%         \toprule
%         \textbf{Technique} & \multicolumn{2}{c}{\textbf{Model (Pearson Correlation, p-value)}} \\
%         \cmidrule(r){2-3}
%          & \textbf{\gptturbo} & \textbf{\codegenmonoC} \\
%         \midrule
%         \SEOriginal & \SENLGPearsonGPT ~(\SENLGPearsonPValueGPT) & \SENLGPearsonSF ~(\SENLGPearsonPValueSF) \\

%         \SESymbolic & \SESymbolicPearsonGPT ~(\SESymbolicPearsonPValueGPT) & \SESymbolicPearsonSF ~(\SESymbolicPearsonPValueSF) \\
%         \SESymbolicRaw & NaN (NaN) & NaN (NaN) \\
%         \midrule
%         \MIOriginal & - & - \\
  
%         \MISymbolic & \MISymbolicPearsonGPT ~(\MISymbolicPearsonPValueGPT) & \MISymbolicPearsonSF ~(\MISymbolicPearsonPValueSF) \\
%         \MISymbolicRaw & - & - \\
%         \midrule
%         \LLMProbability & \LLMProbabilityPearsonGPT ~(\LLMProbabilityPearsonPValueGPT) & \LLMProbabilityPearsonSF ~(\LLMProbabilityPearsonPValueSF) \\
%         \bottomrule
%     \end{tabular}

% \end{table*}

\begin{table*}[ht!]
    \centering
    \small
    \caption{Correlation Results---Pearson coefficient(p-value)---for Different Techniques on Different Difficulty Problems (\gptturbo: Easy/Medium/Hard; \codegenmonoC: Easy). Statistically significant results are highlighted in bold.}
    \label{tab:correlation_results}
    \begin{tabularx}{\linewidth}{Xrrrr}
        \toprule
        \multirow{2}{*}{\textbf{Technique}} 
        & \multicolumn{3}{c}{\textbf{\gptturbo}} & \textbf{\codegenmonoC} \\
        \cmidrule(lr){2-4} \cmidrule(lr){5-5}
        & \textbf{Easy} & \textbf{Medium} & \textbf{Hard} & \textbf{Easy} \\
        \midrule

        \SEOriginal
          & \SENLGPearsonGPTUpdated (\SENLGPearsonPValueGPTUpdated)  
          & \SENLGPearsonGPTMediumUpdated (\SENLGPearsonPValueGPTMediumUpdated)                                         
          & \SENLGPearsonGPTHardUpdated (\SENLGPearsonPValueGPTHardUpdated)                                        
          & \SENLGPearsonSFUpdated (\SENLGPearsonPValueSFUpdated)    
          \\

        \SESymbolic
          & \textbf{\SESymbolicPearsonGPTUpdated (\SESymbolicPearsonPValueGPTUpdated)}
          & \textbf{\SESymbolicPearsonGPTMediumUpdated (\SESymbolicPearsonPValueGPTMediumUpdated)}
          & \textbf{\SESymbolicPearsonGPTHardUpdated (\SESymbolicPearsonPValueGPTHardUpdated)}
          & \textbf{\SESymbolicPearsonSFUpdated (\SESymbolicPearsonPValueSFUpdated)}
          \\

        \SESymbolicUnif
         & \textbf{\SESymbolicUnifPearsonGPTUpdated (\SESymbolicUnifPearsonPValueGPTUpdated)}
         & \textbf{\SESymbolicUnifPearsonGPTMediumUpdated (\SESymbolicUnifPearsonPValueGPTMediumUpdated)}
         & \textbf{\SESymbolicUnifPearsonGPTHardUpdated (\SESymbolicUnifPearsonPValueGPTHardUpdated)}
         & \textbf{\SESymbolicUnifPearsonSFUpdated (\SESymbolicUnifPearsonPValueSFUpdated)}
         \\

        \midrule

        \MIOriginal
         & \MINLGPearsonGPTUpdated (\MINLGPearsonPValueGPTUpdated)  
         & \MINLGPearsonGPTMediumUpdated (\MINLGPearsonPValueGPTMediumUpdated)                                         
         & \MINLGPearsonGPTHardUpdated (\MINLGPearsonPValueGPTHardUpdated)                                        
         & \MINLGPearsonSFUpdated (\MINLGPearsonPValueSFUpdated)    
         \\

        \MISymbolic
          & \textbf{\MISymbolicPearsonGPTUpdated (\MISymbolicPearsonPValueGPTUpdated)}
          & \textbf{\MISymbolicPearsonGPTMediumUpdated (\MISymbolicPearsonPValueGPTMediumUpdated)}
          & \textbf{\MISymbolicPearsonGPTHardUpdated (\MISymbolicPearsonPValueGPTHardUpdated)}
          & \textbf{\MISymbolicPearsonSFUpdated (\MISymbolicPearsonPValueSFUpdated)}
          \\

        \midrule

        \LLMProbability
          & \LLMProbabilityPearsonGPTUpdated (\LLMProbabilityPearsonPValueGPTUpdated)
          & \LLMProbabilityPearsonGPTMediumUpdated (\LLMProbabilityPearsonPValueGPTMediumUpdated)
          & \LLMProbabilityPearsonGPTHardUpdated (\LLMProbabilityPearsonPValueGPTHardUpdated)
          & \LLMProbabilityPearsonSFUpdated (\LLMProbabilityPearsonPValueSFUpdated)
          \\

        \bottomrule
    \end{tabularx}
\end{table*}

% \begin{table*}[ht!]
%     \centering
%     \caption{Average Number of Lines and Tokens for Solutions Generated by \gptturbo and \salesforce/\codegenmonoC}
%     \label{tab:average_lines_tokens}

%     \begin{tabular}{l r r}
%         \toprule
%         \textbf{Metric} & \multicolumn{2}{c}{\textbf{Model}} \\
%         \cmidrule(r){2-3}
%          & \textbf{\gptturbo} & \textbf{\codegenmonoC} \\
%         \midrule
%         Average Lines of Code & \GPTSolutionsLines & \SFSolutionsLines \\
%         Average Tokens & \GPTSolutionsToken & \SFSolutionsToken \\
%         \bottomrule
%     \end{tabular}
% \end{table*}

\begin{table*}[ht!]
    \centering
    \caption{Average Number of Lines and Tokens for Solutions Generated by \gptturbo and \salesforce/\codegenmonoC for Different Classes of Problems}
    \label{tab:average_lines_tokens}

    \begin{tabular}{l r r r r}
        \toprule
        \multirow{2}{*}{\textbf{Metric}} 
          & \multicolumn{3}{c}{\gptturbo}
          & \multicolumn{1}{c}{\codegenmonoC} \\
        \cmidrule(r){2-4}\cmidrule(r){5-5}
          & \textbf{Easy}
          & \textbf{Medium}
          & \textbf{Hard}
          & \textbf{Easy} \\

        \midrule
        Average Lines of Code
          & \GPTSolutionsLines
          & \GPTSolutionsLinesMedium
          & \GPTSolutionsLinesHard
          & \SFSolutionsLines \\
        Average Tokens
          & \GPTSolutionsToken
          & \GPTSolutionsTokenMedium
          & \GPTSolutionsTokenHard
          & \SFSolutionsToken \\
        \bottomrule
    \end{tabular}
\end{table*}

%Table~\ref{tab:correlation_results} summarises the results obtained for the various techniques considered in this experimental evaluation. 
\subsection{Discussion of Results}\label{sec:results-discussion}


\paragraph{RQ1: Is there a correlation between the uncertainty computed by the proposed techniques and correctness for code generation?}

%The experiments presented in Figure~\ref{tab:correlation_results}, we observe

The results in Table~\ref{tab:correlation_results} allow us to make the following observations:

$\bullet$ There is a \emph{\textbf{strong negative correlation between correctness and the uncertainty}} computed by \SESymbolic and \SESymbolicUnif and a \emph{\textbf{weak negative correlation}} for \MISymbolic.

$\bullet$ \emph{\textbf{Semantic equivalence among the LLM responses is a stronger indicator of correctness than the actual reported log-probabilities}}---while the probabilities assigned by the LLM carry minimal signal (see RQ4), semantic clustering provides significant insight. This is suggested by the fact that \SESymbolicUnif, which assumes uniform probabilities over LLM reponses, exhibits a correlation similar to that of \SESymbolic with length-normalization.
%
This finding is important as \SESymbolicUnif to be applied even when the LLM does not expose log-probabilities, such as in the latest GPT models.

An open question emerging from this is whether we can prove a direct correlation between the number of semantic clusters and correctness. As our current focus is on the relationship between uncertainty and correctness, we leave this investigation for future work.

$\bullet$ \emph{\textbf{The correlation between uncertainty and correctness remains stable across different problem complexities and sizes}} (i.e. easy, medium, hard). Our experiments for \SESymbolic, \SESymbolicUnif and \MISymbolic show no significant degradation with increased size and difficulty for \gptturbo.

$\bullet$ \emph{\textbf{There is little variation in the correlation between uncertainty and correctness across the two LLMs of different sizes}}. The results for \SESymbolic, \SESymbolicUnif, and \MISymbolic exhibit slight degradation on \livecodebench-Easy when transitioning from \gptturbo to \codegenmonoC.  However, the correlation remains weak in both cases.

%In contrast to this, our symbolic execution based adaptation that makes use of \textit{length-normaised response probablities} performed better than its contemporaries.
%For \gptturbo, \SESymbolic obtained a moderate negative correlation score of \SESymbolicPearsonGPT~with a statistically significant p-value of \SESymbolicPearsonPValueGPT. 
%The correlation is expected to be negative here as we are working with uncertainty scores hence a higher uncertainty score should be reflective of lower correctness.
%The story for \salesforce/\codegenmonoC was also the same. 


\paragraph{RQ2: How do the techniques based on semantic equivalence compare against those based on mutual information?}


We observed a stronger correlation between uncertainty and correctness in techniques based on semantic equivalence compared to those relying on mutual information. We hypothesize that this is due to the unambiguous nature of the problems in our dataset, reducing the need to account for aleatoric uncertainty. We leave the exploration of ambiguity in the problem formulation as future work.

\paragraph{RQ3: Are the techniques designed for natural language directly applicable to code generation?}
In all our experiments, \SEOriginal and \MIOriginal yield results that are not statistically significant, highlighting the necessity of adapting these techniques to account for the unique characteristics of code.
%The story for \SEOriginal is similar, but from a different perspective. 
%For almost all of the responses, the natural language based clustering mechanism puts all of the model's responses into a single cluster.
%This meant that the semantic-based clustering mechanism did not get a chance to contribute towards the uncertainty scores.
%Consequently, for both \gptturbo and \salesforce/\codegenmonoC, the results are \emph{not} statistically significant with the respective p-value scores of \SENLGPearsonPValueGPT~and \SENLGPearsonPValueSF, which are both larger than the threshold of \pvalue.



%\paragraph{\textbf{RQ4: Is length normalization useful for code generation?}}
%For \SESymbolicRaw, there \textit{raw response probabilities} underflowed to zero for all responses. 
%This is in accordance with the observation from Kuhn~\etal~\cite{kuhnsemantic} when responses have a large number of tokens, which is the case for our generation queries for both models, as shown in Table~\ref{tab:average_lines_tokens}.
%This, in turn, results in all uncertainty computations to decay to zero, which results in an NaN during the correlation computation.

\paragraph{RQ4: Are the probabilities computed by the LLM sufficient as a proxy for correctness?}
In all our experiments, except for \gptturbo on \livecodebench-Easy, the results are not statistically significant. This suggests that there is no meaningful correlation between correctness and the log-probabilities reported by the LLM.

%For \gptturbo while the \LLMProbability metric shows a \emph{small}, statistically significant (with~\LLMProbabilityPearsonPValueGPT~as the p-value) positive\footnote{The correlation here is positive since a high probability of the top-ranked response from the model should indicate a higher level of correctness.} correlation of~\LLMProbabilityPearsonGPT, for \salesforce/\codegenmonoC the results are \emph{not} statistically significant given the p-value \LLMProbabilityPearsonPValueSF~which is greater than the standard \pvalue. This shows the lack of effectiveness of simply using the model's probablities directly to decide the uncertainty in the model's outputs.


\subsection{Using the Uncertainty Estimation for Correctness}
\label{sec:usability}
We envision uncertainty scores being utilized as part of an abstention policy, similar to the approach described by Abbasi~\etal~\cite{abbasi2024believe}.
In this policy, an uncertainty threshold is set such that LLM responses with scores above the threshold are assumed to be good according to some metric, while those below the threshold are considered bad.
For the latter, the LLM may abstain from presenting these responses to the user.

The primary metric of interest in this work is functional correctness. LLM responses are classified as correct if their correctness score exceeds a predefined threshold and incorrect otherwise. To determine the \emph{uncertainty threshold} for the abstention policy, we select the value that maximizes correctness accuracy.

For our experimental evaluation, we considered the best performing techniques for uncertainty assessment, \SESymbolic and \SESymbolicUnif, and \LLMProbability as the baseline.
We then considered our more successful model~\gptturbo and the \livecodebench dataset, using a correctness score threshold of 90\%. 
%
As discussed in Section~\ref{sec:dataset}, \gptturbo's responses exhibit a low unit test passing rate, leading to an imbalanced dataset where incorrect responses significantly outnumber correct ones. This imbalance risks trivializing the classification task, as an ``all-incorrect'' model would dominate. To address this, we followed standard practice and applied random downsampling, removing a portion of incorrect responses to prevent the model from defaulting to a trivial solution. We split the dataset 50/50 into training and validation.
To mitigate overfitting and reduce bias,  we employed 2-fold cross-validation, alternating between training and validation for each fold.

%The metric we are conserned with in this work is functional correctness. Thus, we label LLM responses as \emph{correct}  if their correctness score exceeds the correctness threhold, or \emph{incorrect} otherwise. Then, we obtain the \emph{uncertainty threshold} to be used by the abstention policy by picking the one that maximizes accuracy. 
%In the experiments summarized in Table~\ref{tab:abstention_metrics}, we used the responses provided by \gptturbo to the \livecodebench dataset, with a correctness threshold of 90\%. To reduce overfitting and prevent biases, we used a 2-fold cross-validation, rotating the role of each fold between training and validation. as explained in Section~\ref{sec:dataset}, these responses exhibit relatively low unit test passing rate resulting in an imbalanced dataset with many more incorrect responses, causing a trivial ``all incorrect'' solution to dominate. Following common practice, we downsampled by at random dropping a certain percentage of points that are causing the trivial ``all-incorrect'' response.


%As it is the practice, if extreme class imbalance (\eg many near-zero correctness cases) causes a trivial ``all incorrect'' solution to dominate, we down-sample or partially discard excessively low-correctness samples to ensure a more balanced dataset (and thus avoid collapsing to a trivial threshold).

%Finally, we fix this threshold and measure performance on the validation set, whereby any response with an uncertainty \emph{below} the learned threshold is deemed sufficiently reliable, and any response above it is \emph{abstained} from the user.

%Concretely, we first combine each response's \emph{correctness score}, a measure of how well the solution aligns with ground truth and \emph{uncertainty score} into a single dataset.
%fraction exceeds a specified cutoff (\eg 0.9) 
% Next, we split the dataset into a training set and a validation set, ensuring we do not overfit our threshold to a single subset.

%Next, we carry out training while employing a k-fold cross validation, rotating the role of each fold between training and validation to ensure robust threshold selection.
%We evaluate the accuracy of these predictions against the ground-truth labels and select the threshold that maximizes this accuracy.


%Table~\ref{tab:abstention_metrics} summarizes results for \SESymbolic, \SESymbolicUnif and \LLMProbability for \gptturbo on the whole \livecodebench dataset, with a correctness threshold of 90\%. We used a cross-validation with $k=2$ folds. 


% More precisely, on the training set, we \emph{sort} responses by ascending uncertainty, then \emph{sweep} through possible threshold values; for each candidate threshold, all responses with uncertainty \emph{below} the threshold are predicted as \texttt{correct}, and the rest as \texttt{incorrect}.

%Table~\ref{tab:abstention_metrics}~summarises the performance of using learned abstention threshold for \SESymbolic, \SESymbolicUnif and \LLMProbability for \gptturbo. 

The evaluation results for the uncertainty threshold are presented in Table~\ref{tab:abstention_metrics}. 
Both \SESymbolic and \SESymbolicUnif achieved high accuracy scores of \SENormAcc~and \SEUnifAcc, respectively, whereas \LLMProbability~performed significantly worse, with an accuracy of only~\LLMProbabilityAcc.

A key advantage of our techniques is their extremely low False Positive (FP) rate, with \SENormFP~for both \SESymbolic and \SESymbolicUnif. 
This indicates that our methods are more conservative in accepting LLM-generated responses, a particularly valuable property for code generation. 
Given the potential safety risks associated with incorrect code, especially when LLM performance is inconsistent, this cautious approach enhances the reliability and safety of LLM-assisted coding.


\begin{table*}[ht!]
    \centering
    \caption{Abstention Metrics for \SESymbolic, \SESymbolicUnif and \LLMProbability with \gptturbo}
    \label{tab:abstention_metrics}

    \begin{tabular}{l r r r}
        \toprule
        \textbf{Technique}
          & \multicolumn{1}{c}{\textbf{Accuracy}}
          & \multicolumn{1}{c}{\textbf{False Positives}}
          & \multicolumn{1}{c}{\textbf{False Negatives}} \\
        \midrule
        \SESymbolic
          & \SENormAcc
          & \SENormFP
          & \SENormFN \\
        \SESymbolicUnif
          & \SEUnifAcc
          & \SEUnifFP
          & \SEUnifFN \\
        \LLMProbability
          & \LLMProbabilityAcc
          & \LLMProbabilityFP
          & \LLMProbabilityFN \\
        \bottomrule
    \end{tabular}
\end{table*}

% In practice, one may also tune the threshold for desired precision/coverage trade-offs or use calibration techniques (e.g.\ isotonic regression)
% so that the ``uncertainty score'' better reflects the actual probability of correctness.


%To implement this, we designed a pipeline that processes uncertainty scores and correctness fractions to derive an optimal abstention threshold. 

%The threshold training phase uses a split of the dataset into training and testing subsets, typically with 80\% of the data for training and 20\% for testing. 
%During training, the pipeline iterates over unique uncertainty score thresholds to identify the threshold that maximizes the accuracy metric \ie the number of correct abstentions made for a given correctness criterion (90\% in our case). 
% This metric accounts for both correct predictions, where uncertainty scores are below the threshold and correctness is above a specified value, and correct abstentions, where uncertainty scores are above the threshold and correctness is below this value. 
% The redefinition ensures that the abstention policy rewards both effective predictions and abstentions, promoting balanced decision-making.

%In the evaluation phase, the learned threshold is applied to the test set, where accuracy and abstention rates are calculated. 
% Accuracy is computed as the ratio of correctly handled samples (whether predicted or abstained) to the total number of samples, while the abstention rate reflects the proportion of samples where predictions were withheld due to high uncertainty. 

%We ran this pipeline on the better performing \gptturbo model and achieved a test-set accuracy of \SEAcc~for the SE scores and \MIAcc~for the MI scores. 

% \begin{figure}[ht]
%     \centering
%     % Subfigure 1
%     \begin{subfigure}[b]{0.45\linewidth}
%         \centering
%         \includegraphics[width=\linewidth]{./figures/average_se_bin_090.png}
%         \caption{Correctness Bin with SE scores: 0-90\% and 90-100\%}
%     \end{subfigure}
%     \hfill
%     % Subfigure 2
%     \begin{subfigure}[b]{0.45\linewidth}
%         \centering
%         \includegraphics[width=\linewidth]{./figures/average_mi_bin_090.png}
%         \caption{Correctness Bin with MI scores: 0-90\% and 90-100\%}
%     \end{subfigure}
    
%     \caption{Average SE/MI Scores for \gptturbo responses, showing an abstention threshold value for achieving 90\% correctness.}
%     \label{fig:mise_scores_bins}
% \end{figure}

% \begin{figure}[ht]
%     \centering
%     % Subfigure 1
%     \begin{subfigure}[b]{0.45\linewidth}
%         \centering
%         \includegraphics[width=\linewidth]{./figures/average_mi_bin_060.png}
%         \caption{Correctness Bin: 0-60\% and 60-100\%}
%     \end{subfigure}
%     \hfill
%     % Subfigure 2
%     \begin{subfigure}[b]{0.45\linewidth}
%         \centering
%         \includegraphics[width=\linewidth]{./figures/average_mi_bin_070.png}
%         \caption{Correctness Bin: 0-70\% and 70-100\%}
%     \end{subfigure}
    
%     % Subfigure 3
%     \begin{subfigure}[b]{0.45\linewidth}
%         \centering
%         \includegraphics[width=\linewidth]{./figures/average_mi_bin_080.png}
%         \caption{Correctness Bin: 0-80\% and 80-100\%}
%     \end{subfigure}
%     \hfill
%     % Subfigure 4
%     \begin{subfigure}[b]{0.45\linewidth}
%         \centering
%         \includegraphics[width=\linewidth]{./figures/average_mi_bin_090.png}
%         \caption{Correctness Bin: 0-90\% and 90-100\%}
%     \end{subfigure}
    
%     \caption{Average MI Scores for Different Correctness Bins. Each plot shows the average MI scores computed for two bins.}
%     \label{fig:mi_scores_bins}
% \end{figure}


\section{Related Work}
\label{sec:related}

\section{Related Work} \label{sec:related}

% \textbf{Adversarial Attack}
\textbf{Attacks on SLAM.} 
%With the rise of machine learning, 
The robustness of computer vision systems is being actively investigated. With the emergence of adversarial images in the digital domain by adding optimized noise directly to images~\cite{szegedy2013intriguing,carlini2017towards}, researchers find that such attacks also exist physically in the real world \cite{eykholt2018robust,song2018physical,zhao2019seeing}. To fill the gap between attacks in the digital and physical worlds, recent studies have demonstrated that attacks on real-world computer vision systems are practical \cite{eykholt2018robust,li2019adversarial,man2020ghostimage,sharif2016accessorize,zhao2019seeing,zhou2018invisible}. However, attacks on traditional computer vision methods such as SLAM are relatively less explored. \cite{yoshida2022adversarial} proposes an attack against the scan matching algorithm in LiDAR-based SLAM, while most SLAMs in AR/VR devices rely on different sensors like RGB/depth cameras and IMUs. \cite{ikram2022perceptual} and \cite{chen2024adversary} mislead visual SLAM by poisoning the images with special patterns, and \cite{wang2021can} causes the camera to fail using infrared light. In our work, we demonstrate attacks on Visual-Inertial SLAM (VI-SLAM) by perturbing the IMU readings, rather than cameras, and showing its impact on XR user experience. 

\textbf{Acoustic Injection Attacks.} Among various physical attacks, acoustic injection attacks are attractive due to their low cost. Son~\etal~\cite{son2015rocking} were the first to introduce acoustic attacks on MEMS gyroscopes, demonstrating how these attacks could lead to sensor denial-of-service and result in drone crashes. WALNUT~\cite{trippel2017walnut} expanded on this by developing output biasing and control attacks that enable precise manipulation of MEMS accelerometer outputs using modulated sound waves. Wang et al.~\cite{wang2017sonic} demonstrated a sonic gun, showcasing the vulnerability of various smart devices (\eg drones and self-balancing vehicles) to acoustic attacks. Tu et al. \cite{tu2018injected} designed side-swing and switching attacks to alter the outputs of MEMS gyroscopes and accelerometers. Furthermore, Ji et al. \cite{ji2021poltergeist} fool the object detectors by applying acoustic attack to the image stabilizers commonly used in modern cameras. However, none of the existing works study the relationship between the acoustic injections and SLAM outputs on recent XR devices. 

% \zijian{Do we need one session about security in AR/VR?}
% \yicheng{TODO}
%\jiasi{cite the AIVR paper (UMass Amherst?) paper is we have not already. They add IMU perturbation but w/o SLAM, iirc} \yicheng{Cited}

\textbf{XR Security and Privacy.} 
%Security and privacy concerns in XR systems have gained significant attention. 
For single-user XR systems, researchers have demonstrated various side-channel attacks to extract sensitive information (\eg keystrokes) through video feeds~\cite{ling2019know}, head movements~\cite{nair2023unique, slocum2023going}, architectural hints~\cite{zhang2023its,shang2020arspy}, power usage~\cite{li2024dangers}, and EM side-channel leakages~\cite{al2021vr}. In multi-user XR systems, Su et al.~\cite{su2024remote} use avatar motion data to infer keystrokes in shared VR environments. Slocum et al.~\cite{slocum2024doesn} reveal vulnerabilities in the shared state frameworks of multi-user AR. Similarly, Lebeck et al.~\cite{lebeck2017securing} highlight risks like deceptive virtual objects and emphasize access control for managing shared physical and virtual spaces. Ruth et al.~\cite{ruth2019secure} further propose a secure multi-user AR framework focusing on content sharing and permissions.
Chandio et al.~\cite{chandio2024stealthy} %introduced a multi-modal spatiotemporal attack that 
simultaneously manipulated visual and inertial sensors to disrupt XR pose estimation. However, their study evaluated the attack using offline datasets and assumed the attacker's capability to manipulate IMU data streams through acoustic means, without real experiments. Ours is the first to demonstrate acoustic injection attacks on recent XR devices, like the Hololens 2, in the real world.
 



\section{Conclusion}
\label{sec:conclusion}
\section*{Conclusion}
This paper aims to enhance our understanding of the computational complexity of computing various Shapley value variants. We found that for various ML models --- including decision trees, regression tree ensembles, weighted automata, and linear regression --- both local and global interventional and baseline SHAP can be computed in polynomial time under HMM modeled distributions. This extends popular algorithms, such as TreeSHAP, beyond their empirical distributional scope. We also establish strict complexity gaps between the various SHAP variants (baseline, interventional, and conditional) and prove the intractability of computing SHAP for tree ensembles and neural networks in simplified scenarios. Overall, we present SHAP as a versatile framework whose complexity depends on four key factors: \begin{inparaenum}[(i)] \item model type, \item SHAP variant, \item distribution modeling approach, \item and local vs. global explanations\end{inparaenum}. We believe this perspective provides deeper insight into the computational complexity of SHAP, paving the way for future work.




%We believe that our framework provides a more intricate understanding of SHAP computation complexity across different models, distributions, and variants, paving the way for further research.

Our work opens promising directions for future research. First, expanding our computational analysis to other SHAP-related metrics, such as asymmetric SHAP~\citep{frye20} and SAGE~\citep{covert2020understanding}, would be valuable. Additionally, we aim to explore more expressive distribution classes and relaxed assumptions beyond those in Section \ref{sec:tractable} while maintaining tractable SHAP computation. Finally, when exact computation is intractable (Section \ref{sec:intractable}), investigating the approximability of SHAP metrics through approximation and parameterized complexity theory~\citep{downey2012parameterized} is an important direction.

%Our work opens several promising avenues for future research on the computational properties of explainable AI methods, with a particular focus on SHAP. First, it would be interesting to broaden the computational analysis conducted in this work to include other popular SHAP-related metrics in the literature, such as asymmetric SHAP \cite{frye20} and SAGE \cite{covert2020understanding}. Also, in the future, we aim to explore more expressive distribution classes and relaxed distributional assumptions—extending beyond those examined in Section \ref{sec:tractable} —that still yield tractable SHAP computation. Finally, when exact computation proves intractable (Section \ref{sec:intractable}), it is worthwhile to theoretically investigate the question of the approximability of computing the SHAP metrics across various configurations, through the lens of approximation and parametrized complexity theory \cite{arora2009computational}.

%This paper aims to deepen our understanding of the computational complexity involved in obtaining different Shapley value variants. We found that for a variety of ML models, including decision trees, tree ensembles for regression, weighted automata, and linear regression models — computing both local and global interventional and baseline SHAP can be done in polynomial time when distributions are modeled by HMMs. This extends the distributional scope of popular algorithms like TreeSHAP, which is limited to empirical distributions. Additionally, we demonstrate a strict complexity gap between SHAP variants, showing that interventional and baseline SHAP can be strictly easier to compute than conditional SHAP. Despite these positive results, we uncovered intractability for various SHAP variants in neural networks and tree ensembles. Finally, we provided generalized complexity relations across SHAP variants. We believe that our framework offers a deeper understanding of the complexity involved in computing SHAP across various variants, models, distributions, as well as in both local and global computations, laying the groundwork for future research.

%
% ---- Bibliography ----
%
% BibTeX users should specify bibliography style 'splncs04'.
% References will then be sorted and formatted in the correct style.
%
\bibliographystyle{splncs04}
\bibliography{ref}

\end{document}
