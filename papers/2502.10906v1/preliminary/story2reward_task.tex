Recent advancements in text-based generative models have showcased the potential for translating textual descriptions into diverse domains such as human-like motion \cite{guo2022generating}, high-fidelity images \cite{ramesh2022hierarchical}, music composition \cite{agostinelli2023musiclm}, and game content generation.
The gaming domain has also benefited significantly from text-based generative approaches. For example, text-conditioned generative models have been applied to specific tasks such as generating \textit{Super Mario Bros} levels \cite{sudhakaran2024mariogpt} or \textit{Sokoban} puzzles \cite{todd2023level}, where models synthesize playable and contextually relevant game content.
Extending beyond level design, recent works have explored generating entire games from textual descriptions \cite{zhang2024text, nasir2024word2world}, thereby transforming abstract narratives into interactive environments and mechanics.
The generated content is evaluated to ensure it aligns with the instructions (i.e., textual conditioning).

Our method is evaluated on the text-to-reward generation task, which aims to bridge narrative-driven descriptions with a trainable reward function. This evaluation checks whether the generated content satisfies the given text instructions, such as ensuring that the player encounters specific conditions during gameplay—for example, encountering \textit{Bat} and \textit{Spider} as required objectives. The two input instructions used in this work are as follows:

\begin{itemize}
\item \textit{"The player needs to obtain a key and escape through the door. To pick up the key, the player encounters \textbf{bat} monsters."}
\item \textit{"The player needs to obtain a key and escape through the door. To pick up the key, the player encounters \textbf{bat} and \textbf{spider} monsters."}
\end{itemize}

\begin{figure*}[!t]
    \centering
    \begin{tabular}{cccccc} % 두 개의 열
        \multicolumn{6}{l}{\small{Instruct: \textit{"The player needs to obtain a key and escape through the door. The player encounters \textbf{bat} monsters."}}} \\
        \includegraphics[width=0.14\textwidth]{figure/iteration/chr_1/Iteration_1/images_366_b4f8ee67309be4f5b202.png} & 
        \includegraphics[width=0.14\textwidth]{figure/iteration/chr_1/Iteration_2/images_800_ff4028bd7e4c3f9e2e88.png} &
        \includegraphics[width=0.14\textwidth]{figure/iteration/chr_1/Iteration_3/images_1246_0f2862d8c3c99064a6b0.png} & 
        \includegraphics[width=0.14\textwidth]{figure/iteration/chr_1/Iteration_4/images_1676_32085ed2fc90709921bc.png} &
        \includegraphics[width=0.14\textwidth]{figure/iteration/chr_1/Iteration_5/images_2137_e17a67e7692cda6ed7a6.png} & 
        \includegraphics[width=0.14\textwidth]{figure/iteration/chr_1/Iteration_6/images_2576_5f208debdd0290034fec.png} \\
        $*y=1$ & $y=2$ & $y=3$ & $*y=4$ & $y=5$ & $*y=6$ \\
        \multicolumn{6}{l}{\small{Instruct: \textit{"The player needs to obtain a key and escape through the door. The player encounters \textbf{bat} and \textbf{spider} monsters."}}} \\
        \includegraphics[width=0.14\textwidth]{figure/iteration/chr_2/Iteration_1/images_363_e715df5c06e16b2726d5.png} & 
        \includegraphics[width=0.14\textwidth]{figure/iteration/chr_2/Iteration_2/images_790_1464abf78b01894bc8ff.png} &
        \includegraphics[width=0.14\textwidth]{figure/iteration/chr_2/Iteration_3/images_1220_9ad138b46f393cfe50cc.png} & 
        \includegraphics[width=0.14\textwidth]{figure/iteration/chr_2/Iteration_4/images_1653_0723a73de8f76dddbc03.png} &
        \includegraphics[width=0.14\textwidth]{figure/iteration/chr_2/Iteration_5/images_2067_03674f69ba4823b80e2f.png} & 
        \includegraphics[width=0.14\textwidth]{figure/iteration/chr_2/Iteration_6/images_2504_77418e5af9250f7a1d35.png} \\
        $y=1$ & $y=2$ & $y=3$ & $y=4$ & $*y=5$ & $*y=6$
    \end{tabular}
    \caption{The generated level images are from the iterative reward generation process based on the given instructions. Each map corresponds to an iteration ($y$), which represents the number of times the reward has been generated and revised by LLMs, and is produced by an agent trained using these LLM-generated reward functions. The asterisk (*) denotes that the generated level satisfies the given instructions.}
    \label{fig:iteration_examples}
\end{figure*}


We measure coherence-based accuracy by evaluating how well the player's experience along the path to the door aligns with the given instructional conditions.
To measure the game entities (keys and enemies) encountered by the player, we adopted a deterministic pathfinding algorithm for evaluation.
We evaluate the generated levels based on how well they align with specific gameplay scenarios, placing emphasis on ensuring a coherent player experience.
