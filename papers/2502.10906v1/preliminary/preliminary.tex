\subsection{Procedural Content Generation via RL}
PCGRL \cite{khalifa2020pcgrl}---a DRL-based content generation method---is a machine learning-based content generation methods.
The generator agent is trained with a hand-crafted reward function and gets a positive reward when the content gets closer to the goal condition.
The benefits of PCGRL stem from its data-free nature and computational efficiency during inference, making it well suited for real-time content generation in games \cite{togelius2011search}.
Originally introduced for 2D level generation in games such as \textit{Sokoban} and \textit{Zelda} \cite{khalifa2020pcgrl}, PCGRL has been expanded through subsequent research. These advancements include support for conditional metric inputs \cite{earle2021learning}, the ability to freeze specific tiles during generation \cite{earle2024scaling}, applications in 3D \textit{Minecraft} level generation \cite{jiang2022learning}, and integration with vectorized multiplayer game skill data \cite{jeon2023raidenv}.

In PCGRL, the level design process is framed as a Markov Decision Process (MDP), where level generation is learned through a trial-and-error approach. At each step \( t \), the agent observes the game level as a state \( s_t \), selects an action \( a_t \) to modify a tile of the level, and transitions to a new state \( s_{t+1} \).
The agent then receives a reward: $r_t = R(s_t, s_{t+1})$, determined by a reward function ($R$) that evaluates the transition between states.
In PCGRL, reward function design requires identifying a computable target based on the given generation objective and appropriately combining different weights.
When multiple sub-functions need to be combined to achieve the desired artifact, designing a reward function in a single attempt is highly challenging. Instead, the reward functions has been refined by humans through multiple attempts, with iterative modifications made based on the observed results.
Therefore, the reward design process involves iteratively combining functions to generate a reward function that produces game-like levels while satisfying the given conditions.




\subsection{The 2D Level Generation Environment}

This study uses the \textit{PCGRL-Jax} \cite{earle2024scaling} environment, a GPU-accelerated implementation of the widely used two-dimensional level generation framework \cite{khalifa2020pcgrl,earle2021learning}.
The selected environment ensures a deterministic reward setting compared to the stochastic reward signal used in the previous study \cite{baek2024chatpcg}, so that the generated reward function is relatively accurately evaluated with a consistently trained policy.
Moreover, the environment provides 17$\times$ faster training time than CPU-based environment for multiple reward generate-and-evaluate iterations.

Each episode begins with a randomly initialized 16 $\times{}$ 16 matrix derived from a predefined tile set. The tile set consists of seven types: \textit{Empty} \includegraphics[height=0.8em]{figure/tiles/empty.png}, \textit{Wall} \includegraphics[height=0.8em]{figure/tiles/solid.png}, \textit{Player} \includegraphics[height=0.8em]{figure/tiles/player.png}, \textit{Bat} \includegraphics[height=0.8em]{figure/tiles/bat.png}, \textit{Scorpion} \includegraphics[height=0.8em]{figure/tiles/scorpion.png}, \textit{Spider} \includegraphics[height=0.8em]{figure/tiles/spider.png}, \textit{Key} \includegraphics[height=0.8em]{figure/tiles/key.png}, and \textit{Door} \includegraphics[height=0.8em]{figure/tiles/door.png}. Each tile type is represented numerically in the matrix to indicate its presence.
The agent can modify five types of tiles, except for two unchangeable \textit{Player} and \textit{Door} tiles, along with the $3 \times 3$ area of unchangeable tiles surrounding the tiles.
The two unchangeable tiles are randomly spawned on the opposite corners of the level in the initial state.
The observation space is defined as a 2D array representing the integer tile numbers, along with a channel the location of the tile to be modified.
The discrete action space includes five actions, each corresponding to the specific tile type that replaces the tile at the modification location.
The reward for the agent is determined by an LLM-generated reward function, implemented using JAX-compatible functions \cite{jax2018github}.



