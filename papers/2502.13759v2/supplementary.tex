%%
%% This is file `sample-authordraft.tex',
%% generated with the docstrip utility.
%%
%% The original source files were:
%%
%% samples.dtx  (with options: `authordraft')
%% 
%% IMPORTANT NOTICE:
%% 
%% For the copyright see the source file.
%% 
%% Any modified versions of this file must be renamed
%% with new filenames distinct from sample-authordraft.tex.
%% 
%% For distribution of the original source see the terms
%% for copying and modification in the file samples.dtx.
%% 
%% This generated file may be distributed as long as the
%% original source files, as listed above, are part of the
%% same distribution. (The sources need not necessarily be
%% in the same archive or directory.)
%%
%% Commands for TeXCount
%TC:macro \cite [option:text,text]
%TC:macro \citep [option:text,text]
%TC:macro \citet [option:text,text]
%TC:envir table 0 1
%TC:envir table* 0 1
%TC:envir tabular [ignore] word
%TC:envir displaymath 0 word
%TC:envir math 0 word
%TC:envir comment 0 0
%%
%%
%% The first command in your LaTeX source must be the \documentclass command.
% \documentclass[sigconf,authordraft]{acmart}
\documentclass[sigconf,screen,anonymous]{acmart}
%% NOTE that a single column version may required for 
%% submission and peer review. This can be done by changing
%% the \doucmentclass[...]{acmart} in this template to 
%% \documentclass[manuscript,screen]{acmart}
%% 
%% To ensure 100% compatibility, please check the white list of
%% approved LaTeX packages to be used with the Master Article Template at
%% https://www.acm.org/publications/taps/whitelist-of-latex-packages 
%% before creating your document. The white list page provides 
%% information on how to submit additional LaTeX packages for 
%% review and adoption.
%% Fonts used in the template cannot be substituted; margin 
%% adjustments are not allowed.

%%
%% \BibTeX command to typeset BibTeX logo in the docs
\AtBeginDocument{%
  \providecommand\BibTeX{{%
    \normalfont B\kern-0.5em{\scshape i\kern-0.25em b}\kern-0.8em\TeX}}}

\settopmatter{printacmref=false} % Removes citation information below abstract
\renewcommand\footnotetextcopyrightpermission[1]{} % removes footnote with conference information in first column
\pagestyle{plain} % removes running headers 
\setcopyright{none}

%% These commands are for a PROCEEDINGS abstract or paper.
% \acmConference[Conference acronym 'XX]{Make sure to enter the correct
%   conference title from your rights confirmation emai}{June 03--05,
%   2018}{Woodstock, NY}
%
%  Uncomment \acmBooktitle if th title of the proceedings is different
%  from ``Proceedings of ...''!
%
%\acmBooktitle{Woodstock '18: ACM Symposium on Neural Gaze Detection,
%  June 03--05, 2018, Woodstock, NY} 
% \acmISBN{978-1-4503-XXXX-X/18/06}


%%
%% Submission ID.
%% Use this when submitting an article to a sponsored event. You'll
%% receive a unique submission ID from the organizers
%% of the event, and this ID should be used as the parameter to this command.
% \acmSubmissionID{xxxx}

%%
%% For managing citations, it is recommended to use bibliography
%% files in BibTeX format.
%%
%% You can then either use BibTeX with the ACM-Reference-Format style,
%% or BibLaTeX with the acmnumeric or acmauthoryear sytles, that include
%% support for advanced citation of software artefact from the
%% biblatex-software package, also separately available on CTAN.
%%
%% Look at the sample-*-biblatex.tex files for templates showcasing
%% the biblatex styles.
%%

%%
%% For managing citations, it is recommended to use bibliography
%% files in BibTeX format.
%%
%% You can then either use BibTeX with the ACM-Reference-Format style,
%% or BibLaTeX with the acmnumeric or acmauthoryear sytles, that include
%% support for advanced citation of software artefact from the
%% biblatex-software package, also separately available on CTAN.
%%
%% Look at the sample-*-biblatex.tex files for templates showcasing
%% the biblatex styles.
%%

%%
%% The majority of ACM publications use numbered citations and
%% references.  The command \citestyle{authoryear} switches to the
%% "author year" style.
%%
%% If you are preparing content for an event
%% sponsored by ACM SIGGRAPH, you must use the "author year" style of
%% citations and references.
%% Uncommenting
%% the next command will enable that style.
%%\citestyle{acmauthoryear}

%%
%% end of the preamble, start of the body of the document source.
\begin{document}

%%
%% The "title" command has an optional parameter,
%% allowing the author to define a "short title" to be used in page headers.
\title{Supplementary Materials: Geolocation with Real Human Gameplay Data: A Large-Scale Dataset and Human-Like Reasoning Framework}

%%
%% The "author" command and its associated commands are used to define
%% the authors and their affiliations.
%% Of note is the shared affiliation of the first two authors, and the
%% "authornote" and "authornotemark" commands
%% used to denote shared contribution to the research.
% \author{Ben Trovato}
% \authornote{Both authors contributed equally to this research.}
% \email{trovato@corporation.com}
% \orcid{1234-5678-9012}
% \author{G.K.M. Tobin}
% \authornotemark[1]
% \email{webmaster@marysville-ohio.com}
% \affiliation{%
%   \institution{Institute for Clarity in Documentation}
%   \streetaddress{P.O. Box 1212}
%   \city{Dublin}
%   \state{Ohio}
%   \country{USA}
%   \postcode{43017-6221}
% }

\author{Anonymous Authors}


%%
%% By default, the full list of authors will be used in the page
%% headers. Often, this list is too long, and will overlap
%% other information printed in the page headers. This command allows
%% the author to define a more concise list
%% of authors' names for this purpose.
% \renewcommand{\shortauthors}{Trovato and Tobin, et al.}

%%
%% The abstract is a short summary of the work to be presented in the
%% article.
% \begin{abstract}
%   A clear and well-documented \LaTeX\ document is presented as an
%   article formatted for publication by ACM in a conference proceedings
%   or journal publication. Based on the ``acmart'' document class, this
%   article presents and explains many of the common variations, as well
%   as many of the formatting elements an author may use in the
%   preparation of the documentation of their work.
% \end{abstract}

%%
%% The code below is generated by the tool at http://dl.acm.org/ccs.cfm.
%% Please copy and paste the code instead of the example below.
%%
% \begin{CCSXML}
% <ccs2012>
%  <concept>
%   <concept_id>00000000.0000000.0000000</concept_id>
%   <concept_desc>Do Not Use This Code, Generate the Correct Terms for Your Paper</concept_desc>
%   <concept_significance>500</concept_significance>
%  </concept>
%  <concept>
%   <concept_id>00000000.00000000.00000000</concept_id>
%   <concept_desc>Do Not Use This Code, Generate the Correct Terms for Your Paper</concept_desc>
%   <concept_significance>300</concept_significance>
%  </concept>
%  <concept>
%   <concept_id>00000000.00000000.00000000</concept_id>
%   <concept_desc>Do Not Use This Code, Generate the Correct Terms for Your Paper</concept_desc>
%   <concept_significance>100</concept_significance>
%  </concept>
%  <concept>
%   <concept_id>00000000.00000000.00000000</concept_id>
%   <concept_desc>Do Not Use This Code, Generate the Correct Terms for Your Paper</concept_desc>
%   <concept_significance>100</concept_significance>
%  </concept>
% </ccs2012>
% \end{CCSXML}

% \ccsdesc[500]{Do Not Use This Code~Generate the Correct Terms for Your Paper}
% \ccsdesc[300]{Do Not Use This Code~Generate the Correct Terms for Your Paper}
% \ccsdesc{Do Not Use This Code~Generate the Correct Terms for Your Paper}
% \ccsdesc[100]{Do Not Use This Code~Generate the Correct Terms for Your Paper}

%%
%% Keywords. The author(s) should pick words that accurately describe
%% the work being presented. Separate the keywords with commas.
% \keywords{Do, Not, Us, This, Code, Put, the, Correct, Terms, for,
%   Your, Paper}

%% A "teaser" image appears between the author and affiliation
%% information and the body of the document, and typically spans the
%% page.
% \begin{teaserfigure}
%   \includegraphics[width=\textwidth]{sampleteaser}
%   \caption{Seattle Mariners at Spring Training, 2010.}
%   \Description{Enjoying the baseball game from the third-base
%   seats. Ichiro Suzuki preparing to bat.}
%   \label{fig:teaser}
% \end{teaserfigure}

% \received{20 February 2007}
% \received[revised]{12 March 2009}
% \received[accepted]{5 June 2009}

%%
%% This command processes the author and affiliation and title
%% information and builds the first part of the formatted document.
\maketitle

\appendix


\section{Data Collection Platform User Interface}
To comply with the double-blind review policy, we did not include the URL of our active website in the paper. Instead, we presented selected interface screenshots of the website in Figure ~\ref{fig:UI} while obscuring any elements that could potentially compromise the anonymity required by the policy. 

\begin{figure}
    \centering
    \includegraphics[width=0.8\linewidth]{Images/UI.pdf}
    \caption{UI of Gameplay. UI components that could potentially compromise the double-blind review policy were masked.}
    \label{fig:UI}
\end{figure}


\section{Detail of GeoCoT}
\label{details}
We present the detailed prompt of our GeoCoT process below:



\textit{$\bullet$ \textbf{Question1:} Are there prominent natural features, such as specific types of \textcolor{cyan}{vegetation}, \textcolor{cyan}{landforms} (e.g., \textcolor{orange}{mountains}, \textcolor{orange}{hills}, \textcolor{orange}{plains}), or \textcolor{cyan}{soil characteristics}, that provide clues about the geographical\textcolor{green}{region}?
 \noindent $\bullet$ \textbf{Question2:} Are there any culturally, historically, or architecturally significant \textcolor{cyan}{landmarks}, \textcolor{cyan}{buildings}, or \textcolor{cyan}{structures}, or are there any \textcolor{cyan}{inscriptions} or \textcolor{cyan}{signs} in a specific
    language or script that could help determine the \textcolor{green}{country} \textcolor{green}{or region}?
\noindent $\bullet$  \textbf{Question3:} Are there distinctive road-related features, such as \textcolor{cyan}{traffic direction} (e.g., \textcolor{orange}{left-hand or right-hand driving}), specific types of \textcolor{cyan}{bollards}, unique utility \textcolor{cyan}{pole designs}, or \textcolor{cyan}{license plate}colors and styles, which \textcolor{green}{countries} are known to have these characteristics?
 \noindent $\bullet$ \textbf{Question4:} Are there observable \textcolor{cyan}{urban} or \textcolor{cyan}{rural markers} (e.g., \textcolor{orange}{street signs}, \textcolor{orange}{fire hydrants guideposts}) , or other
    \textcolor{cyan}{infrastructure} elements, that can provide more specific information about the \textcolor{green}{country or city}?
\noindent $\bullet$  \textbf{Question5:} Are there identifiable patterns in \textcolor{cyan}{sidewalks} (e.g., \textcolor{orange}{tile shapes}, \textcolor{orange}{colors}, or \textcolor{orange}{arrangements}), \textcolor{cyan}{clothing styles} worn by people, or other culturally specific details that can help narrow down the \textcolor{green}{city or area}?}


\textit{Let's think step by step. Based on the question I provided, locate the location of the picture as accurately as possible. Identify the continent, country, and city, and summarize it into a paragraph. 
For example: the presence of tropical rainforests, palm trees, and red soil indicates a tropical climate... Signs in Thai, right-side traffic, and traditional Thai architecture further suggest it is in Thailand... Combining these clues, this image was likely taken in a city in \textcolor{red!70!black}{Bangkok, Thailand, Asia.}}

Here, \textcolor{cyan}{cyan} highlights potential clues within the image to help the model infer geographic locations. \textcolor{green}{Green} defines the geographic scope inferred from the clues, such as a region, country, or city. \textcolor{orange}{Orange} provides detailed descriptions of the cyan clues, enhancing the model's understanding. \textcolor{red!70!black}{Red} specifies the expected output format, including city, country, and continent.
% \textit{Let's think step by step. Based on the question I provided, locate the location of the picture as accurately as possible. Identify the continent, country, and city, and summarize it into a paragraph. For example: the presence of tropical rainforests, palm trees, and red soil indicates a tropical climate, most likely located in Southeast Asia. The signs are in Thai and the hydrant is a green circle, indicating that it is in Thailand. Right-side traffic, cylindrical bollards with blue markings, and license plates with black lettering on a white background meet Thai standards. Traditional Thai architecture, such as pitched roofs and wooden structures, further points to a specific location in Thailand. Square gray sidewalk tiles and non-motorized lanes marked with red asphalt are specific urban design features that help narrow down the location. Combining tropical vegetation, Thai-language road signs, traditional architecture, and specific urban design features, this image was most likely taken in a city in \textcolor{red!70!black}{Bangkok, Thailand, Asia.}}


%%
%% The acknowledgments section is defined using the "acks" environment
%% (and NOT an unnumbered section). This ensures the proper
%% identification of the section in the article metadata, and the
%% consistent spelling of the heading.
% \begin{acks}
% To Robert, for the bagels and explaining CMYK and color spaces.
% \end{acks}

\section{Human Annotation Example}
Below we show an example of human annotated ground truth to demonstrate the annotation process, criteria, and the reasoning behind the annotations, where clues are shown in \textcolor[HTML]{2f6eba}{blue}, correct predictions in \textcolor[HTML]{628443}{green}.

\textit{The image shows a rural residential area with dense trees and expansive green lawns. The terrain is flat, and the \textcolor[HTML]{2f6eba}{soil is reddish-brown}, which matches the temperate climate of central Europe, particularly rural areas of France. The architectural style of the house is distinctive: a \textcolor[HTML]{2f6eba}{red-tiled sloped roof, yellow walls, and solar panels}, reflecting the region's focus on renewable energy, a common feature in French countryside homes. The \textcolor[HTML]{2f6eba}{red mailbox} at the gate is a hallmark of rural French residences. The design of the fences and modern \textcolor[HTML]{2f6eba}{gates aligns} with typical styles in the French countryside. The house design and surrounding natural environment suggest a rural European region. Based on the architectural style, natural landscape, and street elements, the image was most likely taken in \textcolor[HTML]{628443}{Aumont, France, Europe}.}



%%
%% The acknowledgments section is defined using the "acks" environment
%% (and NOT an unnumbered section). This ensures the proper
%% identification of the section in the article metadata, and the
%% consistent spelling of the heading.
% \begin{acks}
% To Robert, for the bagels and explaining CMYK and color spaces.
% \end{acks}

%%
%% The next two lines define the bibliography style to be used, and
%% the bibliography file.
% \bibliographystyle{ACM-Reference-Format}
% \bibliography{sample-base}

%%
%% If your work has an appendix, this is the place to put it.
% \appendix

% \section{Research Methods}

% \subsection{Part One}

% Lorem ipsum dolor sit amet, consectetur adipiscing elit. Morbi
% malesuada, quam in pulvinar varius, metus nunc fermentum urna, id
% sollicitudin purus odio sit amet enim. Aliquam ullamcorper eu ipsum
% vel mollis. Curabitur quis dictum nisl. Phasellus vel semper risus, et
% lacinia dolor. Integer ultricies commodo sem nec semper.

% \subsection{Part Two}

% Etiam commodo feugiat nisl pulvinar pellentesque. Etiam auctor sodales
% ligula, non varius nibh pulvinar semper. Suspendisse nec lectus non
% ipsum convallis congue hendrerit vitae sapien. Donec at laoreet
% eros. Vivamus non purus placerat, scelerisque diam eu, cursus
% ante. Etiam aliquam tortor auctor efficitur mattis.

% \section{Online Resources}

% Nam id fermentum dui. Suspendisse sagittis tortor a nulla mollis, in
% pulvinar ex pretium. Sed interdum orci quis metus euismod, et sagittis
% enim maximus. Vestibulum gravida massa ut felis suscipit
% congue. Quisque mattis elit a risus ultrices commodo venenatis eget
% dui. Etiam sagittis eleifend elementum.

% Nam interdum magna at lectus dignissim, ac dignissim lorem
% rhoncus. Maecenas eu arcu ac neque placerat aliquam. Nunc pulvinar
% massa et mattis lacinia.

\end{document}
\endinput
%%
%% End of file `sample-authordraft.tex'.
