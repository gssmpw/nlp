\section{Related work}
\subsection{Image Geolocation Task}
\label{related1}
% Image Geolocalization is an important task in computer vision~\cite{zhu2023difftraj,zhu2023synmob,zhu2024controltraj}, 
% spatial data mining~\cite{zhao2017incorporating,zhao2016exploring,zhang2023promptst,han2023mitigating}, and

% GeoAI~\cite{zhao2017modeling,zhao2022multi,zhang2023mlpst,zhang2023autostl}. 

Image geolocation refers to determining the corresponding location of a given image, a crucial task in computer vision~\cite{zhu2023difftraj,zhu2023synmob,zhu2024controltraj}, spatial data mining~\cite{zhao2017incorporating,zhao2016exploring,zhang2023promptst,han2023mitigating}, and GeoAI~\cite{zhao2017modeling,zhao2022multi,zhang2023mlpst,zhang2023autostl}.
Previous research in image geolocalization could be primarily classified into two approaches: classification-based methods and retrieval-based methods.
(1) \textit{Classification-based methods} partition most regions of the Earth into multiple grid cells.
Models are trained to classify each image into the correct cell~\cite{clark2023, pramanick2022, muller2018, seo2018, weyand2016}. 
The center coordinates of each cell are used as the predicted values. 
However, due to the limited number of cells, the granularity of the predicted values is coarse, preventing precise predictions.
(2) \textit{Retrieval-based methods} establish a database of geographic images with GPS coordinates. 
For a given input image, these methods retrieve the most similar image from the dataset and use its coordinates as the predicted location~\cite{zhu2022,muller2018, zhang2023,workman2015wide,liu2019lending,Zhou_2024}. 
However, constructing a comprehensive global-level image database is clearly impractical.




\subsection{Geolocation Dataset}

Existing geolocation datasets primarily originate from web-scraped or street-view images that have not been human-validated, raising concerns about their quality for effectively evaluating geolocation capabilities. 
For instance, datasets derived from web scraping, such as YFCC100M~\cite{theiner2022} and Im2GPS3K~\cite{vo2017revisiting}, include a significant proportion of images depicting food, art, pets, and personal photographs. 
These types of images are often weakly localizable or entirely non-localizable~\cite{YFCC-Val26k-Interpretable}.
Street-view datasets also face limitations, such as restricted geographic coverage~\cite{astruc2024openstreetview}; for example, \cite{mirowski2019streetlearn-navigation}'s work includes data from only three cities in the United States.
% Online platforms typically associate a location with user-uploaded images, but the coordinates provided are often not strictly accurate. 
Furthermore, dataset collection processes often introduce biases. 
For instance, some commonly used platforms are inaccessible in certain countries, resulting in uneven geographic representation. 
Additionally, the difficulty of individual geolocation tasks varies widely within these datasets, but this aspect has not been comprehensively evaluated. 
For example, images taken at prominent landmarks are relatively easy to geolocate, while others offer no clear hints and are highly challenging~\cite{astruc2024openstreetview}.
These limitations undermine the reliability of current geolocation benchmarks.


% Street view datasets 通常视觉直观上表现出 strongly localizable, 这些图像一般包含了 rich geographical cues. 但是 existing street view datasets 通常有limited 覆盖范围,例如 ~\cite{mirowski2019streetlearn-navigation,nn-UCF-GSV-Dataset-2014}'s work only include 3 cities in US. 同时,现存的所有geolocation datasets, 都缺少大规模的人工验证,来可靠的验证其 localizable and reasoning value.

% From an intuitive perspective, 
% Moreover, all existing geolocation datasets lack extensive human validation, which is essential for reliably assessing their localizability and reasoning value.


\begin{figure*}[htb]
    \centering
    \includegraphics[width=0.8\linewidth]{Images/player.pdf}
    \caption{Performance of game players of different levels in mainstream countries.
    Experts are defined as the top 15\% in performance scores, while beginners are those in the bottom 15\%.
    }
    \label{fig:player_score}
\end{figure*}

\subsection{Large Vision Language Models}

LLMs have exhibited extraordinary emergent abilities by scaling up data and model sizes, notably including instruction following \cite{li2023llama, dai2024instructblip}, in-context learning \cite{brown2020language}, and Chain of Thought (CoT) \cite{kojima2022large}.
Building on these emergent capabilities, significant research efforts have focused on enhancing cross-modality understanding and reasoning capabilities. 
Numerous studies have been conducted on various aspects of LVMs, encompassing structural design \cite{liu2024visual, cai2023benchlmm, liu2024sphinx}, data construction \cite{laion2023gpt4v, zhao2024cobra}, training strategies \cite{zhao2023mamo,mckinzie2024mm1, lu2024deepseek}, evaluations~\cite{bithel2023evaluating}, and the development of lightweight LVMs \cite{ zhu2024comprehensive}.
Additionally, the robust capabilities of LVMs have been applied to other fields, such as medical image understanding \cite{tinyllava, pmc-vqa,zhang2024universal} and document parsing \cite{ye2023mplug, liu2024textmonkey}. 
Furthermore, the development of multi-modal agents has advanced real-world applications, including embodied agents \cite{huang2023embodied, peng2023kosmos} and GUI agents \cite{yang2023appagent, li2024appagent,song2024mmac}.
However, the reasoning capabilities of LVMs in geolocation tasks remain underexplored. 
One of the primary reasons for this limitation is the lack of high-reasoning-value geographic data. 
% Our work addresses this gap by introducing a large-scale, human-validated dataset, offering a robust foundation for advancing LVMs’ exploration in geolocation tasks.




% \subsection{Vision Chain of Thought}
% The concept of Chain of Thought (CoT) was first introduced in the field of natural language processing, where explicit reasoning steps are modeled as intermediate representations to improve complex problem-solving~\cite{wei2022chain,fang2024transformer}. 
% CoT has demonstrated significant improvements in tasks such as arithmetic reasoning, commonsense inference, and multi-step question answering~\cite{dasgupta2022language}.
% Building on this foundation, recent research has adapted CoT reasoning to the realm of LVMs to enhance interpretability and robustness. 
% For instance, \cite{Wu2023The} proposed a Description then Decision strategy to improve task decomposition and step-by-step processing. 
% Furthermore, \cite{Ge2023Chain} advanced Vision CoT techniques to improve generalization in image classification tasks.
% In this work, we extend the CoT framework to the geolocation domain by introducing GeoCoT.