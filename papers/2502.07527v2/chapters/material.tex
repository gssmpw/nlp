\section{Material tasks}\label{sec:material}
To evaluate the capabilities of \ourM{} for material generation, it is prompted to generate material's compositions in both unconditional and conditional way. For unconditional generation, the model is prompted with a special token indicating the start of material (i.e., \material{}) and is expected to generate the composition of the material (Section \ref{sec:uncondition_mat}). For conditional generation, the model is prompted to generate material formula and structure under specific human instructions, including: (1) Composition to material generation (Section \ref{sec:comp_to_mat}); (2) Bulk modulus to material generation (Section \ref{sec:bulk_to_mat}). 

Crystal structures are fundamental to determining the physical, chemical, and mechanical properties of materials. Complementing the sequence outputs of \ourM{}, we finetune \ourM{} into a specialized model, \ourM{}-Mat3D, which predicts crystal structures from the generated chemical formulas (see Section \ref{sec:material_structure_predictor}). After generating material formulas and space groups with \ourM{}, we then utilize \ourM{}-Mat3D to convert them into crystal structures for further evaluation and practical application.


% After generating the chemical formula of a material, we use a dedicated fine-tuned \ourM{} to generate its crystal structures (referred to as \ourM{}-Mat3D), which are then evaluated for their accuracy and stability (see Section \ref{sec:material_structure_predictor}).



\subsection{Unconditional material generation}\label{sec:uncondition_mat}
The model is tasked with generating materials with arbitrary compositions. The input to \ourM{} is \material{}, and it produces material compositions with a specified space group. An example is provided below,
\begin{example}
\noindent\textbf{Instruction}: \material{} \\
\noindent\textbf{Response}: \material{} A A B B B $\langle$sg12$\rangle$\ematerial{}
\end{example}
where A, B refer to elements and $\langle$sg12$\rangle$ denotes the space group.

We evaluated the SMACT validity of the generated materials. % Furthermore, we used the dedicated fine-tuned \ourM{} to autoregressively predict the crystal structures of a randomly chosen subset of valid compositions (see Section \ref{sec:material_structure_predictor}). 
Furthermore, we utilized \ourM{}-Mat3D to predict the crystal structures of a randomly selected subset of valid compositions. 
The energy above hull (abbreviated as ehull) of the predicted structures was then evaluated using MatterSim~\cite{yang2024mattersim}.
% autoregressive model~\cite{antunes2024crystalstructuregenerationautoregressive,gruver2024finetunedlanguagemodelsgenerate} to predict the structures of 
The distribution of ehull is shown in Fig. \ref{fig:mat_uncon}. We also assessed the ratio of stable materials, defining a generated material as stable if its ehull$<0.1$eV/atom. The results are presented in Table \ref{tab:uncon_mat}. It is evident that as the model size increases, the SMACT validity and stability of the generated materials improve.
\begin{table}[!htbp]
    \centering
    \begin{tabular}{lcc}
        \toprule
        Model & SMACT (\%) & Stability (\%) \\
        \midrule
        \ourM{} (1B) & 49.20 & 10.12\\ %5.70 \\
        \ourM{} (8B) & 63.42 & 12.47\\ %10.80\\
        \ourM{} (8x7B) & 66.07 & 17.86\\%17.33\\
        \bottomrule
    \end{tabular}
    \caption{The SMACT validity and stability (with ehull$<0.1$eV/atom) for unconditional material generation. The distribution of ehull for the generated materials is illustrated in Fig. \ref{fig:mat_uncon}.}
    \label{tab:uncon_mat}
\end{table}


\subsection{Composition to material generation}\label{sec:comp_to_mat}
The model is tasked with generating materials containing specific elements:
\begin{example}
\noindent\textbf{Instruction}: \texttt{Build a material that has Li, Ti, Mn, Fe, O}\\
\noindent\textbf{Response}: \material{} Li Li Li Li Ti Ti Ti Mn Mn Fe Fe Fe O O O O O O O O O O O O O O O O 
 $\langle$sg8$ \rangle$\ematerial{}
\end{example}
We evaluated the SMACT validity, stability, novelty, and precision of the generated materials. The novelty is measured as the ratio of unique generated materials that are not present in our instruction tuning data. The composition precision is calculated as 
\begin{equation}
{\rm composition\ precision} = \frac{1}{N} \sum_{i=1}^N \frac{\lvert E_{pi}\cap E_{gi}\rvert}{\lvert E_{pi}\rvert},
\label{eqn:mat_compt_to_mat_precision}
\end{equation}
where $E_{pi}$ and $E_{gi}$ stand for the sets of elements in the $i$-th prompt and corresponding generated material respectively.

The results are demonstrated in Table \ref{tab:mat_comp_to_mat}, and the distribution of ehull is depicted in Figure \ref{fig:mat_comp_to_mat_ehull}. Table \ref{tab:mat_comp_to_mat} shows a significant improvement in SMACT validity scores due to instruction tuning compared to unconditional generation. The precision for all three models is close to 100\%, indicating their strong capability to follow language instructions for generating material formulas with expected elements. Additionally, the high novelty demonstrates the models' generative abilities. Furthermore, stability improves with model size, highlighting their scalability. Figure \ref{fig:mat_comp_to_mat_ehull} illustrates this more clearly: as model size increases, the ehull distribution shifts closer to zero, indicating that more materials have lower energy and are in a more stable state. 
\begin{table}[!htbp]
\centering
\begin{tabular}{lcccc}
\toprule
Model & SMACT (\%) & Stability (\%) & Precision (\%) & Novelty (\%)\\
\midrule
%\ourM{} (1B) & 79.38 & 30.59 & 97.95 & 97.13 \\
%\ourM{} (8B) & 83.36 & 34.90 & 98.44 & 95.51 \\
%\ourM{} (8x7B) & 81.56 & 50.56 & 97.68 & 94.83 \\
\ourM{} (1B) & 79.38 & 31.56 & 97.95 & 97.13 \\
\ourM{} (8B) & 83.36 & 35.56 & 98.44 & 95.51 \\
\ourM{} (8x7B) & 81.56 & 36.46 & 97.68 & 94.83 \\
\bottomrule
\end{tabular}
\caption{The SMACT validity, stability, precision, and novelty for composition to material generation.}
\label{tab:mat_comp_to_mat}
\end{table}

\begin{figure}
    \centering
    \subfigure[\ourM{} (1B)]{
    \includegraphics[width=0.45\linewidth]{figures/mat_comp_to_mat_ehull_1b.pdf}
    }%
    \subfigure[\ourM{} (8B)]{
    \includegraphics[width=0.45\linewidth]{figures/mat_comp_to_mat_ehull_8b.pdf}
    }%
    \vskip\baselineskip
    \subfigure[\ourM{} (8x7B)]{
    \includegraphics[width=0.45\linewidth]{figures/mat_comp_to_mat_ehull_8x7b.pdf}
    }
    \subfigure[Accumulated distribution]{
    \includegraphics[width=0.45\linewidth]{figures/mat_comp_to_mat_accumulate_ehull.pdf}
    }
    \caption{Energy above hull (ehull) distribution for composition to material generation.}
    \label{fig:mat_comp_to_mat_ehull}
\end{figure}


% \subsection{Band gap to material generation}
% The model is prompted to generate materials with specified band gap value (e.g., \textbf{Generate a material with a band gap value 10.0}). Following above task, we evaluate the SMACT and ehull in Table \ref{tab:bandgap_to_mat}. To evaluate how the generated materials follow the instruction, we measure the band gap of the generated materials.
% \begin{table}[!htbp]
%     \centering
%     \begin{tabular}{c|c}
%         \hline
%         Model & SMACT \\
%         \hline
%         \ourM{} 1B & 78.44\%\\
%         \ourM{} 8B & 85.62\%\\
%         \ourM{} 8x7B & 87.01\%\\
%         \hline
%     \end{tabular}
%     \caption{Band gap to material generation}
%     \label{tab:bandgap_to_mat}
% \end{table}

\subsection{Bulk modulus to material generation}\label{sec:bulk_to_mat}
The bulk modulus of a substance is a measure of the resistance of a substance to bulk compression. As a proof-of-concept, the model is prompted to generate materials with specified bulk modulus:
\begin{example}
\noindent\textbf{Instruction}: \texttt{Construct the composition for a material with a specified bulk modulus of 86.39 GPa.}\\
\noindent\textbf{Response}: \material{} Se Se Pd Sc 
 $\langle$sg164$ \rangle$\ematerial{}
\end{example}
We evaluated the SMACT validity, stability, novelty, and precision of the generated materials. Precision is defined as the ratio of generated materials whose bulk modulus is within 10\% of the instructed value, compared to all generated materials. 

The results in Table \ref{tab:bulk_to_mat} indicate improved SMACT validity and stability as the model scales. Figure \ref{fig:mat_bulk_to_mat_ehull} depicts the distribution of ehull for the generated materials, showing a shift closer to zero with increasing model size. 


Further, to demonstrate how \ourM{} follows the instruction to generate materials with expected bulk modulus, we depict the distribution of the bulk modulus of generated materials under the instructions in Figure \ref{fig:mat_bulk_to_mat} where the $x$-axis denotes the bulk modulus in the instruction prompt and the $y$-axis denotes the predicted bulk modulus of the generated materials calculated by MatterSim. We can see that, as the model scales, the distribution aligns more closely with the ideal linear diagonal. 

To assess how many novel materials \ourM{} can generate, we prompted the model with a single instruction and allowed it to produce up to 1,000,000 material formulas. We then plotted the count of novel material formulas against the total number generated. Novel materials are defined as those passing the SMACT validity check, not present in the instruction tuning data, and not previously generated. Figure \ref{fig:mat_novelty} shows that the number of novel materials increases with the total generated. Even at 1 million generated materials, novel ones continue to appear, highlighting the model's strong generative capability.

\begin{table}[!htbp]
\centering
\begin{tabular}{lccccc}
\toprule
Model & SMACT (\%) & Stability (\%) & Precision (\%) & Novelty (\%) \\
\midrule
%\ourM{} (1B) & 86.76 & 36.54 & 16.88 & 52.38 \\
%\ourM{} (8B) & 87.21 & 53.33 & 25.53 & 36.31 \\
%\ourM{} (8x7B) & 94.75 & 52.86 & 35.77 & 32.42 \\
\ourM{} (1B) & 86.76 & 39.34 & 40.00 & 52.38 \\
\ourM{} (8B) & 87.21 & 52.81 & 44.06 & 36.31 \\
\ourM{} (8x7B) & 94.75 & 53.60 & 44.62 & 32.42 \\
\bottomrule
\end{tabular}
\caption{The SMACT validity, stability, precision, and novelty of generated materials conditioned on bulk modulus.}
\label{tab:bulk_to_mat}
\end{table}

\begin{figure}[!htbp]
    \centering
    \includegraphics[width=0.8\linewidth]{figures/mat_bulk_to_mat.pdf}
    \caption{Distribution of predicted bulk modulus values for generated materials. The $x$-axis represents the input bulk modulus values from the instructions, while the $y$-axis shows the predicted values for the generated molecules calculated by MatterSim. }
    \label{fig:mat_bulk_to_mat}
\end{figure}

\begin{figure}[!htbp]
    \centering
    \includegraphics[trim=5cm 3cm 5cm 1cm, clip, width=\linewidth]{figures/NatureLM_material_showCase.pdf}
    \caption{Two cases with bulk modulus values near 400 GPa (evaluated via DFT) were identified. The chemical formulas, space groups, energy above the hull (e\_hull), and bulk modulus values obtained from MatterSim and DFT calculations are provided.}
    \label{fig:bulk_caseStudy}
\end{figure}

Materials with an ultra-high bulk modulus are highly sought after due to their exceptional stiffness and incompressibility, making them indispensable for applications in extreme environments, such as aerospace, industrial tooling, and advanced engineering. To evaluate the potential of \ourM{} in generating materials with high bulk modulus, we conducted a detailed analysis of the generated compositions targeted at a bulk modulus of 400 GPa. From the generated outputs, we manually identified cases where MatterSim~\cite{yang2024mattersim} predicted bulk modulus values within a 5 GPa range of the target. Two such cases were selected for further validation using density functional theory (DFT) calculations (see Fig. \ref{fig:bulk_caseStudy}). The DFT results revealed bulk modulus values of 390 GPa and 394 GPa, which closely align with the target value of 400 GPa. 

Beyond achieving the bulk modulus target, the two generated structures were confirmed to be novel compared to those available in the Materials Project database. This novelty underscores \ourM{}'s potential for discovering new materials with exceptional mechanical properties, thereby broadening the scope of material design and innovation.


\subsection{\ourM{}-Mat3D: a crystal structure predictor for materials}\label{sec:material_structure_predictor}
Crystal material structure prediction (CSP) is a critical problem. Previous works apply random search, particle swarm algorithm, and a few others search algorithms to look for stable crystal structures. More recently, generative models like VAE \cite{cdvae}, diffusion \cite{zeni2023mattergen} and flow matching based methods \cite{flowmm} are applied for such 3D structure generation.   There is also a growing trend towards using Large Language Models (LLMs) for crystal structure generation, which can autoregressively generate the structures \cite{gruver2024finetunedlanguagemodelsgenerate,antunes2024crystalstructuregenerationautoregressive,flowmm}. We fine-tune \ourM{} to act as a crystal structure prediction module that generates 3D structures in an autoregressive manner. 

Using \ourM{} for structure prediction is particularly meaningful because it aligns the sequential modeling capacity of LLMs with the sequential representation of crystal structures. This congruence allows the model to capture the intricate dependencies and patterns inherent in material structures, potentially leading to more accurate and efficient generation of stable crystal configurations.

We represent materials and their 3D structures as 1D sequences in three steps:
\begin{enumerate}
\item {\em Flatten the chemical formula}: Repeat each element according to its count (e.g., \texttt{A2B3} becomes \texttt{A A B B B}). 
\item {\em Add space group information}: Append special tokens $\langle$sg$\rangle$ and $\langle$sg{N}$\rangle$, where \texttt{N} is the space group number.
\item {\em Include coordinate information}: Use the token $\langle$coord$\rangle$ to indicate the start of coordinates. Flatten the lattice parameters into nine float numbers and the fractional atomic coordinates into sequences of float numbers. Numbers are retained to four decimal places and tokenized character-wise (e.g., \texttt{-3.1416} as \texttt{- 3 . 1 4 1 6}).%, using $\langle$cs$\rangle$ as a separator.
\end{enumerate}
For example, the sequence for a material \texttt{A2B3} with space group number 123 is:
\begin{example}
A A B B B $\langle$sg$\rangle$ $\langle$sg123$\rangle$ $\langle$coord$\rangle$ {9 float numbers for lattice} {15 float numbers for atoms}
\end{example}


We collect data from Materials Project \cite{materialsproject}, NOMAD \cite{nomad} and OQMD \cite{oqmd2013,oqmd2015} as our training data which are widely used database for materials with structure information, and test on MP-20, Perov-5 and MPTS-52 following previous works \cite{cdvae,diffcsp,flowmm}. Specially, we remove duplications in the merged training data and remove all the data that appear in the test set in these benchmarks. The final training data contains about 6.5M samples after deduplication and removal of the test set. After training, we also finetuned the model on the training set for each benchmark to mitigate the different distributions between our training data and the benchmark data. We evaluate the match rate of the generated material structures and compare to CDVAE \cite{cdvae}, DiffCSP \cite{diffcsp} and FlowMM \cite{flowmm}. The results are shown in Table \ref{tab:mat_struct_pred}. Experiment results show that our sequence based auto-regressive method achieves comparable or best
performance on MP-20 and MPTS-52 compared to other methods. We will use this for material structure generation in our following experiments. In future work, we will leverage and combine with more advanced methods like MatterGen \cite{zeni2023mattergen} for structure generation.

\begin{table}[!htbp]
\centering
\begin{tabular}{lcccccc}
        \hline
         & \multicolumn{2}{c}{Perov-5} & \multicolumn{2}{c}{MP-20} & \multicolumn{2}{c}{MPTS-52} \\
        & MR (\%)  & RMSE & MR (\%) & RMSE & MR (\%) & RMSE\\
        \hline
        CDVAE  & 45.31 & 0.1138 & 33.90 & 0.1045 & 5.34 & 0.2106 \\
        DiffCSP & 52.02 & \textbf{0.0760} & 51.49 & 0.0631 & 12.19 & 0.1786 \\
        FlowMM & \textbf{53.15} & 0.0992 & 61.39 & 0.0566 & 17.54 & 0.1726 \\
        %MatterAR & \textbf{62.58} & \textbf{0.0415} & \textbf{34.38} & \textbf{0.0889} & 52.23 & 0.0815\\
        \ourM{}-Mat3D (1B) & 50.78 & 0.0856 & \textbf{61.78} & \textbf{0.0436} & \textbf{30.20} & \textbf{0.0837} \\
        \hline
    \end{tabular}
    \caption{The match rate (MR) and RMSE on Perov-5, MP-20 and MPTS-52.}
    \label{tab:mat_struct_pred}
\end{table}

\ourM{}-Mat3D (1B) achieves performance that is comparable to or surpasses other state-of-the-art methods. The high match rates and low RMSE values demonstrate that our model effectively captures the complex spatial arrangements of atoms in crystal structures. Moreover we can see that \ourM{}-Mat3D performs better than other methods as the number of atoms increases, demonstrating the advantage of autoregressive sequence model. As a next step, we plan to further improve the structure prediction quality by incorporating 3D autoregressive data into the pre-training phase of the next version of \ourM{}.
