\section{Literature Review}
\label{Sec-Literature}

Household energy research has historically emphasized the role of demographic and cultural factors in shaping energy consumption patterns but has often failed to fully integrate these insights into methodological approaches, highlighting a gap between theory and practice \cite{crosbieHOUSEHOLDENERGYSTUDIES2006}. The study in \cite{moraEnergyConsumptionResidential2018} highlights the significant impact of occupant behavior, socio-economic characteristics, and preferences (e.g., set-point temperatures and ventilation durations) on residential energy consumption. It emphasizes that variations in energy demand are influenced more by contextual factors like family composition and dwelling usage patterns than by physical attributes such as house size or floor area.

Traditionally, energy researchers made efforts to enable access to relevant anonymized data through direct measurement of the variables of interest, in other words, conducting in-place studies for a determined time and gathering as much data as they can from real places using sensors. The rationale is that by enabling more data, several other lines of research would benefit from diversifying the cases they study. In \cite{cruz2024pattern}, the authors were interested in the pattern-driven behavior on the building side in a demand-side management scenario. For their particular contribution, they used a thorough dataset available through the U.S. Energy Information Administration (EIA) \cite{hronis2020housing}. However, before selecting that dataset, they conducted a review of all the available datasets that contain demand information with the desired granularity and detail. Although they narrowed down the list to 20, the datasets varied considerably--especially in countries like the USA, the UK, and India--and exhibited several sources of bias.

Understanding and optimizing household energy consumption requires multidisciplinary approaches that combine insights from energy modeling, behavioral science, and computational advances. In \cite{duApplianceCommitmentHousehold2011}, the authors present an innovative algorithm that schedules household appliances based on consumption forecasts and user comfort settings, aiming to optimize objectives like minimizing costs or maximizing comfort. This framework underscores the importance of accurately modeling intra-family dynamics to effectively simulate appliance usage patterns. Building upon this, \cite{NICHOLLS2015116} examines how family interactions, such as shared activities and the presence of children, influence peak electricity demand times. The study highlights the necessity of incorporating family dynamics into energy consumption models to accurately predict and manage peak demand periods. Several studies have leveraged weather simulation frameworks to assess HVAC performance under various climatic scenarios. For example, \cite{xuEffectsCoolMaterials2024} analyzed the impact of urban heat islands on residential cooling demand, showing how microclimatic factors intensify energy consumption in densely populated regions. 

{\bf LLMs and Prompt Engineering:} LLMs such as OpenAI’s GPT and Meta’s Llama have transformed how domain-specific tasks are approached, particularly in generating human-like, context-aware outputs \cite{busterSupportingEnergyPolicy2024}. In the energy domain, these models have been employed to automate processes that previously required expert curation. For instance, \cite{busterSupportingEnergyPolicy2024} introduced a novel framework combining LLMs with decision-tree logic to analyze zoning laws for renewable energy siting, achieving high accuracy while enabling large-scale policy research. Additionally, \cite{majumderExploringCapabilitiesLimitations2024} explored the potential of LLMs in energy sector applications, highlighting opportunities for their integration into power system operations and optimization tasks. In \cite{almashor2024can}, the authors explored whether LLM-based agents can synthesize household energy consumption building upon the environment provided by \cite{park2023generative}. Although they obtained some reasonable load patterns, they were limited by the capabilities of the simulation where they ran their experiment; they limited their work to parsing keywords in the agent conversations and associating them to energy consumption to later sum the household load.

Recent studies highlight the potential of large language models to streamline data access and policy analysis in household energy modeling. However, significant gaps persist in integrating these components into a cohesive framework. Current models often rely on limited assumptions about family structures and behaviors, neglecting the nuanced and culturally specific dynamics necessary for accurate simulations. While weather data is commonly used in HVAC modeling, it is rarely contextualized to align with diverse family activities, leaving room for more integrated and adaptive approaches. To address these limitations, this research leverages LLMs to generate a consistent framework that considers diverse family structures and associated behaviors across countries, introducing a structured, multi-stage process for weather data generation. By identifying realistic ranges for key parameters such as temperature, humidity, solar radiation, and wind speed for each country and season, the methodology dynamically generates 24-hour weather profiles that reflect both macro-level seasonal trends and micro-level hourly variations.