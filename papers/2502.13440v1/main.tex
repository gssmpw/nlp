\documentclass[preprint, 3p]{elsarticle} % twocolumn

%% Use the option review to obtain double line spacing
%% \documentclass[authoryear,preprint,review,12pt]{elsarticle}

%% Use the options 1p,twocolumn; 3p; 3p,twocolumn; 5p; or 5p,twocolumn
%% for a journal layout:
%% \documentclass[final,1p,times]{elsarticle}
%% \documentclass[final,1p,times,twocolumn]{elsarticle}
%% \documentclass[final,3p,times]{elsarticle}
%% \documentclass[final,3p,times,twocolumn]{elsarticle}
%% \documentclass[final,5p,times]{elsarticle}
%% \documentclass[final,5p,times,twocolumn]{elsarticle}

\usepackage{amssymb}
\usepackage{comment} 
\usepackage{url} 
\usepackage{amsmath}
\usepackage{booktabs} 

%\journal{Bioacoustics}

% Set custom preprint footer
\makeatletter
\def\ps@pprintTitle{%
   \let\@oddhead\@empty
   \let\@evenhead\@empty
   \def\@oddfoot{\textit{Preprint, under review} \hfill}%
   \let\@evenfoot\@oddfoot
}
\makeatother

\begin{document}

\begin{frontmatter}

\title{Semi-supervised classification of bird vocalizations}

\author[inst1,inst2]{Simen Hexeberg}
\author[inst1,inst2]{Mandar Chitre\corref{cor1}}
\author[inst1,inst2]{Matthias Hoffmann-Kuhnt}
\author[inst3]{Bing Wen Low}

\affiliation[inst1]{organization={ARL, Tropical Marine Science Institute, National University of Singapore}}
\affiliation[inst2]{organization={Department of Electrical \& Computer Engineering, National University of Singapore}}
\affiliation[inst3]{organization={National Parks Board}, country={Singapore}}
\cortext[cor1]{Corresponding author. Email: mandar@nus.edu.sg}

\begin{abstract}
Changes in bird populations can indicate broader changes in ecosystems, making birds one of the most important animal groups to monitor. Combining machine learning and passive acoustics enables continuous monitoring over extended periods without direct human involvement. However, most existing techniques require extensive expert-labeled datasets for training and cannot easily detect time-overlapping calls in busy soundscapes. We propose a semi-supervised acoustic bird detector designed to allow both the detection of time-overlapping calls (when separated in frequency) and the use of few labeled training samples. The classifier is trained and evaluated on a combination of community-recorded open-source data and long-duration soundscape recordings from Singapore. It achieves a mean $F_{0.5}$ score of 0.701 across 315 classes from 110 bird species on a hold-out test set, with an average of 11 labeled training samples per class. It outperforms the state-of-the-art BirdNET classifier on a test set of 103 bird species despite significantly fewer labeled training samples. The detector is further tested on 144 microphone-hours of continuous soundscape data. The rich soundscape in Singapore makes suppression of false positives a challenge on raw, continuous data streams. Nevertheless, we demonstrate that achieving high precision in such environments with minimal labeled training data is possible.


%Birds are among the most surveyed animal groups, as changes in their populations can indicate broader changes in ecosystems. Automated systems that combine passive acoustics with machine learning techniques are increasingly popular, enabling continuous monitoring over extended periods of time without direct human involvement. However, many of these techniques require extensive expert-labeled datasets for training and are not designed to easily detect time-overlapping calls in busy soundscapes. We propose a semi-supervised acoustic bird detector designed to allow detection of time-overlapping calls (when separated in frequency), while requiring only a small number of labeled training samples per class. The classifier is trained and tested on a combination of community-recorded open-source data and long-duration soundscape recordings from the Singapore Botanic Gardens (SBG). It achieves a mean $F_{0.5}$ score of 0.701 across 315 classes from 110 bird species on a hold-out test set, with an average of 11 labeled training samples per class (before augmentation). It outperforms the state-of-the-art BirdNET classifier on a test set of 103 bird species despite far fewer labeled training samples. The detector is further tested on 144 microphone-hours of continuous soundscape data. Among a shortlisted group of well-performing classes, 9 classes have only between 1 and 6 labeled training samples but achieve a mean precision of 0.75. The rich soundscape at the SBG makes suppression of false positives a challenge on raw, continuous data streams. Nevertheless, we demonstrate that achieving high precision in such environments with minimal labeled training data is possible.%


%Biodiversity monitoring is vital for many conservation efforts as it allows detection of changes in ecosystems in both space and time. Birds are one of the most surveyed animal groups as changes in bird populations can indicate broader changes in ecosystems. Well-established bird monitoring techniques, however, typically rely on manual field inspections by domain experts, making these methods both expensive and inconvenient. Automated systems combining passive acoustics with machine learning techniques have become increasingly popular as they enable continuous monitoring over extended periods of time without direct human involvement. Many of these techniques, however, require extensive sets of annotated data for training, which can be expensive to acquire. Many techniques are also not well-suited to handling time-overlapping calls in busy soundscapes. In this work, we propose an acoustic bird detector designed to allow detection of time-overlapping calls (if separated in frequency), while only requiring a small number of labeled training samples per class. We evaluate the method in Singapore, an equatorial city-state with numerous urban green spaces and varied avifauna. The method, however, is not constrained to tropical regions or birds, but can be applied more broadly for clustering and classification of frequency-modulated sounds. The detector consists of a pre-processing step to extract possible bird calls, followed by three artificial neural networks in sequence. The first two neural networks are self-supervised and learn a compressed embedding representation that allows the final network to classify bird species with less annotated training data. The learned embedding also serves as a tool to discover clusters of similar sounds in raw recordings, which can speed up annotation processes. The classifier is trained and tested on a combination of community-recorded open-source data and soundscape data recorded by us at the Singapore Botanic Gardens (SBG). It achieves a mean $F_{0.5}$ score of 0.701 across 315 classes from 110 bird species on a hold-out test set, with an average of 11 labeled training samples per class (before augmentation). It is further tested on 144 microphone-hours of continuous soundscape data. Among a shortlisted group of well-performing classes, 9 classes have only between 1 and 6 labeled training samples but achieve a mean precision of 0.75. The rich soundscape at the SBG makes suppression of false positives a challenge on raw, continuous data streams. Nevertheless, we demonstrate that achieving high precision in such environments with minimal labeled training data is possible.%


\end{abstract}

\begin{comment}
%%Graphical abstract
\begin{graphicalabstract}
\includegraphics{grabs}
\end{graphicalabstract}

%%Research highlights
\begin{highlights}
\item Research highlight 1
\item Research highlight 2
\end{highlights}
\end{comment}

\begin{keyword}
%% keywords here, in the form: keyword \sep keyword
Bioacoustics \sep Passive acoustic monitoring \sep Deep neural network \sep Self-supervised learning \sep Contrastive learning \sep Bird classification  
%% PACS codes here, in the form: \PACS code \sep code
%\PACS 0000 \sep 1111
%% MSC codes here, in the form: \MSC code \sep code
%% or \MSC[2008] code \sep code (2000 is the default)
%\MSC 0000 \sep 1111
\end{keyword}

\end{frontmatter}

%% \linenumbers

%% main text
\section{Introduction}
\label{sec:introduction}
Biodiversity monitoring is a critical aspect of biodiversity conservation, as it helps inform decision making, improves our knowledge and enhances public education and awareness. Birds are one of the most surveyed animal groups in biodiversity monitoring programmes, with point counts and transect surveys being well-established survey techniques for monitoring bird communities~\cite{bibby2000}. However, birds can be very difficult to detect and identify especially in tropical regions characterised by high avian diversity and numerous rare species \cite{Robinson2000}, \cite{Robinson2018}. Additionally, such manned survey techniques are manpower-intensive, require highly specialized expertise, and tend to overlook rare species that are sensitive to human presence \cite{darras2018}, \cite{Darras2019}, \cite{Wheeldon2019}.

Passive monitoring of biodiversity using acoustics is thus an area of great potential, as various animal groups including birds make unique vocalizations, which can be used to validate their presence. Such systems allow for automated collection of large amounts of audio data without human supervision and can survey cryptic species more effectively \cite{darras2018}, \cite{Hoefer2023}. However, the comprehensive analysis of such large volumes of data is prohibitive in terms of the man-hours required \cite{Hingston2018}. This constraint, and the rapid advancement in machine learning techniques, have made data-driven algorithms increasingly popular for bioacoustic species detection and classification tasks. The dominant approach in this space involves feeding time-frequency spectrograms of acoustic recordings to some variant of a Convolutional Neural Network for feature extraction and classification~\cite{Stowell2022}. What many of these methods have in common, however, is the need for extensive sets of expert-labeled training data. As an example, the initial BirdNET classifier~\cite{KAHL2021} was trained to classify close to 1000 different bird species but with about 1,500 spectrograms per class on average. The general need for large datasets may in part be attributed to the excessive information present in broadband spectrograms, which requires the model to learn to distinguish the signal of interest from the noise, and in part because many labeled datasets used for training are weakly labeled, i.e., class labels are typically assigned at spectrogram level without information about the exact time and frequency of the event. Transfer learning has emerged as a common technique to address the shortage of task-specific data. The idea is to improve a model's performance on a specific task by leveraging knowledge gained from a model previously trained on a different but related task. In acoustic classification, it is common to leverage models pre-trained on large datasets of either images or generic audio (typically ImageNet~\cite{ImageNet} or AudioSet~\cite{AudioSet}), and fine-tune these networks on task-specific data~\cite{LeBien2020},~\cite{Manriquez2024},~\cite{Tsalera2021}. Although transfer learning can be effective, one is constrained by the architecture and input format of the pre-trained models, which may be suboptimal for the target task. Models pre-trained on ImageNet, for example, typically require inputs of size $224\times224\times3$, constraining the selection of duration, bandwidth, and time-frequency resolution of the spectrograms. A different but closely related approach, known as meta-learning, trains a model on a set of different tasks with the aim to generalize to new tasks with very few training samples. Several promising few-shot learning approaches on bioacoustic data are presented in~\cite{Nolasco2023}, but one still needs to curate a labeled dataset on a diverse set of tasks to train the initial model before the few-shot learning can take place.

How to handle soundscapes with temporally-dense vocalizations is another challenge in bioacoustics. This is particularly relevant for birds, as many bird species are especially vocal during dawn and dusk, resulting in vocalization-dense soundscapes with frequent time-overlapping calls. One approach is to train multi-label classifiers, i.e., classifiers that can predict multiple target classes from a single input. Multi-labeled datasets, however, as compared to single-labeled datasets, are far less prevalent, harder and more tedious to accurately annotate~\cite{Briggs2012}, and, as a consequence, are not always exhaustive, which can inhibit learning~\cite{Wood2023}. Moreover, multi-label classification requires more training data because the problem is inherently harder. The lack of large, high-quality multi-labeled datasets in bioacoustics may be one reason why most research focuses on single-labeled problems (e.g.~\cite{Lasseck2018},~\cite{Gupta2021},~\cite{Thakur2019},~\cite{Nolasco2023},~\cite{Manriquez2024}). Methodologies targeting multi-label classification tend to divide spectrograms into shorter windows, apply classifiers (either single-label or multi-label) to each window, and aggregate the outputs to obtain predicted scores for all species present in the full spectrogram~\cite{KAHL2021}, \cite{LeBien2020}, \cite{Noumida2022}. The motivation behind this split-and-aggregate strategy is likely to increase the chance of capturing single vocalizations at a time, which reduces the multi-label problem to a set of single-label problems. A shortcoming of this approach, however, is that the models are fed spectrograms of fixed duration and bandwidth, while the duration and bandwidth of vocalizations vary, effectively including excessive information and allowing time-overlapping calls to enter the same input. In \cite{Briggs2012}, the authors address this issue by employing a supervised, pixel-level segmentation technique to separate calls in time and frequency prior to classification. This method, however, requires labeled training data for both segmentation and classification. In \cite{Hexeberg2023}, an object detection technique was used to detect marine mammals acoustically. Although this approach can detect time-overlapping vocalizations, it requires hard labels in the form of bounding boxes enclosing the signals of interest. This, and other labour-intensive annotation processes, may be acceptable if the objective is to detect a few specific vocalizations, but does not scale well to the vast diversity of bird vocalizations.

Lastly, most studies to date in the field of passive acoustics monitoring have involved largely pristine habitats while urban ecosystems have largely been overlooked \cite{Fairbrass2017}, \cite{Gibb2019}. As urban green spaces become increasingly important for bird populations due to the rapid rate of urbanisation \cite{Hughes2022}, evaluating the efficacy of utilising automated analysis for urban soundscapes is particularly urgent. The highly urbanised city-state of Singapore is an ideal study site as it is one of the few tropical cities that has a network of connected urban green spaces close to densely populated urban areas \cite{Wong2023}.

To target these challenges, we propose a semi-supervised, passive-acoustic bird classifier designed to allow detection of time-overlapping vocalizations (when separated in frequency) without requiring a large number of labeled data for training. We assess its performance on both open-source recordings from Singapore and long-duration, continuous soundscape data recorded at two different sites within the Singapore Botanic Gardens (SBG) -- one of the oldest botanic gardens in Southeast Asia which receives millions of visitors annually.

\section{Methodology}

The proposed method consists of four main steps which are explained in detail in the remainder of this section:
\begin{enumerate}
    \item \textbf{Segmentation:} extract individual bird calls with an energy-based segmentation technique. The isolation of individual bird calls limits noise, enabling high data compression, and allows time-overlapping calls to be treated separately as long as they do not also overlap in frequency. A consequence of this approach is that single calls/songs may be split into multiple segments.\label{item:segmentation}
    \item \textbf{Data compression:} learn a compressed representation of the segments while retaining most of the information.\label{item:autoenc}
    \item \textbf{Embedding:} use the representation from step~\ref{item:autoenc} to learn a new representation (embedding) to ensure both translational invariance and that similar sounds have similar embedding -- two key properties for efficient clustering and classification.\label{item:clr}
    \item \textbf{Classification:} curate a set of labeled data and train a classifier using the embeddings from step~\ref{item:clr}. This training process also serves as a final refinement of the embedding.  
\end{enumerate}

Note that step~\ref{item:segmentation} does not involve any learning and step~\ref{item:autoenc}~and~\ref{item:clr} are self-supervised, i.e., no labeled data is required. Training the self-supervised networks on large datasets allows the supervised classifier in the final step to distinguish the classes with a much smaller set of labeled data.

\subsection{Data collection}

\begin{figure*}[t]
\centerline{\includegraphics[width=\linewidth,keepaspectratio]{site1_site2.png}}
    \caption{Left: first deployment at the SBG (site~\#1) from July 4, 2020 until September 20, 2020. The approximate locations of the recording units are marked as yellow circles, with the corresponding external microphones of each unit marked as orange circles. The 6 microphones surround a lake and cover an area of roughly 50~m $\times$ 50~m. Right: a section of the elevated boardwalk used for the second deployment at the SBG (site~\#2) from September 20, 2020 until February 1, 2021. The same 6 microphones from site~\#1 were deployed along the circular boardwalk in a similar constellation, but without direct line of sight between units due to the dense vegetation. Photos from Google Maps.}
    \label{fig:sbg_sites}
\end{figure*}

Our primary source of acoustic data was acquired from two different locations in the SBG over a combined period of 7~months from July 2020 to February 2021. Three \emph{Wildlife Acoustics SongMeter 4 TS} recorders were deployed at both locations. Each recording unit was equipped with two omni-directional microphones, yielding soundscape recordings from 6 microphones simultaneously. The first 2.5~months of data collection took place around a lake with minimal obstruction between the microphones (site~\#1). This setup allows the same call to be detected on multiple recorders (Figure~\ref{fig:sbg_sites}), which we later leverage to train the contrastive network (Subsection~\ref{sec:clr}). The units were then re-located to a second site for the remaining 4.5~months (site~\#2). A similar constellation was used but dense vegetation occluded the direct paths between microphones (Figure~\ref{fig:sbg_sites}). During calibration tests at site~\#2, we emitted high-energy, transient sounds and found that the sounds were mostly audible only on the nearest few microphones. Although low-frequency sounds may attenuate less, site~\#2 is likely to have a significantly lower detection range.

Lastly, the collected soundscape recordings were complemented with 333 carefully selected bird recordings of species known to be present in Singapore from the Xeno-canto database\footnote{https://xeno-canto.org}. These community-curated recordings served as a faster way to obtain an initial set of local bird calls without having to search in long, unprocessed recordings.

\subsection{Time-frequency representation}
\label{sec:tfr}

\begin{figure*}[t]
\centerline{\includegraphics[width=\linewidth,keepaspectratio]{tfr-split.png}}
    \caption{An example illustrating extraction of TFRs from audio recordings. The left panel shows the spectrogram from a 5~second audio clip, with a few detected sounds enclosed by white rectangles. The right panels show the respective extracted TFRs. This example is non-exhaustive, i.e., not all detections in the audio clip are shown here.}
    \label{figTfrSplit}
\end{figure*}

Most bird vocalizations contain time-frequency transients. While vocalizations may overlap in time, they may be separated in frequency. Based on this observation, time-frequency transients are extracted from acoustic recordings as the initial stage of the technique. The extracted time-frequency transients are represented as a $128 \times 256$ matrix of numbers referred to as the time-frequency representation (TFR). TFRs are extracted from acoustic recordings through the following steps:
\begin{enumerate}
    \item Compute a spectrogram of the acoustic data with 2,048 FFT bins, a Hamming window, and an overlap of 1,536 samples between windows. Retain the frequency bins between 500~Hz and 15~kHz only, as this frequency range adequately covers most bird sounds while rejecting other unwanted noises.
    \item Convert the spectrogram to dB. Using the inter-quartile range for each frequency bin to obtain a robust estimate of the noise variance $\sigma$ at that frequency. Normalize each frequency bin by subtracting the median + $2\sigma$, dividing by $2\sigma$, and lastly clipping the resulting data between 0 and 1. This adaptively extracts regions in the time-frequency plane that have significantly higher energy than the background noise at that frequency. Moreover, it reduces the natural dominance of low-frequency calls resulting from higher attenuation of high-frequency signals~\cite{Sutherland1998}.
    \item Reduce the frequency resolution by a factor of 5 by max-pooling.
    \item Blank out time bins with low variance across broad frequency bands, as these represent impulsive sounds not characteristic of birds.
    \item Perform a watershed segmentation of the resulting spectrogram to obtain disconnected regions of high energy in the spectrogram.\label{item:watershed}
    \item Filter out regions with very short durations or very small time-bandwidth products, as these are uncharacteristic of bird sounds.    
\end{enumerate}

The end result of the above steps is best illustrated through an example (Figure~\ref{figTfrSplit}). The spectrogram on the left is from a 5~second long recording with multiple bird vocalizations. After going through the above steps, the spectrogram is converted to a number of TFRs. A few of these TFRs are shown in the panels on the right with corresponding sections marked with white rectangles in the original spectrogram on the left.

Although the TFR representations have variable duration, many of the algorithms used in our signal processing chain require a constant duration input. We pick a constant duration of 2.7~seconds (256 time bins for the TFR), and convert all TFRs to this duration when feeding to algorithms that require a constant duration input. This duration is sufficient to cover most of the TFRs we obtain. For the occasional TFR longer than 2.7~seconds, we randomly select a 256-sample section, while shorter TFRs are randomly zero-padded on both sides to make them 256 samples long. The random padding or clipping is done on the fly to create different versions of the TFR each time it is used. Note that in contrast to methods that operates on entire spectrograms, the complexity level of the downstream classification task is not impacted by the call density as long as single TFRs do not capture multiple calls.

\subsection{Auto-encoding}

While TFRs may be a good visual representation for humans to classify bird vocalizations, it is not necessarily a good representation for a machine. The TFRs are also generally sparse, with most entries containing zero energy. The auto-encoder stage of the processing learns a compressed representation of the TFR that retains most of the information from the original TFR, but using a much smaller number of coefficients. The learning is self-supervised, i.e., no labeled data is required. The auto-encoder simply seeks to reduce the error between the original TFR and a reconstructed TFR with the constraint that the intermediate representation of the TFR (the \emph{latent representation}) only contains 512 values (instead of the $128 \times 256$ values in the original input TFR).

We use a convolutional deep auto-encoder to achieve this compression (Figure~\ref{figAutoEnc}). To train the auto-encoder we build a dataset of TFRs extracted from about 90 microphone-hours of recordings, primarily from the two SBG locations and a smaller subset obtained from the Xeno-canto recordings. To encourage the model to emphasize learning of bird sounds over sounds from other sources, we additionally add three identical sets of the curated TFRs which we later use to train the final classifier (Subsection~\ref{sec:classifier}). This brings the final count to 228,042~TFRs. From these, 5,000~TFRs are used for validation and the balance 223,042~TFRs for training. We use a mean-square error loss function
and train the auto-encoder over 97~epochs using the Adam optimizer~\cite{ADAM}. The auto-encoder is capable of retaining most of the information in the TFRs, as illustrated in Figure~\ref{fig:decoded_tfrs}. After training, the encoder section of the auto-encoder is kept and used as a pre-trained, second-stage processor for the remaining stages of the algorithm.

\begin{figure*}[t]
\centerline{\includegraphics[width=\linewidth,keepaspectratio]{autoenc.png}}
    \caption{Architecture of the convolutional auto-encoder. The network enables a $64 \times$ data compression by learning a latent representation of the TFRs which retains most of the information.}
    \label{figAutoEnc}
\end{figure*}

\subsection{Contrastive representation learning}
\label{sec:clr}

While the latent representation from the previous section holds information related to bird vocalization in a compressed form, it is not a suitable representation for clustering or classification tasks. Firstly, auto-encoder latent spaces are translation equivariant, i.e., when the input TFR is shifted in time, the output TFR also shifts in time. To do this, the latent representation must retain time information. However, a bird sound shifted in time doesn't change the bird, and so we seek translation invariance rather than translation equivariance. Secondly, very similar sounds can have very different latent space representations, as nothing in the training process impose any constraint to force similar sounds to have similar representations. We next use the idea of contrastive learning to discover a preliminary embedding that is invariant to time translation and where similar sounds have similar representations.

Traditional contrastive learning~\cite{chen2020simple} obtains a pair of samples from each training sample by augmenting the training sample in two different ways. The pair is thus guaranteed to be similar, but not the same. Training the contrastive learning network then involves the design of a loss function that requires the samples in each pair to have similar representations, but samples from different pairs to differ as much as possible. This permits self-supervised training without the need for labeled samples, and the network learns a representation that is invariant to the augmentation used to create a pair of samples.

We follow the same basic approach outlined above, but change some of the details in some critical ways. Other than a random translation that is inherent in obtaining a TFR of constant duration (Subsection~\ref{sec:tfr}), we do not perform any augmentation. Instead, sample pairs are derived from recordings of the same bird vocalization on the two microphones connected to each recorder unit. The natural variability in sound propagation thus provides the desired ``augmentation". Since the sound reaches both microphones at potentially different times, some effort is required to associate the sounds on both microphones. Using a combination of time information and a requirement for high cross-correlation between the acoustic time-series of both sounds, we obtain 19,311 reliable TFR pairs from the two SBG deployments. Of these, 500~pairs are used for validation and the balance 18,811 for training. The contrastive learning network outputs a learnt representation space referred to as the \emph{embedding space} hereafter (Figure~\ref{fig:clr}). The training is performed over 50~epochs using the Adam optimizer, but with a loss function that is different from the one proposed in~\cite{chen2020simple}.

\begin{figure}[t]
  \includegraphics[width=\linewidth]{decoded_tfrs.png}
  \caption{Examples of calls from (a): Crimson Sunbird (\textit{Aethopyga siparaja, 558466}), (b): Common Hill Myna (\textit{Gracula religiosa, 179652}), (c): Olive-winged Bulbul (\textit{Pycnonotus plumosus, 562623}) and (d): Lineated Barbet (\textit{Psilopogon lineatus, 1145226}). The top row shows the TFRs after extraction from raw audio recordings, and the bottom row shows the compressed TFRs after passing through the auto-encoder. The high similarity between each pair shows that the compressed latent representation is capable of retaining most of the information in the TFRs. To limit sounds from different sources from merging, TFRs do not capture entire calls/songs if the pause between subsequent vocalizations are too long. The Olive-winged Bulbul in column c is one such example, where only a part of a longer call sequence is captured.}
  \label{fig:decoded_tfrs}
\end{figure}


\begin{figure*}[t]
  \includegraphics[width=\linewidth]{contrastive-network.png}
  \caption{Architecture of the contrastive learning neural network. The $\tanh(\cdot)$ activation function in the last dense layer ensures all entries in the 1024-dimensional embedding space representation are positive, and the final normalization layer ensures that they are scaled such that the embedding space can be thought of as the surface of a hypersphere of unit radius. The similarity between embedding space representations can then be measured in terms of the dot product of the corresponding vectors. The distance between embedding space representations can be measured as the angle between vectors, or equivalently the distance on the surface of the hypersphere.}
  \label{fig:clr}
\end{figure*}



Although the loss function proposed in~\cite{chen2020simple} is theoretically sound, it was developed for a computer vision task quite different from our needs and did not perform well on our problem. Based on the intuition that the embedding space can be modeled as a surface of a 1024-dimensional hypersphere, we do not desire maximal angular separation between dissimilar sounds, but rather orthogonality. Maximal angular separation pushes dissimilar sounds to diametrically opposite sides of the hypersphere, but that can only accommodate two classes. Instead, orthogonality pushes dissimilar sounds to different axes of the hypersphere, and can support up to 1024 distinct classes for a 1024-dimensional hypersphere. With this in mind, the loss function we use is:

\begin{align}
    &L(\mathbf{Z}) &=& \sum_{p=1}^N l_{2p,2p-1} + l_{2p-1,2p}, \label{eq:loss1} \\
    &l_{i,j} &=& -\left[ \mathbf{z}_i^T\mathbf{z}_j + \beta \sum_{k \ne i, k \ne j}^{2N} \min\left( 1 - \mathbf{z}_i^T\mathbf{z}_k, \; 1 \right)^2 \right]. \label{eq:loss2}
\end{align}


Here $\mathbf{Z} = \{\mathbf{z}_i \forall i\}$ is the set of embedding space representations for the training TFRs, organized such that the first pair is $(\mathbf{z}_1,\mathbf{z}_2)$, the second pair is $(\mathbf{z}_3,\mathbf{z}_4)$, and so on. The two terms in~(\ref{eq:loss1}) correspond to two possible orderings of samples in each pair. The first term in~(\ref{eq:loss2}) maximizes the similarity within each pair, while the second term induces orthogonality for non-paired representations. The $\min(\cdot, 1)$ function prevents maximal angular separation and $\beta = 3$ is a hyper-parameter that controls the balance between the two terms. In practice, the loss is not evaluated over the entire dataset, but in mini-batches of $N=512$.  

After training, an average similarity score $\mathbf{z}_i^T\mathbf{z}_j$ of 0.93 was obtained for paired samples, and near orthogonality for non-paired samples. While this ensures that paired samples (similar sounds) have high similarity, it does not guarantee low similarity between every pair of dissimilar sounds. Since the loss function minimizes the \textit{average} similarity between dissimilar sounds, some clusters of dissimilar sounds still experience very similar representations. This is addressed by allowing the embedding space representation to further improve in a final, supervised classification stage.

\subsection{Supervised refinement and classification}
\label{sec:classifier}

\begin{figure*}[t]
  \includegraphics[width=\linewidth]{full-network.png}
  \caption{The complete classification architecture. TFRs extracted from raw recordings are transformed to compressed embedding vectors via a pre-trained auto-encoder followed by a pre-trained contrastive network. These embeddings vectors are then fed as input to a 4-layer classification network (right), which in turn assigns confidence scores to each predefined class. To allow the embedding network to improve further, its model parameters are not frozen during the final, supervised classification training. However, to ensure the embedding network does not forget its primary objective, it is trained once for every 5 epochs of classification training.}
  \label{fig:classifier}
\end{figure*}

The final stage of the model is a classifier with four dense layers and batch normalization between each layer. It takes the preceding embedding vector as input and assigns a confidence score to each predefined class (Figure~\ref{fig:classifier}). As a single species can produce a wide range of calls with very different time-frequency representations, and since the preceding contrastive network operates on compressed TFRs from the auto encoder, the classifier is trained at TFR-level rather than at species-level. Consequently, the classifier does not only learn to differentiate between species but also between different calls from same species. However, TFRs are not guaranteed to capture entire call sequences. As a result, single calls or songs are sometimes fragmented into multiple TFRs and, in turn, assigned to separate classes if significantly distinct.

About 2/3 of the training data is obtained by extracting TFRs from Xeno-Canto files which are first manually labeled. These recordings are typically characterized by high SNR and are relatively short in duration (tens of seconds to a few minutes long), resulting in few but very clean samples for many classes. These clean samples, however, are not necessarily representative of the chaotic soundscape during peak chorus hours. The remaining 1/3 of the data is obtained by applying the pre-trained embedding network (Subsection~\ref{sec:clr}) to search for clusters of similar TFRs in soundscape recordings from the two SBG sites. The embedding enables an efficient labeling process as whole clusters can be labeled by inspecting only a few samples. Classes with less than 3 samples are discarded, resulting in 5727~TFRs distributed over 357 classes, representing 123 bird species. The dataset, however, is highly imbalanced. While most classes only have a handful of samples, others have hundreds. When splitting the dataset into training, validation and test sets, we prioritize TFRs for training while ensuring that each class has at least one validation and one test sample. To account for the class imbalance, we randomly augment original TFRs from low-sample classes with minor shifts in frequency, minor time stretching, and by adding Gaussian and white noise, obtaining a balanced distribution with exactly 50 samples per class for training, and a minimum of 20 samples each for validation and testing. 

The classifier will only extract and learn the necessary features from the input in order to separate the classes in the training set and minimize the cost function. When applying the model on continuous field recordings, however, the model is exposed to a much greater variety of sounds than it has seen during training, many of which will have very similar embeddings as some of the sounds from the trained classes. To make the model more robust against false positives in such environments, two additions to the algorithm are made. First, a separate \textit{bird-pass filter} is trained to distinguish bird sounds from generic non-bird sounds. This binary classifier has three dense layers, takes encoded TFRs from the auto-encoder as input and classify them as either birds or non-birds. The training data for the bird class is the same as for the main classifier, while the non-bird class is trained on 3152~TFRs extracted primarily from the UrbanSound8K dataset~\cite{UrbanSound8k} and in part from underwater recordings of marine mammal vocalizations, previously collected in Singapore waters. To avoid class imbalance, the non-bird sounds are oversampled with random frequency shifts to match the same sample size as the bird class. As even a small false positive rate can translate to a large number of false positives in raw recordings, we value precision over recall for bird classes. To encourage this precision-recall trade-off in the bird-pass filter, false positives are penalized more for the bird class than for the non-bird class, yielding a $F_{0.5}$ score of 0.91 for the bird class on a hold-out test set.

Secondly, an additional ``sink" class is added to the primary classifier. Since most of the non-bird sounds used to train the bird-pass filter are unlikely to be present in our local environment, the sink class is trained on local sounds from the SBG. Specifically, it is trained with 1893~TFRs from site~\#1 which initially confused the model and caused many false positives. The sink class helps the model to learn lower-level features for classes which are easily confused with other local sounds, while also providing the neural network with a class to assign TFRs that do not belong to any of the trained bird classes. We use a standard cross-entropy loss function for the main classifier, but we penalize false negatives for the sink class by an effective\footnote{As the sink class has almost 38 times more samples than each bird class, we weigh the sink class roughly by a factor of $\frac{3}{38}$ in the loss function to achieve an effective sink weight of $3 \times$ that of each bird class.} factor of $3\times$ that of the bird classes, to discourage sink samples from being misclassified as birds.

Lastly, to address the challenge that some clusters of dissimilar sounds occur close in embedding space (Subsection~\ref{sec:clr}), the classifier is trained end-to-end without freezing any of the preceding layers. This allows both the auto-encoder and the contrastive learning network to improve further. During this training, however, the trained embedding network can ``forget" the need for paired samples to have high similarity and unpaired samples to have low average similarity. To reinforce this need, we simultaneously train the embedding network with the same contrastive loss as before (Subsection~\ref{sec:clr}). In practice the training is implemented in sets of 5 epochs of the primary cross-entropy loss, followed by 1 epoch of the auxiliary contrastive loss. We train the network over 95 such sets and retain a mean similarity score of 0.866 for paired samples and near orthogonality (0.046) for unpaired samples.


\section{Results}

We evaluate the performance of the classifier in two settings. First, we assess its performance on a test set that neither the auto-encoder, the contrastive network, nor the classifier have encountered during training. We also compare the test set performance with the prevailing open-source acoustic bird detector, BirdNET~\cite{KAHL2021}. Secondly, we assess how the model performs on raw, continuous soundscape recordings from the SBG to better understand its capabilities and limitations when deployed for extensive periods of time in a typical monitoring setup.

\subsection{Evaluation on test set}
\label{sec:testset}

\begin{table}[b]
    \renewcommand{\arraystretch}{1.2}
    \caption{Classification performance on the test set.}
    \label{tab:ClassifierPerformance}
    \begin{tabular*}{\linewidth}{@{\extracolsep{\fill}}l|cc}  
        \toprule
         & Bird classes & Sink class \\
        \midrule
        Number of classes   & 315      & 1         \\
        Total test samples  & 6435     & 200       \\
        Precision           & 0.799    & 0.076    \\
        Recall              & 0.585    & 0.870    \\
        $F_{0.5}$           & 0.701    & 0.093    \\
        \bottomrule
    \end{tabular*}
\end{table}


Although the TFR samples in each class originate from the same call category, there may still be individual variations, e.g. in duration (particularly for repetitive calls), bandwidth, and frequency modulation rate (for example the steepness of up- and downsweeps). Disturbances from extraneous sources, differences in SNR levels, and artifacts from the data augmentation process can further add to the TFR variation within each class. For classes with many training samples, such diversity can enhance the model's capability to generalize as it forces the model to identify the distinguishing features of each class while ignoring irrelevant information. However, many of our classes have very few training samples, so when evaluating the model on the test set, we do not consider classes where most test samples differ significantly from the training. This reduces the test set to 315 bird classes across 110 species. Moreover, we assign low-confidence predictions (confidence scores below 0.5) to the sink class, which serves as a collector of extraneous TFRs. The sink class precision is consequently low, but more importantly, it obtains a high recall of 0.870, meaning that only 13\% of the sink samples are misclassified as birds (Table~\ref{tab:ClassifierPerformance}). The bird classes achieve a mean $F_{0.5}$ score of 0.701 with a desirable asymmetry between precision (0.799) and recall (0.585). The performance for individual bird classes, however, is skewed: 37 classes have a $F_{0.5}$ score of 0.0 while 111 classes have a $F_{0.5}$ score above 0.9. The zero-score classes fall primarily into one or more of the following categories: 1) correct class but low confidence, 2) class not learned during training, i.e., $F_{0.5} = 0.0$ on the training set, 3) confusion with similar classes, or 4) low similarity between test and training samples.

\subsection{Comparison with BirdNET}

To benchmark the model, we compare its performance against the BirdNET classifier~\cite{KAHL2021}. BirdNET is a leading open-source acoustic bird detector, extensively used by bird enthusiasts and researchers worldwide, and it currently covers more than 6,000~bird species~\cite{birdnet_github}. We use the same test set as in Subsection~\ref{sec:testset} for evaluation but differences between the two models require some adjustments. First, we consider only non-augmented samples. Second, as BirdNET is trained to detect species, we also evaluate our model at species-level rather than at TFR-level. Lastly, we exclude the few species that BirdNET is not trained to detect. The resulting test set consists of 943~samples from 103~species. The number of samples per species ranges from 1 to 101, with a median of 3 samples, meaning that only a handful of samples are available for most species. All samples lie within the 0-15~kHz frequency range covered by BirdNET.  

We test the latest BirdNET version available via the Python Package Index (PyPi) repository~\cite{birdnet_pypi} (version 0.1.6, released September 4, 2024). While our model takes TFRs as input (isolated in time and frequency), BirdNET searches for detections in broadband spectrograms. To reduce the probability of capturing multiple species in the recordings fed to BirdNET, we feed it 3-second-long raw recordings centered at each TFR (i.e., the minimum input duration that avoids signal padding). Since we have only ground truth annotations at TFR-level, we cannot guarantee that other species are absent from the 3-second recordings. However, BirdNET predicts between 0 and 7 species with confidence scores above 0.1 across the test set~\footnote{BirdNET outputs a confidence score between 0.0 and 1.0 for each species (multi-label) but predictions below 0.1 are by default not presented in the Python API}, with an average of 1.3~species, suggesting that multi-species presence is not a substantial issue. Nevertheless, as some samples may contain multiple species, we compare the models using top-k accuracy with $k=1$ and $k=3$, i.e., we consider a prediction correct as long as one of the top-k predictions (ranked by confidence score) matches the ground-truth. We evaluate the mean top-1 and top-3 accuracy, both globally (where each sample is weighted equally) and species-averaged (where each species class is weighted equally) to account for the sample imbalance between species (Table~\ref{tab:BirdNET_vs_BirdWatch}). Our model achieves significantly higher accuracy than BirdNET across all metrics on the test set, despite half of the 103~species are trained with less than 16 labeled samples each.

\begin{table}
    \renewcommand{\arraystretch}{1.2}
    \caption{Performance comparison between BirdNET and our model on 943 test samples across 103 species.}
    \label{tab:BirdNET_vs_BirdWatch}
    \begin{tabular*}{\linewidth}{@{\extracolsep{\fill}}l|cc}  
        \toprule
         & BirdNET & Our model \\
        \midrule
        Top-1 accuracy      & 0.51      & 0.81      \\
        Top-3 accuracy      & 0.66      & 0.89      \\
        Top-1 accuracy (species-averaged)    & 0.50     & 0.73 \\
        Top-3 accuracy (species-averaged)    & 0.64   & 0.82 \\
        \bottomrule
    \end{tabular*}
\end{table}


\subsection{Evaluation on continuous data}

 \begin{figure*}[!thb]
  \includegraphics[width=\linewidth]{diurnal_and_nocturnal_by_hour.png}
  \caption{Mean detections by hour of day for dirunal birds (top row) and nocturnal birds (bottom row) over 10 days of recordings at each site. The top 5 species with highest detection count at each site are listed by name. Note that the y-axis differs between the plots.}
  \label{fig:diurnal_and_nocturnal_species_by_hour}
\end{figure*}

Although the classifier has never seen the exact TFRs in the test set before, they all derive from classes which the model is trained to separate. Deploying the detector out in the field, on the other hand, exposes the classifier to a greater variety of sounds, many of which may be easily confused with some of the trained classes. In this part, we assess the model performance on raw, continuous streams of data, aiming to resemble a typical long-term monitoring situation. We first manually evaluate the model on a single day of soundscape recordings to shortlist classes with a reasonable level of precision. For practical reasons, we estimate class precision but not recall. That is, we verify the number of correct detections but do not attempt to estimate the number of missed detections as it would require very detailed, manual annotations of tens of hours of raw recordings. The shortlisted classes are then analyzed on 10 full days of recordings at both SBG sites to further evaluate the classifier but at an aggregated level without manual verification of each detection.

We randomly select a day in July 2020 from site~\#1 for the manual precision test and run the trained classifier on all 144 microphone-hours of recordings from that day (24 hours $\times$ 6 microphones). This yields about 250,000 TFRs prior to filtering. We disregard TFRs that either 1) do not pass the binary bird-pass filter (threshold = 0.5), 2) have a confidence score below 0.7, or 3) are assigned to the sink class. Additionally, we leave out 68 classes from species that are either known not to be present in the SBG or that are migratory and consequently not present in July. However, we do not ignore classes with low performance on the test set, as low test set performance may be a consequence of the misalignment between training and test samples mentioned earlier (Subsection~\ref{sec:testset}), and not necessarily an indicator of poor model performance.

Among the remaining classes, we focus on those with distinct call characteristics for more reliable verifications. Up to 50 detected TFRs per class are randomly selected for manual verification. We keep classes with a minimum precision of 0.5, yielding 27 classes across 16 species (Table~\ref{tab:precisionContinuousData}). Since the model is trained at TFR level rather than at species level, some species appear in several classes. A few classes achieve very high precision but based on very few detections. The ``training samples" column lists the number of training samples before augmentation and suggests no strong correlation between precision and the number of training samples. The 9 classes with only 6 or fewer samples have a mean precision of 0.75 while the remaining 18 classes have a mean precision of 0.80.

For the aggregated evaluation, we run the detector on 10 randomly selected days between July and September 2020 for site~\#1 and 10 days between October 2020 and January 2021 for site~\#2. With 6 microphones recording in parallel (and occasional downtime), this translates to approximately 1350 microphone-hours of recordings for each site. For each class and hour of day, the detection rate is calculated as total detections divided by total hours of microphone recordings for that hour over the 10-day period. Located only $1.3^\circ$ north of Equator, Singapore experiences minimal changes in the timings of dawn and dusk over the year, with sunrise and sunset around 7~am and 7~pm every day. Detections are therefore naturally aligned with dawn and dusk without additional correction, despite spanning several months in time (Figure~\ref{fig:diurnal_and_nocturnal_species_by_hour}).

At site~\#1, the diurnal birds become vocally active around dawn with a distinct peak between 7 and 8~am. The activity then drops but stays quite high until it reaches a low-point between 2~pm and 3~pm. The activity rebounds in the late afternoon and stays high during dusk. There are very few detections of diurnal birds after 8~pm, until the Red Junglefowl (\textit{Gallus gallus, 176086}) becomes active in the early morning between 5 and 6~am. The only nocturnal bird among the shortlisted ones, the Sunda Scops Owl (\textit{Otus lempiji, 1063234}), is virtually absent during daytime and in the early evening until 10~pm. The majority of its detections occur from 2~am with a peak around 4--5~am, before detections plummet prior to the diurnal morning chorus.

Site~\#2 experiences a similar morning chorus peak as site~\#1 but a more distinct evening peak between 6 and 7~pm -- largely attributed to the Asian Glossy Starling (\textit{Aplonis panayensis, 558724}). This site also experiences few detections from diurnal birds after dusk but relatively many between midnight and dawn. Apart from the overall daily patterns, site~\#1 experiences about ten times higher detection rates compared to site~\#2.

\begin{table}[thbp]
\centering
\small
%\begin{tabular}{@{}ccllcccc@{}}
\begin{tabular}{@{}p{0.7cm}p{1.3cm}llcp{1cm}p{1cm}p{1cm}@{}}
\toprule
\textbf{Class} & \textbf{Training samples} & \textbf{Common name}                & \textbf{Scientific name} & \textbf{TSN} & \textbf{Dete- \newline ctions} & \textbf{True positives} & \textbf{Prec- \newline ision} \\
\midrule
1              & 33                        & Asian Glossy Starling           & \textit{Aplonis panayensis}            & 558724  & 50                   & 44                       & 0.88               \\
2              & 6                         & Black-naped Oriole              & \textit{Oriolus chinensis}    & 561694  & 14                   & 10                       & 0.71                \\
3              & 4                         & Blue-eared Kingfisher           & \textit{Alcedo meninting}    & 554552  & 50                   & 30                       & 0.60                \\
4              & 34                        & Brown-throated Sunbird          & \textit{Anthreptes malacensis}		  & 558621  & 36                   & 34                       & 0.94                \\
5              & 4                         & Brown-throated Sunbird          & \textit{Anthreptes malacensis}		  & 558621  & 50                   & 41                       & 0.82                \\
6              & 3                         & Changeable Hawk-eagle           & \textit{Nisaetus cirrhatus}   		  & 824093  & 2                    & 2                        & 1.00                \\
7              & 50                        & Common Hill Myna                & \textit{Gracula religiosa}    & 179652  & 50                   & 42                       & 0.84                \\
8              & 6                         & Common Hill Myna                & \textit{Gracula religiosa}    & 179652  & 24                   & 22                       & 0.92                \\
9              & 3                         & Common Hill Myna                & \textit{Gracula religiosa}    & 179652  & 1                    & 1                        & 1.00                 \\
10             & 6                         & Crimson Sunbird                 & \textit{Aethopyga siparaja}            & 558466  & 10                   & 7                        & 0.70                 \\
11             & 1                         & Greater Racket-tailed Drongo    & \textit{Dicrurus paradiseus}  		  & 559773  & 14                   & 7                        & 0.50                 \\
12             & 11                        & Olive-backed Sunbird            & \textit{Cinnyris jugularis}   		  & 916461  & 34                   & 33                       & 0.97                \\
13             & 15                        & Olive-backed Sunbird            & \textit{Cinnyris jugularis}   		  & 916461  & 20                   & 15                       & 0.75                \\
14             & 50                        & Olive-winged Bulbul             & \textit{Pycnonotus plumosus}           & 562623  & 50                   & 40                       & 0.80                 \\
15             & 22                        & Olive-winged Bulbul             & \textit{Pycnonotus plumosus}           & 562623  & 32                   & 25                       & 0.78                \\
16             & 30                        & Olive-winged Bulbul             & \textit{Pycnonotus plumosus}           & 562623  & 50                   & 26                       & 0.52                \\
17             & 33                        & Olive-winged Bulbul             & \textit{Pycnonotus plumosus}           & 562623  & 50                   & 42                       & 0.84                \\
18             & 6                         & Red-whiskered Bulbul            & \textit{Pycnonotus jocosus}   		  & 178507  & 50                   & 25                       & 0.50                \\
19             & 33                        & Red Junglefowl                  & \textit{Gallus gallus}        		  & 176086  & 50                   & 44                       & 0.88                \\
20             & 26                        & Red Junglefowl                  & \textit{Gallus gallus}        		  & 176086  & 50                   & 33                       & 0.66                \\
21             & 50                        & Red Junglefowl                  & \textit{Gallus gallus}        		  & 176086  & 6                    & 5                        & 0.83                \\
22             & 50                        & Spotted Dove                    & \textit{Spilopelia chinensis} 		  & 1125210 & 50                   & 48                       & 0.96                \\
23             & 50                        & Sunda Scops Owl                 & \textit{Otus lempiji}         		  & 1063234 & 50                   & 37                       & 0.74                \\
24             & 18                        & White-breasted Waterhen         & \textit{Amaurornis phoenicurus}		  & 176385  & 15                   & 12                       & 0.80                 \\
25             & 11                        & White-breasted Waterhen         & \textit{Amaurornis phoenicurus}        & 176385  & 50                   & 28                       & 0.56                 \\
26             & 26                        & Yellow-vented Bulbul            & \textit{Pycnonotus goiavier}  		  & 562613  & 50                   & 36                       & 0.72                \\
27             & 11                        & Yellow-vented Bulbul            & \textit{Pycnonotus goiavier}  		  & 562613  & 50                   & 43                       & 0.86                \\
\bottomrule
\end{tabular}
\caption{Performance of shortlisted classes from a full day of continuous soundscape recordings at site~\#1. Bird names follow the Checklist of Bird Species for Singapore available at www.nparks.gov.sg/biodiversity/wildlife-in-singapore/species-list/birds. Taxonomic Serial Numbers (TSNs) are obtained from itis.gov.}
\label{tab:precisionContinuousData}
\end{table}



\section{Discussion}

Several classes from the single-day performance test at site~\#1 achieve high precision despite very few training samples (Table~\ref{tab:precisionContinuousData}). This result is encouraging, as it may allow researchers to train bioacoustic classifiers with very little labeled data. This is particularly valuable for monitoring initiatives of endangered or rare species as the low presence of such species makes it inherently hard to acquire acoustic samples for training. The performance gain over BirdNET on 103 bird species further suggests that the proposed methodology may not only be valuable for detecting rare species but may even enhance current state-of-the-art models for more common species, where large labeled datasets already exist.

The aggregated results from the 10-day test at each site add support to the manually verified precision test. The temporal detection patterns show a diurnal morning and evening chorus, few nighttime detections of diurnal birds, and few daytime detections of nocturnal birds, which align well with overall expected behavior. However, the relatively high detection rate between midnight and dawn of diurnal birds at site~\#2, suggests a lower model performance at this site. Most of these detections are classified as White-breasted Waterhen (\textit{Amaurornis phoenicurus, 176385}) and inspection of some of these suggests that these are false positives originating from low-frequency sounds of cars passing by on a nearby road. This road is close enough to be heard at site~\#2 during the night, but likely too far to be detected at site~\#1. Lower model performance at site~\#2 may be a natural consequence of selection bias, as the evaluated classes are selected based on their precision at site~\#1. Nevertheless, this underscores the importance of enhancing classifier robustness to better generalize to new soundscapes, as even nearby survey sites can exhibit significant acoustic variability.

Site~\#2, apart from having more false positives during nighttime, also experiences about one-tenth the detection rate of site~\#1. This difference is likely not linked to variation in model performance between the sites but rather a result of differences in sound propagation. The recorders at site~\#1 were positioned around a lake with minimal obstruction between microphones, potentially allowing the same call to be detected on all 6 microphones simultaneously. In contrast, the recorders at site~\#2 were situated in an area with dense vegetation without direct line-of-sight between microphones. From calibration tests at site~\#2 using impulsive, high-SNR sounds, we found that the sounds were mostly audible on a single microphone at a time. Site~\#1, besides the multiplying effect present there, may further benefit from a longer detection range for each individual microphone.

\section{Challenges and suggestions for future work}
\label{sec:further_work}

The soundscape in SBG is complex, comprising vocalizations not only from birds but also from amphibians, insects, reptiles, certain mammals like bats and squirrels, as well as anthropogenic sounds. The richness of this soundscape makes sound classification in long-duration recordings challenging as sounds from different sources may have high TFR-similarity. As a result, many of the classes that perform well on the test set (Subsection~\ref{sec:testset}) experience low precision in raw soundscape recordings. We propose three possible remedies to address this challenge.

\subsection{More data}
The self-supervised component of the proposed method facilitates higher performance with less labeled training data. Nonetheless, additional training data would still be advantageous, as it would help mitigate confusion with similar sounds and enhance generalization capabilities. A pragmatic solution to handle false positives is to encourage the classifier to learn their features by continue to incorporate them into the training set, either to the sink class or to new or existing bird classes. 

\subsection{Higher resolution}
Various bird species have demonstrated the ability to distinguish fine temporal variations in sound beyond the level of the human auditory system~\cite{Dooling2002}, and other studies emphasize the importance of incorporating high temporal features for acoustic bird classification~\cite{Stowell2014}. For calls with very high temporal variability, the distinguishing features may be obscured or lost in the data compression stages. Retaining more of the information in the raw recording may help improve the performance for such classes. Three alternatives are to:

\begin{enumerate}
    \item Increase the TFR resolution by reducing the max-pooling in the TFR extraction process.
    \item Increase the resolution of the compressed TFR, either by reducing the convolutional filter sizes in the auto-encoder, or by increasing the latent space dimension.
    \item Encourage the auto-encoder to pay more attention to vocalizations with high temporal variation, e.g., by oversampling such TFRs during training.
\end{enumerate}

\subsection{Acoustic-temporal context}
Because the TFR extraction process is designed to separate disconnected regions of high energy in the input spectrogram, some TFRs will inevitably capture only part of the full call or song. In some cases, a small part of a call is sufficiently characteristic to distinguish it from other calls. In other cases, e.g. for many Sunbird species with very specific sequences of transient up- and downsweeps, it is hard to identify the exact species from a single up- or downsweep without also knowing what immediately preceded and followed. The dilation hyper-parameter in the spectrogram segmentation process (step~\ref{item:watershed} in Subsection~\ref{sec:tfr}) can be increased to merge TFRs that are spatially close in the time-frequency space but at the cost of introducing more extraneous signals into the TFRs. A better approach would involve keeping TFRs separated but still incorporate temporal context. One solution could be to train a classifier with not only the target TFR as input, but also with TFRs from the temporal vicinity of the target.



\section{Conclusions}

Protection of biodiversity is a major global concern. For conservation efforts to be effective and to track the effect of such initiatives, one need to monitor target habitats, often over long periods of time. Training machine learning models on bioacoustic data to monitor vocally active species is becoming increasingly common, but many methods rely on extensive annotated datasets for training. When developing bioacoustic monitoring systems in new regions, or when targeting rare or endangered species, annotated training data may be limited or unavailable and is often expensive to acquire -- thereby increasing the barrier to perform studies. Methods that require less annotated data, and tools that can reduce the time and effort needed to obtain relevant training data, are therefore valuable. This is particularly true for birds, as the large number of species and their vast acoustic repertoire benefit from systems that can scale sustainably to new classes, and also because data annotation of bird sounds requires highly specialized expertise.

 We propose a semi-supervised acoustic classification method that allows classification of sounds with limited annotated training data, and we evaluate it on bird recordings from Singapore. The embedding part of the network can accelerate the discovery and annotation of new classes by clustering embedding vectors from raw recordings. Its self-supervised nature, in combination with the pre-processing step that isolates individual bird calls, allows classification with fewer labeled training samples. We show that high classification precision can be achieved with very few labeled training samples, both on a controlled test set and in continuous soundscape recordings. Moreover, the proposed model outperforms the state-of-the-art BirdNET classifier on a test set covering 103 bird species, despite far less labeled training data. However, the rich soundscape in the SBG, with around 40,000 daily TFRs per microphone, makes it challenging to reject false positives. We thus propose various approaches to address this.

Although we validate the methods in this work on bird data, they are by no means restricted to bird vocalizations but can be applied to a wide range of acoustic tasks that involve clustering or classification of frequency-modulated sounds.

%\section*{Model availability}

\section*{Declaration of Competing Interest}
None.

\section*{Acknowledgment}
The National Parks Board – Singapore (NParks) funded this study.

\begin{comment}
%% The Appendices part is started with the command \appendix;
%% appendix sections are then done as normal sections
\appendix

\section{Sample Appendix Section}
\label{sec:appendix}
Lorem ipsum dolor sit amet, consectetur adipiscing elit, sed do eiusmod tempor incididunt ut labore et dolore magna aliqua. Ut enim ad minim veniam, quis nostrud exercitation ullamco laboris nisi ut aliquip ex ea commodo consequat.
\end{comment}

%% If you have bibdatabase file and want bibtex to generate the
%% bibitems, please use
%%

\documentclass{MITstyle}

%\usepackage[table]{xcolor}
\usepackage{chngcntr}
\usepackage{hyperref}
\usepackage{microtype}

\title{A Lightweight and Extensible Cell Segmentation and Classification Model for Whole Slide Images}

\author{Nikita Shvetsov~$^{1, }$\footnote{Correspondence e-mail: nikita.shvetsov@uit.no}, Thomas K. Kilvaer~$^{2, 3}$, Masoud Tafavvoghi~$^{4}$, Anders Sildnes~$^{1}$, \\ Kajsa Møllersen~$^{4}$, Lill-Tove Rasmussen Busund~$^{5, 6}$, Lars Ailo Bongo~$^{1}$ \\
%
\vspace{1em} % Space between authors and afilliations
%
\normalfont{\small $^{1}$Department of Computer Science, UiT The Arctic University of Norway}\\
\normalfont{\small $^{2}$Department of Oncology, University Hospital of North Norway}\\
\normalfont{\small $^{3}$Department of Clinical Medicine, UiT The Arctic University of Norway}\\
\normalfont{\small $^{4}$Department of Community Medicine, UiT The Arctic University of Norway}\\
\normalfont{\small $^{5}$Department of Medical Biology, UiT The Arctic University of Norway} \\
\normalfont{\small $^{6}$Department of Clinical Pathology, University Hospital of North Norway} %\vspace{2em}
}

\begin{document}
\maketitle

\section*{Abstract}

% \begin{abstract}
% Developing clinically useful cell-level analysis tools in digital pathology remains challenging due to limitations in dataset granularity, inconsistent annotations, computational demands of advanced models, and difficulties in integrating new technologies into clinical workflows. To address these challenges, we propose a multi-faceted solution that enhances data quality, model performance, and usability to create a lightweight and extensible cell segmentation and classification model.

% First, we update data labels by employing a cross-relabeling process that refines the labels of two existing datasets, PanNuke and MoNuSAC, to create a new unified dataset with enhanced granularity, encompassing seven distinct cell types. Second, we leverage the H-Optimus foundation model as a fixed encoder to improve feature representation for simultaneous cell segmentation and classification tasks. Third, to address the computational demands of foundation models, we employ knowledge distillation to reduce model size and complexity while maintaining comparable performance. Finally, to facilitate integration into clinical workflows, we integrate the distilled model into the QuPath software, a widely used open-source platform in digital pathology.

% Our results demonstrate improvements in cell segmentation and classification performance using the H‑Optimus-based model compared to a CNN-based model. Specifically, the average $R^2$ improved from 0.575 to 0.871, and the average $PQ$ score improved from 0.450 to 0.492, indicating better alignment with actual cell counts and enhanced segmentation and classification quality. Furthermore, the distilled student model maintains performance comparable to the larger foundation model while reducing the parameter count by a factor of 48.
% Overall, by reducing computational complexity and integrating it into existing workflows, the proposed approach may significantly impact diagnostic processes, reduce the workload of pathologists, and contribute to improved patient outcomes. Though our approach shows potential enhancements in efficiency and usability of cell segmentation and classification models in digital pathology, extensive validation is needed to deploy these models in clinical practice.
% \end{abstract}

%%% shortened abstract
\begin{abstract}
Developing clinically useful cell-level analysis tools in digital pathology remains challenging due to limitations in dataset granularity, inconsistent annotations, high computational demands, and difficulties integrating new technologies into workflows. To address these issues, we propose a solution that enhances data quality, model performance, and usability by creating a lightweight, extensible cell segmentation and classification model. 

First, we update data labels through cross-relabeling to refine annotations of PanNuke and MoNuSAC, producing a unified dataset with seven distinct cell types. Second, we leverage the H-Optimus foundation model as a fixed encoder to improve feature representation for simultaneous segmentation and classification tasks. Third, to address foundation models' computational demands, we distill knowledge to reduce model size and complexity while maintaining comparable performance. Finally, we integrate the distilled model into QuPath, a widely used open-source digital pathology platform. 

Results demonstrate improved segmentation and classification performance using the H-Optimus-based model compared to a CNN-based model. Specifically, average $R^2$ improved from 0.575 to 0.871, and average $PQ$ score improved from 0.450 to 0.492, indicating better alignment with actual cell counts and enhanced segmentation quality. The distilled model maintains comparable performance while reducing parameter count by a factor of 48. By reducing computational complexity and integrating into workflows, this approach may significantly impact diagnostics, reduce pathologist workload, and improve outcomes. Although the method shows promise, extensive validation is necessary prior to clinical deployment.
\end{abstract}
\clearpage

\section{Introduction}
In digital pathology, accurate segmentation and classification of cells are crucial for many diagnostic, prognostic, and predictive analyses \cite{Jaber_Beziaeva_etal._2019,Lin_Pan_etal._2022,Park_Ock_etal._2022,Shen_Choi_etal._2024}. Nowadays, developments in computational pathology offer multiple solutions \cite{H._Qu_P._Wu_etal._2020,Javed_Mahmood_etal._2020} to utilize cell-level datasets to train machine learning models that solve these problems. The quality and specificity of training datasets are critical for robust and accurate models. Adhering to the principle of "garbage in, garbage out", it is essential to ensure that these datasets are extensively and accurately labeled with distinct classes that reflect the diverse biological characteristics of different cell types. Unfortunately, the number of open-source datasets comprising such high-quality annotations is limited. Existing cell segmentation datasets \cite{Gamper_Koohbanani_etal._2019,Graham_Vu_etal._2019,Verma_Kumar_etal._2021} may offer extensive annotations for certain cell types while providing more general labels for others. For example, in PanNuke, which is one of the largest open-source datasets comprising labeled cells, various types of morphologically and functionally different inflammatory cells like macrophages and lymphocytes are clustered in a broad "inflammatory" class. Consequently, these classes are frequently omitted from analyses or aggregated into broader meta-classes \cite{Gamper_Koohbanani_etal._2020} and likely interfere with other cell classes included in the dataset. This and similar inconsistencies in annotation granularity limit the ability of machine learning models to learn the comprehensive and nuanced features necessary for accurate cell segmentation and classification. To address these challenges, methods for refining and standardizing dataset annotations are essential to enhance the quality of training data.

A complementary approach to mitigate the absence of high-quality training data is the use of foundation models. Foundation models as encoders are defined as large-scale, versatile networks pre-trained on vast, diverse datasets using self-supervised learning, contrasting with convolutional neural network (CNN) pre-trained encoders that rely on supervised learning with labeled data. In practice, foundation models leverage enormous amounts of weakly or unlabeled data from millions of whole slide images (WSIs) and employ self-attention mechanisms to capture long-range dependencies and global context \cite{Chen_Ding_etal._2024,Saillard_Jenatton_etal._2024,Vorontsov_Bozkurt_etal._2024,Xu_Usuyama_etal._2024}. As a consequence, foundation models are able to produce transferable feature representations across different cell types and tissue environments. The feature representations can be leveraged by decoder networks to produce segmentation masks and pixel-level classifications. Because foundation models have comprehensive feature representations, they can be effectively fine-tuned using much smaller amounts of cell-level data compared to the large datasets needed to train models from scratch. Furthermore, foundation models incorporate adversarial training elements or contrastive learning \cite{Chen_Ding_etal._2024,Xu_Usuyama_etal._2024}, enhancing their resilience and adaptability by exposing them to challenging and varied scenarios during training. This may result in more generalizable models, often making them well-suited for diverse and complex tasks in digital pathology.

Despite the inherent advantages of foundation models, their deployment for practical use faces its own obstacles. In particular, they require substantial computational power, financial investments and rigorous testing to ensure reliability and efficacy for a given task \cite{Akkus_Dangott_etal._2022,Dragomir_Cocuz_etal._2022,Go_2022,Jafri_Farooqui_etal._2024}. Moreover, while foundation models enhance feature representation and performance, they depend on the quality of available annotations for decoder fine-tuning and, like any other model, cannot resolve existing inconsistencies or ambiguities in data labels. Therefore, there remains a critical need for solutions that address both data quality and practical deployment considerations.
Further, integrating new technologies into existing clinical workflows often encounters resistance, as it necessitates adjustments to established diagnostic processes. So, there is a need to develop solutions that could be integrated into current practices, minimizing the burden on medical professionals to adopt new tools \cite{King_Williams_etal._2023}.

Existing solutions \cite{Goldsborough_Philps_etal._2024,Hörst_Rempe_etal._2024}, while addressing some aspects of these challenges, fall short in providing a comprehensive approach. To address the data quality and clinical deployment issues, we propose a multi-faceted solution that encompasses data refinement, model optimization, and integration with existing pathology tools (\hyperref[fig:fig1]{Figure 1}). The outcome is a lightweight cell segmentation and classification model that can be integrated into digital pathology workflows for practical clinical use.

\begin{figure}[h!]
    \centering
    \includegraphics[width=\textwidth, height=0.82\textheight, keepaspectratio]{images/Figure_1.pdf}
    \caption{Overview of the proposed solution, including 1) Data refinement using cross-relabeling, 2) Teacher model development and fine tuning, 3) Student model optimization with knowledge distillation and 4) Student model and QuPath integration}
    \label{fig:fig1}
\end{figure}
\clearpage

Our approach begins with preparing the data for the fine-tuning and training of the machine learning models. We create a refined dataset, acquired via cross-relabeling two cell-level datasets, enhancing annotation specificity and consistency of the labeled data. Subsequently, we create a cell segmentation and classification model based on the foundation model. We leverage the foundation model as a fixed encoder and fine-tune a decoder using the refined dataset to improve generalization across diverse tissue- and cell types.
To ensure that the model remains lightweight and deployable in a possibly resource-constrained environment, we employ knowledge distillation to approximate the functionality of the foundation model. Finally, to facilitate the practical application of our model in digital pathology workflows, we integrate it with the QuPath \cite{Bankhead_Loughrey_etal._2017} application. Each methodological component contributes to the overarching goal of enhancing model performance, generalizability, and usability in clinical settings.

The primary contributions of this paper are:
\begin{enumerate}
    \item \textit{Data labels refinement through cross-relabeling:}
    
    We propose a new method for refining labels of cell-level datasets through cross-relabeling. This method employs classification models to re-label broad and ambiguous instances, resulting in a more diverse dataset. Our evaluation demonstrates that these classification models achieve high accuracy on test subsets, indicating the reliability of the method for label refinement.

    \item \textit{Enhanced model performance via foundation models:}
    
    We employ a foundation model as a feature extractor for the cell segmentation and classification task. In comparison with training a CNN model from scratch, the foundation model backbone only needs fine-tuning, which significantly reduces training time, computational resources and data requirements. We show that using a foundation model encoder leads to better performance in cell segmentation and classification networks than using a CNN-based encoder. This improvement may enable the model to generalize more effectively across various tissue types and imaging methods.
    
    \item \textit{Model optimization through knowledge distillation:}
    
    We show that a smaller student model trained using knowledge distillation on the refined dataset obtained via our cross-relabeling approach from a foundation model achieves comparable performance in cell segmentation and quantification tasks. As a result, this model is more suitable for deployment in environments without high-performance computing resources.
    
    \item \textit{Integration with QuPath:}
    
    We integrate the distilled cell segmentation and classification model into QuPath, a widely used open-source digital pathology platform, to accelerate clinical adaptation by enabling pathologists to more easily incorporate advanced computational tools into their existing workflows.
\end{enumerate}

Through these methodological steps, we aim to bridge the gap between advanced machine learning techniques and practical clinical applications, making accurate and efficient digital pathology accessible in a broader range of healthcare settings.

\section{Refining Existing Datasets Using Cross-Relabeling}
To address the limitations of sparse and ambiguous labeling of cell-level datasets, we propose a generalizable cross-relabeling strategy that can be applied to any dataset containing broadly categorized or imprecisely labeled cell types. This approach involves training and subsequently leveraging classification models to refine broad categories into more specific or biologically relevant classes.
When applied to cell-level data, the methodology includes extracting individual cell images from the dataset patches, preprocessing these images to standardize the size and accommodate partial cells, and then training deep learning classifiers capable of distinguishing between the finer cell subtypes within the coarser categories. 
To illustrate our approach, we focus on the PanNuke \cite{Gamper_Koohbanani_etal._2020, Gamper_Koohbanani_etal._2019} and MoNuSAC \cite{Verma_Kumar_etal._2021} datasets that we have used to train models for cell quantification in our previous works \cite{Shvetsov_Grønnesby_etal._2022,Shvetsov_Sildnes_etal._2024}. We find that for better cell differentiation we have to introduce more granular labels. PanNuke includes a broad classification of "inflammatory" cells, encompassing lymphocytes, macrophages, and neutrophils. Each cell type differs significantly in structure, function, and clinical relevance. Conversely, MoNuSAC uses the label "epithelial" for a class that comprises both benign epithelial cells and malignant neoplastic cells. This practice makes it challenging to differentiate between benign and malignant epithelial cells in the dataset, which is a critical distinction when identifying tumor areas within tissue samples. To address these issues, we implement a cross-relabeling strategy as shown in \hyperref[fig:fig2]{Figure 2}. The key components are two classification models: one is trained on singular cell images from PanNuke data to classify the epithelial meta-class into epithelial and neoplastic classes. The other is trained on MoNuSAC to refine the inflammatory class into lymphocytes, neutrophils, and macrophages.

\begin{figure}[h!]
    \centering
    \includegraphics[width=\textwidth]{images/Figure_2.pdf}
    \caption{Refined dataset generation via cross relabeling}
    \label{fig:fig2}
\end{figure}

The refining approach consists of three consecutive steps. The first is the preprocessing step, in which we extract individual cells from both datasets (\hyperref[fig:fig3]{Figure 3}). The specifics of PanNuke and MoNuSAC patch preparation before cell preprocessing are provided in \hyperref[chap:S1]{Appendix S1}.

\begin{figure}[h!]
    \centering
    \includegraphics[width=\textwidth]{images/Figure_3.pdf}
    \caption{Cell instances preprocessing including (1) cell map extraction, (2) bounding box delineation, (3) adjusting cell boxes and (4) cropping and resizing of cell images}
    \label{fig:fig3}
\end{figure}

During preprocessing, we extract cell type maps from the ground truth label mask and calculate bounding boxes around each cell instance. To accommodate partial cells at patch borders, a common issue in cropped patch images, we employ mirror padding and extend the field of view of the cell label by 15 pixels to capture adjacent cells. We then crop and resize the identified regions to $64 \times 64$ pixels using bicubic interpolation.

The preprocessed PanNuke dataset comprises 68,031 neoplastic and 23,207 epithelial cell images, while MoNuSAC comprises  33,104 lymphocytes, 1,252 neutrophils, and 1,695 macrophages, which we subsequently use in training cell classification models and classifying the cell image data \hyperref[fig:S2]{Appendix Figure S2 (1)}. 

The next step is to train two distinct ResNet50-based classifiers tailored to address the specific labeling challenges inherent in each dataset. We use ResNet50 for classification models due to its proven effectiveness for image classification tasks in histopathology \cite{pan2022reviewmachinelearningapproaches}, and its compatibility with small images. For the PanNuke dataset, we design the classifier, trained on MoNuSAC data, to disaggregate the heterogeneous "inflammatory" cell category into distinct subtypes: lymphocytes, macrophages, and neutrophils. Similarly, for the MoNuSAC dataset, the classifier is trained on PanNuke data and distinguishes between benign and malignant epithelial cells within the overarching "epithelial" label. By applying these targeted classifiers to their respective datasets, we assign more specific labels to individual cell instances, thus enabling us to create a unified dataset.
To ensure a balanced representation of classes, we train both models on datasets that had been equalized to match the size of the least represented class. Thus, we obtain datasets comprising 23,207 samples per class for PanNuke and 1,252 samples per class for MoNuSAC data. Next, we partition both of them into training (70\%), validation (20\%), and testing (10\%) subsets. To mitigate the risk of overfitting, we use a single dropout layer with a rate of p=0.5 in both models and data augmentation using randomized color perturbations, rotation, and horizontal and vertical flipping. We employ AdamW optimizer and the cross-entropy loss function for the training criterion.

To evaluate the two trained models, we measure the classification accuracy on the respective test subsets. The accuracies on the test subset for both classifiers are presented in \hyperref[tab:1]{Table 1}. The PanNuke model achieves an average accuracy of 93.57\%, with higher accuracy for neoplastic cells (96.06\%) compared to epithelial cells (86.26\%). The confusion matrix in Figure A3.1 shows that the model predominantly distinguishes accurately between epithelial and neoplastic tissues, with a substantial number of correct classifications and relatively few misclassifications. The MoNuSAC model demonstrates an average accuracy of 98.92\%, excelling in classifying lymphocytes (99.67\%) and macrophages (94.12\%), with lower performance for neutrophils (85.71\%). The confusion matrix in Figure A3.2 shows that the model identifies lymphocytes and performs reasonably well with macrophages and neutrophils.

\begin{table}[h!]
\renewcommand{\arraystretch}{1.5}
  \centering
  \caption{Cell classification results for PanNuke and MoNuSAC trained models (CI 95\%).}
  \label{tab:1}
  \begin{tabular}{|l|c|c|}
   \hline
   %\rowcolor{gray!30}
    Accuracy               & PanNuke model              & MoNuSAC model              \\
    \hline
    Average      & 0.936 (0.931--0.941)         & 0.989 (0.986--0.993)        \\
    \hline
    Neoplastic   & 0.961 (0.956--0.965)         & -                          \\
    \hline
    Epithelial   & 0.863 (0.849--0.877)         & -                          \\
    \hline
    Lymphocytes  & -                          & 0.997 (0.995--0.999)        \\
    \hline
    Neutrophils  & -                          & 0.857 (0.796--0.918)        \\
    \hline
    Macrophages  & -                          & 0.941 (0.906--0.976)        \\
    \hline
  \end{tabular}
\end{table}

Finally, during the last step, we use the model trained on PanNuke data for epithelial cells in MoNuSAC and the model trained on MoNuSAC for the inflammatory cells class in PanNuke. Specifically, we use classifier models to relabel epithelial cells in MoNuSAC and inflammatory cells in PanNuke data. Then we combine cells with refined labels and the rest of the cells in both datasets to create a refined dataset (\hyperref[fig:S2]{Appendix Figure S2 (2)}). The process of relabeling cells and visualizing them on a patch is shown in \hyperref[fig:fig4]{Figure 4}. The cell counts in the refined dataset are provided in \hyperref[tab:S4]{Appendix Table S4}.

\begin{figure}[h!]
    \centering
    \includegraphics[width=\textwidth, height=0.42\textheight, keepaspectratio]{images/Figure_4.pdf}
    \caption{Cell relabeling procedure for epithelial and inflammatory cell classes}
    \label{fig:fig4}
\end{figure}

%\hfill

Relabeling and combining datasets have been explored in a prior study \cite{Parulekar_Kanwat_etal._2023}, where consecutive fine-tuning on multiple datasets was employed to account for hierarchical class label structures. While the method presented in \cite{Parulekar_Kanwat_etal._2023} is intuitive, it often lacks consistency and requires multiple fine-tuning runs, which can be cumbersome and time-consuming. 
In contrast, cross-relabeling simplifies this process by using specialized classification models tailored to each dataset's specific labeling challenges. This approach provides better transparency and produces a unified dataset encompassing seven distinct cell types across multiple tissue samples, enhancing data diversity for further model training or fine-tuning.

Despite these improvements, cross-relabeling does not entirely resolve issues related to poor labeling quality or the amount of labeled data. Specifically, our results show lower accuracies persist for underrepresented classes, such as macrophages, which may stem from a limited sample availability and intrinsic challenges in distinguishing these cells based solely on H\&E staining. Furthermore, while our method enhances label specificity, it relies on the initial quality of the broad labels; thus, any fundamental inaccuracies in the original annotations can propagate through the relabeling process. Addressing the overall problem of limited data labels may require integrating additional data sources or utilizing complementary immunohistochemical staining methods.
Although the reported performance metrics are obtained from evaluations on the native test sets of each dataset, it is important to note that the primary application of these classifiers is to perform cross-relabeling, where a model trained on one dataset (e.g., PanNuke) is applied to another (e.g., MoNuSAC) and vice versa. We acknowledge that a more systematic evaluation of cross-dataset generalization is needed and could be performed in future work.

Overall, the refined dataset produced by our approach can enhance the supervised training or fine-tuning of cell segmentation and classification models, especially those that utilize pre-trained foundation models to improve feature extraction robustness. In addition, these models can detect nuanced classes that enable researchers to conduct more detailed analyses of biological processes in computational pathology.

\section{Foundation models for robust cell segmentation and classification}

Accurate cell segmentation and classification in digital pathology are hindered by limited labeled data and the fact that conventional CNNs are unable to capture global contextual information due to their local receptive field constraints \cite{Gheflati_Rivaz_2022,Yang_Marcus_etal.}. Traditional approaches in cell quantification have predominantly relied on CNN encoders, such as ResNet50, given their proven effectiveness in semantic segmentation tasks \cite{Deshmane_2023,Graham_Vu_etal._2019,Mukasheva_Koishiyeva_etal._2024,Stringer_Wang_etal._2021}. However, approaches that include fine-tuning of pretrained CNNs, data augmentation, and stain normalization to partially increase data variability and address staining differences often fail to achieve the necessary generalization and robustness across diverse tissue types and staining conditions \cite{G._Wang_W._Li_etal._2018,Gao_Bagci_etal._2018,Karim_El_Khoury_Martin_Fockedey_etal._2021}.

To overcome these challenges, we leverage an encoder-decoder network that uses a foundation model as the encoder and a CNN upsampling decoder (\hyperref[fig:fig5]{Figure 5}) for simultaneous cell segmentation and classification in 2D patches extracted from WSIs. Foundation models with transformer-based architectures are viable alternatives to CNN-based encoders \cite{Shamshad_Khan_etal._2023,Sourget_2023}. They enable the creation of more advanced architectures that can decode or transform learned features more effectively \cite{Chen_Duan_etal._2023,Cheng_Misra_etal._2022,Xie_Wang_etal._2021}.

\begin{figure}[h!]
    \centering
    \includegraphics[width=\textwidth]{images/Figure_5.pdf}
    \caption{UNETR-like model with foundational model as backbone}
    \label{fig:fig5}
\end{figure}

By utilizing a transformer-based encoder, we incorporate global contextual information into the feature extraction process, which is a key advantage of such architectures \cite{Chen_Lu_etal._2021}. This foundation model integration facilitates accurate pixel-wise segmentation and classification without the need for extensive encoder training, thereby potentially improving generalization across varied cellular structures and tissue types.
In our implementation, we employ a modified UNETR \cite{Hatamizadeh_Tang_etal._2021} architecture that combines a vision transformer (ViT) \cite{Dosovitskiy_Beyer_etal._2021} encoder with a CNN-based decoder. The encoder utilizes the pretrained H-Optimus foundation model, which contains 1.1 billion parameters and is trained on over 500,000 H\&E stained WSIs \cite{Saillard_Jenatton_etal._2024}. We extract outputs from four evenly spaced transformer blocks $Z_i$, where $i \in [1, 14, 26, 38]$, to serve as residual connections for the CNN decoder. We select these blocks based on our observation that features from non-adjacent levels of the encoder lead to better overall performance on the test subset.

The CNN decoder upsamples the feature representations, acquired from the transformer blocks, to generate an intermediate vector that is handled by two task-specific layers that generate cell segmentation and classification masks. The first task-specific layer is the ‘Cellpose head’,  which is used to delineate cell instances. The layer generates horizontal and vertical gradient maps to form vector fields that are refined through gradient tracking in a post-processing step using the Cellpose algorithm \cite{Stringer_Wang_etal._2021}, known for its efficacy in cell segmentation tasks and generalizability across multiple domains \cite{Pachitariu_Stringer_2022,Stringer_Pachitariu_2024}. The second task-specific layer is the "Cell type head", which assigns labels to individual pixels. In the post-processing step, we determine the output classification label of each segmented cell instance by majority voting over the labeled pixels that comprise the cell in the segmentation map.

To evaluate model performance and measure the impact of adding a foundation model as backbone, we compare it to a ResNet50-based model. ResNet50 is a widely used solution for encoders in segmentation architectures in the medical domain \cite{Deshmane_2023,Graham_Vu_etal._2019,Mukasheva_Koishiyeva_etal._2024,Stringer_Wang_etal._2021}. For the H-Optimus-based model, we utilize frozen weights for the encoder and only fine-tune the decoder to take advantage of the extensive pre-training of the foundation model. For the ResNet50-based model we start with ImageNet \cite{Deng_Dong_etal.} weights and train both encoder and decoder parts. Hyperparameters for the training step are set to be identical, where possible, for comparable evaluation. 
For this evaluation, we deliberately use the PanNuke dataset to provide a standardized and controlled comparison between the H‑Optimus and ResNet50-based models (\hyperref[fig:S2]{Appendix Figure S2 (3)}). Specifically, we use two of the default PanNuke dataset splits (66\%) for training and validation, and reserve the third split (33\%) for testing.

To address the challenge of cell class imbalance in the PanNuke dataset, which is a common characteristic in most cell-level H\&E patch datasets, both models’ training processes employ a weighted loss function comprising cross-entropy and focal loss \cite{Lin_Goyal_etal._2018}. The focal loss component is adjusted with coefficients derived from each cell class' instance frequency, emphasizing learning from underrepresented classes and enhancing the model's sensitivity to rare but significant cellular patterns. The cross-entropy loss is augmented with spectral decoupling regularization \cite{Pezeshki_Kaba_etal._2021,Pohjonen_Stürenberg_etal._2022} and spatially varying label smoothing \cite{Islam_Glocker_2021}, which potentially stabilizes training and improves generalization in case of complex tissue morphologies. For optimization, we employ the \textit{AdamW} \cite{Loshchilov_Hutter_2019} to counter unbalanced class scenarios, with cosine annealing learning rate scheduler.

We utilize the scikit-learn library \cite{Van_der_Walt_Schönberger_etal._2014} and HoVer-Net \cite{Graham_Vu_etal._2019} implementations of $R^2$ (the coefficient of determination) and $PQ$ (panoptic quality) to evaluate our experiments. Complete mathematical formulations and detailed explanations of these metrics are provided in \hyperref[chap:S5]{Appendix S5}. To compute confidence intervals, we use nonparametric bootstrapping, where after calculating the metric on the full sample, we generated 1000 bootstrap replicates by resampling with replacement and then determined the 95\% confidence intervals as the 2.5th and 97.5th percentiles of the resulting empirical distribution.

%\hfill

The model comparisons are summarized in \hyperref[tab:2]{Table 2}. The H‑Optimus-based model achieves higher $R^2$ across all cell classes compared to the ResNet50-based model, which means that its predictions are more closely aligned with the PanNuke cell counts, indicating a stronger correlation with the observed data. Notably, the improvement of $R^2_{dead}$ may be an indicator of better global contextual representations provided by the foundation model backbone. In terms of segmentation and classification quality combined, measured by the PQ score, the H‑Optimus-based model demonstrates notable improvements across most cell classes. Overall, the average $R^2$ improved from 0.575 to 0.871, while the average $PQ$ score improved from 0.450 to 0.492, demonstrating better performance of the H-Optimus-based model.

\begin{table}[h!]
\renewcommand{\arraystretch}{1.5}
  \centering
  \caption{Cell quantification metrics for baseline and proposed models (CI 95\%).}
  \label{tab:2}
  \begin{tabular}{|l|c|c|}
    \hline
    %\rowcolor{gray!30}
    Metric             & Resnet50-based            & H-optimus-based              \\
    \hline
    $R^2_{neoplastic}$    & 0.681 (0.576--0.769)       & \textbf{0.941 (0.917--0.960)} \\
    \hline
    $R^2_{inflammatory}$  & 0.863 (0.778--0.903)       & \textbf{0.949 (0.918--0.966)} \\
    \hline
    $R^2_{connective}$    & 0.600 (0.488--0.698)       & 0.609 (0.436--0.772)          \\
    \hline
    $R^2_{dead}$          & 0.097 (-11.389--0.669)     & 0.925 (0.404--0.982)          \\
    \hline
    $R^2_{epithelial}$    & 0.635 (0.490--0.747)       & \textbf{0.930 (0.886--0.964)} \\
    \hline
    $PQ_{neoplastic}$       & 0.517 (0.499--0.535)       & \textbf{0.589 (0.575--0.604)} \\
    \hline
    $PQ_{inflammatory}$     & 0.455 (0.429--0.482)       & \textbf{0.528 (0.507--0.549)} \\
    \hline
    $PQ_{connective}$       & 0.416 (0.400--0.431)       & \textbf{0.451 (0.436--0.465)} \\
    \hline
    $PQ_{dead}$             & 0.374 (0.342--0.408)       & 0.292 (0.209--0.365)          \\
    \hline
    $PQ_{epithelial}$       & 0.488 (0.460--0.519)       & \textbf{0.599 (0.579--0.618)} \\
    \hline
  \end{tabular}
\end{table}

Our results  show that integrating the H‑Optimus foundation model within the UNETR architecture enhances the model's ability to segment and classify cells across diverse tissues from PanNuke data. The pretrained transformer encoder provides robust feature representations, resulting in higher average $R^2$ and $PQ$ scores compared to the CNN-based model. This leads to more reliable cell quantification and more accurate downstream analysis. Additionally, the streamlined fine-tuning process reduces computational overhead and training time, making the model more adaptable for new data.

Despite these advancements, the foundation model-based approach does not fully resolve all challenges related to cell segmentation and classification. We observe lower metric scores for underrepresented classes in the training data. Furthermore, foundation models typically encompass billions of parameters, resulting in substantial computational and memory requirements. It therefore poses challenges for deployment in resource-constrained environments, limiting their practical applicability in certain clinical settings.

\section{Model optimization via Knowledge Distillation}

To address the limitations posed by the extensive size of foundation models, we implement knowledge distillation — a model compression technique that leverages the teacher-student paradigm \cite{Hinton_Vinyals_etal._2015}. By training a smaller, more efficient student model to replicate the output of a larger, pre-trained teacher model, we retain performance while significantly reducing the model's complexity and resource requirements (\hyperref[fig:fig6]{Figure 6}).

\begin{figure}[h!]
    \centering
    \includegraphics[width=\textwidth, height=0.45\textheight, keepaspectratio]{images/Figure_6.pdf}
    \caption{Knowledge distillation framework for training a student model using a pre-trained teacher}
    \label{fig:fig6}
\end{figure}

We employ knowledge distillation to compress the H‑Optimus-based teacher model into a more efficient student model. The teacher model is the modified UNETR architecture with the H‑Optimus foundation model described in the previous chapter. The student model is based on a UNet architecture augmented with residual connections and incorporates a smaller ViT encoder with 9 million parameters \cite{Steiner_Kolesnikov_etal._2022,Wightman_2019}. 

First, we fine-tune the teacher model using the refined dataset from the cross-relabeling procedure (Section 2). Initially we train the decoder of the teacher model while keeping the encoder weights frozen. We split the refined dataset into train (70\%), validation (20\%) and test (10\%) subsets (\hyperref[fig:S2]{Appendix Figure S2 (4)}). During fine-tuning, we use the train and validation subsets, while leaving the test subset for model evaluation. We set the training procedure and model hyperparameters to be identical to those that were used to demonstrate the utility of foundation models for the simultaneous cell segmentation and classification task.

Next, we perform knowledge distillation from teacher to student using the refined dataset used to fine-tune the teacher model. The student model is trained to replicate the teacher model's outputs. We utilize a specialized loss function that aligns the student's predicted probability distribution with the teacher's, incorporating the teacher's class probability distribution derived from the output. Following the methodology of Hinton et al. \cite{Hinton_Vinyals_etal._2015}, we experiment with various hyperparameter settings for the temperature ($T$) and the balancing coefficients ($\alpha$ and $\beta$) in the loss function. We vary $T$ from 1 to 20 and adjust $\alpha$ and $\beta$ to balance the distillation and student losses. Through iterative tuning and evaluation, we identify that setting $T=14$, $\alpha=0.3$, and $\beta=0.7$ yields a configuration that converges and closely approximates the teacher model's performance during training.

Finally, we assess the performance of both models using the $R^2$ and $PQ$ (defined in \hyperref[chap:S5]{Appendix S5}) on the test set of the refined dataset (\hyperref[tab:3]{Table 3}). We observe that the 95\% confidence intervals overlap for most cell types, so we cannot claim statistically significant performance differences between the teacher and student models. One exception appears in the neoplastic class. The teacher model produces an $R^2$ of 0.919, while the student model shows an $R^2$ of 0.852. In addition, the student model achieves higher $PQ$ values for the neoplastic and connective classes, though the confidence intervals show overlap.

\begin{table}[h!]
\renewcommand{\arraystretch}{1.5}
  \centering
  \caption{Cell quantification metrics for teacher and distilled student models (CI 95\%).}
  \label{tab:3}
  \begin{tabular}{|l|c|c|}
    \hline
    %\rowcolor{gray!30}
    Metric & Teacher & Student \\
    \hline
    $R^2_{neoplastic}$    & \textbf{0.919} (0.898--0.939) & 0.852 (0.800--0.891) \\
    \hline
    $R^2_{lymphocyte}$    & 0.969 (0.956--0.977)         & 0.969 (0.956--0.978) \\
    \hline
    $R^2_{connective}$    & 0.694 (0.548--0.809)         & 0.618 (0.469--0.741) \\
    \hline
    $R^2_{dead}$          & 0.755 (0.400--0.908)         & 0.424 (0.100--0.731) \\
    \hline
    $R^2_{epithelial}$    & 0.922 (0.870--0.958)         & 0.843 (0.738--0.917) \\
    \hline
    $R^2_{macrophage}$    & 0.384 (-0.369--0.724)        & 0.704 (0.352--0.859) \\
    \hline
    $R^2_{neutrofil}$     & 0.854 (0.578--0.929)         & 0.833 (0.502--0.925) \\
    \hline
    $PQ_{neoplastic}$       & 0.581 (0.569--0.593)         & 0.601 (0.588--0.613) \\
    \hline
    $PQ_{lymphocyte}$       & 0.536 (0.520--0.553)         & 0.563 (0.544--0.579) \\
    \hline
    $PQ_{connective}$       & 0.436 (0.421--0.451)         & 0.457 (0.441--0.474) \\
    \hline
    $PQ_{dead}$             & 0.272 (0.235--0.315)         & 0.279 (0.201--0.369) \\
    \hline
    $PQ_{epithelial}$       & 0.522 (0.500--0.545)         & 0.530 (0.506--0.555) \\
    \hline
    $PQ_{macrophage}$       & 0.524 (0.459--0.588)         & 0.474 (0.405--0.543) \\
    \hline
    $PQ_{neutrofil}$        & 0.541 (0.490--0.592)         & 0.565 (0.522--0.607) \\
    \hline
  \end{tabular}
\end{table}


We further decompose the $PQ$ metric into its $SQ$ and $DQ$ components (\hyperref[tab:S6]{Appendix Table S6}). Both models produce nearly identical $SQ$ values, which indicates that they predict instance boundaries with similar precision. Although the student model shows some improvement in $DQ$ scores for certain classes, the confidence intervals overlap and do not confirm a statistically significant difference.

We observe that the student and teacher models yield comparable detection performance despite the student model using a much smaller and simpler architecture. A model with fewer parameters reduces the risk of overfitting when training data are scarce relative to the model’s complexity \cite{Farias_Ludermir_etal._2022}. The knowledge distillation process also encourages the student model to focus on the most generalizable detection features learned from the teacher. These factors enable the student model to achieve similar detection performance across different cell types.

Additionally, considering the model sizes reported in \hyperref[tab:4]{Table 4}, the distilled model achieves a significant reduction compared to the teacher model, with a 48-fold decrease in parameter count and a 5.5-fold reduction in on-disk size. In inference mode, the teacher model requires 16 GB of VRAM for a batch size of 32, while the distilled model only needs 3 GB of VRAM for the same batch size. These reductions make the distilled model significantly more practical for fine-tuning and deployment in resource-constrained environments.

\begin{table}[h!]
\renewcommand{\arraystretch}{1.5}
  \centering
  \caption{Parameter counts and size of teacher and distilled model}
  \label{tab:4}
  \adjustbox{max width=\textwidth}{%
  \begin{tabular}{|l|c|c|c|}
    \hline
    %\rowcolor{gray!30}
    Metric & H-optimus-based (Teacher) & mobileViT-based (Student) & Magnitude of difference \\
    \hline
    Parameters count       & 1,158,917,906   & \textbf{24,093,393}   & \textbf{48x}  \\
    \hline
    Estimated Total Size (MB) & 87,912       & \textbf{15,935}    & \textbf{5.5x} \\
    \hline
  \end{tabular}%
}
\end{table}

%\hfill

With recent advancements in complex network architectures and the use of pretrained encoders to achieve state-of-the-art performance \cite{Baumann_Dislich_etal._2024,Hörst_Rempe_etal._2024} in cell segmentation and classification tasks, model size, computational complexity, and processing times have increased. This limits the scalability and accessibility of these models. As we demonstrate, this may be mitigated using knowledge distillation. Studies in the field of natural language processing have demonstrated the efficacy of knowledge distillation in retaining the capabilities of the teacher model while achieving significant reductions in size and complexity \cite{Huangpu_Gao_2024,Sun_Yu_etal.}. 

We demonstrate the feasibility of knowledge distillation in digital pathology, specifically for cell segmentation and classification tasks. Moreover, we achieve this performance while also significantly reducing the parameter count. In addressing the challenge of knowledge transfer, we found that distillation from a transformer-based model to a smaller transformer is more straightforward than attempting to map transformer features to CNN blocks. In our experiments, using a CNN-based network as a student results in worse cell quantification performance due to the structural constraints of CNN feature space dimensions. 

Although our primary approach relies on a transformer-based student model that performs well, it can be further optimized to incorporate advantages from CNN architectures. For example, employing alternative techniques such as using ViT adapters \cite{Chen_Duan_etal._2023} or $1 \times 1$ convolutions to adjust feature map sizes may be beneficial for harnessing CNN advantages like enhanced local feature extraction. Moreover, if additional performance improvements are desired, the process can be further enhanced by applying supplementary knowledge distillation techniques, such as self-distillation \cite{Zhang_Song_etal._2019} or online distillation \cite{Houyon_Cioppa_etal._2023}.

Despite these promising results, further validation on independent datasets is necessary to fully understand the model's limitations. Underrepresented classes may pose challenges when addressing complex cases. Pathologists need to validate these models to adopt them in clinical settings. While the distilled models are smaller and more deployable, a technological gap persists because pathologists traditionally rely on established methods for inspecting WSIs and diagnosing diseases. Addressing the complexities involved in deploying models for inference and supporting pathologists in adopting new tools is essential for integrating these models into clinical workflows.

\section{Model integration with QuPath}
Digital pathology tools with graphical user interfaces are essential for visualizing and analyzing WSIs. To make our student model useful in clinical pathology workflows, it needs to be integrated into a tool that enables inspecting regions, creating annotations, and providing quantitative analyses of biomarkers. Therefore, we integrate the trained student model from the previous chapter into the QuPath open‑source platform \cite{Bankhead_Loughrey_etal._2017}. QuPath provides the required annotation, visualization, and analysis tools to interpret complex histological data, including workflows for cell segmentation, classification, and quantification (\hyperref[fig:fig7]{Figure 7}). 

\begin{figure}[h!]
    \centering
    \includegraphics[width=\textwidth]{images/Figure_7.pdf}
    \caption{Visualization of model-generated cell quantification annotations (left) and the corresponding unannotated slide (right) in QuPath}
    \label{fig:fig7}
\end{figure}

To identify the regions in a WSI critical for prognosticating tumor development, such as specific tumor areas or border regions without overlapping healthy tissue, the pathologist uses QuPath to outline these regions. Then, the pathologist initiates a cell segmentation and classification script through the QuPath interface for the selected regions. The resulting annotations and quantified cell information are then directly overlaid onto the WSI in the QuPath interface. Additional design and implementation details are in \hyperref[chap:S7]{Appendix S7}. 

Two common approaches for integrating deep learning models into QuPath are Java‑based native QuPath extensions \cite{Goldsborough_Philps_etal._2024} and the execution of RESTful API requests to a model server coupled with handling the response via an extension, as demonstrated in the application of cell segmentation models applied to immunofluorescence images \cite{Sugawara_2023}. While the community is actively working on these integration strategies, there is currently no universal solution that fully addresses all integration and performance requirements.

Extensions may offer better integration with QuPath, allowing slightly improved performance and more widespread usage of the built-in QuPath models, but they lack the flexibility to customize models and modify their behavior. For example, the newest version of QuPath includes models such as StarDist \cite{Weigert_Schmidt} and InstanSeg \cite{Goldsborough_Philps_etal._2024} that can perform cell segmentation. Both models pose limitations when applied to simultaneous cell segmentation and classification. StarDist performs well only on convex, round shapes by design, whereas some neoplastic, inflammatory, and connective cells exhibit complex and non-convex shapes. InstanSeg provides only semantic segmentation without assigning classes to the segmented cells.

%\hfill

In contrast, our approach offers an alternative integration strategy. It utilizes the paquo library to directly interact with QuPath’s internal application programming interface from within Python. This enables data exchange and processing without the need for intermediate conversion steps and provides greater control over model customization, retraining, and the incorporation of custom processing steps.

The integration of our custom model with QuPath underscores its potential to significantly enhance the diagnostic process by reducing the time burden on pathologists and enabling them to focus on more complex interpretative tasks using familiar software. Leveraging a tool that is already well-established among pathologists increases the likelihood of its adoption into daily clinical workflows. The quantitative data generated through the automated workflow is critical for both clinical decision-making and research, facilitating more accurate biomarker analysis, enabling robust statistical evaluations, and supporting hypothesis generation and testing. Additionally, by streamlining cell segmentation and classification, the tool enhances the scalability and reproducibility of pathological assessments, ultimately contributing to improved diagnostic accuracy and patient outcomes.

\section{Conclusion and future work}

In this study, we address critical challenges in digital pathology and tackle the usability and deployment issues of the developed models in standard computing environments without the need for high-performance computing systems. Our multi-faceted approach encompasses data refinement through cross-relabeling, leveraging foundation models for robust cell segmentation and classification, optimizing model performance via knowledge distillation, and integrating the optimized model into the QuPath software for practical application. This approach is used to construct a capable, versatile, and adjustable model for cell segmentation and classification, with enhanced performance and usability.

\begin{sloppypar}
While our approach shows potential in the field of computational pathology, certain limitations persist. 
For example, our implementation currently exhibits lower performance in detecting macrophages. 
This serves as an instance of the broader challenge of accurately identifying complex cell types. In order to address this issue, extending our approach to incorporate additional data sources, exploring alternative modeling approaches, and integrating other imaging modalities such as immunohistochemical staining may help improve detection accuracy. Moreover, although the distilled model reduces computational demands, integrating advanced deep learning models into clinical practice requires addressing technological gaps and potential resistance to adopting new tools within established diagnostic processes.
\end{sloppypar}

Future work could focus on several key areas to refine the proposed approach and facilitate its adoption in clinical environments. Enhancing the cell-relabeling process with additional datasets \cite{Graham_Jahanifar_etal._2021} could improve the representation of underrepresented cell types and enhance overall model performance. Also, incorporating additional data sources, such as multi-modal imaging or complementary staining methods, may address limitations related to cell type differentiation and class imbalance. Exploring other foundation models \cite{Vorontsov_Bozkurt_etal._2024,Zimmermann_Vorontsov_etal._2024} or introducing additional modalities \cite{Ding_Wagner_etal._2024,Vaidya_Zhang_etal._2025} may provide alternative architectures better suited to specific tasks or offer improved efficiency. Implementing more complex knowledge distillation techniques \cite{Houyon_Cioppa_etal._2023,Zhang_Song_etal._2019} could further optimize the model's performance and adaptability. Additionally, deeper integration with QuPath or other digital pathology software could provide pathologists more control over cell quantification analysis directly within the QuPath interface, thereby increasing accessibility and usability. Such enhancements would not only refine model performance but also ensure greater adaptability and scalability within various clinical environments. Finally, extensive validation of the model by pathologists and benchmarking against independent datasets are essential steps toward establishing the model's reliability and fostering confidence in its clinical utility.

\section*{Acknowledgments} 
This work was funded in part by the Research Council of Norway grant no. 309439 SFI Visual Intelligence, and the North Norwegian Health Authority grant no. HNF1521-20.

\bibliographystyle{IEEEtran}
\begin{sloppypar}
\begin{thebibliography}{99}

\bibitem{chaplot2020neural} Chaplot, Devendra Singh, et al. "Neural topological slam for visual navigation." Proceedings of the IEEE/CVF conference on computer vision and pattern recognition. 2020.

\bibitem{maksymets2021thda} Maksymets, Oleksandr, et al. "Thda: Treasure hunt data augmentation for semantic navigation." Proceedings of the IEEE/CVF International Conference on Computer Vision. 2021.

\bibitem{mezghan2022memory} Mezghan, Lina, et al. "Memory-augmented reinforcement learning for image-goal navigation." 2022 IEEE/RSJ International Conference on Intelligent Robots and Systems (IROS). IEEE, 2022.

\bibitem{al2022zero} Al-Halah, Ziad, Santhosh Kumar Ramakrishnan, and Kristen Grauman. "Zero experience required: Plug \& play modular transfer learning for semantic visual navigation." Proceedings of the IEEE/CVF Conference on Computer Vision and Pattern Recognition. 2022.

\bibitem{ye2021auxiliary} Ye, Joel, et al. "Auxiliary tasks and exploration enable objectgoal navigation." Proceedings of the IEEE/CVF international conference on computer vision. 2021.

\bibitem{chaplot2020object} Chaplot, Devendra Singh, et al. "Object goal navigation using goal-oriented semantic exploration." Advances in Neural Information Processing Systems 33 (2020)

\bibitem{ramakrishnan2022poni} Ramakrishnan, Santhosh Kumar, et al. "Poni: Potential functions for objectgoal navigation with interaction-free learning." Proceedings of the IEEE/CVF Conference on Computer Vision and Pattern Recognition. 2022.

\bibitem{ramrakhya2022habitat} Ramrakhya, Ram, et al. "Habitat-web: Learning embodied object-search strategies from human demonstrations at scale." Proceedings of the IEEE/CVF Conference on Computer Vision and Pattern Recognition. 2022.

\bibitem{mousavian2019visual} Mousavian, Arsalan, et al. "Visual representations for semantic target driven navigation." 2019 International Conference on Robotics and Automation (ICRA). IEEE, 2019.

\bibitem{dhariwal2021diffusion} Dhariwal, Prafulla, and Alexander Nichol. "Diffusion models beat gans on image synthesis." Advances in neural information processing systems 34 (2021)

\bibitem{ho2022classifier} Ho, Jonathan, and Tim Salimans. "Classifier-free diffusion guidance." arXiv preprint arXiv:2207.12598 (2022).

\bibitem{nichol2021glide} Nichol, Alex, et al. "Glide: Towards photorealistic image generation and editing with text-guided diffusion models." arXiv preprint arXiv:2112.10741 (2021)

\bibitem{brooks2023instructpix2pix} Brooks, Tim, Aleksander Holynski, and Alexei A. Efros. "Instructpix2pix: Learning to follow image editing instructions." Proceedings of the IEEE/CVF Conference on Computer Vision and Pattern Recognition. 2023.

\bibitem{fu2023guiding} Fu, Tsu-Jui, et al. "Guiding instruction-based image editing via multimodal large language models." arXiv preprint arXiv:2309.17102 (2023).

\bibitem{geng2024instructdiffusion} Geng, Zigang, et al. "Instructdiffusion: A generalist modeling interface for vision tasks." Proceedings of the IEEE/CVF Conference on Computer Vision and Pattern Recognition. 2024.

\bibitem{zhou2024minedreamer} Zhou, Enshen, et al. "Minedreamer: Learning to follow instructions via chain-of-imagination for simulated-world control." arXiv preprint arXiv:2403.12037 (2024).

\bibitem{zhou2023esc} Zhou, Kaiwen, et al. "Esc: Exploration with soft commonsense constraints for zero-shot object navigation." International Conference on Machine Learning. PMLR, 2023.

\bibitem{yu2023l3mvn} Yu, Bangguo, Hamidreza Kasaei, and Ming Cao. "L3mvn: Leveraging large language models for visual target navigation." 2023 IEEE/RSJ International Conference on Intelligent Robots and Systems (IROS). IEEE, 2023.

\bibitem{gadre2023cows} Gadre, Samir Yitzhak, et al. "Cows on pasture: Baselines and benchmarks for language-driven zero-shot object navigation." Proceedings of the IEEE/CVF Conference on Computer Vision and Pattern Recognition. 2023.

\bibitem{shah2023navigation} Shah, Dhruv, et al. "Navigation with large language models: Semantic guesswork as a heuristic for planning." Conference on Robot Learning. PMLR, 2023.

\bibitem{cai2024bridging} Cai, Wenzhe, et al. "Bridging zero-shot object navigation and foundation models through pixel-guided navigation skill." 2024 IEEE International Conference on Robotics and Automation (ICRA). IEEE, 2024.

\bibitem{yu2023co} Yu, Bangguo, Hamidreza Kasaei, and Ming Cao. "Co-NavGPT: Multi-robot cooperative visual semantic navigation using large language models." arXiv preprint arXiv:2310.07937 (2023).

\bibitem{wu2024voronav} Wu, Pengying, et al. "Voronav: Voronoi-based zero-shot object navigation with large language model." arXiv preprint arXiv:2401.02695 (2024).

\bibitem{qin2023mp5} Qin, Yiran, et al. "Mp5: A multi-modal open-ended embodied system in minecraft via active perception." arXiv preprint arXiv:2312.07472 (2023).

\bibitem{du2024learning} Du, Yilun, et al. "Learning universal policies via text-guided video generation." Advances in Neural Information Processing Systems 36 (2024).

\bibitem{ajay2024compositional} Ajay, Anurag, et al. "Compositional foundation models for hierarchical planning." Advances in Neural Information Processing Systems 36 (2024).

\bibitem{liang2024skilldiffuser} Liang, Zhixuan, et al. "Skilldiffuser: Interpretable hierarchical planning via skill abstractions in diffusion-based task execution." Proceedings of the IEEE/CVF Conference on Computer Vision and Pattern Recognition. 2024.

\bibitem{heusel2017gans} Heusel, Martin, et al. "Gans trained by a two time-scale update rule converge to a local nash equilibrium." Advances in neural information processing systems 30 (2017).

\bibitem{zhang2018unreasonable} Zhang, Richard, et al. "The unreasonable effectiveness of deep features as a perceptual metric." Proceedings of the IEEE conference on computer vision and pattern recognition. 2018.

\bibitem{brown2020language} Brown, Tom B. "Language models are few-shot learners." arXiv preprint arXiv:2005.14165 (2020).

\bibitem{podell2023sdxl} Podell, Dustin, et al. "Sdxl: Improving latent diffusion models for high-resolution image synthesis." arXiv preprint arXiv:2307.01952 (2023).

\bibitem{brohan2022rt} Brohan, Anthony, et al. "Rt-1: Robotics transformer for real-world control at scale." arXiv preprint arXiv:2212.06817 (2022).

\bibitem{brohan2023rt} Brohan, Anthony, et al. "Rt-2: Vision-language-action models transfer web knowledge to robotic control." arXiv preprint arXiv:2307.15818 (2023).

\bibitem{li2024manipllm} Li, Xiaoqi, et al. "Manipllm: Embodied multimodal large language model for object-centric robotic manipulation." Proceedings of the IEEE/CVF Conference on Computer Vision and Pattern Recognition. 2024.

\bibitem{shah2023vint} Shah, Dhruv, et al. "ViNT: A foundation model for visual navigation." arXiv preprint arXiv:2306.14846 (2023).

\bibitem{liu2024visual} Liu, Haotian, et al. "Visual instruction tuning." Advances in neural information processing systems 36 (2024).

\bibitem{hu2021lora} Hu, Edward J., et al. "Lora: Low-rank adaptation of large language models." arXiv preprint arXiv:2106.09685 (2021).

\bibitem{qin2023supfusion} Qin, Yiran, et al. "SupFusion: Supervised LiDAR-camera fusion for 3D object detection." Proceedings of the IEEE/CVF International Conference on Computer Vision. 2023.

\bibitem{qin2024worldsimbench} Qin, Yiran, et al. "Worldsimbench: Towards video generation models as world simulators." arXiv preprint arXiv:2410.18072 (2024).

\bibitem{yu2025gamefactory} Yu, Jiwen, et al. "GameFactory: Creating New Games with Generative Interactive Videos." arXiv preprint arXiv:2501.08325 (2025).

\bibitem{zhou2024code} Zhou, Enshen, et al. "Code-as-Monitor: Constraint-aware Visual Programming for Reactive and Proactive Robotic Failure Detection." arXiv preprint arXiv:2412.04455 (2024).

\bibitem{zhang2024ad} Zhang, Zaibin, et al. "AD-H: Autonomous Driving with Hierarchical Agents." arXiv preprint arXiv:2406.03474 (2024).

\bibitem{wang2024toward} Wang, Chaoqun, et al. "Toward Accurate Camera-based 3D Object Detection via Cascade Depth Estimation and Calibration." arXiv preprint arXiv:2402.04883 (2024).

\bibitem{huang2024story3d} Huang, Yuzhou, et al. "Story3d-agent: Exploring 3d storytelling visualization with large language models." arXiv preprint arXiv:2408.11801 (2024).

\bibitem{savinov2018semi} Savinov, Nikolay, Alexey Dosovitskiy, and Vladlen Koltun. "Semi-parametric topological memory for navigation." arXiv preprint arXiv:1803.00653 (2018).

\bibitem{majumdar2022zson} Majumdar, Arjun, et al. "Zson: Zero-shot object-goal navigation using multimodal goal embeddings." Advances in Neural Information Processing Systems 35 (2022): 32340-32352.

\bibitem{yadav2023offline} Yadav, Karmesh, et al. "Offline visual representation learning for embodied navigation." Workshop on Reincarnating Reinforcement Learning at ICLR 2023. 2023.

\bibitem{yadav2023ovrl} Yadav, Karmesh, et al. "Ovrl-v2: A simple state-of-art baseline for imagenav and objectnav." arXiv preprint arXiv:2303.07798 (2023).

\bibitem{sun2024fgprompt} Sun, Xinyu, et al. "FGPrompt: fine-grained goal prompting for image-goal navigation." Advances in Neural Information Processing Systems 36 (2024).

\bibitem{zhu2017target} Zhu, Yuke, et al. "Target-driven visual navigation in indoor scenes using deep reinforcement learning." 2017 IEEE international conference on robotics and automation (ICRA). IEEE, 2017.

\bibitem{koh2024generating} Koh, Jing Yu, Daniel Fried, and Russ R. Salakhutdinov. "Generating images with multimodal language models." Advances in Neural Information Processing Systems 36 (2024).

\bibitem{krantz2022instance} Krantz, Jacob, et al. "Instance-specific image goal navigation: Training embodied agents to find object instances." arXiv preprint arXiv:2211.15876 (2022).

\bibitem{schulman2017proximal} Schulman, John, et al. "Proximal policy optimization algorithms." arXiv preprint arXiv:1707.06347 (2017).

\bibitem{anderson2018evaluation} Anderson, Peter, et al. "On evaluation of embodied navigation agents." arXiv preprint arXiv:1807.06757 (2018).

\bibitem{lin2024navcot} Lin, Bingqian, et al. "NavCoT: Boosting LLM-Based Vision-and-Language Navigation via Learning Disentangled Reasoning." arXiv preprint arXiv:2403.07376 (2024).

\bibitem{NavGPT} Zhou, Gengze, Yicong Hong, and Qi Wu. "Navgpt: Explicit reasoning in vision-and-language navigation with large language models." Proceedings of the AAAI Conference on Artificial Intelligence.

\bibitem{hahn2021no} Hahn, Meera, et al. "No rl, no simulation: Learning to navigate without navigating." Advances in Neural Information Processing Systems 34 (2021): 26661-26673.

\bibitem{li2025t2isafety} Li, Lijun, et al. "T2ISafety: Benchmark for Assessing Fairness, Toxicity, and Privacy in Image Generation." arXiv preprint arXiv:2501.12612 (2025).

\bibitem{an2024agfsync} An, Jingkun, et al. "AGFSync: Leveraging AI-Generated Feedback for Preference Optimization in Text-to-Image Generation." arXiv preprint arXiv:2403.13352 (2024).


\end{thebibliography}
\end{sloppypar}

\clearpage
\beginsupplement
\section*{Appendix}
\renewcommand{\thesubsection}{S\arabic{subsection}}

\subsection{\label{chap:S1}PanNuke and MoNuSAC preprocessing}
The PanNuke dataset comprises a set of 7,901 RGB patches, each with dimensions of $256 \times 256$ pixels, which we set as the standard patch size for our analysis. In contrast, the MoNuSAC dataset encompasses 294 images of heterogeneous dimensions. To standardize the MoNuSAC images with our experiments, we implement a standardization protocol. Specifically, for images exceeding the dimensions of $256 \times 256$ pixels, we segment them into equal-sized patches and apply mirror padding to the remaining portions to avoid information loss at the peripherals. Patches with dimensions less than $128 \times 128$ pixels are excluded from the dataset due to the insufficient resolution to capture relevant cellular details. For patches where either dimension falls between 128 and 256 pixels, we employ upsampling to achieve the standard patch size. As a result, we obtain a total of 2,823 RGB patches derived from the MoNuSAC dataset for subsequent analysis. For additional details on the MoNuSAC data preparation process, refer to the source code \cite{Shvetsov_2025a}.
\clearpage

\subsection{\label{chap:S2}Data usage for the methodology}

\counterwithin{figure}{subsection}
\renewcommand{\thefigure}{S\arabic{subsection}}

\begin{figure}[h!]
    \centering
    \includegraphics[width=\textwidth, height=0.85\textheight, keepaspectratio]{images/A2.pdf}
    \caption{Overview of the methodology for cross-labeling, dataset refinement, and model comparison. (1) Cross-relabeling - training and testing cell classification models, (2) Cross-relabeling - using cell classification models to create refined dataset, (3) Fine-tuning and training models for comparison, (4) Student knowledge distillation with refined dataset}
    \label{fig:S2}
\end{figure}
\clearpage

\subsection{\label{chap:S3}Confusion matrices for classification models}
\counterwithin{figure}{subsection}
\renewcommand{\thefigure}{S\arabic{subsection}.\arabic{figure}}

\begin{figure}[h!]
    \centering
    \includegraphics[width=\textwidth, height=0.4\textheight, keepaspectratio]{images/A3_1.pdf}
    \caption{Confusion matrix for PanNuke trained model}
    \label{fig:S3.1}
\end{figure}

\begin{figure}[h!]
    \centering
    \includegraphics[width=\textwidth, height=0.4\textheight, keepaspectratio]{images/A3_2.pdf}
    \caption{Confusion matrix for MoNuSAC trained model}
    \label{fig:S3.2}
\end{figure}

\clearpage

\subsection{\label{chap:S4}Datasets cell counts}

\counterwithin{table}{subsection}
\renewcommand{\thetable}{S\arabic{subsection}}

\begin{table}[h!]
\renewcommand{\arraystretch}{2.0}
\centering
\caption{\label{tab:S4}Cell counts for PanNuke, MoNuSAC and refined datasets. Numbers in parentheses indicate preprocessed cell counts for cell classifier models training and testing.}
%\adjustbox{max width=\textwidth}{%
\begin{tabular}{|l|c|c|c|}
\hline
%\rowcolor{gray!30}
Cell type & PanNuke & MoNuSAC & Refined \\
\hline
Neoplastic & 77,403 (68,031) & - & 105,451 \\
\hline
Epithelial & 26,572 (23,207) & - & 29,926 \\
\hline
Epithelial (benign and malignant) & - & 31,402 & - \\
\hline
Inflammatory & 32,276 & - & - \\
\hline
Lymphocytes & - & 37,045 (33,104) & 65,275 \\
\hline
Neutrophils & - & 1,355 (1,252) & 3,833 \\
\hline
Macrophage & - & 1,842 (1,695) & 3,410 \\
\hline
Dead & 2,908 & - & 2,908 \\
\hline
Connective & 50,585 & - & 50,585 \\
\hline
\end{tabular}
%
%}
\end{table}



\clearpage

\subsection{\label{chap:S5}Definition of validation metrics}
\counterwithin{equation}{subsection}
\renewcommand{\theequation}{\arabic{equation}}

\subsubsection{\label{chap:S5.1}R\textsuperscript{2}}
The coefficient of determination, denoted as $R^2$, is a statistical measure that represents the proportion of variance in the dependent variable that is predictable from the independent variables. In the context of cell quantification in pathology, $R^2$ is used to assess how well the predicted quantities of different cell types in a patch align with the actual quantities observed in the ground truth data, with higher values representing more accurate quantification. $R^2$ is defined as
\begin{equation*}
R^2 = 1 - \frac{\sum_{i=1}^n (y_i - \hat{y}_i)^2}{\sum_{i=1}^n (y_i - \bar{y})^2},
\end{equation*}
where $y_i$ represents the actual number of cells of a specific type in the $i$-th image, $\hat{y}_i$ represents the predicted number of cells of that type in the $i$-th image, $\bar{y}$ is the mean of the actual numbers across all images, and $n$ is the total number of images in the dataset.

The $R^2$ metric has a range of $(-\infty, 1]$. An $R^2$ of 1 indicates perfect prediction, where all predicted values exactly match the actual values. An $R^2$ of 0 suggests that the model explains none of the variability of the response data around its mean. If $R^2$ is negative, it indicates that the model performs worse than a model that simply predicts the mean of the actual values for all observations.

\subsubsection{\label{chap:S5.2}PQ}
Panoptic Quality ($PQ$) is a comprehensive metric used to evaluate the performance of segmentation models in tasks that require both instance segmentation and classification. $PQ$ provides a single score that encapsulates both the detection accuracy (i.e., how many objects were correctly identified) and the segmentation quality (i.e., how accurately the objects' boundaries were delineated). This metric is particularly useful in multiclass scenarios where each pixel is classified into distinct categories, such as different cell types in pathology images.

$PQ$ is calculated as the product of two terms: Detection Quality ($DQ$) and Segmentation Quality ($SQ$). It can be expressed as
\begin{equation*}
PQ = DQ \cdot SQ,
\end{equation*}
where
\begin{equation*}
DQ = \frac{TP}{TP + 0.5\, FP + 0.5\, FN},
\end{equation*}
\begin{equation*}
SQ = \frac{\sum_{(p, g) \in \mathcal{M}} IoU(p, g)}{TP}.
\end{equation*}
In these formulas, $TP$ denotes the number of correctly matched instances between ground truth and prediction, $FP$ denotes the predicted instances that have no corresponding ground truth, $FN$ denotes the ground truth instances that were not detected, $IoU(p, g)$ is the Intersection over Union for a pair of matched instances $p$ (prediction) and $g$ (ground truth), and $\mathcal{M}$ is the set of matched pairs.

The $PQ$ metric is calculated for each class and is averaged across classes to provide a global performance measure.

The $PQ$ score has a range of $[0, 1.0]$, where a higher score indicates better performance in both detecting and segmenting the instances correctly. A $PQ$ of 1 signifies perfect identification and segmentation of all instances, whereas a $PQ$ of 0 indicates that no instances were correctly identified and segmented.

\clearpage

\subsection{\label{chap:S6}Segmentation and Detection quality metrics for teacher and student models}

\begin{table}[h!]
\renewcommand{\arraystretch}{2.0}
\centering
\caption{Segmentation and detection quality for student and teacher models (CI 95\%)}
\label{tab:S6}
%\adjustbox{max width=\textwidth}{%
\begin{tabular}{|l|c|c|}
\hline
%\rowcolor{gray!30}
Metric & Teacher & Student \\
\hline
$SQ_{neoplastic}$ & 0.819 (0.815--0.823) & 0.824 (0.819--0.828) \\
\hline
$SQ_{lymphocyte}$ & 0.795 (0.788--0.802) & 0.790 (0.783--0.796) \\
\hline
$SQ_{connective}$ & 0.770 (0.762--0.776) & 0.780 (0.772--0.786) \\
\hline
$SQ_{dead}$ & 0.659 (0.623--0.688) & 0.657 (0.624--0.695) \\
\hline
$SQ_{epithelial}$ & 0.780 (0.770--0.790) & 0.788 (0.779--0.797) \\
\hline
$SQ_{macrophage}$ & 0.788 (0.760--0.810) & 0.757 (0.730--0.783) \\
\hline
$SQ_{neutrofil}$ & 0.782 (0.761--0.801) & 0.775 (0.759--0.792) \\
\hline
$DQ_{neoplastic}$ & 0.706 (0.692--0.719) & 0.727 (0.712--0.741) \\
\hline
$DQ_{lymphocyte}$ & 0.675 (0.656--0.698) & 0.713 (0.691--0.734) \\
\hline
$DQ_{connective}$ & 0.566 (0.546--0.584) & 0.583 (0.565--0.602) \\
\hline
$DQ_{dead}$ & 0.410 (0.361--0.465) & 0.435 (0.306--0.561) \\
\hline
$DQ_{epithelial}$ & 0.668 (0.639--0.694) & 0.673 (0.644--0.702) \\
\hline
$DQ_{macrophage}$ & 0.657 (0.583--0.727) & 0.615 (0.531--0.703) \\
\hline
$DQ_{neutrofil}$ & 0.691 (0.625--0.753) & 0.729 (0.679--0.778) \\
\hline
\end{tabular}
%
%}
\end{table}

\clearpage

\subsection{\label{chap:S7}QuPath integration method}
We adopt an integration strategy leveraging the paquo \cite{Bayer_AG} library, a Python package that enables direct interaction with QuPath’s internal API, thereby facilitating seamless data exchange without intermediate conversion steps. The data processing pipeline (\hyperref[fig:S7]{Appendix Figure S7}) begins with the acquisition of WSIs and their associated annotations from QuPath, which are represented as Shapely \cite{Gillies_Wel_etal._2024} polygons. Utilizing paquo, we directly read, create, and modify these annotations and detections within a QuPath project in the Python environment. Images are then cropped using these polygons and processed by cell segmentation and classification models employing standard vision processing toolkits such as OpenCV, pyvips, and PyTorch. Additionally, QuPath employs Groovy scripts to initiate a Python process that starts the entire pipeline from QuPath graphical interface: fetching polygons, extracting images from them, and running deep learning model inference on the cropped images. 
The results are returned to QuPath, leveraging paquo's Python bindings to manipulate QuPath data while minimizing the computational overhead typically associated with cross-environment communication.

\counterwithin{figure}{subsection}
\renewcommand{\thefigure}{S\arabic{subsection}}

\begin{figure}[h!]
    \centering
    \includegraphics[width=\textwidth]{images/A7.pdf}
    \caption{QuPath integration workflow using Python environment}
    \label{fig:S7}
\end{figure}

Compared to traditional workflows that involve exporting annotations as GeoJSON, classifying them in Python, and reimporting them into QuPath, our approach offers several advantages. We eliminate the need to switch between programming languages, providing a cohesive and streamlined development process entirely within QuPath software and removing the necessity to use other tools. Meanwhile, we avoid storing annotations as intermediate JSON files unless required for external use or archiving. By conducting the entire inference and post-processing workflow within the Python environment, we leverage the power and flexibility of Python libraries for image processing and machine learning. This approach also enables adjustments to any set of labels and models, thereby improving its applicability.

%\hfill

The distilled model and QuPath integration code are packaged into a Docker container, enabling streamlined execution with the Docker engine. Detailed integration code and deployment instructions can be found in the GitHub repository \cite{Shvetsov_2025b}.

Despite these benefits, we acknowledge that the paquo library is a proof‑of‑concept project in its early development stage and has not been tested across all versions of QuPath.

\clearpage

\subsection{\label{chap:S8}Data and code availability statement}
All datasets, models, and code used in this study are publicly available and can be obtained from the repositories listed below. 
The PanNuke \cite{Gamper_Koohbanani_etal._2019} and MoNuSAC \cite{Verma_Kumar_etal._2021} datasets are publicly accessible, and download information along with detailed descriptions can be found in their respective articles. Preprocessing scripts for PanNuke and MoNuSAC data, as well as individual cell extraction scripts, are available on GitHub \cite{Shvetsov_2025a}. The H-Optimus foundation model used in our experiments can be downloaded from the HuggingFace repository \cite{hoptimus2024}, and model information is available on GitHub \cite{Saillard_Jenatton_etal._2024}. In addition, the integration code for QuPath and the distilled model packaged in a Docker container are provided in the repository \cite{Shvetsov_2025b}, and paquo Python library is available from the authors GitHub repository \cite{Bayer_AG}.
\clearpage

\end{document}

 %\bibliographystyle{elsarticle-num} 
 % \bibliography{references}
 


%% else use the following coding to input the bibitems directly in the
%% TeX file.

% \begin{thebibliography}{00}

% %% \bibitem{label}
% %% Text of bibliographic item

% \bibitem{}

% \end{thebibliography}
\end{document}
\endinput
%%
%% End of file `elsarticle-template-num.tex'.
