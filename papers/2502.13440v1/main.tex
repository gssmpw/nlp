\documentclass[preprint, 3p]{elsarticle} % twocolumn

%% Use the option review to obtain double line spacing
%% \documentclass[authoryear,preprint,review,12pt]{elsarticle}

%% Use the options 1p,twocolumn; 3p; 3p,twocolumn; 5p; or 5p,twocolumn
%% for a journal layout:
%% \documentclass[final,1p,times]{elsarticle}
%% \documentclass[final,1p,times,twocolumn]{elsarticle}
%% \documentclass[final,3p,times]{elsarticle}
%% \documentclass[final,3p,times,twocolumn]{elsarticle}
%% \documentclass[final,5p,times]{elsarticle}
%% \documentclass[final,5p,times,twocolumn]{elsarticle}

\usepackage{amssymb}
\usepackage{comment} 
\usepackage{url} 
\usepackage{amsmath}
\usepackage{booktabs} 

%\journal{Bioacoustics}

% Set custom preprint footer
\makeatletter
\def\ps@pprintTitle{%
   \let\@oddhead\@empty
   \let\@evenhead\@empty
   \def\@oddfoot{\textit{Preprint, under review} \hfill}%
   \let\@evenfoot\@oddfoot
}
\makeatother

\begin{document}

\begin{frontmatter}

\title{Semi-supervised classification of bird vocalizations}

\author[inst1,inst2]{Simen Hexeberg}
\author[inst1,inst2]{Mandar Chitre\corref{cor1}}
\author[inst1,inst2]{Matthias Hoffmann-Kuhnt}
\author[inst3]{Bing Wen Low}

\affiliation[inst1]{organization={ARL, Tropical Marine Science Institute, National University of Singapore}}
\affiliation[inst2]{organization={Department of Electrical \& Computer Engineering, National University of Singapore}}
\affiliation[inst3]{organization={National Parks Board}, country={Singapore}}
\cortext[cor1]{Corresponding author. Email: mandar@nus.edu.sg}

\begin{abstract}
Changes in bird populations can indicate broader changes in ecosystems, making birds one of the most important animal groups to monitor. Combining machine learning and passive acoustics enables continuous monitoring over extended periods without direct human involvement. However, most existing techniques require extensive expert-labeled datasets for training and cannot easily detect time-overlapping calls in busy soundscapes. We propose a semi-supervised acoustic bird detector designed to allow both the detection of time-overlapping calls (when separated in frequency) and the use of few labeled training samples. The classifier is trained and evaluated on a combination of community-recorded open-source data and long-duration soundscape recordings from Singapore. It achieves a mean $F_{0.5}$ score of 0.701 across 315 classes from 110 bird species on a hold-out test set, with an average of 11 labeled training samples per class. It outperforms the state-of-the-art BirdNET classifier on a test set of 103 bird species despite significantly fewer labeled training samples. The detector is further tested on 144 microphone-hours of continuous soundscape data. The rich soundscape in Singapore makes suppression of false positives a challenge on raw, continuous data streams. Nevertheless, we demonstrate that achieving high precision in such environments with minimal labeled training data is possible.


%Birds are among the most surveyed animal groups, as changes in their populations can indicate broader changes in ecosystems. Automated systems that combine passive acoustics with machine learning techniques are increasingly popular, enabling continuous monitoring over extended periods of time without direct human involvement. However, many of these techniques require extensive expert-labeled datasets for training and are not designed to easily detect time-overlapping calls in busy soundscapes. We propose a semi-supervised acoustic bird detector designed to allow detection of time-overlapping calls (when separated in frequency), while requiring only a small number of labeled training samples per class. The classifier is trained and tested on a combination of community-recorded open-source data and long-duration soundscape recordings from the Singapore Botanic Gardens (SBG). It achieves a mean $F_{0.5}$ score of 0.701 across 315 classes from 110 bird species on a hold-out test set, with an average of 11 labeled training samples per class (before augmentation). It outperforms the state-of-the-art BirdNET classifier on a test set of 103 bird species despite far fewer labeled training samples. The detector is further tested on 144 microphone-hours of continuous soundscape data. Among a shortlisted group of well-performing classes, 9 classes have only between 1 and 6 labeled training samples but achieve a mean precision of 0.75. The rich soundscape at the SBG makes suppression of false positives a challenge on raw, continuous data streams. Nevertheless, we demonstrate that achieving high precision in such environments with minimal labeled training data is possible.%


%Biodiversity monitoring is vital for many conservation efforts as it allows detection of changes in ecosystems in both space and time. Birds are one of the most surveyed animal groups as changes in bird populations can indicate broader changes in ecosystems. Well-established bird monitoring techniques, however, typically rely on manual field inspections by domain experts, making these methods both expensive and inconvenient. Automated systems combining passive acoustics with machine learning techniques have become increasingly popular as they enable continuous monitoring over extended periods of time without direct human involvement. Many of these techniques, however, require extensive sets of annotated data for training, which can be expensive to acquire. Many techniques are also not well-suited to handling time-overlapping calls in busy soundscapes. In this work, we propose an acoustic bird detector designed to allow detection of time-overlapping calls (if separated in frequency), while only requiring a small number of labeled training samples per class. We evaluate the method in Singapore, an equatorial city-state with numerous urban green spaces and varied avifauna. The method, however, is not constrained to tropical regions or birds, but can be applied more broadly for clustering and classification of frequency-modulated sounds. The detector consists of a pre-processing step to extract possible bird calls, followed by three artificial neural networks in sequence. The first two neural networks are self-supervised and learn a compressed embedding representation that allows the final network to classify bird species with less annotated training data. The learned embedding also serves as a tool to discover clusters of similar sounds in raw recordings, which can speed up annotation processes. The classifier is trained and tested on a combination of community-recorded open-source data and soundscape data recorded by us at the Singapore Botanic Gardens (SBG). It achieves a mean $F_{0.5}$ score of 0.701 across 315 classes from 110 bird species on a hold-out test set, with an average of 11 labeled training samples per class (before augmentation). It is further tested on 144 microphone-hours of continuous soundscape data. Among a shortlisted group of well-performing classes, 9 classes have only between 1 and 6 labeled training samples but achieve a mean precision of 0.75. The rich soundscape at the SBG makes suppression of false positives a challenge on raw, continuous data streams. Nevertheless, we demonstrate that achieving high precision in such environments with minimal labeled training data is possible.%


\end{abstract}

\begin{comment}
%%Graphical abstract
\begin{graphicalabstract}
\includegraphics{grabs}
\end{graphicalabstract}

%%Research highlights
\begin{highlights}
\item Research highlight 1
\item Research highlight 2
\end{highlights}
\end{comment}

\begin{keyword}
%% keywords here, in the form: keyword \sep keyword
Bioacoustics \sep Passive acoustic monitoring \sep Deep neural network \sep Self-supervised learning \sep Contrastive learning \sep Bird classification  
%% PACS codes here, in the form: \PACS code \sep code
%\PACS 0000 \sep 1111
%% MSC codes here, in the form: \MSC code \sep code
%% or \MSC[2008] code \sep code (2000 is the default)
%\MSC 0000 \sep 1111
\end{keyword}

\end{frontmatter}

%% \linenumbers

%% main text
\section{Introduction}
\label{sec:introduction}
Biodiversity monitoring is a critical aspect of biodiversity conservation, as it helps inform decision making, improves our knowledge and enhances public education and awareness. Birds are one of the most surveyed animal groups in biodiversity monitoring programmes, with point counts and transect surveys being well-established survey techniques for monitoring bird communities~\cite{bibby2000}. However, birds can be very difficult to detect and identify especially in tropical regions characterised by high avian diversity and numerous rare species \cite{Robinson2000}, \cite{Robinson2018}. Additionally, such manned survey techniques are manpower-intensive, require highly specialized expertise, and tend to overlook rare species that are sensitive to human presence \cite{darras2018}, \cite{Darras2019}, \cite{Wheeldon2019}.

Passive monitoring of biodiversity using acoustics is thus an area of great potential, as various animal groups including birds make unique vocalizations, which can be used to validate their presence. Such systems allow for automated collection of large amounts of audio data without human supervision and can survey cryptic species more effectively \cite{darras2018}, \cite{Hoefer2023}. However, the comprehensive analysis of such large volumes of data is prohibitive in terms of the man-hours required \cite{Hingston2018}. This constraint, and the rapid advancement in machine learning techniques, have made data-driven algorithms increasingly popular for bioacoustic species detection and classification tasks. The dominant approach in this space involves feeding time-frequency spectrograms of acoustic recordings to some variant of a Convolutional Neural Network for feature extraction and classification~\cite{Stowell2022}. What many of these methods have in common, however, is the need for extensive sets of expert-labeled training data. As an example, the initial BirdNET classifier~\cite{KAHL2021} was trained to classify close to 1000 different bird species but with about 1,500 spectrograms per class on average. The general need for large datasets may in part be attributed to the excessive information present in broadband spectrograms, which requires the model to learn to distinguish the signal of interest from the noise, and in part because many labeled datasets used for training are weakly labeled, i.e., class labels are typically assigned at spectrogram level without information about the exact time and frequency of the event. Transfer learning has emerged as a common technique to address the shortage of task-specific data. The idea is to improve a model's performance on a specific task by leveraging knowledge gained from a model previously trained on a different but related task. In acoustic classification, it is common to leverage models pre-trained on large datasets of either images or generic audio (typically ImageNet~\cite{ImageNet} or AudioSet~\cite{AudioSet}), and fine-tune these networks on task-specific data~\cite{LeBien2020},~\cite{Manriquez2024},~\cite{Tsalera2021}. Although transfer learning can be effective, one is constrained by the architecture and input format of the pre-trained models, which may be suboptimal for the target task. Models pre-trained on ImageNet, for example, typically require inputs of size $224\times224\times3$, constraining the selection of duration, bandwidth, and time-frequency resolution of the spectrograms. A different but closely related approach, known as meta-learning, trains a model on a set of different tasks with the aim to generalize to new tasks with very few training samples. Several promising few-shot learning approaches on bioacoustic data are presented in~\cite{Nolasco2023}, but one still needs to curate a labeled dataset on a diverse set of tasks to train the initial model before the few-shot learning can take place.

How to handle soundscapes with temporally-dense vocalizations is another challenge in bioacoustics. This is particularly relevant for birds, as many bird species are especially vocal during dawn and dusk, resulting in vocalization-dense soundscapes with frequent time-overlapping calls. One approach is to train multi-label classifiers, i.e., classifiers that can predict multiple target classes from a single input. Multi-labeled datasets, however, as compared to single-labeled datasets, are far less prevalent, harder and more tedious to accurately annotate~\cite{Briggs2012}, and, as a consequence, are not always exhaustive, which can inhibit learning~\cite{Wood2023}. Moreover, multi-label classification requires more training data because the problem is inherently harder. The lack of large, high-quality multi-labeled datasets in bioacoustics may be one reason why most research focuses on single-labeled problems (e.g.~\cite{Lasseck2018},~\cite{Gupta2021},~\cite{Thakur2019},~\cite{Nolasco2023},~\cite{Manriquez2024}). Methodologies targeting multi-label classification tend to divide spectrograms into shorter windows, apply classifiers (either single-label or multi-label) to each window, and aggregate the outputs to obtain predicted scores for all species present in the full spectrogram~\cite{KAHL2021}, \cite{LeBien2020}, \cite{Noumida2022}. The motivation behind this split-and-aggregate strategy is likely to increase the chance of capturing single vocalizations at a time, which reduces the multi-label problem to a set of single-label problems. A shortcoming of this approach, however, is that the models are fed spectrograms of fixed duration and bandwidth, while the duration and bandwidth of vocalizations vary, effectively including excessive information and allowing time-overlapping calls to enter the same input. In \cite{Briggs2012}, the authors address this issue by employing a supervised, pixel-level segmentation technique to separate calls in time and frequency prior to classification. This method, however, requires labeled training data for both segmentation and classification. In \cite{Hexeberg2023}, an object detection technique was used to detect marine mammals acoustically. Although this approach can detect time-overlapping vocalizations, it requires hard labels in the form of bounding boxes enclosing the signals of interest. This, and other labour-intensive annotation processes, may be acceptable if the objective is to detect a few specific vocalizations, but does not scale well to the vast diversity of bird vocalizations.

Lastly, most studies to date in the field of passive acoustics monitoring have involved largely pristine habitats while urban ecosystems have largely been overlooked \cite{Fairbrass2017}, \cite{Gibb2019}. As urban green spaces become increasingly important for bird populations due to the rapid rate of urbanisation \cite{Hughes2022}, evaluating the efficacy of utilising automated analysis for urban soundscapes is particularly urgent. The highly urbanised city-state of Singapore is an ideal study site as it is one of the few tropical cities that has a network of connected urban green spaces close to densely populated urban areas \cite{Wong2023}.

To target these challenges, we propose a semi-supervised, passive-acoustic bird classifier designed to allow detection of time-overlapping vocalizations (when separated in frequency) without requiring a large number of labeled data for training. We assess its performance on both open-source recordings from Singapore and long-duration, continuous soundscape data recorded at two different sites within the Singapore Botanic Gardens (SBG) -- one of the oldest botanic gardens in Southeast Asia which receives millions of visitors annually.

\section{Methodology}

The proposed method consists of four main steps which are explained in detail in the remainder of this section:
\begin{enumerate}
    \item \textbf{Segmentation:} extract individual bird calls with an energy-based segmentation technique. The isolation of individual bird calls limits noise, enabling high data compression, and allows time-overlapping calls to be treated separately as long as they do not also overlap in frequency. A consequence of this approach is that single calls/songs may be split into multiple segments.\label{item:segmentation}
    \item \textbf{Data compression:} learn a compressed representation of the segments while retaining most of the information.\label{item:autoenc}
    \item \textbf{Embedding:} use the representation from step~\ref{item:autoenc} to learn a new representation (embedding) to ensure both translational invariance and that similar sounds have similar embedding -- two key properties for efficient clustering and classification.\label{item:clr}
    \item \textbf{Classification:} curate a set of labeled data and train a classifier using the embeddings from step~\ref{item:clr}. This training process also serves as a final refinement of the embedding.  
\end{enumerate}

Note that step~\ref{item:segmentation} does not involve any learning and step~\ref{item:autoenc}~and~\ref{item:clr} are self-supervised, i.e., no labeled data is required. Training the self-supervised networks on large datasets allows the supervised classifier in the final step to distinguish the classes with a much smaller set of labeled data.

\subsection{Data collection}

\begin{figure*}[t]
\centerline{\includegraphics[width=\linewidth,keepaspectratio]{site1_site2.png}}
    \caption{Left: first deployment at the SBG (site~\#1) from July 4, 2020 until September 20, 2020. The approximate locations of the recording units are marked as yellow circles, with the corresponding external microphones of each unit marked as orange circles. The 6 microphones surround a lake and cover an area of roughly 50~m $\times$ 50~m. Right: a section of the elevated boardwalk used for the second deployment at the SBG (site~\#2) from September 20, 2020 until February 1, 2021. The same 6 microphones from site~\#1 were deployed along the circular boardwalk in a similar constellation, but without direct line of sight between units due to the dense vegetation. Photos from Google Maps.}
    \label{fig:sbg_sites}
\end{figure*}

Our primary source of acoustic data was acquired from two different locations in the SBG over a combined period of 7~months from July 2020 to February 2021. Three \emph{Wildlife Acoustics SongMeter 4 TS} recorders were deployed at both locations. Each recording unit was equipped with two omni-directional microphones, yielding soundscape recordings from 6 microphones simultaneously. The first 2.5~months of data collection took place around a lake with minimal obstruction between the microphones (site~\#1). This setup allows the same call to be detected on multiple recorders (Figure~\ref{fig:sbg_sites}), which we later leverage to train the contrastive network (Subsection~\ref{sec:clr}). The units were then re-located to a second site for the remaining 4.5~months (site~\#2). A similar constellation was used but dense vegetation occluded the direct paths between microphones (Figure~\ref{fig:sbg_sites}). During calibration tests at site~\#2, we emitted high-energy, transient sounds and found that the sounds were mostly audible only on the nearest few microphones. Although low-frequency sounds may attenuate less, site~\#2 is likely to have a significantly lower detection range.

Lastly, the collected soundscape recordings were complemented with 333 carefully selected bird recordings of species known to be present in Singapore from the Xeno-canto database\footnote{https://xeno-canto.org}. These community-curated recordings served as a faster way to obtain an initial set of local bird calls without having to search in long, unprocessed recordings.

\subsection{Time-frequency representation}
\label{sec:tfr}

\begin{figure*}[t]
\centerline{\includegraphics[width=\linewidth,keepaspectratio]{tfr-split.png}}
    \caption{An example illustrating extraction of TFRs from audio recordings. The left panel shows the spectrogram from a 5~second audio clip, with a few detected sounds enclosed by white rectangles. The right panels show the respective extracted TFRs. This example is non-exhaustive, i.e., not all detections in the audio clip are shown here.}
    \label{figTfrSplit}
\end{figure*}

Most bird vocalizations contain time-frequency transients. While vocalizations may overlap in time, they may be separated in frequency. Based on this observation, time-frequency transients are extracted from acoustic recordings as the initial stage of the technique. The extracted time-frequency transients are represented as a $128 \times 256$ matrix of numbers referred to as the time-frequency representation (TFR). TFRs are extracted from acoustic recordings through the following steps:
\begin{enumerate}
    \item Compute a spectrogram of the acoustic data with 2,048 FFT bins, a Hamming window, and an overlap of 1,536 samples between windows. Retain the frequency bins between 500~Hz and 15~kHz only, as this frequency range adequately covers most bird sounds while rejecting other unwanted noises.
    \item Convert the spectrogram to dB. Using the inter-quartile range for each frequency bin to obtain a robust estimate of the noise variance $\sigma$ at that frequency. Normalize each frequency bin by subtracting the median + $2\sigma$, dividing by $2\sigma$, and lastly clipping the resulting data between 0 and 1. This adaptively extracts regions in the time-frequency plane that have significantly higher energy than the background noise at that frequency. Moreover, it reduces the natural dominance of low-frequency calls resulting from higher attenuation of high-frequency signals~\cite{Sutherland1998}.
    \item Reduce the frequency resolution by a factor of 5 by max-pooling.
    \item Blank out time bins with low variance across broad frequency bands, as these represent impulsive sounds not characteristic of birds.
    \item Perform a watershed segmentation of the resulting spectrogram to obtain disconnected regions of high energy in the spectrogram.\label{item:watershed}
    \item Filter out regions with very short durations or very small time-bandwidth products, as these are uncharacteristic of bird sounds.    
\end{enumerate}

The end result of the above steps is best illustrated through an example (Figure~\ref{figTfrSplit}). The spectrogram on the left is from a 5~second long recording with multiple bird vocalizations. After going through the above steps, the spectrogram is converted to a number of TFRs. A few of these TFRs are shown in the panels on the right with corresponding sections marked with white rectangles in the original spectrogram on the left.

Although the TFR representations have variable duration, many of the algorithms used in our signal processing chain require a constant duration input. We pick a constant duration of 2.7~seconds (256 time bins for the TFR), and convert all TFRs to this duration when feeding to algorithms that require a constant duration input. This duration is sufficient to cover most of the TFRs we obtain. For the occasional TFR longer than 2.7~seconds, we randomly select a 256-sample section, while shorter TFRs are randomly zero-padded on both sides to make them 256 samples long. The random padding or clipping is done on the fly to create different versions of the TFR each time it is used. Note that in contrast to methods that operates on entire spectrograms, the complexity level of the downstream classification task is not impacted by the call density as long as single TFRs do not capture multiple calls.

\subsection{Auto-encoding}

While TFRs may be a good visual representation for humans to classify bird vocalizations, it is not necessarily a good representation for a machine. The TFRs are also generally sparse, with most entries containing zero energy. The auto-encoder stage of the processing learns a compressed representation of the TFR that retains most of the information from the original TFR, but using a much smaller number of coefficients. The learning is self-supervised, i.e., no labeled data is required. The auto-encoder simply seeks to reduce the error between the original TFR and a reconstructed TFR with the constraint that the intermediate representation of the TFR (the \emph{latent representation}) only contains 512 values (instead of the $128 \times 256$ values in the original input TFR).

We use a convolutional deep auto-encoder to achieve this compression (Figure~\ref{figAutoEnc}). To train the auto-encoder we build a dataset of TFRs extracted from about 90 microphone-hours of recordings, primarily from the two SBG locations and a smaller subset obtained from the Xeno-canto recordings. To encourage the model to emphasize learning of bird sounds over sounds from other sources, we additionally add three identical sets of the curated TFRs which we later use to train the final classifier (Subsection~\ref{sec:classifier}). This brings the final count to 228,042~TFRs. From these, 5,000~TFRs are used for validation and the balance 223,042~TFRs for training. We use a mean-square error loss function
and train the auto-encoder over 97~epochs using the Adam optimizer~\cite{ADAM}. The auto-encoder is capable of retaining most of the information in the TFRs, as illustrated in Figure~\ref{fig:decoded_tfrs}. After training, the encoder section of the auto-encoder is kept and used as a pre-trained, second-stage processor for the remaining stages of the algorithm.

\begin{figure*}[t]
\centerline{\includegraphics[width=\linewidth,keepaspectratio]{autoenc.png}}
    \caption{Architecture of the convolutional auto-encoder. The network enables a $64 \times$ data compression by learning a latent representation of the TFRs which retains most of the information.}
    \label{figAutoEnc}
\end{figure*}

\subsection{Contrastive representation learning}
\label{sec:clr}

While the latent representation from the previous section holds information related to bird vocalization in a compressed form, it is not a suitable representation for clustering or classification tasks. Firstly, auto-encoder latent spaces are translation equivariant, i.e., when the input TFR is shifted in time, the output TFR also shifts in time. To do this, the latent representation must retain time information. However, a bird sound shifted in time doesn't change the bird, and so we seek translation invariance rather than translation equivariance. Secondly, very similar sounds can have very different latent space representations, as nothing in the training process impose any constraint to force similar sounds to have similar representations. We next use the idea of contrastive learning to discover a preliminary embedding that is invariant to time translation and where similar sounds have similar representations.

Traditional contrastive learning~\cite{chen2020simple} obtains a pair of samples from each training sample by augmenting the training sample in two different ways. The pair is thus guaranteed to be similar, but not the same. Training the contrastive learning network then involves the design of a loss function that requires the samples in each pair to have similar representations, but samples from different pairs to differ as much as possible. This permits self-supervised training without the need for labeled samples, and the network learns a representation that is invariant to the augmentation used to create a pair of samples.

We follow the same basic approach outlined above, but change some of the details in some critical ways. Other than a random translation that is inherent in obtaining a TFR of constant duration (Subsection~\ref{sec:tfr}), we do not perform any augmentation. Instead, sample pairs are derived from recordings of the same bird vocalization on the two microphones connected to each recorder unit. The natural variability in sound propagation thus provides the desired ``augmentation". Since the sound reaches both microphones at potentially different times, some effort is required to associate the sounds on both microphones. Using a combination of time information and a requirement for high cross-correlation between the acoustic time-series of both sounds, we obtain 19,311 reliable TFR pairs from the two SBG deployments. Of these, 500~pairs are used for validation and the balance 18,811 for training. The contrastive learning network outputs a learnt representation space referred to as the \emph{embedding space} hereafter (Figure~\ref{fig:clr}). The training is performed over 50~epochs using the Adam optimizer, but with a loss function that is different from the one proposed in~\cite{chen2020simple}.

\begin{figure}[t]
  \includegraphics[width=\linewidth]{decoded_tfrs.png}
  \caption{Examples of calls from (a): Crimson Sunbird (\textit{Aethopyga siparaja, 558466}), (b): Common Hill Myna (\textit{Gracula religiosa, 179652}), (c): Olive-winged Bulbul (\textit{Pycnonotus plumosus, 562623}) and (d): Lineated Barbet (\textit{Psilopogon lineatus, 1145226}). The top row shows the TFRs after extraction from raw audio recordings, and the bottom row shows the compressed TFRs after passing through the auto-encoder. The high similarity between each pair shows that the compressed latent representation is capable of retaining most of the information in the TFRs. To limit sounds from different sources from merging, TFRs do not capture entire calls/songs if the pause between subsequent vocalizations are too long. The Olive-winged Bulbul in column c is one such example, where only a part of a longer call sequence is captured.}
  \label{fig:decoded_tfrs}
\end{figure}


\begin{figure*}[t]
  \includegraphics[width=\linewidth]{contrastive-network.png}
  \caption{Architecture of the contrastive learning neural network. The $\tanh(\cdot)$ activation function in the last dense layer ensures all entries in the 1024-dimensional embedding space representation are positive, and the final normalization layer ensures that they are scaled such that the embedding space can be thought of as the surface of a hypersphere of unit radius. The similarity between embedding space representations can then be measured in terms of the dot product of the corresponding vectors. The distance between embedding space representations can be measured as the angle between vectors, or equivalently the distance on the surface of the hypersphere.}
  \label{fig:clr}
\end{figure*}



Although the loss function proposed in~\cite{chen2020simple} is theoretically sound, it was developed for a computer vision task quite different from our needs and did not perform well on our problem. Based on the intuition that the embedding space can be modeled as a surface of a 1024-dimensional hypersphere, we do not desire maximal angular separation between dissimilar sounds, but rather orthogonality. Maximal angular separation pushes dissimilar sounds to diametrically opposite sides of the hypersphere, but that can only accommodate two classes. Instead, orthogonality pushes dissimilar sounds to different axes of the hypersphere, and can support up to 1024 distinct classes for a 1024-dimensional hypersphere. With this in mind, the loss function we use is:

\begin{align}
    &L(\mathbf{Z}) &=& \sum_{p=1}^N l_{2p,2p-1} + l_{2p-1,2p}, \label{eq:loss1} \\
    &l_{i,j} &=& -\left[ \mathbf{z}_i^T\mathbf{z}_j + \beta \sum_{k \ne i, k \ne j}^{2N} \min\left( 1 - \mathbf{z}_i^T\mathbf{z}_k, \; 1 \right)^2 \right]. \label{eq:loss2}
\end{align}


Here $\mathbf{Z} = \{\mathbf{z}_i \forall i\}$ is the set of embedding space representations for the training TFRs, organized such that the first pair is $(\mathbf{z}_1,\mathbf{z}_2)$, the second pair is $(\mathbf{z}_3,\mathbf{z}_4)$, and so on. The two terms in~(\ref{eq:loss1}) correspond to two possible orderings of samples in each pair. The first term in~(\ref{eq:loss2}) maximizes the similarity within each pair, while the second term induces orthogonality for non-paired representations. The $\min(\cdot, 1)$ function prevents maximal angular separation and $\beta = 3$ is a hyper-parameter that controls the balance between the two terms. In practice, the loss is not evaluated over the entire dataset, but in mini-batches of $N=512$.  

After training, an average similarity score $\mathbf{z}_i^T\mathbf{z}_j$ of 0.93 was obtained for paired samples, and near orthogonality for non-paired samples. While this ensures that paired samples (similar sounds) have high similarity, it does not guarantee low similarity between every pair of dissimilar sounds. Since the loss function minimizes the \textit{average} similarity between dissimilar sounds, some clusters of dissimilar sounds still experience very similar representations. This is addressed by allowing the embedding space representation to further improve in a final, supervised classification stage.

\subsection{Supervised refinement and classification}
\label{sec:classifier}

\begin{figure*}[t]
  \includegraphics[width=\linewidth]{full-network.png}
  \caption{The complete classification architecture. TFRs extracted from raw recordings are transformed to compressed embedding vectors via a pre-trained auto-encoder followed by a pre-trained contrastive network. These embeddings vectors are then fed as input to a 4-layer classification network (right), which in turn assigns confidence scores to each predefined class. To allow the embedding network to improve further, its model parameters are not frozen during the final, supervised classification training. However, to ensure the embedding network does not forget its primary objective, it is trained once for every 5 epochs of classification training.}
  \label{fig:classifier}
\end{figure*}

The final stage of the model is a classifier with four dense layers and batch normalization between each layer. It takes the preceding embedding vector as input and assigns a confidence score to each predefined class (Figure~\ref{fig:classifier}). As a single species can produce a wide range of calls with very different time-frequency representations, and since the preceding contrastive network operates on compressed TFRs from the auto encoder, the classifier is trained at TFR-level rather than at species-level. Consequently, the classifier does not only learn to differentiate between species but also between different calls from same species. However, TFRs are not guaranteed to capture entire call sequences. As a result, single calls or songs are sometimes fragmented into multiple TFRs and, in turn, assigned to separate classes if significantly distinct.

About 2/3 of the training data is obtained by extracting TFRs from Xeno-Canto files which are first manually labeled. These recordings are typically characterized by high SNR and are relatively short in duration (tens of seconds to a few minutes long), resulting in few but very clean samples for many classes. These clean samples, however, are not necessarily representative of the chaotic soundscape during peak chorus hours. The remaining 1/3 of the data is obtained by applying the pre-trained embedding network (Subsection~\ref{sec:clr}) to search for clusters of similar TFRs in soundscape recordings from the two SBG sites. The embedding enables an efficient labeling process as whole clusters can be labeled by inspecting only a few samples. Classes with less than 3 samples are discarded, resulting in 5727~TFRs distributed over 357 classes, representing 123 bird species. The dataset, however, is highly imbalanced. While most classes only have a handful of samples, others have hundreds. When splitting the dataset into training, validation and test sets, we prioritize TFRs for training while ensuring that each class has at least one validation and one test sample. To account for the class imbalance, we randomly augment original TFRs from low-sample classes with minor shifts in frequency, minor time stretching, and by adding Gaussian and white noise, obtaining a balanced distribution with exactly 50 samples per class for training, and a minimum of 20 samples each for validation and testing. 

The classifier will only extract and learn the necessary features from the input in order to separate the classes in the training set and minimize the cost function. When applying the model on continuous field recordings, however, the model is exposed to a much greater variety of sounds than it has seen during training, many of which will have very similar embeddings as some of the sounds from the trained classes. To make the model more robust against false positives in such environments, two additions to the algorithm are made. First, a separate \textit{bird-pass filter} is trained to distinguish bird sounds from generic non-bird sounds. This binary classifier has three dense layers, takes encoded TFRs from the auto-encoder as input and classify them as either birds or non-birds. The training data for the bird class is the same as for the main classifier, while the non-bird class is trained on 3152~TFRs extracted primarily from the UrbanSound8K dataset~\cite{UrbanSound8k} and in part from underwater recordings of marine mammal vocalizations, previously collected in Singapore waters. To avoid class imbalance, the non-bird sounds are oversampled with random frequency shifts to match the same sample size as the bird class. As even a small false positive rate can translate to a large number of false positives in raw recordings, we value precision over recall for bird classes. To encourage this precision-recall trade-off in the bird-pass filter, false positives are penalized more for the bird class than for the non-bird class, yielding a $F_{0.5}$ score of 0.91 for the bird class on a hold-out test set.

Secondly, an additional ``sink" class is added to the primary classifier. Since most of the non-bird sounds used to train the bird-pass filter are unlikely to be present in our local environment, the sink class is trained on local sounds from the SBG. Specifically, it is trained with 1893~TFRs from site~\#1 which initially confused the model and caused many false positives. The sink class helps the model to learn lower-level features for classes which are easily confused with other local sounds, while also providing the neural network with a class to assign TFRs that do not belong to any of the trained bird classes. We use a standard cross-entropy loss function for the main classifier, but we penalize false negatives for the sink class by an effective\footnote{As the sink class has almost 38 times more samples than each bird class, we weigh the sink class roughly by a factor of $\frac{3}{38}$ in the loss function to achieve an effective sink weight of $3 \times$ that of each bird class.} factor of $3\times$ that of the bird classes, to discourage sink samples from being misclassified as birds.

Lastly, to address the challenge that some clusters of dissimilar sounds occur close in embedding space (Subsection~\ref{sec:clr}), the classifier is trained end-to-end without freezing any of the preceding layers. This allows both the auto-encoder and the contrastive learning network to improve further. During this training, however, the trained embedding network can ``forget" the need for paired samples to have high similarity and unpaired samples to have low average similarity. To reinforce this need, we simultaneously train the embedding network with the same contrastive loss as before (Subsection~\ref{sec:clr}). In practice the training is implemented in sets of 5 epochs of the primary cross-entropy loss, followed by 1 epoch of the auxiliary contrastive loss. We train the network over 95 such sets and retain a mean similarity score of 0.866 for paired samples and near orthogonality (0.046) for unpaired samples.


\section{Results}

We evaluate the performance of the classifier in two settings. First, we assess its performance on a test set that neither the auto-encoder, the contrastive network, nor the classifier have encountered during training. We also compare the test set performance with the prevailing open-source acoustic bird detector, BirdNET~\cite{KAHL2021}. Secondly, we assess how the model performs on raw, continuous soundscape recordings from the SBG to better understand its capabilities and limitations when deployed for extensive periods of time in a typical monitoring setup.

\subsection{Evaluation on test set}
\label{sec:testset}

\begin{table}[b]
    \renewcommand{\arraystretch}{1.2}
    \caption{Classification performance on the test set.}
    \label{tab:ClassifierPerformance}
    \begin{tabular*}{\linewidth}{@{\extracolsep{\fill}}l|cc}  
        \toprule
         & Bird classes & Sink class \\
        \midrule
        Number of classes   & 315      & 1         \\
        Total test samples  & 6435     & 200       \\
        Precision           & 0.799    & 0.076    \\
        Recall              & 0.585    & 0.870    \\
        $F_{0.5}$           & 0.701    & 0.093    \\
        \bottomrule
    \end{tabular*}
\end{table}


Although the TFR samples in each class originate from the same call category, there may still be individual variations, e.g. in duration (particularly for repetitive calls), bandwidth, and frequency modulation rate (for example the steepness of up- and downsweeps). Disturbances from extraneous sources, differences in SNR levels, and artifacts from the data augmentation process can further add to the TFR variation within each class. For classes with many training samples, such diversity can enhance the model's capability to generalize as it forces the model to identify the distinguishing features of each class while ignoring irrelevant information. However, many of our classes have very few training samples, so when evaluating the model on the test set, we do not consider classes where most test samples differ significantly from the training. This reduces the test set to 315 bird classes across 110 species. Moreover, we assign low-confidence predictions (confidence scores below 0.5) to the sink class, which serves as a collector of extraneous TFRs. The sink class precision is consequently low, but more importantly, it obtains a high recall of 0.870, meaning that only 13\% of the sink samples are misclassified as birds (Table~\ref{tab:ClassifierPerformance}). The bird classes achieve a mean $F_{0.5}$ score of 0.701 with a desirable asymmetry between precision (0.799) and recall (0.585). The performance for individual bird classes, however, is skewed: 37 classes have a $F_{0.5}$ score of 0.0 while 111 classes have a $F_{0.5}$ score above 0.9. The zero-score classes fall primarily into one or more of the following categories: 1) correct class but low confidence, 2) class not learned during training, i.e., $F_{0.5} = 0.0$ on the training set, 3) confusion with similar classes, or 4) low similarity between test and training samples.

\subsection{Comparison with BirdNET}

To benchmark the model, we compare its performance against the BirdNET classifier~\cite{KAHL2021}. BirdNET is a leading open-source acoustic bird detector, extensively used by bird enthusiasts and researchers worldwide, and it currently covers more than 6,000~bird species~\cite{birdnet_github}. We use the same test set as in Subsection~\ref{sec:testset} for evaluation but differences between the two models require some adjustments. First, we consider only non-augmented samples. Second, as BirdNET is trained to detect species, we also evaluate our model at species-level rather than at TFR-level. Lastly, we exclude the few species that BirdNET is not trained to detect. The resulting test set consists of 943~samples from 103~species. The number of samples per species ranges from 1 to 101, with a median of 3 samples, meaning that only a handful of samples are available for most species. All samples lie within the 0-15~kHz frequency range covered by BirdNET.  

We test the latest BirdNET version available via the Python Package Index (PyPi) repository~\cite{birdnet_pypi} (version 0.1.6, released September 4, 2024). While our model takes TFRs as input (isolated in time and frequency), BirdNET searches for detections in broadband spectrograms. To reduce the probability of capturing multiple species in the recordings fed to BirdNET, we feed it 3-second-long raw recordings centered at each TFR (i.e., the minimum input duration that avoids signal padding). Since we have only ground truth annotations at TFR-level, we cannot guarantee that other species are absent from the 3-second recordings. However, BirdNET predicts between 0 and 7 species with confidence scores above 0.1 across the test set~\footnote{BirdNET outputs a confidence score between 0.0 and 1.0 for each species (multi-label) but predictions below 0.1 are by default not presented in the Python API}, with an average of 1.3~species, suggesting that multi-species presence is not a substantial issue. Nevertheless, as some samples may contain multiple species, we compare the models using top-k accuracy with $k=1$ and $k=3$, i.e., we consider a prediction correct as long as one of the top-k predictions (ranked by confidence score) matches the ground-truth. We evaluate the mean top-1 and top-3 accuracy, both globally (where each sample is weighted equally) and species-averaged (where each species class is weighted equally) to account for the sample imbalance between species (Table~\ref{tab:BirdNET_vs_BirdWatch}). Our model achieves significantly higher accuracy than BirdNET across all metrics on the test set, despite half of the 103~species are trained with less than 16 labeled samples each.

\begin{table}
    \renewcommand{\arraystretch}{1.2}
    \caption{Performance comparison between BirdNET and our model on 943 test samples across 103 species.}
    \label{tab:BirdNET_vs_BirdWatch}
    \begin{tabular*}{\linewidth}{@{\extracolsep{\fill}}l|cc}  
        \toprule
         & BirdNET & Our model \\
        \midrule
        Top-1 accuracy      & 0.51      & 0.81      \\
        Top-3 accuracy      & 0.66      & 0.89      \\
        Top-1 accuracy (species-averaged)    & 0.50     & 0.73 \\
        Top-3 accuracy (species-averaged)    & 0.64   & 0.82 \\
        \bottomrule
    \end{tabular*}
\end{table}


\subsection{Evaluation on continuous data}

 \begin{figure*}[!thb]
  \includegraphics[width=\linewidth]{diurnal_and_nocturnal_by_hour.png}
  \caption{Mean detections by hour of day for dirunal birds (top row) and nocturnal birds (bottom row) over 10 days of recordings at each site. The top 5 species with highest detection count at each site are listed by name. Note that the y-axis differs between the plots.}
  \label{fig:diurnal_and_nocturnal_species_by_hour}
\end{figure*}

Although the classifier has never seen the exact TFRs in the test set before, they all derive from classes which the model is trained to separate. Deploying the detector out in the field, on the other hand, exposes the classifier to a greater variety of sounds, many of which may be easily confused with some of the trained classes. In this part, we assess the model performance on raw, continuous streams of data, aiming to resemble a typical long-term monitoring situation. We first manually evaluate the model on a single day of soundscape recordings to shortlist classes with a reasonable level of precision. For practical reasons, we estimate class precision but not recall. That is, we verify the number of correct detections but do not attempt to estimate the number of missed detections as it would require very detailed, manual annotations of tens of hours of raw recordings. The shortlisted classes are then analyzed on 10 full days of recordings at both SBG sites to further evaluate the classifier but at an aggregated level without manual verification of each detection.

We randomly select a day in July 2020 from site~\#1 for the manual precision test and run the trained classifier on all 144 microphone-hours of recordings from that day (24 hours $\times$ 6 microphones). This yields about 250,000 TFRs prior to filtering. We disregard TFRs that either 1) do not pass the binary bird-pass filter (threshold = 0.5), 2) have a confidence score below 0.7, or 3) are assigned to the sink class. Additionally, we leave out 68 classes from species that are either known not to be present in the SBG or that are migratory and consequently not present in July. However, we do not ignore classes with low performance on the test set, as low test set performance may be a consequence of the misalignment between training and test samples mentioned earlier (Subsection~\ref{sec:testset}), and not necessarily an indicator of poor model performance.

Among the remaining classes, we focus on those with distinct call characteristics for more reliable verifications. Up to 50 detected TFRs per class are randomly selected for manual verification. We keep classes with a minimum precision of 0.5, yielding 27 classes across 16 species (Table~\ref{tab:precisionContinuousData}). Since the model is trained at TFR level rather than at species level, some species appear in several classes. A few classes achieve very high precision but based on very few detections. The ``training samples" column lists the number of training samples before augmentation and suggests no strong correlation between precision and the number of training samples. The 9 classes with only 6 or fewer samples have a mean precision of 0.75 while the remaining 18 classes have a mean precision of 0.80.

For the aggregated evaluation, we run the detector on 10 randomly selected days between July and September 2020 for site~\#1 and 10 days between October 2020 and January 2021 for site~\#2. With 6 microphones recording in parallel (and occasional downtime), this translates to approximately 1350 microphone-hours of recordings for each site. For each class and hour of day, the detection rate is calculated as total detections divided by total hours of microphone recordings for that hour over the 10-day period. Located only $1.3^\circ$ north of Equator, Singapore experiences minimal changes in the timings of dawn and dusk over the year, with sunrise and sunset around 7~am and 7~pm every day. Detections are therefore naturally aligned with dawn and dusk without additional correction, despite spanning several months in time (Figure~\ref{fig:diurnal_and_nocturnal_species_by_hour}).

At site~\#1, the diurnal birds become vocally active around dawn with a distinct peak between 7 and 8~am. The activity then drops but stays quite high until it reaches a low-point between 2~pm and 3~pm. The activity rebounds in the late afternoon and stays high during dusk. There are very few detections of diurnal birds after 8~pm, until the Red Junglefowl (\textit{Gallus gallus, 176086}) becomes active in the early morning between 5 and 6~am. The only nocturnal bird among the shortlisted ones, the Sunda Scops Owl (\textit{Otus lempiji, 1063234}), is virtually absent during daytime and in the early evening until 10~pm. The majority of its detections occur from 2~am with a peak around 4--5~am, before detections plummet prior to the diurnal morning chorus.

Site~\#2 experiences a similar morning chorus peak as site~\#1 but a more distinct evening peak between 6 and 7~pm -- largely attributed to the Asian Glossy Starling (\textit{Aplonis panayensis, 558724}). This site also experiences few detections from diurnal birds after dusk but relatively many between midnight and dawn. Apart from the overall daily patterns, site~\#1 experiences about ten times higher detection rates compared to site~\#2.

\begin{table}[thbp]
\centering
\small
%\begin{tabular}{@{}ccllcccc@{}}
\begin{tabular}{@{}p{0.7cm}p{1.3cm}llcp{1cm}p{1cm}p{1cm}@{}}
\toprule
\textbf{Class} & \textbf{Training samples} & \textbf{Common name}                & \textbf{Scientific name} & \textbf{TSN} & \textbf{Dete- \newline ctions} & \textbf{True positives} & \textbf{Prec- \newline ision} \\
\midrule
1              & 33                        & Asian Glossy Starling           & \textit{Aplonis panayensis}            & 558724  & 50                   & 44                       & 0.88               \\
2              & 6                         & Black-naped Oriole              & \textit{Oriolus chinensis}    & 561694  & 14                   & 10                       & 0.71                \\
3              & 4                         & Blue-eared Kingfisher           & \textit{Alcedo meninting}    & 554552  & 50                   & 30                       & 0.60                \\
4              & 34                        & Brown-throated Sunbird          & \textit{Anthreptes malacensis}		  & 558621  & 36                   & 34                       & 0.94                \\
5              & 4                         & Brown-throated Sunbird          & \textit{Anthreptes malacensis}		  & 558621  & 50                   & 41                       & 0.82                \\
6              & 3                         & Changeable Hawk-eagle           & \textit{Nisaetus cirrhatus}   		  & 824093  & 2                    & 2                        & 1.00                \\
7              & 50                        & Common Hill Myna                & \textit{Gracula religiosa}    & 179652  & 50                   & 42                       & 0.84                \\
8              & 6                         & Common Hill Myna                & \textit{Gracula religiosa}    & 179652  & 24                   & 22                       & 0.92                \\
9              & 3                         & Common Hill Myna                & \textit{Gracula religiosa}    & 179652  & 1                    & 1                        & 1.00                 \\
10             & 6                         & Crimson Sunbird                 & \textit{Aethopyga siparaja}            & 558466  & 10                   & 7                        & 0.70                 \\
11             & 1                         & Greater Racket-tailed Drongo    & \textit{Dicrurus paradiseus}  		  & 559773  & 14                   & 7                        & 0.50                 \\
12             & 11                        & Olive-backed Sunbird            & \textit{Cinnyris jugularis}   		  & 916461  & 34                   & 33                       & 0.97                \\
13             & 15                        & Olive-backed Sunbird            & \textit{Cinnyris jugularis}   		  & 916461  & 20                   & 15                       & 0.75                \\
14             & 50                        & Olive-winged Bulbul             & \textit{Pycnonotus plumosus}           & 562623  & 50                   & 40                       & 0.80                 \\
15             & 22                        & Olive-winged Bulbul             & \textit{Pycnonotus plumosus}           & 562623  & 32                   & 25                       & 0.78                \\
16             & 30                        & Olive-winged Bulbul             & \textit{Pycnonotus plumosus}           & 562623  & 50                   & 26                       & 0.52                \\
17             & 33                        & Olive-winged Bulbul             & \textit{Pycnonotus plumosus}           & 562623  & 50                   & 42                       & 0.84                \\
18             & 6                         & Red-whiskered Bulbul            & \textit{Pycnonotus jocosus}   		  & 178507  & 50                   & 25                       & 0.50                \\
19             & 33                        & Red Junglefowl                  & \textit{Gallus gallus}        		  & 176086  & 50                   & 44                       & 0.88                \\
20             & 26                        & Red Junglefowl                  & \textit{Gallus gallus}        		  & 176086  & 50                   & 33                       & 0.66                \\
21             & 50                        & Red Junglefowl                  & \textit{Gallus gallus}        		  & 176086  & 6                    & 5                        & 0.83                \\
22             & 50                        & Spotted Dove                    & \textit{Spilopelia chinensis} 		  & 1125210 & 50                   & 48                       & 0.96                \\
23             & 50                        & Sunda Scops Owl                 & \textit{Otus lempiji}         		  & 1063234 & 50                   & 37                       & 0.74                \\
24             & 18                        & White-breasted Waterhen         & \textit{Amaurornis phoenicurus}		  & 176385  & 15                   & 12                       & 0.80                 \\
25             & 11                        & White-breasted Waterhen         & \textit{Amaurornis phoenicurus}        & 176385  & 50                   & 28                       & 0.56                 \\
26             & 26                        & Yellow-vented Bulbul            & \textit{Pycnonotus goiavier}  		  & 562613  & 50                   & 36                       & 0.72                \\
27             & 11                        & Yellow-vented Bulbul            & \textit{Pycnonotus goiavier}  		  & 562613  & 50                   & 43                       & 0.86                \\
\bottomrule
\end{tabular}
\caption{Performance of shortlisted classes from a full day of continuous soundscape recordings at site~\#1. Bird names follow the Checklist of Bird Species for Singapore available at www.nparks.gov.sg/biodiversity/wildlife-in-singapore/species-list/birds. Taxonomic Serial Numbers (TSNs) are obtained from itis.gov.}
\label{tab:precisionContinuousData}
\end{table}



\section{Discussion}

Several classes from the single-day performance test at site~\#1 achieve high precision despite very few training samples (Table~\ref{tab:precisionContinuousData}). This result is encouraging, as it may allow researchers to train bioacoustic classifiers with very little labeled data. This is particularly valuable for monitoring initiatives of endangered or rare species as the low presence of such species makes it inherently hard to acquire acoustic samples for training. The performance gain over BirdNET on 103 bird species further suggests that the proposed methodology may not only be valuable for detecting rare species but may even enhance current state-of-the-art models for more common species, where large labeled datasets already exist.

The aggregated results from the 10-day test at each site add support to the manually verified precision test. The temporal detection patterns show a diurnal morning and evening chorus, few nighttime detections of diurnal birds, and few daytime detections of nocturnal birds, which align well with overall expected behavior. However, the relatively high detection rate between midnight and dawn of diurnal birds at site~\#2, suggests a lower model performance at this site. Most of these detections are classified as White-breasted Waterhen (\textit{Amaurornis phoenicurus, 176385}) and inspection of some of these suggests that these are false positives originating from low-frequency sounds of cars passing by on a nearby road. This road is close enough to be heard at site~\#2 during the night, but likely too far to be detected at site~\#1. Lower model performance at site~\#2 may be a natural consequence of selection bias, as the evaluated classes are selected based on their precision at site~\#1. Nevertheless, this underscores the importance of enhancing classifier robustness to better generalize to new soundscapes, as even nearby survey sites can exhibit significant acoustic variability.

Site~\#2, apart from having more false positives during nighttime, also experiences about one-tenth the detection rate of site~\#1. This difference is likely not linked to variation in model performance between the sites but rather a result of differences in sound propagation. The recorders at site~\#1 were positioned around a lake with minimal obstruction between microphones, potentially allowing the same call to be detected on all 6 microphones simultaneously. In contrast, the recorders at site~\#2 were situated in an area with dense vegetation without direct line-of-sight between microphones. From calibration tests at site~\#2 using impulsive, high-SNR sounds, we found that the sounds were mostly audible on a single microphone at a time. Site~\#1, besides the multiplying effect present there, may further benefit from a longer detection range for each individual microphone.

\section{Challenges and suggestions for future work}
\label{sec:further_work}

The soundscape in SBG is complex, comprising vocalizations not only from birds but also from amphibians, insects, reptiles, certain mammals like bats and squirrels, as well as anthropogenic sounds. The richness of this soundscape makes sound classification in long-duration recordings challenging as sounds from different sources may have high TFR-similarity. As a result, many of the classes that perform well on the test set (Subsection~\ref{sec:testset}) experience low precision in raw soundscape recordings. We propose three possible remedies to address this challenge.

\subsection{More data}
The self-supervised component of the proposed method facilitates higher performance with less labeled training data. Nonetheless, additional training data would still be advantageous, as it would help mitigate confusion with similar sounds and enhance generalization capabilities. A pragmatic solution to handle false positives is to encourage the classifier to learn their features by continue to incorporate them into the training set, either to the sink class or to new or existing bird classes. 

\subsection{Higher resolution}
Various bird species have demonstrated the ability to distinguish fine temporal variations in sound beyond the level of the human auditory system~\cite{Dooling2002}, and other studies emphasize the importance of incorporating high temporal features for acoustic bird classification~\cite{Stowell2014}. For calls with very high temporal variability, the distinguishing features may be obscured or lost in the data compression stages. Retaining more of the information in the raw recording may help improve the performance for such classes. Three alternatives are to:

\begin{enumerate}
    \item Increase the TFR resolution by reducing the max-pooling in the TFR extraction process.
    \item Increase the resolution of the compressed TFR, either by reducing the convolutional filter sizes in the auto-encoder, or by increasing the latent space dimension.
    \item Encourage the auto-encoder to pay more attention to vocalizations with high temporal variation, e.g., by oversampling such TFRs during training.
\end{enumerate}

\subsection{Acoustic-temporal context}
Because the TFR extraction process is designed to separate disconnected regions of high energy in the input spectrogram, some TFRs will inevitably capture only part of the full call or song. In some cases, a small part of a call is sufficiently characteristic to distinguish it from other calls. In other cases, e.g. for many Sunbird species with very specific sequences of transient up- and downsweeps, it is hard to identify the exact species from a single up- or downsweep without also knowing what immediately preceded and followed. The dilation hyper-parameter in the spectrogram segmentation process (step~\ref{item:watershed} in Subsection~\ref{sec:tfr}) can be increased to merge TFRs that are spatially close in the time-frequency space but at the cost of introducing more extraneous signals into the TFRs. A better approach would involve keeping TFRs separated but still incorporate temporal context. One solution could be to train a classifier with not only the target TFR as input, but also with TFRs from the temporal vicinity of the target.



\section{Conclusions}

Protection of biodiversity is a major global concern. For conservation efforts to be effective and to track the effect of such initiatives, one need to monitor target habitats, often over long periods of time. Training machine learning models on bioacoustic data to monitor vocally active species is becoming increasingly common, but many methods rely on extensive annotated datasets for training. When developing bioacoustic monitoring systems in new regions, or when targeting rare or endangered species, annotated training data may be limited or unavailable and is often expensive to acquire -- thereby increasing the barrier to perform studies. Methods that require less annotated data, and tools that can reduce the time and effort needed to obtain relevant training data, are therefore valuable. This is particularly true for birds, as the large number of species and their vast acoustic repertoire benefit from systems that can scale sustainably to new classes, and also because data annotation of bird sounds requires highly specialized expertise.

 We propose a semi-supervised acoustic classification method that allows classification of sounds with limited annotated training data, and we evaluate it on bird recordings from Singapore. The embedding part of the network can accelerate the discovery and annotation of new classes by clustering embedding vectors from raw recordings. Its self-supervised nature, in combination with the pre-processing step that isolates individual bird calls, allows classification with fewer labeled training samples. We show that high classification precision can be achieved with very few labeled training samples, both on a controlled test set and in continuous soundscape recordings. Moreover, the proposed model outperforms the state-of-the-art BirdNET classifier on a test set covering 103 bird species, despite far less labeled training data. However, the rich soundscape in the SBG, with around 40,000 daily TFRs per microphone, makes it challenging to reject false positives. We thus propose various approaches to address this.

Although we validate the methods in this work on bird data, they are by no means restricted to bird vocalizations but can be applied to a wide range of acoustic tasks that involve clustering or classification of frequency-modulated sounds.

%\section*{Model availability}

\section*{Declaration of Competing Interest}
None.

\section*{Acknowledgment}
The National Parks Board – Singapore (NParks) funded this study.

\begin{comment}
%% The Appendices part is started with the command \appendix;
%% appendix sections are then done as normal sections
\appendix

\section{Sample Appendix Section}
\label{sec:appendix}
Lorem ipsum dolor sit amet, consectetur adipiscing elit, sed do eiusmod tempor incididunt ut labore et dolore magna aliqua. Ut enim ad minim veniam, quis nostrud exercitation ullamco laboris nisi ut aliquip ex ea commodo consequat.
\end{comment}

%% If you have bibdatabase file and want bibtex to generate the
%% bibitems, please use
%%

% This must be in the first 5 lines to tell arXiv to use pdfLaTeX, which is strongly recommended.
\pdfoutput=1
% In particular, the hyperref package requires pdfLaTeX in order to break URLs across lines.

\documentclass[11pt]{article}

% Change "review" to "final" to generate the final (sometimes called camera-ready) version.
% Change to "preprint" to generate a non-anonymous version with page numbers.
\usepackage{acl}

% Standard package includes
\usepackage{times}
\usepackage{latexsym}

% Draw tables
\usepackage{booktabs}
\usepackage{multirow}
\usepackage{xcolor}
\usepackage{colortbl}
\usepackage{array} 
\usepackage{amsmath}

\newcolumntype{C}{>{\centering\arraybackslash}p{0.07\textwidth}}
% For proper rendering and hyphenation of words containing Latin characters (including in bib files)
\usepackage[T1]{fontenc}
% For Vietnamese characters
% \usepackage[T5]{fontenc}
% See https://www.latex-project.org/help/documentation/encguide.pdf for other character sets
% This assumes your files are encoded as UTF8
\usepackage[utf8]{inputenc}

% This is not strictly necessary, and may be commented out,
% but it will improve the layout of the manuscript,
% and will typically save some space.
\usepackage{microtype}
\DeclareMathOperator*{\argmax}{arg\,max}
% This is also not strictly necessary, and may be commented out.
% However, it will improve the aesthetics of text in
% the typewriter font.
\usepackage{inconsolata}

%Including images in your LaTeX document requires adding
%additional package(s)
\usepackage{graphicx}
% If the title and author information does not fit in the area allocated, uncomment the following
%
%\setlength\titlebox{<dim>}
%
% and set <dim> to something 5cm or larger.

\title{Wi-Chat: Large Language Model Powered Wi-Fi Sensing}

% Author information can be set in various styles:
% For several authors from the same institution:
% \author{Author 1 \and ... \and Author n \\
%         Address line \\ ... \\ Address line}
% if the names do not fit well on one line use
%         Author 1 \\ {\bf Author 2} \\ ... \\ {\bf Author n} \\
% For authors from different institutions:
% \author{Author 1 \\ Address line \\  ... \\ Address line
%         \And  ... \And
%         Author n \\ Address line \\ ... \\ Address line}
% To start a separate ``row'' of authors use \AND, as in
% \author{Author 1 \\ Address line \\  ... \\ Address line
%         \AND
%         Author 2 \\ Address line \\ ... \\ Address line \And
%         Author 3 \\ Address line \\ ... \\ Address line}

% \author{First Author \\
%   Affiliation / Address line 1 \\
%   Affiliation / Address line 2 \\
%   Affiliation / Address line 3 \\
%   \texttt{email@domain} \\\And
%   Second Author \\
%   Affiliation / Address line 1 \\
%   Affiliation / Address line 2 \\
%   Affiliation / Address line 3 \\
%   \texttt{email@domain} \\}
% \author{Haohan Yuan \qquad Haopeng Zhang\thanks{corresponding author} \\ 
%   ALOHA Lab, University of Hawaii at Manoa \\
%   % Affiliation / Address line 2 \\
%   % Affiliation / Address line 3 \\
%   \texttt{\{haohany,haopengz\}@hawaii.edu}}
  
\author{
{Haopeng Zhang$\dag$\thanks{These authors contributed equally to this work.}, Yili Ren$\ddagger$\footnotemark[1], Haohan Yuan$\dag$, Jingzhe Zhang$\ddagger$, Yitong Shen$\ddagger$} \\
ALOHA Lab, University of Hawaii at Manoa$\dag$, University of South Florida$\ddagger$ \\
\{haopengz, haohany\}@hawaii.edu\\
\{yiliren, jingzhe, shen202\}@usf.edu\\}



  
%\author{
%  \textbf{First Author\textsuperscript{1}},
%  \textbf{Second Author\textsuperscript{1,2}},
%  \textbf{Third T. Author\textsuperscript{1}},
%  \textbf{Fourth Author\textsuperscript{1}},
%\\
%  \textbf{Fifth Author\textsuperscript{1,2}},
%  \textbf{Sixth Author\textsuperscript{1}},
%  \textbf{Seventh Author\textsuperscript{1}},
%  \textbf{Eighth Author \textsuperscript{1,2,3,4}},
%\\
%  \textbf{Ninth Author\textsuperscript{1}},
%  \textbf{Tenth Author\textsuperscript{1}},
%  \textbf{Eleventh E. Author\textsuperscript{1,2,3,4,5}},
%  \textbf{Twelfth Author\textsuperscript{1}},
%\\
%  \textbf{Thirteenth Author\textsuperscript{3}},
%  \textbf{Fourteenth F. Author\textsuperscript{2,4}},
%  \textbf{Fifteenth Author\textsuperscript{1}},
%  \textbf{Sixteenth Author\textsuperscript{1}},
%\\
%  \textbf{Seventeenth S. Author\textsuperscript{4,5}},
%  \textbf{Eighteenth Author\textsuperscript{3,4}},
%  \textbf{Nineteenth N. Author\textsuperscript{2,5}},
%  \textbf{Twentieth Author\textsuperscript{1}}
%\\
%\\
%  \textsuperscript{1}Affiliation 1,
%  \textsuperscript{2}Affiliation 2,
%  \textsuperscript{3}Affiliation 3,
%  \textsuperscript{4}Affiliation 4,
%  \textsuperscript{5}Affiliation 5
%\\
%  \small{
%    \textbf{Correspondence:} \href{mailto:email@domain}{email@domain}
%  }
%}

\begin{document}
\maketitle
\begin{abstract}
Recent advancements in Large Language Models (LLMs) have demonstrated remarkable capabilities across diverse tasks. However, their potential to integrate physical model knowledge for real-world signal interpretation remains largely unexplored. In this work, we introduce Wi-Chat, the first LLM-powered Wi-Fi-based human activity recognition system. We demonstrate that LLMs can process raw Wi-Fi signals and infer human activities by incorporating Wi-Fi sensing principles into prompts. Our approach leverages physical model insights to guide LLMs in interpreting Channel State Information (CSI) data without traditional signal processing techniques. Through experiments on real-world Wi-Fi datasets, we show that LLMs exhibit strong reasoning capabilities, achieving zero-shot activity recognition. These findings highlight a new paradigm for Wi-Fi sensing, expanding LLM applications beyond conventional language tasks and enhancing the accessibility of wireless sensing for real-world deployments.
\end{abstract}

\section{Introduction}

In today’s rapidly evolving digital landscape, the transformative power of web technologies has redefined not only how services are delivered but also how complex tasks are approached. Web-based systems have become increasingly prevalent in risk control across various domains. This widespread adoption is due their accessibility, scalability, and ability to remotely connect various types of users. For example, these systems are used for process safety management in industry~\cite{kannan2016web}, safety risk early warning in urban construction~\cite{ding2013development}, and safe monitoring of infrastructural systems~\cite{repetto2018web}. Within these web-based risk management systems, the source search problem presents a huge challenge. Source search refers to the task of identifying the origin of a risky event, such as a gas leak and the emission point of toxic substances. This source search capability is crucial for effective risk management and decision-making.

Traditional approaches to implementing source search capabilities into the web systems often rely on solely algorithmic solutions~\cite{ristic2016study}. These methods, while relatively straightforward to implement, often struggle to achieve acceptable performances due to algorithmic local optima and complex unknown environments~\cite{zhao2020searching}. More recently, web crowdsourcing has emerged as a promising alternative for tackling the source search problem by incorporating human efforts in these web systems on-the-fly~\cite{zhao2024user}. This approach outsources the task of addressing issues encountered during the source search process to human workers, leveraging their capabilities to enhance system performance.

These solutions often employ a human-AI collaborative way~\cite{zhao2023leveraging} where algorithms handle exploration-exploitation and report the encountered problems while human workers resolve complex decision-making bottlenecks to help the algorithms getting rid of local deadlocks~\cite{zhao2022crowd}. Although effective, this paradigm suffers from two inherent limitations: increased operational costs from continuous human intervention, and slow response times of human workers due to sequential decision-making. These challenges motivate our investigation into developing autonomous systems that preserve human-like reasoning capabilities while reducing dependency on massive crowdsourced labor.

Furthermore, recent advancements in large language models (LLMs)~\cite{chang2024survey} and multi-modal LLMs (MLLMs)~\cite{huang2023chatgpt} have unveiled promising avenues for addressing these challenges. One clear opportunity involves the seamless integration of visual understanding and linguistic reasoning for robust decision-making in search tasks. However, whether large models-assisted source search is really effective and efficient for improving the current source search algorithms~\cite{ji2022source} remains unknown. \textit{To address the research gap, we are particularly interested in answering the following two research questions in this work:}

\textbf{\textit{RQ1: }}How can source search capabilities be integrated into web-based systems to support decision-making in time-sensitive risk management scenarios? 
% \sq{I mention ``time-sensitive'' here because I feel like we shall say something about the response time -- LLM has to be faster than humans}

\textbf{\textit{RQ2: }}How can MLLMs and LLMs enhance the effectiveness and efficiency of existing source search algorithms? 

% \textit{\textbf{RQ2:}} To what extent does the performance of large models-assisted search align with or approach the effectiveness of human-AI collaborative search? 

To answer the research questions, we propose a novel framework called Auto-\
S$^2$earch (\textbf{Auto}nomous \textbf{S}ource \textbf{Search}) and implement a prototype system that leverages advanced web technologies to simulate real-world conditions for zero-shot source search. Unlike traditional methods that rely on pre-defined heuristics or extensive human intervention, AutoS$^2$earch employs a carefully designed prompt that encapsulates human rationales, thereby guiding the MLLM to generate coherent and accurate scene descriptions from visual inputs about four directional choices. Based on these language-based descriptions, the LLM is enabled to determine the optimal directional choice through chain-of-thought (CoT) reasoning. Comprehensive empirical validation demonstrates that AutoS$^2$-\ 
earch achieves a success rate of 95–98\%, closely approaching the performance of human-AI collaborative search across 20 benchmark scenarios~\cite{zhao2023leveraging}. 

Our work indicates that the role of humans in future web crowdsourcing tasks may evolve from executors to validators or supervisors. Furthermore, incorporating explanations of LLM decisions into web-based system interfaces has the potential to help humans enhance task performance in risk control.






\section{Related Work}
\label{sec:relatedworks}

% \begin{table*}[t]
% \centering 
% \renewcommand\arraystretch{0.98}
% \fontsize{8}{10}\selectfont \setlength{\tabcolsep}{0.4em}
% \begin{tabular}{@{}lc|cc|cc|cc@{}}
% \toprule
% \textbf{Methods}           & \begin{tabular}[c]{@{}c@{}}\textbf{Training}\\ \textbf{Paradigm}\end{tabular} & \begin{tabular}[c]{@{}c@{}}\textbf{$\#$ PT Data}\\ \textbf{(Tokens)}\end{tabular} & \begin{tabular}[c]{@{}c@{}}\textbf{$\#$ IFT Data}\\ \textbf{(Samples)}\end{tabular} & \textbf{Code}  & \begin{tabular}[c]{@{}c@{}}\textbf{Natural}\\ \textbf{Language}\end{tabular} & \begin{tabular}[c]{@{}c@{}}\textbf{Action}\\ \textbf{Trajectories}\end{tabular} & \begin{tabular}[c]{@{}c@{}}\textbf{API}\\ \textbf{Documentation}\end{tabular}\\ \midrule 
% NexusRaven~\citep{srinivasan2023nexusraven} & IFT & - & - & \textcolor{green}{\CheckmarkBold} & \textcolor{green}{\CheckmarkBold} &\textcolor{red}{\XSolidBrush}&\textcolor{red}{\XSolidBrush}\\
% AgentInstruct~\citep{zeng2023agenttuning} & IFT & - & 2k & \textcolor{green}{\CheckmarkBold} & \textcolor{green}{\CheckmarkBold} &\textcolor{red}{\XSolidBrush}&\textcolor{red}{\XSolidBrush} \\
% AgentEvol~\citep{xi2024agentgym} & IFT & - & 14.5k & \textcolor{green}{\CheckmarkBold} & \textcolor{green}{\CheckmarkBold} &\textcolor{green}{\CheckmarkBold}&\textcolor{red}{\XSolidBrush} \\
% Gorilla~\citep{patil2023gorilla}& IFT & - & 16k & \textcolor{green}{\CheckmarkBold} & \textcolor{green}{\CheckmarkBold} &\textcolor{red}{\XSolidBrush}&\textcolor{green}{\CheckmarkBold}\\
% OpenFunctions-v2~\citep{patil2023gorilla} & IFT & - & 65k & \textcolor{green}{\CheckmarkBold} & \textcolor{green}{\CheckmarkBold} &\textcolor{red}{\XSolidBrush}&\textcolor{green}{\CheckmarkBold}\\
% LAM~\citep{zhang2024agentohana} & IFT & - & 42.6k & \textcolor{green}{\CheckmarkBold} & \textcolor{green}{\CheckmarkBold} &\textcolor{green}{\CheckmarkBold}&\textcolor{red}{\XSolidBrush} \\
% xLAM~\citep{liu2024apigen} & IFT & - & 60k & \textcolor{green}{\CheckmarkBold} & \textcolor{green}{\CheckmarkBold} &\textcolor{green}{\CheckmarkBold}&\textcolor{red}{\XSolidBrush} \\\midrule
% LEMUR~\citep{xu2024lemur} & PT & 90B & 300k & \textcolor{green}{\CheckmarkBold} & \textcolor{green}{\CheckmarkBold} &\textcolor{green}{\CheckmarkBold}&\textcolor{red}{\XSolidBrush}\\
% \rowcolor{teal!12} \method & PT & 103B & 95k & \textcolor{green}{\CheckmarkBold} & \textcolor{green}{\CheckmarkBold} & \textcolor{green}{\CheckmarkBold} & \textcolor{green}{\CheckmarkBold} \\
% \bottomrule
% \end{tabular}
% \caption{Summary of existing tuning- and pretraining-based LLM agents with their training sample sizes. "PT" and "IFT" denote "Pre-Training" and "Instruction Fine-Tuning", respectively. }
% \label{tab:related}
% \end{table*}

\begin{table*}[ht]
\begin{threeparttable}
\centering 
\renewcommand\arraystretch{0.98}
\fontsize{7}{9}\selectfont \setlength{\tabcolsep}{0.2em}
\begin{tabular}{@{}l|c|c|ccc|cc|cc|cccc@{}}
\toprule
\textbf{Methods} & \textbf{Datasets}           & \begin{tabular}[c]{@{}c@{}}\textbf{Training}\\ \textbf{Paradigm}\end{tabular} & \begin{tabular}[c]{@{}c@{}}\textbf{\# PT Data}\\ \textbf{(Tokens)}\end{tabular} & \begin{tabular}[c]{@{}c@{}}\textbf{\# IFT Data}\\ \textbf{(Samples)}\end{tabular} & \textbf{\# APIs} & \textbf{Code}  & \begin{tabular}[c]{@{}c@{}}\textbf{Nat.}\\ \textbf{Lang.}\end{tabular} & \begin{tabular}[c]{@{}c@{}}\textbf{Action}\\ \textbf{Traj.}\end{tabular} & \begin{tabular}[c]{@{}c@{}}\textbf{API}\\ \textbf{Doc.}\end{tabular} & \begin{tabular}[c]{@{}c@{}}\textbf{Func.}\\ \textbf{Call}\end{tabular} & \begin{tabular}[c]{@{}c@{}}\textbf{Multi.}\\ \textbf{Step}\end{tabular}  & \begin{tabular}[c]{@{}c@{}}\textbf{Plan}\\ \textbf{Refine}\end{tabular}  & \begin{tabular}[c]{@{}c@{}}\textbf{Multi.}\\ \textbf{Turn}\end{tabular}\\ \midrule 
\multicolumn{13}{l}{\emph{Instruction Finetuning-based LLM Agents for Intrinsic Reasoning}}  \\ \midrule
FireAct~\cite{chen2023fireact} & FireAct & IFT & - & 2.1K & 10 & \textcolor{red}{\XSolidBrush} &\textcolor{green}{\CheckmarkBold} &\textcolor{green}{\CheckmarkBold}  & \textcolor{red}{\XSolidBrush} &\textcolor{green}{\CheckmarkBold} & \textcolor{red}{\XSolidBrush} &\textcolor{green}{\CheckmarkBold} & \textcolor{red}{\XSolidBrush} \\
ToolAlpaca~\cite{tang2023toolalpaca} & ToolAlpaca & IFT & - & 4.0K & 400 & \textcolor{red}{\XSolidBrush} &\textcolor{green}{\CheckmarkBold} &\textcolor{green}{\CheckmarkBold} & \textcolor{red}{\XSolidBrush} &\textcolor{green}{\CheckmarkBold} & \textcolor{red}{\XSolidBrush}  &\textcolor{green}{\CheckmarkBold} & \textcolor{red}{\XSolidBrush}  \\
ToolLLaMA~\cite{qin2023toolllm} & ToolBench & IFT & - & 12.7K & 16,464 & \textcolor{red}{\XSolidBrush} &\textcolor{green}{\CheckmarkBold} &\textcolor{green}{\CheckmarkBold} &\textcolor{red}{\XSolidBrush} &\textcolor{green}{\CheckmarkBold}&\textcolor{green}{\CheckmarkBold}&\textcolor{green}{\CheckmarkBold} &\textcolor{green}{\CheckmarkBold}\\
AgentEvol~\citep{xi2024agentgym} & AgentTraj-L & IFT & - & 14.5K & 24 &\textcolor{red}{\XSolidBrush} & \textcolor{green}{\CheckmarkBold} &\textcolor{green}{\CheckmarkBold}&\textcolor{red}{\XSolidBrush} &\textcolor{green}{\CheckmarkBold}&\textcolor{red}{\XSolidBrush} &\textcolor{red}{\XSolidBrush} &\textcolor{green}{\CheckmarkBold}\\
Lumos~\cite{yin2024agent} & Lumos & IFT  & - & 20.0K & 16 &\textcolor{red}{\XSolidBrush} & \textcolor{green}{\CheckmarkBold} & \textcolor{green}{\CheckmarkBold} &\textcolor{red}{\XSolidBrush} & \textcolor{green}{\CheckmarkBold} & \textcolor{green}{\CheckmarkBold} &\textcolor{red}{\XSolidBrush} & \textcolor{green}{\CheckmarkBold}\\
Agent-FLAN~\cite{chen2024agent} & Agent-FLAN & IFT & - & 24.7K & 20 &\textcolor{red}{\XSolidBrush} & \textcolor{green}{\CheckmarkBold} & \textcolor{green}{\CheckmarkBold} &\textcolor{red}{\XSolidBrush} & \textcolor{green}{\CheckmarkBold}& \textcolor{green}{\CheckmarkBold}&\textcolor{red}{\XSolidBrush} & \textcolor{green}{\CheckmarkBold}\\
AgentTuning~\citep{zeng2023agenttuning} & AgentInstruct & IFT & - & 35.0K & - &\textcolor{red}{\XSolidBrush} & \textcolor{green}{\CheckmarkBold} & \textcolor{green}{\CheckmarkBold} &\textcolor{red}{\XSolidBrush} & \textcolor{green}{\CheckmarkBold} &\textcolor{red}{\XSolidBrush} &\textcolor{red}{\XSolidBrush} & \textcolor{green}{\CheckmarkBold}\\\midrule
\multicolumn{13}{l}{\emph{Instruction Finetuning-based LLM Agents for Function Calling}} \\\midrule
NexusRaven~\citep{srinivasan2023nexusraven} & NexusRaven & IFT & - & - & 116 & \textcolor{green}{\CheckmarkBold} & \textcolor{green}{\CheckmarkBold}  & \textcolor{green}{\CheckmarkBold} &\textcolor{red}{\XSolidBrush} & \textcolor{green}{\CheckmarkBold} &\textcolor{red}{\XSolidBrush} &\textcolor{red}{\XSolidBrush}&\textcolor{red}{\XSolidBrush}\\
Gorilla~\citep{patil2023gorilla} & Gorilla & IFT & - & 16.0K & 1,645 & \textcolor{green}{\CheckmarkBold} &\textcolor{red}{\XSolidBrush} &\textcolor{red}{\XSolidBrush}&\textcolor{green}{\CheckmarkBold} &\textcolor{green}{\CheckmarkBold} &\textcolor{red}{\XSolidBrush} &\textcolor{red}{\XSolidBrush} &\textcolor{red}{\XSolidBrush}\\
OpenFunctions-v2~\citep{patil2023gorilla} & OpenFunctions-v2 & IFT & - & 65.0K & - & \textcolor{green}{\CheckmarkBold} & \textcolor{green}{\CheckmarkBold} &\textcolor{red}{\XSolidBrush} &\textcolor{green}{\CheckmarkBold} &\textcolor{green}{\CheckmarkBold} &\textcolor{red}{\XSolidBrush} &\textcolor{red}{\XSolidBrush} &\textcolor{red}{\XSolidBrush}\\
API Pack~\cite{guo2024api} & API Pack & IFT & - & 1.1M & 11,213 &\textcolor{green}{\CheckmarkBold} &\textcolor{red}{\XSolidBrush} &\textcolor{green}{\CheckmarkBold} &\textcolor{red}{\XSolidBrush} &\textcolor{green}{\CheckmarkBold} &\textcolor{red}{\XSolidBrush}&\textcolor{red}{\XSolidBrush}&\textcolor{red}{\XSolidBrush}\\ 
LAM~\citep{zhang2024agentohana} & AgentOhana & IFT & - & 42.6K & - & \textcolor{green}{\CheckmarkBold} & \textcolor{green}{\CheckmarkBold} &\textcolor{green}{\CheckmarkBold}&\textcolor{red}{\XSolidBrush} &\textcolor{green}{\CheckmarkBold}&\textcolor{red}{\XSolidBrush}&\textcolor{green}{\CheckmarkBold}&\textcolor{green}{\CheckmarkBold}\\
xLAM~\citep{liu2024apigen} & APIGen & IFT & - & 60.0K & 3,673 & \textcolor{green}{\CheckmarkBold} & \textcolor{green}{\CheckmarkBold} &\textcolor{green}{\CheckmarkBold}&\textcolor{red}{\XSolidBrush} &\textcolor{green}{\CheckmarkBold}&\textcolor{red}{\XSolidBrush}&\textcolor{green}{\CheckmarkBold}&\textcolor{green}{\CheckmarkBold}\\\midrule
\multicolumn{13}{l}{\emph{Pretraining-based LLM Agents}}  \\\midrule
% LEMUR~\citep{xu2024lemur} & PT & 90B & 300.0K & - & \textcolor{green}{\CheckmarkBold} & \textcolor{green}{\CheckmarkBold} &\textcolor{green}{\CheckmarkBold}&\textcolor{red}{\XSolidBrush} & \textcolor{red}{\XSolidBrush} &\textcolor{green}{\CheckmarkBold} &\textcolor{red}{\XSolidBrush}&\textcolor{red}{\XSolidBrush}\\
\rowcolor{teal!12} \method & \dataset & PT & 103B & 95.0K  & 76,537  & \textcolor{green}{\CheckmarkBold} & \textcolor{green}{\CheckmarkBold} & \textcolor{green}{\CheckmarkBold} & \textcolor{green}{\CheckmarkBold} & \textcolor{green}{\CheckmarkBold} & \textcolor{green}{\CheckmarkBold} & \textcolor{green}{\CheckmarkBold} & \textcolor{green}{\CheckmarkBold}\\
\bottomrule
\end{tabular}
% \begin{tablenotes}
%     \item $^*$ In addition, the StarCoder-API can offer 4.77M more APIs.
% \end{tablenotes}
\caption{Summary of existing instruction finetuning-based LLM agents for intrinsic reasoning and function calling, along with their training resources and sample sizes. "PT" and "IFT" denote "Pre-Training" and "Instruction Fine-Tuning", respectively.}
\vspace{-2ex}
\label{tab:related}
\end{threeparttable}
\end{table*}

\noindent \textbf{Prompting-based LLM Agents.} Due to the lack of agent-specific pre-training corpus, existing LLM agents rely on either prompt engineering~\cite{hsieh2023tool,lu2024chameleon,yao2022react,wang2023voyager} or instruction fine-tuning~\cite{chen2023fireact,zeng2023agenttuning} to understand human instructions, decompose high-level tasks, generate grounded plans, and execute multi-step actions. 
However, prompting-based methods mainly depend on the capabilities of backbone LLMs (usually commercial LLMs), failing to introduce new knowledge and struggling to generalize to unseen tasks~\cite{sun2024adaplanner,zhuang2023toolchain}. 

\noindent \textbf{Instruction Finetuning-based LLM Agents.} Considering the extensive diversity of APIs and the complexity of multi-tool instructions, tool learning inherently presents greater challenges than natural language tasks, such as text generation~\cite{qin2023toolllm}.
Post-training techniques focus more on instruction following and aligning output with specific formats~\cite{patil2023gorilla,hao2024toolkengpt,qin2023toolllm,schick2024toolformer}, rather than fundamentally improving model knowledge or capabilities. 
Moreover, heavy fine-tuning can hinder generalization or even degrade performance in non-agent use cases, potentially suppressing the original base model capabilities~\cite{ghosh2024a}.

\noindent \textbf{Pretraining-based LLM Agents.} While pre-training serves as an essential alternative, prior works~\cite{nijkamp2023codegen,roziere2023code,xu2024lemur,patil2023gorilla} have primarily focused on improving task-specific capabilities (\eg, code generation) instead of general-domain LLM agents, due to single-source, uni-type, small-scale, and poor-quality pre-training data. 
Existing tool documentation data for agent training either lacks diverse real-world APIs~\cite{patil2023gorilla, tang2023toolalpaca} or is constrained to single-tool or single-round tool execution. 
Furthermore, trajectory data mostly imitate expert behavior or follow function-calling rules with inferior planning and reasoning, failing to fully elicit LLMs' capabilities and handle complex instructions~\cite{qin2023toolllm}. 
Given a wide range of candidate API functions, each comprising various function names and parameters available at every planning step, identifying globally optimal solutions and generalizing across tasks remains highly challenging.



\section{Preliminaries}
\label{Preliminaries}
\begin{figure*}[t]
    \centering
    \includegraphics[width=0.95\linewidth]{fig/HealthGPT_Framework.png}
    \caption{The \ourmethod{} architecture integrates hierarchical visual perception and H-LoRA, employing a task-specific hard router to select visual features and H-LoRA plugins, ultimately generating outputs with an autoregressive manner.}
    \label{fig:architecture}
\end{figure*}
\noindent\textbf{Large Vision-Language Models.} 
The input to a LVLM typically consists of an image $x^{\text{img}}$ and a discrete text sequence $x^{\text{txt}}$. The visual encoder $\mathcal{E}^{\text{img}}$ converts the input image $x^{\text{img}}$ into a sequence of visual tokens $\mathcal{V} = [v_i]_{i=1}^{N_v}$, while the text sequence $x^{\text{txt}}$ is mapped into a sequence of text tokens $\mathcal{T} = [t_i]_{i=1}^{N_t}$ using an embedding function $\mathcal{E}^{\text{txt}}$. The LLM $\mathcal{M_\text{LLM}}(\cdot|\theta)$ models the joint probability of the token sequence $\mathcal{U} = \{\mathcal{V},\mathcal{T}\}$, which is expressed as:
\begin{equation}
    P_\theta(R | \mathcal{U}) = \prod_{i=1}^{N_r} P_\theta(r_i | \{\mathcal{U}, r_{<i}\}),
\end{equation}
where $R = [r_i]_{i=1}^{N_r}$ is the text response sequence. The LVLM iteratively generates the next token $r_i$ based on $r_{<i}$. The optimization objective is to minimize the cross-entropy loss of the response $\mathcal{R}$.
% \begin{equation}
%     \mathcal{L}_{\text{VLM}} = \mathbb{E}_{R|\mathcal{U}}\left[-\log P_\theta(R | \mathcal{U})\right]
% \end{equation}
It is worth noting that most LVLMs adopt a design paradigm based on ViT, alignment adapters, and pre-trained LLMs\cite{liu2023llava,liu2024improved}, enabling quick adaptation to downstream tasks.


\noindent\textbf{VQGAN.}
VQGAN~\cite{esser2021taming} employs latent space compression and indexing mechanisms to effectively learn a complete discrete representation of images. VQGAN first maps the input image $x^{\text{img}}$ to a latent representation $z = \mathcal{E}(x)$ through a encoder $\mathcal{E}$. Then, the latent representation is quantized using a codebook $\mathcal{Z} = \{z_k\}_{k=1}^K$, generating a discrete index sequence $\mathcal{I} = [i_m]_{m=1}^N$, where $i_m \in \mathcal{Z}$ represents the quantized code index:
\begin{equation}
    \mathcal{I} = \text{Quantize}(z|\mathcal{Z}) = \arg\min_{z_k \in \mathcal{Z}} \| z - z_k \|_2.
\end{equation}
In our approach, the discrete index sequence $\mathcal{I}$ serves as a supervisory signal for the generation task, enabling the model to predict the index sequence $\hat{\mathcal{I}}$ from input conditions such as text or other modality signals.  
Finally, the predicted index sequence $\hat{\mathcal{I}}$ is upsampled by the VQGAN decoder $G$, generating the high-quality image $\hat{x}^\text{img} = G(\hat{\mathcal{I}})$.



\noindent\textbf{Low Rank Adaptation.} 
LoRA\cite{hu2021lora} effectively captures the characteristics of downstream tasks by introducing low-rank adapters. The core idea is to decompose the bypass weight matrix $\Delta W\in\mathbb{R}^{d^{\text{in}} \times d^{\text{out}}}$ into two low-rank matrices $ \{A \in \mathbb{R}^{d^{\text{in}} \times r}, B \in \mathbb{R}^{r \times d^{\text{out}}} \}$, where $ r \ll \min\{d^{\text{in}}, d^{\text{out}}\} $, significantly reducing learnable parameters. The output with the LoRA adapter for the input $x$ is then given by:
\begin{equation}
    h = x W_0 + \alpha x \Delta W/r = x W_0 + \alpha xAB/r,
\end{equation}
where matrix $ A $ is initialized with a Gaussian distribution, while the matrix $ B $ is initialized as a zero matrix. The scaling factor $ \alpha/r $ controls the impact of $ \Delta W $ on the model.

\section{HealthGPT}
\label{Method}


\subsection{Unified Autoregressive Generation.}  
% As shown in Figure~\ref{fig:architecture}, 
\ourmethod{} (Figure~\ref{fig:architecture}) utilizes a discrete token representation that covers both text and visual outputs, unifying visual comprehension and generation as an autoregressive task. 
For comprehension, $\mathcal{M}_\text{llm}$ receives the input joint sequence $\mathcal{U}$ and outputs a series of text token $\mathcal{R} = [r_1, r_2, \dots, r_{N_r}]$, where $r_i \in \mathcal{V}_{\text{txt}}$, and $\mathcal{V}_{\text{txt}}$ represents the LLM's vocabulary:
\begin{equation}
    P_\theta(\mathcal{R} \mid \mathcal{U}) = \prod_{i=1}^{N_r} P_\theta(r_i \mid \mathcal{U}, r_{<i}).
\end{equation}
For generation, $\mathcal{M}_\text{llm}$ first receives a special start token $\langle \text{START\_IMG} \rangle$, then generates a series of tokens corresponding to the VQGAN indices $\mathcal{I} = [i_1, i_2, \dots, i_{N_i}]$, where $i_j \in \mathcal{V}_{\text{vq}}$, and $\mathcal{V}_{\text{vq}}$ represents the index range of VQGAN. Upon completion of generation, the LLM outputs an end token $\langle \text{END\_IMG} \rangle$:
\begin{equation}
    P_\theta(\mathcal{I} \mid \mathcal{U}) = \prod_{j=1}^{N_i} P_\theta(i_j \mid \mathcal{U}, i_{<j}).
\end{equation}
Finally, the generated index sequence $\mathcal{I}$ is fed into the decoder $G$, which reconstructs the target image $\hat{x}^{\text{img}} = G(\mathcal{I})$.

\subsection{Hierarchical Visual Perception}  
Given the differences in visual perception between comprehension and generation tasks—where the former focuses on abstract semantics and the latter emphasizes complete semantics—we employ ViT to compress the image into discrete visual tokens at multiple hierarchical levels.
Specifically, the image is converted into a series of features $\{f_1, f_2, \dots, f_L\}$ as it passes through $L$ ViT blocks.

To address the needs of various tasks, the hidden states are divided into two types: (i) \textit{Concrete-grained features} $\mathcal{F}^{\text{Con}} = \{f_1, f_2, \dots, f_k\}, k < L$, derived from the shallower layers of ViT, containing sufficient global features, suitable for generation tasks; 
(ii) \textit{Abstract-grained features} $\mathcal{F}^{\text{Abs}} = \{f_{k+1}, f_{k+2}, \dots, f_L\}$, derived from the deeper layers of ViT, which contain abstract semantic information closer to the text space, suitable for comprehension tasks.

The task type $T$ (comprehension or generation) determines which set of features is selected as the input for the downstream large language model:
\begin{equation}
    \mathcal{F}^{\text{img}}_T =
    \begin{cases}
        \mathcal{F}^{\text{Con}}, & \text{if } T = \text{generation task} \\
        \mathcal{F}^{\text{Abs}}, & \text{if } T = \text{comprehension task}
    \end{cases}
\end{equation}
We integrate the image features $\mathcal{F}^{\text{img}}_T$ and text features $\mathcal{T}$ into a joint sequence through simple concatenation, which is then fed into the LLM $\mathcal{M}_{\text{llm}}$ for autoregressive generation.
% :
% \begin{equation}
%     \mathcal{R} = \mathcal{M}_{\text{llm}}(\mathcal{U}|\theta), \quad \mathcal{U} = [\mathcal{F}^{\text{img}}_T; \mathcal{T}]
% \end{equation}
\subsection{Heterogeneous Knowledge Adaptation}
We devise H-LoRA, which stores heterogeneous knowledge from comprehension and generation tasks in separate modules and dynamically routes to extract task-relevant knowledge from these modules. 
At the task level, for each task type $ T $, we dynamically assign a dedicated H-LoRA submodule $ \theta^T $, which is expressed as:
\begin{equation}
    \mathcal{R} = \mathcal{M}_\text{LLM}(\mathcal{U}|\theta, \theta^T), \quad \theta^T = \{A^T, B^T, \mathcal{R}^T_\text{outer}\}.
\end{equation}
At the feature level for a single task, H-LoRA integrates the idea of Mixture of Experts (MoE)~\cite{masoudnia2014mixture} and designs an efficient matrix merging and routing weight allocation mechanism, thus avoiding the significant computational delay introduced by matrix splitting in existing MoELoRA~\cite{luo2024moelora}. Specifically, we first merge the low-rank matrices (rank = r) of $ k $ LoRA experts into a unified matrix:
\begin{equation}
    \mathbf{A}^{\text{merged}}, \mathbf{B}^{\text{merged}} = \text{Concat}(\{A_i\}_1^k), \text{Concat}(\{B_i\}_1^k),
\end{equation}
where $ \mathbf{A}^{\text{merged}} \in \mathbb{R}^{d^\text{in} \times rk} $ and $ \mathbf{B}^{\text{merged}} \in \mathbb{R}^{rk \times d^\text{out}} $. The $k$-dimension routing layer generates expert weights $ \mathcal{W} \in \mathbb{R}^{\text{token\_num} \times k} $ based on the input hidden state $ x $, and these are expanded to $ \mathbb{R}^{\text{token\_num} \times rk} $ as follows:
\begin{equation}
    \mathcal{W}^\text{expanded} = \alpha k \mathcal{W} / r \otimes \mathbf{1}_r,
\end{equation}
where $ \otimes $ denotes the replication operation.
The overall output of H-LoRA is computed as:
\begin{equation}
    \mathcal{O}^\text{H-LoRA} = (x \mathbf{A}^{\text{merged}} \odot \mathcal{W}^\text{expanded}) \mathbf{B}^{\text{merged}},
\end{equation}
where $ \odot $ represents element-wise multiplication. Finally, the output of H-LoRA is added to the frozen pre-trained weights to produce the final output:
\begin{equation}
    \mathcal{O} = x W_0 + \mathcal{O}^\text{H-LoRA}.
\end{equation}
% In summary, H-LoRA is a task-based dynamic PEFT method that achieves high efficiency in single-task fine-tuning.

\subsection{Training Pipeline}

\begin{figure}[t]
    \centering
    \hspace{-4mm}
    \includegraphics[width=0.94\linewidth]{fig/data.pdf}
    \caption{Data statistics of \texttt{VL-Health}. }
    \label{fig:data}
\end{figure}
\noindent \textbf{1st Stage: Multi-modal Alignment.} 
In the first stage, we design separate visual adapters and H-LoRA submodules for medical unified tasks. For the medical comprehension task, we train abstract-grained visual adapters using high-quality image-text pairs to align visual embeddings with textual embeddings, thereby enabling the model to accurately describe medical visual content. During this process, the pre-trained LLM and its corresponding H-LoRA submodules remain frozen. In contrast, the medical generation task requires training concrete-grained adapters and H-LoRA submodules while keeping the LLM frozen. Meanwhile, we extend the textual vocabulary to include multimodal tokens, enabling the support of additional VQGAN vector quantization indices. The model trains on image-VQ pairs, endowing the pre-trained LLM with the capability for image reconstruction. This design ensures pixel-level consistency of pre- and post-LVLM. The processes establish the initial alignment between the LLM’s outputs and the visual inputs.

\noindent \textbf{2nd Stage: Heterogeneous H-LoRA Plugin Adaptation.}  
The submodules of H-LoRA share the word embedding layer and output head but may encounter issues such as bias and scale inconsistencies during training across different tasks. To ensure that the multiple H-LoRA plugins seamlessly interface with the LLMs and form a unified base, we fine-tune the word embedding layer and output head using a small amount of mixed data to maintain consistency in the model weights. Specifically, during this stage, all H-LoRA submodules for different tasks are kept frozen, with only the word embedding layer and output head being optimized. Through this stage, the model accumulates foundational knowledge for unified tasks by adapting H-LoRA plugins.

\begin{table*}[!t]
\centering
\caption{Comparison of \ourmethod{} with other LVLMs and unified multi-modal models on medical visual comprehension tasks. \textbf{Bold} and \underline{underlined} text indicates the best performance and second-best performance, respectively.}
\resizebox{\textwidth}{!}{
\begin{tabular}{c|lcc|cccccccc|c}
\toprule
\rowcolor[HTML]{E9F3FE} &  &  &  & \multicolumn{2}{c}{\textbf{VQA-RAD \textuparrow}} & \multicolumn{2}{c}{\textbf{SLAKE \textuparrow}} & \multicolumn{2}{c}{\textbf{PathVQA \textuparrow}} &  &  &  \\ 
\cline{5-10}
\rowcolor[HTML]{E9F3FE}\multirow{-2}{*}{\textbf{Type}} & \multirow{-2}{*}{\textbf{Model}} & \multirow{-2}{*}{\textbf{\# Params}} & \multirow{-2}{*}{\makecell{\textbf{Medical} \\ \textbf{LVLM}}} & \textbf{close} & \textbf{all} & \textbf{close} & \textbf{all} & \textbf{close} & \textbf{all} & \multirow{-2}{*}{\makecell{\textbf{MMMU} \\ \textbf{-Med}}\textuparrow} & \multirow{-2}{*}{\textbf{OMVQA}\textuparrow} & \multirow{-2}{*}{\textbf{Avg. \textuparrow}} \\ 
\midrule \midrule
\multirow{9}{*}{\textbf{Comp. Only}} 
& Med-Flamingo & 8.3B & \Large \ding{51} & 58.6 & 43.0 & 47.0 & 25.5 & 61.9 & 31.3 & 28.7 & 34.9 & 41.4 \\
& LLaVA-Med & 7B & \Large \ding{51} & 60.2 & 48.1 & 58.4 & 44.8 & 62.3 & 35.7 & 30.0 & 41.3 & 47.6 \\
& HuatuoGPT-Vision & 7B & \Large \ding{51} & 66.9 & 53.0 & 59.8 & 49.1 & 52.9 & 32.0 & 42.0 & 50.0 & 50.7 \\
& BLIP-2 & 6.7B & \Large \ding{55} & 43.4 & 36.8 & 41.6 & 35.3 & 48.5 & 28.8 & 27.3 & 26.9 & 36.1 \\
& LLaVA-v1.5 & 7B & \Large \ding{55} & 51.8 & 42.8 & 37.1 & 37.7 & 53.5 & 31.4 & 32.7 & 44.7 & 41.5 \\
& InstructBLIP & 7B & \Large \ding{55} & 61.0 & 44.8 & 66.8 & 43.3 & 56.0 & 32.3 & 25.3 & 29.0 & 44.8 \\
& Yi-VL & 6B & \Large \ding{55} & 52.6 & 42.1 & 52.4 & 38.4 & 54.9 & 30.9 & 38.0 & 50.2 & 44.9 \\
& InternVL2 & 8B & \Large \ding{55} & 64.9 & 49.0 & 66.6 & 50.1 & 60.0 & 31.9 & \underline{43.3} & 54.5 & 52.5\\
& Llama-3.2 & 11B & \Large \ding{55} & 68.9 & 45.5 & 72.4 & 52.1 & 62.8 & 33.6 & 39.3 & 63.2 & 54.7 \\
\midrule
\multirow{5}{*}{\textbf{Comp. \& Gen.}} 
& Show-o & 1.3B & \Large \ding{55} & 50.6 & 33.9 & 31.5 & 17.9 & 52.9 & 28.2 & 22.7 & 45.7 & 42.6 \\
& Unified-IO 2 & 7B & \Large \ding{55} & 46.2 & 32.6 & 35.9 & 21.9 & 52.5 & 27.0 & 25.3 & 33.0 & 33.8 \\
& Janus & 1.3B & \Large \ding{55} & 70.9 & 52.8 & 34.7 & 26.9 & 51.9 & 27.9 & 30.0 & 26.8 & 33.5 \\
& \cellcolor[HTML]{DAE0FB}HealthGPT-M3 & \cellcolor[HTML]{DAE0FB}3.8B & \cellcolor[HTML]{DAE0FB}\Large \ding{51} & \cellcolor[HTML]{DAE0FB}\underline{73.7} & \cellcolor[HTML]{DAE0FB}\underline{55.9} & \cellcolor[HTML]{DAE0FB}\underline{74.6} & \cellcolor[HTML]{DAE0FB}\underline{56.4} & \cellcolor[HTML]{DAE0FB}\underline{78.7} & \cellcolor[HTML]{DAE0FB}\underline{39.7} & \cellcolor[HTML]{DAE0FB}\underline{43.3} & \cellcolor[HTML]{DAE0FB}\underline{68.5} & \cellcolor[HTML]{DAE0FB}\underline{61.3} \\
& \cellcolor[HTML]{DAE0FB}HealthGPT-L14 & \cellcolor[HTML]{DAE0FB}14B & \cellcolor[HTML]{DAE0FB}\Large \ding{51} & \cellcolor[HTML]{DAE0FB}\textbf{77.7} & \cellcolor[HTML]{DAE0FB}\textbf{58.3} & \cellcolor[HTML]{DAE0FB}\textbf{76.4} & \cellcolor[HTML]{DAE0FB}\textbf{64.5} & \cellcolor[HTML]{DAE0FB}\textbf{85.9} & \cellcolor[HTML]{DAE0FB}\textbf{44.4} & \cellcolor[HTML]{DAE0FB}\textbf{49.2} & \cellcolor[HTML]{DAE0FB}\textbf{74.4} & \cellcolor[HTML]{DAE0FB}\textbf{66.4} \\
\bottomrule
\end{tabular}
}
\label{tab:results}
\end{table*}
\begin{table*}[ht]
    \centering
    \caption{The experimental results for the four modality conversion tasks.}
    \resizebox{\textwidth}{!}{
    \begin{tabular}{l|ccc|ccc|ccc|ccc}
        \toprule
        \rowcolor[HTML]{E9F3FE} & \multicolumn{3}{c}{\textbf{CT to MRI (Brain)}} & \multicolumn{3}{c}{\textbf{CT to MRI (Pelvis)}} & \multicolumn{3}{c}{\textbf{MRI to CT (Brain)}} & \multicolumn{3}{c}{\textbf{MRI to CT (Pelvis)}} \\
        \cline{2-13}
        \rowcolor[HTML]{E9F3FE}\multirow{-2}{*}{\textbf{Model}}& \textbf{SSIM $\uparrow$} & \textbf{PSNR $\uparrow$} & \textbf{MSE $\downarrow$} & \textbf{SSIM $\uparrow$} & \textbf{PSNR $\uparrow$} & \textbf{MSE $\downarrow$} & \textbf{SSIM $\uparrow$} & \textbf{PSNR $\uparrow$} & \textbf{MSE $\downarrow$} & \textbf{SSIM $\uparrow$} & \textbf{PSNR $\uparrow$} & \textbf{MSE $\downarrow$} \\
        \midrule \midrule
        pix2pix & 71.09 & 32.65 & 36.85 & 59.17 & 31.02 & 51.91 & 78.79 & 33.85 & 28.33 & 72.31 & 32.98 & 36.19 \\
        CycleGAN & 54.76 & 32.23 & 40.56 & 54.54 & 30.77 & 55.00 & 63.75 & 31.02 & 52.78 & 50.54 & 29.89 & 67.78 \\
        BBDM & {71.69} & {32.91} & {34.44} & 57.37 & 31.37 & 48.06 & \textbf{86.40} & 34.12 & 26.61 & {79.26} & 33.15 & 33.60 \\
        Vmanba & 69.54 & 32.67 & 36.42 & {63.01} & {31.47} & {46.99} & 79.63 & 34.12 & 26.49 & 77.45 & 33.53 & 31.85 \\
        DiffMa & 71.47 & 32.74 & 35.77 & 62.56 & 31.43 & 47.38 & 79.00 & {34.13} & {26.45} & 78.53 & {33.68} & {30.51} \\
        \rowcolor[HTML]{DAE0FB}HealthGPT-M3 & \underline{79.38} & \underline{33.03} & \underline{33.48} & \underline{71.81} & \underline{31.83} & \underline{43.45} & {85.06} & \textbf{34.40} & \textbf{25.49} & \underline{84.23} & \textbf{34.29} & \textbf{27.99} \\
        \rowcolor[HTML]{DAE0FB}HealthGPT-L14 & \textbf{79.73} & \textbf{33.10} & \textbf{32.96} & \textbf{71.92} & \textbf{31.87} & \textbf{43.09} & \underline{85.31} & \underline{34.29} & \underline{26.20} & \textbf{84.96} & \underline{34.14} & \underline{28.13} \\
        \bottomrule
    \end{tabular}
    }
    \label{tab:conversion}
\end{table*}

\noindent \textbf{3rd Stage: Visual Instruction Fine-Tuning.}  
In the third stage, we introduce additional task-specific data to further optimize the model and enhance its adaptability to downstream tasks such as medical visual comprehension (e.g., medical QA, medical dialogues, and report generation) or generation tasks (e.g., super-resolution, denoising, and modality conversion). Notably, by this stage, the word embedding layer and output head have been fine-tuned, only the H-LoRA modules and adapter modules need to be trained. This strategy significantly improves the model's adaptability and flexibility across different tasks.


\section{Experiment}
\label{s:experiment}

\subsection{Data Description}
We evaluate our method on FI~\cite{you2016building}, Twitter\_LDL~\cite{yang2017learning} and Artphoto~\cite{machajdik2010affective}.
FI is a public dataset built from Flickr and Instagram, with 23,308 images and eight emotion categories, namely \textit{amusement}, \textit{anger}, \textit{awe},  \textit{contentment}, \textit{disgust}, \textit{excitement},  \textit{fear}, and \textit{sadness}. 
% Since images in FI are all copyrighted by law, some images are corrupted now, so we remove these samples and retain 21,828 images.
% T4SA contains images from Twitter, which are classified into three categories: \textit{positive}, \textit{neutral}, and \textit{negative}. In this paper, we adopt the base version of B-T4SA, which contains 470,586 images and provides text descriptions of the corresponding tweets.
Twitter\_LDL contains 10,045 images from Twitter, with the same eight categories as the FI dataset.
% 。
For these two datasets, they are randomly split into 80\%
training and 20\% testing set.
Artphoto contains 806 artistic photos from the DeviantArt website, which we use to further evaluate the zero-shot capability of our model.
% on the small-scale dataset.
% We construct and publicly release the first image sentiment analysis dataset containing metadata.
% 。

% Based on these datasets, we are the first to construct and publicly release metadata-enhanced image sentiment analysis datasets. These datasets include scenes, tags, descriptions, and corresponding confidence scores, and are available at this link for future research purposes.


% 
\begin{table}[t]
\centering
% \begin{center}
\caption{Overall performance of different models on FI and Twitter\_LDL datasets.}
\label{tab:cap1}
% \resizebox{\linewidth}{!}
{
\begin{tabular}{l|c|c|c|c}
\hline
\multirow{2}{*}{\textbf{Model}} & \multicolumn{2}{c|}{\textbf{FI}}  & \multicolumn{2}{c}{\textbf{Twitter\_LDL}} \\ \cline{2-5} 
  & \textbf{Accuracy} & \textbf{F1} & \textbf{Accuracy} & \textbf{F1}  \\ \hline
% (\rownumber)~AlexNet~\cite{krizhevsky2017imagenet}  & 58.13\% & 56.35\%  & 56.24\%& 55.02\%  \\ 
% (\rownumber)~VGG16~\cite{simonyan2014very}  & 63.75\%& 63.08\%  & 59.34\%& 59.02\%  \\ 
(\rownumber)~ResNet101~\cite{he2016deep} & 66.16\%& 65.56\%  & 62.02\% & 61.34\%  \\ 
(\rownumber)~CDA~\cite{han2023boosting} & 66.71\%& 65.37\%  & 64.14\% & 62.85\%  \\ 
(\rownumber)~CECCN~\cite{ruan2024color} & 67.96\%& 66.74\%  & 64.59\%& 64.72\% \\ 
(\rownumber)~EmoVIT~\cite{xie2024emovit} & 68.09\%& 67.45\%  & 63.12\% & 61.97\%  \\ 
(\rownumber)~ComLDL~\cite{zhang2022compound} & 68.83\%& 67.28\%  & 65.29\% & 63.12\%  \\ 
(\rownumber)~WSDEN~\cite{li2023weakly} & 69.78\%& 69.61\%  & 67.04\% & 65.49\% \\ 
(\rownumber)~ECWA~\cite{deng2021emotion} & 70.87\%& 69.08\%  & 67.81\% & 66.87\%  \\ 
(\rownumber)~EECon~\cite{yang2023exploiting} & 71.13\%& 68.34\%  & 64.27\%& 63.16\%  \\ 
(\rownumber)~MAM~\cite{zhang2024affective} & 71.44\%  & 70.83\% & 67.18\%  & 65.01\%\\ 
(\rownumber)~TGCA-PVT~\cite{chen2024tgca}   & 73.05\%  & 71.46\% & 69.87\%  & 68.32\% \\ 
(\rownumber)~OEAN~\cite{zhang2024object}   & 73.40\%  & 72.63\% & 70.52\%  & 69.47\% \\ \hline
(\rownumber)~\shortname  & \textbf{79.48\%} & \textbf{79.22\%} & \textbf{74.12\%} & \textbf{73.09\%} \\ \hline
\end{tabular}
}
\vspace{-6mm}
% \end{center}
\end{table}
% 

\subsection{Experiment Setting}
% \subsubsection{Model Setting.}
% 
\textbf{Model Setting:}
For feature representation, we set $k=10$ to select object tags, and adopt clip-vit-base-patch32 as the pre-trained model for unified feature representation.
Moreover, we empirically set $(d_e, d_h, d_k, d_s) = (512, 128, 16, 64)$, and set the classification class $L$ to 8.

% 

\textbf{Training Setting:}
To initialize the model, we set all weights such as $\boldsymbol{W}$ following the truncated normal distribution, and use AdamW optimizer with the learning rate of $1 \times 10^{-4}$.
% warmup scheduler of cosine, warmup steps of 2000.
Furthermore, we set the batch size to 32 and the epoch of the training process to 200.
During the implementation, we utilize \textit{PyTorch} to build our entire model.
% , and our project codes are publicly available at https://github.com/zzmyrep/MESN.
% Our project codes as well as data are all publicly available on GitHub\footnote{https://github.com/zzmyrep/KBCEN}.
% Code is available at \href{https://github.com/zzmyrep/KBCEN}{https://github.com/zzmyrep/KBCEN}.

\textbf{Evaluation Metrics:}
Following~\cite{zhang2024affective, chen2024tgca, zhang2024object}, we adopt \textit{accuracy} and \textit{F1} as our evaluation metrics to measure the performance of different methods for image sentiment analysis. 



\subsection{Experiment Result}
% We compare our model against the following baselines: AlexNet~\cite{krizhevsky2017imagenet}, VGG16~\cite{simonyan2014very}, ResNet101~\cite{he2016deep}, CECCN~\cite{ruan2024color}, EmoVIT~\cite{xie2024emovit}, WSCNet~\cite{yang2018weakly}, ECWA~\cite{deng2021emotion}, EECon~\cite{yang2023exploiting}, MAM~\cite{zhang2024affective} and TGCA-PVT~\cite{chen2024tgca}, and the overall results are summarized in Table~\ref{tab:cap1}.
We compare our model against several baselines, and the overall results are summarized in Table~\ref{tab:cap1}.
We observe that our model achieves the best performance in both accuracy and F1 metrics, significantly outperforming the previous models. 
This superior performance is mainly attributed to our effective utilization of metadata to enhance image sentiment analysis, as well as the exceptional capability of the unified sentiment transformer framework we developed. These results strongly demonstrate that our proposed method can bring encouraging performance for image sentiment analysis.

\setcounter{magicrownumbers}{0} 
\begin{table}[t]
\begin{center}
\caption{Ablation study of~\shortname~on FI dataset.} 
% \vspace{1mm}
\label{tab:cap2}
\resizebox{.9\linewidth}{!}
{
\begin{tabular}{lcc}
  \hline
  \textbf{Model} & \textbf{Accuracy} & \textbf{F1} \\
  \hline
  (\rownumber)~Ours (w/o vision) & 65.72\% & 64.54\% \\
  (\rownumber)~Ours (w/o text description) & 74.05\% & 72.58\% \\
  (\rownumber)~Ours (w/o object tag) & 77.45\% & 76.84\% \\
  (\rownumber)~Ours (w/o scene tag) & 78.47\% & 78.21\% \\
  \hline
  (\rownumber)~Ours (w/o unified embedding) & 76.41\% & 76.23\% \\
  (\rownumber)~Ours (w/o adaptive learning) & 76.83\% & 76.56\% \\
  (\rownumber)~Ours (w/o cross-modal fusion) & 76.85\% & 76.49\% \\
  \hline
  (\rownumber)~Ours  & \textbf{79.48\%} & \textbf{79.22\%} \\
  \hline
\end{tabular}
}
\end{center}
\vspace{-5mm}
\end{table}


\begin{figure}[t]
\centering
% \vspace{-2mm}
\includegraphics[width=0.42\textwidth]{fig/2dvisual-linux4-paper2.pdf}
\caption{Visualization of feature distribution on eight categories before (left) and after (right) model processing.}
% 
\label{fig:visualization}
\vspace{-5mm}
\end{figure}

\subsection{Ablation Performance}
In this subsection, we conduct an ablation study to examine which component is really important for performance improvement. The results are reported in Table~\ref{tab:cap2}.

For information utilization, we observe a significant decline in model performance when visual features are removed. Additionally, the performance of \shortname~decreases when different metadata are removed separately, which means that text description, object tag, and scene tag are all critical for image sentiment analysis.
Recalling the model architecture, we separately remove transformer layers of the unified representation module, the adaptive learning module, and the cross-modal fusion module, replacing them with MLPs of the same parameter scale.
In this way, we can observe varying degrees of decline in model performance, indicating that these modules are indispensable for our model to achieve better performance.

\subsection{Visualization}
% 


% % 开始使用minipage进行左右排列
% \begin{minipage}[t]{0.45\textwidth}  % 子图1宽度为45%
%     \centering
%     \includegraphics[width=\textwidth]{2dvisual.pdf}  % 插入图片
%     \captionof{figure}{Visualization of feature distribution.}  % 使用captionof添加图片标题
%     \label{fig:visualization}
% \end{minipage}


% \begin{figure}[t]
% \centering
% \vspace{-2mm}
% \includegraphics[width=0.45\textwidth]{fig/2dvisual.pdf}
% \caption{Visualization of feature distribution.}
% \label{fig:visualization}
% % \vspace{-4mm}
% \end{figure}

% \begin{figure}[t]
% \centering
% \vspace{-2mm}
% \includegraphics[width=0.45\textwidth]{fig/2dvisual-linux3-paper.pdf}
% \caption{Visualization of feature distribution.}
% \label{fig:visualization}
% % \vspace{-4mm}
% \end{figure}



\begin{figure}[tbp]   
\vspace{-4mm}
  \centering            
  \subfloat[Depth of adaptive learning layers]   
  {
    \label{fig:subfig1}\includegraphics[width=0.22\textwidth]{fig/fig_sensitivity-a5}
  }
  \subfloat[Depth of fusion layers]
  {
    % \label{fig:subfig2}\includegraphics[width=0.22\textwidth]{fig/fig_sensitivity-b2}
    \label{fig:subfig2}\includegraphics[width=0.22\textwidth]{fig/fig_sensitivity-b2-num.pdf}
  }
  \caption{Sensitivity study of \shortname~on different depth. }   
  \label{fig:fig_sensitivity}  
\vspace{-2mm}
\end{figure}

% \begin{figure}[htbp]
% \centerline{\includegraphics{2dvisual.pdf}}
% \caption{Visualization of feature distribution.}
% \label{fig:visualization}
% \end{figure}

% In Fig.~\ref{fig:visualization}, we use t-SNE~\cite{van2008visualizing} to reduce the dimension of data features for visualization, Figure in left represents the metadata features before model processing, the features are obtained by embedding through the CLIP model, and figure in right shows the features of the data after model processing, it can be observed that after the model processing, the data with different label categories fall in different regions in the space, therefore, we can conclude that the Therefore, we can conclude that the model can effectively utilize the information contained in the metadata and use it to guide the model for classification.

In Fig.~\ref{fig:visualization}, we use t-SNE~\cite{van2008visualizing} to reduce the dimension of data features for visualization.
The left figure shows metadata features before being processed by our model (\textit{i.e.}, embedded by CLIP), while the right shows the distribution of features after being processed by our model.
We can observe that after the model processing, data with the same label are closer to each other, while others are farther away.
Therefore, it shows that the model can effectively utilize the information contained in the metadata and use it to guide the classification process.

\subsection{Sensitivity Analysis}
% 
In this subsection, we conduct a sensitivity analysis to figure out the effect of different depth settings of adaptive learning layers and fusion layers. 
% In this subsection, we conduct a sensitivity analysis to figure out the effect of different depth settings on the model. 
% Fig.~\ref{fig:fig_sensitivity} presents the effect of different depth settings of adaptive learning layers and fusion layers. 
Taking Fig.~\ref{fig:fig_sensitivity} (a) as an example, the model performance improves with increasing depth, reaching the best performance at a depth of 4.
% Taking Fig.~\ref{fig:fig_sensitivity} (a) as an example, the performance of \shortname~improves with the increase of depth at first, reaching the best performance at a depth of 4.
When the depth continues to increase, the accuracy decreases to varying degrees.
Similar results can be observed in Fig.~\ref{fig:fig_sensitivity} (b).
Therefore, we set their depths to 4 and 6 respectively to achieve the best results.

% Through our experiments, we can observe that the effect of modifying these hyperparameters on the results of the experiments is very weak, and the surface model is not sensitive to the hyperparameters.


\subsection{Zero-shot Capability}
% 

% (1)~GCH~\cite{2010Analyzing} & 21.78\% & (5)~RA-DLNet~\cite{2020A} & 34.01\% \\ \hline
% (2)~WSCNet~\cite{2019WSCNet}  & 30.25\% & (6)~CECCN~\cite{ruan2024color} & 43.83\% \\ \hline
% (3)~PCNN~\cite{2015Robust} & 31.68\%  & (7)~EmoVIT~\cite{xie2024emovit} & 44.90\% \\ \hline
% (4)~AR~\cite{2018Visual} & 32.67\% & (8)~Ours (Zero-shot) & 47.83\% \\ \hline


\begin{table}[t]
\centering
\caption{Zero-shot capability of \shortname.}
\label{tab:cap3}
\resizebox{1\linewidth}{!}
{
\begin{tabular}{lc|lc}
\hline
\textbf{Model} & \textbf{Accuracy} & \textbf{Model} & \textbf{Accuracy} \\ \hline
(1)~WSCNet~\cite{2019WSCNet}  & 30.25\% & (5)~MAM~\cite{zhang2024affective} & 39.56\%  \\ \hline
(2)~AR~\cite{2018Visual} & 32.67\% & (6)~CECCN~\cite{ruan2024color} & 43.83\% \\ \hline
(3)~RA-DLNet~\cite{2020A} & 34.01\%  & (7)~EmoVIT~\cite{xie2024emovit} & 44.90\% \\ \hline
(4)~CDA~\cite{han2023boosting} & 38.64\% & (8)~Ours (Zero-shot) & 47.83\% \\ \hline
\end{tabular}
}
\vspace{-5mm}
\end{table}

% We use the model trained on the FI dataset to test on the artphoto dataset to verify the model's generalization ability as well as robustness to other distributed datasets.
% We can observe that the MESN model shows strong competitiveness in terms of accuracy when compared to other trained models, which suggests that the model has a good generalization ability in the OOD task.

To validate the model's generalization ability and robustness to other distributed datasets, we directly test the model trained on the FI dataset, without training on Artphoto. 
% As observed in Table 3, compared to other models trained on Artphoto, we achieve highly competitive zero-shot performance, indicating that the model has good generalization ability in out-of-distribution tasks.
From Table~\ref{tab:cap3}, we can observe that compared with other models trained on Artphoto, we achieve competitive zero-shot performance, which shows that the model has good generalization ability in out-of-distribution tasks.


%%%%%%%%%%%%
%  E2E     %
%%%%%%%%%%%%


\section{Conclusion}
In this paper, we introduced Wi-Chat, the first LLM-powered Wi-Fi-based human activity recognition system that integrates the reasoning capabilities of large language models with the sensing potential of wireless signals. Our experimental results on a self-collected Wi-Fi CSI dataset demonstrate the promising potential of LLMs in enabling zero-shot Wi-Fi sensing. These findings suggest a new paradigm for human activity recognition that does not rely on extensive labeled data. We hope future research will build upon this direction, further exploring the applications of LLMs in signal processing domains such as IoT, mobile sensing, and radar-based systems.

\section*{Limitations}
While our work represents the first attempt to leverage LLMs for processing Wi-Fi signals, it is a preliminary study focused on a relatively simple task: Wi-Fi-based human activity recognition. This choice allows us to explore the feasibility of LLMs in wireless sensing but also comes with certain limitations.

Our approach primarily evaluates zero-shot performance, which, while promising, may still lag behind traditional supervised learning methods in highly complex or fine-grained recognition tasks. Besides, our study is limited to a controlled environment with a self-collected dataset, and the generalizability of LLMs to diverse real-world scenarios with varying Wi-Fi conditions, environmental interference, and device heterogeneity remains an open question.

Additionally, we have yet to explore the full potential of LLMs in more advanced Wi-Fi sensing applications, such as fine-grained gesture recognition, occupancy detection, and passive health monitoring. Future work should investigate the scalability of LLM-based approaches, their robustness to domain shifts, and their integration with multimodal sensing techniques in broader IoT applications.


% Bibliography entries for the entire Anthology, followed by custom entries
%\bibliography{anthology,custom}
% Custom bibliography entries only
\bibliography{main}
\newpage
\appendix

\section{Experiment prompts}
\label{sec:prompt}
The prompts used in the LLM experiments are shown in the following Table~\ref{tab:prompts}.

\definecolor{titlecolor}{rgb}{0.9, 0.5, 0.1}
\definecolor{anscolor}{rgb}{0.2, 0.5, 0.8}
\definecolor{labelcolor}{HTML}{48a07e}
\begin{table*}[h]
	\centering
	
 % \vspace{-0.2cm}
	
	\begin{center}
		\begin{tikzpicture}[
				chatbox_inner/.style={rectangle, rounded corners, opacity=0, text opacity=1, font=\sffamily\scriptsize, text width=5in, text height=9pt, inner xsep=6pt, inner ysep=6pt},
				chatbox_prompt_inner/.style={chatbox_inner, align=flush left, xshift=0pt, text height=11pt},
				chatbox_user_inner/.style={chatbox_inner, align=flush left, xshift=0pt},
				chatbox_gpt_inner/.style={chatbox_inner, align=flush left, xshift=0pt},
				chatbox/.style={chatbox_inner, draw=black!25, fill=gray!7, opacity=1, text opacity=0},
				chatbox_prompt/.style={chatbox, align=flush left, fill=gray!1.5, draw=black!30, text height=10pt},
				chatbox_user/.style={chatbox, align=flush left},
				chatbox_gpt/.style={chatbox, align=flush left},
				chatbox2/.style={chatbox_gpt, fill=green!25},
				chatbox3/.style={chatbox_gpt, fill=red!20, draw=black!20},
				chatbox4/.style={chatbox_gpt, fill=yellow!30},
				labelbox/.style={rectangle, rounded corners, draw=black!50, font=\sffamily\scriptsize\bfseries, fill=gray!5, inner sep=3pt},
			]
											
			\node[chatbox_user] (q1) {
				\textbf{System prompt}
				\newline
				\newline
				You are a helpful and precise assistant for segmenting and labeling sentences. We would like to request your help on curating a dataset for entity-level hallucination detection.
				\newline \newline
                We will give you a machine generated biography and a list of checked facts about the biography. Each fact consists of a sentence and a label (True/False). Please do the following process. First, breaking down the biography into words. Second, by referring to the provided list of facts, merging some broken down words in the previous step to form meaningful entities. For example, ``strategic thinking'' should be one entity instead of two. Third, according to the labels in the list of facts, labeling each entity as True or False. Specifically, for facts that share a similar sentence structure (\eg, \textit{``He was born on Mach 9, 1941.''} (\texttt{True}) and \textit{``He was born in Ramos Mejia.''} (\texttt{False})), please first assign labels to entities that differ across atomic facts. For example, first labeling ``Mach 9, 1941'' (\texttt{True}) and ``Ramos Mejia'' (\texttt{False}) in the above case. For those entities that are the same across atomic facts (\eg, ``was born'') or are neutral (\eg, ``he,'' ``in,'' and ``on''), please label them as \texttt{True}. For the cases that there is no atomic fact that shares the same sentence structure, please identify the most informative entities in the sentence and label them with the same label as the atomic fact while treating the rest of the entities as \texttt{True}. In the end, output the entities and labels in the following format:
                \begin{itemize}[nosep]
                    \item Entity 1 (Label 1)
                    \item Entity 2 (Label 2)
                    \item ...
                    \item Entity N (Label N)
                \end{itemize}
                % \newline \newline
                Here are two examples:
                \newline\newline
                \textbf{[Example 1]}
                \newline
                [The start of the biography]
                \newline
                \textcolor{titlecolor}{Marianne McAndrew is an American actress and singer, born on November 21, 1942, in Cleveland, Ohio. She began her acting career in the late 1960s, appearing in various television shows and films.}
                \newline
                [The end of the biography]
                \newline \newline
                [The start of the list of checked facts]
                \newline
                \textcolor{anscolor}{[Marianne McAndrew is an American. (False); Marianne McAndrew is an actress. (True); Marianne McAndrew is a singer. (False); Marianne McAndrew was born on November 21, 1942. (False); Marianne McAndrew was born in Cleveland, Ohio. (False); She began her acting career in the late 1960s. (True); She has appeared in various television shows. (True); She has appeared in various films. (True)]}
                \newline
                [The end of the list of checked facts]
                \newline \newline
                [The start of the ideal output]
                \newline
                \textcolor{labelcolor}{[Marianne McAndrew (True); is (True); an (True); American (False); actress (True); and (True); singer (False); , (True); born (True); on (True); November 21, 1942 (False); , (True); in (True); Cleveland, Ohio (False); . (True); She (True); began (True); her (True); acting career (True); in (True); the late 1960s (True); , (True); appearing (True); in (True); various (True); television shows (True); and (True); films (True); . (True)]}
                \newline
                [The end of the ideal output]
				\newline \newline
                \textbf{[Example 2]}
                \newline
                [The start of the biography]
                \newline
                \textcolor{titlecolor}{Doug Sheehan is an American actor who was born on April 27, 1949, in Santa Monica, California. He is best known for his roles in soap operas, including his portrayal of Joe Kelly on ``General Hospital'' and Ben Gibson on ``Knots Landing.''}
                \newline
                [The end of the biography]
                \newline \newline
                [The start of the list of checked facts]
                \newline
                \textcolor{anscolor}{[Doug Sheehan is an American. (True); Doug Sheehan is an actor. (True); Doug Sheehan was born on April 27, 1949. (True); Doug Sheehan was born in Santa Monica, California. (False); He is best known for his roles in soap operas. (True); He portrayed Joe Kelly. (True); Joe Kelly was in General Hospital. (True); General Hospital is a soap opera. (True); He portrayed Ben Gibson. (True); Ben Gibson was in Knots Landing. (True); Knots Landing is a soap opera. (True)]}
                \newline
                [The end of the list of checked facts]
                \newline \newline
                [The start of the ideal output]
                \newline
                \textcolor{labelcolor}{[Doug Sheehan (True); is (True); an (True); American (True); actor (True); who (True); was born (True); on (True); April 27, 1949 (True); in (True); Santa Monica, California (False); . (True); He (True); is (True); best known (True); for (True); his roles in soap operas (True); , (True); including (True); in (True); his portrayal (True); of (True); Joe Kelly (True); on (True); ``General Hospital'' (True); and (True); Ben Gibson (True); on (True); ``Knots Landing.'' (True)]}
                \newline
                [The end of the ideal output]
				\newline \newline
				\textbf{User prompt}
				\newline
				\newline
				[The start of the biography]
				\newline
				\textcolor{magenta}{\texttt{\{BIOGRAPHY\}}}
				\newline
				[The ebd of the biography]
				\newline \newline
				[The start of the list of checked facts]
				\newline
				\textcolor{magenta}{\texttt{\{LIST OF CHECKED FACTS\}}}
				\newline
				[The end of the list of checked facts]
			};
			\node[chatbox_user_inner] (q1_text) at (q1) {
				\textbf{System prompt}
				\newline
				\newline
				You are a helpful and precise assistant for segmenting and labeling sentences. We would like to request your help on curating a dataset for entity-level hallucination detection.
				\newline \newline
                We will give you a machine generated biography and a list of checked facts about the biography. Each fact consists of a sentence and a label (True/False). Please do the following process. First, breaking down the biography into words. Second, by referring to the provided list of facts, merging some broken down words in the previous step to form meaningful entities. For example, ``strategic thinking'' should be one entity instead of two. Third, according to the labels in the list of facts, labeling each entity as True or False. Specifically, for facts that share a similar sentence structure (\eg, \textit{``He was born on Mach 9, 1941.''} (\texttt{True}) and \textit{``He was born in Ramos Mejia.''} (\texttt{False})), please first assign labels to entities that differ across atomic facts. For example, first labeling ``Mach 9, 1941'' (\texttt{True}) and ``Ramos Mejia'' (\texttt{False}) in the above case. For those entities that are the same across atomic facts (\eg, ``was born'') or are neutral (\eg, ``he,'' ``in,'' and ``on''), please label them as \texttt{True}. For the cases that there is no atomic fact that shares the same sentence structure, please identify the most informative entities in the sentence and label them with the same label as the atomic fact while treating the rest of the entities as \texttt{True}. In the end, output the entities and labels in the following format:
                \begin{itemize}[nosep]
                    \item Entity 1 (Label 1)
                    \item Entity 2 (Label 2)
                    \item ...
                    \item Entity N (Label N)
                \end{itemize}
                % \newline \newline
                Here are two examples:
                \newline\newline
                \textbf{[Example 1]}
                \newline
                [The start of the biography]
                \newline
                \textcolor{titlecolor}{Marianne McAndrew is an American actress and singer, born on November 21, 1942, in Cleveland, Ohio. She began her acting career in the late 1960s, appearing in various television shows and films.}
                \newline
                [The end of the biography]
                \newline \newline
                [The start of the list of checked facts]
                \newline
                \textcolor{anscolor}{[Marianne McAndrew is an American. (False); Marianne McAndrew is an actress. (True); Marianne McAndrew is a singer. (False); Marianne McAndrew was born on November 21, 1942. (False); Marianne McAndrew was born in Cleveland, Ohio. (False); She began her acting career in the late 1960s. (True); She has appeared in various television shows. (True); She has appeared in various films. (True)]}
                \newline
                [The end of the list of checked facts]
                \newline \newline
                [The start of the ideal output]
                \newline
                \textcolor{labelcolor}{[Marianne McAndrew (True); is (True); an (True); American (False); actress (True); and (True); singer (False); , (True); born (True); on (True); November 21, 1942 (False); , (True); in (True); Cleveland, Ohio (False); . (True); She (True); began (True); her (True); acting career (True); in (True); the late 1960s (True); , (True); appearing (True); in (True); various (True); television shows (True); and (True); films (True); . (True)]}
                \newline
                [The end of the ideal output]
				\newline \newline
                \textbf{[Example 2]}
                \newline
                [The start of the biography]
                \newline
                \textcolor{titlecolor}{Doug Sheehan is an American actor who was born on April 27, 1949, in Santa Monica, California. He is best known for his roles in soap operas, including his portrayal of Joe Kelly on ``General Hospital'' and Ben Gibson on ``Knots Landing.''}
                \newline
                [The end of the biography]
                \newline \newline
                [The start of the list of checked facts]
                \newline
                \textcolor{anscolor}{[Doug Sheehan is an American. (True); Doug Sheehan is an actor. (True); Doug Sheehan was born on April 27, 1949. (True); Doug Sheehan was born in Santa Monica, California. (False); He is best known for his roles in soap operas. (True); He portrayed Joe Kelly. (True); Joe Kelly was in General Hospital. (True); General Hospital is a soap opera. (True); He portrayed Ben Gibson. (True); Ben Gibson was in Knots Landing. (True); Knots Landing is a soap opera. (True)]}
                \newline
                [The end of the list of checked facts]
                \newline \newline
                [The start of the ideal output]
                \newline
                \textcolor{labelcolor}{[Doug Sheehan (True); is (True); an (True); American (True); actor (True); who (True); was born (True); on (True); April 27, 1949 (True); in (True); Santa Monica, California (False); . (True); He (True); is (True); best known (True); for (True); his roles in soap operas (True); , (True); including (True); in (True); his portrayal (True); of (True); Joe Kelly (True); on (True); ``General Hospital'' (True); and (True); Ben Gibson (True); on (True); ``Knots Landing.'' (True)]}
                \newline
                [The end of the ideal output]
				\newline \newline
				\textbf{User prompt}
				\newline
				\newline
				[The start of the biography]
				\newline
				\textcolor{magenta}{\texttt{\{BIOGRAPHY\}}}
				\newline
				[The ebd of the biography]
				\newline \newline
				[The start of the list of checked facts]
				\newline
				\textcolor{magenta}{\texttt{\{LIST OF CHECKED FACTS\}}}
				\newline
				[The end of the list of checked facts]
			};
		\end{tikzpicture}
        \caption{GPT-4o prompt for labeling hallucinated entities.}\label{tb:gpt-4-prompt}
	\end{center}
\vspace{-0cm}
\end{table*}
% \section{Full Experiment Results}
% \begin{table*}[th]
    \centering
    \small
    \caption{Classification Results}
    \begin{tabular}{lcccc}
        \toprule
        \textbf{Method} & \textbf{Accuracy} & \textbf{Precision} & \textbf{Recall} & \textbf{F1-score} \\
        \midrule
        \multicolumn{5}{c}{\textbf{Zero Shot}} \\
                Zero-shot E-eyes & 0.26 & 0.26 & 0.27 & 0.26 \\
        Zero-shot CARM & 0.24 & 0.24 & 0.24 & 0.24 \\
                Zero-shot SVM & 0.27 & 0.28 & 0.28 & 0.27 \\
        Zero-shot CNN & 0.23 & 0.24 & 0.23 & 0.23 \\
        Zero-shot RNN & 0.26 & 0.26 & 0.26 & 0.26 \\
DeepSeek-0shot & 0.54 & 0.61 & 0.54 & 0.52 \\
DeepSeek-0shot-COT & 0.33 & 0.24 & 0.33 & 0.23 \\
DeepSeek-0shot-Knowledge & 0.45 & 0.46 & 0.45 & 0.44 \\
Gemma2-0shot & 0.35 & 0.22 & 0.38 & 0.27 \\
Gemma2-0shot-COT & 0.36 & 0.22 & 0.36 & 0.27 \\
Gemma2-0shot-Knowledge & 0.32 & 0.18 & 0.34 & 0.20 \\
GPT-4o-mini-0shot & 0.48 & 0.53 & 0.48 & 0.41 \\
GPT-4o-mini-0shot-COT & 0.33 & 0.50 & 0.33 & 0.38 \\
GPT-4o-mini-0shot-Knowledge & 0.49 & 0.31 & 0.49 & 0.36 \\
GPT-4o-0shot & 0.62 & 0.62 & 0.47 & 0.42 \\
GPT-4o-0shot-COT & 0.29 & 0.45 & 0.29 & 0.21 \\
GPT-4o-0shot-Knowledge & 0.44 & 0.52 & 0.44 & 0.39 \\
LLaMA-0shot & 0.32 & 0.25 & 0.32 & 0.24 \\
LLaMA-0shot-COT & 0.12 & 0.25 & 0.12 & 0.09 \\
LLaMA-0shot-Knowledge & 0.32 & 0.25 & 0.32 & 0.28 \\
Mistral-0shot & 0.19 & 0.23 & 0.19 & 0.10 \\
Mistral-0shot-Knowledge & 0.21 & 0.40 & 0.21 & 0.11 \\
        \midrule
        \multicolumn{5}{c}{\textbf{4 Shot}} \\
GPT-4o-mini-4shot & 0.58 & 0.59 & 0.58 & 0.53 \\
GPT-4o-mini-4shot-COT & 0.57 & 0.53 & 0.57 & 0.50 \\
GPT-4o-mini-4shot-Knowledge & 0.56 & 0.51 & 0.56 & 0.47 \\
GPT-4o-4shot & 0.77 & 0.84 & 0.77 & 0.73 \\
GPT-4o-4shot-COT & 0.63 & 0.76 & 0.63 & 0.53 \\
GPT-4o-4shot-Knowledge & 0.72 & 0.82 & 0.71 & 0.66 \\
LLaMA-4shot & 0.29 & 0.24 & 0.29 & 0.21 \\
LLaMA-4shot-COT & 0.20 & 0.30 & 0.20 & 0.13 \\
LLaMA-4shot-Knowledge & 0.15 & 0.23 & 0.13 & 0.13 \\
Mistral-4shot & 0.02 & 0.02 & 0.02 & 0.02 \\
Mistral-4shot-Knowledge & 0.21 & 0.27 & 0.21 & 0.20 \\
        \midrule
        
        \multicolumn{5}{c}{\textbf{Suprevised}} \\
        SVM & 0.94 & 0.92 & 0.91 & 0.91 \\
        CNN & 0.98 & 0.98 & 0.97 & 0.97 \\
        RNN & 0.99 & 0.99 & 0.99 & 0.99 \\
        % \midrule
        % \multicolumn{5}{c}{\textbf{Conventional Wi-Fi-based Human Activity Recognition Systems}} \\
        E-eyes & 1.00 & 1.00 & 1.00 & 1.00 \\
        CARM & 0.98 & 0.98 & 0.98 & 0.98 \\
\midrule
 \multicolumn{5}{c}{\textbf{Vision Models}} \\
           Zero-shot SVM & 0.26 & 0.25 & 0.25 & 0.25 \\
        Zero-shot CNN & 0.26 & 0.25 & 0.26 & 0.26 \\
        Zero-shot RNN & 0.28 & 0.28 & 0.29 & 0.28 \\
        SVM & 0.99 & 0.99 & 0.99 & 0.99 \\
        CNN & 0.98 & 0.99 & 0.98 & 0.98 \\
        RNN & 0.98 & 0.99 & 0.98 & 0.98 \\
GPT-4o-mini-Vision & 0.84 & 0.85 & 0.84 & 0.84 \\
GPT-4o-mini-Vision-COT & 0.90 & 0.91 & 0.90 & 0.90 \\
GPT-4o-Vision & 0.74 & 0.82 & 0.74 & 0.73 \\
GPT-4o-Vision-COT & 0.70 & 0.83 & 0.70 & 0.68 \\
LLaMA-Vision & 0.20 & 0.23 & 0.20 & 0.09 \\
LLaMA-Vision-Knowledge & 0.22 & 0.05 & 0.22 & 0.08 \\

        \bottomrule
    \end{tabular}
    \label{full}
\end{table*}




\end{document}

 %\bibliographystyle{elsarticle-num} 
 % \bibliography{references}
 


%% else use the following coding to input the bibitems directly in the
%% TeX file.

% \begin{thebibliography}{00}

% %% \bibitem{label}
% %% Text of bibliographic item

% \bibitem{}

% \end{thebibliography}
\end{document}
\endinput
%%
%% End of file `elsarticle-template-num.tex'.
