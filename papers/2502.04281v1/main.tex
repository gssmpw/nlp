%%%%%%%% ICML 2025 EXAMPLE LATEX SUBMISSION FILE %%%%%%%%%%%%%%%%%

\documentclass{article}

% Recommended, but optional, packages for figures and better typesetting:
\usepackage{microtype}
\usepackage{graphicx}
% \usepackage{subfigure}
\usepackage{booktabs} % for professional tables

% hyperref makes hyperlinks in the resulting PDF.
% If your build breaks (sometimes temporarily if a hyperlink spans a page)
% please comment out the following usepackage line and replace
% \usepackage{icml2025} with \usepackage[nohyperref]{icml2025} above.
\usepackage{hyperref}


% Attempt to make hyperref and algorithmic work together better:
% \newcommand{\theHalgorithm}{\arabic{algorithm}}

% Use the following line for the initial blind version submitted for review:
% \usepackage{icml2025}

% If accepted, instead use the following line for the camera-ready submission:
\usepackage[accepted]{icml2025}

% For theorems and such
\usepackage{amsmath}
\usepackage{amssymb}
\usepackage{mathtools}
\usepackage{amsthm}

% if you use cleveref..
\usepackage[capitalize,noabbrev]{cleveref}


% CUSTOM IMPORTS
\usepackage{balance} % for balancing columns on the final page
\usepackage{multirow} % For merging cells


%CUSTOM COMMANDS
\usepackage{algpseudocode}
\usepackage{amsmath}
\usepackage{subcaption}
\usepackage{amsthm}

%----------------------------------------------------------------------------------------
%	COMMANDS: MATH
%---------------------------------------------------------------------------------------
\def\bigtimes{\prod}
\newcommand{\size}[1]{|#1|}
\def\is{\!=\!}
\def\st{\: | \:}
\def\oof{\:\mathbf{of} \:\:}
\newcommand{\argmin}{\operatornamewithlimits{argmin}}
\newcommand{\argmax}{\operatornamewithlimits{argmax}}


\def\credulous{{\models^c_L}}
\def\skeptical{{\models^s_L}} 
\def\notcredulous{\not\models^c_L}              
\def\notskeptical{\not\models^s_L}   

\newcommand{\var}{\text{var}}
\newcommand{\acronym}{\text{SI}\xspace} %for Simple Incentives
%----------------------------------------------------------------------------------------
%	COMMANDS: FONTS
%----------------------------------------------------------------------------------------
\newcommand{\setf}[1]{{\bf{#1}}}
\newcommand{\scope}[1]{\mathbf{x}^{#1}}


%----------------------------------------------------------------------------------------
%	ENVIRONMENTS: ITEMIZE 
%----------------------------------------------------------------------------------------
\newcommand{\squishlist}{\begin{list}{$\bullet$}{\topsep=1pt \parsep=0pt \itemsep=1pt \leftmargin=1em }} 
\newcommand{\squishend}{\end{list}}

\newcommand{\beitemize}{\begin{list}{$\bullet$}{}} 
\newcommand{\enitemize}{\end{list}}

\newcommand{\memo}[1]{
	\medskip\noindent\fbox{\begin{minipage}[b]{\dimexpr\linewidth-1em} #1 \end{minipage}}\medskip\newline
}


%%%%%%%%%%%%%%%%%%%%%%%%%%%%%%%%
% THEOREMS
%%%%%%%%%%%%%%%%%%%%%%%%%%%%%%%%
\theoremstyle{plain}
\newtheorem{theorem}{Theorem}[section]
\newtheorem{proposition}[theorem]{Proposition}
\newtheorem{lemma}[theorem]{Lemma}
\newtheorem{corollary}[theorem]{Corollary}
\theoremstyle{definition}
\newtheorem{definition}[theorem]{Definition}
\newtheorem{assumption}[theorem]{Assumption}
\theoremstyle{remark}
\newtheorem{remark}[theorem]{Remark}

% Todonotes is useful during development; simply uncomment the next line
%    and comment out the line below the next line to turn off comments
%\usepackage[disable,textsize=tiny]{todonotes}
\usepackage[textsize=tiny]{todonotes}


% The \icmltitle you define below is probably too long as a header.
% Therefore, a short form for the running title is supplied here:
\icmltitlerunning{DECAF: Learning to be Fair in Multi-agent Resource Allocation}

\begin{document}

\twocolumn[
\icmltitle{DECAF: Learning to be Fair in Multi-agent Resource Allocation}

% It is OKAY to include author information, even for blind
% submissions: the style file will automatically remove it for you
% unless you've provided the [accepted] option to the icml2025
% package.

% List of affiliations: The first argument should be a (short)
% identifier you will use later to specify author affiliations
% Academic affiliations should list Department, University, City, Region, Country
% Industry affiliations should list Company, City, Region, Country

% You can specify symbols, otherwise they are numbered in order.
% Ideally, you should not use this facility. Affiliations will be numbered
% in order of appearance and this is the preferred way.
\icmlsetsymbol{equal}{*}

\begin{icmlauthorlist}
\icmlauthor{Ashwin Kumar}{sch}
\icmlauthor{William Yeoh}{sch}
\end{icmlauthorlist}

\icmlaffiliation{sch}{Washington University in St Louis, USA}

\icmlcorrespondingauthor{Ashwin Kumar}{ashwinkumar@wustl.edu}
\icmlcorrespondingauthor{William Yeoh}{wyeoh@wustl.edu}

% You may provide any keywords that you
% find helpful for describing your paper; these are used to populate
% the "keywords" metadata in the PDF but will not be shown in the document
\icmlkeywords{Reinforcement Learning, ICML, Multiagent Systems, Reosurce Allocation, Fairness}

\vskip 0.3in
]

% this must go after the closing bracket ] following \twocolumn[ ...

% This command actually creates the footnote in the first column
% listing the affiliations and the copyright notice.
% The command takes one argument, which is text to display at the start of the footnote.
% The \icmlEqualContribution command is standard text for equal contribution.
% Remove it (just {}) if you do not need this facility.

% \printAffiliationsAndNotice{}  % leave blank if no need to mention equal contribution
% \printAffiliationsAndNotice{\icmlEqualContribution} % otherwise use the standard text.

\begin{abstract}
A wide variety of resource allocation problems operate under resource constraints that are managed by a central arbitrator, with agents who evaluate and communicate preferences over these resources. We formulate this broad class of problems as  \emph{Distributed Evaluation, Centralized Allocation (DECA)} problems and propose methods to learn fair and efficient policies in centralized resource allocation. Our methods are applied to learning long-term fairness in a novel and general framework for fairness in multi-agent systems. We show three different methods based on Double Deep Q-Learning: (1) A joint weighted optimization of fairness and utility, (2) a split optimization, learning two separate Q-estimators for utility and fairness, and (3) an online policy perturbation to guide existing black-box utility functions toward fair solutions. Our methods outperform existing fair MARL approaches on multiple resource allocation domains, even when evaluated using diverse fairness functions, and allow for flexible online trade-offs between utility and fairness.
\end{abstract}



\section{Introduction}

\label{sec:intro}

AI has become an essential component of many modern systems, playing a crucial role in automating complex decision-making processes. Increasingly, AI algorithms are used to make decisions that impact millions of people. In multi-agent settings, these decisions are often optimized for overall system utility. However, this utilitarian approach can introduce biases, making fairness an important consideration in AI-driven decision-making.

We introduce a class of problems called \emph{Distributed Evaluation, Centralized Allocation (DECA)}. To the best of our knowledge, prior work has addressed these problems separately using domain-specific solutions~\citep{shah2020neural, qin2022RLforRides,kube2019allocating, kube2023community}. In this paper, we take a first step toward unifying them under a single DECA framework and propose fairness-oriented methods that apply broadly across DECA problems.

In DECA, multiple agents act within an environment while a central controller coordinates their behavior to ensure resource constraints and environmental requirements are met. Each agent evaluates its own actions (Distributed Evaluation, DE), and the central controller aggregates these evaluations, optimizes for system-wide utility, and assigns actions accordingly (Centralized Allocation, CA). These problems are dynamic in nature, 
with time-varying resources and agents with variable action spaces. 
This makes DECA both computationally challenging and practically significant.\footnote{
Note that DECA is an execution paradigm, not a learning paradigm like Centralized Training/Decentralized Execution (CTDE). CTDE can be used to train agents that operate in a DECA environment, as we do in this work.
}

Fairness in DECA-based decision-making is critical, as algorithmic biases can lead to disparities and decreased trust in automated systems~\citep{mehrabi2021MLBiassurvey}. Beyond ethical considerations, fair resource allocation may also be desirable from the perspective of the central controller. Standard DECA solutions rely on estimating agent utilities and solving a constrained optimization to compute the best action for each agent. To improve fairness, we propose a framework that integrates fairness estimation into DECA, allowing for more balanced allocations. Specifically, we introduce three optimization strategies:

\squishlist
\item \textbf{Joint Optimization (JO)}: A scalarized multi-objective learning approach that jointly optimizes for fairness and utility.
\item \textbf{Split Optimization (SO)}: A method that learns separate fairness and utility estimators, enabling online trade-off adjustments for fairness and utility.
\item \textbf{Fair-Only Optimization (FO)}: A fairness-focused approach that modifies an existing black-box utility function to incorporate fairness considerations.
\squishend

Our approach is broadly applicable across different domains and fairness metrics, as demonstrated through empirical evaluations. We compare these methods and show how each offers unique advantages in different scenarios, using variance in agent utilities as a fairness metric. Notably, the Split and Fair-Only Optimization approaches enable real-time tuning of the fairness-utility trade-off, an important consideration for real-world applications that has been overlooked in prior work on fairness in multi-agent RL.

This work addresses a critical gap in the literature by providing a unified framework for fairness in DECA problems, which encompass a wide range of multi-agent decision-making scenarios. Through our proposed optimization strategies (JO, SO, and FO), we offer a flexible framework for balancing fairness and efficiency in real time. Furthermore, we present the first general approach for integrating fairness into multi-agent resource allocation using Q-learning, paving the way for future advancements in fair AI decision-making.


\section{Related Work}
\label{sec:related}

While DECA has not been formalized in prior work, it has seen application in many domains. From optimizing passenger-driver matches in ridesharing~\citep{shah2020neural, qin2022RLforRides} to efficient allocation of homelessness resources~\citep{kube2019allocating, kube2023community}, many real-world applications follow this general structure. 
It is also analogous to the predict-then-optimize (P+O) approach \citep{wang2021PtO, elmachtoub2022smartPtO}, where a predictive model estimates unknown parameters that are subsequently used in optimization. However, unlike P+O, which may not specifically address resource allocation or multi-agent systems, DECA explicitly focuses on these complexities. This distinction is crucial as it allows us to restrict the problem space and tailor our research towards enhancing fairness within multi-agent resource allocation.

Significant research has addressed algorithmic bias, where ML models, such as those used in hiring decisions \citep{raghavan2020hiringbias}, can exhibit harmful biases. 
We refer readers to an extensive survey by \citet{mehrabi2021MLBiassurvey} for a review of recent work.
These studies typically focus on debiasing the outputs of predictive models to meet fairness criteria such as equalized odds~\citep{hardt2016equality} or demographic parity~\citep{DP_dwork2012}. However, our work diverges from this approach. Instead of correcting biases in predictions, we aim to develop algorithms that inherently promote fair decision-making via the actions they optimize.

Our focus in this paper is on designing ways to learn fair policies in a multi-agent RL setting. \citet{FairRLSurvey} present a survey on RL methods used to improve fairness. We now highlight a few papers that are the closest to our work: FEN~\citep{jiang2019FEN} uses a hierarchical network to learn a fair-efficient policy for multi-agent coordination, learning to optimize the coefficient of variation, with a meta network that selects when each agent behaves greedily or fairly. However, the model does not allow for resource constraints, instead opting for a first-come-first-serve approach. Further, this approach needs communication between agents to allow agents to choose between acting fairly and efficiently. Some methods, on the other hand, propose to optimize fairness in a multi-objective MDP, where each agent's utility is treated as a separate objective, and the goal is to optimize the a social welfare function over agent rewards~\citep{zimmer2021MOMDP,siddique2020MOMDP}. 
This means the learning agent has to predict the utility over the joint action space~\citep{siddique2020MOMDP}, or, as done by SOTO~\citep{zimmer2021MOMDP}, use a decentralized policy gradient based approach, which prevents use of global constraints. 
Finally, SI~\citep{SI_kumar2023} is an approach for improving fairness in rideshare-matching that attempts to improve fairness through myopic fairness post-processing of black-box utility estimates. However, it does not attempt to learn long-term fairness, and is specially designed for the ridesharing domain.

The DECA approach allows us to consider global constraints while allocating resources, opening up the scope for better global solutions, which none of the prior approaches allow. The distributed evaluation allows each agent to only learn a local value function, which reduces the complexity when compared to learning a joint policy. Further, our Split and Fair-Only approaches allow changing the trade-offs between utility and fairness post-training, which provides additional flexibility that previous approaches lack. 


\section{Viewer-provider two-sided systems}

This section models the dynamics of viewer and provider populations on a recommendation platform. 
Specifically, we consider sub-group dynamics where viewers and providers are categorized into $K$ and $L$ subgroups\footnote{We can consider a ``subgroup'' of size 1. In such cases, the viewer ``population'' corresponds to the time spent by an individual viewer, while the provider ``population'' can be the amount of content produced by an individual provider.
}. Then, we model the populations, recommendation policy, payoffs, and social welfare as follows.

\begin{enumerate}[leftmargin=12pt]
    \item (Viewer/provider population)  
    Let $\lambda_{k} \in \mathbb{R}_{\geq 0}$ be the population of the viewer group $k \in [K]$ and $\lambda_{l}$ be that of the provider group $l \in [L]$. Also let $\blambda := (\lambda_{1}, \lambda_{2}, \cdots, \lambda_{K},
    \lambda_{1}, \lambda_{2}, \cdots, \lambda_{L})$ be the joint population vector of viewers and providers.
    \item (Platform's recommendation policy) 
    The platform matches each viewer group $k$ to a provider group $l$ with a recommendation policy denoted by a $K$-by-$L$ matrix $\bpi$. Specifically, its $(k,l)$-th element $\pi_{k,l}$ represents the probability of allocating the provider group $l$ to the viewer group $k$. 
    Thus $\sum_{l=1}^L \pi_{k,l} = 1, \forall k \in [K]$. For example, the uniform random policy, which assigns equal exposure probability across all provider groups is represented as given by $\bpi=\frac{1}{L}\1_{L\times K}$.
    \item (Viewer/provider payoffs) After viewer and provider groups are matched by the policy $\bpi$, their perceived payoffs can be quantified by the following metrics:
    \begin{align}\label{eq:user_satisfaction}
    \text{Viewer Satisfaction: \quad } & s_k = \textstyle \sum_{l=1}^L \pi_{k,l} q_{k,l} \,  , \\\label{eq:content_exposure}
    \text{Provider Exposure: \quad} & e_l = \textstyle\sum_{k=1}^K \pi_{k,l}\lambda_k,
    \end{align}
    where $q_{k,l}$ is the (expected) utility that viewers $k$ receive from the provider groups $l$. Eqs.~\eqref{eq:user_satisfaction} and~\eqref{eq:content_exposure} define viewer satisfaction as determined by the total utility they receive from recommendations, while providers care about the total amount of exposure they receive by recommendation. This model is prevalent is prior works including \citep{singh2018fairness, mladenov2020optimizing}.
    \item (Social welfare) Finally, we consider the following total viewer welfare as the global metric of the platform:
    \begin{align*}
        R(\bpi; \blambda) := \textstyle\sum_{k=1}^{K} \lambda_{k} s_k
    \end{align*}
    Note that here we consider the sum of viewer-side satisfaction as the social welfare, a formulation prevalent in related works~\citep{mladenov2020optimizing, huttenlocher2023matching}.
    The sum of content-side exposure simplifies to the total size of the viewer population.
\end{enumerate}

\subsection{Interaction dynamics and ``population effects''}\label{sec:dynamic_formulation}

We have so far seen a typical formulation in two-sided platforms. However, a key limitation of such formulation is to ignore potential non-stationarity in the viewer and provider populations, which is common in many real-world two-sided systems~\citep{boutilier2023modeling,  deffayet2024sardine}. 

First, consider the impact of provider population growth on the utility experience by viewers, which we call \textit{``population effects''}.
An increase in provider population naturally leads to more high-quality content. 
For example, consider a two-stage recommendation policy, where our higher-level policy $\bpi$ decides the matching between viewer and provider groups, and a second-stage policy selects individual providers among the selected group. 
Any reasonable second stage policy should be able to select a better provider from a larger provider pool~\citep{su2023value, evnine2024achieving}. 
To model such ``population effects'', we introduce the following utility decomposition:
\begin{align}
    q_{k,l} = b_{k,l} + f_{k,l}(\lambda_{l}) \label{eq:reward_decomposition}
\end{align}
where $b_{k,l}$ is the \textit{base} utility, which may indicates the matching between the preference of viewer and provider groups (e.g., this viewer group likes sports articles). In contrast, $f_{k,l}(\cdot)$ represents the quality of the provider which improves as the provider population increases. $f_{k,l}$ might be heterogeneous among different viewer and provider groups because quality might be multi-dimensional (e.g., visuals, intellects, novelty), viewers may have different preferences, and providers may have different abilities. 
We take $f_{k,l}$ to be a monotonically increasing function.

Next, consider the impact of viewer and provider payoffs on the population.
The number of viewers that a platform can maintain is related to the level of satisfaction, similarly the number of providers is related to the exposure.
We assume that viewer and provider subgroups have 
some \textit{``reference''} population $\bar{\lambda}_{k}(s_{k})$ and $\bar{\lambda}_{l}(e_{l})$ given the level of viewer satisfaction $s_k$ and provider exposure $e_l$. We assume that $\bar{\lambda}$ is a monotonically increasing function, so higher viewer satisfaction and provider exposure result in increased populations. 
Based on this, we model the viewer and provider population dynamics as:
\begin{align}
    \text{Viewer: \,}  \lambda_{t+1,k} = (1 - \eta_k) \lambda_{t,k} + \eta_k \bar{\lambda}_{k}(s_{t,k}), \label{eq:user_dynamics} \\
    \text{Content: \,}  \lambda_{t+1,l} = (1 - \eta_l) \lambda_{t,l} + \eta_l \bar{\lambda}_{l}(e_{t,l}), \label{eq:content_dynamics}
\end{align}
where $\eta \in [0, 1]$ are the \textit{reactiveness} hyperparams, determining how fast the population changes. Note that similar models are widely adopted in performative predictions~\citep{perdomo2020performative, brown2022performative}. 
We thus have that the viewer satisfaction $s_k$ depends on the provider population via ``population effects'' $f_{k,l}$, while the provider exposure directly depends on the viewer population.
The two-sided platform has complex dynamics between viewers and providers. 
Our goal will be to consider long-term objectives under such co-evolving and two-sided dynamics.

\subsection{Game-theoretic interpretation}\label{sec:game_formulation}

Next, we provide a further justification of and insight into the dynamics model by introducing a game-theoretic formulation that is equivalent to Eqs. \eqref{eq:user_dynamics} and \eqref{eq:content_dynamics}.

Consider a $(K+L)$-player game involving $K$ viewer groups and $L$ provider groups. Each viewer group selects a pure strategy $\lambda_k \in \RR_{\geq 0}$, and each provider group chooses a pure strategy $\lambda_l \in \RR_{\geq 0}$. The utility functions for the viewer and provider groups, denoted by $\{u_k\}_{k=1}^K$ and $\{v_l\}_{l=1}^L$ are defined as follows:
\begin{align}\label{eq:util_user}
    & u_k(\blambda)= \lambda_k \cdot \bar{\lambda}_k \left(\sum_{l=1}^L \pi_{k,l}\left(b_{k,l}+f_{k,l}(\lambda_l)\right)\right)-\frac{\lambda_k^2}{2}, \\ \label{eq:util_creator}
    & v_l(\blambda)= \lambda_l\cdot \bar{\lambda}_l \left(\textstyle\sum_{k=1}^K \pi_{k,l}\lambda_k\right)-\frac{\lambda_l^2}{2},
\end{align}
We denote this game as $\G(\bpi, B, f, \bar{\lambda})$, where $B$ is a $K$-by-$L$ matrix whose $(k,l)$-element is $b_{k,l}$. Proposition \ref{prop:dynamics_equivalence} establishes a connection between the game instance $\G$ and the 
formulation presented in Section \ref{sec:dynamic_formulation}.

\begin{proposition}\label{prop:dynamics_equivalence}
    If all players in $\G$ apply gradient ascent to optimize their utility functions with learning rates $\{\eta_k\}_{k=1}^K$ and $\{\eta_l\}_{l=1}^L$, the resulting joint strategy evolving dynamics are exactly given by Eqs.~\eqref{eq:user_dynamics} and \eqref{eq:content_dynamics}.
\end{proposition}

Through Proposition \ref{prop:dynamics_equivalence}, our game-theoretic formulation provides a first-principles perspective for understanding the dynamical formulation in Eqs.~\eqref{eq:user_dynamics} and \eqref{eq:content_dynamics}.\footnote{The game $\G$ resembles the Cournot Duopoly competition \cite{cournot1838recherches}. When $K = L = 1$ and $\bar{\lambda}(\mu) = a - b\mu$ and $\bar{\mu}(\lambda) = a - b\lambda$ for some positive constants $a$ and $b$, the game $\G$ corresponds exactly to the Cournot Duopoly model. The key distinction in ours is that $\bar{\mu}$ and $\bar{\lambda}$ are generic increasing functions.} 
That is, 
we can interpret $\bar{\lambda}(\cdot)$ as the marginal gain from increasing the size of a viewer or provider group by one unit. Consequently, the first terms $\lambda_k \cdot \bar{\lambda}_k(\cdot)$ and $\lambda_l \cdot \bar{\lambda}_l(\cdot)$ represent the collective payoffs for viewer and provider groups of sizes $\lambda_k$ and $\lambda_l$. 
The quadratic terms $-\frac{\lambda_k^2}{2}$ and $-\frac{\lambda_l^2}{2}$ capture the congestion costs associated with maintaining larger populations (e.g., if a provider group becomes too large, providers within the group may face intensified competition and thus reduce their productivity due to diminished marginal gains). This suggests that Eqs.~\eqref{eq:user_dynamics} and \eqref{eq:content_dynamics} are quite reasonable formulation to capture real-world interactions.

\begin{figure*}[ht]
    \centering
    \includegraphics[width=0.98\linewidth]{Figures/variance.png}
    \caption{Change in system utility and fairness as $\beta$ is increased, with $\beta=0$ at the top left $\beta=1$ at the bottom-right. For all domains, we can see that split and joint optimization perform similarly, while learning only fairness can sometimes be slightly worse. All our methods Pareto-dominate SOTO and FEN. Each point depicts the average performance over five different models trained at that $\beta$ value, and the lines show the Pareto front for each method.}
    \label{fig:main_results}
\end{figure*}

\section{Experimental Setup}

We conduct experiments for maximizing the objective in Eq.~\ref{eq:objective}, where the system utility is the sum of all agent utilities at the end of an episode, and the fairness is measured as the negative of the variance of agent resources at the end of the episode. We perform experiments for a variety of $\beta$ values, repeating each configuration 5 times for each of our three settings: \textbf{Joint Optimization (JO)}, \textbf{Split Optimization (SO)} and \textbf{Fair-Only Optimization (FO)}. 
% , and the code will be made publicly available after acceptance 
We were unable to use off-the-shelf multi-agent RL libraries because of their lack of support for constrained central decision making. Thus, we also implemented our own versions of the learning algorithm (DDQN with $\epsilon$-greedy TD(0) learning), as described in Section~\ref{sec:fairness_in_decas}.
Each model uses the same network architecture, with two hidden layers of dimension 20, and  output of dimension 1.
The utility model used for FO is randomly selected from the JO models trained with $\beta=0$. We included features indicating the relative advantage of each agent as a signal for fairness, in addition to the features describing the local observation of each agent.

\subsection{Environments}
We adapt the environments from \citet{jiang2019FEN} to align with the DECA framework, reformulating them as resource allocation problems with constraints. Additionally, we introduce a new environment, BiasedDM, featuring a biased decision-maker with differing utilities for agents. Below is a description of each environment.

\noindent \textbf{Matthew:} This environment showcases the Matthew effect~\citep{rigney2010matthew, gao2023matthew}, where the rich get richer. Ten agents move on a continuous unit grid with three resources available at a time. Consuming resources grants agents speed and size boosts, allowing faster access to future resources. Some agents start with inherent advantages. Actions involve assigning resources to agents, preventing others from accessing them, or taking a null action to move randomly. Agents provide utility estimates for each action, and the decision-maker allocates resources, ensuring no two agents share the same resource. Agents always move in straight lines toward their targets and are unable to target new resources while moving to collect a previously allocated resource.%, abstracting motion planning.

\noindent \textbf{Job:} Four agents operate on a discretized grid with a fixed square containing a job. Agents receive rewards for occupying the job's location. Grid locations act as resources, with only one agent allowed per location. Agents move in cardinal directions and communicate directional preferences to the decision-maker, who assigns final moves.

\noindent \textbf{JobAlloc:} This simplified version of the Job environment removes the grid. Agents directly compete to occupy the job, with actions limited to occupying or leaving it. The job can only be claimed if it is unoccupied. This domain's challenge is overcoming the single-step suboptimality when no agent occupies the job.

\noindent \textbf{Plant:} Five agents operate on a discretized grid containing eight resources of three types. Agents must collect specific resource combinations to construct a `unit' and earn rewards. Requirements vary in difficulty across agents. The decision-maker assigns resources based on agents' preferences, ensuring exclusivity. Agents deterministically move toward assigned resources.

\noindent \textbf{BiasedDM:} Unlike the other environments, this environment introduces an explicit bias in decision-making. Five agents compete for a single resource per timestep, where utility to the decision-maker increases with agent index ($0.2 \times i$ for agent $i$). Optimal utility is achieved by always allocating resources to agent 5. Fairness is assessed based on resource distribution over time, highlighting a disconnect between fairness and utility.



\subsection{Baselines}

As we described in our Related Work section, FEN~\cite{jiang2019FEN} and SOTO~\cite{zimmer2021MOMDP} are two most relevant approaches that are generalizable to different domains. We thus compare against them in our experimental evaluations. 
However, they both operate in environments where agents can independently take actions without explicitly accounting for resource constraints, making them incompatible with the DECA framework. We attempt two methods of making constrained decisions with per-agent policies to adapt them:
\squishlist
    \item \textbf{Policy as Q-values:} We treat the action probabilities as Q-values and use them for the central allocation. This is denoted by the \textbf{`\_ILP'} suffix in the experiments.
    \item \textbf{Masked sequential action selection:} We go through agents sequentially and let them sample an action from their policy and assign it to them. Any invalid actions are masked as the resources get consumed. We randomize the order of agents every step to prevent ordering bias. This is denoted by the \textbf{`\_Mask'} suffix in the experiments.
\squishend
We add the extra features that SOTO requires (only for SOTO), and train SOTO with both the $\alpha$-fair and $GGF$ objective described in their paper~\cite{zimmer2021MOMDP}. We also use shared weights across agents in our experiments. 




% Needs the following package: \usepackage{multirow}
\begin{table*}[t]
\centering
\caption{Evaluation of all models on multiple metrics for each environment. For JO, SO, and FO, the values in the bracket denote the $\beta$ value selected based on the model that maximizes 0.1$\cdot U$ - 0.9$\cdot \var(\textbf{Z})$. The selected $\beta$ value is indicated in brackets. The values in bold are the best in each row.}
\label{tab:BaselineResults}
\resizebox{\textwidth}{!}{
\begin{tabular}{clcccccc}
\hline
\multicolumn{1}{l}{\textbf{Environment}} & \textbf{Metric} & \textbf{JO($\beta$)} & \textbf{SO($\beta$)}  & \textbf{FO($\beta$)} & \textbf{FEN}   & \textbf{SOTO($\alpha$-Fair)} & \textbf{SOTO(GGF)} \\
\hline
\multirow{5}{*}{BiasedDM}                & Alpha Fair      & \textbf{-8.09(1)}    & -8.3(0.999)           & -8.19(0.9995)        & -8.15          & -8.15                        & -8.15              \\
                                         & GGF             & \textbf{0.35(1)}     & 0.3(0.999)            & 0.33(0.9995)         & 0.33           & 0.33                         & 0.33               \\
                                         & Maximin         & \textbf{0.16(1)}     & 0.12(0.999)           & 0.15(0.9995)         & 0.15           & 0.15                         & 0.15               \\
                                         & System Utility  & 58.31(1)             & \textbf{63.65(0.999)} & 63.38(0.9995)        & 59.95          & 60.22                        & 60.24              \\
                                         & Variance        & \textbf{-0.0007(1)}  & -0.0033(0.999)        & -0.0022(0.9995)      & -0.0014        & -0.0015                      & -0.0015            \\
\hline
\multirow{5}{*}{JobAlloc}                & Alpha Fair      & \textbf{12.71(0.2)}  & 12.69(0.2)            & \textbf{12.71(0.2)}  & -35.31         & -7.39                        & 1.06               \\
                                         & GGF             & 43.11(0.2)           & 43.41(0.2)            & \textbf{43.8(0.2)}   & 12.59          & 19.5                         & 29.62              \\
                                         & Maximin         & 21.79(0.2)           & 22.21(0.2)            & \textbf{22.71(0.2)}  & 0              & 3.49                         & 10.62              \\
                                         & System Utility  & 96.28(0.2)           & 95.81(0.2)            & 96(0.2)              & \textbf{99.89} & 94.42                        & 90.92              \\
                                         & Variance        & -4.44(0.2)           & -2.54(0.2)            & -1.48(0.2)           & -1839.76       & -923.46                      & -421.93            \\
\hline
\multirow{5}{*}{Job}                & Alpha Fair      & 10.97(0.2)           & \textbf{12.07(0.2)}   & 10.77(0.2)           & -35.89         & -55.21                       & 5.03               \\
                                         & GGF             & 37.12(0.2)           & \textbf{37.88(0.2)}   & 26.65(0.2)           & 11.87          & 0                            & 22.17              \\
                                         & Maximin         & 16.99(0.2)           & \textbf{18.13(0.2)}   & 13.13(0.2)           & 0              & 0                            & 4.9                \\
                                         & System Utility  & 88.43(0.2)           & 88.63(0.2)            & 61.68(0.2)           & \textbf{94.88} & 0                            & 80.57              \\
                                         & Variance        & -25.14(0.2)          & -13.38(0.2)           & -4.59(0.2)           & -1683.49       & \textbf{0}                   & -242.3             \\
\hline
\multirow{5}{*}{Plant}                   & Alpha Fair      & \textbf{15(0.8)}     & 14.77(0.8)            & 14.54(0.8)           & -35.48         & -20.75                       & -20.62             \\
                                         & GGF             & \textbf{35.67(0.8)}  & 34.63(0.8)            & 33.45(0.8)           & 1.4            & 10.28                        & 10.89              \\
                                         & Maximin         & \textbf{16.61(0.8)}  & 15.98(0.8)            & 15.74(0.8)           & 0              & 3.62                         & 4.08               \\
                                         & System Utility  & \textbf{101.72(0.8)} & 99.38(0.8)            & 94.89(0.8)           & 13.31          & 42.57                        & 43.84              \\
                                         & Variance        & -6.34(0.8)           & -6.75(0.8)            & \textbf{-4.99(0.8)}  & -41.64         & -54.29                       & -49.91             \\
\hline
\multirow{5}{*}{Matthew}                 & Alpha Fair      & 20.03(0.2)           & \textbf{23.4(0.5)}    & 20.71(0.5)           & -29.04         & 12.45                        & 12.64              \\
                                         & GGF             & 14.41(0.2)           & \textbf{19(0.5)}      & 11.92(0.5)           & 1.73           & 4.6                          & 4.67               \\
                                         & Maximin         & 3.64(0.2)            & \textbf{8.86(0.5)}    & 5.09(0.5)            & 0.36           & 1.66                         & 1.69               \\
                                         & System Utility  & \textbf{140(0.2)}    & 108.1(0.5)            & 85.5(0.5)            & 47.39          & 42.77                        & 43.02              \\
                                         & Variance        & -26.95(0.2)          & \textbf{-1.28(0.5)}   & -5.17(0.5)           & -45.9          & -3.33                        & -3.32      \\       
\hline
\end{tabular}
}
\end{table*}
\section{Results}

Figure \ref{fig:main_results} shows the performance of all three DECAF methods (JO, SO, FO) on the five domains discussed above. For each method, we varied the hyperparameter $\beta$ controlling the fairness-utility trade-off (Eqs.~\ref{eq:JO}, \ref{eq:SO}, \ref{eq:FO}), starting with $\beta=0$ (top-left) and increasing to $\beta=1$ (bottom-right).  As mentioned in Section \ref{sec:fairness_in_decas}, we present results where we optimize for variance as the fairness function here. Additional results on learning with different fairness functions are included in the supplement.

\subsection{Efficacy of the Fairness-Utility Optimization}
For all domains, all three methods are able to learn expressive policies which lie at various points close to the Pareto front. This shows that the optimization allows the model to trade off utility and fairness to show diverse behaviors as required by the user. This also confirms that the fairness reward proposed for minimizing variance is a good signal.

\subsection{Comparison Against Baselines}

As seen in Figure~\ref{fig:main_results}, our methods Pareto-dominate both FEN and SOTO in all experiments, with SOTO\_Mask with GGF being the most competitive. 
Since the baselines were not trained on variance, we also compare the performance of DECAF on other metrics of interest.
For a more granular comparison, we select one $\beta$ value for JO, SO, and FO each, and compare it to the other methods across a variety of metrics. 
Table~\ref{tab:BaselineResults} shows the results for all domains, where we see that our methods provide better results all across the board.
Between the masked and ILP versions of the baselines, the masked versions perform better. This is to be expected, as using the ILP to select the best actions results in trajectories that are not on-policy for each agent, which breaks the requirements for policy gradient methods. The masked approach, on the other hand, cannot benefit from the centralized decision-making, and the random order of agents can lead to suboptimal behavior and missed opportunities. Further, it is difficult to extend policy gradient methods to variable or combinatorial action spaces, while our approach allows for arbitrary action spaces, as long as reasonable post-decision states can be approximated. Thus, in the DECA setting, Q-learning based methods like DECAF have the upper edge.


\begin{figure}[t]
    \centering
    \subfloat[System Utility]{\includegraphics[width=0.475\linewidth]{Figures/Generalization/Matthew/Split_system_utility.png}
    \label{fig:matthew-a}}
    \subfloat[Variance]{\includegraphics[width=0.475\linewidth]{Figures/Generalization/Matthew/Split_fairness.png}
    \label{fig:matthew-b}}
    \caption{Evaluation of SO models trained on $\beta_{train}$ and evaluated on $\beta_{test}$ for the Matthew environment. Brighter colors indicate better outcomes.}
    \label{fig:matthew}
\end{figure}

\begin{figure}[t]
    \centering
    \subfloat[System Utility]{    
    \includegraphics[width=0.475\linewidth]{Figures/Generalization/Matthew/SplitNoUtility_system_utility.png}
    \label{fig:matthew_FO-a}}
    \subfloat[Variance]{    \includegraphics[width=0.475\linewidth]{Figures/Generalization/Matthew/SplitNoUtility_fairness.png}
    \label{fig:matthew_FO-b}}
    \caption{Evaluation of FO models trained on $\beta_{train}$ and evaluated on $\beta_{test}$ for the Matthew environment. Brighter colors indicate better outcomes.}
    \label{fig:matthew_FO}
\end{figure}
\begin{figure}[t]
    \centering
    \subfloat[Generalization of SO]{
    \includegraphics[width=0.45\linewidth]{Figures/Generalization/Approx_pareto/Matthew/Split_ApproxPareto.png}
    \label{fig:matthew-apprx-Pareto-SO}}
    \subfloat[Generalization of FO]{    \includegraphics[width=0.45\linewidth]{Figures/Generalization/Approx_pareto/Matthew/SplitNoUtility_ApproxPareto.png}
    \label{fig:matthew-apprx-Pareto-FO}}
    \caption{Approximated Pareto fronts using sparse $\beta_{train}$ evaluated on other $\beta$ values for the Matthew domain.}
    \label{fig:matthew_apprx_Pareto}
\end{figure}

\subsection{Comparison of DECAF Methods}
In our results, JO and SO generally exhibit similar performance characteristics, suggesting that simultaneous evolution of utility and fairness estimates is beneficial. FO is also very competetive, but in some cases, it falls below the Pareto front.
This underperformance is likely due to out-of-distribution transitions for the fixed utility model, which are more problematic in FO when a large fairness weight $\beta$ is used, causing a larger shift in the state distribution and resulting in degraded utility estimates. We expect this to be a bigger issue in more complicated environments, or with poor black-box utility functions.

\subsection{Generalization Using Split Optimization (SO)}
Figure~\ref{fig:matthew} shows detailed results for the Matthew environment, when using SO. We evaluate each model trained on a particular $\beta_{train}$ on all other $\beta_{test}$. This allows us to see how well the trained fairness and utility models generalize when the operating $\beta$ is changed. Note that for all these models, $\beta$ is not provided as a feature to the Q-function. 

The diagonal elements show the behavior when training and testing is done on the same $\beta$ value.  
From the plots for system utility (Figure \ref{fig:matthew-a}) and variance (Figure \ref{fig:matthew-b}), we can see that as $\beta_{test}$ increases, variance improves, and as $\beta_{test}$ decreases, utility improves. This is the expected behavior, and the major advantage of SO over JO. With JO, the model only predicts a single value, so we are unable to change the trade-off weight during evaluation, and we require a unique model for each $\beta$ that we want the model to work for. However, with SO, selecting just a few spread out $\beta$ values can allow us to extrapolate between them, providing online adaptability.
This shows that SO has the flexibility to function well at operating points away from the $\beta_{train}$ that it is trained for.

Figure~\ref{fig:matthew-apprx-Pareto-SO} shows how well a few selected models can generalize to the Pareto front. We pick $\beta_{train}$ values evenly spaced across the search space, and evaluate the model on all $\beta_{test}$, picking the closest $\beta_{train}$ in order of the search space. We can see that the approximated Pareto front closely matches the actual Pareto front, even with just 3 models, further demonstrating the strength of SO.
These observations hold for other domains as well, and the results are included in the Supplement.

\subsection{Effectiveness of Fair-Only Optimization (FO)}
Like SO, FO is also able to generalize well when different $\beta_{test}$ are used to evaluate the learned models (Figure~\ref{fig:matthew_FO}). Because the utility model is fixed, all models achieve high utility as $\beta_{test}\xrightarrow{}0$ (Figure~\ref{fig:matthew_FO-a}). Further, all models also improve fairness as $\beta_{test}$ grows larger (Figure~\ref{fig:matthew_FO-b}). The behavior change from utility-oriented to fairness-oriented is much sharper in FO when compared to SO. 
Looking at Figure~\ref{fig:matthew-apprx-Pareto-FO}, we can again see that even FO has the ability to generalize from only a few models to cover the entire Pareto front. 
Despite being Pareto-dominated by SO and JO at intermediate $\beta$ values in some domains, FO has the advantage of reliability: A trusted black-box utility model can be used in conjunction with a possibly smaller fairness model, with the guarantee to behave optimally as $\beta_{test}$ is reduced. When such a model is available, FO is the best choice, given its competent performance and lower computational load.

\section{Conclusions and Future Work}
We proposed DECAF, a framework for learning long-term utility and fairness estimates in multi-agent resource allocation. DECAF is among the first approaches to optimize fair resource allocation under resource constraints, supporting diverse problem settings by decoupling fairness and utility metrics. Split and Fair-Only optimization enable online trade-offs between utility and fairness without retraining, enhancing interpretability. Our results demonstrate the flexibility and effectiveness of our approaches across various scenarios.
 
Our framework currently relies on Q-Learning, as deriving a policy gradient approach for DECA problems is challenging due to the dynamic state-action space and the indirect relationship between agent `policies' and actions resulting from ILP optimization. Addressing this challenge is a promising direction for future research.
Finally, our methods are not the only way to decompose the fairness reward across agents. Techniques like VDN~\citep{sunehag2017VDN} or QMIX~\citep{rashid2020QMIX} could be integrated with our framework to learn credit assignment for fair rewards.


\bibliography{main}
\bibliographystyle{icml2025}


%%%%%%%%%%%%%%%%%%%%%%%%%%%%%%%%%%%%%%%%%%%%%%%%%%%%%%%%%%%%%%%%%%%%%%%%%%%%%%%
%%%%%%%%%%%%%%%%%%%%%%%%%%%%%%%%%%%%%%%%%%%%%%%%%%%%%%%%%%%%%%%%%%%%%%%%%%%%%%%
% APPENDIX
%%%%%%%%%%%%%%%%%%%%%%%%%%%%%%%%%%%%%%%%%%%%%%%%%%%%%%%%%%%%%%%%%%%%%%%%%%%%%%%
%%%%%%%%%%%%%%%%%%%%%%%%%%%%%%%%%%%%%%%%%%%%%%%%%%%%%%%%%%%%%%%%%%%%%%%%%%%%%%%
\newpage
\appendix
% \onecolumn
% \section{You \emph{can} have an appendix here.}

% You can have as much text here as you want. The main body must be at most $8$ pages long.
% For the final version, one more page can be added.
% If you want, you can use an appendix like this one.  

% The $\mathtt{\backslash onecolumn}$ command above can be kept in place if you prefer a one-column appendix, or can be removed if you prefer a two-column appendix.  Apart from this possible change, the style (font size, spacing, margins, page numbering, etc.) should be kept the same as the main body.

\subsection{Lloyd-Max Algorithm}
\label{subsec:Lloyd-Max}
For a given quantization bitwidth $B$ and an operand $\bm{X}$, the Lloyd-Max algorithm finds $2^B$ quantization levels $\{\hat{x}_i\}_{i=1}^{2^B}$ such that quantizing $\bm{X}$ by rounding each scalar in $\bm{X}$ to the nearest quantization level minimizes the quantization MSE. 

The algorithm starts with an initial guess of quantization levels and then iteratively computes quantization thresholds $\{\tau_i\}_{i=1}^{2^B-1}$ and updates quantization levels $\{\hat{x}_i\}_{i=1}^{2^B}$. Specifically, at iteration $n$, thresholds are set to the midpoints of the previous iteration's levels:
\begin{align*}
    \tau_i^{(n)}=\frac{\hat{x}_i^{(n-1)}+\hat{x}_{i+1}^{(n-1)}}2 \text{ for } i=1\ldots 2^B-1
\end{align*}
Subsequently, the quantization levels are re-computed as conditional means of the data regions defined by the new thresholds:
\begin{align*}
    \hat{x}_i^{(n)}=\mathbb{E}\left[ \bm{X} \big| \bm{X}\in [\tau_{i-1}^{(n)},\tau_i^{(n)}] \right] \text{ for } i=1\ldots 2^B
\end{align*}
where to satisfy boundary conditions we have $\tau_0=-\infty$ and $\tau_{2^B}=\infty$. The algorithm iterates the above steps until convergence.

Figure \ref{fig:lm_quant} compares the quantization levels of a $7$-bit floating point (E3M3) quantizer (left) to a $7$-bit Lloyd-Max quantizer (right) when quantizing a layer of weights from the GPT3-126M model at a per-tensor granularity. As shown, the Lloyd-Max quantizer achieves substantially lower quantization MSE. Further, Table \ref{tab:FP7_vs_LM7} shows the superior perplexity achieved by Lloyd-Max quantizers for bitwidths of $7$, $6$ and $5$. The difference between the quantizers is clear at 5 bits, where per-tensor FP quantization incurs a drastic and unacceptable increase in perplexity, while Lloyd-Max quantization incurs a much smaller increase. Nevertheless, we note that even the optimal Lloyd-Max quantizer incurs a notable ($\sim 1.5$) increase in perplexity due to the coarse granularity of quantization. 

\begin{figure}[h]
  \centering
  \includegraphics[width=0.7\linewidth]{sections/figures/LM7_FP7.pdf}
  \caption{\small Quantization levels and the corresponding quantization MSE of Floating Point (left) vs Lloyd-Max (right) Quantizers for a layer of weights in the GPT3-126M model.}
  \label{fig:lm_quant}
\end{figure}

\begin{table}[h]\scriptsize
\begin{center}
\caption{\label{tab:FP7_vs_LM7} \small Comparing perplexity (lower is better) achieved by floating point quantizers and Lloyd-Max quantizers on a GPT3-126M model for the Wikitext-103 dataset.}
\begin{tabular}{c|cc|c}
\hline
 \multirow{2}{*}{\textbf{Bitwidth}} & \multicolumn{2}{|c|}{\textbf{Floating-Point Quantizer}} & \textbf{Lloyd-Max Quantizer} \\
 & Best Format & Wikitext-103 Perplexity & Wikitext-103 Perplexity \\
\hline
7 & E3M3 & 18.32 & 18.27 \\
6 & E3M2 & 19.07 & 18.51 \\
5 & E4M0 & 43.89 & 19.71 \\
\hline
\end{tabular}
\end{center}
\end{table}

\subsection{Proof of Local Optimality of LO-BCQ}
\label{subsec:lobcq_opt_proof}
For a given block $\bm{b}_j$, the quantization MSE during LO-BCQ can be empirically evaluated as $\frac{1}{L_b}\lVert \bm{b}_j- \bm{\hat{b}}_j\rVert^2_2$ where $\bm{\hat{b}}_j$ is computed from equation (\ref{eq:clustered_quantization_definition}) as $C_{f(\bm{b}_j)}(\bm{b}_j)$. Further, for a given block cluster $\mathcal{B}_i$, we compute the quantization MSE as $\frac{1}{|\mathcal{B}_{i}|}\sum_{\bm{b} \in \mathcal{B}_{i}} \frac{1}{L_b}\lVert \bm{b}- C_i^{(n)}(\bm{b})\rVert^2_2$. Therefore, at the end of iteration $n$, we evaluate the overall quantization MSE $J^{(n)}$ for a given operand $\bm{X}$ composed of $N_c$ block clusters as:
\begin{align*}
    \label{eq:mse_iter_n}
    J^{(n)} = \frac{1}{N_c} \sum_{i=1}^{N_c} \frac{1}{|\mathcal{B}_{i}^{(n)}|}\sum_{\bm{v} \in \mathcal{B}_{i}^{(n)}} \frac{1}{L_b}\lVert \bm{b}- B_i^{(n)}(\bm{b})\rVert^2_2
\end{align*}

At the end of iteration $n$, the codebooks are updated from $\mathcal{C}^{(n-1)}$ to $\mathcal{C}^{(n)}$. However, the mapping of a given vector $\bm{b}_j$ to quantizers $\mathcal{C}^{(n)}$ remains as  $f^{(n)}(\bm{b}_j)$. At the next iteration, during the vector clustering step, $f^{(n+1)}(\bm{b}_j)$ finds new mapping of $\bm{b}_j$ to updated codebooks $\mathcal{C}^{(n)}$ such that the quantization MSE over the candidate codebooks is minimized. Therefore, we obtain the following result for $\bm{b}_j$:
\begin{align*}
\frac{1}{L_b}\lVert \bm{b}_j - C_{f^{(n+1)}(\bm{b}_j)}^{(n)}(\bm{b}_j)\rVert^2_2 \le \frac{1}{L_b}\lVert \bm{b}_j - C_{f^{(n)}(\bm{b}_j)}^{(n)}(\bm{b}_j)\rVert^2_2
\end{align*}

That is, quantizing $\bm{b}_j$ at the end of the block clustering step of iteration $n+1$ results in lower quantization MSE compared to quantizing at the end of iteration $n$. Since this is true for all $\bm{b} \in \bm{X}$, we assert the following:
\begin{equation}
\begin{split}
\label{eq:mse_ineq_1}
    \tilde{J}^{(n+1)} &= \frac{1}{N_c} \sum_{i=1}^{N_c} \frac{1}{|\mathcal{B}_{i}^{(n+1)}|}\sum_{\bm{b} \in \mathcal{B}_{i}^{(n+1)}} \frac{1}{L_b}\lVert \bm{b} - C_i^{(n)}(b)\rVert^2_2 \le J^{(n)}
\end{split}
\end{equation}
where $\tilde{J}^{(n+1)}$ is the the quantization MSE after the vector clustering step at iteration $n+1$.

Next, during the codebook update step (\ref{eq:quantizers_update}) at iteration $n+1$, the per-cluster codebooks $\mathcal{C}^{(n)}$ are updated to $\mathcal{C}^{(n+1)}$ by invoking the Lloyd-Max algorithm \citep{Lloyd}. We know that for any given value distribution, the Lloyd-Max algorithm minimizes the quantization MSE. Therefore, for a given vector cluster $\mathcal{B}_i$ we obtain the following result:

\begin{equation}
    \frac{1}{|\mathcal{B}_{i}^{(n+1)}|}\sum_{\bm{b} \in \mathcal{B}_{i}^{(n+1)}} \frac{1}{L_b}\lVert \bm{b}- C_i^{(n+1)}(\bm{b})\rVert^2_2 \le \frac{1}{|\mathcal{B}_{i}^{(n+1)}|}\sum_{\bm{b} \in \mathcal{B}_{i}^{(n+1)}} \frac{1}{L_b}\lVert \bm{b}- C_i^{(n)}(\bm{b})\rVert^2_2
\end{equation}

The above equation states that quantizing the given block cluster $\mathcal{B}_i$ after updating the associated codebook from $C_i^{(n)}$ to $C_i^{(n+1)}$ results in lower quantization MSE. Since this is true for all the block clusters, we derive the following result: 
\begin{equation}
\begin{split}
\label{eq:mse_ineq_2}
     J^{(n+1)} &= \frac{1}{N_c} \sum_{i=1}^{N_c} \frac{1}{|\mathcal{B}_{i}^{(n+1)}|}\sum_{\bm{b} \in \mathcal{B}_{i}^{(n+1)}} \frac{1}{L_b}\lVert \bm{b}- C_i^{(n+1)}(\bm{b})\rVert^2_2  \le \tilde{J}^{(n+1)}   
\end{split}
\end{equation}

Following (\ref{eq:mse_ineq_1}) and (\ref{eq:mse_ineq_2}), we find that the quantization MSE is non-increasing for each iteration, that is, $J^{(1)} \ge J^{(2)} \ge J^{(3)} \ge \ldots \ge J^{(M)}$ where $M$ is the maximum number of iterations. 
%Therefore, we can say that if the algorithm converges, then it must be that it has converged to a local minimum. 
\hfill $\blacksquare$


\begin{figure}
    \begin{center}
    \includegraphics[width=0.5\textwidth]{sections//figures/mse_vs_iter.pdf}
    \end{center}
    \caption{\small NMSE vs iterations during LO-BCQ compared to other block quantization proposals}
    \label{fig:nmse_vs_iter}
\end{figure}

Figure \ref{fig:nmse_vs_iter} shows the empirical convergence of LO-BCQ across several block lengths and number of codebooks. Also, the MSE achieved by LO-BCQ is compared to baselines such as MXFP and VSQ. As shown, LO-BCQ converges to a lower MSE than the baselines. Further, we achieve better convergence for larger number of codebooks ($N_c$) and for a smaller block length ($L_b$), both of which increase the bitwidth of BCQ (see Eq \ref{eq:bitwidth_bcq}).


\subsection{Additional Accuracy Results}
%Table \ref{tab:lobcq_config} lists the various LOBCQ configurations and their corresponding bitwidths.
\begin{table}
\setlength{\tabcolsep}{4.75pt}
\begin{center}
\caption{\label{tab:lobcq_config} Various LO-BCQ configurations and their bitwidths.}
\begin{tabular}{|c||c|c|c|c||c|c||c|} 
\hline
 & \multicolumn{4}{|c||}{$L_b=8$} & \multicolumn{2}{|c||}{$L_b=4$} & $L_b=2$ \\
 \hline
 \backslashbox{$L_A$\kern-1em}{\kern-1em$N_c$} & 2 & 4 & 8 & 16 & 2 & 4 & 2 \\
 \hline
 64 & 4.25 & 4.375 & 4.5 & 4.625 & 4.375 & 4.625 & 4.625\\
 \hline
 32 & 4.375 & 4.5 & 4.625& 4.75 & 4.5 & 4.75 & 4.75 \\
 \hline
 16 & 4.625 & 4.75& 4.875 & 5 & 4.75 & 5 & 5 \\
 \hline
\end{tabular}
\end{center}
\end{table}

%\subsection{Perplexity achieved by various LO-BCQ configurations on Wikitext-103 dataset}

\begin{table} \centering
\begin{tabular}{|c||c|c|c|c||c|c||c|} 
\hline
 $L_b \rightarrow$& \multicolumn{4}{c||}{8} & \multicolumn{2}{c||}{4} & 2\\
 \hline
 \backslashbox{$L_A$\kern-1em}{\kern-1em$N_c$} & 2 & 4 & 8 & 16 & 2 & 4 & 2  \\
 %$N_c \rightarrow$ & 2 & 4 & 8 & 16 & 2 & 4 & 2 \\
 \hline
 \hline
 \multicolumn{8}{c}{GPT3-1.3B (FP32 PPL = 9.98)} \\ 
 \hline
 \hline
 64 & 10.40 & 10.23 & 10.17 & 10.15 &  10.28 & 10.18 & 10.19 \\
 \hline
 32 & 10.25 & 10.20 & 10.15 & 10.12 &  10.23 & 10.17 & 10.17 \\
 \hline
 16 & 10.22 & 10.16 & 10.10 & 10.09 &  10.21 & 10.14 & 10.16 \\
 \hline
  \hline
 \multicolumn{8}{c}{GPT3-8B (FP32 PPL = 7.38)} \\ 
 \hline
 \hline
 64 & 7.61 & 7.52 & 7.48 &  7.47 &  7.55 &  7.49 & 7.50 \\
 \hline
 32 & 7.52 & 7.50 & 7.46 &  7.45 &  7.52 &  7.48 & 7.48  \\
 \hline
 16 & 7.51 & 7.48 & 7.44 &  7.44 &  7.51 &  7.49 & 7.47  \\
 \hline
\end{tabular}
\caption{\label{tab:ppl_gpt3_abalation} Wikitext-103 perplexity across GPT3-1.3B and 8B models.}
\end{table}

\begin{table} \centering
\begin{tabular}{|c||c|c|c|c||} 
\hline
 $L_b \rightarrow$& \multicolumn{4}{c||}{8}\\
 \hline
 \backslashbox{$L_A$\kern-1em}{\kern-1em$N_c$} & 2 & 4 & 8 & 16 \\
 %$N_c \rightarrow$ & 2 & 4 & 8 & 16 & 2 & 4 & 2 \\
 \hline
 \hline
 \multicolumn{5}{|c|}{Llama2-7B (FP32 PPL = 5.06)} \\ 
 \hline
 \hline
 64 & 5.31 & 5.26 & 5.19 & 5.18  \\
 \hline
 32 & 5.23 & 5.25 & 5.18 & 5.15  \\
 \hline
 16 & 5.23 & 5.19 & 5.16 & 5.14  \\
 \hline
 \multicolumn{5}{|c|}{Nemotron4-15B (FP32 PPL = 5.87)} \\ 
 \hline
 \hline
 64  & 6.3 & 6.20 & 6.13 & 6.08  \\
 \hline
 32  & 6.24 & 6.12 & 6.07 & 6.03  \\
 \hline
 16  & 6.12 & 6.14 & 6.04 & 6.02  \\
 \hline
 \multicolumn{5}{|c|}{Nemotron4-340B (FP32 PPL = 3.48)} \\ 
 \hline
 \hline
 64 & 3.67 & 3.62 & 3.60 & 3.59 \\
 \hline
 32 & 3.63 & 3.61 & 3.59 & 3.56 \\
 \hline
 16 & 3.61 & 3.58 & 3.57 & 3.55 \\
 \hline
\end{tabular}
\caption{\label{tab:ppl_llama7B_nemo15B} Wikitext-103 perplexity compared to FP32 baseline in Llama2-7B and Nemotron4-15B, 340B models}
\end{table}

%\subsection{Perplexity achieved by various LO-BCQ configurations on MMLU dataset}


\begin{table} \centering
\begin{tabular}{|c||c|c|c|c||c|c|c|c|} 
\hline
 $L_b \rightarrow$& \multicolumn{4}{c||}{8} & \multicolumn{4}{c||}{8}\\
 \hline
 \backslashbox{$L_A$\kern-1em}{\kern-1em$N_c$} & 2 & 4 & 8 & 16 & 2 & 4 & 8 & 16  \\
 %$N_c \rightarrow$ & 2 & 4 & 8 & 16 & 2 & 4 & 2 \\
 \hline
 \hline
 \multicolumn{5}{|c|}{Llama2-7B (FP32 Accuracy = 45.8\%)} & \multicolumn{4}{|c|}{Llama2-70B (FP32 Accuracy = 69.12\%)} \\ 
 \hline
 \hline
 64 & 43.9 & 43.4 & 43.9 & 44.9 & 68.07 & 68.27 & 68.17 & 68.75 \\
 \hline
 32 & 44.5 & 43.8 & 44.9 & 44.5 & 68.37 & 68.51 & 68.35 & 68.27  \\
 \hline
 16 & 43.9 & 42.7 & 44.9 & 45 & 68.12 & 68.77 & 68.31 & 68.59  \\
 \hline
 \hline
 \multicolumn{5}{|c|}{GPT3-22B (FP32 Accuracy = 38.75\%)} & \multicolumn{4}{|c|}{Nemotron4-15B (FP32 Accuracy = 64.3\%)} \\ 
 \hline
 \hline
 64 & 36.71 & 38.85 & 38.13 & 38.92 & 63.17 & 62.36 & 63.72 & 64.09 \\
 \hline
 32 & 37.95 & 38.69 & 39.45 & 38.34 & 64.05 & 62.30 & 63.8 & 64.33  \\
 \hline
 16 & 38.88 & 38.80 & 38.31 & 38.92 & 63.22 & 63.51 & 63.93 & 64.43  \\
 \hline
\end{tabular}
\caption{\label{tab:mmlu_abalation} Accuracy on MMLU dataset across GPT3-22B, Llama2-7B, 70B and Nemotron4-15B models.}
\end{table}


%\subsection{Perplexity achieved by various LO-BCQ configurations on LM evaluation harness}

\begin{table} \centering
\begin{tabular}{|c||c|c|c|c||c|c|c|c|} 
\hline
 $L_b \rightarrow$& \multicolumn{4}{c||}{8} & \multicolumn{4}{c||}{8}\\
 \hline
 \backslashbox{$L_A$\kern-1em}{\kern-1em$N_c$} & 2 & 4 & 8 & 16 & 2 & 4 & 8 & 16  \\
 %$N_c \rightarrow$ & 2 & 4 & 8 & 16 & 2 & 4 & 2 \\
 \hline
 \hline
 \multicolumn{5}{|c|}{Race (FP32 Accuracy = 37.51\%)} & \multicolumn{4}{|c|}{Boolq (FP32 Accuracy = 64.62\%)} \\ 
 \hline
 \hline
 64 & 36.94 & 37.13 & 36.27 & 37.13 & 63.73 & 62.26 & 63.49 & 63.36 \\
 \hline
 32 & 37.03 & 36.36 & 36.08 & 37.03 & 62.54 & 63.51 & 63.49 & 63.55  \\
 \hline
 16 & 37.03 & 37.03 & 36.46 & 37.03 & 61.1 & 63.79 & 63.58 & 63.33  \\
 \hline
 \hline
 \multicolumn{5}{|c|}{Winogrande (FP32 Accuracy = 58.01\%)} & \multicolumn{4}{|c|}{Piqa (FP32 Accuracy = 74.21\%)} \\ 
 \hline
 \hline
 64 & 58.17 & 57.22 & 57.85 & 58.33 & 73.01 & 73.07 & 73.07 & 72.80 \\
 \hline
 32 & 59.12 & 58.09 & 57.85 & 58.41 & 73.01 & 73.94 & 72.74 & 73.18  \\
 \hline
 16 & 57.93 & 58.88 & 57.93 & 58.56 & 73.94 & 72.80 & 73.01 & 73.94  \\
 \hline
\end{tabular}
\caption{\label{tab:mmlu_abalation} Accuracy on LM evaluation harness tasks on GPT3-1.3B model.}
\end{table}

\begin{table} \centering
\begin{tabular}{|c||c|c|c|c||c|c|c|c|} 
\hline
 $L_b \rightarrow$& \multicolumn{4}{c||}{8} & \multicolumn{4}{c||}{8}\\
 \hline
 \backslashbox{$L_A$\kern-1em}{\kern-1em$N_c$} & 2 & 4 & 8 & 16 & 2 & 4 & 8 & 16  \\
 %$N_c \rightarrow$ & 2 & 4 & 8 & 16 & 2 & 4 & 2 \\
 \hline
 \hline
 \multicolumn{5}{|c|}{Race (FP32 Accuracy = 41.34\%)} & \multicolumn{4}{|c|}{Boolq (FP32 Accuracy = 68.32\%)} \\ 
 \hline
 \hline
 64 & 40.48 & 40.10 & 39.43 & 39.90 & 69.20 & 68.41 & 69.45 & 68.56 \\
 \hline
 32 & 39.52 & 39.52 & 40.77 & 39.62 & 68.32 & 67.43 & 68.17 & 69.30  \\
 \hline
 16 & 39.81 & 39.71 & 39.90 & 40.38 & 68.10 & 66.33 & 69.51 & 69.42  \\
 \hline
 \hline
 \multicolumn{5}{|c|}{Winogrande (FP32 Accuracy = 67.88\%)} & \multicolumn{4}{|c|}{Piqa (FP32 Accuracy = 78.78\%)} \\ 
 \hline
 \hline
 64 & 66.85 & 66.61 & 67.72 & 67.88 & 77.31 & 77.42 & 77.75 & 77.64 \\
 \hline
 32 & 67.25 & 67.72 & 67.72 & 67.00 & 77.31 & 77.04 & 77.80 & 77.37  \\
 \hline
 16 & 68.11 & 68.90 & 67.88 & 67.48 & 77.37 & 78.13 & 78.13 & 77.69  \\
 \hline
\end{tabular}
\caption{\label{tab:mmlu_abalation} Accuracy on LM evaluation harness tasks on GPT3-8B model.}
\end{table}

\begin{table} \centering
\begin{tabular}{|c||c|c|c|c||c|c|c|c|} 
\hline
 $L_b \rightarrow$& \multicolumn{4}{c||}{8} & \multicolumn{4}{c||}{8}\\
 \hline
 \backslashbox{$L_A$\kern-1em}{\kern-1em$N_c$} & 2 & 4 & 8 & 16 & 2 & 4 & 8 & 16  \\
 %$N_c \rightarrow$ & 2 & 4 & 8 & 16 & 2 & 4 & 2 \\
 \hline
 \hline
 \multicolumn{5}{|c|}{Race (FP32 Accuracy = 40.67\%)} & \multicolumn{4}{|c|}{Boolq (FP32 Accuracy = 76.54\%)} \\ 
 \hline
 \hline
 64 & 40.48 & 40.10 & 39.43 & 39.90 & 75.41 & 75.11 & 77.09 & 75.66 \\
 \hline
 32 & 39.52 & 39.52 & 40.77 & 39.62 & 76.02 & 76.02 & 75.96 & 75.35  \\
 \hline
 16 & 39.81 & 39.71 & 39.90 & 40.38 & 75.05 & 73.82 & 75.72 & 76.09  \\
 \hline
 \hline
 \multicolumn{5}{|c|}{Winogrande (FP32 Accuracy = 70.64\%)} & \multicolumn{4}{|c|}{Piqa (FP32 Accuracy = 79.16\%)} \\ 
 \hline
 \hline
 64 & 69.14 & 70.17 & 70.17 & 70.56 & 78.24 & 79.00 & 78.62 & 78.73 \\
 \hline
 32 & 70.96 & 69.69 & 71.27 & 69.30 & 78.56 & 79.49 & 79.16 & 78.89  \\
 \hline
 16 & 71.03 & 69.53 & 69.69 & 70.40 & 78.13 & 79.16 & 79.00 & 79.00  \\
 \hline
\end{tabular}
\caption{\label{tab:mmlu_abalation} Accuracy on LM evaluation harness tasks on GPT3-22B model.}
\end{table}

\begin{table} \centering
\begin{tabular}{|c||c|c|c|c||c|c|c|c|} 
\hline
 $L_b \rightarrow$& \multicolumn{4}{c||}{8} & \multicolumn{4}{c||}{8}\\
 \hline
 \backslashbox{$L_A$\kern-1em}{\kern-1em$N_c$} & 2 & 4 & 8 & 16 & 2 & 4 & 8 & 16  \\
 %$N_c \rightarrow$ & 2 & 4 & 8 & 16 & 2 & 4 & 2 \\
 \hline
 \hline
 \multicolumn{5}{|c|}{Race (FP32 Accuracy = 44.4\%)} & \multicolumn{4}{|c|}{Boolq (FP32 Accuracy = 79.29\%)} \\ 
 \hline
 \hline
 64 & 42.49 & 42.51 & 42.58 & 43.45 & 77.58 & 77.37 & 77.43 & 78.1 \\
 \hline
 32 & 43.35 & 42.49 & 43.64 & 43.73 & 77.86 & 75.32 & 77.28 & 77.86  \\
 \hline
 16 & 44.21 & 44.21 & 43.64 & 42.97 & 78.65 & 77 & 76.94 & 77.98  \\
 \hline
 \hline
 \multicolumn{5}{|c|}{Winogrande (FP32 Accuracy = 69.38\%)} & \multicolumn{4}{|c|}{Piqa (FP32 Accuracy = 78.07\%)} \\ 
 \hline
 \hline
 64 & 68.9 & 68.43 & 69.77 & 68.19 & 77.09 & 76.82 & 77.09 & 77.86 \\
 \hline
 32 & 69.38 & 68.51 & 68.82 & 68.90 & 78.07 & 76.71 & 78.07 & 77.86  \\
 \hline
 16 & 69.53 & 67.09 & 69.38 & 68.90 & 77.37 & 77.8 & 77.91 & 77.69  \\
 \hline
\end{tabular}
\caption{\label{tab:mmlu_abalation} Accuracy on LM evaluation harness tasks on Llama2-7B model.}
\end{table}

\begin{table} \centering
\begin{tabular}{|c||c|c|c|c||c|c|c|c|} 
\hline
 $L_b \rightarrow$& \multicolumn{4}{c||}{8} & \multicolumn{4}{c||}{8}\\
 \hline
 \backslashbox{$L_A$\kern-1em}{\kern-1em$N_c$} & 2 & 4 & 8 & 16 & 2 & 4 & 8 & 16  \\
 %$N_c \rightarrow$ & 2 & 4 & 8 & 16 & 2 & 4 & 2 \\
 \hline
 \hline
 \multicolumn{5}{|c|}{Race (FP32 Accuracy = 48.8\%)} & \multicolumn{4}{|c|}{Boolq (FP32 Accuracy = 85.23\%)} \\ 
 \hline
 \hline
 64 & 49.00 & 49.00 & 49.28 & 48.71 & 82.82 & 84.28 & 84.03 & 84.25 \\
 \hline
 32 & 49.57 & 48.52 & 48.33 & 49.28 & 83.85 & 84.46 & 84.31 & 84.93  \\
 \hline
 16 & 49.85 & 49.09 & 49.28 & 48.99 & 85.11 & 84.46 & 84.61 & 83.94  \\
 \hline
 \hline
 \multicolumn{5}{|c|}{Winogrande (FP32 Accuracy = 79.95\%)} & \multicolumn{4}{|c|}{Piqa (FP32 Accuracy = 81.56\%)} \\ 
 \hline
 \hline
 64 & 78.77 & 78.45 & 78.37 & 79.16 & 81.45 & 80.69 & 81.45 & 81.5 \\
 \hline
 32 & 78.45 & 79.01 & 78.69 & 80.66 & 81.56 & 80.58 & 81.18 & 81.34  \\
 \hline
 16 & 79.95 & 79.56 & 79.79 & 79.72 & 81.28 & 81.66 & 81.28 & 80.96  \\
 \hline
\end{tabular}
\caption{\label{tab:mmlu_abalation} Accuracy on LM evaluation harness tasks on Llama2-70B model.}
\end{table}

%\section{MSE Studies}
%\textcolor{red}{TODO}


\subsection{Number Formats and Quantization Method}
\label{subsec:numFormats_quantMethod}
\subsubsection{Integer Format}
An $n$-bit signed integer (INT) is typically represented with a 2s-complement format \citep{yao2022zeroquant,xiao2023smoothquant,dai2021vsq}, where the most significant bit denotes the sign.

\subsubsection{Floating Point Format}
An $n$-bit signed floating point (FP) number $x$ comprises of a 1-bit sign ($x_{\mathrm{sign}}$), $B_m$-bit mantissa ($x_{\mathrm{mant}}$) and $B_e$-bit exponent ($x_{\mathrm{exp}}$) such that $B_m+B_e=n-1$. The associated constant exponent bias ($E_{\mathrm{bias}}$) is computed as $(2^{{B_e}-1}-1)$. We denote this format as $E_{B_e}M_{B_m}$.  

\subsubsection{Quantization Scheme}
\label{subsec:quant_method}
A quantization scheme dictates how a given unquantized tensor is converted to its quantized representation. We consider FP formats for the purpose of illustration. Given an unquantized tensor $\bm{X}$ and an FP format $E_{B_e}M_{B_m}$, we first, we compute the quantization scale factor $s_X$ that maps the maximum absolute value of $\bm{X}$ to the maximum quantization level of the $E_{B_e}M_{B_m}$ format as follows:
\begin{align}
\label{eq:sf}
    s_X = \frac{\mathrm{max}(|\bm{X}|)}{\mathrm{max}(E_{B_e}M_{B_m})}
\end{align}
In the above equation, $|\cdot|$ denotes the absolute value function.

Next, we scale $\bm{X}$ by $s_X$ and quantize it to $\hat{\bm{X}}$ by rounding it to the nearest quantization level of $E_{B_e}M_{B_m}$ as:

\begin{align}
\label{eq:tensor_quant}
    \hat{\bm{X}} = \text{round-to-nearest}\left(\frac{\bm{X}}{s_X}, E_{B_e}M_{B_m}\right)
\end{align}

We perform dynamic max-scaled quantization \citep{wu2020integer}, where the scale factor $s$ for activations is dynamically computed during runtime.

\subsection{Vector Scaled Quantization}
\begin{wrapfigure}{r}{0.35\linewidth}
  \centering
  \includegraphics[width=\linewidth]{sections/figures/vsquant.jpg}
  \caption{\small Vectorwise decomposition for per-vector scaled quantization (VSQ \citep{dai2021vsq}).}
  \label{fig:vsquant}
\end{wrapfigure}
During VSQ \citep{dai2021vsq}, the operand tensors are decomposed into 1D vectors in a hardware friendly manner as shown in Figure \ref{fig:vsquant}. Since the decomposed tensors are used as operands in matrix multiplications during inference, it is beneficial to perform this decomposition along the reduction dimension of the multiplication. The vectorwise quantization is performed similar to tensorwise quantization described in Equations \ref{eq:sf} and \ref{eq:tensor_quant}, where a scale factor $s_v$ is required for each vector $\bm{v}$ that maps the maximum absolute value of that vector to the maximum quantization level. While smaller vector lengths can lead to larger accuracy gains, the associated memory and computational overheads due to the per-vector scale factors increases. To alleviate these overheads, VSQ \citep{dai2021vsq} proposed a second level quantization of the per-vector scale factors to unsigned integers, while MX \citep{rouhani2023shared} quantizes them to integer powers of 2 (denoted as $2^{INT}$).

\subsubsection{MX Format}
The MX format proposed in \citep{rouhani2023microscaling} introduces the concept of sub-block shifting. For every two scalar elements of $b$-bits each, there is a shared exponent bit. The value of this exponent bit is determined through an empirical analysis that targets minimizing quantization MSE. We note that the FP format $E_{1}M_{b}$ is strictly better than MX from an accuracy perspective since it allocates a dedicated exponent bit to each scalar as opposed to sharing it across two scalars. Therefore, we conservatively bound the accuracy of a $b+2$-bit signed MX format with that of a $E_{1}M_{b}$ format in our comparisons. For instance, we use E1M2 format as a proxy for MX4.

\begin{figure}
    \centering
    \includegraphics[width=1\linewidth]{sections//figures/BlockFormats.pdf}
    \caption{\small Comparing LO-BCQ to MX format.}
    \label{fig:block_formats}
\end{figure}

Figure \ref{fig:block_formats} compares our $4$-bit LO-BCQ block format to MX \citep{rouhani2023microscaling}. As shown, both LO-BCQ and MX decompose a given operand tensor into block arrays and each block array into blocks. Similar to MX, we find that per-block quantization ($L_b < L_A$) leads to better accuracy due to increased flexibility. While MX achieves this through per-block $1$-bit micro-scales, we associate a dedicated codebook to each block through a per-block codebook selector. Further, MX quantizes the per-block array scale-factor to E8M0 format without per-tensor scaling. In contrast during LO-BCQ, we find that per-tensor scaling combined with quantization of per-block array scale-factor to E4M3 format results in superior inference accuracy across models. 

%%%%%%%%%%%%%%%%%%%%%%%%%%%%%%%%%%%%%%%%%%%%%%%%%%%%%%%%%%%%%%%%%%%%%%%%%%%%%%%
%%%%%%%%%%%%%%%%%%%%%%%%%%%%%%%%%%%%%%%%%%%%%%%%%%%%%%%%%%%%%%%%%%%%%%%%%%%%%%%


\end{document}

