\section{Related Work}
\label{sec:related}

While DECA has not been formalized in prior work, it has seen application in many domains. From optimizing passenger-driver matches in ridesharing____ to efficient allocation of homelessness resources____, many real-world applications follow this general structure. 
It is also analogous to the predict-then-optimize (P+O) approach ____, where a predictive model estimates unknown parameters that are subsequently used in optimization. However, unlike P+O, which may not specifically address resource allocation or multi-agent systems, DECA explicitly focuses on these complexities. This distinction is crucial as it allows us to restrict the problem space and tailor our research towards enhancing fairness within multi-agent resource allocation.

Significant research has addressed algorithmic bias, where ML models, such as those used in hiring decisions ____, can exhibit harmful biases. 
We refer readers to an extensive survey by ____ for a review of recent work.
These studies typically focus on debiasing the outputs of predictive models to meet fairness criteria such as equalized odds____ or demographic parity____. However, our work diverges from this approach. Instead of correcting biases in predictions, we aim to develop algorithms that inherently promote fair decision-making via the actions they optimize.

Our focus in this paper is on designing ways to learn fair policies in a multi-agent RL setting. ____ present a survey on RL methods used to improve fairness. We now highlight a few papers that are the closest to our work: FEN____ uses a hierarchical network to learn a fair-efficient policy for multi-agent coordination, learning to optimize the coefficient of variation, with a meta network that selects when each agent behaves greedily or fairly. However, the model does not allow for resource constraints, instead opting for a first-come-first-serve approach. Further, this approach needs communication between agents to allow agents to choose between acting fairly and efficiently. Some methods, on the other hand, propose to optimize fairness in a multi-objective MDP, where each agent's utility is treated as a separate objective, and the goal is to optimize the a social welfare function over agent rewards____. 
This means the learning agent has to predict the utility over the joint action space____, or, as done by SOTO____, use a decentralized policy gradient based approach, which prevents use of global constraints. 
Finally, SI____ is an approach for improving fairness in rideshare-matching that attempts to improve fairness through myopic fairness post-processing of black-box utility estimates. However, it does not attempt to learn long-term fairness, and is specially designed for the ridesharing domain.

The DECA approach allows us to consider global constraints while allocating resources, opening up the scope for better global solutions, which none of the prior approaches allow. The distributed evaluation allows each agent to only learn a local value function, which reduces the complexity when compared to learning a joint policy. Further, our Split and Fair-Only approaches allow changing the trade-offs between utility and fairness post-training, which provides additional flexibility that previous approaches lack. 


\section{Viewer-provider two-sided systems}

This section models the dynamics of viewer and provider populations on a recommendation platform. 
Specifically, we consider sub-group dynamics where viewers and providers are categorized into $K$ and $L$ subgroups\footnote{We can consider a ``subgroup'' of size 1. In such cases, the viewer ``population'' corresponds to the time spent by an individual viewer, while the provider ``population'' can be the amount of content produced by an individual provider.
}. Then, we model the populations, recommendation policy, payoffs, and social welfare as follows.

\begin{enumerate}[leftmargin=12pt]
    \item (Viewer/provider population)  
    Let $\lambda_{k} \in \mathbb{R}_{\geq 0}$ be the population of the viewer group $k \in [K]$ and $\lambda_{l}$ be that of the provider group $l \in [L]$. Also let $\blambda := (\lambda_{1}, \lambda_{2}, \cdots, \lambda_{K},
    \lambda_{1}, \lambda_{2}, \cdots, \lambda_{L})$ be the joint population vector of viewers and providers.
    \item (Platform's recommendation policy) 
    The platform matches each viewer group $k$ to a provider group $l$ with a recommendation policy denoted by a $K$-by-$L$ matrix $\bpi$. Specifically, its $(k,l)$-th element $\pi_{k,l}$ represents the probability of allocating the provider group $l$ to the viewer group $k$. 
    Thus $\sum_{l=1}^L \pi_{k,l} = 1, \forall k \in [K]$. For example, the uniform random policy, which assigns equal exposure probability across all provider groups is represented as given by $\bpi=\frac{1}{L}\1_{L\times K}$.
    \item (Viewer/provider payoffs) After viewer and provider groups are matched by the policy $\bpi$, their perceived payoffs can be quantified by the following metrics:
    \begin{align}\label{eq:user_satisfaction}
    \text{Viewer Satisfaction: \quad } & s_k = \textstyle \sum_{l=1}^L \pi_{k,l} q_{k,l} \,  , \\\label{eq:content_exposure}
    \text{Provider Exposure: \quad} & e_l = \textstyle\sum_{k=1}^K \pi_{k,l}\lambda_k,
    \end{align}
    where $q_{k,l}$ is the (expected) utility that viewers $k$ receive from the provider groups $l$. Eqs.~\eqref{eq:user_satisfaction} and~\eqref{eq:content_exposure} define viewer satisfaction as determined by the total utility they receive from recommendations, while providers care about the total amount of exposure they receive by recommendation. This model is prevalent is prior works including \citep{singh2018fairness, mladenov2020optimizing}.
    \item (Social welfare) Finally, we consider the following total viewer welfare as the global metric of the platform:
    \begin{align*}
        R(\bpi; \blambda) := \textstyle\sum_{k=1}^{K} \lambda_{k} s_k
    \end{align*}
    Note that here we consider the sum of viewer-side satisfaction as the social welfare, a formulation prevalent in related works~\citep{mladenov2020optimizing, huttenlocher2023matching}.
    The sum of content-side exposure simplifies to the total size of the viewer population.
\end{enumerate}

\subsection{Interaction dynamics and ``population effects''}\label{sec:dynamic_formulation}

We have so far seen a typical formulation in two-sided platforms. However, a key limitation of such formulation is to ignore potential non-stationarity in the viewer and provider populations, which is common in many real-world two-sided systems~\citep{boutilier2023modeling,  deffayet2024sardine}. 

First, consider the impact of provider population growth on the utility experience by viewers, which we call \textit{``population effects''}.
An increase in provider population naturally leads to more high-quality content. 
For example, consider a two-stage recommendation policy, where our higher-level policy $\bpi$ decides the matching between viewer and provider groups, and a second-stage policy selects individual providers among the selected group. 
Any reasonable second stage policy should be able to select a better provider from a larger provider pool~\citep{su2023value, evnine2024achieving}. 
To model such ``population effects'', we introduce the following utility decomposition:
\begin{align}
    q_{k,l} = b_{k,l} + f_{k,l}(\lambda_{l}) \label{eq:reward_decomposition}
\end{align}
where $b_{k,l}$ is the \textit{base} utility, which may indicates the matching between the preference of viewer and provider groups (e.g., this viewer group likes sports articles). In contrast, $f_{k,l}(\cdot)$ represents the quality of the provider which improves as the provider population increases. $f_{k,l}$ might be heterogeneous among different viewer and provider groups because quality might be multi-dimensional (e.g., visuals, intellects, novelty), viewers may have different preferences, and providers may have different abilities. 
We take $f_{k,l}$ to be a monotonically increasing function.

Next, consider the impact of viewer and provider payoffs on the population.
The number of viewers that a platform can maintain is related to the level of satisfaction, similarly the number of providers is related to the exposure.
We assume that viewer and provider subgroups have 
some \textit{``reference''} population $\bar{\lambda}_{k}(s_{k})$ and $\bar{\lambda}_{l}(e_{l})$ given the level of viewer satisfaction $s_k$ and provider exposure $e_l$. We assume that $\bar{\lambda}$ is a monotonically increasing function, so higher viewer satisfaction and provider exposure result in increased populations. 
Based on this, we model the viewer and provider population dynamics as:
\begin{align}
    \text{Viewer: \,}  \lambda_{t+1,k} = (1 - \eta_k) \lambda_{t,k} + \eta_k \bar{\lambda}_{k}(s_{t,k}), \label{eq:user_dynamics} \\
    \text{Content: \,}  \lambda_{t+1,l} = (1 - \eta_l) \lambda_{t,l} + \eta_l \bar{\lambda}_{l}(e_{t,l}), \label{eq:content_dynamics}
\end{align}
where $\eta \in [0, 1]$ are the \textit{reactiveness} hyperparams, determining how fast the population changes. Note that similar models are widely adopted in performative predictions~\citep{perdomo2020performative, brown2022performative}. 
We thus have that the viewer satisfaction $s_k$ depends on the provider population via ``population effects'' $f_{k,l}$, while the provider exposure directly depends on the viewer population.
The two-sided platform has complex dynamics between viewers and providers. 
Our goal will be to consider long-term objectives under such co-evolving and two-sided dynamics.

\subsection{Game-theoretic interpretation}\label{sec:game_formulation}

Next, we provide a further justification of and insight into the dynamics model by introducing a game-theoretic formulation that is equivalent to Eqs. \eqref{eq:user_dynamics} and \eqref{eq:content_dynamics}.

Consider a $(K+L)$-player game involving $K$ viewer groups and $L$ provider groups. Each viewer group selects a pure strategy $\lambda_k \in \RR_{\geq 0}$, and each provider group chooses a pure strategy $\lambda_l \in \RR_{\geq 0}$. The utility functions for the viewer and provider groups, denoted by $\{u_k\}_{k=1}^K$ and $\{v_l\}_{l=1}^L$ are defined as follows:
\begin{align}\label{eq:util_user}
    & u_k(\blambda)= \lambda_k \cdot \bar{\lambda}_k \left(\sum_{l=1}^L \pi_{k,l}\left(b_{k,l}+f_{k,l}(\lambda_l)\right)\right)-\frac{\lambda_k^2}{2}, \\ \label{eq:util_creator}
    & v_l(\blambda)= \lambda_l\cdot \bar{\lambda}_l \left(\textstyle\sum_{k=1}^K \pi_{k,l}\lambda_k\right)-\frac{\lambda_l^2}{2},
\end{align}
We denote this game as $\G(\bpi, B, f, \bar{\lambda})$, where $B$ is a $K$-by-$L$ matrix whose $(k,l)$-element is $b_{k,l}$. Proposition \ref{prop:dynamics_equivalence} establishes a connection between the game instance $\G$ and the 
formulation presented in Section \ref{sec:dynamic_formulation}.

\begin{proposition}\label{prop:dynamics_equivalence}
    If all players in $\G$ apply gradient ascent to optimize their utility functions with learning rates $\{\eta_k\}_{k=1}^K$ and $\{\eta_l\}_{l=1}^L$, the resulting joint strategy evolving dynamics are exactly given by Eqs.~\eqref{eq:user_dynamics} and \eqref{eq:content_dynamics}.
\end{proposition}

Through Proposition \ref{prop:dynamics_equivalence}, our game-theoretic formulation provides a first-principles perspective for understanding the dynamical formulation in Eqs.~\eqref{eq:user_dynamics} and \eqref{eq:content_dynamics}.\footnote{The game $\G$ resembles the Cournot Duopoly competition \cite{cournot1838recherches}. When $K = L = 1$ and $\bar{\lambda}(\mu) = a - b\mu$ and $\bar{\mu}(\lambda) = a - b\lambda$ for some positive constants $a$ and $b$, the game $\G$ corresponds exactly to the Cournot Duopoly model. The key distinction in ours is that $\bar{\mu}$ and $\bar{\lambda}$ are generic increasing functions.} 
That is, 
we can interpret $\bar{\lambda}(\cdot)$ as the marginal gain from increasing the size of a viewer or provider group by one unit. Consequently, the first terms $\lambda_k \cdot \bar{\lambda}_k(\cdot)$ and $\lambda_l \cdot \bar{\lambda}_l(\cdot)$ represent the collective payoffs for viewer and provider groups of sizes $\lambda_k$ and $\lambda_l$. 
The quadratic terms $-\frac{\lambda_k^2}{2}$ and $-\frac{\lambda_l^2}{2}$ capture the congestion costs associated with maintaining larger populations (e.g., if a provider group becomes too large, providers within the group may face intensified competition and thus reduce their productivity due to diminished marginal gains). This suggests that Eqs.~\eqref{eq:user_dynamics} and \eqref{eq:content_dynamics} are quite reasonable formulation to capture real-world interactions.

\begin{figure*}[ht]
    \centering
    \includegraphics[width=0.98\linewidth]{Figures/variance.png}
    \caption{Change in system utility and fairness as $\beta$ is increased, with $\beta=0$ at the top left $\beta=1$ at the bottom-right. For all domains, we can see that split and joint optimization perform similarly, while learning only fairness can sometimes be slightly worse. All our methods Pareto-dominate SOTO and FEN. Each point depicts the average performance over five different models trained at that $\beta$ value, and the lines show the Pareto front for each method.}
    \label{fig:main_results}
\end{figure*}