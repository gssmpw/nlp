% This must be in the first 5 lines to tell arXiv to use pdfLaTeX, which is strongly recommended.
\pdfoutput=1
% In particular, the hyperref package requires pdfLaTeX in order to break URLs across lines.

\documentclass[11pt]{article}

% Change "review" to "final" to generate the final (sometimes called camera-ready) version.
% Change to "preprint" to generate a non-anonymous version with page numbers.
\usepackage[preprint]{acl}

% Standard package includes
\usepackage{times}
\usepackage{latexsym}

% For proper rendering and hyphenation of words containing Latin characters (including in bib files)
\usepackage[T1]{fontenc}
% For Vietnamese characters
% \usepackage[T5]{fontenc}
% See https://www.latex-project.org/help/documentation/encguide.pdf for other character sets

% This assumes your files are encoded as UTF8

\usepackage[utf8]{inputenc}


% This is not strictly necessary, and may be commented out,
% but it will improve the layout of the manuscript,
% and will typically save some space.
\usepackage{microtype}

\usepackage{makecell}
\usepackage{subcaption}
\usepackage{graphicx}
% Standard package includes
\usepackage{times}
\usepackage{latexsym}
\usepackage{lipsum}
\usepackage{amsmath}
\usepackage{booktabs}
\usepackage{multirow}
\usepackage{placeins} 

% This is also not strictly necessary, and may be commented out.
% However, it will improve the aesthetics of text in
% the typewriter font.
\usepackage{inconsolata}

%Including images in your LaTeX document requires adding
%additional package(s)
\usepackage{graphicx}

% If the title and author information does not fit in the area allocated, uncomment the following
%
%\setlength\titlebox{<dim>}
%
% and set <dim> to something 5cm or larger.

%%%% copy from here
% Define content to appear before the title (e.g., a horizontal line)


\title{From Perceptions to Decisions: Wildfire Evacuation Decision Prediction with Behavioral Theory-informed LLMs}


\author{ 
  \textbf{Ruxiao Chen}\textsuperscript{*1}, 
  \textbf{Chenguang Wang}\textsuperscript{1}\thanks{Equal Contribution.}, 
  \textbf{Yuran Sun}\textsuperscript{2}, 
  \textbf{Xilei Zhao}\textsuperscript{2}, 
   \textbf{Susu Xu}\textsuperscript{1}\\
  \textsuperscript{1}Johns Hopkins University \textsuperscript{2}University of Florida\\
  \{rchen117@jh.edu,cwang274, sxu83\}@jh.edu \\
}


\begin{document}
\maketitle
\begin{abstract}
Evacuation decision prediction is critical for efficient and effective wildfire response by helping emergency management anticipate traffic congestion and bottlenecks, allocate resources, and minimize negative impacts.
Traditional statistical methods for evacuation decision prediction fail to capture the complex and diverse behavioral logic of different individuals.
In this work, for the first time, we introduce \emph{FLARE}, short for facilitating LLM for advanced reasoning on wildfire evacuation decision prediction, a Large Language Model (LLM)-based framework that integrates behavioral theories and models to streamline the Chain-of-Thought (CoT) reasoning and subsequently integrate with memory-based Reinforcement Learning (RL) module to provide accurate evacuation decision prediction and understanding. 
Our proposed method addresses the limitations of using existing LLMs for evacuation behavioral predictions, such as limited survey data, mismatching with behavioral theory, conflicting individual preferences, implicit and complex mental states, and intractable mental state-behavior mapping.
Experiments on three post-wildfire survey datasets show an average of 20.47\% performance improvement over traditional theory-informed behavioral models, with strong cross-event generalizability. 
Our complete code is publicly available at \url{https://github.com/SusuXu-s-Lab/FLARE}
\end{abstract}

\section{Introduction}
Wildfires are emerging as a significant natural hazard worldwide~\cite{jain2020review,zahura2024impact}.
In the January 2025 Southern California wildfires, more than 200,000 residents received evacuation orders to leave their homes~\cite{nbcnews_california_wildfires_2025}.
There is an urgent demand for emergency planners and policymakers to develop effective evacuation strategies to mitigate wildfire impacts~\cite{mockrin2018does, tapley2023reinforcement}. However, successful evacuations require a clear understanding of the human evacuation decision-making process and outcomes (i.e. whether individuals will follow the order to evacuate or stay) during these events to help policymakers improve evacuation order design, develop more efficient emergency response strategies, and build more resilient communities~\cite{collins2018evacuation, lovreglio2020calibrating, hong2020modeling, sun4953233investigating}. 


Previous studies often construct evacuation choice models through a conceptual framework, Protective Action Decision Model (PADM)~\cite{strahan2019protective, lovreglio2019modelling, santana2021psychological, SUN2024106557}, 
which are designed to incorporate psychological factors, like individual risk perception and threat assessment, into the prediction process. Based on the PADM framework, past methods may employ various statistical models (e.g., logistic regression~\cite{forrister2024analyzing}, multinomial logistic regression~\cite{mccaffrey2018should}) to predict individual-evacuation decisions using socio-demographic information as inputs, trained on post-wildfire survey data.
However, these traditional PADM-type statistical models lack reasoning capabilities to capture the diverse and complex logic underlying human decision-making due to limited data and restrictive modeling structure, even when the survey design is grounded in established behavioral theories. In addition, these statistical methods struggle to integrate qualitative descriptions — such as narrative accounts of wildfire dynamics or contextual details — which are critical for understanding evacuees' perceptions and the rationale behind their evacuation decisions.
To address these limitations, the recent emergence of Large Language Models (LLMs) provides exceptional reasoning capabilities to model and predict evacuation decision-making processes~\cite{huang2022towards, li-etal-2024-quantity, li-etal-2024-superfiltering, nguyen2024predicting, liu2024large, lee2024reasoning}. Compared to traditional statistical models, LLMs display theory of mind (ToM) capabilities and have the potential to bridge the information gap present in survey data by better approximating human decision-making logic. 
LLMs also facilitate the integration of contextual information into the predictive process.

However, employing the existing LLM framework for evacuation decision-making modeling and prediction, with socio-demographic information as inputs, still faces four significant challenges:
(1) \textbf{Mismatching with behavioral theory}: 
Evacuation survey data size is often limited, for example, 334 valid examples for the 2021 Marshall wildfires~\cite{forrister2024analyzing}. LLMs tend to capture only partial reasoning patterns and overfit limited survey data, struggling to align with established behavioral theories~\cite{tjuatja2023llms, petrov2024limited, macmillan2024ir}. 
(2) \textbf{Conflicting preferences in aligning with human thought}:
Reinforcement Learning with Human Feedback (RLHF) offers a promising approach to aligning LLM reasoning with human thought~\cite{sun2023aligninglargemultimodalmodels, zhang2024personalizationlargelanguagemodels, xu2024surveyknowledgedistillationlarge}. However, it is still challenging to accommodate individuals with diverse evacuation patterns~\cite{zhao2022estimating, sun2024social}. For example, some individuals may evacuate immediately upon receiving an official order, prioritizing institutional guidance, while others may rely on social cues, choosing to stay until observing their neighbor evacuating.
(3) \textbf{Incorporating implicit mental states}: 
Previous studies show that integrating mental states will benefit the improvement of human behavior predictions~\cite{gu2024simpletom}. However, in a highly dynamic and chaotic wildfire environment, there exist many implicit, diverse, and complex mental states, perceptions, or beliefs, that drive wildfire evacuation behaviors, making it challenging to learn and incorporate such implicit mental states.
(4) \textbf{One mental state may map to multiple behaviors}: 
Even if individuals share the same mental state, their behaviors can vary due to external factors. For instance, two individuals with identical risk perceptions might make different evacuation decisions due to the different traffic congestion situations and shelter accessibility~\cite{collins2018evacuation, hong2020modeling}.


To address these challenges, we propose \emph{FLARE}, a novel LLM-based framework for evacuation decision prediction. We introduced risk perception and threat assessment~\cite{SUN2024106557}, two critical concepts in traditional evacuation behavioral models, to represent individual mental states. As in Figure~\ref{fig:overview}, we design a classifier based on PADM, constructed using historical datasets and empirical behavioral studies, to select the most relevant input variables to risk perception and threat assessment. We further design a reasoning pattern classifier to assign the most probable reasoning patterns. An LLM further infers the perceptions and assigns corresponding scores from the selected reasoning patterns. Finally, these perceptions — combined with external information and user inputs  — are integrated into a universal CoT template. The CoT is then fed into the LLM along with previous error records and their self-reflected rationale from the training phase for prediction. 

The proposed framework adapts empirical psychological and behavioral knowledge to inform the variable selection, CoT template construction, and model inference to constrain the over-expressiveness of LLMs on small datasets and encourage a better alignment with existing behavioral theories. Both reasoning path classifier and memory-based RL help mitigate RLHF's conflicting preference issues and tailor the prediction to individual behavior. Moreover, we calibrate implicit mental state generation chains through self-validation with a classifier based on PADM and sparsely available but implicit risk/threat-related answers in the survey data. We also augment the evacuation decision prediction with similar past error trials and their reflections from memory, guiding the model toward more accurate outcomes. We also integrate descriptive and descriptive external knowledge (e.g., wildfire progression) as external cues about the individual situation, guiding the LLM to translate the inferred mental state into an accurate behavior prediction.
Our main contributions include:
\vspace{-1em}
\begin{itemize}
    \item We propose a novel framework that integrates LLMs' advanced reasoning capability with psychological and behavioral theories, improving evacuation decision prediction accuracy in small, highly imbalanced datasets.
    \vspace{-0.75em}
    \item We introduce a behavioral theory-informed classifier to distinguish individual reasoning patterns, addressing conflicting preference problems while constraining the reasoning paradigm.
    \vspace{-0.75em}
    \item We introduce implicit mental state learning before predicting evacuation behaviors and augment them with external information and user input, to further improve reasoning capabilities for evacuation decision prediction.
    \vspace{-0.75em}
    \item We incorporate memory of error record and self-reflection mechanisms to refine the model’s reasoning process from mental states to behavioral predictions,  enhancing its alignment with real-world evacuation behavior.
\end{itemize}

\vspace{-0.2cm}
\section{Related Work}
\vspace{-0.2cm}

\subsection{Wildfire Evacuation Decision Prediction}
\vspace{-0.2cm}
Recent research has employed multiple methods to predict wildfire evacuation decisions.
\citet{mccaffrey2018should} employed a multinomial logistic model based on PADM, enhanced by a latent class approach, to predict diverse evacuation decisions in three U.S. fire-prone counties. 
\citet{forrister2024analyzing} applied logistic and linear regression to predict risk perception, evacuation decision, and delay time.
\citet{xu2023predicting} benchmarked seven machine learning approaches (e.g., Random Forest, Classification And Regression Trees (CART), Extreme Gradient Boosting) and identified CART as the top-performing model for predicting evacuation behavior from the 2019 Kincade Fire survey.
Meanwhile, \citet{lovreglio2020calibrating} introduced the Wildfire Decision Model (WDM) calibrated via Hybrid Choice Models (HCM), incorporating latent factors like risk perception and prior experience for more accurate evacuation decision predictions.
\citet{sun4953233investigating} further integrates risk perception and threat assessment as latent variables into an HCM framework, improving prediction accuracy.
Traditional statistical models do not account for the logical flow of decision-making. HCM, in contrast, considers this process.




\vspace{-0.2cm}
\subsection{LLMs for Human Decision and Behavior Prediction}
\vspace{-0.2cm}
Current work increasingly leverages LLMs to model and predict human decisions. 
BigToM~\cite{gandhi2024understanding} evaluates LLMs’ Theory-of-Mind (ToM) reasoning via causal templates and finds that GPT-4 partially matches human ToM performance while other models lag behind. 
Similarly, SUVA~\cite{leng2023llm} utilizes probabilistic modeling and behavioral economics games, revealing that larger LLMs display stronger prosocial and group-identity effects. 
DEBATunE~\cite{li-etal-2024-llms-speak} utilizes a multi-agent debate process for data synthesis on controversial debate topics and utilizes supervised fine-tuning to simulate behaviors on controversial topics. 
Extending these insights, ToM~\cite{amirizaniani2024llms} focuses on open-ended social reasoning from Reddit’s ChangeMyView posts, showing prompt tuning with human intentions and emotions boosts performance but remains below human-level comprehension. 
Although these models can infer mental states, T4D~\cite{zhou2023far} highlights the challenge of converting such inferences into strategic action, as even GPT-4 struggles without structured guidance. 
In parallel, LELMA~\cite{mensfelt2024logic} integrates symbolic AI to verify logical consistency in social simulations like the Prisoner’s Dilemma, demonstrating that self-refinement methods can improve the reliability of LLM-generated reasoning. 
Meanwhile, SimpleToM~\cite{gu2024simpletom} underscores that while LLMs can predict mental states and behavior, they often require deliberate prompting for accurate moral or behavioral judgments. \citet{kang-etal-2023-values} propose the Value Injection Method (VIM), embedding human core values to enhance opinion and choice prediction. 
However, \citet{kuribayashi-etal-2024-psychometric} caution that instruction tuning and prompting do not inherently offer better alignment with human cognition than direct probability outputs from base LLMs. %Turning to multi-agent configurations, Sreedhar et al. \cite{sreedhar2025simulating} show that having multiple LLMs negotiate in the Ultimatum Game leads to more human-like strategic reasoning than single-model setups. 
\citet{zhu2024language} reveal that arithmetic-trained LLMs can surpass classic decision-theoretic models when evaluating risky and time-delayed choices, demonstrating that specialized numerical training augments behavioral prediction. 
\citet{liu2024large} emphasizes a core limitation: LLMs systematically assume people behave more rationally than they do, underestimating well-documented human biases and highlighting a persistent gap between model predictions and real-world decision-making.

 

\vspace{-0.2cm}
\section{Method}
\vspace{-0.2cm}


In this section, we present the development of our LLM-based pipeline for evaluating evacuation decisions using post-wildfire survey data. The pipeline is shown in Figure \ref{fig:overview}. 

\vspace{-0.2cm}
\subsection{Preliminary}
\label{sec:prel}
\vspace{-0.2cm}
\textbf{Evacuation Decision Prediction:} 
The Protective Action Decision Model (PADM) is a conceptual framework designed to explain human cognitive processes and decision-making behaviors in response to hazards and disasters. 
At its core, PADM emphasizes perceptions(e.g., threat perceptions, protective action perceptions, and stakeholder perceptions) that shape individuals’ decisions on how to respond to both immediate and long-term threats \cite{lindell2012protective}.

To advance wildfire evacuation decision prediction, researchers utilized the Hybrid Choice Model (HCM)to integrate the conceptual framework
of the PADM to predict evacuation decisions~\cite{lovreglio2020calibrating, sun4953233investigating}, quantifying two latent variables—threat assessment and risk perception — capturing individuals' assessments of wildfire threats  (e.g., injury, death) and risks (e.g., home/neighborhood damage or destruction)~\cite{Kuligowski2021}. 
By modeling how individuals process risk and threat influences, this approach provides a structured framework for understanding the cognitive processes behind wildfire evacuation behavior.

\textbf{Wildfire Evacuation Survey Data:}
The Wildfire Evacuation Survey collects data about participants' experiences during wildfire events through a set of multiple-choice, scaled, and open-ended questions. Questions~\cite{kuligowski2022modeling} cover topics such as prior wildfire risk knowledge, emergency plans, evacuation experiences, property damage, warning system awareness, and household socio-demographics. Responses range from numerical scales (e.g., personal injury perception rated from 1 to 5) and categorical choices (e.g., ``Yes'' or ``No'') to counts (e.g., number of evacuations), while also capturing qualitative details like medical conditions and household composition. For detailed and complete survey refer to~\cite{KULIGOWSKI2022105541} for Kincade Fire, and ~\cite{FORRISTER2024100729} for the Marshall Fire.



\vspace{-0.2cm}
\begin{figure}[t]
    \centering
    \includegraphics[width=1\linewidth]{figures/overview_main.png}
    \caption{Overview of \emph{FLARE}.}
    \label{fig:overview}
    \vspace{-1.5em}
\end{figure}



\vspace{-0.2cm}
\subsection{Reasoning Process Formulation}
\vspace{-0.2cm}

In this section, we describe how we construct classifiers based on PADM that identify the most probable reasoning patterns from survey data variables. These patterns are derived from the previously introduced risk perception and threat assessment. Once the appropriate reasoning pattern is determined, we demonstrate how an LLM generates the corresponding perception and integrates it into a CoT template, yielding the finalized CoT for evacuation prediction.

\vspace{-0.2cm}
\subsubsection{Variable Selection for Perceptions}
\vspace{-0.2cm}

Building on the HCM framework based on PADM for wildfire evacuation decision prediction~\cite{lovreglio2020calibrating,sun4953233investigating}, our approach aims to develop a unified statistical method to automatically select the key variables that contribute to risk perceptions and threat assessment by examining all available survey questions.

As briefly mentioned in Section \ref{sec:prel}, the survey includes questions capturing socio-demographic data, awareness and understanding, and decision-related factors such as prior wildfire risk awareness, emergency preparedness, evacuation experience, warning system awareness, personal injury perceptions, household income, employment status, and medical conditions.
In the HCM method~\cite{sun4953233investigating}, threat assessment and risk perceptions are validated using specific indicators derived from survey questions, requiring manual selection of variables, fitting them to indicators, and evaluating their alignment.
Our approach automates this process by incorporating all available variables into the fitting process and selecting those with the highest weights, ensuring the strongest contributions to risk perceptions and threat assessments. This minimizes bias in manual selection and enhances the model’s ability to capture key evacuation decision factors.

Formally, we regress each perception indicator on all survey variables:
\begin{equation}
Y_{k} = w_1 X_1 + w_2 X_2 + \cdots + w_n X_n + \epsilon,
\end{equation}
where $Y_k$ is the dependent variable (with $k = T_h$ for threat assessment or $k = R_h$ for risk perception), $w_i$ are the weights for the variables $X_i$, and $\epsilon$ is an error term. Once all variables are fitted in this regression, 
We then select a subset $X'$ of variables whose cumulative weight meets a predefined threshold $\theta$. Mathematically, this criterion is:
\begin{equation}
\sum_{X_i \in X'} |w_i| \ge \theta \cdot \sum_{i=1}^n |w_i|.
\end{equation}
Empirically, $\theta$ corresponds to the elbow point in the weight distribution, ensuring key variables are retained while filtering out less significant ones. The empirical results visualizing this elbow point are provided in Appendix~\ref{appx:weight_dis}, demonstrating the sharp decline in variable importance beyond the selected threshold.

Following the HCM framework, we derive four reasoning patterns by combining threat (e.g., injury, death) and risk (e.g., home/neighborhood damage or destruction) indicators-informed variable subsets, leading to four distinct reasoning processes. The selected variables are detailed in Appendix \ref{appx:select_q}.
With the core variable subsets identified, the next section explores how these guide the construction of the CoT, structuring inference pathways for evacuation decision modeling.


\vspace{-0.2cm}
\subsubsection{CoT Construction based on Perceptions}
\vspace{-0.2cm}

In this section, we describe how we construct the CoT by leveraging the variable subsets selected earlier. 
First, we develop a universal CoT template (see Appendix \ref{sec:Prompt}) that structures reasoning into two stages — threat assessment followed by risk perception — to enforce a behaviorally grounded reasoning path. 
We then introduce a reasoning pattern classifier that selects the most probable reasoning pattern for each individual based on the prediction success rate of each reasoning pattern. 
Finally, we prompt the LLM to generate textual threat assessments and risk perceptions along with their quantitative perception scores for each case, calibrating consistency and establishing a foundation for subsequent evacuation decision prediction.

\emph{\underline{Reasoning Pattern Classifier:}}
We classify individuals into the four reasoning patterns through a statistical machine learning classifier (e.g., random forest), training it on quantified survey data as input and using the LLM’s prediction performance across reasoning patterns as labels to automate pattern selection.
For each individual, we first populate all four reasoning patterns using their survey responses to generate corresponding perceptions, which are then inserted into the CoT template to form a temporary CoT for prediction. 
We then conduct multiple inference trials for each temporary CoT, with each trial producing a predicted evacuation decision (evacuate or stay) that is compared to the individual’s actual response to the evacuation decision. The success rate for each pattern is computed as the proportion of correct predictions, and the pattern with the highest success rate is considered the most probable reasoning pattern for that individual.


We further use the estimated most probable pattern as the label for individuals and train the classifier on these labels and relevant survey variables (e.g., socio-demographics, evacuation order awareness). 
This classifier automates pattern selection, ensuring that the model dynamically adapts to psychological and situational factors, enabling personalized and interpretable evacuation predictions.

\emph{\underline{Perception Inference:}} 
After selecting each individual's most probable reasoning pattern, we refine the generation of risk perceptions and threat assessments to construct a complete CoT for evacuation decision prediction.
The LLM first generates textual threat assessment and risk perceptions while explicitly assigning quantitative perception scores (1–5) as calibration indicators. This dual representation enhances consistency between inferred perceptions and key survey variables. To refine this calibration, we use the first 70\% of the dataset to establish a knowledge base mapping LLM-generated perceptions to survey-derived scores, allowing the model to align qualitative reasoning with numerical assessment.


In the inference stage, we employ Retrieval-Augmented Generation (RAG) to maintain score consistency. The LLM-generated perceptions are compared to stored examples using semantic similarity, retrieving the two most similar instances (based on cosine similarity) along with their scores. This retrieval process aligns predicted scores with established reasoning patterns, ensuring consistency and accuracy in perception inference.

Once robust textual perceptions and their corresponding numerical scores are obtained, we integrate them into the CoT templates along with external information such as environment context derived from survey questions (e.g., ``I'm not in the area ordered to evacuate.'') and user inputs (detailed in Figure \ref{fig:prompt_full}), forming the complete CoT for evacuation decision prediction. 
This construction process ensures that the CoT consistently adapts to each individual's most probable reasoning pattern while leveraging textual information.

With this complete CoT in place, we establish a reasoning process that accommodates multiple threat–risk scenarios, setting the stage for aligning the reasoning process more closely with human decision-making through the memory-based RL module.

\vspace{-0.2cm}
\subsection{Memory-based RL}
\vspace{-0.2cm}
Extending the previously described CoT construction for inference, we further align the reasoning process with human decision-making by incorporating an RL strategy during the LLM inference phase. 
Inspired by the verbal-based RL methodology in Reflexion~\cite{shinn2024reflexion}, our approach introduces a dedicated \emph{Memory} component that records inference errors along with the corresponding LLM-generated rationales. This \emph{Memory} mechanism enables the model to learn from past mistakes and adapt its decision-making, bridging the gap between the metal state prediction (i.e. perceptions) and evacuation decision prediction.

We begin with a training stage to construct the \emph{Memory} for subsequent use. During this stage, the actual evacuation decision reported by each respondent serves as the ground-truth reward signal. 
Whenever the LLM’s predicted decision is incorrect, we store the CoT for inference, the environment context, the LLM-generated rationale, and the correct decision in Memory. 
The model is then prompted to regenerate its reasoning and reflect on the source of the error, with these self-reflection notes also appended to \emph{Memory}.
For subsequent data samples, we retrieve the top‑$k$ most similar past entries — determined via cosine similarity over relevant variable representations — and integrate these entries as contextual information into the current inference. This retrieval mechanism allows the LLM to leverage prior cases with comparable circumstances or error patterns, refining its predictions over time.

After accumulating sufficient history in \emph{Memory} during the training phase, we transition to inference on new data. 
At this stage, self-reflection and error logging are disabled; instead, the \emph{Memory}’s contextual information is directly incorporated into the input, guiding the LLM’s reasoning process. 
The final output comprises the predicted evacuation decision and a supporting rationale derived from the CoT and contextual information retrieved from similar cases in \emph{Memory}. This comprehensive output ensures accurate predictions while providing interpretable insights into individual evacuation decisions.


\vspace{-0.2cm}
\section{Experiment}
\vspace{-0.2cm}

\emph{FLARE}, leverages both a combined dataset and the individual post-disaster survey datasets from the 2018 Carr Fire~\cite{wong2020review}, 2019 Kincade Fire~\cite{KULIGOWSKI2022105541}, and 2021 Marshall Fire~\cite{FORRISTER2024100729}. The characteristics of each dataset, including evacuation ratio, and utilized variables ratio, are detailed in Table \ref{tab:dataset_summary}. 
By integrating these data sources, we facilitate a comprehensive prediction of evacuation behavior while also preserving the unique characteristics of each event through separate analyses. The whole framework is implemented via LangChain. A detailed evaluation using metrics such as Accuracy, Precision, Recall, F1-score, Macro F1-score, and Weighted F1-score MSE is provided, with further details available in Appendix \ref{appx:implement}.

% \section{Experiment Result}
\begin{table*}[htb]
    \centering
    \resizebox{\textwidth}{!}{
    \begin{tabular}{lcccc cccc cccc cccc} 
        \toprule
        \multirow{2}{*}{\textbf{Model}} & \multicolumn{4}{c}{\textbf{Exclusivity ($\tau =0.7)$} $\uparrow$} & \multicolumn{4}{c}{\textbf{Coverage} $\uparrow$} & \multicolumn{4}{c}{\textbf{Balance} $\uparrow$} & \multicolumn{4}{c}{\textbf{Usability (Macro F1 / Micro F1)} $\uparrow$} \\
        \cmidrule(lr){2-5} \cmidrule(lr){6-9} \cmidrule(lr){10-13} \cmidrule(lr){14-17}
        & \textbf{POL73} & \textbf{TUC74} & \textbf{BRY17} & \textbf{FEI23} & \textbf{POL73} & \textbf{TUC74} & \textbf{BRY17} & \textbf{FEI23} & \textbf{POL73} & \textbf{TUC74} & \textbf{BRY17} & \textbf{FEI23}& \textbf{POL73} & \textbf{TUC74} & \textbf{BRY17} & \textbf{FEI23} \\
        \midrule
        Meta-Llama-3-8B-Instruct & 0.477 & 0.772 & \textbf{0.880} & 0.858 & \multirow{4}{*}{0.698} & \multirow{4}{*}{0.160} & \multirow{4}{*}{\textbf{0.979}} & \multirow{4}{*}{0.924} & \multirow{4}{*}{0.687} & \multirow{4}{*}{0.210} & \multirow{4}{*}{0.829} & \multirow{4}{*}{\textbf{0.878}} & 0.026 / 0.109 & 0.005 / 0.006 & 0.047 / 0.156 & \textbf{0.112} / \textbf{0.357} \\
        Mistral-7B-Instruct-v0.2 & 0.317 & 0.095 & 0.495 & \textbf{0.840} &  &  &  &  &  &  &  &  & 0.026 / 0.123 & 0.001 / 0.004 & 0.015 / 0.113 & \textbf{0.085} / \textbf{0.310}\\
        Claude-3-Haiku & 0.609& 0.596 & 0.742 & \textbf{0.815} &  &  &  &  & &  &  &  & 0.101 / 0.218 & 0.023 / 0.055 & \textbf{0.326} / 0.542 & 0.290 / \textbf{0.620} \\
        ChatGPT-4o & 0.842 & 0.703 & \textbf{0.921} & 0.877 &  &  &  &  &  &  &  &  & 0.301 / 0.478 & 0.061 / 0.099 & 0.610 / \textbf{0.760} & \textbf{0.631} / 0.743 \\
        \midrule
        Average & 0.562 & 0.542 & 0.759 & \textbf{0.848} & 0.698 & 0.160 & \textbf{0.979} & 0.924 & 0.687 & 0.210 & 0.829 & \textbf{0.878} & 0.114 / 0.232 & 0.023 / 0.041 & 0.250 / 0.393 & \textbf{0.280} / \textbf{0.508} \\
        \bottomrule
    \end{tabular}}
    \caption{Performance comparison of classification taxonomies. Higher values indicate better performance in all metrics (Exclusivity, Coverage, Balance, and Usability). The computation of Exclusivity and Usability relies on specific LLMs. Exclusivity is calculated using a confidence threshold of $\tau = 0.7$.}
    \label{tab1}
\end{table*}


\vspace{-0.2cm}
\subsection{Main Results}
\vspace{-0.2cm}
In this section, we present a comparative analysis between our approach and several widely adopted methods for wildfire evacuation decision prediction. Specifically, we perform experiments on a consistent dataset to showcase the superior performance of our method, and we further validate its generalizability through cross-dataset testing.

In the consistent dataset experiments (see Table \ref{tab:same_results}), we compare the performance of our method, \emph{FLARE}—which employs three distinct backbones (GPT-o3-mini~\cite{OpenAI2025}, GPT-4o~\cite{hurst2024gpt}, and Claude-3.5~\cite{Anthropic2024}) separately—with four widely adopted prediction methods: Logistic Regression, Random Forest, LLM Inference with GPT-4o, and HCM. 
These comparisons focus on predicting evacuation decisions on a combined dataset. 
The combined dataset is curated by merging survey data from multiple wildfire events. Detailed results for the single-event dataset are provided in Appendix \ref{appx:more_result}.
The results consistently demonstrate that \emph{FLARE} achieves superior accuracy in evacuation prediction compared to the baseline methods. In contrast, the baseline approaches not only deliver lower overall accuracy but also struggle with balanced detection across various predictions, as evidenced by their F1 scores. 
Moreover, \emph{FLARE} exhibits notable adaptability across different state-of-the-art LLMs, consistently enhancing performance when employing various backbones. Notably, when using Claude-3.5 as the backbone, \emph{FLARE} improves accuracy by 13.2\%, Macro F1 by 12.7\%, and Weighted F1 by 11.9\%. These improvements indicate that advancements in LLM reasoning capabilities~\cite{Anthropic2024} could further elevate the performance of \emph{FLARE}.

In the cross-dataset generalization tests (see Table \ref{tab:cross_results}), \emph{FLARE} consistently outperforms the baseline methods. The experiments were conducted using the Kincade Fire and Marshall Fire datasets in a cross-validation setup, where one dataset served as the training set and the other as the test set.
This setup is designed to account for the fact that the wildfire occurred in two different states, as illustrated in Appendix \ref{sec:geo_info}. Within the same state, evacuation laws and processes are generally similar for such events. By considering distinct states, this setup maximizes the differences between wildfire events, allowing for a more rigorous evaluation of the generalizability of our proposed methods.
As shown in Table \ref{tab:cross_results}, \emph{FLARE} achieved superior performance in terms of accuracy, Macro F1, and weighted F1 scores across both configurations. Notably, the baseline methods—particularly Logistic Regression and Random Forest—struggled to accurately classify the ``Stay'' class. In contrast, \emph{FLARE} delivered higher performance metrics and maintained a more balanced detection across classes. These results underscore the robustness of \emph{FLARE} in cross-event scenarios, highlighting its potential for effective knowledge transfer between different wildfire events.

\vspace{-0.2cm}
\subsection{Ablation Study}
\vspace{-0.2cm}

\begin{table}[htbp]
\centering
\scalebox{0.75}{
\begin{tabular}{c|ccc}
\toprule
\textbf{Method} & \textbf{Acc} & \textbf{Macro F1} & \textbf{Weighted F1} \\
\midrule
FLARE w/o CoT and RL  & 0.708  & 0.671  & 0.707 \\
FLARE w/o RL          & 0.740  & 0.727  & 0.890 \\
FLARE w/o perception & 0.768  & 0.726  & 0.756 \\
FLARE w/o CoT         & 0.773  & 0.706  & 0.768 \\
FLARE            & \textbf{0.816}  & \textbf{0.802}  & \textbf{0.824} \\
\bottomrule
\end{tabular}
}
\caption{\textbf{Ablation study of \emph{FLARE}} conducted using GPT-4 on the combined dataset. Key components of the framework were selectively removed, and the impact of each removal was evaluated using Accuracy, Macro F1, and Weighted F1 metrics. Each component's removal resulted in varying degrees of performance degradation.}
\label{tab:ablation_results}
\vspace{-1.5em}
\end{table}

We conduct ablation experiments on the combined dataset using the GPT-4o model to assess the impact of the CoT formulation and memory-based RL module on FLARE's performance. 
As shown in Table \ref{tab:ablation_results}, removing both components leads to a 13.45\% performance drop, confirming their necessity. When only the RL module is removed, the decline is less severe, highlighting the CoT formulation’s robust reasoning capability. 
Furthermore, removing only the RL module results in better performance than both the CoT and RL removed, highlighting the effectiveness of the RL module.
These findings validate that both components are essential for optimizing predictive accuracy and solidifying FLARE’s effectiveness.


\vspace{-0.2cm}
\section{Discussion - Why It Works}
\vspace{-0.2cm}

The framework's effectiveness is driven by the meticulous design of each component, enabling LLMs to generate accurate evacuation predictions and perform complex reasoning based on survey data.
One critical aspect is the use of CoT reasoning is crucial for the prediction. Research shows that without carefully designed instructions, LLMs struggle to grasp human mind states, underscoring the need for advanced mechanisms to improve real-world understanding~\cite{xu-etal-2024-opentom}. Moreover, the evacuation decision process demands complex reasoning, and CoT enables LLMs to mimic human problem-solving—effectively addressing this complexity while enhancing accuracy and robustness~\cite{wei2022chain, wang2022self, kojima2022large}. Thus, our CoT design integrates both inferred perceptions and external information, establishing a strong foundation for accurate evacuation decision prediction.

Another key component is the design of the RL module, which aligns the reasoning process with human thoughts. Although LLMs can closely approximate human performance in mental state prediction with well-crafted CoT, their accuracy drops in behavior prediction without detailed cues~\cite{gu2024simpletom}. However, providing extensive detail increases the input sequence length, straining the LLM’s limited attention and memory and disrupting the coherent CoT required for accurate multi-step reasoning~\cite{li2024happened, li2024longcontextllmsstrugglelong, levy2024tasktokensimpactinput, qian2024longllmsnecessitylongcontexttasks}. To address this, our memory mechanism stores and retrieves the most similar error records during reasoning, ensuring the model obtains sufficient context from previous errors without overextending the input sequence.
Moreover, the LLM's self-reflection capability enables it to autonomously correct errors and refine its responses, thereby enhancing the quality of its output rationale without human intervention~\cite{renze2024self, li2023reflectiontuning, li-etal-2024-selective, shinn2024reflexion}. When a prediction is incorrect, our self-reflection process ensures that high-quality reflective rationales are stored in Memory, making the contextual cues even more valuable. 

Furthermore, the LLM's natural language understanding ability plays a crucial role in effective prediction. As noted earlier, sufficient information and cues lead to accurate behavior prediction~\cite{gu2024simpletom}. For example, survey responses that include self-reported medical conditions (e.g., ``Extremely disabled from stroke'') are difficult for traditional models to incorporate. In contrast, the LLM can both utilize this information and understand its implications, resulting in more accurate predictions~\cite{xie2011values, qin2024large}.

\vspace{-0.2cm}
\section{Conclusion}
\vspace{-0.2cm}

In this study, we introduced \emph{FLARE}, a novel framework that integrates the reasoning capabilities of LLMs with a well-established behavior theory to address the complexities of human mental states in wildfire evacuation decision prediction. By systematically classifying variables and constructing CoT grounded in threat assessments and risk perceptions, our approach captures evacuees’ heterogeneous preferences and interprets essential perceptions. Moreover, we integrate a memory-based RL module that serves as a dynamic repository of previous errors and justifications, guiding the LLM toward spontaneously improved reasoning. Comprehensive experiments on real-world survey data from historical wildfire events demonstrate that \emph{FLARE} not only outperforms established methods but also maintains robust generalizability across different wildfire events. Overall, \emph{FLARE} broadens the innovative applications of LLMs.

\section*{Limitation}

Although \emph{FLARE} demonstrates promising capabilities in analyzing wildfire evacuation decisions, it is subject to several important limitations.
LLMs using CoT reasoning often lack transparency and can produce misleading outputs~\cite{turpin2023language}. This issue undermines trust and limits their adoption in policy planning and decision-making, where reliability and interpretability are essential.
Another concern is the research relies on self-reported survey data. Although the post-wildfire surveys used in this study adhere to strict data collection protocols and provide valuable insights, they were self-reported data, which may introduce potential recall bias and inaccuracies, which could affect the robustness of the conclusions.
A further issue is that, although the PADM framework accounts for geographical, meteorological, and logistical factors (e.g., perception of wildfire impact forecasts, awareness of shelter availability, and knowledge of route alternatives), our survey design did not include these elements. Consequently, our framework may not capture all factors influencing individuals' decision-making processes. Future work will incorporate these variables into the survey to facilitate more precise decision-making.
Future work should address these limitations by incorporating richer datasets that encompass a broader population and greater geographical diversity, as well as integrating more extensive environmental and logistical variables.

\bibliography{ref}

\clearpage

\appendix
\section{Implementation Details}
\label{appx:implement}
\subsection{Datasets}
The survey data for this study were collected from local residents following three distinct wildfire events: the 2018 Carr Fire in California (284 responses)~\cite{wong2020review}, the 2019 Kincade Fire in California~\cite{kuligowski2022modeling}, (270 responses), and the 2021 Marshall Fire in Colorado~\cite{forrister2024analyzing}, (334 responses). Each survey covered varying aspects of evacuation behavior and perceptions. 
The Kincade Fire survey addressed pre-event and event-based factors, household characteristics, and decisions to stay or evacuate. The Marshall Fire survey emphasized pre-fire awareness, warning types, demographics, evacuation decisions and timing, and environmental cues. The Carr Fire survey captured evacuation behaviors, communication approaches, timing, transportation methods, sheltering choices, and perceived governmental response. Though similarly intended, each survey employed different questions and organizational structures. 

\begin{table*}[t]
\centering
\scalebox{0.75}{
\begin{tabular}{c|c|c|c|c|c}  % Updated to match six columns
\toprule
\textbf{Dataset} & \textbf{Valid Samples} & \textbf{Fire Start Time} & \textbf{Survey Period} & \textbf{Utilized Ratio} & \textbf{Evacuation Rate}\\
\midrule
Marshall Fire   & 334 & 12/30/2021 & 5/2022 - 6/2022 & 61 / 71 & 54.19\%\\
\midrule
Kincade Fire    & 270 & 10/23/2019 & 10/2020 - 1/2021 & 66 / 77 & 81.41\% \\
\midrule
Carr Fire & 500 & 7/28/2018 & 3/2019 - 4/2019 & 71 / 75 & 89.4\% \\
\bottomrule
\end{tabular}}
\caption{Overview of wildfire evacuation dataset statistics used in our model. Valid Sample represents the number of valid survey responses. Fire Start Time indicates the date when the wildfire began. Survey Period specifies the duration over which the survey data was collected. Utilized Ratio is defined as the proportion of variables used relative to the total available variables. Evacuation Rate denotes the percentage of respondents who chose to evacuate.}
\label{tab:dataset_summary}
\end{table*}

\subsection{Implementation Details}
In this study, we combined all three wildfires' survey data into a single dataset and split it into training and test sets in an 80\%–20\% ratio for all classification models and the RL component, ensuring consistency in data usage. The classification model is a decision tree with a maximum depth set to 10, while all other hyperparameters remain at their default settings, balancing interpretability with potential model complexity. For the RL module, we similarly reserve 80\% of the data for iterative training—where the \emph{Memory} is updated repeatedly based on feedback—and use the remaining 20\% for direct inference and final performance assessment. The RL module is implemented using LangChain, providing a streamlined and reproducible framework for experimentation. 



\subsection{Evaluation Metrics}
To evaluate the model's effectiveness in predicting evacuation decisions, we compare the predicted results with actual evacuation decisions using a set of well-established metrics: \textbf{Accuracy, Precision, Recall, F1-score, Macro F1-score, and Weighted F1-score}. Accuracy measures the overall correctness of predictions, while Precision and Recall assess the trade-off between false positives and false negatives, respectively. The F1 score combines Precision and Recall into a single metric to balance their trade-offs. Given the potential class imbalance in evacuation decisions, we also utilize the Macro F1-score, which averages F1-scores across all classes equally, and the Weighted F1-score, which accounts for class frequency by weighting each class's F1-score accordingly. This comprehensive multi-metric approach ensures a thorough understanding of the model's reliability and effectiveness in supporting evacuation decision-making. 

Detailed formulation of evaluation metrics shows as follow:

\textbf{Accuracy} measures the proportion of correctly classified instances among all instances and is suitable for balanced datasets. However, it may be misleading for imbalanced data. It is defined as:

\begin{equation}
Accuracy = \frac{TP + TN}{TP + TN + FP + FN}
\end{equation}

\textbf{Precision} calculates the fraction of correctly predicted positive cases out of all predicted positives. It is crucial in scenarios where false positives are costly. The formula is:

\begin{equation}
Precision = \frac{TP}{TP + FP}
\end{equation}

\textbf{Recall} measures the proportion of actual positive instances correctly identified by the model. A high recall is essential when missing positive cases is more critical than incorrectly classifying negatives. It is given by:

\begin{equation}
Recall = \frac{TP}{TP + FN}
\end{equation}

\textbf{F1-score} is the harmonic mean of precision and recall, balancing both metrics to provide a single performance measure, especially useful in imbalanced datasets. It is computed as:

\begin{equation}
F1 = 2 \times \frac{Precision \times Recall}{Precision + Recall}
\end{equation}

\textbf{Macro F1-score} computes the F1-score for each class independently and averages them, treating all classes equally. Since this is a binary classification task, it is equivalent to the standard F1-score:

\begin{equation}
Macro\ F1 = \frac{F1_{pos} + F1_{neg}}{2}
\end{equation}

\textbf{Weighted F1-score} averages F1-scores across classes but assigns a weight based on class frequency, making it more reliable for imbalanced datasets:

\begin{equation}
Weighted\ F1 = \frac{N_{pos} \times F1_{pos} + N_{neg} \times F1_{neg}}{N_{pos} + N_{neg}}
\end{equation}

where \( N_{pos} \) and \( N_{neg} \) are the number of positive and negative samples, respectively.

\textbf{Mean Squared Error (MSE)} measures the average squared difference between predicted and actual values, commonly used in regression tasks. It penalizes larger errors more heavily, making it sensitive to outliers. The formula is:

\begin{equation}
MSE = \frac{1}{n} \sum_{i=1}^{n} (y_i - \hat{y}_i)^2
\end{equation}

where \( y_i \) represents the actual value, \( \hat{y}_i \) is the predicted value, and \( n \) is the total number of samples.


\clearpage
\section{Prompt Design}
\label{sec:Prompt}

\begin{figure}[htbp]
  \centering
  
  \parbox{0.98\textwidth}{
        \rule{0.98\textwidth}{1.5pt} % The line
        Prompt for Threat Assessment \\[-1.5mm]
        \rule{0.98\textwidth}{0.8pt} % The line
        
        \text{[Threat Assessment]}\\
        \textbf{System Prompt} \\
        You are an expert at rational reasoning. \\[2mm]
        \textbf{User Prompt} \\
        Analyze the following scenario: A resident is deciding whether to evacuate during a wildfire. Based on their responses to a wildfire survey, provide a brief summary of the resident's threat assessment. Response to a wildfire survey: \textit{Survey}\\[-1.5mm]
        \rule{0.98\textwidth}{1.5pt} % The line
}\vspace{-2mm}
\caption{\mbox{The prompt used to generate threat assessment} } 
\label{fig:prompt_threat} 
\end{figure}

\vspace{-3mm}

\begin{figure}[htbp]
  \centering
  \parbox{0.98\textwidth}
  {
        \rule{0.98\textwidth}{1.5pt} % The line
        Prompt for Risk Perception\\[-1.5mm]
        \rule{0.98\textwidth}{0.8pt} % The line
    
        \text{[Risk Perception]}\\
        \textbf{System Prompt} \\
        You are an expert at rational reasoning. \\[2mm]
        \textbf{User Prompt} \\
        Consider the following scenario: A resident is deciding whether to evacuate during a wildfire. Based on their Threat Assessment and their responses to a wildfire survey, briefly summarize the resident's Risk Perception. Threat Perception is: \textit{Perception}. \\
        Response to a wildfire survey: \textit{Survey}.\\[-1.5mm]
        \rule{0.98\textwidth}{0.8pt} % The line
    }\vspace{-2mm}
\caption{\mbox{The prompt used to generate risk perception.} } 
\label{fig:prompt_risk} 
\end{figure}

% \clearpage
\vspace{-2mm}

\begin{figure}[!htbp]
  \centering
  \parbox{0.98\textwidth}{
        \rule{0.98\textwidth}{1.5pt} % The line
        Prompt for Evacuation Prediction \\[-1.5mm]
        \rule{0.98\textwidth}{0.8pt} % The line
    
        \text{[Making Decision]}\\
        \textbf{System Prompt} \\
        You are an advanced reasoning agent that can enhance your capabilities by reflecting on your own thought processes.\\[2mm]
        \textbf{User Prompt} \\
        You have access to a post-wildfire survey completed by local residents who experienced a specific wildfire event. Your task is to generate a logical, step-by-step chain of thought to infer whether the resident evacuated during the wildfire. Ensure each step is clearly connected. You must conclude with a definitive YES or NO answer regarding whether the resident evacuated. You will be provided with previous successful examples that have similar information. You may reference the rationale from these examples in your analysis.\\
        Previous Examples: \textit{Examples}\\
        Risk Perception Summary: \textit{Risk}.\\
        External information: \textit{Extras}\\[2mm]
        \textbf{Re-flexion Prompt} \\
        During the fire, this resident \textit{label} from the wildfire. Please reconsider and rethink the original questions to provide another clear and logical rationale on why the resident \textit{Label}:\\[-1.5mm]
        \rule{0.98\textwidth}{0.8pt} % The line

}\vspace{-2mm}
\caption{\mbox{The complete CoT used of evacuation decision prediction.} } 
\label{fig:prompt_full} 
\end{figure}

\clearpage

\section{More result}
\label{appx:more_result}
\begin{table*}[htbp]
\centering
\caption{Overview of Diffusion Models for Tabular Data. The column ``Num" indicates whether it can handle numerical features (`MM' = min-max scaler, 'QT'= quantile transformer, `ZS' = $Z$-score normalization), while ``Cat" indicates whether it can handle categorical features (`OH' = one-hot encoding, `AB'= analog bits encoding, `DC' = dictionary encoding,`DE' = dummy encoding, `IE' = Integer Encoding, `OE' = Ordinal Encoding, `PSK' = PSK encoding, 'FT' = feature tokenization or learned embedding). ``Complete" indicates whether the model requires complete data (i.e., without missing values) to train.}
\label{SummaryPrivacyDataSynthesis}
\resizebox{\linewidth}{!}{
\begin{tabular}{@{}llllllllllll@{}}
\toprule
\textbf{Model Name} & \textbf{Year} & \textbf{Venue} & \textbf{Num} & \textbf{Cat}  & \textbf{Tasks} & \textbf{Backbone} & \#\textbf{Datasets} & \textbf{Metrics} & \textbf{Code} & \textbf{Complete} & \textbf{Domain}\\ 
\midrule
\textbf{SOS} \cite{kim2022sos} & 2022         & KDD & \checkmark & \ding{55} & Synthesis & SDEs & 6 & Utility & Github & \checkmark & Generic\\
\textbf{STaSy} \cite{kimstasy}              & 2023         & ICLR           & \checkmark (MM) & \checkmark (OH) & Synthesis & SDEs & D1 D2 D3 & Fidelity, Utility, Diversity & Github & \checkmark & Generic\\
\textbf{TabDDPM} \cite{kotelnikov2023tabddpm} & 2023         & ICML & \checkmark (QT) & \checkmark (OH) & Synthesis & DDPM+MLD & D1 D2 D3 & Fidelity, Utility, Privacy & Github & \checkmark & Generic\\
 \textbf{CoDi} \cite{lee2023codi}               & 2023         & ICML & \checkmark (MM) & \checkmark (OH) & Synthesis & DDPM+MLD & D1 D2 D3 & Utility, Diversity & Github & \checkmark & Generic\\ 
 \textbf{AutoDiff} \cite{suh2023autodiff} & 2023         & NeurIPSW & \checkmark  & \checkmark  & Synthesis & Any & D1 D2 D3 & Fidelity, Utility, Privacy & Github & \checkmark & Generic\\ 
 \textbf{MissDiff} \cite{ouyang2023missdiff}               & 2023         & ICMLW  & \checkmark (MM) & \checkmark (OH) & Synthesis & SDEs & D1 D2 D3 & Fidelity, Utility & Github & \ding{55} & Generic\\

 \textbf{TabSyn} \cite{zhangmixed}              & 2024         & ICLR    & \checkmark & \checkmark (OH) & Synthesis, Imputation & SDEs & D1 D2 D3 & Fidelity, Utility, Diversity, Privacy & Github & \checkmark & Generic\\
 \textbf{Forest-Diffusion} \cite{jolicoeur2024generating} & 2024 & AISTATS  & \checkmark & \checkmark (DE) & Synthesis, Imputation & SDEs & D1 D2 D3 & Fidelity, Diversity, Utility & Github & \ding{55} & Generic\\
 \textbf{TabDiff} \cite{shi2024tabdiff}            & 2024         & NeurIPSW & \checkmark (MM) & \checkmark (OH) & Synthesis, Imputation & SDEs+MSD & D1 D2 D3 & Fidelity, Diversity, Utility & Github &  \checkmark & Generic\\
 \textbf{TabUnite} \cite{si2024tabunite}            & 2024         & OpenReview   &  \checkmark (QT) & \checkmark (AB, PSK, DC) & Synthesis & SDEs+MSD & D1 D2 D3 & Fidelity, Diversity, Utility & Github &  \checkmark & Generic\\
\textbf{CDTD} \cite{mueller2024continuous}& 2024         & OpenReview   &  \checkmark & \checkmark (FT) & Synthesis & Latent SDEs & D1 D2 D3 & Fidelity, Utility, Privacy & Github &  \checkmark & Generic\\
 \textbf{MedDiff} \cite{he2023meddiff}  & 2023 & ArXiv & \checkmark  & \ding{55} & Synthesis & DDIM & D1 D2 D3 & Fidelity, Utility & Github &  \checkmark & Healthcare\\
\textbf{EHR-TabDDPM} \cite{ceritli2023synthesizing}  & 2023 & ArXiv &\checkmark (QT)  & \checkmark (OH) & Synthesis & DDPM & D1 D2 D3 & Fidelity, Utility, Privacy & Github &  \checkmark & Healthcare\\
 \textbf{DPM-EHR}* \cite{nicholas2023synthetic}           & 2023         & NeurIPSW  &\checkmark  & \checkmark (OH) & Synthesis & DDPM & D1 D2 D3 & Fidelity, Diversity, Utility, Privacy & Github &  \checkmark & Healthcare\\
\textbf{FlexGen-EHR} \cite{he2024flexible} & 2024         & ICLR           &\checkmark + TS (FT)  & \checkmark (FT) & Synthesis & DDPM & D1 D2 D3 & Fidelity, Utility, Privacy & Github &  \ding{55} & Healthcare\\
\textbf{EHRDiff} \cite{yuan2024ehrdiff} & 2024         & TMLR          &\checkmark (MM)   & \checkmark (OH)  & Synthesis & SDEs & D1 D2 D3 & Fidelity, Utility, Privacy & Github &  \checkmark & Healthcare \\
\textbf{EHR-D3PM} \cite{han2024guided}            & 2024         & ArXiv   &\ding{55}  & \checkmark (OH)   & Synthesis & D3PM & D1 D2 D3 & Fidelity, Utility, Privacy & Github &  \checkmark & Healthcare \\
\textbf{FinDiff} \cite{sattarov2023findiff}          & 2023         & ICAIF          &\checkmark (ZS)   & \checkmark (OH)   & Synthesis & DDPM & D1 D2 D3 & Fidelity, Utility, Privacy & Github &  \checkmark & Finance \\
\textbf{EntTabDiff} \cite{liu2024entity}            & 2024         & ICAIF   &\checkmark (ZS)   & \checkmark (OH)   & Synthesis & DDPM & D1 D2 D3 & Fidelity, Utility, Privacy & Github &  \checkmark & Finance \\
\textbf{Imb-FinDiff} \cite{schreyer2024imb}            & 2024         & ICAIF   &\checkmark (ZS)   & \checkmark (OH)   & Synthesis & DDPM & D1 D2 D3 & Fidelity, Utility & Github &  \checkmark & Finance \\
\textbf{ClavaDDPM} \cite{pang2024clavaddpm}             & 2024         & NeurIPS           &\checkmark   & \checkmark(IE)   & Multi-Relational Synthesis & DDPM & D1 D2 D3 & Fidelity, Diversity, Utility, Dependency & Github &  \checkmark & Generic \\

\textbf{GNN-TabSyn} \cite{hudovernik2024relational}  & 2024         & NeurIPSW             &\checkmark   & \checkmark(IE)   & Multi-Relational Synthesis & DDPM & D1 D2 D3 & Fidelity, Utility, Privacy & Github &  \checkmark & Generic \\
\hline
\textbf{TabCSDI} \cite{zheng2022diffusion}  & 2023         & NeurIPSW            & \checkmark & \checkmark (OH, AB, FT) & Imputation & Conditional DDPM & D1 D2 D3 & Accuracy & Github & \ding{55} & Generic\\
\textbf{TabDiff} \cite{shi2024tabdiff}            & 2024         & NeurIPSW    & \checkmark & \checkmark  & Imputation & Conditional DDPM & D1 D2 D3 & Accuracy & Github & \ding{55} & Generic\\
\textbf{SimpDM} \cite{liu2024self}            & 2024         & CIKM   & \checkmark (QT) & \checkmark (OH)  & Imputation & DDPM+MLD & D1 D2 D3 & Accuracy & Github & \ding{55} &  Generic \\
\textbf{MTabGen} \cite{villaizan2024diffusion}        & 2024         & ArXiv   & \checkmark (QT) & \checkmark (OE)  & Imputation & DDPM+MLD & D1 D2 D3 & Utility & Github & \ding{55} &  Generic \\
\textbf{DDPM-Perlin} \cite{wibisono2024natural}            & 2024         & KBS    & \checkmark & \ding{55}  & Imputation & DDPM & D1 D2 D3 & Accuracy & Github & \ding{55} &  Generic \\
\textbf{DiffPuter} \cite{zhang2024unleashing}            & 2024         & OpenReview   & \checkmark & \checkmark (OH)  & Imputation & DDPM & D1 D2 D3 & Accuracy & Github & \ding{55} &  Generic \\
\hline
\textbf{SiloFuse} \cite{shankar2024silofuse}  & 2024         & ICDE  & \checkmark (FT) & \checkmark (FT)  & Trustworthy Synthesis & Latent DDPM & D1 D2 D3 & Fidelity, Utility, Privacy & Github & \checkmark &  Generic \\
\textbf{FedTabDiff} \cite{sattarov2024fedtabdiff}  & 2024         & ArXiv   & \checkmark (QT) & \checkmark (FT) & Trustworthy Synthesis & DDPM & D1 D2 D3 & Fidelity, Utility, Privacy & Github & \checkmark &  Generic \\
\textbf{FairTabDDPM}* \cite{yang2024balanced}            & 2024         & ArXiv   & \checkmark (QT) & \checkmark (OH) & Trustworthy Synthesis & DDPM & D1 D2 D3 & Fidelity, Diversity, Utility, Privacy, Fairness & Github & \checkmark &  Generic \\
\textbf{DP-Fed-FinDiff} \cite{sattarov2024differentially}            & 2024         & ArXiv   & \checkmark (QT) & \checkmark (FT) & Trustworthy Synthesis & DDPM & D1 D2 D3 & Fidelity, Utility, Privacy & Github & \checkmark &  Finance \\
\hline
\textbf{TabADM} \cite{zamberg2023tabadm}  & 2023         & ArXiv & \checkmark & \ding{55} & Anomaly Detection & DDPM & D1 D2 D3 & Accuracy & Github & \checkmark & Generic\\
\textbf{DTE} \cite{livernoche2024diffusion}  & 2024         & ICLR & \checkmark (ZS) & \ding{55}  & Anomaly Detection & DDPM & D1 D2 D3 & Accuracy & Github & \checkmark & Generic\\
\textbf{SDAD} \cite{li2024self}  & 2024         & Inf. Sci. & \checkmark  & \ding{55}  & Anomaly Detection & Latent DDPM & D1 D2 D3 & Accuracy & Github & \checkmark & Generic\\
\textbf{NSCBAD} \cite{anonymous2024anomaly} & 2024         & OpenReview & \checkmark  & \ding{55}  & Anomaly Detection & SDEs & D1 D2 D3 & Accuracy & Github & \checkmark & Generic \\
\textbf{FraudDiffuse} 
\cite{roy2024frauddiffuse}  & 2024         & ICAIF & \checkmark   & \checkmark (FT)  & Anomaly Detection, Synthesis & Latent DDPM & D1 D2 D3 & Utility & Github & \checkmark & Finance \\
 \textbf{FraudDDPM} \cite{pushkarenko2024synthetic}  & 2024         & ISIJ & \checkmark  & \checkmark (OH, FT)  & Anomaly Detection, Synthesis & DDPM & D1 D2 D3 & Utility & Github & \checkmark & Finance \\
\bottomrule
\end{tabular}%
}
\label{tab:tabular_data_synthesis}
\end{table*}

The accuracy of heatmaps for risk perceptions and threat assessment predictions reveals key trends in the model’s performance. Overall, the LLM demonstrates moderate accuracy, with better performance in predicting mid-range values (scores 2–4) while struggling with extreme values (scores 1 and 5). For instance, in risk perception prediction, the model performs best when the actual values are within the 2–4 range, with the highest accuracy (80\%) observed when the actual risk perception is 5, but the model predicts 3, indicating a systematic underestimation of extreme risk perceptions. Similarly, in threat assessment prediction, the model achieves its highest accuracy (50\%) when the actual threat assessment is 1, frequently predicting 2 instead. This pattern suggests that the model is biased toward moderate assessments and struggles to distinguish individuals with extremely high or low-risk perceptions or threat assessments. 

This finding suggests that while the LLM can generate reasonable approximations of threat assessment and risk perception (which are components of mental states), it struggles with capturing the extreme values that often drive actual evacuation decisions
In real-world scenarios, individuals who perceive very high risks are more likely to evacuate, whereas those with very low perceived risks may ignore warnings entirely. However, the model systematically underestimates these extremes, favoring moderate scores instead. This suggests that although it can infer general reasoning patterns, it does not fully capture the high-stakes decision-making process that translates perceptions into action. These findings align with previous research indicating that LLMs perform well in predicting human mental states but have difficulty translating those inferences into precise behavioral predictions.
The model’s tendency to underestimate extreme scores suggests the need for further calibration, like incorporating the evacuation behavioral model, reinforcement learning, and contextual variables during inferences, which is what we did in this paper.

\begin{figure*}[htbp]
    \centering
    \begin{subfigure}{0.48\textwidth}
        \centering
        \includegraphics[width=\linewidth]{figures/threat.png}
        \caption{Accuracy map of predicted and actual threat indicator values}
        \label{fig:left}
    \end{subfigure}
    \hfill
    \begin{subfigure}{0.48\textwidth}
        \centering
        \includegraphics[width=\linewidth]{figures/Risk.png}
        \caption{Accuracy map of predicted and actual risk indicator values}
        \label{fig:right}
    \end{subfigure}
    \caption{Accuracy map tested on Marshall and Kincade dataset using claude-3.5}
    \label{fig:subfig_example}
\end{figure*}

\clearpage


\section{Geographic Information of Wildfire Events}
\label{sec:geo_info}

\subsection{Carr Fire}

On July 23, 2018, the Carr Fire ignited in Shasta County, California, when sparks from a vehicle’s flat tire set nearby dry vegetation ablaze. 

\begin{figure}[htbp]
    \centering
    \includegraphics[width=0.65\linewidth]{figures/carr_area.png}
    \caption{The damage map from the Carr Fire~\cite{schleuss2018carrfiremap}}
    \label{fig:carr_area}
\end{figure}

By August 30, 2018, the Carr Fire had been fully contained after burning over 229,000 acres, destroying approximately 1,600 structures, forcing the evacuation of around 39,000 people, claiming eight lives, and inflicting an estimated \$1.5 billion in damages. The affected area map is shown in Figure \ref{fig:carr_area}.
As it advanced rapidly to the east, the fire prompted the evacuation of French Gulch, Old Shasta, and Keswick, and worsening conditions led officials to evacuate several urban neighborhoods in Redding. Furthermore, the Carr Fire jumped the Sacramento River—partly due to fire whirls induced by the wildfire system. Ultimately, the combined efforts of 4,500 firefighting personnel and favorable weather conditions slowed its progression through Redding and surrounding rural communities, leading to its eventual containment at the end of August 2018~\cite{wong2020review}.

\subsection{Kincade Fire}
On October 23, 2019, at 9:27 P.M., the Kincade Fire ignited northeast of Geyserville in Sonoma County, California, and was ultimately contained on November 6, 2019, at 7:00 P.M. As the largest wildfire of the 2019 California season, it burned 77,758 acres, damaged 60 structures, completely destroyed 374 structures, and injured four individuals. The event prompted the evacuation of more than 186,000 people—the largest evacuation in Sonoma County’s history. To manage this process, emergency officials partitioned the county into designated zones, issuing a mandatory evacuation order in Geyserville on October 26, followed by subsequent orders and warnings extending to areas along the Pacific Ocean and northern sections of Santa Rosa. Figure \ref{fig:kincade_area} illustrates the wildfire’s spatial impact, the delineated evacuation zones, and additional key fire parameters~\cite{SUN2024106557}.

\begin{figure}[htbp]
    \centering
    \includegraphics[width=0.65\linewidth]{figures/kincade_area.jpg}
    \caption{The wildfire impact area and evacuation area map of Kincade Fire~\cite{SUN2024106557}}
    \label{fig:kincade_area}
\end{figure}

\subsection{Marshall Fire}

The Marshall Fire ignited shortly before 10:30 a.m. on December 30, 2021, in Boulder County, Colorado, from two distinct ignition points. 
\begin{figure}[htbp]
    \centering
    \includegraphics[width=0.75\linewidth]{figures/Marshall_area.jpg}
    \caption{The wildfire impact area and evacuation area map of Marshall Fire~\cite{forrister2024analyzing}}
    \label{fig:marshall_area}
\end{figure}

From these points, the fire rapidly spread into several suburban areas across Boulder, Jefferson, and Adams Counties, affecting counties such as Louisville, Superior, Broomfield, Lafayette, Arvada, and Westminster. The preceding unusually humid spring, followed by a warm, dry summer and fall, produced dry fuel conditions that, combined with strong dry winds, accelerated the fire’s spread. Recognized as the most destructive wildfire in Colorado’s history, the Marshall Fire burned over 6,200 acres, destroyed 1,084 homes, damaged 149 additional homes, and resulted in two fatalities. In Boulder County alone, the estimated residential damage exceeded \$513 million. On the day of the fire, more than 30,000 residents in Boulder, Jefferson, and Adams Counties were evacuated. Figure \ref{fig:marshall_area} illustrates the affected evacuation and pre-evacuation zones for these communities~\cite{forrister2024analyzing}.

\clearpage
\section{Weight Distribution for Different Perceptions} 
\label{appx:weight_dis}

\begin{figure*}[htbp]
    \centering
    \begin{subfigure}{0.48\textwidth}
        \centering
        \includegraphics[width=\linewidth]{figures/threat11.png}
        \caption{Weight distribution of variables for Threat Assessment (Injury)}
        \label{fig:subfig1}
    \end{subfigure}
    \hfill
    \begin{subfigure}{0.48\textwidth}
        \centering
        \includegraphics[width=\linewidth]{figures/threat22.png}
        \caption{Weight distribution of variables for Threat Assessment (Death)}
        \label{fig:subfig2}
    \end{subfigure}
    
    \vspace{0.2cm} 
    
    \begin{subfigure}{0.48\textwidth}
        \centering
        \includegraphics[width=\linewidth]{figures/risk11.png}
        \caption{Weight distribution of variables for Risk Perception (Home)}
        \label{fig:subfig3}
    \end{subfigure}
    \hfill
    \begin{subfigure}{0.48\textwidth}
        \centering
        \includegraphics[width=\linewidth]{figures/risk22.png}
        \caption{Weight distribution of variables for Risk Perception (Neighborhood)}
        \label{fig:subfig4}
    \end{subfigure}
    \caption{
     Examples of weight distributions from the logistic regression model are used to symmetrically select variables for each specific perception. The clear elbow points in the figure support the chosen threshold for variable selection.}
    \label{fig:main_figure}
\end{figure*}

To ensure an objective selection of key variables for each perception type, we use a logistic regression model to derive variable weights and identify an appropriate cutoff threshold. Figure \ref{fig:main_figure} illustrates the weight distribution trends for each perception category: Threat Assessment (Injury and Death) and Risk Perception (Home and Neighborhood).

Each plot displays a sharp decline in variable importance, followed by a gradual flattening, indicating the presence of an elbow point. This elbow point serves as the threshold for variable selection, ensuring that the most influential variables are retained while filtering out less significant ones.

\clearpage
\section{Selected Questions for Different Perceptions} 
\label{appx:select_q}
The selection of specific wildfire survey questions based on PADM for each threat assessment and risk perception reflects different reasoning processes in evacuation decision-making. 
An example The selected variables (i.e questions) are detailed in Table~\ref{tab:threat_questions} and Table~\ref{tab:risk_questions}

The questions in the two types of threat assessment reflect two distinct aspects. The first type is driven by direct sensory input—whether individuals observed flames or embers—and their subjective assessment of wildfire likelihood. Factors like residency duration influence familiarity with local fire risks. In contrast, the second type incorporates external cues such as warnings from social networks and educational background. Residents receiving evacuation advice from acquaintances or managing livestock may prioritize economic and logistical concerns alongside personal safety.

The questions in risk perception also follow two distinct patterns. The first emphasizes immediate physical harm, shaped by health conditions, household demographics, and emergency communications. Those with medical conditions or older adults in the household may perceive higher injury risk, while direct evacuation orders heighten urgency. The second type focuses on long-term preparedness, considering financial stability, employment, and proactive fire mitigation efforts. Residents with emergency plans or prior protective measures may perceive lower risk due to a greater sense of control.

Overall, threat assessment is reactive, shaped by real-time environmental and social cues, whereas risk perception is anticipatory, centered on future consequences and preparedness. Structuring these perceptions into distinct reasoning pathways enables LLMs to model diverse decision-making profiles more effectively, improving accuracy and interpretability in wildfire evacuation predictions.



\begin{table*}[htbp]
    \centering
    \caption{Example of the Selected Survey Questions based on PADM from the survey data for Perceptions inference.}
    \label{tab:threat_questions}
    \begin{tabular}{p{0.15\linewidth} | p{0.8\linewidth}}
        \toprule
        \textbf{Category} & \textbf{Survey Question} \\
        \midrule
        \multirow{12}{*}{\makecell{Threat \\ Assessment \\ (Injury)}} 
        & 1. What was your immediate reaction to observing the flames or embers (or both)? (Select only one) \\
        & 2. Before you decided to evacuate (or stay), did you see, hear, or feel flames or embers in your immediate vicinity (that is, your neighborhood)? \\
        & 3. Before the Kincade fire, how would you have described the possibility that a wildfire would threaten your property, on a scale from 1 to 5, where 1 signifies not at all likely and 5 signifies very likely? \\
        & 4. What day did the emergency official(s) first let you know? \\
        & 5. Before you decided to evacuate (or stay), did someone you know tell you to evacuate or that a mandatory evacuation order was issued for your area? \\
        & 6. What was your immediate reaction when the emergency official(s) first let you know? (Select only one) \\
        & 7. Did you or someone in your household, including yourself, have a medical condition at the time of the Kincade fire? \\
        & 8. What time did the emergency official(s) first let you know? \\
        & 9. How long had you lived at that residence? (Select only one) \\
        & 10. Before the Kincade fire, did you know that wildfires could be a problem in your community? \\
        & 11. What were the ways people told you to evacuate or that your area was under a mandatory evacuation order? (Select all that apply) \\
        & 12. How old are you? (Please enter your age at the time you are taking this survey, below) \\
        \midrule
        \multirow{9}{*}{\makecell{Threat \\ Assessment \\ (Death)}} 
        & 1. What was your immediate reaction to observing the flames or embers (or both)? (Select only one) \\
        & 2. What was your immediate reaction when the emergency official(s) first let you know? (Select only one) \\
        & 3. Before the Kincade fire, how would you have described the possibility that a wildfire would threaten your property, on a scale from 1 to 5, where 1 signifies not at all likely and 5 signifies very likely? \\
        & 4. Before you decided to evacuate (or stay), did you see, hear, or feel flames or embers in your immediate vicinity (that is, your neighborhood)? (Mark all that apply) \\
        & 5. How old are you? (Please enter your age at the time you are taking this survey, below) \\
        & 6. Before you decided to evacuate (or stay), did someone you know tell you to evacuate or that a mandatory evacuation order was issued for your area? \\
        & 7. What is the highest level of education you have completed? (Select only one) \\
        & 8. Did you or someone in your household, including yourself, have a medical condition at the time of the Kincade fire? \\
        & 9. How many livestock or other farm animals lived in your household (or on your property) at the time the Kincade fire started (on Wednesday, October 23, 2019)? \\
        \bottomrule
    \end{tabular}
\end{table*}

\begin{table*}[htbp]
    \centering
    \caption{Selected Survey Questions based on PADM for Risk Perception}
    \label{tab:risk_questions}
    \begin{tabular}{p{0.15\linewidth} | p{0.8\linewidth}}
        \toprule
        \textbf{Category} & \textbf{Survey Question} \\
        \midrule
        \multirow{14}{*}{\makecell{Risk \\ Perception \\ (Home)}} 
        & 1. What was your immediate reaction to observing the flames or embers (or both)? (Select only one) \\
        & 2. Before you decided to evacuate (or stay), did one or more emergency officials let you know that you had to evacuate immediately and/or that your area was under a mandatory evacuation order? \\
        & 3. Before the Kincade fire, how would you have described the possibility that a wildfire would threaten your property, on a scale from 1 to 5, where 1 signifies not at all likely and 5 signifies very likely? \\
        & 4. What was your immediate reaction when the emergency official(s) first let you know? (Select only one) \\
        & 5. Did you or someone in your household, including yourself, have a medical condition at the time of the Kincade fire? \\
        & 6. If yes, what was the ultimate goal of this household emergency plan? \\
        & 7. How many adults (including your own adult children): 18 years old to 64 years old lived in your household (or on your property) at the time the Kincade fire started (on Wednesday, October 23, 2019)? \\
        & 8. How long had you lived at that residence? (Select only one) \\
        & 9. What time did the emergency official(s) first let you know? \\
        & 10. What is the highest level of education you have completed? (Select only one) \\
        & 11. Before you decided to evacuate (or stay), did you see, hear, or feel flames or embers in your immediate vicinity (that is, your neighborhood)? (Mark all that apply) \\
        & 12. Do you consider yourself (gender)? \\
        & 13. What day did the emergency official(s) first let you know? \\
        & 14. What time did you notice the fire? \\
        \midrule
        \multirow{11}{*}{\makecell{Risk \\ Perception \\ (Neighborhood)}} 
        & 1. What was your immediate reaction to observing the flames or embers (or both)? (Select only one) \\
        & 2. What was your immediate reaction when the emergency official(s) first let you know? (Select only one) \\
        & 3. Before the Kincade fire, how would you have described the possibility that a wildfire would threaten your property, on a scale from 1 to 5, where 1 signifies not at all likely and 5 signifies very likely? \\
        & 4. Before you decided to evacuate (or stay), did one or more emergency officials let you know that you had to evacuate immediately and/or that your area was under a mandatory evacuation order? \\
        & 5. If yes, what was the ultimate goal of this household emergency plan? (Select only one) \\
        & 6. What time did the emergency official(s) first let you know? \\
        & 7. How many adults (including your own adult children): 18 years old to 64 years old lived in your household (or on your property) at the time the Kincade fire started (on Wednesday, October 23, 2019)? \\
        & 8. Did you or someone in your household, including yourself, have a medical condition at the time of the Kincade fire? \\
        & 9. What day did the emergency official(s) first let you know? \\
        & 10. How long had you lived at that residence? (Select only one) \\
        \bottomrule
    \end{tabular}
\end{table*}

\end{document}


