% This must be in the first 5 lines to tell arXiv to use pdfLaTeX, which is strongly recommended.
\pdfoutput=1
% In particular, the hyperref package requires pdfLaTeX in order to break URLs across lines.

\documentclass[11pt]{article}

% Change "review" to "final" to generate the final (sometimes called camera-ready) version.
% Change to "preprint" to generate a non-anonymous version with page numbers.
%\usepackage[review]{acl}
\usepackage[preprint]{acl}

% Standard package includes
\usepackage{times}
\usepackage{latexsym}
\usepackage{graphicx}
\usepackage{multirow}
\usepackage{amsmath,enumitem}
\usepackage{booktabs}

%\usepackage[table]{xcolor}
%\usepackage{xcolor}

% For proper rendering and hyphenation of words containing Latin characters (including in bib files)
\usepackage[T1]{fontenc}
% For Vietnamese characters
% \usepackage[T5]{fontenc}
% See https://www.latex-project.org/help/documentation/encguide.pdf for other character sets

% This assumes your files are encoded as UTF8
\usepackage[utf8]{inputenc}

% This is not strictly necessary, and may be commented out,
% but it will improve the layout of the manuscript,
% and will typically save some space.
\usepackage{microtype}

% This is also not strictly necessary, and may be commented out.
% However, it will improve the aesthetics of text in
% the typewriter font.
\usepackage{inconsolata}

%Including images in your LaTeX document requires adding
%additional package(s)
\usepackage{graphicx}

% If the title and author information does not fit in the area allocated, uncomment the following
%
%\setlength\titlebox{<dim>}
%
% and set <dim> to something 5cm or larger.

\usepackage{inconsolata}
\usepackage{mdframed}
\usepackage{makecell}
\usepackage{amssymb}

\usepackage{amsmath}
\usepackage{algorithm}
\usepackage{algpseudocode}

\usepackage{xspace}
\newcommand{\modelname}{BDC\xspace}

\newcommand{\weinan}[1]{{\color{blue}[\textbf{weinan: #1}]}}
\newcommand{\minisection}[1]{\vspace{0pt}\noindent\textbf{#1.}}
\newcommand{\dk}[1]{{\color{orange}[\textbf{dk: #1}]}}
\newcommand{\ljx}[1]{{\color{cyan}[\textbf{ljx: #1}]}}
\newcommand{\whj}[1]{{\color{brown}[\textbf{whj: #1}]}}

%\title{\modelname: Learning to Weight Over Disentangled Experts \\ for Robust Code Generation}

\title{Boost, Disentangle, and Customize: \\
A Robust System2-to-System1 Pipeline for Code Generation}


\author{Kounianhua Du$^1$, Hanjing Wang$^1$, Jianxing Liu$^1$, Jizheng Chen$^1$, \\
{\bf Xinyi Dai$^2$, Yasheng Wang$^2$, Ruiming Tang$^2$, Yong Yu$^1$, Jun Wang$^3$, Weinan Zhang$^1$}\\
  $^1$Shanghai Jiao Tong University, $^2$ Huawei Noah’s Ark Lab, $^3$ University College London \\
  Shanghai, China\\
  \texttt{\{kounianhuadu, wnzhang\}@sjtu.edu.cn}
  %\texttt{\{774581965,ruirenting,fatcat,fulingyue,yyu,wnzhang\}@sjtu.edu.cn}, \\ 
  %\texttt{\{xiawei24,wangyasheng,tangruiming\}@huawei.com}
  }

\begin{document}
\maketitle

% \begin{abstract}
% Large language models (LLMs) have demonstrated remarkable capabilities in many domains, yet their ability in System 2 tasks remain opaque. The inherent high-thinking demand and complex latent patterns of data make it hard for llms to accurately infer solutions for these tasks. And the simple train-once-for-all tuning mechanism may then fail under this situation. In this paper, we propose \textbf{\modelname} to \textbf{Disen}tangle the complex problem solving procedure and heterogeneous data distribution into meta-units and adaptively aggregate the resulting \textbf{Lora}-experts to generate problem solver for each problem instance. Concretely, 1) we disentangle the problem-solving stages into problem2thought and thought2solution processes and integrate the search process using a novel multi-expert MCTS algorithm, armed with reflextion-based pruning and refinement. 2) The resulting training data is thereafter disentangled into different meta clusters based on semantic distance, on which we finetune different lora experts capable of different levels of tasks.  3) Then, we train an input-aware hyper-network to adaptively aggregate the lora experts rank-wise for contextualized problem solver. Experiment results and various ablation studies validate the superiority of the data collection algorithm, the effectiveness of the data disentanglement process, and the performance gain brought by the input-aware hyper-network.
% \end{abstract}

%\begin{abstract}
Retrieval-Augmented Generation (RAG) is often used with Large Language Models (LLMs) to infuse domain knowledge or user-specific information. In RAG, given a user query, a retriever extracts chunks of relevant text from a knowledge base. These chunks are sent to an LLM as part of the input prompt. Typically, any given chunk is repeatedly retrieved across user questions. However, currently, for every question, attention-layers in LLMs fully compute the key values (KVs) repeatedly for the input chunks, as state-of-the-art methods cannot reuse KV-caches when chunks appear at arbitrary locations with arbitrary contexts. Naive reuse leads to output quality degradation.  This leads to potentially redundant computations on expensive GPUs and increases latency. In this work, we propose \sys, a system for managing and reusing precomputed KVs corresponding to the text chunks (we call \textit{chunk-caches}) in RAG-based systems. We present how to identify \hl{\textit{chunk-caches} that are reusable}, how to efficiently perform a small fraction of recomputation to \textit{fix} the cache to maintain output quality, and how to efficiently store and evict \textit{chunk-caches} in the hardware for maximizing reuse while masking any overheads. With real production workloads as well as synthetic datasets, we show that \sys reduces redundant computation by \textbf{51\%} over SOTA prefix-caching and \textbf{75\%} over full recomputation.
\hl{Additionally, with continuous batching on a real production workload, we get a \textbf{1.6$\times$} speedup in throughput and a \textbf{2$\times$} reduction in end-to-end response latency over prefix-caching while maintaining quality, for both the \llama-3-8B and \llama-3-70B models. 
}
\end{abstract}





%\section{Introduction}
\label{sec:intro}

\begin{figure*}[tb]
    \centering
    \includegraphics[width=0.848\linewidth]{figs/circuitnn.pdf} 
    \caption{Illustration of differentiable CircuitNN. CircuitNN is designed based on differentiable NAND gates. After DAS is guided by PI and PO pairs of the truth table, CircuitNN can get the precise circuit architecture logic equivalent to the truth table.}
    \label{fig:circuitnn}
\end{figure*}

% 1. Describe the importance of logic synthesis
% 2. Existing Problems
% (a) Neural Architecture Search: Unstable, Predefined Setting, etc.
% (b) Circuit Generation: Probabilistic Model, Logic Equivalence

With the rapid advancement of technology, the scale of integrated circuits (ICs) has expanded exponentially. 
This expansion has introduced significant challenges in chip manufacturing, particularly concerning power and area metrics.
A primary objective in IC design is achieving the same circuit function with fewer transistors, thereby reducing power usage and area occupancy.

Logic synthesis~\cite{hachtel2005logicsynth}, a critical step in electronic design automation (EDA), transforms behavioral-level circuit designs into optimized gate-level circuits, ultimately yielding the final IC layout. 
The primary goal of logic synthesis is to identify the physical implementation with the fewest gates for a given circuit function. 
This task constitutes a challenging NP-hard combinatorial optimization problem. 
Current logic synthesis tools~\cite{brayton2010abc, wolf2013yosys} rely on human-designed heuristics, often leading to sub-optimal outcomes.

Differentiable architecture search (DAS) techniques~\cite{liu2018darts, chu2020darts} offer novel perspectives on addressing challenges in this problem.
Circuit functions can be represented through truth tables, which map binary inputs to their corresponding outputs. 
Truth tables provide a precise representation of input-output relationships, ensuring the design of functionally equivalent circuits.
Inspired by this, researchers~\cite{deepmind2024ai4sys, wang2024tnet} have begun exploring the application of DAS to synthesize circuits directly from truth tables.
Specifically, \citet{deepmind2024ai4sys} proposed CircuitNN, a framework that learns differentiable connection structures with logic gates, enabling the automatic generation of logic circuits from truth tables.
This approach significantly reduces the complexity of traditional circuit generation. 
Building on this, \citet{wang2024tnet} introduced T-Net, a triangle-shaped variant of CircuitNN, incorporating regularization techniques to enhance the efficiency of DAS.

Despite these advancements, several challenges remain. 
The computational complexity of DAS grows quadratically with the number of gates, posing scalability issues.
Although triangle-shaped architecture~\cite{wang2024tnet} partially mitigates this problem, redundancy persists. 
%Additionally, DAS is susceptible to converging to local optima, limiting the ability to search architectures that satisfy the given truth tables~\cite{liu2018darts}. 
%Furthermore, hyperparameters (network depth and layer width) require extensive searches, introducing complexity and prolonging the synthesis process. 
Additionally, DAS is susceptible to converging to local optima~\cite{liu2018darts} and hyperparameters (network depth and layer width) require extensive searches. 
The challenges arise from the vast search space in DAS. 
% Even with predefined settings for CircuitNN, finding a configuration that meets the truth table requires extensive trial and error during the DAS process. 
Intuitively, limiting the search space through predefined parameters (network depth, gates per layer, and connection probabilities) can significantly reduce the complexity.

Recent advances~\cite{openai2023gpt4, abramson2024alphafold3, esser2024sd3, li2024mar} in conditional generative models have demonstrated remarkable performance across language, vision, and graph generation tasks. 
Motivated by these developments, we propose a novel approach to circuit generation that generates preliminary circuit structures to guide DAS in generating refined circuits matching specified truth tables. 
Firstly, we introduce CircuitVQ, a tokenizer with a discrete codebook for circuit tokenization. 
Built upon our Circuit AutoEncoder framework~\cite{hou2022graphmae,li2023maskgae,wu2025mgvga}, CircuitVQ is trained through a circuit reconstruction task. 
Specifically, the CircuitVQ encoder encodes input circuits into discrete tokens using a learnable codebook, while the decoder reconstructs the circuit adjacency matrix based on these tokens.
Subsequently, the CircuitVQ encoder serves as a circuit tokenizer for CircuitAR pretraining, which employs a masked autoregressive modeling paradigm~\cite{chang2022maskgit, li2023mage}. 
In this process, the discrete codes function as supervision signals. 
After training, CircuitAR can generate discrete tokens progressively, which can be decoded into initial circuit structures by the decoder of the CircuitVQ. 
These prior insights can guide DAS in producing refined circuits that match the target truth tables precisely.

Our key contributions can be summarized as follows:
\begin{itemize}
\item We introduce CircuitVQ, a circuit tokenizer that facilitates graph autoregressive modeling for circuit generation, based on our Circuit AutoEncoder framework;
\item Develop CircuitAR, a model trained using masked autoregressive modeling, which generates initial circuit structures conditioned on given truth tables;
\item Propose a refinement framework that integrates differentiable architecture search to produce functionally equivalent circuits guided by target truth tables;
\item Comprehensive experiments demonstrating the scalability and capability emergence of our CircuitAR and the superior performance of the proposed circuit generation approach.
\end{itemize}

% Motivation
% (a) Diffusion (Vision, Graph), Autoregressive (Language, Vision)
% (b) Circuit Generation for Predefined Setting
% (c) Neural Architecture Search for Strict Logic Equivalence

% Contribution
% (a) Circuit Tokenizer (new transformer arch, training strategy)
% (b) CircuitAR (train and gen strategies, post-ar strategy)
% (c) Extensive Evaluation including BitD (Bit Distance) for Scalability



%
\section{Related Work} \label{sec:related}

% \textbf{Adversarial Attack}
\textbf{Attacks on SLAM.} 
%With the rise of machine learning, 
The robustness of computer vision systems is being actively investigated. With the emergence of adversarial images in the digital domain by adding optimized noise directly to images~\cite{szegedy2013intriguing,carlini2017towards}, researchers find that such attacks also exist physically in the real world \cite{eykholt2018robust,song2018physical,zhao2019seeing}. To fill the gap between attacks in the digital and physical worlds, recent studies have demonstrated that attacks on real-world computer vision systems are practical \cite{eykholt2018robust,li2019adversarial,man2020ghostimage,sharif2016accessorize,zhao2019seeing,zhou2018invisible}. However, attacks on traditional computer vision methods such as SLAM are relatively less explored. \cite{yoshida2022adversarial} proposes an attack against the scan matching algorithm in LiDAR-based SLAM, while most SLAMs in AR/VR devices rely on different sensors like RGB/depth cameras and IMUs. \cite{ikram2022perceptual} and \cite{chen2024adversary} mislead visual SLAM by poisoning the images with special patterns, and \cite{wang2021can} causes the camera to fail using infrared light. In our work, we demonstrate attacks on Visual-Inertial SLAM (VI-SLAM) by perturbing the IMU readings, rather than cameras, and showing its impact on XR user experience. 

\textbf{Acoustic Injection Attacks.} Among various physical attacks, acoustic injection attacks are attractive due to their low cost. Son~\etal~\cite{son2015rocking} were the first to introduce acoustic attacks on MEMS gyroscopes, demonstrating how these attacks could lead to sensor denial-of-service and result in drone crashes. WALNUT~\cite{trippel2017walnut} expanded on this by developing output biasing and control attacks that enable precise manipulation of MEMS accelerometer outputs using modulated sound waves. Wang et al.~\cite{wang2017sonic} demonstrated a sonic gun, showcasing the vulnerability of various smart devices (\eg drones and self-balancing vehicles) to acoustic attacks. Tu et al. \cite{tu2018injected} designed side-swing and switching attacks to alter the outputs of MEMS gyroscopes and accelerometers. Furthermore, Ji et al. \cite{ji2021poltergeist} fool the object detectors by applying acoustic attack to the image stabilizers commonly used in modern cameras. However, none of the existing works study the relationship between the acoustic injections and SLAM outputs on recent XR devices. 

% \zijian{Do we need one session about security in AR/VR?}
% \yicheng{TODO}
%\jiasi{cite the AIVR paper (UMass Amherst?) paper is we have not already. They add IMU perturbation but w/o SLAM, iirc} \yicheng{Cited}

\textbf{XR Security and Privacy.} 
%Security and privacy concerns in XR systems have gained significant attention. 
For single-user XR systems, researchers have demonstrated various side-channel attacks to extract sensitive information (\eg keystrokes) through video feeds~\cite{ling2019know}, head movements~\cite{nair2023unique, slocum2023going}, architectural hints~\cite{zhang2023its,shang2020arspy}, power usage~\cite{li2024dangers}, and EM side-channel leakages~\cite{al2021vr}. In multi-user XR systems, Su et al.~\cite{su2024remote} use avatar motion data to infer keystrokes in shared VR environments. Slocum et al.~\cite{slocum2024doesn} reveal vulnerabilities in the shared state frameworks of multi-user AR. Similarly, Lebeck et al.~\cite{lebeck2017securing} highlight risks like deceptive virtual objects and emphasize access control for managing shared physical and virtual spaces. Ruth et al.~\cite{ruth2019secure} further propose a secure multi-user AR framework focusing on content sharing and permissions.
Chandio et al.~\cite{chandio2024stealthy} %introduced a multi-modal spatiotemporal attack that 
simultaneously manipulated visual and inertial sensors to disrupt XR pose estimation. However, their study evaluated the attack using offline datasets and assumed the attacker's capability to manipulate IMU data streams through acoustic means, without real experiments. Ours is the first to demonstrate acoustic injection attacks on recent XR devices, like the Hololens 2, in the real world.
 


%\section{Preliminary}

\paragraph{Notation} Consider a sentence of $T$ tokens $\vx=\{\vx_1,\ldots, \vx_T\}\in\gX$, and let $P$ be the unknown target language distribution, $\tilde P(\vx)$ be the empirical distribution of the training data (which is an approximation of $P$), and $Q$ be the distribution of our model at hand. Since our paper is also closely related to RLHF, we will also use $\pi$ to represent the distributions. In particular, we sometimes write $\pi_\theta$ for a distribution that is parameterized by $\theta$, where $\theta$ is usually the set of trainable parameters of the LLM; we write $\pr$ for a reference distribution that should be clear given the context. The next token prediction loss is minimizing the forward-KL between $P$ and $Q$. 




%% \begin{figure}
%     \centering
%     \includegraphics[width=0.5\linewidth]{Move_teaser.pdf}
%     \caption{Comparison of different dynamic compute approaches. length of arrow indicates residual transformation per token while width indicates velocity of transformation.}
%     \label{fig:enter-label}
% \end{figure}

\section{Method}
\label{sec:method}
Residual connections play a crucial role in shaping token representations, yet their dynamics remain underexplored in the context of efficient decoding. In this work, we delve deeper into transformer residual dynamics and investigate how modulating residual transformation velocity can improve inference efficiency in token-level processing, optimizing both dense and sparse MoE transformers.


\subsection{Residual Dynamics and Motivation for Multi-rate Residuals} \label{sec:motivation}

To analyze how hidden representations evolve across different layers of a transformer architecture, it's crucial to consider the effect of residual connections. Each transformer decoder layer typically has residual connections across attention and MLP submodules. As the residual stream $h_i$ traverses from interval $E_j$ to $E_{j+1}$, it undergoes a residual transformation given by:  
% \begin{equation}
% \label{eq:slow_residual_transformation}
% H_{E_{j+1}} = H_{E_j} \prod_{i=E_j}^{E_{j+1}} \left( I + \mathcal{A}_i \right) \left( I + \mathcal{M}_i \right) \quad \text{where} \quad \mathcal{A}_i = f(c_i, h_{i}), \mathcal{M}_i = g(h_i)
% \end{equation}

\begin{equation} \label{eq:slow_residual_transformation}
h_{E_{j+1}} = h_{E_j} + \sum_{i=E_j}^{E_{j+1}-1} \left( \mathcal{A}_i(h_i) + \mathcal{M}_i(h_i + \mathcal{A}_i(h_i)) \right) \quad \text{where} \quad \mathcal{A}_i = f(c_i, h_{i}), \mathcal{M}_i = g(h_i). 
\end{equation}

Here, \( \mathcal{A}_i \) denotes the non-linear transformation introduced by the multi-head attention mechanism at layer \( i \), while \( \mathcal{M}_i \) corresponds to the non-linear transformation of the MLP block at the same layer. These transformations depend on the input residual stream \( h_i \) and, in the case of \( \mathcal{A}_i \), the previous contextual representation \( c_i \).\footnote{Normalization layers are typically applied in practice but are omitted here for simplicity of the argument.}


% For easy tokens, the magnitude and direction of this delta transformation become progressively smaller with each successive layer as shown in \cref{fig:delta_transformation}. Consequently, it is feasible to predict these tokens after only a few residual connections, whereas harder tokens necessitate more extensive processing through additional layers.

\begin{figure}[ht]
    \centering
    \begin{subfigure}{0.48\textwidth}
        \centering
        \includegraphics[width=\textwidth]{sections/figures/residual_change.pdf}
        \caption{}
        \label{fig:residual_change}
    \end{subfigure}%
    \hfill
    \begin{subfigure}{0.48\textwidth}
        \centering
        \includegraphics[width=\textwidth]{sections/figures/alignment_wrt_dedicated_model.pdf}
        \caption{}
    \label{fig:alignment_wrt_dedicated_model}
    \end{subfigure}
    \caption{(a) As residual streams propagate through the model, the directional shifts in the residuals become progressively smaller. (b) A dedicated model with $k$ layers achieves a faster rate of change in residual streams and higher alignment than base model leveraging early exit mechanisms at layer $k$.}
    \label{fig}
\end{figure}


To examine whether residual transformations can be accelerated across layers, we conducted experiments using a diverse set of prompts on a pre-trained Phi3 model~\cite{phi3_report}. As illustrated in \cref{fig:residual_change}, we measured the directional shift in residual states as \( 1 - \mathcal{C}(h_{i-1}, h_i) \), where \(\mathcal{C}\) denotes normalized cosine similarity. This shift is notably higher in the initial layers, gradually decreasing in subsequent layers. This behavior allows traditional early exit approaches to effectively accelerate decoding by enabling earlier exits for simpler tokens. However, these approaches typically rely on a distance-based approximation, where the full residual transformation of the model is approximated by the residual transformations of the initial layers. To gain deeper insights into the distance versus velocity aspects of residual transformation, we conducted a comparative study. Specifically, we trained an early exit head at layer $k$ of the Phi3 model, which consists of 32 layers, restricting the distance traveled by each token. To accelerate the residual transformation relative to number of layers, we trained a smaller model consisting of only $k$ layers, while keeping all other hyperparameters consistent. We then compared the next-token prediction accuracy of the early exit head of the base model with that of the smaller model. To ensure an equal number of trainable parameters, we inserted low-rank adapters into the smaller model and trained only these adapters, whereas, in the distance-based approach, we trained solely the early exit head. In addition, to accelerate the residual transformation in smaller model, we distilled the residual streams from the larger model by incorporating a distillation loss ~\cite{sanh2019distilbert} between the residual state at layer \(i\) of the smaller model and the residual state at layer \(4 \times i\) of the larger model. As shown in ~\cref{fig:alignment_wrt_dedicated_model} the smaller model demonstrates a significantly faster rate of change in residual streams, leading to higher next token prediction accuracy after $k$ layers compared to the base model that employs traditional early exit mechanisms after $k$ layers \cite{schuster2022confident, chen2023eellm, varshney-etal-2024-investigating}. This experimental setup, which modifies only the rate of change in residual streams while keeping other factors constant, suggests that dense transformers, trained with a fixed number of layers, may inherently possess a slow residual transformation bias.

This observation raises an intriguing question: if the rate of change in residual streams could be accelerated relative to the number of layers, is it possible to facilitate earlier alignment for a greater proportion of tokens? Earlier alignment would be beneficial to not only facilitate dynamic computation but also for generating speculative tokens efficiently with high acceptance rates in speculative decoding setups ~\cite{leviathan2023fast, chen2023accelerating}. 

%thereby enhancing the efficiency of early exiting? 
 % This bias likely constrains the effectiveness of early exiting, particularly for easier tokens. By addressing this limitation through accelerated residual transformations, we hypothesize that it is possible to substantially improve the efficiency and accuracy of early exit strategies in transformer models.

\subsection{Multi-Rate Residual Transformation} \label{m2r2_method}

To address the slow residual transformation bias described in ~\cref{sec:motivation}, we introduce \textit{accelerated residual streams} that operate at rate $R$ relative to original slow residual stream. We pair slow residual stream, $h$ with an accelerated residual stream, $p$, which has an intrinsic bias towards earlier alignment. Relative to ~\cref{eq:slow_residual_transformation}, accelerated residual transformation from interval $E_j$ to $E_{j+1}$ can be represented as: 

% \begin{equation}
% \label{eq:fast_residual_transformation}
% P_{E_{j+1}} = P_{E_j} \prod_{i=E_j}^{E_{j+1}} \left( I + \hat{\mathcal{A}_i} \right) \left( I + \hat{\mathcal{M}_i} \right) \quad \text{where} \quad \hat{\mathcal{A}_i} = \hat{f}(c_i, P_{i}), \hat{\mathcal{M}_i} = \hat{g}(P_{i})
% \end{equation}


\begin{equation} \label{eq:fast_residual_transformation}
p_{E_{j+1}} = p_{E_j} + \sum_{i=E_j}^{E_{j+1}-1} \left( \hat{\mathcal{A}_i}(p_i) + \hat{\mathcal{M}_i}(p_i + \hat{\mathcal{A}_i}(p_i)) \right) \quad \text{where} \quad \hat{\mathcal{A}_i} = \hat{f}(c_i, p_{i}), \hat{\mathcal{M}_i} = \hat{g}(h_i), 
\end{equation}



where $\hat{\mathcal{A}_i}$ and $\hat{\mathcal{M}_i}$ denote non-linear transformation added by layer $i$ to previous accelerated residual $p_{i}$. Similar to $\mathcal{A}_i$, non-linear transformation $\hat{\mathcal{A}_i}$ attends to same context $c_i$ but uses a different transformation $\hat{f}$ for accelerating $p_{E_j}$ relative to $h_{E_j}$. 

We integrate accelerated residual transformation directly into the base network using parallel accelerator adapters such that rank of accelerator adapters $R_p << d$ where $d$ denotes base model hidden dimension. This setup allows the slow residual stream $h_{E_j}$ to pass through the base model layers while the accelerated residual stream $p_{E_j}$ utilizes these parallel adapters as shown in ~\cref{fig:m2r2_main}. Both slow and accelerated residuals are processed in same forward pass via attention masking and incur negligible additional inference latency in memory bound decoding setups, while in compute bound decoding setups where FLOPs optimization is essential, accelerated residual stream utilizes a fraction of attention heads that of slow residual (see ~\cref{sec:flops_optimization}). Additionally, to maximize the utility of accelerated residual transformations without introducing dedicated KV caches, we propose a shared caching mechanism between the slow and accelerated streams which minimally impact alignment benefits of our approach while offering substantial memory savings (see ~\cref{fig:koala_alignment}). Specifically, the attention operation on the slow residuals \( \text{MHA}(h_t, h_{\leq t}, h_{\leq t}) \) is redefined for accelerated residuals as 
\[
\hat{\mathcal{A}} = MHA(p_t, h_{<t} \oplus p_t, h_{<t} \oplus p_t),
\]
where the accelerated residual at time-step $t$, \( p_t \) attends to the slow residual’s KV cache, facilitating the reuse of contextual information across both residual streams without incurring additional caching costs. Here, \(MHA(q, k, v) \) represents multi-head attention between query \( q \), key \( k \), and value \( v \).

\begin{figure}
    \centering
    \includegraphics[width=0.8\linewidth]{sections//figures/m2r2_main2.pdf}
    \caption{Multi-rate Residuals Framework: Slow residual stream of base model is accompanied by a faster stream that operates at a $2-(J+1)\times$ rate relative to the slow stream, undergoing transformations via accelerator adapters as detailed in \cref{m2r2_method}, where J denotes number of early exit intervals. Colors within the slow and fast residual streams indicate similarity, with matching colors representing the most closely aligned residual states. At the beginning of the forward pass and at each exit point, the accelerated residual state is initialized from the corresponding slow residual state to avoid gradient conflict during training (see ~\cref{sec:grad_conflict}). Early exiting decisions are informed by the Accelerated Residual Latent Attention (ARLA) mechanism, described in \cref{method_arla}, which evaluates residual dynamics across consecutive exit gates.}
    \label{fig:m2r2_main}
\end{figure}

% Furthermore. to maximize the benefits of fast residual transformations without using dedicated KV caches, we propose sharing the fast network’s cache with the slow network. Formally speaking, We modify attention operation on slow residuals $MHA(H_t, H_{<=t}, H_{<=t})$ as $MHA(P_{t}, H_{<t} \oplus P_t, H_{<t}  \oplus P_t)$ such that accelerated residuals attend to previous slow context KV cache, where $MHA(q,k,v)$ denotes multi head attention between query, $q$, key $k$ and value $v$.


\subsection{Enhanced Early Residual Alignment}
Early residual alignment is instrumental in optimizing early exiting, speculative decoding, and Mixture-of-Experts (MoE) inference mechanisms. In this section, we provide a detailed analysis of how accelerated residuals enhance these inference setups.

% By aligning the residual states of intermediate layers with the final output representations, the model can maintain high prediction accuracy even when computations are truncated at earlier layers. This enables more reliable early exiting, reducing the overall computational cost while preserving performance. Additionally, in speculative decoding, early residual alignment allows the model to make confident predictions using faster, partial computations, thereby accelerating inference without sacrificing output quality.


\subsubsection{Early Exiting} \label{method_early_exiting}

A prevalent strategy for enabling early exiting at an intermediate layer $E_{j}$ involves approximating the residual transformation between $E_{j}$ and the final layer $N-1$ using a linear, context independent mapping, $\mathcal{T}$, such that $H_{N-1} \approx \mathcal{T}(H_{E_{j}})$. This approximation has been extensively employed in conventional approaches ~\cite{schuster2022confident, chen2023eellm, varshney-etal-2024-investigating}, providing a computationally efficient means to project the output of deeper layers from intermediate states. Specifically, residual state of layer $N-1$ with this approximation can be expressed as:


% \begin{equation}
% \label{eq: vanila_ea_assumption}
% \Phi(H_{E_{j}}) \sim H_{E_{j}} \prod_{i=E_{j}}^{N}\left( I + \mathcal{A}_i \right) \left( I + \mathcal{M}_i \right) \quad \text{where} \quad \Phi \perp C
% \end{equation}

\begin{equation} \label{eq:early_exiting}
h_{E_j} + \sum_{i=E_j}^{N-1} \left( \mathcal{A}_i(h_i) + \mathcal{M}_i(h_i + \mathcal{A}_i(h_i)) \right) \sim \mathcal{T}(h_{E_{j}})  \quad \text{where} \quad \mathcal{T} \perp c. 
\end{equation}


Here, $\mathcal{A}_i$ and $\mathcal{M}_i$ represent the residual contributions of the multi-head attention and MLP layers, respectively, while $\mathcal{T}$ remains independent of $c$, the preceding context.

This approach is inherently limited by two major factors: first, the assumption of linearity between $h_{E_{j}}$ and $h_{N-1}$ may not hold uniformly for all tokens, particularly when $E_j \ll N$. Second, the linear transformation $\mathcal{T}$ disregards the influence of the context $c$ and fails to account for the latent representations of previous contextual states. In contrast, M2R2 accelerated residual states mitigate both of these challenges by approximating the slow residual transformation of all layers via a faster residual transformation of fewer layers as:
% \begin{equation}
% H_{E_j} \prod_{i=E_j}^{N}\left( I + \mathcal{A}_i \right) \left( I + \mathcal{M}_i \right) \sim P_{E_j} \prod_{i=E_j}^{E_j+1}\left( I + \hat{\mathcal{A}_i} \right) \left( I + \hat{\mathcal{M}_i} \right)
% \end{equation}


\begin{equation} \label{eq:m2r2_approximating_ea}
h_{E_j} + \sum_{i=E_j}^{N-1} \left( \mathcal{A}_i(h_i) + \mathcal{M}_i(h_i + \mathcal{A}_i(h_i)) \right) \sim p_{E_j} + \sum_{i=E_j}^{E_{j+1}-1} \left( \hat{\mathcal{A}_i}(p_i) + \hat{\mathcal{M}_i}(p_i + \hat{\mathcal{A}_i}(p_i)) \right), 
\end{equation}

% \begin{equation} \label{eq:fast_residual_transformation}
% p_{E_{j+1}} = p_{E_j} + \sum_{i=E_j}^{E_{j+1}-1} \left( \hat{\mathcal{A}_i}(p_i) + \hat{\mathcal{M}_i}(p_i + \hat{\mathcal{A}_i}(p_i)) \right) \quad \text{where} \quad \hat{\mathcal{A}_i} = \hat{f}(c_i, p_{i}), \hat{\mathcal{M}_i} = \hat{g}(h_i) 
% \end{equation}






where $p_{E_j}$ is initialized from the slow residual state $h_{E_j}$ at each early exit interval $E_j$ using an identity transformation (see ~\cref{fig:m2r2_main}). As shown in ~\cref{fig:m2r2_residual_sim}, accelerated residuals offer a smoother, more consistent shift in residual direction across layers, in contrast to the abrupt changes typically seen at early exit points in standard early exit methods. Moreover, the normalized cosine similarity between accelerated states at early exit intervals and final residual states is substantially higher compared to traditional early exit techniques, highlighting improved alignment with final layer representations. Traditional adaptive compute methods are constrained by two principal factors: the number of tokens eligible for early exit at intermediate layers and the precision of early exit decision. If residual streams fail to saturate early, the majority of tokens remain ineligible for exit, thereby diminishing potential speedups. Additionally, imprecise delineations between tokens suitable for early exit can lead to underthinking (premature exits that adversely affect accuracy) or overthinking (unnecessary processing that compromises efficiency) ~\cite{zhou2020self, dai2020dynamic}. Enhanced early alignment using ~\cref{eq:m2r2_approximating_ea} helps to address  first issue. To address the second issue we introduce Accelerated Residual Latent Attention, which dynamically assesses the saturation of the residual stream, allowing for a more precise differentiation between tokens that can exit early and those requiring further processing.

% This results in uniform change in residual direction    
% % We keep $\mathcal{A} = \hat{\mathcal{A}}$, while $\hat{\mathcal{M}}$ is accelerated by a factor of $2 - (N_{E}+1)X$ relative to the slower residual transformation $\mathcal{M}$, where $N_E$ represents number of early exiting intervals.
% Figure~\cref{fig:rate_change_comparison} illustrates the comparative rate of change between these transformation streams.



% fig:rate_change_comparison
% - grid plot x axis -> layer id (0, 8) , y axis -> layer id -> dark color cell for max similarity , lighter for lower 
% 
-------------------------------------------------------
Let's consider residual stream $h_i$ traverses through interval $E_j$ to $E_{j+1}$ and undergoes residual transformation given by 
\begin{equation}
h_{E_{j+1}} = h_{E_j} \prod_{i=E_j}^{E_{j+1}} \left( 1 + \delta_i \right)    
\end{equation}

where $\delta_i$ denotes non-linear transformation added by layer $i$. Each non-linear transformation of layer $i$ is a function of previous contextual representation, $c_i$ and input residual stream $h_i-1$ as
$\delta_i = f(c_i, h_{i-1})$ 

One way to exit early at exit $E_j+1$ is to assume that residual transformation from $E_j+1$ to final layer $N-1$ can be approximated by a linear function $\phi$ as $h_{N-1} \sim \Phi(h_{E_j+1})$ and most conventional approaches such as \todo{cite EA papers} use this approach. In other words, 

\begin{equation}
\Phi(h_{E_j+1} \sim h_{E_j+1} \prod_{i=E_j+1}^{N} \left( 1 + \delta_i \right)   
\end{equation}

This approach suffers from two primary issues, linearity assumption from $h_E_j+1$ to $H_N-1$ if often incorrect, particularly when $E_j << N$. More importantly, linear transformation $\Phi$ doesn't consider effect of context $C_i$. M2R2  effectively addresses these issues as accelerated residual stream at interval $E_j+1$ can be represented as 

\begin{equation}
r_{E_{j+1}} = r_{E_j} \prod_{i=E_j}^{E_{j+1}} \left( 1 + \gamma_i \right)    
\end{equation}

where $\gamma_i$ denotes non-linear transformation added by layer $i$ to previous accelerated residual $r_i-1$. Similar to $\delta_i$, non-linear transformation $\gamma_i$ considers context $C_i$ as 
$\gamma_i = g(c_i, r_{i-1})$. So in summary, slow residual transformation is approximated by accelerated residual as: 

\begin{equation}
h_{E_j} \prod_{i=E_j}^{N} \left( 1 + \delta_i \right) \sim h_{E_j} \prod_{i=E_j}^{E_j+1} \left( 1 + \gamma_i \right)
\end{equation}

It's worth noting that accelerated residual $r_i$ and slow residual $h_i$ are processed concurrently at layer $i$ by constructing proper attention mask such as attention of slow residual is represented as 

$MHA(H_it, H_{i<=t}, H_{i<=t}$ while attention of fast residual is computed as 

$MHA(r_it, H_{i<=t}, H_{i<=t}$ where $MHA(q,k,v$ denotes multi head attention between query, $q$, key $k$ and value $v$.


------------------------------------------------------------------

Vertical latent attention on accelerated residual is computed as 
$MHA(S_mt, S(Ej<=i<=m)t, S(Ej<=i<=m)t)$ where $Smt$ denotes query/key/value projection in latent domain at layer $m$ at time $t$. 
------------------------------------------------------------------

Gradient conflict Avoidance: 

Let's consider $w_j$ is a trainable parameter that belongs to a layer between $E_j$ and $E_j+1$. Consider early exit loss at gate $E_j+1$, $L_j+1$, gradient propagation of $w_j$ at another trainable parameter $w_j-n$ can be gives as 

$\sum_{k=E_j-n}^{E_j} \beta_k \frac{\partial L_{E_k}}{\partial w_k}$

where $\beta_j$ denotes backward transformation coefficient for weight $w_j$ to reach gate $E_j$. 
 
On the other hand, gradient propagation in proposed approach can be represented as 

\[
\frac{\partial L_{E_j}}{\partial w_j} = 
\begin{cases} 
\beta_j \frac{\partial L_{E_j}}{\partial w_j} & \text{if } E_j \leq w_j \leq E_{j+1} \\
0 & \text{otherwise}
\end{cases}
\]







% \begin{figure}[ht]
%     \centering
%     \includegraphics[width=0.8\textwidth, height=5cm]{rate_change_comparison.png}
%     \caption{Rate of change comparison between fast and slow residual streams.}
%     \label{fig:rate_change_comparison}
% \end{figure}

%vary k and and plot EA accuracy for larger and smaller models. 

% \begin{figure}[ht]
%     \centering
%     \includegraphics[width=0.5\textwidth,height=5cm]{sections/figures/alignment_comparison_dialogsum.pdf}
%     \caption{Alignment of exited tokens for different early exit layers using traditional early exiting heads, dedicated faster networks, and faster residuals.}
%     \label{fig:small_model_early_exiting}
% \end{figure}


\textbf{Accelerated Residual Latent Attention} \label{method_arla}

In the context of residual streams, we observe that the decision to exit at a given layer can be more effectively informed by analyzing the dynamics of residual stream transformations, instead of solely relying on a classification head applied at the early exit interval $E_j$. To capture the subtle dynamics of residual acceleration, we propose a \textit{Accelerated Residual Latent Attention} (ARLA) mechanism. This approach involves making the exit decision at gate $E_j$ by attending to the residuals spanning from gate $E_{j-1}$ to $E_j$, rather than considering only the residual at gate $E_j$. To minimize the computational overhead associated with exit decision-making, the attention mechanism operates within the latent domain as depicted in ~\cref{fig:arla_arch}. Formally, for each interval $[E_j, E_{j+1}]$, the accelerated residuals are projected into Query ($Q^s_{E_j}, \ldots, Q^s_{E_{j+1}}$), Key ($K^s_{E_j}, \ldots, K^s_{E_{j+1}}$), and Value ($V^s_{E_j}, \ldots, V^s_{E_{j+1}}$) vectors, with latent dimension $d^s$ for $Q^s$, $K^s$, and $V^s$ being significantly smaller than hidden dimension of $p$.\footnote{We use $d^s = 64$ for experiments described in ~\cref{sec:experiments}.} Notably, when the router is allowed to make exit decisions at gate $E_j$ based on residual change dynamics, we observe that the attention is not confined to the residual state at $E_j$ but is distributed across residual states from $E_{j-1}$ to $E_j$, %as illustrated in Figure~\ref{fig:vertical_latent_attention_dynamics}. 
This broader focus on residual dynamics significantly reduces decision ambiguity in early exits, as demonstrated in Figure~\ref{fig:roc_arla}, which contrasts routers based on the last hidden state, and the proposed ARLA router.

%show R -> S transformation. 
%show parameter and flop overhead as compared to adapter on last hidden state.

% \begin{figure}[ht]
%     \centering
%     \includegraphics[width=0.5\textwidth,height=5cm]{sections/figures/roc_arla.pdf}
%     \caption{ROC curves of early exit decision strategies: confidence-based methods (CALM/LITE), routers based on the accelerated hidden state, and latent attention routers.}
%     \label{fig:decision_making_comparison}
% \end{figure}

% \begin{figure}[ht]
%     \centering
%     \includegraphics[width=0.5\textwidth,height=5cm]{vertical_latent_attention.png}
%     \caption{Vertical latent attention mechanism for optimizing early exit decisions by considering residuals from gate \(M\) through \(M-1\).}
%     \label{fig:vertical_latent_attention}
% \end{figure}

\begin{figure}[ht]
    \centering
    \begin{subfigure}{0.52\textwidth}
        \centering
        \includegraphics[width=\textwidth, height = 4cm]{sections/figures/arla_arch.pdf}
        \caption{Accelerated Residual Latent Attention (ARLA): Accelerated residuals between early exit gates are projected into latent domain and attention over residual states within the interval is computed to capture residual dynamics and exit decision is made based on residual saturation.}
        \label{fig:arla_arch}
    \end{subfigure}%
    \hfill
    \begin{subfigure}{0.45\textwidth}
        \centering
        \includegraphics[width=\textwidth, height = 4.5cm]{sections/figures/vla_roc.pdf}
        \caption{ROC classification curves of early exit decision strategies using a linear router used on last residual state ~\cite{schuster2022confident, varshney-etal-2024-investigating, chen2023eellm}  and using ARLA approach that considers residual dynamics. }
        \label{fig:roc_arla}
    \end{subfigure}
    \caption{Effectiveness of ARLA in capturing residual dynamics for early exiting decisions.}


\end{figure}



% \begin{figure}[ht]
%     \centering
%     \includegraphics[width=1\textwidth,height=5cm]{sections/figures/arla.pdf}
%     \caption{fig that plots 32 rows 2 cols heatmap showing attention at each gate}
%     \label{fig:vertical_latent_attention_dynamics}
% \end{figure}

\subsubsection{Self Speculative Decoding} \label{method_self_speculative_decoding}

An alternative means to exploit the early alignment properties of our approach is through the use of accelerated residual states for speculative token sampling to accelerate autoregressive decoding. Speculative decoding aims to speed up memory-bound transformer inference by employing a lightweight draft model to predict candidate tokens, while verifying speculated tokens in parallel and advancing token generation by more than one token per full model invocation \cite{leviathan2023fast, chen2023accelerating, xia2023speculative, miao2023specinfer}. Despite its effectiveness in accelerating large language models (LLMs), speculative decoding introduces substantial complexity in both deployment and training. A separate draft model must be specifically trained and aligned with the target model for each application, which increases the training load and operational complexity ~\cite{chen2023accelerating}. Additionally, this approach is resource-inefficient, as it requires both the draft and target models to be simultaneously maintained in memory during inference \cite{leviathan2023fast, chen2023accelerating}. 

One strategy to address this inefficiency is to leverage the initial layers of the target model itself to generate speculative candidates, as depicted in ~\cite{Tang2024}. While this method reduces the autoregressive overhead associated with speculation, it suffers from suboptimal acceptance rates. This occurs because the linear transformation employed for translating hidden states from layer $k$ to the final layer $N$ is typically a poor approximation, as discussed in ~\cref{sec:motivation} and ~\cref{method_early_exiting}. Our approach resolves this limitation by utilizing accelerated residuals, which demonstrate higher fidelity to their slower counterparts. By utilizing accelerated residuals operating at a rate of $N/k$, where $k$ denotes the number of layers used for candidate speculation, we are able to efficiently generate speculative tokens for decoding.\footnote{We typically set $k = 4$ to balance the trade-off between autoregressive drafting overhead and acceptance rate, as discussed in~\cref{sec:experiments}.}
 This technique not only obviates the need for multiple models during inference but also improves the overall efficiency and effectiveness of speculative decoding.

\begin{figure}
    \centering    \includegraphics[width=1\linewidth]{sections/figures/m2r2_aot_loading.pdf}
    \caption{Ahead-of-Time Expert Loading: M2R2 accelerated residual stream predicts experts required for future layers, reducing reliance on on-demand lazy loading. Speculative pre-loading is efficiently overlapped with computation of multi-head attention (MHA) and MLP transformations. Only incorrectly speculated experts are loaded lazily, resulting in faster inference steps and improved computational efficiency. Here, H indicates LBM Host while D indicates HBM Device.}
    \label{fig:moe_expert_aot_loading}
\end{figure}


\subsubsection{Ahead of Time Expert Loading:} \label{method_aot_expert_loading}

Recent advancements in sparse Mixture-of-Experts (MoE) architectures ~\cite{shazeer2017outrageously, fedus2022switch, artetxe2019massively, lepikhin2020gshard, zoph2022designing} have introduced a paradigm shift in token generation by dynamically activating only a subset of experts per input, achieving superior efficiency in comparison to dense models, particularly under memory-bound constraints of autoregressive decoding \cite{fedus2022switch, zoph2022designing}. This sparse activation approach enables MoE-based language models to generate tokens more swiftly, leveraging the efficiency of selective expert usage and avoiding the overhead of full dense layer invocation. In dense transformer models, pre-loading layers is a common strategy to enhance throughput, as computations of current layer can be overlapped with pre-loading of next layer parameters ~\cite{narayanan2021efficient, shoeybi2020megatron}. However, MoE models face a unique challenge: expert selection occurs dynamically based on previous layer’s output, making it infeasible to preload next layer’s experts in parallel. This limitation results in inherent latency, as expert loading becomes a sequential, on-demand process ~\cite{lepikhin2020gshard, fedus2022switch}.

To address this inefficiency, our method introduces a mechanism with \textit{accelerated residuals}, which not only captures key characteristics of base slower residual states but also exhibit high cosine similarity with their final counterparts (as illustrated in \cref{fig:m2r2_residual_sim}). By employing accelerated residual streams, we can effectively predict the necessary experts for future layers well in advance of their actual invocation. Specifically, using a $2\times$ accelerated residual, the experts needed for layers $2i+2$ and $2i+3$ can be identified while still computing in layer $i$, thus overcoming the bottleneck of sequential, on-demand expert selection and mitigating latency in the decoding pipeline, as shown in \cref{fig:moe_expert_aot_loading}. Note that, we use fixed set of accelerator adapters for transforming accelerated residuals (as discussed in ~\cref{m2r2_method}) while slow residual is transformed via expert routing mechanism. 

Furthermore, our approach integrates a Least Recently Used (LRU) caching strategy, which enhances memory efficiency by replacing the least recently used experts with speculated experts that are anticipated to be needed in upcoming layers. This hybrid approach of preemptive expert loading with LRU caching yields substantial improvements over traditional on-demand loading or standalone caching strategies. By minimizing cache misses and efficiently managing memory, this approach addresses both compute and memory bottlenecks, leading to faster, more resource-efficient token generation in MoE architectures. A comprehensive evaluation of this strategy, in relation to state-of-the-art methods, is provided in \cref{experiments_aot}, and the compute and memory traces on an A100 GPU are detailed in \cref{fig:moe_aot_cuda_trace}.



% Recent advancements in sparse Mixture-of-Experts (MoE) architectures have introduced the concept of utilizing distinct computational paths for different tokens \cite{shazeer2017outrageously}. This approach, wherein only a subset of experts are activated per input, enables MoE-based language models to generate tokens more swiftly compared to their dense counterparts due to memory-bound nature of auto-regressive decoding. In dense models, pre-loading layers in advance is a common strategy to enhance computational efficiency. However, this technique is not applicable to MoE models, where expert selection occurs dynamically based on the outputs of previous layers, preventing parallel pre-fetching of experts.

% Our proposed method addresses this inefficiency. Accelerated residuals, which are highly similar to their slower counterparts (see \cref{fig:similarity}), can reliably predict the necessary experts ahead of time. For instance, by utilizing $2X$ accelerated residual stream, we can predict the experts needed for the layer $2i+1$ and $2i+3$ while carrying out computation in layer $i$. This enables us to commence expert loading significantly earlier, as illustrated in \cref{expert_loading}, effectively mitigating the delays observed with the naive on-demand expert loading. Additionally, our method benefits from incorporating a Least Recently Used (LRU) strategy, where speculated experts replace those that are least recently utilized, resulting in improved performance compared to using either strategy alone. For a comprehensive evaluation, refer to \cref{moe_trace}, which provides a CUDA compute and memory trace of our approach executed on <>.



% A naive solution involves using the residual state of the previous layer along with the gating function of the next layer to predict which experts need to be loaded, and initiating the expert loading process in parallel with the attention computation of the next layer. Yet, as shown in \cref{fig:MOE_attn_vs_loading_time}, the attention computation for medium to long contexts is considerably faster than the expert loading time, making this approach inefficient.




\subsection{Training} \label{method_training}
% This approach is feasible due to the absence of gradient conflicts, as discussed in \cref{sec:grad_conflict}.

To accelerate residual streams, we employ parallel accelerator adapters as described in \cref{m2r2_method}.  For the early exiting use-case outlined in \cref{method_early_exiting}, we define the training objective for these adapters using the following loss function, which combines cross-entropy loss at each exit $E_j$ with distillation loss at each layer $i$. Loss weights coefficients $\alpha_0$ and $\alpha_1$ are employed to balance contribution of corresponding losses.

\begin{align} \label{eq:mr_loss}
L_{\text{m2r2}} = \underbrace{-\alpha_0 \sum_{j=1}^{J} \sum_{t=1}^{T} \log p_{\theta} \left( \hat{y}_t^{E_j} \mid y_{<t}, x \right)}_{\text{cross-entropy loss}} 
+ \underbrace{\alpha_1\sum_{i=1}^{E_{J-1}} \sum_{t=1}^{T} \| \mathbf{p}_{t}^{i} - \mathbf{h}_{t}^{((i - E_{j(i)}) \cdot R_i) + E_{j(i)})} \|^2}_{\text{distillation loss}}.
\end{align}

where $\hat{y}_t^{E_j}$ denotes the predictions from the accelerated residual stream at layer $E_j$ and time step $t$, $y_t$ represents the corresponding ground truth tokens, and $x$ indicates previous context tokens. The distillation loss at each layer $i$ is computed by comparing accelerated residuals at layer $i$ with slow residuals at layer $(i - E_{j(i)}) \cdot R_i + E_{j(i)}$, where $R_i$ denotes the rate of accelerated residuals at layer $i$ while $E_{j(i)}$ represents the most recent gate layer index such that $E_{j(i)} <= i$. \( J \) represents the total number of early exit gates, N denotes number of hidden layers and $E_j$ denotes layer index corresponding to gate index $j$ and \( T \) denotes the sequence length. 

In dynamic compute settings, after training of accelerator adapters, we optimize the query, key, and value parameters governing the ARLA routers (see ~\cref{method_arla}) across all exits in parallel on binary cross entropy loss between predicted decision and ground truth exiting decision. The ground truth labels for the router are determined based on whether the application of the final logit head on $\hat{y}_t^{E_j}$ yields the correct next-token prediction. 


% The objective for this optimization is defined by the following loss function:


%TODO are equations required ? 
% \begin{equation} \label{eq:arla_loss_combined}\small
%     L_{\text{arla}} = -\frac{1}{N} \sum_{t=1}^{T} \left( \sum_{j=1}^{E_n} \left[ O_t^{E_j} \log(\hat{O}_t^{E_j}) + (1 - O_t^{E_j}) \log(1 - \hat{O}_t^{E_j}) \right] \right), \quad \text{where} \quad 
%     O_t^{E_j} = \begin{cases} 
%     1, & \text{if } L(\hat{y}_t^{E_j}) = y_t^{E_j} \\
%     0, & \text{otherwise}
%     \end{cases}
% \end{equation}

% where $\hat{O}_t^{E_j}$ represents the binary predicted logits produced by the vertical latent attention router, as described in \cref{sec:arla}, at gate $E_j$ and time step $t$, and $O_t^{E_j}$ denotes the corresponding ground truth labels. The ground truth labels for the router are determined based on whether the application of the logit head on $\hat{y}_t^{E_j}$ yields the correct next-token prediction. The parameters controlling vertical latent attention are trained concurrently to ensure consistency and efficient use of computational resources.

For self-speculative decoding, as described in \cref{method_self_speculative_decoding}, the training objective remains the same as \cref{eq:mr_loss}, but with the number of intervals set to $J = 1$ and the rate of residual transformation set to $R_n = N/k$, where the first $k$ layers generate speculative candidate tokens. In the context of Ahead-of-Time Expert Loading for Mixture-of-Experts (MoE) models (see \cref{method_aot_expert_loading}), setting the rate of residual transformation to $R_n = 2$ typically offers a good trade-off between the accuracy of expert speculation and AoT pre-loading of experts. 

% Thus, we set $J = 1$ and $E_1 = 16$.


~\subsection{FLOPs Optimization} \label{sec:flops_optimization}

Naively implemented, M2R2 incurs higher FLOP overhead compared to traditional speculative decoding and early exiting approaches such as ~\cite{medusa, schuster2022confident, Tang2024}. However, modern accelerators demonstrate compute bandwidth that exceeds memory access bandwidth by an order of magnitude or more~\cite{databricksLLMInference2023, jouppi2021ten}, meaning increased FLOPs do not necessarily translate to increased decoding latency. Nevertheless, to ensure fair comparison and efficiency in compute bound scenarios, we introduce targeted optimizations.

~\textbf{Attention FLOPs Optimization} For medium-to-long context lengths, attention computation dominates FLOPs in the self-attention layer, surpassing the contribution from MLP layers. Specifically, matrix multiplications involving queries, cached keys, and cached values scale with $l_{kv} * l_{q}$ where $l_{kv}$ denotes previous context length and $l_q$ denotes current query length. Since M2R2 pairs accelerated residuals with slow residuals, a naive implementation results in twice the FLOPs consumption compared to a standard attention layer. To address this, we limit the attention of accelerated residual stream to selectively attend to the top-k most relevant tokens, identified by the slow residual stream based on top attention coefficients\footnote{We set to k = 64 and attend to top 64 tokens as identified by the slow residual stream.}. This is possible since slow and accelerated residual streams are processed in same forward pass and accelerated streams have access to attention coefficients of slow stream. Note that, the faster residual stream still retains the flexibility to assign distinct attention coefficients to these tokens. Furthermore, we design the faster residual stream to employ only 8 attention heads, compared to the 32 heads used in the slow residual stream of the Phi-3 model, reducing query, key, value, and output projection FLOPs by a factor of 1/4. ~\cref{fig:m2r2_num_heads_ablation} indicates effect of using a slicker stream on alignment. As depicted, using $\hat{n}_h = 8$ offers a good trade-off between alignment and FLOPs overhead. 

~\textbf{MLP FLOPs Optimization} The accelerator adapters operating on the accelerated residual stream are intentionally designed with lower rank than their counterparts in the base model. This reduces FLOP overhead by a factor proportional to $hiddenSize / rank$. Additionally, since the faster residual stream uses only 8 attention heads (compared to 32 in the slow residual stream of Phi-3), the subsequent MLP layers process a smaller set of activations, further reducing FLOPs by another factor of 1/4.

These optimizations significantly reduce the FLOP overhead per speculative draft generation, as illustrated in ~\cref{fig:flops_optmization}. Notably, while traditional early-exiting speculative approaches such as DEED require propagating the full slow residual state through the initial layers, incurring substantial computational costs, M2R2 achieves efficient token generation via slimmer, low-rank faster residual streams. In contrast, Medusa introduces considerable FLOP overhead due to per-head computations scaling with $d^2+dv$\footnote{Here $d$ denotes hidden state dimension while $v$ denotes vocab size.}, whereas M2R2 employs low-rank layers for both MLP and language modeling heads, maintaining computational efficiency. All experiments involving the M2R2 approach, as detailed in ~\cref{sec:experiments}, are conducted using these FLOPs optimizations.









% \[
% O_t^{E_j} = 
% \begin{cases} 
% 1, & \text{if } L(\hat{y}_t^{E_j}) = y_t^{E_j} \\
% 0, & \text{otherwise}
% \end{cases}
% \]




%add distillation
% We train accelerator adapters described in \cref{m2r2_method} to accelerate residual streams on next token prediction all in parallel since there are no gradient conflict issues as described in \cref{sec:grad_conflict}.

% \begin{align} \label{eq:mr_loss}
% L_{mr} =  & -\sum_{j = 1}^{E_n} (\sum_{t=1}^{T}\log p_{\theta} (\hat{y}_t^{E_j} | \hat{y}_{<t}, x)) \nonumber
% \end{align}

% where $\hat{y_t^{E_j}}$ denotes predicted logits obtained from accelerated residual stream at gate $E_j$ and time-step $t$ while $y_t^{E_j}$ denotes corresponding truth tokens. 

% Upon training of adapters responsible for accelerating residual streams, we train query, key, value parameters responsible for vertical latent attention of all gates in parallel as

% \begin{equation} \label{eq:arla_loss}
%     L_{arla} = -\frac{1}{N} (\sum_{t=1}^{T}(1\sum_{j=1}^{E_n} \left[ O_t^{E_j} \log(\hat{O}_t^{E_j}) + (1 - o_t^{E_j}) \log(1 - \hat{o_t}_{E_j}) \right]))
% \end{equation}

% where $\hat{O_t^{E_j}}$ denotes binary predicted logits obtained from vertical latent attention router described in \cref{sec:arla} at gate $E_j$ and timestep $t$ while $O_t^{E_j}$ denotes corresponding truth label. Truth labels for router are obtained by computing whether logit head application on $\hat{y}_t^j$ results in true next token prediction. Formally speaking, 

% $O_t^{E_j} = 1 if L(\hat{y_t^{E_j}}) == y_t^{E_j} , 0 otherwise$. 

% Parameters responsible for vertical latent attention are also trained in parallel as well. 

%todo: training slow and fast residuals together and distillation can be two training mdoes. 
%Distillation can be an ablation. 




% Although transformer decoding is memory bound on most mainstream accelerators, there could be scenarios where flop savings are crucial. For instance, on on-device settings power consumption is directly correlated with flops per decoding step and reducing flops does help with overall energy consumption. Vanilla early exiting methods help with flop reduction but suffer from mismatch between training and inference due to early exited tokens. If token at decoding step $t$, $T_t$ exited at layer $E_i$, while token $T_{t+k}$ exits at layer $E_j$ such that $E_i < E_j$, hidden state $H_{t+k}l$ does not have corresponding hidden state $H_tl$ to attend to where $E_i < l <= E_j$. One solution that's often used in literature is to rely on last hidden state available, $H_t{E_j}$, however it tends to be sub-optimal and does affect generation quality \cite{ref}.  To alleviate this mismatch while reducing flops, we train router such that attention mask between token $T_{t+k}$ and token $T_{<t+k}$ is given by: 

% \begin{equation}
%     a_{T_{{t+k}{T_{<t+k}}} = 1 if  E_{T_{<t+k}} >= E{T_{t+k}}
%     else 0
% \end{equation}

% This attention mask enables router to account for exited tokens and get trained accordingly. Since attention mechanism during decoding remains exactly same as that during training, impact on generation quality tends to be minimal as noted in \cref{fig:gen_auality_with_and_without_recompute_attention_show_flops}.  Although MoD does not suffer from training and inference mismatch, we observe that it suffers from discountinuity between pre-training and super-vised fine-tuning resulting in sub-optimal perplexity. On the other hand, our method doesn't not require pre-training , doesn't suffer from discountinuity, and achieves much better perplexity in super-vised fine-tuning and instruction tuning setups as shown in \cref{fig:Mod_vs_m2r2_loss_curves}.






% Our techniques are directly applicable in such scenarios.    




%expert loading with cuda streams in experiments
%\documentclass[tikz,border=3.14mm]{standalone}
\usetikzlibrary{trees}

\begin{document}

\begin{tikzpicture}
  \node {-} {
    child {node {123}}
    child {node {1}}
  };
\end{tikzpicture}

%\begin{tikzpicture}
%  % Root node (subtraction)
%  \node {-}
%  child { % Left child (3x^2)
%    node {*}
%      child { % Left child (3)
%        node {3}
%      }
%      child { % Right child (x^2)
%        node {\^}
%          child { % Left child (x)
%            node {x}
%          }
%          child { % Right child (2)
%            node {2}
%          }
%      }
%  }
%  child { % Right child (-1)
%    node {1}
%    edge from parent[draw=none]
%  };
%\end{tikzpicture}

\end{document}

%\section*{Conclusion}
This paper aims to enhance our understanding of the computational complexity of computing various Shapley value variants. We found that for various ML models --- including decision trees, regression tree ensembles, weighted automata, and linear regression --- both local and global interventional and baseline SHAP can be computed in polynomial time under HMM modeled distributions. This extends popular algorithms, such as TreeSHAP, beyond their empirical distributional scope. We also establish strict complexity gaps between the various SHAP variants (baseline, interventional, and conditional) and prove the intractability of computing SHAP for tree ensembles and neural networks in simplified scenarios. Overall, we present SHAP as a versatile framework whose complexity depends on four key factors: \begin{inparaenum}[(i)] \item model type, \item SHAP variant, \item distribution modeling approach, \item and local vs. global explanations\end{inparaenum}. We believe this perspective provides deeper insight into the computational complexity of SHAP, paving the way for future work.




%We believe that our framework provides a more intricate understanding of SHAP computation complexity across different models, distributions, and variants, paving the way for further research.

Our work opens promising directions for future research. First, expanding our computational analysis to other SHAP-related metrics, such as asymmetric SHAP~\citep{frye20} and SAGE~\citep{covert2020understanding}, would be valuable. Additionally, we aim to explore more expressive distribution classes and relaxed assumptions beyond those in Section \ref{sec:tractable} while maintaining tractable SHAP computation. Finally, when exact computation is intractable (Section \ref{sec:intractable}), investigating the approximability of SHAP metrics through approximation and parameterized complexity theory~\citep{downey2012parameterized} is an important direction.

%Our work opens several promising avenues for future research on the computational properties of explainable AI methods, with a particular focus on SHAP. First, it would be interesting to broaden the computational analysis conducted in this work to include other popular SHAP-related metrics in the literature, such as asymmetric SHAP \cite{frye20} and SAGE \cite{covert2020understanding}. Also, in the future, we aim to explore more expressive distribution classes and relaxed distributional assumptions—extending beyond those examined in Section \ref{sec:tractable} —that still yield tractable SHAP computation. Finally, when exact computation proves intractable (Section \ref{sec:intractable}), it is worthwhile to theoretically investigate the question of the approximability of computing the SHAP metrics across various configurations, through the lens of approximation and parametrized complexity theory \cite{arora2009computational}.

%This paper aims to deepen our understanding of the computational complexity involved in obtaining different Shapley value variants. We found that for a variety of ML models, including decision trees, tree ensembles for regression, weighted automata, and linear regression models — computing both local and global interventional and baseline SHAP can be done in polynomial time when distributions are modeled by HMMs. This extends the distributional scope of popular algorithms like TreeSHAP, which is limited to empirical distributions. Additionally, we demonstrate a strict complexity gap between SHAP variants, showing that interventional and baseline SHAP can be strictly easier to compute than conditional SHAP. Despite these positive results, we uncovered intractability for various SHAP variants in neural networks and tree ensembles. Finally, we provided generalized complexity relations across SHAP variants. We believe that our framework offers a deeper understanding of the complexity involved in computing SHAP across various variants, models, distributions, as well as in both local and global computations, laying the groundwork for future research.
%\section{Limitations}
Our model is based on ControlNet~\cite{zhang2023adding} with a CLIP text embedding model~\cite{radford2021learning}, although modified by AnyText~\cite{tuo2023anytext} to incorporate glyph line information. However, the CLIP-text encoder has relatively limited language understanding capabilities compared to state-of-the-art foundation models. Unlike text itself, this limitation affects the model's ability to accurately render complex artistic visual features or backgrounds, which users might specify in their input prompts, such as asking the text to appear like clouds or flames, that go beyond merely the font information.

Additionally, due to limited training resources, our experiments were conducted using a smaller diffusion model as a proof-of-concept compared to commercial ones. Each epoch requires approximately 380 GPU hours on NVIDIA V100 GPUs with 32 GB of memory, but we anticipate significantly improved efficiency on newer hardware and with a larger memory. This constraint may result in suboptimal inpainting of background regions within the text area, as well as instability in the quality of rendered text. The users also have limited controls of background pixels behind the text. 

Some sacrifice in text quality is observed for non-Latin languages on the AnyText-Benchmark in exchange for improved font controllability.

The embedding layers of the glyph controls can also lead to reduced text quality, especially when the text in a font is very small, thin, or excessively long. In such cases, fine details of the font information in the glyphs may be lost. 

As with all other text-to-image algorithms that rely on diffusion models, our approach requires a certain number of denoising steps to generate a single image at inference. End-to-end transformer-based models~\cite{xie2024show} may improve the time efficiency of the generation process.


% The entire pipeline also requires a user-friendly front-end interface to allow users to create glyph control images efficiently, especially working with more complex text layouts or curved typography effects. However, as long as users specify the input glyph controls, our ControlNet is already prepared to follow the pixel-level information to render the image with the required text and font. 


\section{Ethical Impact}
This work is intended solely for academic research purposes. While our algorithm allows users to generate images with customized text, there is a potential risk of misuse for producing harmful or hateful content or misinformation. However, we do not identify any additional ethical concerns compared to existing research on visual text rendering.
%\section*{Acknowledgements}
This is acknowledgment.


\begin{abstract}
%Large language models (LLMs) have demonstrated remarkable capabilities in many domains, yet their ability in System 2 tasks remain opaque. The inherent high-thinking demand and complex latent patterns of data, make it hard for llms to accurately infer solutions. Under this situation, the simple train-once-for-all post-training mechanism may then fail. In this paper, we propose \textbf{\modelname} to \textbf{Disen}tangle the complex problem solving procedure and heterogeneous data distribution into meta-units and adaptively aggregate the resulting \textbf{Lora}-experts to generate problem solver for each problem instance. 
%Concretely, 1) we disentangle the problem-solving stages into problem2thought and thought2solution processes, integrating the search process with a novel multi-expert MCTS algorithm, where reflexion-based pruning and refinement help to boost performance. 2) The resulting training data is thereafter disentangled into different meta clusters based on semantic distance, on which we finetune lora experts capable of different aspects of tasks.  3) Then, we train an input-aware hyper-network to adaptively aggregate the lora experts rank-wise for contextualized problem solver. Experiment results and various ablation studies validate the superiority of the data collection algorithm, the effectiveness of the data disentanglement process, and the performance gain brought by the input-aware hyper-network.
%\whj{第一句要改,如果重点是2->1的话}
Large language models (LLMs) have demonstrated remarkable capabilities in various domains, particularly in system 1 tasks, yet the intricacies of their problem-solving mechanisms in system 2 tasks are not sufficiently explored. 
Recent research on System2-to-System1 methods surge, exploring the System 2 reasoning knowledge via inference-time computation and compressing the explored knowledge into System 1 process. In this paper, we focus on code generation, which is a representative System 2 task, and identify two primary challenges: (1) the complex hidden reasoning processes and (2) the heterogeneous data distributions that complicate the exploration and training of robust LLM solvers. To tackle these issues, we propose a novel BDC framework that explores insightful System 2 knowledge of LLMs using a MC-Tree-Of-Agents algorithm with mutual \textbf{B}oosting, \textbf{D}isentangles the heterogeneous training data for composable LoRA-experts, and obtain \textbf{C}ustomized problem solver for each data instance with an input-aware hypernetwork to weight over the LoRA-experts, offering effectiveness, flexibility, and robustness. This framework leverages multiple LLMs through mutual verification and boosting, integrated into a Monte-Carlo Tree Search process enhanced by reflection-based pruning and refinement. Additionally, we introduce the DisenLora algorithm, which clusters heterogeneous data to fine-tune LLMs into composable Lora experts, enabling the adaptive generation of customized problem solvers through an input-aware hypernetwork. Our contributions include the identification of critical challenges in existing methodologies, the development of the MC-Tree-of-Agents algorithm for insightful data collection, and the creation of a robust and flexible solution for code generation. This work lays the groundwork for advancing LLM capabilities in complex reasoning tasks,  offering a novel System2-to-System1 solution.

%\whj{
%State-of-the-art (SOTA) Large Language Models (LLMs) demonstrate System-2-like intelligence through multi-step logical reasoning, yet remain limited in achieving human-level Artificial General Intelligence (AGI) across diverse tasks. Focusing on code generation, which are proven as an effective proxy for complex problem-solving, we identify two critical challenges: (1) xxxx, and (2) heterogeneous task distributions that resist unified modeling.

%We propose our Boost-Disentangle-Customize (BDC) framework, bridging System-2 collective reasoning with System-1 skill specialization. Our approach features:
%(1) MC-Tree-of-Agent: Collective Monte-Carlo searching and mutual verification boosted reasoning. (2) DisenLoRA Adaption: Parameter-efficient specialization via semantic clustering of reasoning trajectories and dynamic experts composition with input-awareness. Empirical results have shown up to $13.8\%$ improvement over GPT4o-mini on CodeContest-Hard and superior robustness against existing adapter merging algorithms.

%}
\end{abstract} 

\section{Introduction}

%\whj{Some questions here: (1) How should we build the corresponding inter-connections between system-1/2 and our two-stages problem-solving framework? system-1 -> P2T with system-2 -> T2S or system-2 -> P2T + T2S or backbone LLM as system-2 with lora adapters as system-1 }

%\begin{itemize}
%    \item LLM achievement
%    \item System 1-2
%    \item 
%    \item moe, sparsity
%\end{itemize}

%The core contributions of our work are two-fold: (1) We propose a two-stage problem-solving framework, namely Problem2Thought and Thought2Problem to enhance LLMs in complex reasoning tasks. And the performance of our framework, as a strong system-2 solver, is evaluated on enhanced trajectories from SOTA commercial LLMs. (2) Based on the collected trajectory data, we streamline a pipeline to strengthen the reasoning abilities of locally deployed LMs by training specialized LoRA experts and ensembling them in the inference-time with input-awareness, aiming to provide capable and specilized system-1 solvers.

%We like to organize the intro section in several paragraphs:

%1: Basic intro of recent progress of LLMs: The emerging and improving abilities of Large Language Models(LLM) as they scaling have attracted extensive attention from the community. They have exhibited comparable even human-surparsing capabilities in many tasks, for example autodriving, robotics, ... While unlike traditional deep learning models that excel in a certain narrow and highly specialized category of tasks, failing to generalize to a broader range of problems, LLMs produce general answer sequences for tasks grounded in text format, failing in given meaningful solution for specialized tasks. 

%2. If we consider human intelligence, SOTA commercial LLMs(e.g. GPT 4o, Claude-3.5) are close to the idea of system-2, featuring slow, rational and general reasoning capabilities. As contrast, system-1, identified by psycologist, represents quick, instinctful and specialized intelligence.

%3. Previous literature argues that the fusion paradigm of system-1 and system-2, for example materialized by a hierachicial structure contains dual reasoning and controling loop of both large and small models, can be essential for AGI. And LoRA adapters can adapt to various narrow downstream tasks without harming the general world knowledge held in the backbone LMs, which inspires us to adopt peft methods to instantiate system-1 solvers.

%4. Given the success of MoE structure and DeepSeek, the sparsely task-aware activativation of model parameters can be an important road balancing between  scaling for better reason abilities and optimizing inference cost for specific tasks. Thus we propose our HyperNet model for inference-time adatpers ensemble, while also leaving space for future optimization opportunity of sparsity.

%5. Introduce our core contributions.



%--------- \whj{Draft Writing Here} ---------
% Para 1: Importance of llms and the obstacles of system 2 tasks. 

\begin{figure}[t]
    \centering
    \includegraphics[width=1.0\linewidth]{figures/intro.pdf}
    \caption{Illustration of the motivation.}
    \label{fig:intro}
\end{figure}

Large language models show significant intelligence in various domains, striking both the academic and industrial institutions. Despite their prominent problem-solving abilities in system 1 tasks, the mechanism behind the system 2 task solving procedure remain opaque. In this paper, we focus on the code generation task, which emerges as a captivating frontier \citep{zheng2023codegeex, roziere2023code, shen2023pangu}, promising to revolutionize software development by enabling machines to write and optimize code with minimal human intervention. Recent research of llms for code focus on inference-time computation (System 2 methods) \citep{yang2024chain, yao2024tree, zhang2023planning} and post-training. While during post-training, distilling system 2 knowledge into system 1 backbones is important and widely-used \citep{yu2024distilling21}. 

However, the complex hidden reasoning process and the heterogeneous data distribution pose challenges to the existing System2-to-System1 pipeline. On one hand, the hidden reasoning process for code generation is complex and hard to explore \textbf{(C1)}. On the other hand, the heterogeneous data distribution, e.g., jumping structure like branching, recursion, etc., makes the existing train-once-for-all strategy hard to fit the complex latent patterns for robust and generalizable llm solvers \textbf{(C2)}. 

For \textbf{(C1)}, we propose to disentangle the problem solving process into problem2thought and thought2solution stages, exploring the inherent reasoning clues via combining the strengths of multiple llms by mutually-verification and boosting. The exploration is integrated into a Monte-Carlo Tree Search process, where reflexion-based pruning and refinement are designed for more efficient and effective reasoning clues search.

For \textbf{(C2)}, we propose to disentangle the heterogeneous data into clusters, finetuning llms capable of different aspects of tasks to obtain the meta LoRA experts hub, and then adaptively generate customized problem solver for each code problem.  Concretely, we design an input-aware hypernetwork to generate rank-wise weights over meta LoRA experts for customized problem solver, offering robustness and flexibility.

The main contributions of our work can be summarized below.
\begin{itemize}
    \item \textbf{Identification of problems and novel BDC framework.} We identify the high-reasoning demand and heterogeneous latent patterns problems that hinders the performance of existing methods and propose a BDC framework that explores insightful inherent reasoning clues via multi-llms boosting, generates meta-LoRA experts via finetuning on disentangled data, and offer customized problem solver with an input-aware hypernet for rank-wise LoRA merging.
    \item \textbf{Novel MC-Tree-of-Agents algorithm for insightful data collection.} We disentangle the System 2 solving process into problem2thought and thought2solution stages, integrating the exploration process into a reflexion-based monte carlo tree search armed with pruning and refinement, enabling mutually verification and boosting of different agents for insightful data collection. %\whj{Should we reorganize it as a multi-agent problem?}
    \item \textbf{Novel DisenLoRA algorithm that offers customized problem solver for robust code generation.} We disentangle the heterogeneous data distribution into clusters on which meta-LoRA experts are trained, and design an input-aware hypernetwork to weight over the LoRA-experts for customized problem solver, offering robustness and flexibility.
\end{itemize}

%\whj{
%The evolution of large language models (LLMs) has revealed distinct cognitive paradigms mirroring human dual-process theory [1]. Commercial gigantic LMs like GPT-4o and Claude-3.5 exhibit System-2-like characteristics - deliberate, multi-step reasoning through chain-of-thought (CoT) paradigms. This contrasts with earlier deep-learning models' System-1-like narrow intelligence in pattern recognition. Yet as (https://openreview.net/pdf?id=0ofzEysK2D) demonstrate, even state-of-the-art LLMs only reach "Level-2 AGI" - equivalent to unskilled human reasoning in a wide range of non-physical tasks - highlighting the need for frameworks that bridge systematic reasoning with specialized skill acquisition.

%We present a efficient pipeline inspired by human cognitive development: System-2-powered exploration followed by System-1 specialization. Our Boost-Disentangle-Compose (BDC) pipeline addresses two fundamental challenges in code generation: 1) The need for high-quality reasoning traces to bootstrap System-1 skill acquisition, and 2) The heterogeneous latent patterns in real-world coding tasks that resist monolithic adaptation approaches. In this paper, we focus on the code generation task, which is shown to be an effective proxy for various purposed tasks, including robotic control and math solving.

%By decomposing the code generation process into problem-to-thought(refining task specification) and thought-to-solution(synthesizing implementations) phases, we further strengthen the collective searching paradigm of the existing reason-and-reflect frameworks by:
%\begin{itemize}
%    \item Fine-grained mutual verification and pruning.
%    \item Complementary refinement boosting generation process across models.
%    \item Dynamic reward shaping via compilation feedback.
%\end{itemize}
%This collaborative reasoning process boosts System-2-like characteristics, yielding up to $13.8\%$ improvement on complex benchmarks.

%Additionally, the high-quallity samples collected from System-2 exploration unfold a resultant solution space featuring high intrinistic heterogeneity. Our analysis shows coding samples naturally clustered into several semantic groups, as visualized in 
%Fig.\ref{fig:mc}. This pattern underlines the importance of specialized skill acquisition via System-1-like frameworks.  We therefore propose DisenLoRA, which adapts parameter-efficient fine-tuning through:
%\begin{itemize}
%    \item Semantic disentanglement of solution clusters.
%    \item LoRA experts as per cluster.
%    \item A input-aware HyperNet for dynamic expert composition.
%\end{itemize}
%The proposed System-1-like framework achieves robust performance on both IID and OOD benchmarks. The dynamic composition with input-awareness also outperform existing adapter ensembling framework by $26\%$.

%The main contributions of our work can be summarized below.
%\begin{itemize}
%    \item \textbf{Identification of problems and novel BDC framework.} We identify the high-reasoning demand and heterogeneous latent patterns problems that hinders the performance of existing methods and propose a BDC framework that explores insightful inherent reasoning clues via multi-llms boosting, generates meta-lora experts via finetuning on disentangled data, and offer customized problem solver with an input-aware hypernet for rank-wise lora merging.
%    \item \textbf{Novel MC-Tree-of-Agents algorithm for insightful data collection.} We disentangle the System 2 solving process into problem2thought and thought2solution stages, integrating the exploration process into a reflexion-based monte carlo tree search armed with pruning and refinement, enabling mutually verification and boosting of different agents for insightful data collection. 
%    \item \textbf{Novel DisenLora algorithm that offers customized problem solver for robust code generation.} We disentangle the heterogeneous data distribution into clusters on which meta-lora experts are trained, and design an input-aware hypernetwork to weight over the lora-experts for customized problem solver, offering robustness and flexibility.
%\end{itemize}

%}


\section{Related Work}
\subsection{System 2 Methods in LLMs}
Recent research on large language models for System 2 tasks focus on inference-time computation optimization to stimulate the inherent reasoning ability of LLMs. Few-shot learning methods \cite{wang2022code4struct,madaan2022language} utilize the in-context-learning ability of LLMs for enhanced generation. Retrieval-augmented generation (RAG) approaches \cite{nashid2023retrieval,du2024codegragbridginggapnatural} further introduce domain knowledge into LLMs. 
Techniques such as Chain-of-Thought (CoT) \cite{yang2024chain,jiang2024self,li2023structured}, Tree-of-Thought (ToT) \cite{yao2024tree,la2024can}, and Monte Carlo Tree Search (MCTS) \cite{li2024rethinkmcts,zhang2023planning,hu2024uncertainty,hao2023reasoning,feng2024alphazeroliketreesearchguidelarge} are used to explore the inherent reasoning process, often based on the self-play mechanism to reflect on previously generated contents to learn from itself \cite{haluptzok2022language,chen2023gaining,lu2023self,chen2023teaching,madaan2024self,shinn2024reflexion}.
During inference, error position can be beneficial in improving the reliability and performance of the model. With identification and analysis of where and why errors occur, recent research \cite{yao2024mulberry, luo2024improve, wu2025error} has made significant strides in quantifying and mitigating errors during model inference. Refinement \cite{madaan2024self, gou2023critic} and reflexion \cite{shinn2024reflexion, lee2025evolving} are also powerful techniques for enhancing the inference capabilities of LLMs, usually by enabling iterative improvement and self-correction.

\subsection{Model Composition}
Model composition technique gains notable attention in cross-tasks generalization. 
Traditional methods for multiple tasks are to train models on a mixture of datasets of different skills \cite{caruana1997multitask, chen2018gradnorm}, with the high cost of data mixing and lack of scalability of the model though. Model merging is a possible solution to this. Linear merging is a classic merging method that consists of simply averaging the model weights \cite{izmailov2018averaging, smith2017investigation}. Furthermore, Task Arithmetic \cite{ilharco2022editing} computes task vectors for each model, merges them linearly, and then adds back to the base, and SLERP \cite{white2016sampling} spherically interpolates the parameters of two models. Based on Task Arithmetic framework, TIES \cite{yadav2024ties} specifies the task vectors and applies a sign consensus algorithm to resolve interference between models, and DARE \cite{yu2024language} matches the performance of original models by random pruning.

Recently, LoRA merging methods are also widely applied to cross-task generalization. CAT \cite{prabhakar2024lora} introduces learnable linear concatenation of the LoRA layers, and Mixture of Experts(MoE) \cite{buehler2024x, feng2024mixture} method has input-dependent merging coefficients. Other linear merging methods of LoRAs, such as LoRA Hub \cite{huang2023lorahub}, involve additional cross-terms compared to simple concatenation. %\whj{这句没看懂}

\section{Preliminaries}
\subsection{Monte-Carlo Tree Search}
Monte Carlo Tree Search (MCTS) is a decision-making algorithm widely used in environments with large state and action spaces, particularly in game AI and planning.  It incrementally builds search trees to estimate optimal actions by simulating random plays from various nodes and gradually improving action-value estimates based on simulation outcomes. Over iterations, this approach gradually converges to near-optimal decision-making policies. Notably, its integration with reinforcement learning has driven breakthroughs in systems like AlphaGo and AlphaZero \cite{silver2017mastering}, achieving superhuman performance in games.
%The combination of MCTS with reinforcement learning has achieved notable successes, such as in AlphaGo and AlphaZero \cite{silver2017mastering}.

Classical MCTS consists of four stages: selection, expansion, simulation, and backpropagation. It typically employs Upper Confidence Bounds for Trees (UCT) \cite{kocsis2006bandit}, which balances exploration and exploitation by guiding the search to promising nodes. After simulation, results propagate back through the tree, updating node values. However, MCTS struggles in domains with large action spaces, where excessive branching can degrade performance. Progressive Widening and Double Progressive Widening techniques have been proposed to mitigate this by dynamically limiting the number of actions considered at each decision node \cite{coulom2006efficient}.

\subsection{LoRA Finetuning}
%\dk{simple introduction and equations}
LoRA (Low-Rank Adaptation) \cite{hu2021lora} fine-tuning is a technique used to adapt large pre-trained models, such as transformers, to specific tasks with minimal computational overhead. The key idea behind LoRA is to introduce low-rank matrices into the model's weight updates, which reduces the number of trainable parameters and makes fine-tuning more efficient. 

LoRA starts with a model that has been trained on a large dataset. During finetuning, instead of updating the full weight matrix $W \in \mathbb{R}^{m \times n}$, LoRA introduces two low-rank matrices $A \in \mathbb{R}^{m \times r}$ and $B \in \mathbb{R}^{r \times n}$, where $r \ll \min(m, n)$. The updated weight matrix $W'$ is then given by:
\begin{equation}
    W' = W + \Delta W = W + A \cdot B.
\end{equation}

During fine-tuning, only the matrices $A$ and $B$ are updated, while the original weight matrix $W$ remains frozen. This reduces the number of trainable parameters from $m \times n$ to $m \times r + r \times n$, which is much smaller when $r$ is small. For a given task with loss function $\mathcal{L}$, the objective is to minimize:
\begin{equation}
\mathcal{L}(y, f(x; W + A \cdot B)),
\end{equation}
where $y$ is the target output, $x$ is the input, and $f$ is the model's forward function.

By introducing low-rank matrices, both the number of trainable parameters and memory footprint are reduced. This approach is particularly useful in scenarios where computational resources are limited or when fine-tuning needs to be done quickly.


\section{Methodology}
In this section, we introduce the overall methodology of BDC, addressing challenges in the System2-to-System1 pipeline for code generation, specifically the complexity of hidden reasoning processes and heterogeneous data distributions. 
The proposed BDC pipeline consists of three main stages: 1) explore the System 2 knowledge via mutual verification and boosting between LLMs; 2) disentangle the obtained data into clusters over which composible LoRA experts are tuned; 3) customize problem solver by weighting over the composable LoRA experts using an input-aware hypernetwork.

\begin{figure*}
    \centering
    \includegraphics[width=1.0\linewidth]{figures/framework.pdf}
   \caption{Illustration of the overall framework of \modelname.}
    \label{fig:overview}
\end{figure*}

\subsection{System 2 Knowledge Exploration}

%\textbf{Select.} The select phase aims to traverse the decision tree by selecting nodes that balance exploration of unvisited nodes with exploitation of high-value nodes. Starting from the root node, the process iterates through the tree by applying a selection policy. In our implementation, the selection process is governed by the Upper Confidence Bound (UCB) strategy, with a variant of probability-weighted UCB (P-UCB) for better exploration.
%Each node is either a DecisionNode (representing agent decisions) or a ChanceNode (representing state-action transitions). When at a DecisionNode, the node selection is made based on the node's estimated value and visitation frequency, guided by the UCB formula:
%$$\mathrm{PUCB}(node)=Q(s,a)+c\cdot P(a|s)\cdot\frac{\sqrt{\log N(s)}}{1+N(s,a)}$$ 

%where $Q(s, a)$ represents the action value, $P(a|s)$ is the policy’s probability of selecting action $a$ in state $s$, and $N(s)$ is the number of visits to state $s$. The selection proceeds until a terminal node or an unvisited node is encountered.

%\textbf{Expand.} Expansion occurs when a previously unvisited state or a ChanceNode is reached during the selection phase. This step involves adding a new DecisionNode to the tree. In our implementation, when a ChanceNode transitions to a new state $s_t$ after an action $a$, the reward $r(s_t ,a)$ is computed. The reward evaluates the correctness of the state $s_t$ based on the output's pass rate, which is quantified by the evaluation function $R(s_t)$, as seen in the equation:
%$$R(s_t)=\mathrm{PassRate}(s_t)$$
%In our setup, the PassRate is determined by the proportion of correct outputs as evaluated by a code correctness checker. Additionally, future rewards are calculated if the node reaches a terminal state, incorporating the step rewards from the current node to future nodes through rollouts. This is expressed as:
%$$r(s_t,a)=R(s_t)+\gamma\cdot\sum_{i=1}^Tr(s_{t+i},a)$$
%Here, $\gamma$ represents the discount factor applied to future rewards during backpropagation, while the total reward is a combination of the current reward and discounted future rewards. The expansion step also adds new DecisionNodes to the tree, allowing subsequent actions to be explored.

%\textbf{Simulate (Rollout).} The Simulate phase, also known as Rollout, begins after a ChanceNode has expanded. In this phase, a sequence of actions is sampled from the current policy, and the rewards are accumulated until a terminal state is reached or the maximum rollout depth is achieved. Each simulation (rollout) evaluates a potential trajectory from the current state, allowing the model to estimate future rewards. The rollout process can be expressed as: 
%\begin{enumerate}
%  \item For each state $s_t$  and action $a$, the state transitions to $s_{t+1}$ according to the transition function $f(s_t, a)$, which is defined as: $$s_{t+1}=f(s_t,a)=s_t+a$$ where $s_t$ is the prompt already generated before the action and $a$ is the continuing prompt (thought).
%  \item If a terminal condition is met, i.e., if the state reaches the terminal token or exceeds the maximum length, a terminal reward $R(s_t)$ is calculated based on the correctness of the output. Otherwise, an intermediate reward of 0 is assigned: $$r(s_t,a)=\begin{cases}R(s_t)&\text{if terminal,}\\0&\text{otherwise.}\end{cases}$$
%  \item The overall reward for the rollout is the sum of rewards across all steps, discounted by $\gamma$ (the discount factor for future rewards): $$\mathrm{RolloutReward}(s_t)=\sum_{i=0}^T\gamma^i\cdot r(s_{t+i},a)$$ Here, $r(s_{t+i} ,a)$ refers to the reward at each step during the rollout, and $T$ is the depth of the simulation, i.e., the number of steps taken until the terminal state or maximum depth is reached.
%\end{enumerate}

%\textbf{Back Up.} Once a terminal state is reached during the simulation, the back up phase updates the values of nodes from the terminal node back to the root. This step ensures that the accumulated rewards and visit counts are propagated upwards through the tree, refining the decision-making process over time: $$Q(s_t,a)\leftarrow r(s_t,a)+\gamma V(s_{t+1})$$ $$V(s_t)\leftarrow\sum_aN(s_{t+1})Q(s_t,a)/\sum_aN(s_{t+1})$$ $$N(s_t)\leftarrow N(s_t)+1$$ where $\gamma$ is the discount factor.

%\ljx{
%In this paper, we describe the multi-llms MCTS procedure employed in our system, adapted to solve sequential decision-making problems under uncertainty, including 2 parts of problem2thought and thought2solution, leveraging probabilistic state transitions and reward signals. Pruning and refinement mechanism are also applied to minimize the influence of wrong thoughts or make some use of them. The overview of our strategies is shown in \ref{fig:copilot}.

%\textbf{Select.} The select phase aims to traverse the decision tree by selecting nodes that balance exploration of unvisited nodes with exploitation of high-value nodes. Starting from the root node, the process iterates through the tree by applying a selection policy. In our implementation, the selection process is governed by the Upper Confidence Bound (UCB) strategy, with a variant of probability-weighted UCB (P-UCB) for better exploration.
%Node selection begins at the root node. It is made based on the node's estimated value and visitation frequency, guided by the UCB formula:
%$$\mathrm{PUCB}(S^T_d)=Q(s)+c\cdot P(a|s_p)\cdot\frac{\sqrt{\log N(s_p)}}{1+N(s)}$$

%where $s$ is the state of the node $S^T_d$, $s_p$ is its parent's state $Q(s_p, a)$ represents the node value, $P(a|s_p)$ is the policy’s probability of selecting action $a$ in state $s_p$, and $N(s)$ is the number of visits to state $s$. The child of the current node with the highest P-UCB value is chosen for the next selection. The selection proceeds until a terminal node or an unvisited node is encountered.

%\textbf{Expand.} Expansion occurs when a previously unvisited state during the selection phase. This step involves adding candidate reasoning nodes as the children of the current selected node. LLMs($\pi_1, \pi_2$) are given the current node $S^T_d$ with state $s$ to generate next thoughts for solving the problem along with the probability$P(a|s)$ of each thought:
%$$a_i, P(a_i|s) \sim \pi_i(·|Q,s, \text{prompt}_{thought})$$

%This is actually the problem2thought process. New generated nodes can be represented by $S^{T_i}_{d+1}$ with state $s_i=s+a_i$, where $T_i = join(T,i)$ and $i$ means that this node is generated by $\pi_i$.

%\textbf{Simulate.}In this operation, LLMs utilizes the selected thoughts to generate solution, i.e. the thought2solution process. Given the selected state, each LLM generates its own version of solution:
%$$So_i\sim \pi_i(·|Q, s)$$
%And we estimate the node reward through the average passrate of the solutions:
%$$R(s)=\frac{1}{n}\sum_{i=1}^n\text{PassRate}(So_i)$$.
%If a terminal condition is met, i.e., if the state reaches the terminal token or exceeds the maximum length, a terminal reward $r(s_t)$ is calculated and $R(s_t)$ is then set to $0$. Otherwise, an intermediate reward of 0 is assigned: $$r(s_t)=\begin{cases}R(s_t)&\text{if terminal,}\\0&\text{otherwise.}\end{cases}$$

%\textbf{Back Up.}After the simulation operation in which we get the node reward, the back up phase updates the values of nodes from bottom to the root. This step ensures that the accumulated rewards and visit counts are propagated upwards through the tree, refining the decision-making process over time:
%$$Q(s_t) \leftarrow f(Q(s_t),r(s_t)+ \gamma Q(s_{t+1}))$$
%$$N(s_t)\leftarrow N(s_t)+1$$
%where states from root to the bottom is represented as $s_0,s_1,...,s_d$, $Q(s_d)=r(s_d)+\gamma R(s_d)$, $f$ is a value calculation function, normally $max$ or $averaging$ all the values that backpropagate, and $\gamma$ is the discount factor.

%\textbf{Prune.}If the pruning mechanism is applied, the new selected node will be examined to show if it is a wrong node. A node is thought to be a wrong node if its reward is less than its parent's, i.e. $R(s_t) < R(s_{t-1})$, which means the new thought brings no useful information. The wrong node will be pruned, so no expansion will be conducted. But \textbf{Back Up} operation is still carried on to reduce the likelihood of selecting this trajectory.

%\textbf{Refine.}After a wrong node is found, we can get its error information and analysis. Then the information can be summarized in natural language so that LLMs can understand it more easily. With the summary, LLMs generate a new thought to replace the past one, which contains information about avoiding current errors. For the sake of debate, wrong thoughts generated by one LLM is analyzed by another LLM. If the new thought is still recognized as a wrong thought, \textbf{Refine} operation is conducted again. But all rethinking times for one problem can't exceed the max thinking times, and each wrong node is regenerated at least once.
%For coding, we utilizes failed test cases and block analysis to generate error summary:
%$$Su(S^{T_i}_d)\sim\pi_j(·|Q,s_d,\text{BlockAnalysis}(s_d))$$
%And the summary and failed test cases are used for refined thought generation:
%$$a_i' \sim \pi_i(·|Q,s_p,Su(S^{T_i}_d),\text{FailedTestCases}(s_d)$$
%Here, only a part of failed test cases are given to the LLM if there are too many. Then, both the rewards of the first wrong thought and the final thought are backpropagated for fair information, and the node left will be expanded.

%\whj{ -------- my draft of Sec4.1 starts from here -------}
In this subsection, we introduce the mechanism design for the data collection process. Due to the complex reasoning nature embodied, code blocks are hard to evaluate and estimate before mature. Reliable reward signals of a reasoning path therefore mainly depend on the dynamic compilation and execution feedbacks, which are extremely sparse and require extensive simulations. To simplify the generation paradigm and exploit the mutual verification capabilities of the collective searching, we decompose the generation process into two distinct stages: problem-to-thought and thought-to-solution.

\subsubsection{Problem-to-thought}

Traditional Monto-Carlo Tree Searching comprises three key operations in each iteration: (a) Select, (b) Expand, (c) Backup. In the problem-to-thought stage, we further extend MCTS by two distinct operations (d) Prune, and (e) Refine to reduce the searching space. We elaborate on these operations as follows.

\minisection{Select} 
Starting from the root, the reasoning path is prolonged by iteratively adding a specific child of the latest node. The operation is usually governed by certain policies, among which we adopt Probability-weighted Upper Confidence Bound(P-UCB) to balance the exploration and exploitation:
\begin{equation}
\mathrm{PUCB}(S_c)=Q(S)+c\cdot P(a|S_p)\cdot\frac{\sqrt{\log N(S)}}{1+N(S_c)},
\end{equation}
where $S_c$ is the state of the child node. $S$ and $Q(S)$ denote the parent node's state and value. $P(a|S)$ is the conditional probability of sampling the sequence $a$. $N(S)$ is the total number of times the parent node $S$ has been visited during simulations, while $N(S_c)$ tracks visits to the child node $S_c$. The selection process will stop if either a semantic or rule-based(e.g. length limits) terminal state encounters.

\minisection{Expand}
The Expand operation is triggered if a non-terminal leaf node of the tree is selected. A set of predefined LLM polices $\pi_0, \cdots, \pi_n$ generate subsequent thought sequences ${a_i}_n$ given the state $S$ of the current node, forming new leaf nodes:
\begin{equation}
    \forall i\in[n], P(a_i|S) \sim \pi_i(·|S).
\end{equation}

%following sequence concatenation as a deterministic transition function $S^i_c = S + a_i$ and question prompt as the initial state, the newly included leaf nodes are set with states ${S_i_c}$.


\minisection{Backup}
For well-defined problems, a reasoning path ${S_t}$ will eventually end at a terminal leaf node $S_T$ by iterating the Select and Expand operations. The reward $r_T$ is set according to the evaluation. We will skip the definition of reward $r_t$ and passrate $PR(S_t)$, which will be detailed in the explanation of the Simulate operation. The reward value is back-propagated along the reasoning path to update the state values of corresponding ancestor nodes:
\begin{equation}
    Q(S_{t-1}) = f(Q(S_t), r_t + \gamma PR(S_t)),
\end{equation}
where $f$ is the value function.

Additionally, the visit counts of ancestors are updated alongside the reasoning path:
\begin{equation}
    N(S_t) = N(S_t) + 1.
\end{equation}



We further extend and formalize reflective reason settings proposed in CoMCTS into Prune and Refine operations as shown in Figure~\ref{fig:mc}.

\begin{figure}[h]
    \centering
    \includegraphics[width=1.0\linewidth]{figures/datacollection.pdf}
    \caption{Pruning and refinement operations.}
    \label{fig:mc}
\end{figure}

\minisection{Pruning}
The pruning operation on a selected node will examine and compare its passrate with that of its parent. With powerful LLMs, we consider valid and reasonable thoughts to bring non-negative influence solution seeking, thus featuring monotonically non-decreasing values in the passrate $PR(S_t) <= PR(S_{t+1})$.

A child node alleviating this principle will be considered as an ill node that introduces wrong thoughts. The ill node will be removed and trigger an instant Backup operation with zero reward to downweight its ancestors.


\minisection{Refine}
The truncated error and state information left by ill nodes will be analyzed in the Refine operations. To mitigate the bootstrapping bias of LLMs, a distinct policy LLM will be adopted to infer and summarize the information in natural language, which will be later utilized to refine and replace the ill nodes:
\begin{align}
    isIll(S^{\pi_i})&== 1,\nonumber\\
    Summary(S^{\pi_i}) &\sim \pi_{j}(Q(S^{\pi_i}),\\
    S^{\pi_i}, & BlockAnalysis(S^{\pi_i})),\nonumber
\end{align}
where $S^{\pi_i}$ denotes a ill node generated by $\pi_i$. A refined node is generated to replace the ill one:
\begin{equation}
    a' \sim \pi_{i}(Q(S^{\pi_i}), Summary).
\end{equation}

We enforce global and local constraints on possible times of calling Refine operation to avoid infinite loops and balance performance with compute budgets. A successful Refine operation will cause an in-place replacement of the ill-node, triggering another Backup operation to re-weight its ancestors.


\subsubsection{Thought-to-solution}

\minisection{Simulate}
For the thought-to-solution, we repurpose the Simulate operation for the collective solution generation process from the given state $S$. The operation will produce a set of possible solutions, each from a policy LLM:
\begin{equation}
    Solut.(S)_i \sim \pi_i(S).
\end{equation}

We define the passrate of a state as the average passrate of its corresponding solutions:
\begin{equation}
PR(S) = \frac{1}{n}\sum_{i}^{n} Passed(Solut.(S)_i),
\end{equation}
where $Passed(\cdot)$ represents the supervising signal from dynamic compilation and execution feedback.

The node's value $Q(S_t)$ is determined by its $PR(S_t)$ and reward $r_t$. Sincere additional solution string will be appended to a non-terminal state $S_t$ before evaluation, $PR(S_t)$ is an indirect supervising signal for the $S_t$, and the direct signal $r_t$ is set to zero.

The terminal state $S_T$ is treated as the unique solution itself since no string concatenation applies, therefore featuring a non-trivial reward $r_T$. Putting everything together, we have:
\begin{equation}
    Q(S) =\begin{cases} r_T &\text{if terminal,}\\\gamma PR(S_t) &\text{otherwise.}\end{cases}
\end{equation}

%\whj{ -------- my draft of Sec4.1 ends from here -------}
%}

\subsection{System2-to-System1 Training}
\subsubsection{Heterogeneous Distribution Disentanglement}
\label{sec:dis}
After the data collection, the resulting training data obtained from the  MC-Tree-Of-Agents process consists of problem2thought data $D^{p2t}=\{\langle X_i^{p2t},y_i^{p2t}\rangle| i\in\mathbf{P}\}$ and thought2solution data $D^{t2s}=\{\langle X_i^{t2s},y_i^{t2s}\rangle| i\in\mathbf{P}\}$: $D_{train} = \{D^{p2t}, D^{t2s}\}$. As discussed in the introduction section, the latent patterns of coding problems are complex and tend to be heterogeneously distributed, e.g., the branching and recursion flow existing in the code blocks, different strategies of algorithm solutions, etc. Therefore, we disentangle the training data based on the latent semantics of the data into different clusters for fine-grained modeling.

The clustering objective can be summarized as below:
\begin{align}
    minimize_{\mathcal{C}} &\sum_k\sum_{i\in C_k} cosine(e_i, \mu_k), \\
    e_i = &\Phi_\theta (\langle X_i, y_i \rangle),\nonumber \\ 
    \mu_k = &mean\{e_i | i\in C_k\},\nonumber
\end{align}
where $\Phi_{theta}$ is the encoder of a code llm and $\mu_k$ denotes the centroid of cluster $C_k$.

\subsubsection{Composable LoRA Experts Preparation}
Having obtained the disentangled data clusters, we then finetune on them to obtain the meta LoRA experts for specialized experts of different aspects.
\begin{align}
    \forall C_k &\in \mathcal{C}, \nonumber\\
    \pi_{\theta_k} & \leftarrow SFT(\pi_{\theta}, \{\langle X_i, y_i\rangle | i \in C_k\}),
\end{align}
where $\pi_\theta$ denotes the base LLM and $\pi_{\theta_k}$ denotes the parameters of the LoRA adapter obtained by finetuning $\pi_\theta$ on $C_k$.

\subsubsection{Input-Aware Hypernetwork for Customized Solver}

Given specialized LoRA experts ${\pi_{\theta_1},\cdots,\pi_{\theta_K}}$ trained on distinct data clusters, we design an input-aware Hypernetwork $f(\cdot)$ to dynamically compose these experts through rank-wise adaption for customized problem solver.

For each input instance, the hypernetwork generates customized expert weights digesting its encoding and semantic distances to the cluster centroids. we identify "rank" as the minimal unit for aggregation and generate rank-wise weights for different experts at each decoding layer:
\begin{equation}
    G_i \leftarrow f(e_i, \langle cosine(e_i, \mu_1), \dots, cosine(e_i, \mu_K)\rangle),
\end{equation}
where $e_i$ is the encoding of input $X_i$, $G_i\in R^{K\times r\times 1}$ is the output weight matrix, $r$ is the rank of the LoRA matrices, and $K$ is the number of LoRA experts.

The aggregated $\Delta W$ of the linear projection layer is then obtained by
\begin{align}
    \mathbf{A}^* = [A_1, \dots, A_K] \odot G_i, \\
    \mathbf{\Delta W}^* = [B_1A_1^*, \dots, B_KA_K^*],\\
    \Delta W = ReduceSum(\mathbf{\Delta W}^*).
\end{align}

%To keep the magnitude of the customized $\Delta W$ from collision, we regularize the obtained $\Delta W$ using an anchored matrix aggregated using distances from each centroid.
%\begin{equation}
%    \Delta W = \Delta W \frac{||\Delta W_{anchor}||}{||\Delta W||},
%\end{equation}
%where $\Delta W_{anchor}$ is the distance-weighted matrix over the LoRA-experts using the distance of input to each centroid and $||\cdot||$ denotes the infinity norm. 

% The resulting $\Delta W$ is then merged into the original weight matrix for forwarding:
The projection output of $\Delta W$ is then merged during forwarding via:
\begin{equation}
    y= W_0x+\Delta Wx.
\end{equation}

%During the process, only $f(\cdot)$ is trainable while other parameters are frozen. 
We adopt a dedicated training phase for the Hypernetwork where all parameters are frozen except for the $f(\cdot)$. The training is supervised by the cross-entropy loss, with the randomly permuted input-output pairs from $D_{train}$.



\section{Experiments}
We conduct empirical studies starting from the following research questions.
\begin{itemize}[leftmargin=27pt]
    \item [\textbf{RQ1}] Does the proposed data collection algorithm explore insightful reasoning knowledge?
    \item [\textbf{RQ2}] Do the complex latent patterns of reasoning data impact the training performance?
    \item [\textbf{RQ3}] Can the disentangle-and-compose mechanism help to promote performance?
    \item [\textbf{RQ4}] Do the proposed input-aware hypernet work outperform other model composition techniques?
    %Does the proposed \modelname model the high-level thought-of-codes? Can \modelname offer cross-lingual augmentation?
    \item [\textbf{RQ5}] How does DisenLoRA perform on untrained datasets?
\end{itemize}

%\whj{Possible experiment data that can directly show the importance of disentanglement: (1) Different models show diverse performance in different stages(p2t, t2s). (2) Models perform better in verifying decomposed thoughts and solutions.}

\subsection{Setup}
In this section, we provide detailed setup information for the evaluation of the proposed \modelname, including datasets, trajectory data collection, and competing methods. 

The overall evaluation is conducted on two benchmark datasets: the competition-style APPS dataset and the CodeContest dataset. Both datasets categorize problems from easy to hard. We randomly sample problem subsets from each category of these two datasets. Each subset contains approximately 100 problems, except for the CodeContest-Hard category, which consists of around 50 problems due to inherent limitation in size. 

We conduct isolated assessments of both stages of \modelname to ensure a comprehensive comparison. 

\minisection{ Data collection} For Python code generation, we compare the performance of MCTS over different methods: zeroshot, LDB \cite{zhong2024ldb}, RAP \cite{hao2023reasoning}, Reflexion, LATS \cite{zhou2023language}, ToT and RethinkMCTS \cite{li2024rethinkmcts}. To mitigate the influence of factors such as context window limitations and instruction-following capabilities, we employ two advanced base models: GPT-4o-mini and Claude-3.5-Sonnet. Aligned with the purpose, we adopt a greedy decoding strategy for both models. Additionally, we provided peer comparisons between these two base models when driven by the MC-Tree-Of-Agents method in terms of their error position and refinement capability.

\minisection{ Fine-tuning} For fine-tuning, \modelname is compared against several alternative methods, including SFT on clustered subsets, TIES, DARES, and TWINS \cite{liu2023twins}.

%\whj{In this section, we provide detailed setup information for evaluating of the proposed \modelname(ours), including datasets, trajectory data collection and competeting methods. 

%The overall evaluation is conducted on two benchmark datasets: the competition-style APPS dataset and the CodeContest dataset. Both datasets categorize problems from easy to hard. We randomly sample problem subsets from each category of these two datasets. Each subset contains approximately 100 problems, except for the CodeContest-Hard category, which consists of around 50 problems due to inherent limitation in size. 

%We conduct isolated assessments of both stages of \modelname to ensure a comphensive comparsion. 

%\minisection{ Data collection} For Python code generation, we compare the perfomance of MCTS over different methods: zeroshot, LDB\cite{zhong2024ldb}, RAP\cite{hao2023reasoning}, Reflexion, LATS\cite{zhou2023language}, ToT and RethinkMCTS\cite{li2024rethinkmcts}. To mitigate the influence of factors such as context window limitations and instruction-following capabilities, we employ two advanced base models: GPT-4o-mini and Claude-3.5-Sonnet. Aligned with the purpose, we adopt greedy decoding strategy(temperature and other args?) for both models. Additionally, we provided peer comparsions between these two base models when driven by the MC-Tree-Of-Agents method in terms of their error position and refinement capability.

%\minisection{ Fine-tuning} For fine-tuning, \modelname is compared against several alternative methods, including SFT on clustered subsets, TIES, DARES and TWINS\cite{liu2023twins}.}

%In this paper, we evaluate \modelname with the competition-style APPS dataset and the CodeContest dataset. Both datasets are categorized by different difficulty levels. For each level of every dataset, we select\whj{random sample?} about 100 problems for testing, except for CodeContest Hard category, which contains about 50 problems. Greedy decoding strategy is applied to the generation. The evaluation metric is PassRate(PR) and Accuracy(AC).

%For data collection, we evaluate the python code generation abilities of GPT4omini and Claude-3.5-sonnet with different methods: zeroshot, LDB\cite{zhong2024ldb}, RAP\cite{hao2023reasoning}, Reflexion, LATS\cite{zhou2023language}, ToT, MCTS and RethinkMCTS\cite{li2024rethinkmcts}. Based on MC-Tree-Of-Agents, we also evaluate both models with error position and refinement. For tuning, we evaluate different finetuning methods including SFT on different clusters, TIES, DARES, TWINS\cite{liu2023twins} and \modelname(ours).

\begin{table*}[ht]
\centering
\caption{Main results on System 2 knowledge exploration.}
\label{tab:datacollection}
\resizebox{0.9\textwidth}{!}{
\begin{tabular}{c|cccccc|cccc} 
\hline
& \multicolumn{6}{c|}{APPS} & \multicolumn{4}{c}{CodeContest}\\
       Models                 & \multicolumn{2}{c}{Intro.}                            & \multicolumn{2}{c}{Inter.} & \multicolumn{2}{c|}{Comp.} & \multicolumn{2}{c}{Easy} & \multicolumn{2}{c}{Hard} \\ \cline{2-11} 
                        & PR                     & AC                    & PR       & AC       & PR       & AC       & PR          & AC         & PR          & AC         \\ \hline
ZeroShot                & 56.56 & 35.00 &      40.57           &     19.00          &   23.67              &     9.00          &29.03& 19.61 &   28.24                 &19.23              \\
LDB                     & 60.64 & 40.00 &     46.78            &     22.00          &      21.00           &    8.00           & 34.76              & 25.58           & 33.52                   & 16.28                \\
RAP                     & 64.24                         & 39.00                         &   43.32              &     14.00          &         22.83        &    8.00           & 43.08              & 33.33            &39.99  &26.92                 \\
Reflexion               & 60.65                         & 40.00                         &  45.58               &     21.00          &      17.50           &     7.00          & 56.16              & 47.83           &34.09 &21.15                 \\
LATS                    & 69.46                         & 50.00                         &  45.86               &     20.00          &       21.83          &        7.00       &57.70              & 47.83           &39.10                    &30.77                 \\
ToT                     & 74.34                         & 55.00                         &  63.49               &     33.00          &         26.30        &       11.00        & 51.89              & 41.18           &49.07                    & 32.69                \\ 
RethinkMCTS                     & 76.60                         & 59.00                        &  74.35               &     49.00         &         42.50        &       28.00        & 60.84             & 51.53          & 55.79                & 48.04               \\ \hline
%MCTS-Line               &                               &                            &                 &               &                 &               &                    &                 &                    &                 \\
%MCTS-Thought            &                               &                            &                 &               &                 &               &                    &                 &                    &                 \\ \hline
Single (GPT4omini) & 77.99                         & 60.00                         &     72.89            &      50.00            &      44.17         &      25.00     & 55.79     &48.04                &      45.72              & 26.92                \\
%Single-MCTS (Yi)        & 70.92                         & 53.75                      &                 &               &                 &               &                    &                 &                    &                 \\
Single (Claude)    &  73.80  &  61.00   &  73.60   & 57.00 &      54.67           &      42.00         &                  58.75& 53.92                &68.41          &55.76             \\ \hline
MC-Tree-Of-Agents              & 79.72                         & 64.00                      &       79.42          &    63.00           &       59.17          &    45.00           & 62.49               &54.64             & 70.49                   &56.41                 \\ 
%+ Error Position (v1)           & 83.07                         & 69.00                      &       78.65          &    64.00           &       57.67          &    41.00           & 65.92               &58.82             & 70.14                   &51.92                 \\
%+ Refine   (v1)        & 83.24                         & 72.00                      &       79.65          &    63.00           &       55.83         &    38.00           & 62.73               &52.94             & 66.95                   &50.00                 \\
+ Pruning           & 85.18                         & 76.00                     & 81.95                & 67.00             &54.00              & 38.00            & 64.62            &  59.80         &73.12             &59.62                 \\
+ Refine      & 81.29                         & 68.00                    & 78.85              & 62.00            & 60.33             & 44.00            & 63.23             &  56.86          & 73.80                   &63.46\\                 \hline
\end{tabular}
}
\end{table*}

% Please add the following required packages to your document preamble:
% \usepackage[table,xcdraw]{xcolor}
% Beamer presentation requires \usepackage{colortbl} instead of \usepackage[table,xcdraw]{xcolor}
% Please add the following required packages to your document preamble:
% \usepackage[table,xcdraw]{xcolor}
% Beamer presentation requires \usepackage{colortbl} instead of \usepackage[table,xcdraw]{xcolor}
\begin{table*}[ht]
\centering
\caption{Main results on System2-to-System1 tuning.}
\label{tab:tuning}
\resizebox{\textwidth}{!}{
\begin{tabular}{ccccccccccccccc}
\hline
\multicolumn{15}{c}{ Meta-llama-3.1-instruct-8b}                                                                                                                                                                                                                  \\ \hline
\multicolumn{1}{c|}{}                                         & \multicolumn{2}{c}{Intro. (100)} & \multicolumn{2}{c}{Inter. (100)} & \multicolumn{2}{c|}{Comp. (100)}  & \multicolumn{2}{c|}{Overall}      & \multicolumn{2}{c}{Easy (102)} & \multicolumn{2}{c|}{Hard (51)}     & \multicolumn{2}{c}{Overall} \\ \cline{2-15} 
\multicolumn{1}{c|}{\multirow{-2}{*}{Finetune Method}}        & PR              & AC             & PR              & AC             & PR    & \multicolumn{1}{c|}{AC}   & PR    & \multicolumn{1}{c|}{AC}   & PR             & AC            & PR    & \multicolumn{1}{c|}{AC}    & PR           & AC           \\ \hline
\multicolumn{1}{c|}{w/o tuning}                               & 21.14           & 4.00           & 20.72           & 4.00           & 12.83 & \multicolumn{1}{c|}{1.00} & 18.23 & \multicolumn{1}{c|}{3.00} & 25.54          & 17.65         & 15.46 & \multicolumn{1}{c|}{5.77}  & 22.18        & 13.69        \\ \hline
\multicolumn{1}{c|}{SFT on all}                               & 22.55           & 7.00           & 26.40           & 3.00           & 10.67 & \multicolumn{1}{c|}{1.00} & 19.87 & \multicolumn{1}{c|}{3.67} & 25.33          & 17.65         & 16.73 & \multicolumn{1}{c|}{7.69}  & 22.46        & 14.33        \\ \hline
\multicolumn{1}{c|}{ SFT on cluster 0} & 20.67           & 6.00           & 24.23           & 3.00           & 11.50 & \multicolumn{1}{c|}{1.00} & 18.80 & \multicolumn{1}{c|}{3.33} & 27.31          & 17.65         & 11.69 & \multicolumn{1}{c|}{1.92}  & 22.10        & 12.41        \\
\multicolumn{1}{c|}{ SFT on cluster 1} & 21.22           & 4.00           & 20.69           & 4.00           & 12.00 & \multicolumn{1}{c|}{2.00} & 17.97 & \multicolumn{1}{c|}{3.33} & 27.78          & 20.59         & 18.12 & \multicolumn{1}{c|}{9.62}  & 24.56        & 16.93        \\
\multicolumn{1}{c|}{SFT on cluster 2}                         & 16.65           & 7.00           & 23.97           & 3.00           & 17.33 & \multicolumn{1}{c|}{4.00} & 19.32 & \multicolumn{1}{c|}{4.67} & 26.82          & 20.59         & 19.50 & \multicolumn{1}{c|}{9.62}  & 24.38        & 16.93        \\ \hline
\multicolumn{1}{c|}{Ties}                                     & 22.75           & 4.00           & 23.06           & 4.00           & 12.67 & \multicolumn{1}{c|}{4.00} & 19.49 & \multicolumn{1}{c|}{4.00} & 26.64          & 21.57         & 18.71 & \multicolumn{1}{c|}{9.62}  & 24.00        & 17.59        \\
\multicolumn{1}{c|}{Dare}                                     & 24.97           & 7.00           & 26.66           & 5.00           & 12.50 & \multicolumn{1}{c|}{3.00} & 21.38 & \multicolumn{1}{c|}{5.00} & 23.05          & 13.73         & 19.65 & \multicolumn{1}{c|}{15.38} & 21.92        & 14.28        \\
\multicolumn{1}{c|}{Twin}                                     & 19.10           & 5.00           & 23.85           & 5.00           & 8.50  & \multicolumn{1}{c|}{1.00} & 17.15 & \multicolumn{1}{c|}{3.67} & 26.87          & 17.64         & 12.92 & \multicolumn{1}{c|}{9.62}  & 22.22        & 14.97        \\ \hline
\multicolumn{1}{c|}{DisenLoRA}                & \textbf{27.11}           & \textbf{9.00}           & 23.11           & 3.00           & 11.50 & \multicolumn{1}{c|}{\textbf{4.00}} & \textbf{20.57} & \multicolumn{1}{c|}{\textbf{5.33}} & \textbf{32.24}          & \textbf{22.55}         & 19.43 & \multicolumn{1}{c|}{9.62}  & \textbf{27.97}        & \textbf{18.24}        \\ \hline
\end{tabular}
}
\end{table*}

\subsection{Empirical Analysis and Discussion}
\subsubsection{\textbf{RQ1}. MC-Tree-Of-Agents}

We evaluate MC-Tree-Of-Agents against widely-used baseline methods, the results are summarized in Table~\ref{tab:datacollection}.
From the results, we can draw the following conclusions. 
\begin{itemize}[leftmargin=10pt]
    \item The proposed MC-Tree-Of-Agents outperforms all the baseline methods, which effectively explores the insightful  System 2 knowledge. 
    \item Comparing with the single LLM as agents version, MC-Tree-Of-Agents allows for mutual verification and boosting between different LLMs, offering a superior performance over each distinct-LLM-as-agent method. This showcases the effectiveness of the interaction between LLMs of different wisdom.
    \item The pruning and refinement operations both contribute to the final performance, offering a notable accuracy gain. This validates that the designed pruning and refinement mechanism, based on the difference between rewards of parent-child nodes, can save the algorithm from erroneous exploration and lead to beneficial directions in limited rollouts.
\end{itemize}


\subsubsection{\textbf{RQ2}. Impact of latent patterns}

To study the distribution of the latent patterns of coding problems, we first conduct the T-SNE visualization on the encodings of reasoning data collected by MC-Tree-Of-Agents on APPS dataset. The visualization is displayed in Figure~\ref{fig:t-sne}.
%\vspace{-10pt}
\begin{figure}[h!]
    \centering
    \includegraphics[width=0.8\linewidth]{figures/apps_tsne.png}
    \caption{T-sne visualization of the APPS data encoding.}
    \vspace{-10pt}
    \label{fig:t-sne}
\end{figure}

From the visualization, we can see that there different clusters of data distributions existing in the latent reasoning semantic space, which poses a potential challenge to robust and generalizable LLMs on code.

Furthermore, we perform finetuning on different clusters of data obtained in Section~\ref{sec:dis} and evaluate the corresponding models on the test data. From the results in Table~\ref{tab:tuning}, we can see the following conclusions. 1) LLMs finetuning on all the clusters can offer better performance than that of the non-tuning version, validating the quality of the collected System2 knowledge data. 2) Llm experts obtained from different clusters show different performances on different levels of tasks. One expert can demonstrate outstanding capability on one level of tasks, even outperforming the LLM finetuning on all the data, while performing weakly on a different level of task. This phenomenon further justifies the heterogeneous latent patterns of data distribution and serves as supportive evidence for disentangling LLM experts.

\subsubsection{\textbf{RQ3}. Effectiveness of the Experts Composition}

During the empirical study, we test different model merging methods that combine wisdom from different LoRA experts. We evaluate the well-known Ties, Dare, and the recently proposed TWIN merging methods. All of them yield a static composed model that takes in the strength of the candidate experts to be merged via solving parameter interference. From the results, we can see that merging over decomposed LoRA-experts can offer more robust problem solvers, outperforming the simple train-once-for-all mechanism. The experiments justify our major rationale that disentanglement-and-compose pipeline can offer more robust System2-to-System1 performance.

\subsubsection{\textbf{RQ4}. Superiority of DisenLoRA over other composition methods}

Although the static-composed expert model can promote robustness to some extent, its static nature lacks flexibility to different styles of inputs. As discussed in the previous contents, the data distribution of coding problems is complex, making the one-fits-all mechanism easy to fail. Therefore, we design DisenLoRA algorithm to yield a customized problem solver with input-awareness. From the results, we can see that  DisenLoRA outperforms the competing merging methods, validating the effectiveness of the proposed input-aware hypernetwork that dynamically aggregates the candidate composable LoRA experts at a rank-wise level.

\subsubsection{\textbf{RQ5}. Discussion of the Cross-Dataset Generalization of DisenLoRA}

Despite the flexibility offered by the input-aware hypernetwork, its performance may degrade on new datasets where the hypernetwork is not trained. To study this scenario, we use the model trained on APPS to generate solutions for CodeContest and use the model trained on CodeContest to generate solutions for APPS. The results are displayed in Table~\ref{tab:ood4code}.

\begin{table}[h]
    \centering
    \caption{Cross-dataset generalization test.}
    \label{tab:ood4code}
    \resizebox{0.45\textwidth}{!}{
    \begin{tabular}{c|cc|cc}
    \hline
        OOD Dataset & \multicolumn{2}{c|}{APPS}&\multicolumn{2}{c}{CodeContest}\\\hline
      Method   & PR & AC &PR & AC\\
      \hline
      w/o tuning   & 18.23 &	3.00 & 22.18 &	13.69 \\
      w/ SFT & 17.44 &	4.33 & 20.99	&14.29 \\
      \hline
      DisenLoRA & 18.25 &	4.33& 25.09&14.34\\ \hline
    \end{tabular}
    }
\end{table}

From the results, we can see that the proposed DisenLoRA has the generalization ability to the untrained dataset, outperforming the train-once-for-all mechanism still. This demonstrates that the parameters of the trained hypernetwork have the awareness of semantic similarities across datasets.


\section{Conclusion}
We identify the complexity of inherent reasoning exploration and the heterogeneous data distribution problems that hinder the performance of System2-to-System1 methods. Correspondingly, we propose the BDC pipeline that explores insightful System2 knowledge via mutually \textbf{B}oosting between llm agents, \textbf{D}isentangle heterogeneous data distribution for composable LoRA experts, and \textbf{C}ustomize problem solver for each instance, offering flexibility and robustness.
Correspondingly, we propose the MC-Tree-Of-Agents algorithm to efficiently and effectively explore the insightful System2 knowledge via mutual verification and boosting of different LLM agents, armed with reward-guided pruning and refinement to explore more beneficial states in limited rollouts for better performance. 
Additionally, we design an input-aware hypernetwork to aggregate over the disentangled composable LoRA experts trained on different clusters of data collected from MC-Tree-Of-Agents. This mechanism offers a customized problem solver for each data instance.
Various experiments and discussions validate the effectiveness of different model components.

\section*{Limitations}
%\whj{
While our work presents an efficient pipeline for transferring specialized knowledge from collective system-2-like LLMs to locally deployed language models through multiple LoRA adapters—enabling rapid, precise, system-1-like reasoning—three limitations merit discussion. First, despite code generation serving as an effective proxy for complex reasoning, our evaluation is restricted to this domain, leaving open questions about generalizability to broader textual reasoning tasks (e.g., commonsense reasoning and semantic parsing). Second, while we focus on their performance on the specific benchmarks, the safety alignment of derived models remains unaddressed. Systematic evaluation is required to assess whether our distilled experts preserve human values and mitigate harmful outputs. Finally, our ensemble methodology for LoRA experts, while input-aware, does not fully exploit potential sparsity optimizations in parameter activation, leaving room for computational efficiency improvements through advanced routing mechanisms.
%}

\section*{Acknowledgments}

This document has been adapted by Emily Allaway from the instructions for earlier ACL and NAACL proceedings, including those for NAACL 2024 by Steven Bethard, Ryan Cotterell and Rui Yan,
ACL 2019 by Douwe Kiela and Ivan Vuli\'{c},
NAACL 2019 by Stephanie Lukin and Alla Roskovskaya,
ACL 2018 by Shay Cohen, Kevin Gimpel, and Wei Lu,
NAACL 2018 by Margaret Mitchell and Stephanie Lukin,
Bib\TeX{} suggestions for (NA)ACL 2017/2018 from Jason Eisner,
ACL 2017 by Dan Gildea and Min-Yen Kan,
NAACL 2017 by Margaret Mitchell,
ACL 2012 by Maggie Li and Michael White,
ACL 2010 by Jing-Shin Chang and Philipp Koehn,
ACL 2008 by Johanna D. Moore, Simone Teufel, James Allan, and Sadaoki Furui,
ACL 2005 by Hwee Tou Ng and Kemal Oflazer,
ACL 2002 by Eugene Charniak and Dekang Lin,
and earlier ACL and EACL formats written by several people, including
John Chen, Henry S. Thompson and Donald Walker.
Additional elements were taken from the formatting instructions of the \emph{International Joint Conference on Artificial Intelligence} and the \emph{Conference on Computer Vision and Pattern Recognition}.


\bibliography{custom}

\appendix
%\section{Implementation Choices}
\label{sec:impl}

In this section, we briefly discuss the design choices made in our implementation of \lithe.

\vspace{-0.1in}
\subsection{\lithe Parameter Settings}
\label{sec:llm-params}

The \emph{``temperature''} parameter of \gpt, which ranges over [0,1], controls the randomness of the model's response.
While a higher temperature can be useful for creative writing where one would seek diverse and exploratory answers, in our case we want a focused and deterministic answer as far as possible. Hence we set the temperature to 0 which forces the model to greedily sample the next token.


The hyperparameters used by \lithe for MCTS are as follows: The maximum number of iterations $iter_{max}$ is set to 8,  expansion threshold $\theta$ is 0.7, and number of expansions $k$ is 2.
The values of $c_{base}$ and $c$ were set to 10 and 4, respectively.
%
These settings were determined after an empirical evaluation of the various trade-offs, providing a robust balance between efficiency and quality.
%

Finally, we try a maximum of 5 times to fix, via prompt corrections, any rewrite that exhibits syntax errors (Section~\ref{sec:lithe-architecture}).

\vspace{-0.1in}
\subsection{Query Equivalence Testing}
\label{sec:sql-equivalence}
We use a multi-stage approach, described below, to test semantic equivalence between the original query and a candidate rewrite.

\myparagraph{1. Logic-based Equivalence.}
Although verifying the equivalence of a general pair of SQL queries is NP-complete~\cite{queryequivalence}, a variety of logic-based tools (e.g. Cosette\cite{Cosette}, SQL-Solver~\cite{SQLSolver}, VeriEQL~\cite{verieql}, QED~\cite{QED}) are available for proving equivalence over restricted classes of queries, as mentioned in the Introduction. 
%
In \lithe, we use the recently proposed QED~\cite{QED} since it was found to cover a larger set of queries compared to the alternatives. 
%
The advantage of such a logic-based approach is that it is definitive in outcome and relatively inexpensive. 

\myparagraph{2. Result Equivalence via Sampling.}
%
If the original query is not within the QED scope, we alternatively use a sampling-based approach to test equivalence. The idea here is to execute the queries on several small samples of the database and verify equivalence based on the sample results. 
%
However, while this test is a necessary condition for query equivalence, it is not sufficient. That is,  false positives may be present because the sampled database may not cover all the predicates featured in the query. To minimize this likelihood, we use a combination of (1) \textit{correlated sampling}~\cite{cs2} for maintaining join integrity in the sample, (2) adding synthetic tuples in the sample to distinguish outer and inner joins, and (3) adjusting constants in the filter predicates to produce populated results -- the complete details are in the Section~\ref{app:sampling-eq}. 

\myparagraph{3. Result Equivalence on the Entire Database.}
%
Result equivalence could also be evaluated, in principle, on the entire database itself. Of course, this could turn out to be prohibitively expensive, especially if the queries themselves are time-consuming (e.g. due to the scale of the underlying database) and/or if the candidate rewrites happen to be regressions. Therefore, we leave this check as an optional choice for the DBA.


% \section{Methodology}

% \begin{figure*}
%     \centering
%     \includegraphics[width=1.0\linewidth]{figures/DebateThinker.pdf}
%     \caption{Illustration of the overall framework of \modelname.\dk{Todo: replace it.}}
%     \label{fig:copilot}
% \end{figure*}

% \subsection{Data collection}
% In this paper, we describe the MCTS procedure employed in our system, adapted to solve sequential decision-making problems under uncertainty, leveraging probabilistic state transitions and reward signals.

% \textbf{Select.} The select phase aims to traverse the decision tree by selecting nodes that balance exploration of unvisited nodes with exploitation of high-value nodes. Starting from the root node, the process iterates through the tree by applying a selection policy. In our implementation, the selection process is governed by the Upper Confidence Bound (UCB) strategy, with a variant of probability-weighted UCB (P-UCB) for better exploration.
% Each node is either a DecisionNode (representing agent decisions) or a ChanceNode (representing state-action transitions). When at a DecisionNode, the node selection is made based on the node's estimated value and visitation frequency, guided by the UCB formula:
% $$\mathrm{PUCB}(node)=Q(s,a)+c\cdot P(a|s)\cdot\frac{\sqrt{\log N(s)}}{1+N(s,a)}$$

% where $Q(s, a)$ represents the action value, $P(a|s)$ is the policy’s probability of selecting action $a$ in state $s$, and $N(s)$ is the number of visits to state $s$. The selection proceeds until a terminal node or an unvisited node is encountered.

% \textbf{Expand.} Expansion occurs when a previously unvisited state or a ChanceNode is reached during the selection phase. This step involves adding a new DecisionNode to the tree. In our implementation, when a ChanceNode transitions to a new state $s_t$ after an action $a$, the reward $r(s_t ,a)$ is computed. The reward evaluates the correctness of the state $s_t$ based on the output's pass rate, which is quantified by the evaluation function $R(s_t)$, as seen in the equation:
% $$R(s_t)=\mathrm{PassRate}(s_t)$$
% In our setup, the PassRate is determined by the proportion of correct outputs as evaluated by a code correctness checker. Additionally, future rewards are calculated if the node reaches a terminal state, incorporating the step rewards from the current node to future nodes through rollouts. This is expressed as:
% $$r(s_t,a)=R(s_t)+\gamma\cdot\sum_{i=1}^Tr(s_{t+i},a)$$
% Here, $\gamma$ represents the discount factor applied to future rewards during backpropagation, while the total reward is a combination of the current reward and discounted future rewards. The expansion step also adds new DecisionNodes to the tree, allowing subsequent actions to be explored.

% \textbf{Simulate (Rollout).} The Simulate phase, also known as Rollout, begins after a ChanceNode has expanded. In this phase, a sequence of actions is sampled from the current policy, and the rewards are accumulated until a terminal state is reached or the maximum rollout depth is achieved. Each simulation (rollout) evaluates a potential trajectory from the current state, allowing the model to estimate future rewards. The rollout process can be expressed as: 
% \begin{enumerate}
%   \item For each state $s_t$  and action $a$, the state transitions to $s_{t+1}$ according to the transition function $f(s_t, a)$, which is defined as: $$s_{t+1}=f(s_t,a)=s_t+a$$ where $s_t$ is the prompt already generated before the action and $a$ is the continuing prompt (thought).
%   \item If a terminal condition is met, i.e., if the state reaches the terminal token or exceeds the maximum length, a terminal reward $R(s_t)$ is calculated based on the correctness of the output. Otherwise, an intermediate reward of 0 is assigned: $$r(s_t,a)=\begin{cases}R(s_t)&\text{if terminal,}\\0&\text{otherwise.}\end{cases}$$
%   \item The overall reward for the rollout is the sum of rewards across all steps, discounted by $\gamma$ (the discount factor for future rewards): $$\mathrm{RolloutReward}(s_t)=\sum_{i=0}^T\gamma^i\cdot r(s_{t+i},a)$$ Here, $r(s_{t+i} ,a)$ refers to the reward at each step during the rollout, and $T$ is the depth of the simulation, i.e., the number of steps taken until the terminal state or maximum depth is reached.
% \end{enumerate}

% \textbf{Back Up.} Once a terminal state is reached during the simulation, the back up phase updates the values of nodes from the terminal node back to the root. This step ensures that the accumulated rewards and visit counts are propagated upwards through the tree, refining the decision-making process over time: $$Q(s_t,a)\leftarrow r(s_t,a)+\gamma V(s_{t+1})$$ $$V(s_t)\leftarrow\sum_aN(s_{t+1})Q(s_t,a)/\sum_aN(s_{t+1})$$ $$N(s_t)\leftarrow N(s_t)+1$$ where $\gamma$ is the discount factor.

% \subsection{Disentangled Lora Training}

% \subsection{Input-Aware Hypernetwork}

% \section{Experiment}
% \begin{itemize}[leftmargin=27pt]
%     \item [\textbf{RQ1}] Does the proposed \modelname offer performance gain against the base model?
%     \item [\textbf{RQ2}] Does the disentanglement help in training?
%     \item [\textbf{RQ3}] Can soft prompting enhance the capability of the backbone LLMs? Does finetuning with the soft prompting outperforms the simple supervised finetuning?
%     \item [\textbf{RQ4}] Are the proposed pretraining objectives for the GNN expert effective?
%     %Does the proposed \modelname model the high-level thought-of-codes? Can \modelname offer cross-lingual augmentation?
%     \item [\textbf{RQ5}] What is the impact of each of the components of the graphical view?
%     \item [\textbf{RQ6}] How is the compatibility of the graphical view? 
% \end{itemize}

% \subsection{Setup}
% In this paper, we evaluate \modelname with the competition-style APPS dataset and the CodeContest dataset. Both datasets are categorized by different difficulty levels. For each level of every dataset, we select about 100 problems for testing, except for CodeContest Hard category, which contains about 50 problems. Greedy decoding strategy is applied to the generation. The evaluation metric is PassRate(PR) and Accuracy(AC).

% For data collection, we evaluate the python code generation abilities of GPT4omini and Claude-3.5-sonnet with different methods: zeroshot, LDB\cite{zhong2024ldb}, RAP\cite{hao2023reasoning}, Reflexion, LATS\cite{zhou2023language}, ToT, MCTS and RethinkMCTS\cite{li2024rethinkmcts}. Based on multi-MCTS, we also evaluate both models with error position and refinement. For tuning, we evaluate different finetuning methods including SFT on different clusters, TIES, DARES, TWINS\cite{liu2023twins} and \modelname(ours).
% \begin{table*}[ht]
% \centering
% \caption{Main results on data collection.}
% \label{tab:code}
% \resizebox{\textwidth}{!}{
% \begin{tabular}{c|cccccc|cccc} 
% \hline
% & \multicolumn{6}{c|}{APPS} & \multicolumn{4}{c}{CodeContest}\\
%        Models                 & \multicolumn{2}{c}{Intro.}                            & \multicolumn{2}{c}{Inter.} & \multicolumn{2}{c|}{Comp.} & \multicolumn{2}{c}{Easy} & \multicolumn{2}{c}{Hard} \\ \cline{2-11} 
%                         & PR                     & AC                    & PR       & AC       & PR       & AC       & PR          & AC         & PR          & AC         \\ \hline
% ZeroShot                & 56.56 & 35.00 &      40.57           &     19.00          &   23.67              &     9.00          &29.03& 19.61 &   28.24                 &19.23              \\
% LDB                     & 60.64 & 40.00 &     46.78            &     22.00          &      21.00           &    8.00           & 34.76              & 25.58           & 33.52                   & 16.28                \\
% RAP                     & 64.24                         & 39.00                         &   43.32              &     14.00          &         22.83        &    8.00           & 43.08              & 33.33            &39.99  &26.92                 \\
% Reflexion               & 60.65                         & 40.00                         &  45.58               &     21.00          &      17.50           &     7.00          & 56.16              & 47.83           &34.09 &21.15                 \\
% LATS                    & 69.46                         & 50.00                         &  45.86               &     20.00          &       21.83          &        7.00       &57.70              & 47.83           &39.10                    &30.77                 \\
% ToT                     & 74.34                         & 55.00                         &  63.49               &     33.00          &         26.30        &       11.00        & 51.89              & 41.18           &49.07                    & 32.69                \\ 
% RethinkMCTS                     & 76.60                         & 59.00                        &  74.35               &     49.00         &         42.50        &       28.00        & 60.84             & 51.53          & 55.79                & 48.04               \\ \hline
% %MCTS-Line               &                               &                            &                 &               &                 &               &                    &                 &                    &                 \\
% %MCTS-Thought            &                               &                            &                 &               &                 &               &                    &                 &                    &                 \\ \hline
% Single (GPT4omini) & 77.99                         & 60.00                         &     72.89            &      50.00            &      44.17         &      25.00     & 55.79     &48.04                &      45.72              & 26.92                \\
% %Single-MCTS (Yi)        & 70.92                         & 53.75                      &                 &               &                 &               &                    &                 &                    &                 \\
% Single (Claude)    &  73.80  &  61.00   &  73.60   & 57.00 &      54.67           &      42.00         &                  58.75& 53.92                &68.41          &55.76             \\ \hline
% Multi-MCTS              & 79.72                         & 64.00                      &       79.42          &    63.00           &       59.17          &    45.00           & 62.49               &54.64             & 70.49                   &56.41                 \\ 
% + Error Position (v1)           & 83.07                         & 69.00                      &       78.65          &    64.00           &       57.67          &    41.00           & 65.92               &58.82             & 70.14                   &51.92                 \\
% + Refine   (v1)        & 83.24                         & 72.00                      &       79.65          &    63.00           &       55.83         &    38.00           & 62.73               &52.94             & 66.95                   &50.00                 \\
% + Error Position (v2)           & 85.18                         & 76.00                     & 81.95                & 67.00             &54.00              & 38.00            & 64.62            &  59.80         &73.12             &59.62                 \\
% + Refine   (v2)        &                          &                     &               &             & 60.33             & 44.00            &              &            & 73.80                   &63.46\\                 \hline
% \end{tabular}
% }
% \end{table*}

% % Please add the following required packages to your document preamble:
% % \usepackage[table,xcdraw]{xcolor}
% % Beamer presentation requires \usepackage{colortbl} instead of \usepackage[table,xcdraw]{xcolor}
% % Please add the following required packages to your document preamble:
% % \usepackage[table,xcdraw]{xcolor}
% % Beamer presentation requires \usepackage{colortbl} instead of \usepackage[table,xcdraw]{xcolor}
% \begin{table*}[ht]
% \centering
% \caption{Main results on \modelname tuning.}
% \label{tab:tuning}
% \resizebox{\textwidth}{!}{
% \begin{tabular}{ccccccccccccccc}
% \hline
% \multicolumn{15}{c}{\cellcolor[HTML]{EFEFEF}{Meta-llama-3.1-instruct-8b}}\\ \hline
% \multicolumn{1}{c|}{}                                         & \multicolumn{8}{c|}{IID Dataset (APPS Test)}                                                                                               & \multicolumn{6}{c}{OOD Dataset (CodeContest)}                                                     \\ \cline{2-15} 
% \multicolumn{1}{c|}{}                                         & \multicolumn{2}{c}{Intro.   } & \multicolumn{2}{c}{Inter.   } & \multicolumn{2}{c|}{Comp.   }  & \multicolumn{2}{c|}{Overall}      & \multicolumn{2}{c}{Easy   } & \multicolumn{2}{c|}{Hard   }     & \multicolumn{2}{c}{Overall} \\ \cline{2-15} 
% \multicolumn{1}{c|}{\multirow{-3}{*}{Finetune method}}        & PR              & AC             & PR              & AC             & PR    & \multicolumn{1}{c|}{AC}   & PR    & \multicolumn{1}{c|}{AC}   & PR             & AC            & PR    & \multicolumn{1}{c|}{AC}    & PR           & AC           \\ \hline
% \multicolumn{1}{c|}{w/o tuning}                               & 21.14           & 4.00           & 20.72           & 4.00           & 12.83 & \multicolumn{1}{c|}{1.00} & 18.23 & \multicolumn{1}{c|}{3.00} & 25.54          & 17.65         & 15.46 & \multicolumn{1}{c|}{5.77}  & 22.18        & 13.69        \\ \hline
% \multicolumn{1}{c|}{SFT on all}                               & 22.55           & 7.00              & 26.4            & 3.00              & 10.67 & \multicolumn{1}{c|}{1.00}    & 19.87 & \multicolumn{1}{c|}{3.67} & 22.42          & 14.71         & 18.12 & \multicolumn{1}{c|}{13.46} & 20.99        & 14.29        \\ \hline
% \multicolumn{1}{c|}{ SFT on cluster 0} & 20.67           & 6.00              & 24.23           & 3.00              & 11.5  & \multicolumn{1}{c|}{1.00}    & 18.80 & \multicolumn{1}{c|}{3.33} & 31.14          & 20.59         & 19.89 & \multicolumn{1}{c|}{11.54} & 27.39        & 17.57        \\
% \multicolumn{1}{c|}{ SFT on cluster 1} & 21.22           & 4.00              & 20.69           & 4.00              & 12.00    & \multicolumn{1}{c|}{2.00}    & 17.97 & \multicolumn{1}{c|}{3.33} & 29.4           & 21.57         & 17.39 & \multicolumn{1}{c|}{9.62}  & 25.40        & 17.59        \\
% \multicolumn{1}{c|}{SFT on cluster 2}                         & 16.65           & 7.00              & 23.97           & 3.00              & 17.33 & \multicolumn{1}{c|}{4.00}    & 19.32 & \multicolumn{1}{c|}{4.67} & 23.94          & 16.67         & 18.32 & \multicolumn{1}{c|}{7.69}  & 22.07        & 13.68        \\ \hline
% \multicolumn{1}{c|}{Ties}                                     & 22.75           & 4.00              & 23.06           & 4.00              & 12.67 & \multicolumn{1}{c|}{4.00}    & 19.49 & \multicolumn{1}{c|}{4.00} & 24.14          & 13.73         & 11.58 & \multicolumn{1}{c|}{5.77}  & 19.95        & 11.08        \\
% \multicolumn{1}{c|}{Dare}                                     & 24.97           & 7.00              & 26.66           & 5.00              & 12.5  & \multicolumn{1}{c|}{3.00}    & 21.38 & \multicolumn{1}{c|}{5.00} & 30.59          & 24.51         & 12.6  & \multicolumn{1}{c|}{7.69}  & 24.59        & 18.90        \\
% \multicolumn{1}{c|}{Twin}                                     & 19.1            & 5.00              & 23.85           & 5.00              & 8.5   & \multicolumn{1}{c|}{1.00}    & 17.15 & \multicolumn{1}{c|}{3.67} & 26.11          & 16.67         & 15.43 & \multicolumn{1}{c|}{3.85}  & 22.55        & 12.40        \\ \hline
% \multicolumn{1}{c|}{\modelname}                 & \textbf{27.11}           & \textbf{9.00}              & 23.11           & 3.00              & 11.5  & \multicolumn{1}{c|}{\textbf{4.00}}    & \textbf{20.57} & \multicolumn{1}{c|}{\textbf{5.33}} & 27.91          & 18.63         & 19.45 & \multicolumn{1}{c|}{5.77}  & 25.09        & 14.34        \\ \hline
% \end{tabular}
% }
% \end{table*}


% \section{Conclusion}


% \section*{Limitations}
% In this paper, we propose a graphical retrieval augmented generation method that can offer enhanced code generation. Despite the efficiency and effectiveness, there are also limitations within this work.  For example, dependency on the quality of the external knowledge base could be a potential concern. The quality of the external knowledge base could be improved with regular expression extraction on the noisy texts and codes. 


%\section*{Ethics Statement}

%\section*{Acknowledgements}




% \section*{Acknowledgments}

% This document has been adapted by Emily Allaway from the instructions for earlier ACL and NAACL proceedings, including those for NAACL 2024 by Steven Bethard, Ryan Cotterell and Rui Yan,
% ACL 2019 by Douwe Kiela and Ivan Vuli\'{c},
% NAACL 2019 by Stephanie Lukin and Alla Roskovskaya,
% ACL 2018 by Shay Cohen, Kevin Gimpel, and Wei Lu,
% NAACL 2018 by Margaret Mitchell and Stephanie Lukin,
% Bib\TeX{} suggestions for (NA)ACL 2017/2018 from Jason Eisner,
% ACL 2017 by Dan Gildea and Min-Yen Kan,
% NAACL 2017 by Margaret Mitchell,
% ACL 2012 by Maggie Li and Michael White,
% ACL 2010 by Jing-Shin Chang and Philipp Koehn,
% ACL 2008 by Johanna D. Moore, Simone Teufel, James Allan, and Sadaoki Furui,
% ACL 2005 by Hwee Tou Ng and Kemal Oflazer,
% ACL 2002 by Eugene Charniak and Dekang Lin,
% and earlier ACL and EACL formats written by several people, including
% John Chen, Henry S. Thompson and Donald Walker.
% Additional elements were taken from the formatting instructions of the \emph{International Joint Conference on Artificial Intelligence} and the \emph{Conference on Computer Vision and Pattern Recognition}.

% Bibliography entries for the entire Anthology, followed by custom entries
%\bibliography{anthology,custom}
% Custom bibliography entries only
% \bibliography{custom}

% \appendix
% \section{Implementation Details}
% \label{app:setup}
% For the size of retrieval pool, we use 11,913 C++ code snippets and 2,359 python code snippets. Due to the limited access, we do not use a large retrieval corpus for our experiment, which can be enlarged by other people for better performance. We also attach the graph extraction codes for both languages and all other expeirment codes here: https://anonymous.4open.science/r/Code-5970/

% For the fintuning details, the learning rate and weight decay for the expert GNN training is 0.001 and 1e-5, repectively. We apply 8-bit quantization and use LoRA for parameter-efficient fine-tuning. The rank of the low-rank matrices in LoRA is uniformly set to 8, alpha set to 16, and dropout is set to 0.05. The LoRA modules are uniformly applied to the Q and V parameter matrices of the attention modules in each layer of the LLM. All the three models are optimized using the AdamW optimizer. For the CodeContest dataset, totally 10609 datapoints are used, and for APPS dataset, 8691 data samples are used to train the model.


\end{document}
