% This must be in the first 5 lines to tell arXiv to use pdfLaTeX, which is strongly recommended.
\pdfoutput=1
% In particular, the hyperref package requires pdfLaTeX in order to break URLs across lines.

\documentclass[11pt]{article}

% Change "review" to "final" to generate the final (sometimes called camera-ready) version.
% Change to "preprint" to generate a non-anonymous version with page numbers.
%\usepackage[review]{acl}
\usepackage[preprint]{acl}

% Standard package includes
\usepackage{times}
\usepackage{latexsym}
\usepackage{graphicx}
\usepackage{multirow}
\usepackage{amsmath,enumitem}
\usepackage{booktabs}

%\usepackage[table]{xcolor}
%\usepackage{xcolor}

% For proper rendering and hyphenation of words containing Latin characters (including in bib files)
\usepackage[T1]{fontenc}
% For Vietnamese characters
% \usepackage[T5]{fontenc}
% See https://www.latex-project.org/help/documentation/encguide.pdf for other character sets

% This assumes your files are encoded as UTF8
\usepackage[utf8]{inputenc}

% This is not strictly necessary, and may be commented out,
% but it will improve the layout of the manuscript,
% and will typically save some space.
\usepackage{microtype}

% This is also not strictly necessary, and may be commented out.
% However, it will improve the aesthetics of text in
% the typewriter font.
\usepackage{inconsolata}

%Including images in your LaTeX document requires adding
%additional package(s)
\usepackage{graphicx}

% If the title and author information does not fit in the area allocated, uncomment the following
%
%\setlength\titlebox{<dim>}
%
% and set <dim> to something 5cm or larger.

\usepackage{inconsolata}
\usepackage{mdframed}
\usepackage{makecell}
\usepackage{amssymb}

\usepackage{amsmath}
\usepackage{algorithm}
\usepackage{algpseudocode}

\usepackage{xspace}
\newcommand{\modelname}{BDC\xspace}

\newcommand{\weinan}[1]{{\color{blue}[\textbf{weinan: #1}]}}
\newcommand{\minisection}[1]{\vspace{0pt}\noindent\textbf{#1.}}
\newcommand{\dk}[1]{{\color{orange}[\textbf{dk: #1}]}}
\newcommand{\ljx}[1]{{\color{cyan}[\textbf{ljx: #1}]}}
\newcommand{\whj}[1]{{\color{brown}[\textbf{whj: #1}]}}

%\title{\modelname: Learning to Weight Over Disentangled Experts \\ for Robust Code Generation}

\title{Boost, Disentangle, and Customize: \\
A Robust System2-to-System1 Pipeline for Code Generation}


\author{Kounianhua Du$^1$, Hanjing Wang$^1$, Jianxing Liu$^1$, Jizheng Chen$^1$, \\
{\bf Xinyi Dai$^2$, Yasheng Wang$^2$, Ruiming Tang$^2$, Yong Yu$^1$, Jun Wang$^3$, Weinan Zhang$^1$}\\
  $^1$Shanghai Jiao Tong University, $^2$ Huawei Noah’s Ark Lab, $^3$ University College London \\
  Shanghai, China\\
  \texttt{\{kounianhuadu, wnzhang\}@sjtu.edu.cn}
  %\texttt{\{774581965,ruirenting,fatcat,fulingyue,yyu,wnzhang\}@sjtu.edu.cn}, \\ 
  %\texttt{\{xiawei24,wangyasheng,tangruiming\}@huawei.com}
  }

\begin{document}
\maketitle

% \begin{abstract}
% Large language models (LLMs) have demonstrated remarkable capabilities in many domains, yet their ability in System 2 tasks remain opaque. The inherent high-thinking demand and complex latent patterns of data make it hard for llms to accurately infer solutions for these tasks. And the simple train-once-for-all tuning mechanism may then fail under this situation. In this paper, we propose \textbf{\modelname} to \textbf{Disen}tangle the complex problem solving procedure and heterogeneous data distribution into meta-units and adaptively aggregate the resulting \textbf{Lora}-experts to generate problem solver for each problem instance. Concretely, 1) we disentangle the problem-solving stages into problem2thought and thought2solution processes and integrate the search process using a novel multi-expert MCTS algorithm, armed with reflextion-based pruning and refinement. 2) The resulting training data is thereafter disentangled into different meta clusters based on semantic distance, on which we finetune different lora experts capable of different levels of tasks.  3) Then, we train an input-aware hyper-network to adaptively aggregate the lora experts rank-wise for contextualized problem solver. Experiment results and various ablation studies validate the superiority of the data collection algorithm, the effectiveness of the data disentanglement process, and the performance gain brought by the input-aware hyper-network.
% \end{abstract}

%\begin{abstract}  
Test time scaling is currently one of the most active research areas that shows promise after training time scaling has reached its limits.
Deep-thinking (DT) models are a class of recurrent models that can perform easy-to-hard generalization by assigning more compute to harder test samples.
However, due to their inability to determine the complexity of a test sample, DT models have to use a large amount of computation for both easy and hard test samples.
Excessive test time computation is wasteful and can cause the ``overthinking'' problem where more test time computation leads to worse results.
In this paper, we introduce a test time training method for determining the optimal amount of computation needed for each sample during test time.
We also propose Conv-LiGRU, a novel recurrent architecture for efficient and robust visual reasoning. 
Extensive experiments demonstrate that Conv-LiGRU is more stable than DT, effectively mitigates the ``overthinking'' phenomenon, and achieves superior accuracy.
\end{abstract}  
%\section{Introduction}


\begin{figure}[t]
\centering
\includegraphics[width=0.6\columnwidth]{figures/evaluation_desiderata_V5.pdf}
\vspace{-0.5cm}
\caption{\systemName is a platform for conducting realistic evaluations of code LLMs, collecting human preferences of coding models with real users, real tasks, and in realistic environments, aimed at addressing the limitations of existing evaluations.
}
\label{fig:motivation}
\end{figure}

\begin{figure*}[t]
\centering
\includegraphics[width=\textwidth]{figures/system_design_v2.png}
\caption{We introduce \systemName, a VSCode extension to collect human preferences of code directly in a developer's IDE. \systemName enables developers to use code completions from various models. The system comprises a) the interface in the user's IDE which presents paired completions to users (left), b) a sampling strategy that picks model pairs to reduce latency (right, top), and c) a prompting scheme that allows diverse LLMs to perform code completions with high fidelity.
Users can select between the top completion (green box) using \texttt{tab} or the bottom completion (blue box) using \texttt{shift+tab}.}
\label{fig:overview}
\end{figure*}

As model capabilities improve, large language models (LLMs) are increasingly integrated into user environments and workflows.
For example, software developers code with AI in integrated developer environments (IDEs)~\citep{peng2023impact}, doctors rely on notes generated through ambient listening~\citep{oberst2024science}, and lawyers consider case evidence identified by electronic discovery systems~\citep{yang2024beyond}.
Increasing deployment of models in productivity tools demands evaluation that more closely reflects real-world circumstances~\citep{hutchinson2022evaluation, saxon2024benchmarks, kapoor2024ai}.
While newer benchmarks and live platforms incorporate human feedback to capture real-world usage, they almost exclusively focus on evaluating LLMs in chat conversations~\citep{zheng2023judging,dubois2023alpacafarm,chiang2024chatbot, kirk2024the}.
Model evaluation must move beyond chat-based interactions and into specialized user environments.



 

In this work, we focus on evaluating LLM-based coding assistants. 
Despite the popularity of these tools---millions of developers use Github Copilot~\citep{Copilot}---existing
evaluations of the coding capabilities of new models exhibit multiple limitations (Figure~\ref{fig:motivation}, bottom).
Traditional ML benchmarks evaluate LLM capabilities by measuring how well a model can complete static, interview-style coding tasks~\citep{chen2021evaluating,austin2021program,jain2024livecodebench, white2024livebench} and lack \emph{real users}. 
User studies recruit real users to evaluate the effectiveness of LLMs as coding assistants, but are often limited to simple programming tasks as opposed to \emph{real tasks}~\citep{vaithilingam2022expectation,ross2023programmer, mozannar2024realhumaneval}.
Recent efforts to collect human feedback such as Chatbot Arena~\citep{chiang2024chatbot} are still removed from a \emph{realistic environment}, resulting in users and data that deviate from typical software development processes.
We introduce \systemName to address these limitations (Figure~\ref{fig:motivation}, top), and we describe our three main contributions below.


\textbf{We deploy \systemName in-the-wild to collect human preferences on code.} 
\systemName is a Visual Studio Code extension, collecting preferences directly in a developer's IDE within their actual workflow (Figure~\ref{fig:overview}).
\systemName provides developers with code completions, akin to the type of support provided by Github Copilot~\citep{Copilot}. 
Over the past 3 months, \systemName has served over~\completions suggestions from 10 state-of-the-art LLMs, 
gathering \sampleCount~votes from \userCount~users.
To collect user preferences,
\systemName presents a novel interface that shows users paired code completions from two different LLMs, which are determined based on a sampling strategy that aims to 
mitigate latency while preserving coverage across model comparisons.
Additionally, we devise a prompting scheme that allows a diverse set of models to perform code completions with high fidelity.
See Section~\ref{sec:system} and Section~\ref{sec:deployment} for details about system design and deployment respectively.



\textbf{We construct a leaderboard of user preferences and find notable differences from existing static benchmarks and human preference leaderboards.}
In general, we observe that smaller models seem to overperform in static benchmarks compared to our leaderboard, while performance among larger models is mixed (Section~\ref{sec:leaderboard_calculation}).
We attribute these differences to the fact that \systemName is exposed to users and tasks that differ drastically from code evaluations in the past. 
Our data spans 103 programming languages and 24 natural languages as well as a variety of real-world applications and code structures, while static benchmarks tend to focus on a specific programming and natural language and task (e.g. coding competition problems).
Additionally, while all of \systemName interactions contain code contexts and the majority involve infilling tasks, a much smaller fraction of Chatbot Arena's coding tasks contain code context, with infilling tasks appearing even more rarely. 
We analyze our data in depth in Section~\ref{subsec:comparison}.



\textbf{We derive new insights into user preferences of code by analyzing \systemName's diverse and distinct data distribution.}
We compare user preferences across different stratifications of input data (e.g., common versus rare languages) and observe which affect observed preferences most (Section~\ref{sec:analysis}).
For example, while user preferences stay relatively consistent across various programming languages, they differ drastically between different task categories (e.g. frontend/backend versus algorithm design).
We also observe variations in user preference due to different features related to code structure 
(e.g., context length and completion patterns).
We open-source \systemName and release a curated subset of code contexts.
Altogether, our results highlight the necessity of model evaluation in realistic and domain-specific settings.







%\putsec{related}{Related Work}

\noindent \textbf{Efficient Radiance Field Rendering.}
%
The introduction of Neural Radiance Fields (NeRF)~\cite{mil:sri20} has
generated significant interest in efficient 3D scene representation and
rendering for radiance fields.
%
Over the past years, there has been a large amount of research aimed at
accelerating NeRFs through algorithmic or software
optimizations~\cite{mul:eva22,fri:yu22,che:fun23,sun:sun22}, and the
development of hardware
accelerators~\cite{lee:cho23,li:li23,son:wen23,mub:kan23,fen:liu24}.
%
The state-of-the-art method, 3D Gaussian splatting~\cite{ker:kop23}, has
further fueled interest in accelerating radiance field
rendering~\cite{rad:ste24,lee:lee24,nie:stu24,lee:rho24,ham:mel24} as it
employs rasterization primitives that can be rendered much faster than NeRFs.
%
However, previous research focused on software graphics rendering on
programmable cores or building dedicated hardware accelerators. In contrast,
\name{} investigates the potential of efficient radiance field rendering while
utilizing fixed-function units in graphics hardware.
%
To our knowledge, this is the first work that assesses the performance
implications of rendering Gaussian-based radiance fields on the hardware
graphics pipeline with software and hardware optimizations.

%%%%%%%%%%%%%%%%%%%%%%%%%%%%%%%%%%%%%%%%%%%%%%%%%%%%%%%%%%%%%%%%%%%%%%%%%%
\myparagraph{Enhancing Graphics Rendering Hardware.}
%
The performance advantage of executing graphics rendering on either
programmable shader cores or fixed-function units varies depending on the
rendering methods and hardware designs.
%
Previous studies have explored the performance implication of graphics hardware
design by developing simulation infrastructures for graphics
workloads~\cite{bar:gon06,gub:aam19,tin:sax23,arn:par13}.
%
Additionally, several studies have aimed to improve the performance of
special-purpose hardware such as ray tracing units in graphics
hardware~\cite{cho:now23,liu:cha21} and proposed hardware accelerators for
graphics applications~\cite{lu:hua17,ram:gri09}.
%
In contrast to these works, which primarily evaluate traditional graphics
workloads, our work focuses on improving the performance of volume rendering
workloads, such as Gaussian splatting, which require blending a huge number of
fragments per pixel.

%%%%%%%%%%%%%%%%%%%%%%%%%%%%%%%%%%%%%%%%%%%%%%%%%%%%%%%%%%%%%%%%%%%%%%%%%%
%
In the context of multi-sample anti-aliasing, prior work proposed reducing the
amount of redundant shading by merging fragments from adjacent triangles in a
mesh at the quad granularity~\cite{fat:bou10}.
%
While both our work and quad-fragment merging (QFM)~\cite{fat:bou10} aim to
reduce operations by merging quads, our proposed technique differs from QFM in
many aspects.
%
Our method aims to blend \emph{overlapping primitives} along the depth
direction and applies to quads from any primitive. In contrast, QFM merges quad
fragments from small (e.g., pixel-sized) triangles that \emph{share} an edge
(i.e., \emph{connected}, \emph{non-overlapping} triangles).
%
As such, QFM is not applicable to the scenes consisting of a number of
unconnected transparent triangles, such as those in 3D Gaussian splatting.
%
In addition, our method computes the \emph{exact} color for each pixel by
offloading blending operations from ROPs to shader units, whereas QFM
\emph{approximates} pixel colors by using the color from one triangle when
multiple triangles are merged into a single quad.


%\section{Preliminaries}
\label{sec:prelim}
\label{sec:term}
We define the key terminologies used, primarily focusing on the hidden states (or activations) during the forward pass. 

\paragraph{Components in an attention layer.} We denote $\Res$ as the residual stream. We denote $\Val$ as Value (states), $\Qry$ as Query (states), and $\Key$ as Key (states) in one attention head. The \attlogit~represents the value before the softmax operation and can be understood as the inner product between  $\Qry$  and  $\Key$. We use \Attn~to denote the attention weights of applying the SoftMax function to \attlogit, and ``attention map'' to describe the visualization of the heat map of the attention weights. When referring to the \attlogit~from ``$\tokenB$'' to  ``$\tokenA$'', we indicate the inner product  $\langle\Qry(\tokenB), \Key(\tokenA)\rangle$, specifically the entry in the ``$\tokenB$'' row and ``$\tokenA$'' column of the attention map.

\paragraph{Logit lens.} We use the method of ``Logit Lens'' to interpret the hidden states and value states \citep{belrose2023eliciting}. We use \logit~to denote pre-SoftMax values of the next-token prediction for LLMs. Denote \readout~as the linear operator after the last layer of transformers that maps the hidden states to the \logit. 
The logit lens is defined as applying the readout matrix to residual or value states in middle layers. Through the logit lens, the transformed hidden states can be interpreted as their direct effect on the logits for next-token prediction. 

\paragraph{Terminologies in two-hop reasoning.} We refer to an input like “\Src$\to$\brga, \brgb$\to$\Ed” as a two-hop reasoning chain, or simply a chain. The source entity $\Src$ serves as the starting point or origin of the reasoning. The end entity $\Ed$ represents the endpoint or destination of the reasoning chain. The bridge entity $\Brg$ connects the source and end entities within the reasoning chain. We distinguish between two occurrences of $\Brg$: the bridge in the first premise is called $\brga$, while the bridge in the second premise that connects to $\Ed$ is called $\brgc$. Additionally, for any premise ``$\tokenA \to \tokenB$'', we define $\tokenA$ as the parent node and $\tokenB$ as the child node. Furthermore, if at the end of the sequence, the query token is ``$\tokenA$'', we define the chain ``$\tokenA \to \tokenB$, $\tokenB \to \tokenC$'' as the Target Chain, while all other chains present in the context are referred to as distraction chains. Figure~\ref{fig:data_illustration} provides an illustration of the terminologies.

\paragraph{Input format.}
Motivated by two-hop reasoning in real contexts, we consider input in the format $\bos, \text{context information}, \query, \answer$. A transformer model is trained to predict the correct $\answer$ given the query $\query$ and the context information. The context compromises of $K=5$ disjoint two-hop chains, each appearing once and containing two premises. Within the same chain, the relative order of two premises is fixed so that \Src$\to$\brga~always precedes \brgb$\to$\Ed. The orders of chains are randomly generated, and chains may interleave with each other. The labels for the entities are re-shuffled for every sequence, choosing from a vocabulary size $V=30$. Given the $\bos$ token, $K=5$ two-hop chains, \query, and the \answer~tokens, the total context length is $N=23$. Figure~\ref{fig:data_illustration} also illustrates the data format. 

\paragraph{Model structure and training.} We pre-train a three-layer transformer with a single head per layer. Unless otherwise specified, the model is trained using Adam for $10,000$ steps, achieving near-optimal prediction accuracy. Details are relegated to Appendix~\ref{app:sec_add_training_detail}.


% \RZ{Do we use source entity, target entity, and mediator entity? Or do we use original token, bridge token, end token?}





% \paragraph{Basic notations.} We use ... We use $\ve_i$ to denote one-hot vectors of which only the $i$-th entry equals one, and all other entries are zero. The dimension of $\ve_i$ are usually omitted and can be inferred from contexts. We use $\indicator\{\cdot\}$ to denote the indicator function.

% Let $V > 0$ be a fixed positive integer, and let $\vocab = [V] \defeq \{1, 2, \ldots, V\}$ be the vocabulary. A token $v \in \vocab$ is an integer in $[V]$ and the input studied in this paper is a sequence of tokens $s_{1:T} \defeq (s_1, s_2, \ldots, s_T) \in \vocab^T$ of length $T$. For any set $\mathcal{S}$, we use $\Delta(\mathcal{S})$ to denote the set of distributions over $\mathcal{S}$.

% % to a sequence of vectors $z_1, z_2, \ldots, z_T \in \real^{\dout}$ of dimension $\dout$ and length $T$.

% Let $\mU = [\vu_1, \vu_2, \ldots, \vu_V]^\transpose \in \real^{V\times d}$ denote the token embedding matrix, where the $i$-th row $\vu_i \in \real^d$ represents the $d$-dimensional embedding of token $i \in [V]$. Similarly, let $\mP = [\vp_1, \vp_2, \ldots, \vp_T]^\transpose \in \real^{T\times d}$ denote the positional embedding matrix, where the $i$-th row $\vp_i \in \real^d$ represents the $d$-dimensional embedding of position $i \in [T]$. Both $\mU$ and $\mP$ can be fixed or learnable.

% After receiving an input sequence of tokens $s_{1:T}$, a transformer will first process it using embedding matrices $\mU$ and $\mP$ to obtain a sequence of vectors $\mH = [\vh_1, \vh_2, \ldots, \vh_T] \in \real^{d\times T}$, where 
% \[
% \vh_i = \mU^\transpose\ve_{s_i} + \mP^\transpose\ve_{i} = \vu_{s_i} + \vp_i.
% \]

% We make the following definitions of basic operations in a transformer.

% \begin{definition}[Basic operations in transformers] 
% \label{defn:operators}
% Define the softmax function $\softmax(\cdot): \real^d \to \real^d$ over a vector $\vv \in \real^d$ as
% \[\softmax(\vv)_i = \frac{\exp(\vv_i)}{\sum_{j=1}^d \exp(\vv_j)} \]
% and define the softmax function $\softmax(\cdot): \real^{m\times n} \to \real^{m \times n}$ over a matrix $\mV \in \real^{m\times n}$ as a column-wise softmax operator. For a squared matrix $\mM \in \real^{m\times m}$, the causal mask operator $\mask(\cdot): \real^{m\times m} \to \real^{m\times m}$  is defined as $\mask(\mM)_{ij} = \mM_{ij}$ if $i \leq j$ and  $\mask(\mM)_{ij} = -\infty$ otherwise. For a vector $\vv \in \real^n$ where $n$ is the number of hidden neurons in a layer, we use $\layernorm(\cdot): \real^n \to \real^n$ to denote the layer normalization operator where
% \[
% \layernorm(\vv)_i = \frac{\vv_i-\mu}{\sigma}, \mu = \frac{1}{n}\sum_{j=1}^n \vv_j, \sigma = \sqrt{\frac{1}{n}\sum_{j=1}^n (\vv_j-\mu)^2}
% \]
% and use $\layernorm(\cdot): \real^{n\times m} \to \real^{n\times m}$ to denote the column-wise layer normalization on a matrix.
% We also use $\nonlin(\cdot)$ to denote element-wise nonlinearity such as $\relu(\cdot)$.
% \end{definition}

% The main components of a transformer are causal self-attention heads and MLP layers, which are defined as follows.

% \begin{definition}[Attentions and MLPs]
% \label{defn:attn_mlp} 
% A single-head causal self-attention $\attn(\mH;\mQ,\mK,\mV,\mO)$ parameterized by $\mQ,\mK,\mV \in \real^{{\dqkv\times \din}}$ and $\mO \in \real^{\dout\times\dqkv}$ maps an input matrix $\mH \in \real^{\din\times T}$ to
% \begin{align*}
% &\attn(\mH;\mQ,\mK,\mV,\mO) \\
% =&\mO\mV\layernorm(\mH)\softmax(\mask(\layernorm(\mH)^\transpose\mK^\transpose\mQ\layernorm(\mH))).
% \end{align*}
% Furthermore, a multi-head attention with $M$ heads parameterized by $\{(\mQ_m,\mK_m,\mV_m,\mO_m) \}_{m=1}^M$ is defined as 
% \begin{align*}
%     &\Attn(\mH; \{(\mQ_m,\mK_m,\mV_m,\mO_m) \}_{m\in[M]}) \\ =& \sum_{m=1}^M \attn(\mH;\mQ_m,\mK_m,\mV_m,\mO_m) \in \real^{\dout \times T}.
% \end{align*}
% An MLP layer $\mlp(\mH;\mW_1,\mW_2)$ parameterized by $\mW_1 \in \real^{\dhidden\times \din}$ and $\mW_2 \in \real^{\dout \times \dhidden}$ maps an input matrix $\mH = [\vh_1, \ldots, \vh_T] \in \real^{\din \times T}$ to
% \begin{align*}
%     &\mlp(\mH;\mW_1,\mW_2) = [\vy_1, \ldots, \vy_T], \\ \text{where } &\vy_i = \mW_2\nonlin(\mW_1\layernorm(\vh_i)), \forall i \in [T].
% \end{align*}

% \end{definition}

% In this paper, we assume $\din=\dout=d$ for all attention heads and MLPs to facilitate residual stream unless otherwise specified. Given \Cref{defn:operators,defn:attn_mlp}, we are now able to define a multi-layer transformer.

% \begin{definition}[Multi-layer transformers]
% \label{defn:transformer}
%     An $L$-layer transformer $\transformer(\cdot): \vocab^T \to \Delta(\vocab)$ parameterized by $\mP$, $\mU$, $\{(\mQ_m^{(l)},\mK_m^{(l)},\mV_m^{(l)},\mO_m^{(l)})\}_{m\in[M],l\in[L]}$,  $\{(\mW_1^{(l)},\mW_2^{(l)})\}_{l\in[L]}$ and $\Wreadout \in \real^{V \times d}$ receives a sequence of tokens $s_{1:T}$ as input and predict the next token by outputting a distribution over the vocabulary. The input is first mapped to embeddings $\mH = [\vh_1, \vh_2, \ldots, \vh_T] \in \real^{d\times T}$ by embedding matrices $\mP, \mU$ where 
%     \[
%     \vh_i = \mU^\transpose\ve_{s_i} + \mP^\transpose\ve_{i}, \forall i \in [T].
%     \]
%     For each layer $l \in [L]$, the output of layer $l$, $\mH^{(l)} \in \real^{d\times T}$, is obtained by 
%     \begin{align*}
%         &\mH^{(l)} =  \mH^{(l-1/2)} + \mlp(\mH^{(l-1/2)};\mW_1^{(l)},\mW_2^{(l)}), \\
%         & \mH^{(l-1/2)} = \mH^{(l-1)} + \\ & \quad \Attn(\mH^{(l-1)}; \{(\mQ_m^{(l)},\mK_m^{(l)},\mV_m^{(l)},\mO_m^{(l)}) \}_{m\in[M]}), 
%     \end{align*}
%     where the input $\mH^{(l-1)}$ is the output of the previous layer $l-1$ for $l > 1$ and the input of the first layer $\mH^{(0)} = \mH$. Finally, the output of the transformer is obtained by 
%     \begin{align*}
%         \transformer(s_{1:T}) = \softmax(\Wreadout\vh_T^{(L)})
%     \end{align*}
%     which is a $V$-dimensional vector after softmax representing a distribution over $\vocab$, and $\vh_T^{(L)}$ is the $T$-th column of the output of the last layer, $\mH^{(L)}$.
% \end{definition}



% For each token $v \in \vocab$, there is a corresponding $d_t$-dimensional token embedding vector $\embed(v) \in \mathbb{R}^{d_t}$. Assume the maximum length of the sequence studied in this paper does not exceed $T$. For each position $t \in [T]$, there is a corresponding positional embedding  







%\section{Method}\label{sec:method}
\begin{figure}
    \centering
    \includegraphics[width=0.85\textwidth]{imgs/heatmap_acc.pdf}
    \caption{\textbf{Visualization of the proposed periodic Bayesian flow with mean parameter $\mu$ and accumulated accuracy parameter $c$ which corresponds to the entropy/uncertainty}. For $x = 0.3, \beta(1) = 1000$ and $\alpha_i$ defined in \cref{appd:bfn_cir}, this figure plots three colored stochastic parameter trajectories for receiver mean parameter $m$ and accumulated accuracy parameter $c$, superimposed on a log-scale heatmap of the Bayesian flow distribution $p_F(m|x,\senderacc)$ and $p_F(c|x,\senderacc)$. Note the \emph{non-monotonicity} and \emph{non-additive} property of $c$ which could inform the network the entropy of the mean parameter $m$ as a condition and the \emph{periodicity} of $m$. %\jj{Shrink the figures to save space}\hanlin{Do we need to make this figure one-column?}
    }
    \label{fig:vmbf_vis}
    \vskip -0.1in
\end{figure}
% \begin{wrapfigure}{r}{0.5\textwidth}
%     \centering
%     \includegraphics[width=0.49\textwidth]{imgs/heatmap_acc.pdf}
%     \caption{\textbf{Visualization of hyper-torus Bayesian flow based on von Mises Distribution}. For $x = 0.3, \beta(1) = 1000$ and $\alpha_i$ defined in \cref{appd:bfn_cir}, this figure plots three colored stochastic parameter trajectories for receiver mean parameter $m$ and accumulated accuracy parameter $c$, superimposed on a log-scale heatmap of the Bayesian flow distribution $p_F(m|x,\senderacc)$ and $p_F(c|x,\senderacc)$. Note the \emph{non-monotonicity} and \emph{non-additive} property of $c$. \jj{Shrink the figures to save space}}
%     \label{fig:vmbf_vis}
%     \vspace{-30pt}
% \end{wrapfigure}


In this section, we explain the detailed design of CrysBFN tackling theoretical and practical challenges. First, we describe how to derive our new formulation of Bayesian Flow Networks over hyper-torus $\mathbb{T}^{D}$ from scratch. Next, we illustrate the two key differences between \modelname and the original form of BFN: $1)$ a meticulously designed novel base distribution with different Bayesian update rules; and $2)$ different properties over the accuracy scheduling resulted from the periodicity and the new Bayesian update rules. Then, we present in detail the overall framework of \modelname over each manifold of the crystal space (\textit{i.e.} fractional coordinates, lattice vectors, atom types) respecting \textit{periodic E(3) invariance}. 

% In this section, we first demonstrate how to build Bayesian flow on hyper-torus $\mathbb{T}^{D}$ by overcoming theoretical and practical problems to provide a low-noise parameter-space approach to fractional atom coordinate generation. Next, we present how \modelname models each manifold of crystal space respecting \textit{periodic E(3) invariance}. 

\subsection{Periodic Bayesian Flow on Hyper-torus \texorpdfstring{$\mathbb{T}^{D}$}{}} 
For generative modeling of fractional coordinates in crystal, we first construct a periodic Bayesian flow on \texorpdfstring{$\mathbb{T}^{D}$}{} by designing every component of the totally new Bayesian update process which we demonstrate to be distinct from the original Bayesian flow (please see \cref{fig:non_add}). 
 %:) 
 
 The fractional atom coordinate system \citep{jiao2023crystal} inherently distributes over a hyper-torus support $\mathbb{T}^{3\times N}$. Hence, the normal distribution support on $\R$ used in the original \citep{bfn} is not suitable for this scenario. 
% The key problem of generative modeling for crystal is the periodicity of Cartesian atom coordinates $\vX$ requiring:
% \begin{equation}\label{eq:periodcity}
% p(\vA,\vL,\vX)=p(\vA,\vL,\vX+\vec{LK}),\text{where}~\vec{K}=\vec{k}\vec{1}_{1\times N},\forall\vec{k}\in\mathbb{Z}^{3\times1}
% \end{equation}
% However, there does not exist such a distribution supporting on $\R$ to model such property because the integration of such distribution over $\R$ will not be finite and equal to 1. Therefore, the normal distribution used in \citet{bfn} can not meet this condition.

To tackle this problem, the circular distribution~\citep{mardia2009directional} over the finite interval $[-\pi,\pi)$ is a natural choice as the base distribution for deriving the BFN on $\mathbb{T}^D$. 
% one natural choice is to 
% we would like to consider the circular distribution over the finite interval as the base 
% we find that circular distributions \citep{mardia2009directional} defined on a finite interval with lengths of $2\pi$ can be used as the instantiation of input distribution for the BFN on $\mathbb{T}^D$.
Specifically, circular distributions enjoy desirable periodic properties: $1)$ the integration over any interval length of $2\pi$ equals 1; $2)$ the probability distribution function is periodic with period $2\pi$.  Sharing the same intrinsic with fractional coordinates, such periodic property of circular distribution makes it suitable for the instantiation of BFN's input distribution, in parameterizing the belief towards ground truth $\x$ on $\mathbb{T}^D$. 
% \yuxuan{this is very complicated from my perspective.} \hanlin{But this property is exactly beautiful and perfectly fit into the BFN.}

\textbf{von Mises Distribution and its Bayesian Update} We choose von Mises distribution \citep{mardia2009directional} from various circular distributions as the form of input distribution, based on the appealing conjugacy property required in the derivation of the BFN framework.
% to leverage the Bayesian conjugacy property of von Mises distribution which is required by the BFN framework. 
That is, the posterior of a von Mises distribution parameterized likelihood is still in the family of von Mises distributions. The probability density function of von Mises distribution with mean direction parameter $m$ and concentration parameter $c$ (describing the entropy/uncertainty of $m$) is defined as: 
\begin{equation}
f(x|m,c)=vM(x|m,c)=\frac{\exp(c\cos(x-m))}{2\pi I_0(c)}
\end{equation}
where $I_0(c)$ is zeroth order modified Bessel function of the first kind as the normalizing constant. Given the last univariate belief parameterized by von Mises distribution with parameter $\theta_{i-1}=\{m_{i-1},\ c_{i-1}\}$ and the sample $y$ from sender distribution with unknown data sample $x$ and known accuracy $\alpha$ describing the entropy/uncertainty of $y$,  Bayesian update for the receiver is deducted as:
\begin{equation}
 h(\{m_{i-1},c_{i-1}\},y,\alpha)=\{m_i,c_i \}, \text{where}
\end{equation}
\begin{equation}\label{eq:h_m}
m_i=\text{atan2}(\alpha\sin y+c_{i-1}\sin m_{i-1}, {\alpha\cos y+c_{i-1}\cos m_{i-1}})
\end{equation}
\begin{equation}\label{eq:h_c}
c_i =\sqrt{\alpha^2+c_{i-1}^2+2\alpha c_{i-1}\cos(y-m_{i-1})}
\end{equation}
The proof of the above equations can be found in \cref{apdx:bayesian_update_function}. The atan2 function refers to  2-argument arctangent. Independently conducting  Bayesian update for each dimension, we can obtain the Bayesian update distribution by marginalizing $\y$:
\begin{equation}
p_U(\vtheta'|\vtheta,\bold{x};\alpha)=\mathbb{E}_{p_S(\bold{y}|\bold{x};\alpha)}\delta(\vtheta'-h(\vtheta,\bold{y},\alpha))=\mathbb{E}_{vM(\bold{y}|\bold{x},\alpha)}\delta(\vtheta'-h(\vtheta,\bold{y},\alpha))
\end{equation} 
\begin{figure}
    \centering
    \vskip -0.15in
    \includegraphics[width=0.95\linewidth]{imgs/non_add.pdf}
    \caption{An intuitive illustration of non-additive accuracy Bayesian update on the torus. The lengths of arrows represent the uncertainty/entropy of the belief (\emph{e.g.}~$1/\sigma^2$ for Gaussian and $c$ for von Mises). The directions of the arrows represent the believed location (\emph{e.g.}~ $\mu$ for Gaussian and $m$ for von Mises).}
    \label{fig:non_add}
    \vskip -0.15in
\end{figure}
\textbf{Non-additive Accuracy} 
The additive accuracy is a nice property held with the Gaussian-formed sender distribution of the original BFN expressed as:
\begin{align}
\label{eq:standard_id}
    \update(\parsn{}'' \mid \parsn{}, \x; \alpha_a+\alpha_b) = \E_{\update(\parsn{}' \mid \parsn{}, \x; \alpha_a)} \update(\parsn{}'' \mid \parsn{}', \x; \alpha_b)
\end{align}
Such property is mainly derived based on the standard identity of Gaussian variable:
\begin{equation}
X \sim \mathcal{N}\left(\mu_X, \sigma_X^2\right), Y \sim \mathcal{N}\left(\mu_Y, \sigma_Y^2\right) \Longrightarrow X+Y \sim \mathcal{N}\left(\mu_X+\mu_Y, \sigma_X^2+\sigma_Y^2\right)
\end{equation}
The additive accuracy property makes it feasible to derive the Bayesian flow distribution $
p_F(\boldsymbol{\theta} \mid \mathbf{x} ; i)=p_U\left(\boldsymbol{\theta} \mid \boldsymbol{\theta}_0, \mathbf{x}, \sum_{k=1}^{i} \alpha_i \right)
$ for the simulation-free training of \cref{eq:loss_n}.
It should be noted that the standard identity in \cref{eq:standard_id} does not hold in the von Mises distribution. Hence there exists an important difference between the original Bayesian flow defined on Euclidean space and the Bayesian flow of circular data on $\mathbb{T}^D$ based on von Mises distribution. With prior $\btheta = \{\bold{0},\bold{0}\}$, we could formally represent the non-additive accuracy issue as:
% The additive accuracy property implies the fact that the "confidence" for the data sample after observing a series of the noisy samples with accuracy ${\alpha_1, \cdots, \alpha_i}$ could be  as the accuracy sum  which could be  
% Here we 
% Here we emphasize the specific property of BFN based on von Mises distribution.
% Note that 
% \begin{equation}
% \update(\parsn'' \mid \parsn, \x; \alpha_a+\alpha_b) \ne \E_{\update(\parsn' \mid \parsn, \x; \alpha_a)} \update(\parsn'' \mid \parsn', \x; \alpha_b)
% \end{equation}
% \oyyw{please check whether the below equation is better}
% \yuxuan{I fill somehow confusing on what is the update distribution with $\alpha$. }
% \begin{equation}
% \update(\parsn{}'' \mid \parsn{}, \x; \alpha_a+\alpha_b) \ne \E_{\update(\parsn{}' \mid \parsn{}, \x; \alpha_a)} \update(\parsn{}'' \mid \parsn{}', \x; \alpha_b)
% \end{equation}
% We give an intuitive visualization of such difference in \cref{fig:non_add}. The untenability of this property can materialize by considering the following case: with prior $\btheta = \{\bold{0},\bold{0}\}$, check the two-step Bayesian update distribution with $\alpha_a,\alpha_b$ and one-step Bayesian update with $\alpha=\alpha_a+\alpha_b$:
\begin{align}
\label{eq:nonadd}
     &\update(c'' \mid \parsn, \x; \alpha_a+\alpha_b)  = \delta(c-\alpha_a-\alpha_b)
     \ne  \mathbb{E}_{p_U(\parsn' \mid \parsn, \x; \alpha_a)}\update(c'' \mid \parsn', \x; \alpha_b) \nonumber \\&= \mathbb{E}_{vM(\bold{y}_b|\bold{x},\alpha_a)}\mathbb{E}_{vM(\bold{y}_a|\bold{x},\alpha_b)}\delta(c-||[\alpha_a \cos\y_a+\alpha_b\cos \y_b,\alpha_a \sin\y_a+\alpha_b\sin \y_b]^T||_2)
\end{align}
A more intuitive visualization could be found in \cref{fig:non_add}. This fundamental difference between periodic Bayesian flow and that of \citet{bfn} presents both theoretical and practical challenges, which we will explain and address in the following contents.

% This makes constructing Bayesian flow based on von Mises distribution intrinsically different from previous Bayesian flows (\citet{bfn}).

% Thus, we must reformulate the framework of Bayesian flow networks  accordingly. % and do necessary reformulations of BFN. 

% \yuxuan{overall I feel this part is complicated by using the language of update distribution. I would like to suggest simply use bayesian update, to provide intuitive explantion.}\hanlin{See the illustration in \cref{fig:non_add}}

% That introduces a cascade of problems, and we investigate the following issues: $(1)$ Accuracies between sender and receiver are not synchronized and need to be differentiated. $(2)$ There is no tractable Bayesian flow distribution for a one-step sample conditioned on a given time step $i$, and naively simulating the Bayesian flow results in computational overhead. $(3)$ It is difficult to control the entropy of the Bayesian flow. $(4)$ Accuracy is no longer a function of $t$ and becomes a distribution conditioned on $t$, which can be different across dimensions.
%\jj{Edited till here}

\textbf{Entropy Conditioning} As a common practice in generative models~\citep{ddpm,flowmatching,bfn}, timestep $t$ is widely used to distinguish among generation states by feeding the timestep information into the networks. However, this paper shows that for periodic Bayesian flow, the accumulated accuracy $\vc_i$ is more effective than time-based conditioning by informing the network about the entropy and certainty of the states $\parsnt{i}$. This stems from the intrinsic non-additive accuracy which makes the receiver's accumulated accuracy $c$ not bijective function of $t$, but a distribution conditioned on accumulated accuracies $\vc_i$ instead. Therefore, the entropy parameter $\vc$ is taken logarithm and fed into the network to describe the entropy of the input corrupted structure. We verify this consideration in \cref{sec:exp_ablation}. 
% \yuxuan{implement variant. traditionally, the timestep is widely used to distinguish the different states by putting the timestep embedding into the networks. citation of FM, diffusion, BFN. However, we find that conditioned on time in periodic flow could not provide extra benefits. To further boost the performance, we introduce a simple yet effective modification term entropy conditional. This is based on that the accumulated accuracy which represents the current uncertainty or entropy could be a better indicator to distinguish different states. + Describe how you do this. }



\textbf{Reformulations of BFN}. Recall the original update function with Gaussian sender distribution, after receiving noisy samples $\y_1,\y_2,\dots,\y_i$ with accuracies $\senderacc$, the accumulated accuracies of the receiver side could be analytically obtained by the additive property and it is consistent with the sender side.
% Since observing sample $\y$ with $\alpha_i$ can not result in exact accuracy increment $\alpha_i$ for receiver, the accuracies between sender and receiver are not synchronized which need to be differentiated. 
However, as previously mentioned, this does not apply to periodic Bayesian flow, and some of the notations in original BFN~\citep{bfn} need to be adjusted accordingly. We maintain the notations of sender side's one-step accuracy $\alpha$ and added accuracy $\beta$, and alter the notation of receiver's accuracy parameter as $c$, which is needed to be simulated by cascade of Bayesian updates. We emphasize that the receiver's accumulated accuracy $c$ is no longer a function of $t$ (differently from the Gaussian case), and it becomes a distribution conditioned on received accuracies $\senderacc$ from the sender. Therefore, we represent the Bayesian flow distribution of von Mises distribution as $p_F(\btheta|\x;\alpha_1,\alpha_2,\dots,\alpha_i)$. And the original simulation-free training with Bayesian flow distribution is no longer applicable in this scenario.
% Different from previous BFNs where the accumulated accuracy $\rho$ is not explicitly modeled, the accumulated accuracy parameter $c$ (visualized in \cref{fig:vmbf_vis}) needs to be explicitly modeled by feeding it to the network to avoid information loss.
% the randomaccuracy parameter $c$ (visualized in \cref{fig:vmbf_vis}) implies that there exists information in $c$ from the sender just like $m$, meaning that $c$ also should be fed into the network to avoid information loss. 
% We ablate this consideration in  \cref{sec:exp_ablation}. 

\textbf{Fast Sampling from Equivalent Bayesian Flow Distribution} Based on the above reformulations, the Bayesian flow distribution of von Mises distribution is reframed as: 
\begin{equation}\label{eq:flow_frac}
p_F(\btheta_i|\x;\alpha_1,\alpha_2,\dots,\alpha_i)=\E_{\update(\parsnt{1} \mid \parsnt{0}, \x ; \alphat{1})}\dots\E_{\update(\parsn_{i-1} \mid \parsnt{i-2}, \x; \alphat{i-1})} \update(\parsnt{i} | \parsnt{i-1},\x;\alphat{i} )
\end{equation}
Naively sampling from \cref{eq:flow_frac} requires slow auto-regressive iterated simulation, making training unaffordable. Noticing the mathematical properties of \cref{eq:h_m,eq:h_c}, we  transform \cref{eq:flow_frac} to the equivalent form:
\begin{equation}\label{eq:cirflow_equiv}
p_F(\vec{m}_i|\x;\alpha_1,\alpha_2,\dots,\alpha_i)=\E_{vM(\y_1|\x,\alpha_1)\dots vM(\y_i|\x,\alpha_i)} \delta(\vec{m}_i-\text{atan2}(\sum_{j=1}^i \alpha_j \cos \y_j,\sum_{j=1}^i \alpha_j \sin \y_j))
\end{equation}
\begin{equation}\label{eq:cirflow_equiv2}
p_F(\vec{c}_i|\x;\alpha_1,\alpha_2,\dots,\alpha_i)=\E_{vM(\y_1|\x,\alpha_1)\dots vM(\y_i|\x,\alpha_i)}  \delta(\vec{c}_i-||[\sum_{j=1}^i \alpha_j \cos \y_j,\sum_{j=1}^i \alpha_j \sin \y_j]^T||_2)
\end{equation}
which bypasses the computation of intermediate variables and allows pure tensor operations, with negligible computational overhead.
\begin{restatable}{proposition}{cirflowequiv}
The probability density function of Bayesian flow distribution defined by \cref{eq:cirflow_equiv,eq:cirflow_equiv2} is equivalent to the original definition in \cref{eq:flow_frac}. 
\end{restatable}
\textbf{Numerical Determination of Linear Entropy Sender Accuracy Schedule} ~Original BFN designs the accuracy schedule $\beta(t)$ to make the entropy of input distribution linearly decrease. As for crystal generation task, to ensure information coherence between modalities, we choose a sender accuracy schedule $\senderacc$ that makes the receiver's belief entropy $H(t_i)=H(p_I(\cdot|\vtheta_i))=H(p_I(\cdot|\vc_i))$ linearly decrease \emph{w.r.t.} time $t_i$, given the initial and final accuracy parameter $c(0)$ and $c(1)$. Due to the intractability of \cref{eq:vm_entropy}, we first use numerical binary search in $[0,c(1)]$ to determine the receiver's $c(t_i)$ for $i=1,\dots, n$ by solving the equation $H(c(t_i))=(1-t_i)H(c(0))+tH(c(1))$. Next, with $c(t_i)$, we conduct numerical binary search for each $\alpha_i$ in $[0,c(1)]$ by solving the equations $\E_{y\sim vM(x,\alpha_i)}[\sqrt{\alpha_i^2+c_{i-1}^2+2\alpha_i c_{i-1}\cos(y-m_{i-1})}]=c(t_i)$ from $i=1$ to $i=n$ for arbitrarily selected $x\in[-\pi,\pi)$.

After tackling all those issues, we have now arrived at a new BFN architecture for effectively modeling crystals. Such BFN can also be adapted to other type of data located in hyper-torus $\mathbb{T}^{D}$.

\subsection{Equivariant Bayesian Flow for Crystal}
With the above Bayesian flow designed for generative modeling of fractional coordinate $\vF$, we are able to build equivariant Bayesian flow for each modality of crystal. In this section, we first give an overview of the general training and sampling algorithm of \modelname (visualized in \cref{fig:framework}). Then, we describe the details of the Bayesian flow of every modality. The training and sampling algorithm can be found in \cref{alg:train} and \cref{alg:sampling}.

\textbf{Overview} Operating in the parameter space $\bthetaM=\{\bthetaA,\bthetaL,\bthetaF\}$, \modelname generates high-fidelity crystals through a joint BFN sampling process on the parameter of  atom type $\bthetaA$, lattice parameter $\vec{\theta}^L=\{\bmuL,\brhoL\}$, and the parameter of fractional coordinate matrix $\bthetaF=\{\bmF,\bcF\}$. We index the $n$-steps of the generation process in a discrete manner $i$, and denote the corresponding continuous notation $t_i=i/n$ from prior parameter $\thetaM_0$ to a considerably low variance parameter $\thetaM_n$ (\emph{i.e.} large $\vrho^L,\bmF$, and centered $\bthetaA$).

At training time, \modelname samples time $i\sim U\{1,n\}$ and $\bthetaM_{i-1}$ from the Bayesian flow distribution of each modality, serving as the input to the network. The network $\net$ outputs $\net(\parsnt{i-1}^\mathcal{M},t_{i-1})=\net(\parsnt{i-1}^A,\parsnt{i-1}^F,\parsnt{i-1}^L,t_{i-1})$ and conducts gradient descents on loss function \cref{eq:loss_n} for each modality. After proper training, the sender distribution $p_S$ can be approximated by the receiver distribution $p_R$. 

At inference time, from predefined $\thetaM_0$, we conduct transitions from $\thetaM_{i-1}$ to $\thetaM_{i}$ by: $(1)$ sampling $\y_i\sim p_R(\bold{y}|\thetaM_{i-1};t_i,\alpha_i)$ according to network prediction $\predM{i-1}$; and $(2)$ performing Bayesian update $h(\thetaM_{i-1},\y^\calM_{i-1},\alpha_i)$ for each dimension. 

% Alternatively, we complete this transition using the flow-back technique by sampling 
% $\thetaM_{i}$ from Bayesian flow distribution $\flow(\btheta^M_{i}|\predM{i-1};t_{i-1})$. 

% The training objective of $\net$ is to minimize the KL divergence between sender distribution and receiver distribution for every modality as defined in \cref{eq:loss_n} which is equivalent to optimizing the negative variational lower bound $\calL^{VLB}$ as discussed in \cref{sec:preliminaries}. 

%In the following part, we will present the Bayesian flow of each modality in detail.

\textbf{Bayesian Flow of Fractional Coordinate $\vF$}~The distribution of the prior parameter $\bthetaF_0$ is defined as:
\begin{equation}\label{eq:prior_frac}
    p(\bthetaF_0) \defeq \{vM(\vm_0^F|\vec{0}_{3\times N},\vec{0}_{3\times N}),\delta(\vc_0^F-\vec{0}_{3\times N})\} = \{U(\vec{0},\vec{1}),\delta(\vc_0^F-\vec{0}_{3\times N})\}
\end{equation}
Note that this prior distribution of $\vm_0^F$ is uniform over $[\vec{0},\vec{1})$, ensuring the periodic translation invariance property in \cref{De:pi}. The training objective is minimizing the KL divergence between sender and receiver distribution (deduction can be found in \cref{appd:cir_loss}): 
%\oyyw{replace $\vF$ with $\x$?} \hanlin{notations follow Preliminary?}
\begin{align}\label{loss_frac}
\calL_F = n \E_{i \sim \ui{n}, \flow(\parsn{}^F \mid \vF ; \senderacc)} \alpha_i\frac{I_1(\alpha_i)}{I_0(\alpha_i)}(1-\cos(\vF-\predF{i-1}))
\end{align}
where $I_0(x)$ and $I_1(x)$ are the zeroth and the first order of modified Bessel functions. The transition from $\bthetaF_{i-1}$ to $\bthetaF_{i}$ is the Bayesian update distribution based on network prediction:
\begin{equation}\label{eq:transi_frac}
    p(\btheta^F_{i}|\parsnt{i-1}^\calM)=\mathbb{E}_{vM(\bold{y}|\predF{i-1},\alpha_i)}\delta(\btheta^F_{i}-h(\btheta^F_{i-1},\bold{y},\alpha_i))
\end{equation}
\begin{restatable}{proposition}{fracinv}
With $\net_{F}$ as a periodic translation equivariant function namely $\net_F(\parsnt{}^A,w(\parsnt{}^F+\vt),\parsnt{}^L,t)=w(\net_F(\parsnt{}^A,\parsnt{}^F,\parsnt{}^L,t)+\vt), \forall\vt\in\R^3$, the marginal distribution of $p(\vF_n)$ defined by \cref{eq:prior_frac,eq:transi_frac} is periodic translation invariant. 
\end{restatable}
\textbf{Bayesian Flow of Lattice Parameter \texorpdfstring{$\boldsymbol{L}$}{}}   
Noting the lattice parameter $\bm{L}$ located in Euclidean space, we set prior as the parameter of a isotropic multivariate normal distribution $\btheta^L_0\defeq\{\vmu_0^L,\vrho_0^L\}=\{\bm{0}_{3\times3},\bm{1}_{3\times3}\}$
% \begin{equation}\label{eq:lattice_prior}
% \btheta^L_0\defeq\{\vmu_0^L,\vrho_0^L\}=\{\bm{0}_{3\times3},\bm{1}_{3\times3}\}
% \end{equation}
such that the prior distribution of the Markov process on $\vmu^L$ is the Dirac distribution $\delta(\vec{\mu_0}-\vec{0})$ and $\delta(\vec{\rho_0}-\vec{1})$, 
% \begin{equation}
%     p_I^L(\boldsymbol{L}|\btheta_0^L)=\mathcal{N}(\bm{L}|\bm{0},\bm{I})
% \end{equation}
which ensures O(3)-invariance of prior distribution of $\vL$. By Eq. 77 from \citet{bfn}, the Bayesian flow distribution of the lattice parameter $\bm{L}$ is: 
\begin{align}% =p_U(\bmuL|\btheta_0^L,\bm{L},\beta(t))
p_F^L(\bmuL|\bm{L};t) &=\mathcal{N}(\bmuL|\gamma(t)\bm{L},\gamma(t)(1-\gamma(t))\bm{I}) 
\end{align}
where $\gamma(t) = 1 - \sigma_1^{2t}$ and $\sigma_1$ is the predefined hyper-parameter controlling the variance of input distribution at $t=1$ under linear entropy accuracy schedule. The variance parameter $\vrho$ does not need to be modeled and fed to the network, since it is deterministic given the accuracy schedule. After sampling $\bmuL_i$ from $p_F^L$, the training objective is defined as minimizing KL divergence between sender and receiver distribution (based on Eq. 96 in \citet{bfn}):
\begin{align}
\mathcal{L}_{L} = \frac{n}{2}\left(1-\sigma_1^{2/n}\right)\E_{i \sim \ui{n}}\E_{\flow(\bmuL_{i-1} |\vL ; t_{i-1})}  \frac{\left\|\vL -\predL{i-1}\right\|^2}{\sigma_1^{2i/n}},\label{eq:lattice_loss}
\end{align}
where the prediction term $\predL{i-1}$ is the lattice parameter part of network output. After training, the generation process is defined as the Bayesian update distribution given network prediction:
\begin{equation}\label{eq:lattice_sampling}
    p(\bmuL_{i}|\parsnt{i-1}^\calM)=\update^L(\bmuL_{i}|\predL{i-1},\bmuL_{i-1};t_{i-1})
\end{equation}
    

% The final prediction of the lattice parameter is given by $\bmuL_n = \predL{n-1}$.
% \begin{equation}\label{eq:final_lattice}
%     \bmuL_n = \predL{n-1}
% \end{equation}

\begin{restatable}{proposition}{latticeinv}\label{prop:latticeinv}
With $\net_{L}$ as  O(3)-equivariant function namely $\net_L(\parsnt{}^A,\parsnt{}^F,\vQ\parsnt{}^L,t)=\vQ\net_L(\parsnt{}^A,\parsnt{}^F,\parsnt{}^L,t),\forall\vQ^T\vQ=\vI$, the marginal distribution of $p(\bmuL_n)$ defined by \cref{eq:lattice_sampling} is O(3)-invariant. 
\end{restatable}


\textbf{Bayesian Flow of Atom Types \texorpdfstring{$\boldsymbol{A}$}{}} 
Given that atom types are discrete random variables located in a simplex $\calS^K$, the prior parameter of $\boldsymbol{A}$ is the discrete uniform distribution over the vocabulary $\parsnt{0}^A \defeq \frac{1}{K}\vec{1}_{1\times N}$. 
% \begin{align}\label{eq:disc_input_prior}
% \parsnt{0}^A \defeq \frac{1}{K}\vec{1}_{1\times N}
% \end{align}
% \begin{align}
%     (\oh{j}{K})_k \defeq \delta_{j k}, \text{where }\oh{j}{K}\in \R^{K},\oh{\vA}{KD} \defeq \left(\oh{a_1}{K},\dots,\oh{a_N}{K}\right) \in \R^{K\times N}
% \end{align}
With the notation of the projection from the class index $j$ to the length $K$ one-hot vector $ (\oh{j}{K})_k \defeq \delta_{j k}, \text{where }\oh{j}{K}\in \R^{K},\oh{\vA}{KD} \defeq \left(\oh{a_1}{K},\dots,\oh{a_N}{K}\right) \in \R^{K\times N}$, the Bayesian flow distribution of atom types $\vA$ is derived in \citet{bfn}:
\begin{align}
\flow^{A}(\parsn^A \mid \vA; t) &= \E_{\N{\y \mid \beta^A(t)\left(K \oh{\vA}{K\times N} - \vec{1}_{K\times N}\right)}{\beta^A(t) K \vec{I}_{K\times N \times N}}} \delta\left(\parsn^A - \frac{e^{\y}\parsnt{0}^A}{\sum_{k=1}^K e^{\y_k}(\parsnt{0})_{k}^A}\right).
\end{align}
where $\beta^A(t)$ is the predefined accuracy schedule for atom types. Sampling $\btheta_i^A$ from $p_F^A$ as the training signal, the training objective is the $n$-step discrete-time loss for discrete variable \citep{bfn}: 
% \oyyw{can we simplify the next equation? Such as remove $K \times N, K \times N \times N$}
% \begin{align}
% &\calL_A = n\E_{i \sim U\{1,n\},\flow^A(\parsn^A \mid \vA ; t_{i-1}),\N{\y \mid \alphat{i}\left(K \oh{\vA}{KD} - \vec{1}_{K\times N}\right)}{\alphat{i} K \vec{I}_{K\times N \times N}}} \ln \N{\y \mid \alphat{i}\left(K \oh{\vA}{K\times N} - \vec{1}_{K\times N}\right)}{\alphat{i} K \vec{I}_{K\times N \times N}}\nonumber\\
% &\qquad\qquad\qquad-\sum_{d=1}^N \ln \left(\sum_{k=1}^K \out^{(d)}(k \mid \parsn^A; t_{i-1}) \N{\ydd{d} \mid \alphat{i}\left(K\oh{k}{K}- \vec{1}_{K\times N}\right)}{\alphat{i} K \vec{I}_{K\times N \times N}}\right)\label{discdisc_t_loss_exp}
% \end{align}
\begin{align}
&\calL_A = n\E_{i \sim U\{1,n\},\flow^A(\parsn^A \mid \vA ; t_{i-1}),\N{\y \mid \alphat{i}\left(K \oh{\vA}{KD} - \vec{1}\right)}{\alphat{i} K \vec{I}}} \ln \N{\y \mid \alphat{i}\left(K \oh{\vA}{K\times N} - \vec{1}\right)}{\alphat{i} K \vec{I}}\nonumber\\
&\qquad\qquad\qquad-\sum_{d=1}^N \ln \left(\sum_{k=1}^K \out^{(d)}(k \mid \parsn^A; t_{i-1}) \N{\ydd{d} \mid \alphat{i}\left(K\oh{k}{K}- \vec{1}\right)}{\alphat{i} K \vec{I}}\right)\label{discdisc_t_loss_exp}
\end{align}
where $\vec{I}\in \R^{K\times N \times N}$ and $\vec{1}\in\R^{K\times D}$. When sampling, the transition from $\bthetaA_{i-1}$ to $\bthetaA_{i}$ is derived as:
\begin{equation}
    p(\btheta^A_{i}|\parsnt{i-1}^\calM)=\update^A(\btheta^A_{i}|\btheta^A_{i-1},\predA{i-1};t_{i-1})
\end{equation}

The detailed training and sampling algorithm could be found in \cref{alg:train} and \cref{alg:sampling}.




%\documentclass[tikz,border=3.14mm]{standalone}
\usetikzlibrary{trees}

\begin{document}

\begin{tikzpicture}
  \node {-} {
    child {node {123}}
    child {node {1}}
  };
\end{tikzpicture}

%\begin{tikzpicture}
%  % Root node (subtraction)
%  \node {-}
%  child { % Left child (3x^2)
%    node {*}
%      child { % Left child (3)
%        node {3}
%      }
%      child { % Right child (x^2)
%        node {\^}
%          child { % Left child (x)
%            node {x}
%          }
%          child { % Right child (2)
%            node {2}
%          }
%      }
%  }
%  child { % Right child (-1)
%    node {1}
%    edge from parent[draw=none]
%  };
%\end{tikzpicture}

\end{document}

%\section{Conclusion}
In this work, we propose a simple yet effective approach, called SMILE, for graph few-shot learning with fewer tasks. Specifically, we introduce a novel dual-level mixup strategy, including within-task and across-task mixup, for enriching the diversity of nodes within each task and the diversity of tasks. Also, we incorporate the degree-based prior information to learn expressive node embeddings. Theoretically, we prove that SMILE effectively enhances the model's generalization performance. Empirically, we conduct extensive experiments on multiple benchmarks and the results suggest that SMILE significantly outperforms other baselines, including both in-domain and cross-domain few-shot settings.
%\section{Limitations \& a Brief Discussion}
% \blueitemize{
%     \item world generator: need for a manual solution as the Expert solution for each world
%     \item ACE: the benefits of ACE in non-iterative tasks that will not provide any actual feedback, ground truth, during the interaction with the problem is not clear yet. Our hypothesis is that having a ground truth is essential to make the triad of thesis, antithesis, and synthesis work in practice for a LLM. To investigate that, we evaluated a modified version of ACE that does not get any feedback signal from the world in MMLU benchmarks.
%     \item our finding shows that ACE will not provide with tangible improvements over the default. We saw that as an evidence for our previous hypothesis, while that will require more investigation    
% }
\textbf{WorldGen's Limitation:}
WorldGen effectively generates worlds with adjustable complexity for testing LLMs in SOPs, but relies on manually designed Expert solutions to solve the SOP. This dependence on human expertise for robust baselines can be time-intensive and limit the automation potential of the approach. We leave addressing the fully automated objective to future work.

\textbf{Limitations of ACE:}
% ACE and its dialectical framework have demonstrated great performance in our main targeted scenarios, SOPs. However, the effectiveness of ACE in other domains, particularly in settings where real-time feedback is unavailable, remains an open question. For instance, in static question-answering tasks or myopic scenarios where answers are final and lack iterative refinement opportunities, the benefits of ACE’s approach might not be as pronounced. To fully understand its potential and limitations in these contexts, extensive future evaluations are necessary. Moreover, by treating LLMs as black boxes, ACE performance is inherently constrained by the capabilities of the underlying base model. Lastly, the dialectical approach of ACE results in a slight natural increase in token consumption, though ACE is more token-efficient than multi-agent schemes like Debate or Majority. This overhead may pose challenges in scenarios where computational resources or cost efficiency are critical considerations, though where finding a robust solution outweighs minor increases in token costs, as in complex SOP tasks, this trade-off is less significant.
ACE's dialectical framework has demonstrated great performance in our main targeted domain, SOPs, but its effectiveness in other domains, particularly those lacking real-time feedback, remains an open question. In static question-answering or static tasks without iterative refinement, the benefits of ACE may be limited, warranting further evaluation. Additionally, by treating LLMs as black boxes, ACE's performance is inherently bound by the capabilities of the underlying model. Moreover, while its token consumption is lower than multi-agent schemes like Debate or Majority, ACE incurs a slight overhead compared to single-agent approaches. This trade-off is minor in complex SOP tasks but could pose challenges in resource-constrained scenarios.

\textbf{On the Potential of ACE:} 
% We think that the potentials of ACE and its Hegelian dialectical roots extend beyond solving SOPs. First, ACE and its dialectical aspect mirroring human-like problem-solving processes encourage the development of solutions that are not only accurate but also deeply contextual and well-reasoned. Moreover, the well-established philosophical frameworks such as Hegelian dialects, have the abilities to elucidate why other prompt engineering techniques, such as self-reflection, are effective. By framing these techniques within a dialectical structure, we can have a foundation to understand existing methods. Additionally, our Hegelian-inspired framework provides a robust structure for \textit{generating synthetic data}. The iterative nature of the dialectical process enables the creation of diverse and high-quality datasets that capture a wide range of perspectives and solutions. These datasets can be instrumental in training and fine-tuning LLMs, ensuring they are better equipped to handle complex and nuanced tasks.
% The potential of ACE, rooted in its Hegelian dialectical framework, extends beyond solving SOPs. Its dialectical approach, mirroring human-like problem-solving processes, fosters solutions that are not only accurate but also deeply contextual and well-reasoned. Furthermore, Hegelian philosophy provides a foundation to explain the effectiveness of other prompt engineering techniques, such as self-reflection, by framing them within a structured dialectical process. This perspective helps deepen our understanding of existing methods and their mechanisms. Additionally, the Hegelian-inspired framework offers a powerful structure for \textit{generating synthetic data}. Its iterative nature facilitates the creation of diverse, high-quality datasets that reflect a broad range of perspectives and solutions, making them invaluable for training and fine-tuning LLMs to tackle complex and nuanced tasks effectively.
The potential of ACE, rooted in its Hegelian dialectical framework, extend beyond solving SOPs. Its dialectical approach, mirroring human-like problem-solving processes, fosters solutions that are not only accurate but also deeply contextual and well-reasoned. Furthermore, Hegelian philosophy provides a foundation to explain the effectiveness of other prompt engineering techniques, such as self-reflection, by framing them within a structured dialectical process. This perspective can deepen our understanding of existing methods and their mechanisms.
Additionally, the Hegelian-inspired framework offers a powerful structure for \textit{generating synthetic data}. Its iterative nature facilitates the creation of diverse, high-quality datasets that reflect a broad range of perspectives and solutions, making them invaluable for training and fine-tuning LLMs to tackle complex and nuanced tasks effectively.

\textbf{LLM$^+$ Could Have Been Better!}
A fair criticism might be that LLMs might perform better in solving SOPs with improved prompt engineering. We are not claiming that LLM$^+$ represents the optimal default scheme; rather, we argue that it serves as a robust baseline. Even with carefully designed prompts, LLMs' performance in this setting remains limited, highlighting the need for approaches like ACE to unlock their full potential and deliver superior performance.

\textbf{What If the Next LLM Becomes Very Capable?} 
% Having a more capable LLM does not risk ACE becoming less useful. On the contrary, as demonstrated in section~\ref{sec:eval}, a better base model provides a stronger foundation for achieving even higher performance. In other words, ACE can make a great LLM even greater!
A more capable LLM makes ACE even more useful, not less. As demonstrated in Section~\ref{sec:eval}, a better base model serves as a stronger foundation, enabling even greater performance improvements. In essence, ACE with its dialectical base is designed to complement and amplify the capabilities of any LLM, regardless of its initial proficiency in SOP context. By leveraging ACE, we can transform an already impressive LLM into an extraordinary one, pushing what is possible and unlocking new levels of performance in this domain.

% While ACE demonstrates great advancements in enhancing LLMs performance across SOPs, it is not without limitations. Firstly, while ACE excels in our targeted scenarios (SOPs) involving sequential decision-making and dynamic feedback, its performance in other tasks such as myopic ones needs to be seen and requires more evaluations. Although ACE's dialectical method benefits in myopic tasks such as answering multiple-choice questions, the benefits might not as pronounced as in SOP setting, as no access to iterative feedback for antithesis and synthesis processes limits sufficient grounding to generate meaningful refinements. Second, ACE operates as a meta-framework, treating LLMs as black boxes without retraining or modifying their internal weights. As a result, its performance is naturally bounded by the inherent capabilities of the underlying base model. Third, the dialectical approach of ACE introduces a slight natural increase in token consumption due to the iterative exchange of Thesis, Antithesis, and Synthesis, along with associated explanations and strategy descriptions. While ACE is more token-efficient than multi-agent schemes such as Debate and Majority, it still incurs additional token cost compared to a single-agent scheme (LLM$^+$). This overhead may present a challenge in scenarios where computational resources or cost-efficiency are critical considerations, though this is less relevant in complex SOP tasks that finding a proper solution overrules the slight increase in the token cost.

% While ACE demonstrates great advancements in enhancing LLMs performance across SOPs, it is not without limitations. These limitations arise primarily from its dependency on the base model's capabilities, the token overhead associated with its dialectical structure, and its reliance on the availability and quality of feedback.

% ACE's architecture heavily relies on feedback from the World to generate effective Antitheses. This feedback acts as a guiding ground truth, allowing ACE to iteratively refine its reasoning and improve decision-making. In settings where real-time feedback is unavailable (e.g., static question-answering tasks), ACE's effectiveness diminishes, as evidenced by the reduced performance of ACE$^\ast$ in the MMLU benchmark. The lack of feedback in such scenarios limits the model's ability to effectively challenge and improve upon its initial hypotheses, thereby narrowing its scope of applicability.

% ACE operates as a meta-framework, treating LLMs as black boxes without retraining or modifying their internal weights. As a result, its performance is bounded by the inherent capabilities of the underlying base model. For example, in cases where the base model lacks foundational knowledge to solve a problem (e.g., achieving a 0\% success rate in certain tasks), ACE cannot overcome this limitation, as its dialectical structure builds upon the model's existing abilities.

% The dialectical approach of ACE introduces a natural increase in token consumption due to the iterative exchange of Thesis, Antithesis, and Synthesis, along with associated explanations and strategy descriptions. While ACE is more token-efficient than multi-agent schemes like Debate$^\star$ and Majority$^\star$, it still incurs approximately $2.27\times$ the token cost of a single-agent scheme (LLM$^+$). This overhead may present a challenge in scenarios where computational resources or cost-efficiency are critical considerations.

% Deploying ACE in real-world applications may involve additional complexity due to its multi-step process. Configuring the interplay between Thesis, Antithesis, and Synthesis, as well as integrating feedback mechanisms, requires careful design to ensure optimal performance.

% Although the WorldGen block is able to generate worlds with flexible complexity to test future LLMs with increased capabilities, we still require the design of an Expert solution to solve the SOP in a generated world. This is essential to establish access to the query budget and to provide a benchmark of efficient solutions for comparison with the performance of LLMs. That said, our approach is not fully automated. The reliance on a manually designed Expert solution introduces a dependency on human expertise for creating robust baselines, which may limit scalability and adaptability for new or unforeseen SOP scenarios. Additionally, designing such Expert solutions can be time-consuming and may not always generalize well to highly dynamic or unconventional world configurations generated by WorldGen. However, WorldGen provides a crucial first step toward automating this process by offering a structured and adaptable environment that can facilitate the eventual development of more autonomous and generalized expert systems.

% In summary, while ACE offers a powerful framework for enhancing LLM capabilities, its reliance on feedback, sensitivity to the base model, and token inefficiency highlight areas for further research and refinement. Future work could explore strategies to mitigate these limitations, such as incorporating lightweight feedback approximations, adapting ACE for tasks without dynamic feedback, and optimizing token usage.
%% \smallskip
% \myparagraph{Acknowledgments} We thank the reviewers for their comments.
% The work by Moshe Tennenholtz was supported by funding from the
% European Research Council (ERC) under the European Union's Horizon
% 2020 research and innovation programme (grant agreement 740435).



\begin{abstract}
%Large language models (LLMs) have demonstrated remarkable capabilities in many domains, yet their ability in System 2 tasks remain opaque. The inherent high-thinking demand and complex latent patterns of data, make it hard for llms to accurately infer solutions. Under this situation, the simple train-once-for-all post-training mechanism may then fail. In this paper, we propose \textbf{\modelname} to \textbf{Disen}tangle the complex problem solving procedure and heterogeneous data distribution into meta-units and adaptively aggregate the resulting \textbf{Lora}-experts to generate problem solver for each problem instance. 
%Concretely, 1) we disentangle the problem-solving stages into problem2thought and thought2solution processes, integrating the search process with a novel multi-expert MCTS algorithm, where reflexion-based pruning and refinement help to boost performance. 2) The resulting training data is thereafter disentangled into different meta clusters based on semantic distance, on which we finetune lora experts capable of different aspects of tasks.  3) Then, we train an input-aware hyper-network to adaptively aggregate the lora experts rank-wise for contextualized problem solver. Experiment results and various ablation studies validate the superiority of the data collection algorithm, the effectiveness of the data disentanglement process, and the performance gain brought by the input-aware hyper-network.
%\whj{第一句要改,如果重点是2->1的话}
Large language models (LLMs) have demonstrated remarkable capabilities in various domains, particularly in system 1 tasks, yet the intricacies of their problem-solving mechanisms in system 2 tasks are not sufficiently explored. 
Recent research on System2-to-System1 methods surge, exploring the System 2 reasoning knowledge via inference-time computation and compressing the explored knowledge into System 1 process. In this paper, we focus on code generation, which is a representative System 2 task, and identify two primary challenges: (1) the complex hidden reasoning processes and (2) the heterogeneous data distributions that complicate the exploration and training of robust LLM solvers. To tackle these issues, we propose a novel BDC framework that explores insightful System 2 knowledge of LLMs using a MC-Tree-Of-Agents algorithm with mutual \textbf{B}oosting, \textbf{D}isentangles the heterogeneous training data for composable LoRA-experts, and obtain \textbf{C}ustomized problem solver for each data instance with an input-aware hypernetwork to weight over the LoRA-experts, offering effectiveness, flexibility, and robustness. This framework leverages multiple LLMs through mutual verification and boosting, integrated into a Monte-Carlo Tree Search process enhanced by reflection-based pruning and refinement. Additionally, we introduce the DisenLora algorithm, which clusters heterogeneous data to fine-tune LLMs into composable Lora experts, enabling the adaptive generation of customized problem solvers through an input-aware hypernetwork. Our contributions include the identification of critical challenges in existing methodologies, the development of the MC-Tree-of-Agents algorithm for insightful data collection, and the creation of a robust and flexible solution for code generation. This work lays the groundwork for advancing LLM capabilities in complex reasoning tasks,  offering a novel System2-to-System1 solution.

%\whj{
%State-of-the-art (SOTA) Large Language Models (LLMs) demonstrate System-2-like intelligence through multi-step logical reasoning, yet remain limited in achieving human-level Artificial General Intelligence (AGI) across diverse tasks. Focusing on code generation, which are proven as an effective proxy for complex problem-solving, we identify two critical challenges: (1) xxxx, and (2) heterogeneous task distributions that resist unified modeling.

%We propose our Boost-Disentangle-Customize (BDC) framework, bridging System-2 collective reasoning with System-1 skill specialization. Our approach features:
%(1) MC-Tree-of-Agent: Collective Monte-Carlo searching and mutual verification boosted reasoning. (2) DisenLoRA Adaption: Parameter-efficient specialization via semantic clustering of reasoning trajectories and dynamic experts composition with input-awareness. Empirical results have shown up to $13.8\%$ improvement over GPT4o-mini on CodeContest-Hard and superior robustness against existing adapter merging algorithms.

%}
\end{abstract} 

\section{Introduction}

%\whj{Some questions here: (1) How should we build the corresponding inter-connections between system-1/2 and our two-stages problem-solving framework? system-1 -> P2T with system-2 -> T2S or system-2 -> P2T + T2S or backbone LLM as system-2 with lora adapters as system-1 }

%\begin{itemize}
%    \item LLM achievement
%    \item System 1-2
%    \item 
%    \item moe, sparsity
%\end{itemize}

%The core contributions of our work are two-fold: (1) We propose a two-stage problem-solving framework, namely Problem2Thought and Thought2Problem to enhance LLMs in complex reasoning tasks. And the performance of our framework, as a strong system-2 solver, is evaluated on enhanced trajectories from SOTA commercial LLMs. (2) Based on the collected trajectory data, we streamline a pipeline to strengthen the reasoning abilities of locally deployed LMs by training specialized LoRA experts and ensembling them in the inference-time with input-awareness, aiming to provide capable and specilized system-1 solvers.

%We like to organize the intro section in several paragraphs:

%1: Basic intro of recent progress of LLMs: The emerging and improving abilities of Large Language Models(LLM) as they scaling have attracted extensive attention from the community. They have exhibited comparable even human-surparsing capabilities in many tasks, for example autodriving, robotics, ... While unlike traditional deep learning models that excel in a certain narrow and highly specialized category of tasks, failing to generalize to a broader range of problems, LLMs produce general answer sequences for tasks grounded in text format, failing in given meaningful solution for specialized tasks. 

%2. If we consider human intelligence, SOTA commercial LLMs(e.g. GPT 4o, Claude-3.5) are close to the idea of system-2, featuring slow, rational and general reasoning capabilities. As contrast, system-1, identified by psycologist, represents quick, instinctful and specialized intelligence.

%3. Previous literature argues that the fusion paradigm of system-1 and system-2, for example materialized by a hierachicial structure contains dual reasoning and controling loop of both large and small models, can be essential for AGI. And LoRA adapters can adapt to various narrow downstream tasks without harming the general world knowledge held in the backbone LMs, which inspires us to adopt peft methods to instantiate system-1 solvers.

%4. Given the success of MoE structure and DeepSeek, the sparsely task-aware activativation of model parameters can be an important road balancing between  scaling for better reason abilities and optimizing inference cost for specific tasks. Thus we propose our HyperNet model for inference-time adatpers ensemble, while also leaving space for future optimization opportunity of sparsity.

%5. Introduce our core contributions.



%--------- \whj{Draft Writing Here} ---------
% Para 1: Importance of llms and the obstacles of system 2 tasks. 

\begin{figure}[t]
    \centering
    \includegraphics[width=1.0\linewidth]{figures/intro.pdf}
    \caption{Illustration of the motivation.}
    \label{fig:intro}
\end{figure}

Large language models show significant intelligence in various domains, striking both the academic and industrial institutions. Despite their prominent problem-solving abilities in system 1 tasks, the mechanism behind the system 2 task solving procedure remain opaque. In this paper, we focus on the code generation task, which emerges as a captivating frontier \citep{zheng2023codegeex, roziere2023code, shen2023pangu}, promising to revolutionize software development by enabling machines to write and optimize code with minimal human intervention. Recent research of llms for code focus on inference-time computation (System 2 methods) \citep{yang2024chain, yao2024tree, zhang2023planning} and post-training. While during post-training, distilling system 2 knowledge into system 1 backbones is important and widely-used \citep{yu2024distilling21}. 

However, the complex hidden reasoning process and the heterogeneous data distribution pose challenges to the existing System2-to-System1 pipeline. On one hand, the hidden reasoning process for code generation is complex and hard to explore \textbf{(C1)}. On the other hand, the heterogeneous data distribution, e.g., jumping structure like branching, recursion, etc., makes the existing train-once-for-all strategy hard to fit the complex latent patterns for robust and generalizable llm solvers \textbf{(C2)}. 

For \textbf{(C1)}, we propose to disentangle the problem solving process into problem2thought and thought2solution stages, exploring the inherent reasoning clues via combining the strengths of multiple llms by mutually-verification and boosting. The exploration is integrated into a Monte-Carlo Tree Search process, where reflexion-based pruning and refinement are designed for more efficient and effective reasoning clues search.

For \textbf{(C2)}, we propose to disentangle the heterogeneous data into clusters, finetuning llms capable of different aspects of tasks to obtain the meta LoRA experts hub, and then adaptively generate customized problem solver for each code problem.  Concretely, we design an input-aware hypernetwork to generate rank-wise weights over meta LoRA experts for customized problem solver, offering robustness and flexibility.

The main contributions of our work can be summarized below.
\begin{itemize}
    \item \textbf{Identification of problems and novel BDC framework.} We identify the high-reasoning demand and heterogeneous latent patterns problems that hinders the performance of existing methods and propose a BDC framework that explores insightful inherent reasoning clues via multi-llms boosting, generates meta-LoRA experts via finetuning on disentangled data, and offer customized problem solver with an input-aware hypernet for rank-wise LoRA merging.
    \item \textbf{Novel MC-Tree-of-Agents algorithm for insightful data collection.} We disentangle the System 2 solving process into problem2thought and thought2solution stages, integrating the exploration process into a reflexion-based monte carlo tree search armed with pruning and refinement, enabling mutually verification and boosting of different agents for insightful data collection. %\whj{Should we reorganize it as a multi-agent problem?}
    \item \textbf{Novel DisenLoRA algorithm that offers customized problem solver for robust code generation.} We disentangle the heterogeneous data distribution into clusters on which meta-LoRA experts are trained, and design an input-aware hypernetwork to weight over the LoRA-experts for customized problem solver, offering robustness and flexibility.
\end{itemize}

%\whj{
%The evolution of large language models (LLMs) has revealed distinct cognitive paradigms mirroring human dual-process theory [1]. Commercial gigantic LMs like GPT-4o and Claude-3.5 exhibit System-2-like characteristics - deliberate, multi-step reasoning through chain-of-thought (CoT) paradigms. This contrasts with earlier deep-learning models' System-1-like narrow intelligence in pattern recognition. Yet as (https://openreview.net/pdf?id=0ofzEysK2D) demonstrate, even state-of-the-art LLMs only reach "Level-2 AGI" - equivalent to unskilled human reasoning in a wide range of non-physical tasks - highlighting the need for frameworks that bridge systematic reasoning with specialized skill acquisition.

%We present a efficient pipeline inspired by human cognitive development: System-2-powered exploration followed by System-1 specialization. Our Boost-Disentangle-Compose (BDC) pipeline addresses two fundamental challenges in code generation: 1) The need for high-quality reasoning traces to bootstrap System-1 skill acquisition, and 2) The heterogeneous latent patterns in real-world coding tasks that resist monolithic adaptation approaches. In this paper, we focus on the code generation task, which is shown to be an effective proxy for various purposed tasks, including robotic control and math solving.

%By decomposing the code generation process into problem-to-thought(refining task specification) and thought-to-solution(synthesizing implementations) phases, we further strengthen the collective searching paradigm of the existing reason-and-reflect frameworks by:
%\begin{itemize}
%    \item Fine-grained mutual verification and pruning.
%    \item Complementary refinement boosting generation process across models.
%    \item Dynamic reward shaping via compilation feedback.
%\end{itemize}
%This collaborative reasoning process boosts System-2-like characteristics, yielding up to $13.8\%$ improvement on complex benchmarks.

%Additionally, the high-quallity samples collected from System-2 exploration unfold a resultant solution space featuring high intrinistic heterogeneity. Our analysis shows coding samples naturally clustered into several semantic groups, as visualized in 
%Fig.\ref{fig:mc}. This pattern underlines the importance of specialized skill acquisition via System-1-like frameworks.  We therefore propose DisenLoRA, which adapts parameter-efficient fine-tuning through:
%\begin{itemize}
%    \item Semantic disentanglement of solution clusters.
%    \item LoRA experts as per cluster.
%    \item A input-aware HyperNet for dynamic expert composition.
%\end{itemize}
%The proposed System-1-like framework achieves robust performance on both IID and OOD benchmarks. The dynamic composition with input-awareness also outperform existing adapter ensembling framework by $26\%$.

%The main contributions of our work can be summarized below.
%\begin{itemize}
%    \item \textbf{Identification of problems and novel BDC framework.} We identify the high-reasoning demand and heterogeneous latent patterns problems that hinders the performance of existing methods and propose a BDC framework that explores insightful inherent reasoning clues via multi-llms boosting, generates meta-lora experts via finetuning on disentangled data, and offer customized problem solver with an input-aware hypernet for rank-wise lora merging.
%    \item \textbf{Novel MC-Tree-of-Agents algorithm for insightful data collection.} We disentangle the System 2 solving process into problem2thought and thought2solution stages, integrating the exploration process into a reflexion-based monte carlo tree search armed with pruning and refinement, enabling mutually verification and boosting of different agents for insightful data collection. 
%    \item \textbf{Novel DisenLora algorithm that offers customized problem solver for robust code generation.} We disentangle the heterogeneous data distribution into clusters on which meta-lora experts are trained, and design an input-aware hypernetwork to weight over the lora-experts for customized problem solver, offering robustness and flexibility.
%\end{itemize}

%}


\section{Related Work}
\subsection{System 2 Methods in LLMs}
Recent research on large language models for System 2 tasks focus on inference-time computation optimization to stimulate the inherent reasoning ability of LLMs. Few-shot learning methods \cite{wang2022code4struct,madaan2022language} utilize the in-context-learning ability of LLMs for enhanced generation. Retrieval-augmented generation (RAG) approaches \cite{nashid2023retrieval,du2024codegragbridginggapnatural} further introduce domain knowledge into LLMs. 
Techniques such as Chain-of-Thought (CoT) \cite{yang2024chain,jiang2024self,li2023structured}, Tree-of-Thought (ToT) \cite{yao2024tree,la2024can}, and Monte Carlo Tree Search (MCTS) \cite{li2024rethinkmcts,zhang2023planning,hu2024uncertainty,hao2023reasoning,feng2024alphazeroliketreesearchguidelarge} are used to explore the inherent reasoning process, often based on the self-play mechanism to reflect on previously generated contents to learn from itself \cite{haluptzok2022language,chen2023gaining,lu2023self,chen2023teaching,madaan2024self,shinn2024reflexion}.
During inference, error position can be beneficial in improving the reliability and performance of the model. With identification and analysis of where and why errors occur, recent research \cite{yao2024mulberry, luo2024improve, wu2025error} has made significant strides in quantifying and mitigating errors during model inference. Refinement \cite{madaan2024self, gou2023critic} and reflexion \cite{shinn2024reflexion, lee2025evolving} are also powerful techniques for enhancing the inference capabilities of LLMs, usually by enabling iterative improvement and self-correction.

\subsection{Model Composition}
Model composition technique gains notable attention in cross-tasks generalization. 
Traditional methods for multiple tasks are to train models on a mixture of datasets of different skills \cite{caruana1997multitask, chen2018gradnorm}, with the high cost of data mixing and lack of scalability of the model though. Model merging is a possible solution to this. Linear merging is a classic merging method that consists of simply averaging the model weights \cite{izmailov2018averaging, smith2017investigation}. Furthermore, Task Arithmetic \cite{ilharco2022editing} computes task vectors for each model, merges them linearly, and then adds back to the base, and SLERP \cite{white2016sampling} spherically interpolates the parameters of two models. Based on Task Arithmetic framework, TIES \cite{yadav2024ties} specifies the task vectors and applies a sign consensus algorithm to resolve interference between models, and DARE \cite{yu2024language} matches the performance of original models by random pruning.

Recently, LoRA merging methods are also widely applied to cross-task generalization. CAT \cite{prabhakar2024lora} introduces learnable linear concatenation of the LoRA layers, and Mixture of Experts(MoE) \cite{buehler2024x, feng2024mixture} method has input-dependent merging coefficients. Other linear merging methods of LoRAs, such as LoRA Hub \cite{huang2023lorahub}, involve additional cross-terms compared to simple concatenation. %\whj{这句没看懂}

\section{Preliminaries}
\subsection{Monte-Carlo Tree Search}
Monte Carlo Tree Search (MCTS) is a decision-making algorithm widely used in environments with large state and action spaces, particularly in game AI and planning.  It incrementally builds search trees to estimate optimal actions by simulating random plays from various nodes and gradually improving action-value estimates based on simulation outcomes. Over iterations, this approach gradually converges to near-optimal decision-making policies. Notably, its integration with reinforcement learning has driven breakthroughs in systems like AlphaGo and AlphaZero \cite{silver2017mastering}, achieving superhuman performance in games.
%The combination of MCTS with reinforcement learning has achieved notable successes, such as in AlphaGo and AlphaZero \cite{silver2017mastering}.

Classical MCTS consists of four stages: selection, expansion, simulation, and backpropagation. It typically employs Upper Confidence Bounds for Trees (UCT) \cite{kocsis2006bandit}, which balances exploration and exploitation by guiding the search to promising nodes. After simulation, results propagate back through the tree, updating node values. However, MCTS struggles in domains with large action spaces, where excessive branching can degrade performance. Progressive Widening and Double Progressive Widening techniques have been proposed to mitigate this by dynamically limiting the number of actions considered at each decision node \cite{coulom2006efficient}.

\subsection{LoRA Finetuning}
%\dk{simple introduction and equations}
LoRA (Low-Rank Adaptation) \cite{hu2021lora} fine-tuning is a technique used to adapt large pre-trained models, such as transformers, to specific tasks with minimal computational overhead. The key idea behind LoRA is to introduce low-rank matrices into the model's weight updates, which reduces the number of trainable parameters and makes fine-tuning more efficient. 

LoRA starts with a model that has been trained on a large dataset. During finetuning, instead of updating the full weight matrix $W \in \mathbb{R}^{m \times n}$, LoRA introduces two low-rank matrices $A \in \mathbb{R}^{m \times r}$ and $B \in \mathbb{R}^{r \times n}$, where $r \ll \min(m, n)$. The updated weight matrix $W'$ is then given by:
\begin{equation}
    W' = W + \Delta W = W + A \cdot B.
\end{equation}

During fine-tuning, only the matrices $A$ and $B$ are updated, while the original weight matrix $W$ remains frozen. This reduces the number of trainable parameters from $m \times n$ to $m \times r + r \times n$, which is much smaller when $r$ is small. For a given task with loss function $\mathcal{L}$, the objective is to minimize:
\begin{equation}
\mathcal{L}(y, f(x; W + A \cdot B)),
\end{equation}
where $y$ is the target output, $x$ is the input, and $f$ is the model's forward function.

By introducing low-rank matrices, both the number of trainable parameters and memory footprint are reduced. This approach is particularly useful in scenarios where computational resources are limited or when fine-tuning needs to be done quickly.


\section{Methodology}
In this section, we introduce the overall methodology of BDC, addressing challenges in the System2-to-System1 pipeline for code generation, specifically the complexity of hidden reasoning processes and heterogeneous data distributions. 
The proposed BDC pipeline consists of three main stages: 1) explore the System 2 knowledge via mutual verification and boosting between LLMs; 2) disentangle the obtained data into clusters over which composible LoRA experts are tuned; 3) customize problem solver by weighting over the composable LoRA experts using an input-aware hypernetwork.

\begin{figure*}
    \centering
    \includegraphics[width=1.0\linewidth]{figures/framework.pdf}
   \caption{Illustration of the overall framework of \modelname.}
    \label{fig:overview}
\end{figure*}

\subsection{System 2 Knowledge Exploration}

%\textbf{Select.} The select phase aims to traverse the decision tree by selecting nodes that balance exploration of unvisited nodes with exploitation of high-value nodes. Starting from the root node, the process iterates through the tree by applying a selection policy. In our implementation, the selection process is governed by the Upper Confidence Bound (UCB) strategy, with a variant of probability-weighted UCB (P-UCB) for better exploration.
%Each node is either a DecisionNode (representing agent decisions) or a ChanceNode (representing state-action transitions). When at a DecisionNode, the node selection is made based on the node's estimated value and visitation frequency, guided by the UCB formula:
%$$\mathrm{PUCB}(node)=Q(s,a)+c\cdot P(a|s)\cdot\frac{\sqrt{\log N(s)}}{1+N(s,a)}$$ 

%where $Q(s, a)$ represents the action value, $P(a|s)$ is the policy’s probability of selecting action $a$ in state $s$, and $N(s)$ is the number of visits to state $s$. The selection proceeds until a terminal node or an unvisited node is encountered.

%\textbf{Expand.} Expansion occurs when a previously unvisited state or a ChanceNode is reached during the selection phase. This step involves adding a new DecisionNode to the tree. In our implementation, when a ChanceNode transitions to a new state $s_t$ after an action $a$, the reward $r(s_t ,a)$ is computed. The reward evaluates the correctness of the state $s_t$ based on the output's pass rate, which is quantified by the evaluation function $R(s_t)$, as seen in the equation:
%$$R(s_t)=\mathrm{PassRate}(s_t)$$
%In our setup, the PassRate is determined by the proportion of correct outputs as evaluated by a code correctness checker. Additionally, future rewards are calculated if the node reaches a terminal state, incorporating the step rewards from the current node to future nodes through rollouts. This is expressed as:
%$$r(s_t,a)=R(s_t)+\gamma\cdot\sum_{i=1}^Tr(s_{t+i},a)$$
%Here, $\gamma$ represents the discount factor applied to future rewards during backpropagation, while the total reward is a combination of the current reward and discounted future rewards. The expansion step also adds new DecisionNodes to the tree, allowing subsequent actions to be explored.

%\textbf{Simulate (Rollout).} The Simulate phase, also known as Rollout, begins after a ChanceNode has expanded. In this phase, a sequence of actions is sampled from the current policy, and the rewards are accumulated until a terminal state is reached or the maximum rollout depth is achieved. Each simulation (rollout) evaluates a potential trajectory from the current state, allowing the model to estimate future rewards. The rollout process can be expressed as: 
%\begin{enumerate}
%  \item For each state $s_t$  and action $a$, the state transitions to $s_{t+1}$ according to the transition function $f(s_t, a)$, which is defined as: $$s_{t+1}=f(s_t,a)=s_t+a$$ where $s_t$ is the prompt already generated before the action and $a$ is the continuing prompt (thought).
%  \item If a terminal condition is met, i.e., if the state reaches the terminal token or exceeds the maximum length, a terminal reward $R(s_t)$ is calculated based on the correctness of the output. Otherwise, an intermediate reward of 0 is assigned: $$r(s_t,a)=\begin{cases}R(s_t)&\text{if terminal,}\\0&\text{otherwise.}\end{cases}$$
%  \item The overall reward for the rollout is the sum of rewards across all steps, discounted by $\gamma$ (the discount factor for future rewards): $$\mathrm{RolloutReward}(s_t)=\sum_{i=0}^T\gamma^i\cdot r(s_{t+i},a)$$ Here, $r(s_{t+i} ,a)$ refers to the reward at each step during the rollout, and $T$ is the depth of the simulation, i.e., the number of steps taken until the terminal state or maximum depth is reached.
%\end{enumerate}

%\textbf{Back Up.} Once a terminal state is reached during the simulation, the back up phase updates the values of nodes from the terminal node back to the root. This step ensures that the accumulated rewards and visit counts are propagated upwards through the tree, refining the decision-making process over time: $$Q(s_t,a)\leftarrow r(s_t,a)+\gamma V(s_{t+1})$$ $$V(s_t)\leftarrow\sum_aN(s_{t+1})Q(s_t,a)/\sum_aN(s_{t+1})$$ $$N(s_t)\leftarrow N(s_t)+1$$ where $\gamma$ is the discount factor.

%\ljx{
%In this paper, we describe the multi-llms MCTS procedure employed in our system, adapted to solve sequential decision-making problems under uncertainty, including 2 parts of problem2thought and thought2solution, leveraging probabilistic state transitions and reward signals. Pruning and refinement mechanism are also applied to minimize the influence of wrong thoughts or make some use of them. The overview of our strategies is shown in \ref{fig:copilot}.

%\textbf{Select.} The select phase aims to traverse the decision tree by selecting nodes that balance exploration of unvisited nodes with exploitation of high-value nodes. Starting from the root node, the process iterates through the tree by applying a selection policy. In our implementation, the selection process is governed by the Upper Confidence Bound (UCB) strategy, with a variant of probability-weighted UCB (P-UCB) for better exploration.
%Node selection begins at the root node. It is made based on the node's estimated value and visitation frequency, guided by the UCB formula:
%$$\mathrm{PUCB}(S^T_d)=Q(s)+c\cdot P(a|s_p)\cdot\frac{\sqrt{\log N(s_p)}}{1+N(s)}$$

%where $s$ is the state of the node $S^T_d$, $s_p$ is its parent's state $Q(s_p, a)$ represents the node value, $P(a|s_p)$ is the policy’s probability of selecting action $a$ in state $s_p$, and $N(s)$ is the number of visits to state $s$. The child of the current node with the highest P-UCB value is chosen for the next selection. The selection proceeds until a terminal node or an unvisited node is encountered.

%\textbf{Expand.} Expansion occurs when a previously unvisited state during the selection phase. This step involves adding candidate reasoning nodes as the children of the current selected node. LLMs($\pi_1, \pi_2$) are given the current node $S^T_d$ with state $s$ to generate next thoughts for solving the problem along with the probability$P(a|s)$ of each thought:
%$$a_i, P(a_i|s) \sim \pi_i(·|Q,s, \text{prompt}_{thought})$$

%This is actually the problem2thought process. New generated nodes can be represented by $S^{T_i}_{d+1}$ with state $s_i=s+a_i$, where $T_i = join(T,i)$ and $i$ means that this node is generated by $\pi_i$.

%\textbf{Simulate.}In this operation, LLMs utilizes the selected thoughts to generate solution, i.e. the thought2solution process. Given the selected state, each LLM generates its own version of solution:
%$$So_i\sim \pi_i(·|Q, s)$$
%And we estimate the node reward through the average passrate of the solutions:
%$$R(s)=\frac{1}{n}\sum_{i=1}^n\text{PassRate}(So_i)$$.
%If a terminal condition is met, i.e., if the state reaches the terminal token or exceeds the maximum length, a terminal reward $r(s_t)$ is calculated and $R(s_t)$ is then set to $0$. Otherwise, an intermediate reward of 0 is assigned: $$r(s_t)=\begin{cases}R(s_t)&\text{if terminal,}\\0&\text{otherwise.}\end{cases}$$

%\textbf{Back Up.}After the simulation operation in which we get the node reward, the back up phase updates the values of nodes from bottom to the root. This step ensures that the accumulated rewards and visit counts are propagated upwards through the tree, refining the decision-making process over time:
%$$Q(s_t) \leftarrow f(Q(s_t),r(s_t)+ \gamma Q(s_{t+1}))$$
%$$N(s_t)\leftarrow N(s_t)+1$$
%where states from root to the bottom is represented as $s_0,s_1,...,s_d$, $Q(s_d)=r(s_d)+\gamma R(s_d)$, $f$ is a value calculation function, normally $max$ or $averaging$ all the values that backpropagate, and $\gamma$ is the discount factor.

%\textbf{Prune.}If the pruning mechanism is applied, the new selected node will be examined to show if it is a wrong node. A node is thought to be a wrong node if its reward is less than its parent's, i.e. $R(s_t) < R(s_{t-1})$, which means the new thought brings no useful information. The wrong node will be pruned, so no expansion will be conducted. But \textbf{Back Up} operation is still carried on to reduce the likelihood of selecting this trajectory.

%\textbf{Refine.}After a wrong node is found, we can get its error information and analysis. Then the information can be summarized in natural language so that LLMs can understand it more easily. With the summary, LLMs generate a new thought to replace the past one, which contains information about avoiding current errors. For the sake of debate, wrong thoughts generated by one LLM is analyzed by another LLM. If the new thought is still recognized as a wrong thought, \textbf{Refine} operation is conducted again. But all rethinking times for one problem can't exceed the max thinking times, and each wrong node is regenerated at least once.
%For coding, we utilizes failed test cases and block analysis to generate error summary:
%$$Su(S^{T_i}_d)\sim\pi_j(·|Q,s_d,\text{BlockAnalysis}(s_d))$$
%And the summary and failed test cases are used for refined thought generation:
%$$a_i' \sim \pi_i(·|Q,s_p,Su(S^{T_i}_d),\text{FailedTestCases}(s_d)$$
%Here, only a part of failed test cases are given to the LLM if there are too many. Then, both the rewards of the first wrong thought and the final thought are backpropagated for fair information, and the node left will be expanded.

%\whj{ -------- my draft of Sec4.1 starts from here -------}
In this subsection, we introduce the mechanism design for the data collection process. Due to the complex reasoning nature embodied, code blocks are hard to evaluate and estimate before mature. Reliable reward signals of a reasoning path therefore mainly depend on the dynamic compilation and execution feedbacks, which are extremely sparse and require extensive simulations. To simplify the generation paradigm and exploit the mutual verification capabilities of the collective searching, we decompose the generation process into two distinct stages: problem-to-thought and thought-to-solution.

\subsubsection{Problem-to-thought}

Traditional Monto-Carlo Tree Searching comprises three key operations in each iteration: (a) Select, (b) Expand, (c) Backup. In the problem-to-thought stage, we further extend MCTS by two distinct operations (d) Prune, and (e) Refine to reduce the searching space. We elaborate on these operations as follows.

\minisection{Select} 
Starting from the root, the reasoning path is prolonged by iteratively adding a specific child of the latest node. The operation is usually governed by certain policies, among which we adopt Probability-weighted Upper Confidence Bound(P-UCB) to balance the exploration and exploitation:
\begin{equation}
\mathrm{PUCB}(S_c)=Q(S)+c\cdot P(a|S_p)\cdot\frac{\sqrt{\log N(S)}}{1+N(S_c)},
\end{equation}
where $S_c$ is the state of the child node. $S$ and $Q(S)$ denote the parent node's state and value. $P(a|S)$ is the conditional probability of sampling the sequence $a$. $N(S)$ is the total number of times the parent node $S$ has been visited during simulations, while $N(S_c)$ tracks visits to the child node $S_c$. The selection process will stop if either a semantic or rule-based(e.g. length limits) terminal state encounters.

\minisection{Expand}
The Expand operation is triggered if a non-terminal leaf node of the tree is selected. A set of predefined LLM polices $\pi_0, \cdots, \pi_n$ generate subsequent thought sequences ${a_i}_n$ given the state $S$ of the current node, forming new leaf nodes:
\begin{equation}
    \forall i\in[n], P(a_i|S) \sim \pi_i(·|S).
\end{equation}

%following sequence concatenation as a deterministic transition function $S^i_c = S + a_i$ and question prompt as the initial state, the newly included leaf nodes are set with states ${S_i_c}$.


\minisection{Backup}
For well-defined problems, a reasoning path ${S_t}$ will eventually end at a terminal leaf node $S_T$ by iterating the Select and Expand operations. The reward $r_T$ is set according to the evaluation. We will skip the definition of reward $r_t$ and passrate $PR(S_t)$, which will be detailed in the explanation of the Simulate operation. The reward value is back-propagated along the reasoning path to update the state values of corresponding ancestor nodes:
\begin{equation}
    Q(S_{t-1}) = f(Q(S_t), r_t + \gamma PR(S_t)),
\end{equation}
where $f$ is the value function.

Additionally, the visit counts of ancestors are updated alongside the reasoning path:
\begin{equation}
    N(S_t) = N(S_t) + 1.
\end{equation}



We further extend and formalize reflective reason settings proposed in CoMCTS into Prune and Refine operations as shown in Figure~\ref{fig:mc}.

\begin{figure}[h]
    \centering
    \includegraphics[width=1.0\linewidth]{figures/datacollection.pdf}
    \caption{Pruning and refinement operations.}
    \label{fig:mc}
\end{figure}

\minisection{Pruning}
The pruning operation on a selected node will examine and compare its passrate with that of its parent. With powerful LLMs, we consider valid and reasonable thoughts to bring non-negative influence solution seeking, thus featuring monotonically non-decreasing values in the passrate $PR(S_t) <= PR(S_{t+1})$.

A child node alleviating this principle will be considered as an ill node that introduces wrong thoughts. The ill node will be removed and trigger an instant Backup operation with zero reward to downweight its ancestors.


\minisection{Refine}
The truncated error and state information left by ill nodes will be analyzed in the Refine operations. To mitigate the bootstrapping bias of LLMs, a distinct policy LLM will be adopted to infer and summarize the information in natural language, which will be later utilized to refine and replace the ill nodes:
\begin{align}
    isIll(S^{\pi_i})&== 1,\nonumber\\
    Summary(S^{\pi_i}) &\sim \pi_{j}(Q(S^{\pi_i}),\\
    S^{\pi_i}, & BlockAnalysis(S^{\pi_i})),\nonumber
\end{align}
where $S^{\pi_i}$ denotes a ill node generated by $\pi_i$. A refined node is generated to replace the ill one:
\begin{equation}
    a' \sim \pi_{i}(Q(S^{\pi_i}), Summary).
\end{equation}

We enforce global and local constraints on possible times of calling Refine operation to avoid infinite loops and balance performance with compute budgets. A successful Refine operation will cause an in-place replacement of the ill-node, triggering another Backup operation to re-weight its ancestors.


\subsubsection{Thought-to-solution}

\minisection{Simulate}
For the thought-to-solution, we repurpose the Simulate operation for the collective solution generation process from the given state $S$. The operation will produce a set of possible solutions, each from a policy LLM:
\begin{equation}
    Solut.(S)_i \sim \pi_i(S).
\end{equation}

We define the passrate of a state as the average passrate of its corresponding solutions:
\begin{equation}
PR(S) = \frac{1}{n}\sum_{i}^{n} Passed(Solut.(S)_i),
\end{equation}
where $Passed(\cdot)$ represents the supervising signal from dynamic compilation and execution feedback.

The node's value $Q(S_t)$ is determined by its $PR(S_t)$ and reward $r_t$. Sincere additional solution string will be appended to a non-terminal state $S_t$ before evaluation, $PR(S_t)$ is an indirect supervising signal for the $S_t$, and the direct signal $r_t$ is set to zero.

The terminal state $S_T$ is treated as the unique solution itself since no string concatenation applies, therefore featuring a non-trivial reward $r_T$. Putting everything together, we have:
\begin{equation}
    Q(S) =\begin{cases} r_T &\text{if terminal,}\\\gamma PR(S_t) &\text{otherwise.}\end{cases}
\end{equation}

%\whj{ -------- my draft of Sec4.1 ends from here -------}
%}

\subsection{System2-to-System1 Training}
\subsubsection{Heterogeneous Distribution Disentanglement}
\label{sec:dis}
After the data collection, the resulting training data obtained from the  MC-Tree-Of-Agents process consists of problem2thought data $D^{p2t}=\{\langle X_i^{p2t},y_i^{p2t}\rangle| i\in\mathbf{P}\}$ and thought2solution data $D^{t2s}=\{\langle X_i^{t2s},y_i^{t2s}\rangle| i\in\mathbf{P}\}$: $D_{train} = \{D^{p2t}, D^{t2s}\}$. As discussed in the introduction section, the latent patterns of coding problems are complex and tend to be heterogeneously distributed, e.g., the branching and recursion flow existing in the code blocks, different strategies of algorithm solutions, etc. Therefore, we disentangle the training data based on the latent semantics of the data into different clusters for fine-grained modeling.

The clustering objective can be summarized as below:
\begin{align}
    minimize_{\mathcal{C}} &\sum_k\sum_{i\in C_k} cosine(e_i, \mu_k), \\
    e_i = &\Phi_\theta (\langle X_i, y_i \rangle),\nonumber \\ 
    \mu_k = &mean\{e_i | i\in C_k\},\nonumber
\end{align}
where $\Phi_{theta}$ is the encoder of a code llm and $\mu_k$ denotes the centroid of cluster $C_k$.

\subsubsection{Composable LoRA Experts Preparation}
Having obtained the disentangled data clusters, we then finetune on them to obtain the meta LoRA experts for specialized experts of different aspects.
\begin{align}
    \forall C_k &\in \mathcal{C}, \nonumber\\
    \pi_{\theta_k} & \leftarrow SFT(\pi_{\theta}, \{\langle X_i, y_i\rangle | i \in C_k\}),
\end{align}
where $\pi_\theta$ denotes the base LLM and $\pi_{\theta_k}$ denotes the parameters of the LoRA adapter obtained by finetuning $\pi_\theta$ on $C_k$.

\subsubsection{Input-Aware Hypernetwork for Customized Solver}

Given specialized LoRA experts ${\pi_{\theta_1},\cdots,\pi_{\theta_K}}$ trained on distinct data clusters, we design an input-aware Hypernetwork $f(\cdot)$ to dynamically compose these experts through rank-wise adaption for customized problem solver.

For each input instance, the hypernetwork generates customized expert weights digesting its encoding and semantic distances to the cluster centroids. we identify "rank" as the minimal unit for aggregation and generate rank-wise weights for different experts at each decoding layer:
\begin{equation}
    G_i \leftarrow f(e_i, \langle cosine(e_i, \mu_1), \dots, cosine(e_i, \mu_K)\rangle),
\end{equation}
where $e_i$ is the encoding of input $X_i$, $G_i\in R^{K\times r\times 1}$ is the output weight matrix, $r$ is the rank of the LoRA matrices, and $K$ is the number of LoRA experts.

The aggregated $\Delta W$ of the linear projection layer is then obtained by
\begin{align}
    \mathbf{A}^* = [A_1, \dots, A_K] \odot G_i, \\
    \mathbf{\Delta W}^* = [B_1A_1^*, \dots, B_KA_K^*],\\
    \Delta W = ReduceSum(\mathbf{\Delta W}^*).
\end{align}

%To keep the magnitude of the customized $\Delta W$ from collision, we regularize the obtained $\Delta W$ using an anchored matrix aggregated using distances from each centroid.
%\begin{equation}
%    \Delta W = \Delta W \frac{||\Delta W_{anchor}||}{||\Delta W||},
%\end{equation}
%where $\Delta W_{anchor}$ is the distance-weighted matrix over the LoRA-experts using the distance of input to each centroid and $||\cdot||$ denotes the infinity norm. 

% The resulting $\Delta W$ is then merged into the original weight matrix for forwarding:
The projection output of $\Delta W$ is then merged during forwarding via:
\begin{equation}
    y= W_0x+\Delta Wx.
\end{equation}

%During the process, only $f(\cdot)$ is trainable while other parameters are frozen. 
We adopt a dedicated training phase for the Hypernetwork where all parameters are frozen except for the $f(\cdot)$. The training is supervised by the cross-entropy loss, with the randomly permuted input-output pairs from $D_{train}$.



\section{Experiments}
We conduct empirical studies starting from the following research questions.
\begin{itemize}[leftmargin=27pt]
    \item [\textbf{RQ1}] Does the proposed data collection algorithm explore insightful reasoning knowledge?
    \item [\textbf{RQ2}] Do the complex latent patterns of reasoning data impact the training performance?
    \item [\textbf{RQ3}] Can the disentangle-and-compose mechanism help to promote performance?
    \item [\textbf{RQ4}] Do the proposed input-aware hypernet work outperform other model composition techniques?
    %Does the proposed \modelname model the high-level thought-of-codes? Can \modelname offer cross-lingual augmentation?
    \item [\textbf{RQ5}] How does DisenLoRA perform on untrained datasets?
\end{itemize}

%\whj{Possible experiment data that can directly show the importance of disentanglement: (1) Different models show diverse performance in different stages(p2t, t2s). (2) Models perform better in verifying decomposed thoughts and solutions.}

\subsection{Setup}
In this section, we provide detailed setup information for the evaluation of the proposed \modelname, including datasets, trajectory data collection, and competing methods. 

The overall evaluation is conducted on two benchmark datasets: the competition-style APPS dataset and the CodeContest dataset. Both datasets categorize problems from easy to hard. We randomly sample problem subsets from each category of these two datasets. Each subset contains approximately 100 problems, except for the CodeContest-Hard category, which consists of around 50 problems due to inherent limitation in size. 

We conduct isolated assessments of both stages of \modelname to ensure a comprehensive comparison. 

\minisection{ Data collection} For Python code generation, we compare the performance of MCTS over different methods: zeroshot, LDB \cite{zhong2024ldb}, RAP \cite{hao2023reasoning}, Reflexion, LATS \cite{zhou2023language}, ToT and RethinkMCTS \cite{li2024rethinkmcts}. To mitigate the influence of factors such as context window limitations and instruction-following capabilities, we employ two advanced base models: GPT-4o-mini and Claude-3.5-Sonnet. Aligned with the purpose, we adopt a greedy decoding strategy for both models. Additionally, we provided peer comparisons between these two base models when driven by the MC-Tree-Of-Agents method in terms of their error position and refinement capability.

\minisection{ Fine-tuning} For fine-tuning, \modelname is compared against several alternative methods, including SFT on clustered subsets, TIES, DARES, and TWINS \cite{liu2023twins}.

%\whj{In this section, we provide detailed setup information for evaluating of the proposed \modelname(ours), including datasets, trajectory data collection and competeting methods. 

%The overall evaluation is conducted on two benchmark datasets: the competition-style APPS dataset and the CodeContest dataset. Both datasets categorize problems from easy to hard. We randomly sample problem subsets from each category of these two datasets. Each subset contains approximately 100 problems, except for the CodeContest-Hard category, which consists of around 50 problems due to inherent limitation in size. 

%We conduct isolated assessments of both stages of \modelname to ensure a comphensive comparsion. 

%\minisection{ Data collection} For Python code generation, we compare the perfomance of MCTS over different methods: zeroshot, LDB\cite{zhong2024ldb}, RAP\cite{hao2023reasoning}, Reflexion, LATS\cite{zhou2023language}, ToT and RethinkMCTS\cite{li2024rethinkmcts}. To mitigate the influence of factors such as context window limitations and instruction-following capabilities, we employ two advanced base models: GPT-4o-mini and Claude-3.5-Sonnet. Aligned with the purpose, we adopt greedy decoding strategy(temperature and other args?) for both models. Additionally, we provided peer comparsions between these two base models when driven by the MC-Tree-Of-Agents method in terms of their error position and refinement capability.

%\minisection{ Fine-tuning} For fine-tuning, \modelname is compared against several alternative methods, including SFT on clustered subsets, TIES, DARES and TWINS\cite{liu2023twins}.}

%In this paper, we evaluate \modelname with the competition-style APPS dataset and the CodeContest dataset. Both datasets are categorized by different difficulty levels. For each level of every dataset, we select\whj{random sample?} about 100 problems for testing, except for CodeContest Hard category, which contains about 50 problems. Greedy decoding strategy is applied to the generation. The evaluation metric is PassRate(PR) and Accuracy(AC).

%For data collection, we evaluate the python code generation abilities of GPT4omini and Claude-3.5-sonnet with different methods: zeroshot, LDB\cite{zhong2024ldb}, RAP\cite{hao2023reasoning}, Reflexion, LATS\cite{zhou2023language}, ToT, MCTS and RethinkMCTS\cite{li2024rethinkmcts}. Based on MC-Tree-Of-Agents, we also evaluate both models with error position and refinement. For tuning, we evaluate different finetuning methods including SFT on different clusters, TIES, DARES, TWINS\cite{liu2023twins} and \modelname(ours).

\begin{table*}[ht]
\centering
\caption{Main results on System 2 knowledge exploration.}
\label{tab:datacollection}
\resizebox{0.9\textwidth}{!}{
\begin{tabular}{c|cccccc|cccc} 
\hline
& \multicolumn{6}{c|}{APPS} & \multicolumn{4}{c}{CodeContest}\\
       Models                 & \multicolumn{2}{c}{Intro.}                            & \multicolumn{2}{c}{Inter.} & \multicolumn{2}{c|}{Comp.} & \multicolumn{2}{c}{Easy} & \multicolumn{2}{c}{Hard} \\ \cline{2-11} 
                        & PR                     & AC                    & PR       & AC       & PR       & AC       & PR          & AC         & PR          & AC         \\ \hline
ZeroShot                & 56.56 & 35.00 &      40.57           &     19.00          &   23.67              &     9.00          &29.03& 19.61 &   28.24                 &19.23              \\
LDB                     & 60.64 & 40.00 &     46.78            &     22.00          &      21.00           &    8.00           & 34.76              & 25.58           & 33.52                   & 16.28                \\
RAP                     & 64.24                         & 39.00                         &   43.32              &     14.00          &         22.83        &    8.00           & 43.08              & 33.33            &39.99  &26.92                 \\
Reflexion               & 60.65                         & 40.00                         &  45.58               &     21.00          &      17.50           &     7.00          & 56.16              & 47.83           &34.09 &21.15                 \\
LATS                    & 69.46                         & 50.00                         &  45.86               &     20.00          &       21.83          &        7.00       &57.70              & 47.83           &39.10                    &30.77                 \\
ToT                     & 74.34                         & 55.00                         &  63.49               &     33.00          &         26.30        &       11.00        & 51.89              & 41.18           &49.07                    & 32.69                \\ 
RethinkMCTS                     & 76.60                         & 59.00                        &  74.35               &     49.00         &         42.50        &       28.00        & 60.84             & 51.53          & 55.79                & 48.04               \\ \hline
%MCTS-Line               &                               &                            &                 &               &                 &               &                    &                 &                    &                 \\
%MCTS-Thought            &                               &                            &                 &               &                 &               &                    &                 &                    &                 \\ \hline
Single (GPT4omini) & 77.99                         & 60.00                         &     72.89            &      50.00            &      44.17         &      25.00     & 55.79     &48.04                &      45.72              & 26.92                \\
%Single-MCTS (Yi)        & 70.92                         & 53.75                      &                 &               &                 &               &                    &                 &                    &                 \\
Single (Claude)    &  73.80  &  61.00   &  73.60   & 57.00 &      54.67           &      42.00         &                  58.75& 53.92                &68.41          &55.76             \\ \hline
MC-Tree-Of-Agents              & 79.72                         & 64.00                      &       79.42          &    63.00           &       59.17          &    45.00           & 62.49               &54.64             & 70.49                   &56.41                 \\ 
%+ Error Position (v1)           & 83.07                         & 69.00                      &       78.65          &    64.00           &       57.67          &    41.00           & 65.92               &58.82             & 70.14                   &51.92                 \\
%+ Refine   (v1)        & 83.24                         & 72.00                      &       79.65          &    63.00           &       55.83         &    38.00           & 62.73               &52.94             & 66.95                   &50.00                 \\
+ Pruning           & 85.18                         & 76.00                     & 81.95                & 67.00             &54.00              & 38.00            & 64.62            &  59.80         &73.12             &59.62                 \\
+ Refine      & 81.29                         & 68.00                    & 78.85              & 62.00            & 60.33             & 44.00            & 63.23             &  56.86          & 73.80                   &63.46\\                 \hline
\end{tabular}
}
\end{table*}

% Please add the following required packages to your document preamble:
% \usepackage[table,xcdraw]{xcolor}
% Beamer presentation requires \usepackage{colortbl} instead of \usepackage[table,xcdraw]{xcolor}
% Please add the following required packages to your document preamble:
% \usepackage[table,xcdraw]{xcolor}
% Beamer presentation requires \usepackage{colortbl} instead of \usepackage[table,xcdraw]{xcolor}
\begin{table*}[ht]
\centering
\caption{Main results on System2-to-System1 tuning.}
\label{tab:tuning}
\resizebox{\textwidth}{!}{
\begin{tabular}{ccccccccccccccc}
\hline
\multicolumn{15}{c}{ Meta-llama-3.1-instruct-8b}                                                                                                                                                                                                                  \\ \hline
\multicolumn{1}{c|}{}                                         & \multicolumn{2}{c}{Intro. (100)} & \multicolumn{2}{c}{Inter. (100)} & \multicolumn{2}{c|}{Comp. (100)}  & \multicolumn{2}{c|}{Overall}      & \multicolumn{2}{c}{Easy (102)} & \multicolumn{2}{c|}{Hard (51)}     & \multicolumn{2}{c}{Overall} \\ \cline{2-15} 
\multicolumn{1}{c|}{\multirow{-2}{*}{Finetune Method}}        & PR              & AC             & PR              & AC             & PR    & \multicolumn{1}{c|}{AC}   & PR    & \multicolumn{1}{c|}{AC}   & PR             & AC            & PR    & \multicolumn{1}{c|}{AC}    & PR           & AC           \\ \hline
\multicolumn{1}{c|}{w/o tuning}                               & 21.14           & 4.00           & 20.72           & 4.00           & 12.83 & \multicolumn{1}{c|}{1.00} & 18.23 & \multicolumn{1}{c|}{3.00} & 25.54          & 17.65         & 15.46 & \multicolumn{1}{c|}{5.77}  & 22.18        & 13.69        \\ \hline
\multicolumn{1}{c|}{SFT on all}                               & 22.55           & 7.00           & 26.40           & 3.00           & 10.67 & \multicolumn{1}{c|}{1.00} & 19.87 & \multicolumn{1}{c|}{3.67} & 25.33          & 17.65         & 16.73 & \multicolumn{1}{c|}{7.69}  & 22.46        & 14.33        \\ \hline
\multicolumn{1}{c|}{ SFT on cluster 0} & 20.67           & 6.00           & 24.23           & 3.00           & 11.50 & \multicolumn{1}{c|}{1.00} & 18.80 & \multicolumn{1}{c|}{3.33} & 27.31          & 17.65         & 11.69 & \multicolumn{1}{c|}{1.92}  & 22.10        & 12.41        \\
\multicolumn{1}{c|}{ SFT on cluster 1} & 21.22           & 4.00           & 20.69           & 4.00           & 12.00 & \multicolumn{1}{c|}{2.00} & 17.97 & \multicolumn{1}{c|}{3.33} & 27.78          & 20.59         & 18.12 & \multicolumn{1}{c|}{9.62}  & 24.56        & 16.93        \\
\multicolumn{1}{c|}{SFT on cluster 2}                         & 16.65           & 7.00           & 23.97           & 3.00           & 17.33 & \multicolumn{1}{c|}{4.00} & 19.32 & \multicolumn{1}{c|}{4.67} & 26.82          & 20.59         & 19.50 & \multicolumn{1}{c|}{9.62}  & 24.38        & 16.93        \\ \hline
\multicolumn{1}{c|}{Ties}                                     & 22.75           & 4.00           & 23.06           & 4.00           & 12.67 & \multicolumn{1}{c|}{4.00} & 19.49 & \multicolumn{1}{c|}{4.00} & 26.64          & 21.57         & 18.71 & \multicolumn{1}{c|}{9.62}  & 24.00        & 17.59        \\
\multicolumn{1}{c|}{Dare}                                     & 24.97           & 7.00           & 26.66           & 5.00           & 12.50 & \multicolumn{1}{c|}{3.00} & 21.38 & \multicolumn{1}{c|}{5.00} & 23.05          & 13.73         & 19.65 & \multicolumn{1}{c|}{15.38} & 21.92        & 14.28        \\
\multicolumn{1}{c|}{Twin}                                     & 19.10           & 5.00           & 23.85           & 5.00           & 8.50  & \multicolumn{1}{c|}{1.00} & 17.15 & \multicolumn{1}{c|}{3.67} & 26.87          & 17.64         & 12.92 & \multicolumn{1}{c|}{9.62}  & 22.22        & 14.97        \\ \hline
\multicolumn{1}{c|}{DisenLoRA}                & \textbf{27.11}           & \textbf{9.00}           & 23.11           & 3.00           & 11.50 & \multicolumn{1}{c|}{\textbf{4.00}} & \textbf{20.57} & \multicolumn{1}{c|}{\textbf{5.33}} & \textbf{32.24}          & \textbf{22.55}         & 19.43 & \multicolumn{1}{c|}{9.62}  & \textbf{27.97}        & \textbf{18.24}        \\ \hline
\end{tabular}
}
\end{table*}

\subsection{Empirical Analysis and Discussion}
\subsubsection{\textbf{RQ1}. MC-Tree-Of-Agents}

We evaluate MC-Tree-Of-Agents against widely-used baseline methods, the results are summarized in Table~\ref{tab:datacollection}.
From the results, we can draw the following conclusions. 
\begin{itemize}[leftmargin=10pt]
    \item The proposed MC-Tree-Of-Agents outperforms all the baseline methods, which effectively explores the insightful  System 2 knowledge. 
    \item Comparing with the single LLM as agents version, MC-Tree-Of-Agents allows for mutual verification and boosting between different LLMs, offering a superior performance over each distinct-LLM-as-agent method. This showcases the effectiveness of the interaction between LLMs of different wisdom.
    \item The pruning and refinement operations both contribute to the final performance, offering a notable accuracy gain. This validates that the designed pruning and refinement mechanism, based on the difference between rewards of parent-child nodes, can save the algorithm from erroneous exploration and lead to beneficial directions in limited rollouts.
\end{itemize}


\subsubsection{\textbf{RQ2}. Impact of latent patterns}

To study the distribution of the latent patterns of coding problems, we first conduct the T-SNE visualization on the encodings of reasoning data collected by MC-Tree-Of-Agents on APPS dataset. The visualization is displayed in Figure~\ref{fig:t-sne}.
%\vspace{-10pt}
\begin{figure}[h!]
    \centering
    \includegraphics[width=0.8\linewidth]{figures/apps_tsne.png}
    \caption{T-sne visualization of the APPS data encoding.}
    \vspace{-10pt}
    \label{fig:t-sne}
\end{figure}

From the visualization, we can see that there different clusters of data distributions existing in the latent reasoning semantic space, which poses a potential challenge to robust and generalizable LLMs on code.

Furthermore, we perform finetuning on different clusters of data obtained in Section~\ref{sec:dis} and evaluate the corresponding models on the test data. From the results in Table~\ref{tab:tuning}, we can see the following conclusions. 1) LLMs finetuning on all the clusters can offer better performance than that of the non-tuning version, validating the quality of the collected System2 knowledge data. 2) Llm experts obtained from different clusters show different performances on different levels of tasks. One expert can demonstrate outstanding capability on one level of tasks, even outperforming the LLM finetuning on all the data, while performing weakly on a different level of task. This phenomenon further justifies the heterogeneous latent patterns of data distribution and serves as supportive evidence for disentangling LLM experts.

\subsubsection{\textbf{RQ3}. Effectiveness of the Experts Composition}

During the empirical study, we test different model merging methods that combine wisdom from different LoRA experts. We evaluate the well-known Ties, Dare, and the recently proposed TWIN merging methods. All of them yield a static composed model that takes in the strength of the candidate experts to be merged via solving parameter interference. From the results, we can see that merging over decomposed LoRA-experts can offer more robust problem solvers, outperforming the simple train-once-for-all mechanism. The experiments justify our major rationale that disentanglement-and-compose pipeline can offer more robust System2-to-System1 performance.

\subsubsection{\textbf{RQ4}. Superiority of DisenLoRA over other composition methods}

Although the static-composed expert model can promote robustness to some extent, its static nature lacks flexibility to different styles of inputs. As discussed in the previous contents, the data distribution of coding problems is complex, making the one-fits-all mechanism easy to fail. Therefore, we design DisenLoRA algorithm to yield a customized problem solver with input-awareness. From the results, we can see that  DisenLoRA outperforms the competing merging methods, validating the effectiveness of the proposed input-aware hypernetwork that dynamically aggregates the candidate composable LoRA experts at a rank-wise level.

\subsubsection{\textbf{RQ5}. Discussion of the Cross-Dataset Generalization of DisenLoRA}

Despite the flexibility offered by the input-aware hypernetwork, its performance may degrade on new datasets where the hypernetwork is not trained. To study this scenario, we use the model trained on APPS to generate solutions for CodeContest and use the model trained on CodeContest to generate solutions for APPS. The results are displayed in Table~\ref{tab:ood4code}.

\begin{table}[h]
    \centering
    \caption{Cross-dataset generalization test.}
    \label{tab:ood4code}
    \resizebox{0.45\textwidth}{!}{
    \begin{tabular}{c|cc|cc}
    \hline
        OOD Dataset & \multicolumn{2}{c|}{APPS}&\multicolumn{2}{c}{CodeContest}\\\hline
      Method   & PR & AC &PR & AC\\
      \hline
      w/o tuning   & 18.23 &	3.00 & 22.18 &	13.69 \\
      w/ SFT & 17.44 &	4.33 & 20.99	&14.29 \\
      \hline
      DisenLoRA & 18.25 &	4.33& 25.09&14.34\\ \hline
    \end{tabular}
    }
\end{table}

From the results, we can see that the proposed DisenLoRA has the generalization ability to the untrained dataset, outperforming the train-once-for-all mechanism still. This demonstrates that the parameters of the trained hypernetwork have the awareness of semantic similarities across datasets.


\section{Conclusion}
We identify the complexity of inherent reasoning exploration and the heterogeneous data distribution problems that hinder the performance of System2-to-System1 methods. Correspondingly, we propose the BDC pipeline that explores insightful System2 knowledge via mutually \textbf{B}oosting between llm agents, \textbf{D}isentangle heterogeneous data distribution for composable LoRA experts, and \textbf{C}ustomize problem solver for each instance, offering flexibility and robustness.
Correspondingly, we propose the MC-Tree-Of-Agents algorithm to efficiently and effectively explore the insightful System2 knowledge via mutual verification and boosting of different LLM agents, armed with reward-guided pruning and refinement to explore more beneficial states in limited rollouts for better performance. 
Additionally, we design an input-aware hypernetwork to aggregate over the disentangled composable LoRA experts trained on different clusters of data collected from MC-Tree-Of-Agents. This mechanism offers a customized problem solver for each data instance.
Various experiments and discussions validate the effectiveness of different model components.

\section*{Limitations}
%\whj{
While our work presents an efficient pipeline for transferring specialized knowledge from collective system-2-like LLMs to locally deployed language models through multiple LoRA adapters—enabling rapid, precise, system-1-like reasoning—three limitations merit discussion. First, despite code generation serving as an effective proxy for complex reasoning, our evaluation is restricted to this domain, leaving open questions about generalizability to broader textual reasoning tasks (e.g., commonsense reasoning and semantic parsing). Second, while we focus on their performance on the specific benchmarks, the safety alignment of derived models remains unaddressed. Systematic evaluation is required to assess whether our distilled experts preserve human values and mitigate harmful outputs. Finally, our ensemble methodology for LoRA experts, while input-aware, does not fully exploit potential sparsity optimizations in parameter activation, leaving room for computational efficiency improvements through advanced routing mechanisms.
%}

\section*{Acknowledgments}

This document has been adapted by Emily Allaway from the instructions for earlier ACL and NAACL proceedings, including those for NAACL 2024 by Steven Bethard, Ryan Cotterell and Rui Yan,
ACL 2019 by Douwe Kiela and Ivan Vuli\'{c},
NAACL 2019 by Stephanie Lukin and Alla Roskovskaya,
ACL 2018 by Shay Cohen, Kevin Gimpel, and Wei Lu,
NAACL 2018 by Margaret Mitchell and Stephanie Lukin,
Bib\TeX{} suggestions for (NA)ACL 2017/2018 from Jason Eisner,
ACL 2017 by Dan Gildea and Min-Yen Kan,
NAACL 2017 by Margaret Mitchell,
ACL 2012 by Maggie Li and Michael White,
ACL 2010 by Jing-Shin Chang and Philipp Koehn,
ACL 2008 by Johanna D. Moore, Simone Teufel, James Allan, and Sadaoki Furui,
ACL 2005 by Hwee Tou Ng and Kemal Oflazer,
ACL 2002 by Eugene Charniak and Dekang Lin,
and earlier ACL and EACL formats written by several people, including
John Chen, Henry S. Thompson and Donald Walker.
Additional elements were taken from the formatting instructions of the \emph{International Joint Conference on Artificial Intelligence} and the \emph{Conference on Computer Vision and Pattern Recognition}.


\bibliography{custom}

\appendix
%\section{Implementation Choices}
\label{sec:impl}

In this section, we briefly discuss the design choices made in our implementation of \lithe.

\vspace{-0.1in}
\subsection{\lithe Parameter Settings}
\label{sec:llm-params}

The \emph{``temperature''} parameter of \gpt, which ranges over [0,1], controls the randomness of the model's response.
While a higher temperature can be useful for creative writing where one would seek diverse and exploratory answers, in our case we want a focused and deterministic answer as far as possible. Hence we set the temperature to 0 which forces the model to greedily sample the next token.


The hyperparameters used by \lithe for MCTS are as follows: The maximum number of iterations $iter_{max}$ is set to 8,  expansion threshold $\theta$ is 0.7, and number of expansions $k$ is 2.
The values of $c_{base}$ and $c$ were set to 10 and 4, respectively.
%
These settings were determined after an empirical evaluation of the various trade-offs, providing a robust balance between efficiency and quality.
%

Finally, we try a maximum of 5 times to fix, via prompt corrections, any rewrite that exhibits syntax errors (Section~\ref{sec:lithe-architecture}).

\vspace{-0.1in}
\subsection{Query Equivalence Testing}
\label{sec:sql-equivalence}
We use a multi-stage approach, described below, to test semantic equivalence between the original query and a candidate rewrite.

\myparagraph{1. Logic-based Equivalence.}
Although verifying the equivalence of a general pair of SQL queries is NP-complete~\cite{queryequivalence}, a variety of logic-based tools (e.g. Cosette\cite{Cosette}, SQL-Solver~\cite{SQLSolver}, VeriEQL~\cite{verieql}, QED~\cite{QED}) are available for proving equivalence over restricted classes of queries, as mentioned in the Introduction. 
%
In \lithe, we use the recently proposed QED~\cite{QED} since it was found to cover a larger set of queries compared to the alternatives. 
%
The advantage of such a logic-based approach is that it is definitive in outcome and relatively inexpensive. 

\myparagraph{2. Result Equivalence via Sampling.}
%
If the original query is not within the QED scope, we alternatively use a sampling-based approach to test equivalence. The idea here is to execute the queries on several small samples of the database and verify equivalence based on the sample results. 
%
However, while this test is a necessary condition for query equivalence, it is not sufficient. That is,  false positives may be present because the sampled database may not cover all the predicates featured in the query. To minimize this likelihood, we use a combination of (1) \textit{correlated sampling}~\cite{cs2} for maintaining join integrity in the sample, (2) adding synthetic tuples in the sample to distinguish outer and inner joins, and (3) adjusting constants in the filter predicates to produce populated results -- the complete details are in the Section~\ref{app:sampling-eq}. 

\myparagraph{3. Result Equivalence on the Entire Database.}
%
Result equivalence could also be evaluated, in principle, on the entire database itself. Of course, this could turn out to be prohibitively expensive, especially if the queries themselves are time-consuming (e.g. due to the scale of the underlying database) and/or if the candidate rewrites happen to be regressions. Therefore, we leave this check as an optional choice for the DBA.


% \section{Methodology}

% \begin{figure*}
%     \centering
%     \includegraphics[width=1.0\linewidth]{figures/DebateThinker.pdf}
%     \caption{Illustration of the overall framework of \modelname.\dk{Todo: replace it.}}
%     \label{fig:copilot}
% \end{figure*}

% \subsection{Data collection}
% In this paper, we describe the MCTS procedure employed in our system, adapted to solve sequential decision-making problems under uncertainty, leveraging probabilistic state transitions and reward signals.

% \textbf{Select.} The select phase aims to traverse the decision tree by selecting nodes that balance exploration of unvisited nodes with exploitation of high-value nodes. Starting from the root node, the process iterates through the tree by applying a selection policy. In our implementation, the selection process is governed by the Upper Confidence Bound (UCB) strategy, with a variant of probability-weighted UCB (P-UCB) for better exploration.
% Each node is either a DecisionNode (representing agent decisions) or a ChanceNode (representing state-action transitions). When at a DecisionNode, the node selection is made based on the node's estimated value and visitation frequency, guided by the UCB formula:
% $$\mathrm{PUCB}(node)=Q(s,a)+c\cdot P(a|s)\cdot\frac{\sqrt{\log N(s)}}{1+N(s,a)}$$

% where $Q(s, a)$ represents the action value, $P(a|s)$ is the policy’s probability of selecting action $a$ in state $s$, and $N(s)$ is the number of visits to state $s$. The selection proceeds until a terminal node or an unvisited node is encountered.

% \textbf{Expand.} Expansion occurs when a previously unvisited state or a ChanceNode is reached during the selection phase. This step involves adding a new DecisionNode to the tree. In our implementation, when a ChanceNode transitions to a new state $s_t$ after an action $a$, the reward $r(s_t ,a)$ is computed. The reward evaluates the correctness of the state $s_t$ based on the output's pass rate, which is quantified by the evaluation function $R(s_t)$, as seen in the equation:
% $$R(s_t)=\mathrm{PassRate}(s_t)$$
% In our setup, the PassRate is determined by the proportion of correct outputs as evaluated by a code correctness checker. Additionally, future rewards are calculated if the node reaches a terminal state, incorporating the step rewards from the current node to future nodes through rollouts. This is expressed as:
% $$r(s_t,a)=R(s_t)+\gamma\cdot\sum_{i=1}^Tr(s_{t+i},a)$$
% Here, $\gamma$ represents the discount factor applied to future rewards during backpropagation, while the total reward is a combination of the current reward and discounted future rewards. The expansion step also adds new DecisionNodes to the tree, allowing subsequent actions to be explored.

% \textbf{Simulate (Rollout).} The Simulate phase, also known as Rollout, begins after a ChanceNode has expanded. In this phase, a sequence of actions is sampled from the current policy, and the rewards are accumulated until a terminal state is reached or the maximum rollout depth is achieved. Each simulation (rollout) evaluates a potential trajectory from the current state, allowing the model to estimate future rewards. The rollout process can be expressed as: 
% \begin{enumerate}
%   \item For each state $s_t$  and action $a$, the state transitions to $s_{t+1}$ according to the transition function $f(s_t, a)$, which is defined as: $$s_{t+1}=f(s_t,a)=s_t+a$$ where $s_t$ is the prompt already generated before the action and $a$ is the continuing prompt (thought).
%   \item If a terminal condition is met, i.e., if the state reaches the terminal token or exceeds the maximum length, a terminal reward $R(s_t)$ is calculated based on the correctness of the output. Otherwise, an intermediate reward of 0 is assigned: $$r(s_t,a)=\begin{cases}R(s_t)&\text{if terminal,}\\0&\text{otherwise.}\end{cases}$$
%   \item The overall reward for the rollout is the sum of rewards across all steps, discounted by $\gamma$ (the discount factor for future rewards): $$\mathrm{RolloutReward}(s_t)=\sum_{i=0}^T\gamma^i\cdot r(s_{t+i},a)$$ Here, $r(s_{t+i} ,a)$ refers to the reward at each step during the rollout, and $T$ is the depth of the simulation, i.e., the number of steps taken until the terminal state or maximum depth is reached.
% \end{enumerate}

% \textbf{Back Up.} Once a terminal state is reached during the simulation, the back up phase updates the values of nodes from the terminal node back to the root. This step ensures that the accumulated rewards and visit counts are propagated upwards through the tree, refining the decision-making process over time: $$Q(s_t,a)\leftarrow r(s_t,a)+\gamma V(s_{t+1})$$ $$V(s_t)\leftarrow\sum_aN(s_{t+1})Q(s_t,a)/\sum_aN(s_{t+1})$$ $$N(s_t)\leftarrow N(s_t)+1$$ where $\gamma$ is the discount factor.

% \subsection{Disentangled Lora Training}

% \subsection{Input-Aware Hypernetwork}

% \section{Experiment}
% \begin{itemize}[leftmargin=27pt]
%     \item [\textbf{RQ1}] Does the proposed \modelname offer performance gain against the base model?
%     \item [\textbf{RQ2}] Does the disentanglement help in training?
%     \item [\textbf{RQ3}] Can soft prompting enhance the capability of the backbone LLMs? Does finetuning with the soft prompting outperforms the simple supervised finetuning?
%     \item [\textbf{RQ4}] Are the proposed pretraining objectives for the GNN expert effective?
%     %Does the proposed \modelname model the high-level thought-of-codes? Can \modelname offer cross-lingual augmentation?
%     \item [\textbf{RQ5}] What is the impact of each of the components of the graphical view?
%     \item [\textbf{RQ6}] How is the compatibility of the graphical view? 
% \end{itemize}

% \subsection{Setup}
% In this paper, we evaluate \modelname with the competition-style APPS dataset and the CodeContest dataset. Both datasets are categorized by different difficulty levels. For each level of every dataset, we select about 100 problems for testing, except for CodeContest Hard category, which contains about 50 problems. Greedy decoding strategy is applied to the generation. The evaluation metric is PassRate(PR) and Accuracy(AC).

% For data collection, we evaluate the python code generation abilities of GPT4omini and Claude-3.5-sonnet with different methods: zeroshot, LDB\cite{zhong2024ldb}, RAP\cite{hao2023reasoning}, Reflexion, LATS\cite{zhou2023language}, ToT, MCTS and RethinkMCTS\cite{li2024rethinkmcts}. Based on multi-MCTS, we also evaluate both models with error position and refinement. For tuning, we evaluate different finetuning methods including SFT on different clusters, TIES, DARES, TWINS\cite{liu2023twins} and \modelname(ours).
% \begin{table*}[ht]
% \centering
% \caption{Main results on data collection.}
% \label{tab:code}
% \resizebox{\textwidth}{!}{
% \begin{tabular}{c|cccccc|cccc} 
% \hline
% & \multicolumn{6}{c|}{APPS} & \multicolumn{4}{c}{CodeContest}\\
%        Models                 & \multicolumn{2}{c}{Intro.}                            & \multicolumn{2}{c}{Inter.} & \multicolumn{2}{c|}{Comp.} & \multicolumn{2}{c}{Easy} & \multicolumn{2}{c}{Hard} \\ \cline{2-11} 
%                         & PR                     & AC                    & PR       & AC       & PR       & AC       & PR          & AC         & PR          & AC         \\ \hline
% ZeroShot                & 56.56 & 35.00 &      40.57           &     19.00          &   23.67              &     9.00          &29.03& 19.61 &   28.24                 &19.23              \\
% LDB                     & 60.64 & 40.00 &     46.78            &     22.00          &      21.00           &    8.00           & 34.76              & 25.58           & 33.52                   & 16.28                \\
% RAP                     & 64.24                         & 39.00                         &   43.32              &     14.00          &         22.83        &    8.00           & 43.08              & 33.33            &39.99  &26.92                 \\
% Reflexion               & 60.65                         & 40.00                         &  45.58               &     21.00          &      17.50           &     7.00          & 56.16              & 47.83           &34.09 &21.15                 \\
% LATS                    & 69.46                         & 50.00                         &  45.86               &     20.00          &       21.83          &        7.00       &57.70              & 47.83           &39.10                    &30.77                 \\
% ToT                     & 74.34                         & 55.00                         &  63.49               &     33.00          &         26.30        &       11.00        & 51.89              & 41.18           &49.07                    & 32.69                \\ 
% RethinkMCTS                     & 76.60                         & 59.00                        &  74.35               &     49.00         &         42.50        &       28.00        & 60.84             & 51.53          & 55.79                & 48.04               \\ \hline
% %MCTS-Line               &                               &                            &                 &               &                 &               &                    &                 &                    &                 \\
% %MCTS-Thought            &                               &                            &                 &               &                 &               &                    &                 &                    &                 \\ \hline
% Single (GPT4omini) & 77.99                         & 60.00                         &     72.89            &      50.00            &      44.17         &      25.00     & 55.79     &48.04                &      45.72              & 26.92                \\
% %Single-MCTS (Yi)        & 70.92                         & 53.75                      &                 &               &                 &               &                    &                 &                    &                 \\
% Single (Claude)    &  73.80  &  61.00   &  73.60   & 57.00 &      54.67           &      42.00         &                  58.75& 53.92                &68.41          &55.76             \\ \hline
% Multi-MCTS              & 79.72                         & 64.00                      &       79.42          &    63.00           &       59.17          &    45.00           & 62.49               &54.64             & 70.49                   &56.41                 \\ 
% + Error Position (v1)           & 83.07                         & 69.00                      &       78.65          &    64.00           &       57.67          &    41.00           & 65.92               &58.82             & 70.14                   &51.92                 \\
% + Refine   (v1)        & 83.24                         & 72.00                      &       79.65          &    63.00           &       55.83         &    38.00           & 62.73               &52.94             & 66.95                   &50.00                 \\
% + Error Position (v2)           & 85.18                         & 76.00                     & 81.95                & 67.00             &54.00              & 38.00            & 64.62            &  59.80         &73.12             &59.62                 \\
% + Refine   (v2)        &                          &                     &               &             & 60.33             & 44.00            &              &            & 73.80                   &63.46\\                 \hline
% \end{tabular}
% }
% \end{table*}

% % Please add the following required packages to your document preamble:
% % \usepackage[table,xcdraw]{xcolor}
% % Beamer presentation requires \usepackage{colortbl} instead of \usepackage[table,xcdraw]{xcolor}
% % Please add the following required packages to your document preamble:
% % \usepackage[table,xcdraw]{xcolor}
% % Beamer presentation requires \usepackage{colortbl} instead of \usepackage[table,xcdraw]{xcolor}
% \begin{table*}[ht]
% \centering
% \caption{Main results on \modelname tuning.}
% \label{tab:tuning}
% \resizebox{\textwidth}{!}{
% \begin{tabular}{ccccccccccccccc}
% \hline
% \multicolumn{15}{c}{\cellcolor[HTML]{EFEFEF}{Meta-llama-3.1-instruct-8b}}\\ \hline
% \multicolumn{1}{c|}{}                                         & \multicolumn{8}{c|}{IID Dataset (APPS Test)}                                                                                               & \multicolumn{6}{c}{OOD Dataset (CodeContest)}                                                     \\ \cline{2-15} 
% \multicolumn{1}{c|}{}                                         & \multicolumn{2}{c}{Intro.   } & \multicolumn{2}{c}{Inter.   } & \multicolumn{2}{c|}{Comp.   }  & \multicolumn{2}{c|}{Overall}      & \multicolumn{2}{c}{Easy   } & \multicolumn{2}{c|}{Hard   }     & \multicolumn{2}{c}{Overall} \\ \cline{2-15} 
% \multicolumn{1}{c|}{\multirow{-3}{*}{Finetune method}}        & PR              & AC             & PR              & AC             & PR    & \multicolumn{1}{c|}{AC}   & PR    & \multicolumn{1}{c|}{AC}   & PR             & AC            & PR    & \multicolumn{1}{c|}{AC}    & PR           & AC           \\ \hline
% \multicolumn{1}{c|}{w/o tuning}                               & 21.14           & 4.00           & 20.72           & 4.00           & 12.83 & \multicolumn{1}{c|}{1.00} & 18.23 & \multicolumn{1}{c|}{3.00} & 25.54          & 17.65         & 15.46 & \multicolumn{1}{c|}{5.77}  & 22.18        & 13.69        \\ \hline
% \multicolumn{1}{c|}{SFT on all}                               & 22.55           & 7.00              & 26.4            & 3.00              & 10.67 & \multicolumn{1}{c|}{1.00}    & 19.87 & \multicolumn{1}{c|}{3.67} & 22.42          & 14.71         & 18.12 & \multicolumn{1}{c|}{13.46} & 20.99        & 14.29        \\ \hline
% \multicolumn{1}{c|}{ SFT on cluster 0} & 20.67           & 6.00              & 24.23           & 3.00              & 11.5  & \multicolumn{1}{c|}{1.00}    & 18.80 & \multicolumn{1}{c|}{3.33} & 31.14          & 20.59         & 19.89 & \multicolumn{1}{c|}{11.54} & 27.39        & 17.57        \\
% \multicolumn{1}{c|}{ SFT on cluster 1} & 21.22           & 4.00              & 20.69           & 4.00              & 12.00    & \multicolumn{1}{c|}{2.00}    & 17.97 & \multicolumn{1}{c|}{3.33} & 29.4           & 21.57         & 17.39 & \multicolumn{1}{c|}{9.62}  & 25.40        & 17.59        \\
% \multicolumn{1}{c|}{SFT on cluster 2}                         & 16.65           & 7.00              & 23.97           & 3.00              & 17.33 & \multicolumn{1}{c|}{4.00}    & 19.32 & \multicolumn{1}{c|}{4.67} & 23.94          & 16.67         & 18.32 & \multicolumn{1}{c|}{7.69}  & 22.07        & 13.68        \\ \hline
% \multicolumn{1}{c|}{Ties}                                     & 22.75           & 4.00              & 23.06           & 4.00              & 12.67 & \multicolumn{1}{c|}{4.00}    & 19.49 & \multicolumn{1}{c|}{4.00} & 24.14          & 13.73         & 11.58 & \multicolumn{1}{c|}{5.77}  & 19.95        & 11.08        \\
% \multicolumn{1}{c|}{Dare}                                     & 24.97           & 7.00              & 26.66           & 5.00              & 12.5  & \multicolumn{1}{c|}{3.00}    & 21.38 & \multicolumn{1}{c|}{5.00} & 30.59          & 24.51         & 12.6  & \multicolumn{1}{c|}{7.69}  & 24.59        & 18.90        \\
% \multicolumn{1}{c|}{Twin}                                     & 19.1            & 5.00              & 23.85           & 5.00              & 8.5   & \multicolumn{1}{c|}{1.00}    & 17.15 & \multicolumn{1}{c|}{3.67} & 26.11          & 16.67         & 15.43 & \multicolumn{1}{c|}{3.85}  & 22.55        & 12.40        \\ \hline
% \multicolumn{1}{c|}{\modelname}                 & \textbf{27.11}           & \textbf{9.00}              & 23.11           & 3.00              & 11.5  & \multicolumn{1}{c|}{\textbf{4.00}}    & \textbf{20.57} & \multicolumn{1}{c|}{\textbf{5.33}} & 27.91          & 18.63         & 19.45 & \multicolumn{1}{c|}{5.77}  & 25.09        & 14.34        \\ \hline
% \end{tabular}
% }
% \end{table*}


% \section{Conclusion}


% \section*{Limitations}
% In this paper, we propose a graphical retrieval augmented generation method that can offer enhanced code generation. Despite the efficiency and effectiveness, there are also limitations within this work.  For example, dependency on the quality of the external knowledge base could be a potential concern. The quality of the external knowledge base could be improved with regular expression extraction on the noisy texts and codes. 


%\section*{Ethics Statement}

%\section*{Acknowledgements}




% \section*{Acknowledgments}

% This document has been adapted by Emily Allaway from the instructions for earlier ACL and NAACL proceedings, including those for NAACL 2024 by Steven Bethard, Ryan Cotterell and Rui Yan,
% ACL 2019 by Douwe Kiela and Ivan Vuli\'{c},
% NAACL 2019 by Stephanie Lukin and Alla Roskovskaya,
% ACL 2018 by Shay Cohen, Kevin Gimpel, and Wei Lu,
% NAACL 2018 by Margaret Mitchell and Stephanie Lukin,
% Bib\TeX{} suggestions for (NA)ACL 2017/2018 from Jason Eisner,
% ACL 2017 by Dan Gildea and Min-Yen Kan,
% NAACL 2017 by Margaret Mitchell,
% ACL 2012 by Maggie Li and Michael White,
% ACL 2010 by Jing-Shin Chang and Philipp Koehn,
% ACL 2008 by Johanna D. Moore, Simone Teufel, James Allan, and Sadaoki Furui,
% ACL 2005 by Hwee Tou Ng and Kemal Oflazer,
% ACL 2002 by Eugene Charniak and Dekang Lin,
% and earlier ACL and EACL formats written by several people, including
% John Chen, Henry S. Thompson and Donald Walker.
% Additional elements were taken from the formatting instructions of the \emph{International Joint Conference on Artificial Intelligence} and the \emph{Conference on Computer Vision and Pattern Recognition}.

% Bibliography entries for the entire Anthology, followed by custom entries
%\bibliography{anthology,custom}
% Custom bibliography entries only
% \bibliography{custom}

% \appendix
% \section{Implementation Details}
% \label{app:setup}
% For the size of retrieval pool, we use 11,913 C++ code snippets and 2,359 python code snippets. Due to the limited access, we do not use a large retrieval corpus for our experiment, which can be enlarged by other people for better performance. We also attach the graph extraction codes for both languages and all other expeirment codes here: https://anonymous.4open.science/r/Code-5970/

% For the fintuning details, the learning rate and weight decay for the expert GNN training is 0.001 and 1e-5, repectively. We apply 8-bit quantization and use LoRA for parameter-efficient fine-tuning. The rank of the low-rank matrices in LoRA is uniformly set to 8, alpha set to 16, and dropout is set to 0.05. The LoRA modules are uniformly applied to the Q and V parameter matrices of the attention modules in each layer of the LLM. All the three models are optimized using the AdamW optimizer. For the CodeContest dataset, totally 10609 datapoints are used, and for APPS dataset, 8691 data samples are used to train the model.


\end{document}
