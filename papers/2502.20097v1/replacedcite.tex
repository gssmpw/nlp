\section{Related works}
The task of estimating treatment effects becomes considerably more complex in the presence of interference. Hence, despite early influential works in causal inference such as____, only recently has a large body of literature emerged to tackle scenarios with interference. One of the most commonly studied settings is clustered network interference, where treatment units form independent clusters____, a condition also known as partial interference. Interference also naturally arises in network data, where some units are related to other units by being neighbors in e.g. a social network____. In some cases, an experimental study design can be constructed in a way to detect and reduce bias from interference, for instance, through a two-stage randomization design____ or by stratified randomization across different blocks____. Previous works have covered specific tasks under interference such as heterogeneous treatment effect estimation____ or policy evaluation/learning____. To our knowledge, there is no prior work on the problem of estimating Qini curves in the presence of interference.

Evaluating treatment prioritization rules using Qini curves in settings without interference has gained more attention in recent years____. The development of estimations procedures with better statistical inference guarantees has enabled the use of Qini curves in this context____. While none of these works consider interference, ____ considers the related problem of estimating Qini curves for combinatorial multi-armed treatments. There is an inherent connection between combinatorial treatment problems and clustered network interference, as the treatment assignment of units within a single cluster can be seen as a combinatorial treatment decision. This connection also underscores the challenge of estimating Qini curves under clustered network interference: as cluster size grows the combinatorial space of possible treatments expands exponentially, leading to a corresponding increase in interactions among units within the cluster. To address this challenge, we propose estimation strategies designed to more accurately estimate Qini curves, even as the cluster size grows.