% \clearpage
\section{Experiment Settings}
\label{appendix-parameters-settings-experiments}

We deploy several SOTA baselines in our benchmark to evaluate their robustness across various challenging scenarios. The implementation parameters associated with these methods are provided in Tables \ref{appendix-paremetrs:PPO-MAPPO}-\ref{appendix-para:PCRPO-CRPO}.

Since RL performance can be significantly influenced by different random seeds \citep{henderson2018deep, colas2018many}, we aim to balance computational costs and experimental rigor by typically using 3–5 seeds in our experiments. For single-agent settings, we use the same 3 seeds across all baselines to ensure a fair comparison. In multi-agent settings, where variance tends to be higher, we employ the same 5 seeds across all baselines to achieve a more reliable evaluation. We recognize the importance of robust experimental evaluation and intend to include additional seeds in future studies to further examine RL robustness.

Moreover, when selecting different robust disturbance parameters, the choice can significantly affect the evaluation of various RL algorithms. For instance, in standard RL, disturbances can be modeled as Gaussian distributions, such as $\mathcal{N}(0, 0.1)$ or $\mathcal{N}(0, 0.15)$, applied to the state or action space, which can notably influence the performance of algorithms like PPO. Alternatively, uniform disturbances within the range $[0.2, 0.8]$ can be used to effectively assess the robustness of standard RL approaches. For robust RL, additional parameters are often employed to evaluate algorithm robustness. For example, as for the evaluation robustness of MOPO method, wind speed may follow a uniform distribution $U(0.8, 1.2)$, while robot gravity may vary uniformly within $U(9.81, 19.82)$. Other factors include variations in the robot's physical dimensions, such as the torso length, which can be expressed as the original length plus $0.1\sin(0.2 \cdot \text{iteration number})$, and the foot length, which follows a similar perturbation. Our benchmark also incorporates robust parameters to evaluate the safety of RL algorithms. For example, Gaussian disturbances $\mathcal{N}(0, 0.3)$ are particularly effective for assessing the robustness of safe RL algorithms such as PCRPO and CRPO. In the context of multi-agent RL, robustness can be evaluated by selectively perturbing partial agents. Gaussian disturbances, such as $\mathcal{N}(0, 0.1)$ or $\mathcal{N}(0, 0.15)$, applied to the state or action space, can provide significant insights into the robustness of algorithms like MAPPO and IPPO.

\begin{table}[htbp]
\centering
\begin{tabular}{cccc}
\hline
\textbf{Parameters}      & \textbf{Value} & \textbf{Parameters}       & \textbf{Value} \\ \hline
buffer size                   & 4096          & hidden size                       & [64, 64]           \\ 
lr                 & 3e-4           & gamma                   & 0.99           \\ 
epoch                  & 100          & steps per epoch               & 30000 \\ 
steps per collect                    & 2048            & repeat per collect             & 10           \\ 
batch size                   & 64            & training num           & 8            \\ 
testing num        & 10             & rew norm              & True            \\ 
vf coef            & 0.25           & ent coef     & 0.0  
  \\ 
gae lambda        & 0.95             & bound action clip             & clip            \\ 
lr decay       & True             & max grad norm          & 0.5            \\ 
eps clip        & 0.2             &  dual clip             & None    
  \\ 
value clip        & 0             & norm adv              & 0            
  \\ 
recompute adv       & 0             &                &             
      
\\
\hline
\end{tabular}
\caption{Parameter values used for PPO \citep{schulman2017proximal}, MAPPO \citep{yu2022surprising} and IPPO \citep{de2020independent} in  experiments.}
\label{appendix-paremetrs:PPO-MAPPO}
\end{table}


\begin{table}[htbp]
\centering
\begin{tabular}{cccc}
\hline
\textbf{Parameters}      & \textbf{Value} & \textbf{Parameters}       & \textbf{Value} \\ \hline
buffer size                   & 4096          & hidden size                       & [64, 64]           \\ 
actor lr                 & 1e-3 & critic lr & 1e-3 \\           gamma                   & 0.99   & tau & 0.005        \\ 
alpha                   & 0.0.2   & auto alpha & False        \\ 
epoch                  & 100          & steps per epoch               & 30000 \\ 
steps per collect                    & 2048            & update per step             & 1           \\
start time step & 10000 & n step & 1 \\
batch size                   & 64            & training num           & 8            \\ 
testing num        & 10            &  &
\\
\hline
\end{tabular}
\caption{Parameter values used for SAC \citep{haarnoja2018soft} in the experiment.}
\end{table}

\begin{table}[htbp]
\centering
\begin{tabular}{cccc}
\hline
\textbf{Parameters}      & \textbf{Value} & \textbf{Parameters}       & \textbf{Value} \\ \hline
start steps                   & 5000          & num steps                       & 300000          \\ 
eval                 & True           & eval episode                   & 10           \\ 
eval times                  & 10          & local reply size               & 1000 \\ 
gamma                   & 0.99            & tau            & 0.005           \\ 
lr                   & 3e-4            & alpha           & 0.2            \\ 
batch size        & 256             & update per step             & 3            \\ 
target update interval            & 2           & hidden size     & 256  
  \\ 
gail batch        & 256             & exponent             & 1.5            \\ 
tomac alpha      & 1e-3             & reward max          & 1            \\ 
\\
\hline
\end{tabular}
\caption{Parameter values used for OMPO \citep{luoompo2024} in non-stationary MuJoCo experiments.}
\end{table}

\begin{table}[htbp]
\centering
\begin{tabular}{cccc}
\hline
\textbf{Parameters}      & \textbf{Value} & \textbf{Parameters}       & \textbf{Value} \\ \hline
image obs                   & False          & actor lr                       & 3e-4          \\ 
critic lr                 & 1e-3           & gamma                   & 0.99           \\ 
tau                  & 5e-3          & alpha               & 0.1 \\ 
auto alpha                   & True            & alpha lr             & 3e-4           \\ 
hidden size                   & [256, 256, 256]            & n steps           & 4            \\ 
buffer size        & 1e6             & step per epoch              & 1e4            \\ 
step per collect            & 20           & batch size     & 128  
  \\ 
start time step        & 0             & exploration noise             & 0            \\ 
horizon        & 300             & camera          & agentview           \\ 
height        & 128             &  width            & 128    
  \\ 
encoder type        & mlp & training num             & 10            
  \\ 
test num       & 10             &       sigma         & 0.01\\
bound & 0.01 & augmented ratio & 0.5 \\
vae sigma & 1.0 & control frequency & 20
      
\\
\hline
\end{tabular}
\caption{Parameter values used for RSC \citep{ding2024seeing} in the causaldoor/causallift experiments; for DBC \citep{zhang2021learning}, based on above parameters, transition model type is probabilistic, encoder feature dim is 256, encoder lr is 1e-4, decoder lr is 1e-4, bisim coef is 0.5, log std min is -10, log std max is 2; for ATLA \citep{zhang2021robust}, policy update max is 100, adv update max is 100, and adv eps is 0.01.}
\end{table}


% \citep{ding2024seeing}, ATLA \citep{zhang2021robust}, and DBC \citep{zhang2021learning}



\begin{table}[htbp]
\centering
\begin{tabular}{cccc}
\hline
\textbf{Parameters}      & \textbf{Value} & \textbf{Parameters}       & \textbf{Value} \\ \hline
gamma                  & 0.995          & hidden layer dim                       & 64           \\ 
cost limit                 & 0.04          & slack bound                   & 5e-3           \\ 
exploration iteration                  & 40          & epoch               & 500 \\ 
tau                    & 0.97            & l2 reg             & 1e-3           \\ 
 max  kl                & 1e-2            & damping           & 1e-1            \\ 
batch size        & 150000             & gradient wr              & 0.4            \\ 
     gradient wc       & 0.6   & &
\\
\hline
\end{tabular}
\caption{Parameter values used for PCRPO \citep{gu2024balance} and CRPO \citep{xu2021crpo} in the safety experiments.}
\label{appendix-para:PCRPO-CRPO}
\end{table}



% \begin{table}[htbp]
% \centering
% \begin{tabular}{cccc}
% \hline
% \textbf{Parameters}      & \textbf{Value} & \textbf{Parameters}       & \textbf{Value} \\ \hline
% buffer size                   & 4096          & hidden size                       & [64, 64]           \\ 
% lr                 & 3e-4           & gamma                   & 0.99           \\ 
% epoch                  & 100          & steps per epoch               & 30000 \\ 
% steps per collect                    & 2048            & repeat per collect             & 10           \\ 
% batch size                   & 64            & training num           & 8            \\ 
% testing num        & 10             & rew norm              & True            \\ 
% vf coef            & 0.25           & ent coef     & 0.0  
%   \\ 
% gae lambda        & 0.95             & bound action clip             & clip            \\ 
% lr decay       & True             & max grad norm          & 0.5            \\ 
% eps clip        & 0.2             &  dual clip             & None    
%   \\ 
% value clip        & 0             & norm adv              & 0            
%   \\ 
% recompute adv       & 0             &                &             
      
% \\
% \hline
% \end{tabular}
% \caption{Parameter values used for MAPPO in the experiment.}
% \end{table}

% \begin{table}[htbp]
% \centering
% \begin{tabular}{cccc}
% \hline
% \textbf{Parameters}      & \textbf{Value} & \textbf{Parameters}       & \textbf{Value} \\ \hline
% buffer size                   & 4096          & hidden size                       & [64, 64]           \\ 
% lr                 & 3e-4           & gamma                   & 0.99           \\ 
% epoch                  & 100          & steps per epoch               & 30000 \\ 
% steps per collect                    & 2048            & repeat per collect             & 10           \\ 
% batch size                   & 64            & training num           & 8            \\ 
% testing num        & 10             & rew norm              & True            \\ 
% vf coef            & 0.25           & ent coef     & 0.0  
%   \\ 
% gae lambda        & 0.95             & bound action clip             & clip            \\ 
% lr decay       & True             & max grad norm          & 0.5            \\ 
% eps clip        & 0.2             &  dual clip             & None    
%   \\ 
% value clip        & 0             & norm adv              & 0            
%   \\ 
% recompute adv       & 0             &                &             
      
% \\
% \hline
% \end{tabular}
% \caption{Parameter values used for IPPO in the experiment.}
% \end{table}