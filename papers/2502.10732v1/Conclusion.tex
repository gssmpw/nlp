\section{Conclusion}
\label{sec:conclusion}


The \rbrl framework takes an important step towards addressing the critical challenge of balancing decision efficacy and interpretability in constrained allocation tasks. By synergizing the generative capabilities of LLMs with the optimization power of RL, \rbrl introduces a novel paradigm where structured rules, generated dynamically by LLMs, guide RL policies to make transparent and actionable decisions. Evaluations across healthcare and public health domains demonstrate \rbrl’s competitive performance against traditional RL baselines while significantly enhancing the quality of human-understandable explanations. Human and LLM-judged surveys further validate that RBRL’s explanations improve trust and clarity, contributing towards addressing the longstanding barrier to deploying AI in high-stakes scenarios.



\section*{Impact and Ethics Statement}
This work advances the development of transparent AI systems for high-stakes decision-making in domains like healthcare and public policy. By enabling RL agents to generate human-readable rules and explanations, \rbrl improves trust and accountability, critical for ethical deployment in settings where lives and resources are at stake.

While the framework prioritizes alignment with human reasoning, potential risks include over-reliance on imperfect LLM-generated rules or explanations that may inadvertently obscure biases in training data. Mitigation requires rigorous validation of rules by domain experts and ongoing monitoring of LLM outputs. Additionally, \rbrl’s reliance on LLMs raises computational and accessibility challenges in resource-constrained environments. By addressing these considerations, this research contributes to safer, more equitable AI systems that empower—rather than replace—human decision-makers.

Notice that the Uganda and Heat Alerts datasets used in this study is derived from a simulator provided by \citet{boehmer2024optimizing} and \citet{considine2023optimizing}. These simulators do not include any feature information or identifying details of real patients. Thus, the data generated by the simulator cannot be traced to or represent actual individuals, ensuring privacy and ethical compliance. We emphasize that this is purely a simulated study; and recognize that for any next steps towards real world use, there is a need to conduct rigorous simulation studies on a large scale with real patient data, with detailed assessments of potential biases, verification of policy convergence and its robustness to distribution shifts in patient populations, and making necessary adjustments. Beyond that, there will be a need to obtain ethics and regulatory approval to test the policy in a real-world setting for future comprehensive field-testing, addressing issues of participant consent and privacy; and ultimately there would need to be sufficient human oversight for any future deployment.