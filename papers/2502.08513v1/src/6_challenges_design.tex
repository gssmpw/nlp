\section{Creation Challenges}
\label{sec:challenges}

We identify seven challenges including a total of seventeen issues (\textbf{RQ2}), as listed in Table~\ref{tab:challenges}. Among them, eight issues echo the findings in previous empirical studies on general XR applications~\cite{krauss2021current, ashtari2020creating, krauss2022elements, liu2023challenges}. Therefore, we will only elaborate on the nine newly identified issues that exhibit certain uniqueness arising from the integration of story elements with VR and computer animation.


\begin{table*}
\setlength{\aboverulesep}{0.5pt}
    \setlength{\belowrulesep}{0.5pt}
\setlength{\tabcolsep}{6pt}
\definecolor{lightbluecolor}{HTML}{EDF5F7}

\caption{An overview of challenges in creating animated VR stories by creators, according to our interview studies. Items with a \colorbox{lightbluecolor}{colored background} are newly identified issues.}
\label{tab:challenges}
\Description{Overview of seven challenges.}
\begin{tabular}{@{}lp{5.1cm}m{7.6cm}p{3.2cm}@{}}
\toprule
{\#} & \multicolumn{1}{c}{\textbf{Challenge}}                                             & \multicolumn{1}{c}{\textbf{Issue}}                                                                                                                            & \multicolumn{1}{c@{}}{\textbf{Stage}}                                                     \\
\midrule
                &                           &  \cellcolor[HTML]{edf5f7}{C1-1: Lack of guidelines about balancing creators' narrative intent with viewers' autonomy}                                                                                            &  \\ %\hhline{~~-~} 
                                 &                                                                                      & \cellcolor[HTML]{edf5f7}{C1-2: Hard to understand VR story experiences and guidelines outside a VR environment}  &                                                                                           \\ %\cline{3-3}
                      \multirow{-4}{*}{\textbf{C1}}      &    \multirow{-4}{*}{ \parbox{5.1cm}{Insufficient VR storytelling guidelines}}                                                                              &  {C1-3: Limited guidelines for creators with different backgrounds to avoid applying incompatible knowledge (e.g., game design, 2D audio-visual language) }                                                                                    & \multirow{-5}{*}{\parbox{3.2cm}{Idea Generation, Story Creation, Scriptwriting, Storyboarding, Previs, Design}}                                                                                        \\ \hline
                    
             &               & \cellcolor[HTML]{edf5f7}{C2-1: Hard to describe and access multi-element plots based on story elements and desired outcomes}                                                                                     &                                     \\ %\hhline{~~-~}
                            \multirow{-3}{*}{\textbf{C2}}    &    \multirow{-3}{*}{\parbox{4.5cm}{Hard to find references for multi-element VR plots}}                                                                                     & {C2-2: Limited high-quality VR plots that align with creators' creative needs}                                                               &     \multirow{-2.5}{*}{\parbox{3.2cm}{Storyboarding, Previs, Design}}                                                                                        \\ \hline
                                       &       &  \cellcolor[HTML]{edf5f7}{C3-1: Difficult to plan and manage visual details under VR’s real-time performance constraints}                                                                                                   &                                             \\ %\hhline{~~-~}
                            \multirow{-3}{*}{\textbf{C3}}      &    \multirow{-3}{*}{\parbox{4.5cm}{Struggle to achieve satisfactory VR visuals for artistic expression}}                                                                                    &  \cellcolor[HTML]{edf5f7}{C3-2: Hard to achieve professional-grade visual quality with immersive tools alone}     &            \multirow{-3}{*}{Asset Development}                                                                                     \\ \hline
         &              & {C4-1: Hard to realize mismatches between creation mindsets and philosophies of DCCs and immersive tools}                                            &                                                   \\ %\hhline{~~-~}
                             \multirow{-3}{*}{\textbf{C4}}   &   \multirow{-3}{*}{\parbox{4.5cm}{Difficult to fluidly use non-immersive and immersive software}}                                                                               &  {C4-2: Inconvenient user interfaces and input interactions in DCCs and immersive tools}                                                                                    &     \multirow{-3}{*}{Asset Development}                                                                                  \\ \hline

             &    & \cellcolor[HTML]{edf5f7}{C5-1: Hard to plan and coordinate various story elements spatially and temporally}                                                                     &                               \\ %\hhline{~~-~}
                                &                                                                                     & \cellcolor[HTML]{edf5f7}{C5-2: Hard to switch between multiple narrative perspectives without confusing viewers about their roles}                                                                                     &                                                                                         \\ %\hhline{~~-~}
                                \multirow{-4.5}{*}{\textbf{C5}} &   \multirow{-4.5}{*}{\parbox{4.5cm}{Missing integrated building blocks for core VR story experiences}}                                                                               &  {C5-3: Hard to implement semantically-rich interactions related to story content}                                                                                     &     \multirow{-4.5}{*}{\parbox{2.5cm}{Scene Assembly, Story Integration}}                                                                                          \\ \hline
            &      & \cellcolor[HTML]{edf5f7}{C6-1: Uncertainty in the relationships between various design parameters,  emotions, and viewer comfort}                                 &                                   \\ %\hhline{~~-~}
                         \multirow{-3}{*}{\textbf{C6}}        &      \multirow{-3}{*}{\parbox{4.5cm}{Tedious parameter adjustment for optimal audience experience}}                                                                               &  {C6-2: Inefficient transition between a desktop and a VR environment to adjust various design factors based on firsthand VR experience}                                                         &    \multirow{-3}{*}{\parbox{2.5cm}{Scene Assembly, Story Integration}}        \\ \hline
                                      &    &  {C7-1: Unaware of key aspects during assessment}                                                                 &                               \\ %\hhline{~~-~}
                                &                                                                                     &  {C7-2: Hard to deal with the complex nature of individual audience experiences}                                                                                     &                                                                                         \\ %\hhline{~~-~}
                                \multirow{-4}{*}{\textbf{C7}} &   \multirow{-4}{*}{\parbox{4.5cm}{Lack data collection and analysis methods for evaluation}}                                                                              &  \cellcolor[HTML]{edf5f7}{C7-3: Insufficient support for collecting and analyzing viewers' watching behavior data}                                                                                      &     \multirow{-4.5}{*}{\parbox{2.5cm}{Evaluation}}                                                                                          \\
                                \bottomrule
\end{tabular}
\end{table*}

\subsection{C1: Insufficient VR Storytelling Guidelines}
\label{sec:challenge_guideline}

During the pre-production phase, our interviewees often found that existing storytelling guidelines were inadequate for addressing unique VR considerations and remained hard to understand.

\textbf{C1-1: Lack of guidelines about balancing creators' narrative intent with viewers' autonomy.}
Our interviewees were aware that viewers in VR have the autonomy to choose what to focus on and for how long. While some embraced this autonomy and told their stories indirectly (Sec.~\ref{sec:story_creation}), most interviewees experienced frustration as it often interfered with their narrative intent.
The first frustration is out-of-order exploration, where viewers encounter plot elements in a random order, disrupting planned revelations and suspense.
The second frustration is missing pivotal moments, where viewers focus on unintended aspects of the experience, weakening the intended emotional impact.
The third frustration is the disruption of narrative pacing, where interactive elements absorb viewers’ attention, causing temporal disconnects from the broader storyline.
Despite these frustrations, our interviewees were reluctant to excessively control or restrict viewers' autonomy. Approaches like limiting interactive elements or constraining movement and even gaze will undermine VR's unique advantages. However, they could not find relevant guidelines:

\q{In my story, you'll be Alice adventuring in Wonderland! You can explore freely, establish connections with the characters, and feel their emotions. However, I also have a predetermined storyline. Then, numerous questions emerge. How can I accommodate viewers' curiosity and self-exploration without risking narrative distraction? Can this curiosity be strategically used to enhance narrative engagement? How should I advance the storyline to maintain good narrative pacing, especially when interactive elements are more attractive?} (P13)

\textbf{C1-2: Hard to understand VR story experiences and guidelines outside a VR environment.}
Our interviewees indicated that the prevailing formats (e.g., text, images, or videos) of VR storytelling guidelines lack intuitiveness. This difficulty applies not only to the crafting of overall VR story experiences but also to specific design aspects. Managing visual hierarchy and weight within a boundless 360-degree canvas is one such design aspect. For example, P19, who learned about using 3D perspective lines and vanishing points from an online 2D video, found it difficult to apply these concepts in VR to establish a consistent focal point from different angles. 
Additionally, newcomers who had not yet fully understood the viewers' experience in VR stories also reported difficulties in understanding certain pieces of advice.

\q{... I questioned my teacher's advice to mirror real life in my designs. Later, I realized that in immersive VR, too many unfamiliar elements could disrupt narrative engagement more than on 2D screens... floating without gravity once took me out of the story, making me wonder why it happened.} (P5)


\begin{figure*}[!tb]
    % \centering
    \includegraphics[width=\linewidth]{figures/updated/reference.png}
    % \vspace{-8mm}
    \caption{Examples of different types of references. (A) Online images for the visual design of a character (\CR{\copyright Angela Cai}). (B) Self-taken photos to inform the design of a scene (\CR{\copyright AMAO}). (C) A multi-element plot from an animated VR story that showcases how to use spatial layering for aesthetic scenes and use a moving vehicle as symbolism tied to the story's theme (\CR{\copyright Xiaoka}).} 
    \label{fig:design_references}
    \Description{This figure provides examples of different types of references. Part A illustrates "Online Images," where various animal pictures (e.g., lion, whale) are combined to inspire a creature design. Part B shows "Self-taken Photos," where a real-world alleyway inspires a stylized 3D environment. Part C features "Animated VR Stories," showing the inspiration from a well-known VR story.}
\end{figure*}

\begin{figure*}[!tb]
    \centering
    \includegraphics[width=\linewidth]{figures/updated/tricks.png}
    \caption{Examples of scenes that suffer from performance failure and relevant remedies. (A) A high-fidelity scene exceeding memory quota (\CR{\copyright Luna Han}). (B) Employing low-poly models to create scenes (\CR{\copyright Bigz}). (C) Listing assets to animate in advance to optimize capacity distribution (\CR{\copyright Bigz}). (D) Changing the level of details based on visibility (\CR{\copyright AMAO}).} 
    \label{fig:capacity_fidelity}
    \Description{This figure illustrates a scene suffering from performance failure and potential remedies. Part A shows an overloaded cityscape. Part B suggests using low-poly brushes to paint trees. Part C reduces complexity by spotlighting specific areas, such as a fox and grasses. Part D demonstrates controlling the details of different parts of a house based on visibility.}
\end{figure*}

\subsection{C2: Hard to Find References for Multi-Element VR Plots}
\label{sec:reference_plots}

Our interviewees underscored the critical role of references (Fig.~\ref{fig:design_references}) during the storyboarding, previs, and design stages. They particularly needed segments of animated VR stories to steer their design decisions to effectively interweave multiple visual, auditory, and interactive elements into engaging plots. For brevity, we refer to such segments as multi-element plots. 
As shown in Fig.~\ref{fig:design_references}-C, an interviewee took cues from the opening of \textit{Baba Yaga}~\cite{Babayaga}, imitating its spatial layering and lighting to craft aesthetic scenes, and mimicking its use of a vehicle as symbolism tied to her story's theme. However, finding multi-element plot examples is challenging.

\textbf{C2-1: Hard to describe and access multi-element plots based on story elements and desired outcomes.}
Our interviewees relied on describing their target plots to more experienced individuals to receive recommended references. They often began by describing their desired outcomes, including specific narrative effects, emotional impacts, and sensory experiences. However, they were not satisfied with simple and vague descriptions like \q{a symbolic plot evoking deep thoughts} for narrative effects (P14), \q{a plot giving viewers a sense of reverence} for emotional impacts (P10), or \q{a plot providing a cold atmosphere and physical sensation} for sensory experiences (P2). They wished to obtain more accurate recommendations by specifying preferences, such as viewers' position and viewpoint, the composition of multiple elements, and the integration of interactive elements like user choices. Articulating this combination of multiple elements and the resultant outcomes proved more difficult in animated VR stories than in films and animations. This complexity arose because, in VR, elements unfold not only visually and audibly but also spatially and interactively:

\q{If I ask for a plot where a crowd running around, I want to specify my preferences for camera movement and how the viewers engage within the story, such as their spatial relationship with the surroundings. Otherwise, my friend’s recommendations may not align with my expected outcome, but it is hard to express these specifics.} (P6)