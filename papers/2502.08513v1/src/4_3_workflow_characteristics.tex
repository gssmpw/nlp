\subsection{Characteristics of Workflows}

\subsubsection{Various Tools and Toolchains}
\label{sec:process_toolchains}
We found that three categories of tools and software were employed during the production phase. The first category was digital content creation tools (DCCs, e.g., Maya, Blender) and digital audio workstations (DAWs, e.g., Pro Tools, Adobe Audition), which were used to develop individual visual and auditory assets. The second category was game engines (e.g., Unreal Engine and Unity), which helped to integrate DCC and DAW outputs into cohesive scenes, implement interactivity with visual scripting, and add visual effects or physics with game engines' built-in functions. 
The third category was immersive tools (e.g., Quill, Gravity Sketch, Open Brush) that offered 360-degree authoring spaces and enabled direct 3D content manipulation in either specific tasks (e.g., modeling and animating) or the comprehensive crafting of their entire stories. 

Our interviewees adaptively used these tools based on the stories, the tools' affordance, and their skillsets. Common practices include the combination of DCCs and DAWs with or without game engines (14/21), exclusive use of immersive tools (5/21), or a blend of immersive tools and DCCs or game engines (2/21). 

Interestingly, P12 and his team innovatively used VRChat~\footnote{\url{https://hello.vrchat.com/}} to simulate the live-action film shooting process for their animated VR story. They viewed VRChat as a tool, rather than a social VR platform, and incorporated it into their workflow. Figure~\ref{fig:vrchat} shows their use of VRChat during the production phase. They developed characters as VRChat avatars and assembled scenes in Unity before uploading them to VRChat. During story integration, the team entered their scene in VRChat, utilizing its head, hand and full-body tracking capabilities to control avatars and act within virtual settings. 
This process mimicked live-action film shooting. Some members controlled different characters, akin to actors performing. One member shot with the virtual camera as a cinematographer, while another acted as a director to coordinate the characters and the cinematographer.





\subsubsection{Iterative Cycles}
\label{sec:processes_iterative_cycles}
We identified two iterative cycles in the creation workflows. The first cycle happened between the story creation, scriptwriting, and the production phase. During production, our interviewees often found that some planned plots and settings were impractical within time constraints or totally infeasible.
This necessitated revisiting and adjusting previous storylines, which risked introducing inconsistencies into the final narrative and caused additional workload. However, they thought this cycle was unavoidable because they were unable to assess the feasibility of each plot during the early story creation or scriptwriting stage. 
Thus, many interviewees progressively discovered the constraints of their selected tools, accepted extra workload, and compromised their stories to the tools and their abilities.

The second iterative cycle manifested between the production phase and the evaluation stage. Interviewees who conducted the evaluation would typically revisit and modify the spatial layout, pacing, and cameras to address complaints about user comfort. While they (e.g., P2, P10) also received comments like \q{hard to follow the story}, they seldom revised the storylines or associated assets. Such revisions were perceived as time-consuming, so they chose to take these comments as lessons learned for their upcoming VR story projects.
