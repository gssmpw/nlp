\subsection{Story-Driven and Visual-Driven Workflows}
\label{sec:diverse_workflow}

Our interviewees embraced diverse workflows that connected the above ten stages, although some might omit some stages. Only \RR{nine, ten, eleven, and sixteen} interviewees were involved in scriptwriting, storyboarding, previs, and evaluation. The sequence in which these stages were carried out also varied. For instance, while some prioritized design before storyboarding and previs, others followed the reverse order. Two common iterative cycles emerged in their workflows. The first cycle occurred between story creation, scriptwriting, and production, while the second cycle appeared between production and evaluation.
Despite the diversity in these workflows, they can be classified as story-driven or visual-driven. Figure~\ref{fig:diverse_workflow} provides representative examples for each type. 

Most interviewees (16/21) adopted story-driven workflows (Fig.~\ref{fig:diverse_workflow}-A), usually in an order of the pre-production phase, production phase, and evaluation stage. They prioritized narrative coherence, ensuring that the visuals were designed to support and enhance the story. When certain visual elements proved difficult to implement, they often adjusted their designs to more practical or achievable solutions. For example, P3 changed his design from dreamlike to realistic to avoid additional workload.

In contrast, in visual-driven workflows (Fig.~\ref{fig:diverse_workflow}-B), interviewees (5/21; P1, P11, P14, P15, P16) exhibited a pronounced focus on visuals. They invested significant effort in exploration and experimentation. For example, P14 and P15 delved into rendering and shaders for stylized visuals, whereas P18 spent three months translating her 2D illustration visual style into VR using Quill. In visual-driven workflows, the stages in the pre-production and production phases were carried out alternately. Specifically, story creation followed the design and asset development stage, with the story often shaped by available assets, resulting in relatively simpler storylines.
