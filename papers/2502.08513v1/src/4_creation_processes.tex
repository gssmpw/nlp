\section{Creation Process}
\label{sec:processes}

This section reports findings on the creation processes of animated VR stories (\textbf{RQ1}), including individual stages and overall workflows.

\begin{figure*}[!tb]
    \centering
    \includegraphics[width=\linewidth]{figures/updated/storyboard.png}
    \caption{Various practices and purposes in storyboarding. (A) A high-fidelity storyboard for post-hoc refinement (\CR{\copyright AMAO}). The creator inspected the compact overview of three world settings (A1-A3) and directly inserted placeholders or added sketches on areas to enhance. (B) A storyboard with interconnected nodes detailing the entire storyline for team communication (\CR{\copyright Hedi's team}). (C) A storyboard in two forms by a solo creator that selectively captures some moments for self-evaluation (\CR{\copyright Ocean Hu}).} 
    \label{fig:storyboarding}
    \Description{This figure showcases different practices in storyboarding. Part A presents a colorful storyboard sequence progressing from everyday life to dreamlike and futuristic scenarios. Part B illustrates a mind-map style diagram connecting various narrative elements and scenes with sketches. Part C contains rough black-and-white storyboard panels, showing a character's journey through outdoor landscapes using minimal color and shading for depth.}
\end{figure*}

\begin{figure*}[!tb]
    \centering
    \includegraphics[width=\linewidth]{figures/updated/previs.png}
    \caption{Examples of previs to assess scenes and the functionality of interactions before developing high-fidelity models. (A) Low-fidelity models in the previs stage and high-fidelity scenes in the ultimate story (\CR{\copyright Hedi's team}). (B) Interactive elements that (B1) started as sketches, (B2) evolved into low-fidelity prototypes, and (B3) were incorporated into the final high-fidelity story (\CR{B1-B3 \copyright Coin's team}).} 
    \label{fig:previsualization}
    \Description{This figure showcases examples from the previsualization stage. Part A compares low-fidelity and high-fidelity scenes. Part B focuses on object interaction, starting with a low-fi sketch (B1), followed by low-fi 3D models (B2), and concluding with high-fi 3D interactive models (B3), illustrating the evolution from rough concepts to fully interactive, realistic environments.}
\end{figure*}


\subsection{Common Stages}
\label{sec:common_stages}

Our interviewees went through 10 common stages to transform their ideas into an animated VR story. 
Figure~\ref{fig:workflow_example_1} illustrates a concrete example. 
For brevity, we use the pre-production phase and production phase when discussing multiple stages later. Specifically, the pre-production phase refers to the six stages (Sec.~\ref{sec:idea_generation}-\ref{sec:design}) from idea generation to design, where an animated VR story is conceptualized and planned. The production phase refers to asset development (Sec.~\ref{sec:asset_development}), scene assembly (Sec.~\ref{sec:scene_assembly}), and story integration (Sec.~\ref{sec:story_integration}), marking the hands-on execution that brings the story to life. Our interviewees did not have post-production activities like editing before conducting the evaluation (Sec.~\ref{sec:evaluation}).
Below describe each stage and summarize corresponding practices.


\wrap{figures/icons/idea.png}
\subsubsection{Idea Generation}
\label{sec:idea_generation}
Creators draw their initial ideas from various sparks, such as personal experiences, dreams, and existing artworks. They then think about what themes, messages, and emotions to convey. 
They may select visual styles (e.g., realistic versus cartoonish), set an overall mood or atmosphere, and have rough pictures of characters. 
Creators may assess if VR is the right medium for their vision, making decisions on aspects that require particular attention in this medium, such as 3-DoF or 6-DoF camera control, passive or active viewer roles, levels of interactivity, and the incorporation of multiple perspectives. 

\wrap{figures/icons/story.png}
\subsubsection{Story Creation}
\label{sec:story_creation}
Creators outline the backbone of a VR story and determine what viewers would experience sequentially (see Fig.~\ref{fig:workflow_example_1}-B1). They also create the main story elements, such as characters, worlds, and storylines (see Fig.~\ref{fig:workflow_example_1}-B2). Our interviewees employed direct (e.g., linear narrative) and/or indirect (e.g., leaving clues in VR environments) ways to tell their stories.

When aiming for maximal clarity in conveying messages to the viewers, interviewees adopted a more direct approach. They prioritized the development of characters and central conflicts. Most interviewees adopted a linear narrative structure and organized the story's events in chronological order. Only P16 incorporated a branching narrative, offering viewers the opportunity to influence the story's direction through choices of their roles in the story.

Conversely, when aiming for an open-ended and exploratory story experience, interviewees preferred indirect approaches, mainly using environmental storytelling to exploit VR's immersive and multi-sensory benefits. As an example, P2 shared, \q{Inspired by a science fiction novel, I used five interconnected environments to indirectly tell my story. Each environment left subtle clues, such as sun positioning for a timeline and color changes for different places. I allowed viewers to explore freely to feel and find these clues and form their own interpretation of the story.}

\wrap{figures/icons/script.png}
\subsubsection{Scriptwriting}
\label{sec:script_writing}
Scripts are a written and structured format of stories. Interviewees' scripts varied. Some might only include common elements like character dialogue and narration, and others might also include audiovisual cues (e.g., camera setting, sound effects) and interactions at different levels of detail. 

\wrap{figures/icons/storyboard.png}
\subsubsection{Storyboarding}
\label{sec:storyboarding}


\begin{figure*}[!tb]
    \centering
    \includegraphics[width=0.98\linewidth]{figures/updated/interaction.png}
    % \vspace{-8mm}
    \caption{An example of interaction design while connecting both the story plots and scenes (\CR{\copyright Coin's team}). (A) Brainstorming on a whiteboard about possible interactions with available objects within a scene. (B) Marking places where interactions will be triggered on a scene image. (C) Representing interactions in a fishbone diagram to examine whether these interactions can propel the plots forward.}
    \label{fig:interaction_design_1}
    \Description{This figure demonstrates how interactions are designed in relation to story plots and scenes. Part A shows a structured plan with story events and gameplay elements arranged in columns and sticky notes. Part B presents a 3D kitchen scene, highlighting interaction points such as the oven (B2) and cooktop (B1), where specific actions occur. Part C illustrates a flowchart of story progression, moving from an omniscient to a subjective perspective as the adventure begins, with interactions like lighting candles and managing objects integrated into the plot.}
\end{figure*}

\begin{figure*}[!tb]
    \centering
    % \includegraphics[width=\linewidth]{figures/scene_assembly.png}
    \includegraphics[width=\linewidth]{figures/updated/assembly.png}
    % \vspace{-8mm}
    \caption{Key considerations in the scene assembly stage. (A) Optimizing the coherence of 3D spatial relationships amongst assets across varied camera angles (\CR{\copyright Jiaming's team}). (B) Calibrating light and color settings to convey subtle ambiance (\CR{\copyright Tiemu}). (C-E): Guiding viewers' attention by (C) enlarging the subject of interest, (D) creating color contrast, and (E) integrating sound effects (\CR{C-E \copyright AMAO}).} 
    \label{fig:scene_assembly}
    \Description{This figure highlights key considerations in the scene assembly stage. Part A shows a scene with a rabbit and a turtle from different camera angles. Part B compares two versions of the same scene to show how lighting and coloring affect mood. Part C emphasizes the subject, demonstrating how framing draws attention to key actions. Part D explores color contrast. Part E integrates sound, linking it to visual elements, such as a vehicle with a satellite dish.}
\end{figure*}


Storyboards are a series of sketches that visualize the story flow, key moments, and plots.  Our interviewees configured initial scene compositions, camera settings, and character poses in storyboards. Because viewers control cameras in VR, interviewees also needed to guess how viewers would observe and interact with the scenes and then annotated visual and auditory cues to direct viewers' attention. As shown in Fig.~\ref{fig:storyboarding}, our interviewees used storyboards in different ways for various purposes.

\wrap{figures/icons/previs.png}
\subsubsection{Previs}
\label{sec:previs}
Previs (or previsualization), involves the use of rough 3D visual representations for story segments or whole stories, previously described in natural language or 2D pictures. 
As shown in Fig.~\ref{fig:previsualization}, 
low-detail 3D models were laid out in 3D spaces to represent scenes and interactions, serving as proxies for the final, polished versions. 
A few interviewees used previs selectively for key story segments. They focused on exploring spatial relationships, experimenting with dynamic factors like camera and character movement, and prototyping VR interactivity. 
Some interviewees from larger teams used previs to map out the entire stories to aid the team and stakeholder communication.
For example, P4 acted as a director and utilized Quill to create comprehensive previs that conveyed her vision to Unity developers without art backgrounds and helped secure approval from her superiors.

\wrap{figures/icons/audio.png}
\subsubsection{Visual, Audio, and Interaction Design}
\label{sec:design}

Creators transform their imagination and visions into physical or digital representations (e.g., drawings and videos) of visual, auditory, and interactive elements. 
Our interviewees designed with VR's characteristics in mind, building upon previous settings on the story's themes, mood, and atmosphere.
For visual design, they used drawings to establish the look of the characters, props, and environments. 
For audio design, they determined the rhythm and functions of background music, sound effects, and spatial audio that reinforced the story's emotional beats and atmosphere.
For interaction design, they particularly cared about whether the interaction enhanced viewers' engagement rather than distracting viewers from the story. P9 stated, \q{If an interaction was isolated or distracted viewers, I would leave it out.} Thus, our interviewees thought about where and when to introduce interactions based on scenes and storylines in order to connect viewers with the scene and propel the plot forward. Figure~\ref{fig:interaction_design_1} shows an example in which a group of creators brainstormed and annotated where interaction happened, and used a fishbone diagram to integrate interaction and plots.



\wrap{figures/icons/asset.png}
\subsubsection{Asset Development}
\label{sec:asset_development}

Creators develop 3D models, textures, rigs, and animations for characters, props, and environments, transforming the design concepts into digital assets. They also produce auditory elements like music and voice-over and implement interaction logic for interactive elements.
Our interviewees dedicated much effort to achieving specific visual aesthetics, either as an overall style (e.g., \q{ink painting} by P1) or for individual elements (e.g., \q{sparkling water} by P2 and \q{mysterious mist} by P7).




\begin{figure*}[!tb]
    \centering
    \includegraphics[width=\linewidth]{figures/updated/story.png}
    \caption{Practices of within-scene integration that synchronizes multiple story elements temporally. (A) Listing by-plot set-ups with a table, which covers character actions, scenes, and configurations of the camera, light, and sound (\CR{\copyright Coin's team}). (B) Aligning the camera and voice-over with the scene to guide viewers on a tour through various cities (\CR{\copyright ZhangXS}).  (C) Illustrating how lighting should change along a timeline with marked events (\CR{\copyright Coin's team}).} 
    \label{fig:within_scene_integration}
    \Description{This figure illustrates practices for within-scene integration that coordinate multiple story elements over time. Part A presents a detailed chart mapping out the plot, character actions, scenes, camera angles, lighting, and sound. Part B demonstrates camera movement and voice-over integration, with a scene flow depicted visually to show the progression of events. Part C highlights the technical setup for lighting and coloring, displaying a timeline that links character actions and scene transitions.}
\end{figure*}

\begin{figure*}[!tb]
    \centering
    \includegraphics[width=\linewidth]{figures/updated/scene_transition.png}
    \caption{Practices of between-scene integration that concern how a scene transits to another (\CR{\copyright Angela Cai}). (A) Panning transition: cameras brought viewers to mythical worlds by moving through a mythical monster's mouth and a keyhole. (B) Teleportation: a rotating turntable teleported viewers between multiple parallel universes through similar visual elements.} 
    \label{fig:scene_transition}
    \Description{This figure illustrates practices for between-scene integration, focusing on how one scene transitions to another. Part A showcases a "Panning Transition," where camera movement smoothly follows a subject to transition between scenes. Part B demonstrates "Teleportation," where the transition occurs by switching to another scene with matching elements.}
\end{figure*}


\wrap{figures/icons/assembly.png}
\subsubsection{Scene Assembly}
\label{sec:scene_assembly}

The scene assembly stage focuses on the spatial aspects of the story. 
With individual assets (e.g., characters and props) at hand, our interviewees assembled these elements in 3D spaces to construct narrative scenes. Relying on prior storyboards and previs, they undertook multiple iterations to fine-tune the placement of assets, cameras, lights, and sounds. The placement involved several key considerations (Fig.~\ref{fig:scene_assembly}) for both aesthetic cohesion and narrative efficacy. 
First, they carefully considered the 3D spatial layouts amongst assets from multiple angles (Fig.~\ref{fig:scene_assembly}-A), such as placing assets at different layers to create depth. 
Second, they fine-tuned lighting, color, and special effects to craft the desired atmosphere and aesthetic within individual scenes. For instance, an interviewee adjusted a room scene to be darker and dimmer to express a feeling of depression (Fig.~\ref{fig:scene_assembly}-B). Third, they paid particular attention to guiding viewers' attention by experimenting with various audiovisual cues. For example, they enlarged the points of interest as an alternative to close-up camera shots (Fig.~\ref{fig:scene_assembly}-C). They might also simplify surroundings but emphasize the points of interest with colors (Fig.~\ref{fig:scene_assembly}-D) or sounds (Fig.~\ref{fig:scene_assembly}-E). 


\wrap[1.45cm]{figures/icons/integration.png}
\subsubsection{Story Integration}
\label{sec:story_integration}


With the spatial stages already set, this stage primarily involves the temporal orchestration (e.g., timing, duration, and sequencing) of story elements within and between scenes to form a complete story. This stage consists of within-scene integration (Fig.~\ref{fig:within_scene_integration}) and between-scene integration (Fig.~\ref{fig:scene_transition}).

For within-scene integration, interviewees engaged in two key tasks. First, they gave temporal dynamics to story elements such as characters, lighting, and cameras by specifying their movements, which included trajectory, duration, and speed. Second, they coordinated and synchronized these dynamic elements to construct plots. They strove to make all the elements flow together cohesively over time to deliver the intended messages or emotions. For example, an interviewee used a table (Fig.~\ref{fig:within_scene_integration}-A) to detail how each element should move in each plot with timestamp and duration specifications. The interviewee also used an illustration (Fig.~\ref{fig:within_scene_integration}-C) to think about how lighting should change along events. 
Another interviewee aligned the camera and voice-over with the environment (Fig.~\ref{fig:within_scene_integration}-B) to guide viewers on a tour through various cities.

During between-scene integration, interviewees worked on connecting individual scenes smoothly. They often chose transition methods that seamlessly fit into the story's context and utilized unique experiences provided by VR. For example, an interviewee adopted two transition methods: one used panning cameras to transition from one scene to another (Fig.~\ref{fig:scene_transition}-A), while the other offered teleportation experiences with matching elements (Fig.~\ref{fig:scene_transition}-B).

\begin{figure*}[!tb]
    \centering
     \includegraphics[width=\linewidth]{figures/updated/workflow.png}
    \caption{Typical workflows for crafting animated VR stories. (A) Story-driven workflow: prioritizing narrative coherence with visuals serving the story. (B) Visual-driven workflow: pursuing compelling visual effects that shape the story content.} 
    \label{fig:diverse_workflow}
    \Description{This figure illustrates typical story-driven (A) and visual-driven (B) workflows to craft animated VR stories.}
\end{figure*}


\wrap{figures/icons/eval.png}
\subsubsection{Evaluation}
\label{sec:evaluation}

Creators invite viewers to experience the final VR story and give feedback. Our interviewees primarily sought feedback on narrative comprehension, engagement, and user comfort through informal interviews or casual chats. 
They used two main ways to present their VR stories to viewers. First, many showcased their VR stories at exhibitions for walk-in visitors. Second, to reach a wider audience, they converted their VR stories into 360-degree panoramic videos, viewable on mobile phones and computers. Some also created 2D videos by recording their VR stories for online streaming platforms. 

\subsection{Story-Driven and Visual-Driven Workflows}
\label{sec:diverse_workflow}

Our interviewees embraced diverse workflows that connected the above ten stages, although some might omit some stages. Only \RR{nine, ten, eleven, and sixteen} interviewees were involved in scriptwriting, storyboarding, previs, and evaluation. The sequence in which these stages were carried out also varied. For instance, while some prioritized design before storyboarding and previs, others followed the reverse order. Two common iterative cycles emerged in their workflows. The first cycle occurred between story creation, scriptwriting, and production, while the second cycle appeared between production and evaluation.
Despite the diversity in these workflows, they can be classified as story-driven or visual-driven. Figure~\ref{fig:diverse_workflow} provides representative examples for each type. 

Most interviewees (16/21) adopted story-driven workflows (Fig.~\ref{fig:diverse_workflow}-A), usually in an order of the pre-production phase, production phase, and evaluation stage. They prioritized narrative coherence, ensuring that the visuals were designed to support and enhance the story. When certain visual elements proved difficult to implement, they often adjusted their designs to more practical or achievable solutions. For example, P3 changed his design from dreamlike to realistic to avoid additional workload.

In contrast, in visual-driven workflows (Fig.~\ref{fig:diverse_workflow}-B), interviewees (5/21; P1, P11, P14, P15, P16) exhibited a pronounced focus on visuals. They invested significant effort in exploration and experimentation. For example, P14 and P15 delved into rendering and shaders for stylized visuals, whereas P18 spent three months translating her 2D illustration visual style into VR using Quill. In visual-driven workflows, the stages in the pre-production and production phases were carried out alternately. Specifically, story creation followed the design and asset development stage, with the story often shaped by available assets, resulting in relatively simpler storylines.
