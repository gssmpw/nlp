\section{Introduction}
An animated virtual reality (VR) story is a sequence of connected events, crafted to convey messages and evoke emotions, portrayed through computer animation, and unfolded within an interactive and three-dimensional VR environment~\cite{franklin2021fairy,cutler2019making,dooley2021cinematic, bosworth2018craftingVRstory}. 
Combining the high level of presence offered by VR~\cite{bindman2018bunny} with the boundless creative possibilities of computer animation~\cite{cutler2019making}, animated VR stories can provide unparalleled experiences.
These advantages have attracted many creators to tell their stories in this form. 
Early pioneers included filmmakers, animators, and transmedia artists from Oculus Story Studio and Google's Spotlight Stories, who created groundbreaking shorts like \textit{Henry} (2015) and \textit{Pearl} (2016). Building on this momentum, animated VR stories have now become mainstream within the VR sections~\cite{bosworth2018craftingVRstory} of prominent film festivals \cite{CannesAward} like Cannes and Sundance, as evidenced by award-winning stories \textit{Invasion!} (2017), \textit{The Dream Collector} (2017), \textit{Baba Yaga} (2020), and \textit{Namoo} (2022). 
Recently, with an expanding VR consumer market and a growing demand for VR content, more and more creators are releasing their animated stories~\cite{CannesAward} as standalone VR programs or on platforms like Oculus Video Animation Players.

\begin{figure*}[!tb]
    \centering
    \includegraphics[width=\linewidth]{figures/updated/instance.png}
    \caption{An illustration of 9 stages in creating an animated VR story (\CR{\copyright Coin's team}), where the red lines indicate how a storyline evolves.
    (A) \textbf{Idea generation}: brainstorming general ideas. (B) \textbf{Story creation}: conceiving the main character's activities using sketches (B1) and developing the story by including other characters with scattered word pieces (B2). (C) \textbf{Scriptwriting}: transforming the stories into textual scripts. (D) \textbf{Storyboarding}: sketching the key moments while configuring the timing, camera setups \& visual cues. (E) \textbf{Design}: specifying visual (e.g., character appearance) \& auditory (e.g., music) elements. (F) \textbf{Asset development}: developing rigged and animated 3D models \& audio clips. (G) \textbf{Scene assembly}: assembling individual assets spatially into scenes in the story. (H) \textbf{Story integration}: aligning multiple story elements on the timeline and integrating the whole story. (I) \textbf{Evaluation}: collecting feedback from viewers.} 
    \label{fig:workflow_example_1}
    \Description{This figure showcases the nine stages (labeled with Part A-I). Part A ("Idea Generation") shows a person thinking about various themes, such as "Counting sheep to fall asleep" and "Why VR?". Part B ("Story Creation") contains a rough sketch and text snippets illustrating the main character's activities, such as animals flying into the sky. Part C ("Scriptwriting") provides a script sample, describing how "a blue light beam lifted the lambs". Part D ("Storyboarding") presents a sequence of small visual frames, detailing information such as "Time: Night”, "Camera: Fixed”, and "Visual cue: Light”. Part E ("Design") features concepts for the visual appearance of various characters, such as a camel and a monkey. Part F ("Asset Development") focuses on audio and 3D models. It lists sound files and displays basic 3D model designs, including textured models of the characters. Part G ("Scene Assembly") contains two screenshots of a 3D environment being built, showing the placement of various elements in a virtual scene. Part H ("Story Integration") shows two additional 3D-rendered images, illustrating how the scenes are temporally connected. Part I ("Evaluation") shows a set of thumbs-up/thumbs-down buttons to indicate feedback or assessment of the final outcome.}
\end{figure*}

Despite the increasing interest in animated VR stories, dedicated support for creators has not received adequate attention from industry or academia.
For example, commercial software either lacks specialized design for animated VR stories (e.g., Blender, Unity) or is no longer actively maintained (e.g., Quill).
Although researchers have proposed several authoring tools~\cite{galvane2019vr, stemasov2023sampling, wang2022videoposevr, nguyen2017collavr, vogel2018animationvr}, these tools are fragmented and limited to a subset of tasks like VR storyboards~\cite{henrikson2016multi, henrikson2016storeoboard, galvane2019vr} and asset animation~\cite{lamberti2020immersive, vogel2018animationvr, wang2022videoposevr}. Many other essential tasks reported by creators~\cite{ward2021tinker,gipson2018disneycicles,cutler2019making, darnell2016invasion} are not adequately supported, such as composing VR scenes~\cite{cutler2019making,gipson2018disneycicles} and enabling interactions inside stories~\cite{ward2021tinker,darnell2016invasion}.
The insufficient support compels VR story creators to bridge general-purpose software, demanding significant time and effort to learn and experiment~\cite{gipson2018disneycicles}. 
They wish for streamlined tools that blend into their processes and address the challenges they face~\cite{cutler2019making}.
However, a comprehensive understanding of current creation processes and challenges remains absent, making the future research directions in supporting animated VR story creation unclear.

Recent HCI studies~\cite{ashtari2020creating, krauss2021current,  krauss2022elements} have provided empirical insights into the current practices and challenges of crafting VR applications. Although they acknowledge the complexities of prototyping story-driven VR experiences~\cite{ashtari2020creating, krauss2021current}, these studies generally aim to accommodate a wide array of VR use cases such as training and rehabilitation. 
Consequently, creators' unique considerations for animated VR stories have been overlooked.
The first consideration is rooted in the essence of stories.
Animated VR stories prioritize aspects such as narrative engagement, empathy, and emotional resonance~\cite{bindman2018bunny, bahng2020reflexive}. To achieve so, creators carefully consider a blend of visual, auditory, and interactive story elements~\cite{cutler2019making,ward2021tinker} and may face challenges in orchestrating them cohesively in both spatial and temporal aspects.
The second consideration is rooted in the use of VR and computer animation technologies to tell the story. 
Creators need to consider the benefits and constraints these technologies present. For example, though VR enables storytelling with a high level of presence, it also requires appealing 360-degree visuals~\cite{cutler2019making,gipson2018disneycicles}, which may bring challenges to balance visual quality and runtime performance.
We have yet to adequately understand how the interplay of the two considerations poses challenges to animated VR story creators within their creation processes.

To fill the above gaps, this study aims to answer the following research questions:

\begin{itemize}[left=0pt]
    \item \textbf{RQ1}: What are the creation processes that creators usually follow when crafting animated VR stories?
    \item \textbf{RQ2}: What challenges do animated VR creators face, especially when marrying story elements with the benefits and constraints of VR and computer animation technologies?
\end{itemize}


We conducted semi-structured interviews with 21 animated VR story creators. To achieve a broad understanding, we ensured our interviewees' backgrounds covered experiences in crafting diverse stories. The stories varied in their visual styles and levels of interactivity, from head-tracking to gesture interactions with in-story characters. They were made with various non-immersive (e.g., Blender, Unity) or immersive (e.g., Quill, Open Brush) software.
%
% findings
For RQ1, we identify ten common stages in the creation processes, from idea generation to evaluation (Fig.~\ref{fig:workflow_example_1}). The inclusion and order of these stages can vary and form diverse workflows. Two types of workflows emerge: story-driven and visual-driven workflows that prioritize story content or visuals, respectively. 
%
For RQ2, we identify seven challenges including a total of seventeen issues (Table~\ref{tab:challenges}) at different stages. Among them, nine issues are relatively unique to animated VR stories, which highlight narrative intent and viewer autonomy, satisfactory visuals for artistic expression, and multiple narrative perspectives. 
Based on the findings, we offer several future research opportunities to support animated VR story creation and discuss the differences between animated VR stories and general XR applications~\cite{krauss2021current, ashtari2020creating, krauss2022elements}. 

In summary, our contributions around animated VR stories are threefold: (1) identification of ten common stages and two types of workflows in the creation processes (\RR{Sec.~\ref{sec:processes}}), (2) summarization of nine unique issues in crafting them (\RR{Sec.~\ref{sec:challenges}}), and (3) provision of future research opportunities to support their creation (\RR{Sec.~\ref{sec:opportunities}}).