\section{Validation}
\label{sec:validation}

We invited both the original interviewees (N=21) and \RR{additional animated VR story creators (N=15)} to review the findings reported in Sec.
~\ref{sec:common_stages} and Sec.~\ref{sec:challenges}. The purpose was to assess the applicability, perceived importance, and perceived difficulty of the summarized stages and challenges.


\subsection{Setup}
To validate our findings, we designed two online questionnaires regarding the creation stages and challenges, respectively. Then we asked each respondent to complete the two online questionnaires. 

\textbf{Questionnaire for creation stages.}  The questionnaire described the ten stages and showed Fig.~\ref{fig:workflow_example_1} as a visual depiction. It then asked: (1) \q{Please select the three most important stages in completing an animated VR story.} (2) \q{Are there any stages in your creation of animated VR stories that these ten stages do not encompass?}


\begin{figure*}[!tb]
    \centering
    \includegraphics[width=\linewidth]{figures/updated/internal_external.png}
    \vspace{-6mm}
    \caption{Summary responses from our 21 original interviewees (A1-A4) and \RR{15 additional creators (B1-B4)}, including the number of votes for each stage regarding the importance (A1, \RR{B1}), subjective ratings on the uniqueness (A2, \RR{B2}) and difficulty (A3, \RR{B3}) of newly identified issues, and importance (A4, \RR{B4}) to address the issues. 1: Not unique/difficult/important at all. 7: Extremely unique/difficult/important.} 
    \label{fig:interviewee_rating}
    \Description{This figure has two parts, summarizing validation results for the importance of stages and challenges from our original interviewees and additional creators. Part A is the results from our original interviewees, and Part B is the results from the additional creators. From left to right, each part consists of votes for important stages, uniqueness of issues, difficulty of issues, and the importance of addressing these issues.}
\end{figure*}

\textbf{Questionnaire for creation challenges.}
The questionnaire had two parts. The first part described each newly identified issue in turn, followed by three questions rated on a 7-point Likert scale: (1) \q{Compared to other forms of stories, to what extent is this issue unique to animated VR stories?} (2) \q{How much difficulty does this issue pose during your VR story creation?} (3) \q{Assuming we want to offer better creativity support, how important is it to solve this issue?} 
The second part presented all 17 identified issues and asked whether any additional issues were missed.

\RR{\textbf{Respondents.}}
\RR{We had a total of 36 respondents, consisting of 21 original interviewees (Sec.~\ref{sec:interview_recruitment}) and 15 additional creators. The interviewees helped verify our interpretations of their input, while the additional creators could further assess the comprehensiveness of our findings.
To recruit additional creators, we distributed our questionnaires to several VR creators' online communities and art schools.
After excluding three respondents who had not completed one animated VR story, we had 15 valid additional creators (aged 22-35; 7 females and 8 males). All creators were from China, except one from Japan.
Similar to our interviewees, all creators had formal art training and experience in different artistic fields such as digital media arts and animation. Their experience in non-VR art averaged 4.9 (min=2, max=9) years and their VR creation experience averaged 2.2 (min=1, max=5) years.
They had created an average of 2 (min=1, max=5) VR stories, with the creation time per story averaging 2.7 (min=1, max=6) months.}


\subsection{Analysis and Results}
\RR{The votes and ratings were generally consistent between the original interviewees and additional creators, with both groups agreeing on our findings. The detailed analysis and results are as follows.}

\textbf{Importance of stages.} Figure~\ref{fig:interviewee_rating} shows the number of votes for each stage selected as the top three important stages by both our original interviewees (Fig.~\ref{fig:interviewee_rating}-A1) and \RR{additional creators (Fig.~\ref{fig:interviewee_rating}-B1)}.
\RR{Both groups consistently identified story creation (14/21 original, 9/15 additional) and design (15/21 original, 12/15 additional) as the most crucial stages.} This consensus aligns with our findings on story-driven and visual-driven workflows (Sec.~\ref{sec:diverse_workflow}), where the story and design determine the direction of subsequent stages and outline the final output content.
\RR{Scene assembly (2/21 original, 2/15 additional) and story integration (3/21 original, 1/15 additional) received limited votes across both groups}, likely because the respondents prioritized stages demanding creativity. 
They viewed scene assembly and story integration as basic and manual stages of production, where the quality of results depends on human labor and skill level rather than creativity.
Nevertheless, all stages were considered important by at least one respondent in each group. 

\textbf{Uniqueness, difficulty, and importance of challenges.}
All the nine issues received mean ratings above 4 on a 7-point Likert scale across all three aspects from the interviewees (Fig~\ref{fig:interviewee_rating}-A2, A3, A4) and \RR{creators (Fig~\ref{fig:interviewee_rating}-B2, B3, B4), suggesting that both groups recognized the nine issues as unique, difficult, and important.} \RR{Furthermore, we performed Mann-Whitney U tests to compare the ratings of the two groups for each issue across the three aspects. Under a significance level of 0.05, no significant differences were found between the two groups for any issue. While there were variations in mean ratings between groups, such variations were expected and natural, as individual experiences could influence subjective ratings. Based on these non-significant test results, we proceeded to aggregate the raw ratings from both groups. The aggregated mean ratings show that the three most unique issues were C5-2 (mean=5.72), C6-1 (mean=5.69), and C1-1 (mean=5.64); the three most difficult issues were C3-1 (mean=5.31), C6-1 (mean=5.06), and C5-2 (mean=4.89); the three most important issues were C6-1 (mean=6.00), C3-1 (mean=5.78), and C1-1 (mean=5.72).} 
\RR{Interestingly, C1-1, C5-2, and C6-1 all relate to viewers, focusing on viewer autonomy, perspective management, and comfort. The results highlight that respondents were particularly concerned with effectively delivering their narratives while accommodating the unique affordances of VR, where viewers possess unprecedented agency and presence within the story space.}

\textbf{Comprehensiveness of stages and challenges.} 
None of the interviewees or \RR{additional creators} reported creation stages beyond our findings, indicating that our ten stages can describe the animated VR story creation process well. \RR{Respondents valued our clear stage identification, with one noting \q{my stages often overlap and blend together, so it's helpful to see them clearly laid out in this structured way, which will help organize my workflow in future projects.}}
In terms of challenges, our interviewees found that our summary reflected their input well and did not report any new challenges. \RR{Most additional creators also agreed that the challenges encompassed the obstacles they encountered. One creator suggested the challenge of \q{managing the intensity of VR experiences to avoid motion sickness or fatigue}, which fell under C6-1. Another creator mentioned \q{struggling to optimize VR story content for VR devices with varying specifications}, which aligned with C6-2. Therefore, no distinct challenges were identified beyond our findings.}