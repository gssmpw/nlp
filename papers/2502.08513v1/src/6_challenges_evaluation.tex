\subsection{C7: Lack Data Collection and Analysis Methods for Evaluation}

In the evaluation stage, though our interviewees wanted to enhance their VR stories based on audience feedback, they had difficulties in collecting and analyzing data.

\textbf{C7-3: Insufficient support for collecting and analyzing viewers’ watching behavior data.}
Our interviewees found it difficult to understand viewers' experiences based on verbal communication. Thus, they wanted support to collect and visualize viewers' behavior, such as their gaze and exploration trajectories. 
For example, an interviewee wanted to identify the differences between the viewers' actual exploration and her intended path (Fig.~\ref{fig:quantitative_evaluation_support}-A) and \RR{points of interest (POIs)} (Fig.~\ref{fig:quantitative_evaluation_support}-B-E) to refine the storylines. However, most interviewees found that such support was limited due to the complexity of VR stories:

\q{Guiding a viewer's gaze in VR spaces requires me to place various cues. I am not sure whether my visual cues are effective, so I want a function for data visualization. Besides, my story is continuous in both time and space, and the viewers' exploration is also continuous in time and space. How can I analyze them together?} (P16)


