\section{Background and Related Work}

Our study builds on existing research into storytelling guidelines and design considerations for animated VR stories. To contextualize our research outcomes within HCI literature, we examine empirical studies on the creation of general XR applications and authoring tools for animated VR stories.

\subsection{Guidelines for Animated VR Stories}
Animated VR stories differ from traditional animations in two main characteristics~\cite{cutler2019making, godde2018cinematic}. First, they are set within immersive 3D spaces that encircle viewers~\cite{godde2018cinematic, serrano2017movie}, rather than being projected onto a flat screen. Second, they offer viewers various levels of interactivity~\cite{tong2021viewer, rothe2019interactioninCVR}, such as turning heads, walking around, and directly interacting with characters. 
These characteristics reconfigure the conventional relationships between the audience, camera, and story~\cite{henrikson2016multi, tong2022adaptive}. 
In contrast to traditional animations where creators fully control the camera and story, in VR, viewers pilot the cameras and can actively participate in and influence the story. 
As a result, some typical storytelling techniques, such as frame control and camera movement~\cite{serrano2017movie,henrikson2016multi}, may not be applicable to VR. 
Therefore, practitioners and researchers have proposed tailored storytelling guidelines~\cite{cutler2019making, williams2021virtual, aitamurto2021fomo, gupta2020roleplaying}, such as guiding user attention through audiovisual cues~\cite{rothe2019guidance, schmitz2020directing}, and strategically distributing story elements across 3D spaces~\cite{kvisgaard2019frames, pope2017geometry}.

HCI research has further broadened the research scope by associating VR storytelling with user experiences (e.g., presence~\cite{bindman2018bunny, kroma2022technical, men2017impact}, narration comprehension~\cite{bindman2018bunny, gupta2020investigating}, motion sickness~\cite{han2022evaluating, rahimi2018scene}, embodiment~\cite{liu2022generating}), emotional responses (e.g., empathy~\cite{bindman2018bunny}, affect~\cite{norouzi2021virtual}), and cognitive processes (e.g., self-reflection~\cite{bahng2020reflexive}, knowledge acquisition~\cite{zhang2019exploring, hwang2022being, lee2020data}, situational awareness~\cite{zhu2024reader}), as well as providing design recommendations based on study results. 
For instance, Bindman~\etal~\cite{bindman2018bunny} discovered that viewers' narrative engagement and empathy were more influenced by their perceived roles within the story rather than by the level of device immersion. 
Bahng~\etal~\cite{bahng2020reflexive} created an interactive story about death and loneliness, identified four reflexive design factors, and suggested incorporating these reflexive factors in future VR storytelling experiences, particularly when the goal is to provoke thoughtful self and social reflection.

While these studies alleviate creative hurdles, they neglect the practical difficulties of execution. Our study builds on these guidelines, examining how creators consider them during the creation process and identifying barriers to applying them in their stories.

\subsection{Empirical Studies for General XR Creation}
Recent HCI research has explored the current practices, challenges, and opportunities associated with creating XR applications. These studies target various types of creators (e.g., hobbyists~\cite{ashtari2020creating}, professional developers~\cite{krauss2021current, liu2023challenges, borsting2022software}, and professional designers~\cite{ashtari2020creating, krauss2021current, krauss2022elements}), settings (e.g., industry~\cite{krauss2022elements} and non-industry~\cite{ashtari2020creating, shin2023space}), and phases (e.g., the whole process~\cite{ashtari2020creating, krauss2021current, borsting2022software} and testing phase~\cite{liu2023challenges}). 

Some of these studies~\cite{ashtari2020creating, krauss2021current, borsting2022software, krauss2022elements, mendez2025immersivesurvey} recognize the significant difficulties in immersive storytelling. 
For example, Ashtari~\etal~\cite{ashtari2020creating} reported the difficulty in designing an immersive story with real-world sensory experiences and engagement, although VR provides immersive environments with reduced real-world distractions. 
Krauß~\etal~\cite{krauss2022elements} noted that typical manifestations of prototypes like text and storyboards fell short in effectively conveying the feeling of XR. However, these studies primarily focus on general XR applications and lack deeper insights into the unique characteristics of animated VR stories. Furthermore, their interviewees did not fully cover the current main creators of animated VR stories, who are probably filmmakers, transmedia artists, and animators~\cite{henrikson2016multi,rall2022pericles, cutler2019making}. 
Shin and Woo~\cite{shin2023space} found that creators adopted different creative strategies to associate story events with physical landmarks in AR. However, these findings may not be applicable to animated VR stories, which take place in virtual worlds and whose storylines are not constrained by physical sites.

Our study complements this line of research by identifying the creation processes and challenges of animated VR stories. 
Since creators need to consider multiple story elements, as well as benefits and constraints brought by VR and computer animation technologies, we are particularly interested in how these considerations influence their processes and pose challenges. 
Based on our findings, we compare animated VR stories to general XR applications. 


\begin{table*}[t]
    %\scriptsize
    \setlength{\aboverulesep}{0.2pt}
    \setlength{\belowrulesep}{0.2pt}
\setlength{\tabcolsep}{2pt}
\centering
\caption{A summary of interviewees' background and relevant experience. From left to right, each column shows the interview ID, background, years (Y) of experiences in general art/design and VR, number (\#) of animated VR stories and all VR artworks being created, and a list of different types of non-VR artworks.}
\label{tab:participant_information}
\Description{A summary of participants’ background and relevant experience.}
\resizebox{\textwidth}{!}{\begin{tabular}{llccccl}
 \toprule
 \multirow{2}[0]{*}{\textbf{ID}}  & \multicolumn{1}{c}{\multirow{2}[0]{*}{\textbf{Background}}} &  \multicolumn{2}{c}{\textbf{Experience} (Y)} & \multicolumn{2}{c}{\textbf{VR Artworks } (\#)} & \multicolumn{1}{c}{\multirow{2}[0]{*}{\textbf{Experience in Non-VR Artworks} (Types) } } \\ \cline{3-6}
 & & \multicolumn{1}{c}{\textbf{Art}}  & \multicolumn{1}{c}{\textbf{VR}} & \multicolumn{1}{c}{\textbf{Story}} & \multicolumn{1}{c}{\textbf{All}} &  \\ \midrule                                                               
P1          & Digital Media Arts                 & 10                             & 4           & 2                          & 2              & Painting, 2D/3D Animation                                                           \\
P2          & Transmedia Arts                    & 10                             & 2           & 2                          & 2              & Comics,  Painting, 3D Animation                                                         \\
P3          & UX Design, Product Design & 9                              & 1           & 1                          & 1              & Painting, Printmaking, Glitch Art,   Handicraft                   \\
P4          & Animation                          & 14                             & 3           & 4                          & 9              & Illustration, 2D Animation                                                              \\
P5          & Digital Media Arts                 & 10                             & 6           & 3                          & 8              & Motion Comics, Promotional Video,   
 Micro Movie, 3D Game, Theater     \\
P6          & Music Production                   & 13                             & 3           & 1                          & 2              & Dynamic Visuals and Sound-based Creation                                               \\
P7          & Transmedia Arts                    & 14                             & 4           & 1                          & 3              & 3D Game, Painting, Sculpture, Performance Arts, 2D/3D Animation          \\
P8          & Transmedia Arts                    & 8                              & 3           & 1                          & 1              & 3D Game, 3D Animation, Audio-Visual                                                        \\
P9          & Virtual Reality                    & 5                              & 3           & 1                          & 3              & Painting                                                                                \\
P10         & Transmedia Arts                    & 16                             & 6           & 1                          & 4              & Painting, 2D/3D Animation, Film                                                          \\
P11         & Advertising, Graphic Design        & 29                             & 4           & 1                          & 4              & Graphic Design, Sculpture, 2D Animation \\
P12         & Digital Media Technology           & 3                              & 2           & 1                          & 1              & Sculpture, Installation, Experimental Film, 3D Animation, 3D Game                     \\
P13         & Digital Media Technology           & 3                              & 1           & 1                          & 1              & Computer Graphics Art, 3D Game                                                                         \\
P14         & Virtual Reality                    & 6                              & 4           & 1                          & 1              & 3D Game                                                                                 \\
P15         & Motion Graphics                    & 7                              & 4           & 2                          & 2              & 2D Animated Shorts                                                                      \\
P16         & Graphic Design                     & 16                             & 3           & 3                          & 4              & Painting, Graphic Design, Video                                                         \\
P17         & Mural Arts                         & 20                             & 7           & 2                          & 8              & Kinetic Sculpture, Installation, Mural Arts, Computer Graphics Art                                     \\
P18         & Digital Media Arts, Animation      & 15                             & 4           & 2                          & 2              & Illustration                                                                            \\
P19         & Fine Arts                          & 10                             & 1           & 5                          & 10             & Illustration                                                                            \\
P20         & Digital Media Arts                 & 6                              & 3           & 2                          & 2              & Micro Movie, Photography, Painting, 2D/3D Animation, Installation                    \\
P21         & Motion Graphics                    & 24                             & 4           & 1                          & 1              & Illustration, GIF Animation, 2D Animated Shorts \\                               \bottomrule
\end{tabular}}
\end{table*}

\subsection{Authoring Tools for Animated VR Stories}
\label{sec:related_authoring_story}

Various authoring tools have been proposed in the literature for creating animated VR stories, which can be categorized into non-immersive, immersive, and hybrid based on the environments in which they are used. 

Non-immersive tools~\cite{henrikson2016storeoboard, zhao2020shadowplay2} can supplement existing commercial software (e.g., Blender and Unreal Engine), which is versatile but complex and oriented to broader 2D/3D game or animation creation. 
For example, Henrikson~\etal~\cite{henrikson2016storeoboard} designed an interactive tablet system for artists without 3D modeling skills to create stereoscopic storyboards with fluid pen-and-touch input.  
Similarly, ShadowPlay2.5D~\cite{zhao2020shadowplay2} allows novices to create 360-degree poetry stories, offering tailored features like an image repository and pen-based image animation. However, their 2D interfaces might cause a cognitive disconnect when considering spatial relationships and peripheral vision inherent in a 3D VR environment. 

These limitations drive interest in immersive tools, such as commercial tools (e.g., Quill and AnimVR) and research prototypes~\cite{galvane2019vr, stemasov2023sampling, wang2022videoposevr, nguyen2017collavr, vogel2018animationvr}. 
For example, Galvane~\etal~\cite{galvane2019vr} proposed a VR tool for storyboard creation that allows creators to arrange virtual spaces, capture snapshots, and then convert these into storyboards. This immersive approach gives creators a better understanding of spatial relationships.
AnimationVR~\cite{vogel2018animationvr} further allows direct manipulation via 6DoF controllers to animate characters, bypassing 2D gizmos.
Despite these advantages, immersive tools may bear accuracy issues, with limited feature support compared to their non-immersive counterparts.

To address these issues, research has explored hybrid tools~\cite{henrikson2016multi, lamberti2020immersive} to integrate the strengths of both classes. 
For example, VR Blender~\cite{lamberti2020immersive} combines the extensive support of Blender with the immersive authoring benefits of VR, thereby enhancing key animation tasks and facilitating reuse and modification. 
Henrikson~\etal~\cite{henrikson2016multi} proposed a multi-device system that facilitates artists to create storyboards on tablets while allowing directors to see them in VR.

While effective, these three types of tools remain fragmented and limited to certain tasks like storyboarding~\cite{henrikson2016multi, henrikson2016storeoboard, galvane2019vr}, asset animation~\cite{lamberti2020immersive, vogel2018animationvr, wang2022videoposevr}, and camera control~\cite{galvane2019vr}. 
Creators need to perform many other tasks~\cite{gipson2018disneycicles} to create animated VR stories and desire tools that blend into their creation processes~\cite{cutler2019making}.
To inform future research in creativity support for VR stories, we provide a nuanced understanding of the creation processes and challenges based on interviews with creators using both non-immersive and immersive tools.
