\section{Interview Study Design and Analysis}
To answer RQ1 and RQ2, we conducted semi-structured interviews with 21 creators to collect qualitative and in-depth insights on their creation processes and challenges. 

\subsection{Positionality Statement}
The study design, data collection, and data analysis were mainly performed by five authors with interdisciplinary backgrounds (denoted as A1--A5). A1 has 4-year experience in HCI research and 3-year experience in developing VR with game engines. A2 has 13 years of experience in digital entertainment development and production, sophisticated in 3D animation, VR shorts, and interactive films. A3 is an award-winning animation artist with 11 years of experience in directing 2D/3D animation and 3 years of experience in VR storytelling practice and research. A4 and A5 are technical HCI researchers with 6 and 15 years of experience, respectively.

\subsection{Recruitment and Interviewees}
\label{sec:interview_recruitment}
We recruited creators in various ways and interviewed those who had crafted at least one animated VR story.
First, we searched for VR stories on popular local content platforms and found the corresponding creators. We sent eleven interview requests and obtained five acceptances. 
Second, we distributed our advertisement to several art schools and received seven qualified responses.
Lastly, we reached out to nine creators from our personal network.

We stopped recruiting once interviewees' insights converged, ending up with 21 creators (all of them Chinese; 13 females and 8 males; age groups: 22--27 (14), 28--33 (4), 34--39 (2), 40--46 (1)). Table~\ref{tab:participant_information} lists their professional backgrounds and experiences. Specifically, all interviewees had formal art training but came from different subfields, such as digital media arts and graphic design. Their experience in non-VR art and VR art averaged 11.8 (min=3, max=29) years and 3.4 (min=1, max=7) years. On average, they created 1.8 (min=1, max=5) VR stories, spending approximately 3.4 (min=0.5, max=12) months per story. 
They also had abundant experience in various types of VR artworks (e.g., VR paintings) and non-VR artworks (e.g., 2D/3D animation, desktop games, and films).

\RR{In terms of their sociocultural backgrounds, fourteen interviewees were studying or working in various cities across China, while the others were in the United States, the United Kingdom, Japan, and Finland. Their creative works reflected a blend of cultural influences, incorporating both traditional and contemporary elements. Notably, Chinese traditional culture (N=16) serves as a foundational influence, encompassing classical literature (e.g., Tang poetry), religious philosophies (e.g., Buddhism and Taoism), and ethnic minority art. There is also a strong influence from anime culture (N=5), popular culture (N=5), science fiction (N=3), and gaming culture (N=2).} 


\subsection{\RR{Question Design} and Data Collection}

\RR{We developed our interview questions through brainstorming, refinement, and testing. Starting from our two research questions, A1 and A2 independently brainstormed potential interview questions and then collaborated to organize them.
A1 then refined the questions with A4 and A5. To test the questions, A1 conducted a mock interview with A3, an animated VR creator. A3's feedback highlighted the need to contextualize questions with concrete projects to elicit insights, which led us to incorporate project-specific questions. To control interview duration, we divided the resulting questions into two parts: (1) a pre-interview questionnaire (Appendix~\ref{appendix:questionnaire}) that included project-specific questions about interviewees' favorite VR stories, and (2) an interview guide (Appendix~\ref{appendix:guide}) to facilitate the discussion of creation processes and associated challenges.
}

The data collection with each interviewee consisted of three steps. First, we obtained their consent to join the interview and be video and audio recorded. Thirteen interviewees permitted us to use their intermediate and final artwork in our manuscript for research purposes, with their preferred pseudonyms included in the figure captions. Next, they completed the questionnaire at least two days before the interviews and prepared demos and materials related to their crafted stories. Detailed information about their favorite VR story projects can be found in Appendix~\ref{appendix:project}. Finally, we conducted individual semi-structured interviews \RR{online via video conferencing software}. The interview sessions averaged two hours, and all audio recordings were automatically transcribed for analysis.


\subsection{Interview Data Analysis}
In terms of \textbf{RQ1} about creation processes, we analyzed the interview data in three steps. 
Firstly, A1 extracted stages from each interviewee's descriptions, created a flow map to show their sequence, and annotated each stage with its inputs, outputs, and supporting quotes on Miro whiteboards. This process resulted in 21 whiteboards, each corresponding to one animated VR story. 
Secondly, A2 and A3 went through each whiteboard and discussed together with A1. For each whiteboard, we examined whether any mistakes existed in A1's initial organization and checked whether some original stages mentioned by the interviewees could be merged or split. 
\RR{For example, we merged technical stages like modeling, texturing, and rigging into a broader stage of asset development, as they collectively create usable 3D assets.}
Thirdly, we unified stage names across all whiteboards. 
\RR{Since our interview protocol did not prescribe any specific stages, the interviewees often used different terms to refer to the same stage. For example, \qq{whiteboxing} and \qq{3D layout} were both used to describe the previsualization practices. To determine appropriate stage names, we first referenced 3D animation~\cite{beane20123dAnimation} and filmmaking~\cite{jeffrey2020ves}, adopting terms such as \qq{idea}, \qq{story}, \qq{script}, \qq{storyboard}, \qq{previsualization}, and \qq{design}. We then refined these terms to better align with our collected data. For instance, we specified the design stage as \qq{visual, audio, and interaction design}. Additionally, we introduced the terms \qq{scene assembly} and \qq{story integration} to describe interviewees spatially and temporally composited all story elements.}
A1, A2, and A3 iterated between the second and third steps until we were satisfied with the results and then discussed them with A4 and A5.

To address \textbf{RQ2} about challenges, we conducted a thematic analysis~\cite{braun2019reflecting}. A1, A2, and A4 independently went through \RR{the same transcripts from four interviewees (approximately 20\% of all interview data}), highlighting sentences related to challenges and generating initial codes. Utilizing affinity diagrams, A1, A2, and A4 compared, discussed, and grouped their codes into themes. The themes were then reviewed together with A5 to develop an initial codebook. Subsequently, A1 coded the remaining transcripts using this codebook \RR{while maintaining sensitivity to new themes. When potential new themes emerged, A1 documented relevant instances and discussed them with A2 and A4 to refine the codebook.}
The final coding results were reviewed and refined through group discussions with all researchers.
