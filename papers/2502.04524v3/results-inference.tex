Here, the experimental characterization of the key metrics of the CMO/HfO\textsubscript{\textnormal{x}} ReRAM array relevant to inference performance is presented.
Specifically, the continuous conductance tuning capability is demonstrated over a range spanning approximately one order of magnitude. The trade-off between weight transfer programming noise of CMO/HfO\textsubscript{\textnormal{x}} ReRAM devices and number of required iterations for programming convergence is analyzed across different acceptance ranges. Furthermore, conductance relaxation—defined as the change in conductance over time after programming—is characterized. Finally, the combined impact of weight transfer, conductance relaxation, limited input/output quantization of the digital-to-analog converter (DAC) and analog-to-digital converter (ADC), and IR drop on the array wires is evaluated with respect to MVM accuracy.

\subsubsection{Weight transfer accuracy}\label{subsubsecMultibit}
In memristor-based AIMC inference accelerators, pre-trained normalized weights are initially mapped into target conductances and subsequently programmed into hardware in an iterative process known as weight transfer. This iterative process, which stops once the programmed conductance converges to the target value within a defined acceptance range, inherently introduces an error due to the analog nature of conductance weights. This error, described by a normal distribution with the standard deviation referred to as programming noise ($\sigma_{\rm prog}$), leads to a drop in MVM accuracy. To quantify this non-ideality, the non-volatile multi-level capability of the CMO/HfO\textsubscript{\textnormal{x}} ReRAM array is characterized. Fig. \ref{fig3}a shows the experimental cumulative distribution of conductance values for 35 representative levels, with all states sharply separated and without any overlap. Fig. \ref{fig3}b shows a schematic representation of the closed-loop (i.e., program-verify) scheme, where identical \textit{set} and \textit{reset} pulse trains are employed to program each ReRAM cell to its target conductance within a desired acceptance range (see ”Methods” section ”Identical-pulse closed-loop scheme” for details). Selecting programming conditions involves a fundamental trade-off: a narrower acceptance range can improve programming precision by reducing programming noise, but it increases the number of iterations required for convergence (see Fig. \ref{fig3}d). Besides the longer programming time, other non-idealities to consider when choosing the acceptance range are (1) the conductance relaxation immediately after programming, which is characterized in \ref{subsubsecRelaxation} for CMO/HfO\textsubscript{\textnormal{x}} ReRAM devices, and (2) read noise, which has already been characterized between 0.2\% and 2\% of \textit{G}\textsubscript{\textnormal{target}} for CMO/HfO\textsubscript{\textnormal{x}} ReRAM devices \cite{Davide_DRC} within a similar conductance range used in this work. 
\\
\\
The trade-off between the programming noise and the number of iterations is characterized for two representative acceptance range intervals: 0.2\% and 2\% of \textit{G}\textsubscript{\textnormal{target}}, respectively. Fig. \ref{fig3}c illustrates the experimental number of pulses needed to converge to the \textit{G}\textsubscript{\textnormal{target}} using the two representative acceptance ranges. On average, each cell requires approximately 11 and 89 \textit{set}/\textit{reset} pulses for acceptance ranges of 2\% and 0.2\% of \textit{G}\textsubscript{\textnormal{target}}, respectively. Since the acceptance range is defined as a percentage of \textit{G}\textsubscript{\textnormal{target}}, the number of iterations required for convergence is almost independent of the target conductance value. In the Supplementary Information, Fig. \ref{figS5}a shows the experimental cumulative distribution of conductance values for the same 35 representative levels presented in Fig. \ref{fig3}a, but using 2\% \textit{G}\textsubscript{\textnormal{target}} as acceptance range. The standard deviation of the representative conductance levels is extracted and fitted as a linear function of the target conductance (dashed lines), as shown in Fig. \ref{fig3}e, for both acceptance ranges. For all conductance levels, a standard deviation of less than 0.1~\textmu S (1~\textmu S) is achieved considering 0.2\% \textit{G}\textsubscript{\textnormal{target}} (2\% \textit{G}\textsubscript{\textnormal{target}}) as the acceptance range. This is more than one order of magnitude lower compared to other memristive technologies, such as phase-change memory (PCM) arrays, targeting similar conductance ranges \cite{Joshi2020, Tsai2019, LeGallo18}. These results demonstrate that CMO/HfO\textsubscript{\textnormal{x}} ReRAM cells achieve an almost ideal weight transfer during programming, enabling the distinction of more than 32 states (5 bits).

\begin{figure}[H]
\centering
\includegraphics[width=1.0\textwidth]{./Fig3.pdf}
\caption{\textbf{Weight transfer characterization.} \textbf{a} Cumulative distributions of 35 conductance states obtained using an identical-pulse closed-loop scheme with a 0.2\% \textit{G}\textsubscript{\textnormal{target}} acceptance range. For each distribution, the entire CMO/HfO\textsubscript{\textnormal{x}} ReRAM array was programmed to the corresponding \textit{G}\textsubscript{\textnormal{target}}, and the conductance values measured during the final closed-loop iteration (during programming) is reported. Each dot represents a 1T1R cell. \textbf{b} An example sequence of the identical-pulse closed-loop programming scheme utilized in this work. \textbf{c} Experimental number of closed-loop iterations as a function of \textit{G}\textsubscript{\textnormal{target}} for the two representative acceptance ranges. Each semitransparent point represents a 1T1R cell, the opaque points represent the average number of iterations per \textit{G}\textsubscript{\textnormal{target}}, and the horizontal dashed line indicates the overall average of the opaque points. \textbf{d} Graphical representation of the trade-off between programming noise and the number of iterations required for convergence, as a function of the acceptance range. \textbf{e} Experimental programming noise as a function of \textit{G}\textsubscript{\textnormal{target}} for the two representative acceptance ranges. 
Each point represents the standard deviation of the normal distribution measured across the entire array. The dashed lines in black indicate the corresponding linear fits.}
\label{fig3}
\end{figure}

\subsubsection{Conductance relaxation and matrix-vector multiplication accuracy}\label{subsubsecRelaxation}
In addition to the excellent weight transfer accuracy during programming as presented in the previous section, the characterization of temporal conductance relaxation is critical to estimate the MVM accuracy over time. In analog ReRAM devices, a significant conductance relaxation has been observed immediately after programming (within 1 second) \cite{Wan2022}. Following this initial abrupt conductance change, the relaxation process slows considerably \cite{Zhao2018,Wan2022}. The physical cause of retention degradation is attributed to the Brownian motion of defects in the resistive switching layer \cite{Zhao2018}. In this section, the conductance relaxation of the CMO/HfO\textsubscript{\textnormal{x}} ReRAM array after programming is characterized. Fig. \ref{fig4}a shows the relaxation of the distributions previously reported in Fig. \ref{fig3}a, approximately 10 minutes after programming. The 35 levels remain distinguishable 10 minutes after programming, with an average overlap of 9.6\% between adjacent states gaussians, while the average standard deviation of the distributions increases to 0.6~\textmu S, showing almost independence from the \textit{G}\textsubscript{\textnormal{target}} (see Fig. \ref{fig4}b). 
\\
\\
The stability of the CMO/HfO\textsubscript{\textnormal{x}} ReRAM conductance states is further assessed on a longer time-scale, up to 1 hour. To achieve so, a linearly spaced \textit{G}\textsubscript{\textnormal{target}} vector within the experimental conductance range of 10~\textmu S to 90 \textmu S is defined, with a fine step of 0.2~\textmu S (400 points). Each \textit{G}\textsubscript{\textnormal{target}} value is programmed into a single ReRAM device within the array. Due to the size mismatch between the array (32 devices) and the \textit{G}\textsubscript{\textnormal{target}} vector (size 400), multiple measurement batches are needed. Fig. \ref{fig4}c shows the experimental relaxation of the 400 programmed states within the entire conductance window, 1 second and 1 hour after programming, executed with the closed-loop scheme (see ”Methods” section
”Identical-pulse closed-loop scheme” for details) and with a 0.2\% \textit{G}\textsubscript{\textnormal{target}} acceptance range. The exhibited conductance error induced by the relaxation process after 1 hour, computed as \(G_{\mathrm{1h}} - G_{\mathrm{prog.}}\), is plotted as a function of the programmed conductances in Fig. \ref{fig4}d. After 1 hour, although both positive and negative relaxation errors are recorded, an average decrease in conductance is observed across all programmed states, with a relaxation error averaging around -0.7~\textmu S. This highlights that the relaxation process in CMO/HfO\textsubscript{\textnormal{x}} ReRAM devices leads, on average, to a decrease in the mean and an increase in the standard deviation of the Gaussian distributions regardless of the initial conductance state. Since the absolute magnitudes of the mean decrease and the standard deviation increase are independent of \textit{G}\textsubscript{\textnormal{target}}, an extended characterization of the relaxation process up to 1 week is conducted for a representative conductance state (50~\textmu S). To achieve this, the array's CMO/HfO\textsubscript{\textnormal{x}} ReRAM devices are programmed using the identical-pulse closed-loop scheme to \textit{G}\textsubscript{\textnormal{target}} of 50~\textmu S, with a 0.2\% \textit{G}\textsubscript{\textnormal{target}} acceptance range. Fig. \ref{fig4}e illustrates the experimental array relaxation over 1 week. The insets display the evolution of both the mean and standard deviation as a function of the logarithm of time after programming (in seconds), using a linear fit to predict the conductance distribution over a 10-year period. 
\\
\\
To assess the accuracy of analog MVM, a comprehensive set of non-idealities—both intrinsic to CMO/HfO\textsubscript{\textnormal{x}} ReRAM devices and at the architecture level—is considered, including finite programming resolution with 0.2\% \textit{G}\textsubscript{\textnormal{target}} acceptance range, conductance relaxation, limited ADC and DAC quantization, and IR-drop across array wires. Fig. \ref{fig4}f shows the hardware-aware simulation results of the analog MVM using CMO/HfO\textsubscript{\textnormal{x}} ReRAM cells, projected for up to 10 years from programming, compared to the expected floating-point (FP) result. The results are generated using a single 64×64 normally distributed random weight matrix and 100 normally distributed input vectors within the range [-1, 1] (see ”Methods” section ”HW-aware simulation of analog MVM” for details). Considering the input and output quantization of 6-bit and 8-bit respectively, the inset illustrates the time evolution of the root-mean-square error (RMSE) of the simulated analog MVM compared to the FP expected result. These results show that the CMO/HfO\textsubscript{\textnormal{x}} ReRAM core enables accurate MVM operations, achieving an RMSE ranging from 0.06 at 1 second to 0.2 at 10 years after programming, compared to the ideal FP case. Fig. \ref{figS6} in the Supplementary Information illustrates the impact of IR-drop and input/output quantization on the RMSE of an MVM performed on a 64×64 array. Over short time scales (within 1 hour), the primary accuracy bottleneck is the limited input/output quantization of 6-bit and 8-bit, respectively. Over longer periods, relaxation effects become the dominant source of non-ideality. In a larger 256×256 array, IR-drop emerges as the main accuracy bottleneck for analog MVM. Compared to the analog ReRAMs studied by Wan et al. \cite{Wan2022}, who report an experimentally determined RMSE of approximately 0.58 under conditions similar to those of this work, CMO/HfO\textsubscript{\textnormal{x}} ReRAMs demonstrate a potential improvement in MVM accuracy by a factor of 10 and 3, 1 second and 10 years after programming, respectively. The excellent MVM accuracy results demonstrate the suitability of CMO/HfO\textsubscript{\textnormal{x}} ReRAM devices for long-term AI inference applications, and lay the foundation for AI training acceleration, where short-term forward and backward MVMs are key steps.

\begin{figure}[H]
\centering
\includegraphics[width=1\textwidth]{./Fig4.pdf}
\caption{\textbf{Conductance relaxation and MVM accuracy.} \textbf{a} Probability density distributions of 35 conductance states approximately 10 minutes after programming. The black areas between adjacent Gaussian distributions represent the overlap of their tails. On average, an overlap of 9.6\% is observed after 10 minutes. \textbf{b} The standard deviations of the 35 conductance states during programming (in purple) and 10 minutes after it (light blue). \textbf{c} Relaxation of 400 conductance states, with one device per G-state, measured 1 second and 1 hour after programming. \textbf{d} Relaxation error 1 hour after programming. A negative and nearly G-independent average error (dashed line) indicates that relaxation in CMO/HfO\textsubscript{\textnormal{x}} ReRAMs tends toward a slight conductance decrease and is state-independent. \textbf{e} Experimental array relaxation of a representative 50 \textmu S state, up to 1 week after programming with 0.2\% \textit{G}\textsubscript{\textnormal{target}} acceptance range. The experimental data used to extract the distributions are represented as points aligned to the y=0 horizontal axis. Insets show the time dependence of the mean and standard deviation. Dashed blue lines represent the conditions during programming, once the convergence to \textit{G}\textsubscript{\textnormal{target}} is reached, while a linear fit (green dashed line) extrapolates the distribution 10 years after programming (dashed black line).
\textbf{f} Analog MVM accuracy simulations using a 64x64 CMO/HfO\textsubscript{\textnormal{x}} ReRAM array as a function of time after programming (indicated by different colors). The inset shows the expected RMSE compared to the ideal FP result.  Experimental programming noise, conductance relaxation, limited input/output quantization and IR-drop are considered in this assessment.}
\label{fig4}
\end{figure}



