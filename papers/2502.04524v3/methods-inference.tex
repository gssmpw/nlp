The 'aihwkit' \cite{Rasch2021AFlexible} simulation tool was used to perform MVM assessments including non-ideal behaviors and noise, and their effect on the computation accuracy with respect to floating-point operations. The MVM simulation included the exhibited programming noise, conductance relaxation, input and output quantization, and IR-drop across array wires.
The 'aihwkit' allows to configure such noisy effects for dedicated memristive devices such as PCM by Nandakumar et al. \cite{Nandakumar2019PhaseChange} and ReRAM by Wan et al. \cite{Wan2022}. Therefore, a unique phenomenological noise model for CMO/HfO\textsubscript{\textnormal{x}} ReRAM devices for inference is developed to incorporate into the simulation both the characterized programming noise and conductance relaxation. Additionally, input and output are quantized with 6-bit and 8-bit resolution, respectively, and the IR-drop is considered, with 100~\textmu S as the maximum ReRAM conductance level and a default segment wire resistance of 0.35~$\Omega$. 
\subsubsection{Modelling the programming noise}
For a target conductance  \textit{G}\textsubscript{\textnormal{target}}, the device's programmed conductance is defined as the target value plus normally distributed noise with a standard deviation $\sigma_{\rm prog}$, which is a function of \textit{G}\textsubscript{\textnormal{target}}. As depicted in Fig. \ref{fig3}e, the programming noise ($\sigma_{\rm prog}$) of the CMO/HfO\textsubscript{\textnormal{x}} ReRAM  devices is statistically described by a first-order polynomial equation for a given acceptance range. The polynomial coefficients for acceptance ranges of 2\% and 0.2\% of \textit{G}\textsubscript{\textnormal{target}} are extracted from the characterization and introduced into the simulation environment. To assess the effects of the programming noise, each weight in the normalized matrix (ranging from [-1, 1]) is mapped to its corresponding conductance value (within the range [9, 89]~\textmu S from Fig. \ref{fig3}a), and is then further adjusted by the programming noise described by the extracted linear functions. Therefore, the MVM accuracy can be assessed immediately after programming ($t=0$), see Fig. \ref{fig4}f.

\subsubsection{Modelling the conductance relaxation}
After programming, the conductance levels exhibit relaxation over time, as shown in Fig. \ref{fig4}. Unlike previous ReRAM drift characterizations reported by Wan et al. \cite{Wan2022} the observed relaxation in CMO/HfO\textsubscript{\textnormal{x}} ReRAM is approximately independent of the initial programmed conductance. Consequently, a new modelling approach in the 'aihwkit' is needed to accurately simulate the conductance relaxation effect, which differs from the methods derived from previous literature on ReRAM \cite{Wan2022}. The conductance relaxation mean and standard deviation are modelled independently of \textit{G}\textsubscript{\textnormal{target}} and solely as a function of time after programming. 
The coefficients of the first-order polynomials describing the time dependence of both the mean and standard deviation of the programmed conductance are incorporated into the simulation environment to estimate conductance variations at any given inference time. By doing so, the MVM accuracy can be estimated after a period of time up to 10 years.
