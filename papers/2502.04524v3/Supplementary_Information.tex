\section*{Supplementary Information}%\label{secA1}
\begin{figure}[H]
\centering
\includegraphics[width=1.0\textwidth]{./Fig_S1.pdf}
\caption{\textbf{ReRAM forming modelling.} The CMO/HfO\textsubscript{\textnormal{x}} ReRAM device is simulated using a 3D FEM in COMSOL Multiphysics 5.2 software. \textbf{a} The bird’s-eye view and \textbf{b} the lateral y-z view of the device’s geometry and material stack are shown. Due to the temperature and electric field confinement, an effective device area of 200$\times$200~nm$^2$ is considered for the simulation to reduce computational resource demands. \textbf{c} The experimental array forming data in the low-voltage linear regime (from 0 to \(0.2 \,\mathrm{V}\)) are fitted to extract the average filament radius. \textbf{d} The increase in experimental conductance resulting from a negative voltage sweep after the forming event is modelled as an effective increase in the electrical conductivity of the CMO layer, due to a radial redistribution of defects.}
\label{figS1}
\end{figure}

\begin{figure}[H]
\centering
\includegraphics[width=1.0\textwidth]{./Fig_S2.pdf}
\caption{\textbf{Experimental CMO/HfO\textsubscript{\textnormal{x}} ReRAM array forming data.} \textbf{a} The forming voltage distribution of the 1T1R cells within the array, defined as the voltage required to trigger the highest current increase during the quasi-static voltage sweep from 0 to \(3.6 \,\mathrm{V}\) in Fig. \ref{fig2}a of the manuscript. \textbf{b} The array forming current distribution corresponding to \( V = V_{\mathrm{forming}}^{\mathrm{1T1R}} \). \textbf{c} The experimental resistance of the transistor in the triode region at \(V_\mathrm{G} = \mathrm{1.2 \, V}\), extracted from a linear fit from 0 to \(0.2 \,\mathrm{V}\) of the transistor output characteristic. \textbf{d} The forming voltage distribution of the ReRAM array, shown in Fig. \ref{fig2}c of the manuscript, computed as \(V_{\mathrm{forming}}^{\mathrm{ReRAM}} \)=  \(V_{\mathrm{forming}}^{\mathrm{1T1R}} \) - \(R_{\mathrm{DS}}^{\mathrm{triode}} \)* \(I_{\mathrm{forming}}^{\mathrm{1T1R}} \).}
\label{figS2}
\end{figure}

\begin{figure}[H]
\centering
\includegraphics[width=0.6\textwidth]{./Fig_S3.pdf}
\caption{The experimental array’s response to the voltage sweep from 0 to \(-1.5 \,\mathrm{V}\), following the positive forming and the initial negative voltage sweep (denoted as step (1) and (2) in Fig. \ref{fig2}a of the manuscript, respectively). The oxygen vacancies in the CMO layer radially spread outward, depleting the CMO defect sub-band within a half-spherical volume at the interface with the conductive filament, leading to a \textit{reset} process.}
\label{figS3}
\end{figure}

\begin{figure}[H]
\centering
\includegraphics[width=1.0\textwidth]{./Fig_S4.pdf}
\caption{
The voltage-dependent evolution of \textbf{a} the average temperature and \textbf{b} electric field within a 3D half-spherical volume of the CMO layer situated atop the conductive filament in both HRS and LRS is presented. These trends serve as inputs for equation (6) of the manuscript.}
\label{figS4}
\end{figure}

\begin{figure}[H]
\centering
\includegraphics[width=1.0\textwidth]{./Fig_S5.pdf}
\caption{\textbf{a} The experimental cumulative distribution of conductance values for 35 representative programmed levels using 2\% of \textit{G}\textsubscript{\textnormal{target}} as acceptance range. The closed-loop scheme based on identical pulses shown in Fig. \ref{fig3}b of the manuscript and detailed in Methods is used. \textbf{b} Flowchart illustrating the identical-pulse closed-loop technique used for programming the ReRAM array into target analog conductance range.}
\label{figS5}
\end{figure}

\begin{figure}[H]
\centering
\includegraphics[width=1.0\textwidth]{./Fig_S6.pdf}
\caption{ \textbf{The individual impact of IR-drop across array wires and input/output bit quantization on MVM accuracy.} \textbf{a} Simulated RMSE compared to FP ideal results using 64x64 analog CMO/HfOx ReRAM array, shown as a function of the time after programming. Dashed horizontal lines represent the RMSE during programming, considering programming noise (with 0.2\% \textit{G}\textsubscript{\textnormal{target}} as the acceptance range) but excluding relaxation effects. With 32-bit input/output quantization and no IR-drop (orange dots), an RMSE as low as $10^{-2}$ is achieved during programming, which immediately triples (see the arrow) after relaxation (within \(\mathrm{1 \, s}\)). Including the realistic IR-drop results in an overall RMSE increase (blue squares). Finally, reducing input/output quantization to 6/8 bits, respectively, leads to a further accuracy loss (green crosses), demonstrating that at short timescales (within 1 hour), the main analog MVM accuracy bottleneck is the reduced input/output quantization. After 1 hour, all cases converge, showing that the accuracy bottleneck is then dominated by the relaxation process. \textbf{b} By scaling up to a 256x256 array size (grey diamonds) and considering input/output quantization of 6/8 bits, IR-drop emerges as the primary bottleneck for analog MVM accuracy.}
\label{figS6}
\end{figure}

\begin{figure}[H]
\centering
\includegraphics[width=1.0\textwidth]{./Fig_S7.pdf}
\caption{The experimental response of the 8x4 CMO/HfO\textsubscript{\textnormal{x}} ReRAM devices within the array to the open-loop programming pulse scheme (shown in Fig. \ref{fig5}b of the manuscript) is shown. The \textit{set} and \textit{reset} pulse amplitudes are \(1.35 \,\mathrm{V}\) (\(V_\mathrm{G} = \mathrm{1.4 \, V}\)) and \(-1.3 \,\mathrm{V}\) (\(V_\mathrm{G} = \mathrm{3.3 \, V}\)), respectively, with a constant width of 2.5~\textmu s due to setup limitations.}
\label{figS7}
\end{figure}

\begin{figure}[H]
\centering
\includegraphics[width=1.0\textwidth]{./Fig_S8.pdf}
\caption{The experimental open-loop pulsed response of a representative CMO/HfO\textsubscript{\textnormal{x}} ReRAM device within the array shows that the potentiation and depression characteristics do not inherently saturate at the upper and lower boundaries. The generalized soft bounds model (yellow line) better captures this experimental trend compared to the saturated soft bounds model (black line).}
\label{figS8}
\end{figure}

\begin{figure}[H]
\centering
\includegraphics[width=1.0\textwidth]{./Fig_S9.pdf}
\caption{
Multi-variate Gaussian distributions to reproduce the experimental inter-device variability. \textbf{a} Multi-variate gaussian G1 distribution of the experimental number of states and device asymmetry ($up\_down$). \textbf{b} Gaussian G2 distribution of the analytical parameters $\gamma$ and $\gamma_{\rm up\_down}$ extracted from the generalized soft bounds model fitting to the experimental traces.}
\label{figS9}
\end{figure}

\subsection*{Device modelling}
\subsubsection*{$up\_down$ parameter}
The $up\_down$ parameter is defined for the generalized soft bounds model in the simulation environment of the ‘aihwkit’ as the directional bias between the up and down update size ($\Delta G^+$ and $\Delta G^-$). In addition, the minimum step in each direction \textit{d} is described by the following expression \cite{Rasch2021AFlexible}.
\begin{align}
    \Delta G^d = \Delta G_{SP} (1 + d\beta + \sigma_{d-to-d})
   \label{eqS1}
\end{align}
where \textit{d} is -1 or 1 depending on the update direction. In contrast, the symmetry point is defined for each device as follows \cite{Rasch2024Agad}:
\begin{align}
    SP = [\Delta G^+ - \Delta G^-]/[\Delta G^+ / (b_{\rm max} - \Delta G^+/b_{\rm min})]
   \label{eqS2}
\end{align}
Where $\Delta G^+$, $\Delta G^-$ define the minimum step size in the up and down direction respectively, and $b_{\rm max}$ and $b_{\rm min}$ represent the upper and lower bounds of the conductance. Therefore, considering an independent definition of each device (i.e. zero d-to-d variability) and a normalized conductance range between -1 and 1, the symmetry point device-level characteristic and the $up\_down$ analytical parameter are equivalent.

\subsubsection*{Training setup}
For result replicability, the experimental parameters are incorporated into the simulation environment, where the Noise-to-Signal Ratio (NSR) is represented by  'dw\_min\_std', normalized SP by ‘up\_down’, normalized maximum and minimum conductances by ‘w\_max’ and ‘w\_min’ and min conductance step by ‘dw\_min’. 
\\
\\
From this device model, analog training simulations were performed using AGAD considering a learning rate to update the weights of 1e-2, ‘fast\_lr’ of 0.1 to update matrix, ‘transfer\_every’ 3 iterations and batch size of 32. The FP baseline was obtained with SGD training using a learning rate of 1e-3 and batch size of 32. 
