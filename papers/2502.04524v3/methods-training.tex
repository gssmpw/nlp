\subsubsection{Generalized soft bounds model}
The generalized soft bounds model (SBM) selection was based on the observed characteristics of the potentiation and depression since the devices did not strictly exhibit thorough saturation at the upper and lower boundaries (see Fig. \ref{figS8} in Supplementary Information). The generalized SBM incorporates a tunable scale exponent ($\gamma$) that describes abrupt and gradual trends toward the maximum and minimum conductance levels. This exponent parameter also varies depending on the conductance update direction. Therefore, the analytical expression of the generalized SBM implemented in the 'aihwkit' includes an asymmetry factor ($\gamma_{\rm up\_down}$) to account for this behavior\cite{aihwkit}. However, these two parameters do not have a direct physical equivalence, and therefore, cannot be derived from experimental traces. Hereby, $\gamma$ and $\gamma_{\rm up\_down}$ are obtained for each device through an independent linear fitting of the generalized SBM to the experimental response. In addition to the analytical parameters of the generalized SBM, devices in the 'aihwkit' are defined by a set of parameters that can be extracted from experimental traces. More precisely, the empirical maximum and minimum conductance, minimum conductance step size and its standard deviation, and the asymmetry between up and down response are considered ($G_{\rm max}$, $G_{\rm min}$, $\Delta G_{\rm sp}$, $\sigma_{\Delta G_{\rm sp}}$, and $up\_down$). More details on the $up\_down$ parameter are provided in the Supplementary Information. In this regard, each simulated device is defined by 6 parameters: four empirically obtained ($G_{\rm max}$, $G_{\rm min}$, $\Delta G_{\rm sp}$ and $up\_down$) and two analytically modelled from SBM linear fitting ($\gamma$ and $\gamma_{\rm up\_down}$).

\subsubsection{Intra and inter-device variability}
By extracting the standard deviation of the minimum conductance step size ($\sigma_{\Delta G_{\rm sp}}$) from the experimental traces and incorporating it into the simulation's device model, the device response intrinsically includes noise from cycle to cycle. This provides a realistic device behavior with intra-device variability. Furthermore, the network devices shall include inter-device variabilities to perform physically accurate simulations. To achieve this, two multi-variate Gaussian distributions, G$_{\rm 1}$ and G$_{\rm 2}$, are created (see Fig. \ref{figS9} in Supplementary Information). G$_{\rm 1}$ is extracted from the experimentally obtained parameters: N$_{\rm states}$ (which accounts for variations across devices in the G-range and step) and SP in the normalized G-range, whereas G$_{\rm 2}$ is fitted to the analytical model parameters obtained from the fitted generalized SBM ($\gamma$ and $\gamma_{\rm up\_down}$). Therefore, variables from G$_{\rm 1}$ showed statistical independence from those of G$_{\rm 2}$. New device instances are independently sampled from the two Gaussian distributions to represent synapses on the DNN layers. The instantiated CMO/HfO\textsubscript{\textnormal{x}} ReRAM devices include variations in the device response, conductance ranges, and asymmetrical behavior, thus providing a more hardware-aware and realistic scenario for analog training simulation. 