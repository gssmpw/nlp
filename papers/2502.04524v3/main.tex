\documentclass[pdflatex,sn-mathphys-num]{sn-jnl}% Math 

\usepackage{graphicx}%
\usepackage{multirow}%
\usepackage{amsmath,amssymb,amsfonts}%
\usepackage{amsthm}%
\usepackage{mathrsfs}%
\usepackage[title]{appendix}%
\usepackage{xcolor}%
\usepackage{textcomp}%
\usepackage{manyfoot}%
\usepackage{booktabs}%
\usepackage{algorithm}%
\usepackage{algorithmicx}%
\usepackage{algpseudocode}%
\usepackage{listings}%
\usepackage{soul}
%Mine
\usepackage{verbatim}
\usepackage{amssymb}
\usepackage{chngcntr}
%\usepackage{caption}

\usepackage[font=small]{caption}  
%%%%

\theoremstyle{thmstyleone}
\newtheorem{theorem}{Theorem}
\newtheorem{proposition}[theorem]{Proposition}% 


\theoremstyle{thmstyletwo}%
\newtheorem{example}{Example}%
\newtheorem{remark}{Remark}%
\theoremstyle{thmstylethree}%
\newtheorem{definition}{Definition}%
\raggedbottom

\begin{document}
\title[Article Title]{All-in-One Analog AI Accelerator: On-Chip Training and Inference with Conductive-Metal-Oxide/HfO\textsubscript{\textnormal{x}} ReRAM Devices}

\author*[1]{\fnm{Donato Francesco} \sur{Falcone}}\email{dof@zurich.ibm.com}
\author[1]{\fnm{Victoria} \sur{Clerico}}
\author[1]{\fnm{Wooseok} \sur{Choi}}
\author[1]{\fnm{Tommaso} \sur{Stecconi}}
\author[1]{\fnm{Folkert} \sur{Horst}}
\author[1]{\fnm{Laura} \sur{Bégon-Lours}}
\author[1]{\fnm{Matteo} \sur{Galetta}}
\author[1]{\fnm{Antonio} \sur{La Porta}}
\author[2,3]{\fnm{Nikhil} \sur{Garg}}
\author[2,3]{\fnm{Fabien} \sur{Alibart}}
\author[1]{\fnm{Bert Jan} \sur{Offrein}}
\author[1]{\fnm{Valeria} \sur{Bragaglia}}


\affil[1]{\orgname{IBM Research - Europe}, \orgaddress{\city{Rüschlikon}, \postcode{8803}, \state{Zürich}, \country{Switzerland}}}

\affil[2]{\orgdiv{Institut Interdisciplinaire d’Innovation Technologique (3IT)}, \orgname{Université de Sherbrooke}, \orgaddress{\city{Sherbrooke}, \postcode{QC J1K 0A5}, \state{Quebec}, \country{Canada}}}

\affil[3]{\orgdiv{Institute of Electronics, Microelectronics and Nanotechnology (IEMN)}, \orgname{Université de Lille}, \orgaddress{\city{Villeneuve d’Ascq}, \postcode{59650}, \country{France}}}

%TO DO: CAMBIARE 512X512 IN 256X256
%TO DO: MODIFICARE LA FORMULA CHE PUO ESSERE MISLEADING IN FIG. 3E, vedi la TESI
\abstract{Analog in-memory computing is an emerging paradigm designed to efficiently accelerate deep neural network workloads. Recent advancements have focused on either inference or training acceleration. However, a unified analog in-memory technology platform—capable of on-chip training, weight retention, and long-term inference acceleration—has yet to be reported. This work presents an all-in-one analog AI accelerator, combining these capabilities to enable energy-efficient, continuously adaptable AI systems. The platform leverages an array of analog filamentary conductive-metal-oxide (CMO)/HfO\textsubscript{\textnormal{x}} resistive switching memory cells (ReRAM) integrated into the back-end-of-line (BEOL). The array demonstrates reliable resistive switching with voltage amplitudes below 1.5~V, compatible with advanced technology nodes. The array’s multi-bit capability (over 32 stable states) and low programming noise (down to 10~nS) enable a nearly ideal weight transfer process, more than an order of magnitude better than other memristive technologies. Inference performance is validated through matrix-vector multiplication simulations on a 64×64 array, achieving a root-mean-square error improvement by a factor of 10 at 1 second and 3 at 10 years after programming, compared to state-of-the-art. Training accuracy closely matching the software equivalent is achieved across different datasets. The CMO/HfO\textsubscript{\textnormal{x}} ReRAM technology lays the foundation for efficient analog systems accelerating both inference and training in deep neural networks.}


\keywords{In-memory computing, Analog ReRAM, Deep Neural Networks, Training, Inference}

\maketitle
\section{Introduction}\label{sec1}
Modern computing systems rely on von Neumann architectures, where instructions and data must be transferred between memory and the processing unit to perform computational tasks. This data transfer, particularly recurrent and massive in prominent artificial intelligence (AI)-related workloads, results in significant latency and energy overhead \cite{epoch2024datamovementbottlenecksscalingpast1e28flop}. Digital AI accelerators address this challenge through computational parallelism, bringing memory closer to the processing units, and exploiting application-specific processors \cite{Jouppi2017,Sze2017}. This approach has demonstrated to bring significant improvements in throughput and efficiency for running deep neural networks (DNNs) \cite{Haensch2019}, but the physical separation between memory and compute units persists. Analog in-memory computing (AIMC) \cite{Sebastian2020} is a promising approach to eliminate this separation and so achieve further power and efficiency improvements in deep-learning workloads \cite{Mutlu2019}, by enabling some arithmetic and logic operations to be performed directly at the location where the data is stored. By mapping the weights of DNNs onto crossbar arrays of resistive devices and by leveraging Ohm's and Kirchhoff’s physical laws, matrix-vector multiplications (MVMs)—the most recurrent operation in AI-workloads \cite{Tsai2023}—are performed in memory with $O(1)$ time complexity \cite{Sebastian2020,Burr2017,Haensch2019}. Recent demonstrations of the AIMC paradigm have primarily focused on accelerating the inference step of digitally trained DNNs \cite{Wan2022,Yao2020, Ambrogio2023,Hermes64cores}. However, the increasing computing demands of modern AI models make the training phase orders of magnitude more costly in time and expenses than inference, highlighting the need for efficient hardware acceleration based on the AIMC paradigm. For instance, Gemini 1.0 Ultra required over $5 \cdot 10^{25}$ floating-point operations (FLOPs), approximately 100 days, \(\mathrm{24 \, MW}\) of power, and an estimated cost of 30 million dollars for training \cite{Gemini}. 
\\
\\
Analog training acceleration imposes even more stringent requirements on resistive devices. In addition to inference (i.e., the forward pass), the back-propagation of errors, gradient computation, and weight update steps must be performed during the learning phase. However, in the digital domain updating the weights of a matrix of size NxN requires $O(N^2)$ digital operations, leading to a significant drop in efficiency and speed. Beyond the forward pass, the AIMC approach enables acceleration of (1) backward pass through MVMs transposing the inputs and outputs, (2) gradient computation, and (3) the weight update through gradual bidirectional conductance changes upon external stimuli, all with $O(1)$ time complexity. To achieve this, the ideal analog resistive device should exhibit bidirectional, linear, and symmetric conductance updates in response to an open-loop programming pulse scheme (i.e., without the need for verification following each pulse) \cite{Haensch2019,Woo2018}. 
Promising technologies include redox-based resistive switching memory (ReRAM) \cite{Yin2020,Zahoor2020}, electro-chemical random access memory (ECRAM) \cite{Tang2019}, and capacitive weight elements \cite{Li2018}. Addressing the various non-idealities of these technologies \cite{Ielmini2016} requires the co-optimization of technology and designated training algorithms.
\\
\\
Gokmen et al. \cite{Gokmen2016} proposed an efficient, fully parallel approach that leverages the coincidence of stochastic voltage pulse trains to carry out outer-product calculations and weight updates entirely within memory, in $O(1)$ time complexity. To relax the device symmetry requirements, a novel training algorithm, known as Tiki-Taka, was designed based on this parallel scheme \cite{Gokmen2020}. The primary advantage of the Tiki-Taka approach lies in reduced device symmetry constraints across the entire conductance (G) range, focusing instead on a localized symmetry point where increases and decreases in G are balanced \cite{Gokmen2020}. More recently, the Tiki-Taka version 2 (TTv2) algorithm was demonstrated in hardware \cite{Gong2022} on small-scale tasks using optimized analog ReRAM technology in a 6-Transistor-1ReRAM unit cell crossbar array configuration. However, TTv2 faces some convergence issues when the reference conductance is not programmed with high precision \cite{Rasch2024Agad}. Analog gradient accumulation with dynamic reference (AGAD) learning algorithm (i.e., TTv4) was proposed to overcome the reference conductance limitation, providing enhanced and robust performance \cite{Rasch2024Agad}. 
\\
\\
From a technology perspective, the addition of an engineered conductive-metal-oxide (CMO) layer in a conventional HfO\textsubscript{\textnormal{x}}-based ReRAM metal/insulator/metal (M/I/M) stack has been shown to improve switching characteristics in terms of the number of analog states, stochasticity, symmetry point, and endurance, compared to conventional M/I/M technology \cite{Stecconi2024, Davide_DRC,NHFalcone2024}. However, while CMO/HfO\textsubscript{\textnormal{x}} ReRAM technology has proven to meet all the fundamental device criteria for on-chip training \cite{Stecconi2024}, array-level assessment and BEOL integration remain unexplored. Furthermore, although accelerating DNN training using AIMC is more challenging than inference, a unified technology platform capable of performing on-chip training, retaining the weights, and enabling long-term inference acceleration has yet to be reported. 
\\
\\
This work fills this gap by demonstrating an all-in-one AI accelerator based on CMO/HfO\textsubscript{\textnormal{x}} ReRAM technology, able to perform analog acceleration of both training and long-term inference operations. Such an integrated approach paves the way for highly autonomous, energy-efficient, and continuously adaptable AI systems, opening new paths for real-time learning and inference applications. The flowchart in Fig. \ref{fig1}a illustrates the all-in-one analog training and inference challenge addressed in this study. To achieve this goal, CMO/HfO\textsubscript{\textnormal{x}} ReRAM devices, integrated into the BEOL of a \(\mathrm{130 \, nm}\) complementary metal-oxide-semiconductor (CMOS) technology node with copper interconnects (see "Methods" section "Device fabrication" for details), are arranged in an array architecture using a 1T1R unit cell. Compared to implementations that use multiple transistors to control the resistive switching, the 1T1R unit cell maximizes memory density, which is crucial for storing large AI models on a single chip. Fig. \ref{fig1}b shows an image of the all-in-one analog ReRAM-based AI core used in this work, with the corresponding 8x4 array architecture and the schematic of the BEOL integrated 1T1R cells. The CMO/HfO\textsubscript{\textnormal{x}} ReRAM array is first studied in a quasi-static regime by statistically characterizing the devices' electro-forming step and quasi-static switching response. A physical 3D finite-element model (FEM) is developed to represent the geometry of the conductive filament and analytically describe the charge transport mechanism within these cells. Subsequently, the weight transfer accuracy and conductance relaxation are experimentally characterized on the 8x4 array. These measurements enable the demonstration of the core's inference capabilities, validated through representative MVM accuracy simulations on a 64×64 array. After demonstrating the MVM accuracy of the CMO/HfO\textsubscript{\textnormal{x}} ReRAM core, analog switching experiments using an open-loop identical pulse scheme demonstrated the suitability
of the same core for analog on-chip training acceleration. To assess the training performance, a realistic device model was used in the simulation, accounting for measured characteristics such as non-linear and asymmetric switching behavior, as well as inter- and intra-device variabilities. The training performance was validated using AGAD on fully connected and long short-term memory (LSTM) neural networks, demonstrating scalability from small to large-scale neural networks.
\begin{figure}[H]
\centering
\includegraphics[width=1.0\textwidth]{./Fig1.pdf}
\caption{\textbf{All-in-one AIMC challenge.} \textbf{a} Schematic representation of the key steps required to perform on-chip training and inference with analog acceleration. Each step is executed using a crossbar array of resistive devices. \textbf{b} CMO/HfO\textsubscript{\textnormal{x}} ReRAM AI core used in this work, consisting of an 8×4 array of 1T1R unit cells. From a fabrication perspective, each ReRAM cell is integrated into the BEOL of a \(\mathrm{130 \, nm}\) NMOS transistor with copper interconnects.}
\label{fig1}
\end{figure}

\section{Results}\label{sec2}
\subsection{Quasi-static array characterization and modelling}\label{subsecMemoryCore}
The quasi-static electrical characterization and analytical transport modelling of the 8x4 CMO/HfO\textsubscript{\textnormal{x}} ReRAM array are presented here.
\subsubsection{Filament forming} Fig. \ref{fig2}a shows the current-voltage characteristic of the ReRAM devices in the array, undergoing a soft-dielectric breakdown process, commonly referred to as forming \cite{Padovani2015}. During this step, a quasi-static voltage sweep up to \(\mathrm{3.6 \, V}\) is applied to the top electrode of each ReRAM device, while grounding the source and driving the gate of the corresponding NMOS selector with a constant \(V_\mathrm{G} = \mathrm{1.2 \, V}\) ensuring current compliance. This process leads to the formation of a highly defect-rich conductive filament in the HfO\textsubscript{\textnormal{x}} layer. Due to the high oxygen vacancy ($\rm V_{\rm O}^{\rm \cdot \cdot}$ in Kröger–Vink notation \cite{Kroger}) formation energy, ranging from \(\mathrm{2.8 \, eV}\) to \(\mathrm{4.6 \, eV}\) in HfO\textsubscript{\textnormal{x}} depending on the stoichiometry \cite{Padovani2012,Padovani2013}, defect generation occurs with statistical relevance only during the forming sweep within the HfO\textsubscript{\textnormal{x}} layer \cite{NHFalcone2024}. The subsequent application of a negative voltage sweep up to \(-1.4 \, \mathrm{V}\), with a constant \(V_\mathrm{G} = \mathrm{3.3 \, V}\), induces a radial redistribution of the defects within the CMO layer, consistent with findings in literature \cite{NHFalcone2024}. This process leads to an increase of the ReRAM conductance and is modelled by considering a constant average radius of the conductive filament, with a local electrical conductivity increase of the CMO layer on top of the filament. Refer to the "Methods" section "ReRAM forming modelling" for details. To determine the experimental ReRAM forming voltage, the voltage drop across the NMOS selector must be subtracted from the voltage applied to the 1T1R cell. Fig. \ref{fig2}b shows the experimental transistor output characteristic, from which the resistance in the triode region at \(V_\mathrm{G} = \mathrm{1.2 \, V}\) is measured and used to extract the distribution of \(V_{\mathrm{forming}}^{\mathrm{ReRAM}} \) within the CMO/HfO\textsubscript{\textnormal{x}} ReRAM array (reported in Fig. \ref{fig2}c). Refer to the ”Methods” section ”ReRAM forming voltage extraction” for details. The highly reproducible CMO/HfO\textsubscript{\textnormal{x}} ReRAM forming step exhibits a 100\% yield with a narrow distribution (\(\sigma = \mathrm{75 \, mV}\)) around \(V_{\mathrm{forming}}^{\mathrm{ReRAM}} \approx \mathrm{3.2 \, V}\), making it suitable for integration with \(\mathrm{130 \, nm}\) NMOS transistors rated for \(\mathrm{3.3 \, V}\) operation. 

\subsubsection{Resistive switching and polarity optimization}
The underlying physical mechanism behind the resistive switching in analog CMO/HfO\textsubscript{\textnormal{x}} ReRAM devices has been recently unveiled \cite{NHFalcone2024, IMWFalcone2023, ESSDERC24_Galetta}. The current transport is explained by a trap-to-trap tunneling process, and the resistive switching by a modulation of the defect density within the conductive sub-band of the CMO that behaves as electric field and temperature confinement layer. In these works, the analog CMO/HfO\textsubscript{\textnormal{x}} ReRAM device shows a counter-eightwise (C8W) switching polarity, according to the definition proposed in literature \cite{Dittmann2021}. The intrinsically gradual \textit{reset} (from low to high resistance) process, marked by a temperature decrease, occurs during the positive voltage sweep on the ReRAM top electrode, while the exponential \textit{set} (from high to low resistance) process, involving a rapid temperature increase, occurs on the negative side \cite{NHFalcone2024}. However, when arranged in a 1T1R cell configuration based on an NMOS selector, the C8W switching polarity prevents direct control of the transistor's \(V_\mathrm{GS}\) during the exponential \textit{set} process. This results in reduced switching uniformity, which is critical for the array-level adoption of analog CMO/HfO\textsubscript{\textnormal{x}} ReRAM devices. 
\\
\\
For this reason, in this work the analog CMO/HfO\textsubscript{\textnormal{x}} ReRAM devices within the 1T1R cells are optimized to exhibit the desirable 8W switching polarity by extending the current switching model in literature \cite{NHFalcone2024}. To achieve this, following the positive forming and the initial negative voltage sweep, each device in the array is subjected to a forward and backward voltage sweep from 0 to \(-1.5 \, \mathrm{V}\). During this process, oxygen vacancies in the CMO layer radially spread outward, depleting the CMO defect sub-band within a half-spherical volume at the interface with the conductive filament, leading to a \textit{reset} process (Fig. \ref{figS3} in Supplementary Information shows the experimental array's response). Conversely, a voltage sweep from 0 to \(1.3 \, \mathrm{V}\) enables the migration of oxygen vacancies in the CMO layer in the reverse direction, resulting in a \textit{set} transition, controlled by the transistor gate. For each 1T1R cell within the 8x4 array, Fig. \ref{fig2}d shows 5 quasi-static I-V cycling sweeps to experimentally assess the reproducibility of the optimized 8W switching polarity. The electronic transport in both the low-resistive state (LRS) and high-resistive state (HRS) is modelled as a trap-to-trap tunneling process, described by the Mott and Gurney analytical formulation. The physical parameters characterizing the transport in both LRS and HRS ($N_{\rm e}$, $\Delta E_{\rm e}$, $a_{\rm e}$, $\sigma_{\rm CMO}$ and $r_{\rm CF}$) are shown in Fig. \ref{fig2}d. Refer to the ”Methods” section ”Analytical ReRAM transport modelling” for details on the LRS and HRS modelling. Fig. \ref{fig2}e illustrates the cumulative probability distribution of the experimental LRS and HRS within the array, demonstrating device-to-device uniformity and a resistance ratio HRS/LRS of approximately 15, with absolute switching voltages \(\leq \mathrm{1.5 \, V}\). The excellent uniformity of the forming and the optimized 8W-cycling characteristics set the groundwork for AIMC-based inference and training AI-accelerators using the CMO/HfO\textsubscript{\textnormal{x}} ReRAM technology.
\begin{figure}[H]
\centering
\includegraphics[width=1.0\textwidth]{./Fig2.pdf}
\caption{ \textbf{ReRAM array quasi-static electrical characterization and modelling.} \textbf{a} (1) Experimental positive forming sweeps (with \(V_\mathrm{G} = \mathrm{1.2 \, V}\)) of the 8x4 CMO/HfO\textsubscript{\textnormal{x}} ReRAM devices in the array. This process results in an average filament radius of \(11\, \mathrm{nm}\) in the HfO\textsubscript{\textnormal{x}} layer. (2) Negative voltage sweeps (with \(V_\mathrm{G} = \mathrm{3.3 \, V}\)) to enable defect redistribution within the CMO layer, resulting in an increase in the conductance of the ReRAM cells. A representative sweep is shown in black. The insets illustrate a schematic representation of the defect arrangement within the stack. \textbf{b} Experimental NMOS transistor output characteristic, with \(V_\mathrm{G}\) up to \(\mathrm{3 \, V}\). \textbf{c} Experimental ReRAM forming voltage distribution measured from the CMO/HfO\textsubscript{\textnormal{x}} ReRAM array. The experimental data used to extract the distribution are represented as green points. \textbf{d} Superposition of 5 I-V quasi-static 8W-cycles (in blue) for each of the 32 devices in the array, using \(V_\mathrm{set} = \mathrm{1.3 \, V}\), \( V_\mathrm{G} = \mathrm{1.1 \, V}\) and \(V_\mathrm{reset} = \mathrm{-1.5 \, V}\), \(V_\mathrm{G} = \mathrm{3.3 \, V}\) for \textit{set} and \textit{reset} processes, respectively. The analytical trap-to-trap tunneling model effectively captures the electron transport in both the LRS and HRS (yellow dashed lines). The physical parameters characterizing the transport, extracted from the model, and a schematic representation of the defect distribution, are presented for both resistive states. \textbf{e} Cumulative probability distributions for both LRS and HRS. For each array cell, the average resistance over 5 I-V cycles in LRS and HRS is defined at a read voltage of \(\mathrm{0.2 \, V}\).}
\label{fig2}
\end{figure}

\subsection{Analog inference with CMO/HfO\textsubscript{\textnormal{x}} ReRAM core}\label{subsecAnalogInference}
Here, the experimental characterization of the key metrics of the CMO/HfO\textsubscript{\textnormal{x}} ReRAM array relevant to inference performance is presented.
Specifically, the continuous conductance tuning capability is demonstrated over a range spanning approximately one order of magnitude. The trade-off between weight transfer programming noise of CMO/HfO\textsubscript{\textnormal{x}} ReRAM devices and number of required iterations for programming convergence is analyzed across different acceptance ranges. Furthermore, conductance relaxation—defined as the change in conductance over time after programming—is characterized. Finally, the combined impact of weight transfer, conductance relaxation, limited input/output quantization of the digital-to-analog converter (DAC) and analog-to-digital converter (ADC), and IR drop on the array wires is evaluated with respect to MVM accuracy.

\subsubsection{Weight transfer accuracy}\label{subsubsecMultibit}
In memristor-based AIMC inference accelerators, pre-trained normalized weights are initially mapped into target conductances and subsequently programmed into hardware in an iterative process known as weight transfer. This iterative process, which stops once the programmed conductance converges to the target value within a defined acceptance range, inherently introduces an error due to the analog nature of conductance weights. This error, described by a normal distribution with the standard deviation referred to as programming noise ($\sigma_{\rm prog}$), leads to a drop in MVM accuracy. To quantify this non-ideality, the non-volatile multi-level capability of the CMO/HfO\textsubscript{\textnormal{x}} ReRAM array is characterized. Fig. \ref{fig3}a shows the experimental cumulative distribution of conductance values for 35 representative levels, with all states sharply separated and without any overlap. Fig. \ref{fig3}b shows a schematic representation of the closed-loop (i.e., program-verify) scheme, where identical \textit{set} and \textit{reset} pulse trains are employed to program each ReRAM cell to its target conductance within a desired acceptance range (see ”Methods” section ”Identical-pulse closed-loop scheme” for details). Selecting programming conditions involves a fundamental trade-off: a narrower acceptance range can improve programming precision by reducing programming noise, but it increases the number of iterations required for convergence (see Fig. \ref{fig3}d). Besides the longer programming time, other non-idealities to consider when choosing the acceptance range are (1) the conductance relaxation immediately after programming, which is characterized in \ref{subsubsecRelaxation} for CMO/HfO\textsubscript{\textnormal{x}} ReRAM devices, and (2) read noise, which has already been characterized between 0.2\% and 2\% of \textit{G}\textsubscript{\textnormal{target}} for CMO/HfO\textsubscript{\textnormal{x}} ReRAM devices \cite{Davide_DRC} within a similar conductance range used in this work. 
\\
\\
The trade-off between the programming noise and the number of iterations is characterized for two representative acceptance range intervals: 0.2\% and 2\% of \textit{G}\textsubscript{\textnormal{target}}, respectively. Fig. \ref{fig3}c illustrates the experimental number of pulses needed to converge to the \textit{G}\textsubscript{\textnormal{target}} using the two representative acceptance ranges. On average, each cell requires approximately 11 and 89 \textit{set}/\textit{reset} pulses for acceptance ranges of 2\% and 0.2\% of \textit{G}\textsubscript{\textnormal{target}}, respectively. Since the acceptance range is defined as a percentage of \textit{G}\textsubscript{\textnormal{target}}, the number of iterations required for convergence is almost independent of the target conductance value. In the Supplementary Information, Fig. \ref{figS5}a shows the experimental cumulative distribution of conductance values for the same 35 representative levels presented in Fig. \ref{fig3}a, but using 2\% \textit{G}\textsubscript{\textnormal{target}} as acceptance range. The standard deviation of the representative conductance levels is extracted and fitted as a linear function of the target conductance (dashed lines), as shown in Fig. \ref{fig3}e, for both acceptance ranges. For all conductance levels, a standard deviation of less than 0.1~\textmu S (1~\textmu S) is achieved considering 0.2\% \textit{G}\textsubscript{\textnormal{target}} (2\% \textit{G}\textsubscript{\textnormal{target}}) as the acceptance range. This is more than one order of magnitude lower compared to other memristive technologies, such as phase-change memory (PCM) arrays, targeting similar conductance ranges \cite{Joshi2020, Tsai2019, LeGallo18}. These results demonstrate that CMO/HfO\textsubscript{\textnormal{x}} ReRAM cells achieve an almost ideal weight transfer during programming, enabling the distinction of more than 32 states (5 bits).

\begin{figure}[H]
\centering
\includegraphics[width=1.0\textwidth]{./Fig3.pdf}
\caption{\textbf{Weight transfer characterization.} \textbf{a} Cumulative distributions of 35 conductance states obtained using an identical-pulse closed-loop scheme with a 0.2\% \textit{G}\textsubscript{\textnormal{target}} acceptance range. For each distribution, the entire CMO/HfO\textsubscript{\textnormal{x}} ReRAM array was programmed to the corresponding \textit{G}\textsubscript{\textnormal{target}}, and the conductance values measured during the final closed-loop iteration (during programming) is reported. Each dot represents a 1T1R cell. \textbf{b} An example sequence of the identical-pulse closed-loop programming scheme utilized in this work. \textbf{c} Experimental number of closed-loop iterations as a function of \textit{G}\textsubscript{\textnormal{target}} for the two representative acceptance ranges. Each semitransparent point represents a 1T1R cell, the opaque points represent the average number of iterations per \textit{G}\textsubscript{\textnormal{target}}, and the horizontal dashed line indicates the overall average of the opaque points. \textbf{d} Graphical representation of the trade-off between programming noise and the number of iterations required for convergence, as a function of the acceptance range. \textbf{e} Experimental programming noise as a function of \textit{G}\textsubscript{\textnormal{target}} for the two representative acceptance ranges. 
Each point represents the standard deviation of the normal distribution measured across the entire array. The dashed lines in black indicate the corresponding linear fits.}
\label{fig3}
\end{figure}

\subsubsection{Conductance relaxation and matrix-vector multiplication accuracy}\label{subsubsecRelaxation}
In addition to the excellent weight transfer accuracy during programming as presented in the previous section, the characterization of temporal conductance relaxation is critical to estimate the MVM accuracy over time. In analog ReRAM devices, a significant conductance relaxation has been observed immediately after programming (within 1 second) \cite{Wan2022}. Following this initial abrupt conductance change, the relaxation process slows considerably \cite{Zhao2018,Wan2022}. The physical cause of retention degradation is attributed to the Brownian motion of defects in the resistive switching layer \cite{Zhao2018}. In this section, the conductance relaxation of the CMO/HfO\textsubscript{\textnormal{x}} ReRAM array after programming is characterized. Fig. \ref{fig4}a shows the relaxation of the distributions previously reported in Fig. \ref{fig3}a, approximately 10 minutes after programming. The 35 levels remain distinguishable 10 minutes after programming, with an average overlap of 9.6\% between adjacent states gaussians, while the average standard deviation of the distributions increases to 0.6~\textmu S, showing almost independence from the \textit{G}\textsubscript{\textnormal{target}} (see Fig. \ref{fig4}b). 
\\
\\
The stability of the CMO/HfO\textsubscript{\textnormal{x}} ReRAM conductance states is further assessed on a longer time-scale, up to 1 hour. To achieve so, a linearly spaced \textit{G}\textsubscript{\textnormal{target}} vector within the experimental conductance range of 10~\textmu S to 90 \textmu S is defined, with a fine step of 0.2~\textmu S (400 points). Each \textit{G}\textsubscript{\textnormal{target}} value is programmed into a single ReRAM device within the array. Due to the size mismatch between the array (32 devices) and the \textit{G}\textsubscript{\textnormal{target}} vector (size 400), multiple measurement batches are needed. Fig. \ref{fig4}c shows the experimental relaxation of the 400 programmed states within the entire conductance window, 1 second and 1 hour after programming, executed with the closed-loop scheme (see ”Methods” section
”Identical-pulse closed-loop scheme” for details) and with a 0.2\% \textit{G}\textsubscript{\textnormal{target}} acceptance range. The exhibited conductance error induced by the relaxation process after 1 hour, computed as \(G_{\mathrm{1h}} - G_{\mathrm{prog.}}\), is plotted as a function of the programmed conductances in Fig. \ref{fig4}d. After 1 hour, although both positive and negative relaxation errors are recorded, an average decrease in conductance is observed across all programmed states, with a relaxation error averaging around -0.7~\textmu S. This highlights that the relaxation process in CMO/HfO\textsubscript{\textnormal{x}} ReRAM devices leads, on average, to a decrease in the mean and an increase in the standard deviation of the Gaussian distributions regardless of the initial conductance state. Since the absolute magnitudes of the mean decrease and the standard deviation increase are independent of \textit{G}\textsubscript{\textnormal{target}}, an extended characterization of the relaxation process up to 1 week is conducted for a representative conductance state (50~\textmu S). To achieve this, the array's CMO/HfO\textsubscript{\textnormal{x}} ReRAM devices are programmed using the identical-pulse closed-loop scheme to \textit{G}\textsubscript{\textnormal{target}} of 50~\textmu S, with a 0.2\% \textit{G}\textsubscript{\textnormal{target}} acceptance range. Fig. \ref{fig4}e illustrates the experimental array relaxation over 1 week. The insets display the evolution of both the mean and standard deviation as a function of the logarithm of time after programming (in seconds), using a linear fit to predict the conductance distribution over a 10-year period. 
\\
\\
To assess the accuracy of analog MVM, a comprehensive set of non-idealities—both intrinsic to CMO/HfO\textsubscript{\textnormal{x}} ReRAM devices and at the architecture level—is considered, including finite programming resolution with 0.2\% \textit{G}\textsubscript{\textnormal{target}} acceptance range, conductance relaxation, limited ADC and DAC quantization, and IR-drop across array wires. Fig. \ref{fig4}f shows the hardware-aware simulation results of the analog MVM using CMO/HfO\textsubscript{\textnormal{x}} ReRAM cells, projected for up to 10 years from programming, compared to the expected floating-point (FP) result. The results are generated using a single 64×64 normally distributed random weight matrix and 100 normally distributed input vectors within the range [-1, 1] (see ”Methods” section ”HW-aware simulation of analog MVM” for details). Considering the input and output quantization of 6-bit and 8-bit respectively, the inset illustrates the time evolution of the root-mean-square error (RMSE) of the simulated analog MVM compared to the FP expected result. These results show that the CMO/HfO\textsubscript{\textnormal{x}} ReRAM core enables accurate MVM operations, achieving an RMSE ranging from 0.06 at 1 second to 0.2 at 10 years after programming, compared to the ideal FP case. Fig. \ref{figS6} in the Supplementary Information illustrates the impact of IR-drop and input/output quantization on the RMSE of an MVM performed on a 64×64 array. Over short time scales (within 1 hour), the primary accuracy bottleneck is the limited input/output quantization of 6-bit and 8-bit, respectively. Over longer periods, relaxation effects become the dominant source of non-ideality. In a larger 256×256 array, IR-drop emerges as the main accuracy bottleneck for analog MVM. Compared to the analog ReRAMs studied by Wan et al. \cite{Wan2022}, who report an experimentally determined RMSE of approximately 0.58 under conditions similar to those of this work, CMO/HfO\textsubscript{\textnormal{x}} ReRAMs demonstrate a potential improvement in MVM accuracy by a factor of 10 and 3, 1 second and 10 years after programming, respectively. The excellent MVM accuracy results demonstrate the suitability of CMO/HfO\textsubscript{\textnormal{x}} ReRAM devices for long-term AI inference applications, and lay the foundation for AI training acceleration, where short-term forward and backward MVMs are key steps.

\begin{figure}[H]
\centering
\includegraphics[width=1\textwidth]{./Fig4.pdf}
\caption{\textbf{Conductance relaxation and MVM accuracy.} \textbf{a} Probability density distributions of 35 conductance states approximately 10 minutes after programming. The black areas between adjacent Gaussian distributions represent the overlap of their tails. On average, an overlap of 9.6\% is observed after 10 minutes. \textbf{b} The standard deviations of the 35 conductance states during programming (in purple) and 10 minutes after it (light blue). \textbf{c} Relaxation of 400 conductance states, with one device per G-state, measured 1 second and 1 hour after programming. \textbf{d} Relaxation error 1 hour after programming. A negative and nearly G-independent average error (dashed line) indicates that relaxation in CMO/HfO\textsubscript{\textnormal{x}} ReRAMs tends toward a slight conductance decrease and is state-independent. \textbf{e} Experimental array relaxation of a representative 50 \textmu S state, up to 1 week after programming with 0.2\% \textit{G}\textsubscript{\textnormal{target}} acceptance range. The experimental data used to extract the distributions are represented as points aligned to the y=0 horizontal axis. Insets show the time dependence of the mean and standard deviation. Dashed blue lines represent the conditions during programming, once the convergence to \textit{G}\textsubscript{\textnormal{target}} is reached, while a linear fit (green dashed line) extrapolates the distribution 10 years after programming (dashed black line).
\textbf{f} Analog MVM accuracy simulations using a 64x64 CMO/HfO\textsubscript{\textnormal{x}} ReRAM array as a function of time after programming (indicated by different colors). The inset shows the expected RMSE compared to the ideal FP result.  Experimental programming noise, conductance relaxation, limited input/output quantization and IR-drop are considered in this assessment.}
\label{fig4}
\end{figure}





\subsection{Analog training with CMO/HfO\textsubscript{\textnormal{x}} ReRAM core}\label{subsecAnalogTraining}
To efficiently tackle deep learning workloads, the analog AI accelerator must not only perform forward and backward passes (MVMs), but most importantly, allow for weight updates \cite{aihwkit}. During backpropagation, the synaptic weights are modified according to the gradient of the corresponding layer. Therefore, the device conductance must be gradually modified in both positive and negative directions to represent analog weight changes. Analog CMO/HfO\textsubscript{x} ReRAM arrays not only allow for bidirectional conductance updates, but additionally enable parallel weight updating by following a stochastic open-loop pulse scheme \cite{Gokmen2016,Gokmen2020}. Remarkably, the parallel and open-loop update scheme significantly accelerates training compared to serial and closed-loop methods, providing efficiency gains of several orders of magnitude and advantages in system design complexity \cite{Chen2023}. In this section, the bidirectional open-loop response of the CMO/HfO\textsubscript{x} ReRAM array, required during Tiki-Taka training, is characterized. Specifically, the analog conductance potentiation, depression and symmetry point are measured. Subsequently, the devices' responses are statistically reproduced in the open-source 'aihwkit' simulation platform developed by IBM \cite{aihwkit}. Finally, this hardware-aware device model, which includes device variabilities, is used to simulate the training of representative neural networks using the AGAD learning algorithm. This novel analog training algorithm relaxes the symmetry requirements of previous Tiki-Taka versions by incorporating additional digital computations on-the-fly \cite{Rasch2024Agad}.

\subsubsection{Open-loop ReRAM array characterization}\label{openloop_characterization}
Fig. \ref{fig5}a shows the experimental conductance change of a representative CMO/HfO\textsubscript{x} ReRAM device within the array upon applying identical-voltage pulse trains with alternating polarity in batches of 400. Subsequently, a sequence of 500 pulses with alternating polarity, consisting of 1-pulse-up followed by 1-pulse-down, is applied to experimentally determine the symmetry point.
The same open-loop programming scheme, with \(V_{\rm set} = 1.35 \, \mathrm{V}\) (\(V_{\rm G} = 1.4 \, \mathrm{V}\)) and \(V_{\rm reset} = -1.3 \, \mathrm{V}\) (\(V_{\rm G} = 3.3 \, \mathrm{V}\)), each lasting 2.5~\textmu s, is applied to all devices in the 8x4 array. The \textit{set}/\textit{reset} pulse width is limited by the experimental setup, although previous work has demonstrated CMO/HfO\textsubscript{\textnormal{x}} ReRAM switching with pulses as short as \(60 \, \mathrm{ns}\) \cite{Davide_DRC}. Due to inter-device (device-to-device) and intra-device (cycle-to-cycle) variabilities, the experimental response of each device to a given number of identical pulses exhibits some level of variability (see Fig. \ref{figS7} in the Supplementary Information). Therefore, for each pulse, a Gaussian distribution of the measured conductance states among the devices is extracted. For statistical relevance, Fig. \ref{fig5}b shows the experimental standard deviation of the array response to the open-loop scheme as a function of the pulse number, represented in grey. To realistically assess the accuracy of analog training with CMO/HfO\textsubscript{x} ReRAM devices, the key figures of merit of the device training characterization—such as the number of states, the symmetry point skew, and the noise-to-signal ratio (NSR)—are first extracted from experimental data, as defined below.%, and then used to feed the device model in the aihwkit.
\begin{align}
   \mathrm{N}_{\rm states} = \frac{G_{\rm max} - G_{\rm min}}{\overline{\Delta G_{\rm sp}}}
   \label{nstates}
\end{align}
\begin{align}
   \mathrm{SP}_{\rm skew} = \frac{G_{\rm max} - \overline{G_{\rm sp}}}{G_{\rm max} - G_{\rm min}}
   \label{skew}
\end{align}
\begin{align}
   \mathrm{NSR} = \frac{ \sigma_{\Delta G_{\rm sp}}  } {\overline{\Delta G_{\rm sp}}}
   \label{NSR}
\end{align}
$G_{\rm max}$ and $G_{\rm min}$ represent the maximum and minimum values extracted from the full conductance swings, while $\overline{G_{\rm sp}}$, $\overline{\Delta G_{\rm sp}}$ and $\sigma_{\Delta G_{sp}}$ denote the values of the mean conductance, mean conductance update and standard deviation of the conductance update at the symmetry point during the 1-pulse-up, 1-pulse-down procedure, respectively. Fig. \ref{fig5}c shows the experimental Gaussian distributions of these metrics for the 32 devices within the array. The results indicate an average of 22 states, with a range from 16 to 33. A shift in the $G_{\rm sp}$ (or SP$_{\rm skew}$) of 61\% is measured, reflecting a negative trend in the device asymmetry where the down response is steeper than the up response. An average NSR of 90\% among the devices is obtained, demonstrating the capability to discriminate between pulses up and down around the symmetry point. This parameter reflects the intrinsic noise on the device's response under identical conditions, highlighting an intra-device variation \cite{aihwkit}. Previous studies on similar CMO/HfO\textsubscript{\textnormal{x}} ReRAM systems \cite{Stecconi2024} extracted these metrics from isolated 1R devices using an optimized open-loop scheme tailored to each device. In contrast, this work demonstrates for the first time that a single open-loop identical pulse scheme enables reliable operation of the entire CMO/HfO\textsubscript{\textnormal{x}} 1T1R array, ensuring consistent performance across the array.
\begin{figure}[H]
\centering
\includegraphics[width=1.0\textwidth]{./Fig5.pdf}
\caption{\textbf{Open-loop array characterization for on-chip training.} \textbf{a} Bidirectional accumulative response and symmetry point of a representative device in the array. The top inset shows the open-loop identical pulse scheme used for the synaptic potentiation (red) and depression (blue). A conceptual illustration of the 8x4 CMO/HfO\textsubscript{\textnormal{x}} ReRAM array is depicted on the left. \textbf{b} Array statistical open-loop response to identical pulses. The grey area represents the standard deviation of the experimental Gaussian distributions, each corresponding to a specific pulse number. The inset shows a representative example of the experimental G-distribution at pulse number 1200. The raw data can be found in Fig. \ref{figS7} of the Supplementary Information. \textbf{c} The experimental probability densities of N$_{\rm states}$, SP$_{\rm skew}$ and NSR, respectively. The experimental data used to extract the distributions are represented as points aligned along the y=0 horizontal axis.}
\label{fig5}
\end{figure}

\subsubsection{Tiki-Taka training simulations}\label{subsubsecTT}
To perform realistic hardware-aware training simulations, the experimental device response is reproduced on software using the generalized soft bounds model implemented in the 'aihwkit' \cite{Frascaroli2018}, which better captures the bidirectional resistive switching behavior (see Fig. \ref{figS8} in Supplementary Information) and accounts for intra- and inter-device variabilities (see cycle-to-cycle and device-to-device variations in Fig. \ref{fig6}a). Additionally, Gaussian distributions are modelled based on parameters extracted from device characterization ($G_{\rm max}$, $G_{\rm min}$, $\Delta G_{\rm sp}$, NSR, SP$_{\rm skew}$) to account for device-to-device variability observed in the experimental characterization (see "Methods" section "Intra and inter-device variability" for details). This Gaussian fitting approach allows defining various device presets—characterized by the same model but with different parameter settings—to represent the synapses across the neural network. A realistic simulation setup is obtained by exclusively considering experimentally obtained parameters to reproduce the device trace (see "Methods" section "Generalized soft bounds model" for details). The device model is defined based on the observed conductance window and number of states, without assuming asymptotic behavior for an infinite number of pulses. This prevents overestimation of both the conductance window and the number of states (material states), enhancing the fidelity of the simulation.
\\
\\
To validate analog training with CMO/HfO\textsubscript{\textnormal{x}} ReRAM technology, a 3-layer fully connected (FC) neural network was trained on the MNIST dataset for image classification. In addition, the impact of the device's number of states, asymmetry, and noise-to-signal ratio on accuracy and convergence time is evaluated by simulating identical networks in which each property is individually enhanced, while keeping the others fixed at the experimentally derived values. Literature has shown that these device characteristics critically influence the convergence of analog training algorithms \cite{Rasch2024Agad}. Therefore, this method assesses the deviation of the current CMO/HfO\textsubscript{\textnormal{x}} ReRAM device properties from the ideal analog resistive device scenario. Moreover, to show the scalability of the CMO/HfO\textsubscript{\textnormal{x}} ReRAM technology to more computationally-intensive tasks, such as time series processing, a 2-layer long short-term memory (LSTM) network was trained on \textit{War and Peace} text sequences to predict the next token.  
Each network is initially trained using conventional stochastic gradient descent (SGD) based backpropagation with 32-bit FP precision, serving as the baseline performance. Fig. \ref{fig6}b illustrates the accuracy per epoch for the FP-baseline trained with SGD (in green) and the analog network trained using AGAD, evaluated under four different parameter settings: (1) properties extracted from the experimental array (in yellow), (2) reduced NSR to 20\% (in red), (3) average of N$_{\rm states}$ = 100 states (in blue), and (4) zero average device asymmetry (in orange). Using symmetrical device presets, i.e. with an average SP$_{\rm skew}$ of 50\%, improves accuracy by 0.7\% with respect to analog training with CMO/HfO\textsubscript{\textnormal{x}} ReRAM experimentally derived configuration (96.9\%), landing an accuracy of 97.6\%, a 0.7\% lower than the FP-SGD baseline (98.3\%). The other two configurations show less performance improvement, indicating more resilience of the AGAD-training to device's N$_{\rm states}$ and NSR.
\\
\\
Additionally, a 2-layer LSTM network with 64 memory states each (see Fig. \ref{fig6}c), is trained with the experimentally obtained configuration. The performance is measured using the exponential of the cross-entropy loss, i.e. the test perplexity metric, which quantifies the certainty of the token prediction. Results in Fig. \ref{fig6}d demonstrate the capabilities of the CMO/HfO\textsubscript{\textnormal{x}} ReRAM technology on more complex network architectures, such as LSTMs, and computationally demanding tasks, exhibiting performance comparable to the FP-equivalent, with an approximate 0.7\% difference in test perplexity.
\begin{figure}[H]
\centering
\includegraphics[width=1.0\textwidth]{./Fig6.pdf} 
\caption{\textbf{Device model and on-chip training simulations.} \textbf{a} Device presets generated using the generalized soft bounds model with experimentally extracted parameters of CMO/HfO\textsubscript{\textnormal{x}}  devices, including inter- and intra-device variabilities. \textbf{b} Training simulations of a 3-layer fully-connected neural network on MNIST (235K parameters), using 32-bit FP precision trained on SGD (in green). Analog training simulations were performed using AGAD considering the empirical distribution of the parameters (in yellow), enhanced NSR (in red), increased N$_{\rm states}$ (in blue), and symmetrical device configurations (in orange). \textbf{c} LSTM network architecture for text forecasting on the \textit{War and Peace} dataset (79K parameters). The architecture  considers a sequence length of 100 tokens and accounts for 2 layers with 64 hidden units. \textbf{d} Training results of the FP baseline (in green) and the analog training with AGAD on the experimental device configuration (in yellow). The training setup can be found in the Supplementary Information.}
\label{fig6}
\end{figure}



\section{Discussion}\label{sec13}
An all-in-one technology platform based on analog filamentary CMO/HfO\textsubscript{\textnormal{x}} ReRAM devices is presented. This platform addresses critical challenges in modern digital AI accelerators by overcoming the physical separation between memory and compute units. It enables the execution of forward and backward MVMs, along with weight updates and gradient computations, directly on a unified analog in-memory platform with $O(1)$ time complexity. This all-in-one approach fundamentally differs from DNN inference-only \cite{Wan2022} and training-only \cite{Stecconi2024,Nate2023} analog accelerators. In inference-only accelerators, DNN weights are trained in software (i.e., off-chip) using traditional digital CPUs or GPUs and then programmed once onto the analog AI hardware accelerator. In training-only accelerators, the long-term retention capabilities and overall MVM accuracy for large array tiles are not assessed. In this work, a novel all-in-one analog computing platform, capable of both on-chip training and inference acceleration, is unveiled.
\\
\\
The CMO/HfO\textsubscript{\textnormal{x}} ReRAM devices are integrated in the BEOL of a NMOS transistor platform in a scalable 1T1R array architecture. The highly reproducible forming step demonstrates compatibility with NMOS rated for \(\mathrm{3.3 \, V}\) operation, while the uniform quasi-static 8W-cycling characteristics, achieved with voltage amplitudes of less than $\pm$ \(\mathrm{1.5 \, V}\), exhibit a significant conductance window and a low off-state. The multi-bit capability of more than 32 states (5 bits), distinguishable after 10 minutes with less than 10\% overlap error, is experimentally demonstrated using an identical-pulse closed-loop scheme. The characterization of the weight transfer reveals record-low programming noise ranging from \(\mathrm{10 \, nS}\) to \(\mathrm{100 \, nS}\), more than one order of magnitude lower than that of other memristive technologies targeting similar conductance ranges \cite{Joshi2020, Tsai2019, LeGallo18}. 
Each conductance distribution exhibits a state-independent relaxation process over time, characterized by a slight shift of the mean toward lower conductance and an increase in the standard deviation. This independence of the relaxation process from the target conductance is advantageous for implementing effective compensation schemes in the future.
\\
\\
Realistic MVM simulations on a 64x64 array tile, considering CMO/HfO\textsubscript{\textnormal{x}} ReRAM device non-idealities such as finite weight transfer resolution, conductance relaxation, limited input/output quantization, and IR-drop across array wires, show an RMSE as low as 0.2 compared to the ideal FP-case, even 10 years after programming. This demonstrates that the CMO/HfO\textsubscript{\textnormal{x}} ReRAM devices improve analog MVM accuracy by a factor of 10 and 3 compared to the state of the art \cite{Wan2022}, 1 second and 10 years after programming, respectively. Although this study was performed at room temperature, previous characterization of a similar CMO/HfO\textsubscript{\textnormal{x}} ReRAM stack demonstrated the thermal stability of the analog states at high temperature (less than 4\% drift after 72 hours at 85~\textdegree C)\cite{Stecconi2024}. Future studies will focus on incorporating the experimental read noise of CMO/HfO\textsubscript{\textnormal{x}} ReRAM devices, characterized between 0.2\% and 2\% of \textit{G}\textsubscript{\textnormal{target}} within a similar conductance range as used in this work \cite{Davide_DRC}, into MVM accuracy simulations. Although read noise is not included in the MVM simulations of this study, no significant additional drop in MVM accuracy is anticipated. In fact, the magnitude of read noise is much smaller than that of the relaxation process and of the effect of reduced input/output quantization, which dominate the RMSE on different timescales. Furthermore, simulation results demonstrate the suitability of CMO/HfO\textsubscript{\textnormal{x}} ReRAM technology for large 256x256 array, with the IR-drop expected to become the primary accuracy bottleneck in this case.
\\
\\
Finally, the electrical response of the CMO/HfO\textsubscript{\textnormal{x}} ReRAM array to an open-loop scheme with identical pulses demonstrates the viability of this technology for on-chip training applications. A realistic device model, accounting for both inter- and intra-device variability, is derived from experimental data. Table \ref{table1} benchmarks the representative device model used in this work on the MNIST dataset against other approaches, highlighting its high fidelity in reproducing experimental device responses. 
\begin{table}[h]
\caption{Device model benchmarking: from simplified approaches to realistic non-ideality modeling}\label{tab1}%
\setlength{\tabcolsep}{2pt}  
\begin{tabular}{@{}p{2.3cm}p{1.8cm}p{2.2cm}p{2.3cm}p{1.4cm}p{1.2cm}p{1.25cm}@{}}
\toprule
ReRAM & Device \space Asymmetry & Analog States & Experimental Data Origin & Algorithm & Model Fidelity & MNIST Accuracy \\
\midrule
Ti/HfO\textsubscript{\textnormal{x}} \cite{Nate2023} & Not-included  & exp. states\footnotemark[2] & BEOL array & TTv2  & Medium & 90.5 \% \\
Ta/TaO\textsubscript{\textnormal{x}} \cite{Nate2023} & Not-included  & exp. states\footnotemark[2] & BEOL array & TTv2 & Medium  & 96.4 \%\\
TaO\textsubscript{\textnormal{x}}/HfO\textsubscript{\textnormal{x}} \cite{Stecconi2024} & included   & material states\footnotemark[3] & Single ReRAMs & TTv2 & Medium & 97.4 \% \\
\textbf{CMO\textsubscript{\textnormal{x}}/HfO\textsubscript{\textnormal{x}}}\footnotemark[1] & \textbf{included}  & \textbf{exp. states}\footnotemark[2] & \textbf{BEOL array} & \textbf{AGAD} & \textbf{High}  & \textbf{96.9 \%} \\
\botrule
\end{tabular}
\footnotetext[1]{\textbf{This work.}}
\footnotetext[2]{Measured number of analog states during open-loop device characterization.}
\footnotetext[3]{The asymptotic number of states under an infinite number of pulses.}
\label{table1}
\end{table}
\\
The impact of the device's number of states, asymmetry and noise-to-signal ratio on training accuracy using the AGAD algorithm on MNIST is evaluated. This analysis demonstrates that, with the current device's experimental properties, AGAD analog training achieves 96.9\% accuracy, comparable to the ideal FP-baseline of 98.3\%. To further improve analog training performance and bring results closer to the software equivalent, the key metric to enhance in the device is the symmetry. Finally, the on-chip analog training capabilities of the CMO/HfO\textsubscript{\textnormal{x}} ReRAM technology are demonstrated on a more complex 2-layer LSTM network, showing comparable performance to its floating-point equivalent.
\\
\\
In conclusion, the novel CMO/HfO\textsubscript{\textnormal{x}} ReRAM all-in-one technology platform presented in this work lays the foundation for efficient and versatile analog chips capable of combining both training and inference capabilities, enabling autonomous, energy-efficient, and adaptable AI systems.


\section{Methods}\label{secMethod}
\subsection{Device fabrication}\label{subsecDeviceFabrication}
The CMO/HfO\textsubscript{\textnormal{x}} ReRAM array is based on 1T1R unit cells. In this configuration, the bottom electrode of the ReRAM device is connected in series to the drain of an n-type metal–oxide–semiconductor (NMOS) selector transistor. The transistor blocks sneak paths and ensures current compliance during electro-forming and programming of the ReRAM device. The NMOS transistors, rated for \(\mathrm{3.3 \, V}\) operation, are fabricated using a standard \(\mathrm{130 \, nm}\) foundry process with copper BEOL interconnects. The ReRAM devices are integrated on metal-8 layer. To prevent the oxidation of the copper vias during the ReRAM stack deposition, the \(\mathrm{70 \, nm}\) thick silicon nitride (SiN\textsubscript{\textnormal{x}}) passivation layer from the foundry is used as a protective layer. On top of that, a \(\mathrm{20 \, nm}\) thick titanium nitride (TiN) bottom electrode and a \(\mathrm{4 \, nm}\) thick hafnium oxide (HfO\textsubscript{\textnormal{x}}) layers are deposited by Plasma-Enhanced Atomic Layer Deposition (PEALD) process at 300~\textdegree C, while maintaining vacuum conditions to avoid oxidation of the TiN layer. Subsequently, a stack of layers consisting of a \(\mathrm{20 \, nm}\) thick conductive metal-oxide (CMO), a \(\mathrm{20 \, nm}\) thick titanium nitride (TiN), and a \(\mathrm{50 \, nm}\) thick tungsten (W) is deposited by sputtering and patterned through a lithography step. A \(\mathrm{100 \, nm}\) thick silicon oxide (SiO\textsubscript{\textnormal{x}}) layer is sputtered as passivation. The passivation layer is then patterned to expose the W top electrode and the copper via in the metal-8 layer beneath the bottom electrode. The ReRAM fabrication is completed using a titanium/gold lift-off process. In this approach, the TiN bottom electrode is connected to the metal-8 via through its vertical sidewalls using gold. The ReRAM BEOL patterning steps are performed through mask-based photolithography performed on a 6$\times$6~mm$^2$ die issued from a Multi Project Wafer (MPW). The area of the CMO/HfO\textsubscript{\textnormal{x}} ReRAM devices presented in this work is 12$\times$12~\textmu m$^2$. Previous studies on CMO/HfO\textsubscript{\textnormal{x}} ReRAM devices have demonstrated scalability down to 200$\times$200~nm$^2$ \cite{Stecconi2024, NHFalcone2024, Davide_DRC}. Due to their filament-type nature, the performance of the ReRAM devices presented in this work is expected to remain similar for smaller areas.

\subsection{ReRAM forming modelling}\label{subsecFormingModelling}
A 3D FEM of the CMO/HfO\textsubscript{\textnormal{x}} ReRAM device, after the forming event, is used to simulate electronic transport by solving the continuity (\ref{Comsol_eq1}) and the Joule-heating (\ref{Comsol_eq2}) equations in steady state: 
\begin{align}
  \nabla \cdot J_{\rm e} = \nabla \cdot (\sigma (-\nabla V) = 0
  \label{Comsol_eq1}
\end{align}
\begin{align}
  \nabla \cdot (- k \nabla T) = J_{\rm e} \cdot E = Q_{\rm e}
   \label{Comsol_eq2}
\end{align}
where $J_{\rm e}$ is the electric current density, $\sigma$ the electrical conductivity, $V$ the electric potential, $k$ the thermal conductivity and $Q_{\rm e}$ the heat source due to Joule heating. From the fit of the experimental array forming data in the low-voltage linear regime (from 0 to \(0.2 \,\mathrm{V}\)), an average filament radius of \(11\, \mathrm{nm}\) is extracted. The electrical and thermal conductivities of the materials in the ReRAM stack are taken from literature \cite{NHFalcone2024}, by considering \( \sigma_{\mathrm{CMO}} = 5 \,\mathrm{S/cm}\) and \( k_{\mathrm{CMO}} = 4 \,\mathrm{W/m K}\) for the CMO layer used in this work. During the subsequent negative voltage sweep, the electrical conductivity of the CMO layer was used as a fitting parameter to model the radial redistribution of defects within the layer. Using experimental array data in the low-voltage linear regime (from 0 to \(\mathrm{-0.2 \, V}\)), the resulting CMO electrical conductivity is \( 37 \,\mathrm{S/cm}\). Fig. \ref{figS1} in Supplementary Information shows the results of the simulations.

\subsection{ReRAM forming voltage extraction}\label{subsecForming}
The forming voltage of each 1T1R cell (\(V_{\mathrm{forming}}^{\mathrm{1T1R}}\)) is defined as the voltage required to trigger the highest current increase (\(\max \left(\frac{dI}{dV}\right)\)) during the quasi-static voltage sweep from 0 to \(3.6 \, \mathrm{V}\) (see Supplementary Information Fig. \ref{figS2}a). The corresponding current is defined as the forming current (\(I_{\mathrm{forming}}^{\mathrm{1T1R}}\)) (see Supplementary Information Fig. \ref{figS2}b). Being the transistor driven by a constant \(V_\mathrm{G} = 1.2 \, \mathrm{V}\), it acts as a series resistor in the triode region before the forming event, when the ReRAM stack is highly insulating. After the forming event, when a conductive filament is created in the ReRAM device, the transistor ensures current compliance in the saturation region. The resistance of the transistor in the triode region at \(V_\mathrm{G} = 1.2 \, \mathrm{V}\) is measured to be \( R_\mathrm{DS} \approx 0.8 \, \mathrm{k\Omega} \) (see Supplementary Information Fig. \ref{figS2}c). Therefore, for each 1T1R cell, the actual ReRAM forming voltage is computed as \(V_{\mathrm{forming}}^{\mathrm{ReRAM}} = V_{\mathrm{forming}}^{\mathrm{1T1R}} - R_{\mathrm{DS}}^{\mathrm{triode}} \cdot I_{\mathrm{forming}}^{\mathrm{1T1R}}\) and reported in Fig. \ref{fig2}c. 

\subsection{Analytical ReRAM transport modelling}\label{subsecTransportModel}
In the 1T1R cell, the electronic current $I_{\rm e}$ is modelled as a trap-to-trap tunneling process within the CMO layer, as described in equation (\ref{I_e_MG}), following the model proposed by Mott and Gurney \cite{Mott_Gurney1950}. This model accounts for electron-hopping conduction across an energy barrier $\Delta E_{\rm e}$, which remains uniform in all directions when there is no electric field applied. However, when an electric field is introduced, it modifies the energy barrier by $\mp$ $ea_{\rm e}E$/2 for forward (backward) jumps, leading to a reduction (increase) in the barrier height. 
\begin{align}
   I_{\rm e}^{\rm Mott-Gurney} = 2 A e a_{\rm e} \nu_{\rm 0,e} N_{\rm e} \exp{(\frac{-\Delta E_{\rm e}}{k_{\rm B} T})} \sinh{(\frac{a_{\rm e}eE}{2k_{\rm B} T})}
   \label{I_e_MG}
\end{align}
In equation \eqref{I_e_MG}, $e$ is the elementary charge, $k_{\rm B}$ is the Boltzmann’s constant, $a_{\rm e}$ is the hopping distance, $\nu_{\rm 0,e}$ is the electron attempt frequency, $N_{\rm e}$ is the density of electronic defect states in the sub-band of the CMO layer, $\Delta E_{\rm e}$ is the zero-field hopping energy barrier, $T$ and $E$ are the local temperature and electric field, respectively, and $A = \rm \pi \it r_{\rm CF}^{\rm 2}$, $r_{\rm CF}$ being the filament radius, is the cross-sectional area of the filament at the interface with the CMO layer. The temperature and electric field in the CMO layer, for both LRS and HRS, are simulated by solving equations (\ref{Comsol_eq1}) and (\ref{Comsol_eq2}), while accounting for the experimental I-V non-linearity (see Supplementary Fig. \ref{figS4} for details). The trap-to-trap tunneling parameters ($N_{\rm e}$, $\Delta E_{\rm e}$, $a_{\rm e}$) are extracted from the fit using the same approach as described in previous works \cite{NHFalcone2024, IMWFalcone2023}.

\subsection{Identical-pulse closed-loop scheme}\label{subsubsecPVInference}
The procedure begins with a quasi-static voltage sweep from 0 to \(-1.5 \, \mathrm{V}\) to \textit{reset} each cell within the array to the HRS. Subsequently, a closed-loop scheme is initiated, which iteratively repeats the following two steps until convergence to \textit{G}\textsubscript{\textnormal{target}} within an acceptance range: (1) read the conductance of the ReRAM cell, and (2) if the measured value is below (above) the target conductance, apply a \textit{set} (\textit{reset}) programming pulse. During this iterative process, the cell conductance may fluctuate multiple times before eventually reaching the acceptance range. Starting from the HRS, this procedure is applied to the CMO/HfO\textsubscript{\textnormal{x}} ReRAM array to sequentially program 35 representative conductance levels, ranging from approximately 10~\textmu S to 90~\textmu S, using acceptance ranges of both 0.2\% \textit{G}\textsubscript{\textnormal{target}} and 2\% \textit{G}\textsubscript{\textnormal{target}}. Unlike the conventional incremental-pulse closed-loop technique previously used for ReRAM \cite{Wan2022, Alibart2012}, where the amplitudes of \textit{set} and \textit{reset} pulses are gradually increased to achieve convergence, this work employs an identical-pulse closed-loop scheme to simplify the pulse generation circuitry design, using only two fixed amplitude values for the \textit{set} (\(1.35 \, \mathrm{V}\) or \(1.5 \, \mathrm{V}\)) and two for the \textit{reset} (\(-1.3 \, \mathrm{V}\) or \(-1.5 \, \mathrm{V}\)) pulses. Specifically, depending on \textit{G}\textsubscript{\textnormal{target}}, three ranges are used: from approximately 10~\textmu S to 30~\textmu S  with \(V_{\rm set} = 1.35 \, \mathrm{V}\) and \(V_{\rm reset} = -1.5 \, \mathrm{V}\); from 30~\textmu S to 60~\textmu S \(V_{\rm set} = 1.35 \, \mathrm{V}\) and \(V_{\rm reset} = -1.3 \, \mathrm{V}\) ; and from 60~\textmu S to 90~\textmu S \(V_{\rm set} = 1.5 \, \mathrm{V}\) and \(V_{\rm reset} = -1.3 \, \mathrm{V}\). Fig. \ref{figS5}b in Supplementary Information shows the flowchart of the identical-pulse closed-loop technique used in this work. The \textit{set}/\textit{reset} pulse width is fixed at 2.5~\textmu s due to setup limitations, even though previous work has demonstrated CMO/HfO\textsubscript{\textnormal{x}} ReRAM switching with pulse width as short as \(60 \, \mathrm{ns}\) \cite{Davide_DRC}. The reading pulse amplitude and width are \(V_{\rm read} = 0.2 \, \mathrm{V}\) and 300~\textmu s, respectively. During the \textit{set}, \textit{reset}, and read operations of each 1T1R cell, the transistor's gate voltage is controlled with constant values of \(V_{\rm G}\) equal to \(1.4 \, \mathrm{V}\), \(3.3 \, \mathrm{V}\), and \(3.3 \, \mathrm{V}\), respectively.

\subsection{HW-aware simulation of analog MVM}\label{subsecInference}
The 'aihwkit' \cite{Rasch2021AFlexible} simulation tool was used to perform MVM assessments including non-ideal behaviors and noise, and their effect on the computation accuracy with respect to floating-point operations. The MVM simulation included the exhibited programming noise, conductance relaxation, input and output quantization, and IR-drop across array wires.
The 'aihwkit' allows to configure such noisy effects for dedicated memristive devices such as PCM by Nandakumar et al. \cite{Nandakumar2019PhaseChange} and ReRAM by Wan et al. \cite{Wan2022}. Therefore, a unique phenomenological noise model for CMO/HfO\textsubscript{\textnormal{x}} ReRAM devices for inference is developed to incorporate into the simulation both the characterized programming noise and conductance relaxation. Additionally, input and output are quantized with 6-bit and 8-bit resolution, respectively, and the IR-drop is considered, with 100~\textmu S as the maximum ReRAM conductance level and a default segment wire resistance of 0.35~$\Omega$. 
\subsubsection{Modelling the programming noise}
For a target conductance  \textit{G}\textsubscript{\textnormal{target}}, the device's programmed conductance is defined as the target value plus normally distributed noise with a standard deviation $\sigma_{\rm prog}$, which is a function of \textit{G}\textsubscript{\textnormal{target}}. As depicted in Fig. \ref{fig3}e, the programming noise ($\sigma_{\rm prog}$) of the CMO/HfO\textsubscript{\textnormal{x}} ReRAM  devices is statistically described by a first-order polynomial equation for a given acceptance range. The polynomial coefficients for acceptance ranges of 2\% and 0.2\% of \textit{G}\textsubscript{\textnormal{target}} are extracted from the characterization and introduced into the simulation environment. To assess the effects of the programming noise, each weight in the normalized matrix (ranging from [-1, 1]) is mapped to its corresponding conductance value (within the range [9, 89]~\textmu S from Fig. \ref{fig3}a), and is then further adjusted by the programming noise described by the extracted linear functions. Therefore, the MVM accuracy can be assessed immediately after programming ($t=0$), see Fig. \ref{fig4}f.

\subsubsection{Modelling the conductance relaxation}
After programming, the conductance levels exhibit relaxation over time, as shown in Fig. \ref{fig4}. Unlike previous ReRAM drift characterizations reported by Wan et al. \cite{Wan2022} the observed relaxation in CMO/HfO\textsubscript{\textnormal{x}} ReRAM is approximately independent of the initial programmed conductance. Consequently, a new modelling approach in the 'aihwkit' is needed to accurately simulate the conductance relaxation effect, which differs from the methods derived from previous literature on ReRAM \cite{Wan2022}. The conductance relaxation mean and standard deviation are modelled independently of \textit{G}\textsubscript{\textnormal{target}} and solely as a function of time after programming. 
The coefficients of the first-order polynomials describing the time dependence of both the mean and standard deviation of the programmed conductance are incorporated into the simulation environment to estimate conductance variations at any given inference time. By doing so, the MVM accuracy can be estimated after a period of time up to 10 years.

\subsection{HW-aware simulation of analog training}\label{devicemodel}
\subsubsection{Generalized soft bounds model}
The generalized soft bounds model (SBM) selection was based on the observed characteristics of the potentiation and depression since the devices did not strictly exhibit thorough saturation at the upper and lower boundaries (see Fig. \ref{figS8} in Supplementary Information). The generalized SBM incorporates a tunable scale exponent ($\gamma$) that describes abrupt and gradual trends toward the maximum and minimum conductance levels. This exponent parameter also varies depending on the conductance update direction. Therefore, the analytical expression of the generalized SBM implemented in the 'aihwkit' includes an asymmetry factor ($\gamma_{\rm up\_down}$) to account for this behavior\cite{aihwkit}. However, these two parameters do not have a direct physical equivalence, and therefore, cannot be derived from experimental traces. Hereby, $\gamma$ and $\gamma_{\rm up\_down}$ are obtained for each device through an independent linear fitting of the generalized SBM to the experimental response. In addition to the analytical parameters of the generalized SBM, devices in the 'aihwkit' are defined by a set of parameters that can be extracted from experimental traces. More precisely, the empirical maximum and minimum conductance, minimum conductance step size and its standard deviation, and the asymmetry between up and down response are considered ($G_{\rm max}$, $G_{\rm min}$, $\Delta G_{\rm sp}$, $\sigma_{\Delta G_{\rm sp}}$, and $up\_down$). More details on the $up\_down$ parameter are provided in the Supplementary Information. In this regard, each simulated device is defined by 6 parameters: four empirically obtained ($G_{\rm max}$, $G_{\rm min}$, $\Delta G_{\rm sp}$ and $up\_down$) and two analytically modelled from SBM linear fitting ($\gamma$ and $\gamma_{\rm up\_down}$).

\subsubsection{Intra and inter-device variability}
By extracting the standard deviation of the minimum conductance step size ($\sigma_{\Delta G_{\rm sp}}$) from the experimental traces and incorporating it into the simulation's device model, the device response intrinsically includes noise from cycle to cycle. This provides a realistic device behavior with intra-device variability. Furthermore, the network devices shall include inter-device variabilities to perform physically accurate simulations. To achieve this, two multi-variate Gaussian distributions, G$_{\rm 1}$ and G$_{\rm 2}$, are created (see Fig. \ref{figS9} in Supplementary Information). G$_{\rm 1}$ is extracted from the experimentally obtained parameters: N$_{\rm states}$ (which accounts for variations across devices in the G-range and step) and SP in the normalized G-range, whereas G$_{\rm 2}$ is fitted to the analytical model parameters obtained from the fitted generalized SBM ($\gamma$ and $\gamma_{\rm up\_down}$). Therefore, variables from G$_{\rm 1}$ showed statistical independence from those of G$_{\rm 2}$. New device instances are independently sampled from the two Gaussian distributions to represent synapses on the DNN layers. The instantiated CMO/HfO\textsubscript{\textnormal{x}} ReRAM devices include variations in the device response, conductance ranges, and asymmetrical behavior, thus providing a more hardware-aware and realistic scenario for analog training simulation. 

% .bib or .bbl
%\bibliography{sn-bibliography}
\documentclass{MITstyle}

%\usepackage[table]{xcolor}
\usepackage{chngcntr}
\usepackage{hyperref}
\usepackage{microtype}

\title{A Lightweight and Extensible Cell Segmentation and Classification Model for Whole Slide Images}

\author{Nikita Shvetsov~$^{1, }$\footnote{Correspondence e-mail: nikita.shvetsov@uit.no}, Thomas K. Kilvaer~$^{2, 3}$, Masoud Tafavvoghi~$^{4}$, Anders Sildnes~$^{1}$, \\ Kajsa Møllersen~$^{4}$, Lill-Tove Rasmussen Busund~$^{5, 6}$, Lars Ailo Bongo~$^{1}$ \\
%
\vspace{1em} % Space between authors and afilliations
%
\normalfont{\small $^{1}$Department of Computer Science, UiT The Arctic University of Norway}\\
\normalfont{\small $^{2}$Department of Oncology, University Hospital of North Norway}\\
\normalfont{\small $^{3}$Department of Clinical Medicine, UiT The Arctic University of Norway}\\
\normalfont{\small $^{4}$Department of Community Medicine, UiT The Arctic University of Norway}\\
\normalfont{\small $^{5}$Department of Medical Biology, UiT The Arctic University of Norway} \\
\normalfont{\small $^{6}$Department of Clinical Pathology, University Hospital of North Norway} %\vspace{2em}
}

\begin{document}
\maketitle

\section*{Abstract}

% \begin{abstract}
% Developing clinically useful cell-level analysis tools in digital pathology remains challenging due to limitations in dataset granularity, inconsistent annotations, computational demands of advanced models, and difficulties in integrating new technologies into clinical workflows. To address these challenges, we propose a multi-faceted solution that enhances data quality, model performance, and usability to create a lightweight and extensible cell segmentation and classification model.

% First, we update data labels by employing a cross-relabeling process that refines the labels of two existing datasets, PanNuke and MoNuSAC, to create a new unified dataset with enhanced granularity, encompassing seven distinct cell types. Second, we leverage the H-Optimus foundation model as a fixed encoder to improve feature representation for simultaneous cell segmentation and classification tasks. Third, to address the computational demands of foundation models, we employ knowledge distillation to reduce model size and complexity while maintaining comparable performance. Finally, to facilitate integration into clinical workflows, we integrate the distilled model into the QuPath software, a widely used open-source platform in digital pathology.

% Our results demonstrate improvements in cell segmentation and classification performance using the H‑Optimus-based model compared to a CNN-based model. Specifically, the average $R^2$ improved from 0.575 to 0.871, and the average $PQ$ score improved from 0.450 to 0.492, indicating better alignment with actual cell counts and enhanced segmentation and classification quality. Furthermore, the distilled student model maintains performance comparable to the larger foundation model while reducing the parameter count by a factor of 48.
% Overall, by reducing computational complexity and integrating it into existing workflows, the proposed approach may significantly impact diagnostic processes, reduce the workload of pathologists, and contribute to improved patient outcomes. Though our approach shows potential enhancements in efficiency and usability of cell segmentation and classification models in digital pathology, extensive validation is needed to deploy these models in clinical practice.
% \end{abstract}

%%% shortened abstract
\begin{abstract}
Developing clinically useful cell-level analysis tools in digital pathology remains challenging due to limitations in dataset granularity, inconsistent annotations, high computational demands, and difficulties integrating new technologies into workflows. To address these issues, we propose a solution that enhances data quality, model performance, and usability by creating a lightweight, extensible cell segmentation and classification model. 

First, we update data labels through cross-relabeling to refine annotations of PanNuke and MoNuSAC, producing a unified dataset with seven distinct cell types. Second, we leverage the H-Optimus foundation model as a fixed encoder to improve feature representation for simultaneous segmentation and classification tasks. Third, to address foundation models' computational demands, we distill knowledge to reduce model size and complexity while maintaining comparable performance. Finally, we integrate the distilled model into QuPath, a widely used open-source digital pathology platform. 

Results demonstrate improved segmentation and classification performance using the H-Optimus-based model compared to a CNN-based model. Specifically, average $R^2$ improved from 0.575 to 0.871, and average $PQ$ score improved from 0.450 to 0.492, indicating better alignment with actual cell counts and enhanced segmentation quality. The distilled model maintains comparable performance while reducing parameter count by a factor of 48. By reducing computational complexity and integrating into workflows, this approach may significantly impact diagnostics, reduce pathologist workload, and improve outcomes. Although the method shows promise, extensive validation is necessary prior to clinical deployment.
\end{abstract}
\clearpage

\section{Introduction}
In digital pathology, accurate segmentation and classification of cells are crucial for many diagnostic, prognostic, and predictive analyses \cite{Jaber_Beziaeva_etal._2019,Lin_Pan_etal._2022,Park_Ock_etal._2022,Shen_Choi_etal._2024}. Nowadays, developments in computational pathology offer multiple solutions \cite{H._Qu_P._Wu_etal._2020,Javed_Mahmood_etal._2020} to utilize cell-level datasets to train machine learning models that solve these problems. The quality and specificity of training datasets are critical for robust and accurate models. Adhering to the principle of "garbage in, garbage out", it is essential to ensure that these datasets are extensively and accurately labeled with distinct classes that reflect the diverse biological characteristics of different cell types. Unfortunately, the number of open-source datasets comprising such high-quality annotations is limited. Existing cell segmentation datasets \cite{Gamper_Koohbanani_etal._2019,Graham_Vu_etal._2019,Verma_Kumar_etal._2021} may offer extensive annotations for certain cell types while providing more general labels for others. For example, in PanNuke, which is one of the largest open-source datasets comprising labeled cells, various types of morphologically and functionally different inflammatory cells like macrophages and lymphocytes are clustered in a broad "inflammatory" class. Consequently, these classes are frequently omitted from analyses or aggregated into broader meta-classes \cite{Gamper_Koohbanani_etal._2020} and likely interfere with other cell classes included in the dataset. This and similar inconsistencies in annotation granularity limit the ability of machine learning models to learn the comprehensive and nuanced features necessary for accurate cell segmentation and classification. To address these challenges, methods for refining and standardizing dataset annotations are essential to enhance the quality of training data.

A complementary approach to mitigate the absence of high-quality training data is the use of foundation models. Foundation models as encoders are defined as large-scale, versatile networks pre-trained on vast, diverse datasets using self-supervised learning, contrasting with convolutional neural network (CNN) pre-trained encoders that rely on supervised learning with labeled data. In practice, foundation models leverage enormous amounts of weakly or unlabeled data from millions of whole slide images (WSIs) and employ self-attention mechanisms to capture long-range dependencies and global context \cite{Chen_Ding_etal._2024,Saillard_Jenatton_etal._2024,Vorontsov_Bozkurt_etal._2024,Xu_Usuyama_etal._2024}. As a consequence, foundation models are able to produce transferable feature representations across different cell types and tissue environments. The feature representations can be leveraged by decoder networks to produce segmentation masks and pixel-level classifications. Because foundation models have comprehensive feature representations, they can be effectively fine-tuned using much smaller amounts of cell-level data compared to the large datasets needed to train models from scratch. Furthermore, foundation models incorporate adversarial training elements or contrastive learning \cite{Chen_Ding_etal._2024,Xu_Usuyama_etal._2024}, enhancing their resilience and adaptability by exposing them to challenging and varied scenarios during training. This may result in more generalizable models, often making them well-suited for diverse and complex tasks in digital pathology.

Despite the inherent advantages of foundation models, their deployment for practical use faces its own obstacles. In particular, they require substantial computational power, financial investments and rigorous testing to ensure reliability and efficacy for a given task \cite{Akkus_Dangott_etal._2022,Dragomir_Cocuz_etal._2022,Go_2022,Jafri_Farooqui_etal._2024}. Moreover, while foundation models enhance feature representation and performance, they depend on the quality of available annotations for decoder fine-tuning and, like any other model, cannot resolve existing inconsistencies or ambiguities in data labels. Therefore, there remains a critical need for solutions that address both data quality and practical deployment considerations.
Further, integrating new technologies into existing clinical workflows often encounters resistance, as it necessitates adjustments to established diagnostic processes. So, there is a need to develop solutions that could be integrated into current practices, minimizing the burden on medical professionals to adopt new tools \cite{King_Williams_etal._2023}.

Existing solutions \cite{Goldsborough_Philps_etal._2024,Hörst_Rempe_etal._2024}, while addressing some aspects of these challenges, fall short in providing a comprehensive approach. To address the data quality and clinical deployment issues, we propose a multi-faceted solution that encompasses data refinement, model optimization, and integration with existing pathology tools (\hyperref[fig:fig1]{Figure 1}). The outcome is a lightweight cell segmentation and classification model that can be integrated into digital pathology workflows for practical clinical use.

\begin{figure}[h!]
    \centering
    \includegraphics[width=\textwidth, height=0.82\textheight, keepaspectratio]{images/Figure_1.pdf}
    \caption{Overview of the proposed solution, including 1) Data refinement using cross-relabeling, 2) Teacher model development and fine tuning, 3) Student model optimization with knowledge distillation and 4) Student model and QuPath integration}
    \label{fig:fig1}
\end{figure}
\clearpage

Our approach begins with preparing the data for the fine-tuning and training of the machine learning models. We create a refined dataset, acquired via cross-relabeling two cell-level datasets, enhancing annotation specificity and consistency of the labeled data. Subsequently, we create a cell segmentation and classification model based on the foundation model. We leverage the foundation model as a fixed encoder and fine-tune a decoder using the refined dataset to improve generalization across diverse tissue- and cell types.
To ensure that the model remains lightweight and deployable in a possibly resource-constrained environment, we employ knowledge distillation to approximate the functionality of the foundation model. Finally, to facilitate the practical application of our model in digital pathology workflows, we integrate it with the QuPath \cite{Bankhead_Loughrey_etal._2017} application. Each methodological component contributes to the overarching goal of enhancing model performance, generalizability, and usability in clinical settings.

The primary contributions of this paper are:
\begin{enumerate}
    \item \textit{Data labels refinement through cross-relabeling:}
    
    We propose a new method for refining labels of cell-level datasets through cross-relabeling. This method employs classification models to re-label broad and ambiguous instances, resulting in a more diverse dataset. Our evaluation demonstrates that these classification models achieve high accuracy on test subsets, indicating the reliability of the method for label refinement.

    \item \textit{Enhanced model performance via foundation models:}
    
    We employ a foundation model as a feature extractor for the cell segmentation and classification task. In comparison with training a CNN model from scratch, the foundation model backbone only needs fine-tuning, which significantly reduces training time, computational resources and data requirements. We show that using a foundation model encoder leads to better performance in cell segmentation and classification networks than using a CNN-based encoder. This improvement may enable the model to generalize more effectively across various tissue types and imaging methods.
    
    \item \textit{Model optimization through knowledge distillation:}
    
    We show that a smaller student model trained using knowledge distillation on the refined dataset obtained via our cross-relabeling approach from a foundation model achieves comparable performance in cell segmentation and quantification tasks. As a result, this model is more suitable for deployment in environments without high-performance computing resources.
    
    \item \textit{Integration with QuPath:}
    
    We integrate the distilled cell segmentation and classification model into QuPath, a widely used open-source digital pathology platform, to accelerate clinical adaptation by enabling pathologists to more easily incorporate advanced computational tools into their existing workflows.
\end{enumerate}

Through these methodological steps, we aim to bridge the gap between advanced machine learning techniques and practical clinical applications, making accurate and efficient digital pathology accessible in a broader range of healthcare settings.

\section{Refining Existing Datasets Using Cross-Relabeling}
To address the limitations of sparse and ambiguous labeling of cell-level datasets, we propose a generalizable cross-relabeling strategy that can be applied to any dataset containing broadly categorized or imprecisely labeled cell types. This approach involves training and subsequently leveraging classification models to refine broad categories into more specific or biologically relevant classes.
When applied to cell-level data, the methodology includes extracting individual cell images from the dataset patches, preprocessing these images to standardize the size and accommodate partial cells, and then training deep learning classifiers capable of distinguishing between the finer cell subtypes within the coarser categories. 
To illustrate our approach, we focus on the PanNuke \cite{Gamper_Koohbanani_etal._2020, Gamper_Koohbanani_etal._2019} and MoNuSAC \cite{Verma_Kumar_etal._2021} datasets that we have used to train models for cell quantification in our previous works \cite{Shvetsov_Grønnesby_etal._2022,Shvetsov_Sildnes_etal._2024}. We find that for better cell differentiation we have to introduce more granular labels. PanNuke includes a broad classification of "inflammatory" cells, encompassing lymphocytes, macrophages, and neutrophils. Each cell type differs significantly in structure, function, and clinical relevance. Conversely, MoNuSAC uses the label "epithelial" for a class that comprises both benign epithelial cells and malignant neoplastic cells. This practice makes it challenging to differentiate between benign and malignant epithelial cells in the dataset, which is a critical distinction when identifying tumor areas within tissue samples. To address these issues, we implement a cross-relabeling strategy as shown in \hyperref[fig:fig2]{Figure 2}. The key components are two classification models: one is trained on singular cell images from PanNuke data to classify the epithelial meta-class into epithelial and neoplastic classes. The other is trained on MoNuSAC to refine the inflammatory class into lymphocytes, neutrophils, and macrophages.

\begin{figure}[h!]
    \centering
    \includegraphics[width=\textwidth]{images/Figure_2.pdf}
    \caption{Refined dataset generation via cross relabeling}
    \label{fig:fig2}
\end{figure}

The refining approach consists of three consecutive steps. The first is the preprocessing step, in which we extract individual cells from both datasets (\hyperref[fig:fig3]{Figure 3}). The specifics of PanNuke and MoNuSAC patch preparation before cell preprocessing are provided in \hyperref[chap:S1]{Appendix S1}.

\begin{figure}[h!]
    \centering
    \includegraphics[width=\textwidth]{images/Figure_3.pdf}
    \caption{Cell instances preprocessing including (1) cell map extraction, (2) bounding box delineation, (3) adjusting cell boxes and (4) cropping and resizing of cell images}
    \label{fig:fig3}
\end{figure}

During preprocessing, we extract cell type maps from the ground truth label mask and calculate bounding boxes around each cell instance. To accommodate partial cells at patch borders, a common issue in cropped patch images, we employ mirror padding and extend the field of view of the cell label by 15 pixels to capture adjacent cells. We then crop and resize the identified regions to $64 \times 64$ pixels using bicubic interpolation.

The preprocessed PanNuke dataset comprises 68,031 neoplastic and 23,207 epithelial cell images, while MoNuSAC comprises  33,104 lymphocytes, 1,252 neutrophils, and 1,695 macrophages, which we subsequently use in training cell classification models and classifying the cell image data \hyperref[fig:S2]{Appendix Figure S2 (1)}. 

The next step is to train two distinct ResNet50-based classifiers tailored to address the specific labeling challenges inherent in each dataset. We use ResNet50 for classification models due to its proven effectiveness for image classification tasks in histopathology \cite{pan2022reviewmachinelearningapproaches}, and its compatibility with small images. For the PanNuke dataset, we design the classifier, trained on MoNuSAC data, to disaggregate the heterogeneous "inflammatory" cell category into distinct subtypes: lymphocytes, macrophages, and neutrophils. Similarly, for the MoNuSAC dataset, the classifier is trained on PanNuke data and distinguishes between benign and malignant epithelial cells within the overarching "epithelial" label. By applying these targeted classifiers to their respective datasets, we assign more specific labels to individual cell instances, thus enabling us to create a unified dataset.
To ensure a balanced representation of classes, we train both models on datasets that had been equalized to match the size of the least represented class. Thus, we obtain datasets comprising 23,207 samples per class for PanNuke and 1,252 samples per class for MoNuSAC data. Next, we partition both of them into training (70\%), validation (20\%), and testing (10\%) subsets. To mitigate the risk of overfitting, we use a single dropout layer with a rate of p=0.5 in both models and data augmentation using randomized color perturbations, rotation, and horizontal and vertical flipping. We employ AdamW optimizer and the cross-entropy loss function for the training criterion.

To evaluate the two trained models, we measure the classification accuracy on the respective test subsets. The accuracies on the test subset for both classifiers are presented in \hyperref[tab:1]{Table 1}. The PanNuke model achieves an average accuracy of 93.57\%, with higher accuracy for neoplastic cells (96.06\%) compared to epithelial cells (86.26\%). The confusion matrix in Figure A3.1 shows that the model predominantly distinguishes accurately between epithelial and neoplastic tissues, with a substantial number of correct classifications and relatively few misclassifications. The MoNuSAC model demonstrates an average accuracy of 98.92\%, excelling in classifying lymphocytes (99.67\%) and macrophages (94.12\%), with lower performance for neutrophils (85.71\%). The confusion matrix in Figure A3.2 shows that the model identifies lymphocytes and performs reasonably well with macrophages and neutrophils.

\begin{table}[h!]
\renewcommand{\arraystretch}{1.5}
  \centering
  \caption{Cell classification results for PanNuke and MoNuSAC trained models (CI 95\%).}
  \label{tab:1}
  \begin{tabular}{|l|c|c|}
   \hline
   %\rowcolor{gray!30}
    Accuracy               & PanNuke model              & MoNuSAC model              \\
    \hline
    Average      & 0.936 (0.931--0.941)         & 0.989 (0.986--0.993)        \\
    \hline
    Neoplastic   & 0.961 (0.956--0.965)         & -                          \\
    \hline
    Epithelial   & 0.863 (0.849--0.877)         & -                          \\
    \hline
    Lymphocytes  & -                          & 0.997 (0.995--0.999)        \\
    \hline
    Neutrophils  & -                          & 0.857 (0.796--0.918)        \\
    \hline
    Macrophages  & -                          & 0.941 (0.906--0.976)        \\
    \hline
  \end{tabular}
\end{table}

Finally, during the last step, we use the model trained on PanNuke data for epithelial cells in MoNuSAC and the model trained on MoNuSAC for the inflammatory cells class in PanNuke. Specifically, we use classifier models to relabel epithelial cells in MoNuSAC and inflammatory cells in PanNuke data. Then we combine cells with refined labels and the rest of the cells in both datasets to create a refined dataset (\hyperref[fig:S2]{Appendix Figure S2 (2)}). The process of relabeling cells and visualizing them on a patch is shown in \hyperref[fig:fig4]{Figure 4}. The cell counts in the refined dataset are provided in \hyperref[tab:S4]{Appendix Table S4}.

\begin{figure}[h!]
    \centering
    \includegraphics[width=\textwidth, height=0.42\textheight, keepaspectratio]{images/Figure_4.pdf}
    \caption{Cell relabeling procedure for epithelial and inflammatory cell classes}
    \label{fig:fig4}
\end{figure}

%\hfill

Relabeling and combining datasets have been explored in a prior study \cite{Parulekar_Kanwat_etal._2023}, where consecutive fine-tuning on multiple datasets was employed to account for hierarchical class label structures. While the method presented in \cite{Parulekar_Kanwat_etal._2023} is intuitive, it often lacks consistency and requires multiple fine-tuning runs, which can be cumbersome and time-consuming. 
In contrast, cross-relabeling simplifies this process by using specialized classification models tailored to each dataset's specific labeling challenges. This approach provides better transparency and produces a unified dataset encompassing seven distinct cell types across multiple tissue samples, enhancing data diversity for further model training or fine-tuning.

Despite these improvements, cross-relabeling does not entirely resolve issues related to poor labeling quality or the amount of labeled data. Specifically, our results show lower accuracies persist for underrepresented classes, such as macrophages, which may stem from a limited sample availability and intrinsic challenges in distinguishing these cells based solely on H\&E staining. Furthermore, while our method enhances label specificity, it relies on the initial quality of the broad labels; thus, any fundamental inaccuracies in the original annotations can propagate through the relabeling process. Addressing the overall problem of limited data labels may require integrating additional data sources or utilizing complementary immunohistochemical staining methods.
Although the reported performance metrics are obtained from evaluations on the native test sets of each dataset, it is important to note that the primary application of these classifiers is to perform cross-relabeling, where a model trained on one dataset (e.g., PanNuke) is applied to another (e.g., MoNuSAC) and vice versa. We acknowledge that a more systematic evaluation of cross-dataset generalization is needed and could be performed in future work.

Overall, the refined dataset produced by our approach can enhance the supervised training or fine-tuning of cell segmentation and classification models, especially those that utilize pre-trained foundation models to improve feature extraction robustness. In addition, these models can detect nuanced classes that enable researchers to conduct more detailed analyses of biological processes in computational pathology.

\section{Foundation models for robust cell segmentation and classification}

Accurate cell segmentation and classification in digital pathology are hindered by limited labeled data and the fact that conventional CNNs are unable to capture global contextual information due to their local receptive field constraints \cite{Gheflati_Rivaz_2022,Yang_Marcus_etal.}. Traditional approaches in cell quantification have predominantly relied on CNN encoders, such as ResNet50, given their proven effectiveness in semantic segmentation tasks \cite{Deshmane_2023,Graham_Vu_etal._2019,Mukasheva_Koishiyeva_etal._2024,Stringer_Wang_etal._2021}. However, approaches that include fine-tuning of pretrained CNNs, data augmentation, and stain normalization to partially increase data variability and address staining differences often fail to achieve the necessary generalization and robustness across diverse tissue types and staining conditions \cite{G._Wang_W._Li_etal._2018,Gao_Bagci_etal._2018,Karim_El_Khoury_Martin_Fockedey_etal._2021}.

To overcome these challenges, we leverage an encoder-decoder network that uses a foundation model as the encoder and a CNN upsampling decoder (\hyperref[fig:fig5]{Figure 5}) for simultaneous cell segmentation and classification in 2D patches extracted from WSIs. Foundation models with transformer-based architectures are viable alternatives to CNN-based encoders \cite{Shamshad_Khan_etal._2023,Sourget_2023}. They enable the creation of more advanced architectures that can decode or transform learned features more effectively \cite{Chen_Duan_etal._2023,Cheng_Misra_etal._2022,Xie_Wang_etal._2021}.

\begin{figure}[h!]
    \centering
    \includegraphics[width=\textwidth]{images/Figure_5.pdf}
    \caption{UNETR-like model with foundational model as backbone}
    \label{fig:fig5}
\end{figure}

By utilizing a transformer-based encoder, we incorporate global contextual information into the feature extraction process, which is a key advantage of such architectures \cite{Chen_Lu_etal._2021}. This foundation model integration facilitates accurate pixel-wise segmentation and classification without the need for extensive encoder training, thereby potentially improving generalization across varied cellular structures and tissue types.
In our implementation, we employ a modified UNETR \cite{Hatamizadeh_Tang_etal._2021} architecture that combines a vision transformer (ViT) \cite{Dosovitskiy_Beyer_etal._2021} encoder with a CNN-based decoder. The encoder utilizes the pretrained H-Optimus foundation model, which contains 1.1 billion parameters and is trained on over 500,000 H\&E stained WSIs \cite{Saillard_Jenatton_etal._2024}. We extract outputs from four evenly spaced transformer blocks $Z_i$, where $i \in [1, 14, 26, 38]$, to serve as residual connections for the CNN decoder. We select these blocks based on our observation that features from non-adjacent levels of the encoder lead to better overall performance on the test subset.

The CNN decoder upsamples the feature representations, acquired from the transformer blocks, to generate an intermediate vector that is handled by two task-specific layers that generate cell segmentation and classification masks. The first task-specific layer is the ‘Cellpose head’,  which is used to delineate cell instances. The layer generates horizontal and vertical gradient maps to form vector fields that are refined through gradient tracking in a post-processing step using the Cellpose algorithm \cite{Stringer_Wang_etal._2021}, known for its efficacy in cell segmentation tasks and generalizability across multiple domains \cite{Pachitariu_Stringer_2022,Stringer_Pachitariu_2024}. The second task-specific layer is the "Cell type head", which assigns labels to individual pixels. In the post-processing step, we determine the output classification label of each segmented cell instance by majority voting over the labeled pixels that comprise the cell in the segmentation map.

To evaluate model performance and measure the impact of adding a foundation model as backbone, we compare it to a ResNet50-based model. ResNet50 is a widely used solution for encoders in segmentation architectures in the medical domain \cite{Deshmane_2023,Graham_Vu_etal._2019,Mukasheva_Koishiyeva_etal._2024,Stringer_Wang_etal._2021}. For the H-Optimus-based model, we utilize frozen weights for the encoder and only fine-tune the decoder to take advantage of the extensive pre-training of the foundation model. For the ResNet50-based model we start with ImageNet \cite{Deng_Dong_etal.} weights and train both encoder and decoder parts. Hyperparameters for the training step are set to be identical, where possible, for comparable evaluation. 
For this evaluation, we deliberately use the PanNuke dataset to provide a standardized and controlled comparison between the H‑Optimus and ResNet50-based models (\hyperref[fig:S2]{Appendix Figure S2 (3)}). Specifically, we use two of the default PanNuke dataset splits (66\%) for training and validation, and reserve the third split (33\%) for testing.

To address the challenge of cell class imbalance in the PanNuke dataset, which is a common characteristic in most cell-level H\&E patch datasets, both models’ training processes employ a weighted loss function comprising cross-entropy and focal loss \cite{Lin_Goyal_etal._2018}. The focal loss component is adjusted with coefficients derived from each cell class' instance frequency, emphasizing learning from underrepresented classes and enhancing the model's sensitivity to rare but significant cellular patterns. The cross-entropy loss is augmented with spectral decoupling regularization \cite{Pezeshki_Kaba_etal._2021,Pohjonen_Stürenberg_etal._2022} and spatially varying label smoothing \cite{Islam_Glocker_2021}, which potentially stabilizes training and improves generalization in case of complex tissue morphologies. For optimization, we employ the \textit{AdamW} \cite{Loshchilov_Hutter_2019} to counter unbalanced class scenarios, with cosine annealing learning rate scheduler.

We utilize the scikit-learn library \cite{Van_der_Walt_Schönberger_etal._2014} and HoVer-Net \cite{Graham_Vu_etal._2019} implementations of $R^2$ (the coefficient of determination) and $PQ$ (panoptic quality) to evaluate our experiments. Complete mathematical formulations and detailed explanations of these metrics are provided in \hyperref[chap:S5]{Appendix S5}. To compute confidence intervals, we use nonparametric bootstrapping, where after calculating the metric on the full sample, we generated 1000 bootstrap replicates by resampling with replacement and then determined the 95\% confidence intervals as the 2.5th and 97.5th percentiles of the resulting empirical distribution.

%\hfill

The model comparisons are summarized in \hyperref[tab:2]{Table 2}. The H‑Optimus-based model achieves higher $R^2$ across all cell classes compared to the ResNet50-based model, which means that its predictions are more closely aligned with the PanNuke cell counts, indicating a stronger correlation with the observed data. Notably, the improvement of $R^2_{dead}$ may be an indicator of better global contextual representations provided by the foundation model backbone. In terms of segmentation and classification quality combined, measured by the PQ score, the H‑Optimus-based model demonstrates notable improvements across most cell classes. Overall, the average $R^2$ improved from 0.575 to 0.871, while the average $PQ$ score improved from 0.450 to 0.492, demonstrating better performance of the H-Optimus-based model.

\begin{table}[h!]
\renewcommand{\arraystretch}{1.5}
  \centering
  \caption{Cell quantification metrics for baseline and proposed models (CI 95\%).}
  \label{tab:2}
  \begin{tabular}{|l|c|c|}
    \hline
    %\rowcolor{gray!30}
    Metric             & Resnet50-based            & H-optimus-based              \\
    \hline
    $R^2_{neoplastic}$    & 0.681 (0.576--0.769)       & \textbf{0.941 (0.917--0.960)} \\
    \hline
    $R^2_{inflammatory}$  & 0.863 (0.778--0.903)       & \textbf{0.949 (0.918--0.966)} \\
    \hline
    $R^2_{connective}$    & 0.600 (0.488--0.698)       & 0.609 (0.436--0.772)          \\
    \hline
    $R^2_{dead}$          & 0.097 (-11.389--0.669)     & 0.925 (0.404--0.982)          \\
    \hline
    $R^2_{epithelial}$    & 0.635 (0.490--0.747)       & \textbf{0.930 (0.886--0.964)} \\
    \hline
    $PQ_{neoplastic}$       & 0.517 (0.499--0.535)       & \textbf{0.589 (0.575--0.604)} \\
    \hline
    $PQ_{inflammatory}$     & 0.455 (0.429--0.482)       & \textbf{0.528 (0.507--0.549)} \\
    \hline
    $PQ_{connective}$       & 0.416 (0.400--0.431)       & \textbf{0.451 (0.436--0.465)} \\
    \hline
    $PQ_{dead}$             & 0.374 (0.342--0.408)       & 0.292 (0.209--0.365)          \\
    \hline
    $PQ_{epithelial}$       & 0.488 (0.460--0.519)       & \textbf{0.599 (0.579--0.618)} \\
    \hline
  \end{tabular}
\end{table}

Our results  show that integrating the H‑Optimus foundation model within the UNETR architecture enhances the model's ability to segment and classify cells across diverse tissues from PanNuke data. The pretrained transformer encoder provides robust feature representations, resulting in higher average $R^2$ and $PQ$ scores compared to the CNN-based model. This leads to more reliable cell quantification and more accurate downstream analysis. Additionally, the streamlined fine-tuning process reduces computational overhead and training time, making the model more adaptable for new data.

Despite these advancements, the foundation model-based approach does not fully resolve all challenges related to cell segmentation and classification. We observe lower metric scores for underrepresented classes in the training data. Furthermore, foundation models typically encompass billions of parameters, resulting in substantial computational and memory requirements. It therefore poses challenges for deployment in resource-constrained environments, limiting their practical applicability in certain clinical settings.

\section{Model optimization via Knowledge Distillation}

To address the limitations posed by the extensive size of foundation models, we implement knowledge distillation — a model compression technique that leverages the teacher-student paradigm \cite{Hinton_Vinyals_etal._2015}. By training a smaller, more efficient student model to replicate the output of a larger, pre-trained teacher model, we retain performance while significantly reducing the model's complexity and resource requirements (\hyperref[fig:fig6]{Figure 6}).

\begin{figure}[h!]
    \centering
    \includegraphics[width=\textwidth, height=0.45\textheight, keepaspectratio]{images/Figure_6.pdf}
    \caption{Knowledge distillation framework for training a student model using a pre-trained teacher}
    \label{fig:fig6}
\end{figure}

We employ knowledge distillation to compress the H‑Optimus-based teacher model into a more efficient student model. The teacher model is the modified UNETR architecture with the H‑Optimus foundation model described in the previous chapter. The student model is based on a UNet architecture augmented with residual connections and incorporates a smaller ViT encoder with 9 million parameters \cite{Steiner_Kolesnikov_etal._2022,Wightman_2019}. 

First, we fine-tune the teacher model using the refined dataset from the cross-relabeling procedure (Section 2). Initially we train the decoder of the teacher model while keeping the encoder weights frozen. We split the refined dataset into train (70\%), validation (20\%) and test (10\%) subsets (\hyperref[fig:S2]{Appendix Figure S2 (4)}). During fine-tuning, we use the train and validation subsets, while leaving the test subset for model evaluation. We set the training procedure and model hyperparameters to be identical to those that were used to demonstrate the utility of foundation models for the simultaneous cell segmentation and classification task.

Next, we perform knowledge distillation from teacher to student using the refined dataset used to fine-tune the teacher model. The student model is trained to replicate the teacher model's outputs. We utilize a specialized loss function that aligns the student's predicted probability distribution with the teacher's, incorporating the teacher's class probability distribution derived from the output. Following the methodology of Hinton et al. \cite{Hinton_Vinyals_etal._2015}, we experiment with various hyperparameter settings for the temperature ($T$) and the balancing coefficients ($\alpha$ and $\beta$) in the loss function. We vary $T$ from 1 to 20 and adjust $\alpha$ and $\beta$ to balance the distillation and student losses. Through iterative tuning and evaluation, we identify that setting $T=14$, $\alpha=0.3$, and $\beta=0.7$ yields a configuration that converges and closely approximates the teacher model's performance during training.

Finally, we assess the performance of both models using the $R^2$ and $PQ$ (defined in \hyperref[chap:S5]{Appendix S5}) on the test set of the refined dataset (\hyperref[tab:3]{Table 3}). We observe that the 95\% confidence intervals overlap for most cell types, so we cannot claim statistically significant performance differences between the teacher and student models. One exception appears in the neoplastic class. The teacher model produces an $R^2$ of 0.919, while the student model shows an $R^2$ of 0.852. In addition, the student model achieves higher $PQ$ values for the neoplastic and connective classes, though the confidence intervals show overlap.

\begin{table}[h!]
\renewcommand{\arraystretch}{1.5}
  \centering
  \caption{Cell quantification metrics for teacher and distilled student models (CI 95\%).}
  \label{tab:3}
  \begin{tabular}{|l|c|c|}
    \hline
    %\rowcolor{gray!30}
    Metric & Teacher & Student \\
    \hline
    $R^2_{neoplastic}$    & \textbf{0.919} (0.898--0.939) & 0.852 (0.800--0.891) \\
    \hline
    $R^2_{lymphocyte}$    & 0.969 (0.956--0.977)         & 0.969 (0.956--0.978) \\
    \hline
    $R^2_{connective}$    & 0.694 (0.548--0.809)         & 0.618 (0.469--0.741) \\
    \hline
    $R^2_{dead}$          & 0.755 (0.400--0.908)         & 0.424 (0.100--0.731) \\
    \hline
    $R^2_{epithelial}$    & 0.922 (0.870--0.958)         & 0.843 (0.738--0.917) \\
    \hline
    $R^2_{macrophage}$    & 0.384 (-0.369--0.724)        & 0.704 (0.352--0.859) \\
    \hline
    $R^2_{neutrofil}$     & 0.854 (0.578--0.929)         & 0.833 (0.502--0.925) \\
    \hline
    $PQ_{neoplastic}$       & 0.581 (0.569--0.593)         & 0.601 (0.588--0.613) \\
    \hline
    $PQ_{lymphocyte}$       & 0.536 (0.520--0.553)         & 0.563 (0.544--0.579) \\
    \hline
    $PQ_{connective}$       & 0.436 (0.421--0.451)         & 0.457 (0.441--0.474) \\
    \hline
    $PQ_{dead}$             & 0.272 (0.235--0.315)         & 0.279 (0.201--0.369) \\
    \hline
    $PQ_{epithelial}$       & 0.522 (0.500--0.545)         & 0.530 (0.506--0.555) \\
    \hline
    $PQ_{macrophage}$       & 0.524 (0.459--0.588)         & 0.474 (0.405--0.543) \\
    \hline
    $PQ_{neutrofil}$        & 0.541 (0.490--0.592)         & 0.565 (0.522--0.607) \\
    \hline
  \end{tabular}
\end{table}


We further decompose the $PQ$ metric into its $SQ$ and $DQ$ components (\hyperref[tab:S6]{Appendix Table S6}). Both models produce nearly identical $SQ$ values, which indicates that they predict instance boundaries with similar precision. Although the student model shows some improvement in $DQ$ scores for certain classes, the confidence intervals overlap and do not confirm a statistically significant difference.

We observe that the student and teacher models yield comparable detection performance despite the student model using a much smaller and simpler architecture. A model with fewer parameters reduces the risk of overfitting when training data are scarce relative to the model’s complexity \cite{Farias_Ludermir_etal._2022}. The knowledge distillation process also encourages the student model to focus on the most generalizable detection features learned from the teacher. These factors enable the student model to achieve similar detection performance across different cell types.

Additionally, considering the model sizes reported in \hyperref[tab:4]{Table 4}, the distilled model achieves a significant reduction compared to the teacher model, with a 48-fold decrease in parameter count and a 5.5-fold reduction in on-disk size. In inference mode, the teacher model requires 16 GB of VRAM for a batch size of 32, while the distilled model only needs 3 GB of VRAM for the same batch size. These reductions make the distilled model significantly more practical for fine-tuning and deployment in resource-constrained environments.

\begin{table}[h!]
\renewcommand{\arraystretch}{1.5}
  \centering
  \caption{Parameter counts and size of teacher and distilled model}
  \label{tab:4}
  \adjustbox{max width=\textwidth}{%
  \begin{tabular}{|l|c|c|c|}
    \hline
    %\rowcolor{gray!30}
    Metric & H-optimus-based (Teacher) & mobileViT-based (Student) & Magnitude of difference \\
    \hline
    Parameters count       & 1,158,917,906   & \textbf{24,093,393}   & \textbf{48x}  \\
    \hline
    Estimated Total Size (MB) & 87,912       & \textbf{15,935}    & \textbf{5.5x} \\
    \hline
  \end{tabular}%
}
\end{table}

%\hfill

With recent advancements in complex network architectures and the use of pretrained encoders to achieve state-of-the-art performance \cite{Baumann_Dislich_etal._2024,Hörst_Rempe_etal._2024} in cell segmentation and classification tasks, model size, computational complexity, and processing times have increased. This limits the scalability and accessibility of these models. As we demonstrate, this may be mitigated using knowledge distillation. Studies in the field of natural language processing have demonstrated the efficacy of knowledge distillation in retaining the capabilities of the teacher model while achieving significant reductions in size and complexity \cite{Huangpu_Gao_2024,Sun_Yu_etal.}. 

We demonstrate the feasibility of knowledge distillation in digital pathology, specifically for cell segmentation and classification tasks. Moreover, we achieve this performance while also significantly reducing the parameter count. In addressing the challenge of knowledge transfer, we found that distillation from a transformer-based model to a smaller transformer is more straightforward than attempting to map transformer features to CNN blocks. In our experiments, using a CNN-based network as a student results in worse cell quantification performance due to the structural constraints of CNN feature space dimensions. 

Although our primary approach relies on a transformer-based student model that performs well, it can be further optimized to incorporate advantages from CNN architectures. For example, employing alternative techniques such as using ViT adapters \cite{Chen_Duan_etal._2023} or $1 \times 1$ convolutions to adjust feature map sizes may be beneficial for harnessing CNN advantages like enhanced local feature extraction. Moreover, if additional performance improvements are desired, the process can be further enhanced by applying supplementary knowledge distillation techniques, such as self-distillation \cite{Zhang_Song_etal._2019} or online distillation \cite{Houyon_Cioppa_etal._2023}.

Despite these promising results, further validation on independent datasets is necessary to fully understand the model's limitations. Underrepresented classes may pose challenges when addressing complex cases. Pathologists need to validate these models to adopt them in clinical settings. While the distilled models are smaller and more deployable, a technological gap persists because pathologists traditionally rely on established methods for inspecting WSIs and diagnosing diseases. Addressing the complexities involved in deploying models for inference and supporting pathologists in adopting new tools is essential for integrating these models into clinical workflows.

\section{Model integration with QuPath}
Digital pathology tools with graphical user interfaces are essential for visualizing and analyzing WSIs. To make our student model useful in clinical pathology workflows, it needs to be integrated into a tool that enables inspecting regions, creating annotations, and providing quantitative analyses of biomarkers. Therefore, we integrate the trained student model from the previous chapter into the QuPath open‑source platform \cite{Bankhead_Loughrey_etal._2017}. QuPath provides the required annotation, visualization, and analysis tools to interpret complex histological data, including workflows for cell segmentation, classification, and quantification (\hyperref[fig:fig7]{Figure 7}). 

\begin{figure}[h!]
    \centering
    \includegraphics[width=\textwidth]{images/Figure_7.pdf}
    \caption{Visualization of model-generated cell quantification annotations (left) and the corresponding unannotated slide (right) in QuPath}
    \label{fig:fig7}
\end{figure}

To identify the regions in a WSI critical for prognosticating tumor development, such as specific tumor areas or border regions without overlapping healthy tissue, the pathologist uses QuPath to outline these regions. Then, the pathologist initiates a cell segmentation and classification script through the QuPath interface for the selected regions. The resulting annotations and quantified cell information are then directly overlaid onto the WSI in the QuPath interface. Additional design and implementation details are in \hyperref[chap:S7]{Appendix S7}. 

Two common approaches for integrating deep learning models into QuPath are Java‑based native QuPath extensions \cite{Goldsborough_Philps_etal._2024} and the execution of RESTful API requests to a model server coupled with handling the response via an extension, as demonstrated in the application of cell segmentation models applied to immunofluorescence images \cite{Sugawara_2023}. While the community is actively working on these integration strategies, there is currently no universal solution that fully addresses all integration and performance requirements.

Extensions may offer better integration with QuPath, allowing slightly improved performance and more widespread usage of the built-in QuPath models, but they lack the flexibility to customize models and modify their behavior. For example, the newest version of QuPath includes models such as StarDist \cite{Weigert_Schmidt} and InstanSeg \cite{Goldsborough_Philps_etal._2024} that can perform cell segmentation. Both models pose limitations when applied to simultaneous cell segmentation and classification. StarDist performs well only on convex, round shapes by design, whereas some neoplastic, inflammatory, and connective cells exhibit complex and non-convex shapes. InstanSeg provides only semantic segmentation without assigning classes to the segmented cells.

%\hfill

In contrast, our approach offers an alternative integration strategy. It utilizes the paquo library to directly interact with QuPath’s internal application programming interface from within Python. This enables data exchange and processing without the need for intermediate conversion steps and provides greater control over model customization, retraining, and the incorporation of custom processing steps.

The integration of our custom model with QuPath underscores its potential to significantly enhance the diagnostic process by reducing the time burden on pathologists and enabling them to focus on more complex interpretative tasks using familiar software. Leveraging a tool that is already well-established among pathologists increases the likelihood of its adoption into daily clinical workflows. The quantitative data generated through the automated workflow is critical for both clinical decision-making and research, facilitating more accurate biomarker analysis, enabling robust statistical evaluations, and supporting hypothesis generation and testing. Additionally, by streamlining cell segmentation and classification, the tool enhances the scalability and reproducibility of pathological assessments, ultimately contributing to improved diagnostic accuracy and patient outcomes.

\section{Conclusion and future work}

In this study, we address critical challenges in digital pathology and tackle the usability and deployment issues of the developed models in standard computing environments without the need for high-performance computing systems. Our multi-faceted approach encompasses data refinement through cross-relabeling, leveraging foundation models for robust cell segmentation and classification, optimizing model performance via knowledge distillation, and integrating the optimized model into the QuPath software for practical application. This approach is used to construct a capable, versatile, and adjustable model for cell segmentation and classification, with enhanced performance and usability.

\begin{sloppypar}
While our approach shows potential in the field of computational pathology, certain limitations persist. 
For example, our implementation currently exhibits lower performance in detecting macrophages. 
This serves as an instance of the broader challenge of accurately identifying complex cell types. In order to address this issue, extending our approach to incorporate additional data sources, exploring alternative modeling approaches, and integrating other imaging modalities such as immunohistochemical staining may help improve detection accuracy. Moreover, although the distilled model reduces computational demands, integrating advanced deep learning models into clinical practice requires addressing technological gaps and potential resistance to adopting new tools within established diagnostic processes.
\end{sloppypar}

Future work could focus on several key areas to refine the proposed approach and facilitate its adoption in clinical environments. Enhancing the cell-relabeling process with additional datasets \cite{Graham_Jahanifar_etal._2021} could improve the representation of underrepresented cell types and enhance overall model performance. Also, incorporating additional data sources, such as multi-modal imaging or complementary staining methods, may address limitations related to cell type differentiation and class imbalance. Exploring other foundation models \cite{Vorontsov_Bozkurt_etal._2024,Zimmermann_Vorontsov_etal._2024} or introducing additional modalities \cite{Ding_Wagner_etal._2024,Vaidya_Zhang_etal._2025} may provide alternative architectures better suited to specific tasks or offer improved efficiency. Implementing more complex knowledge distillation techniques \cite{Houyon_Cioppa_etal._2023,Zhang_Song_etal._2019} could further optimize the model's performance and adaptability. Additionally, deeper integration with QuPath or other digital pathology software could provide pathologists more control over cell quantification analysis directly within the QuPath interface, thereby increasing accessibility and usability. Such enhancements would not only refine model performance but also ensure greater adaptability and scalability within various clinical environments. Finally, extensive validation of the model by pathologists and benchmarking against independent datasets are essential steps toward establishing the model's reliability and fostering confidence in its clinical utility.

\section*{Acknowledgments} 
This work was funded in part by the Research Council of Norway grant no. 309439 SFI Visual Intelligence, and the North Norwegian Health Authority grant no. HNF1521-20.

\bibliographystyle{IEEEtran}
\begin{sloppypar}
\begin{thebibliography}{99}

\bibitem{chaplot2020neural} Chaplot, Devendra Singh, et al. "Neural topological slam for visual navigation." Proceedings of the IEEE/CVF conference on computer vision and pattern recognition. 2020.

\bibitem{maksymets2021thda} Maksymets, Oleksandr, et al. "Thda: Treasure hunt data augmentation for semantic navigation." Proceedings of the IEEE/CVF International Conference on Computer Vision. 2021.

\bibitem{mezghan2022memory} Mezghan, Lina, et al. "Memory-augmented reinforcement learning for image-goal navigation." 2022 IEEE/RSJ International Conference on Intelligent Robots and Systems (IROS). IEEE, 2022.

\bibitem{al2022zero} Al-Halah, Ziad, Santhosh Kumar Ramakrishnan, and Kristen Grauman. "Zero experience required: Plug \& play modular transfer learning for semantic visual navigation." Proceedings of the IEEE/CVF Conference on Computer Vision and Pattern Recognition. 2022.

\bibitem{ye2021auxiliary} Ye, Joel, et al. "Auxiliary tasks and exploration enable objectgoal navigation." Proceedings of the IEEE/CVF international conference on computer vision. 2021.

\bibitem{chaplot2020object} Chaplot, Devendra Singh, et al. "Object goal navigation using goal-oriented semantic exploration." Advances in Neural Information Processing Systems 33 (2020)

\bibitem{ramakrishnan2022poni} Ramakrishnan, Santhosh Kumar, et al. "Poni: Potential functions for objectgoal navigation with interaction-free learning." Proceedings of the IEEE/CVF Conference on Computer Vision and Pattern Recognition. 2022.

\bibitem{ramrakhya2022habitat} Ramrakhya, Ram, et al. "Habitat-web: Learning embodied object-search strategies from human demonstrations at scale." Proceedings of the IEEE/CVF Conference on Computer Vision and Pattern Recognition. 2022.

\bibitem{mousavian2019visual} Mousavian, Arsalan, et al. "Visual representations for semantic target driven navigation." 2019 International Conference on Robotics and Automation (ICRA). IEEE, 2019.

\bibitem{dhariwal2021diffusion} Dhariwal, Prafulla, and Alexander Nichol. "Diffusion models beat gans on image synthesis." Advances in neural information processing systems 34 (2021)

\bibitem{ho2022classifier} Ho, Jonathan, and Tim Salimans. "Classifier-free diffusion guidance." arXiv preprint arXiv:2207.12598 (2022).

\bibitem{nichol2021glide} Nichol, Alex, et al. "Glide: Towards photorealistic image generation and editing with text-guided diffusion models." arXiv preprint arXiv:2112.10741 (2021)

\bibitem{brooks2023instructpix2pix} Brooks, Tim, Aleksander Holynski, and Alexei A. Efros. "Instructpix2pix: Learning to follow image editing instructions." Proceedings of the IEEE/CVF Conference on Computer Vision and Pattern Recognition. 2023.

\bibitem{fu2023guiding} Fu, Tsu-Jui, et al. "Guiding instruction-based image editing via multimodal large language models." arXiv preprint arXiv:2309.17102 (2023).

\bibitem{geng2024instructdiffusion} Geng, Zigang, et al. "Instructdiffusion: A generalist modeling interface for vision tasks." Proceedings of the IEEE/CVF Conference on Computer Vision and Pattern Recognition. 2024.

\bibitem{zhou2024minedreamer} Zhou, Enshen, et al. "Minedreamer: Learning to follow instructions via chain-of-imagination for simulated-world control." arXiv preprint arXiv:2403.12037 (2024).

\bibitem{zhou2023esc} Zhou, Kaiwen, et al. "Esc: Exploration with soft commonsense constraints for zero-shot object navigation." International Conference on Machine Learning. PMLR, 2023.

\bibitem{yu2023l3mvn} Yu, Bangguo, Hamidreza Kasaei, and Ming Cao. "L3mvn: Leveraging large language models for visual target navigation." 2023 IEEE/RSJ International Conference on Intelligent Robots and Systems (IROS). IEEE, 2023.

\bibitem{gadre2023cows} Gadre, Samir Yitzhak, et al. "Cows on pasture: Baselines and benchmarks for language-driven zero-shot object navigation." Proceedings of the IEEE/CVF Conference on Computer Vision and Pattern Recognition. 2023.

\bibitem{shah2023navigation} Shah, Dhruv, et al. "Navigation with large language models: Semantic guesswork as a heuristic for planning." Conference on Robot Learning. PMLR, 2023.

\bibitem{cai2024bridging} Cai, Wenzhe, et al. "Bridging zero-shot object navigation and foundation models through pixel-guided navigation skill." 2024 IEEE International Conference on Robotics and Automation (ICRA). IEEE, 2024.

\bibitem{yu2023co} Yu, Bangguo, Hamidreza Kasaei, and Ming Cao. "Co-NavGPT: Multi-robot cooperative visual semantic navigation using large language models." arXiv preprint arXiv:2310.07937 (2023).

\bibitem{wu2024voronav} Wu, Pengying, et al. "Voronav: Voronoi-based zero-shot object navigation with large language model." arXiv preprint arXiv:2401.02695 (2024).

\bibitem{qin2023mp5} Qin, Yiran, et al. "Mp5: A multi-modal open-ended embodied system in minecraft via active perception." arXiv preprint arXiv:2312.07472 (2023).

\bibitem{du2024learning} Du, Yilun, et al. "Learning universal policies via text-guided video generation." Advances in Neural Information Processing Systems 36 (2024).

\bibitem{ajay2024compositional} Ajay, Anurag, et al. "Compositional foundation models for hierarchical planning." Advances in Neural Information Processing Systems 36 (2024).

\bibitem{liang2024skilldiffuser} Liang, Zhixuan, et al. "Skilldiffuser: Interpretable hierarchical planning via skill abstractions in diffusion-based task execution." Proceedings of the IEEE/CVF Conference on Computer Vision and Pattern Recognition. 2024.

\bibitem{heusel2017gans} Heusel, Martin, et al. "Gans trained by a two time-scale update rule converge to a local nash equilibrium." Advances in neural information processing systems 30 (2017).

\bibitem{zhang2018unreasonable} Zhang, Richard, et al. "The unreasonable effectiveness of deep features as a perceptual metric." Proceedings of the IEEE conference on computer vision and pattern recognition. 2018.

\bibitem{brown2020language} Brown, Tom B. "Language models are few-shot learners." arXiv preprint arXiv:2005.14165 (2020).

\bibitem{podell2023sdxl} Podell, Dustin, et al. "Sdxl: Improving latent diffusion models for high-resolution image synthesis." arXiv preprint arXiv:2307.01952 (2023).

\bibitem{brohan2022rt} Brohan, Anthony, et al. "Rt-1: Robotics transformer for real-world control at scale." arXiv preprint arXiv:2212.06817 (2022).

\bibitem{brohan2023rt} Brohan, Anthony, et al. "Rt-2: Vision-language-action models transfer web knowledge to robotic control." arXiv preprint arXiv:2307.15818 (2023).

\bibitem{li2024manipllm} Li, Xiaoqi, et al. "Manipllm: Embodied multimodal large language model for object-centric robotic manipulation." Proceedings of the IEEE/CVF Conference on Computer Vision and Pattern Recognition. 2024.

\bibitem{shah2023vint} Shah, Dhruv, et al. "ViNT: A foundation model for visual navigation." arXiv preprint arXiv:2306.14846 (2023).

\bibitem{liu2024visual} Liu, Haotian, et al. "Visual instruction tuning." Advances in neural information processing systems 36 (2024).

\bibitem{hu2021lora} Hu, Edward J., et al. "Lora: Low-rank adaptation of large language models." arXiv preprint arXiv:2106.09685 (2021).

\bibitem{qin2023supfusion} Qin, Yiran, et al. "SupFusion: Supervised LiDAR-camera fusion for 3D object detection." Proceedings of the IEEE/CVF International Conference on Computer Vision. 2023.

\bibitem{qin2024worldsimbench} Qin, Yiran, et al. "Worldsimbench: Towards video generation models as world simulators." arXiv preprint arXiv:2410.18072 (2024).

\bibitem{yu2025gamefactory} Yu, Jiwen, et al. "GameFactory: Creating New Games with Generative Interactive Videos." arXiv preprint arXiv:2501.08325 (2025).

\bibitem{zhou2024code} Zhou, Enshen, et al. "Code-as-Monitor: Constraint-aware Visual Programming for Reactive and Proactive Robotic Failure Detection." arXiv preprint arXiv:2412.04455 (2024).

\bibitem{zhang2024ad} Zhang, Zaibin, et al. "AD-H: Autonomous Driving with Hierarchical Agents." arXiv preprint arXiv:2406.03474 (2024).

\bibitem{wang2024toward} Wang, Chaoqun, et al. "Toward Accurate Camera-based 3D Object Detection via Cascade Depth Estimation and Calibration." arXiv preprint arXiv:2402.04883 (2024).

\bibitem{huang2024story3d} Huang, Yuzhou, et al. "Story3d-agent: Exploring 3d storytelling visualization with large language models." arXiv preprint arXiv:2408.11801 (2024).

\bibitem{savinov2018semi} Savinov, Nikolay, Alexey Dosovitskiy, and Vladlen Koltun. "Semi-parametric topological memory for navigation." arXiv preprint arXiv:1803.00653 (2018).

\bibitem{majumdar2022zson} Majumdar, Arjun, et al. "Zson: Zero-shot object-goal navigation using multimodal goal embeddings." Advances in Neural Information Processing Systems 35 (2022): 32340-32352.

\bibitem{yadav2023offline} Yadav, Karmesh, et al. "Offline visual representation learning for embodied navigation." Workshop on Reincarnating Reinforcement Learning at ICLR 2023. 2023.

\bibitem{yadav2023ovrl} Yadav, Karmesh, et al. "Ovrl-v2: A simple state-of-art baseline for imagenav and objectnav." arXiv preprint arXiv:2303.07798 (2023).

\bibitem{sun2024fgprompt} Sun, Xinyu, et al. "FGPrompt: fine-grained goal prompting for image-goal navigation." Advances in Neural Information Processing Systems 36 (2024).

\bibitem{zhu2017target} Zhu, Yuke, et al. "Target-driven visual navigation in indoor scenes using deep reinforcement learning." 2017 IEEE international conference on robotics and automation (ICRA). IEEE, 2017.

\bibitem{koh2024generating} Koh, Jing Yu, Daniel Fried, and Russ R. Salakhutdinov. "Generating images with multimodal language models." Advances in Neural Information Processing Systems 36 (2024).

\bibitem{krantz2022instance} Krantz, Jacob, et al. "Instance-specific image goal navigation: Training embodied agents to find object instances." arXiv preprint arXiv:2211.15876 (2022).

\bibitem{schulman2017proximal} Schulman, John, et al. "Proximal policy optimization algorithms." arXiv preprint arXiv:1707.06347 (2017).

\bibitem{anderson2018evaluation} Anderson, Peter, et al. "On evaluation of embodied navigation agents." arXiv preprint arXiv:1807.06757 (2018).

\bibitem{lin2024navcot} Lin, Bingqian, et al. "NavCoT: Boosting LLM-Based Vision-and-Language Navigation via Learning Disentangled Reasoning." arXiv preprint arXiv:2403.07376 (2024).

\bibitem{NavGPT} Zhou, Gengze, Yicong Hong, and Qi Wu. "Navgpt: Explicit reasoning in vision-and-language navigation with large language models." Proceedings of the AAAI Conference on Artificial Intelligence.

\bibitem{hahn2021no} Hahn, Meera, et al. "No rl, no simulation: Learning to navigate without navigating." Advances in Neural Information Processing Systems 34 (2021): 26661-26673.

\bibitem{li2025t2isafety} Li, Lijun, et al. "T2ISafety: Benchmark for Assessing Fairness, Toxicity, and Privacy in Image Generation." arXiv preprint arXiv:2501.12612 (2025).

\bibitem{an2024agfsync} An, Jingkun, et al. "AGFSync: Leveraging AI-Generated Feedback for Preference Optimization in Text-to-Image Generation." arXiv preprint arXiv:2403.13352 (2024).


\end{thebibliography}
\end{sloppypar}

\clearpage
\beginsupplement
\section*{Appendix}
\renewcommand{\thesubsection}{S\arabic{subsection}}

\subsection{\label{chap:S1}PanNuke and MoNuSAC preprocessing}
The PanNuke dataset comprises a set of 7,901 RGB patches, each with dimensions of $256 \times 256$ pixels, which we set as the standard patch size for our analysis. In contrast, the MoNuSAC dataset encompasses 294 images of heterogeneous dimensions. To standardize the MoNuSAC images with our experiments, we implement a standardization protocol. Specifically, for images exceeding the dimensions of $256 \times 256$ pixels, we segment them into equal-sized patches and apply mirror padding to the remaining portions to avoid information loss at the peripherals. Patches with dimensions less than $128 \times 128$ pixels are excluded from the dataset due to the insufficient resolution to capture relevant cellular details. For patches where either dimension falls between 128 and 256 pixels, we employ upsampling to achieve the standard patch size. As a result, we obtain a total of 2,823 RGB patches derived from the MoNuSAC dataset for subsequent analysis. For additional details on the MoNuSAC data preparation process, refer to the source code \cite{Shvetsov_2025a}.
\clearpage

\subsection{\label{chap:S2}Data usage for the methodology}

\counterwithin{figure}{subsection}
\renewcommand{\thefigure}{S\arabic{subsection}}

\begin{figure}[h!]
    \centering
    \includegraphics[width=\textwidth, height=0.85\textheight, keepaspectratio]{images/A2.pdf}
    \caption{Overview of the methodology for cross-labeling, dataset refinement, and model comparison. (1) Cross-relabeling - training and testing cell classification models, (2) Cross-relabeling - using cell classification models to create refined dataset, (3) Fine-tuning and training models for comparison, (4) Student knowledge distillation with refined dataset}
    \label{fig:S2}
\end{figure}
\clearpage

\subsection{\label{chap:S3}Confusion matrices for classification models}
\counterwithin{figure}{subsection}
\renewcommand{\thefigure}{S\arabic{subsection}.\arabic{figure}}

\begin{figure}[h!]
    \centering
    \includegraphics[width=\textwidth, height=0.4\textheight, keepaspectratio]{images/A3_1.pdf}
    \caption{Confusion matrix for PanNuke trained model}
    \label{fig:S3.1}
\end{figure}

\begin{figure}[h!]
    \centering
    \includegraphics[width=\textwidth, height=0.4\textheight, keepaspectratio]{images/A3_2.pdf}
    \caption{Confusion matrix for MoNuSAC trained model}
    \label{fig:S3.2}
\end{figure}

\clearpage

\subsection{\label{chap:S4}Datasets cell counts}

\counterwithin{table}{subsection}
\renewcommand{\thetable}{S\arabic{subsection}}

\begin{table}[h!]
\renewcommand{\arraystretch}{2.0}
\centering
\caption{\label{tab:S4}Cell counts for PanNuke, MoNuSAC and refined datasets. Numbers in parentheses indicate preprocessed cell counts for cell classifier models training and testing.}
%\adjustbox{max width=\textwidth}{%
\begin{tabular}{|l|c|c|c|}
\hline
%\rowcolor{gray!30}
Cell type & PanNuke & MoNuSAC & Refined \\
\hline
Neoplastic & 77,403 (68,031) & - & 105,451 \\
\hline
Epithelial & 26,572 (23,207) & - & 29,926 \\
\hline
Epithelial (benign and malignant) & - & 31,402 & - \\
\hline
Inflammatory & 32,276 & - & - \\
\hline
Lymphocytes & - & 37,045 (33,104) & 65,275 \\
\hline
Neutrophils & - & 1,355 (1,252) & 3,833 \\
\hline
Macrophage & - & 1,842 (1,695) & 3,410 \\
\hline
Dead & 2,908 & - & 2,908 \\
\hline
Connective & 50,585 & - & 50,585 \\
\hline
\end{tabular}
%
%}
\end{table}



\clearpage

\subsection{\label{chap:S5}Definition of validation metrics}
\counterwithin{equation}{subsection}
\renewcommand{\theequation}{\arabic{equation}}

\subsubsection{\label{chap:S5.1}R\textsuperscript{2}}
The coefficient of determination, denoted as $R^2$, is a statistical measure that represents the proportion of variance in the dependent variable that is predictable from the independent variables. In the context of cell quantification in pathology, $R^2$ is used to assess how well the predicted quantities of different cell types in a patch align with the actual quantities observed in the ground truth data, with higher values representing more accurate quantification. $R^2$ is defined as
\begin{equation*}
R^2 = 1 - \frac{\sum_{i=1}^n (y_i - \hat{y}_i)^2}{\sum_{i=1}^n (y_i - \bar{y})^2},
\end{equation*}
where $y_i$ represents the actual number of cells of a specific type in the $i$-th image, $\hat{y}_i$ represents the predicted number of cells of that type in the $i$-th image, $\bar{y}$ is the mean of the actual numbers across all images, and $n$ is the total number of images in the dataset.

The $R^2$ metric has a range of $(-\infty, 1]$. An $R^2$ of 1 indicates perfect prediction, where all predicted values exactly match the actual values. An $R^2$ of 0 suggests that the model explains none of the variability of the response data around its mean. If $R^2$ is negative, it indicates that the model performs worse than a model that simply predicts the mean of the actual values for all observations.

\subsubsection{\label{chap:S5.2}PQ}
Panoptic Quality ($PQ$) is a comprehensive metric used to evaluate the performance of segmentation models in tasks that require both instance segmentation and classification. $PQ$ provides a single score that encapsulates both the detection accuracy (i.e., how many objects were correctly identified) and the segmentation quality (i.e., how accurately the objects' boundaries were delineated). This metric is particularly useful in multiclass scenarios where each pixel is classified into distinct categories, such as different cell types in pathology images.

$PQ$ is calculated as the product of two terms: Detection Quality ($DQ$) and Segmentation Quality ($SQ$). It can be expressed as
\begin{equation*}
PQ = DQ \cdot SQ,
\end{equation*}
where
\begin{equation*}
DQ = \frac{TP}{TP + 0.5\, FP + 0.5\, FN},
\end{equation*}
\begin{equation*}
SQ = \frac{\sum_{(p, g) \in \mathcal{M}} IoU(p, g)}{TP}.
\end{equation*}
In these formulas, $TP$ denotes the number of correctly matched instances between ground truth and prediction, $FP$ denotes the predicted instances that have no corresponding ground truth, $FN$ denotes the ground truth instances that were not detected, $IoU(p, g)$ is the Intersection over Union for a pair of matched instances $p$ (prediction) and $g$ (ground truth), and $\mathcal{M}$ is the set of matched pairs.

The $PQ$ metric is calculated for each class and is averaged across classes to provide a global performance measure.

The $PQ$ score has a range of $[0, 1.0]$, where a higher score indicates better performance in both detecting and segmenting the instances correctly. A $PQ$ of 1 signifies perfect identification and segmentation of all instances, whereas a $PQ$ of 0 indicates that no instances were correctly identified and segmented.

\clearpage

\subsection{\label{chap:S6}Segmentation and Detection quality metrics for teacher and student models}

\begin{table}[h!]
\renewcommand{\arraystretch}{2.0}
\centering
\caption{Segmentation and detection quality for student and teacher models (CI 95\%)}
\label{tab:S6}
%\adjustbox{max width=\textwidth}{%
\begin{tabular}{|l|c|c|}
\hline
%\rowcolor{gray!30}
Metric & Teacher & Student \\
\hline
$SQ_{neoplastic}$ & 0.819 (0.815--0.823) & 0.824 (0.819--0.828) \\
\hline
$SQ_{lymphocyte}$ & 0.795 (0.788--0.802) & 0.790 (0.783--0.796) \\
\hline
$SQ_{connective}$ & 0.770 (0.762--0.776) & 0.780 (0.772--0.786) \\
\hline
$SQ_{dead}$ & 0.659 (0.623--0.688) & 0.657 (0.624--0.695) \\
\hline
$SQ_{epithelial}$ & 0.780 (0.770--0.790) & 0.788 (0.779--0.797) \\
\hline
$SQ_{macrophage}$ & 0.788 (0.760--0.810) & 0.757 (0.730--0.783) \\
\hline
$SQ_{neutrofil}$ & 0.782 (0.761--0.801) & 0.775 (0.759--0.792) \\
\hline
$DQ_{neoplastic}$ & 0.706 (0.692--0.719) & 0.727 (0.712--0.741) \\
\hline
$DQ_{lymphocyte}$ & 0.675 (0.656--0.698) & 0.713 (0.691--0.734) \\
\hline
$DQ_{connective}$ & 0.566 (0.546--0.584) & 0.583 (0.565--0.602) \\
\hline
$DQ_{dead}$ & 0.410 (0.361--0.465) & 0.435 (0.306--0.561) \\
\hline
$DQ_{epithelial}$ & 0.668 (0.639--0.694) & 0.673 (0.644--0.702) \\
\hline
$DQ_{macrophage}$ & 0.657 (0.583--0.727) & 0.615 (0.531--0.703) \\
\hline
$DQ_{neutrofil}$ & 0.691 (0.625--0.753) & 0.729 (0.679--0.778) \\
\hline
\end{tabular}
%
%}
\end{table}

\clearpage

\subsection{\label{chap:S7}QuPath integration method}
We adopt an integration strategy leveraging the paquo \cite{Bayer_AG} library, a Python package that enables direct interaction with QuPath’s internal API, thereby facilitating seamless data exchange without intermediate conversion steps. The data processing pipeline (\hyperref[fig:S7]{Appendix Figure S7}) begins with the acquisition of WSIs and their associated annotations from QuPath, which are represented as Shapely \cite{Gillies_Wel_etal._2024} polygons. Utilizing paquo, we directly read, create, and modify these annotations and detections within a QuPath project in the Python environment. Images are then cropped using these polygons and processed by cell segmentation and classification models employing standard vision processing toolkits such as OpenCV, pyvips, and PyTorch. Additionally, QuPath employs Groovy scripts to initiate a Python process that starts the entire pipeline from QuPath graphical interface: fetching polygons, extracting images from them, and running deep learning model inference on the cropped images. 
The results are returned to QuPath, leveraging paquo's Python bindings to manipulate QuPath data while minimizing the computational overhead typically associated with cross-environment communication.

\counterwithin{figure}{subsection}
\renewcommand{\thefigure}{S\arabic{subsection}}

\begin{figure}[h!]
    \centering
    \includegraphics[width=\textwidth]{images/A7.pdf}
    \caption{QuPath integration workflow using Python environment}
    \label{fig:S7}
\end{figure}

Compared to traditional workflows that involve exporting annotations as GeoJSON, classifying them in Python, and reimporting them into QuPath, our approach offers several advantages. We eliminate the need to switch between programming languages, providing a cohesive and streamlined development process entirely within QuPath software and removing the necessity to use other tools. Meanwhile, we avoid storing annotations as intermediate JSON files unless required for external use or archiving. By conducting the entire inference and post-processing workflow within the Python environment, we leverage the power and flexibility of Python libraries for image processing and machine learning. This approach also enables adjustments to any set of labels and models, thereby improving its applicability.

%\hfill

The distilled model and QuPath integration code are packaged into a Docker container, enabling streamlined execution with the Docker engine. Detailed integration code and deployment instructions can be found in the GitHub repository \cite{Shvetsov_2025b}.

Despite these benefits, we acknowledge that the paquo library is a proof‑of‑concept project in its early development stage and has not been tested across all versions of QuPath.

\clearpage

\subsection{\label{chap:S8}Data and code availability statement}
All datasets, models, and code used in this study are publicly available and can be obtained from the repositories listed below. 
The PanNuke \cite{Gamper_Koohbanani_etal._2019} and MoNuSAC \cite{Verma_Kumar_etal._2021} datasets are publicly accessible, and download information along with detailed descriptions can be found in their respective articles. Preprocessing scripts for PanNuke and MoNuSAC data, as well as individual cell extraction scripts, are available on GitHub \cite{Shvetsov_2025a}. The H-Optimus foundation model used in our experiments can be downloaded from the HuggingFace repository \cite{hoptimus2024}, and model information is available on GitHub \cite{Saillard_Jenatton_etal._2024}. In addition, the integration code for QuPath and the distilled model packaged in a Docker container are provided in the repository \cite{Shvetsov_2025b}, and paquo Python library is available from the authors GitHub repository \cite{Bayer_AG}.
\clearpage

\end{document}


\backmatter

\bmhead{Supplementary information}
This manuscript is supported by additional supplementary information provided in a separate document.

\bmhead{Acknowledgements}
The authors acknowledge the Binnig and Rohrer Nanotechnology Center (BRNC) at IBM Research Europe - Zurich. Special thanks go to Jean-Michel Portal, Eloi Muhr and Dominique Drouin for their contributions to the design of the NMOS transistors used in this work. The authors also extend their gratitude to Stephan Menzel for the fruitful discussions and to Ralph Heller for his assistance in wire-bonding the chip. This work is funded by SNSF \textit{ALMOND} (grantID: 198612), by the European Union and Swiss state secretariat SERI within the H2020 \textit{MeM-Scales} (grantID: 871371), \textit{MANIC} (grantID: 861153), \textit{PHASTRAC} (grantID: 101092096) and \textit{CHIST-ERA UNICO} (20CH21-186952) projects.

\bmhead{Author contributions}
Conceptualization: D. F. F. and V. B.; hardware fabrication: D. F. F. and L. B. L.; electrical characterization: D. F. F, W. C., T. S., F. H., physical simulations: D. F. F.; inference and training simulations: V. C., D. F. F.; NMOS transistor design : N. G., F. A.; result interpretation: D. F. F., V. C., W. C., V. B., M. G., A. L. P. and B. J. O., supervision: V. B. and B. J. O.; manuscript writing: D. F. F., V. C.; data curation: D. F. F., V. C. and V. B.; manuscript review and editing: all authors; funding acquisition: B. J. O. and V. B.

\bmhead{Competing interests}
The authors declare no competing interests.

\bmhead{Data availability}
The data that support the plots within this paper and other findings of this study are available from the corresponding authors upon reasonable request.

\bmhead{Code availability}
The repositories containing the source codes used in this work for analog inference simulations and CMO/HfO\textsubscript{\textnormal{x}} ReRAM noise model can be found at \href{https://github.com/IBM/aihwkit/blob/master/examples/34_ReRAM_inference.py}{this link} and \href{https://github.com/IBM/aihwkit/blob/master/src/aihwkit/inference/noise/reram.py}{this link}, respectively.

%\appendix
%\renewcommand{\thefigure}{S\arabic{figure}} 
%\setcounter{figure}{0} 


\appendix
\renewcommand{\thefigure}{S\arabic{figure}} 
\setcounter{figure}{0} % Reset della numerazione delle figure

% ****** Start of file aipsamp.tex ******
%
%   This file is part of the AIP files in the AIP distribution for REVTeX 4.
%   Version 4.1 of REVTeX, October 2009
%
%   Copyright (c) 2009 American Institute of Physics.
%
%   See the AIP README file for restrictions and more information.
%
% TeX'ing this file requires that you have AMS-LaTeX 2.0 installed
% as well as the rest of the prerequisites for REVTeX 4.1
% 
% It also requires running BibTeX. The commands are as follows:
%
%  1)  latex  aipsamp
%  2)  bibtex aipsamp
%  3)  latex  aipsamp
%  4)  latex  aipsamp
%
% Use this file as a source of example code for your aip document.
% Use the file aiptemplate.tex as a template for your document.
\documentclass[%
 aip,
% jmp,
% bmf,
% sd,
% rsi,
 amsmath,amssymb,
%preprint
reprint, onecolumn %%%%% Comment onecolumn to override one column format %%%%%%%
%author-year,%
%author-numerical,%
% Conference Proceedings
]{revtex4-1}

\usepackage{graphicx}% Include figure files
\usepackage{dcolumn}% Align table columns on decimal point
\usepackage{bm}% bold math
%\usepackage[mathlines]{lineno}% Enable numbering of text and display math
%\linenumbers\relax % Commence numbering lines

\usepackage[utf8]{inputenc}
\usepackage[T1]{fontenc}
\usepackage{mathptmx}
\usepackage{etoolbox}

\usepackage{graphicx}% Include figure files
\usepackage{dcolumn}% Align table columns on decimal point
\usepackage{bm}% bold math
%\usepackage[mathlines]{lineno}% Enable numbering of text and display math
%\linenumbers\relax % Commence numbering lines

\usepackage[inkscapelatex=false]{svg}
\usepackage[utf8]{inputenc}
\usepackage[T1]{fontenc}
\usepackage{mathptmx}
\usepackage{etoolbox}

% Load the setspace package
\usepackage{setspace}
% Using \doublespacing in the preamble 
% changes the text to double-line spacing

%% Apr 2021: AIP requests that the corresponding 
%% email to be moved after the affiliations
\makeatletter
\def\@email#1#2{%
 \endgroup
 \patchcmd{\titleblock@produce}
  {\frontmatter@RRAPformat}
  {\frontmatter@RRAPformat{\produce@RRAP{*#1\href{mailto:#2}{#2}}}\frontmatter@RRAPformat}
  {}{}
}%
\makeatother
\begin{document}

\preprint{AIP/123-QED}

\title{\centering Epitaxial high-K barrier AlBN/GaN HEMTs}
%\title{AlBN: Integrating High-K Dielectric Functionality in GaN High Electron Mobility Transistors}

\author{\fontsize{13}{13}Chandrashekhar Savant\textsuperscript{1}$^\dagger$, Thai-Son Nguyen\textsuperscript{1}$^\dagger$, Kazuki Nomoto\textsuperscript{2}$^\dagger$, Saurabh Vishwakarma\textsuperscript{3}, Siyuan Ma\textsuperscript{1}, Akshey Dhar\textsuperscript{1}, Yu-Hsin Chen\textsuperscript{1}, Joseph Casamento\textsuperscript{1,4}, David J. Smith\textsuperscript{5}, Huili Grace Xing\textsuperscript{1,2,6}, and Debdeep Jena\textsuperscript{1,2,6,7}\\
\fontsize{10}{11}\selectfont \textsuperscript{1}Department of Materials Science and Engineering, Cornell University, Ithaca, NY 14853, USA, \\ \textsuperscript{2}School of Electrical and Computer Engineering, Cornell University, Ithaca, NY 14853, USA, 
\\ \textsuperscript{3}School for Engineering of Matter, Transport and Energy, Arizona State University, Tempe, AZ 85287, USA, \\\textsuperscript{4}Kavli Institute at Cornell for Nanoscale Science, Cornell University, Ithaca, NY 14853, USA, \\ \textsuperscript{5}Department of Physics, Arizona State University, Tempe, AZ 85287, USA, \\\textsuperscript{6}School of Applied and Engineering Physics, Cornell University, Ithaca, NY 14853, USA
\\($^\dagger$Equal Contribution. Email: cps259@cornell.edu)
}

\maketitle


\textbf{SUPPLEMENTARY MATERIAL}

\setcounter{figure}{0}
\renewcommand{\figurename}{Fig.}
\renewcommand{\thefigure}{S\arabic{figure}}

\vspace{2mm}
\begin{figure*}[h]
\includegraphics[width=\textwidth]{SFig1.eps}
\caption{\label{figS1} Room temperature (a) sheet resistance, (b) Hall mobility, and (c) 2DEG sheet density maps for 2.2 nm AlBN/GaN heterostructure mapped across a 3-inch diameter wafer. Though there is some non-uniformity across the wafer, the large area growth of AlBN/GaN heterostructures is demonstrated.}
\end{figure*}


\vspace{0mm}
\begin{figure*}[h]
\includegraphics[width=\textwidth]{SFig-2_CVdiffdies.eps}
\caption{\label{figS2} (a), (b), (c) Measured CV characteristics of 2.2 nm AlBN/GaN-based metal-insulator-semiconductor (MIS) capacitors and control 2 nm AlN/GaN-based MIS capacitors at 500 kHz, 20 $^\circ$C with top electrodes of 20 $\mu$m, 40 $\mu$m, and 80 $\mu$m diameters, respectively. A higher capacitance was observed for the AlBN barrier-based devices than that of the control AlN barrier-based devices with comparable top electrode sizes and barrier thicknesses.}
\end{figure*}

\vspace{0mm}
\begin{figure*}[h]
\includegraphics[width=15cm]{SFig-3_CVdifffreq.eps}
\caption{\label{figS3} Measured CV characteristics of a 2.2 nm AlBN/GaN-based metal-insulator-semiconductor (MIS) capacitor and a control 2 nm AlN/GaN-based MIS capacitor at 20 $^\circ$C with top electrode diameter of 20 $\mu$m  at (a) 100 kHz, (b) 500 kHz, (c) 1 MHz and (d) 2 MHz frequencies, respectively.}
\end{figure*}

\vspace{1mm}
\begin{figure*}
\includegraphics[width=8cm]{SFig-4_Cvsfreq.eps}
\caption{\label{figS4} Measured capacitance at 0 V vs. frequency plot for a 2.2 nm AlBN/GaN-based MIS capacitor and a control 2 nm AlN/GaN MIS capacitor at 20 $^\circ$C.}
\end{figure*}


\end{document}
%
% ****** End of file aipsamp.tex ****** 

\end{document}
\typeout{get arXiv to do 4 passes: Label(s) may have changed. Rerun}
