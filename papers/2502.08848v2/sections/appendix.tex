\section{Appendix}

\subsection{Localization algorithm}  \label{label:gcc_phat}

The following GCC-PHAT formulation was used: 
\begin{equation}
G(f) = \mathcal{F}^{-1}\bigg(\frac{X_1(f)[X_2(f)]^*}{(|X_1(f)[X_2(f)]^*|)^{-0.3}}\bigg) ,
\end{equation}

where $X_1(f)$ and $X_2(f)$ are the Fourier transforms of the two microphone signals, $[]^*$ denotes complex conjugate, and $\mathcal{F}^{-1}$ is the inverse Fourier transform. $G(f)$ is the resulting cross-correlation. 
We use partial normalization to the power of -0.3 since it provides more robustness to noise by giving less weight to delays around end-fires, which are more likely due to noise. Partial normalization deviates from the original GCC-PHAT as it uses full normalization to weigh all delays equally.

The time delay between the two microphones is extracted from the cross-correlation in the following way: 
\begin{equation}
\Delta t = \frac{argmax(G(f))}{f_s} ,  
\end{equation}

Where $f_s$ is the audio sampling frequency (44.1 kHz), and $argmax$ is the index of the maximum peak in the cross-correlation, corresponding to delay in samples. 

To convert the time delay into azimuth angle (in the microphone plane), the microphone spacing needs to be known to calculate the maximum delay: 

\begin{equation}
\Delta t_{max} = \frac{\Delta d}{c} ,
\end{equation}

where $c = 343 m/sec$ is the speed of sound, and $\Delta d$ is the distance between the microphones. 

Assuming far-field sound waves, we can use a simple formula to calculate the azimuth angle. The far-field approximation assumes planar sound waves and is valid approximately if the microphone is a meter or more away from the source. 
\begin{equation}
\theta_{azimuth} = cos^{-1}(\frac{\Delta t}{ \Delta t_{max}})    
\end{equation}