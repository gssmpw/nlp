\section{Introduction}
\label{sec:intro}
In recent years, blockchain technology has been widely applied to solve problems in various domains, developing innovative solutions previously considered impossible.  
It all began with an inventive ledger design to record all transactions generated in a decentralized system called Bitcoin, invented by \cite{nakamoto2008bitcoin}.  
This revolutionary ledger design maintains a sequentially growing list of blocks, each containing several transactions and linked to the preceding block using cryptographic techniques.  
The process of adding new blocks is governed by a consensus mechanism called Proof of Work (PoW), in which participants, called miners, use their computational power to calculate a hash function.
The miner who finds a valid solution for this hash function can have the right to produce a new block and earn rewards for generating it.  
In the design of the Bitcoin protocol, each miner is expected to broadcast the generated block to the network immediately.  
As long as all participants behave honestly, the expected revenue will be proportional to their computational power.  

However, in practice, miners 
are economically rational and profit-seeking.
They may adopt strategic behaviors, implying that honestly broadcasting a block is not necessarily the most rewarding strategy.
This is indeed the case, as evidenced by the study in \cite{eyal2014majority}, which introduced the selfish mining strategy—arguably the most notorious mining attack in blockchain.
In this attack, a miner strategically delays broadcasting the blocks they mine, causing other miners to generate blocks at invalid positions and inducing honest miners to waste their mining power.  
As a result, the strategic miner can earn more (expected) revenue than their fair share. 
Pushing this approach to the extreme, \cite{sapirshtein2016optimal} expanded the action space of selfish mining, modeling it as a Markov Decision Process (MDP), and analyzed the optimal mining strategy for a miner when facing other honest miners.  
A series of works have since been initiated to study mining strategies using this approach, such as \cite{feng2019selfish,grunspan2020selfish,li2021new,ferreira2022optimal}.  

However, with the continuous updates of blockchain consensus and the introduction of new protocols, directly computing a miner's strategy using MDP faces computational difficulties. 
To address this issue,~\cite{hou2019squirrl} proposed a generalizable framework for using reinforcement learning (RL) to analyze blockchain incentive mechanisms.  
Using this approach, researchers only need to model the states and strategies from the miner's perspective in an MDP and then use machine learning methods to learn an approximate optimal strategy.  
This method provides a framework for the detailed analysis of various blockchain protocols and attack patterns, including~\cite{bar2022werlman,bar2023deep,sarenche2024deep}.

In this survey, we provide a comprehensive overview of blockchain strategic mining analysis.  
We first summarize the MDP modeling approaches to analyze miners' strategic mining behavior and review the resulting security thresholds for attackers in different types of consensus protocols.  
Next, we focus on summarizing the existing findings on miners' strategic behavior using RL methods and compare the learning techniques employed as well as the resulting security thresholds.  
Finally, we introduce the MDP modeling paradigms for other consensus protocols in blockchain, such as voting-based and parallel confirmation protocols. 
We also propose several open problems and discuss the potential of using reinforcement learning to analyze these protocols.

\paragraph{The differences between our survey and others.}
Past surveys on strategic mining mainly focus on how to prevent mining attacks.
~\cite{madhushanie2024selfish} focuses solely on the harms of selfish mining attacks and analyzes existing detection and mitigation methods,
while ~\cite{nicolas2020blockchain} focuses on defending against double-spending attacks and selfish mining attacks, proposing various defense strategies.
However, no existing work has systematically surveyed the analytical methods for strategic mining. 
Our paper fills this important gap.

\paragraph{Roadmap.}
In Section~\ref{sec:MDP}, we introduce the MDP modeling method for consensus strategies and the definition and results of the security threshold.  
Then, in Section~\ref{sec:RL}, we present the results of using RL methods to analyze the strategy for mining blocks.  
After that, in Section~\ref{sec:consensus}, we provide an overview of the classification of blockchain consensus mechanisms and how analysis methods are applied within each category. 
Finally, we summarize this survey in Section~\ref{sec:conclusion}.  