\section{Conclusion and Open Problems}

In this section, we introduce the blockchain system model and give an overview of different types of blockchain consensus.

\subsection{System Model of Blockchain}
The blockchain is a continuously growing ledger composed of records known as blocks, which are interlinked through cryptographic mechanisms. 
Each block contains a cryptographic hash of the previous block, a timestamp, and transaction data typically structured in a Merkle tree. 
This architecture ensures the blockchain's resilience against unauthorized data modifications while providing transparent and traceable information accessible to all participants. 
Essentially, the blockchain operates as a decentralized and shared database, where each participant maintains a copy of the ledger, ensuring consensus and data integrity across the network.

A block within the blockchain is divided into two main components: the block header and the block body. 
The block header includes critical elements such as the version number, timestamp, target hash bits, nonce, hash value of the preceding block, and the Merkle root, which represents the cryptographic summary of all transactions within the block. 
The block body, on the other hand, contains the transactional details. Each block is cryptographically linked to its predecessor, creating a chain that allows any node within the distributed network to continuously verify the integrity of the entire blockchain. 
This linkage not only secures the data but also enables the blockchain's decentralized nature, where no single entity has control, and all participants can independently validate the blockchain's integrity.

\subsection{Proof-based Consensus}

The miner granted the privilege of block validation integrates the new block into the existing chain, thereby extending the blockchain. 

This process results in the formation of the longest main chain, stretching from the Genesis block to the most recent one, effectively recording the comprehensive history of blockchain transactions.

\subsection{Byzantine-based Consensus}

In the early studies, the consensus protocol was mainly aimed at solving the problem of reaching an agreement among nodes in the distributed system (\cite{lamport1982byzantine,fischer1985impossibility,biely2011synchronous}. 
With the emergence of malicious nodes, the focus of consensus research is gradually developing towards Byzantine fault-tolerant protocol
\cite{castro1999practical,castro2022practical,cachin2001secure}. 
The blockchain, especially the public blockchain, must adopt the BFT consensus protocol because the system needs to resist the attack of malicious nodes, while the consortium blockchain and the private
blockchain can use the Byzantine fault-tolerant or crash fault-tolerant protocols because of the strict access mechanism for the nodes.


% 之后用一两句话总结我们看过的Consensus,简单总结+突出各自的特点。
% TBFT - 宋斌峰
Tendermint BFT, as a consensus algorithm within the Byzantine Fault Tolerant (BFT) category, distinguishes itself from the conventional Practical Byzantine Fault Tolerance (PBFT) by introducing innovative mechanisms aimed at enhancing the system's security and efficiency. Through the implementation of a precommit confirmation rule requiring assent from over two-thirds of participants, TBFT maintains its ability to effectively thwart the submission of malicious blocks even in scenarios where more than one-third of nodes are Byzantine. Furthermore, TBFT places a significant emphasis on a stake-weighted voting mechanism, aligning voting power directly with the economic stakes of nodes, thereby augmenting the network's decentralization and security. By adopting a weak synchrony model and incorporating timeout procedures, TBFT ensures high efficiency and practicality when confronted with network delays and node failures, thus exemplifying superior performance in sustaining the stable operation of blockchain networks.\cite{buchman2016tendermint}

% Maxbft - 曾婉莹
Another BFT-based consensus, MaxBFT,  is an optimized three-phase BFT consensus protocol for ChainMaker. It achieves high fault tolerance while ensuring system security, efficiency, and eventual consistency. In comparison to other BFT algorithms, MaxBFT introduces several optimizations. 
Notably, it streamlines the voting process, and simplifies the three-phase consensus message types (proposal and vote messages), reducing communication complexity. 
Besides, communication occurs in a star-shaped network, minimizing message propagation and reducing network communication overhead to \(O(n)\). 
Additionally, MaxBFT decouples liveness rules from safety rules, which enhances flexibility.\cite{yin2019hotstuff}
% DPoS - 梁诺靖

% Avalanche - 谢丽佳
Avalanche\cite{rocket2019scalable} employs network subsampling to design its consensus mechanisms, where nodes repeatedly select a small group of nodes at random to converge on a common bivalent state. 
Unlike traditional BFT-based and Nakamoto mechanisms, leaderless BFT protocols in Avalanche maintain probabilistic safety guarantees without requiring global communication or precise participant knowledge. 
Avalanche leverages DAG structure to facilitate concurrent operations, achieving high throughput and scalability while compromising strict consistency to some extent.

% Diem BFT - 李济宸
Diem BFT\cite{team2021diembft}
DiemBFT designs a novel leader election mechanism that achieves leader utilization. That is, the number of times a crashed leader is elected as leader is bounded. 
Second, DiemBFT changes leaders when they are considered faulty. The previous version used the HotStuff linear view change mechanism to replace faulty leaders.
DiemBFT embraces this design choice and adopts a quadratic view-change of failed leaders, which enables it to commit blocks in two steps instead of three in the steady state.
Introduce a leader reputation mechanism that provides leader utilization under crash faults while making sure that not all committed blocks are proposed by Byzantine leaders.

% PoST - 周波
The PoST (Proof of Space and Time) consensus \cite{moran2019simple} is different from other proof-of-work mechanisms in that it consumes a lot of computing power to compete to obtain block rewards. It is mainly used to build decentralized storage networks. In the PoST consensus, miners provide storage space to obtain incentive distribution, and the miner's mining efficiency is related to the miner's storage activity. In a blockchain network built on PoST, miners can obtain tokens by providing storage, and customers can hire miners to store data by spending tokens. The big difference here is that miners aren't just keeping the blockchain running, they're also providing storage services.

% Arweave, Everfinance - 陈宏崟

\subsubsection{DAG-based Consensus}

% Sui - 黄瀚霆,王菲帆
LUTRIS\cite{blackshear2023sui} is the basic protocol of the smart-contract platform Siu, which combines DAG-based consensuses with consensus-less approaches, thus allowing it to achieve high throughput paired with valuable features such as low latency and long-term stability. 
In its design, DAG-based consensus protocols such as Narwhale\cite{danezis2022narwhal} and Bullshark\cite{spiegelman2022bullshark} can realize massive concurrency of transactions, thus achieving the design goal of high throughput. 
However, DAG-based protocols introduce a certain amount of latency, and consensus-free protocol designs such as FastPay\cite{baudet2020fastpay} serve an important role in reducing latency to increase scalability.The combination of the two types of approaches achieves superior performance metrics of sub-second latency and sustained throughput of thousands of transactions per second.

%Maybe have others