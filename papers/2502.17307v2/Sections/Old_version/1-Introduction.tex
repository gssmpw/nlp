\section{Introduction}

%blockchain technique cryptocurrency: block contains transactions and participants' incentive/consensus mechanism
In recent years, blockchain technology has been widely applied to handle problems in various domains, developing innovative solutions in ways that have not been considered possible in the past.
It all started with an inventive ledge design to record all transactions generated in a decentralized system called bitcoin, invented by~\cite{nakamoto2008bitcoin}. 
This revolutionary ledger design maintains a sequentially growing list of blocks, each containing several transactions and linked with a preceding block with cryptographic techniques. 
The process of adding new blocks is governed by a consensus mechanism called Proof of Work, in which miners are rewarded with newly minted bitcoins.
This mechanism secures the network from malicious threats and stabilizes the rate of block creation, ensuring steady transaction processing. 
As a result, a permissionless and incentivized cryptocurrency system is established.

%A standout innovation of Bitcoin is its use of the consensus mechanism known as Proof of Work (PoW), where participants should complete the right to add new blocks by solving cryptographic puzzles.
%The first miner to resolve the puzzle receives newly minted bitcoins and transaction fees as rewards.
%This mechanism secures the network from malicious threats and stabilizes the rate of block creation, ensuring steady transaction processing. 
%Moreover, as long as participants act honestly, their expected earnings correlate directly with their computational power.

%下面是不是要说矿工incentive problem
Following this, numerous blockchain-based cryptocurrency protocols have been proposed, yet all participants face a common block competition issue.
For the convenience of readers, we choose a simple setting where each participant of the distributed system has its own account.
A transaction is either a creation of a fixed number of tokens assigned to the block miner, or a certain number of tokens switched hands. 
The owner of an account aims to optimize the total number of tokens in its account. 
The system is consistent if each transaction, supposedly known only to the involved participants, becomes the common knowledge of all.  

The problem cannot be reduced to a broadcast problem because of an incentive complication: 
Each participant may not be truthful if lying can bring in more tokens than executing the broadcast protocol honestly. 
It is one of the most fundamental problems in the study of distributed systems, as well as multi-agent systems. 

%multi-agent, MDP problem, Stocastic game. 有incentive, thus security problem become a game problem.

%Selfishmining is MDP strategy.

%Blockchain economic problem = solving game.

In the design of the consensus protocol
of the pioneering blockchain ecosystem, bitcoin \cite{nakamoto2008bitcoin} 
created a decentralized electronic payment system, operating without intermediaries or trusted third parties.
This marks a significant departure from traditional financial systems, offering a novel approach to secure transactions and store value in a distributed manner, across a network of nodes. 
%Each transaction is cryptographically linked to the preceding one, forming a chain of blocks, hence the term "blockchain". 
This design ensures immutability and transparency. 

A standout innovation of Bitcoin is its use of the consensus mechanism known as Proof of Work (PoW), where participants should complete the right to add new blocks by solving cryptographic puzzles.
The first miner to resolve the puzzle receives newly minted bitcoins and transaction fees as rewards.
This mechanism secures the network from malicious threats and stabilizes the rate of block creation, ensuring steady transaction processing. 
Moreover, as long as participants act honestly, their expected earnings correlate directly with their computational power.

%Bitcoin makes use of a consensus mechanism known as Proof of Work (PoW) to validate and to add a new block a step by a step to the blockchain. 

%In the PoW protocol, miners compete to solve complex mathematical puzzles. 
%The first miner to find a valid solution is rewarded with newly minted bitcoins and transaction fees. 
%This process secures the network against malicious actors and regulates the block creation rate, maintaining a consistent transaction throughput.
%Meanwhile, so long as all participants behave honestly in this system, one's expected revenue will be proportional to its computational power.


As the pioneering blockchain ecosystem, bitcoin \cite{nakamoto2008bitcoin} introduced a decentralized electronic payment system, operating without intermediaries or trusted third parties.
This marked a significant departure from traditional financial systems, offering a novel approach to secure transactions and store value in a distributed manner.
The concept of blockchain, as introduced by Bitcoin, revolves around a decentralized ledger that records all transactions across a network of nodes. 
Each transaction is cryptographically linked to the preceding one, forming a chain of blocks, hence the term "blockchain". 
This design ensures immutability and transparency. 
Altering any past transaction would require consensus from most network participants, making it computationally infeasible to tamper with the data.
%
% blockchain consensus and mining strategy
To maintain transaction execution, Bitcoin employs a consensus mechanism known as Proof of Work (PoW) to validate and add new blocks to the blockchain. 
In the PoW protocol, miners compete to solve complex mathematical puzzles. 
The first miner to find a valid solution is rewarded with newly minted bitcoins and transaction fees. 
This process secures the network against malicious actors and regulates the block creation rate, maintaining a consistent transaction throughput.
Meanwhile, so long as all participants behave honestly in this system, one's expected revenue will be proportional to its computational power.
%
% Safety and security issues on blockchain consensus problem and miner's strategies behaviors.
However, in practice, miners are rational and may act strategically to get more revenue.
\cite{eyal2014majority} firstly proposed a mining strategy called Selfish mining, in which miners can strategically hide and propose the block to maintain more mining profits.
This indicates that the Bitcoin mining protocol is not incentive-compatible.
The key idea behind the attack is to induce honest miners to waste
their mining power. 
As a result, the selfish pool could obtain more revenue than its fair share.
Because of this, many researchers have studied the safety and security problems of consensus protocols and achieved some results.
%
% Our contribution
In this paper, we present a comprehensive survey focusing on the safety and security aspects of blockchain consensus protocols and their related concepts.
The rest of the paper is organized as follows:
% 别人做的哪些,以及把我们要做的放进去衔接起来。

% Open problem: 
% 1 Multi-agent情景,可以做insightful mining。
% 2 安全指标和度量框架
% 3 延迟的模型刻画
% 在这篇论文中,我们系统性得整理了区块链中策略性挖矿,以及其对共识安全影响的相关研究。同时,我们也