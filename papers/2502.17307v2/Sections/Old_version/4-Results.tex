\section{Results}

% 1.在基础Consensus基础上,考虑了哪些特殊的环境设计
% 2.Selfish(Strategic) miner strategy space
% 3.Honest(Other) miner's action
% 4.MDP/Learning Solving Methods or Technique
% 5.Results: security threshold, some insight to explain it.



%\begin{table*}[ht]
%\centering
%\caption{Consensus Mechanism Classification}
%\renewcommand{\arraystretch}{1.5} % Adjust row height
%\setlength{\tabcolsep}{5pt} % Adjust column spacing
%\begin{tabular}{|p{3cm}|p{3cm}|p{5cm}|p{6cm}|}
%\hline
%\multicolumn{2}{|c|}{\textbf{Category}} & \multicolumn{1}{c|}{\textbf{Consensus Protocols}} & \multicolumn{1}{c|}{\textbf{Description}} \\ \hline
\multicolumn{1}{|c|}{}                                    & Longest Chain Rule  & Bitcoin\cite{nakamoto2008bitcoin}, Ethereum 1.0\cite{wood2014ethereum}, FruitChain\cite{pass2017fruitchains}                        & Based on PoW, selecting the longest chain by computational work. \\ \cline{2-4}
\multicolumn{1}{|c|}{\multirow{-2}{*}{Chain-based Rules}} & Heaviest Chain Rule & Ouroboros\cite{kiayias2017ouroboros}, PoST \cite{moran2019simple} & Based on PoS or storage, selecting the chain with the highest accumulated weight. \\ \hline
\multicolumn{2}{|c|}{Vote-based Rules}                                          & PBFT\cite{castro1999practical}, Tendermint \cite{buchman2016tendermint}, Diem BFT \cite{team2021diembft}, HotStuff\cite{yin2019hotstuff}, Ethereum 2.0\cite{buterin2020combining} & Deterministic consensus with a voting mechanism, ensuring finality without forks and randomness. \\ \hline
\multicolumn{2}{|c|}{Parallel Confirmation Rules}                                           & Avalanche \cite{rocket2019scalable}, Sui \cite{blackshear2023sui} & Probabilistic consensus based on DAG, supporting high throughput and scalability. \\ \hline
\end{tabular}
\end{table*}


\subsection{Proof of Work Consensus}

\textbf{Selfish mining} - Jichen Li
It has long been believed that the Bitcoin protocol is incentive-compatible. However, Eyal and Sirer~\cite{eyal2014majority} indicate this is not the case. It describes a well-known attack called selfish mining. A pool could receive higher rewards than its fair share via the selfish mining strategy. This attack ingeniously exploits the conflict-resolution rule of the Bitcoin protocol, in which when encountering a fork, only one chain of blocks will be considered valid. With the selfish mining strategy, the attacker deliberately creates a fork and forces honest miners to waste efforts on a stale branch. Specifically, the selfish pool strategically keeps its newly found block secret rather than publishing it immediately. Afterward, it continues to mine on the head of this private branch. When the honest miners generate a new block, the selfish pool will correspondingly publish one private block at the same height and thus create a fork. Once the selfish pool's leads are reduced to two, an honest block will prompt it to reveal all its private blocks. As a well-known conclusion, assuming that the honest miners apply the uniform tie-breaking rule, if the fraction of the selfish pool's mining power is greater than $25\%$, it will always get more benefit than behaving honestly.

\textbf{Optimal Selfish mining} - Jichen Li
\cite{sapirshtein2016optimal} extend the underlying model for selfish mining attacks, and provide an algorithm to find Q-optimal policies for attackers within the model, as well as tight upper bounds on the revenue of optimal policies. 
As a consequence, the algorithm are able to provide lower bounds on the computational power an attacker needs in order to benefit from selfish mining. 
Paper find that the profit threshold – the minimal fraction of resources required for a profitable attack – is strictly lower than the one induced by the \cite{eyal2014majority} scheme. Indeed, the policies given by our algorithm dominate orginal selfish mining strategy, by better regulating attack-withdrawals.

\textbf{A Better Method to Analyze Blockchain Consistency} - Jichen Li
When considering how to analyze a PoW blockchain protocol, the formal frameworks only analyze the consistency and liveness of the chain.
Paper \cite{kiffer2018better} provides a Markov-chain based method for analyzing the consistency properties of blockchain protocols.
They consider a partially synchronous network that proposed blocks in the rounds and an adaptive corrupt adversary.
The adversary can break blockchain consistency by processing a family of delaying attacks, in which they will withhold their block and broadcast it when any honest miner finds a block.
By analyzing this attack, the authors show strong concentration bounds to demonstrate how long a participant should wait before considering a high-value transaction to be confirmed.

\textbf{Lay Down the Common Metrics: Evaluating Proof-of-Work Consensus Protocols’ Security} - Binfeng Song, Lijia Xie
\cite{zhang2019lay} employed a multi-metric evaluation framework based on Markov decision processes to quantitatively assess the chain quality and attack resistance of several existing enhanced PoW protocols. In this framework, the metric of incentive compatibility serves as an indicator of a protocol's resistance to selfish mining.
In the analysis of selfish mining strategies using MDP modeling, 
selfish miners can secretly withhold their discovered blocks, revealing their chain only when it surpasses the length of the public chain or when their lead is reduced to a single block. Conversely, honest miners adheres to protocol rules, promptly publishing blocks and choosing the longest chain for mining. The conclusions suggest that current enhanced PoW protocols fail to achieve optimal chain quality or robust attack resistance. This shortfall is attributed to the inherent dilemma in existing protocols between rewarding malicious actions and penalizing adherence to the protocol.

\textbf{SquirRL} - Wanying Zeng, Bo Zhou
\cite{hou2019squirrl} introduce SquirRL, a framework that leverages deep reinforcement learning to explore vulnerabilities in blockchain incentive mechanisms and recover adversarial strategies. Application of SquirRL successfully uncovers previously known attacks, including the optimal selfish mining attack in Bitcoin \cite{sapirshtein2016optimal}, and the Nash equilibrium in block withholding attacks\cite{eyal2015miner}. 

SquirRL applies to selfish-mining evaluation in blockchain consensus/incentive protocols through the following steps:
(a)Environment Construction: The environment involves features and action spaces reflecting the views and capabilities of participating agents. SquirRL considers different environment designs, including single selfish miner, and multiple selfish miners, as well as dynamic and stochastic environment designs, on top of the basic consensus protocol.
(b)Adversarial Model Selection: The protocol designer selects an adversarial model, including the numbers and types of agents, to explore various adversarial strategies.
(c)RL Algorithm Selection: The protocol designer selects an RL algorithm appropriate for the environment and adversarial model, associating it with a reward function and hyperparameters. SquirRL utilizes deep reinforcement learning (DRL) algorithms to solve Markov Decision Processes (MDPs), including both value-based methods and policy gradient methods.
(d)Training and Evaluation: SquirRL trains DRL agents in the selected environment and evaluates the performance of various strategies, including selfish mining, against baseline strategies and known theoretical results.

The paper yields novel empirical insights. Firstly, a counterintuitive flaw is identified in the widely used rushing adversary model when applied to multi-agent Markov games with incomplete information.
Secondly, contrary to previous assumptions, the optimal selfish mining strategy identified in \cite{sapirshtein2016optimal} is not a Nash equilibrium in the multi-agent selfish mining context. The results suggest (though not conclusively proven) that when more than two competing agents engage in selfish mining, no profitable Nash equilibrium exists. This observation aligns with the lack of observed selfish mining behavior in real-world scenarios.
Thirdly, a novel attack is uncovered on a simplified version of Ethereum’s finalization mechanism, Casper the Friendly Finality Gadget (FFG). This attack allows a strategic agent to amplify her rewards by up to 30\%.


\textbf{WeRLman} - Hanting Huang, Nuojing Liang
Incentives are crucial to ensuring the security of proof-of-work blockchain protocols. In the operation of blockchain systems, whale transactions, characterized by offering additional transaction costs, occur occasionally. WeRLman is the first selfish-mining analysis model considering both subsidy and transaction fees\cite{bar2022werlman}. Like Squirl, WeRLman describes blockchain through a Markov Decision Process (MDP). To cope with the complexity of the model and the large policy space, WeRLman utilizes a deep reinforcement learning framework inspired by the principles of AlphaGo Zero and incorporates Monte Carlo Tree Search (MCTS) and Deep Q Network (DQN) methods. Through experiments extracted from the Bitcoin platform data, analysis based on WeRLman reveals a clear inverse relationship between fee changes and the system security threshold. It is worth noting that the security threshold is considerably lower in the case of whale transactions compared to the security threshold of 0.25 analyzed by the previous model built under constant rewards. Specifically, based on Bitcoin's historical fees with its minting strategy, the deviation thresholds would be reduced to 0.2 in 10 years, 0.17 in 20 years, and 0.12 in 30 years. Based on the current transaction costs of the Ethernet smart contract platform, the security threshold would be reduced to 0.17, which is below the common sizes of large miners.

\textbf{Deep Bribe: Predicting the Rise of Bribery in Blockc-hain Mining with Deep RL} - Jiawei Nie, Feifan Wang 
\cite{bar2023deep}Building upon WeRLman, the article further considers the impact of petty compliant miners on the security threshold. Specifically, as transaction fees relative to the block's inherent reward (BTC) gradually increase, some miners tend to favor chains with more transaction fees when producing forks, leading to the emergence of undercutting attacks. In this environment, selfish miners can attract petty compliant miners to mine on their forked chain by creating equally long chains with lower transaction fees, resulting in a lower security threshold. For the MDP Solving Method, PTO is first used for MDP transformation, followed by WeRLman for solving (based on MCTS and DQN). In the model of this article, when the proportion of petty compliant miners $\beta\geq$  0.75, the obtained security thresholds are respectively below 0.19 (additional transaction fee F = 0.14) and 0.13 (F = 0.74). This reflects that when considering the MEV represented by transaction fee differences, undercutting attacks (or bribery) can exacerbate the threat of selfish mining.

\textbf{Insightful Mining} - Jichen Li
In this paper, first, we propose a novel strategy called insightful mining to counteract the selfish mining attack. 
By infiltrating an undercover miner into the selfish pool, the insightful pool could acquire the number of its hidden blocks.
We prove that, with this extra insight, the utility of the insightful pool is strictly greater than the selfish pool’s when they have the same mining power. 
Then we investigate the mining game where all pools can choose to be honest or take the insightful mining strategy. 
We characterize the Nash equilibrium of such a game and derive three corollaries: 
(a) each mining game has a pure Nash equilibrium; 
(b) there are at most two insightful pools under some equilibrium no matter how the mining power is distributed; 
(c) honest mining is a Nash equilibrium if the largest mining pool has a fraction of mining power no more than 1/3.
Our work explores, for the first time, the idea of spying in the selfish mining attack, which might shed new light on researchers in the field.

\textbf{Undetectable Selfish Mining} - Feifan Wang, Jichen Li
\cite{bahrani2023undetectable}This paper builds upon the selfish mining model proposed by Eyal and Sirer\cite{eyal2014majority}, introducing the concept of statistically detectable actions and designing selfish mining strategies that achieve statistical undetectability. Firstly, the paper extends the Nakamoto Consensus Game (NCG) to l-NCG by introducing a delay parameter, innovatively characterizing the natural occurrence rate and statistical distribution of orphan blocks using this delay parameter. Subsequently, the paper demonstrates that two classical selfish mining strategies—Selfish Mining and Strong Selfish Mining—are statistically detectable due to the presence of strategic players, which leads to non-independent and non-identically distributed probabilities of generating orphan blocks at different heights. Then, starting from Selfish Mining and Strong Selfish Mining, it achieves undetectability by modifying the block publication method while ensuring strict profitability. Specifically, strategic players devise appropriate Labeling Strategy and Broadcasting Strategy based on the state of the previous height, ensuring that the probability of generating orphan blocks at the next height remains constant at a specified value. Through simple mathematical derivation and exact MDP solutions verification, the paper indicates that when the hash rate of strategic players exceeds 0.382, suitable strategies can always be found to achieve additional profits in the sense of statistical undetectability.


\subsection{Proof of Stake Consensus}
\textbf{Deep Selfish Proposing in Longest-Chain Proof-of-Stake Protocols} - Hanting Huang, Qingmao Yao

Though the longest chain rule was originally designed for features of the proof-of-work (PoW) protocol, some PoS-based blockchain protocols have also adopted it as their consensus protocol. However, there is no mining process in PoS; instead, there is the election of proposers. The selfish mining attack is one of the most important threats that can undermine the longest-chain blockchain protocol. Not only PoW-based blockchains but also POS-based blockchains using the longest chain paradigm face the threat of selfish mining. Since POS-based blockchains do not involve a mining process, the term "selfish proposing" is used instead of selfish mining to describe this attack. Compared to PoW protocols, generating valid blocks in LC-PoS is effortless. Based on the phenomenon called nothing-at-stake, attackers can expand their attacking strategies and generate new attack actions. Also, PoS protocols suffer from a degree of predictability for the lottery mechanism that specifies the block proposer(s) for each slot. This work \cite{sarenche2024deep} considers the selfish proposing attack in the longest-chain PoS (LC-PoS) blockchains. By generalizing the nothing-at-stake selfish proposing attack with different levels of predictability, it is concluded that the proposing block ratio will be slightly increased with the Nothing-at-Stake phenomenon, and proposer predictability will considerably increase the attack block ratio. To analyze the selfish proposing attack in more complicated scenarios, this work uses a Deep Q-learning tool to analyze the selfish proposing attack and obtain the near-optimal attack strategy with varying stake shares.

\textbf{Formal Barriers to Longest-Chain Proof-of-Stake Protocols} - Lijia Xie, Binfeng Song

In the study by \cite{brown2019formal}, a model for PoS (Proof of Stake) cryptocurrencies was proposed to analyze security issues driven by incentives. The paper establishes two key properties: predictability and recency, ensuring that every protocol within the model satisfies at least one of these properties. Moreover, the security implications of predictability and recency indicate that every protocol is vulnerable to at least one type of attack, including selfish mining, double spending, or Nothing-at-Stake attacks. In contrast to the adversarial corruption found in PoW (Proof of Work) systems, the main findings of this paper highlight the formal barriers to designing incentive-compatible PoS cryptocurrencies, stemming from the inherent pseudorandomness of PoS systems.

\textbf{Optimal Strategic Mining Against Cryptographic Self-Selection in Proof-of-stake} - Bo Zhou, Wanying Zeng

The paper \cite{ferreira2022optimal} considers an adversary who aims to lead and win in the PoS election protocol, maximizing the number of rounds they can win, and indicates that the presence of an adversary can gain more benefits through strategy, regardless of the proportion of stacks they control. In the Cryptographic self-selection protocol, a leader needs to be elected for each round to generate blocks, and the election of the leader is related to the stakes held by the players and the seeds of current round of. The leader can generate biased seeds to make them advantageous in subsequent rounds of elections. Ferreira et al. showed that when the stack ratio mastered by strategy players is less than 0.38, they can win at most one round. They also proposed a 1-Lookahead strategy that can strictly outperform any honest strategy under any stack ratio. Finally, they proposed an algorithm to find the optimal strategy through MDP.

\section{Conclusion and Discussion}

