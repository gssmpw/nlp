\section{Consensus Protocol Classification and Open Problems}
\label{sec:consensus}

Consensus protocols are fundamental to blockchain systems, ensuring that all participants agree on transaction validity and the blockchain’s state, thereby preventing unauthorized modifications and preserving the network’s integrity. 
These protocols can be classified on the basis of their chain selection rules, which include chain-based, vote-based, and DAG-based (parallel confirmation) approaches.
When analyzing strategic mining across different consensus protocols, a key step is constructing the environment, tailored to the specific incentive mechanism and underlying consensus algorithm. 
In this section, we will explore the similarities and differences in constructing environments for various consensus protocols and propose meaningful open questions for future research.

\subsection{Consensus Protocol Overview}

\begin{figure*}[ht]
\centering

\begin{tikzpicture}[  
    node distance=0.5cm, % 节点之间的默认距离  
    every node/.style={align=center, font=\small, text centered, minimum height=1.5cm, rounded corners}, % 圆角矩形  
    every edge/.style={ultra thin, draw=black},  
]  

% 根节点  
\node[ellipse, draw=red!30, fill=red!10, text width=2cm] (root) {Consensus Mechanism};  

% 第一层节点(不同颜色)  
\node[rectangle, draw=blue!50, fill=blue!30, text width=3cm, right=of root, yshift=3.2cm] (chain) {Chain-based Rules};  
\node[rectangle, draw=orange!50, fill=orange!30, text width=6.75cm, right=of root] (vote) {Vote-based Rules};  
\node[rectangle, draw=green!50, fill=green!30, text width=6.75cm, right=of root, yshift=-2.2cm] (parallel) {Parallel Confirmation Rules};  

% 连接根节点到第一层节点  
\draw (root.east) -- ++(0.25,0) |- (chain.west);  
\draw (root.east) -- ++(0.25,0) |- (vote.west);  
\draw (root.east) -- ++(0.25,0) |- (parallel.west);  

% Chain-based Rules 的子节点(浅色版本)  
\node[rectangle, draw=blue!40, fill=blue!20, text width=3cm, right=of chain, yshift=1cm] (longest) {Longest Chain Rule};  
\node[rectangle, draw=blue!40, fill=blue!20, text width=3cm, right=of chain, yshift=-1cm] (heaviest) {Heaviest Chain Rule};  

% 连接 Chain-based Rules 到子节点  
\draw (chain.east) -- ++(0.25,0) |- (longest.west);  
\draw (chain.east) -- ++(0.25,0) |- (heaviest.west);  

% 最下面一层共识协议节点(浅色版本)  
\node[rectangle, draw=blue!40, fill=blue!10, text width=5cm, right=of longest] (bitcoin) {Bitcoin \cite{nakamoto2008bitcoin},\\ Ethereum 1.0 \cite{wood2014ethereum},\\ FruitChain \cite{pass2017fruitchains}};  
\node[rectangle, draw=blue!40, fill=blue!10, text width=5cm, right=of heaviest] (ouroboros) {Ouroboros \cite{kiayias2017ouroboros}, PoST \cite{moran2019simple}};  
\node[rectangle, draw=orange!40, fill=orange!10, text width=5cm, right=of vote] (pbft) {PBFT \cite{castro1999practical}, \\Tendermint \cite{buchman2016tendermint}, \\Diem BFT \cite{team2021diembft},\\ HotStuff \cite{yin2019hotstuff},\\ Ethereum 2.0 \cite{buterin2020combining},\\Algorand\cite{chen2019algorand}};  
\node[rectangle, draw=green!40, fill=green!10, text width=5cm, right=of parallel] (avalanche) {Avalanche \cite{rocket2019scalable}, \\ 
Sui \cite{blackshear2023sui}, \\
Conflux \cite{li2020decentralized}};  

% 连接最下面一层节点  
\draw (longest.east) -- ++(0.25,0) |- (bitcoin.west);  
\draw (heaviest.east) -- ++(0.25,0) |- (ouroboros.west);  
\draw (vote.east) -- ++(0.25,0) |- (pbft.west);  
\draw (parallel.east) -- ++(0.25,0) |- (avalanche.west);  
\end{tikzpicture}
\caption{Blockchain Consensus Protocol Classification}
\end{figure*}

%Consensus protocols are critical in blockchain systems as they ensure that all participants agree on the validity of transactions and the state of the blockchain. These protocols prevent unauthorized changes and secure the integrity of the distributed network by enabling all nodes to reach a common agreement. 
Blockchain consensus mechanisms can be classified based on their chain selection rules, including chain-based rules, vote-based rules, DAG-based rules.

\paragraph{Chain-based Consensus Rules.} 
Chain-based consensus rules rely on a linear blockchain structure, where blocks are linked in a chain, and the main chain is determined by the accumulated work or weight. 
The Longest Chain Rule, used in Proof-of-Work (PoW) systems like Bitcoin \cite{nakamoto2008bitcoin} and Ethereum 1.0 \cite{wood2014ethereum}, selects the chain with the most computational work. FruitChain \cite{pass2017fruitchains} extends the Longest Chain Rule by incorporating a dual-reward system, where miners earn rewards not only for block creation but also for broadcasting ``fruit'' structures, thus enhancing system security. 
Similarly, the Heaviest Chain Rule selects the chain with the greatest accumulated weight, typically based on stake, as seen in systems like Ouroboros \cite{kiayias2017ouroboros}. Additionally, Proof of Space and Time (PoST) \cite{moran2019simple} relies on the amount of storage and time invested to secure the chain, using a similar structure where the "heaviest" chain is the one that accumulates the most storage and time.

\paragraph{Vote-based Consensus Rules.} 
These consensus protocols involve nodes casting votes on the validity of transactions or blocks. 
In vote-based consensus systems, such as Practical Byzantine Fault Tolerance (PBFT) \cite{castro1999practical}, Tendermint \cite{buchman2016tendermint}, Algorand \cite{chen2019algorand} and HotStuff \cite{yin2019hotstuff}, a multi-phase voting process ensures that once a block is committed, it cannot be reverted, thus preventing forks. 
These protocols offer strong finality guarantees, with blocks being immediately finalized once they receive a sufficient number of votes.

\paragraph{Parallel Confirmation Rules.} Parallel confirmation rules deviate from traditional chain-based structures by allowing multiple branches to exist in parallel. 
In protocols like Avalanche \cite{rocket2019scalable}, Conflux \cite{li2020decentralized}, and Sui \cite{blackshear2023sui}, consensus is achieved probabilistically, with transactions validated concurrently across different branches. 
Since there is no main chain, consensus is reached without relying on a single-chain structure. 

It is also important to note that some consensus mechanisms may exhibit both primary and secondary attributes, blending features from different categories. 
For example, Ethereum 2.0 \cite{buterin2020combining} primarily relies on vote-based consensus (LMD-GHOST), while also incorporating the heaviest chain rule (via stake-weighted mechanisms) as a secondary feature.

\subsection{Design Components Across Different Consensus Protocols}

%The general MDP-based automated analysis framework for strategic mining focuses on identifying adversarial strategies targeting an incentive mechanism. 
%Constructing the environment and selecting an appropriate RL algorithm are key steps in this process. 
%The environment construction depends on the specific incentive mechanism and the underlying consensus algorithm. 
In this section, we outline the similarities and differences in environment construction for various consensus protocols, highlighting key design components such as \textit{state space}, \textit{action space}, and \textit{reward design}.

\paragraph{State Space.}
The state space must encode three critical components: (1) Action availability: features representing permissible actions in the current state, such as the fork status in Bitcoin (to track competing chains) \cite{sapirshtein2016optimal} or the match flag indicating active participation in protocols like LC-PoS \cite{sarenche2024deep}.
(2) Reward computation: features enabling reward calculation based on the canonical chain or subgraph. For example, in Bitcoin, this involves tracking the lengths of competing chains ($l_a, l_h$) to resolve forks \cite{nakamoto2008bitcoin}. For protocols with non-linear reward mechanisms (e.g., FruitChain’s "fruits" \cite{pass2017fruitchains,zhang2019lay} or Ethereum 2.0’s attestations \cite{zhang2024max}), additional metrics are required to compute relative rewards.
(3) State transition: The system state evolves through block generation, a discrete-time process with probability determined by mining power in PoW or stake in PoS. Transitions follow the consensus protocol’s stochastic rules and network assumptions, including idealized instant block propagation. During ties, honest nodes adopt adversarial blocks with a probability (rushing factor), modeling latency exploitation.


%As new blocks are generated, the system state transitions according to the consensus protocol's random rules and underlying network assumptions. Block generation is modeled as a discrete-time process, with the probability determined by mining power in PoW protocols or stake in PoS protocols. The network is assumed to propagate blocks instantly, and in the event of a tie, honest nodes follow the adversarial block with a probability known as the rushing factor.Accordingly, the state space must incorporate two key elements: i) the actions available in the current state, such as the fork status in Bitcoin \cite{sapirshtein2016optimal} and the match flag to indicate whether the action is active in LC-PoS protocol \cite{sarenche2024deep}; and ii) the ability to compute rewards based on specific states and actions. To calculate rewards, the state space should first involve the features that enable the identification of the canonical chain or subgraph, which in Bitcoin typically requires maintaining records such as the length of competing chains $l_a$ and $l_h$. Additionally, for rewards not solely based on block length, more detailed features need to be employed to compute relative rewards, such as the number of fruits in FruitChain \cite{pass2017fruitchains,zhang2019lay} or attestations in Ethereum 2.0 \cite{zhang2024max}.

\paragraph{Action Space.}
The design of the action space is closely tied to the adversarial model. 
For each consensus mechanism, different types of adversaries can be defined. 
For example, in the Bitcoin protocol, the action space is defined as {\textit{adopt, override, wait, match}}. 
Other attack strategies may involve actions outside this predefined space, such as the consideration of petty compliant miners in \cite{bar2023deep}. 
In the selfish proposing attack targeting the LC-PoS protocol \cite{sarenche2024deep}, the action space includes sub-actions to capture the `jump' strategy, allowing a selfish proposer to alter the parent block due to the ``nothing-at-stake'' property. 
In Ethereum 2.0, the action space may exclude the `match' action, as the rule for determining the canonical chain is based on the heaviest chain rather than the longest chain. 
This distinction removes the need to consider network propagation when competing chains have equal block lengths.

\paragraph{Reward Design.}
In reward design, previous approaches have primarily focused on relative rewards, defined as the attacker's rewards as a fraction of the total network rewards. 
These rewards are typically established on a per-unit basis, such as for blocks, as seen in earlier works. 
However, rewards can also be more intricately characterized, including transaction rewards \cite{bar2022werlman} and attestation rewards in Ethereum 2.0 \cite{zhang2024max}. 
The goal of the analysis is to identify a strategy $\pi: S \to \Delta(A)$ that maximizes the expected reward $R(s,a)$. 

%\paragraph{Objective Function.}The objective of analysis is to identify a strategy $\pi: S \to \Delta(A)$ that maximizes the expected reward $R(s,a)$. Accordingly, the \textit{state space} must incorporate two key elements: i) the actions available in the current state, such as the fork status in Bitcoin \cite{sapirshtein2016optimal} and the match flag to indicate whether the action is active in LC-PoS protocol \cite{sarenche2024deep}; and ii) the ability to compute rewards based on specific states and actions. To calculate rewards, the state space should first involve the features that enable the identification of the canonical chain or subgraph, which in Bitcoin typically requires maintaining records such as the length of competing chains $l_a$ and $l_h$. Additionally, for rewards not solely based on block length, more detailed features need to be employed to compute relative rewards, such as the number of fruits in FruitChain \cite{pass2017fruitchains,zhang2019lay} or attestations in Ethereum 2.0 \cite{zhang2024max}.

\subsection{Open Problems}
The evolution of blockchain technology and the digital economy has led to a significant shift from traditional longest-chain consensus mechanisms to alternative models. 
However, the economic security of these systems hinges on resolving critical open problems that remain inadequately addressed.
Moreover, the growing sophistication of miners in employing strategic mining techniques has introduced the risk of multi-agent strategic behaviors. These behaviors could destabilize the ecosystem, leading to economic inefficiencies or even systemic failures.
This section identifies important open problems in using RL for blockchain security analysis. 
Each problem highlights the need for advanced modeling techniques to address the complexities of modern blockchain systems.
%经济学安全
%每一组前面讲如果open problem做好了会有什么效果。

\paragraph{Open problem 1: How can extend strategic mining attack analysis to non-longest-chain consensus protocols?}

Existing research on strategic mining attacks has predominantly centered on longest-chain consensus protocols~\cite{eyal2014majority,sapirshtein2016optimal,sarenche2024deep}. 
To generalize this framework to alternative consensus mechanisms, three directions emerge:
\begin{itemize}
    \item \textbf{Weight-Based Protocols:} The state space must explicitly model weight accumulation dynamics. This requires incorporating weight-related parameters into the strategy space while preserving backward compatibility with existing analysis methods for longest-chain systems.
    \item \textbf{Parallel Proof-of-Work Protocols:}  The state space can be generalized to include additional features, such as topological order and uncle block rewards. These additions introduce multi-dimensional optimization challenges absent in linear chain protocols.
    \item \textbf{Vote-Based Consensus:} Participants attempt to minimize the costs associated with validating blocks and sending votes, which can result in coordination failures that undermine the validity of consensus protocols \cite{amoussou2020governing}. Furthermore, attackers can execute censorship attacks \cite{srivastava2024towards} by strategically excluding specific information from being incorporated into the final consensus.
\end{itemize}
Multiple tricks such as imposing artificial limits within the adversarial model \cite{sapirshtein2016optimal,zur2020efficient,hou2019squirrl} can help maintain a manageable state space size.
The application of the analysis framework to adversarial models targeting the aforementioned attacks still requires further exploration.

%For heaviest chain rule consensus, the state space can be modified to include weight-related information. 
%Similarly, for parallel proof-based consensus, the state space can be generalized to include additional features, such as topological order and uncle block rewards.
%The key challenge, however, is to incorporate the necessary features while avoiding excessive redundancy in the state space.
%Multiple tricks such as imposing artificial limits within the adversarial model \cite{sapirshtein2016optimal,zur2020efficient,hou2019squirrl} can help maintain a manageable state space size.

%For vote-based consensus, attackers exhibit distinct strategic behaviors.
% As mentioned above, strategic participants aim to maximize their expected rewards and deviate from the prescribed protocol only when such deviations offer higher expected payoffs.
% In vote-based consensus, participants attempt to minimize the costs associated with validating blocks and sending votes, which can result in coordination failures that undermine the validity or termination properties of consensus protocols \cite{amoussou2020governing}. Furthermore, attackers can execute censorship attacks \cite{srivastava2024towards} by strategically excluding specific information, such as transactions, from being incorporated into the final consensus.
% The application of the analysis framework to adversarial models targeting the aforementioned attacks still requires further exploration.

\paragraph{Open problem 2: How to develop more realistic MDP models for blockchain security?}

Current RL models for blockchain security rely on simplified assumptions, such as synchronized networks and fixed miner strategies \cite{zhang2019lay,sarenche2024deep}, to reduce complexity. However, in real-world environments, network conditions, miner strategies, and blockchain dynamics are highly unpredictable. For example, attackers may exploit network latency to conduct undetectable attacks, limiting the applicability of existing models \cite{bahrani2023undetectable}.

Future research should focus on relaxing these assumptions by incorporating dynamic, time-varying environments. RL models can account for unpredictable network delays, adaptive miner strategies, and changing blockchain dynamics. 
These uncertain conditions can be handle by using advanced RL algorithms, while ensuring robust security analysis under evolving threats.

\paragraph{Open problem 3: How can strategic mining be analyzed in multi-agent environments using RL?}
A key challenge is applying RL in multi-agent environments, where multiple miners or validators interact strategically to maximize rewards. In these environments, agents may compete, complicating the analysis of selfish mining attacks. 

\cite{marmolejo2019competing} use a Markov chain model to analyze multi-agent mining dynamics by simplifying the state space of selfish miners. This analysis restrict to semi-selfish mining, where miners only maintain private chains of length at most two. 
The Partially Observed Markov Game (POMG) extends MDP to multi-agent environments with partial information, allowing it to model strategic behaviors like selfish mining, such as SquirRL \cite{hou2019squirrl}. 
However, POMG has limitations, including assumptions about partial observability and challenges with scalability as the number of miners increases. Future research should focus on improving its scalability and refining agent interaction models to better capture the complexities and unpredictability of real-world blockchain dynamics.