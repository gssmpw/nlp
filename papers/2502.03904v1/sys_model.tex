\vspace{-1mm}
\section{Zak-OTFS System Model}
\label{sec2}
Figure \ref{fig1} shows the block diagram of a Zak-OTFS transceiver.
\begin{figure*}
\centering    \includegraphics[width=0.95\linewidth]{Figures/continuous_zak_otfs_bd.eps}
\caption{Block diagram of Zak-OTFS transceiver.}
\label{fig1}      
\vspace{-4mm}
\end{figure*}
In Zak-OTFS, a pulse in the DD domain is the basic information carrier. A DD pulse is a quasi-periodic localized function defined by a delay period $\tau_{\mathrm{p}}$ and a Doppler period $\nu_{\mathrm{p}}=\frac{1}{\tau_{\mathrm{p}}}$. The fundamental period in the DD domain is defined as 
$\mathcal{D}_{0}= \{(\tau,\nu): 0\leq\tau<\tau_{\mathrm p}, 0\leq\nu<\nu_{\mathrm p}\}$,
where $\tau$ and $\nu$ represent the delay and Doppler variables, respectively. The fundamental period is discretized into $M$ bins on the delay axis and $N$ bins on the Doppler axis, as 
$\big\{(k\frac{\tau_{{\mathrm p}}}{M},l\frac{\nu_{{\mathrm p}}}{N}) | k=0,\ldots,M-1,l=0,\ldots,N-1\big\}$. The time domain Zak-OTFS frame is limited to a time duration $T=N\tau_{\mathrm p}$ and a bandwidth $B=M\nu_{\mathrm p}$. In each frame, $MN$ information symbols drawn from a modulation alphabet ${\mathbb A}$, $x[k,l]\in {\mathbb A}$, $k=0,\ldots,M-1$, $l=0,\ldots,N-1$, are multiplexed in the DD domain. The information symbol $x[k,l]$ is carried by DD domain pulse $x_{\mathrm{dd}}[k,l]$, which is a quasi-periodic function with period $M$ along the delay axis and period $N$ along the Doppler axis, i.e., for any $n,m\in\mathbb{Z}$,  
\begin{equation}
x_{\mathrm{dd}}[k+nM,l+mN]=x[k,l]e^{j2\pi n\frac{l}{N}}.
\end{equation}
These discrete DD domain signals $x_{\mathrm{dd}}[k,l]$s are supported on the information lattice 
$\Lambda_{\mathrm{dd}}=
\big\{\big(k\frac{\tau_{\mathrm p}}{M},l\frac{\nu_{\mathrm p}}{N}\big) | k,l\in \mathbb{Z}\big\}$.
The continuous DD domain information signal is given by
\vspace{-1mm}
\begin{equation}
x_{\mathrm{dd}}(\tau,\nu)=\sum_{k,l\in \mathbb{Z}} x_{\mathrm{dd}}[k,l] \delta\Big(\tau-\frac{k\tau_{\mathrm p}}{M}\Big)\delta\Big(\nu-\frac{l\nu_{\mathrm p}}{N}\Big),
\end{equation}
where $\delta(.)$ denotes the Dirac-delta impulse function. For any $n,m\in \mathbb{Z}$, we have
$x_{\mathrm{dd}}(\tau+n\tau_{\mathrm{p}},\nu+m\nu_{\mathrm{p}})=e^{j2\pi n\nu \tau_{\mathrm{p}}}x_{\mathrm{dd}}(\tau,\nu)$,
so that $x_{\mathrm{dd}}(\tau,\nu)$ is periodic with period $\nu_{\mathrm p}$ along the Doppler axis and quasi-periodic with period $\tau_{\mathrm p}$ along the delay axis.

The DD domain transmit signal $x_{\mathrm{dd}}^{w_{\mathrm{tx}}}(\tau,\nu)$ is given by the twisted convolution of the transmit pulse shaping filter $w_{\mathrm{tx}}(\tau,\nu)$ with $x_{\mathrm{dd}}(\tau,\nu)$ as $x_{\mathrm{dd}}^{w_{\mathrm{tx}}}(\tau,\nu) = w_{\mathrm{tx}}(\tau,\nu)*_{\sigma}x_{\mathrm{dd}}(\tau,\nu)$,
where $*_{\sigma}$ denotes the twisted convolution\footnote{Twisted convolution of two DD functions $a(\tau,\nu)$ and $b(\tau,\nu)$ is defined as 
$a(\tau,\nu) \ast_\sigma b(\tau,\nu) \overset{\Delta}{=} \iint a(\tau', \nu') b(\tau-\tau',\nu-\nu')e^{j2\pi\nu'(\tau-\tau')}d\tau'  d\nu'$.}. The transmitted time domain (TD) signal $s_{\mathrm{td}}(t)$ is the TD realization of $x_{\mathrm{dd}}^{w_{\mathrm{tx}}}(\tau,\nu)$, given by
$s_{\mathrm{td}}(t)=Z_{t}^{-1}\left(x_{\mathrm{dd}}^{w_{\mathrm{tx}}}(\tau,\nu)\right)$, where $Z_{t}^{-1}$ denotes the inverse time-Zak transform operation\footnote{Inverse time-Zak transform of a DD function $a(\tau,\nu)$ is defined as $Z_{t}^{-1}(a(\tau,\nu)) \overset{\Delta}{=} \sqrt{\tau_{\mathrm p}} \int_0^{\nu_{\mathrm p}} a(t,\nu) d\nu$.}. The transmit pulse shaping filter $w_{\mathrm{tx}}(\tau,\nu)$ 
limits the time and bandwidth of the transmitted signal $s_{\mathrm{td}}(t)$. The transmit signal $s_{\mathrm{td}}(t)$ passes through a doubly-selective channel to give the output signal $r_{\mathrm{td}}(t)$. The DD domain impulse response of the physical channel $h_{\mathrm{phy}}(\tau,\nu)$ is given by
\begin{equation}
h_{\mathrm{phy}}(\tau,\nu)=\sum_{i=1}^{P}h_{i}\delta(\tau-\tau_{i})\delta(\nu-\nu_{i}),
\end{equation}
where $P$ denotes the number of DD paths, and the $i$th path has gain $h_{i}$, delay shift $\tau_{i}$, and Doppler shift $\nu_{i}$. 

The received TD signal $y(t)$ at the receiver is given by $y(t)=r_{\mathrm{td}}(t)+n(t)$,
where $n(t)$ is AWGN with variance $N_{0}$, i.e., $\mathbb{E}[n(t)n(t+t')]=N_{0}\delta(t')$. The TD signal $y(t)$ is converted to the corresponding DD domain signal $y_{\mathrm{dd}}(\tau,\nu)$ by applying Zak transform\footnote{Zak transform of a continuous TD signal $a(t)$ is defined as
$Z_t\left(a(t)\right) \overset{\Delta}{=} \sqrt{\tau_p} \sum_{k \in \mathbb{Z}} a(\tau + k \tau_{\mathrm p}) e^{-j2\pi\nu k\tau_{\mathrm p}}$.}, i.e.,
\begin{eqnarray}
\hspace{-6mm}
y_{\mathrm{dd}}(\tau,\nu) = Z_{t}(y(t)) 
= r_{\mathrm{dd}}(\tau,\nu)+n_{\mathrm{dd}}(\tau,\nu),
\end{eqnarray}
where $r_{\mathrm{dd}}(\tau,\nu)=h_{\mathrm{phy}}(\tau,\nu)*_{\sigma}w_{\mathrm{tx}}(\tau,\nu)*_{\sigma}x_{\mathrm{dd}}(\tau,\nu)$ is the Zak transform of $r_{\mathrm{td}}(t)$, given by the twisted convolution cascade of $x_{\mathrm{dd}}(\tau,\nu)$, $w_{\mathrm{tx}}(\tau,\nu)$, and $h_{\mathrm{phy}}(\tau,\nu)$,  and $n_{\mathrm{dd}}(\tau,\nu)$ is the Zak transform of $n(t)$. The receiver filter $w_{\mathrm{rx}}(\tau,\nu)$ acts on $y_{\mathrm{dd}}(\tau,\nu)$ through twisted convolution to give the output 
\begin{eqnarray}
\hspace{-4mm} 
y_{\mathrm{dd}}^{w_{\mathrm{rx}}}(\tau,\nu) & \hspace{-2mm} = & \hspace{-2mm} w_{\mathrm{rx}}(\tau,\nu)*_{\sigma}y_{\mathrm{dd}}(\tau,\nu) \nonumber \\ 
& \hspace{-22mm} = & \hspace{-12mm} \underbrace{w_{\mathrm{rx}}(\tau,\nu)*_{\sigma}h_{\mathrm{phy}}(\tau,\nu)*_{\sigma}w_{\mathrm{tx}}(\tau,\nu)}_{\overset{\Delta}{=} \ h_{\mathrm{eff}}(\tau,\nu)}*_{\sigma}x_{\mathrm{dd}}(\tau,\nu) \nonumber \\ 
&\hspace{-22mm} & \hspace{-12mm} + \ \underbrace{w_{\mathrm{rx}}(\tau,\nu)*_{\sigma}n_{\mathrm{dd}}(\tau,\nu)}_{\overset{\Delta}{=} \ n_{\mathrm{dd}}^{w_{\mathrm{rx}}}(\tau,\nu)}, 
\label{cont1}
\end{eqnarray}
where $h_{\mathrm{eff}}(\tau,\nu)$ denotes the effective channel consisting of the twisted convolution cascade of $w_{\mathrm{tx}}(\tau,\nu),\ h_{\mathrm{phy}}(\tau,\nu)$, and $w_{\mathrm{rx}}(\tau,\nu)$, and $n_{\mathrm{dd}}^{w_{\mathrm{rx}}}(\tau,\nu)$ denotes the noise filtered through the Rx filter. The DD signal $y_{\mathrm{dd}}^{w_{\mathrm{rx}}}(\tau,\nu)$ is sampled on the information lattice, resulting in the discrete quasi-periodic DD domain received signal $y_{\mathrm{dd}}[k,l]$ as
\vspace{0mm}
\begin{equation}
y_{\mathrm{dd}}[k,l]=y_{\mathrm{dd}}^{w_{\mathrm{rx}}}\left(\tau=\frac{k\tau_{\mathrm p}}{M},\nu=\frac{l\nu_{\mathrm p}}{N}\right), \ \ k,l\in\mathbb{Z},
\end{equation} 
which is given by
$y_{\mathrm{dd}}[k,l]=h_{\mathrm{eff}}[k,l]*_{\sigma\text{d}}x_{\mathrm{dd}}[k,l]+n_{\mathrm{dd}}[k,l]$,
where $*_{\sigma\text{d}}$ is twisted convolution in discrete DD domain, i.e., 
$h_{\mathrm{eff}}[k,l]*_{\sigma\text{d}}x_{\mathrm{dd}}[k,l] = \sum_{k',l'\in\mathbb{Z}}h_{\mathrm{eff}}[k-k',l-l']x_{\mathrm{dd}}[k',l'] e^{j2\pi\frac{k'(l-l')}{MN}}$, where the effective channel filter $h_{\mathrm{eff}}[k,l]$ and filtered noise samples $n_{\mathrm{dd}}[k,l]$ are given by
\begin{align}
h_{\text{eff}}[k,l]=h_{\text{eff}}\left(\tau=\frac{k\tau_{p}}{M},\nu=\frac{l\nu_{p}}{N}\right), \label{discr2} \\ 
n_{\text{dd}}[k,l]=n_{\text{dd}}^{w_{\mathrm{rx}}}\left(\tau=\frac{k\tau_{p}}{M},\nu=\frac{l\nu_{p}}{N}\right).
\label{discr3}
\end{align}
Owing to the quasi-periodicity in the DD domain, it is sufficient to consider the received samples $y_{\mathrm{dd}}[k,l]$ within the fundamental period $\mathcal{D}_0$. Writing the $y_{\mathrm{dd}}[k,l]$ samples as a vector, the received signal model can be written in matrix-vector form as \cite{zak_otfs1},\cite{zak_otfs2}
\begin{equation}
\mathbf{y}=\mathbf{H_\text{eff}x}+\mathbf{n},
\label{sys_mod}
\end{equation}
where $\mathbf{x,y,n} \in\mathbb{C}^{MN\times 1}$, such that their $(kN+l+1)$th entries are given by $x_{kN+l+1}=x_{\mathrm{dd}}[k,l]$, $y_{kN+l+1}=y_{\mathrm{dd}}[k,l]$, $n_{kN+l+1}=n_{\mathrm{dd}}[k,l]$, and $\mathbf{H}_{\text{eff}}\in\mathbb{C}^{MN\times MN}$ is the effective channel matrix such that
\vspace{-1mm}
\begin{eqnarray}
\mathbf{H}_\text{eff}[k'N+l'+1,kN+l+1] & \hspace{-2mm} = & \hspace{-2mm} \sum_{m,n\in\mathbb{Z}}h_{\mathrm{eff}}[k'-k-nM, \nonumber \\
& \hspace{-45mm} & \hspace{-35mm} l'-l-mN]e^{j2\pi nl/N}e^{j2\pi\frac{(l'-l-mN)(k+nM)}{MN}},
\label{eqn_channel_matrix}
\vspace{-4mm}
\end{eqnarray}
where $k',k=0,\ldots,M-1$, $l',l=0,\ldots,N-1$. 

\vspace{-2mm}
\subsection{DD pulse shaping filters}
In the absence of pulse shaping, i.e., $w_\text{tx}(\tau,\nu)=\delta(\tau,\nu)$, the transmit signal has infinite time duration and bandwidth. Pulse shaping limits the time and bandwidth of transmission. We consider transmit DD pulse shaping filters of the form $w_\text{tx}(\tau,\nu)=w_1(\tau)w_2(\nu)$ \cite{zak_otfs2},\cite{zak_otfs6}. The time duration $T'$ of each frame is approximately related to the spread of $w_2({\nu})$ along the Doppler axis as $\frac{1}{T'}$. Likewise, the bandwidth $B'$ is approximately related to the spread of $w_1(\tau)$ along the delay axis as $\frac{1}{B'}$. That is, a larger bandwidth and time duration implies a smaller DD spread of $w_\text{tx}(\tau,\nu)$, and hence a smaller contribution to the spread of $h_\text{eff}(\tau,\nu)$. Sinc, RRC, and Gaussian filters have been considered in the Zak-OTFS literature and are described below. 

{\em Sinc filter:}
For sinc filter, $w_1({\tau})$ and $w_2({\nu})$ are given by
$w_1({\tau}) = \sqrt{B}\text{sinc}(B\tau)$ and $w_2({\nu}) = \sqrt{T}\text{sinc}(T\nu)$,
so that 
\begin{equation}
w_\text{tx}(\tau,\nu)
=\underbrace{\sqrt{B}\text{sinc}(B\tau)}_{w_1(\tau)} \underbrace{\sqrt{T}\text{sinc}(T\nu)}_{w_2(\nu)}. 
\label{eq:sinc1}
\end{equation}
For sinc filter, the frame duration $T'=T$ and frame bandwidth $B'=B$ (i.e., there is no time or bandwidth expansion), resulting in a spectral efficiency of $\frac{BT}{B'T'}=1$ symbol/dimension. 

{\em RRC filter:}
For RRC filter, $w_\text{tx}(\tau,\nu)$ is given by
\begin{equation}
w_\text{tx}(\tau,\nu) = \underbrace{\sqrt{B} \ \text{rrc}_{\beta_\tau}(B\tau)}_{w_1(\tau)} \ \underbrace{\sqrt{T} \ \text{rrc}_{\beta_\nu}(T\nu)}_{w_2(\nu)},
\label{eq:rrc1}
\end{equation}
where $0\leq \beta_{\tau},\beta_\nu \leq 1$ and
\begin{equation}
\text{rrc}_{\beta}(x)=\frac{\sin \left(\pi x(1-\beta) \right)+4\beta x\cos \left(\pi x(1+\beta) \right)}{\pi x \left(1-(4\beta x)^2 \right)}. 
\end{equation}
It can be seen that the choice of $\beta_\tau=\beta_\nu=0$ in the RRC filter (\ref{eq:rrc1}) specializes to the sinc filter. Also, for $\beta_\nu>0$ and $\beta_\tau>0$, there is time and bandwidth expansion such that  $T'=T(1+\beta_{\nu})$ and $B'=B(1+\beta_{\tau})$, resulting in a spectral efficiency of $\frac{BT}{B'T'}<1$ symbol/dimension. 

{\em Gaussian filter:}
For Gaussian filter, $w_\text{tx}(\tau,\nu)$ is given by \cite{zak_otfs7}
\begin{equation}
\hspace{-2mm}
w_\text{tx}(\tau,\nu) \hspace{-0.5mm} = \hspace{-0.5mm} \underbrace{\left(\frac{2\alpha_{\tau}B^2}{\pi}\right)^{\frac{1}{4}}e^{-\alpha_{\tau}B^{2}\tau^{2}}}_{w_1(\tau)} \ \underbrace{\left(\frac{2\alpha_{\nu}T^2}{\pi}\right)^{\frac{1}{4}}e^{-\alpha_{\nu}T^{2}\nu^{2}}}_{w_2(\nu)}\hspace{-1mm}. \hspace{-0mm}
\label{eq:gauss1}
\end{equation}
The Gaussian pulse can be configured by adjusting the parameters $\alpha_{\tau}$ and $\alpha_{\nu}$. Because of the infinite support in Gaussian pulse, a time duration $T'$ where 99$\%$ of the frame energy is localized in the time domain and a bandwidth $B'$ where 99$\%$ of the frame 
energy is localized in the frequency domain are considered \cite{zak_otfs7}. No time and bandwidth expansion (i.e., $T'=T$, $B'=B$) in the Gaussian pulse corresponds to setting 
$\alpha_{\tau}=\alpha_{\nu}=1.584$. 
