\vspace{-1mm}
\section{Proposed Gaussian-Sinc DD filter} 
\label{sec4}
In the Zak-OTFS literature, sinc, RRC, and Gaussian pulse shaping filters have been considered. In this section, we present the rationale for a new pulse shaping filter for Zak-OTFS, the proposed Gaussian-sinc (GS) filter, and the derivation of closed-form expressions for the I/O relation and noise covariance with the proposed GS filter.  

\begin{figure}
\centering
\includegraphics[width=9cm,height=6cm]{Figures/Pulse_Shape.eps}
\caption{Delay pulse magnitude $|w_{1}(\tau)|$ (in dB) as a function of the normalized delay $B\tau$.}
\label{pulse_shapes}
\end{figure}

\vspace{-3mm}
\subsection{Rationale for a new filter}
In Fig. \ref{pulse_shapes}, we plot the delay pulse magnitude $|w_1(\tau)|$ in dB scale as a function of the normalized delay $B\tau$ for sinc, RRC, and Gaussian filters. Similar characteristics can be observed for Doppler pulse magnitude $|w_2(\nu)|$ as a function of the normalized Doppler $T\nu$. The sinc filter has ideal main lobe characteristics with nulls at the Nyquist sampling points on the DD grid (i.e., at $\tau=\frac{k}{B}$, $\nu=\frac{l}{T}$, $k,l \in \mathbb{Z}\backslash 0$). But it has the drawback of high side lobe levels. The RRC filter alleviates the issue of high side lobes in sinc filter through the choice of $\beta_\tau$ and $\beta_\nu$ parameters. But this is achieved at the expense of increased time and bandwidth, since $T'=T(1+\beta_\nu)$, $B'=B(1+\beta_\tau)$, and $0< \beta_\tau,\beta_\nu \leq 1$ . The Gaussian filter has very low side lobe levels, but it does not have good main lobe characteristics. In particular, it does not have nulls at the Nyquist sampling points. Instead, it has a high value closer to the peak value. These varied characteristics of the sinc, RRC, and Gaussian filters affect the receiver performance in different ways. For example, presence of nulls at the Nyquist points positively influences the equalization/detection performance, while having high non-zero values at these points has a negative influence on the equalization/detection performance. Likewise, very low side lobes positively influences the I/O relation estimation performance, while high side lobes influences it negatively. We illustrate the above points through the performance plots in Figs. \ref{fig:motiv_a}, \ref{fig:motiv_b}, \ref{fig:motiv_c}.
\begin{table}
\centering
\begin{tabular}{|c|c|c|c|c|c|c|}
\hline
Path index ($i$)         & 1 & 2    & 3    & 4    & 5    & 6    \\ \hline
Delay $\tau_{i}$ ($\mu s$)      & 0 & 0.31 & 0.71 & 1.09 & 1.73 & 2.51 \\ \hline
Relative power 
(dB) & 0 & -1   & -9   & -10  & -15  & -20  \\ \hline
\end{tabular}
\caption{Power delay profile of Veh-A channel model.}
\label{tab_pdp}
\vspace{-5mm}
\end{table}

\begin{figure*}
\hspace{2mm}
\subfloat[BER vs SNR with perfect CSI]  {\includegraphics[width=6cm,height=4.5cm]{Figures/BER_12X14_CSI.eps} \label{fig:motiv_a}}\hfill  \subfloat[MSE vs SNR with embedded pilot]
{\includegraphics[width=6cm, height=4.5cm]{Figures/MSE_12X14_embedded.eps} \label{fig:motiv_b}}
\subfloat[BER vs SNR with embedded pilot]
{\includegraphics[width=6cm, height=4.5cm]{Figures/BER_12X14_embedded.eps} \label{fig:motiv_c}}
\caption{Performance of sinc and Gaussian filters (a) with perfect CSI and (b),(c) with model-free I/O relation estimation.}
\vspace{-4mm}
\label{fig:motiv}
\end{figure*}

For generating the performance plots in Figs. \ref{fig:motiv_a}, \ref{fig:motiv_b}, \ref{fig:motiv_c}, the following system parameters are used. A Zak-OTFS system with $M=12$, $N=14$, and BPSK is considered. The Doppler period taken to be $\nu_{\mathrm p}=15$ kHz. Therefore, the delay period is $\tau_{\mathrm p}=\frac{1}{\nu_{\mathrm p}}=66.66\ \mu$s. Consequently, the time duration of a Zak-OTFS frame is $T=N\tau_{\mathrm p}=0.93$ ms and the bandwidth is $B=M\nu_{\mathrm p}=180$ kHz. The receive filter is matched to the transmit filter \cite{zak_otfs6}-\cite{zak_otfs11}, i.e., 
\begin{equation}
w_{\mathrm{rx}}(\tau,\nu)=w^{*}_{\mathrm{tx}}(-\tau,-\nu) e^{j2\pi \nu\tau}.
\label{mat_filtering}
\end{equation}
The Veh-A fractional DD channel model \cite{ITU_VehA} having $P=6$ channel paths whose power delay profile is shown in Table \ref{tab_pdp} and a maximum Doppler shift of $\nu_{\max}=815$ Hz is considered. The Doppler shift of the $i$th path is modeled as $\nu_{i}=\nu_{\mathrm{max}}\cos\theta_{i},i=1,\ldots,P$, where $\theta_{i}$s are independent and uniformly distributed in $[0,2\pi)$. The considered $\tau_p,\nu_p$ values and channel spreads satisfy the crystallization condition. Minimum mean square error (MMSE) detection is used. For RRC filter, $\beta_\tau=0.05$ and $\beta_\nu=0.1$ are used. For Gaussian filter, $\alpha_\tau$ and $\alpha_\nu$ are taken to be 1.584. 

First, let us see how the choice of the filter affects the equalization/detection performance at the receiver. For this, we assume perfect channel state information (CSI). Figure \ref{fig:motiv_a} shows the BER performance of Zak-OTFS using sinc, RRC, and Gaussian filters with perfect CSI. We observe that the sinc filter achieves nearly 5 dB better performance compared to Gaussian filter. Note that, because of the perfect CSI assumption, there is no effect of I/O relation estimation on the detection performance. Consequently, the better performance of sinc filter with perfect CSI is attributed to the fact that it has nulls at Nyquist sampling points (leaving only a weak influence by the physical channel spread), whereas Gaussian filter has a high non-zero value at the $\tau=\frac{1}{B}$, $\nu=\frac{1}{T}$ sampling points (as per Fig. \ref{pulse_shapes}, this value is just 7 dB below the main lobe peak), which leads to high inter-symbol interference. With bandwidth and time expansion, RRC filter achieves slightly better performance compared to sinc filter performance.

Now, let us see how the filters affect performance when there is no perfect CSI assumption and a model-free I/O relation estimation scheme is used with embedded pilot frame. An embedded pilot frame structure shown in Fig. \ref{fig:embedded_pilot} with a PDR of 5 dB is considered.
As in Fig. \ref{fig:embedded_pilot}, the pilot is located at $(k_{\text{p}},l_{\text{p}})=(M/2,N/2)$  and the embedded frame parameters are fixed as $p_1=p_2=1$, $g_1=1$, $g_2=2$,  and $k_{\text{max}}=\lceil B\max(\tau_{i}) \rceil=1 $. Figure \ref{fig:motiv_b} shows the MSE performance of I/O relation estimation using sinc, RRC, and Gaussian filters. It is interesting to observe that while Gaussian filter's BER performance with perfect CSI is the worst (Fig. \ref{fig:motiv_a}), its MSE performance of I/O relation estimation is the best (Fig. \ref{fig:motiv_b}). This is attributed to the Gaussian filter's very low side lobes compared to those of sinc and RRC filters (see Fig. \ref{pulse_shapes}), which help to isolate the influence of interference from data/pilot replicas on estimation. However, for sinc filter, because of its high side lobes and consequent high interference levels, the MSE floors at a high value.  Figure \ref{fig:motiv_c} shows the BER performance comparison corresponding to the MSE performance comparison in Fig. \ref{fig:motiv_b}. From Fig. \ref{fig:motiv_c}, it is seen that, though Gaussian filter performs better than sinc filter in terms of MSE, there is a cross-over in their BER performance. This can be explained as follows. Because of its very good I/O relation estimation, the Gaussian filter's BER for perfect CSI and estimated CSI are very close (see BER plots of Gaussian filter in Figs. \ref{fig:motiv_a} and \ref{fig:motiv_c}). Whereas, because of its poor I/O relation estimation, the sinc filter's BER degrades significantly and floors at high SNRs, where the MSE floor (due to high data/pilot replicas' interference and pilot-data interference) dominates BER performance over noise variance. At low SNRs, the sinc filter has the advantage of low inter-symbol interference due to its nulls, whereas Gaussian filter suffers from high inter-symbol interference because of its poor main lobe characteristics leading to its poorer performance compared to sinc filter. Also, RRC filter performs slightly better than sinc filter, and this comes at the cost of time and bandwidth expansion.    
 
The above observations indicate that the Gaussian and sinc filters have complementary merits with respect to I/O relation estimation and detection tasks. Therefore, a filter which possesses the merits of both without bandwidth or time expansion is of interest, and this forms the essence of the GS filter proposed in the following subsection.

\vspace{-3mm}
\subsection{Proposed GS filter}
The proposed GS filter aims to simultaneously achieve the complementary strengths of Gaussian filter (good I/O relation estimation) and sinc filter (good equalization/detection) without bandwidth or time expansion. Towards this, the proposed filter is devised in a separable form $w_\text{tx}(\tau,\nu)=w_1(\tau)w_2(\nu)$, where $w_1(\tau)$ is a product function in $\tau$ variable of the form    
\begin{equation}
w_{1}(\tau)=\Omega_{\tau}\sqrt{B}\mathrm{sinc}(B\tau)e^{-\alpha_{\tau}B^{2}\tau^{2}},
\label{delay_domain}
\end{equation}   
and $w_2(\nu)$ is a product function in $\nu$ variable of the form
\begin{equation}
w_{2}(\nu)=\Omega_{\nu}\sqrt{T}\mathrm{sinc}(T\nu)e^{-\alpha_{\nu}T^{2}\nu^{2}},
\label{Doppler_domain}
\end{equation}
so that the overall proposed filter is given by
\begin{equation}
w_\text{tx}(\tau,\nu)=\Omega_\tau\Omega_\nu\sqrt{BT} \mathrm{sinc}(B\tau)\mathrm{sinc}(T\nu)e^{-\alpha_{\tau}B^{2}\tau^{2}} e^{-\alpha_{\nu}T^{2}\nu^{2}}.
\label{gsf}
\end{equation}
Note that $w_1(\tau)$ and $w_2(\nu)$ are constructed as product of sinc and Gaussian shaping functions with energy normalization parameters $\Omega_\tau$ and $\Omega_\nu$. 
The parameters $\alpha_{\tau}$ and $\alpha_{\nu}$ fix the bandwidth $B$ and the time duration $T$, respectively, and 
the parameters $\Omega_{\tau}$ and $\Omega_{\nu}$ are used to normalize the energy of the filter to unity, i.e.,  
$\int |w_1(\tau)|^{2}d\tau =\int |w_{2}(\nu)|^{2} d\nu=1$.
The  expressions for $\Omega_\tau$ and $\Omega_\nu$ in terms of $\alpha_{\tau}$ and
$\alpha_{\nu}$, respectively, for unit energy normalization are obtained in Appendix \ref{appxA}. 

The delay pulse characteristics of the proposed GS filter is plotted in Fig. \ref{pulse_shapes} (along with those of sinc, RRC, and Gaussian filters). It can be seen that the GS filter retains the nulls of the sinc filter while reducing the side lobe levels without bandwidth and time expansion. The values of $\alpha_\tau$ and $\alpha_\nu$ in the proposed filter in (\ref{gsf}) for which there is no bandwidth and time expansion ($B'=B, T'=T$) and 99\% energy is contained within bandwidth $B$ and time duration $T$ are $\alpha_{\tau}=\alpha_{\nu}=0.044$, and the corresponding values of $\Omega_{\tau}$ and $\Omega_{\nu}$ are $\Omega_{\tau}=\Omega_{\nu}=1.0278$. 

\vspace{-2mm}
\subsection{Closed-form expressions for I/O relation/noise covariance} 
To facilitate performance analysis/simulation of Zak-OTFS with the proposed GS filter, here we derive closed-form expressions for the DD domain I/O relation and noise covariance with the proposed filter. A receive filter matched to the proposed filter (as per Eq. (\ref{mat_filtering})) is considered. The effective channel in the continuous DD domain 
can be written as 
%\vspace{0mm}
\begin{eqnarray}
h_{\mathrm{eff}}(\tau,\nu) & \hspace{-2mm} = & \hspace{-2mm} w_{\mathrm{rx}}(\tau,\nu)*_{\sigma}h_{\mathrm{phy}}(\tau,\nu)*_{\sigma}w_{\mathrm{tx}}(\tau,\nu) \nonumber \\ 
& \hspace{-30mm} = & \hspace{-17mm} w_{\mathrm{rx}}(\tau,\nu)*_{\sigma}\left(\sum_{i=1}^{P}h_{i}\delta(\tau-\tau_{i})\delta(\nu-\nu_{i})\right)*_{\sigma}w_{1}(\tau)w_{2}(\nu) \nonumber    \\ 
& \hspace{-30mm} = & \hspace{-17mm} w_{1}^{*}(-\tau)w_{2}^{*}(-\nu)e^{j2\pi\nu\tau}*_{\sigma}\bigg(\sum_{i=1}^{P}h_{i}w_{1}(\tau-\tau_{i})w_{2}(\nu-\nu_{i}) \nonumber \\ 
& \hspace{-30mm} & \hspace{-17mm} e^{j2\pi\nu_{i}(\tau-\tau_{i})}\bigg) \nonumber \\
& \hspace{-30mm} = & \hspace{-17mm} \sum_{i=1}^{P}
\underbrace{\left(\int w_{1}^{*}(-\tau')w_{1}(\tau-\tau_{i}-\tau') e^{-j2\pi\nu_{i}\tau'}d\tau'\right)}_{\overset{\Delta}{=}I_{i}^{(1)}(\tau)} \nonumber \\
& \hspace{-35mm} & \hspace{-20mm} 
\underbrace{\left(\int \hspace{-1mm} w_{2}^{*}(-\nu')w_{2}(\nu-\nu_{i}-\nu')e^{j2\pi\nu'\tau}d\nu'\hspace{-0.5mm} \right)}_{\overset{\Delta}{=}I_{i}^{(2)}(\tau,\nu)}
\hspace{-0.5mm} h_{i}e^{j2\pi\nu_{i}(\tau-\tau_{i})}.
\label{eqn:channel_matched}
\end{eqnarray}
\vspace{0mm}
We note that a general expression for $h_{\mathrm{eff}}(\tau,\nu)$ for an arbitrary pulse-shaping filter in the matched filter configuration is presented in \cite{zak_otfs5} (see Eq. (57) in \cite{zak_otfs5}). This $h_{\mathrm{eff}}(\tau,\nu)$ expression in \cite{zak_otfs5} is given in a form of two separable integrals, corresponding to the auto-ambiguity functions of the time and frequency domain representations of the Doppler and delay domain components of the pulse-shaping filter, respectively. Observe that the integrals in (\ref{eqn:channel_matched}) are similar to those of the auto-ambiguity integrals in \cite{zak_otfs5}. Here, we further simplify the integrals in (\ref{eqn:channel_matched}) to closed-form for the proposed GS filter. Accordingly, we specialize $w_1(\tau)$ and $w_2(\nu)$ in (\ref{eqn:channel_matched}) with those of the GS filter given in (\ref{delay_domain}) and (\ref{Doppler_domain}), respectively, and obtain closed-form expression for $h_{\mathrm{eff}}(\tau,\nu)$ for the proposed GS filter (see Theorem \ref{Thm1} below and Appendix \ref{appxB}).
\begin{theorem} 
The DD domain effective channel $h_{\mathrm{eff}}(\tau, \nu)$ in closed-form for GS filter is given by
\begin{eqnarray}
\hspace{-6mm}
h_{\mathrm{eff}}(\tau,\nu) & \hspace{-2mm} = & \hspace{-2mm} \sum_{i=1}^{P}h_{i}e^{j2\pi\nu_{i}(\tau-\tau_i)} \nonumber \\
& \hspace{-2mm} & \hspace{-2mm}
\cdot\left(I_{i,1}^{(1)}(\tau)\mathbbm{1}_{\{\tau\neq\tau_i\}}+I_{i,2}^{(1)}(\tau)\mathbbm{1}_{\{\tau=\tau_i\}}\right) \nonumber \\
& \hspace{-2mm} & \hspace{-2mm} \cdot \left(I_{i,1}^{(2)}(\tau,\nu)\mathbbm{1}_{\{\nu\neq\nu_i\}}+I_{i,1}^{(2)}(\tau,\nu)\mathbbm{1}_{\{\nu=\nu_i\}}\right),
\label{eqn:GS_match_channel}
\end{eqnarray}
where 
$\mathbbm{1}_{\{.\}}$ denotes the indicator function, and  
$I_{i,1}^{(1)}(\tau)$, $I_{i,2}^{(1)}(\tau)$, $I_{i,1}^{(2)}(\tau,\nu)$, and $I_{i,2}^{(2)}(\tau,\nu)$ are defined in Appendix \ref{appxB}. 
\label{Thm1}
\end{theorem}
\vspace{0mm}
\vspace{-1mm}
\begin{IEEEproof}
See Appendix \ref{appxB}.
\end{IEEEproof}

Now, the continuous DD domain noise at the output is given by
\begin{eqnarray}
n_{\mathrm{dd}}^{w_{\mathrm{rx}}}(\tau,\nu) & \hspace{-2mm} = & \hspace{-2mm} w_{\mathrm{rx}}(\tau,\nu)*_{\sigma}n_{\mathrm{dd}}(\tau,\nu) \nonumber \\ 
& \hspace{-33mm} = & \hspace{-18mm} w_{1}^{*}(-\tau)w_{2}^{*}(-\nu)e^{j2\pi\nu\tau}\hspace{-1mm} *_{\sigma}\hspace{-0.5mm} \Big(\hspace{-0.5mm} \sqrt{\tau_{p}}\sum_{q\in\mathbb{Z}}n(\tau+q\tau_{p})e^{-j2\pi\nu q\tau_{p}}\Big) \nonumber \\ 
& \hspace{-33mm} = & \hspace{-18mm} \sqrt{\tau_{p}}\sum_{q=-\infty}^{\infty}e^{-j2\pi\nu q\tau_{p}}\left(\int w_{1}^{*}(-\tau')n(\tau-\tau'+q\tau_{p})d\tau'\right) \nonumber \\ 
& \hspace{-33mm} & \hspace{-18mm}  \underbrace{\left(\int w_{2}^{*}(-\nu')e^{j2\pi\nu'(\tau+q\tau_{p})}\right)}_{\overset{\Delta}{=}I_{q}^{(3)}(\tau)}.
\label{eqn:noise_matched}
\end{eqnarray}
A general expression for the noise covariance for an arbitrary pulse shape with the matched filter configuration is presented in \cite{zak_otfs5} (see the expression after Eq. (47) in page 4469 of \cite{zak_otfs5}). This noise covariance expression in \cite{zak_otfs5} consists of integrals in the form of auto-ambiguity function of the delay domain component 
of the pulse-shaping filter, and auto-correlation function of the Doppler domain component 
of the pulse-shaping filter. Here, using the DD domain noise in (\ref{eqn:noise_matched}), we derive the covariance of the noise in closed-form for the proposed GS filter (see Theorem \ref{Thm2} below and Appendix \ref{appxC}).

\begin{theorem}
For all $k_1,k_2=0,1,...,M-1, \ l_1,l_2=0,1,...,N-1$, the $(k_{1}N+l_1+1,k_{2}N+l_{2}+1)$th element of the noise covariance matrix is given by
\begin{eqnarray}
\hspace{-1mm}
\mathbbm{E}[n_{\mathrm{dd}}[k_1,l_1],n_{\mathrm{dd}}^{*}[k_2,l_2]] & \hspace{-2mm} = & \hspace{-2mm} \tau_p\sum_{q_1=-\infty}^{\infty}\sum_{q_2=-\infty}^{\infty}e^{j2\pi\frac{q_2l_2-q_1l_1}{N}} \nonumber \\
& \hspace{-53mm} & \hspace{-43mm} \cdot g\left(\frac{k_1\tau_p}{M}+q_1\tau_p\right)g^{*}\left(\frac{k_2\tau_p} {M}+q_2\tau_p\right) \nonumber \\
& \hspace{-53mm} & \hspace{-43mm} 
\cdot \Big(S_{\{k_1,k_2,q_1,q_2\}}^{(1)}\mathbbm{1}_{\{x_{\{k_1,k_2,q_1,q_2\}}\neq 0\}} 
+S^{(2)}\mathbbm{1}_{\{x_{\{k_1,k_2,q_1,q_2\}}= 0\}} \Big), \nonumber \\
\label{eqn:noise_expectation}
\end{eqnarray}
where $g(.)$, $x_{\{k_1,k_2,q_1,q_2\}}$, 
$S^{(1)}_{\{k_1,k_2,q_1,q_2\}}$, and $S^{(2)}$ are defined in Appendix \ref{appxC}.
\label{Thm2}
\end{theorem}
\vspace{-1mm}
\begin{IEEEproof}
See Appendix \ref{appxC}.
\end{IEEEproof}



