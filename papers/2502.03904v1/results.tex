\section{Results and Discussions}
\label{sec5}
In this section, we present the numerical results on the MSE and BER performance of Zak-OTFS for sinc, RRC, Gaussian, and GS filters with model-free I/O relation estimation using exclusive and embedded pilots. We consider a system with $M=32$, $N=48$, and fix the pilot location at $(k_{\mathrm{p}},l_{\mathrm{p}})=(M/2,N/2)$. The Doppler period is fixed at $\nu_{\mathrm p}=15$ kHz and the delay period is $\tau_{\mathrm p}=\frac{1}{\nu_{\mathrm p}}=66.66\ \mu$s. The time duration of a frame is $T=N\tau_{\mathrm p}=3.2$ ms and the bandwidth is $B=M\nu_{\mathrm p}=480$ kHz. 
Receive filter $w_\text{rx}(\tau,\nu)$ is matched to the transmit filter $w_\text{tx}(\tau,\nu)$ (see Eq. (\ref{mat_filtering})).
We consider the Veh-A channel model \cite{ITU_VehA} having $P=6$ paths with fractional DDs and a PDP as detailed in Table \ref{tab_pdp}. The maximum Doppler shift is $\nu_{\mathrm{max}}=815$ Hz, and the Doppler shift of the $i$th path is modeled as $\nu_{i}=\nu_{\mathrm{max}}\cos\theta_{i},i=1,\ldots,P$, where $\theta_{i}$s are independent and uniformly distributed in $[0,2\pi)$. Also, in the simulations, the range of values of $m$ and $n$ in (\ref{eqn_channel_matrix}) is limited to -1 to 1, and this is found to ensure an adequate support set of $h_{\mathrm{eff}}[k,l]$ that captures the channel spread accurately. BPSK and 8-QAM modulation alphabets are considered.
MMSE detection is used at the receiver. No bandwidth/time expansion ($B'=B, T'=T$) is considered for the filters except RRC filter. For RRC filter, an expanded bandwidth of $B'=1.05B$ $(\beta_{\tau}=0.05)$ and an expanded time duration of $T'=1.1T$ $(\beta_{\nu}=0.1)$ are considered. 


\begin{figure}[!t]
\centering
\includegraphics[width=9.0cm,height=6.5cm]{Figures/MSE_32X48_exclusive.eps}
\caption{MSE vs pilot SNR performance for different filters with exclusive pilot frame.}
\label{fig:mse_32X48_exclusive}
\vspace{-4mm}
\end{figure}

\vspace{-2mm}
\subsection{Performance with exclusive pilot frame}
Figures \ref{fig:mse_32X48_exclusive} 
and \ref{fig:BER_32X48_exclusive} show the MSE and BER performance of sinc, RRC, Gaussian, and the proposed GS filters using exclusive pilot frame. Figure 7 presents the effect of limited read-off from the exclusive pilot frame for I/O relation estimation.

\vspace{1mm}
\subsubsection{MSE performance}
Figure \ref{fig:mse_32X48_exclusive} shows the MSE performance as a function of pilot SNR. It is observed that the Gaussian filter performs better than the other three filters in terms of MSE performance. This characteristic is attributed to the highly localized nature  of the Gaussian filter with very low side lobes, resulting in negligible spread of the effective channel \big($h_{\mathrm{eff}}[k,l]$\big) outside the fundamental region $\mathcal{D}_{0}$, and this results in a very good estimate of the effective channel matrix $\hat{\textbf{H}}_{\mathrm{eff}}$. The MSE performance of the sinc filter is the poorest among all, which is due to its high side lobe levels that result in high effective channel spreads outside $\mathcal{D}_{0}$, leading to poor estimates. RRC filter performs slightly better than sinc filter, which is an artifact of the comparatively lower side lobe levels due to bandwidth and time expansion (refer Fig. \ref{pulse_shapes}). The proposed GS filter has low side lobe levels and achieves almost the same MSE performance as that of the RRC filter, but it achieves it without bandwidth and time expansion. 

\begin{figure}[!t]
\centering
\includegraphics[width=9.0cm,height=6.5cm]{Figures/BER_32X48_exclusive.eps}
\caption{BER vs data SNR performance of different filters with exclusive pilot frame at 30 dB pilot SNR. Perfect CSI performance is also shown.}
\label{fig:BER_32X48_exclusive}
\vspace{-4mm}
\end{figure}

\begin{figure}[!t]
\centering
\includegraphics[width=9.0cm,height=6.5cm]{Figures/MSE_different_delay_bins.eps}
\caption{MSE vs pilot SNR performance for sinc and Gaussian filters with limited read-off  in exclusive pilot frame.}
\label{fig:MSE_vs_SNR_different_delay_bins}
\vspace{-4mm}
\end{figure}

\vspace{1mm}
\subsubsection{BER performance}
Figure \ref{fig:BER_32X48_exclusive} shows the corresponding BER performance as a function of data SNR with BPSK at a pilot SNR of 30 dB. 
Performance with perfect CSI is also plotted for comparison. The sinc filter performs the best with perfect CSI, because its nulls at the sampling points cause weak inter-symbol interference among the data symbols which aids good detection performance. However, with I/O relation estimation using exclusive pilot, the presence of high side lobe levels and consequent high interference from pilot replicas results in a higher MSE (as seen in Fig. \ref{fig:BER_32X48_exclusive}), and this makes the BER to floor. RRC filter performs 
better due to time and bandwidth expansion. The Gaussian filter performs the worst with perfect CSI because of the absence of nulls at the sampling points, thereby causing high inter-symbol interference which results in poor data detection performance. However, its highly localized pulse shape results in a very accurate I/O relation estimation, and hence its performance with estimated CSI closely follows its own perfect CSI performance.
With I/O relation estimation, the proposed GS filter strikes a good balance between estimation and detection performance (with its low side lobes and nulls at sampling points) and achieves very good BER performance. 

\begin{figure}[!t]
\centering
\includegraphics[width=9.0cm,height=6.5cm]{Figures/MSE_32x48_embedded_0PDR.eps}
\caption{MSE vs data SNR performance of different filters with embedded pilot frame at 0 dB PDR.}
\label{fig_mse}
\vspace{-4mm}
\end{figure}

\vspace{3mm}
\subsubsection{Effect of limited read-off for I/O relation estimation}
In model-free I/O relation estimation, we estimate the effective channel coefficients by reading off the received samples (see Eq. (\ref{eqn:href_est})) which are used in the summation for constructing the effective channel matrix (see Eq. (\ref{eqn_channel_matrix})). Let $n_{\mathrm{dc}}$ denote the number of delay columns around the pilot considered in the read-off. This determines the support of the estimated effective channel.  Note that, $n_{\mathrm{dc}}=M$ is used in exclusive pilot-based estimation, i.e., the entire received frame is read off.  
In embedded pilot-based estimation, the received samples are read off only from the pilot region, i.e.,  $n_{\text{dc}}<M$. Figure \ref{fig:MSE_vs_SNR_different_delay_bins} presents an assessment of the effect of  $n_{\mathrm{dc}}<M$ (i.e., limited read-off) in exclusive pilot-based estimation. The figure shows the MSE performance for sinc and Gaussian filters for $n_{\mathrm{dc}}=M, M/2,M/4,M/8$. 
We observe that $n_{\text{dc}}$ affects the MSE performance differently at low and high SNR regions. That is, at low SNRs, 
smaller $n_{\mathrm{dc}}$ gives better MSE, whereas, at high SNRs, 
larger $n_\text{dc}$ provides better MSE. This is because a smaller $n_\text{dc}$ means a fewer terms to be summed up to obtain the estimate of $\textbf{H}_{\mathrm{eff}}$ as per (\ref{eqn_channel_matrix}), 
which reduces the effect of noise on this estimate at low SNRs (where noise is dominant), leading to better MSE. Whereas, a larger $n_\text{dc}$ incorporates more number of terms in (\ref{eqn_channel_matrix}) in the construction of $\textbf{H}_{\mathrm{eff}}$, which, at high SNRs (where signal terms are dominant), is beneficial to achieve better MSE.


\vspace{-2mm}
\subsection{Performance with embedded pilot frame}
Here, we present the MSE and BER performance with embedded pilot frame for different filters. As shown in Fig. \ref{fig:embedded_pilot}, the pilot symbol is placed at $(k_{\text{p}},l_{\text{p}})=(M/2,N/2)$ and the embedded pilot frame parameters are fixed as $p_1=3$, $p_2=1$, $g_1=2$, $g_2=3$,  and $k_{\text{max}}=\lceil B\max(\tau_{i}) \rceil=2$.

\vspace{1mm}
\subsubsection{MSE performance}
In Fig. \ref{fig_mse}, the MSE performance of I/O relation estimation is plotted as a function of data SNR at a fixed PDR of 0 dB. It can be observed that the Gaussian filter consistently demonstrates the highest estimation accuracy, followed by the proposed GS filter, the RRC filter, and finally the sinc filter. Two sources of interference arise due to the higher side lobe levels inherent in non-Gaussian filters: 1) aliasing due to interference from quasi-periodic replicas which is self-interaction and 2) pilot-data interference within the same frame. While aliasing affects both exclusive and embedded pilot scenarios, pilot-data interference is an additional challenge specific to the embedded pilot setting. This combined interference significantly degrades the estimation accuracy of non-Gaussian filters, particularly at higher SNRs where the impact of noise diminishes and these interference effects become more prominent. This manifests as flooring in the MSE performance at high SNRs, and a wider performance gap emerges between Gaussian and non-Gaussian filters. Comparing the MSE performance of exclusive and embedded pilots in Figs. \ref{fig:mse_32X48_exclusive} and \ref{fig_mse}, respectively, we see a similar trend of MSE performance at low and high SNRs reported in Fig. \ref{fig:MSE_vs_SNR_different_delay_bins}, i.e., at low SNRs, embedded pilot-based estimation is better because of the small $n_{\mathrm{dc}}$ for the read-off. 


\begin{figure}[!t]
\centering
\includegraphics[width=9.0cm,height=6.5cm]{Figures/BER_vs_PDR_32X48_embedded.eps}
\caption{BER vs PDR performance of different filters with embedded pilot frame at 15 dB data SNR.}
\label{fig_ber_pdr}
\vspace{-4mm}
\end{figure}


\vspace{1mm}
\subsubsection{BER performance}
Figure \ref{fig_ber_pdr} shows the BER performance of different filters as a function of PDR at a fixed data SNR of 15 dB. 
The BER curves exhibit U-shaped characteristics with respect to PDR. At low PDRs, poor estimation due to low pilot SNR degrades data detection and increases BER.  The BER improves with increase in PDR due to more accurate estimation. Conversely, excessive pilot power at high PDRs leads to significant pilot-data interference more than the noise effect, particularly more detrimental to non-Gaussian filters due to their higher side lobe levels. This manifests in a steeper increase in BER for non-Gaussian filters compared to Gaussian filter at high PDR values, indicating a higher sensitivity to strong pilot signals. The Gaussian filter, with its better DD localization and lower side lobe levels, is less susceptible to this interference, resulting in a more gradual increase in BER at high PDRs. The proposed GS filter consistently demonstrates the best performance because of its pulse shape which balances the complementary strengths of Gaussian and sinc filters. 


\begin{figure}[!t]
\centering
\includegraphics[width=9.0cm,height=6.5cm]{Figures/BER_32x48_embedded_0PDR.eps}
\caption{BER vs data SNR performance of different filters with embedded pilot frame at 0 dB PDR for BPSK.}
\label{fig_ber}
\vspace{-5mm}
\end{figure}

\begin{figure}[!t]
\centering
\includegraphics[width=9.0cm,height=6.5cm]{Figures/BER_embedded_8QAM_0dB.eps}
\caption{BER vs data SNR performance of different filters with embedded pilot frame at 0 dB PDR for 8-QAM.}
\label{fig_ber_8qam}
\vspace{-5mm}
\end{figure}



In Fig. \ref{fig_ber}, we present the BER performance of different filters corresponding to the MSE performance depicted in Fig. \ref{fig_mse}. At low SNRs, the Gaussian filter exhibits the poorest performance among the evaluated filters. This can be attributed to its relatively high main lobe value at the sampling instants, significantly increasing inter-symbol interference and consequently degrading data detection performance when the noise effects are more pronounced. While the sinc filter outperforms the Gaussian filter at low SNRs, its significant side lobe levels cause substantial pilot-data interference. As a result, the BER performance of the sinc filter starts to floor around 20 dB data SNR, as evident from the corresponding MSE performance. The proposed GS filter consistently outperforms the sinc filter with flooring beyond 25 dB. This improved performance is attributed to its lower side lobe levels compared to the sinc filter, which reduces interference, consequently improving both channel estimation and data detection accuracy. The RRC filter with bandwidth/time expansion, also due to its favorable side lobe properties, outperforms sinc filter but performs poorer than the proposed GS filter. However, the Gaussian filter has crossovers with these non-Gaussian filters. Notably, these crossover points in BER performance closely align with the SNR values where the MSE performance of the filters begin to floor as shown in Fig. \ref{fig_mse}. As the MSE of non-Gaussian filters floors, further SNR improvements do not significantly enhance channel estimation accuracy, which ultimately impacts the detection performance.

Figure \ref{fig_ber_8qam} shows the BER performance of different filters with 8-QAM and embedded pilot frame at a PDR of 0 dB. It can be seen that the performance of the proposed GS filter is superior compared to Gaussian and sinc filters. For example, at a BER of $10^{-2}$, the GS filter achieves an SNR gain of about 4 dB compared to Gaussian and sinc filters. Corresponding to the uncoded BER performance in Fig. \ref{fig_ber_8qam}, Fig. \ref{fig_ber_8qam_coded} shows the coded BER performance with a rate-$1/2$ convolutional code with constraint length 7. From Fig. \ref{fig_ber_8qam_coded}, we can see that the proposed GS filter achieves an SNR gain in excess of 6 dB at a coded BER of $10^{-4}$ compared to Gaussian and sinc filters.

\begin{figure}[!t]
\centering
\includegraphics[width=9.0cm,height=6.5cm]{Figures/Coded_BER_32X48_embedded.eps}
\caption{Coded BER vs data SNR performance of different filters with embedded pilot frame at 0 dB PDR for 8-QAM and rate-1/2 coding.}
\label{fig_ber_8qam_coded}
\vspace{-7mm}
\end{figure}
