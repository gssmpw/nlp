\vspace{-1mm}
\section{Delay-Doppler I/O Relation Estimation}
\label{sec3}
To perform the equalization/detection task at the receiver, knowledge of the DD I/O relation, i.e., the effective channel matrix ${\bf H_\text{eff}}$ in (\ref{sys_mod}), is needed. This can be obtained using two approaches, namely, model-dependent and model-free approaches \cite{zak_otfs2}.
In model-dependent approach, the parameters of the physical channel $h_\text{phy}(\tau,\nu)$, i.e., $\{\tau_i$, $\nu_i$, $h_i$\}s, are estimated using a channel estimation scheme and these estimated parameters are then used to construct the I/O relation. That is, use the estimated $\{\tau_i,\nu_i,h_i\}$s to compute $h_\text{eff}(\tau,\nu)$ defined in (\ref{cont1}) 
and sample it to obtain $h_\text{eff}[k,l]$ as in (\ref{discr2}), which when substituted in (\ref{eqn_channel_matrix}) gives the estimated I/O relation $\hat{\bf H}_\text{eff}$.

Model-free approach does not require explicit estimation of the physical channel parameters
$\{\tau_i,\nu_i,h_i\}$s. Instead, the I/O relation can be obtained by sending a pilot symbol in a frame and directly reading out the corresponding DD domain output samples in $\mathcal{D}_0$ at the receiver. Due to its simplicity, we consider model-free approach for I/O relation estimation, which is presented below for exclusive and embedded pilot frames. 

\vspace{-2mm}
\subsection{Exclusive pilot frame}
For a pilot at the origin (0,0), the channel response consists of two terms, the first term due to the pilot at the origin and the other term due to its quasi-periodic replicas, i.e., the response is given by 
\begin{equation}
h_\text{eff}[k,l] + \sum_{n,m\in\mathbb{Z},\newline (n,m)\neq(0,0)}\hspace{-6mm}h_\text{eff}[k-nM,l-mN]e^{j2\pi\frac{nl}{N}}.
\label{c_rsp}
\end{equation}
In the crystalline regime of operation, where the maximum delay and Doppler spreads of the effective channel (denoted by $\tau_\text{max}$ and $\nu_\text{max}$, respectively) are less than the delay and Doppler periods, respectively (i.e., $\tau_{\mathrm{max}}<\tau_{\mathrm{p}}$ and $\nu_{\mathrm{max}}<\nu_{\mathrm{p}}$), the local responses do not interfere with each other significantly \cite{zak_otfs2}. Therefore, the model-free I/O relation estimation approach considers only the $(0,0)$th local response.  

In an exclusive pilot frame, a pilot $x_{\text{p}}[k,l]$ located at $(k_{\text{p}}, l_{\text{p}})=(M/2, N/2)$ and zeros at other locations is sent to estimate the effective channel $h_{\mathrm{eff}}[k,l]$. The channel response for this exclusive pilot frame is given by 
\begin{align}
y_{\text{p}}[k,l] = & \ h_{\mathrm{eff}}[k,l]*_{\sigma\text{d}}x_{\text{p}}[k,l] \nonumber \\
=&\sum_{m,n\in\mathbb{Z}}h_{\mathrm{eff}}[k-(k_{\text{p}}+nM),l-(l_{\text{p}}+mN)] \nonumber \\
&e^{j2\pi\frac{nl_{\text{p}}}{N}}e^{j2\pi\frac{(l-l_{\text{p}}-mN)(k_{\text{p}}+nM)}{MN}}.
\label{c_rsp2}
\end{align}
In the crystalline regime, the total response in the fundamental period coincides with the $(0,0)$th local response ($m=n=0$), given by 
\begin{equation}
y_{\text{p}}[k,l]=h_{\mathrm{eff}}[k-M/2,l-N/2]e^{j\pi\frac{\left(l-\frac{N}{2}\right)}{N}},
\end{equation}
for $0\leq k<M$ and $0\leq l<N$. Consequently, the effective channel estimate is obtained as 
\begin{eqnarray}
\hat{h}_{\mathrm{eff}}[k,l] \hspace{-0.5mm} = \hspace{-0.5mm}
\begin{cases}
y_{\text{p}}\left[k+\frac{M}{2},l+\frac{N}{2}\right]e^{-j\pi\frac{l}{N}}, & \hspace{-2mm} -\frac{M}{2}\leq k<\frac{M}{2}, \\
& \hspace{-2mm} -\frac{N}{2}\leq l<\frac{N}{2}, \\ 
0, \ \ \mathrm{otherwise}.
\end{cases}
\hspace{-1mm}
\label{eqn:href_est}
\end{eqnarray}
The above estimated coefficients $\hat{h}_{\mathrm{eff}}[k,l]$ are used in (\ref{eqn_channel_matrix}) to obtain the $\hat{\bf H}_\text{eff}$. Note that the accuracy of this estimate in terms of normalized mean squared error (MSE), defined as the average of $\frac{||\bf{H}_{\text{eff}}-\hat{\bf H}_{\text{eff}}||_{F}^{2}}{||\bf{H}_{\text{eff}}||_{F}^{2}}$, is influenced by the choice of the filter, particularly the side lobe characteristics of the filter. Lower the side lobe levels, better will be the accuracy, because lower side lobes result in a weak second term in the channel response due to the replicas (see Eq. (\ref{c_rsp})). 

\vspace{0mm}
\begin{figure}
\hspace{2mm}
\includegraphics[width=9cm,height=6cm]{Figures/Embedded_pilot.eps}
\caption{Embedded pilot frame with pilot symbol, pilot region, guard region, and data region.}
\label{fig:embedded_pilot}
\vspace{-5mm}
\end{figure}

\vspace{-4mm}
\subsection{Embedded pilot frame}
We consider the embedded pilot frame shown in Fig. \ref{fig:embedded_pilot} \cite{zak_otfs7}. It consists of a pilot symbol located at $(k_{\text{p}}, l_{\text{p}})=(M/2,N/2)$, a data region $\mathcal{D}=\mathcal{D}_1\cup \mathcal{D}_2$ in which data symbols are transmitted, and a region in between (pilot region $\mathcal{P}$ + guard region $\mathcal{G}=\mathcal{G}_1\cup \mathcal{G}_2$) where no symbols are transmitted. 
The support of the effective channel, denoted by $\mathcal{S}$, is marked/represented by the ellipse in Fig. \ref{fig:embedded_pilot}. The pilot region is designed to encompass $\mathcal{S}$, and the guard regions act as buffers between the pilot and data regions to mitigate interference between them.
The pilot region spans from $k_{\text{p}}-p_1$ to $k_{\text{p}}+k_{\max}+p_2$, and the guard region is defined by the boundaries $k_{\text{p}}-k_{\max}-g_1$ and $k_{\text{p}}+k_{\max}+g_2$. Here, $k_{\max} = \lceil B\max(\tau_{i})  \rceil$ represents the maximum delay spread of the physical channel and $p_1, p_2, g_1, g_2$ are non-negative integers. The additional bins within these regions represented by $p_1, p_2, g_1, g_2$ accommodate the signal spread caused by the pulse shaping filters and can be chosen according to the system bandwidth.  

For $0\leq k\leq M-1$ and $0\leq l\leq N-1$, the symbol $x[k,l]$ in the frame is given by
\begin{eqnarray}
x[k,l]=
\begin{cases}
\sqrt{E_\text{p}}, \quad \quad \quad \quad \ (k,l)=(k_\text{p},l_\text{p}), \\
\sqrt{\frac{E_\text{d}}{|\mathcal{D}|}}x_{\text{d}}[k,l], \quad \ (k,l)\in \mathcal{D}, \\
0, \qquad \qquad \qquad \  \mathrm{otherwise,}
\end{cases}
\label{eqn:noise_integral_closed}
\end{eqnarray}
where $x_\text{d}[k,l]$ is the information symbol at location $(k,l)$. Taking $\mathbb{E}[|x_\text{d}[k,l]|^{2}]=1$, the average energy transmitted in a frame is $E_\text{p}+E_\text{d}$ and the average transmitted power is $(E_\text{p}+E_\text{d})/T'$. Normalizing the channel gains as $\sum_{i=1}^{P}\mathbb{E}[|h_i|^{2}]=1$, the data SNR is given by
$\gamma_{\text{d}}=\frac{E_\text{d}}{N_{0}B'T'}$ and the pilot SNR is given by
$\gamma_{\text{p}}=\frac{E_\text{p}}{N_{0}B'T'}$. The term  $E_{\text{p}}/E_{\text{d}}$ is the ratio of the pilot power to data power ratio (PDR).

In each frame, the  effective channel coefficients $\{h_{\mathrm{eff}}[k,l]\}$ are estimated based on the received pilot symbols at locations within the pilot region ${\mathcal{P}}$ using (\ref{eqn:href_est}). These estimates are used to construct the estimated effective channel matrix $\hat{\mathbf{H}}_{\text{eff}}$. Note that the estimation accuracy here is affected by the interference from data symbols (in addition to self-interference due to pilot symbol replicas and noise), which is determined by the pulse shape. The received DD symbols $y_{\mathrm{dd}}[k,l],\ (k,l)\in \mathcal{D} \hspace{0.5mm} \cup  \hspace{0.5mm} \mathcal{G}$ are arranged as a vector of length $MN-|\mathcal{P}|$, which is the vector of the $|\mathcal{D}|$ transmitted symbols times the effective channel matrix plus the noise vector \cite{zak_otfs7}. Information symbols are detected from the vector of received symbols in $\mathcal{D} \cup \mathcal{G}$. Note that, among other things, the equalizer/detection performance here is affected by the interference from pilot. 
