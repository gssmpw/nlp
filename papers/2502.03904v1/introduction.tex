\vspace{-2mm}
\section{Introduction}
\label{sec1}
Orthogonal time frequency space (OTFS) modulation is a delay-Doppler (DD) domain modulation suited for doubly-selective channels. In multicarrier OTFS (MC-OTFS) modulation introduced in \cite{otfs1}, the information symbols in the DD domain are converted to time-frequency (TF) domain following which conversion to time domain is carried out using a legacy multicarrier modulation scheme \cite{otfs2}-\cite{h_b_mishra}. In Zak transform based OTFS (Zak-OTFS) modulation, the information symbols multiplexed in the DD domain are converted to time domain for transmission using inverse Zak transform \cite{zak_otfs1},\cite{zak_otfs2},\cite{zak_otfs3}. At the receiver, the received time domain signal is converted back to DD domain using Zak transform for data detection. In this paper, we consider Zak-OTFS. Two key aspects are central to Zak-OTFS. First, it provides a formal mathematical framework using Zak theory for describing OTFS and studying its fundamental properties \cite{zak_otfs1},\cite{zak_otfs3}. This is analogous to how Fourier theory provides an appropriate mathematical framework for describing and understanding OFDM. Second, Zak-OTFS waveform is more robust to a larger range of delay and Doppler spreads of the channel. This is because the input-output (I/O) relation in Zak-OTFS is non-fading and predictable, even in the presence of significant delay and Doppler spreads, and, as a consequence, the channel can be efficiently acquired and equalized \cite{zak_otfs2}. Recent works on Zak-OTFS have been reported in \cite{zak_otfs4}-\cite{zak_otfs8}.

An important building block in the Zak-OTFS transmitter is the DD domain transmit pulse shaping filter. The basic information-bearing carrier in Zak-OTFS is a pulse in the DD domain which is a quasi-periodic localized function. The Zak-OTFS performance is influenced by how well these pulses are localized in the DD domain. A DD filter matched to the transmit filter is used at the receiver. The `effective' channel in Zak-OTFS includes the cascade of the transmit DD filter, the physical channel, and the receive DD filter. Consequently, the choice of the pulse shaping filter influences the DD spread of the effective channel. Estimating the DD domain input-output (I/O) relation in Zak-OTFS amounts to estimating the coefficients of the effective channel. The estimated I/O relation is used for subsequent equalization/detection in the DD domain. Therefore, the pulse shape influences the performance of the two important receiver functions, namely, I/O relation estimation and equalization/detection. 
    
In the Zak-OTFS literature, the following DD pulse shaping filters have been considered: 1) sinc filter
\cite{zak_otfs2}-\cite{zak_otfs5},\cite{zak_otfs9},\cite{zak_otfs8},
2) root raised cosine (RRC) filter \cite{zak_otfs2}-\cite{zak_otfs9}, and 3) Gaussian filter \cite{zak_otfs7}. The sinc filter has the benefit of good main lobe characteristics with nulls at the Nyquist sampling points in the DD domain (i.e., nulls at the information grid points). This attribute has a positive influence on achieving good equalization/detection performance. However, sinc filter has the drawback of high side lobes which plays a negative role in I/O relation estimation. Specifically, pulse shaping filters cause aliasing between the received pilot and its own quasi-periodic replicas (a.k.a. self-interaction). Because of this, the high side lobes in the sinc filter result in increased aliasing (self-interference due to quasi-periodic replicas) that leads to poor I/O relation estimation. The RRC filter achieves reduced side lobe levels compared to sinc filter, but this side lobe reduction is achieved with bandwidth and time expansion. More the bandwidth and time expansion, better will be the side lobe reduction. The Gaussian filter, on the other hand, has the advantage of good DD localization with very low side lobe levels, but it has poor main lobe characteristics without nulls at the Nyquist sampling points. This makes the Gaussian filter superior for I/O relation estimation but inferior for equalization/detection compared to sinc and RRC filters. 

Based on the above observations, in this paper, we propose a new pulse shaping filter, termed {\em Gaussian-sinc (GS) filter}, which inherits the complementary strengths of Gaussian and sinc filters. Unlike RRC filter, the proposed filter does not incur time and bandwidth expansion. We derive closed-form expressions for the I/O relation and noise covariance of Zak-OTFS with the proposed GS filter. We evaluate the Zak-OTFS performance for different pulse shaping filters with I/O relation estimated using exclusive and embedded pilots. We consider ITU Vehicular-A (Veh-A) channel model \cite{ITU_VehA} with fractional delays and Dopplers in performance evaluation. Our simulation results show that the proposed GS filter achieves better bit error rate (BER) performance compared to other filters reported in the literature.  
For example, with model-free I/O relation estimation using embedded pilot and 8-QAM, the proposed GS filter achieves an SNR gain of about 4 dB at $10^{-2}$ uncoded BER compared to Gaussian and sinc filters, and the SNR gain becomes more than 6 dB  at a coded BER of $10^{-4}$ with rate-1/2 coding.

The rest of the paper is organized as follows. The Zak-OTFS system model and the sinc, RRC, and Gaussian filters are introduced in Sec. \ref{sec2}. The model-free I/O relation estimation using exclusive and embedded pilot frames is presented in Sec. \ref{sec3}. The proposed GS filter and the derivation of closed-form expressions for the I/O relation and noise covariance are presented in Sec. \ref{sec4}. Performance results and discussions are presented in Sec. \ref{sec5}. Conclusions and future work are presented in Sec. \ref{sec6}. 