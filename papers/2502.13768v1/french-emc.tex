%!TEX encoding =  UTF-8 Unicode
%!TEX root = main-emc.tex


\selectlanguage{french}
\def\frenchproofname{\textsl{Démonstration}}

\setcounter{page}{1} 
\setcounter{section}{0}
\setcounter{subsection}{0}
\setcounter{equation}{0}

\selectlanguage{french}
\def\frenchproofname{\textsl{Démonstration}}


\FrenchFootnotes

%!TEX encoding =  UTF-8 Unicode
%!TEX root =  main-emc.tex

\theoremstyle{plain}
\newtheorem{ftheorem}{Théorème}[subsection]
\newtheorem{fproposition}[ftheorem]{Proposition}
\newtheorem{fpropdef}[ftheorem]{Proposition et définition}
\newtheorem{flemma}[ftheorem]{Lemme}
\newtheorem{fcorollary}[ftheorem]{Corollaire}

\theoremstyle{definition}
\newtheorem{fdefinition}[ftheorem]{Définition}
\newtheorem{fnotation}[ftheorem]{Notation}
\newtheorem{fexample}[ftheorem]{Exemple}
\newtheorem{fexamples}[ftheorem]{Exemples}

\theoremstyle{remark}
\newtheorem{fremark}[ftheorem]{Remarque}
\newtheorem{fremarks}[ftheorem]{Remarques}


%!TEX encoding =  UTF-8 Unicode
%!TEX root =  main-emc.tex

\renewcommand\csl{{\sC_{\rm csl}}}
\renewcommand\cbo{{\sC_{\rm cb}}}
\renewcommand\ckf{{\sC_{\rm KF}}}
\renewcommand\caf{{\sC_{\rm ca}}}
\renewcommand\capo{{\sC_{\rm cap}}}
\renewcommand\cw{{\sC_{\rm W}}}
\renewcommand\crf{{\sC_{\rm frf}}}
\renewcommand\csr{{\sC_{\rm sfrf}}}
\renewcommand\csp{{\sC_{\rm sp}}}

\renewcommand\CB{{\rm CB}}
\renewcommand\CA{{\rm CA}}

\renewcommand\ysl{{\Y_{\rm csl}}}
\newcommand\ybo{{\Y_{\rm cb}}}
\renewcommand\ykf{{\Y_{\rm KF}}}
\renewcommand\yaf{{\Y_{\rm ca}}}
\renewcommand\yap{{\Y_{\rm cap}}}
\renewcommand\yw{{\Y_{\rm W}}}
\renewcommand\yrf{{\Y_{\rm frf}}}
\renewcommand\ysr{{\Y_{\rm sfrf}}}
\renewcommand\ysp{{\Y_{\rm sp}}}

\renewcommand \loca {localement\xspace}
\renewcommand \uni {uniformément\xspace}
\newcommand \uniz {uniformément}
\renewcommand \mcu {module de continuité uniforme\xspace}
\newcommand \mcuz {module de continuité uniforme}
\renewcommand \unico {\uni continue\xspace}
\newcommand \unicoz {\uni continue}
\newcommand \unicos {\uni continues\xspace}
\newcommand \unicosz {\uni continues}

\renewcommand \com {complexité\xspace}
\newcommand \comz {complexité}
\newcommand \pres {présentation\xspace}
\newcommand \equivas {équivalentes\xspace}
\renewcommand \equiva {équivalente\xspace}
\renewcommand \etpo {en temps \poll}
\newcommand \elem {élément\xspace}

\newcommand \pol{polynôme\xspace}
\newcommand \pols{polynômes\xspace}

\newcommand \polt{polynomialement\xspace}
\newcommand \poll{polynomial\xspace}
\newcommand \polle{polynomiale\xspace}
\newcommand \polles{polynomiales\xspace}


\renewcommand\rp{\pres rationnelle\xspace}
\renewcommand\rps{présentations rationnelles\xspace}
\renewcommand\rapr{rationnellement présenté\xspace}
\newcommand\rapre{rationnellement présentée\xspace}
\newcommand\raprs{rationnellement présentés}


\newcommand\ta{{\rm t}}
\newcommand\prof{{\rm prof}}

\renewcommand\ni{\noindent }


\thickmuskip = 7mu plus 2mu


\pagestyle{headings}
\patchcmd{\sectionmark}{\MakeUppercase}{}{}{}

\title{Espaces métriques rationnellement présentés et complexité, le cas 
de l'espace des fonctions réelles \unicos sur un intervalle compact}
\author{S. Labhalla \\ Dépt. de Mathématiques \\ Univ. de Marrakech, Maroc 
\\{\tt labhalla@ucam.ac.ma} 
\and H. Lombardi \\ Laboratoire de Mathématiques de Besançon \\
Université Marie et Louis Pasteur, France \\  {\tt Henri.Lombardi@univ-fcomte.fr}
\and  E. Moutai\\ Dépt. de Mathématiques \\ Univ. de Marrakech, Maroc}

\maketitle


\begin{quotation} 
\small Cet article est paru en 2001 dans \emph{Theoretical Computer Science.} {\bf 250}, \num 1-2, 265--332  (reçu en avril 1997; révisé en mars 1999). 
Nous avons corrigé quelques erreurs de détail.

\end{quotation}

\rdb
\label{beginfrench}

\begin{abstract}
 Nous définissons la notion de {\em \rp d'un
espace métrique complet} comme moyen d'étude des espaces métriques et des
fonctions continues du point de vue de la complexité algorithmique. Nous
étudions dans ce cadre différentes manières de présenter l'espace $\czu$ 
des fonctions réelles \unicos sur l'intervalle $[0,1]$, muni de la norme 
usuelle:
 $\norme{f}_{\infty} = {\bf Sup} \{ \abs{f(x)}  \;
 0 \leq x \leq 1\}.$ Ceci nous permet de
faire une comparaison de nature globale entre les notions de complexité
attachées à ces présentations. En particulier, nous obtenons une
généralisation des résultats de Hoover concernant le {\em théorème 
d'approximation de Weierstrass en temps \poll}. Nous obtenons également une
généralisation des résultats de Ker-I. Ko, H. Friedman et N. Müller 
concernant les fonctions analytiques calculables en temps \poll.
\end{abstract}

\newpage 
\startcontents[french]

\setcounter{tocdepth}{4}
\markboth{Table des matières}{Table des matières}

\printcontents[french]{}{1}{}
\normalsize
\newpage




%!TEX encoding =  UTF-8 Unicode
%!TEX root =  main-emc.tex


\section*{Introduction}\label{fsec0}
\addcontentsline{toc}{section}{Introduction}
\markboth{Introduction}{Introduction}
Notons   $\czu$ l'espace des fonctions réelles \unicos sur l'intervalle 
$[0,1]$.\\
Dans \cite{fKF82}, Ker-I. Ko et Friedman ont introduit et étudié la notion de 
\com des fonctions réelles, définie via une machine de Turing à oracle. \\
Dans l'article  \cite{fHo90}, Hoover a étudié les présentations de l'espace 
$\czu$ via les circuits booléens et les circuits arithmétiques.\\
Dans ces deux articles la \com dans l'espace  $\czu$  est étudiée ``point 
par point''.  \\
Autrement dit, sont définies des phrases comme : 
%-----------------begin item------------------
\begin{itemize}
\item $f$ est un $\p$-point au sens de Ker-I. Ko et Friedman.
\item $f$ est un $\p$-point au sens des circuits booléens.
\item $f$ est un $\p$-point au sens des circuits arithmétiques.
 \end{itemize}
%-----------------end item------------------
Ko et Friedman ont étudié les propriétés des  $\p$-points  (au sens de 
Ker-I. Ko et Friedman).   
Hoover~a démontré que les  $\p$-points au sens des circuits arithmétiques sont 
les mêmes que les $\p$-points au sens de Ker-I. Ko et Friedman.

\smallskip  Dans cet article, nous étudions différentes présentations de 
l'ensemble  $\czu$  en nous basant sur la notion de {\em présentation 
rationnelle d'un espace métrique complet}. Ceci nous permet de faire une 
comparaison de nature globale entre les notions de \com attachées à 
différentes présentations.


 
\smallskip  Après quelques préliminaires dans la section~\ref{fsec1}, 
la section ~\ref{fsec2} contient  essentiellement une exposition de ce 
qui nous semble être une problématique naturelle concernant les 
questions de \com relatives aux espaces métriques complets séparables. 

\smallskip 
Usuellement un espace métrique contient des objets de nature infinie (le 
paradigme étant un nombre réel défini à la Cauchy), 
ce qui exclut une présentation informatique directe (c.-à-d. codée sur un 
alphabet fini) de ces objets. 
Pour contourner cette difficulté, on procède comme il est usuel pour 
l'espace  $\RR$. On considère une partie dense  Y  de l'espace métrique 
considéré  X, qui soit suffisamment simple pour que 
%-----------------begin item------------------
\begin{itemize}
\item ses éléments puissent être codés comme (certains) mots sur un 
alphabet fini fixé.
\item la fonction distance restreinte à  $Y$  soit calculable, c.-à-d. 
donnée par une fonction calculable:
\[
\delta\colon  Y\times Y\times \NN_1\rightarrow \QQ \; \qquad \hbox{avec}\qquad \; 
\abs {d_X(x,y) - \delta(x,y,n)} \;  \leq 1/2^n
\]
\end{itemize}
%-----------------end item------------------
L'espace  $X$  apparaît alors comme le séparé-complété de  $Y$.\\
On dira que le codage proposé pour  $Y$  et la description proposée pour la fonction distance constituent une {\em présentation rationnelle} de l'espace 
métrique  $X$.
 
Il est à noter qu'on n'envisage pas de traiter des espaces métriques non 
complets, pour lesquels les difficultés de codage semblent insurmontables, et pour lesquels on ne peut pratiquement rien démontrer de sérieux en analyse constructive.

Pareillement les espaces métriques traités sont ``à base dénombrable'' 
(on dit aussi ``séparables'').

\smallskip Les espaces métriques étudiés en analyse constructive (cf. 
\cite{fBB}) sont très souvent définis via une présentation rationnelle de 
ce type, ou au moins faciles à définir selon ce schéma. Le problème qui 
se pose est en général de définir constructivement une partie 
dénombrable dense de l'espace considéré. C'est évidemment impossible 
pour des espaces de Banach classiques non séparables du style  $L^\infty$, 
mais justement, ces espaces ne sont pas traitables sous leur forme classique par 
les méthodes 
constructives. Le problème est plus délicat pour des espaces qui sont 
classiquement séparables mais pour lesquels il n'y a pas de procédé 
naturel constructif qui donne une partie dénombrable dense. Par exemple c'est 
le cas pour un sous-espace fermé arbitraire d'un espace complet séparable, 
et c'est encore le cas pour certains espaces de fonctions continues.

\smallskip La présentation en unaire et la présentation en binaire de  
$\ZZ$  ne sont pas \equivas du point de vue la 
\com en temps \poll. La présentation en unaire est une présentation 
naturelle de  $\ZZ$  comme groupe tandis que 
la présentation en binaire est une \pres naturelle de  $\ZZ$  comme anneau. 
On peut se poser des questions analogues concernant les espaces métriques 
classiques usuels.

\smallskip La première question qui se pose est la comparaison des 
différentes présentations d'un espace métrique usuel.  Notez que la 
présentation usuelle de  $\RR$  est considérée comme la seule naturelle 
et que c'est à partir de cette présentation de  $\RR$  qu'est définie la 
\com d'une présentation  $(Y_1, \delta_1)$  de  $X$  ou la \com de 
l'application $\Id_X$  de  $(Y_1, \delta_1)$  vers  $(Y_2, \delta_2)$  lorsqu'on 
compare deux présentations distinctes de  $X $.

\smallskip La question qui se pose ensuite est celle de la \com des fonctions continues calculables entre espaces métriques.  S'il~y~a une notion naturelle de \com des fonctions dans le cas des espaces compacts, la question est nettement plus délicate dans le cas général, car elle renvoie à la \com des fonctionnelles de type 2 arbitraires.
 
\smallskip Une autre question est celle de la complexité d'objets liés de 
manière naturelle à l'espace métrique qu'on étudie. Par exemple dans le 
cas des réels, et en ne considérant que la structure algébrique de  $\RR$  
on est intéressé par le fait que la \com de l'addition, celle de la 
multiplication, ou celle de la recherche des racines complexes d'un \pol 
à coefficients réels soient toutes ``de bas niveau''. 
Ce genre de résultats légitime a posteriori le choix qui est fait usuellement pour présenter $\RR$, et les résultats de nature inverse disqualifient d'autres présentations (cf. \cite{fLL}), moins efficaces que la \pres via les suites de Cauchy de rationnels écrits en binaire{\footnote{Néanmoins, le test de signe est indécidable pour les réels présentés à la Cauchy. 
On se contente d'avoir en temps \poll le test constructif:  
$x + y \geq  1/2^n  \Rightarrow   ( x \geq  1/2^{n+2} \; \hbox{ou}\;  
y \geq  1/2^{n+2} )$.}}. 
De même, un espace métrique usuel est en général muni d'une structure 
plus riche que la seule structure métrique, et il s'agit alors d'étudier,
 pour chaque présentation, la \com de ces ``éléments naturels de 
structure''.


\medskip 
Dans la section \ref{fsec3}, nous introduisons l'espace  $\czu$  des fonctions réelles \uni continues sur l'intervalle $[0,1]$ du point de vue de ses présentations rationnelles.
 
La fonction d'évaluation  $(f, x) \mapsto f(x)$  n'est pas \loca \unico. Vue son importance, nous discutons ce que signifie la \com de cette fonction lorsqu'on a choisi une \rp  de l'espace  $\czu$. 

Nous donnons ensuite deux exemples significatifs de telles présentations 
rationnelles:
%-----------------begin item------------------
\begin{itemize}
\item Une \pres par circuits semilinéaires binaires notée  $\csl$, où les 
points rationnels sont exactement les fonctions semilinéaires binaires. Une telle fonction peut être définie par un circuit semilinéaire binaire (cf. définition~\ref{f321}) et codée par un programme d'évaluation correspondant au circuit.
\item une présentation, notée $\crf$, via des fractions rationnelles 
convenablement contrôlées et données en \pres par formule (cf. définition~\ref{f325}).
\end{itemize}
%-----------------end item------------------
Enfin nous établissons des résultats de \com liés au théorème 
d'approximation de Newman. En bref, le théorème de Newman a une \com 
\polle et en conséquence les fonctions linéaires par morceaux sont, 
chacune individuellement, des points de \com $\p$  dans l'espace~$\crf$.


\medskip Dans la section~\ref{fsec4} nous définissons et étudions des \rps  
``naturelles'' de  $\czu$,  \equivas du point de vue de la \com en temps 
\poll:
%-----------------begin item------------------
\begin{itemize}
\item La \pres dite ``à la Ko-Friedman'', et notée $\ckf$, pour laquelle un point rationnel est donné par un quadruplet $(Pr, n, m, T)$ 
où  $Pr$  est un programme de machine de Turing. Les entiers $n$, $m$,  $T$  sont des paramètres de contrôle.  
Nous faisons le lien avec la notion de \com introduite par Ko et Friedman. Nous démontrons une propriété universelle 
qui caractérise cette \rp  du point de vue de la \com en temps \poll.
\item La \pres par circuits booléens qu'on note par $\cbo$ et pour laquelle un point rationnel est donné par un quadruplet  $(C,n,m,k)$  où  $C$  code un circuit booléen et  $n$,  $m$,  $k$  sont des paramètres de contrôle.
\item La \pres par circuits arithmétiques fractionnaires (avec magnitude) 
notée  $\caf$. 
Un point rationnel de cette \pres est donné par un couple $(C,M)$ où $C$  est le code d'un circuit arithmétique, et $M$ est un paramètre de contrôle.
\item La \pres par circuits arithmétiques polynomiaux (avec magnitude) notée  $\capo$ analogue à la précédente, mais ici le circuit est \poll 
(c.-à-d. ne contient pas les portes ``passage à l'inverse''). 
\end{itemize}
%-----------------end item------------------
Nous démontrons dans la section \ref{fsubsec42} que les présentations  $\ckf$,  
$\cbo$,  $\csl$,  $\caf$  et  $\capo$  sont \equivas en temps \poll. Ce 
résultat généralise et précise les résultats de Hoover. Non seulement 
les $\p$-points de  $\ckf$  sont ``les mêmes'' que les $\p$-points de  $\capo$, 
mais bien mieux, les bijections  
$$ \ckf \rightarrow \capo \;\; \;  \hbox{et} \; \; \; \capo \rightarrow \ckf $$
qui représentent l'identité de  $\czu$  sont, globalement, calculables en 
temps \poll. 
Ainsi nous obtenons une formulation complètement contrôlée du point de vue 
algorithmique pour le théorème d'approximation de Weierstrass. \\

\smallskip Dans la section~\ref{fsubsec43} , nous démontrons que ces 
présentations ne sont pas de classe  $\p$  en établissant la non 
faisabilité du calcul de la norme (si  $\p\neq \np$). Plus précisément, 
nous définissons convenablement ``le problème de la norme'' et nous 
démontrons qu'il s'agit d'un problème $\np$-complet pour les présentations 
considérées. 

Pour ce qui concerne le test d'appartenance à l'ensemble des (codes des) 
points rationnels,  c'est un problème co-$\np$-complet pour les 
présentations $\ckf$  et  $\cbo$, tandis qu'il est en temps linéaire pour la 
\pres  $\csl$. Cette dernière est donc 
légèrement plus satisfaisante.

\medskip Nous définissons dans la section \ref{fsec5}  d'autres présentations 
de l'espace  $\czu$  à savoir:
%-----------------begin item------------------
\begin{itemize}
\item La \pres notée  $\cw$  (comme Weierstrass) pour laquelle l'ensemble des 
points rationnels est l'ensemble des \pols (à une variable) à 
coefficients rationnels donnés en \pres dense.
\item La \pres notée  $\csp$  pour laquelle l'ensemble des points rationnels 
est l'ensemble des fonctions \polles par morceaux, chaque \pol 
étant donné comme pour  $\cw$.
\item La \pres notée  $\csr$  est obtenue à partir de  $\crf$  de la même 
manière que   $\csp$    est obtenue à partir de  $\cw$.
\end{itemize}
%-----------------end item------------------
Nous essayons de voir jusqu'à quel point le caractère de classe $\p$ de ces 
présentations fournit un cadre de travail adéquat pour l'analyse 
numérique.\\
Nous caractérisons les $\p$-points de  $\cw$ en établissant l'équivalence 
entre les propriétés suivantes: (cf. théorème \ref{f527})
\begin{itemize}
 \item[a)] $f$  est un $\p$-point de  $\cw$
\item[b)] la suite   $A_n(f)$ (qui donne le développement de  $f$  en série 
de Chebyshev) est une $\p$-suite dans  $\RR$   et vérifie une majoration : 
\[
\abS{A_n(f)}\;  \leq Mr^{n^{\gamma}} \;\hbox{avec}\;M > 0,~\gamma > 0,~0< r < 1.
\]


\item[c)] $f$ est un $\p$-point de  $\ckf$  et est dans la classe de Gevrey 
\end{itemize}
Nous en déduisons une équivalence analogue entre (cf. \thref{f528}):
\begin{itemize}
\item[a)] $f$ est une fonction analytique et est un $\p$-point de  $\cw$.
\item[b)] $f$ est une fonction analytique et est un $\p$-point de  $\ckf$.
\end{itemize}


\smallskip Les calculs usuels sur les $\p$-points de  $\cw$ (calcul de la norme, du maximum et de l'intégrale d'un $\p$-point de  $\cw$) sont en temps \poll (cf. proposition \ref{f526}). 

De plus, un assez bon comportement de la dérivation vis à vis de la \com est obtenu en démontrant que pour tout $\p$-point (ou toute $\p$-suite) de  $\cw$  la suite (ou la suite double) de ses dérivées est une $\p$-suite de  $\cw$ (cf. théorème \ref{f5210} et corollaire \ref{f5211}). 

Dans le théorème \ref{f5213} et son corollaire \ref{f5214}, nous 
donnons une version plus uniforme des résultats précédents. Combinés avec la proposition \ref{f526} on obtient une amélioration sensible des résultats de  \cite{fMu87}  et  \cite{fKF88}. En outre nos preuves sont plus conceptuelles. 

\smallskip Par ailleurs, ces résultats, combinés avec le théorème de Newman (dans sa version précisée à la section \ref{fsubsec33}), démontrent que le passage de la \pres par fraction rationnelles  $\crf$  à la \pres par \pols (denses)  $\cw$ n'est pas en temps \poll. Le théorème de Newman démontre également que les présentations  $\crf$  et  $\csr$  sont \polt 
équivalentes. 


\bigskip En résumant les résultats des sections \ref{fsec4} et \ref{fsec5}, 
nous obtenons  que l'identité de $\czu$ est \uni de classe  $\p$  dans les cas suivants: 
\[
\cw \rightarrow \csp \rightarrow \crf \equiv \csr 
\rightarrow \ckf \equiv \cbo \equiv \csl \equiv \caf 
\equiv \capo.
\] 
et aucune des flèches dans la ligne ci-dessus n'est une $\p$-équivalence 
sauf peut être  $\csp \rightarrow \crf $  et très éventuellement  $\crf 
\rightarrow \ckf$  (mais cela impliquerait $\p = \np$).
 

\bigskip Nous terminons cette introduction par quelques remarques sur le 
constructivisme.

Le travail présenté ici est écrit dans le style des mathématiques 
constructives à la Bishop telles qu'elles ont été développées 
notamment dans les livres de Bishop \cite{fBi}, Bishop \& Bridges \cite{fBB},  
Mines, Richman \& Ruitenburg \cite{fmrr}. Il s'agit d'un corps de mathématiques constructives en quelque sorte minimal\footnote{Un travail précurseur dans un style purement formel est celui de Goodstein \cite{fGo1,fGo}.}. Tous les théorèmes 
démontrés de cette manière sont en effet également valables pour les 
mathématiciens classiques, le bénéfice étant qu'ici les théorèmes et 
les preuves ont toujours un contenu algorithmique. Le lecteur ou la lectrice classique peut donc lire ce travail comme un prolongement des travaux en analyse récursive par Kleene et Turing puis plus tard notamment par Ker-I. Ko \& H. Friedman (\cite{fKF82}, \cite{fKo91}), N. Th. Müller \cite{fMu86,fMu87}, Pour El \& Richards \cite{fPR}, qui se situent eux dans un cadre de mathématiques classiques.

Par ailleurs les théorèmes et leurs preuves dans le style de Bishop sont 
également acceptables dans le cadre des autres variantes du constructivisme, comme l'intuitionnisme de Brouwer ou l'école constructive russe de A.A. Markov, G.S. Ceitin, N.A. Shanin, B.A. Kushner et leurs élèves. On trouvera une discussion éclairante sur ces différentes variantes de constructivisme dans le livre de Bridges \& Richman 
\cite{fBR}. 

Le livre très complet de Beeson \cite{fBe} donne également des 
discussions approfondies de ces points de vue et de leurs variantes, notamment en s'appuyant sur une étude remarquable des travaux des logiciens qui ont tenté de formaliser les mathématiques constructives.  

Les mathématiques constructives russes peuvent être découvertes à 
travers les livres de Kushner \cite{fKu} et O. Aberth \cite{fAb}. Un article historique très pertinent sur la question est écrit par M.~Margenstern \cite{fMa}. 
Le chapitre IV de \cite{fBe} est également très instructif. Beeson déclare page 58: ``nous espérons démontrer que [cet] univers [mathématique] est un endroit extrêmement divertissant, plein de surprises (comme n'importe quel pays étranger), mais en aucun cas trop chaotique ni invivable''. Les mathématiques constructives russes restreignent leurs objets d'étude aux ``êtres récursifs'' et se rapprochent en cela de certains travaux d'analyse récursive classique développés 
notamment par Grzegorczyk, Kreisel, Lacombe, Schoenfield et Specker, avec 
lesquels elles partagent de nombreux résultats. Un des principes des mathématiques constructives russes est par exemple que seuls existent les réels récursifs donnés par des algorithmes (qui calculent des suites de rationnels à vitesse de convergence controlée). Et une fonction réelle définie sur les réels est un algorithme  $F$  qui prend en entrée un algorithme  $x$  et donne en sortie, sous la condition que  $x$  est un algorithme produisant un réel récursif, un algorithme  $y$  produisant un réel récursif. Ceci conduit par exemple au théorème de Ceitin, faux classiquement, selon lequel toute fonction réelle définie sur les réels est continue en tout point. 
L'analyse récursive classique énonce quant à elle: si un algorithme 
termine chaque fois que l'entrée est le code d'un réel récursif et donne 
alors en sortie le code d'un réel récursif, et s'il définit une fonction 
(c.-à-d. deux codes du même réel récursif en entrée conduisent à 
deux codes d'un même réel récursif en sortie) alors il définit une 
fonction continue en tout point réel récursif (théorème de Kreisel-
Lacombe-Schoenfield). Beeson a analysé la preuve de Kreisel-Lacombe-
Schoenfield pour en préciser les aspects non constructifs. La preuve de 
Ceitin, qui est plus constructive que celle de 
Kreisel-Lacombe-Schoenfield, est analysée dans \cite{fBR}.
 
Dans le style de Bishop, comme on considère que la notion d'effectivité est une notion primitive qui ne se réduit pas nécessairement à la 
récursivité, et que la récursivité, au contraire, ne peut être définie sans cette notion primitive d'effectivité, les nombres réels et les fonctions réelles sont des objets qui conservent une plus grande part de 
``liberté'' et sont donc plus proches des réels et des fonctions réelles 
telles que les conçoivent intuitivement les mathématiciens classiques. 
Et le théorème ``russe'' précédent est donc indémontrable dans le style 
Bishop. De manière symétrique, les théorèmes de mathématiques 
classiques qui contredisent directement des résultats du 
constructivisme russe (comme par exemple la possibilité de définir sans 
aucune ambigüité des fonctions réelles discontinues) ne peuvent être 
démontrés dans les mathématiques du style Bishop. 

En conclusion, les mathématiques constructives russes présentent un 
intérêt historique indéniable, développent une philosophie 
mathématique très cohérente et peuvent sans doute trouver un renouveau à 
partir de préoccupations d'informatique théorique. Il serait donc également intéressant de faire une étude de ces mathématiques du point de vue de la complexité algorithmique.    

\section{Préliminaires}\label{fsec1}
\subsection{Notations}\label{fsubsec11}
%-----------------begin item------------------
\begin{description}
\item[$\NN_1$] ensemble des entiers naturels en unaire.
\item[$\NN$] ensemble des entiers naturels en binaire. Du point de vue de la
 \com, $\NN_1$ est isomorphe à la partie de  $\NN$  formée des puissances 
de  $2$.
\item[$\ZZ$] ensemble des entiers relatifs en binaire.
\item[$\QQ$] ensemble des rationnels présentés sous forme d'une fraction 
avec numérateur et dénominateur en binaire.
\item[$\QQ^{\NN_1}$] ensemble des suites de rationnels, où l'indice est en 
unaire.
\item[$\DD$] ensemble des  nombres de la forme  $k/2^n$  avec $(k,n) \in  \ZZ \times  \NN_1$.
\item[$\DD_{[0,1]}$] $\DD \cap  [0,1]$.
\item[$\DD_n$] ensemble  des nombres de la forme  $k/2^n $ avec $k \in  \ZZ$.
\item[$\DD_{n,[0,1]}$] $\DD_n \cap  [0,1]$.
\item[${\DD[X]}$] ensemble des \pols à coefficients dans  $\DD$   
donnés en \pres dense.
\item[${\DD[X]_f}$] ensemble des \pols à coefficients dans  $\DD$   
donnés en \pres par formule.
\item[$\RR$] ensemble des nombres réels présentés par les suites de 
Cauchy dans  $\QQ$.
\item [$\CB$] ensemble des codes de circuits booléens.
\item [$\CA$] ensemble des codes de circuits arithmétiques.
\item[{\rm MT}] machine de Turing.
\item[{\rm MTO}] machine de Turing à oracle.
\item[$\mu$] \mcu (cf.  définition \ref{f221})
\item[$\Lg(a)$] la longueur du codage binaire du nombre dyadique $\vert a \vert$ (pour  $a \in  \DD$).
\item[$\ta(C)$] taille du circuit  $C$  (le nombre de ses portes).
\item[$\prof(C)$] profondeur du circuit  $C$.
\item[$\Mag(C)$] magnitude du circuit arithmétique  $C$ (cf. définition 
\ref{f4110}).
\item[$\wi{f}$] fonction continue codée par  $f$.
\item[$\log(n)$] pour un entier naturel $n$ c'est la longueur de son codage binaire.
\item[$\M(n)$] \com du calcul de la multiplication de deux entiers en représentation binaire (en multiplication rapide $\M(n) = \Oo(n \log(n) \log\log(n)$).
\item[$\flo{d}$]	la longueur (du codage discret) de l'objet  $d$  (le codage de  $d$  est un mot sur un alphabet fini fixé).
\item[$\Flo{x}$] la partie entière du nombre réel $x$.  
\end{description}
%-----------------end item------------------

%---- paragraph{para }----
\paragraph{Présentations rationnelles de l'espace  $\czu$}~
%------------------------------------------------------------------

%-----------------begin item------------------
\begin{itemize}\itemsep2pt
\item[$\ckf$] \pres par Machine de Turing, à la Ko-Friedman\\
$\ykf$ est l'ensemble (des codes de) points rationnels (cf. définition \ref{f411})
%
\item[$\caf$] \pres par circuits arithmétiques\\
$\yaf$ est l'ensemble (des codes de) points rationnels (cf. définition \ref{f4110})
%
\item[$\capo$] \pres par circuits arithmétiques sans divisions (ou polynomiaux)\\
$\yap$ est l'ensemble (des codes de) points rationnels (cf. section \ref{fsubsubsec414})
% 
\item[$\cbo$] \pres par circuits booléens\\
$\ybo$ est l'ensemble (des codes de) points rationnels  (cf. définition \ref{f418})
%
\item[$\csl$] \pres par circuits semilinéaires binaires\\
$\ysl$ est l'ensemble (des codes de) points rationnels (cf. définition 
\ref{f321})
%
\item[$\crf$] \pres par fractions rationnelles dans un codage par formules\\
$\yrf$ est l'ensemble (des codes de) points rationnels (cf. 
définition \ref{f325})
%
\item[$\cw$]  \pres ``à la Weierstrass'' par \pols dans un codage dense\\
$\yw$ est l'ensemble (des codes de) points rationnels (cf. 
section \ref{fsubsec51})
%
\item[$\csp$] \pres par fonctions \polles par morceaux dans un codage dense\\
$\ysp$ est l'ensemble (des codes de) points rationnels (cf. section \ref{fsubsubsec512})
%
\item[$\csr$] \pres par fractions rationnelles par morceaux dans un codage par formules\\
$\ysr$ est l'ensemble (des codes de) points rationnels (cf. section \ref{fsubsubsec513})
\end{itemize}
%-----------------end item------------------

\subsection{Classes de fonctions discrètes intéressantes}\label{fsubsec12} 
Nous considérerons des classes de fonctions discrètes~$\ca$   (une fonction 
discrète est une fonction de $A^{\star}$
vers  $B^{\star}$  où  $A$  et  $B$  sont deux alphabets finis{\footnote{Si 
on veut donner une allure plus mathématique et moins informatique à la 
chose, on pourra remplacer  $A^{\star}$  et  $B^{\star}$  par des ensembles  
$\NN^k$.}}) jouissant des propriétés de stabilité élémentaires 
suivantes
%-----------------begin item------------------
\begin{itemize}
\item $\ca$ contient les fonctions arithmétiques usuelles et le test de 
comparaison dans  $\ZZ$   (pour  $\ZZ$  codé en binaire).
\item $\ca$ contient les fonctions calculables en temps linéaire  (c.-à-d. 
$\LINT  \subset  \ca$  )
\item $\ca$ est stable par composition.
\item $\ca$ est stable par listes:  si  $f\colon A^{\star} \rightarrow
B^{\star}$ est dans~$\ca$ alors $\lst(f) \colon  \lst(A^{\star})
\rightarrow \lst(B^{\star})$ est également dans~$\ca$ 
($\lst(f)[x_1, x_2,\ldots, x_n] = [f(x_1), f(x_2),\ldots, f(x_n)]$).
\end{itemize}
%-----------------end item------------------
Une classe vérifiant les propriétés de stabilité précédentes sera 
dite {\em élémentairement stable}.
En pratique, on sera particulièrement intéressé par les classes 
élémentairement stables suivantes:
%-----------------begin item------------------
\begin{itemize}
\item $\Rec :$ la classe des fonctions récursives.
\item $\Fnc :$ la classe des fonctions constructivement définies  (en 
mathématiques constructives, ce concept est un concept primitif qui ne 
coïncide pas avec le précédent, et il est nécessaire de l'avoir au 
préalable pour pouvoir définir le précédent, la récursivité étant 
interprétée comme une constructivité purement mécanique dans son 
déroulement comme processus de calcul)
\item $\Prim :$ la classe des fonctions primitives récursives.
\item $\p :$ la classe des fonctions calculables en temps \poll.
\item $\cE:$ la classe des fonctions élémentairement récursives, c.-à-d. 
encore calculables en temps majoré par une composée d'exponentielles.
\item $\PSP :$ la classe des fonctions calculables en espace \poll (avec 
une sortie 
polynomialement majorée en taille).
\item $\LINS :$ la classe des fonctions calculables en espace linéaire (avec une sortie linéairement majorée en taille).
\item $\DSRT(\Lin, \Lin, \Poly):$ la classe des fonctions calculables en espace 
linéaire, \etpo  et avec une sortie linéairement majorée en taille.
\item $\DSRT(\Lin, \Lin, \Oo(n^k)):$ la classe des fonctions calculables en espace 
linéaire, en temps $\Oo(n^k)$ (avec  $k > 1$)  et avec une sortie 
linéairement majorée en taille.
\item $\DSRT(\Lin, \Lin, \Exp):$ la classe des fonctions calculables en espace linéaire, en temps $\exp(\Oo(n))$  et avec une sortie linéairement majorée en taille.
\item $\DRT(\Lin,\Oo(n^k)):$ la classe des fonctions calculables en temps 
$\Oo(n^k)$, (avec  $k > 1$)  et avec une sortie linéairement majorée en 
taille.
\item $\DSR(\Poly, \Lin):$ la classe des fonctions calculables en espace 
polynomial  avec une sortie linéairement majorée en taille.
\item $\DSR(\Oo(n^k), \Lin):$ la classe des fonctions calculables en espace 
$\Oo(n^k)$, (avec  $k > 1$)   avec une sortie linéairement majorée en taille.
\item $\QL:$ la classe des fonctions calculables en temps 
quasilinéaire (cf. Schnorr \cite{fSc}). \\ ($\QL: = \cup_b \DTI 
(\Oo(n.\log^b(n))) = \DTI (\QLin)$).
\end{itemize}
%-----------------end item------------------

\smallskip On remarquera que, hormis les classes $\Rec$  et $\Fnc$, toutes les classes que nous avons considérées sont des classes de \com (au sens de Blum). 
Il n'y a cependant aucune nécessité à cela, comme le montrent justement les exemples de $\Rec$  et $\Fnc$. 
Lorsque nous considérons uniquement la classe $\Fnc$, nous développons un chapitre de mathématiques constructives abstraites.
Notez que $\DTI (\Oo(n^k))$ pour $k > 1$ n'est pas stable par composition. Mais, le plus souvent, les calculs en temps $\Oo(n^k)$ que nous aurons à considérer sont dans la classe $\DRT(\Lin, \Oo(n^k))$  ou même $\DSRT(\Lin, \Lin, \Oo(n^k))$  et ces classes ont les bonnes propriétés de stabilité.

\subsection{Complexité d'une Machine de Turing Universelle}\label{fsubsec13} 
Nous aurons besoin dans la suite d'utiliser une Machine de Turing Universelle et d'estimer sa \com algorithmique. 
Le résultat suivant, pour lequel nous n'avons pas trouvé de référence, semble faire partie du folklore, il nous a été signalé par M. Margenstern.

\begin{flemma} \label{f131} 
Il existe une machine de Turing universelle $MU$ qui fait le travail suivant.\\
Elle prend en entrée:\\
---  le code  (dont la taille est  $p$)  d'une machine de Turing $M_0$ 
(fonctionnant  sur un alphabet fixé, avec une bande d'entrée, une bande de 
sortie, plusieurs bandes de travail)  supposée être de \com en temps  $T $ 
et en espace  $S$  (avec  $S(n) \geq  n$)\\ 
--- une entrée  $x$   (de taille $n$)  pour $M_0$.\\
Elle donne en sortie le résultat du calcul exécuté par  $M_0$  pour 
l'entrée  $x$.\\
Elle exécute cette tâche en un temps  $\Oo(T(n)(S(n)+p))$  et en utilisant un 
espace  $\Oo(S(n)+p).$
\end{flemma} 

\proof 
La machine $MU$ utilise une bande de travail pour y écrire, à chaque étape 
élémentaire de la machine $M_0$,  qu'elle simule, la liste des contenus de 
chacune des variables de~$M_0.$ Pour simuler une étape de $M_0$ la machine  
$MU$  a besoin de  $\Oo(S(n)+p)$  étapes élémentaires, en utilisant un 
espace du même ordre de grandeur.\eop

\subsection{Circuits et programmes d'évaluation}\label{fsubsec14} 
Les familles de circuits constituent des modèles de calcul intéressants, 
notamment du point de vue du parallélisme. Les familles de circuits booléens constituent dans une certaine mesure une alternative au modèle standard des Machines de Turing. Des circuits arithmétiques de faible taille sont capables de calculer des \pols de très grand degré. C'est par exemple le cas pour un circuit arithmétique qui simule une itération de Newton pour une fonction donnée par une fraction rationnelle.
 
Dans tous les cas se pose le problème de savoir quel codage on adopte pour un circuit. Nous choisirons de toujours coder un circuit par un des programmes d'évaluation (ou straight-line program) qui exécutent la même tâche que lui\footnote{À un même circuit peuvent correspondre différents programmes d'évaluation selon l'ordre dans lequel on écrit les instructions à exécuter.}.

Par ailleurs, concernant les circuits arithmétiques, qui représentent des 
\pols ou des fractions rationnelles, nous les envisagerons non pas du point de vue du calcul exact (ce qui serait trop coûteux), mais du point de vue du calcul approché. Se pose alors la question d'évaluer leur temps 
d'exécution lorsqu'on veut garantir une précision donnée sur le 
résultat, pour des entrées dans  $\DD$  données elles-mêmes avec une 
certaine précision. Aucune majoration raisonnable du temps d'exécution ne 
peut être obtenue par des arguments d'ordre général si la profondeur du 
circuit n'est pas très faible, car les degrés obtenus sont trop grands et 
les nombres calculés risquent de voir leur taille exploser. Comme on n'arrive pas toujours à se limiter à des circuits de très faible profondeur, il s'avère indispensable de donner un paramètre de contrôle, appelé magnitude, qui assure que, malgré un éventuel très grand degré, la 
taille de tous les nombres calculés par le circuit (qui seront évalués 
avec une précision elle même limitée) n'est pas trop grande lorsque 
l'entrée représente un réel variant dans un intervalle compact. 



\section{Espaces métriques complets \raprs   }\label{fsec2}

\subsection[Présentation rationnelle d'un espace métrique, \com des 
points \ldots ]{Présentation rationnelle d'un espace métrique, \com des points 
et des familles de points}\label{fsubsec21}
Sauf mention expresse du contraire, les classes de fonctions discrètes que 
nous considérons seront des classes élémentairement stables.
\begin{fdefinition}[\rp  de classe~$\ca$  pour un espace métrique complet] \rm \label{f211}
~\\
Un espace métrique complet  $(X, d_X)$  est {\em donné dans une \pres  
rationnelle de classe~$\ca$} de la manière suivante. On donne un triplet  
$(Y, \delta, \eta)$  où
%-----------------begin item------------------
\begin{itemize}
\item $Y$  est une~$\ca$-partie d'un langage  $A^{\star}$
\item $\eta$ est une application de $Y$ dans $X$
\item$\delta\colon  Y\times Y\times \NN_1\rightarrow
\DD$   est une fonction dans la classe~$\ca$   et vérifiant  (pour  
$n\in\NN_1$  et  $x ,  y , z   \in  Y$):\\
--- $\abs{d_X(x,y)-\delta(\eta(x),\eta(y))}\; \leq 1/2^n$\\
--- $\delta(x,y,n) \in \DD_n $\\ 
--- $\delta(x,y,n) = \delta(y,x,n)$\\
--- $\delta(x,x,n) = 0$\\ 
--- $\abs{\delta(x,y,n+1)-\delta(x,y,n)}\;  \leq 1/2^{n+1}$\\ 
--- $\delta(x,z,n) \leq \delta(x,y,n) + \delta(y,z,n) + 2/2^n$ 
\end{itemize}
%-----------------end item------------------
Si on définit l'écart  $d_Y(x,y)$   comme la limite de $\delta(x,y,n)$ 
lorsque $n$  tend vers l'infini, l'application  $\eta$  de $Y$ dans $X$ 
identifie $(X,d_X)$
au séparé complété de $(Y,d_Y)$.
L'ensemble  $\eta(Y)$  est appelé l'ensemble des {\em points rationnels}  de  
$X$  pour la \pres considérée.  Si  $y \in  Y$  on notera souvent   
$\wi{y}$   pour le point rationnel $\eta(y)$,  et  $y$  est appelé le code de  
$\wi{y}$.
\end{fdefinition}
Nous abrégerons parfois  ``$(Y,\delta,\eta)$ est une \pres  rationnelle de 
classe~$\ca$   pour  $(X,d_X)$''  en  ``$(Y,\delta)$ est une~$\ca$-\pres de  
$(X,d_X)$''.

\begin{fremarks} \label{f212}
1) On peut remplacer  $\DD$  par  $\QQ$  dans la définition ci-dessus sans 
modifier la notion qui est définie. 
Le fait de choisir  $\QQ$  est plus joli mathématiquement, tandis que le fait de choisir   $\DD$  est plus naturel d'un point de vue informatique. 
En outre la contrainte  $\delta(x,y,n) \in \DD_n$  répond à la requête naturelle de ne pas utiliser plus de place que nécessaire pour représenter une approximation à $1/2^n$  près d'un nombre réel.\\
2) Lorsqu'on a donné une \rp  pour un espace métrique abstrait  $(X,d_X)$, 
on dira qu'{\em on l'a muni d'une structure de calculabilité}. Cela revient essentiellement à définir un langage  
$Y \subset  A^{\star}$  puis une application $\eta$ de  $Y$  vers  $X$  dont l'image soit une partie dense de  $X$. La \pres est complètement définie uniquement lorsqu'on a aussi donné une application  
$\delta \colon  Y \times  Y \times  \NN_1  \rightarrow   \DD$  
vérifiant les requêtes de la définition \ref{f211}. 
{\em Dans la suite, on se permettra néanmoins de dire qu'un codage d'une 
partie dense  $Y$  de  $X$  définit une \pres de classe~$\ca$   de  $X$  
lorsqu'on démontre que les requêtes de la définition \ref{f211} peuvent être satisfaites.} \\
3)	Comprise au sens constructif, la phrase incluse dans la définition 	
	``on donne une application  $\eta$  de  $Y$  vers $X$''	 
réclame qu'on ait	 
	``un certificat que  $\eta(y)$  soit bien un élément de  $X$  pour tout $y$  de $Y$''. 	
Il se peut que ce certificat d'appartenance implique lui-même une notion 
naturelle de \com. Lorsque ce sera le cas, par exemple pour l'espace  
$\czu$, il sera inévitable de prendre en compte cette \com. 
{\em Ainsi, la définition \ref{f211} doit en l'état actuel être considérée comme incomplète et à préciser au cas par cas.} C'est sans doute dommage si on se place du point de vue l'élégance formelle des définitions 
générales. Mais c'est une situation de fait qui semble très difficile à 
contourner. 
\end{fremarks}

\begin{fexamples} \label{f213}~\\
--- L'espace  $\RR$  est défini usuellement dans la \pres où l'ensemble des 
points rationnels est  $\QQ$.  De manière équivalente, et c'est ce que nous 
ferons dans la suite, on peut considérer comme ensemble des points rationnels 
de  $\RR$  l'ensemble  $\DD$  des nombres dyadiques{\footnote{Cela correspond 
à la notation $\RR_{\rm conv}$  dans \cite{fLL}  et   $\RR_{\rm con}$  dans 
\cite{fKo83}.}}.\\ 
--- On définira sans difficulté la \pres produit de deux présentations 
pour deux espaces métriques.\\ 
--- Tout {\em ensemble discret}  $Z$  ($Z$  est donné comme une partie d'un 
langage  $A^{\star}$  avec une relation d'équivalence qui définit 
l'égalité dans  $Z$  et qui est testable dans la classe~$\ca$  )  donne 
lieu à un {\em espace métrique discret}, c.-à-d. dans lequel la distance 
de deux points distincts est égale à  $1$. \\
--- Les espaces métriques complets des mathématiques constructives dans le 
style Bishop \cite{fBB} admettent en général une \rp  de classe  $\Fnc$. 
Dans chaque cas concret la \pres s'avère être une \pres de classe $\Prim$ ou 
même  $\p$. 
\end{fexamples}

\begin{fdefinition}[complexité d'un point dans un 
espace métrique rationnellement présenté] \label{f214}~\\
On considère un espace métrique complet  $X$  donné dans  une \pres  
$(Y,\delta,\eta)$  de classe $\ca'$. Un point  $x$  de  $X$  est dit {\em de 
classe}~$\ca$   lorsqu'on connaît une suite  ($n \mapsto y_n$)  de classe 
$\ca$  (en tant que fonction   $\NN_1   \rightarrow   Y$)   avec   
$d_X(x,\eta(y_n)) \leq 1/2^n$.
 On dit encore que  $x$  est {\em un~$\ca$-point dans  $X$}.
\end{fdefinition}

\begin{fexample}  \label{f215}
Lorsque  $X = \RR$, la définition ci-dessus correspond à la notion usuelle 
de ``nombre réel de classe~$\ca$''    (au sens de Cauchy).
\end{fexample}

\begin{fremark}\label{f216}
Il semblerait naturel de demander que la classe $\ca'$ contienne la classe  
$\ca$, mais ce n'est pas complètement indispensable. Cette remarque vaut pour  à peu près toutes les définitions qui suivent.
\end{fremark}


%--- defbis{214bis}-------------------- 
\mni
{\bf Définition \ref{f214}bis  ~}
{\em (Complexité d'un point dans un espace métrique 
rationnellement présenté, version plus explicite)} 
On considère un espace métrique complet  $X$  donné dans  une \pres de 
classe  $\ca'$: $(Y, \delta,\eta)$. \\
Un point  $x$  de classe~$\ca$   dans  $X$  est donné par une suite  ($n 
\mapsto y_n$)  de classe~$\ca$   
(en tant que fonction   $\NN_1   \rightarrow   Y$)   
vérifiant les conditions
$\delta(y_n,y_{n+1},n+1) \leq 1/2^n$ 
et $x=\lim_{n\rightarrow \infty}\eta(y_n)$.
%--- end-defbis---------------------------


\medskip 
Nous laissons à la lectrice ou au lecteur le soin de vérifier que les deux 
définitions \ref{f214} et \ref{f214}bis sont équivalentes.
Nous passons maintenant à la définition de la \com pour une famille de 
points (avec un ensemble d'indices discret)

\begin{fdefinition}[complexité d'une famille de points dans un espace métrique rationnellement présenté] \label{f217} 
On considère un préensemble discret  $Z$  (c'est l'ensemble des indices de 
la famille, il est donné comme une partie d'un langage  $A^{\star}$  sur un alphabet fini  $A$)  et un espace métrique complet~$X$  donné dans  une \pres de classe $\ca':~(Y, \delta,\eta).$  Une fonction (ou famille)   
$f\colon  Z \rightarrow X$  est dite de classe~$\ca$  lorsqu'on connaît une fonction $\varphi : Z \times \NN_1 \rightarrow Y$ qui est de classe~$\ca$   et qui vérifie 
$d_X(f(z),\eta(\varphi(z,n))) \leq 1/2^n$  pour tout $z \in Z$. 
On dit alors que  $\varphi$  est {\em une \pres de classe~$\ca$  de la famille $f(z)_{z\in Z}$ de points de  $X$.}
\end{fdefinition}

\begin{fexamples} \label{f218}~\\
--- Lorsque  $Z = \NN_1$  et  $X = \RR$  la définition ci-dessus correspond 
à la notion usuelle de ``suite de réels de classe~$\ca$'' (on dit encore: 
{\em une~$\ca$-suite dans  $\RR$}). \\
--- La définition d'un espace \rapr dans la classe~$\ca$  peut être relue de 
la manière suivante. On donne: \\
\spa -- une~$\ca$-partie  $Y$  d'un langage  $A^{\star}$. \\
\spa -- une fonction $\varphi \colon  Y \rightarrow X$ telle que $\varphi(Y)$ soit 
dense dans  $X$  et telle que la famille de nombres réels  $\; \; Y \times Y
\rightarrow \RR$, $\; \; (y_1, y_2) \mapsto  d_X(\varphi(y_1), \varphi(y_2))$ 
soit de classe~$\ca$. 
\end{fexamples}
 
\begin{fproposition} \label{f219}
Soit  $(x_n)$  une suite de classe~$\ca$   (l'ensemble d'indices est  $\NN_1$) 
dans un espace métrique \rapr  $(X,d)$.  Si la suite est explicitement de 
Cauchy avec la majoration   $d(x_n, x_{n+1}) \leq 1/2^n$  alors la limite de la 
suite est un point de classe~$\ca$   dans $X$.
\end{fproposition}
\proof Soient  $(Y, \delta, \eta)$  une re\rp  de  $(X,d)$  et  $x$  la limite de la suite  $(x_n)$.
La suite  $(x_n)$  est de classe~$\ca$   dans  $X$, donc il existe une fonction  $\psi \colon  \NN_1 \times \NN_1 \rightarrow Y$  de classe~$\ca$  telle que: 
$$d(x_n,\psi(n,m)) \leq 1/2^m \; \;\hbox{pour tout}\; n, m \in \NN_1.
$$
On pose   $z_n = \psi(n+1,n+1)$, c'est une suite de classe~$\ca$    et
$$d(x,z_n) \leq d(x,x_{n+1}) + d(x_{n+1},\psi(n+1,n+1)) \leq 1/2^{n+1} +
1/2^{n+1} = 1/2^n.
$$
Donc  $x$  est un~$\ca$-point dans  $X$.
\eop

\begin{fremark}\label{f2110}
Si on ralentit suffisamment la vitesse de convergence de la suite de Cauchy, on 
peut obtenir comme point limite d'une suite de classe  $\DTI (n^2)$  un point 
récursif arbitraire de~$X$  (cf. \cite{fKF82} pour l'espace  $[0,1]$).
\end{fremark} 
%--- defbis{217bis}-------------------- 
\mni
{\bf Définition \ref{f217}bis  }
{\em (complexité d'une famille de points dans un espace métrique 
rationnellement présenté, version plus explicite)} 
On considère un préensemble discret  $Z$  (une partie d'un 
langage~$A^{\star}$)  et un espace métrique complet  $X$  donné dans  une \pres de classe $\ca': (Y,\delta,\eta).$  
Une famille $ f \colon  Z \rightarrow X$ est dite  de classe~$\ca$    si on a une fonction  
$ \varphi \colon  Z \times \NN_1 \rightarrow Y$ de classe~$\ca$   qui vérifie $\delta(\varphi(z,n),\varphi(z,n+1),n+1) \leq 1/2^n  $
et $f(z)=\lim_{n\rightarrow \infty}\eta(\varphi(z,n))$ pour tout $z\in Z$.
%--- end-defbis---------------------------

\subsection{Complexité des fonctions \uni continues} \label{fsubsec22}

Nous commençons par donner une définition ``raisonnable'' qui sera 
justifiée par les exemples et propositions qui suivent.

\begin{fdefinition}[\com des fonctions \unicos entre espaces métriques rationnellement présentés] \label{f221}
On considère deux espaces métriques complets  $X_1$  et  $X_2$  donnés 
dans  des présentations de classe  $\ca'$:   $(Y_1,\delta_1,\eta_1)$  et  
$(Y_2,\delta_2,\eta_2).$  Soient  $f \colon  X_1 \rightarrow X_2 $  une fonction \unico 
et  $\mu \colon  \NN_1 \rightarrow \NN_1$ une suite d'entiers.\\ 
On dira que $\mu$ est {\em un \mcu pour la fonction   $f$}  si on a :  
$$d_{X_1}(x,y) \leq 1/2^{\mu(n)} \Rightarrow d_{X_2}(f(x),f(y)) \leq
1/2^n \; \hbox{pour tous} \; x, y \in X_1 \;\hbox{et}\; n \in \NN_1$$
On dira que la fonction   $f$   est {\em \uni de classe}~$\ca$    (pour les 
présentations considérées) lorsque\\
--- elle possède un \mcu  $\mu \colon  \NN_1  \rightarrow   \NN_1$   dans la 
classe~$\ca$\\  
--- la restriction de  $f$  à  $Y_1$  est dans la classe~$\ca$   au sens de 
la définition \ref{f217}, c.-à-d. qu'elle est présentée par une fonction   
$\varphi\colon \NN_1 \times Y_1 \rightarrow Y_2 $  qui est de classe~$\ca$   et qui 
vérifie:
$$d_{X_2}(f(y),\eta_2(\varphi(y,n))) \leq 1/2^n \; \hbox{pour tout}\,  y \in 
Y_1.$$	
Lorsque  $X_1 = X_2 =X$  la fonction  $\Id_X$  admet  $\Id_{\NN_1}$  pour 
\mcu. Si les deux fonctions  $\Id_X$  (de  $X_1$  vers  $X_2$  et de  $X_2$  
vers  $X_1$)  sont dans la classe~$\ca$   on dira que {\em les deux 
présentations sont 
(\uniz)~$\ca$-équivalentes.}
\end{fdefinition}

\begin{fremarks}\label{f222}~\\
1)	On ne demande pas que, pour  $n$  fixé, la fonction  $y \mapsto
\varphi(y,n)$  soit \unico sur  $Y_1$  ni même qu'elle soit continue en chaque 
point de  $Y_1$.\\
2)	Notez que la définition ici donne un \mcu correspondant (à très peu 
près) à la définition donnée en  théorie 
classique de l'approximation. On appelle en général \mcu une suite croissante d'entiers  $\NN\to\ZZ:\;n \mapsto \nu(n)$  qui tend vers $+\infty$ et qui vérifie:
\[
d_{X_1}(x,y) \leq 1/2^n \Rightarrow d_{X_2}(f(x),f(y)) \leq 1/2^{\nu(n)}
\]	
La fonction $\nu$ est en quelques sorte ``réciproque'' de la fonction $\mu$.
\end{fremarks}

\begin{fexamples}\label{f223}~\\
--- Lorsque  $X_1 = [0,1]$ et $X_2 = \RR$  et lorsque~$\ca$    est une classe 
de \com en temps ou en espace {\em élémentairement stable}, la définition 
ci-dessus est \equiva à la notion usuelle de fonction réelle calculable dans 
la classe~$\ca$  (cf. \cite{fKF82}, \cite{fKo91}), comme nous le prouverons en 
détail à la proposition~\ref{f415}.\\
--- Lorsque  $X_1$ est un espace métrique discret la définition~\ref{f221} 
redonne la définition~\ref{f217}. 
\end{fexamples}
La définition \ref{f221} rend accessible la notion de \com pour les fonctions 
\unicos. Cela traite donc toutes les fonctions continues dans le cas où 
l'espace de départ est compact. Rappelons à ce sujet qu'en analyse 
constructive on définit, dans le cas d'un espace compact, la continuité 
comme signifiant la continuité uniforme (cf. \cite{fBB}) et non pas la 
continuité en tout point. 

\begin{fremark}\label{f224} 
Le contrôle de la continuité donné par le \mcu est essentiel dans la 
définition \ref{f221}. On peut par exemple définir une fonction  $\varphi \colon [0,1] \rightarrow \RR$  qui est continue au sens classique, dont la restriction 
à  $\DD_{[0,1]} = \DD\cap [0,1]$  est $\LINT$  mais qui ne possède pas de 
\mcu récursif. Pour cela on considère une fonction  $\theta \colon  \NN 
\rightarrow \NN$  de classe $\LINT $ injective et d'image non récursive. Pour 
chaque $m \in \NN_1$ on considère  $n = \theta(m)\; , a_n = 1/(3.2^n)$  et on 
définit  $\psi_m \colon  [0,1] \rightarrow\RR$  partout nulle sauf sur un 
intervalle centré en   $a_n$  sur lequel le graphe de  $\psi_m$  fait une 
pointe de hauteur  $1/2^n$  avec une pente égale à  $1/2^m$.  Enfin, on 
définit  $\varphi$  comme la somme de la série  $\sum_m \psi_m$.  Bien que 
la suite  $\varphi(a_n)$  ne soit pas une suite récursive de nombres réels 
(ce qui implique d'ailleurs que $\varphi$ ne puisse avoir de \mcu récursif), 
la restriction de $\varphi$ aux nombres dyadiques est très facile à calculer  
(les nombres réels litigieux  $a_n$  ne sont pas des dyadiques). Cet exemple 
met bien en évidence que {\em la définition classique de la continuité (la 
continuité en tout point) ne permet pas d'avoir accès au calcul des valeurs 
de la fonction à partir de sa restriction à une partie dense de l'espace de 
départ.}
\end{fremark}

\begin{fremark} \label{f225}
Dans \cite{fKF82}  il est ``démontré'' que si une fonctionnelle définie via 
une machine de Turing à oracle (MTO) calcule une fonction de  $[0,1]$  
vers  $\RR$,  alors la fonction possède un module de continuité uniforme 
récursif. Si on évite tout recours aux principes non constructifs (caché 
dans le théorème de Heine-Borel), la preuve de \cite{fKF82} peut être 
facilement transformée en une preuve constructive du théorème suivant:  
si une fonction  $f\colon [0,1] \rightarrow \RR$  est \unico et calculable par une MTO, alors son \mcu est récursif.\\ 
A contrario, si on fait l'hypothèse (nullement invraisemblable) selon laquelle tout oracle d'une machine de Turing serait fourni par un procédé mécanique (mais inconnu), il est possible de définir des fonctions ``pathologiques'' de  $[0,1]$  dans  $\RR$  au moyen de machines de Turing à oracles: ce sont des fonctions continues en tout point réel récursif mais non \unicos. 
Ceci est basé sur l'``arbre singulier de Kleene'': un arbre binaire récursif infini qui ne possède aucune branche infinie récursive. Cf. \cite{fBe}, théorème de la section 7 du chapitre 4, page 70 où est donnée une fonction  $t(x)$  continue en tout point réel récursif sur $[0,1]$, mais qui n'est pas bornée (donc pas \unicoz) sur cet intervalle.
\\  
Il semble étrange que dans une preuve concernant les questions de 
calculabilité, on puisse utiliser sans même la mentionner l'hypothèse 
que les oracles d'une MTO se comportent de manière antagonique avec la 
Thèse de Church (du moins sous la forme où elle est admise dans le
 constructivisme russe).
\end{fremark}

On vérifie sans peine que la définition \ref{f221} peut être traduite sous 
la forme plus explicite suivante. 

%--- defbis{221bis}--------------------
\mni
{\bf Définition \ref{f221}bis  ~}
{\em (\com des fonctions \unicos entre espaces métriques rationnellement 
présentés, forme plus explicite)}
On considère deux espaces métriques complets  $X_1$  et  $X_2$  donnés 
dans  des présentations de classe  $\ca'$:   $(Y_1,\delta_1,\eta_1)$  et  
$(Y_2,\delta_2,\eta_2).$  Une fonction \unico  $f \colon  X_1 \rightarrow X_2 $ est 
dite {\em  uniformément de classe~$\ca$}   (pour les présentations 
considérées) lorsqu'elle est présentée au moyen de deux données:\\
--- la restriction de  $f$  à $Y_1$   est présentée par une fonction   
$\varphi \colon  Y_1\times \NN_1 \rightarrow Y_2$ qui est de classe~$\ca$   et qui vérifie:
$$\delta_2(\varphi(y,n),\varphi(y,n+1),n+1) \leq 1/2^n \; \hbox{pour tout} \; 
y \in Y_1.$$
avec $f(y)=\lim_{n\rightarrow \infty}\eta_2(\varphi(y,n))$.\\
--- une suite  $\mu\colon  \NN_1  \rightarrow   \NN_1$   dans la classe~$\ca$    
vérifiant:  pour tous  $x$  et  $y$  dans  $Y_1$
$$\delta_1(x,y,\mu(n))\le 1/2^{\mu(n)} \; \Rightarrow\;  
\delta_2(\varphi(x,n+2),\varphi(y,n+2),n+2) \leq 1/2^n.$$
%--- end-defbis---------------------------

\medskip 
Les résultats qui suivent sont faciles à établir.
\begin{fproposition} \label{f226}
La composée de deux fonctions \uni de classe~$\ca$  est une fonction \uni de 
classe~$\ca$. 
\end{fproposition}
\begin{fproposition} \label{f227}
L'image d'un point  (respectivement d'une famille de points)  de classe~$\ca$  
par une fonction \uni de classe~$\ca$   est un point  (respectivement d'une 
famille de points)  de classe~$\ca$. 
\end{fproposition}

\begin{fremark}\label{f228}
Considérons deux \rps  d'un même espace métrique complet~$X$ données 
par les deux familles de points rationnels respectifs  $(\wi{y})_{y \in 
Y}$  et  $(\wi{z})_{z \in Z}$.  Si la première famille est $\LINT $ pour 
la première présentation elle peut naturellement être de plus grande 
\com pour la deuxième. \\
De manière générale, {\em dire que l'identité de  $X$  est \uni de classe  
$\ca$   lorsqu'on passe de la première à la deuxième \pres revient 
exactement à dire que la famille $(\wi{y})_{y \in Y}$  est une~$\ca$-
famille de points pour la deuxième présentation.} \\
Ainsi  deux présentations de $X$ sont~$\ca$-\equivas si et seulement si la famille  $(\wi{y})_{y \in Y}$  est une famille de classe~$\ca$  dans la seconde \pres et $(\wi{z})_{z \in Z}$   est une famille de classe~$\ca$   dans la première. 
La proposition~\ref{f227} nous permet de formuler un résultat analogue et plus 
intrinsèque: {\em deux \rps  d'un même espace métrique complet sont 
$\ca$-\equivas si et seulement si elles définissent les mêmes familles de 
points de classe~$\ca$}.
\end{fremark}
Nous passons maintenant à la définition de la \com pour une famille de 
fonctions \unicos (avec un ensemble d'indices discret).

\begin{fdefinition}[\com d'une famille de fonctions \unicos entre espaces métriques 
rationnellement présentés]\label{f229}

On considère un préensemble discret  $Z$  (c'est l'ensemble des indices de 
la famille, il est donné comme une partie d'un langage  $A^{\star}$  sur un alphabet fini  $A$)  et deux espaces métriques complets  $X_1$  et  $X_2$  donnés dans  des présentations de classe  $\ca'$:   $(Y_1,\delta_1,\eta_1)$  et  $(Y_2,\delta_2,\eta_2)$.\\  
Nous notons  $\U(X_1,X_2)$  l'ensemble des fonctions \unicos de  $X_1$  dans  $X_2$.  
Une famille de fonctions \unicos $\wi{f}\colon Z\rightarrow 
\U(X_1,X_2)$ est dite {\em \uni de classe~$\ca$} (pour les présentations considérées) lorsque\\
--- la famille possède un \mcu  $\mu \colon  Z \times \NN_1 \rightarrow\NN_1$   
dans la classe~$\ca$ :   la fonction  $\mu$  doit vérifier     
\[
\forall  n \in \NN_1\;\forall  x,
y \in X_1 \;\; \; \;  \big(d_{X_1}(x,y) \leq 1/2^{\mu(f,n)} \Rightarrow 
d_{X_2}(\wi{f}(x),\wi{f}(y)) \leq 1/2^n \big)
\] 
--- la famille   $(f,x) \mapsto \wi{f}(x) \colon  Z \times X_2 \rightarrow
X_2$  est une famille de points dans  $X_2$  de classe~$\ca$   au sens de la 
définition~\ref{f217}, c.-à-d. qu'elle est présentée par une fonction    
$\varphi\colon  Z \times Y_1 \times \NN_1 \rightarrow Y_2 $    qui est de classe  
$\ca$   et qui vérifie:
$$d_{X_2}\big(\wi{f}(x),\varphi(f,x,n)\big) \leq 1/2^n \; \hbox{pour tout} \; 
(f, x, n) \in Z \times Y_1 \times \NN_.$$
\end{fdefinition}

\begin{fremark} \label{f2210}
La notion définie en 2.2.9 est naturelle car elle est la relativisation à la classe~$\ca$ de la notion constructive de famille de fonctions \unicos. 
Mais cette notion naturelle ne semble pas pouvoir se déduire de la 
définition \ref{f221} en munissant   $Z \times  X_1$  d'une structure 
convenable d'espace métrique \rapr et en demandant que la fonction   $(f,x) \mapsto \wi{f}(x) \,;\, Z \times X_2 \rightarrow X_2$   soit \uni de classe  
$\ca$. Si par exemple, on prend sur   $Z \times  X_1$  la métrique déduite 
de la métrique discrète de  $Z$  et de la métrique de  $X_1$  on obtiendra 
la deuxième des conditions de la définition  \ref{f229}  mais la première 
sera remplacée par la demande que toutes les fonctions de la famille aient un 
même \mcu de classe~$\ca$. C'est-à-dire que la famille devrait être \uni 
équicontinue. Cette condition est intuivement trop forte. Les propositions 
\ref{f2211} et \ref{f2212} qui suivent sont une confirmation que la définition 
\ref{f229} est convenable.
\end{fremark}
Les propositions \ref{f226}  et \ref{f227} se généralisent au cas des familles 
de fonctions. Les preuves ne présentent aucune difficulté.
\begin{fproposition} \label{f2211}
Soient  deux espaces métriques complets  $X_1$  et  $X_2$  
\raprs.  Soit   $\big(\wi{f}\,\big)_{f \in Z}$  une famille dans   $\U(X_1,X_2)$  
\uni de classe~$\ca$    et  $(x_n)_{n \in \NN_1}$  une famille de classe~$\ca$   
dans  $X_1$.\\  
Alors la famille  $(\wi{f}(x_n))_{(f,n) \in Z \times \NN_1}$  est une 
famille de classe~$\ca$   dans  $X_2$.
\end{fproposition}

\begin{fproposition} \label{f2212}
Soient  trois espaces métriques complets  $X_1$, $X_2$  et  $X_3$  \raprs.  
Soit   $\big(\wi{f}\,\big)_{f \in Z}$ une famille dans  $\U(X_1,X_2)$  \uni de 
classe~$\ca$    et    $(\wi{g})_{g \in Z'}$  une famille dans  
$\U(X_2,X_3)$  \uni de classe~$\ca$. \\ 
Alors la famille  
$(\wi{f} \circ \wi{g})_{(f,g) \in Z \times Z'}$  
dans  $\U(X_1,X_3)$  est \uni de classe~$\ca$.
\end{fproposition}

\subsection{Complexité des fonctions ``\loca \unicos''} \label{fsubsec23}
La notion de \com définie au paragraphe précédent pour les fonctions 
\unicos est entièrement légitime lorsque l'espace de définition est 
compact. Dans le cas d'un espace \loca compact au sens de Bishop{\footnote{I.e., toute partie bornée est contenu dans un compact, \cite{fBB}.}} les fonctions continues sont les fonctions \unicos sur tout borné (du point de vue classique c'est un théorème, du point de vue constructif, c'est une définition). Ceci conduit à la notion de \com naturelle suivante.

\begin{fdefinition} \label{f231}
On considère deux espaces métriques complets  $X_1$  et  $X_2$  donnés 
dans  des présentations de classe  $\ca'$:   $(Y_1,\delta_1,\eta_1)$  et  
$(Y_2,\delta_2,\eta_2).$  On suppose avoir spécifié un point  $x_0$  de  
$X_1$  et un point  $y_0$  de  $X_2$. Une fonction  $f\colon  X_1  \rightarrow   X_2$   
est dite {\em \loca \unicoz}   si elle est \unico et bornée sur toute partie 
bornée.\\
Elle est dite {\em \loca \uni de classe~$\ca$}   (pour les présentations 
considérées) lorsque\\
--- elle possède dans la classe~$\ca$   une borne sur tout borné, c.-à-d. 
une~$\ca$-suite  $\beta \colon  \NN_1 \rightarrow \NN_1$ vérifiant, pour tout  $x$  dans  
$X_1$  et tout  $m$   dans  $\NN_1$:    
\[
d_{X_1}(x_0,x) \leq 1/2^n \Rightarrow d_{X_2}(y_0,f(x)) \leq 1/2^{\beta(n)};
\]
--- elle possède dans la classe~$\ca$   un \mcu sur tout borné, c.-à-d. 
une~$\ca$-fonction  $\mu \colon  \NN_1 \times \NN_1 \rightarrow \NN_1$   vérifiant,   pour   
$x, z$  dans  $X_1$  et   $n, m$  dans  $\NN_1$:    
\[
\left( d_{X_1}(x_0,x) \leq 1/2^m, \;d_{X_1}(x_0,z) \leq 1/2^m, \; d_{X_1}(x,z) \leq 1/2^{\mu(m,n)}\right) \; \Rightarrow\; d_{X_2}(f(x),f(z)) \leq 1/2^n;
\] 
--- la restriction de $f$ à $Y_1$ est dans la classe~$\ca$ au sens de 
la définition \ref{f217}, c.-à-d. qu'elle est présentée par une~$\ca$-fonction    
$\varphi\colon \NN_1 \times Y_1 \rightarrow Y_2 $     qui vérifie:
\[
d_{X_2}(f(y),\varphi(y,n)) \leq 1/2^n \;\hbox{pour tout} \; y \in Y_1.
\]
\end{fdefinition}
 
Notez que la notion définie ci-dessus ne dépend pas du choix des points
$x_0$ et $y_0$.

\begin{fexamples}\label{f232}~\\
--- Lorsque  $X_1 = \RR$  et  $X_2 = \RR$  la définition ci-dessus est 
\equiva à la notion naturelle de fonction réelle calculable dans la classe  
$\ca$   telle qu'on la trouve dans \cite{fHo90} et \cite{fKo91}.\\
--- La fonction  $x \mapsto x^2$  est \loca \unico de classe $\QL$ mais elle 
n'est pas \unico sur $\RR$.
\end{fexamples}

\begin{fremarks}\label{f233} ~\\
1)	Lorsque  $f$  est une fonction \unico, la définition \ref{f231} et la 
définition \ref{f221} se ressemblent beaucoup. Cependant la définition 
\ref{f221} est a priori plus contraignante. En effet, si $\mu(n)$  est un module 
de continuité dans la définition \ref{f221}, alors on peut prendre dans la 
définition \ref{f231}  $\mu'(m,n) = \mu(n)$. Mais réciproquement, supposons 
qu'on ait un \mcu sur tout borné vérifiant  $\mu'(m,n) = \inf(m, 2^n)$   
alors, la fonction est \unico mais le meilleur \mcu qu'on puisse en déduire 
est  
$$\mu(n) = {\bf Sup} \left\{ \mu'(m,n)\,;\, m \in \NN \right\} = 2^n$$
et il a un taux de croissance exponentiel alors que  $\mu'(m,n)$  est 
linéaire. Ainsi la fonction  $f$  peut être de \com linéaire en tant que 
fonction \loca \unico, et exponentielle en tant que fonction \unico.  
Les deux définitions sont \equivas si l'espace  $X_1$  a un diamètre fini.
\\
2)	En analyse constructive un espace compact est un espace précompact et 
complet. Il n'est pas possible de démontrer constructivement que toute partie 
fermée d'un espace compact est compacte. Un espace 
(\uniz) \loca compact est un espace métrique complet dans lequel tout borné 
est contenu dans un compact  $K_n$,  où la suite  $(K_n)$  est une suite 
croissante donnée une fois pour toutes (par exemple le compact  $K_n$  
contient la boule  $B(x_0, 2^n)$). Une fonction définie sur un tel espace est 
alors dite continue si elle est \unico sur toutes les parties bornées.  En 
particulier elle est bornée sur toute partie bornée. La définition 
\ref{f231} permet donc de donner dans ce cas une version ``\com'' de la 
définition constructive de la continuité.
\\
3)	La proposition \ref{f226} sur la composition des fonctions \uni de classe 
$\ca$  reste valable pour les fonctions \loca \uni de classe~$\ca$. La 
proposition \ref{f227} également.
\end{fremarks}


\subsection{Une approche générale de la \com des fonctions continues  } 
\label{fsubsec24} 
La question de la \com des fonctions continues n'est manifestement pas 
épuisée. 
Comme nous l'avons déjà signalé, s'il est bien vrai qu'une fonction 
continue  $f(x)$  est classiquement bien connue à partir d'une \pres  $(y, 
n) \mapsto \varphi(y,n)$  qui permet de la calculer avec une précision 
arbitraire sur une partie dense  $Y$  de  $X$, la \com de  $\varphi$  ne peut 
être tenue que pour un pâle reflet de la \com de  $f$  (cf. les remarques 
\ref{f224} et \ref{f233}(1)).
Une question cruciale, et peu étudiée jusqu'à présent, est de savoir 
jusqu'à quel point on peut certifier qu'une telle donnée  $\varphi$  
correspond bien à une fonction continue~$f$. Dans le cas d'une réponse 
positive, il faut expliquer par quelle procédure on peut calculer des valeurs approchées de~$f(x)$  lorsque~$x$  est un point arbitraire de  $X$  (donné par exemple par une suite de Cauchy de points rationnels de la \pres dans le cas d'un espace métrique rationnellement présenté).\\
Nous proposons une approche un peu informelle de cette question.
Soit  $\phi \colon  X_1 \rightarrow X_2$  une fonction entre espaces métriques et  
$(F_{\alpha})_{\alpha \in M}$  une famille de parties de $X_1$.  {\em Un \mcu 
pour  $\phi$  près de chaque partie  $F_{\alpha}$ } est par définition une 
fonction   $\mu \colon M \times \NN_1 \rightarrow \NN_1$  vérifiant:	
\[ 
\forall \alpha \in M \; \forall x \in F_{\alpha} \; \forall x'\in X_1\; \; \; \left(d_{X_1}(x,x') 
\leq 1/2^{\mu(\alpha,n)} \Rightarrow d_{X_2}(\phi(x),\phi(x'))
 \leq 1/2^n \right).
\]

\begin{fdefinition}[définition générale mais un peu informelle pour ce qu'est une fonction continue et ce qu'est sa \com] \label{f241}
On considère deux espaces métriques complets  $X_1$  et  $X_2$  donnés 
dans  des \rps  de classe  $\ca'$:   $(Y_1,\delta_1,\eta_1)$  et  
$(Y_2,\delta_2,\eta_2).$  On suppose avoir spécifié un point  $y_0$ de 
$X_2$.\\ 
Supposons que nous ayons défini une famille  $F_{X_1} = (F_{\alpha})_{\alpha \in M}$  de parties de  $X_1$  avec la propriété suivante:\\
--- (2.4.1.1)   tout compact de  $X_1$  est contenu dans un des  $F_{\alpha}$. 
\\
On dira qu'une fonction  $\phi\colon  X_1 \rightarrow X_2$  est   {\em $F_{X_1}$-
\unicoz}   si elle est bornée sur chaque partie  $F_{\alpha}$  et si elle 
possède un \mcu près de chaque partie $F_{\alpha}$.\\  
Supposons maintenant en plus que la famille $(F_{\alpha})_{\alpha \in M}$  ait 
la propriété suivante:\\
--- (2.4.1.2)   l'ensemble d'indices  $M$  a une certaine ``structure de 
calculabilité''\\
On dira qu'une fonction  $\phi\colon  X_1 \rightarrow X_2$  est  {\em $F_{X_1}$-\unico 
de classe~$\ca$  }  si elle vérifie les propriétés suivantes:\\
--- une borne sur chaque partie  $F_{\alpha}$  peut être calculée en 
fonction de $\alpha$   dans la classe~$\ca$, c.-à-d. qu'on a une fonction de 
classe~$\ca$,     $ \beta \colon  M \rightarrow \NN_1$   vérifiant:  
\[
\forall \alpha \in M \; \forall x \in F_{\alpha} \; \;
 d_{X_2}(y_0, \phi (x)) \leq 2^{\beta(\alpha)},
\] 
--- un \mcu près de  $F_{\alpha}$  peut être calculé dans la classe  
$\ca$:  une fonction   $\mu \colon  M \times \NN_1 \rightarrow \NN_1$   dans la 
classe~$\ca$   vérifiant:
\[
\forall \alpha \in M \; \forall x \in F_{\alpha} \; \forall x' \in X_1\; \; \; \left(d_{X_1}(x,x') 
\leq 1/2^{\mu(\alpha,n)} \Rightarrow d_{X_2}(\phi(x),\phi(x'))
 \leq 1/2^n \right),
\]
--- la restriction de  $\phi$  à  $Y_1$ est calculable dans la classe~$\ca$.
\end{fdefinition}
Il s'agit manifestement d'une extension des définitions \ref{f221} et \ref{f231}.
Le caractère informel de la définition tient évidemment à ``la 
structure de calculabilité'' de  $M$.\\
A priori on voudrait prendre pour  $(F_{\alpha})_{\alpha \in M}$ une famille de 
parties suffisamment simple pour vérifier la condition (2.4.1.2) et 
suffisamment grande pour vérifier la condition (2.4.1.1).\\
Mais ces deux conditions tirent dans deux sens opposés.\\ 
Notez que la définition \ref{f241} est inspirée de la notion de fonction 
continue définie par D. Bridges dans \cite[Constructive functional analysis]{fBr}.

\subsection {Ouverts et fermés} \label{fsubsec25}
Nous passons à la définition d'une structure de calculabilité pour un 
sous-espace ouvert $U$ d'un espace métrique $X$ muni d'une structure de 
calculabilité. La métrique induite par $X$ sur $U$ ne saurait en général 
nous satisfaire car l'espace obtenu n'est généralement pas complet. On a 
néanmoins une construction qui fonctionne dans un cas particulier important.\\ 
Soient $X$ un espace métrique complet et   $f\colon  X \rightarrow \RR$  une 
fonction 
\loca \unico.  L'ouvert  $U_f = \{x \in X\,;\, f(x) > 0 \}$  est un espace 
métrique complet pour la distance  $d_f$  définie par:		
$$d_f(x,y) = d_X(x,y) + \abs{1/f(x) - 1/f(y)}$$

\begin{fpropdef} \label{f251}
Soient $X$ un espace métrique complet donné avec une~$\ca$-\pres $(Y, 
\delta, \eta)$, et   $f\colon  X \rightarrow \RR$  une fonction \loca \uni de classe  
$\ca$    représentée par un \mcu et une fonction discrète de classe  
$\ca$,
$ \varphi \colon  Y \times \NN_1 \rightarrow \DD$. Alors l'espace métrique complet  
$(U_f, d_f)$ peut être muni d'une~$\ca$-\pres où l'ensemble des points 
rationnels est (codé par) l'ensemble  $Y_U$ des couples  $(y,n)$  de  $Y 
\times  \NN_1$  vérifiant:		$$n \geq 10 \; \; \hbox{et} \; \; 
\varphi(y,n) \geq 1/2^{n/8}$$
\end{fpropdef}

\proof Une preuve détaillée est donnée dans \cite{fMo}.\eop

\medskip Nous attaquons maintenant la définition d'une structure de calculabilité 
pour un sous-espace fermé $F$ d'un espace métrique $X$ muni d'une structure 
de calculabilité. En analyse constructive, un sous-ensemble fermé n'est 
vraiment utile que lorsqu'il est situé, c.-à-d. lorsque la fonction  $D_F$  
distance au fermé est calculable. 
Au moment de traduire cette notion en termes de calculabilité récursive ou de \com, nous devons prendre garde que dans la définition constructive, le fait que la fonction  $D_F$  est la fonction distance au fermé doit être aussi rendu explicite.

\begin{fdefinition} \label{f252}
On considère un espace métrique complet $X$ donné par une~$\ca$-\pres  
$(Y, \delta, \eta)$.  Une partie  $F$  de $X$ est appelée {\em un fermé 
$\ca$-situé de $X$}  si:
\begin{itemize}
%
\item [i)]	la fonction  $D_F\colon   x \mapsto d_X(x,F)$   de $X$ vers  $\RR^{\geq 0}$   est 
calculable dans la classe~$\ca$,
%
\item [ii)]	il existe une fonction calculable dans la classe~$\ca$ :    $P_F \colon  Y 
\times \NN_1\rightarrow X$ qui certifie la fonction  $D_F$  au sens suivant: pour tout   $(y,n) \in  Y \times  \NN_1$,
\[
D_F(P_F(y,n)) = 0,\; \; \; \;  d_X(y,P_F(y,n)) \leq D_F(y) + 1/2^n 
\] 
%
%\item  
\end{itemize}
\end{fdefinition}

\begin{fremarks}\label{f253}~\\
1)	La fonction  $P_F(y,n)$  calcule un élément de  $F$  dont la distance 
à  $y$  est suffisamment proche de  $D_F(y)$. Cependant, la fonction  
$P_F(y,n)$  ne définit pas en général, par prolongement par continuité~à~$X$ et par passage à la limite lorsque  $n$  tend vers  $\infty$, un 
projecteur sur le fermé $F$. \\
2)	On peut démontrer que les points  $P_F(y,n)$  (codés par les couples  $(y,n) 
\in  Y \times  \NN_1$)  forment une partie dénombrable dense de  $F$  qui est 
l'ensemble des points rationnels d'une~$\ca$-\pres de  $F$  (cf. \cite{fMo}).
\end{fremarks}


\subsection{Espaces de Banach rationnellement présentés}\label{fsubsec26}
Nous donnons une définition minimale. Il va de soi que pour chaque espace de Banach particulier, des notions de \com naturellement attachées à l'espace considéré peuvent éventuellement être prises en compte en plus pour obtenir une notion vraiment raisonnable. 

\begin{fdefinition}[\rp  d'un espace de Banach]\label{f261} 
Une \rp  d'un espace de Banach séparable $X$ sur le corps  $\KK$  ($\RR$  ou $\CC$) sera dite de classe~$\ca$ si, d'une part elle est de classe~$\ca$  en tant que \pres de l'espace métrique et, d'autre part les opérations 
d'espace vectoriel suivantes sont dans la classe~$\ca$\\  
--- le produit par un scalaire,\\ 
--- la somme d'une liste de vecteurs choisis parmi les points rationnels.
\end{fdefinition}

Dans le contexte du corps des complexes  $\CC$  nous désignerons par  
$\DD_\KK$  l'ensemble des complexes dont les parties réelle et imaginaire 
sont dans  $\DD$. Dans le contexte réel  $\DD_\KK$  sera seulement une 
autre dénomination de  $\DD$.\\
La proposition suivante n'est pas difficile à établir. 

\begin{fproposition} \label{f262}
Donner une \rp  de classe~$\ca$   d'un espace de Banach $X$ au sens de la 
définition ci-dessus revient à donner: \\
--- un codage de classe~$\ca$   pour  une partie dénombrable $G$ de $X$ qui 
engendre $X$ en tant qu'espace de Banach{\footnote{On peut supposer que tous les éléments de $G$ sont des vecteurs de norme comprise entre 1/2 et 1.}},\\    
--- une application   $\nu \colon  \lst(\DD_{\KK} \times G) \times \NN_1 
\rightarrow
 \DD$  qui est dans la classe~$\ca$   (pour le codage de $G$ considéré)  et 
qui calcule la norme d'une combinaison linéaire d'éléments de  $G$  au sens suivant:\\   	
pour tout $\big([(x_1,g_1),(x_2,g_2),\ldots,(x_n,g_n)],m\big)$ dans  $\lst(\DD_{\KK} \times G) \times \NN_1$, on a la majoration
\[
\Abs{\;\nu\big([(x_1,g_1),(x_2,g_2),\ldots,(x_n,g_n)],m\big) - \Norme{ x_1.g_1 + x_2.g_2 +\cdots.+ x_n.g_n }_X \; }\; \leq 1/2^m.
\]
\end{fproposition}

\begin{fremarks} \label{f263}~\\
1)	Comme pour la définition \ref{f211}  (cf. remarque \ref{f212} (3)), il se 
peut que le certificat d'inclusion de $G$ dans $X$ implique une notion de 
\com, qu'il sera inévitable de prendre en compte dans une définition plus 
précise, au cas par cas.\\
2)	Les espaces  $L^p(\RR)$  de l'analyse fonctionnelle, avec  $1 \leq p < 
\infty$  peuvent être rationnellement présentés de différentes 
manières, selon différents choix possibles pour l'ensemble des points 
rationnels et un codage de cet ensemble. Tous les choix raisonnables 
s'avèrent donner des présentations  $\Prim$-équivalentes. \\
3)	Il serait intéressant de savoir si le cadre de travail proposé par M. 
Pour El et I. Richards \cite{fPR} concernant la calculabilité dans les espaces 
de Banach peut avoir des conséquences concrètes qui iraient au delà de ce 
qui peut être traité par les \rps  (lesquelles offrent un cadre naturel non seulement pour les problèmes de récursivité mais aussi pour les 
problèmes de \comz). Le ``contre-exemple'' concernant  $L^{\infty}$  donné 
dans \cite{fPR} incite à penser le contraire. 
\end{fremarks}

\section[Fonctions réelles continues sur un intervalle compact 
\ldots]{Espace des fonctions réelles continues sur un intervalle compact, 
premières propriétés}\label{fsec3}
Dans cette section, nous introduisons le problème des \rps  pour l'espace  
$\czu$:  l'espace des fonctions réelles \unicos sur l'intervalle  [0,1], muni 
de la norme usuelle:
\[
\norme{f}_{\infty} = {\bf Sup}\{ \abs{f(x)} ; 0 \leq x \leq 1 \}.
\] 
Considérons une \rp  de l'espace  $\czu$  donnée par une famille  
$\big(\wi{f}\,\big)_{f \in Y}$  de fonctions \unicos indexée par une partie  $Y$  
d'un langage $A^{\star}$. Nous sommes intéressés par les problèmes de 
\com suivants:\\
--- la \com de l'ensemble des codes des points rationnels de la 
présentation, c.-à-d. plus précisément la \com de  $Y$  en tant que 
partie du langage  $A^{\star}$   (c.-à-d. la \com du test d'appartenance);\\
--- la \com des opérations d'espace vectoriel (le produit par un scalaire 
d'une part, la somme d'une liste de vecteurs d'autre part);\\
--- la \com du calcul de la norme (ou de la fonction distance);\\ 
--- la \com de l'ensemble  $\big(\wi{f}\,\big)_{f \in Y}$  des points rationnels de 
la présentation, en tant que famille de fonctions \unicos sur  $[0,1]$;\\
--- la \com de la fonction d'évaluation  $\Ev \colon  \czu \times [0,1] 
\rightarrow \RR$: $ (g,x) \mapsto g(x)$.\\
Il va de soi que l'on peut remplacer l'intervalle  $[0,1]$  par un autre 
intervalle  $[a,b]$  avec  $a$  et  $b$  dans~$\DD$  ou de faible \com dans  
$\RR$. 

\subsection{La définition d'une \rp  de l'espace des fonctions 
continues}\label{fsubsec31} 
La \com de l'ensemble  $\big(\wi{f}\,\big)_{f \in Y}$  des points rationnels de la 
présentation, en tant que famille de fonctions \unicos sur  $[0,1]$    n'est 
rien d'autre que la \com de l'application  $f \mapsto \wi{f}$    de 
l'ensemble des codes de points rationnels $Y$ vers l'espace  $\czu$.  Nous 
devons donc, conformément à ce que nous disions dans la remarque 
\ref{f212}(3), inclure dans la définition de ce qu'est une \rp  de classe  
$\ca$   de  $\czu$, le fait que  $\big(\wi{f}\,\big)_{f \in Y}$  est une famille 
\uni de classe~$\ca$   au sens de la définition \ref{f229}.
Le problème de la \com de la fonction d'évaluation est également un 
problème important car il serait ``immoral''  que la fonction d'évaluation ne 
soit pas une fonction de classe~$\ca$   lorsqu'on a une \rp  de classe~$\ca$. 
Cependant, la fonction d'évaluation n'est pas \unico, ni même 
\loca \unico.\\
Pour traiter en général la question des fonctions continues mais 
non \loca \unicos sur l'espace  $\czu$  nous faisons appel à la définition 
informelle \ref{f241} avec la famille suivante de parties de  $\czu$:  

\begin{fnotation}  \label{f311}
Si  $\alpha$  est une fonction croissante de  $\NN_1$ vers  $\NN_1$  et  $r 
\in \NN_1$,  on note  $F_{\alpha,r}$  la partie de  $\czu$  formée par toutes 
les fonctions qui, d'une part acceptent  $\alpha$   comme \mcu, et d'autre part 
ont leur norme majorée par  $2^r$.
\end{fnotation}

La ``structure de calculabilité''  sur l'ensemble d'indices  
$$M := \{ (\alpha,r); \alpha \; \hbox{est une  fonction croissante de} \;
 \NN_1 \;\hbox{vers} \; \NN_1 \; \hbox{et} \; r \in \NN_1 \}$$
n'est pas une chose bien définie dans la littérature, mais nous n'aurons 
besoin de faire appel qu'à des opérations parfaitement élémentaires 
comme ``évaluer  $\alpha$  en un entier  $n$''{\footnote{Notez que le 
théorème d'Ascoli classique affirme que toute partie compacte de $\czu$ 
est contenue dans une partie $F_{\alpha,r}$ , et que les parties  $F_{\alpha,r}$   
sont compactes. En mathématiques constructives la partie directe est encore 
valable, mais la deuxième partie de l'énoncé doit être raffinée, cf 
\cite{fBB} chap 4 théorème 4.8, pages 96 à 98.}}.
Le module de continuité de la fonction d'évaluation est alors très 
simple (\uni linéaire pour toute définition raisonnable de cette notion).\\
En effet, près de la partie  $F_{\alpha, r} \times [0,1]$    de  $\czu \times 
[0,1]$   un module de continuité de la fonction $\Ev$ est donné par:
$$\mu(n,\alpha,r) = \max (\alpha(n+1),n+1) \; \hbox{pour} \; n \in \NN_1 \; 
\hbox{et} \; (\alpha,r) \in M
$$                             
comme il est très facile de le vérifier. Et la borne sur  $F_{\alpha,r}$  
est évidemment donnée par  $\beta(\alpha,r) = r$.\\
Toute la question de la \com de la fonction d'évaluation dans une \pres 
donnée est donc concentrée sur la question de la \com de la fonction 
d'évaluation restreinte à l'ensemble des points rationnels  
$$(f,x) \mapsto \wi{f}(x) \qquad  Y \times \DD_{[0,1]} \rightarrow \RR.$$
Or cette \com est subordonnée à celle de  
$\big(\wi{f}\,\big)_{f \in Y}$  en tant que famille de fonctions \unicosz: c'est 
ce que nous précisons dans la proposition suivante (dont la démonstration est immédiate).

\begin{fproposition} \label{f312}
Considérons sur l'espace  $\czu$  la famille de parties  
$(F_{\alpha,r})_{(\alpha,r) \in M}$  pour contrôler les questions de 
continuité sur  $\czu$  (cf. notation \ref{f311} et définition \ref{f241}).\\  
Alors si   $\big(\wi{f}\,\big)_{f \in Y}$  est une famille \uni de classe~$\ca$    
et si on considère la \rp  de l'espace métrique  $\czu$  attachée à 
cette famille considérée comme ensemble des points rationnels de la 
présentation, la fonction d'évaluation      
$$\Ev \colon  \czu \times [0,1] \rightarrow \RR \colon  (g,x) \mapsto g(x)$$
est elle même de classe~$\ca$.
\end{fproposition}

Comme en outre nous demandons que la structure d'espace de Banach soit elle-même de classe~$\ca$, cela nous donne finalement la définition suivante.

\begin{fdefinition} \label{f313}
Une \rp  de classe~$\ca$   de l'espace  $\czu$  est donnée par une famille de 
fonctions  $\big(\wi{f}\,\big)_{f \in Y}$  qui est une famille \uni de classe  
$\ca$, dense dans $\czu$  et telle que soient également dans la classe~$\ca$   
les calculs suivants:\\
--- le produit par un scalaire,\\  
--- la somme d'une liste de fonctions choisies parmi les points rationnels,\\
--- le calcul de la norme.
\end{fdefinition}

Dans toute la suite, nous nous intéresserons à une étude précise des 
complexités impliquées dans la définition \ref{f313}. Notre conclusion 
est qu'il n'existe pas de paradis \etpo des fonctions continues, du moins si  
$\p \not= \np$.  


\subsection{Deux exemples significatifs de \rps  de l'espace 
\texorpdfstring{$\czu$}{C[0,1]}}\label{fsubsec32}

Nous donnons maintenant deux exemples significatifs de présentations de  
$\czu$ (d'autres exemples seront donnés plus loin)   

\subsubsection{Présentation par circuits semilinéaires binaires} 
\label{fsubsubsec321}  
Cette \pres et l'ensemble des codes des points rationnels seront notés 
respectivement  $\csl$  et  $\ysl$.  Nous appellerons {\em  fonction 
semilinéaire à coefficients dans  $\DD$ }  une fonction linéaire par 
morceaux qui est égale à une combinaison par  $\max$  et  $\min$  de 
fonctions  $x \mapsto ax+b$  avec  $a$  et  $b$  dans  $\DD$.


\begin{fdefinition} \label{f321}
Un {\em  circuit semilinéaire binaire} est un circuit qui a pour portes 
d'entrée des variables réelles  $x_i $  (ici, une seule suffira parce que 
le circuit calcule une fonction d'une seule variable) et les deux constantes  0  
et  1. Il y a une seule porte de sortie.\\
Les portes qui ne sont pas des portes d'entrée sont de l'un des types 
suivants:\\
--- des portes à une entrée, des types suivants:  $x \mapsto 2x$ ,$x 
\mapsto x/2$, $x \mapsto -x$\\ 
--- des portes à deux entrées, des types suivants:  
$(x,y) \mapsto x + y$, $(x,y) \mapsto \max(x,y)$, $(x,y) \mapsto \min(x,y)$.\\ 
Un circuit semilinéaire binaire avec une seule variable d'entrée définit 
une fonction semilinéaire à coefficients dans  $\DD$. Un tel circuit peut 
être codé par un programme d'évaluation. L'ensemble  $\ysl$   est 
l'ensemble (des codes) de ces circuits semilinéaires: ses éléments codent les 
points rationnels de la \pres  $\csl$.
\end{fdefinition}

Nous verrons plus loin que cette \pres est en quelque sorte la plus naturelle, 
mais qu'elle manque d'être une \pres de classe  $\p$   à cause du calcul de 
la norme.\\
On a une majoration facile d'un module de Lipschitz de la fonction définie par le circuit:  
\[
\abs{\wi{f}(x) - \wi{f}(y)}\;  \leq 2^p \abs{ x- y} \; \hbox{où} \; p \;  \hbox{est la profondeur du circuit}
\]
Ceci donne pour \mcu  $\mu(k) = k+n$.  Ceci implique en particulier qu'on n'a 
pas besoin de contrôler la précision dans  $\DD_{[0,1]}$   dans la 
proposition suivante.  

\begin{fproposition}[complexité de la famille de fonctions  $\big(\wi{f}\,\big)_{f \in \ysl }$] \label{f322}
~ \\
La famille de fonctions  $\big(\wi{f}\,\big)_{f \in \ysl }$  est  \uni de classe  
$\p$. Précisément, cette famille admet  
$\mu(f,k) = k+\prof(f)$
  pour  \mcu et on peut expliciter une fonction  $\varphi \colon  \ysl  \times \DD_{[0,1]} 
\times \NN_1 \rightarrow \DD_{[0,1]}$   de classe $\DRT(\Lin, \Oo(N^2))$  (où $N$ est 
la taille de l'entrée  $(f,x,k)$)  avec
\[
 \forall (f, x, k) \in \ysl  \times 
\DD_{[0,1]} \times \NN_1 \;
\abs{\wi{f}(x) - \varphi(f,x,k)}\;  \leq 1/2^k.
\]
Plus précisément encore, comme il n'est pas nécessaire de lire $x$ en 
entier,  la taille de $x$ n'intervient pas, et la fonction  $\varphi$   est dans 
les classes  $\DRT(\Lin, \Oo(\ta(f)(\prof(f)+k)))$  et  $\DSPA(\Oo(\prof(f) 
(\prof(f)+k)))$.
\end{fproposition}

\proof 
Pour calculer  $\varphi (f,x,k)$  on évalue le circuit $f$ sur l'entrée $x$ 
dont on ne considère que les  $k + 2 \prof(f)$  premiers bits, en tronquant le 
résultat intermédiaire calculé à la porte  $\pi$ à la précision  
$k+2\prof(f)-\prof(\pi)$.  Enfin, pour le résultat final on ne garde que la 
précision  $k$.\\
Une telle méthode appliquée naïvement nécessite de garder stockés 
tous les résultats obtenus à une profondeur fixée  $p$  pendant qu'on 
calcule des résultats à la profondeur  $p+1$.
Nous faisons alors  $\ta(f)$  calculs élémentaires  $(\bullet + \bullet, 
\bullet - \bullet,\bullet \times 2, \bullet/2, \max(\bullet, \bullet),
\min(\bullet, \bullet))$  
sur des objets de taille $\leq  k + 2 \prof(f)$.  Chaque calcul élémentaire 
prend un temps $\Oo(k+\prof(f))$, et donc le calcul global se fait en temps  
$\Oo(\ta(f)(\prof(f)+k))$.  Et cela prend aussi un $\Oo(\ta(f)(\prof(f)+k))$ comme 
espace de calcul.\\  
Il existe une autre méthode d'évaluation d'un circuit, un peu moins 
économe en temps mais nettement plus économe en espace, suivant l'idée de 
Borodin \cite{fBo}. Avec une telle méthode on économise l'espace de calcul 
qui devient un  $\Oo(\prof(f)(\prof(f)+k))$.
\eop

\begin{fremark} \label{f323}
Nous n'avons pas pris en compte dans notre calcul le problème posé par la 
gestion de  $t = \ta(f)$  objets (ici des nombres dyadiques) de tailles 
majorées par  $s = \prof(f)+k$. Dans le modèle RAM cette gestion serait a 
priori en temps   $\Oo(t \Lg(t) s)$  ce qui n'augmente pas 
sensiblement le  $\Oo(t s)$  que nous avons trouvé, et ce qui reste en  
$\Oo(N^2)$  si on se rappelle que le codage du circuit semilinéaire par un 
programme d'évaluation lui donne une taille de $\Oo(t \Lg(t))$. Dans le 
modèle des machines de Turing par contre, cette gestion réclame a priori un 
temps  $\Oo(t^2s)$  car il faut parcourir~$t$ fois la bande où sont stockés 
les objets sur une longueur totale  $\leq  ts$. Nous avons donc commis une 
certaine sous-estimation en nous concentrant sur le problème que nous 
considérons comme central: estimer le coût total des 
opérations arithmétiques proprement dites. Nous omettrons dans la suite 
systématiquement le calcul du {\em temps de gestion} des valeurs 
intermédiaires (très sensible au modèle de calcul choisi) chaque fois 
qu'il s'agira d'évaluer des circuits.
\end{fremark}

\subsubsection{Présentation \texorpdfstring{$\crf$}{Cfrac}
(via des fractions rationnelles contrôlées et données en \pres par formule)} \label{fsubsubsec322}
La \pres précédente de l'espace  $\czu$  n'est pas de 
classe  $\p$  (sauf si $\p$ = $\np$ comme nous le verrons à la section 4.3)  parce que la norme n'est pas calculable en temps \poll.  
Pour obtenir une \pres de classe $\p$ il est nécessaire de restreindre assez considérablement l'ensemble des points rationnels de la présentation, de manière à ce que la norme devienne une fonction calculable en temps \poll. 
Un exemple significatif est celui où les points rationnels sont des fractions rationnelles bien contrôlées et données dans une \pres du 
type dense.
On a le choix entre plusieurs variantes et nous avons choisi de donner le 
dénominateur et le numérateur dans une \pres dite ``par formules''.  Une 
formule est un arbre dont les feuilles sont étiquetées par la variable $X$ 
ou par un élément de  $\DD$  et dont chaque noeud est étiqueté par un 
opérateur arithmétique.  
Dans les formules que nous considérons, les seuls opérateurs utilisés  sont $\bullet + \bullet, \bullet - \bullet$ et $\bullet \times  \bullet,$ de sorte que l'arbre est un arbre binaire (chaque noeud de l'arbre est une sous-formule et représente un \pol de $\DD$[X].) 



\begin{flemma} 
\label{f324} Désignons par $\DD[X]_f$  l'ensemble des \pols à 
coefficients dans $\DD$, donnés en \pres par formule.  Pour un \pol à 
une variable et à coefficients dans  $\DD$   le passage de la re\pres dense 
à la re\pres par formule est  $\LINT $  et le passage de la re\pres par 
formule à la re\pres dense est \poll. Plus précisément si on 
procède de manière naïve on est en  $\DTI (\Oo(N^2{\cal M}(N)))$).
\end{flemma}

\proof  Tout d'abord, la re\pres dense peut être considérée comme un cas 
particulier de re\pres par formule, selon le schéma de Horner.\\ 
Ensuite, passer de la re\pres par formule à la re\pres dense revient à 
évaluer la formule dans  $\DD[X]$. Introduisons les paramètres de controle 
suivants. Un \pol $P \in \DD[X]$   a un degré noté  $d_P$  et la taille 
de ses coefficients est controlée par l'entier $\sigma(P) := \log(\sum_i \abs{a_i})$ où les  $a_i$  sont les coefficients de  $P$.  Une formule  $F 
\in \DD[X]_f$  contient un nombre d'opérateurs arithmétiques noté  $t_F$ 
et la taille de ses coefficients est controlée par l'entier $\lambda(F) = 
\sum_i (\Lg(b_i))$   où les  $b_i$ sont les dyadiques apparaissant dans la 
formule.\\  
La taille  $\flo{F}$ de la formule $F$ est évidemment un majorant de   
$t_F$ et  $\lambda (F)$.\\
Pour deux dyadiques  $a$  et  $b$  on a toujours  $\Lg(a \pm b) \leq \Lg(a) + 
\Lg(b)$  et $ \Lg(ab) \leq  \Lg(a)+\Lg(b)$.\\
On vérifie alors facilement que $\sigma (P \pm Q) \leq \sigma(P) + \sigma(Q)$   
et  $\sigma (PQ) \leq \sigma (P) + \sigma (Q)$.  Le temps de calcul (naïf) 
de  $PQ$  est un  $\Oo(d_Pd_Q {\cal M}(\sigma(P) + \sigma(Q)))$.\\
On démontre ensuite par récurrence sur la taille de la formule $F$ que le 
\pol correspondant  $P \in \DD[X]$  vérifie $d_P \leq t_F$ et  $\sigma 
(P) \leq \lambda (F)$. On démontre également par récurrence que le temps pour 
calculer  $P$  à partir de $F$ est majoré par  $t_F^2{ \cal M}(\lambda(F))$. 
\eop

\begin{fdefinition} \label{f325}
L'ensemble  $\yrf  \subset \DD[X]_f \times \DD[X]_f$  est l'ensemble des 
fractions rationnelles (à une variable) à coefficients dans  $\DD$,  dont 
le dénominateur est minoré par  $1$  sur l'intervalle $[0,1]$.  L'espace  
$\czu$  muni de l'ensemble  $\yrf$  comme famille des codes des points 
rationnels est noté~$\crf$.
\end{fdefinition} 

\begin {fproposition}[complexité de la famille de fonctions  $\big(\wi{f}\,\big)_{f \in \yrf}$] \label{f326}
~ \\
La famille de fonctions  $\big(\wi{f}\,\big)_{f \in \yrf }$   est  \uni de classe  
$\p$, plus précisément de classe  
$\DRT(\Lin, \Oo({\M}(N)N))$, où $\M(N)$ est la \com de la multiplication de deux entiers de taille~$N$.
\end{fproposition}

\proof Nous devons calculer un \mcu pour la famille $\big(\wi{f}\,\big)_{f \in \yrf}$.  Nous devons aussi expliciter une fonction $\varphi$ de classe  $\DRT(\Lin,\Oo(\M(N) N))$
\[
\varphi \colon  \yrf  \times \DD_{[0,1]} \times \NN_1 \rightarrow \DD \; ,\;
(f,x,n) \mapsto \varphi(f,x,n)
\]
vérifiant   $\abs{\wi{f}(x) - \varphi(f,x,n)}\;  \leq 1/2^n$.\\
En fait, le \mcu va se déduire du calcul de  $\varphi $.\\ 
Puisque le dénominateur de la fraction est minoré par  $1$, il nous suffit 
de donner un \mcu et une procédure de calcul en temps  $\Oo(\M(N) N)$ pour 
évaluer une formule  $F \in \DD[X]_f$  avec une précision  $1/2^n$  sur 
l'intervalle $[0,1]$   ($N = n+ \flo{F}$).  Nous supposons sans perte de 
généralité que la taille  $m = \flo{F}$  de la formule  $F = F_1 * 
F_2$  est égale à $ m_1+m_2+2$  si  $m_1 = \flo{F_1}$  et  $m_2 = 
\flo{F_2}$   ( $*$  désigne un des opérateurs $+, \; - \;$ ou $\;\times$).  
On établit alors (par récurrence sur la profondeur de la formule) les deux 
faits suivants:\\ 
--- lorsqu'on évalue de manière exacte la formule  $F \in \DD[X]_f$  en un  
$x \in [0,1]$    le résultat est toujours majoré en valeur absolue par  
$2^m$. \\
--- lorsqu'on évalue la formule $F$ en un  $x \in [0,1]$    de manière 
approchée, en prenant $x$ et  tous les résultats intermédiaires avec une 
précision (absolue)  $1/2^{n+m}$  le résultat final est garanti avec la 
précision  $1/2^n$. \\
On conclut ensuite sans difficulté.
 \eop
 
\begin{fproposition} \label{f327}~\\
a) La famille de nombres réels  
$(\Norme{ \wi{f}})_{f \in \yrf }$  est de \com  $\p$.\\
b) Le test d'appartenance  ``$\,f \in\yrf\; ?\,$''  est de \com  $\p$.
\end{fproposition}
\proof 
a) Soit  $f = P/Q \in \yrf $.  Pour calculer une valeur approchée à  $1/2^n$  
de la norme de  $\wi{f}$, on procède comme suit:\\
--- calculer  $(P'Q - Q'P)$  en tant qu'élément de  $\DD[X]$  (lemme 
\ref{f324})\\
--- calculer  $m$: la précision requise sur $x$ pour pouvoir évaluer  
$\wi{f} (x)$   avec la précision  $1/2^n$  (proposition \ref{f326})\\
--- calculer les racines   $(\alpha_i)_{1 \leq i \leq n})$  de  $(P'Q - Q'P)$   
sur  $[0,1]$    avec la précision  $2^{-m}$ \\
--- calculer  $\max  \{\wi{f}(0);\wi{f}(1);\wi{f}(\alpha_i)
 \; 1 \leq i \leq s \}$ avec la précision  $1/2^n$ (proposition \ref{f326})\\
b) Le code $f$ contient les codes de $P$ et $Q$. Il s'agit de voir qu'on peut tester en temps \poll que le dénominateur $Q$ est minoré par $1$ sur 
l'intervalle $[0,1]$. 
Ceci est un résultat classique concernant les calculs avec les nombres réels algébriques: il s'agit de comparer à $1$ le inf des $Q(\alpha_i)$ avec $\alpha_i=0,1$ ou un zéro de $Q'$ sur l'intervalle. 
\eop

\begin{fremarks} \label{f328}~\\
1)	Il est bien connu que le calcul des racines réelles d'un \pol de  
$\ZZ[X]$  situées dans un intervalle rationnel donné est un calcul de 
classe  $\p$. Peut-être la méthode la plus performante n'est pas celle que 
nous avons indiquée, mais une légère variante. 
En effet la recherche des 
racines complexes d'un \pol (pour une précision donnée) est aujourd'hui 
extrêmement rapide (cf. \cite {fPa}). 
Plutôt que de chercher spécifiquement les zéros réels sur l'intervalle $[0,1]$ on pourrait donc chercher avec la précision  $2^{-m}$  les zéros réels ou complexes  $(\beta_j)_{1 \leq j \leq t}$ suffisamment proches de l'intervalle   $[0,1]$    (i.e. leur partie imaginaire est en valeur absolue  $\leq2^{-m}$ et leur partie réelle est sur  $[0,1]$  à   $2^{-m}$  près) et évaluer $\wi{f}$ en les  $Re(\beta_j)$.\\    
2) Comme cela résulte de la proposition \ref{f335} ci-dessous,  toute 
fonction semilinéaire à coefficients dans $\DD$  est un point de \com  
$\p$ dans $\crf$. 
Le fait que la \pres $\csl$ ne soit pas de classe  $\p$ (si $\p\neq\np$, cf. section \ref{fsubsec43}) implique par contre que la famille de fonctions $\big(\wi{f}\,\big)_{f \in \ysl }$ n'est pas une famille de classe $\p$   dans $\crf$.\\
3) On démontre facilement que les opérations d'espace vectoriel sont aussi en temps \poll. 
\end{fremarks}

Les résultats précédents se résument comme suit.

\begin{ftheorem} \label{f329}
La \pres  $\crf$  de  $\czu$  est de classe  $\p$.
\end{ftheorem}

\subsection{Le théorème d'approximation de Newman et sa \com 
algorithmique}\label{fsubsec33}
Le théorème de Newman est un théorème fondamental en théorie de 
l'approximation. L'énoncé ci-après est un cas particulier{\footnote{Nous 
avons pris la borne  $n^2/2$ sur les degrés de manière à obtenir une 
majoration en  $e^{-n}.$}}.

\begin{ftheorem}[Théorème de Newman, \cite{fNe}, voir par exemple \cite{fPP} p. 73--75] \label{f331}~  

\noindent   
Soit  $n$  entier  $\geq  6$,   définissons
\[
H_n(x) = \prod_{ 1 \leq k < n^2}(x + e^{-k/n}).
\]
et considérons les deux \pols  $P_n(x)$  et 
  $Q_n(x)$, de degrés majorés par  $n^2/2$,  donnés par
\[
P_n(x^2) = x(H_n(x) - H_n(-x)) \; \; \hbox{et} \; \; Q_n(x^2) = H_n(x) + H_n(-x)
\]
Alors on a pour tout  $x \in  [-1,1]$  
\[
\abS{ \; \abs{x}\; - \;(P_n/Q_n)(x^2) \; } \; \leq 3e^{-n}\; \leq 2^{-(n+1)}
\]
et 
\[\abS{Q_n(x^2)}\; \geq \;2H_n(0) = 2/e^{(n^3 - n)/2} \;\geq 1/2^{3(n^3 - 
n)/4} 
\]
\end{ftheorem}

Du théorème de Newman découle que les fonctions semilinéaires se 
laissent individuellement bien approcher par des fractions rationnelles 
faciles à écrire et bien contrôlées. C'est ce que précisent le 
lemme \ref{f333}  ci-après et ses corollaires, les propositions et 
théorèmes qui suivent. Nous rappelons d'abord le résultat suivant (cf. 
\cite{fBrent}).
\begin{flemma}[théorème de Brent] \label{f332}  
 Soit   $a \in  [-1,1]\cap \DD_m$. Le calcul de  
$\exp(a)$  avec la précision~$2^{-m}$ peut être fait en temps $\Oo(\M(m)\log(m))$.   
En d'autres termes, la fonction exponentielle sur l'intervalle $[-1,1]$ est de \com $\DTI(\Oo(\M(m)\log(m)))$   
\end{flemma} 

On en déduit facilement, en notant $\DD[X]_f$ la présentation de $\DD[X]$ 
par formules.

\begin{flemma} \label{f333}
Il existe une suite 
\[
\NN_1 \rightarrow \DD[X]_f \times \DD[X]_f \; ; \; n \mapsto (u_n, v_n),
\] 
de classe $\DRT(\Oo(n^5), \Oo({\M}(n^3) n^2 \log^3(n)))$, telle que pour tout   $x \in [0,1]$      
\[
\abs{\;\abs{x} - p_n(x^2)/q_n(x^2) \;} \;\;\leq \;2^{-n}.
\]
Les degrés des \pols $p_n$ et $q_n$ sont majorés par $n^2/2$, leurs tailles en présentation par formule sont majorées par un $\Oo(n^5)$,  
et $q_n(x^2)$ est minoré par  $1/e^{(n^3-n)}$.
\end{flemma} 

\proof On définit $p_n$ et  $q_n$  comme   $P_n$  et  $Q_n$  en rempla\c{c}ant 
dans la définition le réel  $e^{-k/n}$ par une approximation  dyadique  
$c_{n,k}$  suffisante calculée au moyen du lemme \ref{f332}. \\ 
Si  $\abs{e^{-k/n}-c_{n,k}}\;\leq \varepsilon$ on vérifie que 
pour tout   $x\in [0,1]$  
\[
\abs{P_n(x) - p_n(x)}\;  \leq  (n^{2}-1) 2^{n^2} \varepsilon  \quad  
\hbox{et}  \quad  \abs{Q_n(x)- q_n(x)}\; \leq  (n^2-1) 2^{n^2} \varepsilon  = \varepsilon_1. 
\]
Pour l'écart entre les fractions rationnelles on utilise
\[
\abs{A/B - a/b};  \leq \;  
\abs{A/B} \;  \abs{b - B}  / b + \; \abs{A - a}  / b 
\leq   3 \varepsilon_1/ b. 
\]
Comme $ B \geq  1/2^{3(n^3-n)/4} $ on a  $1/b\leq2.2^{3(n^3-n)/4}$ si  
$\abs{b - B}\;\leq  B/2$, en particulier si  
\[
(n^{2}-1) 2^{n^2} \varepsilon \leq (1/2) \cdot 1/2^{3(n^3-n)/4}.
\]
On est alors conduit à prendre un  $\varepsilon $  tel que   
\[
3\varepsilon_1/b\leq 6(n^{2}-1)2^{n^2}2^{3(n^3-n)/4}\varepsilon 
\leq 1/2^{n+1}.
\]
Il suffit donc de prendre  $\varepsilon  \leq  2 ^{-n^3}$ (pour  $n$  assez 
grand).\\
On va donc être amené à décrire   $p_n$  et  $q_n$  par des formules de 
taille  algébrique   $\Oo(n^2)$   portant sur des termes de base  $(x+c_{n,k})$  
où  $c_{n,k}$  est un dyadique de taille  $\Oo(n^{3}).$ 
La taille (booléenne) de la formule est donc un  $\Oo(n^{5}).$ 
L'essentiel du temps de calcul est absorbé par le calcul des  $c_{n,k}$  en 
utilisant le lemme \ref{f332}.  
\eop

Notez que la majoration du temps de calcul est à peine moins bonne que la 
taille du résultat.

On en déduit immédiatement les résultats suivants.
\begin{ftheorem} \label{f334}
La fonction   $x \mapsto \abs{ x - 1/2 }$ est un point de \com $\p$   (plus 
précisément  $\DRT(\Oo(n^5), \Oo({\M}(n^3) n^2 \log^3(n)))$   
dans l'espace  $\crf$. 
En d'autres termes, il existe une suite  
$\NN_1 \rightarrow \yrf  \; , \; 
n \mapsto (u_n,v_n)$, de classe
 $\DRT(\Oo(n^5), \Oo({\M}(n^3) n^2 \log^3(n)))$, telle que         
\[
\Norme{ \; \abs{ x-1/2 } - u_n(x)/v_n(x)\; } \; \leq 2^{-n}
\]
Les degrés des \pols  $u_n$  et  $v_n$ sont majorés par   $n^2$.
\end{ftheorem}
\begin{fproposition} \label{f335}
La fonction  $x \mapsto \abs{x}$ sur l'intervalle  $[-2^m , 2^m]$  peut 
être approchée à  $1/2^n$   près  par une fraction rationnelle  
$p_{n,m}/q_{n,m}$  dont le dénominateur est minoré par  $1$  (sur le même 
intervalle),  et le calcul  
$$(n,m) \mapsto (p_{n,m},q_{n,m}) \; \; \NN_1 \times \NN_1 \rightarrow
 \DD[X]_f \times \DD[X]_f $$
est de \com  $\DRT(\Oo(N^5), \Oo({\M}(N^3) N^2 \log^3(N)))$  
où  $N = n+m$.  Les degrés des \pols  
$p_{n,m}$  et  $q_{n,m}$ sont majorés par   $N^2$.
De même la fonction $(x,y) \mapsto \max(x,y)$ 
(resp. $(x,y) \mapsto \min(x,y)$)  sur le carré  
$[-2^m,2^m] \times  [-2^m,2^m]$  peut être approchée à  $1/2^n$   
près  par une fraction rationnelle du même type et de même \com 
que les précédentes.
\end{fproposition}

\proof 	Pour la fonction valeur absolue:\\
Si   $x \in [-2^m , 2^m]$  on écrit   
\[
\abs{x}\;  = 2^m \abs{ x/2^m } 
\]
avec  $x/2^m \in [-1 , 1]$. Donc, avec les notations de la preuve du lemme 
\ref{f333}, il suffit de prendre    $p_{n,m}(x) = 2^m p_{n+m}(x/2^m)$  et   
$q_{n,m}(x) = 2^m q_{n+m}(x/2^m)$.\\
Pour les fonctions $\max$ et $\min$, il suffit d'utiliser les formules:	
\[
\max(x,y) = \frac{x+y\; + \abs{x-y}} {2}\quad  \hbox{et} \quad  
\min (x,y) = \frac{x+y\; - \abs{x-y}}{2} 
\]
\eop

\medskip Dans la suite, on aura besoin d'``approcher'' la fonction discontinue
\[  
C_a(x) = \left\{
\begin{array}{cl}
1, & \hbox{si} \; x \geq a \\
0, & \hbox{sinon} 
\end{array}
\right.
\] 
Une telle ``approximation''  est donnée par la fonction semilinéaire continue  
$ C_{p,a}$:
\[
C_{p,a} = \min(1, \max(0, 2^p(x - a))) \; \hbox{où} \; p \in \NN_1 
\; \hbox{et} \; a \in \DD_{[0,1]}
\] 


\begin{figure}[htbp]  
\begin{center}
\includegraphics*[width=14cm]{fi336}
\end{center}
\caption[Courbe représentative de la fonction $C_{p,a}$]{\label{ffi336}  
la fonction $C_{p,a}$}  
\end{figure}  

\noindent    
La \com de la famille de fonctions  $(p,a) \mapsto C_{p,a}$  est donnée dans la 
proposition suivante.

\begin{fproposition} \label{f336}
La famille de fonctions   
\[
\NN_1 \times \DD_{[0,1]} \; \; (p,a) \mapsto C_{p,a}
\]
définie par:
$
 C_{p,a} = \min(1,\max(0, 2^p(x - a)))
$  
est une famille de classe $\p$. 
Plus précisément, elle est de \com  
$\DRT(\Oo(N^5), \Oo({\M}(N^3) N^2 \log^3(N)))$   où  $N = \max(\Lg(a), n+p)$.
\end{fproposition}

La proposition suivante concerne la fonction racine carrée.

\begin{fproposition} \label{f337}
La fonction  $ x \mapsto \sqrt {\abs { x - 1/2 }} $  sur $[0,1]$   est un  
$\p$-point de  $\crf$.
\end{fproposition}
\begin{figure}[htbp]  
\begin{center}
\includegraphics*[width=10cm]{fi339}
\end{center}
\caption[Courbe de la fonction $\sqrt {\abs{ x - 1/2 }} $]{\label{ffi339}  
courbe de la fonction $ x \mapsto \sqrt {\abs { x - 1/2}} $}  
\end{figure}  

\proof (esquisse) 
Dans les \pols  $P_n(x^2)$ et $Q_n(x^2)$  du théorème de Newman, si on 
remplace  $x^2$  par une bonne approximation de $\abs{x}$ sur  $[0,1]$  
obtenue elle même par le théorème de Newman, on obtiendra une fraction 
rationnelle  $R_n(x)/S_n(x)$  telle que 
\[
\forall x \in [0,1] \abs{\sqrt{ \abs{x}} - (R_n/S_n)(x) }\;  \leq 2^{-
n}.
\]
Les degrés de  $R_n(x)$  et  $S_n(x)$  seront en  $\Oo(n^4)$. \\  
Notez que la fraction  $(P_n/Q_n)(x)$  n'est pas impaire, elle fournit donc une 
bonne approximation de la fonction  $\sqrt{ \abs{x}}$  uniquement sur 
l'intervalle $[0,1]$.
\eop

\section[Une \pres naturelle de l'espace \texorpdfstring{$\czu$}{C[0,1]}]{Une \pres naturelle de l'espace \texorpdfstring{$\czu$}{C[0,1]} 
et quelques présentations 
équivalentes}\label{fsec4}

Une notion naturelle de \com pour les ``points'' et pour les ``suites de points'' 
de l'espace   $\czu$  est donnée par  K. I Ko et H. Friedman dans \cite{fKF82}. 
Nous étudions dans cette section une \rp  de l'espace  $\czu$  qui donne (à très peu près) les mêmes notions de \com. Nous étudions également 
d'autres présentations utilisant des circuits, qui s'avèrent \equivas du 
point de vue de la \com  $\p$. La preuve de ces équivalences est basée sur 
la preuve d'un résultat analogue mais moins général donnée par 
\cite{fHo87,fHo90}. 

\subsection {Définitions de quelques présentations de l'espace 
 \texorpdfstring{$\czu$}{C[0,1]}}\label{fsubsec41}
 
\subsubsection{Presentation KF (KF comme Ko-Friedman) }\label{fsubsubsec411}
Rappelons qu'on note  $\DD_n = \{ k/2^n ; k \in \ZZ \}$  et  $\DD_{n,[0,1]} = 
\DD_n \cap [0,1]$.   \\ 
Dans la définition donnée par Ko et Friedman, on considère, pour  
$f \in \czu$, une machine de Turing à Oracle  (MTO)  qui ``calcule'' la 
fonction $f$ au sens suivant. 
 L'oracle délivre, pour la question  ``$\,m\;?\,$'' écrite en unaire  une approximation 
  $\xi \in\DD_{m,[0,1]}$ de $x$ avec la précision  $2^{-m}$. 
Pour l'entrée  $n \in  \NN_1$, la machine calcule, en utilisant l'oracle, une 
approximation $ \zeta  \in  \DD_n$  de  $f(x)$  à $1/2^n$  près.\\
La seule lecture du programme de la  MTO  ne permet évidemment pas de savoir si la  MTO  calcule bien une fonction de  $\czu$.  En conséquence, puisque nous souhaitons avoir des objets clairement identifiés comme ``points rationnels'' de la \pres que nous voulons définir, nous prenons le parti d'introduire des paramètres de contrôle. 
Mais pour que de tels paramètres soient vraiment efficaces, il est nécessaire de limiter l'exécution de la machine à une précision de sortie donnée a priori.  
En conséquence la précision nécessaire en entrée est également 
limitée a priori. Ainsi, la  MTO  est approximée par une suite de machines 
de Turing ordinaires (sans oracle) n'exécutant chacune qu'un nombre fini de calculs.
Nous sommes donc conduits à définir une \rp  de  $\czu$  qui sera notée  
$\ckf$ (l'ensemble des codes des points rationnels  sera noté  $\ykf$)  de la 
manière suivante. 
 
\begin{fdefinition}\label{f411}
On considère un langage, choisi une fois pour toutes, pour décrire les 
programmes de machines de Turing avec une seule entrée, dans $\DD_{[0,1]}$  
(notée  $x$), et avec une seule sortie, dans $\DD$  (notée  $y$). Soit  
${\bf Prog}$  la partie de ce langage formée par les programmes bien écrits (conformément à une syntaxe précisée une fois pour toutes).
Soit  $f = (Pr, n, m, T) \in {\bf Prog} \times \NN_1 \times \NN_1 
\times \NN_1$  où  on a:\\
--- $n$  est la précision réclamée pour  $y$,\\
--- $m$  est la précision avec laquelle est donnée  $x$.\\
Le quadruplet  $(Pr,n,m,T)$  est dit {\em correct} lorsque sont vérifiées 
les conditions suivantes:\\
--- le programme  $Pr$  calcule une fonction de $\DD_{m,[0,1]}$  vers  
$\DD_n$,  c.-à-d. pour une entrée dans  $\DD_{m,[0,1]}$   on obtient en 
sortie un élément de  $\DD_n$;\\
--- $T$  majore le temps d'exécution maximum pour tous les $x$ de  
$\DD_{m,[0,1]}$; \\
--- pour deux éléments consécutifs (distants de  $1/2^m$) de  
$\DD_{m,[0,1]}$  en entrée, le programme donne en sortie deux éléments de  
$\DD_n$  distants d'au plus  $1/2^n$.\\
L'ensemble des quadruplets  $(Pr,n,m,T)$  corrects est noté  $\ykf$\label{fykf}.
  Lorsque la donnée $f$ est correcte, elle définit le point rationnel  
$\wi{f}$  suivant:  c'est la fonction linéaire par morceaux qui
 joint les points du graphe donnés sur la grille  
$\DD_{m,[0,1]} \times  \DD_n$  par l'exécution du programme $Pr$  pour 
toutes les entrées 
possibles dans  $\DD_{m,[0,1]}$.\\
Ainsi $\ckf=(\ykf,\eta,\delta)$ où  \smash{$\eta(f)=\wi f$} et $\delta$ doit être précisé conformément à la définition~\ref{f211}. La complexité de $\eta$ et $\delta$ est traitée dans la proposition \ref{f413}. 
\end{fdefinition} 

\begin{fremark}\label{f412}
 On notera que les deux premières conditions pourraient être réalisées 
de manière automatique par des contraintes de type syntaxique faciles à 
mettre en oeuvre. Par contre, comme on le verra dans la proposition \ref{f441}, 
la troisième est incontrôlable \etpo (sauf si  $\p$ = $\np$ ) (cette 
condition de correction définit un problème $\cnp$-complet). 
Notez aussi que si on n'avait pas imposé la condition de correction pour les points de  $\ykf$  
on n'aurait eu aucun contrôle a priori du \mcu pour la fonction linéaire par 
morceaux définie par une donnée, et la famille  $\big(\wi{f}\,\big)_{f \in \ykf 
}$  n'aurait pas été \uni de classe  $\p$.
\end{fremark}

\begin{fproposition}[complexité de la famille de fonctions attachées à  $\ykf$] \label{f413}~\\
La famille de fonctions continues  $\big(\wi{f}\,\big)_{f \in \ykf}$  
est \uni de classe  $\DSRT(\Lin, \Lin, \Oo(N^2))$. 
\end{fproposition}

\proof 
Tout d'abord, on remarque que la fonction  $\wi{f}$  correspondant à  
$f = (Pr, n, m, T)$ est linéaire par morceaux et que, puisque la donnée est 
correcte, la pente de chaque morceau est majorée en valeur absolue par  
$2^{m-n}$  ce qui donne le \mcu  $\mu(f,k) = k+m-n$  pour la famille  
$\big(\wi{f}\,\big)_{f \in \ykf }$. \\
Nous devons exhiber une fonction  
$\varphi \colon  \ykf  \times \DD_{[0,1]} \times \NN_1 \rightarrow \DD$ de \com  
$\DSRT(\Lin, \Lin, N^2)$  telle que:
\[
\forall (f,x,k) \in \ykf  \times \DD_{[0,1]} \times \NN_1 \; \; 
\abs{ \varphi(f,x,k) - \wi{f}(x) }\;  \leq 2^{-k}.
\]
Soit  $z=(f,x,k)=((Pr,n,m,T),x,k) \in \ykf  \times \DD_{[0,1]} \times \NN_1$.
  Nous supposerons sans perte de généralité que  $k \geq  n$  et que $x$ 
est donné avec au moins $ m $ bits.\\
Par simple lecture de $x$ on repère deux éléments consécutifs  $a$  et  
$b$  de $\DD_{m,[0,1]}$   tels que   $a \leq  x \leq  b$  et on trouve le 
dyadique  $r \in \DD_{[0,1]}$  tel que  $x=a+r/2^n$. \\
Notons  $\varepsilon \in \{-1,0,1\}$   l'entier vérifiant   
$\wi{f}(a)=\wi{f}(b)+ \varepsilon/2^n$. \\  
Alors   $\wi{f}(x) = \wi{f}(a) + \varepsilon r/2^n$.  En outre  
$\wi{f}(a)= \Exec (Pr,a)$  est le résultat de l'exécution du 
programme  $Pr$   pour l'entrée  $a$.
Le calcul complet consiste donc essentiellement à \\ 
--- lire $x$  (et en déduire  $a$,  $b$  et  $r$), puis\\
--- calculer  $\Exec (Pr,a)$  et   $\Exec (Pr,b)$.\\
La \com est donc majorée par la \com de la Machine de Turing Universelle que nous utilisons pour exécuter le programme  $Pr$. D'après le lemme \ref{f131} cela se fait en temps $\Oo(T(T+ \flo{Pr}))$ et en espace $\Oo(T+ \flo{Pr})$. Et la taille de la sortie est majorée par $T$.
\eop

\medskip Le fait que la \pres  $\ckf$  qui résulte de la définition \ref{f411} soit 
appelée ``\pres à la Ko-Friedman'' est justifié par la proposition 
suivante.

\begin{fproposition} \label{f414}
Une fonction   $f \colon  [0,1] \rightarrow \RR$ est calculable \etpo  au sens de Ko-Friedman si et seulement si elle est un  $\p$-point de $\ckf$. Plus précisément 
\begin{itemize}
%
\item [a)] Si la fonction $f$ est de \com en temps  $T(n)$  au sens de Ko-Friedman alors  
elle est un  $\DTI (T)$-point de $\ckf$,
%
\item [b)]	Si la fonction $f$ est un  $\DTI (T)$-point de $\ckf$ alors elle est de \com en temps $T^2(n)$ au sens de Ko-Friedman. 
\end{itemize}
\end{fproposition}
Nous prouverons d'abord une caractérisation de la \com d'une fonction au sens de Ko-Friedman, pour une classe~$\ca$   de \com en temps ou en espace élémentairement stable. Nous l'avons déjà affirmée sans preuve dans le premier exemple \ref{f223}. Nous en donnons un énoncé plus précis ici.

\begin{fproposition} \label{f415}
Soit~$\ca$    une classe élémentairement stable de type $\DTI (\bullet)$  
ou  $\DSPA(\bullet)$  ou $\DSRT(\bullet, \bullet, \bullet)$  ou  $\DRT(\bullet, 
\bullet)$  ou  $\DSR(\bullet, \bullet)$  et soit une fonction continue  
$f \colon  [0,1] \rightarrow \RR$. Les propriétés suivantes sont équivalentes. 
%
\begin{enumerate}
%
\item la fonction $f$ est calculable au sens de Ko-Friedman dans la classe~$\ca$.
%
\item la fonction $f$ est \uni de classe~$\ca$.
%
\end{enumerate}
{\bf NB}: 
Pour ce qui concerne la \com d'une MTO, nous entendons ici que les questions 
posées à l'oracle doivent être comptées parmi les sorties de la machine. 
Autremant dit la taille des entiers~$m$ qui sont les questions à l'oracle 
doivent avoir la majoration requise pour la sortie dans les classes du type  $\DSRT(\bullet,\bullet,\bullet)$ ou $\DRT(\bullet,\bullet,\bullet)$  ou  $\DSR(\bullet,\bullet,\bullet)$.
\end{fproposition}

\proof (de la proposition \ref{f415})
Rappelons qu'une fonction est \uni de classe~$\ca$   (cf. définition 
\ref{f211}) lorsque:\\
a)	la fonction $f$ possède un \mcu dans la classe~$\ca$.\\
b)	la famille   $(f(a))_{a \in \DD}$   est une~$\ca$-famille de nombres 
réels.\\
Soit tout d'abord  $f \in \czu$ et   $M$  une  MTO  qui, pour toute entrée  
$n \in \NN_1$ et tout  $x \in [0,1]$   (donné comme oracle) calcule  $f(x)$  
à  $2^{-n}$   près dans la classe~$\ca$.  Si on remplace l'oracle, qui 
donne à la demande une approximation de $x$ à   $2^{-m}$  près dans  
$\DD_{m}$,  par la lecture d'un nombre dyadique  $a$  arbitraire, on obtient 
une machine de Turing usuelle qui calcule  $(f(a))_{a \in \DD}$   en tant que 
famille de nombres réels, ceci dans la classe~$\ca$. \\
Voyons la question du \mcu. Sur l'entrée  $n$  (précision requise pour la 
sortie) et pour n'importe quel oracle pour n'importe quel  $x \in [0,1]$, la MTO 
va calculer  $y=f(x)$  à~$2^{-n}$  près en interrogeant l'oracle pour 
certaines précisions  $m$. Puisque~$\ca$   est une classe de \com du type 
prévu, le plus grand des entiers  $m$  utilisés sur l'entrée  $n$  peut 
être majoré par $\mu(n)$ où $\mu\colon  \NN_1 \rightarrow  \NN_1$  est une 
fonction dans la classe~$\ca${\footnote{Par exemple pour une classe de 
complexité où on spécifie que les sorties sont de taille linéaire,  la 
taille de $m$ dépend linéairement de $n$ (indépendamment de l'oracle) 
puisque $m$, en tant que question posée à l'oracle, est une des sorties.}}. 
C'est le \mcu que nous cherchons.\\
Supposons maintenant que la fonction $f$ soit  \uni de classe~$\ca$.
 Soit  $M$  la machine de Turing (sans oracle) qui calcule  
$(f(x))_{x \in \DD}$   en tant que famille de nombres réels dans la classe~$\ca$.            
Soit  $M'$ une machine de Turing qui calcule un \mcu  $\mu \colon  \NN_1 \rightarrow 
\NN_1$  dans la classe~$\ca$. La MTO à la Ko-Friedman est alors la suivante. 
Sur l'entrée  $n$  elle calcule  $m = \mu(n+1)$  (en utilisant  $M'$), elle 
interroge l'oracle avec la précision  m, l'oracle donne un 
\elem  $a \in \DD_m$. La MTO utilise alors  $M$  pour calculer  $f(a)$  avec 
une approximation de $1/2^{n+1}$, qui, privée de son dernier bit, constitue la 
sortie de la MTO. \eop  

\medskip La proposition \ref{f415} démontre, par contraste, que la non parfaite adéquation 
obtenue dans la proposition \ref{f414} est seulement due à la difficulté de 
faire entrer exactement et à tout coup la notion de complexité d'une 
fonction \unico (en tant que fonction) dans le moule de la notion de 
``complexité d'un point dans un espace métrique''. Cette adéquation 
parfaite a lieu pour des classes comme  $\p$, $\Prim $ ou $\Rec $ mais pas pour  
$\DRT(\Lin, \Oo(n^k))$. \\
Enfin on pourra noter la grande similitude entre les preuves des propositions 
\ref{f415} et \ref{f414}, avec une complication un peu plus grande pour cette 
dernière.   
\sni \proof (de la proposition \ref{f414}) Soit tout d'abord  $f \in \czu$  et   
$M$  une  MTO  qui, pour toute entrée  $n \in \NN_1$  et tout $x \in [0,1]$ 
(donné comme oracle) calcule  $f(x)$ à  $2^{-n}$ près, en temps majoré 
par  $T(n)$. \\
Considérons la suite  $f_n = (Pr_n, n, T(n), T(n))_{n \in \NN_1}$ où on 
a:\\
$Pr_n$  est obtenu à partir du programme  $Pr$  de la machine  $M$  en 
rempla\c{c}ant la réponse  $(x_m)$  de l'oracle  à la question  ``$\,m\;?\,$''  
par l'instruction   ``lire  $x$  avec la précision  $m$'' \\ 
La correction de la donnée $f_n$ est claire. Soit  $\wi{f_n}$  la 
fonction linéaire par morceaux correspondant à la donnée  $f_n$.  On a  
pour  $\xi \in [0,1]$  et  $d$  une approximation de  $\xi$ à  $1/2^{T(n)}$  
près
$$ {\Exec}(Pr_n, d) = {\Exec} (Pr, n, \hbox{Oracle pour } \xi) 
= M^{\xi}(n)$$
et donc
$$\abS{ \wi{f_n}(\xi) - f(\xi) }\;  \leq 
\abs{f(d) - {\Exec}(Pr_n, \xi) } + \abS{ \wi{f_n}(d) - {\Exec} (Pr_n, \xi) }\;  \leq 2^{-n} + 2^{-n} \leq 2^{-(n-1)}$$ 
Enfin la suite  $n \mapsto f_{n+1}$  est de complexité  $\DTI (\Oo(n))$.\\
Réciproquement, supposons que  $\norme{ f_n - f}  \leq 2^{-n}$  avec    
$n \mapsto f_n$   de classe  $\DTI (T(n))$. \\
On a  $f_n = (Pr_n, q(n), m(n), t(n))$. Considérons la MTO  $M$ qui, pour tout 
entier  $n \in \NN_1$ et tout  $\xi \in [0,1]$, effectue les tâches 
suivantes: \\
--- prendre l'élément  $f_{n+1} = (Pr_{n+1}, q(n+1), m(n+1), t(n+1))$, de la 
suite, correspondant à  $n+1$. \\
--- poser à l'oracle la question  $Q(n) = m(n+1) + n+1 - q(n)$  (on obtient 
une approximation  $d$  de  $\xi$ à $2^{-Q(n)}$ près). \\
--- calculer  $\wi{f_{n+}}(d)$ par interpolation linéaire (cf. 
proposition \ref{f413}): ceci est la sortie de la machine~$M$.\\   
On a alors pour tout $ \xi \in [0,1]$
\[
\abS{\wi{f_n}(\xi) - M^{\xi}(n) }\;  \leq 
\abS{\wi{f}_n(\xi) - \wi{f}_{n+1}(\xi)} + 
\abS{\wi{f}_{n+1}(\xi) - \wi{f}_{n+1}(d) }\;  
\leq 2^{-(n+1)} + 2^{-(n+1)} = 2^{-n}  
\] 
donc 
\[
\Norme{ f - M^{\xi}(n)} \; \leq\;  \Norme{ f - \wi{f_n}} + 
\Norme{ \wi{f_n} - M^{\xi}(n) } 
\; \leq\;  2^{-n} + 2^{-n}
\]
De plus la machine  $M$ est de complexité $\Oo(T^2(n))$ (cf. proposition 
\ref{f413}).\eop

\medskip \noindent Dans le même style que la proposition \ref{f415}, on obtient deux 
caractérisations naturelles, à $\p$-équivalence près, de la \rp   $\ckf$  
de  $\czu$.  

\begin{fproposition} \label{f416}
Soit $\sC^{\star}$  une \rp  de l'espace  $\czu$.  
Alors on a l'équivalence des deux assertions suivantes.   
%
\begin{enumerate}
%
\item La \pres  $\sC^{\star}$ est $\p$-\equiva à  $\ckf$.
%
\item Une famille  $\big(\wi{f}\,\big)_{f \in Z}$  dans  $\czu$  est \uni de classe  
$\p$  si et seulement si elle est une $\p$-famille de points de  $\sC^{\star}$.
%
\end{enumerate}
\end{fproposition}

\proof 
Pour n'importe quelle classe élémentairement stable~$\ca$, deux \rps  d'un espace métrique arbitraire sont~$\ca$-\equivas si et seulement si elles définissent les mêmes~$\ca$-familles de points (cf. remarque \ref{f228}). Il suffit donc de vérifier que la \rp   $\ckf$  vérifie la condition (2). La preuve est identique à celle de la proposition \ref{f414}. 
\eop


\begin{ftheorem} \label{f417}
La \pres  $\ckf$  est universelle pour l'évaluation au sens suivant.\\
 Si  $\sC_Y=(Y,\delta,\eta)$  est une \pres rationnelle de  $\czu$  pour laquelle la famille  
$\big(\wi{f}\,\big)_{f \in Y}$  est \uni de classe $\p$,  
alors la fonction  $\Id_\czu$  de $\sC_Y$  vers  $\ckf$  est \uni de 
classe~$\p$.\\ 
Le résultat se généralise à toute classe de \com~$\ca$   
élémentairement stable vérifiant la propriété suivante:
si  $F \in \ca$  alors le temps de calcul de $F\colon  \NN_1 \rightarrow \NN_1,\; N 
\mapsto F(N)$,  est majoré par une fonction de classe~$\ca$.
\end{ftheorem}

\proof 
Cela résulte de \ref{f416}. Donnons néanmoins la preuve.  Nous devons démontrer 
que la famille   $\big(\wi{f}\,\big)_{f \in Y}$   est une $\p$-famille de points   pour la \pres  $\ckf$.  Puisque  la famille  $\big(\wi{f}\,\big)_{f 
\in Y}$  est \uni de classe $\p$, on a deux fonctions $\psi$ et  $\mu$  de classe  
$\p$:
$$ \psi \colon  Y \times \DD_{[0,1]} \times \NN_1 \rightarrow \DD \qquad  
(f,d,n) \mapsto \psi(f,d,n) = \wi{f}(d) \;  \hbox {avec  la  
précision} \; 1/2^n$$
et 			
$$ \mu\colon  Y \times \NN_1 \rightarrow \DD \qquad  (f,n) \mapsto \mu(f,d,n) = m $$
où  $\mu$  est un \mcu pour la famille.  
Soit $S \colon   \NN_1 \rightarrow \NN_1$   une fonction de classe~$\p$  qui donne une majoration du temps de calcul de $\psi$. \\  
Considérons alors la fonction  $\varphi \colon  Y \times \NN_1 \rightarrow \ykf $   
définie par:
$$\varphi(f,n) = (Pr(f,n), n, \mu(f,n), S(\flo{f} + \mu(f,n) + n))$$
où  $\flo{f}$  est la taille de $f$ et  $Pr$  est le programme calculant 
$\psi$ à peine modifié: 
il prend en entrée seulement les  $d \in \DD_{m,[0,1]}$ et ne garde pour la sortie que les chiffres significatifs dans $\DD_n$.\\
La fonction $\varphi$  est aussi dans la classe $\p$. \\
De plus il est clair que le point rationnel codé par $(Pr, n, m, S(\flo{f} + m+ 
n))$  ainsi construit approxime~$\wi f$~à~$1/2^n$   prés.  	\eop

\medskip Notez que cette propriété universelle de la \pres  $\ckf$  est évidemment 
partagée par toute autre \pres  $\p$-\equiva à  $\ckf$.

\subsubsection{Présentation  par circuits booléens} \label{fsubsubsec412}
La \pres par circuits booléens est essentiellement la même chose que la 
\pres à la Ko-Friedman. Elle est un peu plus naturelle dans le cadre que nous 
nous sommes fixés (\rp  de l'espace métrique  $\czu$). 

Notons   $\CB$  l'ensemble des (codes des) circuits booléens. Codé 
informatiquement, un circuit booléen  $C$  est simplement donné par un 
programme d'évaluation booléen (boolean straight-line program) qui représente l'exécution du circuit  $C$.
Un point rationnel de la \pres par circuits booléens est une fonction 
linéaire par morceaux  $\wi{f}$    codée par un quadruplet  
$f=(C,n,m,k)$ où $C$ est (le code d')un circuit booléen,  $n$  est la 
précision réclamée pour  $f(x)$, $m$ la précision nécessaire en 
entrée, et $2^{k+1}$  majore la norme de  $\wi{f}$. 

\begin{fdefinition} \label{f418}
Un point rationnel de la \pres par circuits booléens  $\cbo$  est codé par 
un quadruplet  $f = (C, n, m, k) \in \CB \times \NN_1 \times \NN_1 \times 
\NN_1$  où  $C$   est (le code d')un circuit booléen ayant  $m$  portes 
d'entrée et  $k+n+1$   portes de sortie:  les $m$   portes d'entrée 
permettent de coder un élément de $\DD_{m,[0,1]}$  et  les $k+n+1$  portes 
de sortie permettent de coder les éléments de  $\DD$  de la forme 
$\pm 2^k\left( \sum_{i=1,\ldots,n+k}b_i2^{-i}\right)$  (où les $b_i$ sont 
égaux à $0$ ou $1$, $b_0$ code le signe $\pm$).\\
La donnée  $f = (C, n, m, k)$  est dite {\em correcte} lorsque deux entrées 
codant des éléments de  $\DD_{m,[0,1]}$  distants de  $1/2^m$  donnent en 
sortie des codages de nombres distants de au plus  $1/2^n$.  L'ensemble des 
données correctes est  $\ybo$.\\
Lorsqu'une donnée $f$ est correcte, elle définit une fonction linéaire par 
morceau  $\wi{f}$   qui est bien contrôlée (c'est le ``point 
rationnel'' défini par la donnée).
\end{fdefinition}

On notera que les paramètres de contrôle  $n, m$ et $k$  sont surtout 
indiqués pour le confort de l'utilisateur. En fait  $n+k$   et  $m$  sont 
directement lisibles sur le circuit  $C$. \\
De ce point de vue, la situation est un peu améliorée par rapport à la 
\pres  $\ckf$  pour laquelle les paramètres de contrôle sont absolument 
indispensables.
La taille de  $C$  contrôle à elle seule le temps d'exécution du circuit  
(c.-à-d. la fonction qui calcule, à partir du code d'un circuit booléen et 
de la liste de ses entrées, la liste de ses sorties est une fonction de très 
faible \comz).
\begin{fproposition} \label{f419} 
{\em  (Complexité de la famille de fonctions attachées à   $\ybo$)}.  
La famille de fonctions continues  $\big(\wi{f}\,\big)_{f \in \ybo }$ est \uni de 
classe  $\DSRT(\Lin, \Lin, \QLin)$.
\end{fproposition}

\proof
On recopie presque à l'identique la preuve de la proposition \ref{f413} 
concernant la famille  $\big(\wi{f}\,\big)_{f \in \ykf}$. La seule différence est 
le remplacement de la fonction  $\Exec(Pr,a)$  par l'évaluation d'un circuit 
booléen.   La fonction qui calcule, à partir du code d'un circuit booléen 
et de la liste de ses entrées, la liste de ses sorties est une fonction de 
\com en temps $\Oo(t\log(t))$ (avec $t = \ta(C)$), donc de classe  $\DSRT(\Lin, 
\Lin, \QLin)$. 	\eop

\medskip \noindent  {\bf NB}. Si on avait pris en compte ``le temps de gestion'' (cf. remarque \ref{f323}) on aurait trouvé une \com  $\DSRT(\Lin, \Lin, \Oo(N^2))$, c.-à-d. le même résultat que pour la famille  $\big(\wi{f}\,\big)_{f \in \ykf}$  (la \com de  $\Exec(Pr,a)$  prend justement en compte ``le temps de gestion'').

\subsubsection{Présentation par circuits arithmétiques fractionnaires (avec magnitude)}\label{fsubsubsec 413}
Rappelons qu'un circuit {\em arithmétique (fractionnaire)} est par 
définition un circuit qui a pour portes d'entrées des variables ``réelles''  $x_i$  et des constantes dans $\QQ$  (on pourrait évidemment ne tolérer que 
les deux constantes  0  et  1  sans changement significatif). \\
Les autres portes sont:\\
--- des portes à une entrée, des types suivants:  
$x \mapsto x^{-1}, \; x \mapsto -x$ \\
--- des portes à deux entrées, des deux types suivants:  
$x+y, \; x \times y$. \\ 
Un circuit arithmétique calcule une fraction rationnelle, (ou éventuellement 
``error'' si on demande d'inverser la fonction identiquement nulle).
Nous considérerons des circuits arithmétiques avec une seule variable en 
entrée et une seule porte de sortie et nous noterons  $\CA$  l'ensemble des 
(codes de) circuits arithmétiques à une seule entrée.  Le code d'un 
circuit est le texte d'un programme d'évaluation (dans un langage et avec une 
syntaxe fixés une fois pour toutes) correspondant au circuit arithmétique 
considéré.


\begin{fdefinition} \label{f4110}
Un entier  $M$  est appelé {\em un coefficient de magnitude} pour un circuit 
arithmétique $\alpha$ à une seule entrée lorsque $2^M$  majore en valeur 
absolue toutes les fractions rationnelles du circuit (c.-à-d. celles 
calculées à toutes les portes du circuit) en tout point de l'intervalle 
$[0,1]$. \\
Un couple  $f = (\alpha, M) \in \CA \times \NN_1$ est une donnée {\em 
correcte} lorsque $M$ est un coefficient de magnitude pour le circuit  $\alpha$.  
Cela définit les éléments de  $\yaf$.  Un élément $f$ de  $\yaf$  
définit la fraction rationnelle notée  $\wi{f}$    lorsqu'elle est 
vue comme un élément de  $\czu$.  
La famille  $\big(\wi{f}\,\big)_{f \in \yaf }$ est la famille des points rationnels 
d'une \pres de  $\czu$  qui sera notée  $\caf$
\end{fdefinition}
 
Notez que, à cause de la présence des multiplications et du passage à 
l'inverse, la taille de  $M$ peut être exponentielle par rapport à celle de  
$\alpha$, même lorsqu'aucune fraction rationnelle de $\alpha$ n'a de pole sur  
$[0,1]$.  Ce genre de mésaventure n'avait pas lieu avec les circuits 
semilinéaires binaires.
Il semble même improbable que la correction d'un couple 
$(\alpha, M) \in \CA \times \NN_1$  puisse être testée \etpo (par rapport 
à la taille de la donnée $(\alpha, M)$. 

\begin{fproposition}[complexité de la famille de fonctions attachées à  $\yaf$] \label{f4111}~\\    
La famille de fonctions continues  $\big(\wi{f}\,\big)_{f \in \yaf }$  est \uni de 
classe  $\p$, et plus précisément de classe  $\DSRT(\Oo(N^3), \Lin, 
\Oo(N{\M}(N^2)))$.
\end{fproposition}

\proof 	Soit $ f = (\alpha,M) $ un élément de  $\yaf$. Soit $ p $ la 
profondeur du cirduit $\alpha$. On démontre par récurrence sur $\pi$ que, pour 
une porte de profondeur $\pi$  la fonction 
rationnelle correspondante a une dérivée majorée par $2^{2M \pi}$ sur 
l'intervalle $[0,1]$. Ceci fournit le \mcu  $\mu(f,k) = k + 2Mp$  (qui est en 
$\Oo(N^2)$) pour la famille  $\big(\wi{f}\,\big)_{f \in \yaf }$. \\
Nous devons maintenant exhiber une fonction  
$\psi \colon  \yaf  \times \DD_{[0,1]} \times \NN_1 \rightarrow \DD$  
de \com $ \DRT(\Lin,\Oo(N\M(N^2))) $ telle que
\[
\forall (f, x, k) \in \yaf  \times \DD_{[0,1]} \times \NN_1 \; \;
\abs{ \psi(f,x,k) - \wi{f}(x)} \;  \leq 2^{-k}.
\]
Soit  $(f, x, k) \in \yaf  \times \DD_{[0,1]} \times \NN_1$  où  
$f = (\alpha,M)$.  Soit $ t = \ta(\alpha) $ le nombre des portes du circuit 
$\alpha$. La taille des entrées est $N = t+M+k$. \\ 
Pour calculer $\psi((\alpha,M), x, k)$  on procède comme suit.  
On lit $ m = k+2Mp+p $ bits de $x$ ce qui donne les deux points consécutifs
 $ a $ et $ b $ de $\DD_{m,[0,1]}$   tels que $a \leq x \leq b$.  On exécute 
le circuit $ \alpha $ au point $ a $ en tronquant le calcul effectué sur 
chaque porte aux $ m $ premiers bits significatifs. La valeur obtenue à la 
sortie, tronquée aux $ k $ premiers bits significatifs, est l'élément 
$\psi((\alpha,M), x, k)$. Comme une multiplication (ou une division) se fait en 
temps ${\M}(m)$, le temps des opérations arithmétiques proprement dites est 
un $\Oo(t \M(k+2Mp+p)) = \Oo(N\M(N^2))$ et l'espace occupé est en $\Oo(N^3)$.
	\eop 

\mni {\bf NB}. Si on avait pris en compte ``le temps de gestion'' (cf. remarque 
\ref{f323}) on aurait dit: \\ 	
le temps des opérations arithmétiques proprement dites est $\Oo(N \M(N^2)),$ 
et la gestion des objets est en $\Oo(N^4).$ Ainsi si on considère la 
multiplication rapide (respectivement naïve), la fonction d'évaluation 
est dans $ \DRT(\Lin,\Oo(N^4))$ (respectivement $\DRT(\Lin,\Oo(N^5))$) où $N$ est 
la taille des données.
\subsubsection{Présentation par circuits arithmétiques polynomiaux (avec 
magnitude)}\label{fsubsubsec414}

La \pres suivante de  $\czu$  sera notée   $\capo$. L'ensemble des codes des points rationnels  sera noté  $\yap$. 

C'est la même chose que les circuits arithmétiques fractionnaires, sauf 
qu'on supprime les portes ``passage à l'inverse''. Il n'est donc pas 
nécessaire d'écrire la définition en détail.\\
On retrouve les mêmes difficultés concernant la magnitude, en un peu moins 
grave. Il ne semble pas que l'on puisse obtenir pour les circuits polynomiaux de 
majorations sensiblement meilleures que celles obtenues à la proposition 
\ref{f4111} pour les circuits avec divisions.


\subsection{Comparaisons des présentations précédentes}\label{fsubsec42}

Nous démontrons dans cette section un résultat important, l'équivalence entre 
la \pres à la Ko-Friedman et les quatre présentations ``par circuits'' de 
$\czu$ citées dans la section précédente, ceci du point de vue de la \com 
en temps \poll. En particulier nous obtenons 
une formulation complètement contrôlée du point de vue algorithmique 
pour le théorème d'approximation de Weierstrass.  \\
Pour démontrer l'équivalence entre ces cinq présentations du point de vue de 
la \com en temps \poll, nous  suivrons le plan ci-dessous:

\begin{figure}[htbp]
\begin{center}
$$
\xymatrix @R=15pt{
  \ckf\ar[r] &  \cbo\ar[r] &  \csl\ar[r] &  \caf\ar[r] & \capo\ar@/^25pt/[llll] 
}
$$ 
\end{center}
\medskip\caption[schéma de preuve]{\label{ffi421}schéma de la preuve  
}  
\end{figure}  

Pour donner l'équivalence entre ces différentes présentations, nous devons 
construire des fonctions qui permettent de transformer un point rationnel d'une 
\pres donnée en un point rationnel d'une autre présentation, qui approche 
convenablement le premier.

\begin{fproposition} \label{f421}
L'identité de  $\czu$  de  $\ckf$  vers  $\cbo$  est \uni de classe  $\p$,  en 
fait de classe  $\DTI (N^{14})$.
\end{fproposition}

La preuve de cette proposition est basée sur le Lemme \ref{f131} qui décrit 
la \com d'une Machine de Turing Universelle et sur le lemme suivant.

\begin{flemma} \label{f422}
{\rm (cf. \cite{fSt} et \cite{fMo}) }  \\
Soient $ M $ une machine de Turing fixée, $ T $ et $ m $ des éléments de 
$\NN_1$. Alors on a une fonction calculable en temps $\Oo((T+m)^7)$  qui calcule  
$\gamma_{T,m}$:  un circuit booléen simulant les $ T $ premières 
configurations de $ M $ pour n'importe quelle entrée dans  $\{0,1\}^m$.
\end{flemma} 

\begin{fremark} \label{f423}
J. Stern a donné dans \cite {fSt}, pour tout entier $T \in \NN_1$, une 
construction assez simple d'un circuit  $\gamma_T$  qui calcule la configuration 
de $ M $ obtenue après $ T $ pas de calcul à partir d'une configuration 
initiale de taille $\leq  T$  (on peut supposer que $T \geq  m$ ou prendre 
$\max(T,m)$). La taille du circuit  $\gamma_T$  est un  $\Oo(T^2)$. L'auteur a 
mentionné aussi que cette construction se fait en temps \poll. Une \com 
plus précise (de l'ordre de  $\Oo(T^7)$) est donnée dans \cite{fMo}.
\end{fremark}

\proof (de la proposition \ref{f421})	On considère la Machine de Turing 
Universelle MU du lemme \ref{f131}.  Soit $ f = (Pr,n,m,T) $ un élément de  
$\ykf$.  Notons $ p $ la taille du programme $Pr.$ 
La taille de  $f$ est $N = p+n+m+T.$  La machine $ MU $ prend en entrée le 
programme $ Pr $  ainsi qu'une entrée
 $ x \in  \DD_{m,[0,1]}$  et le nombre d'étapes $ T.$ 
Elle exécute en $\Oo(T (\max(T,m)+p)) = \Oo(N^2)$ étapes (cf. lemme \ref{f131})  
la tâche de calculer la sortie pour le programme $ Pr $ sur la même entrée 
$x$ après $ T $ étapes de calcul.  
En appliquant le lemme \ref{f422}  on obtient un temps  $\Oo(N^{14})$  pour 
calculer à partir de l'entrée $f$ un \elem $ g $ de  $\ybo$  
(un circuit booléen ainsi que ses paramètres de contrôle)  
pour lequel on a 
$ \wi{g} = \wi{f}$.
	\eop

\begin{fproposition} \label{f424}
L'identité de  $\czu$  de  $\cbo$  vers  $\csl$  est \uni de classe    $\LINT 
$.  Plus précisément, on a une fonction discrète qui, à partir d'un 
élément $ f = (\gamma,n,m,k) $ de $\ybo$  et d'un entier 
$ q \in \NN_1$,  calcule en temps $ \Oo(N) $ (où $ N $ est la taille de 
l'entrée $ ((\gamma,n,m,k),q)$), un circuit semilinéaire binaire $ g $ tel 
que  
\[
\forall x \in [0,1]\;\; 
\abs{ \wi{f}(x) - \wi{g}(x)} \;  \leq 2^{-q} \; \;  \eqno (\F4.2.4) .
\]  
\end{fproposition}

La preuve de cette proposition utilise le lemme suivant:
\begin{flemma} \label{f425} 
Il existe une fonction calculable en temps $\Oo(N)$ qui transforme tout 
élément $f = ((\gamma,n,m,k),h)$ de $\ybo \times  \NN_1$   en un élément 
$f' = (\gamma',n+h,m+h,k)$ de $\ybo$  correspondant à la même fonction 
semilinéaire.
\end{flemma}

\proof (du lemme \ref{f425})
Supposons que le circuit $ \gamma $ calcule, pour l'entrée $x = u/2^m$ où
 $0 \leq u \leq 2^m - 1$, la valeur $y = \ell/2^n$, et pour $x' = (u+1)/2^m$, la 
valeur $y' = \ell'/2^n$ avec $\ell , \ell' \in  \ZZ$  et  
$\abs{ \ell - \ell'} \;  \leq 1$.
 Alors le circuit   $\gamma'$  calcule, pour l'entrée  $x+r2^{-h}2^{-m}$, 
$0 \leq r \leq 2^h$, la valeur  $y+(y'-y)r2^{-h}$. \\
Noter de plus que  $  \ta(\gamma') = \Oo(\ta(\gamma )+h)  $  et  $\prof(\gamma') = 
\Oo(\prof(\gamma)+h)$.	\eop

\proof (de la proposition \ref{f424})
D'après le lemme \ref{f425}, la condition (\F4.2.4) de la 
proposition \ref{f423} peut être remplacée par la condition:   
\[
\forall x \in [0,1]\;\; \abs {\wi{f}(x) - \wi{g}(x)} \;  \leq 2^{-(n-1)}  
   \eqno{(\F4.2.5)}
\] 
En effet: si   $q < n$  alors (\F4.2.4) découle de (\F4.2.5), sinon, on utilise 
l'interpolation linéaire donnée dans la preuve du lemme \ref{f425}.\\
Nous cherchons donc maintenant à simuler le circuit booléen   
$f = (\gamma,n,m,k)$   par un circuit semilinéaire binaire  $g$  avec la 
précision  $ 1/2^n$. \\
Tout d'abord remarquons qu'il est facile de simuler de manière exacte toutes 
les portes, ``sauf l'entrée'', par un circuit semilinéaire simple  $\lambda$ 
qui consiste en:\\
1) remplacer les constantes booléennes $ 0 $ et $ 1 $ de $ \gamma $ par les 
constantes rationnelles $ 0$ et $1$.\\
2) remplacer chaque sommet de $ \gamma $ calculant $\neg u$ par un sommet de 
$ \lambda $ calculant $1-u$. \\
3) remplacer chaque sommet de $ \gamma $ calculant $u \land v$ par un sommet 
de $ \lambda $ calculant $\min(u,v)$.\\
4) remplacer chaque sommet de $ \gamma $ calculant $u \lor v$ par un sommet
 de $ \lambda $ calculant $\max(u,v)$. \\
5) calculer, à partir des sorties $c_0, c_1, \ldots,c_{n+k}$ du 
circuit $ \gamma, $ le point rationnel:
\[
\pm 2^k \sum_{i=1,\ldots,n+k}c_i2^{-i} \quad (c_0  \hbox{ code le  signe})
\]
Il est clair que le circuit $ \lambda $ se construit en temps linéaire et 
admet une taille $ \ta(\lambda) = \Oo(\ta(\gamma)) $ et une profondeur 
$ \prof(\lambda) = \Oo(\prof(\gamma))$.\\
Maintenant nous cherchons à ``simuler l'entrée'' du circuit $\gamma,$ 
c.-à-d. les bits codant $x$, par un circuit semilinéaire binaire.\\
Usuellement, pour déterminer le développement binaire d'un nombre réel 
$x$, on utilise le pseudo-algorithme  suivant qui utilise la fonction 
discontinue $ C $ définie par:
$$ C(x) = \left\{
\begin{array}{cl} 
1 & \hbox{ si  } x \geq 1/2 \\ 
0 &
\hbox{sinon} 
\end{array}
\right.
$$

\sni {\bf Algorithme 1} (``calcul'' des $m$ premiers bits d'un nombre réel $x\in [0,1]$)
\begin{itemize}
\item[] Entrées: $ x \in [0,1], \; m \in \NN$ \\
        Sortie:   la liste $(b_1, b_2,\ldots, b_m) \in \{ 0,1 \}^m$ 
	\begin{itemize}
	\item[] Pour $j:=1$ à $m$ Faire
		\begin{itemize}
		\item[] $ b_j \leftarrow C(x)$ \\
			 $ x \leftarrow 2x - b_j$ 
		\end{itemize}
	  Fait \\
 	Fin.
	\end{itemize}
\end{itemize}

\sni La fonction $C$ est discontinue, donc l'algorithme 1 {\em n'est pas vraiment 
un algorithme}; et il ne peut pas être simulé par un circuit semilinéaire 
binaire. On considère alors la 
fonction continue
$$C_p(x) := C_{p,1/2} = \min(1, \max(0,2^p(x-1/2))) \eqno \hbox{(cf. figure \ref{ffi336})}
$$
qui ``approche'' la 
fonction  $C$.  Et l'on considère l'algorithme suivant:

%:\newpage
\newpage
\mni {\bf Algorithme 2} (calcul ``approximatif'' des $ m $ premiers  bits qui 
codent un nombre réel $x$)
\begin{itemize}
\item[] Entrées: $ x \in [0,1], \; m \in \NN_1$ \\
  Sortie: une liste $(b_1, b_2,\ldots, b_m) \in [ 0,1 ]^m$ 
	\begin{itemize}
	\item[] $p \leftarrow m+2$ \\
			 Pour $j:=1$ à $m$ Faire
		\begin{itemize}
		\item[] $ b_j \leftarrow C_p(x)$ \\
			    $ x \leftarrow 2x - b_j$  \\
				$p \leftarrow p-1$
	     \end{itemize}
	  Fait \\
 	Fin.
	\end{itemize}
\end{itemize}

\sni Ceci est réalisé par un circuit de taille et profondeur $\Oo(m)$ sous la 
forme suivante:

\mni {\bf Algorithme 2bis} (forme circuit semilinéaire de l'algorithme 2)
\begin{itemize}
\item[] Entrées: $ x \in [0,1], \; m \in \NN_1$ \\
 Sortie: une liste $(b_1, b_2,\ldots, b_m) \in [ 0,1 ]^m$ 
	\begin{itemize}
	 \item[] $p \leftarrow m+2 , \; q \leftarrow 2^p$ \\
			 Pour $j:=1$ à $m$ Faire
				\begin{itemize}
			 \item[] $ y \leftarrow q(x-1/2)$ \\
			 		 $ b_j \leftarrow \min (1, \max(0,y))$ \\
					$x \leftarrow 2x - b_j$ \\
 				    $q \leftarrow q/2$
				 \end{itemize}
	  Fait \\
 	Fin.
	 \end{itemize}
\end{itemize}

\sni Mais un problème crucial se pose quand le réel $x$ en entrée est dans 
un intervalle de type  $\; ]\,k/2^m, k/2^m + 1/2^p\,[\; $   où 
$0 \leq k \leq 2^m-1$. Dans ce cas au moins un bit calculé dans l'algorithme 2 est dans $]0,1[$. Par conséquent le résultat final de la 
simulation du circuit booléen peut être incohérent. 
Pour contourner cette difficulté, on utilise une technique introduite par 
Hoover \cite{fHo87,fHo90}. On fait les remarques suivantes:\\
--- Pour tout  $x \in [0,1]$  au plus une des valeurs  
$x_{\sigma} = x + \frac \sigma   {2^{m+2}}$  (où  $ \sigma \in \{-1, 0, 1 \} $) est dans un  intervalle de type précédent.\\
--- Notons  $z_{\sigma} = \sum_{j=1,\ldots,m}b_j2^{-j}$  où les $b_j$  sont 
fournis par l'algorithme 2bis sur l'entrée  $x_{\sigma}$. D'après la remarque précédente, au moins deux valeurs  $\lambda(\wi {\lambda}(z_{\sigma}) \; (\sigma \in \{-1,0,1\})$, 
correspondent exactement à la sortie du circuit arithmétique  $\gamma$  
lorsqu'on met en entrée les  $m$ premiers bits de~$x_{\sigma}$. 
C.-à-d.,  pour au moins deux valeurs de  $\sigma$, on a  $z_{\sigma} \in 
\DD_m$  et $\wi {\lambda}(z_{\sigma}) = \wi{f}(z_{\sigma})$   avec en outre $\abs{ z_{\sigma} - x }\;  \leq 1/2^m + 1/2^{m+2} \leq 1/2^{m-1}$ 
(d'où  $\abS{ \wi{f}(z_{\sigma}) - \wi{f}(x) }\;  \leq 1/2^{n-1}$).\\
--- Donc, une éventuelle mauvaise valeur calculée par le circuit 
semilinéaire ne peut être que la première ou la troisième des valeurs si 
on les ordonne par ordre croissant.
Ainsi, si   $y_{-1}$, $y_0$ et $y_1$,  sont les trois valeurs respectivement calculées par le circuit aux points $x_{\sigma}$  ($\sigma\in  \{-1, 0,1\}$), alors la deuxième des 3 valeurs approche correctement $\wi{f}(x)$. 
Etant donnés trois nombres réels  $y_{-1}$, $y_0$ et $y_1$, le  2ème par 
ordre croissant, $\theta(y_{-1},y_0,y_1)$,  est calculé par exemple par l'un des deux circuits semilinéaires binaires représentés par le 
deuxième membre dans les équations:
\[
\theta (y_{-1},y_0,y_1) = y_{-1}+y_0+y_1 - \min (y_{-1},y_0,y_1) - 
\max (y_{-1},y_0,y_1)
\]
\[\theta (y_{-1},y_0,y_1) = \min (\max (y_0,y_1), \max (y_1,y_{-1}), 
\max (y_0,y_{-1}))
\]
Résumons: à partir de l'entrée $x$ nous calculons les  
$x_{\sigma} = x + \frac \sigma   {2^{m+2}}$  (où  $\sigma \in \{-1,0,1 \}$), 
et nous appliquons successivement:\\
--- le circuit $\varepsilon$ (de taille $\Oo(m) = \Oo(N)$)  qui décode 
correctement les $m$ premiers digits de $x_{\sigma}$  pour au moins deux des 
trois $x_{\sigma}$; \\
--- le circuit  $\lambda$  qui simule le circuit booléen proprement dit et 
recode les digits de sortie sous forme d'un dyadique, ce circuit est de taille 
$\Oo(N)$;\\
--- puis le circuit  $\theta$  qui choisit la  2ème  valeur calculée par 
ordre croissant.\\
Nous obtenons donc un circuit qui calcule la fonction:
\[
\wi{g}(x) = \theta (\lambda(\varepsilon (x-1/2^p)),
\lambda (\varepsilon (x)), \lambda (\varepsilon (x+1/2^p))) \quad \hbox{avec} 
\; p = m+2
\]
 de taille et de profondeur \Oo(N) et tel que: 
\[
\abs{ \wi{f}(x) - \wi{g}(x) }\;  \leq 2^{-n} \; \forall x \in 
[0,1].
\]
En outre, le temps mis pour calculer  (le code de)  $g$  à partir de 
l'entrée $f$ est également en $\Oo(N)$.

\begin{figure}[htbp]  
\begin{center}
\includegraphics*[width=3cm]{fi422}
\end{center}
\caption[vue globale du calcul]{\label{ffi422}  
vue globale du calcul}  
\end{figure}  
\eop

\medskip Nous passons à la simulation d'un circuit semilinéaire binaire par un circuit arithmétique (avec divisions). 
Nous donnons tout d'abord une version ``circuit'' de la proposition \ref{f335}.

\begin{fproposition} \label{f426}
Les fonctions  $(x,y) \mapsto \max(x,y)$ et $(x,y) \mapsto \min(x,y)$   sur le carré  $[-2^m, 2^m] \times [-2^m, 2^m]$   peuvent être approchées à  $1/2^n$ près par des circuits arithmétiques de taille  $\Oo(N^3)$, 
de magnitude  $\Oo(N^3)$ et qui peuvent être calculés dans la classe 
$\DTI (\Oo(N^3))$ où $(N = n+m)$.
\end{fproposition} 

\proof 
On reprend la preuve du lemme \ref{f333}. La re\pres de \pols approchant 
convenablement  $ P_n $ et $ Q_n $  au moyen de circuits est plus économique en espace. Tout d'abord il faut construire un circuit qui calcule une approximation de $ e^{-1/n} $ avec la précision $1/2^{n^3}$. 
On considère le développement de Taylor  
\[
F_m(z) = \sum_{0\leq k\leq m} \frac{(-1)^k}{k!} z^k
\] 
qui approche $ e^{z} $ à $1/2^{m+2}$  près si  $m\geq 5$ et $0\leq z\leq 1$. 
Ici, on peut se contenter de donner un circuit qui calcule $d_n = F_{n^3}(1/n)$ dont la taille et le temps de calcul sont a priori en $ \Oo(n^3) $ (alors qu'on était obligé de donner explicitement une approximation dyadique $c_{n,1}$  de $d_n$). 
Ensuite on construit un circuit qui calcule une bonne approximation 
de  $H_n(x) = \prod_{1 \leq k < n^2} (x+e^{-k/n})$ sous la forme  
$h_n(x) = (x+d_n)(x+d_n^2)\cdots(x+d_n^{n^2-1})$.
Ce qui réclame un temps de calcul (et une taille de circuit) en  $\Oo(n^2)$. 
Ceci fait que le circuit arithmétique qui calcule une approximation  à 
$ 1/2^n $ près  de  $\abs{x}$  sur  $[0,1]$   
est calculé en temps $\Oo(n^3)$. \\ 
On voit aussi facilement que le coefficient de magnitude est majoré par la 
taille de $ 1/h_n(0) $ c.-à-d. également un $ \Oo(n^3).$ 	\eop
 
\smallskip On remarquera que le coefficient de magnitude peut difficilement être amélioré. Par contre, il ne semble pas impossible que $ d_n$ puisse être calculé par un circuit de taille plus petite que la taille naïve en $\Oo(n^3)$.

\begin{fproposition} \label{f427}
L'identité de  $\czu$  de  $\csl$  vers  $\caf$  est \uni de classe  $\p$, en fait de classe $ \DTI (N^4). $ 
Plus précisément, on a une fonction discrète qui, à partir d'un élément $ g $ de $ \ysl $ de taille $ \ta(g) = t $ et de profondeur $ \prof(g) = p, $ et d'un entier $ n \in  \NN_1$,  calcule en temps $ \Oo( t (n+p)^3 ) $ un élément $ f = (\alpha,M)  \in \yaf$\\ 
\spa de taille	 $ \ta(\alpha)  =  \ta(g) \Oo((n+p)^3), $  \\
\spa de profondeur	 $ \prof(\alpha)  = \Oo(p(n+p)^3) $ \\
\spa avec coefficient de magnitude	 $ M = \Mag(\alpha) = \Oo((n+p)^3), $  
\\
et tel que 	
$\abS{\wi{f}(x)-\wi{g}(x)}\; \leq 2^{-n} \; \; \forall x \in [0,1]$. 
\end{fproposition} 

\proof 
Notons   $ p = \prof(g). $   Les portes de $ g $ pour une entrée 
$ x \in  [0,1]$    prennent des valeurs dans l'intervalle $[-2^p,2^p]$.  
On veut simuler à  $2^{-n}$  près le circuit $ g $ par un circuit 
arithmétique.  Il suffit de simuler les portes $\max$ et $\min$ du circuit 
semilinéaire à $2^{-(n+p)}$ près sur l'intervalle $[-2^p,2^p]$.  
Chaque simulation réclame, d'après la proposition \ref{f426}, un circuit 
arithmétique (avec division) dont toutes les caractéristiques sont 
majorées par un  $\Oo(n+p)^3$.	\eop

Nous passons à la simultation d'un circuit arithmétique avec divisions par 
un circuit arithmétique \poll.

\begin{fproposition} \label{f428}
L'identité de  $\czu$  de  $\caf$  vers  $\capo$  est \uni de classe  $\p$, en 
fait de classe  $\DTI (N^2)$.  
Plus précisément, on a une fonction discrète qui, à partir d'un 
élément  $f=(\alpha,M)$ de  $\yaf$ et d'un entier  
$n\in\NN_1$,  calcule en temps  $\Oo(N^2)$  un circuit arithmétique 
\poll   $(\Gamma, M') \in \yap $ ($N$ est la taille de l'entrée 
$((\alpha,M),n)$ \\
--- de taille  $ \ta(G) = \Oo( \ta(\alpha) (n + M) ), $  \\
--- de profondeur  $ \prof(G) = \Oo( \prof(\alpha) (n + M) ) $, \\
--- de magnitude  $ \Mag(G) = M'= \Oo( n + M ) $,  \\
--- et tel que:  
$\abS{ \wi{f}(x) - \wi{g}(x) }\;  \leq 2^{-n}$  
pour tout $x \in [0,1]$.
\end{fproposition}
 
Le problème se pose seulement au niveau des portes ``passage à l'inverse''. 
Nous cherchons donc à les simuler  par des circuits polynomiaux tout en 
gardant la magnitude bien majorée.

\begin{flemma} \label{f429}
La fonction   $ x \mapsto 1/x  $  sur l'intervalle  $[2^{-m}, 2^m]$  peut être 
réalisée avec la précision   $ 1/2^n  $   par un circuit \poll de 
magnitude $\Oo(m)$,  de taille  $ \Oo(m+n) $   et qui est construit en temps 
linéaire.
\end{flemma}

\proof Nous utilisons comme Hoover la méthode de Newton qui permet de calculer l'inverse d'un réel $z$ à $2^{-n}$ prés en un nombre raisonnable 
d'additions et multiplications en fonction de $n$.\\
{\bf Méthode de Newton} (pour le calcul de l'inverse de $z$)\\
Pour   $2^{-m} < z < 2^m$  on pose 
\[
C(x) = \left\{
\begin{array}{l} 
y_0 = 2^{-m} \\ 
y_{i+1} = y_i (2- zy_i) 
\end{array}
\right.
\]
On vérifie facilement que, pour    $i \geq 3m + \log(m+n)$,  on a :
\[
\abs{ z^{-1} - y_i }\;  \leq 2^{-n}.
\]
Ainsi pour l'entrée  $(n,m) \in \NN_1 \times \NN_1$,  
l'application de la méthode de Newton jusqu'à l'itération  
$i = 3m + \log(m+n)$  est représentée par un circuit \poll de taille 
\Oo(m+n) et de magnitude  $\Oo(m+n)$.  De plus, il est facile de vérifier que la 
construction de ce circuit se fait en temps linéaire.
  \eop

\begin{fproposition} \label{f4210}
L'identité de  $\czu$  de  $\capo$  vers  $\ckf$  est \uni de classe  $\p$, en 
fait de classe  $\DRT(\Lin, \Oo(N^4))$.
\end{fproposition} 

\proof 
Le fait d'être de classe  $\p$   résulte du théorème \ref{f417} et de la 
proposition \ref{f4111}.  La lecture précise des démonstrations du théorème 
\ref{f417} et de la proposition \ref{f4111} donne le résultat    
$\DRT(\Lin,\Oo(N^4))$   (en tenant compte du Nota bene après \ref{f4111}).  
\eop

\smallskip En résumant les résultats qui précèdent nous obtenons le théorème 
suivant.
\begin{ftheorem} \label{f4211}
Les cinq présentations    $\ckf$,  $\cbo$,  $\csl$,  $\caf$   et   $\capo$  de  
$\czu$  sont   $\p$-équivalentes.
\end{ftheorem}

\begin{fnotation} \label{f4212}
Pour autant qu'on se situe à un niveau de \com suffisamment élevé pour 
rendre le théorème de comparaison valable (en particulier la classe  $\p$  
suffit)  il n'y a pas de raison de faire de différence entre les cinq 
présentations  $\ckf$,  $\cbo$,  $\csl$,  $\caf$   et   $\capo$  de  $\czu$. 
En conséquence, désormais la notation  $\czu$  signifiera qu'on considère 
l'espace  $\czu$  avec la structure de calculabilité  $\csl$.
\end{fnotation}

\subsection{Complexité du problème de la norme}\label{fsubsec43}
On sait que la détermination du maximum,  sur un intervalle  $[0,1]$   d'une 
fonction calculable \etpo  $f \colon  \NN \rightarrow \{ 0,1 \}$ 
 est à très peu près la même chose que le problème $\np$-complet le 
plus classique:  SAT. 
 
Il n'est donc pas étonnant de trouver comme problème  
$\np$-complet un problème relié au calcul de la norme pour une fonction 
continue. Il nous faut tout d'abord formuler le problème de la norme attaché 
à une \rp  donnée de l'espace  $\czu$  d'une manière suffisamment 
précise et invariante. 
   
\begin{fdefinition}\label{f431}~\\
Nous appelons ``problème de la norme'', relativement à une \pres $\ca_1 = 
(Y_1, \delta_1, \eta_1)$ de  $\czu$)  le problème:\\
--- Résoudre {\em approximativement}  la question  
``$\; a \leq \norme{ f }_{\infty}\;?\;$''  dans la \pres  $\ca_1$  de $\czu$.\\
La formulation précise de ce problème est la suivante:\\
--- Entrées:  $(f,a,n) \in Y_1 \times \DD \times \NN_1$ \\
--- Sortie:    fournir correctement une des deux réponses:\\
\hspace*{1cm}--- voici un $x \in \DD$  tel que 
$\abs{f(x)}  \geq a - 1/2^n$, i.e.\  $\norme{f}_{\infty}\geq a - 1/2^n$, \\
\hspace*{1cm}--- il n'existe pas de  $x \in \DD$  vérifiant  
$\abs{f(x)}  \geq a$,  i.e.\  $\norme{f}_{\infty}\leq a$.       
\end{fdefinition}

Cette définition est justifiée par le lemme suivant.

\begin{flemma} \label{f432}
Pour deux \rps   $\ca_1$ et $\ca_2$  de  $\czu$  \polt équivalentes, 
les problèmes de la norme correspondants sont aussi \polt 
équivalents.
\end{flemma}

\proof 
La transformation du problème correspondant à une \pres $\ca_1$ au 
problème correspondant à une autre \pres  $\ca_2$  se fait par un algorithme 
ayant la même \com que l'algorithme qui permet de présenter la fonction 
identité entre  $\ca_1$  et  $\ca_2$. En effet, pour les données  $(f,a,n) 
\in Y_1 \times \DD \times \NN_1$, on cherche 
$g \in Y_2$ telle que  $\norme{ f-g }_{\infty} \leq 2^{-(n+2)}$,
 puis on résout le problème avec les entrées  $(g,a-1/2^{n+1}, n+2)$. 
 \\
Si on trouve  $x \in \DD$  tel que 
$\abs{g(x)}  \geq a - 2^{-(n+1)}- 2^{-(n+2)}$,  alors 
$\abs{f(x)}  \geq a - 2^{-n}$. 
\\ 
Si on déclare forfait, c'est qu'il n'existe pas de $x \in \DD$ tel que 
$\abs{g(x)}\;  \geq a - 2^{-(n+1)}$. 
A fortiori, il n'existe pas de  
$x \in \DD$  tel que  $\abs{f(x)} \geq a$.   
\eop

\begin{ftheorem} \label{f433}
Le problème de la norme est $\np$-complet pour les présentations  $\ckf$,  
$\cbo$,  $\csl$,  $\caf$   et   $\capo$.
\end{ftheorem}

\proof 
D'après le lemme \ref{f432} il suffit de faire la preuve pour la \pres $\cbo$. 
Le caractère  $\np$  du problème est immédiat.  
Pour voir la  $\np$-dureté, nous considérons le problème de la norme 
limité aux entrées  $((\gamma,1,m,1), 3/4, 2)$  où  $\gamma$  
est un circuit booléen arbitraire à  $m$  entrées et une sortie 
(le quadruplet est alors évidement correct), et la réponse oui 
correspond à la satisfiabilité du circuit  $\gamma$.  \eop

\medskip On a également le résultat suivant, essentiellement 
négatif{\footnote{Puisque  $\p \neq \np$ !}}, et donc moins intéressant.

\begin{fproposition} \label{f434}
Pour les présentations considérées, la fonction norme  
$f \mapsto \norme{ f }_{\infty}$ de $\czu$  vers  $\RR^{+}$ est \uni 
de classe $\p$  si et seulement si  $\p = \np$.
\end{fproposition}

\proof 
Il suffit de raisonner avec la \pres  $\cbo$. Si la fonction norme est  $\p$-
calculable{\footnote{Nous utiliserons quelquefois la terminologie  $\p$-
calculable   comme abréviation pour  calculable en temps \poll.}}, le 
problème de la norme se résout en temps \poll, et donc  $\p$ = $\np$.\\
Si  $\p = \np$, le problème de la norme se résout en temps \poll, ce 
qui permet de calculer la norme par dichotomie en initialisant avec la 
majoration $2^k$, jusqu'à obtenir la précision  $1/2^q$. Ceci réclame  
$k+q$ étapes de dichotomie. 
L'ensemble du calcul est \etpo sur l'entrée  
$(g,q) \in \ybo  \times \NN_1$.  \eop

\begin{fcorollary} \label{f435}
L'identité de  $\czu$  de  $\ckf$  vers  $\crf$  n'est pas calculable en temps 
\poll, au moins si  $\p \neq \np$.
\end{fcorollary}

\proof 
La fonction norme est calculable \etpo pour la \pres  $\crf$  d'après la 
proposition  \ref{f327}.  On conclut par la proposition précédente.  \eop 

\begin{fproposition} \label{f436}
Pour les cinq présentations précédentes de $\czu$, si l'évaluation est 
dans  $\DSPA(S(N))$  avec $S(N) \geq N$, alors la fonction norme est aussi dans 
$\DSPA(S(N))$.
\end{fproposition}

\proof 
Si   $k \mapsto \mu(k)$  est un module de continuité d'un point rationnel  
$\wi{f}$    d'une \pres donnée de $\czu$, alors pour calculer la norme 
avec une précision  $n$  il suffit d'évaluer  $\wi{f}$    sur les 
éléments de $\DD_{m,[0,1]}$  (où  $ m = \mu(n)$)  et prendre la valeur 
maximale. 
Puisque les résultats intermédiaires inutiles sont immédiatement 
effacés, et puisque  $S(N) \geq N \geq m$ l'espace de calcul de la fonction 
norme est le même que celui de la fonction d'évaluation.  \eop


\subsection{Complexité des \rps  étudiées}\label{fsubsec44}
Dans cette section, nous présentons rapidement la complexité du test 
d'appartenance et des opérations d'espace vectoriel, pour les cinq 
présentations considérées, et nous récapitulons l'ensemble des 
résultats obtenus. 

\begin{fproposition} \label{f441}
Le test d'appartenance (à l'ensemble des codes des points rationnels) est:\\
--- $\LINS$  et $\cnp$-complet pour les présentations  $\ckf$ et  $\cbo$;  \\
--- $\LINT $ pour la \pres $\csl$.
\end{fproposition}

\proof    Pour la \pres  $\csl$, c'est évident. Les preuves sont 
essentiellement les mêmes pour les deux présentations  $\ckf$  et  $\cbo$. 
Nous n'en donnons qu'une chaque fois.\\
Voyons que le test d'appartenance est  $\LINS$  pour  $\ykf$.  
Pour une entrée  $(Pr,n,m,T)$  on fait le calcul suivant: \\
 pour  $i = 1,\ldots,2^m$   vérifier que
\[
Pr(i/2^m) \in \DD_n \quad  \hbox{et} \quad  \Abs{ Pr((i-1)/2^m) - Pr(i/2^m) }\;  
\leq 1/2^n
\]
Ce calcul est $\LINS$.\\
Pour la $\cnp$-complétude du test d'appartenance, nous donnons la preuve pour 
les circuits booléens. On peut se limiter aux entrées  $(\gamma,2,m,0)$  
où  $\gamma$  est un circuit qui ne calcule qu'une sortie correspondant au 
premier bit, les deux autres bits sont nuls. On demande la cohérence sur deux 
points consécutifs de la grille. Seules les fonctions constantes sont donc 
tolérées.\\
Le problème opposé du test d'appartenance revient à savoir si un circuit 
booléen est non constant, ce qui implique la résolution du problème de 
satisfiabilité.  \eop

\begin{fproposition} \label{f442}
 Les opérations d'espace vectoriel (sur l'ensemble des points rationnels) sont 
dans $\LINT $ pour les cinq présentations  $\ckf$,  $\cbo$,  $\csl$,  $\caf$   
et   $\capo$  de  $\czu$.
\end{fproposition} 

\proof Les calculs sont évidents. Par exemple si 
$(f_1,\ldots,f_s) \in \lst(\ykf )$  et  $n \in \NN_1$, on peut calculer 
facilement 
$f \in \ykf $  tel que: \\
$$\NOrme{ \wi{f} - \sum_{i=1,\ldots,s} \wi{f_i} } \leq 
1/2^n$$ 
car il suffit de connaître chaque $\wi{f_i}$  avec la précision 
$1/2^{n+\log(s)}$.  \eop


Le seul ``drame'' est évidemment que les présentations $\ckf$ et $\cbo$ ne 
sont pas des $\p$-présentations de  $\czu$ (sauf si $\p = \np$  cf. la 
proposition \ref{f434}.) \\
Pour terminer cette section nous donnons un tableau récapitulatif dans lequel 
nous regroupons presque tous les résultats de \com établis pour les cinq 
\rps   $\ckf$, $\cbo$,  $\csl$,  $\caf$   et   $\capo$  de l'espace $\czu$.

\bni
\begin{tabular}{ c l p{3.5cm} c  c }  
& Evaluation & Fonction  & Test 
& Opérations\cr
&  & Norme & d'appartenance 
&  d'espace \cr
&&&&vectoriel\cr\cr
\ni $\ckf$  & $\DSRT(\Lin,\Lin,\Oo(N^2))$ & $\LINS$ et 
&  $\LINS$ et & $\LINT $\cr
\ni et  $\cbo$ &  &$\np$-complet  
&   $\cnp$-complet & \cr\cr
\ni $\csl$ & $\DSRT(\Oo(N^2),\Lin,\Oo(N^2))$ & $\DSPA(\Oo(N^2))$ 
& $\LINT $ & $\LINT $\cr
\ni  &  &et $\np$-complet\cr\cr
\ni $\caf$  & $\DSRT(\Oo(N^3),\Lin,\Oo(N^4))$ & 
$\DSPA(\Oo(N^3))$ 
& $\PSP$ & $\LINT $\cr
\ni et  $\capo$ &  &et $\np$-complet
\end{tabular}

\bni Pour le test d'appartenance $\PSP $ il est probable que sa complexité 
soit bien moindre.

\begin{fremark} \label{f444}
Malgré la facilité de calcul de la fonction d'évaluation pour les 
présentations $\ckf$  et   $\cbo$, c'est encore la \pres par circuits 
semilinéaires binaires qui semble au fond la plus simple. 
Sa considération a permis en outre d'éclairer le théorème de comparaison 
\ref{f4211}, qui est une version renforcée, uniforme, des résultats établis 
par Hoover. \\
Le défaut inévitable (si $\p \neq \np$) des présentations définies 
jusqu'à maintenant est la non faisabilité du calcul de la norme. 
Ceci empêche d'avoir une procédure de contrôle faisable pour les suites 
de Cauchy de points rationnels. 
Ceci diminue d'autant l'intérêt des  $\p$-points de  $\csl$. Cela souligne 
bien le fait qu'il est un peu artificiel d'étudier les  $\p$-points d'un 
espace qui est donné dans une \pres de \com non \polle.\\
En outre des problèmes a priori au moins aussi difficiles que la calcul de la 
norme, comme par exemple le calcul d'une primitive ou la solution d'une 
équation différentielle, sont également sans espoir de solution 
raisonnable dans le cadre des présentations que nous venons d'étudier.\\ 
Il est donc légitime de se tourner vers d'autres \rps  de l'espace  $\czu$  
pour voir dans quelle mesure elles sont mieux adaptées aux objectifs de 
l'analyse numérique.
\end{fremark}




\section{Quelques présentations de classe  \texorpdfstring{$\p$}{P} 
pour l'espace   \texorpdfstring{$\czu$}{C[0,1]}}\label{fsec5}
Dans cette section on aborde la question de savoir jusqu'à quel point des \rps  
dans la classe  $\p$  de l'espace  $\czu$  fournissent un cadre de travail 
adéquat pour l'analyse numérique. Il ne s'agit que d'une première étude, 
qui devrait être sérieusement développée.

\subsection {Définitions de quelques présentations de classe  
$\p$}\label{fsubsec51}

\subsubsection{Présentation  \texorpdfstring{$\cw$}{Cw}
 (à la Weierstrass)}\label{fsubsubsec511}
L'ensemble  $\yw$ des codes des points rationnels   de la \pres  $\cw$  est simplement l'ensemble  $\DD[X]$  des \pols (à une variable) à coefficients dans  $\DD$   donnés en \pres dense.

\smallskip Ainsi $\cw=(\yw,\eta,\delta)$ où la lectrice donnera précisément $\eta$ et $\delta$ conformément à la définition~\ref{f211}.


\smallskip Un  $\p$-point $f$ de  $\cw$  est donc donné par une suite $\p$-calculable:  
\[
m \mapsto u_m \; : \; \NN_1 \rightarrow \DD[X] \quad \hbox{avec} \quad 
\forall m \; \norme{ u_m - f }_{\infty} \leq 1/2^m.
\]

Et une  $\p$-suite  $f_n$ de  $\cw$  est donnée par une suite double  $\p$-calculable: 
\[
(n,m) \mapsto u_{n,m} \; : \; \NN_1 \times \NN_1 \rightarrow \DD[X] \quad \hbox{avec} \quad \; \forall n,m \; \norme{ u_{n,m} - f_n }_{\infty} 
\leq 1/2^m.
\]

\begin{fremark} \label{f511}
Une définition \equiva pour un $\p$-point $f$ de  $\cw$  est obtenue en 
demandant que $f$ s'écrive comme somme d'une série $\sum_m s_m$,
  où  $(s_m)_{m\in\NN_1}$  est une suite  $\p$-calculable dans~$\DD[X]$   vérifiant:  
$\norme{s_m}_{\infty} \leq 1/2^m$.  Ceci donne une manière 
agréable de présenter les  $\p$-points de~$\cw$. 
En effet on peut contrôler \etpo (par rapport à  $m$)  le fait que la suite est correcte pour les termes de $1$ à  $m$.  
En outre, dans l'optique d'un calcul paresseux, on peut contrôler la somme de série $\sum_m s_m$ au fur et à mesure que la précision requise augmente. 
Cette remarque est valable pour toute autre \rp  de classe  $\p$  alors qu'elle ne le serait pas pour les présentations étudiées dans la section \ref{fsec4}.
\end{fremark}

\smallskip Le résultat suivant est immédiat.

\begin{fproposition} \label{f512}
La \pres  $\cw$  de  $\czu$  est de classe  $\p$.
\end{fproposition}


Voici un résultat comparant les \rps   $\crf$  et $\cw$.  


\begin{fproposition} \label{f513}~\\
--- L'identité de  $\czu$  de  $\cw$  vers  $\crf$  est  $\LINT $.\\
--- L'identité de  $\czu$  de  $\crf$  vers  $\cw$  n'est pas de classe  $\p$.
\end{fproposition}

\proof La première affirmation est triviale. 
La seconde résulte du fait que la fonction  $x \mapsto \abs{ x-1/2} $  est 
un  $\p$-point de  $\crf$  (théorème \ref{f334}) tandis que tous les 
$\p$-points de  $\cw$  sont des fonctions infiniment dérivables 
(cf. ci-dessous le théorème \ref{f527}).  \eop 

\smallskip L'intérêt de la \pres   $\cw$  est notamment souligné par les 
théorèmes de caractérisation (cf. section \ref{fsubsec52}) 
qui précisent des phénomènes ``bien connus'' en analyse numérique, avec 
les \pols de Chebyshev comme méthode d'attaque des problèmes.



\subsubsection{Présentation   \texorpdfstring{$\csp$}{Csp}
(via des semi-\pols en \pres 
dense)}\label{fsubsubsec512}
Il s'agit d'une \pres qui augmente notablement l'ensemble des $\p$-points (par 
rapport à~$\cw$). Un élément de  $\ysp$   représente une fonction 
\polle par morceaux (ou encore un  semi-\pol) donné ``en \pres 
dense''. 

Plus précisément $\ysp \subset \lst(\DD) \times \lst(\DD[X])$, et les 
deux listes dans $\DD$ et $\DD[X]$ sont assujetties aux conditions 
suivantes:\\
--- la liste  $(x_i)_{0 \leq i \leq t}$ de nombres rationnels dyadiques est 
ordonnée par ordre croissant: 
$$0 = x_0 < x_1 < x_2 <\cdots< x_{t-1} < x_t =1\,;$$
--- la liste  $(P_i)_{1 \leq i \leq t}$ dans $\DD[X]$
vérifie  $P_i(x_i) = P_{i+1}(x_i)$  pour  $ 1 \leq i \leq t-1$.\\
Le code $f = ((x_i)_{0 \leq i \leq t},(P_i)_{1 \leq i \leq t})$ 
définit le point rationnel  $\wi{f}$: la fonction continue qui coïncide avec $\wi{P_i}$ sur l'intervalle $[x_{i-1},x_i]$. 

\smallskip Ainsi $\csp=(\ysp,\eta,\delta)$ où \smash{$\eta(f)=\wi f$} et le lecteur donnera  $\delta$ conformément à la définition~\ref{f211}.

\smallskip La \pres  $\csp$  de  $\czu$  est clairement de classe  $\p$.

La proposition suivante se démontre comme la proposition \ref{f513}.

\begin{fproposition} \label{f514} ~\\
--- L'identité de  $\czu$  de  $\cw$  vers  $\csp$  est  $\LINT $.\\
--- L'identité de  $\czu$  de  $\csp$  vers  $\cw$  n'est pas de classe  $\p$.
\end{fproposition} 

\subsubsection{Présentation $\csr$ (via des semi-fractions rationnelles 
contrôlées et données en \pres par formule)}\label{fsubsubsec513}
L'ensemble  $\ysr$ des codes des points rationnels  est maintenant un ensemble qui code des fonctions rationnelles par morceaux (ou semi-fractions rationnelles) à coefficients dyadiques et qui sont convenablement contrôlées. 

\smallskip Plus précisément, $\ysr  \subset \lst(\DD) \times \lst(\DD[X]_f \times \DD[X]_f)$,  et les deux listes dans $\DD$ et~$\DD[X]_f \times \DD[X]_f$  sont assujetties aux conditions suivantes:\\
--- la liste  $(x_i)_{0 \leq i \leq t}$ de nombres rationnels dyadiques est 
ordonnée par ordre croissant: 
$$0 = x_0 < x_1 < x_2 <\cdots< x_{t-1} < x_t =1;$$
--- chaque couple   $(P_i,Q_i) \; (1 \leq i \leq t$ de la 2ème liste 
représente une fraction rationnelle  $R_i = P_i/Q_i$    avec le 
dénominateur $Q_i$ minoré par  $1$  sur l'intervalle $[x_{i-1} , x_i]$;\\
--- la liste  $(R_i)_{1 \leq i \leq t}$  vérifie   $R_i(x_i) = R_{i+1}(x_i)$
   pour  $1 \leq i < t$.
\\
Le code  $f = ((x_i)_{0 \leq i \leq t},(P_i,Q_i)_{1 \leq i \leq t})$  
définit le point rationnel \smash{$\wi{f}$}: la fonction continue qui coïncide avec $\wi{R_i}$  sur chaque intervalle  $[x_{i-1} , x_i]$.

\smallskip Ainsi $\csr=(\ysr,\eta,\delta)$ avec $\eta(f)=\wi f$ et la lectrice donnera $\delta$ conformément à la définition~\ref{f211}.

\smallskip La \pres  $\csr=(\ysr,\eta,\delta)$  de  $\czu$  est clairement de classe  $\p$.

%%%%%%%%%%%%%%%%%%%%%%%%%%%%%%%%%%%%%%%%%%%%%%%%%%%%%%%%%%%%%%%%%%%%
\subsection {Résultats concernant la \pres à la Weierstrass}\label{fsubsec52}

Cette section est pour l'essentiel un développement de la section C-c de l'article \cite{fLo89} qui reprend la troisième partie la thèse du deuxième auteur.

Les références classiques de base pour la théorie de l'approximation sont
\cite[Bakhvalov, 1973]{fBa},
\cite[Cheney, 1966]{fCh} et \cite[Rivlin, 1974]{fRi}. Pour la classe de Gevrey on se réfère à \cite[Hörmander, 1983]{fHo}. Nous suivons les notations de \cite{fCh}.  

\smallskip Avant de caractériser les $\p$-points de $\cw$, il nous faut rappeler quelques 
résultats classiques de la théorie de l'approximation uniforme par des 
\pols.

\mni {\bf Attention~!} Vu la manière usuelle dont est formulée la théorie 
de l'approximation, on utilisera l'intervalle     $[-1,1]$ pour donner les 
résultats et les preuves concernant  $\cw$.


\subsubsection{Quelques définitions et résultats de la théorie de 
l'approximation uniforme par des \pols}\label{fsubsubsecf521}

Voir par exemple  \cite{fBa}, \cite{fRi}  et   \cite{fCh}. 

\begin{fnotation} \label{f521} ~ 
%-----------------begin item------------------
\begin{itemize}\itemsep2pt
\item  $\cab$ est l'espace des fonctions réelles continues sur le segment 
$[a,b]$.
\item  
$\C$ est l'espace  $\cuu$,  la norme uniforme sur cet intervalle est notée  
$\norme{ f }_{\infty}$   et  la distance correspondante $d_{\infty}$. 
\item   
$\C^{(k)}$  est l'espace des fonctions $k$ fois continûment dérivables sur 
$[-1,1]$. 
%
\item  
$\Ci$  est l'espace des fonctions  indéfiniment dérivables sur  $[-1,1]$.  
%
\item  
$\po_n$ est l'espace des \pols de degré  $\leq n$.  
%
\item  
$\Tch_n$  est le \pol de Chebyshev de degré $n$:  
\[
\Tch_n\big(\varphi (z)\big)= \varphi(z^n) \; \hbox{avec} \; \varphi(z) = \frac {1} {2} (z 
+ 1/z)
\]
on peut également les définir par    $\Tch_n\big(\cos (x)\big) = \cos (nx)$    ou par  
\[
F(u,x) = \frac {1-u\,x} {1-u^2-2u\,x} = \sum_{n=0}^{\infty} \Tch_n(x)u^n.
\]
%
\item  
On note: \fbox{$E_n(f) = d_{\infty} (f, \po_n)$}    pour  $f \in \C$.    
%
\item  On considère   sur $\C$  le produit scalaire  
\[
\left < g,h \right > := \int_{-1}^1 \frac {g(x)\,h(x)} {\sqrt {1-x^2}}\, dx 
= \int_0^{\pi} g\big(\cos (x)\big)\,h \big(\cos (x)\big)dx. 
\]
On notera  $\norme{ f }_2$  la norme au sens de ce produit scalaire.  
Les \pols  $(\Tch_i)_{0 \leq i \leq n}$   forment une base orthogonale de  
$\po_n$   pour ce produit scalaire, avec  
\[
\left < \Tch_0,\Tch_0 \right > = \pi \; \hbox{et} \; \left < \Tch_i,\Tch_i \right > = 
\pi /2 \; \hbox{pour} \; i>0.
\]
\item  On note
\[
A_k = A_k(f):= \frac {2} {\pi} \int_{-1}^{1}  \frac {f(x)\,\Tch_k(x)} {\sqrt {1-x^2}}\, {dx}= \frac {2} {\pi} \int_{0}^{\pi} \cos (kx) f(\cos (x))dx
\]
Les  $A_k$  sont appelés les  {\em coefficients de Chebyshev}  de  $f$.
\item  La fonction 
\[
s_n(f) := A_0/2 + \sum_{i=1}^{n}A_i\Tch_i, \quad   \hbox{ aussi notée } 
 {\sum_{i=0}^n}\,'\, A_i\Tch_i
\]  
est la projection orthogonale de $f$ sur $\po_n$ au sens du produit scalaire considéré.    
\item  
La série correspondante est appelée  {\em la série de Chebyshev}  de 
$f$~{\footnote{Elle converge au sens de  $L^2$  pour le produit scalaire 
considéré.  Les séries de Chebyshev sont aux fonctions continues sur  
$[-1,1]$  ce que les séries de Fourier sont aux fonctions continues 
périodiques, ce qui se comprend bien en considérant le ``changement de 
variable''  $z\mapsto 1/2(z + 1/z)$   qui transforme le cercle unité du plan complexe en le segment   $[-1,1]$  et la fonction  $z\mapsto z^n$  en le \pol~$\Tch_n.$}}. 
\item  
On note  \fbox{$S_n(f) = \norme{ f - s_n(f) }_{\infty}$}.        
On a immédiatement   $\abs{ A_{n+1}(f) }\;  \leq S_n(f) + S_{n+1}(f)$. 
\item  
Les  zéros de  $\Tch_n$   sont les 
\[
\xi_i^{[n]} = \cos \left(\frac{2i-1} {n}\cdot \frac {\pi} {2}\right) \; \; \; i =1,\ldots,n
\]
et l'on a  
\[
\Tch_n(x) = 2^{n-1} \prod_{i=1}^{n} (x - \xi_i^{[n]}) \; \; (  
\hbox{pour}\; n \geq 1).
\]          
\item  
Les  extrema de  $\Tch_n$  sur  $[-1,1]$  sont égaux à $\pm 1$ et obtenus aux points
\[
\eta_i^{[n]} = \cos \left(\frac{i} {n}\cdot \pi\right) \; \; \; i=0,\ldots,n.
\] 
\item  
Des valeurs approchées de  $s_n(f)$ peuvent être calculées au moyen de 
formules d'interpolation:  on pose   
\[
\alpha_k^{[m]} = \frac {2} {m} \;
{\sum_{i=0}^m}\,'f(\xi_i^{[m]})\Tch_k(\xi_i^{[m]}), \; \; \; \; u_n^{[m]} = 
\sum_{k=1}^{n} \alpha_k^{[m]}\Tch_k(x)
\]
et l'on a:  $u_n^{[n+1]}$   est le \pol qui interpole $f$ aux zéros de  
$\Tch_{n}$
\end{itemize}
%-----------------end item------------------
\end{fnotation}
La théorie de l'approximation uniforme par des \pols établit des liens 
étroits entre ``être suffisamment bien approchable par des \pols''  et  
``être suffisamment régulière''.


\subsubsection{Quelques résultats classiques}\label{fsubsubsec 522}

Ces résultats se trouvent pour l'essentiel dans \cite{fCh}.

\smallskip  Dans ce paragraphe les fonctions sont dans $\C=\cuu$.

\smallskip \noindent {\bf Évaluation d'un \pol} $P=\sum_{k=0}^{n}a_k\Tch_k$\\
 Les formules récurrentes  $\Tch_{m+1}(x) = 2x\Tch_m(x) - \Tch_{m-1}(x)$  conduisent à un algorithme à la Horner:
\[
B_{n+1} = B_{n+2} = 0, \quad  B_k = 2xB_{k+1} - B_{k+2} + a_k, \quad  P(x) = \frac 
{B_0 - B_2} {2}.
\]

\smallskip\noindent {\bf Inégalités de Markov} \\
Si  $g \in \po_n$ alors (A.A. Markov, \cite[page 91]{fCh})    
%--------------------begin equation---------------
\begin{equation} \label{fF 5.2.1}
\Norme{ g' }_{\infty} \leq n^2 \norme{ g }_{\infty}
\end{equation}
%---------------------end equation--------------
et pour  $k \geq 2$ (V.A. Markov, \cite[Theorem 2.24]{fRi})   
%--------------------begin equation---------------
\begin{equation} \label{fF 5.2.2}
\Norme{ g^{(k)} }_{\infty} \leq \Tch_n^{(k)}(1) \norme{ g 
}_{\infty} = \frac {n^2(n^2-1)\cdots(n^2-(k-1)^2)} {1.3.5\cdots(2k-1)} 
\,\norme{ g }_{\infty}
\end{equation}
%---------------------end equation--------------


\smallskip\noindent {\bf Comparaison de $E_n(f)$ et $S_n(f)$}
%--------------------begin equation---------------
\begin{equation} \label{fF 5.2.3}
E_n(f) \leq S_n(f) \leq \left(4+ \frac {4} {\pi^{2}} \log (n)\right)\,E_n(f)
\end{equation}
%---------------------end equation--------------

\smallskip\noindent {\bf  Comparaison de $E_n(f)$ et $A_{n+1}(f)$}\\
Pour  $n \geq  1$  on a	
\[
\int_{-1}^{1} \frac {\abs{ \Tch_n(x) }} {\sqrt {1-x^2}}\,dx = 2
\]
d'où l'on déduit   
%--------------------begin equation---------------
\begin{equation} \label{fF 5.2.4}
(\pi /4) \abS{A_{n+1}(f)}   \leq E_n(f)
\end{equation}
%---------------------end equation--------------

\smallskip\noindent{\bf Théorèmes de Jackson}\\
Soit $f \in \C$. Pour tout entier $n \geq 1$ on a 
%--------------------begin equation---------------
\begin{equation} \label{fF 5.2.5}
E_n(f) \leq \pi \lambda /(2n+2) \; \; \; \hbox{si} \; \; \; \abs{f(x)-f(y)}\;  \leq \lambda \abs{x-y}
\end{equation}
%---------------------end equation--------------
%--------------------begin equation---------------
\begin{equation} \label{fF 5.2.6}
E_n(f) \leq (\pi /2)^k \Norme{ f^{(k)} }_{\infty} \big/ \big((n+1)(n)(n-
1)\cdots(n-k+2)\big) \; \;\; \hbox{si} \; f \in \C^{(k)} \; \hbox{et} \; n\geq k
\end{equation}
%---------------------end equation--------------

\smallskip\noindent{\bf Convergence de la série de Chebyshev d'une fonction} 
\\
La série de Chebyshev d'une fonction  $f \in \C^{(k)}$  converge \uni  vers 
$f$ si  $k \geq 1$,   et elle est absolument convergente (pour la norme  
$\norme{ f }_{\infty}$)  si  $k \geq 2$.   
%--------------------begin equation---------------
\begin{equation} \label{fF 5.2.7}
S_n(f) = \norme{ s_n(f)-f }_{\infty}\; \leq \sum_{j=n+1}^{\infty} 
\abS{A_j}
\end{equation}
%---------------------end equation--------------
et (cf. \cite{fRi} Theorem 3.12 p. 182)
%--------------------begin equation---------------
\begin{equation} \label{fF 5.2.8}
\NOrme{ s_n(f)- u_n^{[n+1]} }_{\infty}\; \leq \sum_{j=n+2}^{\infty} 
\abS{A_j}
\end{equation}
%---------------------end equation--------------

\smallskip\noindent{\bf Approximation uniforme des fonctions dans
 $\Ci$  par des \pols} \\
Les propriétés suivantes sont équivalentes. 
%-----------------begin item------------------
\begin{itemize}\itemsep2pt
\item [(i)]  $\forall k \; \; \exists M > 0 \; \; \forall n> 0 \; \; E_n(f) \leq M/n^k$; 
\item [(ii)]
 $\forall k \; \; \exists M > 0 \; \; \forall n> 0 \; \; S_n(f) \leq M/n^k$;
\item [(iii)] 
$\forall k \; \; \exists M > 0 \; \; \forall n> 0 \; \; \abS{A_n(f)}  \leq 
M/n^k$;
\item [(iv)] $\forall k \; \exists M > 0 \; \forall n> 0 \; \NOrme{ u_n^{[n+1]} - f }_{\infty} \leq M/n^k$;
\item [(v)] 
La fonction  $f$ est de classe $\ca^{\infty}$ (i.e., $f \in \Ci$).
\end{itemize}
%-----------------end item------------------

\proof 
(i)  et  (ii)  sont équivalents d'après  (\ref{fF 5.2.3}). \\   
(iv)$\Rightarrow$ (i)  trivialement.  \\  
(ii) $\Rightarrow$ (iii)  parce que  $\abS{A_n(f)}\;  \leq S_n(f) + S_{n-
1}(f)$.  \\
(iii) $\Rightarrow$ (iv)  d'après  (\ref{fF 5.2.7})  et  (\ref{fF 5.2.8}).  \\  
(iii) $\Rightarrow$ (v): la série  $\sum' A_i\Tch_i^{(h)}$  est  absolument 
convergente d'après  (\ref{fF 5.2.2})  et les majorations  (iii)~;  donc on 
peut dériver $h$ fois terme à terme la série de Chebyshev.    \\ 
(v) $\Rightarrow$ (i)  d'après  (\ref{fF 5.2.6}).  \eop


\smallskip\noindent{\bf Analyticité et approximation uniforme par des 
\pols} \\
Les propriétés suivantes sont \equivas:   
%-----------------begin item------------------
\begin{itemize}\itemsep2pt
\item [(i)] $ \exists M > 0, \; \; r < 1 \; \; \forall n> 0, \; \; E_n(f) \leq Mr^n$;

\item [(ii)] $\exists M > 0, \; \; r < 1 \; \; \forall n> 0, \; \; S_n(f) \leq Mr^n$; 

\item [(iii)] $\exists M >0, \; \; r < 1 \; \forall n> 0, \; \; \abS{A_n(f)} \leq Mr^n $;

\item [(iv)] $ \exists M > 0, \;  r < 1 \; \; \forall n> 0,  \; \NOrme{ u_n^{[n+1]} - f }_{\infty}   \leq Mr^n$;

\item [(v)] $\exists r < 1$ 
telle que  $f$ est analytique dans le plan complexe à l'intérieur                      
de l'ellipse  $\sE_\rho$  de foyers  $1$, $-1$  et dont le demi-somme des                      
diamètres principaux est égale  à  $\rho = 1/r$;

\item [(vi)] $ \exists M > 0, \;  R > 1 \; \; \forall n, \;  \NOrme{f^{(n)} }_{\infty}  \leq MR^nn!$\,.

\item [(vii)]  $f$ est analytique sur l'intervalle $[-1,1]$.
\end{itemize}
%-----------------end item------------------
En outre la limite inférieure des valeurs de  $r$  possibles est la même 
dans les  5 premiers cas{\footnote{Les équivalences   (i) \ldots (iv)  se 
démontrent comme pour la proposition précédente.  Pour l'équivalence avec  
(v)  voir par exemple \cite{fRi}.  La condition  (vi)   représente à très 
peu près l'analyticité dans l'ouvert  $U_R$  formé des points dont la 
distance à l'intervalle est inférieure à  $1/R$.}}.

\begin{figure}[htbp]  
\begin{center}
\includegraphics*[width=10cm]{fi521}
\end{center}
% \epsfxsize=10cm
% $$\epsfbox{f521.eps}$$
\caption[L'ellipse $\sE_\rho$]{\label{ffi521}  
L'ellipse $\sE_\rho$}  
\end{figure}  

\begin{fremarks}\label{frem-anal} ~\\
1) L'espace des fonctions analytiques sur un intervalle compact possède donc 
une bonne description constructive, en termes de série de Chebyshev par 
exemple.  Il apparaît comme une réunion dénombrable emboîtée 
d'espaces métriques complets  (ceux obtenus en utilisant la définition  
(iii)  et en fixant  $M$  et  $r$  rationnels par exemple).  
L'espace des fonctions  $\Ci$  est beaucoup plus difficile à décrire 
constructivement, essentiellement parce qu'il n'existe pas de manière 
agréable d'engendrer les suites de rationnels à 
décroissance rapide{\footnote{Cela tient au  $\forall k  \; \exists M $  dans 
la définition de la décroissance rapide.  
Cette alternance de quantificateurs prend une forme explicite lorsqu'on donne  
$M$  en fonction de  $k$  explicitement.  Mais, en vertu de l'argument diagonal 
de Cantor, il n'y a pas de manière effective d'engendrer les fonctions 
effectives de $\NN$   vers $\NN$.}}. \\
2) La condition  (i)  peut être également lue comme suit:  
la fonction $f$ peut être approchée à  $1/2^n$  (pour la norme uniforme)  
par un \pol de degré $\leq  c.n$,  où  $c$  est une constante 
fixée,  c.-à-d. encore: il existe un entier $h$ tel que
  $E_{hn}(f) \leq 1/2^n$. \\
   Même remarque pour les conditions  (ii),  (iii)  et  (iv). Cela implique 
que la fonction $f$ peut être approchée à  $1/2^n$  par un \pol à 
coefficients dyadiques dont la taille (en \pres dense sur la base des  $X^n$  ou 
sur la base des  $\Tch_n$)  est   en  $\Oo(n^2)$.  La taille de la somme des 
valeurs absolues des coefficients est, elle, en  $\Oo(n)$.    
Bakhvalov  (cf.  \cite {fBa}  IV-8  Th. p. 233)  donne une condition 
suffisante du même genre pour qu'une fonction $f$ soit analytique dans une 
lentille d'extrémités   $-1$  et  $1$  du plan complexe  (et non plus dans 
un voisinage du segment):  il suffit que la somme des valeurs absolues des 
coefficients d'un \pol donnant $f$ à  $1/2^n$  près soit majorée par  
$M2^{qn}$  (où~$M$ et $q$  sont des constantes fixées). C.-à-d. encore: 
la taille de la somme des valeurs absolues des coefficients d'un \pol 
approchant $f$ à  $1/2^n$  est en  $\Oo(n)$.
\begin{figure}[htbp]  
\begin{center}
\includegraphics*[width=12cm]{fi522}
\end{center}
% \epsfxsize=12cm
% $$\epsfbox{f522.eps}$$
\caption[La lentille de Bakhvalov]{\label{ffi522}  
La lentille de Bakhvalov}  
\end{figure}  

\end{fremarks}



\smallskip\noindent{\bf Classe de Gevrey et approximation uniforme par des 
\pols} \\
Si $f$ est un  $\p$-point de  $\cw$   donné par une suite 
$\; m \mapsto P_m\; $
 $\p$-calculable (avec $\norme{ f - P_m}_{\infty} \leq 2^m$), alors 
le degré de $P_m$  
est majoré par un \pol en  $m$,  donc il existe un entier $ k $ et une 
constante~$ B $ telles que le degré de $ P_m $ soit majoré par $(Bm)^k$. 
Soit alors $ n $ arbitraire, et considérons le plus grand entier $ m $ tel que  
$(Bm)^k \leq n$,  c.-à-d.  $m := \Flo{\sqrt[k]{n}/B}$. 
On a donc  $m+1 \geq \sqrt[k]{n}/B$.  En posant  $r := 1/2^{1/B}$
  et   $\gamma := 1/k$,  on obtient:  
$$E_n(f) \leq 1/2^m \leq 2.r^{n^{\gamma}}, \; \hbox{avec} \; r \in ]0,1[, \; 
\gamma > 0.$$
En particulier, la suite  $E_n(f)$  est à décroissance rapide et $f\in\Ci$.
Ceci nous amène à étudier les fonctions $f$ pour lesquelles ce genre de 
majoration est obtenu.

\begin{fdefinition}[classe de Gevrey{\footnote{Cf. par exemple  Hörmander \cite{fHo}: The 
Analysis of Linear Partial Differential Operators  I  p 281  (Springer 1983).  
Une fonction est Gevrey d'ordre $ 1 $ si et seulement si elle est 
analytique.}}] \label{f523}

Une fonction $f\in \Ci$ est dite dans la classe de Gevrey d'ordre $\alpha > 0 $ 
si ses dérivées vérifient une majoration:   
\[
\NOrme{f^{(n)} }_{\infty}\leq  MR^n n^{\alpha n}.
\]
La classe de Gevrey est obtenue lorsqu'on ne précise pas l'ordre  $\alpha$. 
\end{fdefinition}

\begin{ftheorem} \label{f524}
Soit $f \in \C$.  Les propriétés suivantes sont équivalentes. 
%-----------------begin item------------------
\begin{itemize}\itemsep2pt
\item [(i)] 
$ \exists M > 0, \; \; r < 1, \; \; \gamma > 0 \; \; 
\forall n> 0, \; \; E_n(f) \leq Mr^{n^{\gamma}}$,
\item [(ii)] 
$ \exists M > 0, \; \; r < 1, \; \; \gamma > 0 \; \; 
\forall n> 0, \; \; S_n(f) \leq Mr^{n^{\gamma}}$,
\item [(iii)] 
$ \exists M > 0, \; \; r < 1, \; \; \gamma > 0 \; \; 
\forall n> 0, \; \; \abS{A_n(f)}\;  
 \leq Mr^{n^{\gamma}}$, 
\item [(iv)] 
$ \exists M > 0, \; r < 1 \; \forall n> 0, \; \NOrme{ u_n^{[n+1]} - f }_{\infty} \leq Mr^{n^{\gamma}}$,
\item [(j)] 
$ \exists c, \beta > 0 \; \forall n> 0,\; \forall m \geq cn^{\beta} 
\; E_m(f) \leq 1/2^n $,
\item [(jj)]
$ \exists c, \beta > 0 \; \forall n> 0,\; \forall m \geq cn^{\beta} \; S_m(f) \leq 1/2^n $,
\item [(jjj)] 
$ \exists c, \beta > 0 \; \forall n> 0,\; \forall m \geq cn^{\beta} \; 
\abS{A_m(f)}\;  \leq 1/2^n $,
\item [(jw)] 
$ \exists c, \beta > 0 \; \forall n> 0,\;\forall m \geq cn^{\beta} \NOrme{ u_m^{[m+1]} - f }_{\infty} \leq 1/2^n $,
\item [(k)]  $f$ est dans la classe de Gevrey.
\end{itemize}
%-----------------end item------------------
\end{ftheorem}

\proof
(i) $\Leftrightarrow$ (ii)  à partir de l'équation (\ref{fF 5.2.3}). \\  
(i)  $\Rightarrow$ (iii) à partir de l'équation (\ref{fF 5.2.4}). \\ 
(iv)  $\Rightarrow$ (i)  est triviale.\\  
Les  4  équivalences du type (i) $\Leftrightarrow$ (j)  résultent du même 
genre de calcul que celui qui a été fait avant le théorème.\\
L'implication  (jjj) $\Rightarrow$ (jw)  résulte d'un calcul de majoration 
simple utilisant  les inégalités (\ref{fF 5.2.7}) et (\ref{fF 5.2.8}). \\
Supposons  (k), c.-à-d. que $f$ soit Gevrey d'ordre  $\alpha$, et démontrons 
(i). 
Le problème de majoration n'est délicat que pour  $\alpha \geq 1$,
 ce qu'on supposera maintenant. En appliquant le théorème de Jackson, on 
obtient une majoration  
$E_n(f) \leq \pi^{k} \Norme{f^{(k)} }_{\infty} /n^k$
  dès que  $n \geq 2k$,  ce qui donne avec la majoration de Gevrey    
$E_n(f) \leq A(Ck^{\alpha}/n)^k$.  
On peut supposer  $C^{1/ \alpha} \geq 2$  et on prend pour  $k$  un entier 
proche de  $(n/2C)^{1/ \alpha}$ ($ \leq n/2$),
 d'où à très peu près:
$$E_n(f) \leq A(1/2)^{(n/2C)^{1/ \alpha}} = Ar^{n^{\gamma}}, \; \hbox{avec} \; 
\gamma = 1/ \alpha.$$
Supposons maintenant que $f$ vérifie (j)  et démontrons que $f$ est Gevrey.  \\
Le problème de majoration n'est délicat que pour  $\beta \geq 1$, ce qu'on 
supposera maintenant. On écrit  $f^{(k)} = \sum' A_m \Tch_m^{(k)}$.   
D'où   $\Norme{f^{(k)} }_{\infty} \leq \sum' \abS{A_m} m^{2k}$ d'après l'inégalité de V.A. Markov (\ref{fF 5.2.2}).  
On utilise maintenant la majoration  (jjj). On prend $c$ et $\beta$ 
entiers pour simplifier (ce n'est pas une restriction). Dans la somme ci-dessus, 
on regroupe les termes pour $ m $ compris entre $cn^{\beta}$ et 
$c(n+1)^{\beta}$. Dans le paquet obtenu, on majore chaque terme par $ (1/2^n) 
m^2k$, et on majore le nombre de termes par $ c(n+1)b,$ d'où:
\[
\Norme{f^{(k)} }_\infty \leq \sum_n (c(n+1)^{\beta}/2^n)(c(n+1)^{\beta})^{2k} \leq 2c^{2k+1} \sum_n (n+1)^{\beta (2k+1)}/2^n
\]
\[
\leq 4c^{2k+1} \sum_n n^h/2^n, \; \hbox{où} \; h = \beta(2k+1).
\] 
On majore cette série par la série obtenue en dérivant $h$ fois  la 
série  $\sum_n x^n$  (puis en faisant  $x=1/2$)  et on obtient que $f$ est 
Gevrey d'ordre  $2\beta$.  \eop


\begin{fremarks} \label{f525}~\\
1)  L'espace des fonctions Gevrey possède donc une \pres constructive 
agréable.\\
2)  Pour  $\gamma = 1$ on obtient les fonctions analytiques. Pour  $\gamma > 1$, 
on obtient des fonctions entières.\\
3)  Pour $\gamma \leq 1$, la limite supérieure des $ \gamma $ possibles est la 
même dans  (i),  (ii),  (iii)  et  (iv),  la limite inférieure des $\beta$ 
possibles est la même dans  (j),  (jj),  (jjj)  et  (jw),  avec
$\gamma = 1/ \beta$. \\
4)  Si on se base sur le cas des fonctions analytiques  
($\alpha = \beta = \gamma = 1$), 
on peut espérer, pour l'implication  (j) $\Rightarrow$ (k),
 obtenir que $f$ soit  Gevrey d'ordre $\beta$  au moyen d'un calcul de 
majoration plus sophistiqué.\\
5)  Dans  (j),  (jj),  (jw)  on peut supprimer le quantificateur $\forall m$   
si on prend $ c $ et $\beta$ entiers  et $m = cn^{\beta}$.
\end{fremarks} 


\subsubsection{Retour aux questions de \com dans l'espace  $\cw$}
\label{fsubsubsec 523}
Nous commençons par une remarque importante.

\smallskip \noindent {\bf  Remarque importante.}  En ce qui concerne  $\DD[X]$, la \pres dense ordinaire  (sur la base des  $X^n$)  et la \pres dense sur la base des \pols de Chebyshev   $\Tch_n$, sont \equivas en temps \poll. Nous utiliserons indifféremment l'une ou l'autre des deux bases, selon la commodité du moment.

\medskip  Rappelons également que la norme 
$\; P \to \norme{ P }_{\infty}\; $ 
 est une fonction $\p$-calculable de $\DD[X]$  vers $\RR$.

La preuve de la proposition suivante est immédiate. En fait toute 
fonctionnelle définie sur  $\cw$  qui a un \mcu \poll et dont la 
restriction à  $\DD[X]$  est ``facile à calculer'' est elle même ``facile 
à calculer''. Cette proposition prend toute sa valeur au vu du théorème de 
caractérisation \ref{f527}. 

\begin{fproposition}[bon comportement des fonctionnelles usuelles] \label{f526}~ \\ 
Les fonctionnelles:
\[
\cw \to \RR \quad  f \mapsto \norme{f}_{\infty}, \; \norme{f}_2, \; \norme{f}_1
\]
  sont  \uni de classe  $\p$.\\
Les fonctionnelles:
\[
\cw \times [0,1] \times [0,1] \to \RR \quad  (f,a,b) \mapsto \sup_{x 
\in [a,b]} (f(x)), \; \int_{a}^{b} f(x) dx
\]
sont  \uni de classe  $\p$.
\end{fproposition}

\begin{ftheorem}[caractérisation des  $\p$-points de  $\cw$] \label{f527}
~\\
Soit   $f \in \C$.  Les propriétés suivantes sont équivalentes.
\begin{itemize}\itemsep2pt
%
\item [a)]  La fonction $f$  est  un $\p$-point de  $\ckf$  et est dans la classe de Gevrey.
%
\item  [b)]  La suite   $A_n(f)$   est une  $\p$-suite dans  $\RR$   et vérifie une majoration     $\abS{A_n(f)}   \leq Mr^{n^{\gamma}}$
avec   $M > 0$, $\gamma > 0$ et $0 < r < 1$.
%
\item  [c)]  La fonction $f$  est un  $\p$-point de  $\cw$.
\end{itemize}

\end{ftheorem}

\proof  Les implications  (c) $\Rightarrow$ (a)  et   (c)   $\Rightarrow$  (b) sont faciles à partir du théorème \ref{f524}. \\   
(b)  $\Rightarrow$  (c). Un \pol (en \pres dense sur la base des $\Tch_n$)  approchant $f$  avec la précision  $1/2^{n+1}$  
est obtenu avec la somme partielle extraite de la série de Chebyshev de $f$ en s'arrêtant à l'indice  $(Bn)^h$   (où  $B$  et $h$ se calculent à partir 
de  $M$ et $\gamma$).  
Il reste à remplacer chaque coefficient de Chebyshev par un dyadique l'approchant avec la précision: 
\[
1/ \Flo{(Bn)^h2^{n+1}}  = 1/2^{n+1+h \log(Bn)}.
\]
(a)  $\Rightarrow$  (c).  Un \pol  approchant  $f$  avec la précision  
$1/2^{n+1}$  est obtenu avec   $u_m^{[m+1]}$  (où  $m = (Cn)^k$),
$C$ et $k$  se calculent à partir de  $M$ et $\gamma$, en tenant compte des inégalités  (\ref{fF 5.2.7}) et (\ref{fF 5.2.8})).   
La formule définissant  $u_m^{[m+1]}$   fournit ses coefficients sur la base des  $\Tch_n$  et on peut calculer (en temps \poll)  une approximation à  $1/2^{n+1+k \log(Cn)}$  près de ces coefficients en profitant du fait que la suite double $\xi_i^{[n]}$ est une  $\p$-suite de réels et que la fonction $f$ est  un $\p$-point de  $\ckf$.  \eop

\smallskip Une conséquence immédiate du théorème précédent est obtenue dans le 
cas des fonctions analytiques.

\begin{ftheorem} \label{f528}
Soit   $f \in \C$.  Les propriétés suivantes sont équivalentes. 
%-----------------begin item------------------
\begin{itemize}

\item [(a)] La fonction $f$ est une fonction analytique et c'est un $\p$-point 
de  $\ckf$.

\item [(b)] La suite  $A_n(f)$  est une  $\p$-suite dans  $\RR$  et vérifie 
une majoration
$$ \abS{A_n(f)}\;  \leq Mr^n \; (M > 0, \; r < 1).$$

\item [(c)] La fonction $f$ est une fonction analytique et est un  $\p$-point de $\cw$.
\end{itemize}
%-----------------end item------------------
\end{ftheorem}

\begin{fdefinition}[fonctions $\p$-analytiques]  \label{f529} 
~\\ 
Lorsque ces propriétés sont vérifiées, on dira que la fonction $f$ est  
$\p$-analytique sur l'intervalle  $[-1,1]$.
 \end{fdefinition}

\begin{ftheorem}[assez bon comportement de la dérivation vis à vis de la \com] \label{f5210} ~\\
Soit $\wi f$ un  $\p$-point de  $\cw$.  Alors  la suite $k \mapsto \wi{f^{(k)}}$   est  une  $\p$-suite de  $\cw$.  
Plus généralement, si $\big(\wi{f_n}\big)$ est une $\p$-suite de  $\cw$  alors la suite 
double $\bigg(\wi{f_n^{(k)}}\bigg)$ est une $\p$-suite de  $\cw$.
\end{ftheorem}

\proof Nous donnons la preuve pour la première partie de la proposition. Elle 
s'appliquerait sans changement pour le cas d'une $\p$-suite de $\cw$. \\
La fonction $\wi f$ est un $\p$-point de $\cw$ donné comme limite d'une suite   $\p$-calculable $n \mapsto P_n$. 
La suite double  $P_n^{(k)}$  est  $\p$-calculable  ($n,k\in\NN_1$).  
Il existe deux entiers  $a$  et  $b$  tels que le degré de~$P_n$ soit majoré par  $2^a n^b$.  
Donc,  d'après l'inégalité de V.A. Markov  (\ref{fF 5.2.2})  on a la majoration:
\[
\NOrme{ P_n^{(k)} - P_{n-1}^{(k)} }_\infty \leq (2^an^{2b})^k \Norme{ P_n - P_{n-1} }_{\infty} \leq (2^an^{2b})^k /2^{n-2} = 1/2^{n-(k.(a+2b \log(n))+2)}
\]
On détermine alors aisément une constante  $n_0$  telle que, pour 
$n \geq 2n_0k$, on ait:   $$n \geq 2(k.(a+2b \log(n))+2)$$ 
et donc  
\[
\NOrme{ P_n^{(k)} - P_{n-1}^{(k)} }_\infty \leq 1/2^{n/2}
\] 
de sorte qu'en posant  $\nu(n) := 2 \sup(n_0k,n)$,  on a, pour 
$q \geq \nu(n),$  
\[
\NOrme{ P_{\nu(n)}^{(k)} - P_{\nu(n+1)}^{(k)} }_\infty \leq 1/2^{n-1}
\]
et donc, puisque  
$\nu(n+1) = \nu(n)$ ou  $\nu(n)+2,$ 
\[
\NOrme{ P_{\nu(n)}^{(k)} - P_{\nu(n+1)}^{(k)} }_\infty \leq 1/2^{n-1}
\]
 d'où enfin:   
\[
\NOrme{ P_{\nu(n)}^{(k)} - f^{(k)} }_\infty \leq 1/2^{n-2}
\]
On termine en notant que la suite double  $(n,k) \mapsto P_{\nu(n+2)}^{(k)}$  
est  $\p$-calculable.    \eop

\begin{fcorollary} \label{f5211}
Si $f$ est un  $\p$-point de  $\cw$   et   $a, b$   deux  $\p$-points de  
$[-1,1]$, alors les suites  
\[
\Norme{f^{(n)} }_{\infty}, \; \Norme{f^{(n)} }_2, \; \Norme{f^{(n)} }_1, \; \Norme{f^{(n)}(a)} \;\hbox{et} \; \sup\nolimits_{x \in [a,b]} (f^{(n)}(x))
\]
sont des $\p$-suites dans  $\RR$.\\
Plus généralement, si $(f_p)$ est une $\p$-suite de  $\cw$  alors les suites 
doubles  
\[
\norme{f_p^{(n)}}_\infty, \; \norme{f_p^{(n)} }_2, \; \norme{f_p^{(n)} }_1, \; \norme{f_p^{(n)}(a)} \; \hbox{et} \; \sup\nolimits_{x \in [a,b]} (f_p^{(n)}(x))
\]
sont des $\p$-suites dans  $\RR$.
\end{fcorollary}

La preuve du théorème \ref{f5210} (et donc du corollaire \ref{f5211}) est en 
quelque sorte uniforme et a une signification plus générale. Nous allons 
maintenant définir le cadre naturel dans lequel s'applique ce théorème et donner un nouvel énoncé, plus général et plus satisfaisant.

\begin{fdefinition} \label{f5212}
Pour  $c$ et $\beta > 0$  on note $\Gv_{c, \beta}$ la classe des fonctions 
Gevrey vérifiant la majoration (du type (jjj) dans \ref{f524})
$$ \forall m > cn^{\beta} \quad  \abS{A_m(f)}\;  \leq 1/2^n$$
C'est une partie convexe fermée de $\C$.   
Pour  $c$ et $\beta$  entiers, on note  $\Y_{\Gv_{c, \beta}}$  les éléments de  $\DD[X]$  qui sont dans la classe  $\Gv_{c, \beta}$. 
Cet ensemble $\Y_{\Gv_{c, \beta}}$  peut être pris 
pour ensemble des codes des points rationnels d'une \pres rationnelle $\sC_{\Gv, c, \beta}$  de $\Gv_{c, \beta}$.
\end{fdefinition}

On notera que le test d'appartenance à la partie $\Y_{\Gv_{c, \beta}}$  de  
$\DD[X]$   est en temps \poll, puisque les $A_m(f)$ pour un \pol $f$ 
sont ses coefficients sur la base de Chebyshev.
Dans ce nouveau cadre, le théorème \ref{f528} admet une formulation plus 
uniforme et plus efficace.

\begin{ftheorem} \label{f5213}
Chaque fonctionnelle $f \mapsto f^{(k)}$  est une fonction \uni de classe  $\p$   
de $\sC_{\Gv_{c, \beta}}$ vers  $\cw$.   Plus précisément la suite de 
fonctionnelles 
$$ (k,f) \mapsto f^{(k)} \; : \; \NN_1 \times \sC_{\Gv_{c, \beta}} \rightarrow 
\cw $$
est \uni de classe  $\p$   (au sens de la définition \ref{f229}).
\end{ftheorem}

\proof 
La suite double   $(k,f) \mapsto f^{(k)}$   est de faible \com en tant que 
fonction de  $\NN_1 \times  \DD[X]$   
vers  $\DD[X]$  donc aussi en tant que fonction de 
 $\NN_1 \times \Y_{\Gv, \beta}$
  vers  $\DD[X]$. \\
Tout le problème est donc de démontrer que l'on a un \mcu \poll (au sens de 
\ref{f229}).  
Nous devons calculer une fonction  $\mu(k,h)$  telle que l'on ait pour tous $f$ 
et $g$  dans  $\Y_{\Gv, \beta}$:
\[
\norme{ f - g }_\infty \leq 1/2^{\mu (k,h)} \Rightarrow \NOrme{f^{(k)} - g^{(k)} }_\infty \leq 1/2^h. 
\] 
Ce calcul de majoration est assez proche de celui qui a été fait dans la 
preuve du théorème \ref{f528}. On écrit   
\[
\NOrme{f^{(k)} - g^{(k)} }_\infty \leq \NOrme{f^{(k)} - s_n(f)^{(k)} }_{\infty} + \NOrme{ g^{(k)} - s_n(g)^{(k)} }_{\infty} + \NOrme{ s_n(f-g)^{(k)} }_{\infty}.
\]
Dans la somme du second membre les deux premiers termes sont majorés comme 
suit
\[
\NOrme{f^{(k)} - s_n(f)^{(k)} }_{\infty} \;\leq\;\sum_{q>n} \abS{A_q(f)} \NOrme{ T_q^{(k)} }_\infty \;\leq\; \sum_{q>n} \abS{A_q(f)} q^{2k}.
\]
Comme on a:  $\forall q > cn^{\beta} \; \abS{A_q(f)}\;  \leq 1/2^n$, 
$\sum_{q>n} \abS{A_q(f)} q^{2k}$ 
est ``bien'' convergente et on peut expliciter un  $\alpha(k,h)$  \poll en 
$k,h$  tel que  (voir l'explicitation en fin de preuve),
\[
\hbox{avec} \; n = \alpha(k,h) \; \; \forall f \in \Y_{\Gv,c,\beta} \; : \; \; 
\sum_{q>n} \abS{A_q(f)} q^{2k} \leq 1/2^{h+2}.
\]
Une fois fixé  $n = \alpha(k,h)$ il nous reste à rendre petit le terme 
$\NOrme{ s_n(f-g)^{(k)} }_{\infty}$. 
L'inégalité de Markov  (\ref{fF 5.2.2}) implique que 
\[
\NOrme{ s_n(f-g)^{(k)} }_\infty \leq \norme{ s_n(f-g) }_{\infty} n^{2k}.
\]
Il ne reste plus qu'à obtenir une majoration convenable de  
$\norme{ s_n(f-g) }_{\infty}$ à partir de  
$\norme{ f-g }_{\infty}$.
Par exemple, on peut utiliser la majoration  
$S_n(f) \leq (4+ \log (n)) E_n(f)$ 
  (d'après la formule (\ref{fF 5.2.3})) d'où
\[
\NOrme{f^{(k)} }_\infty \leq \norme{f}_\infty + S_n(f) \leq \norme{f}_\infty + (4+ \log (n)) E_n(f) \leq (5+ \log (n)) \norme{f}_\infty.
\]
Donnons pour terminer l'explicitation de $\alpha(k,h)$.  On a les inégalités
\[
\sum_{q \geq cn_0^{\beta}} \abS{A_q(f)} q^{2k} \leq \sum_{n \geq n_0}\sum_{q \leq c(n+1)^{\beta}}q^{2k}/2^n \leq \sum_{n \geq n_0} c(n+1)^{\beta}(c(n+1)^{\beta})^(2k)/2^n,
\]
donc
\[
\sum_{q \geq cn_0^{\beta}} \abS{A_q(f)} q^{2k} \leq \sum_{n \geq n_0} (c(n+1)^{\beta})^{2k+1}/2^n \leq \sum_{n \geq n_0} 1/2^{\varphi (n, \beta ,c, k)}
\]
avec, si  $2^a \geq c$,
\[
\varphi (n, \beta ,c, k) \geq n- (2k+1)a-(2k+1) \beta \log(n+1).
\]
Si l'on a   
\[
\hbox{pour} \; n \geq n_0 \; \; \varphi (n, \beta ,c, k) \geq h+n/2+4, \eqno (\star)
\]
on obtient
\[
\sum_{q \geq cn_0^{\beta}} \abS{A_q(f)} q^{2k} \leq \sum_{n \geq n_0} 1/2^{\varphi (n, \beta ,c, k)} \leq (1/2^{h+2}(1/4) \sum_{n \geq n_0} 1/2^{n/2} \leq 1/2^{h+2}.
\]
Et l'on peut prendre $\alpha (k,h) = cn_0^{\beta}.$ \\
Il reste à voir comment on peut réaliser la condition $(\star)$. \\
Pour tout entier $b$  on a un  entier  $\nu (b) \leq \max (8,b^2)$  pour lequel  
\[
n > \nu (b) \; \Rightarrow\; n \geq b \log (n+1).
\]
Si donc $n \geq \nu (4(2k+1) \beta)$  on obtient
\[
\varphi (n, \beta ,c, k) \geq n-(2k+1)a-(2k+1)\beta \log(n+1) \geq (3n/4)-(2k+1)a
\]
et la condition $\varphi (n, \beta ,c, k)\geq h+n/2+4$ est réalisée  si    
$n/4 \geq (2k+1)a +h +4.$
D'où finalement  
$\alpha (h,k) = c \max(\nu (4(2k+1)\beta ),4((2k+1)a + h +4))^{\beta}.$    \eop
 
En appliquant la proposition \ref{f526}, on obtient le corollaire suivant.

\begin{fcorollary} \label{f5214}~\\
i) Les trois suites de fonctionnelles 
\[
(n,f) \mapsto \Norme{f^{(n)} }_{\infty} , \;  \Norme{f^{(n)} }_2, \; \Norme{f^{(n)} }_1 \qquad  \NN_1 \times \sC_{\Gv,c, \beta} \to \RR
\]
sont \uni de classe  $\p$   (au sens de la définition \ref{f229}).\\
ii) La suite de fonctionnelles
\[(n,f,x) \mapsto f^{(n)}(x) \qquad 
\NN_1 \times \sC_{\Gv,c, \beta} \times [-1,1] \to \RR  
\] 
est \uni de classe  $\p$.\\
iii) La suite de fonctionnelles
\[  (n,f,a, b) \mapsto \sup_{x \in [a,b]} (f^{(n)}(x)) \qquad 
\NN_1 \times \sC_{\Gv,c, \beta} \times [-1,1] \times [-1,1] \to \RR 
\] 
est \uni de classe  $\p$.  
\end{fcorollary}

\begin{fremark} 
Les théorèmes  \ref{f527}, \ref{f528}, \ref{f5210}, \ref{f5213}, la proposition 
\ref{f526} et les corollaires \ref{f5211} et \ref{f5214}  améliorent sensiblement 
les résultats de  \cite{fKF82}, \cite{fKF88}  et  \cite{fMu87}  sur les fonctions 
analytiques calculables \etpo (au sens de Ko-Friedman).
\end{fremark}


\subsection {Comparaisons de différentes présentations de classe  
$\p$}\label{fsubsec53}
Dans cette section, nous obtenons la chaîne suivante de fonctions \uni de 
classe  $\p$  pour l'identité de $\czu$. 
\[
\cw \to \csp \to \crf \equiv \csr \to \ckf 
\] 
et aucune des flèches  $\rightarrow$   dans la ligne ci-dessus n'est une $\p$-équivalence sauf peut-être  
$\csp  \rightarrow   \crf$  et très éventuellement  $\crf  \rightarrow   
\ckf$  (mais cela impliquerait $\p = \np$).
Tout d'abord, il est clair que l'identité de $\czu$  est de classe  $\LINT $ 
pour les cas suivants:  
\[
\cw \rightarrow \csp; \quad  \csp \to \csr ;  \quad  \crf \rightarrow \csr. 
\]
Par ailleurs l'identité de $\czu$  est de classe  $\p$   dans le cas suivant 
\[
\crf \rightarrow \caf 
\] 
(en fait, seul le calcul de la magnitude n'est pas complètement trivial, et il est sûrement dans $\DTI (\Oo(N^2))$). 

Il nous reste à démontrer que l'identité de $\czu$ de  $\csp$  vers  $\crf$ et celle de  $\csr$  vers  $\crf$  
sont de classe~$\p$.

\begin{ftheorem} \label{f531}
L'identité de  $\czu$  de   $\csp$  vers  $\crf$  est \uni de classe $\p$.
\end{ftheorem} 
\proof 
Soit  $n \in  \NN_1$  et  $f \in  \ysp$. On doit calculer un élément  
$g \in \yrf $  tel que  
\[
\Norme{ \wi{f} - \wi{g} }_{\infty} \leq 1/2^n.
\] 
On a $f = ((x_0,x_1,\ldots,x_t),(P_1,P_2,\ldots,P_t))$ avec $x_0=0$, $x_t=1$ et 
$P_i(x_i) = P_{i+1}(x_i)$   pour  $i = 1,\ldots,t-1$.   
On calcule   $m \in \NN_1$   tel que  $2^m$  majore les 
$\norme{P_i}_{\infty}$ et $\norme{P_i'}_{\infty}$.  \\
On pose $p=m+n+1$ et $z_i = x_i - 1/2^{p+1}$ pour $i=1,\dots,t-1$. Pour  $i = 0$ on pose $h_0 := -C_{p,z_1}$. Pour  $i = t-1$ on pose $h_{t-1} := C_{p,z_{t-1}}$. Pour  $i = 1,\ldots,t-2$ on pose 
 $h_i := C_{p,z_i} - C_{p,z_{i+1}}$   (voir la figure \ref{ffi531}). 

\begin{figure}[htbp]  
\begin{center}
\includegraphics*[width=12cm]{fi531}
\end{center}
\caption[La fonction $h_i$]{\label{ffi531}  
La fonction $h_i$}  
\end{figure}  

\noindent Alors la fonction  $\wi{f}$ est à très peu près égale à  
$h=\sum_i h_iP_{i+1}$:  sur les intervalles  
$[z_i+1/2^p,z_{i+1}]$ on a   $h =\wi{f}$, 
tandis que sur un intervalle  
\[
[z_i, z_i +1/2^p] = [x_i - 1/2^{p+1}, x_i +1/2^{p+1}]
\]
on obtient  $h = h_{i-1}P_i + h_i P_{i+1}$  qui est une moyenne pondérée de  
$P_i$ et $P_{i+1}$,   au lieu de  $P_i$   ou  $P_{i+1}$. 
En un point $x$ de cet intervalle, on a  
$\abs { x-x_i }\;  \leq 1/2^{p+1}$, on applique le théorème des 
accroissements finis et en utilisant $P_i(x_i) =P_{i+1}(x_i)$  on obtient
\[
\abs{P_i(x) - P_{i+1}(x)}\; \leq \abs{P_i(x) - P_i(x_i)} + \abs{P_{i+1}(x) - P_{i+1}(x_i)}\; \leq 2^{m+1}/2^{p+1} \leq 1/2^{n+1}
\]
et donc:
\[
\Norme{ \wi{f} - h }_{\infty} \leq 1/2^{n+1}.
\]
Il reste à remplacer chaque  $h_i$  par un élément  $g_i$ de  $\yrf$  
vérifiant   
$\Norme{ \wi{g_i} - h_i }_{\infty} \leq 1/(t2^{p})$
 (de sorte que
$\Norme{\wi{g_i}P_{i+1}-h_iP_{i+1}}_{\infty}\leq 1/(t2^{n+1})$
et donc
\[
\NOrme{ h - \sum\nolimits_i \wi{g_i}P_{i+1} }_\infty \leq 1/2^{n+1}.
\]
Vue la proposition \ref{f337} concernant l'approximation des fonctions  $C_{p,a}$  
par des fractions rationnelles, le calcul des  $g_i$  se fait \etpo à partir 
de la donnée  $(f,n)$. Il reste à exprimer  
$\sum_i\wi{g_i}P_{i+1}$  
 sous forme $\wi{g}$   avec  $ g \in \yrf$, ce qui n'offre pas de 
difficulté, pour obtenir  
$\Norme{ \wi{f} - \wi{g} }_{\infty} \leq 1/2^n.$  
\eop

\begin{ftheorem} \label{f532}
Les représentations   $\crf$  et   $\csr$  de  $\czu$  sont $\p$-
équivalentes.
\end{ftheorem}
\proof
À partir d'un élément  
$f=((x_0,x_1,\ldots,x_t),((P_1,Q_1),\ldots,(P_t,Q_t)))$  arbitraire de  $\ysr$  
on peut calculer \etpo des nombres dyadiques $d_1, d_2,\ldots,d_t \geq 1$   
vérifiant   
$d_iQ_i(x_i) = d_{i+1}Q_{i+1}(x_i)$  (et donc aussi  $d_iP_i(x_i) = 
d_{i+1}P_{i+1}(x_i)$)  pour  $i = 1, \ldots, t-1$ 
Alors  $\wi{f} = \wi{g} / \wi{h}$    où  
$g,h \in \Y_{Sp}$  sont donnés par:
\[
g = ((x_0,x_1,\ldots,x_t),(d_1P_1,\ldots,d_tP_t)) \quad  \hbox{et} \quad  h = ((x_0,x_1,\ldots,x_t),(d_1Q_1,\ldots,d_tQ_t)).
\]
On conclut en utilisant le \thref{f531} qui permet d'approcher 
convenablement  $\wi g$  et $\wi h$ par des fractions rationnelles.  \eop 



%--- SECTION{sec Conclusion}---------------
\section*{Conclusion}\label{fsec Conclusion}
\addcontentsline{toc}{section}{Conclusion}
%-------------------------------------------
Hoover a relié de manière intéressante la notion naturelle de \com des 
fonctions réelles continues donnée par Ko et Friedman à une autre notion, 
basée sur les circuits arithmétiques. Il a donné de cette manière une 
certaine version ``en temps \poll'' du théorème d'approximation de 
Weierstrass. 

Dans cet article nous avons généralisé l'approche de Hoover, en 
introduisant un point de vue uniforme grâce à la notion de \rp  d'un espace 
métrique.
Ceci fournit un cadre de travail général satisfaisant pour l'étude de 
très nombreux problèmes de \com algorithmique en analyse. Grâce à cette approche
nous avons également généralisé  les 
résultats de Ko, Friedman et Müller concernant les fonctions analytiques 
calculables en temps \poll.  
La \rp  $\csl$ $\p$-\equiva à  $\ckf$ est la plus naturelle du point de vue de 
l'informatique théorique. Cependant ce n'est pas une $\p$-\pres et elle est 
mal adaptée à l'analyse numérique dés qu'on se pose des problèmes plus 
compliqués que l'évaluation (le calcul de la norme, ou d'une primitive par 
exemple).

Parmi les représentations que nous avons étudiées, la \pres  $\cw$  semble 
être la plus facile à utiliser pour de nombreux problèmes de l'analyse 
numérique. 

Quant à la \pres par les fractions rationnelles, elle mérite 
une étude plus approfondie. On aimerait obtenir pour cette \pres un analogue du théorème (donné pour la \pres~$\cw$) concernant la caractérisation des $\p$-points. 
Il serait également intéressant d'obtenir pour $\crf$ la 
$\p$-calculabilité de certaines opérations usuelles de l'analyse numérique, 
comme le calcul d'une primitive ou plus généralement le calcul de la 
solution d'une équation différentielle ordinaire. 


\bni{\bf Remerciements}
Nous remercions Maurice Margenstern et le referee pour leurs remarques 
pertinentes.

%---------------- biblio ----------------------
\addcontentsline{toc}{section}{Références bibliographiques}
\begin{thebibliography}{50}
\bibitem{fAb} Aberth O.: {\em Computable analysis.}
McGraw-Hill (1980).

\bibitem{fBa} Bakhvalov: {\em Méthodes Numériques.} Editions MIR. Moscou 
(1973).

\bibitem{fBe} Beeson M.: {\em Foundations of Constructive Mathematics.} Springer-Verlag (1985).

\bibitem{fBi} Bishop E.: {\em Foundations of Constructive Analysis}  McGraw-Hill, New York, (1967).

\bibitem{fBB} Bishop E., Bridges D.: {\em Constructive Analysis.} Springer-Verlag (1985).

\bibitem{fBo} Borodin A.: {\em On relating time and space to size and depth.} SIAM J. Comput. {\bf 6} (4), 733--744 (1977).

\bibitem{fBrent} Brent R.: {\em Fast multiple-precision evaluation of elementary functions.} Journal of the ACM {\bf 23} (2) 242--251 (1976).

\bibitem{fBr} Bridges D.: {\em Constructive Functional Analysis.} Pitman, London (1979).

\bibitem{fBR} Bridges D., Richman F. : {\em Varieties of Constructive 
Mathematics.} London Math. Soc. LNS 97. Cambridge University Press (1987)

\bibitem{fCh} Cheney E. W.: {\em Introduction to Approximation Theory.}
 Mc Graw Hill Book Company (1966).

\bibitem{fGo1} Goodstein R.: {\em Recursive number theory. A development of recursive arithmetic in a logic-free equation calculus}. Studies in Logic and the Foundations of Mathematics. Amsterdam, North-Holland  (1957).

\bibitem{fGo} Goodstein R.: {\em Recursive Analysis}. Studies in Logic and the Foundations of Mathematics. Amsterdam, North-Holland  (1961).

\bibitem{fHo87} Hoover J.: {\em Feasibly constructive analysis.} PhD (1987).

\bibitem{fHo90} Hoover J.: {\em Feasible real functions and arithmetic circuits.}
Siam J. Comput. {\bf 19} (1), 182--204 (1990).

\bibitem{fHo} Hörmander: {\em The Analysis of Linear Partial Differential Operators I.} Springer (1983).

\bibitem{fKF82} Ker-I. Ko, Friedman H.: {\em Computational complexity of real functions.}
 Theoretical Computer Science {\bf 20}, 323--352 (1982).

\bibitem{fKF88} Ker-I. Ko, Friedman H.: {\em Computing power series in \poll time.}
Adv. Appl. Math. {\bf 9}, 40--50 (1988).

\bibitem{fKo83} Ker I. Ko: {\em On the definitions of some complexity classes of real numbers.}
Math System Theory {\bf 16}, 95--109 (1983).

\bibitem{fKo91} Ker I. Ko: {\em Complexity theory of real functions.}
 Birkhäuser (1991).

\bibitem{fKu} Kushner B. A.: {\em Lectures on constructive mathematical 
analysis.} AMS Translations of Mathematical monographs \num60 (1984)  (la version russe est de 1973).

\bibitem{fLL} Labhalla S., Lombardi H.: {\em Real numbers, continued fractions and complexity classes.}
Annals of Pure and Applied Logic {\bf 50}, 1--28 (1990).

\bibitem{fLo89} Lombardi H.: {\em Nombres algébriques et approximations.}
Publications Mathématiques de l'Université (Besançon) 
Théorie des Nombres. Fascicule 2 (1989). \url{http://hlombardi.free.fr/publis/these_part3.pdf}

%\bibitem{fLo90} Lombardi H.: {\em Complexité des nombres réels et des 
%fonctions réelles.} CALSYF 90, journées du GRECO de Calcul Formel (1990).
%\url{http://hlombardi.free.fr/publis/NbAlgD5.pdf}

\bibitem{fMa} Margenstern M.: {\em L'école constructive de Markov}. Revue d'Histoire des Mathématiques, {\bf 1} (2),  271--305 (1995)

\bibitem{fmrr} Mines R., Richman F., Ruitenburg W. {\em A Course in Constructive Algebra.}  Springer-Verlag. Universitext. (1988).

\bibitem{fMo} Moutai E.M.: {\em Complexité des fonctions réelles, comparaison de différentes présentations.}
 Thèse de Troisième Cycle, Marrakech.(1995).

\bibitem{fMu86} N. Th. Müller: {\em Sub\poll complexity classes of real functions and real numbers.}
Proc 13th ICALP LNCS \num226. 284--293 (1986).

\bibitem{fMu87} N. Th. Müller: {\em Uniform computational complexity 
classes of Taylor series.}
Lecture Notes in Computer Science 267 (1987).

\bibitem{fNe} Newman D.J.: {\em Rational approximation to $\vert x\vert$.}
 Michigan Math. Journal (1964).

\bibitem{fPa} Pan V.: {\em Solving a \poll equation: some history and recent progress.} SIAM Rev.  {\bf 39}  (2), 187--220 (1997).

\bibitem{fPP} Petrushev P.P., Popov V.A.: {\em Rational approximation of real functions.}
Encyclopedia of Mathematics and its applications. Cambridge University Press (1987).

\bibitem{fPR} Pour El M., Richards I.: {\em Computability in Analysis and 
Physics.} Perspectives in Mathematical logic. Springer Verlag (1989).

\bibitem{fRi} Th. J. Rivlin: {\em The Chebyshev Polynomials.}
A Wiley Interscience Publication. Wiley \& Sons. New York (1974). 

\bibitem{fSc} Schnorr C.P.: {\em Satisfiability is quasilinear complete in NQL.}
J. Ass. Comput. Machinery {\bf 25} (1) 136--145 (1978).

\bibitem{fSt} Stern J.: {\em Fondements mathématiques de l'informatique.}
Mc Graw-Hill, Paris, (1990).

\end{thebibliography}





 

