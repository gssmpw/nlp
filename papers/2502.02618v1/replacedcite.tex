\section{Related Work}
In this section, we examine the literature to identify studies that assess the impact of age on FER tasks, as well as prior reviews on FER. Our objective is twofold: first, to underscore the lack of FER reviews focused on specific age groups, particularly the elderly, and second, to provide evidence of the age bias often introduced by age-agnostic approaches. This analysis aims to highlight the need for age-sensitive methodologies in FER and to emphasize the importance of addressing this gap in existing reviews.

    \subsection{Effects of Aging on Facial Expression Recognition}

        Numerous psychological studies have explored the effects of aging on facial expression recognition, demonstrating that observers are influenced by their own age. For example, older individuals tend to exhibit deficits in decoding specific emotions ____. However, aging impacts not only the observers but also the individuals displaying the target expressions. In this regard, ____ examined how age-related changes in the face, such as wrinkles and folds, influence the decoding process of emotional expressions, concluding that the age of the face is a critical factor. This finding is supported by subsequent studies, such as ____, which identified variations in facial expression intensities and muscle usage across different age groups. For instance, elderly individuals tend to display more negative emotions and engage more muscles in the lower face compared to younger people. Similarly, ____ used generative adversarial networks (GAN) to investigate the increased difficulty in identifying expressions of older adults and attributed this challenge to the decline in facial muscle function with age. ____ explored the effect of training FER models with data from different age groups and found that recognizing expressions in elderly faces posed the greatest difficulty. A review by ____ on the effect of facial aging on FER identified several open challenges, including the scarcity of images capturing the same individuals across different ages (which hinders the learning of aging patterns), the tendency of models to overlook important facial features, and the considerable variation in aging effects across subjects. Moreover, the training of a Siamese CNN on the ElderReact and EmoReact datasets by ____ highlighted differences in the expressions of elderly individuals and children. ____ demonstrated the importance of addressing age bias in FER, showing that age-specific training yielded a 22\% improvement in accuracy compared to using a non-age-specific dataset.

        Automatic FER systems are heavily influenced by the age of the faces in the dataset, making the choice of dataset critical. Datasets such as FACES ____ and LifeSpan ____, are frequently employed ____, largely due to their inclusion of subjects from a broad age range, which facilitates the development of age-invariant models. To address the effects of aging, several strategies have been proposed. One approach involves removing aging features through facial smoothing techniques that eliminate age-related details without compromising essential structural information ____. Another method incorporates age information during training using Bayesian networks, while marginalizing over age during testing ____. ____ first analyzed age-related differences in facial characteristics, then later incorporated age as a key feature in their model ____, using a weighted combination of age group estimators and age-specific expression recognizers. 
        
        Other studies have focused on specific age groups rather than addressing aging effects comprehensively. Datasets utilized in such approaches include ElderReact ____ and Tsinghua ____ for elderly subjects, CK+ ____, JAFFE ____, AFEW ____, and FER-2013 ____ for adults, and LIRIS ____, CAFE ____, DEFSS ____, and EmoReact ____ for children, among others.
        
    
    \subsection{Summary of Previous Reviews}
        % Background 
        
        Mining the literature, we identified several existing reviews addressing facial expression recognition. Although most of these do not focus specifically on elderly populations and therefore do not consider the effects of human aging discussed in the previous section, they do provide valuable insights into the datasets and techniques commonly employed in FER. In this section, we analyze relevant reviews and surveys from the past decade, highlighting that many critical aspects addressed in our current study have not been thoroughly examined in prior reviews.
    
        \begin{table*}
            \centering
            % \begin{adjustbox}{width=.65\textwidth}
            \begin{tabular*}{.71\textwidth}{l|ll|ll|lllll}
                 & \multicolumn{2}{c|}{\textbf{Process}} & \multicolumn{2}{c|}{\textbf{Focus}} &\multicolumn{5}{c}{\textbf{Researched data}} \\
                 \multicolumn{1}{c|}{} & \rotatebox{90}{Systematic} & \rotatebox{90}{Prev. reviews} & \rotatebox{90}{Elderly} & \multicolumn{1}{c|}{\rotatebox{90}{DL}} & \rotatebox{90}{Datasets} & \rotatebox{90}{Multimodality} & \rotatebox{90}{Deployment} & \rotatebox{90}{Privacy} & \rotatebox{90}{XAI}\\ \midrule
                ____ &  &  &  & \cmark &  &  &  &  \\
                ____ &  &  &  &  &  &  &  &  &  \\
                ____ &  & \cmark &  &  & \cmark &  &  &  &  \\
                ____ &  &  &  &  &  &  &  &  &  \\
                ____ &  &  &  &  &  &  &  &  &  \\
                ____ &  &  & \cmark &  &  &  &  &  &  \\
                ____ &  &  &  &  &  &  &  &  &  \\
                ____ &  & \cmark &  &  & \cmark &  &  &  &  \\
                ____ & \cmark &  &  &  & \cmark & \cmark &  &  &  \\
                ____ &  &  &  & \cmark & \cmark &  &  &  &  \\
                ____ &  &  &  &  & \cmark &  &  &  &  \\
                ____ &  &  &  & \cmark & \cmark &  &  &  &  \\
                ____ &  &  &  & \cmark & \cmark & \cmark &  &  &  \\
                ____ &  & \cmark &  &  & \cmark &  &  &  &  \\
                ____ & \cmark & \cmark &  &  & \cmark & \cmark &  &  &  \\
                ____ &  &  &  &  & \cmark &  &  &  &  \\
                ____ & \cmark & \cmark &  &  & \cmark & \cmark &  &  &  \\
                ____ &  &  &  &  & \cmark &  &  &  &  \\
                ____ &  &  &  &  & \cmark &  &  &  &  \\
                ____ & \cmark &  &  &  & \cmark &  &  &  &  \\
                ____ &  &  &  &  & \cmark &  &  &  &  \\
                ____ &  &  &  & \cmark &  &  &  &  &  \\
                ____ &  &  &  &  & \cmark &  &  &  &  \\
                ____ &  &  &  &  & \cmark & \cmark &  &  &  \\
                ____ &  &  &  &  &  &  &  &  &  \\
                ____ & \cmark & \cmark &  &  & \cmark &  &  &  &  \\
                ____ & \cmark &  &  &  & \cmark &  & \cmark &  &  \\
                ____ & \cmark &  &  &  & \cmark & \cmark &  &  &  \\
                ____ &  & \cmark &  &  & \cmark &  &  &  &  \\
                ____ &  &  &  &  & \cmark & \cmark &  &  &  \\
                ____ &  &  &  &  &  &  &  & \cmark &  \\
                ____ &  &  &  & \cmark & \cmark & \cmark &  &  &  \\
                ____ &  &  & \cmark &  & \cmark &  &  &  &  \\
                ____ & \cmark &  &  & \cmark & \cmark &  &  &  &  \\
                ____ &  &  &  & \cmark & \cmark & \cmark &  &  &  \\
                ____ & \cmark & \cmark &  &  &  &  & \cmark &  &  \\
                ____ &  &  &  &  & \cmark &  &  &  &  \\
                ____ &  &  &  & \cmark & \cmark &  &  &  &  \\
                ____ & \cmark & \cmark &  &  & \cmark & \cmark &  &  &  \\
                ____ & \cmark & \cmark &  &  & \cmark &  &  &  &  \\
                This study & \cmark & \cmark & \cmark & \cmark & \cmark & \cmark & \cmark & \cmark & \cmark \\
                \bottomrule
            \end{tabular*}
            % \end{adjustbox}
            \caption{Comparison of previous reviews with our work, considering the key aspects of the current systematic review. The columns represent the following criteria, from left to right: whether the review is systematic; it includes an exploration of prior reviews; it focuses on the elderly population; it focuses on DL-based approaches; it provides a list of datasets; it investigates the use of multimodal data; it examines the deployment of FER systems in real-world environments; it considers privacy a critical issue; and it discusses the application of XAI techniques in FER systems.}
            \label{tab:review-comparison}
        \end{table*}
        
        To begin with, it is important to consider the methodological rigor of prior review processes. A systematic review methodology promotes a thorough and unbiased selection and analysis of studies by precisely defining information sources, search strategies, study selection criteria, and data extraction procedures. Additionally, quality assessment steps are essential to ensure that only high-quality, relevant studies are included. However, only 12 of the 41 reviews listed in Table \ref{tab:review-comparison} followed a systematic review process. ____ also emphasize the importance of analyzing existing reviews in the field to identify gaps in the literature and avoid duplicate efforts. Yet, only 11 studies accounted for previous reviews, as illustrated in the table.
        
        The main distinction between the current study and earlier reviews lies in our focus on FER applications for the elderly and deep learning tools. Only two previous reviews have concentrated specifically on elderly populations. ____ discussed 11 studies on elderly-focused FER, covering both traditional and deep learning methods and their associated datasets. ____ also reviewed FER in elderly populations, primarily as a tool for early diagnosis of mild cognitive impairment. However, their work lacks coverage of recent studies from the last five years and focuses primarily on traditional computer vision approaches rather than DL-based methods. Consequently, there is a gap in the literature regarding recent DL-based FER research specific to elderly individuals.
        
        Ten reviews were identified that discuss DL in FER ____. Of these, only the work by Pinto et al. employed a systematic review process, and none of these reviews examined previous FER reviews, which may have contributed to overlapping lists of studies. While many of these reviews included extensive lists of commonly used datasets--an important consideration in DL-based approaches--none were focused on elderly populations, nor did they address critical aspects for the current work such as deployment in real-world environments, privacy preservation, or explainable AI techniques.
        
        Three critical aspects—deployment in real--world settings, privacy preservation, and XAI techniques--remain largely unexplored in prior reviews. Deployment in real-world scenarios was examined in only two reviews, which described applications involving mobile apps and robots ____ and smart home integration ____. Although ethical concerns regarding privacy were noted in two reviews ____, only one review explored this issue in depth ____. Furthermore, despite the growing importance of XAI, this area was not thoroughly investigated in any review, even though some listed it as a future challenge ____.
        
        Given the evident lack of reviews focused on elderly populations using DL-based techniques for FER, the present study aims to address this gap. Additionally, as few reviews have examined factors critical for the practical deployment of FER systems in real-world environments, and none have explored XAI in detail, our work may offer valuable insights for the development of practical, reliable applications.