\section{Related Work}
Laparoscopy surgical action recognition is a crucial research area in surgical training, for provision of real-time feedback to trainees during procedures. Lea et al \cite{lea2016segmental}. proposed Segmental Spatiotemporal Convolutional Neural Networks (CNNs) for capturing fine-grained temporal information to perform action segmentation, achieving significant results in surgical action recognition. Similarly, Twinanda et al \cite{twinanda2016endonet}. developed EndoNet, an advanced neural network architecture specifically designed for laparoscopic surgical videos, capable of performing multitask recognition, including action classification. The multitask structure of EndoNet effectively improves the recognition of complex actions, offering a novel approach for action detection in laparoscopic videos. In surgical data processing, class imbalance is a common phenomenon. Chawla et al \cite{chawla2002smote}. proposed SMOTE (Synthetic Minority Over-sampling Technique), which generates synthetic minority samples to balance the dataset distribution and improve recognition of minority class actions. Additionally, the survey by He and Garcia \cite{he2009learning} provides a comprehensive review of various techniques for handling class imbalance, including oversampling, undersampling, and weighted methods, which play a critical role in enhancing minority class recognition.

In medical applications, model interpretability is essential for machine learning model adoption. Rudin \cite{rudin2019stop} emphasized the necessity of using interpretable models in high-stakes domains and recommended prioritizing interpretable models over black-box models to ensure higher transparency for clinicians and trainers. Moreover, the work by Doshi-Velez and Kim \cite{doshi2017towards} explores various techniques for enhancing model interpretability, presenting a framework for applying interpretable models in medical applications, providing theoretical support for the use of interpretable models in clinical scenarios. Regarding model stability, Breiman's random forest algorithm \cite{breiman2001random}, as an ensemble learning method, enhances model stability and generalization by combining multiple decision trees, making it widely applicable to medical data processing where high stability is required. To further improve model stability and automate the optimization process, Feurer et al. \cite{feurer2015efficient} developed an automated machine learning system that combines automated model selection and hyperparameter tuning, effectively reducing the complexity of manual tuning and improving model adaptability and stability on surgical data. In surgical training, real-time feedback systems assist trainees by providing immediate feedback during procedures. Salvador et al \cite{salvador2024effects} investigated the application of real-time visual feedback in laparoscopic training, examining its impact on novices' learning curves. The results showed that trainees receiving real-time feedback demonstrated higher precision in tissue handling skills, significantly shortening their learning curves. The introduction of real-time feedback enabled trainees to master essential skills more quickly, improving training efficiency.


In summary, current laparoscopy surgical action recognition systems, class imbalance handling techniques, model interpretability methods, stability-enhancing techniques, and real-time feedback systems have shown progress in surgical training, yet limitations remain. Addressing these gaps, this study aims to optimize surgical action recognition systems using automated machine learning, combining multiple techniques to improve model performance in accuracy, interpretability, stability, and real-time functionality, providing a more intelligent and efficient feedback mechanism for surgical training.


\begin{figure*}[h!]
	\centering
	\includegraphics[width=\textwidth]{framework.png}
	\caption{Overall AutoML workflow including meta learning warmstart for bayesian optimization efficient model selection and ensemble building for laparoscopy surgical suturing action detection. 
	}
	\label{framework}
\end{figure*}