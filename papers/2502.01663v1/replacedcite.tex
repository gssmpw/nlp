\section{Literature Review}
Human Metapneumovirus (hMPV) was discovered in 2001 and is a leading cause of respiratory infections, particularly among young children, the elderly, and immune-compromised individuals____. This virus is responsible for 5-10\% of pediatric hospitalizations due to acute respiratory infections and can lead to severe conditions like bronchiolitis and pneumonia. The symptoms of hMPV infection often resemble those of respiratory syncytial virus, and RT-PCR is the preferred diagnostic method due to the virus's slow growth in cell culture. Although vaccine candidates are under development, no vaccines are currently commercially available, and research continues to improve the understanding and treatment strategies for hMPV.

A case report by ____ discusses a 68-year-old immunocompetent male who developed severe pneumonia caused by hMPV. Despite the absence of significant comorbidities, the patient required hospitalization due to worsening respiratory symptoms. Diagnosis was confirmed with multiplex RT-PCR, and imaging revealed viral pneumonia. The patient fully recovered with supportive care, underscoring the importance of molecular diagnostics in accurate diagnosis, reducing unnecessary antibiotic use, and highlighting the potential for future vaccines such as IVX-A12.

Kannappan et al.____ examine the growing significance of sentiment analysis in the digital landscape. Their study highlights its role in website development, social media profile creation, and managing digital platforms. Sentiment analysis aids in addressing customer inquiries, evaluating product feedback, and safeguarding a company's reputation by ensuring positive reviews. The research explores the synergy between Natural Language Processing (NLP) and Machine Learning (ML), focusing on how NLP tools process human language and how ML, particularly with Python, facilitates effective sentiment analysis.

Kavitha et al.____ explore the application of sentiment analysis on social media data using NLP and ML techniques. Their study emphasizes the importance of extracting insights from user-generated content like tweets and blogs, showing how ML algorithms such as Random Forest and Logistic Regression can effectively process and analyze sentiment. This research demonstrates the potential of combining NLP and ML to derive meaningful conclusions from social media data and enhance decision-making.

Srivastava et al.____ review various approaches to sentiment analysis within NLP, focusing on techniques that detect the emotional tone of text. With the exponential growth of online content, including text, photos, audio, and video, their study illustrates how sentiment analysis can extract valuable insights. The authors examine widely-used methods such as Naïve Bayes, Support Vector Machines (SVM), and the lexicon-based approach in NLP, and discuss challenges in sentiment analysis, predicting greater accessibility for smaller businesses and the public as technology evolves.

Jim et al.____ present a comprehensive review of recent advancements and challenges in sentiment analysis, a critical area within NLP. They examine its application in classifying textual data as positive, negative, or neutral, providing businesses with crucial insights into customer emotions. The study delves into various domains, pre-processing techniques, datasets, and evaluation metrics that contribute to sentiment analysis, as well as the roles of Machine Learning, Deep Learning, and Large Language Models. They propose future research directions to address the challenges and limitations identified in state-of-the-art studies.

Gunasekaran____ reviews various sentiment analysis techniques within NLP, including lexicon-based, machine learning, deep learning, and hybrid approaches. This research highlights the importance of sentiment analysis in customer feedback analysis, marketing, and politics. Using Twitter as a case study, it discusses applications across different sectors and offers a comparative analysis of techniques, datasets, and metrics aimed at improving the accuracy and efficiency of sentiment analysis.

Chong et al.____ explore sentiment analysis on tweets using NLP techniques, focusing on three key steps: subjectivity classification, semantic association, and polarity classification. They employ sentiment lexicons and grammatical relationships, achieving superior results over traditional sentiment analysis tools. Their findings emphasize the importance of tailored NLP approaches for analyzing sentiment in social media contexts.

Jain et al.____ focus on real-time sentiment analysis using multimedia inputs such as audio, video, and text to interpret emotions. They compare techniques like Support Vector Machines (SVMs), Bayesian Networks, and Convolutional Neural Networks (CNNs) for sentiment classification. Their research leads to the development of a real-time sentiment analysis system to help users assess daily attitudes and receive relevant recommendations.

Wankhade et al.____ provide a comprehensive survey on sentiment analysis, reviewing methods, applications, and challenges. The study explores how sentiment analysis can gather and interpret opinions from internet-based platforms like social media and blogs. It compares various techniques and addresses challenges in accurately determining sentiment polarity, offering suggestions for future research to improve the effectiveness and accuracy of sentiment analysis.

  
The reviewed literature highlights the broad range of applications and techniques in sentiment analysis, particularly in the domains of customer feedback, social media analysis, and emotion detection from multimedia data. The ongoing advancements in NLP and ML offer significant potential for improving sentiment analysis methods, with future research expected to address the current challenges and further refine these technologies for diverse real-world applications.