\documentclass[sigconf]{acmart}
\acmYear{2025}
\acmConference[arXiv]{February}{February}{2025}
\acmISBN{$^{*}$Corresponding author: arissa@google.com}
\setcopyright{none}
\copyrightyear{2025}
\settopmatter{printfolios=true}
\settopmatter{printacmref=false}
% \usepackage[dvipsnames]{xcolor}
\usepackage{algorithmic}
\usepackage{textcomp}
\usepackage{fancyhdr}
\usepackage{hyperref}
\usepackage{multirow}
\usepackage{graphicx}  %
\usepackage{multirow}  %
\usepackage{array}     %
\usepackage{booktabs}  %
\usepackage{amsmath}
\usepackage{circledsteps}
\usepackage{xspace}
\usepackage[font=footnotesize]{caption}

\hypersetup{
    colorlinks=true,
    linkcolor=blue,
    filecolor=magenta,      
    urlcolor=cyan,
    citecolor=blue,
    anchorcolor=blue,
    pdftitle={Overleaf Example},
    pdfpagemode=FullScreen,
}
\usepackage{microtype}
\usepackage{subfigure}

\newcommand{\theHalgorithm}{\arabic{algorithm}}

\newcommand{\todo}[1]{{\color{red} #1}}

\setlength{\belowcaptionskip}{-10pt}

\AtBeginDocument{%
  \providecommand\BibTeX{{%
    Bib\TeX}}}

\def\sectionautorefname{Section}

\begin{document}

\title{Machine Learning Fleet Efficiency: Analyzing and Optimizing Large-Scale Google TPU Systems with ML Productivity Goodput\\[.25em] \subtitle{\emph{A Preprint}}}

\author{\Large
    Arissa Wongpanich$^{\dagger}$
    Tayo Oguntebi$^{\dagger}$
    Jose Baiocchi Paredes$^{\dagger}$\\
    Yu Emma Wang$^{\dagger}$
    Phitchaya Mangpo Phothilimthana$^{\dagger}$
    Ritwika Mitra$^{\dagger}$\\
    Zongwei Zhou$^{\dagger}$
    Naveen Kumar$^{\dagger}$
    Vijay Janapa Reddi$^{\dagger\ddagger}$ \\[1em]
    $^{\dagger}$Google \hspace{1cm} $^{\ddagger}$Harvard University\\[1em]
}

  \renewcommand{\shortauthors}{}



\begin{abstract}
Recent years have seen the emergence of machine learning (ML) workloads deployed in warehouse-scale computing (WSC) settings, also known as ML fleets. As the computational demands placed on ML fleets have increased due to the rise of large models and growing demand for ML applications, it has become increasingly critical to measure and improve the efficiency of such systems. However, there is not yet an established methodology to characterize ML fleet performance and identify potential performance optimizations accordingly. This paper presents a large-scale analysis of an ML fleet based on Google's TPUs, introducing a framework to capture fleet-wide efficiency, systematically evaluate performance characteristics, and identify optimization strategies for the fleet. We begin by defining an ML fleet, outlining its components, and analyzing an example Google ML fleet in production comprising thousands of accelerators running diverse workloads. Our study reveals several critical insights: first, ML fleets extend beyond the hardware layer, with model, data, framework, compiler, and scheduling layers significantly impacting performance; second, the heterogeneous nature of ML fleets poses challenges in characterizing individual workload performance; and third, traditional utilization-based metrics prove insufficient for ML fleet characterization. To address these challenges, we present the ``ML Productivity Goodput'' (MPG) metric to measure ML fleet efficiency. We show how to leverage this metric to characterize the fleet across the ML system stack. We also present methods to identify and optimize performance bottlenecks using MPG, providing strategies for managing warehouse-scale ML systems in general. Lastly, we demonstrate quantitative evaluations from applying these methods to a real ML fleet for internal-facing Google TPU workloads, where we observed tangible improvements.

\end{abstract}




\newcommand{\mpg}{ML Productivity Goodput\xspace}
\newcommand{\sg}{Scheduling Goodput\xspace}
\newcommand{\pg}{Program Goodput\xspace}
\newcommand{\rg}{Runtime Goodput\xspace}
\newcommand{\google}{Google\xspace}
\maketitle
\pagestyle{plain}

\section{Introduction}


\begin{figure}[t]
\centering
\includegraphics[width=0.6\columnwidth]{figures/evaluation_desiderata_V5.pdf}
\vspace{-0.5cm}
\caption{\systemName is a platform for conducting realistic evaluations of code LLMs, collecting human preferences of coding models with real users, real tasks, and in realistic environments, aimed at addressing the limitations of existing evaluations.
}
\label{fig:motivation}
\end{figure}

\begin{figure*}[t]
\centering
\includegraphics[width=\textwidth]{figures/system_design_v2.png}
\caption{We introduce \systemName, a VSCode extension to collect human preferences of code directly in a developer's IDE. \systemName enables developers to use code completions from various models. The system comprises a) the interface in the user's IDE which presents paired completions to users (left), b) a sampling strategy that picks model pairs to reduce latency (right, top), and c) a prompting scheme that allows diverse LLMs to perform code completions with high fidelity.
Users can select between the top completion (green box) using \texttt{tab} or the bottom completion (blue box) using \texttt{shift+tab}.}
\label{fig:overview}
\end{figure*}

As model capabilities improve, large language models (LLMs) are increasingly integrated into user environments and workflows.
For example, software developers code with AI in integrated developer environments (IDEs)~\citep{peng2023impact}, doctors rely on notes generated through ambient listening~\citep{oberst2024science}, and lawyers consider case evidence identified by electronic discovery systems~\citep{yang2024beyond}.
Increasing deployment of models in productivity tools demands evaluation that more closely reflects real-world circumstances~\citep{hutchinson2022evaluation, saxon2024benchmarks, kapoor2024ai}.
While newer benchmarks and live platforms incorporate human feedback to capture real-world usage, they almost exclusively focus on evaluating LLMs in chat conversations~\citep{zheng2023judging,dubois2023alpacafarm,chiang2024chatbot, kirk2024the}.
Model evaluation must move beyond chat-based interactions and into specialized user environments.



 

In this work, we focus on evaluating LLM-based coding assistants. 
Despite the popularity of these tools---millions of developers use Github Copilot~\citep{Copilot}---existing
evaluations of the coding capabilities of new models exhibit multiple limitations (Figure~\ref{fig:motivation}, bottom).
Traditional ML benchmarks evaluate LLM capabilities by measuring how well a model can complete static, interview-style coding tasks~\citep{chen2021evaluating,austin2021program,jain2024livecodebench, white2024livebench} and lack \emph{real users}. 
User studies recruit real users to evaluate the effectiveness of LLMs as coding assistants, but are often limited to simple programming tasks as opposed to \emph{real tasks}~\citep{vaithilingam2022expectation,ross2023programmer, mozannar2024realhumaneval}.
Recent efforts to collect human feedback such as Chatbot Arena~\citep{chiang2024chatbot} are still removed from a \emph{realistic environment}, resulting in users and data that deviate from typical software development processes.
We introduce \systemName to address these limitations (Figure~\ref{fig:motivation}, top), and we describe our three main contributions below.


\textbf{We deploy \systemName in-the-wild to collect human preferences on code.} 
\systemName is a Visual Studio Code extension, collecting preferences directly in a developer's IDE within their actual workflow (Figure~\ref{fig:overview}).
\systemName provides developers with code completions, akin to the type of support provided by Github Copilot~\citep{Copilot}. 
Over the past 3 months, \systemName has served over~\completions suggestions from 10 state-of-the-art LLMs, 
gathering \sampleCount~votes from \userCount~users.
To collect user preferences,
\systemName presents a novel interface that shows users paired code completions from two different LLMs, which are determined based on a sampling strategy that aims to 
mitigate latency while preserving coverage across model comparisons.
Additionally, we devise a prompting scheme that allows a diverse set of models to perform code completions with high fidelity.
See Section~\ref{sec:system} and Section~\ref{sec:deployment} for details about system design and deployment respectively.



\textbf{We construct a leaderboard of user preferences and find notable differences from existing static benchmarks and human preference leaderboards.}
In general, we observe that smaller models seem to overperform in static benchmarks compared to our leaderboard, while performance among larger models is mixed (Section~\ref{sec:leaderboard_calculation}).
We attribute these differences to the fact that \systemName is exposed to users and tasks that differ drastically from code evaluations in the past. 
Our data spans 103 programming languages and 24 natural languages as well as a variety of real-world applications and code structures, while static benchmarks tend to focus on a specific programming and natural language and task (e.g. coding competition problems).
Additionally, while all of \systemName interactions contain code contexts and the majority involve infilling tasks, a much smaller fraction of Chatbot Arena's coding tasks contain code context, with infilling tasks appearing even more rarely. 
We analyze our data in depth in Section~\ref{subsec:comparison}.



\textbf{We derive new insights into user preferences of code by analyzing \systemName's diverse and distinct data distribution.}
We compare user preferences across different stratifications of input data (e.g., common versus rare languages) and observe which affect observed preferences most (Section~\ref{sec:analysis}).
For example, while user preferences stay relatively consistent across various programming languages, they differ drastically between different task categories (e.g. frontend/backend versus algorithm design).
We also observe variations in user preference due to different features related to code structure 
(e.g., context length and completion patterns).
We open-source \systemName and release a curated subset of code contexts.
Altogether, our results highlight the necessity of model evaluation in realistic and domain-specific settings.





\section{Background}\label{sec:backgrnd}

\subsection{Cold Start Latency and Mitigation Techniques}

Traditional FaaS platforms mitigate cold starts through snapshotting, lightweight virtualization, and warm-state management. Snapshot-based methods like \textbf{REAP} and \textbf{Catalyzer} reduce initialization time by preloading or restoring container states but require significant memory and I/O resources, limiting scalability~\cite{dong_catalyzer_2020, ustiugov_benchmarking_2021}. Lightweight virtualization solutions, such as \textbf{Firecracker} microVMs, achieve fast startup times with strong isolation but depend on robust infrastructure, making them less adaptable to fluctuating workloads~\cite{agache_firecracker_2020}. Warm-state management techniques like \textbf{Faa\$T}~\cite{romero_faa_2021} and \textbf{Kraken}~\cite{vivek_kraken_2021} keep frequently invoked containers ready, balancing readiness and cost efficiency under predictable workloads but incurring overhead when demand is erratic~\cite{romero_faa_2021, vivek_kraken_2021}. While these methods perform well in resource-rich cloud environments, their resource intensity challenges applicability in edge settings.

\subsubsection{Edge FaaS Perspective}

In edge environments, cold start mitigation emphasizes lightweight designs, resource sharing, and hybrid task distribution. Lightweight execution environments like unikernels~\cite{edward_sock_2018} and \textbf{Firecracker}~\cite{agache_firecracker_2020}, as used by \textbf{TinyFaaS}~\cite{pfandzelter_tinyfaas_2020}, minimize resource usage and initialization delays but require careful orchestration to avoid resource contention. Function co-location, demonstrated by \textbf{Photons}~\cite{v_dukic_photons_2020}, reduces redundant initializations by sharing runtime resources among related functions, though this complicates isolation in multi-tenant setups~\cite{v_dukic_photons_2020}. Hybrid offloading frameworks like \textbf{GeoFaaS}~\cite{malekabbasi_geofaas_2024} balance edge-cloud workloads by offloading latency-tolerant tasks to the cloud and reserving edge resources for real-time operations, requiring reliable connectivity and efficient task management. These edge-specific strategies address cold starts effectively but introduce challenges in scalability and orchestration.

\subsection{Predictive Scaling and Caching Techniques}

Efficient resource allocation is vital for maintaining low latency and high availability in serverless platforms. Predictive scaling and caching techniques dynamically provision resources and reduce cold start latency by leveraging workload prediction and state retention.
Traditional FaaS platforms use predictive scaling and caching to optimize resources, employing techniques (OFC, FaasCache) to reduce cold starts. However, these methods rely on centralized orchestration and workload predictability, limiting their effectiveness in dynamic, resource-constrained edge environments.



\subsubsection{Edge FaaS Perspective}

Edge FaaS platforms adapt predictive scaling and caching techniques to constrain resources and heterogeneous environments. \textbf{EDGE-Cache}~\cite{kim_delay-aware_2022} uses traffic profiling to selectively retain high-priority functions, reducing memory overhead while maintaining readiness for frequent requests. Hybrid frameworks like \textbf{GeoFaaS}~\cite{malekabbasi_geofaas_2024} implement distributed caching to balance resources between edge and cloud nodes, enabling low-latency processing for critical tasks while offloading less critical workloads. Machine learning methods, such as clustering-based workload predictors~\cite{gao_machine_2020} and GRU-based models~\cite{guo_applying_2018}, enhance resource provisioning in edge systems by efficiently forecasting workload spikes. These innovations effectively address cold start challenges in edge environments, though their dependency on accurate predictions and robust orchestration poses scalability challenges.

\subsection{Decentralized Orchestration, Function Placement, and Scheduling}

Efficient orchestration in serverless platforms involves workload distribution, resource optimization, and performance assurance. While traditional FaaS platforms rely on centralized control, edge environments require decentralized and adaptive strategies to address unique challenges such as resource constraints and heterogeneous hardware.



\subsubsection{Edge FaaS Perspective}

Edge FaaS platforms adopt decentralized and adaptive orchestration frameworks to meet the demands of resource-constrained environments. Systems like \textbf{Wukong} distribute scheduling across edge nodes, enhancing data locality and scalability while reducing network latency. Lightweight frameworks such as \textbf{OpenWhisk Lite}~\cite{kravchenko_kpavelopenwhisk-light_2024} optimize resource allocation by decentralizing scheduling policies, minimizing cold starts and latency in edge setups~\cite{benjamin_wukong_2020}. Hybrid solutions like \textbf{OpenFaaS}~\cite{noauthor_openfaasfaas_2024} and \textbf{EdgeMatrix}~\cite{shen_edgematrix_2023} combine edge-cloud orchestration to balance resource utilization, retaining latency-sensitive functions at the edge while offloading non-critical workloads to the cloud. While these approaches improve flexibility, they face challenges in maintaining coordination and ensuring consistent performance across distributed nodes.


\section{Anatomy of an ML Fleet}\label{sec:fleet}

\begin{figure}[t]
    \centering
    \includegraphics[width=\columnwidth]{final_figs/ml_stack.png}
    \caption{The ML fleet system stack of a production system at Google. The multi-layered architecture of a fleet is complex; each layer is a critical component in the ML system, with interactions between layers affecting overall performance and efficiency. Segmenting the fleet based on these layers provides actionable metrics which can be used to improve performance. }
    \label{fig:ml_stack}
\end{figure}
In this section, we dive into the anatomy of a production ML fleet to provide perspective on the complexity of managing it. We begin by dissecting the fleet based on its distinctive components and characteristics, starting with the hardware foundation and progressing to the user application level.  \autoref{fig:ml_stack} shows the various layers that comprise the system stack and mediate user access to the ML fleet. This stack illustrates the intricate ecosystem that underpins modern ML operations. 

We supplement our discussion in this section with data from a snapshot of Google's TPU fleet for internal workloads, providing concrete examples of the challenges these systems face in practice. By examining actual usage patterns, resource allocation, and performance metrics from a production ML fleet, we can ground our discussion in real-world scenarios and offer insights based on empirical evidence. This data-driven approach will allow us to illustrate the complexities of managing large-scale ML operations and demonstrate how theoretical concepts translate into practical challenges and opportunities for optimization. %
\subsection{Accelerators}


ML fleets are distinguished from other types of large-scale compute systems by their accelerator-centric architecture. The ML computing landscape is dominated by domain-specific hardware, such as GPUs and other ASICs. In order to tailor to the vector and matrix intensive operations that underpin ML workloads, new accelerators such as Google's Tensor Processing Units (TPUs) \cite{jouppi2018motivation} have been developed. In general, there has been a Cambrian explosion of ML hardware accelerators \cite{hennessy2019anewgoldenage}, with new accelerators being deployed at an unprecedented rate compared to traditional WSC fleets \cite{jouppi2021ten}. \autoref{fig:five-years} vividly illustrates this dynamism, revealing dramatic shifts in our ML fleet's hardware makeup for internal workloads over just a few years.

ML fleets typically incorporate a diverse array of hardware including CPUs, GPUs, TPUs, and other accelerators, each fulfilling specific roles. For example, CPUs may be responsible for scheduling, GPUs for training tasks, and edge accelerators \cite{yazdanbakhsh2021edge} for deployment and serving. The challenge lies in effectively orchestrating these heterogeneous accelerators to maximize their individual strengths---a complexity rarely encountered in general compute fleets.

Moreover, the heterogeneity extends beyond just accelerator type. Even within a single class of hardware accelerators, there are many different versions of the hardware, adding another layer of complexity to fleet management. Each hardware generation introduces unique features that require significant optimizations to extract peak ML workload efficiency. One notable example is the integration of the SparseCore (SC) in TPUv4 \cite{jouppi2023tpuv4opticallyreconfigurable}, which was designed to significantly boost performance for embedding-heavy models. Subsequent large-embedding model teams would likely then consider the hardware specifications of the SparseCore when designing their embedding configurations. Design points such as embedding dimension, vocabulary size, valence, and others might also be co-designed to optimize performance on the hardware platform. This demonstrates how hardware-software co-design is becoming increasingly important in improving the efficiency of these diverse accelerators, forming a symbiotic relationship where the computational needs of future workloads affect the next generation of hardware, and the hardware capabilities inform the types of workloads that the ML fleet is best equipped to handle \cite{shi2020learned}.







\subsection{Scheduler}\label{sec:scheduler}
\begin{figure}[t!]
    \centering
    \includegraphics[width=3in]{final_figs/job_size_bars.png}
    \caption{A sample breakdown of Google's ML fleet for internal workloads, segmenting on workload topology size (the number of accelerators requested by a given job). Progressive snapshots over the course of one year illustrate the ML fleet's growing share of jobs using an "extra-large" number of accelerators. This demonstrates how an ML fleet scheduler must be able to adapt to changing conditions, as the evolution of job sizes and topologies in response to shifting ML workloads presents unique challenges for the entire fleet.}
    \label{fig:job_size}
\end{figure}
The scheduler directly manages the hardware in a fleet by coordinating the allocation of resources; for the case study presented in this paper, it coordinates TPU allocations for Google's internal-facing ML workloads. There are two interconnected challenges that a scheduler must address when allocating hardware for an ML fleet: (1)~optimizing performance across various hardware types, and (2)~balancing utilization with stability and fault tolerance.

\autoref{fig:job_size} illustrates these challenges. It shows the allocation of workloads in Google's internal-facing ML fleet with different chip requirements over time, categorized into sizes based on the total number of TPU chips in the required topology. In this categorization, workloads with size "small" refer to jobs that request a single TPU or a handful of TPUs, while workloads with size "extra-large" refer to jobs that request the largest number of TPUs (often requiring multiple pods, as described in \citet{kumar2021exploring}). \autoref{fig:job_size} demonstrates that over the course of just one year, the allocation distribution can shift dramatically, reflecting the changing nature of ML workloads in the fleet. As large-scale ML models become more prevalent in an ML fleet, an increasing number of workloads will require correspondingly larger meshes of connected accelerators.







Optimizing the scheduling of jobs while meeting these resource requirements is difficult because it presents an NP-hard bin-packing problem. Each workload may specify a different accelerator type, chip topology, and location requirement and needs to be scheduled according to fleet constraints in a way that reduces overall fragmentation of the fleet. Since workloads are constantly being started and completed, the machine availability of the fleet is constantly changing, requiring a robust defragmentation algorithm. In addition, latency requirements may require accelerators for a workload to be grouped together near certain locations or data cells, adding another constraint to the scheduling optimization problem. 

The utilization of fleet resources must also be balanced with stability and fault tolerance. For example, to reduce disruptions, some machines may intentionally remain underutilized so that higher priority jobs may be more easily scheduled when needed. While high utilization is desirable for cost-efficiency, pushing hardware to its limits can lead to thermal issues, increased failure rates, and unpredictable performance. In large-scale ML fleets, hardware failures are inevitable, and the scheduler must be robust enough to handle these failures gracefully, redistributing workloads and ensuring job continuity without significant performance degradation.






\begin{figure}[t!]
    \centering
    \includegraphics[width=3in]{final_figs/life_of_mlapp.png}
    \caption{An ML workload requires all requested TPUs to be allocated before the task can start. In this example of a training workload, forward progress is saved via checkpoints. Delays during workload initialization and checkpoint writing, which are part of the Runtime and Framework layers, can reduce overall system efficiency.}
    \label{fig:life_of_mlapp}
\end{figure}

\subsection{Runtime/Compiler}

The runtime and compiler layers form an important component in the ML fleet system stack. They are responsible for bridging the gap between high-level ML models and the underlying hardware accelerators. The runtime layer focuses on the execution environment of ML programs. It handles important tasks such as program setup, data feeding, result management, and checkpoint creation, as illustrated in \autoref{fig:life_of_mlapp}. Depending on the system design, it either triggers just-in-time compilation of user-written code into accelerator-specific instructions or invokes pre-compiled operation kernels from vendor-specific libraries. The runtime layer can also manage the distribution strategy of code execution, as with notable runtimes like Pathways \cite{barham2022pathways}.

\autoref{fig:pathways} shows the growth of Pathways-based workloads in our production fleet. It highlights the demand for runtimes that support efficient distributed execution for ML workloads. It also emphasizes the rapidly shifting distribution of workload runtimes in a fleet.




The compiler layer, working with the runtime, transforms high-level ML model code into executable code optimized for specific accelerators. It operates on graph intermediate representations, applying both platform-independent and platform-dependent optimizations. The output is a program tailored to the target accelerator, such as a specific version of a TPU. Domain-specific compilers, like XLA (Accelerated Linear Algebra) \cite{xla}, have significantly improved the performance of ML workloads. For instance, in MLPerf BERT benchmarks \cite{mattson2020mlperf}, XLA demonstrated a remarkable 7$\times$ performance boost and 5$\times$ batch size improvement \cite{kumar2021exploring} over previous records, emphasizing the potential of specialized compilation techniques. We note that there are many types of accelerators, some of which do not require an explicit compiler for code generation. 

Compiler optimization in ML fleets faces unique challenges due to the rapid evolution of hardware accelerators, requiring frequent updating of optimization strategies to leverage the specific features of each new hardware generation. Moreover, the impact of optimizations can be difficult to generalize, as an optimization that improves one workload may degrade another due to differences in computation or communication patterns. This emphasizes the need for a balanced approach to optimization, considering both platform-independent techniques for flexibility and platform-specific optimizations for maximum performance.






\begin{figure}[t!]
    \centering
    \includegraphics[width=\columnwidth]{final_figs/pathways.png}
    \caption{The prevalence of fleet-wide workloads using the Pathways runtime over a sample of one year, illustrating the rapid shift of fleet-wide runtimes to accommodate changing workloads. Pathways adoption has increased rapidly, as it provides better support for distributed execution and data processing.}
    \label{fig:pathways}
\end{figure}
\subsection{Framework}

The framework layer sits on top of the runtime/compiler. It is the interface between ML practitioners and the underlying complex hardware and software infrastructure. This layer encompasses various ML frameworks and libraries, such as TensorFlow \cite{abadi2016tensorflow}, JAX \cite{frostig2018compiling}, and PyTorch \cite{paszke2019pytorch}, each offering unique features and optimizations. 

The framework layer provides high-level abstractions and APIs that allow developers to build and deploy ML models efficiently. These frameworks are responsible for translating user-written code into representations that can be understood and optimized by lower-level layers such as compilers and runtimes. This translation process bridges the gap between user intent and system execution.



One of the key responsibilities of ML frameworks is defining the structure of distributed ML applications. For example, TensorFlow's Distribution Strategy \cite{abadi2016tensorflowlargescalemachinelearning} provides a framework for distributing training across multiple devices or machines. These can have single-client or multi-client architectures, depending on workload needs, as shown in \autoref{fig:multi_single_frameworks}. These frameworks must also map ML primitives to hardware-specific designs to achieve optimal performance. This is important for specialized hardware like TPUs, which are designed for bulk-synchronous training. Frameworks like JAX are more targeted towards ML workloads, with features that facilitate ease of interpretability when analyzing ML performance, such as high-level tracing for just-in-time compilation. In the ML fleet, JAX usage has increased over time, most likely due to these features and the emergence of more ML-heavy workloads \cite{frostig2018compiling}.


In addition, ML frameworks often provide auxiliary services to improve efficiency of the entire ML fleet. For instance, TensorFlow's \texttt{tf.data} \cite{murray2021tfdata} service optimizes the performance of the data pipeline. These features, while abstracted from the user, can impact the overall system efficiency, as shown in \autoref{fig:life_of_mlapp}. Underneath these high-level frameworks lies a foundation of general-purpose libraries and datacenter services. Frameworks like TensorFlow utilize libraries such as gRPC \cite{grpc}, protobuf \cite{protobuf}, and tcmalloc \cite{tcmalloc} for various low-level operations, and interface with datacenter services for storage (e.g., Colossus \cite{ghemawat2003gfs} \cite{colossus}) and monitoring (e.g., Monarch \cite{adams2020monarch}). 


\begin{figure}[t!]
    \centering
    \includegraphics[width=.75\linewidth]{final_figs/multi_single_frameworks.png}
    \caption{
Comparing single-client frameworks with multi-client frameworks. 
    }
    \label{fig:multi_single_frameworks}
\end{figure}







As the primary point of interaction for users, the framework layer serves as a key bridge in the ML system stack, as it not only abstracts underlying complexities but also plays an important role in determining the overall efficiency and capabilities of the fleet. Frameworks must balance the need for user-friendly APIs with the need to leverage underlying hardware-specific optimizations, while also managing the complexities of distributed computing, data pipeline optimization, and integration with lower-level services. 









\subsection{ML Model \& Data}

To characterize the ML Fleet at the highest level of the stack, we generally want to know: What types of workloads are we spending most of our compute cycles on? This is a critical question because it drives nearly every design decision we make for the ML Fleet at every level of the stack, from the hardware (how many training vs. inference vs. other chips) to the software (JAX vs. other frameworks, runtime distribution strategies, and compiler optimizations). Workload heterogeneity analysis is useful for understanding what kind of models are prevalent in the production fleet, especially since different workloads stress the hardware in different ways. This understanding can drive decisions about which accelerators to deploy and how many, or which compiler optimizations to carry out.







In practice, we observe that the model and data layer of the ML fleet stack are the most affected by fluctuating user demands. User requirements such as the model architecture, size of the training dataset, or even use of different numerical formats in the training model can impact the efficiency of the job, which can have a cascading effect on the efficiency of the overall ML fleet.

While ML workloads share some computational patterns, particularly in their use of matrix operations and data-intensive processing, the specific architectures and resource requirements can vary significantly.  As new model architectures and learning tasks emerge, they prompt rapid shifts in workload composition, leading to fluctuations in resource demands across various model types. 

In a production ML fleet, there are varying proportions of workloads dedicated to each phase of the ML model life cycle; training, bulk inference, and real-time serving. Thus, the fleet must be flexible enough to handle the requirements of each of these phases; for example, training workloads may be compute intensive while real-time serving workloads may focus on minimizing latency.

















\section{ML Productivity Goodput}\label{sec:goodput}

The optimization of ML fleet efficiency is a complex, cyclical challenge, as illustrated in \autoref{fig:ml_stack}. The first challenge is measuring, understanding, and reporting fleetwide efficiency, establishing a baseline for current performance. Second, we must identify and quantify fleet-wide inefficiencies, pinpointing areas that require improvement. Third, we must eliminate these inefficiencies by implementing changes across the fleet, which in turn leads back to the first stage as we measure the impact of these changes. This cycle ensures ongoing optimization and adaptation to the ever-evolving landscape of ML workloads and hardware. To keep pace with this lifecycle, we require a metric that not only quantifies current performance but also guides future optimization efforts across the fleet, which is why Google has developed the \mpg metric. %

In this section, we present an in-depth discussion of MPG, a new metric for quantifying ML fleet efficiency. We refer to this as the iron law of performance for ML fleets, drawing a corollary to the iron law of processor performance~\cite{emer1984ironlaw}. MPG, defined in \autoref{fig:mpg}, is a means for measuring efficiency gains and guiding exploration of optimization strategies across various fleet components.


\begin{figure}[t!]
    \centering
    \includegraphics[width=\columnwidth]{final_figs/mpg.png}
    \caption{ML Productivity Goodput (MPG) and its components.}
    \label{fig:mpg}
\end{figure}





\subsection{Pitfalls \& Myths of Traditional Metrics}


Before we set the stage for the new metric, we examine the common pitfalls of historical approaches for fleetwide measurement. \autoref{fig:utilization} illustrates how computer architects have historically tended to think about performance metrics \cite{li2023analyzing, mars2011bubble,kanev2015profiling}. These traditional performance metrics can sometimes fall short in providing a holistic view, given the unique challenges we have discussed in \autoref{sec:background} and \autoref{sec:fleet}. 





\textbf{Myth 1: High \underline{Capacity} equates to high resource availability.}
While capacity can tell us how many individual accelerators may be available in the fleet at a given time, it does not take into account the topological shape of those accelerators. For example, an ML training workload requesting thousands of chips in a certain physical mesh shape may never be scheduled if the only available accelerators are fragmented across different clusters or data centers. Other factors, such as the geographical location of data storage cells and accelerators, are not included in the capacity metric, even though they significantly affect job scheduling. Therefore, high capacity by itself as a metric does not necessarily effectively translate to high availability for workloads, and we should instead opt for scheduling efficiency as a more robust metric.

\textbf{Myth 2: High \underline{Occupancy} guarantees productivity.}
Occupancy is defined as the fraction of accelerators allocated to jobs and is often measured by the scheduler (e.g. Borg \cite{verma2015borg}). Occupancy is traditionally seen as a key efficiency indicator, but it can be misleading as it masks inefficiencies in the system stack. For example, an accelerator might be successfully allocated but stuck in I/O wait or running poorly optimized code, thus resulting in a high occupancy but very little actual progress being made towards the workload task. This is important for long-running tasks such as ML model training, where frequent pre-emptions may hinder checkpoint progress but still result in a nominally high occupancy. The traditional occupancy metric therefore does not distinguish between productive and unproductive use of allocated resources.

\textbf{Myth 3: \underline{Duty Cycle} accurately represents useful work.}
Duty Cycle measures whether an accelerator is in use, not how much of its compute capacity is used. When looking at an ML workload running on a TPU, duty cycle does not provide any signal on how much the matrix-multiply units (MXUs) are utilized \cite{jouppi2023tpuv4opticallyreconfigurable}. It is agnostic of the program-level efficiency and does not take into account the effectiveness of the operations being performed. An accelerator could have a high duty cycle while executing unnecessary or redundant computations. So, we require a more sophisticated metric.

\begin{figure}[t]
\centering
\includegraphics[width=\columnwidth]{final_figs/efficiency_util.png}
\caption{Historical utilization-based fleet efficiency metrics. We propose replacing this approach and using goodput as a measure of fleet efficiency rather than utilization.}
\label{fig:utilization}
\end{figure}

\textbf{The Overarching Misconception: Utilization Equals Productivity.}
The common thread among these metrics is the assumption that keeping accelerators busy equates to productive work. However, this overlooks critical factors. (1) \textit{Quality of Computations:} None of these metrics assess whether the operations being performed are actually contributing to the desired output. (2) \textit{Workload Efficiency:} They do not consider whether the workloads are optimally designed for the hardware. (3) \textit{System-level Bottlenecks:} Focusing solely on accelerator usage ignores potential bottlenecks in data loading, memory access, or inter-accelerator communication. (4) \textit{Forward Progress:} These metrics provide no insight into how much useful work is being accomplished towards completing an actual ML task.







\subsection{Metric Features}
Ideally, the MPG metric must be a clearly defined and accurate measure of forward progress; improvements in the metric must also reflect real improvements in the efficiency of the fleet. This metric must be capable of overcoming two significant challenges.

\begin{enumerate}
\item \textbf{It must capture the dynamic nature of ML fleets}: The fleet is constantly fluctuating due to variables such as changes in workload composition, updates to the code stack, and evolving hardware. To effectively improve efficiency, we must ensure that any change in the metric is explainable despite these fluctuating variables.
\item \textbf{It must explain the trade-offs between individual and aggregate efficiency}. At a fleetwide scale, jobs must be scheduled in concert with one another to ensure maximum aggregate efficiency of the fleet. However, individual jobs may have certain service-level requirements, meaning that this metric must be decomposable based on workload characteristics.
\end{enumerate}

\begin{figure}[t]
    \centering
    \includegraphics[width=\columnwidth]{final_figs/mpg_breakdown.png}
    \caption{Breakdown of a ML workload using \mpg. 
    }
    \label{fig:mpg_breakdown}
\end{figure}

\subsection{A New Approach: ML Productivity Goodput}

\mpg (MPG) is designed to address the myriad challenges discussed in Section~\ref{sec:fleet}, as well as to overcome the limitations of existing approaches. Just as the Iron Law of Processor Performance~\cite{emer1984ironlaw} breaks down CPU performance into $\frac{instructions}{program}$$\times$ $\frac{cycles}{instruction}$$\times$$\frac{time}{cycle}$, the MPG metric decomposes ML fleet efficiency into scheduling, runtime, and program components (see \autoref{fig:mpg}).



This multi-layered structure, as illustrated in \autoref{fig:mpg_breakdown}, offers several advantages over the traditional metrics. First, it allows for precise identification of performance bottlenecks or improvements at specific layers of the stack, facilitates a more granular analysis of efficiency trends over time, and mitigates the risk of misleading interpretations that can arise from aggregated data, such as Simpson's paradox.\footnote{A statistical phenomenon where a trend that is evident within individual groups disappears or reverses when the population groups are combined.} Second, by decoupling these submetrics, we enable more targeted optimization efforts and gain deeper insights into the complex interactions within the ML fleet. Finally, this approach not only enhances our ability to measure current performance but also provides a framework for guiding improvements, discussed in \autoref{sec:improvements}.  

\textbf{Scheduling Goodput:}
\emph{How often does an application have all necessary resources to make progress?} 

Scheduling Goodput (SG) quantifies the efficiency of resource allocation in an ML fleet. It measures the fraction of time that an application has all the required resources simultaneously available to make progress. This metric can be lower than traditional Occupancy, particularly in distributed, bulk-synchronous applications where all required chips must be available concurrently. The numerator of SG is calculated as the simultaneous uptime of all tasks in a distributed ML application that must be connected to make synchronous progress, as shown in \autoref{fig:scheduling}. This is referred to as ``allocated chip-time'' or ``all-allocated'' time. The denominator is fleet capacity, expressed as chip-time. This provides a full view of how effectively the scheduling layer is using the fleet's resources. 

Scheduling Goodput offers insights into potential inefficiencies in resource allocation, such as fragmentation of available resources, delays in coordinating multiple chips for distributed applications, and mismatches between application requirements and available resources. By optimizing SG, we can improve the overall efficiency of resource utilization in the ML fleet, ensuring that applications have the necessary resources to make consistent progress.

\begin{figure}[t!]
    \centering
    \includegraphics[width=\columnwidth]{final_figs/scheduling.png}
    \caption{The scheduling goodput for training workloads measures the percentage of time when all of the TPU workers are available to work at the same time. In other words, it measures the portion of time that all of the necessary resources are available to make progress. 
}
    \label{fig:scheduling}
\end{figure}

\textbf{Runtime Goodput:}
\emph{Of the time that an application has all necessary resources, how often is it making progress?} 

\rg (RG) measures the efficiency of the orchestration layers in managing the execution of ML applications once resources are allocated. This metric focuses on the actual productive time of an application, accounting for various overheads in the runtime environment. The orchestration layer is responsible for critical tasks such as initializing chips, connecting them into slices for bulk-synchronous progress, loading and compiling programs, feeding data to these programs, and ensuring that training progress is regularly saved through checkpoints. The numerator of RG is the productive chip-time of the application's progress that has been saved in checkpoints; work done between the last checkpoint and failure (or preemption) doesn't count as "productive" time and is therefore not included in RG. The denominator of RG is the allocated chip-time defined as the numerator of SG. 

\rg can help with identifying bottlenecks in the runtime environment, such as slow data loading, inefficient checkpointing, or suboptimal program compilation. It can guide the efforts to streamline the execution pipeline and improve the overall throughput of ML workloads.

\textbf{Program Goodput:}
\emph{Of the time that an application is making progress, how close is it to the ideal roofline?}

Program Goodput (PG) assesses the efficiency of the application code itself, measuring how effectively it utilizes the available computational resources. While a traditional roofline performance model~\cite{williams2009roofline} might seem suitable for this purpose, it falls short in capturing the true efficiency of modern ML workloads. The traditional roofline model is highly sensitive to compiler decisions, such as how ML operators are fused or rematerialized \cite{briggs1992rematerialization}, or which operands are placed in which memory space. It rewards individual ops that are close to peak utilization, but penalizes correct optimizations that result in computation graphs where the utilization may be lower, but overall execution time is shorter. 

To overcome these limitations, we use a compute-based roofline model that compares the ideal execution time of the workload against its actual execution time. The ideal predicted execution time, which is the numerator of PG, can be computed from intrinsic properties of the machine learning model being run. By analyzing the shape of the unoptimized  high-level operations (HLO) graph, we can estimate how many floating point operations (FLOPs) the program would require at its theoretical peak performance. Since we are analyzing the computation graph before any compiler optimizations, this prediction is agnostic to compiler decisions. 

The denominator of PG is the actual execution time. The PG metric can thus be interpreted as a percentage reflecting how well optimized the ML program is, with a score of 100\% indicating perfect performance matching the theoretical peak.





\section{Improving Fleetwide Efficiency}\label{sec:improvements}
 \begin{figure}[t]
    \centering
    \includegraphics[height=2in]{final_figs/pg_cl.png}
    \caption{This figure demonstrates the effect of an XLA algebraic simplification optimization on Program Goodput (PG) across a benchmark of the top 150 fleet workloads. Looking at the PG in this way allows us to bisect which code changes improved or regressed overall fleet efficiency.}
    \label{fig:pg_cl}
\end{figure}

We show how \mpg is a robust quantifier of ML Fleet performance through optimization examples from Google's ML Fleet in production. We present a breakdown of the MPG components using segmented fleet data and demonstrate how this procedure can help identify potential optimization techniques. Additionally, we showcase the effects of deploying these optimizations and how MPG helps verify and track performance improvements.

Looking at the aggregated MPG of the fleet does not necessarily help ML practitioners identify what kinds of improvements will make the largest impact on the fleet; this is where the decomposability of the metric comes into play, as shown in \autoref{tab:improvements}. By breaking MPG into its three components; Program Goodput, Runtime Goodput, and Scheduling Goodput, we can diagnose fleetwide issues and identify the types of optimizations that would most improve fleet efficiency. Furthermore, we can segment the fleet using the characteristics described in Section~\ref{sec:fleet} in order to identify issues with specific workload types and propose model-level optimizations. 


\renewcommand*{\arraystretch}{1.25}
\begin{table*}[h!]
\caption{Optimizing different components of \mpg.}
\label{tab:improvements}
\resizebox{\textwidth}{!}{%
\scriptsize
\begin{tabular}{@{}lp{2cm}p{3cm}p{3cm}p{4cm}@{}}
\toprule
\textbf{ML Fleet Stack Layer} & \textbf{Program Goodput $\times$} & \textbf{Runtime Goodput $\times$} & \textbf{Scheduling Goodput $=$} & \textbf{Workload \mpg} \\ \bottomrule
\textbf{Compiler:} \\ On-duty step time decreases &  \textbf{Increases} & Decreases if device-bound \newline Decreases if host-bound & Decreases if device-bound \newline No change if host-bound & \textbf{Increases if device-bound}  \newline No change if host-bound
 \\ \midrule
\textbf{Runtime:} \\ Off-duty time or preemption waste decreases & No change & \textbf{Increases} & Decreases & \textbf{Increases} \\ \midrule
\textbf{Scheduler:} \\ Partially-allocated time decreases & No change & No change & \textbf{Increases}  & \textbf{Increases} 
\\ \bottomrule
\end{tabular}%
}
\end{table*}

\subsection{Program Goodput Optimizations}
We present various techniques and strategies that have been employed at \google over recent years for improving Program Goodput in our ML fleets, ranging from parallelization methods to compiler optimizations. Recall that PG measures the effective utilization of computational resources. As ML models grow in size and complexity, optimizing PG becomes increasingly important to make efficient use of hardware and reduce compute times. With PG instrumentation, we have been able to pinpoint which segments of the fleet require further optimization at the compiler or ML model level. 

 \begin{figure}[t]
    \centering
    \includegraphics[height=2in]{final_figs/pg_chip.png}
    \caption{Tracking the Program Goodput (PG) versus allocation trends for a particular domain-specific chip in an ML fleet. Looking at the disaggregated segments of MPG can help reveal fleetwide trends and interactions between different layers in the ML fleet stack, informing future design decisions.}
    \label{fig:pg_chip_type}
\end{figure}


 



 \textbf{Overlapping communication and computation.}
To identify potential system optimizations that can improve fleet efficiency, we can look at the PG of workloads segmented by performance characteristics. In other words, how many of the workloads in the fleet are compute-bound versus communication-bound? By segmenting the PG in this way, it is possible for us to identify that many high-cost workloads are communication-bound. 

To address this issue at the high-level operation (HLO) level, a technique that overlaps communication with computation was developed and deployed in our production fleet (described in ~\citet{wang2022overlap}). This technique decomposes communication collectives, along with the dependent computation operations, into a sequence of finer-grained operations to hide data transfer latency so that better system utilization is achieved. This approach improved the overall system throughput by up to 1.38$\times$ and achieves 72\% FLOPS utilization on 1024 TPU chips for a large language model with 500 billion parameters~\cite{wang2022overlap}.

\textbf{Compiler autotuning.}
At the fleet level, we have also developed and deployed optimizations that improve code-generation quality and can be generalized to any workload in the fleet. XTAT~\cite{phothilimthana2021flexible} is an autotuner for production ML compilers that tunes multiple compiler stages, including tensor layouts, operator fusion decisions, tile sizes and code generation parameters. Evaluated over 150 ML training and inference models on TPUs, XTAT offers speedups over the heavily-optimized XLA compiler in the fleet.

\textbf{Example: Quantifying the impact of an XLA optimization on the TPU fleet.}
It is rare for any single optimization to have a significant impact on overall fleet-wide PG. But we can track the impact of these optimizations by looking at the change in PG for a fixed set of benchmarked workloads or segment of the production fleet over time. For example, looking at a benchmark of the top 150 most costly workloads in the fleet, \autoref{fig:pg_cl} pinpoints the effect of a code change that was submitted to the XLA compiler - in this case, an algebraic simplification in the compiler graph. The dramatic increase in PG for the benchmark of 150 workloads suggests that the positive impact of this optimization can be generalized to the ML fleet as a whole. 

It is also helpful to look at PG fluctuations across hardware segments of the fleet. \autoref{fig:pg_chip_type} illustrates a notional example where looking at segmented PG can uncover insights that would otherwise be hidden by looking at aggregate metrics. In this case, the segmented data suggests that when a new ML accelerator chip is introduced to the fleet, the workloads running on that chip may initially have a low PG, since the model / compiler code has not been fully tailored for that chip yet. As user adoption increases and accelerator-specific software optimizations are rolled out to the fleet, PG gets closer to theoretical peak efficiency. In other words, hardware accelerator maturity tends to yield greater PG over time. As the chip nears the end of its lifecycle (represented by decreasing allocation in the fleet, and illustrated in \autoref{fig:pg_chip_type} by the ``Chip decommissioned'' label), the PG decreases due to lower chip usage and natural workload/compiler drift. This highlights the importance of co-design across all layers of the ML fleet to make sure that both the software (compiler) and hardware (chips) are optimized for the latest workloads.
\subsection{Runtime Goodput Optimizations}
Outside of device time, host overhead and pre-emptions can be major bottlenecks for some workloads, and can be tracked by measuring Runtime Goodput. For example, training jobs usually use input pipelines to ingest and transform input data, which could be bottlenecks for certain models that ingest large amounts of data. Some solutions have been proposed to reduce host overhead, such as Plumber~\cite{kuchnik2022plumber}, a tool to find bottlenecks in ML input pipelines. 

We can improve the RG of the ML fleet with asynchronous strategies such as sharding the dataflow graph, as proposed by Pathways \cite{barham2022pathways}. The segmented analysis of RG shows that the particular workloads on Pathways tend to have higher RG scores over time, validating the benefits of Pathways for our particular ML Fleet. Also, techniques such as asynchronous checkpointing \citep{maurya2024datastates, nicolae2020deepfreeze} can reduce the time spent fetching previous model training checkpoints where the accelerators temporarily pause training and are completely idle.  


Other strategies, such as ahead-of-time compilation, where programs are compiled on less expensive hardware such as CPUs and then executed on TPUs, can also improve RG. By offloading compilation to a less expensive chip and storing the results in a compilation cache, we can reduce the total runtime of more specialized accelerators. These techniques are often implemented in common ML frameworks, such as TensorFlow \cite{aotTF} and JAX \cite{aotJAX}.


To demonstrate the benefits of these framework-specific optimizations, we can examine the RG of the ML Fleet at the framework level of the system stack described in \autoref{fig:ml_stack}. This can help us understand which frameworks or runtime strategies may be better suited for which workloads.
\autoref{fig:segmented_runtime_goodput} shows RG scores from a sample of Google's ML fleet for internal workloads, segmented based on characteristics such as model architecture, product area or workload phase (training, real-time serving, or bulk inference), and compared to a baseline of top fleet workloads. Although the segments in \autoref{fig:segmented_runtime_goodput} are not explicitly identified, we demonstrate how segmenting RG based on workload characteristics (Segment~A, Segment~B, and Segment~C) can reveal trends that would otherwise be hidden by aggregate fleet metrics (represented by the "Top Fleet Workloads" segment). For example, training workloads running JAX with Pathways may tend to have a higher RG, possibly due to the fact that Pathways is single-client \cite{barham2022pathways} and therefore better optimized for training than multi-client frameworks.

\begin{figure}[t]
    \centering
    \includegraphics[width=\columnwidth]{final_figs/segmented_runtime_goodput.png}
    \caption{Runtime goodput speedups over the course of one quarter, segmented by fleet workload types. Speedup is normalized to the top N workloads in the fleet, measured at the beginning of the quarter. }
    \label{fig:segmented_runtime_goodput}
\end{figure}

Examining the data along a different axis, training versus real-time serving versus bulk inference, can also be helpful. Using a sample from Google's ML fleet, \autoref{fig:train_inf_rg} illustrates that training workloads tend to have a higher RG than serving workloads. This is most likely due to the inherently different nature of serving and training.  Typically, training workloads have more constant computational demands, while real-time serving can fluctuate based on user demand. The slight decrease in serving RG can be attributed to transitory demands on the fleet, but it remains relatively stable in comparison to the bulk inference segment. The huge fluctuation in RG for bulk inference highlights the changing nature of production fleet demands. Previously, the bulk inference segment of the fleet was dominated by workloads running on a single core, where each chip contained a replica of the model, resulting in more easily accessible checkpoints/data and less accelerator wait time. However, as we move to larger models, the weights must be sharded across multiple chips, resulting in more expensive data reads. Additionally, the rise of expert-based models \cite{shazeer2017moe} has made bulk inference runtime much more complex to optimize, as some machines must wait for others for distillation of weight updates in a student-teacher model. This has resulted in an temporary decrease of RG for the bulk inference segment between "Month 3" and "Month 6" of \autoref{fig:train_inf_rg}. This example illustrates how analyzing disaggregated RG can allow ML fleet architects to make informed decisions about their runtime stack by pinpointing segments that may be more susceptible to shifting fleet demands.

\begin{figure}[t]
    \centering
    \includegraphics[height=1.8in]{final_figs/train_inf_rg.png}
    \caption{Runtime Goodput trends for a notional slice of a sample ML fleet over a period of six months, segmented by workload phase.}
    \label{fig:train_inf_rg}
\end{figure}



\subsection{Scheduling Goodput Optimizations}

The optimization of Scheduling Goodput (SG) can be presented as a bin packing problem, as described in \autoref{sec:scheduler}. Users launch workloads with varying TPU topology sizes, and the scheduling algorithm must determine how to best fit these workloads into the existing fleet of allocated chips. This process presents numerous challenges, primarily due to the wide variety of job sizes in the fleet, ranging from single-chip to multipod configurations \cite{kumar2021exploring}.

A significant complication in job scheduling is that it requires more than mere availability in the fleet. The topology of the available hardware must also satisfy the topology requirements of the workload, which is sometimes impossible without first pre-empting other jobs. Consequently, suboptimal scheduling can have a cascading effect on other components of \mpg.



We can identify availability issues in the ML Fleet by looking at the SG for jobs with different chip allocation requirements, as shown in \autoref{fig:sg_job_size}. The data shows that the overall SG is already close to optimal, due to defragmentation techniques and scheduling optimizations. However, it is interesting to note which jobs tend to have the highest SG: extra-large jobs which require the greatest number of chips or possibly multiple TPU pods, as well as smaller jobs which require only a single chip or a few chips. 

This is likely due to the way the scheduler deals with evictions; evicting extremely large jobs would have a severe negative impact on the overall \mpg score due to their huge startup overhead. Once the extra-large job is running, it is also immensely dependent on checkpointing and data sharding, affecting the Runtime and Program Goodput components as well. In short, evicting extra-large jobs from the hardware would present a cascading series of failures, strongly incentivizing the scheduler to reduce churn for these jobs and evict medium-sized jobs instead. 

On the other hand, extremely small jobs usually do not get prematurely evicted since they are more likely to finish quickly, and if pre-empted, it is usually quicker to find topologically matching availability. With extremely small jobs, the scheduler has more flexibility to intelligently allocate the workloads to optimal compute cells in order to defragment the overall ML fleet availability. It is also important to note that for workloads of all sizes, the SG is greater than 95\% due to the particular pre-emption preferences of the scheduler. The pre-emption preferences of the scheduler can be tuned, e.g. to require a SG of greater than 95\% for medium-sized jobs, but this could reduce the SG for other segments of the fleet.


\begin{figure}[t]
    \centering
    \includegraphics[height=1.8in]{final_figs/sg_job_size.png}
    \caption{Scheduling goodput by job size. Extra-large and small jobs tend to have better scheduling goodput due to the scheduler's preemption algorithm.}
    \label{fig:sg_job_size}
\end{figure}


\section{Discussion of Assumptions}\label{sec:discussion}
In this paper, we have made several assumptions for the sake of clarity and simplicity. In this section, we discuss the rationale behind these assumptions, the extent to which these assumptions hold in practice, and the consequences for our protocol when these assumptions hold.

\subsection{Assumptions on the Demand}

There are two simplifying assumptions we make about the demand. First, we assume the demand at any time is relatively small compared to the channel capacities. Second, we take the demand to be constant over time. We elaborate upon both these points below.

\paragraph{Small demands} The assumption that demands are small relative to channel capacities is made precise in \eqref{eq:large_capacity_assumption}. This assumption simplifies two major aspects of our protocol. First, it largely removes congestion from consideration. In \eqref{eq:primal_problem}, there is no constraint ensuring that total flow in both directions stays below capacity--this is always met. Consequently, there is no Lagrange multiplier for congestion and no congestion pricing; only imbalance penalties apply. In contrast, protocols in \cite{sivaraman2020high, varma2021throughput, wang2024fence} include congestion fees due to explicit congestion constraints. Second, the bound \eqref{eq:large_capacity_assumption} ensures that as long as channels remain balanced, the network can always meet demand, no matter how the demand is routed. Since channels can rebalance when necessary, they never drop transactions. This allows prices and flows to adjust as per the equations in \eqref{eq:algorithm}, which makes it easier to prove the protocol's convergence guarantees. This also preserves the key property that a channel's price remains proportional to net money flow through it.

In practice, payment channel networks are used most often for micro-payments, for which on-chain transactions are prohibitively expensive; large transactions typically take place directly on the blockchain. For example, according to \cite{river2023lightning}, the average channel capacity is roughly $0.1$ BTC ($5,000$ BTC distributed over $50,000$ channels), while the average transaction amount is less than $0.0004$ BTC ($44.7k$ satoshis). Thus, the small demand assumption is not too unrealistic. Additionally, the occasional large transaction can be treated as a sequence of smaller transactions by breaking it into packets and executing each packet serially (as done by \cite{sivaraman2020high}).
Lastly, a good path discovery process that favors large capacity channels over small capacity ones can help ensure that the bound in \eqref{eq:large_capacity_assumption} holds.

\paragraph{Constant demands} 
In this work, we assume that any transacting pair of nodes have a steady transaction demand between them (see Section \ref{sec:transaction_requests}). Making this assumption is necessary to obtain the kind of guarantees that we have presented in this paper. Unless the demand is steady, it is unreasonable to expect that the flows converge to a steady value. Weaker assumptions on the demand lead to weaker guarantees. For example, with the more general setting of stochastic, but i.i.d. demand between any two nodes, \cite{varma2021throughput} shows that the channel queue lengths are bounded in expectation. If the demand can be arbitrary, then it is very hard to get any meaningful performance guarantees; \cite{wang2024fence} shows that even for a single bidirectional channel, the competitive ratio is infinite. Indeed, because a PCN is a decentralized system and decisions must be made based on local information alone, it is difficult for the network to find the optimal detailed balance flow at every time step with a time-varying demand.  With a steady demand, the network can discover the optimal flows in a reasonably short time, as our work shows.

We view the constant demand assumption as an approximation for a more general demand process that could be piece-wise constant, stochastic, or both (see simulations in Figure \ref{fig:five_nodes_variable_demand}).
We believe it should be possible to merge ideas from our work and \cite{varma2021throughput} to provide guarantees in a setting with random demands with arbitrary means. We leave this for future work. In addition, our work suggests that a reasonable method of handling stochastic demands is to queue the transaction requests \textit{at the source node} itself. This queuing action should be viewed in conjunction with flow-control. Indeed, a temporarily high unidirectional demand would raise prices for the sender, incentivizing the sender to stop sending the transactions. If the sender queues the transactions, they can send them later when prices drop. This form of queuing does not require any overhaul of the basic PCN infrastructure and is therefore simpler to implement than per-channel queues as suggested by \cite{sivaraman2020high} and \cite{varma2021throughput}.

\subsection{The Incentive of Channels}
The actions of the channels as prescribed by the DEBT control protocol can be summarized as follows. Channels adjust their prices in proportion to the net flow through them. They rebalance themselves whenever necessary and execute any transaction request that has been made of them. We discuss both these aspects below.

\paragraph{On Prices}
In this work, the exclusive role of channel prices is to ensure that the flows through each channel remains balanced. In practice, it would be important to include other components in a channel's price/fee as well: a congestion price  and an incentive price. The congestion price, as suggested by \cite{varma2021throughput}, would depend on the total flow of transactions through the channel, and would incentivize nodes to balance the load over different paths. The incentive price, which is commonly used in practice \cite{river2023lightning}, is necessary to provide channels with an incentive to serve as an intermediary for different channels. In practice, we expect both these components to be smaller than the imbalance price. Consequently, we expect the behavior of our protocol to be similar to our theoretical results even with these additional prices.

A key aspect of our protocol is that channel fees are allowed to be negative. Although the original Lightning network whitepaper \cite{poon2016bitcoin} suggests that negative channel prices may be a good solution to promote rebalancing, the idea of negative prices in not very popular in the literature. To our knowledge, the only prior work with this feature is \cite{varma2021throughput}. Indeed, in papers such as \cite{van2021merchant} and \cite{wang2024fence}, the price function is explicitly modified such that the channel price is never negative. The results of our paper show the benefits of negative prices. For one, in steady state, equal flows in both directions ensure that a channel doesn't loose any money (the other price components mentioned above ensure that the channel will only gain money). More importantly, negative prices are important to ensure that the protocol selectively stifles acyclic flows while allowing circulations to flow. Indeed, in the example of Section \ref{sec:flow_control_example}, the flows between nodes $A$ and $C$ are left on only because the large positive price over one channel is canceled by the corresponding negative price over the other channel, leading to a net zero price.

Lastly, observe that in the DEBT control protocol, the price charged by a channel does not depend on its capacity. This is a natural consequence of the price being the Lagrange multiplier for the net-zero flow constraint, which also does not depend on the channel capacity. In contrast, in many other works, the imbalance price is normalized by the channel capacity \cite{ren2018optimal, lin2020funds, wang2024fence}; this is shown to work well in practice. The rationale for such a price structure is explained well in \cite{wang2024fence}, where this fee is derived with the aim of always maintaining some balance (liquidity) at each end of every channel. This is a reasonable aim if a channel is to never rebalance itself; the experiments of the aforementioned papers are conducted in such a regime. In this work, however, we allow the channels to rebalance themselves a few times in order to settle on a detailed balance flow. This is because our focus is on the long-term steady state performance of the protocol. This difference in perspective also shows up in how the price depends on the channel imbalance. \cite{lin2020funds} and \cite{wang2024fence} advocate for strictly convex prices whereas this work and \cite{varma2021throughput} propose linear prices.

\paragraph{On Rebalancing} 
Recall that the DEBT control protocol ensures that the flows in the network converge to a detailed balance flow, which can be sustained perpetually without any rebalancing. However, during the transient phase (before convergence), channels may have to perform on-chain rebalancing a few times. Since rebalancing is an expensive operation, it is worthwhile discussing methods by which channels can reduce the extent of rebalancing. One option for the channels to reduce the extent of rebalancing is to increase their capacity; however, this comes at the cost of locking in more capital. Each channel can decide for itself the optimum amount of capital to lock in. Another option, which we discuss in Section \ref{sec:five_node}, is for channels to increase the rate $\gamma$ at which they adjust prices. 

Ultimately, whether or not it is beneficial for a channel to rebalance depends on the time-horizon under consideration. Our protocol is based on the assumption that the demand remains steady for a long period of time. If this is indeed the case, it would be worthwhile for a channel to rebalance itself as it can make up this cost through the incentive fees gained from the flow of transactions through it in steady state. If a channel chooses not to rebalance itself, however, there is a risk of being trapped in a deadlock, which is suboptimal for not only the nodes but also the channel.

\section{Conclusion}
This work presents DEBT control: a protocol for payment channel networks that uses source routing and flow control based on channel prices. The protocol is derived by posing a network utility maximization problem and analyzing its dual minimization. It is shown that under steady demands, the protocol guides the network to an optimal, sustainable point. Simulations show its robustness to demand variations. The work demonstrates that simple protocols with strong theoretical guarantees are possible for PCNs and we hope it inspires further theoretical research in this direction.
\section{Conclusion}
In this work, we propose a simple yet effective approach, called SMILE, for graph few-shot learning with fewer tasks. Specifically, we introduce a novel dual-level mixup strategy, including within-task and across-task mixup, for enriching the diversity of nodes within each task and the diversity of tasks. Also, we incorporate the degree-based prior information to learn expressive node embeddings. Theoretically, we prove that SMILE effectively enhances the model's generalization performance. Empirically, we conduct extensive experiments on multiple benchmarks and the results suggest that SMILE significantly outperforms other baselines, including both in-domain and cross-domain few-shot settings.

\bibliographystyle{ACM-Reference-Format}
\bibliography{refs}

\end{document}
