\documentclass[sigconf]{acmart}
\acmYear{2025}
\acmConference[arXiv]{February}{February}{2025}
\acmISBN{$^{*}$Corresponding author: arissa@google.com}
\setcopyright{none}
\copyrightyear{2025}
\settopmatter{printfolios=true}
\settopmatter{printacmref=false}
% \usepackage[dvipsnames]{xcolor}
\usepackage{algorithmic}
\usepackage{textcomp}
\usepackage{fancyhdr}
\usepackage{hyperref}
\usepackage{multirow}
\usepackage{graphicx}  %
\usepackage{multirow}  %
\usepackage{array}     %
\usepackage{booktabs}  %
\usepackage{amsmath}
\usepackage{circledsteps}
\usepackage{xspace}
\usepackage[font=footnotesize]{caption}

\hypersetup{
    colorlinks=true,
    linkcolor=blue,
    filecolor=magenta,      
    urlcolor=cyan,
    citecolor=blue,
    anchorcolor=blue,
    pdftitle={Overleaf Example},
    pdfpagemode=FullScreen,
}
\usepackage{microtype}
\usepackage{subfigure}

\newcommand{\theHalgorithm}{\arabic{algorithm}}

\newcommand{\todo}[1]{{\color{red} #1}}

\setlength{\belowcaptionskip}{-10pt}

\AtBeginDocument{%
  \providecommand\BibTeX{{%
    Bib\TeX}}}

\def\sectionautorefname{Section}

\begin{document}

\title{Machine Learning Fleet Efficiency: Analyzing and Optimizing Large-Scale Google TPU Systems with ML Productivity Goodput\\[.25em] \subtitle{\emph{A Preprint}}}

\author{\Large
    Arissa Wongpanich$^{\dagger}$
    Tayo Oguntebi$^{\dagger}$
    Jose Baiocchi Paredes$^{\dagger}$\\
    Yu Emma Wang$^{\dagger}$
    Phitchaya Mangpo Phothilimthana$^{\dagger}$
    Ritwika Mitra$^{\dagger}$\\
    Zongwei Zhou$^{\dagger}$
    Naveen Kumar$^{\dagger}$
    Vijay Janapa Reddi$^{\dagger\ddagger}$ \\[1em]
    $^{\dagger}$Google \hspace{1cm} $^{\ddagger}$Harvard University\\[1em]
}

  \renewcommand{\shortauthors}{}



\begin{abstract}
Recent years have seen the emergence of machine learning (ML) workloads deployed in warehouse-scale computing (WSC) settings, also known as ML fleets. As the computational demands placed on ML fleets have increased due to the rise of large models and growing demand for ML applications, it has become increasingly critical to measure and improve the efficiency of such systems. However, there is not yet an established methodology to characterize ML fleet performance and identify potential performance optimizations accordingly. This paper presents a large-scale analysis of an ML fleet based on Google's TPUs, introducing a framework to capture fleet-wide efficiency, systematically evaluate performance characteristics, and identify optimization strategies for the fleet. We begin by defining an ML fleet, outlining its components, and analyzing an example Google ML fleet in production comprising thousands of accelerators running diverse workloads. Our study reveals several critical insights: first, ML fleets extend beyond the hardware layer, with model, data, framework, compiler, and scheduling layers significantly impacting performance; second, the heterogeneous nature of ML fleets poses challenges in characterizing individual workload performance; and third, traditional utilization-based metrics prove insufficient for ML fleet characterization. To address these challenges, we present the ``ML Productivity Goodput'' (MPG) metric to measure ML fleet efficiency. We show how to leverage this metric to characterize the fleet across the ML system stack. We also present methods to identify and optimize performance bottlenecks using MPG, providing strategies for managing warehouse-scale ML systems in general. Lastly, we demonstrate quantitative evaluations from applying these methods to a real ML fleet for internal-facing Google TPU workloads, where we observed tangible improvements.

\end{abstract}




\newcommand{\mpg}{ML Productivity Goodput\xspace}
\newcommand{\sg}{Scheduling Goodput\xspace}
\newcommand{\pg}{Program Goodput\xspace}
\newcommand{\rg}{Runtime Goodput\xspace}
\newcommand{\google}{Google\xspace}
\maketitle
\pagestyle{plain}

\section{Introduction}
\label{sec:intro}

\begin{figure*}[tb]
    \centering
    \includegraphics[width=0.848\linewidth]{figs/circuitnn.pdf} 
    \caption{Illustration of differentiable CircuitNN. CircuitNN is designed based on differentiable NAND gates. After DAS is guided by PI and PO pairs of the truth table, CircuitNN can get the precise circuit architecture logic equivalent to the truth table.}
    \label{fig:circuitnn}
\end{figure*}

% 1. Describe the importance of logic synthesis
% 2. Existing Problems
% (a) Neural Architecture Search: Unstable, Predefined Setting, etc.
% (b) Circuit Generation: Probabilistic Model, Logic Equivalence

With the rapid advancement of technology, the scale of integrated circuits (ICs) has expanded exponentially. 
This expansion has introduced significant challenges in chip manufacturing, particularly concerning power and area metrics.
A primary objective in IC design is achieving the same circuit function with fewer transistors, thereby reducing power usage and area occupancy.

Logic synthesis~\cite{hachtel2005logicsynth}, a critical step in electronic design automation (EDA), transforms behavioral-level circuit designs into optimized gate-level circuits, ultimately yielding the final IC layout. 
The primary goal of logic synthesis is to identify the physical implementation with the fewest gates for a given circuit function. 
This task constitutes a challenging NP-hard combinatorial optimization problem. 
Current logic synthesis tools~\cite{brayton2010abc, wolf2013yosys} rely on human-designed heuristics, often leading to sub-optimal outcomes.

Differentiable architecture search (DAS) techniques~\cite{liu2018darts, chu2020darts} offer novel perspectives on addressing challenges in this problem.
Circuit functions can be represented through truth tables, which map binary inputs to their corresponding outputs. 
Truth tables provide a precise representation of input-output relationships, ensuring the design of functionally equivalent circuits.
Inspired by this, researchers~\cite{deepmind2024ai4sys, wang2024tnet} have begun exploring the application of DAS to synthesize circuits directly from truth tables.
Specifically, \citet{deepmind2024ai4sys} proposed CircuitNN, a framework that learns differentiable connection structures with logic gates, enabling the automatic generation of logic circuits from truth tables.
This approach significantly reduces the complexity of traditional circuit generation. 
Building on this, \citet{wang2024tnet} introduced T-Net, a triangle-shaped variant of CircuitNN, incorporating regularization techniques to enhance the efficiency of DAS.

Despite these advancements, several challenges remain. 
The computational complexity of DAS grows quadratically with the number of gates, posing scalability issues.
Although triangle-shaped architecture~\cite{wang2024tnet} partially mitigates this problem, redundancy persists. 
%Additionally, DAS is susceptible to converging to local optima, limiting the ability to search architectures that satisfy the given truth tables~\cite{liu2018darts}. 
%Furthermore, hyperparameters (network depth and layer width) require extensive searches, introducing complexity and prolonging the synthesis process. 
Additionally, DAS is susceptible to converging to local optima~\cite{liu2018darts} and hyperparameters (network depth and layer width) require extensive searches. 
The challenges arise from the vast search space in DAS. 
% Even with predefined settings for CircuitNN, finding a configuration that meets the truth table requires extensive trial and error during the DAS process. 
Intuitively, limiting the search space through predefined parameters (network depth, gates per layer, and connection probabilities) can significantly reduce the complexity.

Recent advances~\cite{openai2023gpt4, abramson2024alphafold3, esser2024sd3, li2024mar} in conditional generative models have demonstrated remarkable performance across language, vision, and graph generation tasks. 
Motivated by these developments, we propose a novel approach to circuit generation that generates preliminary circuit structures to guide DAS in generating refined circuits matching specified truth tables. 
Firstly, we introduce CircuitVQ, a tokenizer with a discrete codebook for circuit tokenization. 
Built upon our Circuit AutoEncoder framework~\cite{hou2022graphmae,li2023maskgae,wu2025mgvga}, CircuitVQ is trained through a circuit reconstruction task. 
Specifically, the CircuitVQ encoder encodes input circuits into discrete tokens using a learnable codebook, while the decoder reconstructs the circuit adjacency matrix based on these tokens.
Subsequently, the CircuitVQ encoder serves as a circuit tokenizer for CircuitAR pretraining, which employs a masked autoregressive modeling paradigm~\cite{chang2022maskgit, li2023mage}. 
In this process, the discrete codes function as supervision signals. 
After training, CircuitAR can generate discrete tokens progressively, which can be decoded into initial circuit structures by the decoder of the CircuitVQ. 
These prior insights can guide DAS in producing refined circuits that match the target truth tables precisely.

Our key contributions can be summarized as follows:
\begin{itemize}
\item We introduce CircuitVQ, a circuit tokenizer that facilitates graph autoregressive modeling for circuit generation, based on our Circuit AutoEncoder framework;
\item Develop CircuitAR, a model trained using masked autoregressive modeling, which generates initial circuit structures conditioned on given truth tables;
\item Propose a refinement framework that integrates differentiable architecture search to produce functionally equivalent circuits guided by target truth tables;
\item Comprehensive experiments demonstrating the scalability and capability emergence of our CircuitAR and the superior performance of the proposed circuit generation approach.
\end{itemize}

% Motivation
% (a) Diffusion (Vision, Graph), Autoregressive (Language, Vision)
% (b) Circuit Generation for Predefined Setting
% (c) Neural Architecture Search for Strict Logic Equivalence

% Contribution
% (a) Circuit Tokenizer (new transformer arch, training strategy)
% (b) CircuitAR (train and gen strategies, post-ar strategy)
% (c) Extensive Evaluation including BitD (Bit Distance) for Scalability

\section{Basic Background: Supervised Learning and the PAC Model}
\label{sec:background}

At this point almost everyone has heard of machine learning (ML). Anyone likely to stumble upon this article will have also heard of its most influential special case, supervised learning, and those theoretically inclined will also be familiar with the PAC model. Nonetheless, I will set the stage by  recapping the basics.

\subsection{Basics of Supervised Learning}%Let's set the stage in any case

\emph{Supervised Learning} is the task of ``coming up'' with a function $f: \X \to \Y$ to ``explain'' or ``fit'' a sequence of input/output examples   $(x_1,y_1), \ldots, (x_n,y_n)$, with $x_i \in \X$ and $y_i \in \Y$.  Here $\X$ is a \emph{data domain} consisting of \emph{datapoints} $x \in \X$, $\Y$ is a \emph{label set} consisting of \emph{labels} $y \in \Y$, and the sequence $(x_1,y_1),\ldots,(x_n,y_n)$ is the \emph{training data} consisting of \emph{labeled examples (a.k.a. samples)}~$(x_i,y_i)$.  I~will refer to the chosen function $f$ as a \emph{predictor}, and to $n$ as the \emph{sample size}. A \emph{learning algorithm} takes as input training data, and outputs (some representation of) a predictor $f \in \Y^\X$.\footnote{Note that this describes the usual \emph{batch}, a.k.a.~\emph{offline}, setting of supervised learning. I do not discuss other paradigms such as online or active learning in this article.} 



Success in supervised learning is defined as \emph{generalization} to  future examples: For a typical \emph{test example}  $(x_{\tst},y_{\tst})$, the predicted label $y'_{\tst}=f(x_{\tst})$ should ``equal'' $y_{\tst}$, perhaps approximately. We usually assume the test example is drawn from the same  ``source'' as the training data  --- commonly, i.i.d.~from the same distribution. The quality of the prediction is quantified by $\ell(y'_{\tst},y_{\tst})$, where $\ell:~\Y~\times~\Y \to \RR_{\geq 0}$ is a \emph{loss function} chosen as part of the problem definition. Common loss functions include the 0-1 loss $\ell_{0-1}(y',y) = [y' \neq y]$ for \emph{classification} problems,\footnote{The notation $[P]$ denotes $1$ when predicate $P$ is true, and denotes $0$ when $P$ is false.} as well as the absolute loss $|y'-y|$ or squared loss $(y'-y)^2$ for \emph{regression problems} featuring $\Y  \sse \RR$.

Nontrivial generalization properties are typically only possible if one assumes something about the data.\footnote{The need for such an assumption is formalized by the  \emph{no free lunch theorems} of supervised learning \cite{wolpert_connection_1992,wolpert_lack_1996,schaffer_conservation_1994}.} The Bayesian approach to  machine learning, common in many applications, assumes some parametric form for the distribution generating the data, and postulates a prior on the parameters. This is not the approach I will take in this article. Instead, I will focus on the frequentist --- and some would say ``worst-case'' or ``adversarial'' ---  approach that is common in the computational learning theory community, embodied by the PAC model. Here we assume that the (training and test) data can be explained, perhaps approximately, by a function in some ``simple enough to learn'' class of functions $\H \sse \Y^\X$, often called the \emph{hypotheses}. Equivalently, we  seek a predictor which explains the unseen data roughly  as well as the best hypothesis $h^* \in \H$, whether or not we assume that $h^*$ itself provides a perfect explanation.



 \paragraph{Common Algorithmic Templates.} Perhaps the best known general-purpose supervised learning algorithm is \emph{empirical risk minimization (ERM)}, which chooses as its predictor a hypothesis $f \in \H$ minimizing $\frac{1}{n} \sum_{i=1}^n \ell(f(x_i),y_i)$ --- a quantity called the \emph{training error}, \emph{empirical error}, or \emph{empirical risk} of $f$. %\footnote{When multiple hypotheses minimize the empirical risk, we assume ERM breaks ties arbitrarily.}
A common template for generalizing ERM involves adding a \emph{regularization term} $\psi(f)$ to the  objective function, typically chosen to measure some notion of ``hypothesis complexity.'' An algorithm instantiating this template is known as a \emph{structural risk minimizer (SRM)}, and chooses as its predictor the hypothesis $f \in \H$ minimizing the \emph{structural risk} $\frac{1}{n} \sum_{i=1}^n \ell(f(x_i),y_i) + \psi(f)$. Other well-known algorithms, such as gradient descent and its variations,  can frequently be interpreted as approximate implementations of ERM or SRM.


\paragraph{Proper vs Improper Learning.} A learning algorithm is said to be \emph{proper} if its predictor $f$ is always chosen from the hypothesis class, i.e., $f \in \H$, otherwise it is said to be \emph{improper}. ERM  is an example of a proper learning algorithm, as are SRM algorithms of the form described above.  In the \emph{proper regime} of learning, algorithms are required to be proper. This article will be concerned with the more flexible \emph{improper regime} (a.k.a \emph{representation-independent learning}), where no such constraint is placed on the learner. In other words, all we care about is predictive power at test time, rather than any insights derived from the functional form or representation of the predictor~itself.


\subsection{The PAC Model}
A standard mathematical setup for evaluation of supervised learning algorithms, at least in the theoretical computer science community, is Valiant's \emph{Probably Approximately Correct (PAC) model} of learning (see e.g.~\cite{kearns_introduction_1994,mohri_foundations_2018}). Here, we assume there is an unknown distribution $\D$ on $\X \times \Y$ from which training and test data are  drawn.  Specifically, the labeled datapoints of the training set  $(x_1,y_1), \ldots, (x_n,y_n)$, as well as the test data  $(x_\tst,y_\tst)$, are i.i.d.~from $\D$. Often it is assumed that $\D$ lies in some class of distributions of interest. The \emph{true expected loss}, or simply \emph{loss}, of a predictor $f: \X \to \Y$ is the expected loss it incurs on draws from $\D$, written $L_\D(f) = \Ex_{(x,y) \sim \D} \ell(f(x),y)$.


There are two main ``settings'' in PAC learning. The  \emph{realizable setting} only requires that the data be perfectly explained by some hypothesis in $\H$. More generally, the \emph{agnostic setting} makes no assumption relating the data to the hypotheses, but shifts the goalposts as necessary to allow nontrivial guarantees: the expected loss at test time is evaluated only ``relative'' to that of the best hypothesis $h^* \in \H$. There are other settings which make more nuanced assumptions, such as $\D$ being of a particular parametric form or its support living in some (unknown) lower-dimensional space, etc. I will mostly discuss the realizable and agnostic settings in this article, those being the simplest and most studied from a theoretical perspective. %TODO:We will briefly discuss other settings in Section ??

The PAC model demands high probability guarantees of learners, in the worst case over distributions of interest. Consider first the realizable setting, where $\D$ is such that $\min_{h \in \H} L_{\D}(h) = 0$. A PAC learner has \emph{error} $\epsilon=\epsilon(n)$ and \emph{confidence} $\delta=\delta(n)$ if, when training data consists of $n$ i.i.d~samples from a realizable distribution $\D$, it produces a predictor $f$  satisfying $L_\D(f) \leq \epsilon$ with probability at least $1-\delta$. In the agnostic setting, where $\D$ can be arbitrary, we require $L_\D(f) - \min_{h \in \H} L_\D(h) \leq \epsilon$ with probability $1-\delta$.

In both the realizable and agnostic settings, we look for PAC learners with small $\epsilon$ and $\delta$ as a function of the sample size $n$. An equivalent perspective looks at the sample complexity $m(\epsilon,\delta)$, which is the minimum sample size which guarantees error  at most $\epsilon$ with probability at least $1-\delta$. We say a problem is \emph{PAC learnable} if its PAC sample complexity is finite whenever $\epsilon,\delta > 0$.

For most PAC learning problems, learnability and sample complexity are characterized in terms of a  ``dimension'' of the hypothesis class. Most prominently this is the \emph{VC dimension} for binary classification, the \emph{fat shattering dimension} for agnostic regression, and the \emph{DS dimension} for multiclass classification (see \cite{anthony_neural_1999,daniely_optimal_2014,brukhim_characterization_2022}). Treatment of these is beyond the scope of this article. The unfamiliar reader need not worry, however,  as dimensions will feature only tangentially in our~discussion.




%\paragraph{Learning settings: Realizable, Agnostic, etc.} In learning theory, evaluating a supervised learning algorithm requires specifying a data model and an objective. We will leave the details of the data model flexible for now, to allow for both the PAC model and the adversarial transductive model. Nonetheless we will describe two variations, which we call ``settings'', which cut across different models. The  \emph{realizable setting}  requires only that the data be perfectly explained by some hypothesis $h \in \H$ --- i.e., there exists a hypothesis which is guaranteed to suffer a loss of $0$ on training and test data. The performance of the learning algorithm is its expected loss at test time for some ``worst case'' realizable instance. More generally, the \emph{agnostic setting} makes no assumption relating the data to the hypotheses, but shifts the goalposts as necessary to allow nontrivial guarantees: the expected loss at test time is evaluated only ``relative'' to that of the best hypothesis $h^* \in \H$, again for some ``worst case'' instance. There are other settings which make more nuanced assumptions about the data, such as it is drawn from a distribution of a particular parametric form, or that it lives in some (unknown) lower-dimensional space, etc. We will mostly discuss the realizable and agnostic settings, those being the simplest and most studied from a theoretical perspective.




%%% Local Variables:
%%% mode: latex
%%% TeX-master: "learning_matching"
%%% End:

\section{Anatomy of an ML Fleet}\label{sec:fleet}

\begin{figure}[t]
    \centering
    \includegraphics[width=\columnwidth]{final_figs/ml_stack.png}
    \caption{The ML fleet system stack of a production system at Google. The multi-layered architecture of a fleet is complex; each layer is a critical component in the ML system, with interactions between layers affecting overall performance and efficiency. Segmenting the fleet based on these layers provides actionable metrics which can be used to improve performance. }
    \label{fig:ml_stack}
\end{figure}
In this section, we dive into the anatomy of a production ML fleet to provide perspective on the complexity of managing it. We begin by dissecting the fleet based on its distinctive components and characteristics, starting with the hardware foundation and progressing to the user application level.  \autoref{fig:ml_stack} shows the various layers that comprise the system stack and mediate user access to the ML fleet. This stack illustrates the intricate ecosystem that underpins modern ML operations. 

We supplement our discussion in this section with data from a snapshot of Google's TPU fleet for internal workloads, providing concrete examples of the challenges these systems face in practice. By examining actual usage patterns, resource allocation, and performance metrics from a production ML fleet, we can ground our discussion in real-world scenarios and offer insights based on empirical evidence. This data-driven approach will allow us to illustrate the complexities of managing large-scale ML operations and demonstrate how theoretical concepts translate into practical challenges and opportunities for optimization. %
\subsection{Accelerators}


ML fleets are distinguished from other types of large-scale compute systems by their accelerator-centric architecture. The ML computing landscape is dominated by domain-specific hardware, such as GPUs and other ASICs. In order to tailor to the vector and matrix intensive operations that underpin ML workloads, new accelerators such as Google's Tensor Processing Units (TPUs) \cite{jouppi2018motivation} have been developed. In general, there has been a Cambrian explosion of ML hardware accelerators \cite{hennessy2019anewgoldenage}, with new accelerators being deployed at an unprecedented rate compared to traditional WSC fleets \cite{jouppi2021ten}. \autoref{fig:five-years} vividly illustrates this dynamism, revealing dramatic shifts in our ML fleet's hardware makeup for internal workloads over just a few years.

ML fleets typically incorporate a diverse array of hardware including CPUs, GPUs, TPUs, and other accelerators, each fulfilling specific roles. For example, CPUs may be responsible for scheduling, GPUs for training tasks, and edge accelerators \cite{yazdanbakhsh2021edge} for deployment and serving. The challenge lies in effectively orchestrating these heterogeneous accelerators to maximize their individual strengths---a complexity rarely encountered in general compute fleets.

Moreover, the heterogeneity extends beyond just accelerator type. Even within a single class of hardware accelerators, there are many different versions of the hardware, adding another layer of complexity to fleet management. Each hardware generation introduces unique features that require significant optimizations to extract peak ML workload efficiency. One notable example is the integration of the SparseCore (SC) in TPUv4 \cite{jouppi2023tpuv4opticallyreconfigurable}, which was designed to significantly boost performance for embedding-heavy models. Subsequent large-embedding model teams would likely then consider the hardware specifications of the SparseCore when designing their embedding configurations. Design points such as embedding dimension, vocabulary size, valence, and others might also be co-designed to optimize performance on the hardware platform. This demonstrates how hardware-software co-design is becoming increasingly important in improving the efficiency of these diverse accelerators, forming a symbiotic relationship where the computational needs of future workloads affect the next generation of hardware, and the hardware capabilities inform the types of workloads that the ML fleet is best equipped to handle \cite{shi2020learned}.







\subsection{Scheduler}\label{sec:scheduler}
\begin{figure}[t!]
    \centering
    \includegraphics[width=3in]{final_figs/job_size_bars.png}
    \caption{A sample breakdown of Google's ML fleet for internal workloads, segmenting on workload topology size (the number of accelerators requested by a given job). Progressive snapshots over the course of one year illustrate the ML fleet's growing share of jobs using an "extra-large" number of accelerators. This demonstrates how an ML fleet scheduler must be able to adapt to changing conditions, as the evolution of job sizes and topologies in response to shifting ML workloads presents unique challenges for the entire fleet.}
    \label{fig:job_size}
\end{figure}
The scheduler directly manages the hardware in a fleet by coordinating the allocation of resources; for the case study presented in this paper, it coordinates TPU allocations for Google's internal-facing ML workloads. There are two interconnected challenges that a scheduler must address when allocating hardware for an ML fleet: (1)~optimizing performance across various hardware types, and (2)~balancing utilization with stability and fault tolerance.

\autoref{fig:job_size} illustrates these challenges. It shows the allocation of workloads in Google's internal-facing ML fleet with different chip requirements over time, categorized into sizes based on the total number of TPU chips in the required topology. In this categorization, workloads with size "small" refer to jobs that request a single TPU or a handful of TPUs, while workloads with size "extra-large" refer to jobs that request the largest number of TPUs (often requiring multiple pods, as described in \citet{kumar2021exploring}). \autoref{fig:job_size} demonstrates that over the course of just one year, the allocation distribution can shift dramatically, reflecting the changing nature of ML workloads in the fleet. As large-scale ML models become more prevalent in an ML fleet, an increasing number of workloads will require correspondingly larger meshes of connected accelerators.







Optimizing the scheduling of jobs while meeting these resource requirements is difficult because it presents an NP-hard bin-packing problem. Each workload may specify a different accelerator type, chip topology, and location requirement and needs to be scheduled according to fleet constraints in a way that reduces overall fragmentation of the fleet. Since workloads are constantly being started and completed, the machine availability of the fleet is constantly changing, requiring a robust defragmentation algorithm. In addition, latency requirements may require accelerators for a workload to be grouped together near certain locations or data cells, adding another constraint to the scheduling optimization problem. 

The utilization of fleet resources must also be balanced with stability and fault tolerance. For example, to reduce disruptions, some machines may intentionally remain underutilized so that higher priority jobs may be more easily scheduled when needed. While high utilization is desirable for cost-efficiency, pushing hardware to its limits can lead to thermal issues, increased failure rates, and unpredictable performance. In large-scale ML fleets, hardware failures are inevitable, and the scheduler must be robust enough to handle these failures gracefully, redistributing workloads and ensuring job continuity without significant performance degradation.






\begin{figure}[t!]
    \centering
    \includegraphics[width=3in]{final_figs/life_of_mlapp.png}
    \caption{An ML workload requires all requested TPUs to be allocated before the task can start. In this example of a training workload, forward progress is saved via checkpoints. Delays during workload initialization and checkpoint writing, which are part of the Runtime and Framework layers, can reduce overall system efficiency.}
    \label{fig:life_of_mlapp}
\end{figure}

\subsection{Runtime/Compiler}

The runtime and compiler layers form an important component in the ML fleet system stack. They are responsible for bridging the gap between high-level ML models and the underlying hardware accelerators. The runtime layer focuses on the execution environment of ML programs. It handles important tasks such as program setup, data feeding, result management, and checkpoint creation, as illustrated in \autoref{fig:life_of_mlapp}. Depending on the system design, it either triggers just-in-time compilation of user-written code into accelerator-specific instructions or invokes pre-compiled operation kernels from vendor-specific libraries. The runtime layer can also manage the distribution strategy of code execution, as with notable runtimes like Pathways \cite{barham2022pathways}.

\autoref{fig:pathways} shows the growth of Pathways-based workloads in our production fleet. It highlights the demand for runtimes that support efficient distributed execution for ML workloads. It also emphasizes the rapidly shifting distribution of workload runtimes in a fleet.




The compiler layer, working with the runtime, transforms high-level ML model code into executable code optimized for specific accelerators. It operates on graph intermediate representations, applying both platform-independent and platform-dependent optimizations. The output is a program tailored to the target accelerator, such as a specific version of a TPU. Domain-specific compilers, like XLA (Accelerated Linear Algebra) \cite{xla}, have significantly improved the performance of ML workloads. For instance, in MLPerf BERT benchmarks \cite{mattson2020mlperf}, XLA demonstrated a remarkable 7$\times$ performance boost and 5$\times$ batch size improvement \cite{kumar2021exploring} over previous records, emphasizing the potential of specialized compilation techniques. We note that there are many types of accelerators, some of which do not require an explicit compiler for code generation. 

Compiler optimization in ML fleets faces unique challenges due to the rapid evolution of hardware accelerators, requiring frequent updating of optimization strategies to leverage the specific features of each new hardware generation. Moreover, the impact of optimizations can be difficult to generalize, as an optimization that improves one workload may degrade another due to differences in computation or communication patterns. This emphasizes the need for a balanced approach to optimization, considering both platform-independent techniques for flexibility and platform-specific optimizations for maximum performance.






\begin{figure}[t!]
    \centering
    \includegraphics[width=\columnwidth]{final_figs/pathways.png}
    \caption{The prevalence of fleet-wide workloads using the Pathways runtime over a sample of one year, illustrating the rapid shift of fleet-wide runtimes to accommodate changing workloads. Pathways adoption has increased rapidly, as it provides better support for distributed execution and data processing.}
    \label{fig:pathways}
\end{figure}
\subsection{Framework}

The framework layer sits on top of the runtime/compiler. It is the interface between ML practitioners and the underlying complex hardware and software infrastructure. This layer encompasses various ML frameworks and libraries, such as TensorFlow \cite{abadi2016tensorflow}, JAX \cite{frostig2018compiling}, and PyTorch \cite{paszke2019pytorch}, each offering unique features and optimizations. 

The framework layer provides high-level abstractions and APIs that allow developers to build and deploy ML models efficiently. These frameworks are responsible for translating user-written code into representations that can be understood and optimized by lower-level layers such as compilers and runtimes. This translation process bridges the gap between user intent and system execution.



One of the key responsibilities of ML frameworks is defining the structure of distributed ML applications. For example, TensorFlow's Distribution Strategy \cite{abadi2016tensorflowlargescalemachinelearning} provides a framework for distributing training across multiple devices or machines. These can have single-client or multi-client architectures, depending on workload needs, as shown in \autoref{fig:multi_single_frameworks}. These frameworks must also map ML primitives to hardware-specific designs to achieve optimal performance. This is important for specialized hardware like TPUs, which are designed for bulk-synchronous training. Frameworks like JAX are more targeted towards ML workloads, with features that facilitate ease of interpretability when analyzing ML performance, such as high-level tracing for just-in-time compilation. In the ML fleet, JAX usage has increased over time, most likely due to these features and the emergence of more ML-heavy workloads \cite{frostig2018compiling}.


In addition, ML frameworks often provide auxiliary services to improve efficiency of the entire ML fleet. For instance, TensorFlow's \texttt{tf.data} \cite{murray2021tfdata} service optimizes the performance of the data pipeline. These features, while abstracted from the user, can impact the overall system efficiency, as shown in \autoref{fig:life_of_mlapp}. Underneath these high-level frameworks lies a foundation of general-purpose libraries and datacenter services. Frameworks like TensorFlow utilize libraries such as gRPC \cite{grpc}, protobuf \cite{protobuf}, and tcmalloc \cite{tcmalloc} for various low-level operations, and interface with datacenter services for storage (e.g., Colossus \cite{ghemawat2003gfs} \cite{colossus}) and monitoring (e.g., Monarch \cite{adams2020monarch}). 


\begin{figure}[t!]
    \centering
    \includegraphics[width=.75\linewidth]{final_figs/multi_single_frameworks.png}
    \caption{
Comparing single-client frameworks with multi-client frameworks. 
    }
    \label{fig:multi_single_frameworks}
\end{figure}







As the primary point of interaction for users, the framework layer serves as a key bridge in the ML system stack, as it not only abstracts underlying complexities but also plays an important role in determining the overall efficiency and capabilities of the fleet. Frameworks must balance the need for user-friendly APIs with the need to leverage underlying hardware-specific optimizations, while also managing the complexities of distributed computing, data pipeline optimization, and integration with lower-level services. 









\subsection{ML Model \& Data}

To characterize the ML Fleet at the highest level of the stack, we generally want to know: What types of workloads are we spending most of our compute cycles on? This is a critical question because it drives nearly every design decision we make for the ML Fleet at every level of the stack, from the hardware (how many training vs. inference vs. other chips) to the software (JAX vs. other frameworks, runtime distribution strategies, and compiler optimizations). Workload heterogeneity analysis is useful for understanding what kind of models are prevalent in the production fleet, especially since different workloads stress the hardware in different ways. This understanding can drive decisions about which accelerators to deploy and how many, or which compiler optimizations to carry out.







In practice, we observe that the model and data layer of the ML fleet stack are the most affected by fluctuating user demands. User requirements such as the model architecture, size of the training dataset, or even use of different numerical formats in the training model can impact the efficiency of the job, which can have a cascading effect on the efficiency of the overall ML fleet.

While ML workloads share some computational patterns, particularly in their use of matrix operations and data-intensive processing, the specific architectures and resource requirements can vary significantly.  As new model architectures and learning tasks emerge, they prompt rapid shifts in workload composition, leading to fluctuations in resource demands across various model types. 

In a production ML fleet, there are varying proportions of workloads dedicated to each phase of the ML model life cycle; training, bulk inference, and real-time serving. Thus, the fleet must be flexible enough to handle the requirements of each of these phases; for example, training workloads may be compute intensive while real-time serving workloads may focus on minimizing latency.

















\section{ML Productivity Goodput}\label{sec:goodput}

The optimization of ML fleet efficiency is a complex, cyclical challenge, as illustrated in \autoref{fig:ml_stack}. The first challenge is measuring, understanding, and reporting fleetwide efficiency, establishing a baseline for current performance. Second, we must identify and quantify fleet-wide inefficiencies, pinpointing areas that require improvement. Third, we must eliminate these inefficiencies by implementing changes across the fleet, which in turn leads back to the first stage as we measure the impact of these changes. This cycle ensures ongoing optimization and adaptation to the ever-evolving landscape of ML workloads and hardware. To keep pace with this lifecycle, we require a metric that not only quantifies current performance but also guides future optimization efforts across the fleet, which is why Google has developed the \mpg metric. %

In this section, we present an in-depth discussion of MPG, a new metric for quantifying ML fleet efficiency. We refer to this as the iron law of performance for ML fleets, drawing a corollary to the iron law of processor performance~\cite{emer1984ironlaw}. MPG, defined in \autoref{fig:mpg}, is a means for measuring efficiency gains and guiding exploration of optimization strategies across various fleet components.


\begin{figure}[t!]
    \centering
    \includegraphics[width=\columnwidth]{final_figs/mpg.png}
    \caption{ML Productivity Goodput (MPG) and its components.}
    \label{fig:mpg}
\end{figure}





\subsection{Pitfalls \& Myths of Traditional Metrics}


Before we set the stage for the new metric, we examine the common pitfalls of historical approaches for fleetwide measurement. \autoref{fig:utilization} illustrates how computer architects have historically tended to think about performance metrics \cite{li2023analyzing, mars2011bubble,kanev2015profiling}. These traditional performance metrics can sometimes fall short in providing a holistic view, given the unique challenges we have discussed in \autoref{sec:background} and \autoref{sec:fleet}. 





\textbf{Myth 1: High \underline{Capacity} equates to high resource availability.}
While capacity can tell us how many individual accelerators may be available in the fleet at a given time, it does not take into account the topological shape of those accelerators. For example, an ML training workload requesting thousands of chips in a certain physical mesh shape may never be scheduled if the only available accelerators are fragmented across different clusters or data centers. Other factors, such as the geographical location of data storage cells and accelerators, are not included in the capacity metric, even though they significantly affect job scheduling. Therefore, high capacity by itself as a metric does not necessarily effectively translate to high availability for workloads, and we should instead opt for scheduling efficiency as a more robust metric.

\textbf{Myth 2: High \underline{Occupancy} guarantees productivity.}
Occupancy is defined as the fraction of accelerators allocated to jobs and is often measured by the scheduler (e.g. Borg \cite{verma2015borg}). Occupancy is traditionally seen as a key efficiency indicator, but it can be misleading as it masks inefficiencies in the system stack. For example, an accelerator might be successfully allocated but stuck in I/O wait or running poorly optimized code, thus resulting in a high occupancy but very little actual progress being made towards the workload task. This is important for long-running tasks such as ML model training, where frequent pre-emptions may hinder checkpoint progress but still result in a nominally high occupancy. The traditional occupancy metric therefore does not distinguish between productive and unproductive use of allocated resources.

\textbf{Myth 3: \underline{Duty Cycle} accurately represents useful work.}
Duty Cycle measures whether an accelerator is in use, not how much of its compute capacity is used. When looking at an ML workload running on a TPU, duty cycle does not provide any signal on how much the matrix-multiply units (MXUs) are utilized \cite{jouppi2023tpuv4opticallyreconfigurable}. It is agnostic of the program-level efficiency and does not take into account the effectiveness of the operations being performed. An accelerator could have a high duty cycle while executing unnecessary or redundant computations. So, we require a more sophisticated metric.

\begin{figure}[t]
\centering
\includegraphics[width=\columnwidth]{final_figs/efficiency_util.png}
\caption{Historical utilization-based fleet efficiency metrics. We propose replacing this approach and using goodput as a measure of fleet efficiency rather than utilization.}
\label{fig:utilization}
\end{figure}

\textbf{The Overarching Misconception: Utilization Equals Productivity.}
The common thread among these metrics is the assumption that keeping accelerators busy equates to productive work. However, this overlooks critical factors. (1) \textit{Quality of Computations:} None of these metrics assess whether the operations being performed are actually contributing to the desired output. (2) \textit{Workload Efficiency:} They do not consider whether the workloads are optimally designed for the hardware. (3) \textit{System-level Bottlenecks:} Focusing solely on accelerator usage ignores potential bottlenecks in data loading, memory access, or inter-accelerator communication. (4) \textit{Forward Progress:} These metrics provide no insight into how much useful work is being accomplished towards completing an actual ML task.







\subsection{Metric Features}
Ideally, the MPG metric must be a clearly defined and accurate measure of forward progress; improvements in the metric must also reflect real improvements in the efficiency of the fleet. This metric must be capable of overcoming two significant challenges.

\begin{enumerate}
\item \textbf{It must capture the dynamic nature of ML fleets}: The fleet is constantly fluctuating due to variables such as changes in workload composition, updates to the code stack, and evolving hardware. To effectively improve efficiency, we must ensure that any change in the metric is explainable despite these fluctuating variables.
\item \textbf{It must explain the trade-offs between individual and aggregate efficiency}. At a fleetwide scale, jobs must be scheduled in concert with one another to ensure maximum aggregate efficiency of the fleet. However, individual jobs may have certain service-level requirements, meaning that this metric must be decomposable based on workload characteristics.
\end{enumerate}

\begin{figure}[t]
    \centering
    \includegraphics[width=\columnwidth]{final_figs/mpg_breakdown.png}
    \caption{Breakdown of a ML workload using \mpg. 
    }
    \label{fig:mpg_breakdown}
\end{figure}

\subsection{A New Approach: ML Productivity Goodput}

\mpg (MPG) is designed to address the myriad challenges discussed in Section~\ref{sec:fleet}, as well as to overcome the limitations of existing approaches. Just as the Iron Law of Processor Performance~\cite{emer1984ironlaw} breaks down CPU performance into $\frac{instructions}{program}$$\times$ $\frac{cycles}{instruction}$$\times$$\frac{time}{cycle}$, the MPG metric decomposes ML fleet efficiency into scheduling, runtime, and program components (see \autoref{fig:mpg}).



This multi-layered structure, as illustrated in \autoref{fig:mpg_breakdown}, offers several advantages over the traditional metrics. First, it allows for precise identification of performance bottlenecks or improvements at specific layers of the stack, facilitates a more granular analysis of efficiency trends over time, and mitigates the risk of misleading interpretations that can arise from aggregated data, such as Simpson's paradox.\footnote{A statistical phenomenon where a trend that is evident within individual groups disappears or reverses when the population groups are combined.} Second, by decoupling these submetrics, we enable more targeted optimization efforts and gain deeper insights into the complex interactions within the ML fleet. Finally, this approach not only enhances our ability to measure current performance but also provides a framework for guiding improvements, discussed in \autoref{sec:improvements}.  

\textbf{Scheduling Goodput:}
\emph{How often does an application have all necessary resources to make progress?} 

Scheduling Goodput (SG) quantifies the efficiency of resource allocation in an ML fleet. It measures the fraction of time that an application has all the required resources simultaneously available to make progress. This metric can be lower than traditional Occupancy, particularly in distributed, bulk-synchronous applications where all required chips must be available concurrently. The numerator of SG is calculated as the simultaneous uptime of all tasks in a distributed ML application that must be connected to make synchronous progress, as shown in \autoref{fig:scheduling}. This is referred to as ``allocated chip-time'' or ``all-allocated'' time. The denominator is fleet capacity, expressed as chip-time. This provides a full view of how effectively the scheduling layer is using the fleet's resources. 

Scheduling Goodput offers insights into potential inefficiencies in resource allocation, such as fragmentation of available resources, delays in coordinating multiple chips for distributed applications, and mismatches between application requirements and available resources. By optimizing SG, we can improve the overall efficiency of resource utilization in the ML fleet, ensuring that applications have the necessary resources to make consistent progress.

\begin{figure}[t!]
    \centering
    \includegraphics[width=\columnwidth]{final_figs/scheduling.png}
    \caption{The scheduling goodput for training workloads measures the percentage of time when all of the TPU workers are available to work at the same time. In other words, it measures the portion of time that all of the necessary resources are available to make progress. 
}
    \label{fig:scheduling}
\end{figure}

\textbf{Runtime Goodput:}
\emph{Of the time that an application has all necessary resources, how often is it making progress?} 

\rg (RG) measures the efficiency of the orchestration layers in managing the execution of ML applications once resources are allocated. This metric focuses on the actual productive time of an application, accounting for various overheads in the runtime environment. The orchestration layer is responsible for critical tasks such as initializing chips, connecting them into slices for bulk-synchronous progress, loading and compiling programs, feeding data to these programs, and ensuring that training progress is regularly saved through checkpoints. The numerator of RG is the productive chip-time of the application's progress that has been saved in checkpoints; work done between the last checkpoint and failure (or preemption) doesn't count as "productive" time and is therefore not included in RG. The denominator of RG is the allocated chip-time defined as the numerator of SG. 

\rg can help with identifying bottlenecks in the runtime environment, such as slow data loading, inefficient checkpointing, or suboptimal program compilation. It can guide the efforts to streamline the execution pipeline and improve the overall throughput of ML workloads.

\textbf{Program Goodput:}
\emph{Of the time that an application is making progress, how close is it to the ideal roofline?}

Program Goodput (PG) assesses the efficiency of the application code itself, measuring how effectively it utilizes the available computational resources. While a traditional roofline performance model~\cite{williams2009roofline} might seem suitable for this purpose, it falls short in capturing the true efficiency of modern ML workloads. The traditional roofline model is highly sensitive to compiler decisions, such as how ML operators are fused or rematerialized \cite{briggs1992rematerialization}, or which operands are placed in which memory space. It rewards individual ops that are close to peak utilization, but penalizes correct optimizations that result in computation graphs where the utilization may be lower, but overall execution time is shorter. 

To overcome these limitations, we use a compute-based roofline model that compares the ideal execution time of the workload against its actual execution time. The ideal predicted execution time, which is the numerator of PG, can be computed from intrinsic properties of the machine learning model being run. By analyzing the shape of the unoptimized  high-level operations (HLO) graph, we can estimate how many floating point operations (FLOPs) the program would require at its theoretical peak performance. Since we are analyzing the computation graph before any compiler optimizations, this prediction is agnostic to compiler decisions. 

The denominator of PG is the actual execution time. The PG metric can thus be interpreted as a percentage reflecting how well optimized the ML program is, with a score of 100\% indicating perfect performance matching the theoretical peak.






%
%
%


\begin{comment}
\subsection{Extension to Many Candidate Parents}
The basic provenance problem described thus far assumes we have one candidate parent $f$ that we want to test our child model $g$ against. 
We now extend it to the general case when there is a set of parents:
\begin{definition}[Model Provenance Problem with  Unspecified Parent]
Given only query access to models, determine whether a model $g$ is derived from some model from the set ${f_1,\ldots,f_s}$ of candidate parent models.
\end{definition}

\prateek{Explain how this Algo works as follows. It finds the most similar model to the given model $g$ among all the control models $C$ and candidate parents $F$. If that model is a control model, the Algorithm terminates with False. Otherwise, the Algorithm goes on to test whether the FWER
of this model is overall below $\alpha$, the desired significance level. The test for the latter is the same as in Algorithm 1, except now all the alternate hypotheses (including control and candidates) are in the family tested against. When the Algorithm return True, it has the guarantee that the most similar model is one of the candidate models and that the total significance level across all hypotheses meets the threshold $\alpha$.}

While running the basic tester $s$ times (once for each provenance pair $(f_i,g)$) would solve the unspecified parent problem, this approach besides requiring more effort, also would require additional correction for multiple testing to maintain the same level of confidence. The probability of false positives would grow with the number of candidate parents $s$ unless appropriate adjustments (such as Bonferroni correction) are made to the significance level. 
%
We thus consider improved tester given in Algorithm~\ref{alg:unknown_parent_tester}. Our  tester avoids this issue by conducting a single set of hypothesis tests after identifying the most similar candidate.

\begin{algorithm}[t]
    \caption{Provenance Tester for $g$ Given a Candidate Parent Set}
    \label{alg:unknown_parent_tester}
    \begin{algorithmic}
      \Require{Model $g$, candidate set $F=\{f_1,\ldots,f_s\}$, set of control models $C=\{c_1,\ldots,c_m\}$, prompt space $\Omega$, number of prompts $T$, significance  parameter $\alpha$, statistical test ZTest.
      }
      \State $x_1,\ldots,x_T \stackrel{\text{iid}}{\sim} \Omega$ \Comment{Sample T prompts}
  
      \For{$i \gets 1$ to $s$}
          \State $\mu_i \gets \frac{1}{T}\sum_{j=1}^T \mathds{1}(f_i(x_j)=g(x_j))$ \Comment{Calc sim of candidates}
      \EndFor
      
      \For{$i \gets 1$ to $m$}
          \State $\mu'_i \gets \frac{1}{T}\sum_{j=1}^T \mathds{1}(c_i(x_j)=g(x_j))$ \Comment{Calc sim of controls}
      \EndFor
        \State $\mathcal{M} \gets \{\mu_1,\ldots,\mu_s\} \cup \{\mu'_1,\ldots,\mu'_m\}$ \Comment{Set of all sims}
      \State $\mu_{max} \gets \max(\mathcal{M})$ \Comment{Find highest sim}
      
      \If{$\mu_{max} \notin \{\mu_1,\ldots,\mu_s\}$}
          \Return \textsc{False} \Comment{Highest not from $F$, but from $C$, so cannot be parent}
      \EndIf      
      \For{$\mu' \in \mathcal{M} \setminus \{\mu_{max}\}$}
          \State $p_i \gets \text{ZTest}(\mu_{max}, \mu', T)$ \Comment{Compare against  other sims}
      \EndFor
      
      \State $(p_{(1)},\ldots,p_{(s+m-1)}) \gets \text{Sort}(p_1,\ldots,p_{s+m-1})$
      
      \For{$k \gets 1$ to $s+m-1$}
          \State $\alpha_k \gets \alpha/(s+m-k)$ \Comment{Holm-Bonferroni adjustment}
          \If{$p_{(k)} > \alpha_k$}
              \Return \textsc{False}
          \EndIf
      \EndFor
      
      \Return $(\textsc{True}, \argmax_{i \in [s]} \mu_i)$ \Comment{Return parent }
    \end{algorithmic}
  \end{algorithm}


\end{comment}

\subsection{Reducing Query Complexity}
\label{sec:query}

Most of LLMs available currently allow cheap (even free) API access, thus the monetary query cost of running our testers is insignificant. 
When this is not the case, for example, either when the cost of queries is high (e.g. one query to OpenAI model O1 can cost more than \$1~\cite{openai_pricing}), or the models have some rate restrictions, one can consider enhancements to our testers from Algorithms~\ref{alg:basic_tester},~\ref{alg:unknown_parent_tester}. Furthermore, there are use cases when query complexity can be reduced without any side effects, thus it makes sense from optimization perspective. 
Note that our proposed enhancements for query reduction are not meant to preserve the theoretical guarantees of classical hypothesis testing that our previous testers inherit, but they can be useful in setups where query costs are prohibitive. 

We can divide the queries used in the tester (see Algorithms~\ref{alg:basic_tester},~\ref{alg:unknown_parent_tester}) into two distinctive types: \emph{online queries} made to the tested child model $g$, and \emph{offline queries} made to the parent model $f$ (or models $f_1,\ldots,f_s$) and to the control models $c_1,\ldots,c_m$. We make this distinction for two reasons. First, often offline queries are much cheaper, as the potential parent models (and the control models  as we will see in the Section~\ref{sec:eval}) are well established, and available from multiple sources, thus they are usually cheaper or free. Second, in some use cases, we can reuse the offline queries to perform many provenance tests of different $g_i$. Thus further we analyze separately these two scenarios.

\vspace{10pt}
\noindent
\textbf{Reducing Online Complexity.}
Since our tester is fundamentally based on statistical hypothesis testing, any reduction in query complexity must be compensated by increasing the statistical power of individual queries. Rather than querying model $g$ with $T$ random prompts, we can strategically select a smaller set of $T'<T$ prompts that yield comparable statistical power for detecting model provenance\footnote{It means in  Algorithms~\ref{alg:basic_tester},~\ref{alg:unknown_parent_tester}, instead of random sampling $x_1,\ldots,x_T \stackrel{\text{iid}}{\sim} \Omega$, the goal is to find set $x_1,\ldots,x_{T'}$  from $x_1,\ldots,x_T$ and $F,C$.}. We achieve this through an informed sampling approach: instead of uniform sampling from $\Omega$, we employ rejection sampling with an entropy-based selection criterion. Specifically, to generate each prompt in $T'$, we sample $k$ candidate prompts from $\Omega$ and select the one that maximizes the entropy of output tokens across all parent and control models. The selection criterion is dynamically weighted to favor prompts that have stronger discriminative power between similar models. While this approach introduces dependencies between the sampled  prompts (so the theoretical guarantees of classical hypothesis testing used in Algorithm~\ref{alg:basic_tester} and~\ref{alg:unknown_parent_tester} do not carry over), our empirical results in Section~\ref{sec:eval:online} demonstrate its practical effectiveness.
Full details about the approach are given in Appendix~\ref{sec:appendix:advanced_sampling}.

\vspace{10pt}
\noindent
\textbf{Reducing Offline Complexity.}
In non-adversarial settings where multiple provenance tests are performed against the same parent model $f$, we can trivially reduce the offline complexity by reusing the same set of offline queries across all tests. 
A concrete example of such scenarios arises when issues are discovered in a pre-trained LLM, such as problematic training data or generation of harmful content. A recent example is the lawsuit against the pre-trained model LLama for using copyrighted data in its training set~\cite{Reuters2024Meta}.  Since various teams and organizations may have fine-tuned their applications using this model, but precise provenance information is not readily available, there is a need to identify which models are derived from this problematic base model. In this case, the same set of offline queries to the base model and control models can be reused across all provenance tests.

We further consider the case of reducing offline complexity in settings where offline queries cannot be reused. The current version of our provenance tester samples $T$ prompts for each parent/control model, then runs the hypothesis test to discover the most similar candidate to the tested model $g$ and shows it has significantly higher similarity.
The key observation for reducing offline query complexity is that we may not need an equal number of queries to all parent/control models to identify the most similar one. If a particular parent model shows consistently higher similarity to $g$ compared to other models, we might be able to confirm it as the top candidate with fewer queries to the clearly dissimilar models. The challenge lies in determining when we have sufficient statistical evidence to conclude that one model is significantly more similar than the others, while maintaining our desired confidence levels.

This observation naturally leads us to formulate the problem as a Best Arm Identification (BAI)~\cite{audibert2010best} problem in the Multi-Armed Bandit (MAB) setting. In this formulation, each parent or control model represents an ``arm'' of the bandit, and querying a model with a prompt corresponds to ``pulling'' that arm. The ``reward'' for each pull is the binary outcome indicating whether the model's output matches that of the tested model $g$. The goal is to identify the arm (model) with the highest expected reward (similarity to $g$) while minimizing the total number of pulls (queries). So, we can leverage well-studied MAB algorithms that adaptively allocate queries, focusing more on promising candidates while quickly eliminating clearly dissimilar ones. The implementation of the tester based on BAI is detailed in Appendix~\ref{sec:appendix:offline_bai}. Theoretical guarantees from the MAB literature could be applied to bound the number of queries needed to identify the correct parent model with high probability, but this is beyond our goals.

This work identifies signal collapse as a critical bottleneck in one-shot neural network pruning. Performance loss in pruned networks is due to \textbf{signal collapse} in addition to the removal of critical parameters. We propose \textbf{REFLOW} (\textbf{Re}storing \textbf{F}low of \textbf{Low}-variance signals), a simple yet effective method that mitigates signal collapse without computationally expensive weight updates. By focusing on signal preservation, REFLOW highlights the importance of mitigating signal collapse in sparse networks and enables magnitude pruning to match or surpass state-of-the-art one-shot pruning methods such as CHITA, CBS, and WF.

REFLOW consistently achieves state-of-the-art accuracy across diverse architectures, restoring ResNeXt-101 from under 4.1\% to 78.9\% top-1 accuracy at 80\% sparsity on ImageNet. Its lightweight design makes it a practical solution for both research and deployment, delivering high-quality sparse models without the overhead of traditional approaches. These findings challenge the traditional emphasis on weight selection strategies and underscore the critical role of signal propagation for achieving high-quality sparse networks in the context of one-shot pruning.



\section*{Conclusion}
This paper aims to enhance our understanding of the computational complexity of computing various Shapley value variants. We found that for various ML models --- including decision trees, regression tree ensembles, weighted automata, and linear regression --- both local and global interventional and baseline SHAP can be computed in polynomial time under HMM modeled distributions. This extends popular algorithms, such as TreeSHAP, beyond their empirical distributional scope. We also establish strict complexity gaps between the various SHAP variants (baseline, interventional, and conditional) and prove the intractability of computing SHAP for tree ensembles and neural networks in simplified scenarios. Overall, we present SHAP as a versatile framework whose complexity depends on four key factors: \begin{inparaenum}[(i)] \item model type, \item SHAP variant, \item distribution modeling approach, \item and local vs. global explanations\end{inparaenum}. We believe this perspective provides deeper insight into the computational complexity of SHAP, paving the way for future work.




%We believe that our framework provides a more intricate understanding of SHAP computation complexity across different models, distributions, and variants, paving the way for further research.

Our work opens promising directions for future research. First, expanding our computational analysis to other SHAP-related metrics, such as asymmetric SHAP~\citep{frye20} and SAGE~\citep{covert2020understanding}, would be valuable. Additionally, we aim to explore more expressive distribution classes and relaxed assumptions beyond those in Section \ref{sec:tractable} while maintaining tractable SHAP computation. Finally, when exact computation is intractable (Section \ref{sec:intractable}), investigating the approximability of SHAP metrics through approximation and parameterized complexity theory~\citep{downey2012parameterized} is an important direction.

%Our work opens several promising avenues for future research on the computational properties of explainable AI methods, with a particular focus on SHAP. First, it would be interesting to broaden the computational analysis conducted in this work to include other popular SHAP-related metrics in the literature, such as asymmetric SHAP \cite{frye20} and SAGE \cite{covert2020understanding}. Also, in the future, we aim to explore more expressive distribution classes and relaxed distributional assumptions—extending beyond those examined in Section \ref{sec:tractable} —that still yield tractable SHAP computation. Finally, when exact computation proves intractable (Section \ref{sec:intractable}), it is worthwhile to theoretically investigate the question of the approximability of computing the SHAP metrics across various configurations, through the lens of approximation and parametrized complexity theory \cite{arora2009computational}.

%This paper aims to deepen our understanding of the computational complexity involved in obtaining different Shapley value variants. We found that for a variety of ML models, including decision trees, tree ensembles for regression, weighted automata, and linear regression models — computing both local and global interventional and baseline SHAP can be done in polynomial time when distributions are modeled by HMMs. This extends the distributional scope of popular algorithms like TreeSHAP, which is limited to empirical distributions. Additionally, we demonstrate a strict complexity gap between SHAP variants, showing that interventional and baseline SHAP can be strictly easier to compute than conditional SHAP. Despite these positive results, we uncovered intractability for various SHAP variants in neural networks and tree ensembles. Finally, we provided generalized complexity relations across SHAP variants. We believe that our framework offers a deeper understanding of the complexity involved in computing SHAP across various variants, models, distributions, as well as in both local and global computations, laying the groundwork for future research.

\bibliographystyle{ACM-Reference-Format}
\bibliography{refs}

\end{document}
