\section{Related Works}
\label{sec:related}

\paragraph{Safety Alignment of LLMs.}
Safety alignment ensures that LLMs adhere to human values, reducing their susceptibility to malicious instructions \citep{yi2024vulnerability}. \citet{li2024safety} identified particular ``safety layers” that differentiate malicious from normal queries, revealing distinct behaviors when models process adversarial versus benign prompts. Common alignment methods include RLHF \citep{ouyang_training_2022, bai2022training} and DPO \citep{rafailov2024direct}, but researchers have also explored alternatives. For instance, Safety Arithmetic~\citep{hazra2024safety} is a training-free technique employing parameter arithmetic to suppress harmful outputs while promoting safer ones, and SAFEPATCHING~\citep{zhao2024towards, kim2024decoupling} refines alignment by selectively adjusting model parameters. Another training-free approach \citep{zhou2024emulated} can even reverse an LLM’s safety alignment.

\paragraph{Markov chains and LLMs.}
While they may seem unrelated, Markov chains and LLMs share a core principle. Autoregressive LLMs can be viewed as Markov chains with a large but finite state space, and their token-by-token generation mirrors the “memorylessness” property of Markov processes. \citet{zekri2024large} formally demonstrated that an LLM with vocabulary size $D$ and context length $K$ can be represented by a Markov chain of size $O(D^K)$, offering a theoretical lens for studying convergence and generalization properties in LLMs.

\paragraph{Group theory and LLMs.}  
Group theory focuses on symmetry, manifesting in phenomena from crystal structures to fundamental forces\footnote{Not to be confused with Group Relative Policy Optimization (GRPO) \citep{shao2024deepseekmath} by \citep{guo2025deepseek}.}. In the LLM context, it has been used to test algebraic properties—such as closure, identity, inverse, and associativity—revealing that LLMs often fail to maintain these properties under various testing regimes \citep{imani2024exploring}. For instance, LLMs may produce skewed outputs or show abrupt performance drops beyond certain sequence lengths. Conversely, \citet{chang2024unraveling} explored a more constructive angle, illustrating how carefully curated training data can help LLMs learn and uphold algebraic structures more reliably.