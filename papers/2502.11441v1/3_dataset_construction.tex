
\section{Dataset Construction}
We consider two scenarios for entity unlearning: the fictitious author scenario (TOFU) and a real-world scenario involving actual individuals.  This section details the construction of the forget set and the various neighbor sets for each scenario.

\subsection{Target Entity Selection}

For the real-world scenario, we first select 10 prominent figures across professions: actors, singers, politicians, and business leaders, etc.  These individuals are chosen based on their public visibility and the availability of information about them~\citep{rwku, choi2024opt}. In the TOFU scenario, we follow the method outlined in~\citet{maini2024tofu}, employing a 1\% forget ratio to determine the number of fictitious authors to be included in the forget set.

\subsection{Neighboring Entity Selection}
\label{sec:neighborentityselection}
The selection process for each type of neighbor set varies depending on the specific criteria for each.

\paragraph{Domain Neighbor Set.} For the real-world scenario, domain neighbor entities are constructed by selecting individuals within the same professional domain as the target entities following~\citet{closerlookat, liu2024revisitingharrypotter}. In the TOFU scenario, the domain neighbors provided in~\citet{maini2024tofu} are used.
\paragraph{Entity Neighbor Set.}  For the real-world scenario, entity neighbor entities are selected based on the following criteria adapted from~\citet{choi2024opt, rwku}: 1) a bidirectional relationship exists between the target entity and the potential neighbor, meaning both entities link to each other via hyperlinks on their respective Wikipedia pages and are mentioned at least once on those pages; and 2) the neighboring pages all represent people.  These criteria aim to identify entities closely associated with the target entities, reflecting real-world relationships and connections. For the TOFU scenario, given its fictitious nature, and the absence of a defined entity neighbor concept in~\citet{maini2024tofu}, entity neighbors are not applicable.

\paragraph{Syntactically Similar Neighbor Set.} Unlike the other neighbor sets, which are based on entities, the syntactically similar neighbor set is constructed using questions in $\mathcal{D}_f$. This set consists of questions in the retain set that share a similar syntactic structure with those in the $\mathcal{D}_f$. To construct this set, we first compute the pairwise Levenshtein similarity, as defined in equation~\ref{eq:LevenshteinSimilarity}, between all questions in $\mathcal{D}_f$. Then, we group questions ensuring that each question within a cluster is syntactically similar to the others in that cluster.

\subsection{Generating QA Pairs}
Based on the selected entities, we generate QA pairs that capture key information about each entity.
\paragraph{Real-world Scenario.}
We utilize Wikipedia as a knowledge source following~\citet{rwku}.

For the forget set, domain neighbor set, and entity neighbor set, we employ GPT-4o to generate QA pairs for each entity.  We first gather relevant passages from Wikipedia pages corresponding to each target entity.  These passages serve as the context for prompting GPT-4o to generate QA pairs related to the targets.  To validate the model's knowledge and the quality of the generated pairs, we use these QA pairs to probe the evaluated model. We retain only those QA pairs for which the model successfully recalls the correct answer. This validation ensures both the consistency of the QA pairs and confirms the model's existing knowledge. 
%We also manually verify these probes to ensure correct format and type. For the forget set, we specifically select 15 QA pairs per entity.

For constructing the syntactically similar neighbor set, we first identify entities from the retain set that are not included in any of the other neighbor sets (forget, domain, or entity). Using the syntactic clusters identified in Section~\ref{sec:neighborentityselection}, we generate QA pairs that align with the syntactic structures of these clusters.

Specifically, we adopt the masking approach used in Section~\ref{subsec:EvaluatingRetainSetPreservation} when computing Levenshtein similarity. We first mask entity within the clustered questions and then generate new QA pairs by filling these masked structures with entities from the identified retain set. This ensures that the generated questions maintain syntactic similarity to the existing clusters while introducing new entities. We follow the same verification process (model probing and manual verification) as for the other neighbor sets to ensure the dataset's validity. The detailed procedure for constructing the syntactically similar neighbor set is outlined in Algorithm~\ref{algorithm:constructingDataset}.
% For constructing the syntactically similar neighbor set, we leverage the syntactic clusters identified in Section~\ref{sec:neighborentityselection}. We first identify entities from the retain set that are not included in any of the other neighbor sets (forget, domain, or entity). For these entities, we use GPT-4o to generate QA pairs that align with the syntactic structures of the identified clusters. Specifically, we prompt GPT-4o to create questions that exhibit high Levenshtein similarity to the questions within those clusters. These generated questions form the syntactically similar neighbor set. We follow the same verification process (model probing and manual verification) as for the other neighbor sets to ensure the dataset's validity. The detailed procedure for constructing the syntactically similar neighbor set is outlined in Algorithm~\ref{algorithm:constructingDataset}.


\paragraph{TOFU.} For the TOFU, the forget set and domain neighbor entities are defined by the benchmark itself~\citep{maini2024tofu}. To identify the syntactically similar neighbor set, we compare the provided neighbor sets against the forget set using the same syntactic similarity clustering method described above. Critically, we ensure that there is no overlap with the domain neighbor set. This approach ensures that the syntactically similar neighbor set reflects the structural patterns present in the forget set while maintaining distinctness from other neighbor sets. 

Further details and dataset statistics are provided in the appendix~\ref{appendix:dataset_construction}.