
\pdfoutput=1


\documentclass[11pt]{article}


\usepackage{acl}
\usepackage{algorithm}
\usepackage{algpseudocode}
\usepackage{geometry}
\usepackage[utf8]{inputenc}
\usepackage{tcolorbox}
\usepackage{times}
\usepackage{latexsym}
\usepackage[T1]{fontenc}
\usepackage[utf8]{inputenc}
\usepackage{microtype}
\usepackage{inconsolata}
\usepackage{graphicx}
\usepackage{subcaption}
\usepackage{bm}
\usepackage{comment}
\usepackage{enumitem}
\usepackage{amsmath}
\usepackage{amssymb}
\usepackage{makecell}
\usepackage{tabularx}



\title{Which Retain Set Matters for LLM Unlearning?

A Case Study on Entity Unlearning}

\author{Hwan Chang \and Hwanhee Lee\thanks{Corresponding author.} \\
    Department of Artificial Intelligence, Chung-Ang University, Seoul, Korea\\
    \texttt{\{hwanchang, hwanheelee\}@cau.ac.kr}
}
\begin{document}
\maketitle
\begin{abstract}

% Recent works to jointly reconstruct 3D human and object from a single RGB image, are mostly model-based, that fail to capture the fine details of the clothed human body and object surface. In this paper, we introduce ReCHOR, a novel, model-free, first-method to produce realistic clothed human-object reconstructions from a monocular view. This is extremely challenging due to human-object occlusions, diverse interactions and depth ambiguity, as it needs to infer both 3D spatial awareness and high resolution details. Our core idea is based on estimating neural implicit representations for human and object respectively by an attention-based neural implicit model that attends to pixel-aligned features from both the global human-object image for spatial awareness and  the local separate view of human and object images for high quality details. Additionally, the network is conditioned on semantic features from an initial estimated human-object pose prior and a generative diffusion model that inpaints occluded regions, thus enabling the retrieval of details from them.
% We also propose a synthetic dataset with rendered scenes of diverse, inter-occluded 3D human and object scans, to train our network. We evaluate our method on the synthetic and real world BEHAVE dataset. Our experiments show that our method outperforms the SOTA in achieving realistic clothed human-object reconstructions.
Recent approaches to jointly reconstruct 3D humans and objects from a single RGB image represent 3D shapes with template-based or coarse models, which fail to capture details of loose clothing on human bodies. In this paper, we introduce a novel implicit approach for jointly reconstructing realistic 3D clothed humans and objects from a monocular view. For the first time, we model both the human and the object with an implicit representation, allowing to capture more realistic details such as clothing. This task is extremely challenging due to human-object occlusions and the lack of 3D information in 2D images, often leading to poor detail reconstruction and depth ambiguity. To address these problems, we propose a novel attention-based neural implicit model that leverages image pixel alignment from both the input human-object image for a global understanding of the human-object scene and from local separate views of the human and object images to improve realism with, for example, clothing details. Additionally, the network is conditioned on semantic features derived from an estimated human-object pose prior, which provides 3D spatial information about the shared space of humans and objects. To handle human occlusion caused by objects, we use a generative diffusion model that inpaints the occluded regions, recovering otherwise lost details. For training and evaluation, we introduce a synthetic dataset featuring rendered scenes of inter-occluded 3D human scans and diverse objects. Extensive evaluation on both synthetic and real-world datasets demonstrates the superior quality of the proposed human-object reconstructions over competitive methods.
\end{abstract}
\section{Introduction}\label{sec:intro}

In computational finance, Monte Carlo simulations are used extensively to estimate the expected value of financial payoffs based on the solution of stochastic differential equations (SDEs) which model the evolution of stock prices, interest rates, exchange rates and other quantities \cite{glasserman04}.  Monte Carlo methods are very general and flexible, but for high accuracy it requires generating a large number of costly SDE path approximations, which has motivated research into a number of variance reduction or, equivalently, cost reduction techniques. One such method is
Multilevel Monte Carlo (MLMC), which was proposed in \cite{GILES2008} and was adapted for various applications that are summarised in \cite{Giles_overview17} and successfully combined with other methods such as quasi-Monte Carlo methods. The main idea of MLMC is to approximate the payoff using different time stepping resolutions when numerically solving the underlying SDE and to generate an optimal number of samples on each level, such that the overall computational cost is minimised subject to the desired bound on the variance. %, such that the total computational cost is minimised. 
The computational savings come from the fact that most samples are computed on the coarser levels and hence are less expensive while only a few samples from the finest levels are required \cite{GILES2008}.


Among the directions in which the computational cost 
of MLMC methods could further be reduced, an important avenue is the use of lower precision calculations, especially for the first Monte Carlo levels where the targeted accuracy is relatively low. 
 An overview of the research on mixed precision for the standard Monte Carlo (MC) framework is provided in \cite{ChowMixedPrecisionStandardMC} but only a few references study the potential of low precision computation in the MLMC framework \cite{Rounding_error_oliver}. To the best of our knowledge, the only MLMC framework with customised precision in the literature is \cite{brugger2014mixed}, but they use a uniform precision for all operations on each Monte Carlo level instead of optimising 
 the precision of each intermediary variable to reduce as much as possible the cost of path generation.
 
An important motivation for an MLMC framework with variable precision would be performing the low precision computations on reconfigurable hardware devices such as Field Programmable Gate Arrays (FPGAs). FPGAs contain customizable logic blocks and connectors that make it easy to adapt the digital circuit architecture for a specific application, leading to a highly parallel and optimised implementation. Therefore they are successfully exploited in applications that require high speed and have high computational workload, such as signal processing \cite{woods2008fpga}, and real time applications like high frequency trading \cite{HFT1,HFT2}. That is why a number of previous works in hardware architecture design implemented the MLMC algorithm to price financial options using FPGAs as accelerators, which resulted in improved speed and power efficiency compared to full CPU architectures \cite{Schryver2013AMM}. The paper \cite{lindsey2016domain} also proposed 
a Domain Specific Language to automate the configuration of FPGAs for this specific application. However, only \cite{brugger2014mixed} proposed a heuristic to reduce the precision in calculations.

In addition, all aforementioned works considered that the random number generation (RNG) is performed in single or double precision. Yet in most cases an important portion of the workload in the overall MLMC simulation comes from the RNG and in \cite{brugger2014mixed} this limited the total computational savings.
To reduce the cost of MLMC simulations in particular those based on the Geometric Brownian Motion (GBM), \cite{approximateICDF_Oliver, NestedOliver} have proposed to use approximate random numbers that are generated by applying an approximation of the inverse CDF to uniform random numbers. In \cite{NestedOliver}, the authors proposed a way to integrate these lower precision random variables into a \textit{nested} MLMC framework and completed a numerical analysis to bound the resulting error at each MC level by a product of the time step and the error in the random number approximation. The same authors show in \cite{approximateICDF_Oliver} that using approximate random variables reduces the cost of path generation by a factor 7.


In this paper we propose a nested MLMC framework that combines the use of approximate random normal variables and lower precision calculations to reduce the computational cost of MLMC even further than \cite{brugger2014mixed,NestedOliver}. We illustrate the efficiency of our framework in Matlab, after making several assumptions on the cost of operations and size of the errors that we carefully justify. We focus on the case of GBM and use the approximate RNG methods presented in \cite{approximateICDF_Oliver} as well as a new slightly modified method that combines CDF inversion and the central limit theorem. To choose the precision of the variables in the low precision path generation, we introduce a novel method to optimise the bit-widths. This optimisation is performed before the main path generation loop is executed and is based on a linear model of the payoff error  
due to rounding when computing in low precision. The error model relies on algorithmic differentiation in a similar manner to \cite{unifying-bwoptim,bitwidth-AD,ADAPT}. The bit-width optimisation procedure can be performed off-line, so this stage can be excluded from the on-line time complexity of our framework. The user specified desired accuracy is then enforced by calculating on-line the number of samples that need to be generated.

In terms of hardware design, we suggest implementing the low precision path generation on FPGAs and the full-precision ones on a CPU or GPU. 
The FPGA offers enough flexibility to define a separate bit-width for every variable in the low precision path generation, and can be reconfigured periodically to update the bit-widths when the market parameters have changed considerably. 


The paper is organized as follows : \Cref{sec:MLMC} introduces MLMC and nested MLMC to make clear the estimator that is implemented in our framework. Then in \Cref{sec:RNG} we detail the methods that could be used to obtain approximate random normally distributed numbers very cheaply for the low precision path generation. In \Cref{sec:error_model} and \Cref{sec:costModel} we propose an error model and a cost model (resp.) that we then use to formulate the optimisation problem that is solved to obtain the optimal bit-widths of fixed point variables in \Cref{sec:optimisation}. Finally we summarise our results and future directions in \Cref{sec:conclusion}.



\section{Background}
\label{sec:background}

\begin{figure*}[htbp]
\centering
\includegraphics[width=\textwidth]{Fig_background.pdf}
\caption{Ciphertext side-channel examples and revisiting vulnerabilities from the perspective of compilation.}
\label{fig:background}
\end{figure*}

\subsection{Ciphertext Side-Channel Attacks}
\label{subsec:ciphertext}

The ciphertext side channel originates from the deterministic memory encryption implemented in AMD's TEE.
The encrypted memory is calculated by an XOR-Encrypt-XOR (XEX) mode, expressed as: $c = ENC(m \oplus T(P_{m})) \oplus T(P_{m})$, where the plaintext $m$ undergoes the XOR operations before and after AES-128 encryption with a tweak value $T(P_{m})$ that incorporates the physical address $P_{m}$.
Without freshness in the encryption process, the encryption of the same plaintext at a given physical address produces the identical ciphertext.
It is crucial to acknowledge that this vulnerability extends to other deterministic encryption-based TEE architectures as long as attackers have read accesses to ciphertext (via software access~\cite{li2021cipherleaks} or memory bus snooping~\cite{lee2020off}).

% \begin{figure}[htbp]
% \vspace{-5pt}
% \begin{minipage}[c]{0.5\linewidth}
%     \begin{subfigure}[b]{\linewidth}
%     \centering
%     \footnotesize
%     \begin{tabular}{l}
%         1: pbit $\leftarrow$ 1;\\
%         2: \textbf{for}\ i $\leftarrow$ cardinality\_bit - 1\ downto\ 0$\lbrace$\\
%         3: $\quad$ kbit $\leftarrow$ BN\_is\_bit\_set(k, i) $\wedge$ pbit;\\
%         4: $\quad$ EC\_POINT\_CSWAP(kbit, r, s, ...);\\
%         5: $\quad$ ...\\
%         6: $\quad$ pbit $\leftarrow$ pbit $\wedge$ kbit;$\rbrace$\\
%     \end{tabular}
%     \caption{ossl\_ec\_scalar\_mul\_ladder.}
%     \label{fig:channel1}
%     \end{subfigure}
% \end{minipage}
% \hspace{15pt}
% \begin{minipage}[c]{0.4\linewidth}
%     \begin{subfigure}[b]{0.9\linewidth}
%     \centering
%     \footnotesize
%     \begin{tabular}{l}
%         1: \textbf{for}\ i $\leftarrow$ 0\ to\ nwords - 1$\lbrace$\\
%         2: $\quad$ t $\leftarrow$ (a.d[i] $\wedge$ b.d[i])\\
%         3: $\quad \quad \quad$ \&\ condition;\\
%         4: $\quad$ a.d[i] $\leftarrow$ a.d[i] $\wedge$ t;\\
%         5: $\quad$ b.d[i] $\leftarrow$ b.d[i] $\wedge$ t;$\rbrace$\\
%     \end{tabular}
%     \caption{BN\_constant\_swap.}
%     \label{fig:channel2}
%     \end{subfigure}
% \end{minipage}
% \caption{Ciphertext side-channel examples.}%\yz{change font in figures.}
% \label{fig:channels}
% \vspace{-5pt}
% \end{figure}

Two attack schemes are introduced in~\cite{li2022systematic}.
The \textit{Dictionary} attack involves the continuous monitoring of the ciphertext at a fixed memory address to construct a dictionary containing mappings of ciphertext-plaintext pairs.
Consider the code snippet shown in \F~\ref{fig:background}(a), extracted from the ECDSA Montgomery ladder algorithm implemented in OpenSSL-3.0.2.
In each loop iteration, the \texttt{BN\_is\_bit\_set} function (denoted by $k_{i}$ in line 3) is utilized to obtain one bit of the secret $k$.
Following this, the $kbit$ variable is computed through an XOR operation with the value in $pbit$, which is then written back to $pbit$.
Given the dual XOR operations in lines 3 and 6, $pbit$ ultimately stores each bit of the secret $k$.
The attacker records consecutive ciphertext pairs ($pbit$-$kbit$) both before and after the \texttt{BN\_is\_bit\_set} function, aiming to deduce $k_{i}$ in each iteration based on the changes observed in ciphertext pairs.
In contrast, the \textit{Collision} attack focuses on identifying repetitions or alterations in certain ciphertexts to break the constant-time mechanism.
\F~\ref{fig:background}(b) shows the constant-time-swap function \texttt{BN\_constant\_swap}.
This function takes two variables $a$ and $b$, along with a decision $C$ (e.g., $kbit$ in line 4 of \F~\ref{fig:background}(a)).
If $C$ is set to 1, the values of $a$ and $b$ are exchanged, leading to observable changes in the ciphertext. Conversely, if $C$ is 0, the ciphertext remains unaltered.
In this way, the \textit{Collision} attack recovers the decision $C$, undermining the constant-time component.

Currently, many well-known cryptographic applications are vulnerable to this attack, including RSA and ECDSA (such as \textit{secp256k1} and \textit{secp384r1}) equipped with constant-time algorithms, ECDSA from WolfSSL-5.3.0, ECDSA and RSA from MbedTLS-3.1.0, as well as EdDSA (\textit{Ed25519}) from OpenSSH adopted by Ubuntu LTS 20.04~\cite{li2021cipherleaks, li2022systematic}.

\subsection{Countermeasures to Ciphertext Side-channels}
\label{subsec:countermeasures}

Hardware-based countermeasures provide stronger security by eliminating ciphertext side channels, but they require extensive validation before chip manufacturing. In contrast, we choose a software-based approach, enabling quicker implementation and deployment without modifying hardware.
Unfortunately, existing countermeasures for cache and timing side channels~\cite{percival2005cache, osvik2006cache, zhang2012cross, yarom2014flush, liu2015last, yarom2014recovering, ryan2019return, aranha2020ladderleak}, like constant-time cryptography, cannot mitigate ciphertext side channels. While constant-time cryptography avoids secret-dependent branches and memory accesses, it has been shown to be ineffective against ciphertext side-channel attacks~\cite{li2021cipherleaks, li2022systematic, deng2023cipherh}.

% Previous efforts adhering to this concept can be categorized into three classes. 
% 1) Researchers verify whether a cryptography program satisfies the constant-time criterion using various approaches, including the program counter model~\cite{agat2000transforming, molnar2005program, barthe2006preventing, kopf2007transformational, almeida2013certified, mantel2015transforming}, observation-equivalence-based noninterference~\cite{barthe2014system, almeida2016verifiable, rodrigues2016sparse, dehesa2017verifying}, and self-composition-based noninterference~\cite{almeida2013formal, almeida2016verifying, chen2017precise, antonopoulos2017decomposition, yang2018lazy, blazy2019verifying, daniel2020binsec}.
% 2) Conceptually, formally constructing high-assurance cryptography libraries shall fundamentally resolve the constant-time issues, leveraging formal languages like F$^{*}$~\cite{zinzindohoue2016verified}, HACL$^{*}$~\cite{zinzindohoue2017hacl}, Vale~\cite{bond2017vale}, Jasmin~\cite{almeida2017jasmin} and Fact~\cite{cauligi2019fact}.
% 3) Transforming existing programs into constant-time equivalents also significantly contributes to resisting side channels. For instance, some approaches~\cite{wu2018eliminating,soares2021memory} execute both real and decoy paths; Constantine~\cite{borrello2021constantine} leverages the linearization of control-flow and data-flow.

Without detailed implementation, AMD's whitepaper~\cite{amdmeasures} and Li et al.~\cite{li2022systematic} proposed countermeasures as follows, but no single software-based scheme is perfectly suited for both methodology and implementation. 
Therefore, exploring different mitigation approaches, particularly through compiler-level optimizations and combinations, offers valuable insights for improving defenses.

\begin{packed_itemize}
\item[1)] Preserving secret variables in registers instead of memory enhances security~\cite{li2022systematic}, but faces implementation challenges due to limited register availability.

\item[2)] Avoiding the reuse of fixed memory addresses ensures fresh ciphertexts~\cite{li2022systematic, amdmeasures}, but requires extra memory and precise runtime reference management, potentially leading to significant performance overhead.

\item[3)] Introducing a random nonce to the plaintext with each memory write increases ciphertext unpredictability~\cite{li2022systematic}. This includes masking and padding strategies~\cite{amdmeasures}, where padding requires extended data structures.
\end{packed_itemize}


\section{Dataset Construction}
We consider two scenarios for entity unlearning: the fictitious author scenario (TOFU) and a real-world scenario involving actual individuals.  This section details the construction of the forget set and the various neighbor sets for each scenario.

\subsection{Target Entity Selection}

For the real-world scenario, we first select 10 prominent figures across professions: actors, singers, politicians, and business leaders, etc.  These individuals are chosen based on their public visibility and the availability of information about them~\citep{rwku, choi2024opt}. In the TOFU scenario, we follow the method outlined in~\citet{maini2024tofu}, employing a 1\% forget ratio to determine the number of fictitious authors to be included in the forget set.

\subsection{Neighboring Entity Selection}
\label{sec:neighborentityselection}
The selection process for each type of neighbor set varies depending on the specific criteria for each.

\paragraph{Domain Neighbor Set.} For the real-world scenario, domain neighbor entities are constructed by selecting individuals within the same professional domain as the target entities following~\citet{closerlookat, liu2024revisitingharrypotter}. In the TOFU scenario, the domain neighbors provided in~\citet{maini2024tofu} are used.
\paragraph{Entity Neighbor Set.}  For the real-world scenario, entity neighbor entities are selected based on the following criteria adapted from~\citet{choi2024opt, rwku}: 1) a bidirectional relationship exists between the target entity and the potential neighbor, meaning both entities link to each other via hyperlinks on their respective Wikipedia pages and are mentioned at least once on those pages; and 2) the neighboring pages all represent people.  These criteria aim to identify entities closely associated with the target entities, reflecting real-world relationships and connections. For the TOFU scenario, given its fictitious nature, and the absence of a defined entity neighbor concept in~\citet{maini2024tofu}, entity neighbors are not applicable.

\paragraph{Syntactically Similar Neighbor Set.} Unlike the other neighbor sets, which are based on entities, the syntactically similar neighbor set is constructed using questions in $\mathcal{D}_f$. This set consists of questions in the retain set that share a similar syntactic structure with those in the $\mathcal{D}_f$. To construct this set, we first compute the pairwise Levenshtein similarity, as defined in equation~\ref{eq:LevenshteinSimilarity}, between all questions in $\mathcal{D}_f$. Then, we group questions ensuring that each question within a cluster is syntactically similar to the others in that cluster.

\subsection{Generating QA Pairs}
Based on the selected entities, we generate QA pairs that capture key information about each entity.
\paragraph{Real-world Scenario.}
We utilize Wikipedia as a knowledge source following~\citet{rwku}.

For the forget set, domain neighbor set, and entity neighbor set, we employ GPT-4o to generate QA pairs for each entity.  We first gather relevant passages from Wikipedia pages corresponding to each target entity.  These passages serve as the context for prompting GPT-4o to generate QA pairs related to the targets.  To validate the model's knowledge and the quality of the generated pairs, we use these QA pairs to probe the evaluated model. We retain only those QA pairs for which the model successfully recalls the correct answer. This validation ensures both the consistency of the QA pairs and confirms the model's existing knowledge. 
%We also manually verify these probes to ensure correct format and type. For the forget set, we specifically select 15 QA pairs per entity.

For constructing the syntactically similar neighbor set, we first identify entities from the retain set that are not included in any of the other neighbor sets (forget, domain, or entity). Using the syntactic clusters identified in Section~\ref{sec:neighborentityselection}, we generate QA pairs that align with the syntactic structures of these clusters.

Specifically, we adopt the masking approach used in Section~\ref{subsec:EvaluatingRetainSetPreservation} when computing Levenshtein similarity. We first mask entity within the clustered questions and then generate new QA pairs by filling these masked structures with entities from the identified retain set. This ensures that the generated questions maintain syntactic similarity to the existing clusters while introducing new entities. We follow the same verification process (model probing and manual verification) as for the other neighbor sets to ensure the dataset's validity. The detailed procedure for constructing the syntactically similar neighbor set is outlined in Algorithm~\ref{algorithm:constructingDataset}.
% For constructing the syntactically similar neighbor set, we leverage the syntactic clusters identified in Section~\ref{sec:neighborentityselection}. We first identify entities from the retain set that are not included in any of the other neighbor sets (forget, domain, or entity). For these entities, we use GPT-4o to generate QA pairs that align with the syntactic structures of the identified clusters. Specifically, we prompt GPT-4o to create questions that exhibit high Levenshtein similarity to the questions within those clusters. These generated questions form the syntactically similar neighbor set. We follow the same verification process (model probing and manual verification) as for the other neighbor sets to ensure the dataset's validity. The detailed procedure for constructing the syntactically similar neighbor set is outlined in Algorithm~\ref{algorithm:constructingDataset}.


\paragraph{TOFU.} For the TOFU, the forget set and domain neighbor entities are defined by the benchmark itself~\citep{maini2024tofu}. To identify the syntactically similar neighbor set, we compare the provided neighbor sets against the forget set using the same syntactic similarity clustering method described above. Critically, we ensure that there is no overlap with the domain neighbor set. This approach ensures that the syntactically similar neighbor set reflects the structural patterns present in the forget set while maintaining distinctness from other neighbor sets. 

Further details and dataset statistics are provided in the appendix~\ref{appendix:dataset_construction}.
\begin{figure*}[t]
    \centering
    \begin{subfigure}{0.49\linewidth}
        \includegraphics[width=\linewidth]{figure/01_01_realworld_dissimilarity_similarity_comparison.pdf}
        \caption{Real-world Scenario}
        \label{fig:real-world_main}
    \end{subfigure}
    \hfill
    \begin{subfigure}{0.49\linewidth}
    \label{fig:tofu_main}
        \includegraphics[width=\linewidth]{figure/01_02_tofu_dissimilarity_similarity_comparison.pdf}
        \caption{TOFU}
        \label{fig:tofu_main}
    \end{subfigure}
    \caption{Relative Utility Drop (\%) for different neighbor sets across real-world scenario (left) and TOFU (right). Each method (GA, NPO, IDK, DPO) is evaluated based on its model utility before and after unlearning, with lower bars indicating greater utility loss. Model utility values before and after unlearning are provided in Appendix~\ref{appendix:detailedResultsPerMethods}}
    \label{fig:experiment1}
\end{figure*}
\section{Experimental Setup}
\subsection{Evaluation Metrics}
We evaluate the unlearned model using several metrics to assess its performance from various perspectives~\citep{closerlookat, maini2024tofu}. Specifically, we employ \emph{ROUGE} to measure word-level similarity, \emph{BERT Cosine Similarity} to assess semantic consistency between outputs before and after unlearning, \emph{Probability} to evaluate the model's confidence to predict the ground truth answer, and \emph{Entailment Score} to assess factual correctness relative to the ground truth.

\noindent
Since all metrics range from zero to one, we aggregate them using the arithmetic mean. Applying this to the retain set defines \textbf{Model Utility (MU)}, while applying it to the forget set defines \textbf{Forget Efficacy (FE)}.

\noindent
To quantify the impact of unlearning on neighbor sets, we define the \textbf{Relative Utility Drop (RUD)} as:
\begin{equation}
\resizebox{0.6\hsize}{!}{$
\textstyle \text{RUD} = \frac{MU_{\text{after}} - MU_{\text{before}}}{MU_{\text{before}}} \times 100.$}
\end{equation}
Since unlearning typically reduces MU, RUD is usually negative, indicating the degree of performance drop. This metric shows which neighbor set suffers the most performance decline after unlearning. Further details on metric computation are provided in Appendix~\ref{appendix:evaluationMetrics}.
\subsection{Unlearning Methods}
We explore various unlearning strategies, each of which aims to erase knowledge of target entities in distinct ways. A comprehensive explanation of these methods is provided in Appendix~\ref{appendix:overviewUnlearningMethods}.
\begin{itemize}[leftmargin=6pt]
    \item \textbf{GA}~\cite{jang2023knowledgeunlearning}: Utilizes gradient ascent on the forget set to counteract learned knowledge.
    \item \textbf{DPO}~\cite{rafailov2023dpo}: Treats unlearning as a preference optimization problem by applying the standard DPO loss. It uses answers in the forget set as negative samples and rejection templates (e.g., ``I don't know'') as positive samples to guide the model's response.
    \item \textbf{NPO}~\cite{zhang2024npo}: A variant of DPO that removes positive samples from the optimization process. It retains only negative examples from the forget set, encouraging the model to suppress forgotten information without explicit reinforcement of alternative responses.
    \item \textbf{IDK}~\cite{maini2024tofu}: Fine-tunes the model to default to ``I don't know'' responses when queried about the forget set.
\end{itemize}


\subsection{Implementation Details}
For the TOFU benchmark~\citep{maini2024tofu}, we utilize fine-tuned Llama-2-7b-chat~\citep{touvron2023llama2}, which has been trained on the constructed dataset to ensure it precisely answers questions in TOFU. For the real-world scenario benchmark, we employ Llama-3-8B-Instruct~\citep{dubey2024llama3}.
\noindent
To enable a fair comparison of different unlearning methods at similar levels of forgetting, we adjust the hyperparameters to keep Forget Efficacy between 0.65 and 0.75. Further details are provided in Appendix~\ref{appendix:implementationDetails}.


\section{How does Performance Degradation Vary across Different Neighbor Sets?}
\label{sec:problem}
This section investigates how performance degradation after unlearning varies across different neighbor sets.
First, we examine which neighbor sets experience the most significant performance degradation. (Section~\ref{sec:5_1}) If similar syntactic structures sets are the most vulnerable to forgetting, we further examine whether domain differences within these structures lead to varying effects. (Section~\ref{sec:5_2}) We then examine the robustness of these forgetting patterns when questions are paraphrased. (Section~\ref{sec:5_3}) Finally, we analyze gradient updates during unlearning to understand the underlying mechanisms driving the observed patterns. (Section~\ref{sec:5_4})
% First, we examine which neighbor sets experience the most significant performance degradation. (Section~\ref{sec:5_1}) Finding that syntactically similar sets show the highest vulnerability, we then investigate whether domain differences within syntactic similarities create varying effects. (Section~\ref{sec:5_2}) We then examine the robustness of these forgetting patterns when questions are paraphrased. (Section~\ref{sec:5_3}) Finally, we analyze gradient updates during unlearning to understand the underlying mechanisms driving the observed patterns. (Section~\ref{sec:5_4})

\subsection{Analyzing Performance Drops Across Neighbor Sets}
\label{sec:5_1}
\vspace{2pt}
\textbf{Syntactically Similar Neighbor Set Experiences Higher Forgetting.} Across both real-world scenario and  TOFU evaluations (Figure~\ref{fig:real-world_main} and Figure~\ref{fig:tofu_main}), $\mathcal{N}_{\text{syntactically}}$ consistently demonstrates a higher utility drop compared to both $\mathcal{N}_{\text{domain}}$ and $\mathcal{N}_{\text{entity}}$. The greater utility drop suggests that syntactic similarity plays a crucial role in the forgetting phenomenon.  When the model is unlearning specific data, it appears to struggle more with retaining information that shares similar sentence structures, regardless of the specific domain or entities involved.

\noindent \textbf{No Significant Difference among Existing Neighbor Sets.} In the real-world scenario, a notable observation is the lack of significant performance differences between $\mathcal{N}_{\text{domain}}$ and $\mathcal{N}_{\text{entity}}$. As depicted in Figure~\ref{fig:real-world_main}, both sets exhibit similar RUD across all methods. Our results show that, despite different ways of defining neighbor sets in previous studies~\citep{choi2024opt, closerlookat}, the impact caused by unlearning is similar across them.

\subsection{Exploring Domain Effects on Forgetting in Syntactically Similar Cases}
\label{sec:5_2}
To examine the domain-specific effects of unlearning in syntactically similar cases, we conduct experiments in real-world scenario across five distinct categories: \textit{Human, Company, Creative Works, Fictional Character,} and \textit{Product}. This analysis builds on our previous findings that syntactically similar neighbor sets exhibit more pronounced forgetting than those based on domain or entity similarity.

Prior studies~\cite{rwku,maini2024tofu} define data distributions similar to $\mathcal{D}_f$ using either entity similarity or professional domain association, assuming that these sets would experience the most significant forgetting. Under this assumption, the Human category would be expected to exhibit the highest degree of forgetting, as entities within this category tend to be closely related. However, as shown in Figure~\ref{fig:variousfig}, our results reveal the opposite trend—across all evaluated methods except for DPO, non-human categories exhibit substantially higher forgetting rates. This finding challenges the conventional assumption that neighbor sets constructed based on entity or domain similarity necessarily lead to higher forgetting.
\begin{figure}[t]
    \centering
    \includegraphics[width=1\linewidth]{figure/03_various_entity_types.pdf}
    \caption{Relative Utility Drop across different entity categories (Human, Company, Creative Works, Fictional Character, and Product) for various unlearning methods.}
    \label{fig:variousfig}
\end{figure}
\begin{figure}[t]
    \centering
    \includegraphics[width=1\linewidth]{figure/02_paraphrase_experiments.pdf}
    \caption{Relative Utility Drop for syntactically similar and different neighbor sets across different unlearning methods, measured over three paraphrases per question. A larger drop indicates higher semantic forgetting.}
    \label{fig:paraphrasefig}
\end{figure}
\subsection{Robustness of Forgetting Patterns in Paraphrased Scenarios}
\label{sec:5_3}
Our previous experiments reveal that syntactically similar neighbor sets experience higher levels of forgetting compared to other neighbor sets. To validate the robustness of this finding, we investigate whether this performance gap persists even when questions are paraphrased. 

Specifically, we generate paraphrased versions for each question for syntactically similar and different neighbor sets using GPT-4o Then, we filter out cases where the pre-unlearning model fails to provide correct answers, ensuring that each question has three valid paraphrases. We then measure the RUD for these paraphrased questions using the post-unlearning model and compare the forgetting rates across the two groups.

Figure~\ref{fig:paraphrasefig} shows that even after paraphrasing, syntactically similar neighbors exhibit greater utility drops than dissimilar neighbors. This suggests that the model's increased forgetting isn't solely due to shared syntax, but also reflects a sensitivity to underlying semantic relationships. The consistent performance gap after paraphrasing reinforces the role of syntactic similarity in forgetting, highlighting its influence beyond surface-level wording.





% Specifically, we generate three paraphrased versions for each question in both syntactically similar and different neighbor sets. We then measure the RUD for these paraphrased questions and compare the forgetting rates across the two groups.


\subsection{Gradient Analysis}
\label{sec:5_4}
To further investigate the underlying mechanisms behind the forgetting patterns observed in syntactically similar and dissimilar neighbor sets, we analyze the gradient updates during the unlearning process. Our primary goal is to understand how the model's gradient norms evolve when encountering different types of neighbors, particularly whether syntactically similar instances influence each other more strongly than dissimilar ones.

In our experimental setup, we perform gradient ascent on a syntactically similar set and track the changes in gradient norms as the model encounters other syntactically similar or syntactically different instances. Specifically, we measure the Frobenius norm of the model’s weight gradients at each unlearning step, comparing how the gradients behave when interacting with different types of data points.


\begin{figure}[t]
    \centering
    \includegraphics[width=1\linewidth]{figure/04_gradient_norm_plot.pdf}
    \caption{Frobenius norm of model weight gradients across unlearning steps. The gradient norms for syntactically similar instances (red) increase more steeply than those for syntactically different instances (blue).}
    \label{fig:gradientanalysis}
\end{figure}
As shown in Figure~\ref{fig:gradientanalysis}, the gradient norms of syntactically similar instances exhibit a steeper increase over unlearning steps compared to syntactically different instances. Notably, the initial gap between their gradient norms at the first checkpoint widens progressively as unlearning proceeds. This suggests that forgetting syntactically similar knowledge amplifies gradient updates in a way that reinforces the distinction between similar and dissimilar instances.
\begin{figure*}[t]
    \centering
    \includegraphics[width=0.9\linewidth]{figure/05_heatmap_figure.pdf}
    \caption{Relative utility drop (\%) averaged across all unlearning methods (GA, DPO, NPO, and IDK) under different retain set configurations using GD (left) and KL (right) regularization. The x-axis represents the type of train retain set, while the y-axis represents the type of test retain set. A higher value (darker color) indicates better utility retention. Detailed relative utility drop results for each individual unlearning method can be found in Appendix~\ref{appendix:detailedResultsPerMethods}.}
    \label{fig:regularizationheapmap}
    %\vspace{-3mm}
\end{figure*}
\section{What is the Optimal Neighbor Set for Effective Regularization?}
\label{sec:solution}
To preserve model utility during unlearning, regularization losses on a subset of the retain set are commonly employed during the unlearning process~\cite{closerlookat,maini2024tofu}. Based on the findings of the previous section, we aim to identify the optimal configuration of the retain set used for regularization, to optimize model utility while ensuring successful forgetting, specifically from a data perspective.

\paragraph{Regularization loss.} It encourages the unlearned model parameters $\bm \theta$ to preserves model utility. A typical unlearning objective function, computed on a subset of $\mathcal{D}_{\text{R}}$, is formulated as follows:
\begin{equation}
    \resizebox{0.71\hsize}{!}{$
    \underset{\bm \theta}{\min} \mathcal{L}(\bm \theta) = \underset{\bm \theta}{\min} - \mathcal{L}_f(\bm \theta) +  \mathcal{L}_{\text{R}}(\theta;\mathcal{D}_{\text{R}}).$}
    \label{eq:previous_objective}
\end{equation}
Our analysis considers two primary regularization approaches: Gradient Descent (GD) and Kullback-Leibler Divergence (KL). A comprehensive explanation of these methods is provided in Appendix~\ref{appendix:overviewUnlearningMethods}.
\paragraph{Setup.} To determine the optimal train retain set configuration, we conduct comprehensive experiments examining nine different combinations of train and test retain sets, using $\mathcal{N}_{\text{domain}}$, $\mathcal{N}_{\text{entity}}$, and $\mathcal{N}_{\text{syntactically}}$ for both training and evaluation. For each train retain set, we apply different unlearning methods (GA, DPO, NPO, and IDK) with regularization loss and report the average RUD across test retain sets.

\paragraph{Results.} We visualize the results separately for GD and KL regularization in Figure~\ref{fig:regularizationheapmap}. The results reveal two key findings:

\noindent
\textbf{1) Training with $\mathcal{N}_{\text{syntactically}}$ effectively preserves performance on $\mathcal{N}_{\text{syntactically}}$.} In both GD and KL regularization heatmaps, the bottom row (Test Retain Set: Syntactically Similar) shows that training with $\mathcal{N}_{\text{syntactically}}$ preserves utility best, with average differences of 14.7\% point and 7.35\% point compared to other training sets, respectively.

\noindent
\textbf{2) Training with $\mathcal{N}_{\text{syntactically}}$ contributes to robust performance across various neighbor sets.} Beyond preserving performance on syntactically similar data, training with $\mathcal{N}_{\text{syntactically}}$ also yields competitive results when evaluated on $\mathcal{N}_{\text{entity}}$ and $\mathcal{N}_{\text{domain}}$. In many cases, it surpasses or closely matches the performance achieved by training with other neighbor sets. These findings emphasize the role of syntactically similar examples in reducing utility loss while unlearning.
\section{Related Work}
LLM unlearning~\citep{jang2023knowledgeunlearning, yao2023llmunlearningsurvey, lynch2024eight} has gained significant attention as a method to enhance privacy. Various approaches~\citep{sinha2024unstar, zhang2024npo} have been proposed to ensure that models effectively erase specific information while maintaining overall performance. A key challenge in unlearning is assessing whether knowledge unrelated to the forget set is inadvertently affected. To evaluate this, researchers commonly examine general knowledge~\citep{hendrycks2021measuring, cobbe2021training} as well as a designated subset of the retain set that shares a similar distribution with the forget set but excludes the targeted information. These subsets, often referred to as neighbor sets~\citep{closerlookat}, help determine the extent of unintended degradation in model performance.

In hazardous knowledge unlearning, prior work has leveraged domain-relevant general knowledge as a benchmark. For instance,~\citet{li2024wmdp} employs general biology knowledge to assess the impact of bioweapon-related unlearning and general computer security knowledge to evaluate the removal of information related to Attacking Critical Infrastructure. For entity unlearning~\citep{maini2024tofu, rwku}, previous studies have used entities from similar professions or those closely linked to the target entity as neighbor sets. While these approaches provide an initial framework, they lack a systematic investigation of which aspects of the retain set suffer the most from unlearning. Our study addresses this gap by systematically investigating the impact of unlearning on different types of neighbor sets more clearly and identifying which knowledge components experience the highest degree of forgetting.
\section{Summary and Conclusion}
\label{sec:conclusion}


In this paper, we introduced \ToolName{}, a method for discovering fine-grained \emph{sub-activities} from unlabeled smart home sensor data without relying on pre-segmentation. Our pipeline is organized into two core steps: Clustering and Labeling. 
The \textbf{Clustering step} consists of:

\begin{itemize}
    \item \textbf{Encoder Pre-Training:} We leverage a pre-trained BERT model adapted with sensor-specific tokens and train it using a masked language modeling (MLM) objective to generate context-rich embeddings for raw sensor sequences.
    
    \item \textbf{Clustering Model Fine-Tuning:} Using the SCAN loss function, we fine-tune these embeddings to form more homogeneous and distinct clusters of sensor sequences.
\end{itemize}

The \textbf{Labeling step} comprises:

\begin{itemize}
    \item \textbf{Cluster Centroid Annotation:} Representative sequences from each cluster are visualized with a custom tool, enabling expert annotators to assign meaningful sub-activity labels to the centroids.
    
    \item \textbf{Label Propagation:} The centroid labels are propagated to all sequences within their respective clusters, resulting in a fully labeled dataset with minimal manual effort.
    
    \item \textbf{Re-annotation of Original Time-Series Data:} 
    Finally, these propagated labels are mapped back onto the original time-series data, preserving temporal continuity and facilitating the analysis of longitudinal activity patterns.
\end{itemize}


Our approach addresses important challenges in HAR, including the high cost and effort of manual data annotation, the limitations of coarse activity labels, and the need for scalable and generalizable models. \ToolName{} offers an open source tool that facilitates the HAR annotation and re-annotation process and enables the dynamic discovery and validation of sub-activities, thus capturing a broader spectrum of behaviors observed in real homes.
\section{Limitations}

Although our method is generally applicable to all common LLM architectures, as they share the same language modeling head and embeddings, only dense decoders were used in our experiments. 
In addition, only models with up to $N=2.6\B$ parameters have been tested.
The cosine decay learning rate schedule was applied throughout all experiments (App.~\ref{app:hyperparameters}). Alternatives such as an infinite learning rate schedule are not incorporated in our study.
Furthermore, as mentioned at the end of Sec.~\ref{sec:results}, we have not explicitly verified that the slight residual shift of the mean embedding, which is observed even for Coupled Adam, is caused by weight tying.
Finally, we have used a straightforward implementation of Coupled Adam, closely following Algorithm~\ref{alg:algorithm_adam}. More sophisticated implementations might lead to increased efficiency and further improvements; we leave it for future work to investigate this.


\section*{Acknowledgement}
This research was supported by Institute for Information \& Communications Technology Planning \& Evaluation (IITP) through the Korea government (MSIT) under Grant No. 2021-0-01341 (Artificial Intelligence Graduate School Program (Chung-Ang University)).


\bibliography{acl}
\subsection{Additional details for Theorem~1}\label{app:decomp_lattice_proof}
We provide details that were omitted in the proof of Theorem~1. First, we derive the matrix $P$.
Given the planes $H$ and  $H_0:=\{x_{d+1}=0\}$, we wish  to find a plane $H_{ref}$ that is half-way (angle-wise) between $H$ and $H_0$. This would allow to reflect points in $H$ onto $H_0$ through $H_{ref}$ where the reflection is achieved using the Householder matrix $P:=I-2\hat{n}_{ref}\hat{n}^t_{ref}$, where $\hat{n}_{ref}\in \dR^{d+1}$ is the normal of $H_{ref}$~\cite{householder1958unitary}. That is, we reflect a lattice point $p\in \dR^{d+1}$ by computing the value \({p_{\text{reflected}}=P\cdot p}\).

Next, we show that the normal 
\begin{equation*}
    \hat{n}_{ref}:=\frac{1}{\sqrt{2-\frac{2}{\sqrt{d+1}}}}\cdot \left(-\tfrac{1}{\sqrt{d+1}},\dots,-\tfrac{1}{\sqrt{d+1}},1-\tfrac{1}{\sqrt{d+1}}\right)
\end{equation*}
satisfies those requirements.\footnote{We obtained the expression for $\hat{n}_{ref}$ by first considering $d=2$, where the task is more tangible, and then generalizing to higher dimensions.}  Consider the Householder matrix
\begin{align}\label{eq:reflection}
 P&=% I-2\hat{n}\hat{n}^t= 
 I-2\hat{n}_{ref}\hat{n}^t_{ref}\nonumber\\
 &=\begin{pNiceArray}{cw{c}{1cm}c|c}[margin]
            \Block{3-3}<\Large>{I_d - \frac{1}{D-\sqrt{D}}\mathds{1}} 
            & & & \dfrac{1}{\sqrt{D}} \\
            & & & \Vdots \\
            & & & \dfrac{1}{\sqrt{D}} \\
            \hline
            \dfrac{1}{\sqrt{D}} & \dots& \dfrac{1}{\sqrt{D}} & \dfrac{1}{\sqrt{D}}
        \end{pNiceArray},
    \end{align}
where $D:=d+1$, $I_d$ is an $d\times d$ identity matrix, and $\mathds{1}$ is the $d\times d$ matrix with $1$s in all its entries. 

Consider a point $p\in A^*_d$. Next, we show that it is reflected onto the plane  $H_0$, i.e., for $v=P\cdot p$, we get $v_{d+1}=0$. To do that, we move to the basis of the integer vector space, and show that for all $1\leq i\leq d$, taking the base element $e_i=(0,\dots,1,\dots,0)$, the $(d+1)$th element of $v=PG^t\cdot e_i$ (i.e., using the generator and then the reflector) is zero. First, for all $i<d$ it holds that 
    \begin{align*}
        PG^t\cdot e_i=P\cdot
        \begin{pmatrix}
        1 &  0&  \dots&  0& -1& 0 &\dots& 0
            % 1 \\
            % 0 \\
            % \vdots \\
            % 0 \\
            % -1 \\
            % 0 \\
            % \vdots \\
            % 0
        \end{pmatrix}^t.
    \end{align*}
Now, considering that the elements of the final row of $P$ are all equal to $1/\sqrt{D}$, we obtain a zero in the $(d+1)$th dimension. It remains to calculate the expression resulting from multiplying with $e_d$:
        \begin{align*}
        PG^t\cdot e_d=P\cdot
        \begin{pmatrix}
        -\frac{D-1}{D} &  \frac{1}{D}&  \dots&  \frac{1}{D}
        \end{pmatrix}^t.
    \end{align*}
    Looking specifically at the last element, we see that it is equal to     \begin{align*}
        \frac{1}{\sqrt{D}}\cdot\frac{1-D}{D} + (D-1)\frac{1}{D}\frac{1}{\sqrt{D}}=\frac{1-D+D-1}{D\sqrt{D}}=0.
    \end{align*}

    That is, by applying the transformation $P$ on the lattice points, we reflect them onto the $x_{d+1}=0$ plane. It remains to get rid of the $(d+1)$th dimension. This is accomplished by the mapping
\begin{align*}
        E=
        \begin{pmatrix}
            1 & 0 & \dots & 0 & 0 \\
            0 & 1 & \dots & 0 & 0 \\
            \vdots & \vdots & \ddots & \vdots & 0 \\
            0 & 0 & \dots & 1 & 0
        \end{pmatrix}_{d\times(d+1)}.
    \end{align*}
    
It remains to compute the  explicit embedding $T(g):=EPG^t(g)$, for $g\in \dZ^d$. We first calculate 
    \begin{align*}
        \left(EP\right)^t=
        \begin{pNiceArray}{cw{c}{1cm}c}[margin]
            \Block{3-3}<\Large>{I_d - \frac{1}{D-\sqrt{D}}\mathds{1}} 
            & &  \\
            & &  \\
            & &  \\
            \hline
            \dfrac{1}{\sqrt{D}} & \dots & \dfrac{1}{\sqrt{D}}
        \end{pNiceArray}_{d\times(d+1)}.
    \end{align*}
 Next, it can be shown that
    \begin{align}
        T^t&=G\left(EP\right)^t\nonumber\\
        &=\begin{pmatrix}
            1 & -1 &  0  & \dots & 0 \\
            1 & 0  &  -1 & \dots & 0 \\
            \vdots & \vdots  &  \vdots  & \ddots & \vdots \\
            1 & 0  &  0  & \dots & -1 \\
            \frac{1}{D - \sqrt{D}} - 1 & \frac{1}{D - \sqrt{D}} & \frac{1}{D - \sqrt{D}} & \dots & \frac{1}{D - \sqrt{D}}
        \end{pmatrix}_{d\times(d+1)}\!\!\!\!\!\!.
    \end{align}

    %      \begin{figure}[H]
    %     \centering
    %     \begin{subfigure}[b]{0.3\textwidth}
    %         \includegraphics[width=\textwidth]{Images/EPGt_visual_explanation1.png}
    %         %\caption{Sample points in $H_0$}
    %         %\label{fig:epgt_visual1}
    %     \end{subfigure}
    %     \hfill
    %     \begin{subfigure}[b]{0.3\textwidth}
    %         \includegraphics[width=\textwidth]{Images/EPGt_visual_explanation2.png}
    %         %\caption{$H_0$ rotated to the "floor"}
    %         %\label{fig:epgt_visual2}
    %     \end{subfigure}
    %     \hfill
    %     \begin{subfigure}[b]{0.3\textwidth}
    %         \includegraphics[width=\textwidth]{Images/EPGt_visual_explanation3.png}
    %         %\caption{The samples as they look in $\dR^2$}
    %         %\label{fig:epgt_visual3}
    %     \end{subfigure}
    %     \caption{Visualization of embedding the lattice  $A_2^*$ originally defined in $\dR^3$ onto $\dR^2$ via the mapping $T$. [Left] The blue rectangle represents the plane $H$, where the corresponding $A_2^*$ lattice points are drawn in red. The points are generated by taking integer vectors in $\dR^d$ and applying the mapping $G^t$.  [Center] $H$ and $A_2^*$ is reflected onto the plane $H_0=\{x_3=0\}$ using the mapping $PG^t$. [Right] The third dimension is removed, via the mapping $E$, to yield the embedding of $A_2^*$ in $\dR^2$.}
    %     \label{fig:egpt_visual}
    % \end{figure}


\subsection{Additional details for Theorem~3}\label{app:CC}
We provide details omitted from the main body of the text. 
We start with a simplified derivation of a single annulus, which would inform the more advanced construction. Fix ${0<r_1<
  r^*}$ forced it to a single line, and define $\btheta_{r'} := \frac{r'}{{\beta^*}}f_\Lambda$, and observe that 
\begin{align}
  CC_\X&\leq  r^*\cdot
\left|\X\cap (\B_{r^*}\setminus \B_{r_1})\right| + r_1\cdot
\left|\X\cap \B_{r_1}\right|\nonumber \\ & = r^*\left(|\X\cap \B_{r^*}|-|\X \cap \B_{r_1}|\right) + r_1 \left|\X\cap \B_{r_1}\right| \nonumber\\
& = r^*|\X\cap \B_{r^*}|+ (r_1-r^*) |\X\cap \B_{r_1}| \nonumber \\  
  & = r^*\frac{\partial(B_1)}{\sqrt{\det(\Lambda)}}\btheta^d_{r^*} +r^* P_d(\btheta_{r^*}) + (r_1-r^*) \frac{\partial(B_1)}{\sqrt{\det(\Lambda)}}\btheta^d_{r_1}\nonumber\\& + (r_1-r^*) P_d(\btheta_{r_1}) \nonumber
\\
& = \frac{\partial(B_1)}{\sqrt{\det(\Lambda)}}\theta^d\left({r^*}^{d+1}+{r_1}^{d+1}-r^*{r_1}^{d}\right)\nonumber\\&+ r P_d(\btheta_{r^*}) + (r_1-r^*) P_d(\btheta_{r_1})\nonumber \\ & = \frac{\partial(B_1)}{\sqrt{\det(\Lambda)}}\theta^d\left({r^*}^{d+1}+{r_1}^{d+1}-r^*{r_1}^{d}\right)+ r^* P_d(\btheta_{r^*}), \label{eq:CC1}
\end{align}
where the sample complexity bound in Equation~(5) is used. For simplicity, we bound throughout the error term with $r^* P_d(\btheta_{r^*})$.
Next, we optimize the value $r_1$ to minimize the expression in Equation~\eqref{eq:CC1}.

Consider the function $f(r_1)={r^*}^{d+1}-{r^*} r^d_1 + {r^*}^{d+1}_1$. We look for the minimum of $f(r_1)$ by requiring that
\begin{align*}
            f'(r_1)=-{r^*} dr_1^{d-1}+(d+1)r_1^d=0,
\end{align*}
which yields the value $r'_1:=\frac{d}{d+1}{r^*}$. This value is  a minimum since
\begin{align*}
 f^{(2)}(r_1)|_{r'_1}=&\left(-{r^*}(d-1)r_1^{d-2}+d(d+1)r_1^{d-1}\right)|_{r_1'}\\
        =&{r^*}^{d-1}\left(\frac{d^d}{(d+1)^{d-2}}-\frac{d^{d-2}(d-1)}{(d+1)^{d-2}}\right)\\
        =&{r^*}^{d-1}d^{d-2}\frac{d^2-d+1}{(d+1)^{d-2}},
    \end{align*}
    and we know that $d^2-d+1>0$ for all $d\geq 2$.%, then $r_1=\frac{d}{d+1}r$ indeed minimizes $f(r_1)$.
    
Now, we apply the above line of reasoning in a recursive manner by considering a sequence of $k+1\geq 2$ radii ${0<r_k<\ldots<r_0={r^*}}$ where $r_i:=\td^i r^*$, where $\td:=\frac{d}{d+1}$. This leads to the bound
\begin{align}
\label{eq:cc_eval_app}
CC_\X&\leq \sum_{i=0}^{k-1}r_i |\X\cap (\B_{r_i}\setminus \B_{r_{i+1}})| + r_k|\X\cap \B_{r_k}|\nonumber\\
  &= \frac{\partial(B_1)}{\sqrt{\det(\Lambda)}} \left(\underbrace{{r^*} \btheta^d_{r^*} + \sum_{i=1}^k(r_i-r_{i-1}) \btheta^d_{r_i}}_{:=\gamma}\right) + {r^*} P_d(\btheta_{r^*}).
\end{align}

We show that 
\[\gamma:=r \btheta^d_{r^*} + \sum_{i=1}^k(r_i-r_{i-1}) \btheta^d_{r_i}= {r^*} \btheta^d_{r^*} \left(1 - \frac{\xi^{d+2} - \xi}{ d\xi - (d+1)}\right),\]
where $r_i=\td^i {r^*},\td:=\frac{d}{d+1}, \btheta_{r_i}= r_i\frac{\btheta_{r^*}}{r^*}, k=d,$ and $\xi:=\td^d=\left(\frac{d}{d+1}\right)^d$. In particular,
\begin{align}
  \gamma &={r^*} \btheta^d_{r^*} + \sum_{i=1}^k(r_i-r_{i-1}) r_i^d\frac{\btheta^d_{r^*}}{{r^*}^d} \nonumber\\
  & = {r^*} \btheta^d_{r^*} + \sum_{i=1}^k{r^*}\td^{i-1}(\td-1) \td^{di} {r^*}^d\frac{\btheta^d_{r^*}}{{r^*}^d} \nonumber\\ 
  &=  {r^*} \btheta^d_{r^*} + \sum_{i=1}^k{r^*} (\td-1) \td^{di+ i -1} \btheta^d_{r^*} \nonumber\\ 
  &=   {r^*} \btheta^d_{r^*} \left(1 + \sum_{i=1}^k (\td-1) \td^{di+ i -1} \right)\nonumber\\
  &= {r^*} \btheta^d_{r^*} \left(1 + \frac{\td-1}{\td}\sum_{i=1}^k \td^{(d+1)i} \right)\nonumber\\
  &= {r^*} \btheta^d_{r^*} \left(1 + \frac{\td-1}{\td}\frac{\left(\td^{d+1}\right)^{k+1} - \td^{d+1}}{\td^{d+1} - 1} \right)\nonumber\\
  &= {r^*} \btheta^d_{r^*} \left(1 + \td^d(\td-1)\frac{\left(\td^{d+1}\right)^k - 1}{\td^{d+1} - 1} \right).\nonumber\\
  % & = r \btheta^d_{r} + \sum_{i=1}^kr\td^{i-1}(\td-1) \td^{di} r^d\frac{\btheta^d_{r}}{r^d} \\ &  =  r \btheta^d_{r} + \sum_{i=1}^kr (\td-1) \td^{di+ i -1} \btheta^d_{r} \\ & =   r \btheta^d_{r} \left(1 + \sum_{i=1}^k (\td-1) \td^{di+ i -1} \right)\\
\end{align}

Taking $k=d$ results in $r_k=(\frac{d}{d+1})^d {r^*}\approx\frac{1}{e}{r^*}$. 
To use the sample set analysis, we need a large enough $r$ value, so assuming the original $r$ is large enough, we can deduce safely that $\frac{r}{e}$ is also large enough. 
Notice also that $\td - 1 = \frac{-1}{d+1}$, and thus $(d+1)\td=d$, so returning to our expression, and substituting $\xi:=\td^d=\left(\frac{d}{d+1}\right)^d$, we obtain 
\begin{align*}
    \gamma&={r^*} \btheta^d_{r^*} \left(1 - \frac{\td^d\left(\td^{d(d+1)} - 1\right)}{(d+1)(\td^{d+1} - 1)}\right)\\
    &= {r^*} \btheta^d_{r^*} \left(1 - \frac{\xi\left(\xi^{d+1} - 1\right)}{(d+1)(\td \xi - 1)}\right) 
    \\&= {r^*} \btheta^d_{r^*} \left(1 - \frac{\xi^{d+2} - \xi}{ d\xi - (d+1)}\right)
    :={r^*} \btheta^d_{r^*}\zeta.
\end{align*}

We finish this section with a plot of the value $\gamma$ in Figure~\ref{fig:annuli_bound:app}.

\begin{figure}[thb]
\centering  
\includegraphics[width=0.9\columnwidth]{Images/annuli_bound.pdf}
\caption{Plot of the improvement factor $\gamma$.}
\label{fig:annuli_bound:app}
\end{figure}

\subsection{Additional experimental results}
Additional scenarios, which were omitted from the main paper, are given in Figure~\ref{fig:scenarios:app}. Extended results comparing lattice-based samples using the \loc algorithm are provided in Table~\ref{tbl:lattice_comparison:app}.

\begin{figure*}[tbh]
  \centering
%     \hspace*{-0.66cm}
% \subfloat[Zigzag-bypass (long)]{\includegraphics[width=2.18\columnwidth,clip]{Images/Scenarios/ZZB3H_scenario.png}
%     %\label{fig:3d_lattices:da}
%     }
%     \newline
\subfloat[Zigzag-bypass (short)]{\includegraphics[width=1.15\columnwidth,clip]{Images/Scenarios/ZZB2H_scenario.png}
    %\label{fig:3d_lattices:da}
    }
\subfloat[Narrow (more scenarios)]{\includegraphics[width=0.465\columnwidth,clip]{Images/Scenarios/N1_scenarios.png}
    %\label{fig:3d_lattices:da}
    }
  \caption{Additional scenarios used in the experiments. The scenario ZZB3, which is not illustrated here, is similar to ZZB2, only that the horizontal hallways are twice as long.}
  \label{fig:scenarios:app}
\end{figure*}

\begin{table}[tbh]
\caption{Extended comparison of running time and solution length using lattices-based sample sets (where the underlying lattice is denoted in the table) in the iA*-\loc algorithm. Solution length is normalized with respect to the length obtained using $\XA$. }
\centering
\label{tbl:lattice_comparison:app}
\begin{tabular}{|c||ccc|cc|}
\hline
 & \multicolumn{3}{c|}{\cellcolor[HTML]{EFEFEF} Time (s)} & \multicolumn{2}{c|}{\cellcolor[HTML]{EFEFEF} Length (r)} \\ \cline{2-6} 
\multirow{-2}{*}{\begin{tabular}[c]{@{}c@{}}Scenario\\ (robot \#)\end{tabular}} & \multicolumn{1}{c|}{\cellcolor[HTML]{FFFFC7}$\ZN$} & \multicolumn{1}{c|}{\cellcolor[HTML]{FFFFC7}$\DN$} & \cellcolor[HTML]{FFFFC7}$\AN$ & \multicolumn{1}{c|}{\cellcolor[HTML]{FFFFC7}$\ZN$} & \cellcolor[HTML]{FFFFC7}$\DN$ \\ \hline \hline
\cellcolor[HTML]{ECF4FF}N4(2) & \multicolumn{1}{c|}{0.00} & \multicolumn{1}{c|}{0.00} & 0.00 & \multicolumn{1}{c|}{0.62} & 0.74 \\
\cellcolor[HTML]{ECF4FF}N1(5) & \multicolumn{1}{c|}{165.35} & \multicolumn{1}{c|}{4.59} & 0.36 & \multicolumn{1}{c|}{0.65} & 0.79 \\
\cellcolor[HTML]{ECF4FF}N2(5) & \multicolumn{1}{c|}{62.68} & \multicolumn{1}{c|}{1.81} & 0.41 & \multicolumn{1}{c|}{0.85} & 0.95 \\
\cellcolor[HTML]{ECF4FF}N3(5) & \multicolumn{1}{c|}{142.27} & \multicolumn{1}{c|}{2.91} & 0.59 & \multicolumn{1}{c|}{0.65} & 0.87 \\
\cellcolor[HTML]{ECF4FF}N5(5) & \multicolumn{1}{c|}{dnf} & \multicolumn{1}{c|}{4.82} & 3.32 & \multicolumn{1}{c|}{dnf} & 0.82 \\
\cellcolor[HTML]{ECF4FF}N1B(6) & \multicolumn{1}{c|}{dnf} & \multicolumn{1}{c|}{328.30} & 15.08 & \multicolumn{1}{c|}{dnf} & 0.89 \\ \hline
\cellcolor[HTML]{ECF4FF}BT4(2) & \multicolumn{1}{c|}{0.04} & \multicolumn{1}{c|}{0.01} & 0.01 & \multicolumn{1}{c|}{0.69} & 0.85 \\
\cellcolor[HTML]{ECF4FF}BT10(2) & \multicolumn{1}{c|}{-} & \multicolumn{1}{c|}{1.20} & 0.30 & \multicolumn{1}{c|}{-} & 0.92 \\
\cellcolor[HTML]{ECF4FF}BT5(3) & \multicolumn{1}{c|}{0.54} & \multicolumn{1}{c|}{0.14} & 0.06 & \multicolumn{1}{c|}{0.38} & 0.51 \\
\cellcolor[HTML]{ECF4FF}BT1(4) & \multicolumn{1}{c|}{146.69} & \multicolumn{1}{c|}{50.81} & 3.51 & \multicolumn{1}{c|}{0.95} & 1.03 \\
\cellcolor[HTML]{ECF4FF}BT6(4) & \multicolumn{1}{c|}{dnf} & \multicolumn{1}{c|}{153.40} & 12.36 & \multicolumn{1}{c|}{dnf} & 1.04 \\
\cellcolor[HTML]{ECF4FF}BT7(4) & \multicolumn{1}{c|}{240.88} & \multicolumn{1}{c|}{5.38} & 4.36 & \multicolumn{1}{c|}{0.95} & 0.96 \\ \hline
\cellcolor[HTML]{ECF4FF}K1(3) & \multicolumn{1}{c|}{32.31} & \multicolumn{1}{c|}{4.97} & 1.37 & \multicolumn{1}{c|}{0.82} & 0.89 \\ \hline
\cellcolor[HTML]{ECF4FF}UM4(2) & \multicolumn{1}{c|}{-} & \multicolumn{1}{c|}{8.47} & 2.43 & \multicolumn{1}{c|}{-} & 0.90 \\
\cellcolor[HTML]{ECF4FF}UM1(3) & \multicolumn{1}{c|}{482.17} & \multicolumn{1}{c|}{25.15} & 6.68 & \multicolumn{1}{c|}{0.84} & 1.16 \\
\cellcolor[HTML]{ECF4FF}UM2(3) & \multicolumn{1}{c|}{13.35} & \multicolumn{1}{c|}{1.22} & 0.04 & \multicolumn{1}{c|}{1.04} & 1.52 \\
\cellcolor[HTML]{ECF4FF}UM4B3(3) & \multicolumn{1}{c|}{99.35} & \multicolumn{1}{c|}{1.03} & 0.66 & \multicolumn{1}{c|}{1.59} & 0.89 \\
\cellcolor[HTML]{ECF4FF}UM3(4) & \multicolumn{1}{c|}{236.31} & \multicolumn{1}{c|}{223.87} & 64.57 & \multicolumn{1}{c|}{0.63} & 0.97 \\ \hline
\cellcolor[HTML]{ECF4FF}ZZB1(2) & \multicolumn{1}{c|}{1.93} & \multicolumn{1}{c|}{1.01} & 0.44 & \multicolumn{1}{c|}{0.94} & 0.94 \\
\cellcolor[HTML]{ECF4FF}ZZB2(2) & \multicolumn{1}{c|}{2.91} & \multicolumn{1}{c|}{0.93} & 0.71 & \multicolumn{1}{c|}{0.94} & 0.94 \\
\cellcolor[HTML]{ECF4FF}ZZB3(2) & \multicolumn{1}{c|}{2.26} & \multicolumn{1}{c|}{0.84} & 0.47 & \multicolumn{1}{c|}{0.95} & 0.95 \\ \hline\end{tabular}
\end{table}

\begin{table*}[tbh]
\centering
\begin{tabular}{|c|cccl|ccl|cl|cl|}
\hline
 & \multicolumn{4}{c|}{\cellcolor[HTML]{EFEFEF} Total time (s)} & \multicolumn{3}{c|}{\cellcolor[HTML]{EFEFEF} Search time (s)} & \multicolumn{2}{c|}{\cellcolor[HTML]{EFEFEF}Length (r)} & \multicolumn{2}{c|}{\cellcolor[HTML]{EFEFEF}Success (\%)} \\ \cline{2-12} 
\multirow{-2}{*}{\begin{tabular}[c]{@{}c@{}}Scenario\\ (Robot \#)\end{tabular}} & \multicolumn{1}{c|}{\cellcolor[HTML]{FFFFC7}\begin{tabular}[c]{@{}c@{}}$\AN$\\ \loc\end{tabular}} & \multicolumn{1}{c|}{\cellcolor[HTML]{FFFFC7}\begin{tabular}[c]{@{}c@{}}$\AN$\\ \glo\end{tabular}} & \multicolumn{1}{c|}{\cellcolor[HTML]{FFFFC7}\begin{tabular}[c]{@{}c@{}}\rnd\\ \glo\end{tabular}} & \multicolumn{1}{c|}{\cellcolor[HTML]{FFFFC7}\begin{tabular}[c]{@{}c@{}}\rndm\\ \glo\end{tabular}} & \multicolumn{1}{c|}{\cellcolor[HTML]{FFFFC7}\begin{tabular}[c]{@{}c@{}}$\AN$\\ \glo\end{tabular}} & \multicolumn{1}{c|}{\cellcolor[HTML]{FFFFC7}\begin{tabular}[c]{@{}c@{}}\rnd\\ \glo\end{tabular}} & \multicolumn{1}{c|}{\cellcolor[HTML]{FFFFC7}\begin{tabular}[c]{@{}c@{}}\rndm\\ \glo\end{tabular}} & \multicolumn{1}{c|}{\cellcolor[HTML]{FFFFC7}\begin{tabular}[c]{@{}c@{}}\rnd\\ \glo\end{tabular}} & \multicolumn{1}{c|}{\cellcolor[HTML]{FFFFC7}\begin{tabular}[c]{@{}c@{}}\rndm\\ \glo\end{tabular}} & \multicolumn{1}{c|}{\cellcolor[HTML]{FFFFC7}\begin{tabular}[c]{@{}c@{}}\rnd\\ \glo\end{tabular}} & \multicolumn{1}{c|}{\cellcolor[HTML]{FFFFC7}\begin{tabular}[c]{@{}c@{}}\rndm\\ \glo\end{tabular}} \\ \hline
\cellcolor[HTML]{ECF4FF}N1(5) & \multicolumn{1}{c|}{0.36} & \multicolumn{1}{c|}{3.05} & \multicolumn{1}{c|}{4.16} & 3.40 & \multicolumn{1}{c|}{0.84} & \multicolumn{1}{c|}{3.37} & 2.59 & \multicolumn{1}{c|}{1.48} & 1.46 & \multicolumn{1}{c|}{80.00} & 90 \\
\cellcolor[HTML]{ECF4FF}N2(5) & \multicolumn{1}{c|}{0.41} & \multicolumn{1}{c|}{2.67} & \multicolumn{1}{c|}{2.74} & 4.28 & \multicolumn{1}{c|}{0.82} & \multicolumn{1}{c|}{2.11} & 3.62 & \multicolumn{1}{c|}{2.43} & 3.31 & \multicolumn{1}{c|}{65.00} & 95 \\
\cellcolor[HTML]{ECF4FF}N3(5) & \multicolumn{1}{c|}{0.59} & \multicolumn{1}{c|}{3.83} & \multicolumn{1}{c|}{5.44} & 4.22 & \multicolumn{1}{c|}{1.72} & \multicolumn{1}{c|}{4.65} & 3.39 & \multicolumn{1}{c|}{2.02} & 1.56 & \multicolumn{1}{c|}{85.00} & 85 \\
\cellcolor[HTML]{ECF4FF}N5(5) & \multicolumn{1}{c|}{3.32} & \multicolumn{1}{c|}{31.48} & \multicolumn{1}{c|}{23.42} & 26.19 & \multicolumn{1}{c|}{20.02} & \multicolumn{1}{c|}{18.62} & 21.14 & \multicolumn{1}{c|}{0.89} & 0.88 & \multicolumn{1}{c|}{100.00} & 100 \\ \hline
\cellcolor[HTML]{ECF4FF}BT9(2) & \multicolumn{1}{c|}{0.13} & \multicolumn{1}{c|}{0.13} & \multicolumn{1}{c|}{0.77} & 0.42 & \multicolumn{1}{c|}{0.13} & \multicolumn{1}{c|}{0.77} & 0.42 & \multicolumn{1}{c|}{1.10} & 1.41 & \multicolumn{1}{c|}{95.00} & 40 \\
\cellcolor[HTML]{ECF4FF}BT10(2) & \multicolumn{1}{c|}{0.30} & \multicolumn{1}{c|}{0.31} & \multicolumn{1}{c|}{1.16} & 0.46 & \multicolumn{1}{c|}{0.31} & \multicolumn{1}{c|}{1.16} & 0.46 & \multicolumn{1}{c|}{1.13} & 1.27 & \multicolumn{1}{c|}{95.00} & 75 \\
\cellcolor[HTML]{ECF4FF}BT1B(3) & \multicolumn{1}{c|}{34.83} & \multicolumn{1}{c|}{47.58} & \multicolumn{1}{c|}{118.27} & 62.88 & \multicolumn{1}{c|}{47.27} & \multicolumn{1}{c|}{118.11} & 62.70 & \multicolumn{1}{c|}{0.93} & 0.99 & \multicolumn{1}{c|}{100.00} & 100 \\
\cellcolor[HTML]{ECF4FF}BT2(3) & \multicolumn{1}{c|}{5.62} & \multicolumn{1}{c|}{7.08} & \multicolumn{1}{c|}{22.67} & 28.17 & \multicolumn{1}{c|}{6.97} & \multicolumn{1}{c|}{22.61} & 28.11 & \multicolumn{1}{c|}{0.93} & 1.12 & \multicolumn{1}{c|}{100.00} & 95 \\
\cellcolor[HTML]{ECF4FF}BT2B(3) & \multicolumn{1}{c|}{9.58} & \multicolumn{1}{c|}{14.40} & \multicolumn{1}{c|}{41.67} & 21.23 & \multicolumn{1}{c|}{14.13} & \multicolumn{1}{c|}{4.36} & 21.09 & \multicolumn{1}{c|}{1.00} & 1.05 & \multicolumn{1}{c|}{100.00} & 95 \\
\cellcolor[HTML]{ECF4FF}BT3(3) & \multicolumn{1}{c|}{5.38} & \multicolumn{1}{c|}{14.15} & \multicolumn{1}{c|}{62.22} & 32.10 & \multicolumn{1}{c|}{12.80} & \multicolumn{1}{c|}{61.54} & 31.39 & \multicolumn{1}{c|}{1.05} & 1.11 & \multicolumn{1}{c|}{100.00} & 100 \\
\cellcolor[HTML]{ECF4FF}BT5(3) & \multicolumn{1}{c|}{0.06} & \multicolumn{1}{c|}{0.38} & \multicolumn{1}{c|}{0.27} & 0.19 & \multicolumn{1}{c|}{0.12} & \multicolumn{1}{c|}{0.14} & 0.05 & \multicolumn{1}{c|}{0.57} & 0.57 & \multicolumn{1}{c|}{100.00} & 85 \\
\cellcolor[HTML]{ECF4FF}BT8(3) & \multicolumn{1}{c|}{12.17} & \multicolumn{1}{c|}{19.31} & \multicolumn{1}{c|}{169.32} & 79.52 & \multicolumn{1}{c|}{18.87} & \multicolumn{1}{c|}{169.12} & 79.31 & \multicolumn{1}{c|}{1.00} & 1.05 & \multicolumn{1}{c|}{100.00} & 100 \\
\cellcolor[HTML]{ECF4FF}BT8B(3) & \multicolumn{1}{c|}{3.17} & \multicolumn{1}{c|}{3.60} & \multicolumn{1}{c|}{41.63} & 24.23 & \multicolumn{1}{c|}{3.55} & \multicolumn{1}{c|}{41.60} & 24.20 & \multicolumn{1}{c|}{1.16} & 1.20 & \multicolumn{1}{c|}{100.00} & 100 \\
\cellcolor[HTML]{ECF4FF}BT11(3) & \multicolumn{1}{c|}{17.21} & \multicolumn{1}{c|}{35.21} & \multicolumn{1}{c|}{51.19} & 31.23 & \multicolumn{1}{c|}{34.34} & \multicolumn{1}{c|}{50.77} & 30.80 & \multicolumn{1}{c|}{0.88} & 0.95 & \multicolumn{1}{c|}{100.00} & 100 \\
\cellcolor[HTML]{ECF4FF}BT1(4) & \multicolumn{1}{c|}{3.51} & \multicolumn{1}{c|}{97.33} & \multicolumn{1}{c|}{63.69} & 68.39 & \multicolumn{1}{c|}{13.88} & \multicolumn{1}{c|}{18.83} & 22.18 & \multicolumn{1}{c|}{1.02} & 1.04 & \multicolumn{1}{c|}{100.00} & 100 \\
\cellcolor[HTML]{ECF4FF}BT6(4) & \multicolumn{1}{c|}{12.36} & \multicolumn{1}{c|}{124.16} & \multicolumn{1}{c|}{106.73} & 95.75 & \multicolumn{1}{c|}{43.87} & \multicolumn{1}{c|}{61.96} & 50.18 & \multicolumn{1}{c|}{1.01} & 1.03 & \multicolumn{1}{c|}{100.00} & 100 \\
\cellcolor[HTML]{ECF4FF}BT7(4) & \multicolumn{1}{c|}{4.36} & \multicolumn{1}{c|}{95.89} & \multicolumn{1}{c|}{60.65} & 56.06 & \multicolumn{1}{c|}{15.06} & \multicolumn{1}{c|}{15.24} & 9.29 & \multicolumn{1}{c|}{1.00} & 1.01 & \multicolumn{1}{c|}{100.00} & 100 \\ \hline
\cellcolor[HTML]{ECF4FF}UM4(2) & \multicolumn{1}{c|}{2.43} & \multicolumn{1}{c|}{2.93} & \multicolumn{1}{c|}{12.71} & 2.26 & \multicolumn{1}{c|}{2.90} & \multicolumn{1}{c|}{12.69} & 2.24 & \multicolumn{1}{c|}{0.96} & 1.41 & \multicolumn{1}{c|}{70.00} & 5 \\
\cellcolor[HTML]{ECF4FF}UM4B1(2) & \multicolumn{1}{c|}{4.81} & \multicolumn{1}{c|}{5.68} & \multicolumn{1}{c|}{17.38} & 5.03 & \multicolumn{1}{c|}{5.64} & \multicolumn{1}{c|}{17.35} & 5.01 & \multicolumn{1}{c|}{0.86} & 1.12 & \multicolumn{1}{c|}{90.00} & 45 \\
\cellcolor[HTML]{ECF4FF}UM1(3) & \multicolumn{1}{c|}{6.68} & \multicolumn{1}{c|}{58.62} & \multicolumn{1}{c|}{49.35} & 31.92 & \multicolumn{1}{c|}{47.14} & \multicolumn{1}{c|}{42.58} & 24.86 & \multicolumn{1}{c|}{0.98} & 1.09 & \multicolumn{1}{c|}{100.00} & 100 \\
\cellcolor[HTML]{ECF4FF}UM2(3) & \multicolumn{1}{c|}{0.04} & \multicolumn{1}{c|}{2.94} & \multicolumn{1}{c|}{4.49} & 3.44 & \multicolumn{1}{c|}{0.21} & \multicolumn{1}{c|}{2.97} & 1.83 & \multicolumn{1}{c|}{1.95} & 2.23 & \multicolumn{1}{c|}{75.00} & 40 \\
\cellcolor[HTML]{ECF4FF}UM5(3) & \multicolumn{1}{c|}{2.87} & \multicolumn{1}{c|}{31.79} & \multicolumn{1}{c|}{29.71} & 21.31 & \multicolumn{1}{c|}{19.01} & \multicolumn{1}{c|}{22.18} & 13.63 & \multicolumn{1}{c|}{1.05} & 1.26 & \multicolumn{1}{c|}{95.00} & 85 \\ \hline
\cellcolor[HTML]{ECF4FF}ZZB1(2) & \multicolumn{1}{c|}{0.44} & \multicolumn{1}{c|}{0.49} & \multicolumn{1}{c|}{10.44} & 0.47 & \multicolumn{1}{c|}{0.48} & \multicolumn{1}{c|}{10.43} & 0.45 & \multicolumn{1}{c|}{0.89} & 1.01 & \multicolumn{1}{c|}{100.00} & 5 \\
\cellcolor[HTML]{ECF4FF}ZZB2(2) & \multicolumn{1}{c|}{0.71} & \multicolumn{1}{c|}{2.51} & \multicolumn{1}{c|}{272.70} & 2.66 & \multicolumn{1}{c|}{1.22} & \multicolumn{1}{c|}{271.96} & 1.83 & \multicolumn{1}{c|}{0.89} & 2.62 & \multicolumn{1}{c|}{100.00} & 65 \\
\cellcolor[HTML]{ECF4FF}ZZB3(2) & \multicolumn{1}{c|}{0.47} & \multicolumn{1}{c|}{7.69} & \multicolumn{1}{c|}{341.84} & 5.33 & \multicolumn{1}{c|}{1.18} & \multicolumn{1}{c|}{338.15} & 1.45 & \multicolumn{1}{c|}{0.88} & 2.58 & \multicolumn{1}{c|}{100.00} & 65 \\ \hline
\end{tabular}
\caption{Comparison of running time and solution length between $\XA$ (using \loc and \glo) and uniform random sampling. For random sampling we report the average values in terms of running and solution length (the latter is given as normalized value with  respect to the solution length with $\XA$). }
\label{tbl:lattice_vs_random:app}
\end{table*}

\begin{figure*}[th]
  \centering
%     \hspace*{-0.66cm}
% \subfloat[Zigzag-bypass (long)]{\includegraphics[width=2.18\columnwidth,clip]{Images/Scenarios/ZZB3H_scenario.png}
%     %\label{fig:3d_lattices:da}
%     }
%     \newline
\subfloat{\includegraphics[width=\columnwidth,clip]{Images/tuning1.pdf}
    }
    \subfloat{\includegraphics[width=\columnwidth,clip]{Images/tuning2.pdf}
    }
    \newline
\subfloat{\includegraphics[width=\columnwidth,clip]{Images/tuning3.pdf}
    }
\subfloat{\includegraphics[width=\columnwidth,clip]{Images/tuning4.pdf}
    }
      \caption{Effect of the parameters $\delta,\epsilon$ on the performance of \loc with $\XA$ for $\delta=2.5$ (left) and $\delta=4$ (right). We report the running time (top) and solution length (bottom). The absence of data points for the parameters $\delta=4, \eps\in \{2,4,5\}$ indicates a solution failure.  
  }
  \label{fig:parameters:app}
\end{figure*}

\subsection{Comparison with Random Sampling}
Extended results where $\XA$-samples are compared with \rnd are given in Table~\ref{tbl:lattice_vs_random:app}. Here, we consider two versions of random sampling. The first version, denoted by \rnd, which is identical to the one considered in the main paper, uses random sampling together with the asymptotically optimal connection radius $r_{\textup{rnd}}(n)$, which is commonly used in practice. The second version, denoted by \rndm uses the radius as ${r^*}$ used for lattice-based sampling. The latter is used to further emphasize the inferiority of uniform random sampling as compared to $\XA$ due to identical parameters between $\XA$-\glo and \rndm-\glo (except for the sampling distribution). In particular, the move to \rndm  severely reduces the success rates in some of the scenarios.

Another addition in Table~\ref{tbl:lattice_vs_random:app} is the running time of the search algorithm (under "search time"). Recall that the total running time for \glo consists of the (i) construction of the sample set and the nearest-neighbor data structure and the (ii) running the search algorithm. Although both $\XA$-\glo and \rnd use the same number of samples, the construction time is usually larger in the former due to an additional step of constructing the lattice samples over the entire configuration space, which is currently implemented in a naive and unoptimized manner. In this sense, the comparison between $\XA$-\glo and \rnd is not entirely fair. Thus, we also report the running time of the search algorithm, which can be the computational bottleneck, especially for more complicated robot geometries where the collision-check operation is more expensive~\cite{KleinbortSH16}. Although the search time for $\XA$-\glo is usually lower for most scenarios, we argue that with more expensive collision checks, the advantage of lattice-based sample sets would be even more prominent.

\subsection{Effect of parameter choice}
We report the effect of the choice of the $\delta$ and $\eps$ parameters on solution length and running time for the \loc algorithm using $\XA$ sampling. We specifically focus on the ZZB3 scenario due to the availability of several homotopy classes for the solution, where each class has a different length and level of difficulty. For instance, in one class, the robots use the rightmost part of the workspace, which consists of a winding path, and exchange positions halfway between---leading to a relatively short solution length. In a second class, the robots use the long passage to the left, which consists of long straight-line motions and yields a significantly longer solution length.


We set $\delta\in \{2.75,4\}$ and report the solution length and running time in Figure~\ref{fig:parameters:app} for $\eps\in \{0.5,0.6,\ldots,1,2,\ldots,10\}$. Observe that for $\delta=2.75$ the planner obtains a low-length solution already for high $\eps$ values, whereas $\delta=4$ initially uncovers an inefficient solution length-wise but eventually settles on the better homotopy class when $\eps$ is reduced. From values of $\eps\leq 1$ the length relatively stabilizes, while the runtime jumps at several orders of magnitude, which highlights the exponential dependence of sample and collision-check complexity on the value $\eps$. Finding the middle-ground $\eps$ value is an important goal, which we leave for future work. 

Notice that the planner fails to find a solution for $\delta=4$ and $\eps\in\{2,4,5\}$. Due to our \decomps result, this implies that no $4$-clear solution exists. Despite this, the planner does succeed for some values of $\eps$, which suggests that our sufficient conditions for \decomps are not necessary. The success could also be explained by the specific arrangement of the points in $\XA$, which coincidentally induces a connected component via the second homotopy class for this specific scenario. It should also be noted that the sample set $\X_{\AN}^{4,\eps}$ can be viewed (via Lemma~1) as the sample set $\X_{\AN}^{2.5,\eps'}$ for $\eps$ small enough, which explains the success of the planner with  $\delta=4$ and smaller $\eps$ values. 

% Please add the following required packages to your document preamble:
% \usepackage{multirow}
% \usepackage[table,xcdraw]{xcolor}
% Beamer presentation requires \usepackage{colortbl} instead of \usepackage[table,xcdraw]{xcolor}



\end{document}
