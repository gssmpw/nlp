
\pdfoutput=1


\documentclass[11pt]{article}


\usepackage{acl}
\usepackage{algorithm}
\usepackage{algpseudocode}
\usepackage{geometry}
\usepackage[utf8]{inputenc}
\usepackage{tcolorbox}
\usepackage{times}
\usepackage{latexsym}
\usepackage[T1]{fontenc}
\usepackage[utf8]{inputenc}
\usepackage{microtype}
\usepackage{inconsolata}
\usepackage{graphicx}
\usepackage{subcaption}
\usepackage{bm}
\usepackage{comment}
\usepackage{enumitem}
\usepackage{amsmath}
\usepackage{amssymb}
\usepackage{makecell}
\usepackage{tabularx}



\title{Which Retain Set Matters for LLM Unlearning?

A Case Study on Entity Unlearning}

\author{Hwan Chang \and Hwanhee Lee\thanks{Corresponding author.} \\
    Department of Artificial Intelligence, Chung-Ang University, Seoul, Korea\\
    \texttt{\{hwanchang, hwanheelee\}@cau.ac.kr}
}
\begin{document}
\maketitle
\begin{abstract}


The choice of representation for geographic location significantly impacts the accuracy of models for a broad range of geospatial tasks, including fine-grained species classification, population density estimation, and biome classification. Recent works like SatCLIP and GeoCLIP learn such representations by contrastively aligning geolocation with co-located images. While these methods work exceptionally well, in this paper, we posit that the current training strategies fail to fully capture the important visual features. We provide an information theoretic perspective on why the resulting embeddings from these methods discard crucial visual information that is important for many downstream tasks. To solve this problem, we propose a novel retrieval-augmented strategy called RANGE. We build our method on the intuition that the visual features of a location can be estimated by combining the visual features from multiple similar-looking locations. We evaluate our method across a wide variety of tasks. Our results show that RANGE outperforms the existing state-of-the-art models with significant margins in most tasks. We show gains of up to 13.1\% on classification tasks and 0.145 $R^2$ on regression tasks. All our code and models will be made available at: \href{https://github.com/mvrl/RANGE}{https://github.com/mvrl/RANGE}.

\end{abstract}


\section{Introduction}

Video generation has garnered significant attention owing to its transformative potential across a wide range of applications, such media content creation~\citep{polyak2024movie}, advertising~\citep{zhang2024virbo,bacher2021advert}, video games~\citep{yang2024playable,valevski2024diffusion, oasis2024}, and world model simulators~\citep{ha2018world, videoworldsimulators2024, agarwal2025cosmos}. Benefiting from advanced generative algorithms~\citep{goodfellow2014generative, ho2020denoising, liu2023flow, lipman2023flow}, scalable model architectures~\citep{vaswani2017attention, peebles2023scalable}, vast amounts of internet-sourced data~\citep{chen2024panda, nan2024openvid, ju2024miradata}, and ongoing expansion of computing capabilities~\citep{nvidia2022h100, nvidia2023dgxgh200, nvidia2024h200nvl}, remarkable advancements have been achieved in the field of video generation~\citep{ho2022video, ho2022imagen, singer2023makeavideo, blattmann2023align, videoworldsimulators2024, kuaishou2024klingai, yang2024cogvideox, jin2024pyramidal, polyak2024movie, kong2024hunyuanvideo, ji2024prompt}.


In this work, we present \textbf{\ours}, a family of rectified flow~\citep{lipman2023flow, liu2023flow} transformer models designed for joint image and video generation, establishing a pathway toward industry-grade performance. This report centers on four key components: data curation, model architecture design, flow formulation, and training infrastructure optimization—each rigorously refined to meet the demands of high-quality, large-scale video generation.


\begin{figure}[ht]
    \centering
    \begin{subfigure}[b]{0.82\linewidth}
        \centering
        \includegraphics[width=\linewidth]{figures/t2i_1024.pdf}
        \caption{Text-to-Image Samples}\label{fig:main-demo-t2i}
    \end{subfigure}
    \vfill
    \begin{subfigure}[b]{0.82\linewidth}
        \centering
        \includegraphics[width=\linewidth]{figures/t2v_samples.pdf}
        \caption{Text-to-Video Samples}\label{fig:main-demo-t2v}
    \end{subfigure}
\caption{\textbf{Generated samples from \ours.} Key components are highlighted in \textcolor{red}{\textbf{RED}}.}\label{fig:main-demo}
\end{figure}


First, we present a comprehensive data processing pipeline designed to construct large-scale, high-quality image and video-text datasets. The pipeline integrates multiple advanced techniques, including video and image filtering based on aesthetic scores, OCR-driven content analysis, and subjective evaluations, to ensure exceptional visual and contextual quality. Furthermore, we employ multimodal large language models~(MLLMs)~\citep{yuan2025tarsier2} to generate dense and contextually aligned captions, which are subsequently refined using an additional large language model~(LLM)~\citep{yang2024qwen2} to enhance their accuracy, fluency, and descriptive richness. As a result, we have curated a robust training dataset comprising approximately 36M video-text pairs and 160M image-text pairs, which are proven sufficient for training industry-level generative models.

Secondly, we take a pioneering step by applying rectified flow formulation~\citep{lipman2023flow} for joint image and video generation, implemented through the \ours model family, which comprises Transformer architectures with 2B and 8B parameters. At its core, the \ours framework employs a 3D joint image-video variational autoencoder (VAE) to compress image and video inputs into a shared latent space, facilitating unified representation. This shared latent space is coupled with a full-attention~\citep{vaswani2017attention} mechanism, enabling seamless joint training of image and video. This architecture delivers high-quality, coherent outputs across both images and videos, establishing a unified framework for visual generation tasks.


Furthermore, to support the training of \ours at scale, we have developed a robust infrastructure tailored for large-scale model training. Our approach incorporates advanced parallelism strategies~\citep{jacobs2023deepspeed, pytorch_fsdp} to manage memory efficiently during long-context training. Additionally, we employ ByteCheckpoint~\citep{wan2024bytecheckpoint} for high-performance checkpointing and integrate fault-tolerant mechanisms from MegaScale~\citep{jiang2024megascale} to ensure stability and scalability across large GPU clusters. These optimizations enable \ours to handle the computational and data challenges of generative modeling with exceptional efficiency and reliability.


We evaluate \ours on both text-to-image and text-to-video benchmarks to highlight its competitive advantages. For text-to-image generation, \ours-T2I demonstrates strong performance across multiple benchmarks, including T2I-CompBench~\citep{huang2023t2i-compbench}, GenEval~\citep{ghosh2024geneval}, and DPG-Bench~\citep{hu2024ella_dbgbench}, excelling in both visual quality and text-image alignment. In text-to-video benchmarks, \ours-T2V achieves state-of-the-art performance on the UCF-101~\citep{ucf101} zero-shot generation task. Additionally, \ours-T2V attains an impressive score of \textbf{84.85} on VBench~\citep{huang2024vbench}, securing the top position on the leaderboard (as of 2025-01-25) and surpassing several leading commercial text-to-video models. Qualitative results, illustrated in \Cref{fig:main-demo}, further demonstrate the superior quality of the generated media samples. These findings underscore \ours's effectiveness in multi-modal generation and its potential as a high-performing solution for both research and commercial applications.
\section{Preliminaries}
\label{sec:preliminaries}

\subsection{Supervised Binary Classification}
In many real-world tasks, one commonly encounters binary classification problems, in which an input $x \in \mathbb{R}^d$ is presented, and its label $y \in \{\pm 1\}$ needs to be predicted. Each sample is assumed to be independently and identically drawn from an unknown joint distribution $p(x,y)$. Let $\pi_{+} = p(y=+1)$ be the prior probability of the positive class (positive prior), and define
\begin{align*}
p_{\mathrm{p}}(x) = p(x \mid y=+1)
\text{, }
p_{\mathrm{n}}(x) = p(x \mid y=-1).
\end{align*}
Then, the marginal distribution of $x$ is given by
\begin{align*}
p(x)
= \pi_{+}p_{\mathrm{p}}(x) + (1-\pi_{+})p_{\mathrm{n}}(x).
\end{align*}

A classifier $g: \mathbb{R}^d \to \mathbb{R}$ outputs a real-valued score, whose sign determines the predicted label. For instance, a neural network can serve as $g$. A loss function $\ell:\mathbb{R} \times \{\pm 1\} \to [0,\infty)$ then measures how much the prediction disagrees with the true label. Let $R^+_p(g) = \mathbb{E}_{x \sim p_p} [\ell(g(x), +1)]$ denote the loss for true positive data, and $R^-_n(g) = \mathbb{E}_{x \sim p_n} [\ell(g(x), -1)]$ denote the loss for the true negative data. Then, the true risk is expressed as
\begin{align}
    R_{\mathrm{pn}}(g) =& \mathbb{E}_{(x,y)\sim p}[\ell(g(x),y)] \notag \\
    =& \pi_{+}R^+_p + (1 - \pi_{+})R^-_n \label{eq:risk} 
\end{align}

In supervised learning, positive dataset $\mathcal{C}_p = \{x^p_m\}_{m=1}^{m_p} \sim p_p(x)$ and negative dataset $\mathcal{C}_n = \{ x^n_m \}_{m=1}^{m_n} \sim p_n(x)$ are accessible. Replacing the expectations in \eqref{eq:risk} with sample mean, one obtains the empirical risk, and $g$ is trained to minimize it.


It is well known that having sufficient positive and negative samples typically allows one to train a highly accurate classifier for many tasks. However, in practice, obtaining large-scale positive and negative datasets with annotations is often challenging, especially in specialized domains where annotation costs become a significant obstacle.

\subsection{Unlabeled-Unlabeled (UU) Learning}
\label{subsec:uu}
UU learning~\citep{Lu2019-sd} is a technique that allows training a classifier without fully labeled positive and negative datasets, leveraging two unlabeled datasets with different class priors.

Concretely, suppose unlabeled corpora, $\widetilde{\mathcal{C}}_p = \{\widetilde{x}^p_m\}_{m=1}^{m_p}$ and $\widetilde{\mathcal{C}}_n = \{\widetilde{x}^n_m\}_{m=1}^{m_n}$, drawn from different mixture distributions. We denote $\theta_p = p(y=+1 \mid \widetilde{x}\in \widetilde{\mathcal{C}}_p)$ and $\theta_n = p(y=+1 \mid \widetilde{x}\in \widetilde{\mathcal{C}}_n)$ the \emph{positive prior} of these unlabeled corpora. In other words, $\theta_p$ is the fraction of true positives in $\widetilde{\mathcal{C}}_p$, and $\theta_n$ is the fraction of true positives in $\widetilde{\mathcal{C}}_n$. Then, the mixture distribution of each corpus is given as
\begin{align*}
    \widetilde{p}_{p}(x) &= \theta_p\, p_{p}(x) \;+\; \bigl(1 - \theta_p\bigr)\, p_{n}(x) \\
    \widetilde{p}_{_n}(x) &= \theta_n\, p_{p}(x) \;+\; \bigl(1 - \theta_n\bigr)\, p_{n}(x).        
\end{align*}

When $\theta_p > \theta_n$, we can treat $\widetilde{\mathcal{C}}_p$ as a pseudo-positive corpus (due to its larger proportion of actual positives) and $\widetilde{\mathcal{C}}_n$ as a pseudo-negative corpus (having a smaller proportion of actual positives). 

By appropriately combining these two unlabeled sets, one can construct an unbiased estimate of the true binary classification risk~\eqref{eq:risk}. Specifically, let $R_{\tilde{p}}^{\pm}(g)=\mathbb{E}_{x\sim \widetilde{p}_p}[\ell(g(x),\pm 1)]$, and $R_{\tilde{n}}^{\pm} (g)=\mathbb{E}_{x\sim \widetilde{p}_n}[\ell(g(x),\pm 1)]$. Then, the UU learning risk is given by
\begin{align}
    &R_{\mathrm{uu}}(g)\label{eq:uu}
    \\
    &\hspace{1em}= a R_{\tilde{p}}^+(g) - b R_{\tilde{p}}^-(g) - c R_{\tilde{n}}^+(g) + d R_{\tilde{n}}^-(g),\notag
\end{align}
where the coefficients $a$, $b$, $c$, $d$ are computed from $\pi_+$, $\theta_p$, and $\theta_n$ as $a = \frac{(1-\theta_n)\,\pi_+}{\theta_p - \theta_n}$, $b = \frac{\theta_n\,(1-\pi_+)}{\theta_p - \theta_n}$, $c = \frac{(1-\theta_p)\,\pi_+}{\theta_p - \theta_n}$, $d = \frac{\theta_p\,(1-\pi_+)}{\theta_p - \theta_n}$. When $\theta_p = 1$ and $\theta_n = 0$, that is, when using the same dataset as standard supervised learning, equation~\eqref{eq:uu} reduces to the standard supervised learning risk equation~\eqref{eq:risk}. In other words, supervised learning can be considered a special case of UU learning.

\subsection{Robust UU Learning}
\label{subsec:ruu}
While UU learning \eqref{eq:uu} does allow model training without explicit positive/negative labels, comparing the original binary classification risk \eqref{eq:risk}—which remains nonnegative—against the UU risk \eqref{eq:uu} shows the UU risk includes negative terms such as $-b R_{\tilde{p}}^-(g)$ and $-c R_{\tilde{n}}^+(g)$. It has been observed that these negative risk terms can lead to overfitting~\citep{Lu2020-dx}.

To mitigate this, \emph{Robust UU Learning} introduces a generalized Leaky ReLU function $f$ to moderate the reduction of negative risk. Concretely, it normalizes each term of the loss function as~\citep{Lu2020-dx}
\begin{align}
    R_{\mathrm{ruu}}(g)
    &= f\left(a R_{\tilde{p}}^+(g) - c R_{\tilde{n}}^+(g) \right) \notag \\
    &\hspace{2em}+ f\left(d R_{\tilde{n}}^-(g) - b R_{\tilde{p}}^-(g)\right) \label{eq:ruu}
\end{align}
where each bracketed term resembles a “normalized” risk under the hypothetical label of being positive or negative, respectively. The function $f$ is given by
\begin{align}
    f(x) =
    \begin{cases}
    x & \text{if } x > 0 \\
    \lambda x & \text{if } x < 0
    \end{cases}
    \quad (\lambda < 0).
    \label{eq:relu}
\end{align}

Intuitively, $f$ leaves the risk value unchanged when the risk is positive, but for negative risk, it uses $\lambda < 0$ to convert it into a positive quantity, thus mitigating the overfitting by negative risk.


\section{Dataset Construction}
We consider two scenarios for entity unlearning: the fictitious author scenario (TOFU) and a real-world scenario involving actual individuals.  This section details the construction of the forget set and the various neighbor sets for each scenario.

\subsection{Target Entity Selection}

For the real-world scenario, we first select 10 prominent figures across professions: actors, singers, politicians, and business leaders, etc.  These individuals are chosen based on their public visibility and the availability of information about them~\citep{rwku, choi2024opt}. In the TOFU scenario, we follow the method outlined in~\citet{maini2024tofu}, employing a 1\% forget ratio to determine the number of fictitious authors to be included in the forget set.

\subsection{Neighboring Entity Selection}
\label{sec:neighborentityselection}
The selection process for each type of neighbor set varies depending on the specific criteria for each.

\paragraph{Domain Neighbor Set.} For the real-world scenario, domain neighbor entities are constructed by selecting individuals within the same professional domain as the target entities following~\citet{closerlookat, liu2024revisitingharrypotter}. In the TOFU scenario, the domain neighbors provided in~\citet{maini2024tofu} are used.
\paragraph{Entity Neighbor Set.}  For the real-world scenario, entity neighbor entities are selected based on the following criteria adapted from~\citet{choi2024opt, rwku}: 1) a bidirectional relationship exists between the target entity and the potential neighbor, meaning both entities link to each other via hyperlinks on their respective Wikipedia pages and are mentioned at least once on those pages; and 2) the neighboring pages all represent people.  These criteria aim to identify entities closely associated with the target entities, reflecting real-world relationships and connections. For the TOFU scenario, given its fictitious nature, and the absence of a defined entity neighbor concept in~\citet{maini2024tofu}, entity neighbors are not applicable.

\paragraph{Syntactically Similar Neighbor Set.} Unlike the other neighbor sets, which are based on entities, the syntactically similar neighbor set is constructed using questions in $\mathcal{D}_f$. This set consists of questions in the retain set that share a similar syntactic structure with those in the $\mathcal{D}_f$. To construct this set, we first compute the pairwise Levenshtein similarity, as defined in equation~\ref{eq:LevenshteinSimilarity}, between all questions in $\mathcal{D}_f$. Then, we group questions ensuring that each question within a cluster is syntactically similar to the others in that cluster.

\subsection{Generating QA Pairs}
Based on the selected entities, we generate QA pairs that capture key information about each entity.
\paragraph{Real-world Scenario.}
We utilize Wikipedia as a knowledge source following~\citet{rwku}.

For the forget set, domain neighbor set, and entity neighbor set, we employ GPT-4o to generate QA pairs for each entity.  We first gather relevant passages from Wikipedia pages corresponding to each target entity.  These passages serve as the context for prompting GPT-4o to generate QA pairs related to the targets.  To validate the model's knowledge and the quality of the generated pairs, we use these QA pairs to probe the evaluated model. We retain only those QA pairs for which the model successfully recalls the correct answer. This validation ensures both the consistency of the QA pairs and confirms the model's existing knowledge. 
%We also manually verify these probes to ensure correct format and type. For the forget set, we specifically select 15 QA pairs per entity.

For constructing the syntactically similar neighbor set, we first identify entities from the retain set that are not included in any of the other neighbor sets (forget, domain, or entity). Using the syntactic clusters identified in Section~\ref{sec:neighborentityselection}, we generate QA pairs that align with the syntactic structures of these clusters.

Specifically, we adopt the masking approach used in Section~\ref{subsec:EvaluatingRetainSetPreservation} when computing Levenshtein similarity. We first mask entity within the clustered questions and then generate new QA pairs by filling these masked structures with entities from the identified retain set. This ensures that the generated questions maintain syntactic similarity to the existing clusters while introducing new entities. We follow the same verification process (model probing and manual verification) as for the other neighbor sets to ensure the dataset's validity. The detailed procedure for constructing the syntactically similar neighbor set is outlined in Algorithm~\ref{algorithm:constructingDataset}.
% For constructing the syntactically similar neighbor set, we leverage the syntactic clusters identified in Section~\ref{sec:neighborentityselection}. We first identify entities from the retain set that are not included in any of the other neighbor sets (forget, domain, or entity). For these entities, we use GPT-4o to generate QA pairs that align with the syntactic structures of the identified clusters. Specifically, we prompt GPT-4o to create questions that exhibit high Levenshtein similarity to the questions within those clusters. These generated questions form the syntactically similar neighbor set. We follow the same verification process (model probing and manual verification) as for the other neighbor sets to ensure the dataset's validity. The detailed procedure for constructing the syntactically similar neighbor set is outlined in Algorithm~\ref{algorithm:constructingDataset}.


\paragraph{TOFU.} For the TOFU, the forget set and domain neighbor entities are defined by the benchmark itself~\citep{maini2024tofu}. To identify the syntactically similar neighbor set, we compare the provided neighbor sets against the forget set using the same syntactic similarity clustering method described above. Critically, we ensure that there is no overlap with the domain neighbor set. This approach ensures that the syntactically similar neighbor set reflects the structural patterns present in the forget set while maintaining distinctness from other neighbor sets. 

Further details and dataset statistics are provided in the appendix~\ref{appendix:dataset_construction}.
\begin{figure*}[t]
    \centering
    \begin{subfigure}{0.49\linewidth}
        \includegraphics[width=\linewidth]{figure/01_01_realworld_dissimilarity_similarity_comparison.pdf}
        \caption{Real-world Scenario}
        \label{fig:real-world_main}
    \end{subfigure}
    \hfill
    \begin{subfigure}{0.49\linewidth}
    \label{fig:tofu_main}
        \includegraphics[width=\linewidth]{figure/01_02_tofu_dissimilarity_similarity_comparison.pdf}
        \caption{TOFU}
        \label{fig:tofu_main}
    \end{subfigure}
    \caption{Relative Utility Drop (\%) for different neighbor sets across real-world scenario (left) and TOFU (right). Each method (GA, NPO, IDK, DPO) is evaluated based on its model utility before and after unlearning, with lower bars indicating greater utility loss. Model utility values before and after unlearning are provided in Appendix~\ref{appendix:detailedResultsPerMethods}}
    \label{fig:experiment1}
\end{figure*}
\section{Experimental Setup}
\subsection{Evaluation Metrics}
We evaluate the unlearned model using several metrics to assess its performance from various perspectives~\citep{closerlookat, maini2024tofu}. Specifically, we employ \emph{ROUGE} to measure word-level similarity, \emph{BERT Cosine Similarity} to assess semantic consistency between outputs before and after unlearning, \emph{Probability} to evaluate the model's confidence to predict the ground truth answer, and \emph{Entailment Score} to assess factual correctness relative to the ground truth.

\noindent
Since all metrics range from zero to one, we aggregate them using the arithmetic mean. Applying this to the retain set defines \textbf{Model Utility (MU)}, while applying it to the forget set defines \textbf{Forget Efficacy (FE)}.

\noindent
To quantify the impact of unlearning on neighbor sets, we define the \textbf{Relative Utility Drop (RUD)} as:
\begin{equation}
\resizebox{0.6\hsize}{!}{$
\textstyle \text{RUD} = \frac{MU_{\text{after}} - MU_{\text{before}}}{MU_{\text{before}}} \times 100.$}
\end{equation}
Since unlearning typically reduces MU, RUD is usually negative, indicating the degree of performance drop. This metric shows which neighbor set suffers the most performance decline after unlearning. Further details on metric computation are provided in Appendix~\ref{appendix:evaluationMetrics}.
\subsection{Unlearning Methods}
We explore various unlearning strategies, each of which aims to erase knowledge of target entities in distinct ways. A comprehensive explanation of these methods is provided in Appendix~\ref{appendix:overviewUnlearningMethods}.
\begin{itemize}[leftmargin=6pt]
    \item \textbf{GA}~\cite{jang2023knowledgeunlearning}: Utilizes gradient ascent on the forget set to counteract learned knowledge.
    \item \textbf{DPO}~\cite{rafailov2023dpo}: Treats unlearning as a preference optimization problem by applying the standard DPO loss. It uses answers in the forget set as negative samples and rejection templates (e.g., ``I don't know'') as positive samples to guide the model's response.
    \item \textbf{NPO}~\cite{zhang2024npo}: A variant of DPO that removes positive samples from the optimization process. It retains only negative examples from the forget set, encouraging the model to suppress forgotten information without explicit reinforcement of alternative responses.
    \item \textbf{IDK}~\cite{maini2024tofu}: Fine-tunes the model to default to ``I don't know'' responses when queried about the forget set.
\end{itemize}


\subsection{Implementation Details}
For the TOFU benchmark~\citep{maini2024tofu}, we utilize fine-tuned Llama-2-7b-chat~\citep{touvron2023llama2}, which has been trained on the constructed dataset to ensure it precisely answers questions in TOFU. For the real-world scenario benchmark, we employ Llama-3-8B-Instruct~\citep{dubey2024llama3}.
\noindent
To enable a fair comparison of different unlearning methods at similar levels of forgetting, we adjust the hyperparameters to keep Forget Efficacy between 0.65 and 0.75. Further details are provided in Appendix~\ref{appendix:implementationDetails}.


\section{How does Performance Degradation Vary across Different Neighbor Sets?}
\label{sec:problem}
This section investigates how performance degradation after unlearning varies across different neighbor sets.
First, we examine which neighbor sets experience the most significant performance degradation. (Section~\ref{sec:5_1}) If similar syntactic structures sets are the most vulnerable to forgetting, we further examine whether domain differences within these structures lead to varying effects. (Section~\ref{sec:5_2}) We then examine the robustness of these forgetting patterns when questions are paraphrased. (Section~\ref{sec:5_3}) Finally, we analyze gradient updates during unlearning to understand the underlying mechanisms driving the observed patterns. (Section~\ref{sec:5_4})
% First, we examine which neighbor sets experience the most significant performance degradation. (Section~\ref{sec:5_1}) Finding that syntactically similar sets show the highest vulnerability, we then investigate whether domain differences within syntactic similarities create varying effects. (Section~\ref{sec:5_2}) We then examine the robustness of these forgetting patterns when questions are paraphrased. (Section~\ref{sec:5_3}) Finally, we analyze gradient updates during unlearning to understand the underlying mechanisms driving the observed patterns. (Section~\ref{sec:5_4})

\subsection{Analyzing Performance Drops Across Neighbor Sets}
\label{sec:5_1}
\vspace{2pt}
\textbf{Syntactically Similar Neighbor Set Experiences Higher Forgetting.} Across both real-world scenario and  TOFU evaluations (Figure~\ref{fig:real-world_main} and Figure~\ref{fig:tofu_main}), $\mathcal{N}_{\text{syntactically}}$ consistently demonstrates a higher utility drop compared to both $\mathcal{N}_{\text{domain}}$ and $\mathcal{N}_{\text{entity}}$. The greater utility drop suggests that syntactic similarity plays a crucial role in the forgetting phenomenon.  When the model is unlearning specific data, it appears to struggle more with retaining information that shares similar sentence structures, regardless of the specific domain or entities involved.

\noindent \textbf{No Significant Difference among Existing Neighbor Sets.} In the real-world scenario, a notable observation is the lack of significant performance differences between $\mathcal{N}_{\text{domain}}$ and $\mathcal{N}_{\text{entity}}$. As depicted in Figure~\ref{fig:real-world_main}, both sets exhibit similar RUD across all methods. Our results show that, despite different ways of defining neighbor sets in previous studies~\citep{choi2024opt, closerlookat}, the impact caused by unlearning is similar across them.

\subsection{Exploring Domain Effects on Forgetting in Syntactically Similar Cases}
\label{sec:5_2}
To examine the domain-specific effects of unlearning in syntactically similar cases, we conduct experiments in real-world scenario across five distinct categories: \textit{Human, Company, Creative Works, Fictional Character,} and \textit{Product}. This analysis builds on our previous findings that syntactically similar neighbor sets exhibit more pronounced forgetting than those based on domain or entity similarity.

Prior studies~\cite{rwku,maini2024tofu} define data distributions similar to $\mathcal{D}_f$ using either entity similarity or professional domain association, assuming that these sets would experience the most significant forgetting. Under this assumption, the Human category would be expected to exhibit the highest degree of forgetting, as entities within this category tend to be closely related. However, as shown in Figure~\ref{fig:variousfig}, our results reveal the opposite trend—across all evaluated methods except for DPO, non-human categories exhibit substantially higher forgetting rates. This finding challenges the conventional assumption that neighbor sets constructed based on entity or domain similarity necessarily lead to higher forgetting.
\begin{figure}[t]
    \centering
    \includegraphics[width=1\linewidth]{figure/03_various_entity_types.pdf}
    \caption{Relative Utility Drop across different entity categories (Human, Company, Creative Works, Fictional Character, and Product) for various unlearning methods.}
    \label{fig:variousfig}
\end{figure}
\begin{figure}[t]
    \centering
    \includegraphics[width=1\linewidth]{figure/02_paraphrase_experiments.pdf}
    \caption{Relative Utility Drop for syntactically similar and different neighbor sets across different unlearning methods, measured over three paraphrases per question. A larger drop indicates higher semantic forgetting.}
    \label{fig:paraphrasefig}
\end{figure}
\subsection{Robustness of Forgetting Patterns in Paraphrased Scenarios}
\label{sec:5_3}
Our previous experiments reveal that syntactically similar neighbor sets experience higher levels of forgetting compared to other neighbor sets. To validate the robustness of this finding, we investigate whether this performance gap persists even when questions are paraphrased. 

Specifically, we generate paraphrased versions for each question for syntactically similar and different neighbor sets using GPT-4o Then, we filter out cases where the pre-unlearning model fails to provide correct answers, ensuring that each question has three valid paraphrases. We then measure the RUD for these paraphrased questions using the post-unlearning model and compare the forgetting rates across the two groups.

Figure~\ref{fig:paraphrasefig} shows that even after paraphrasing, syntactically similar neighbors exhibit greater utility drops than dissimilar neighbors. This suggests that the model's increased forgetting isn't solely due to shared syntax, but also reflects a sensitivity to underlying semantic relationships. The consistent performance gap after paraphrasing reinforces the role of syntactic similarity in forgetting, highlighting its influence beyond surface-level wording.





% Specifically, we generate three paraphrased versions for each question in both syntactically similar and different neighbor sets. We then measure the RUD for these paraphrased questions and compare the forgetting rates across the two groups.


\subsection{Gradient Analysis}
\label{sec:5_4}
To further investigate the underlying mechanisms behind the forgetting patterns observed in syntactically similar and dissimilar neighbor sets, we analyze the gradient updates during the unlearning process. Our primary goal is to understand how the model's gradient norms evolve when encountering different types of neighbors, particularly whether syntactically similar instances influence each other more strongly than dissimilar ones.

In our experimental setup, we perform gradient ascent on a syntactically similar set and track the changes in gradient norms as the model encounters other syntactically similar or syntactically different instances. Specifically, we measure the Frobenius norm of the model’s weight gradients at each unlearning step, comparing how the gradients behave when interacting with different types of data points.


\begin{figure}[t]
    \centering
    \includegraphics[width=1\linewidth]{figure/04_gradient_norm_plot.pdf}
    \caption{Frobenius norm of model weight gradients across unlearning steps. The gradient norms for syntactically similar instances (red) increase more steeply than those for syntactically different instances (blue).}
    \label{fig:gradientanalysis}
\end{figure}
As shown in Figure~\ref{fig:gradientanalysis}, the gradient norms of syntactically similar instances exhibit a steeper increase over unlearning steps compared to syntactically different instances. Notably, the initial gap between their gradient norms at the first checkpoint widens progressively as unlearning proceeds. This suggests that forgetting syntactically similar knowledge amplifies gradient updates in a way that reinforces the distinction between similar and dissimilar instances.
\begin{figure*}[t]
    \centering
    \includegraphics[width=0.9\linewidth]{figure/05_heatmap_figure.pdf}
    \caption{Relative utility drop (\%) averaged across all unlearning methods (GA, DPO, NPO, and IDK) under different retain set configurations using GD (left) and KL (right) regularization. The x-axis represents the type of train retain set, while the y-axis represents the type of test retain set. A higher value (darker color) indicates better utility retention. Detailed relative utility drop results for each individual unlearning method can be found in Appendix~\ref{appendix:detailedResultsPerMethods}.}
    \label{fig:regularizationheapmap}
    %\vspace{-3mm}
\end{figure*}
\section{What is the Optimal Neighbor Set for Effective Regularization?}
\label{sec:solution}
To preserve model utility during unlearning, regularization losses on a subset of the retain set are commonly employed during the unlearning process~\cite{closerlookat,maini2024tofu}. Based on the findings of the previous section, we aim to identify the optimal configuration of the retain set used for regularization, to optimize model utility while ensuring successful forgetting, specifically from a data perspective.

\paragraph{Regularization loss.} It encourages the unlearned model parameters $\bm \theta$ to preserves model utility. A typical unlearning objective function, computed on a subset of $\mathcal{D}_{\text{R}}$, is formulated as follows:
\begin{equation}
    \resizebox{0.71\hsize}{!}{$
    \underset{\bm \theta}{\min} \mathcal{L}(\bm \theta) = \underset{\bm \theta}{\min} - \mathcal{L}_f(\bm \theta) +  \mathcal{L}_{\text{R}}(\theta;\mathcal{D}_{\text{R}}).$}
    \label{eq:previous_objective}
\end{equation}
Our analysis considers two primary regularization approaches: Gradient Descent (GD) and Kullback-Leibler Divergence (KL). A comprehensive explanation of these methods is provided in Appendix~\ref{appendix:overviewUnlearningMethods}.
\paragraph{Setup.} To determine the optimal train retain set configuration, we conduct comprehensive experiments examining nine different combinations of train and test retain sets, using $\mathcal{N}_{\text{domain}}$, $\mathcal{N}_{\text{entity}}$, and $\mathcal{N}_{\text{syntactically}}$ for both training and evaluation. For each train retain set, we apply different unlearning methods (GA, DPO, NPO, and IDK) with regularization loss and report the average RUD across test retain sets.

\paragraph{Results.} We visualize the results separately for GD and KL regularization in Figure~\ref{fig:regularizationheapmap}. The results reveal two key findings:

\noindent
\textbf{1) Training with $\mathcal{N}_{\text{syntactically}}$ effectively preserves performance on $\mathcal{N}_{\text{syntactically}}$.} In both GD and KL regularization heatmaps, the bottom row (Test Retain Set: Syntactically Similar) shows that training with $\mathcal{N}_{\text{syntactically}}$ preserves utility best, with average differences of 14.7\% point and 7.35\% point compared to other training sets, respectively.

\noindent
\textbf{2) Training with $\mathcal{N}_{\text{syntactically}}$ contributes to robust performance across various neighbor sets.} Beyond preserving performance on syntactically similar data, training with $\mathcal{N}_{\text{syntactically}}$ also yields competitive results when evaluated on $\mathcal{N}_{\text{entity}}$ and $\mathcal{N}_{\text{domain}}$. In many cases, it surpasses or closely matches the performance achieved by training with other neighbor sets. These findings emphasize the role of syntactically similar examples in reducing utility loss while unlearning.
\section{Related Work on Cultural Change}
\label{sec:related_work}

%Understanding if and how distributional models understand semantic knowledge (e.g., is ``dog'' a mammal) is an important research question. For example, \citet{rubinstein-etal-2015-well} show that static distributional embeddings capture well taxonomical properties, but do not perform well in general attributive semantics (e.g., predicting the color of something). Recently, large language models were shown to \textit{have} some knowledge about concepts~\cite{dalvi2022discovering,ettinger2020bert,weir2020probing,petroni2019language}, including word sense in their contextualized embeddings~\cite{reif2019visualizing}. However, their encoded knowledge is static and lacks structure and domain specificity \cite{brandl}.

%\cnote{..}

%\paragraph{Word similarity.} Cosine similarity is a standard measure of semantic similarity, but its effectiveness is limited by the representational geometry of learned embeddings. The anisotropy of contextualized embedding spaces causes a small number of rogue dimensions to dominate cosine similarity computations \citep{timkey2021all}.
%Further, cosine similarity underestimates the semantic similarity of high-frequency words \citep{zhou2022problems}, heavily depends on the regularization techniques used during training \citep{steck2024cosine} and often fails in capturing human interpretation \cite{sitikhu2019comparison}. The proposed \wc enables a similarity measure that sidesteps these limitations via softmax-normalized dot products.



% % Recent work has explored the limits of cosine similarity


%\paragraph{Asymmetry.} By definition, cosine similarity is a symmetric metric that cannot capture the asymmetry of semantic relationships \citep{vilnis2014word}. Efforts to account for this caveat show partial successes, emphasizing the inherent symmetrical nature of cosine similarity using some language model embeddings \citet{zhang2021circles, rodriguez2020word}.

%\paragraph{Word embeddings for semantic and cultural change.} 
Both static and contextualized embedding spaces contain semantically meaning dimensions that align with high-level linguistic and cultural features \citep{bolukbasi2016man, DBLP:journals/corr/abs-1906-02715}. These embeddings have enabled a large number of quantitative analyses of temporal shifts in meaning and links to cultural or social scientific variables. For example early on, using static embeddings, \citet{hamilton2016cultural} measured linguistic drifts in global semantic space as well as cultural shifts in particular local semantic neighborhoods. \citet{garg2018word} demonstrated that changes in word embeddings correlated with demographic and occupation shifts through the 1900s.

Analyzes of contextualized embeddings have identified semantic axes based on pairs of ``seed words'' or ``poles'' \citep{soler2020bert, lucy2022discovering, grand2022semantic}. Across the temporal dimension, such axes can measure the evolution of gender and class \citep{kozlowski2019geometry}, internet slang \citep{keidar-etal-2022-slangvolution}, and more \citep{madani2023measuring, lyu2023representation, erk2024adjusting}. \citet{bravzinskas2017embedding} proposes a probabilistic measure for lexical similarity. 

It's also instructive to consider the similarity of our method  with tasks like word sense disambiguation (WSD) and named entity recognition (NER). The central idea behind \wc of mapping from embeddings to categories are also found in NER and WSD. What differs is the dynamic nature of the categories. Where NER focuses on pre-defined concept hierarchies and WSD on pre-defined senses per word,  \wc  focuses on a coherent but dynamic grouping of words that is interpretable for a given task.

% an important direction for future work in computational social science (see 3.1.10 in \citet{ziems2023can}).

% \subsection{Semantic change}


% Survey \citep{de2024survey} (should prob look more)

% Diachronic word embeddings 

%  \citet{di2019training}






% Similarly to named entity recognition, \ac{ourmethod} attempts to map words to classes that \textit{might be} types (such as ORGANIZATION or PLACE). However, \ac{ourmethod} is partially self-supervised and it is entirely user-driven.

% Similarly to semantic change detection, \ac{ourmethod} attempts to capture the usage of a word in different contexts. However, \ac{ourmethod} offers the possibility of defining the axis (the seed words) onto which the user wants to project words. 


\section{Conclusion \& Future Work}\label{conclusion}
This work presents XAMBA, the first framework optimizing SSMs on COTS NPUs, removing the need for specialized accelerators. XAMBA mitigates key bottlenecks in SSMs like CumSum, ReduceSum, and activations using ActiBA, CumBA, and ReduBA, transforming sequential operations into parallel computations. These optimizations improve latency, throughput (Tokens/s), and memory efficiency. Future work will extend XAMBA to other models, explore compression, and develop dynamic optimizations for broader hardware platforms.



% This work introduces XAMBA, the first framework to optimize SSMs on COTS NPUs, eliminating the need for specialized hardware accelerators. XAMBA addresses key bottlenecks in SSM execution, including CumSum, ReduceSum, and activation functions, through techniques like ActiBA, CumBA, and ReduBA, which restructure sequential operations into parallel matrix computations. These optimizations reduce latency, enhance throughput, and improve memory efficiency. 
% Experimental results show up to 2.6$\times$ performance improvement on Intel\textregistered\ Core\texttrademark\ Ultra Series 2 AI PC. 
% Future work will extend XAMBA to other models, incorporate compression techniques, and explore dynamic optimization strategies for broader hardware platforms.


% This work presents XAMBA, an optimization framework that enhances the performance of SSMs on NPUs. Unlike transformers, SSMs rely on structured state transitions and implicit recurrence, which introduce sequential dependencies that challenge efficient hardware execution. XAMBA addresses these inefficiencies by introducing CumBA, ReduBA, and ActiBA, which optimize cumulative summation, ReduceSum, and activation functions, respectively, significantly reducing latency and improving throughput. By restructuring sequential computations into parallelizable matrix operations and leveraging specialized hardware acceleration, XAMBA enables efficient execution of SSMs on NPUs. Future work will extend XAMBA to other state-space models, integrate advanced compression techniques like pruning and quantization, and explore dynamic optimization strategies to further enhance performance across various hardware platforms and frameworks.
% This work presents XAMBA, an optimization framework that enhances the performance of SSMs on NPUs. Key techniques, including CumBA, ReduBA, and ActiBA, achieve significant latency reductions by optimizing operations like cumulative summation, ReduceSum, and activation functions. Future work will focus on extending XAMBA to other state-space models, integrating advanced compression techniques, and exploring dynamic optimization strategies to further improve performance across various hardware platforms and frameworks.

% This work introduces XAMBA, an optimization framework for improving the performance of Mamba-2 and Mamba models on NPUs. XAMBA includes three key techniques: CumBA, ReduBA, and ActiBA. CumBA reduces latency by transforming cumulative summation operations into matrix multiplication using precomputed masks. ReduBA optimizes the ReduceSum operation through matrix-vector multiplication, reducing execution time. ActiBA accelerates activation functions like Swish and Softplus by mapping them to specialized hardware during the DPU’s drain phase, avoiding sequential execution bottlenecks. Additionally, XAMBA enhances memory efficiency by reducing SRAM access, increasing data reuse, and utilizing Zero Value Compression (ZVC) for masks. The framework provides significant latency reductions, with CumBA, ReduBA, and ActiBA achieving up to 1.8X, 1.1X, and 2.6X reductions, respectively, compared to the baseline.
% Future work includes extending XAMBA to other state-space models (SSMs) and exploring further hardware optimizations for emerging NPUs. Additionally, integrating advanced compression techniques like pruning and quantization, and developing adaptive strategies for dynamic optimization, could enhance performance. Expanding XAMBA's compatibility with other frameworks and deployment environments will ensure broader adoption across various hardware platforms.
\section*{Limitations}
Our study focuses on entity unlearning, leaving hazardous knowledge and copyrighted content unlearning unexplored. These cases may require different evaluation strategies.

Additionally, our experiments use mid-sized models (LLaMA-2-7B-Chat, LLaMA-3-8B-Instruct). Larger models, with their computational demands and structural differences, may respond differently. Future research should assess their applicability to such models.



\section*{Acknowledgement}
This research was supported by Institute for Information \& Communications Technology Planning \& Evaluation (IITP) through the Korea government (MSIT) under Grant No. 2021-0-01341 (Artificial Intelligence Graduate School Program (Chung-Ang University)).


\bibliography{acl}
\newpage
\appendix
\appendices

\section{Robot Setups}\label{app:robot_setup}

\subsection{Simulation Robot Setups}
To ensure fairness, we utilize the same Franka Panda arm for evaluations in both the LIBERO~\cite{LIBERO23} and our Open6DOR V2 benchmarks. For SIMPLER~\cite{simplerenv24}, we use the Google Robot exclusively to conduct the baseline experiments, adhering to all configurations outlined in SIMPLER, as presented in Table~\ref{tab:simpler_env}. 

\subsection{Real World Robot Setups}
As for manipulation tasks, in \cref{fig:robots}, we perform 6-DoF rearrangement tasks using the Franka Panda equipped with a gripper and the UR robot arm with a LeapHand, while articulated object manipulation is conducted using the Flexiv arm equipped with a suction tool. All the robot arms mount a Realsense D415 camera to its end for image capturing.
\begin{figure}[h!]
\centering
\includegraphics[width=0.96\linewidth]{figs/src/robots.pdf}
\vspace{-5pt}
\captionof{figure}{\textbf{The robots used in our real-world experiments.}}
\vspace{-5pt}
\label{fig:robots}
\end{figure}


In \cref{fig:franka_setup}, we present the workspace and robotic arm for real-world 6-DoF rearrangement. Unlike Rekep~\cite{ReKep24}, CoPa~\cite{CoPa24} et al., we utilize only a single RealSense D415 camera. This setup significantly reduces the additional overhead associated with environmental setup and multi-camera calibration, and it is more readily reproducible.
\begin{figure}[h!]
\centering
\includegraphics[width=1.0\linewidth]{figs/src/franka_setup.pdf}
\captionof{figure}{\textbf{6-DoF rearrangement robot setup.}}
\vspace{-10pt}
\label{fig:franka_setup}
\end{figure}


As for navigation tasks, we provide a visualization of our robotic dog in~\cref{fig:dog_setup}. Following Uni-Navid~\cite{uninavid24}, our robotic dog is Unitree GO2 and we mount a RealSense D455 camera on the head of the robotic dog. Here, we only use the RGB frames with a resolution of $640\times480$ in the setting of  $90^\circ$ HFOV. We also mount a portable Wi-Fi at the back of the robot dog, which is used to communicate with the remote server (send captured images and receive commands). Unitree GO2 is integrated with a LiDAR-L1, which is only used for local motion planning. 
\begin{figure}
\begin{center}
  \includegraphics[width=0.7\linewidth]{figs/src/robotdog.pdf}
\end{center}
   \caption{\textbf{Navigation robot setup.} We use Unitree GO2 as our embodiment, and we mount RealSense D455, a portable Wi-Fi and a LiDAR-L1. Note that, our model only takes RGB frames as input. The portable Wi-Fi is used for communication with the remote server and the Lidar is used for the local controller API of Unitree Dog.}
   \vspace{-10pt}
\label{fig:dog_setup}
\end{figure}


\section{Additional Experiments}\label{app:add_exp}

\subsection{Articulated Objects Manipulation Evaluation}
We further integrate \ours~with articulated object manipulation, as illustrated in \cref{tab:manip}, and evaluate its practicality in robotic manipulation tasks using the PartNet-Mobility Dataset within the SAPIEN~\cite{SAPIEN20} simulator. Our experimental setup follows ManipLLM~\cite{ManipLLM24}, employing the same evaluation metrics. Specifically, we directly utilize the segmentation centers provided by SAM as contact points, leverage PointSO to generate contact directions, and use VLM to determine subsequent motion directions. The results demonstrate significant improvements over the baseline. Notably, our model achieves this performance without dividing the data into training and testing sets, operating instead in a fully zero-shot across most tasks. This underscores the robustness and generalization of our approach.
% !TEX root = ../../top.tex
% !TEX spellcheck = en-US

% Temporary modification of the table column separation width
\setlength\mytabcolsep{\tabcolsep}
\setlength\tabcolsep{2pt}


\renewcommand{\manipimgcar}[1]{\includegraphics[width=0.13\linewidth]{#1}}
\renewcommand{\manipimgmix}[1]{\includegraphics[width=0.07\linewidth]{#1}}
\renewcommand{\manipimg}[1]{\includegraphics[width=0.08\linewidth]{#1}}


\begin{figure*}[t]
	\centering
	\small
%	\vspace{-3mm}
	\begin{tabular}{c|c|c}
		% Cars 
		\begin{tabular}{c|c}
			\manipimgcar{fig/manip/car_olivier_lat_489_277/init} & \manipimgcar{fig/manip/car_olivier_lat_34_308/init} \\
			\manipimgcar{fig/manip/car_olivier_lat_489_277/final} & \manipimgcar{fig/manip/car_olivier_lat_34_308/final} \\
		\end{tabular}
		&
		% Mixers 
		\begin{tabular}{cc|cc}
			\manipimgmix{fig/manip/mixer_lat_1424_171/init} & \manipimgmix{fig/manip/mixer_lat_1424_171/final} & \manipimgmix{fig/manip/mixer_lat_957_629/init} & \manipimgmix{fig/manip/mixer_lat_957_629/final} \\
		\end{tabular}
		&
		% Chairs
		\begin{tabular}{cc|cc}
			\manipimg{fig/manip/chair_sepreg_lat_550_531/init} & \manipimg{fig/manip/chair_sepreg_lat_550_531/final} & \manipimg{fig/manip/chair_sepreg_lat_745_952/init} & \manipimg{fig/manip/chair_sepreg_lat_745_952/final} \\
		\end{tabular} \\
		\midrule
		% Cars 2
		\begin{tabular}{c|c}
		\manipimgcar{fig/manip/car_olivier_pose_290/init} & \manipimgcar{fig/manip/car_olivier_pose_691/init} \\
		\manipimgcar{fig/manip/car_olivier_pose_290/final} & \manipimgcar{fig/manip/car_olivier_pose_691/final} \\
		\end{tabular}
		&
		% Mixers 2
		\begin{tabular}{cc|cc}
		\manipimgmix{fig/manip/mixer_pose_872/init} & \manipimgmix{fig/manip/mixer_pose_872/final} & \manipimgmix{fig/manip/mixer_pose_183/init} & \manipimgmix{fig/manip/mixer_pose_183/final} \\
		\end{tabular}
		&
		% Chairs 2
		\begin{tabular}{cc|cc}
		\manipimg{fig/manip/chair_sepreg_pose_422/init} & \manipimg{fig/manip/chair_sepreg_pose_422/final} & \manipimg{fig/manip/chair_sepreg_pose_859/init} & \manipimg{fig/manip/chair_sepreg_pose_859/final} \\
		\end{tabular} \\
	\end{tabular}
	\caption{\textbf{Additional shape manipulation.} We manipulate four shapes per dataset: (\textit{top}) by changing the latent of specific parts (car body, mixer helix, and all chair parts) and (\textit{bottom}) by editing part poses (car wheels, mixer width, chair width and height). In all cases, the parts adapts to the modifications and to each other, maintaining a coherent whole.}
	\label{fig:supp-manip}
\end{figure*}


% Restore table column separation width
\setlength{\tabcolsep}{\mytabcolsep}


\subsection{Spatial VQA on EmbSpatial-Bench~\cite{embspatial24} \& SpatialBot-Bench~\cite{SpatialBot24}}
To further demonstrate \sofar's spatial reasoning capabilities, we conducted Spatial VQA tests within the EmbSpatial-Bench~\cite{embspatial24} and SpatialBot-Bench~\cite{SpatialBot24}. As shown in \cref{tab:embspatial,tab:spatialbot}, \sofar~significantly outperformed all baselines, achieving more than a 20\% performance improvement in EmbSpatial-Bench.
\begin{table}[h!]
\centering
\setlength{\tabcolsep}{8.5pt}
\caption{Zero-shot performance of LVLMs in EmbSpatial-Bench~\cite{embspatial24}. \textbf{Bold} indicates the best results.}
\resizebox{0.97\linewidth}{!}{
\begin{tabular}{lcc}
\toprule
Model & Generation & Likelihood \\
\midrule
BLIP-2~\cite{BLIP223} & 37.99 & 35.71 \\
InstructBLIP~\cite{InstructBLIP23} & 38.85 & 33.41 \\
MiniGPT4~\cite{MiniGPT4_23} & 23.54 & 31.70 \\
LLaVA-1.6~\cite{LLaVA23} & 35.19 & 38.84 \\
\midrule
GPT-4V~\cite{GPT4Vision23} & 36.07 & - \\
\rowcolor{linecolor1}Qwen-VL-Max~\cite{qwenvl23} & 49.11 & - \\
\rowcolor{linecolor2}\textbf{\ours} & \textbf{70.88} & - \\
\bottomrule
\end{tabular}
}
\label{tab:embspatial}
\end{table}
\begin{table}[h!]
\centering
\setlength{\tabcolsep}{3pt}
\caption{Zero-shot performance of LVLMs in SpatialBot-Bench~\cite{SpatialBot24}.
SpatialBot-3B: SpatialBot-Phi2-3B-RGB, SpatialBot-8B: SpatialBot-Llama3-8B-RGB.
}
\resizebox{1.00\linewidth}{!}{
\begin{tabular}{lcccccc}
\toprule
Model & Pos & Exist & Count & Reach & Size & Avg \\
\midrule
ChatGPT-4o~\cite{GPT4o24} & 70.6 & 85.0 & 84.5 & 51.7 & \textbf{43.3} & 67.0\\
SpatialBot-3B~\cite{SpatialBot24} & 64.7 & 80.0 & 88.0 & \textbf{61.7} & 28.3 & 64.5\\
SpatialBot-8B~\cite{SpatialBot24} & 55.9 & 80.0 & \textbf{91.2} & 40.0 & 20.0 & 57.4\\
\midrule
\rowcolor{linecolor2}\textbf{\ours} & \textbf{76.5} & \textbf{87.5} & 80.0 & 57.5 & 40.0 & \textbf{68.3}\\
\bottomrule
\end{tabular}
}
\label{tab:spatialbot}
\end{table}


\subsection{Close-Loop Execution Experiment}\label{app:close_loop}
We demonstrate the closed-loop replan capabilities of \sofar~within Simpler-Env~\cite{simplerenv24} in \cref{fig:close_loop}. The instruction for both tasks is ``pick the coke can'' In \cref{fig:close_loop} (a), the model initially misidentified the coke can as a Fanta can. After correction by the VLM, the model re-identified and located the correct object. In \cref{fig:close_loop} (b), the model accidentally knocks over the Coke can during motion due to erroneous motion planning. Subsequently, the model re-plans and successfully achieves the grasp.

\subsection{Long Horizon Object Manipulation Experiment}\label{app:long_horizon}
\cref{fig:long_horizon} illustrates the execution performance of our model on long-horizon tasks. Through the VLM~\cite{GPT4o24,gemini23}, complex instructions such as ``making breakfast'' and ``cleaning up the desktop'' can be decomposed into sub-tasks. In the second example, we deliberately chose complex and uncommon objects as assets, such as ``Aladdin's lamp'' and ``puppets'', but \sofar~is able to successfully complete all tasks.
\begin{figure}[h!]
\centering
\includegraphics[width=1.0\linewidth]{figs/src/close_loop_execution.pdf}
\vspace{-15pt}
\captionof{figure}{\textbf{Close-loop execution of our \sofar.}}
\label{fig:close_loop}
\end{figure}


\subsection{In the Wild Evaluation of Semantic Orientation}
We provide a qualitative demonstration of the accuracy of PointSO under in-the-wild conditions, as shown in \cref{fig:in_the_wild}, where the predicted Semantic Orientation is marked in the images. We obtained single-sided point clouds by segmenting objects using Florence-2~\cite{florence2} and SAM~\cite{SAM23} and fed them into PointSO. It can be observed that our model achieves good performance across different views, objects, and instructions, which proves the effectiveness and generalization of PointSO.
\begin{figure*}[h!]
\centering
\includegraphics[width=1.0\linewidth]{figs/src/long_horizon.pdf}
\captionof{figure}{\textbf{Long-horizon object manipulation experiment of our \sofar.}}
\label{fig:long_horizon}
\end{figure*}

\begin{figure}[h!]
\centering
\includegraphics[width=1.0\linewidth]{figs/src/in_the_wild.pdf}
\vspace{-15pt}
\captionof{figure}{\textbf{In the wild evaluation of PointSO.}}
\label{fig:in_the_wild}
\end{figure}

\begin{figure}[t!]
\centering
\includegraphics[width=1.0\linewidth]{figs/src/cross_view.pdf}
\captionof{figure}{\textbf{Cross view generalization of our \sofar.}}
\label{fig:cross_view}
\end{figure}


\subsection{Cross-View Generalization}
\sofar~gets point clouds in the world coordinate system using an RGB-D camera to obtain grasping poses, and it is not limited to a fixed camera perspective. In addition, PointSO generates partial point clouds from different perspectives through random camera views to serve as data augmentation for training data, which also generalizes to camera perspectives in the real world. \cref{fig:cross_view} illustrates \sofar's generalization capability for 6-DoF object manipulation across different camera poses. It can be observed that whether it's a front view, side view, or ego view, \sofar~can successfully execute the ``upright the bottle'' instruction.

\subsection{Failure Case Distribution Analysis}
Based on the failure cases from real-world experiments, we conducted a quantitative analysis of the failure case distribution for \sofar, with the results shown in \cref{fig:failure_case}. It can be observed that 31\% of the failures originated from grasping issues, including objects being too small, inability to generate reasonable grasping poses, and instability after grasping leading to sliding or dropping. Next, 23\% were due to incorrect or inaccurate Semantic Orientation prediction. For tasks such as upright or upside - down, highly precise angle estimation (<5°) is required for smooth execution. Object analysis and detection accounted for approximately 20\% of the errors. The instability of open-vocabulary detection modules like Florence2~\cite{florence2} and Grounding DINO~\cite{groundingdino23} often led to incorrect detection of out-of-distribution objects or object parts. In addition, since our Motion Planning did not take into account the working space range of the robotic arm and potential collisions of the manipulated object, occasional deadlocks and collisions occurred during motion. Finally, there were issues with the Task Planning of the VLM~\cite{GPT4o24,gemini23}. For some complex Orientations, the VLM occasionally failed to infer the required angles and directions to complete the task. Employing a more powerful, thought-enabled VLM~\cite{gpt_o1} might alleviate such errors.

\begin{figure*}[t]
\centering
\includegraphics[width=\linewidth]{CameraReady/Figures/failure_case_image.pdf}
\caption{Failure cases of our approach.}
\label{fig:failure_case}
\end{figure*}

\subsection{Ablation Study}\label{app:ablation}
\subsubsection{Scaling Law}
The scaling capability of models and data is one of the most critical attributes today and a core feature of foundation models~\cite{FoundationModel21}. We investigate the performance of PointSO across different data scales, as illustrated in \cref{tab:scaling_law}. 
We obtain the subset for OrienText300K from Objaverse-LVIS, which consists of approximately 46,000 3D objects with category annotations. The selection was based on the seven criteria mentioned in the main text. Objects meeting all seven criteria formed the strict subset, comprising around 15k objects. When including objects without textures and those of lower quality, the total increases to approximately 26k objects.
It can be seen that the increase in data volume is the most significant factor driving the performance improvement of PointSO. It can be anticipated that with further data expansion, such as Objaverse-XL~\cite{ObjaverseXL23}, PointSO will achieve better performance.
\begin{table}[t!]
\setlength{\tabcolsep}{7pt}
\caption{\textbf{Data scaling property} of semantic orientation with different training data scales evaluated on OrienText300K test split. All experiments are conducted with PointSO-Base.
}
\label{tab:scaling_law}
\centering
\resizebox{1.0\linewidth}{!}{
\begin{tabular}{lccccc}
\toprule[0.95pt]
    Data Scale & \texttt{45°} & \texttt{30°} & \texttt{15°} & \texttt{5°} & Average \\ 
    \midrule[0.6pt]
    5\% & 57.03 & 46.09 & 39.84 & 27.34 & 42.58 \\
    10\% & 61.72 & 53.13 & 43.75 & 30.47 & 47.27 \\
    50\% & 76.56 & 72.66 & 66.41 & 56.25 & 67.97 \\
    \rowcolor{linecolor2}100\% & \textbf{79.69} & \textbf{77.34} & \textbf{70.31} & \textbf{62.50} & \textbf{72.46} \\
    \bottomrule[0.95pt]
\end{tabular}
}
\end{table}

\subsubsection{Cross-Modal Fusion Choices}\label{app:fusion}
We further conduct an ablation study on the multi-modal fusion methods in PointSO, testing commonly used feature fusion techniques such as cross-attention, multiplication, addition, and concatenation, as shown in \cref{tab:fusion}. The results indicate that simple addition achieves the best performance. This may be attributed to the fact that instructions in the semantic domain are typically composed of short phrases or sentences, and the text CLS token already encodes sufficiently high-level semantic information.
% \usepackage{multirow}
% \usepackage{booktabs}


\begin{table}[h]
	\centering
	\caption{Model performance under two coefficient fusion methods.}
	\begin{tabular}{c|ccc} 
		\toprule
		\multirow{2}{*}{Fusion mode} & \multicolumn{3}{c}{R40}                           \\
		& Easy           & Moderate       & Hard            \\ 
		\hline
		straight                     & 90.92          & 82.84          & 80.29           \\
		our                          & \textbf{91.96} & \textbf{83.31} & \textbf{80.59}  \\
		\bottomrule
	\end{tabular}
\label{tabel6}
\end{table}

\begin{table*}[h!]
\setlength{\tabcolsep}{3pt}
\caption{\textbf{Ablation study of open vocabulary detection modules} on Open6DOR~\cite{Open6DOR24} perception tasks.
}
\label{tab:detection_ab}
\centering
\resizebox{0.98\linewidth}{!}{
\begin{tabular}{lccccccccccc}
\toprule[0.95pt]
\multirow{2}{*}[-0.5ex]{Method} & \multicolumn{3}{c}{\textbf{Position Track}} & \multicolumn{4}{c}{\textbf{Rotation Track}} & \multicolumn{3}{c}{\textbf{6-DoF Track}} & \multirow{2}{*}[-0.5ex]{Time Cost (s)}\\
\cmidrule(lr){2-4} \cmidrule(lr){5-8} \cmidrule(lr){9-11}
& Level 0 & Level 1 & Overall & Level 0 & Level 1 & Level 2 & Overall & Position & Rotation & Overall \\ 
\midrule[0.6pt]
YOLO-World~\cite{yoloworld24} & 59.0 & 37.7 & 53.3 & 48.3 & 36.1 & 62.0 & 44.9 & 53.4 & 44.6 & 27.8 & \textbf{7.4s}\\
\rowcolor{linecolor1}Grounding DINO~\cite{groundingdino23} & 92.2 & 71.5 & 86.7 & 64.7 & 41.1 & 69.8 & 55.5 & 87.2 & 51.6 & 44.6 & 9.2s\\
\rowcolor{linecolor2}Florence-2~\cite{florence2} & \textbf{96.0} & \textbf{81.5} & \textbf{93.0} & \textbf{68.6} & \textbf{42.2} & \textbf{70.1} & \textbf{57.0} & \textbf{92.7} & \textbf{52.7} & \textbf{48.7} & \textbf{8.5s}\\
\bottomrule[0.95pt]
\end{tabular}
}
\end{table*}

\subsubsection{Open Vocabulary Object Detection Module}
\sofar~utilize a detection foundation model to localize the interacted objects or parts, then generate masks with SAM~\cite{SAM23}. Although not the SOTA performance on the COCO benchmark, Florence-2~\cite{florence2} exhibits remarkable generalization in in-the-wild detection tasks, even in simulator scenarios. \cref{tab:detection_ab} illustrates the performance of various detection modules in Open6DOR~\cite{Open6DOR24} Perception, where Florence-2 achieves the best results and outperforms Grounding DINO~\cite{groundingdino23} and YOLO-World~\cite{yoloworld24}.

\vspace{3pt}
\section{Additional Implementation Details}\label{app:implementation_details}

\subsection{Detail Real World Experiment Results}\label{app:detail_realworld}
To fully demonstrate the generalization of \sofar~rather than cherry-picking, we carefully design 60 different real-world experimental tasks, covering more than 100 different and diverse objects. Similar to the Open6DOR~\cite{Open6DOR24} benchmark in the simulator, we divide these 60 tasks into three parts: position-track, orientation-track, and the most challenging comprehensive \& 6-DoF-track. Each track is further divided into simple and hard levels. The position-simple track includes tasks related to front \& back \& left \& right spatial relationships, while the position-hard track includes tasks related to between, center, and customized. The orientation-simple track includes tasks related to the orientation of object parts, and the orientation-hard track includes tasks related to whether the object is upright or flipped (with very strict requirements for angles in both upright and flipped cases). Comprehensive tasks involve complex instruction understanding and long-horizon tasks; 6-DoF tasks simultaneously include requirements for both object position and orientation instructions. In \cref{tab:detailed_realworld}, we present the complete task instructions, as well as the performance metrics of \sofar~and the baseline. Due to the large number of tasks, we performed each task three times. It can be seen that \sofar~achieves the best performance in all tracks, especially in the orientation-track and comprehensive \& 6-DoF-track. We also show all the objects used in the real-world experiments in \cref{fig:real_obj}, covering a wide range of commonly and uncommonly used objects in daily life.

\vspace{-10pt}
\begin{table*}[t!]
\setlength{\tabcolsep}{6pt}
\caption{\textbf{Detailed zero-shot real-world 6-DoF rearrangement results}.}
\label{tab:detailed_realworld}
\centering
\resizebox{0.96\linewidth}{!}{
\begin{tabular}{lcccc}
\toprule[0.95pt]
    Task & CoPa~\cite{CoPa24} & ReKep-Auto~\cite{ReKep24} & \sofar-LLaVA~(Ours) & \sofar~(Ours) \\ 
    \midrule[0.6pt]
    \multicolumn{5}{c}{\textit{Positional Object Manipulation}}\\
    \midrule
    Move the soccer ball to the right of the bread. & 2/3 & 3/3 & 3/3 & \textbf{3/3} \\
    Place the doll to the right of the lemon. & 3/3 & 3/3 & 3/3 & \textbf{3/3} \\
    Put the pliers on the right side of the soccer ball. & 1/3 & 1/3 & 3/3 & \textbf{2/3} \\
    Move the pen to the right of the doll. & 3/3 & 2/3 & 3/3 & \textbf{3/3} \\
    Place the carrot on the left of the croissant. & 2/3 & 3/3 & 2/3 & \textbf{2/3} \\
    Move the avocado to the left of the baseball. & 3/3 & 2/3 & 2/3 & \textbf{3/3} \\
    Pick the pepper and place it to the left of the charger. & 1/3 & 2/3 & 2/3 & \textbf{2/3} \\
    Place the baseball on the left side of the mug. & 3/3 & 2/3 & 2/3 & \textbf{3/3} \\
    Arrange the flower in front of the potato. & 2/3 & 3/3 & 2/3 & \textbf{3/3} \\
    Put the volleyball in front of the knife. & 3/3 & 3/3 & 3/3 & \textbf{3/3} \\
    Place the ice cream cone in front of the potato. & 2/3 & 3/3 & 2/3 & \textbf{3/3} \\
    Move the bitter melon to the front of the forklift. & 2/3 & 1/3 & 2/3 & \textbf{2/3} \\
    Place the orange at the back of the stapler. & 3/3 & 2/3 & 3/3 & \textbf{3/3} \\
    Move the panda toy to the back of the shampoo bottle. & 2/3 & 3/3 & 3/3 & \textbf{2/3} \\
    pick the pumpkin and place it behind the pomegranate. & \textbf{3/3} & 2/3 & 1/3 & 2/3 \\
    Place the basketball at the back of the board wipe. & 2/3 & 2/3 & 3/3 & \textbf{2/3} \\
    Put the apple inside the box. & 3/3 & 2/3 & 3/3 & \textbf{3/3} \\
    Place the waffles on the center of the plate. & 3/3 & 2/3 & 3/3 & \textbf{3/3} \\
    Move the hamburger into the bowl.& 2/3 & 2/3 & 2/3 & \textbf{3/3} \\
    Pick the puppet and put it into the basket. & 1/3 & 2/3 & 2/3 & \textbf{2/3} \\
    Drop the grape into the box. & 2/3 & 3/3 & 3/3 & \textbf{2/3} \\
    Put the doll between the lemon and the USB. & 2/3 & 2/3 & 2/3 & \textbf{3/3} \\
    Set the duck toy in the center of the cart, bowl, and camera. & 2/3 & 1/3 & 2/3 & \textbf{2/3} \\
    Place the strawberry between the Coke bottle and the glue. & 2/3 & 2/3 & 3/3 & \textbf{3/3} \\
    Put the pen behind the basketball and in front of the vase. & 2/3 & 1/3 & 2/3 & \textbf{2/3} \\
    Total success rate& 74.7\% & 72.0\% & 81.3\% & \textbf{85.3\%} \\
    \midrule
    \multicolumn{5}{c}{\textit{Orientational Object Manipulation}}\\
    \midrule
    Turn the yellow head of the toy car to the right. & 2/3 & 2/3 & 1/3 & \textbf{2/3} \\
    Adjust the knife handle so it points to the right. & 2/3 & 1/3 & 2/3 & \textbf{2/3} \\
    Rotate the cap of the bottle towards the right. & 2/3 & 2/3 & 2/3 & \textbf{2/3}\\
    Rotate the tip of the screwdriver to face the right. & 0/3 & 0/3 & 1/3 & \textbf{1/3}\\
    Rotate the stem of the apple to the right. & 0/3 & 1/3 & 1/3 & \textbf{2/3}\\
    Turn the front of the toy car to the left. & 0/3 & 0/3 & 2/3 & \textbf{2/3} \\
    Rotate the cap of the bottle towards the left. & 2/3 & 1/3 & 1/3 & \textbf{2/3}\\
    Adjust the pear's stem to the right. & 1/3 & 1/3 & 1/3 & \textbf{1/3}\\
    Turn the mug handle to the right. & 1/3 & 1/3 & 2/3 & \textbf{2/3}\\
    Rotate the handle of the mug to towards right.& 2/3 & 1/3 &\textbf{2/3} & 1/3\\
    Rotate the box so the text side faces forward. & 0/3 & 1/3 & 0/3 & \textbf{1/3}\\
    Adjust the USB port to point forward. & 0/3 & 0/3 & 1/3 & \textbf{1/3}\\
    Set the bottle upright. & 0/3 & 1/3 & 0/3 & \textbf{1/3}\\
    Place the coffee cup in an upright position. & 1/3 & 1/3 & 2/3 & \textbf{2/3}\\
    Upright the statue of liberty& 0/3 & 0/3 & \textbf{1/3} & 0/3\\
    Stand the doll upright. & 0/3 & 1/3 & 0/3 & \textbf{1/3}\\
    Right the Coke can. & 0/3 & 0/3 & 1/3 & \textbf{1/3}\\
    Flip the bottle upside down. & 0/3 & 0/3 & 0/3 & \textbf{1/3}\\
    Turn the coffee cup upside down. & 0/3 & 0/3 & 1/3 & \textbf{1/3}\\
    Invert the shampoo bottle upside down. & 0/3 & 0/3 & 0/3 & \textbf{0/3}\\
    Total success rate& 21.7\% & 23.3\% & 35.0\% & \textbf{43.3\%} \\
    \midrule
    \multicolumn{5}{c}{\textit{Comprehensive 6-DoF Object Manipulation}}\\
    \midrule
    Pull out a tissue.& 3/3 & 3/3 & 2/3 & \textbf{3/3}\\
    Place the right bottle into the
    box and arrange it in a 3×3 pattern. & 0/3 & 0/3 & 0/3 & \textbf{1/3}\\
    Take the tallest box and position it on the right side. & 1/3 & 1/3 & 3/3 & \textbf{3/3}\\
    Grasp the error bottle and put it on the right side. & 1/3 & 2/3 & 1/3 & \textbf{2/3} \\
    Take out the green test tube and place it between the two bottles. & 2/3 & 2/3 & 3/3 & \textbf{3/3}\\
    Pack the objects on the table into the box one by one. & 1/3 & 1/3 & 0/3 & \textbf{1/3}\\
    Rotate the loopy doll to face the yellow dragon doll & 0/3 & 1/3 & 1/3 & \textbf{1/3}\\
    Right the fallen wine glass and arrange it neatly in a row. & 0/3 & 0/3 & 0/3 & \textbf{0/3}\\
    Grasp the handle of the knife and cut the bread.& 0/3 & 0/3 & 0/3 & \textbf{1/3}\\
    Pick the baseball into the cart and turn the cart to facing right. & 0/3 & 0/3 & 1/3 & \textbf{2/3}\\
    Place the mug on the left of the ball and the handle turn right. & 0/3 & 0/3 & 1/3 & \textbf{1/3}\\
    Aim the camera at the toy truck. & 1/3 & 0/3 & 1/3 & \textbf{1/3}\\
    Rotate the flashlight to illuminate the loopy. & 0/3 & 0/3 & 1/3 & \textbf{1/3}\\
    Put the pen into the pen container. & 0/3 & 1/3 & 0/3 & \textbf{1/3} \\
    Pour out chips from the chips cylinder to the plate. & 0/3 & 1/3 & 1/3 & \textbf{1/3} \\
    Total success rate& 20.0\% & 26.7\% & 33.3\% & \textbf{48.9\%} \\
    \bottomrule[0.95pt]
\end{tabular}
}
\end{table*}
\vspace{-10pt}
\begin{figure}[t!]
\centering
\includegraphics[width=1.0\linewidth]{figs/src/real_obj.pdf}
\captionof{figure}{\textbf{The real-world assets used in our real-world experiments.} More than 100 diverse objects are used in our 6-DoF rearrangement experiments.}
\label{fig:real_obj}
\vspace{-10pt}
\end{figure}


\input{tabs/PointSO_configurations}
\usepackage{graphicx}
\usepackage{url}
\usepackage{color, soul}
\usepackage{bm}
\usepackage{amsmath} % for the equation* environment
\usepackage{lineno}

\subsection{PointSO Model Details}\label{app:pointso_details}
For PointSO, we utilize FPS + KNN to perform patchify and employ a small PointNet~\cite{PointNet} as the patch encoder. Subsequently, a standard Transformer encoder is adopted as the backbone, followed by a single linear layer to map the output to a three-dimensional vector space. All parameter configurations follow prior work on point cloud representation learning~\cite{ACT23,ReCon23,ShapeLLM24}. Detailed hyperparameter and model configurations are provided in \cref{tab:hyper_params,tab:PointSO_configs}.

\subsection{SoFar-LLaVA Model Details}\label{app:model_details}
\begin{figure*}[t!]
\begin{center}
\includegraphics[width=0.89\linewidth]{figs/src/sofar_llava.pdf}
\caption{\textbf{Pipeline of \sofar-LLaVA}, a fine-tuned vision language model based on visual instruction tuning.
}
\label{fig:sofar_llava}
\vspace{-5pt}
\end{center}
\end{figure*}

In addition to leveraging the extensive knowledge and strong generalization capabilities of closed-source/open-source pretrained VLMs~\cite{ChatGPT22,gemini23,qwenvl23} for zero-shot or in-context learning, \sofar~can also enhance the planning performance of open-source models through visual instruction tuning for rapid fine-tuning. The pipeline of the model is illustrated in \cref{fig:sofar_llava}. A JSON-formatted 6-DoF scene graph, processed through a text tokenizer, along with the image refined by SoM~\cite{SoM23}, is fed into an LLM (\eg, LLaMA~\cite{LLaMA23,LLaMA2_23}) for supervised fine-tuning~\cite{LLaVA23}.
In the Open6DOR~\cite{Open6DOR24} task, we supplement the training dataset with additional samples retrieved and manually annotated from Objaverse~\cite{objaverse23}, ensuring alignment with the object categories in the original benchmark. This dataset includes approximately 3,000 6-DoF object manipulation instructions. Using this data, we construct dialogue-style training data based on ChatGPT and train the \sofar-LLaVA model. The training hyperparameters are detailed in \cref{tab:hyper_params}. Similarly, we finetune PointSO on this training dataset and achieve superior performance on the Open6DOR task.

\subsection{ChatGPT API Costs}
The knowledge of OrienText300K is derived from the annotations of 3D modelers on Sketchfab, combined with ChatGPT's filtering and comprehension capabilities. To generate semantic direction annotations, we filter the 800K dataset of Objaverse~\cite{objaverse23} and apply ChatGPT to approximately 350K of the filtered data to generate semantic text-view index pairs. The OpenAI official API was used for these calls, with the GPT-4o version set to 2024-08-06 and the output format configured as JSON. The total cost for debugging and execution amounted to approximately \$10K.


\section{Additional Benchmark Statistic Analysis}
\subsection{6-DoF SpatialBench Analysis}
We conduct a statistical analysis of the manually constructed 6-DoF SpatialBench, with category comparisons and word cloud visualizations shown in \cref{fig:spatialvqa_statistic}. We collect diverse image data from the internet, encompassing scenes such as indoor, outdoor, and natural landscapes. The questions may involve one or multiple objects, with varying levels of uncertainty in image resolution. Most importantly, we are the first to propose a VQA benchmark for orientation understanding, focusing on both quantitative and qualitative evaluation of orientation.


\subsection{Open6DOR V2 Analysis}
Open6DOR V2 builds upon Open6DOR V1 by removing some incorrectly labeled data and integrating assets and metrics into Libero, enabling closed-loop policy evaluation. The detailed number of tasks is presented in \cref{tab:open6dorv2_statistic}, comprising over 4,500 tasks in total. Notably, we remove level 2 of the position track in Open6DOR V1~\cite{Open6DOR24} because it requires manual inspection, which is not conducive to open-source use and replication by the community. Besides, due to the randomness of object drops in the scene, approximately 8\% of the samples already satisfy the evaluation metrics in their initial state.

\vspace{3pt}
\section{Additional Related Works}\label{app:related_work}
\subsection{3D Representation Learning}
Research on 3D Representation Learning encompasses various methods, including point-based~\cite{PointNet,PointNet++}, voxel-based~\cite{voxelnet15}, and multiview-based approaches~\cite{MVCNN3D15,MVTN}. 
Point-based methods~\cite{PointNext,PointTrans21} have gained prominence in object classification~\cite{ModelNet15,ScanObjectNN19} due to their sparsity yet geometry-informative representation. On the other hand, voxel-based methods~\cite{voxelrcnn21,SyncSpecCNN17,VPP23} offer dense representation and translation invariance, leading to a remarkable performance in object detection~\cite{ScanNet17} and segmentation~\cite{ShapeNetPart16, S3DIS16}.
The evolution of attention mechanisms~\cite{AttentionIsAllYouNeed,ReKo23} has also contributed to the development of effective representations for downstream tasks, as exemplified by the emergence of 3D Transformers~\cite{PointTrans21,groupfree21, voxeltransformer21}. Notably, 3D self-supervised representation learning has garnered significant attention in recent studies. PointContrast~\cite{PointContrast20} utilizes contrastive learning across different views to acquire discriminative 3D scene representations. Innovations such as Point-BERT~\cite{PointBERT} and Point-MAE~\cite{PointMAE} introduce masked modeling~\cite{MAE,BERT} pretraining into the 3D domain. 
ACT~\cite{ACT23} pioneers cross-modal geometry understanding through 2D or language foundation models such as CLIP~\cite{CLIP} or BERT~\cite{BERT}. 
Following ACT, {\scshape ReCon}~\cite{ReCon23} further proposes a learning paradigm that unifies generative and contrastive learning. PPT~\cite{ppt24} highlights the significance of positional encoding in 3D representation learning
Additionally, leveraging foundation vision-language models like CLIP~\cite{ACT23,CLIP} has spurred the exploration of a new direction in open-world 3D representation learning. This line of work seeks to extend the applicability and adaptability of 3D representations in diverse and open-world/vocabulary scenarios~\cite{OpenScene23,CLIPFO3D23,PLA23,Lowis3D23,OVIR3D23,PointGCC23}.

\section{Additional Discussions}
\subsection{Relation to Affordance \& 6-DoF Pose Estimation}
Conceptually, this semantic orientation is a counterpart of \textit{affordance}~\citep{Affordance77,AffordanceHRI16,MoveWithAffordanceMaps20,HandsAsAffordancesProbes22} but beyond,
as SO and affordance all present potential actions and interactions with objects.
However, SO also contains the spatial understanding of intra-object part-level attributes more than affordance learning.
Compared to vanilla 6-DoF pose estimation, our proposed SO combined with the 3-DoF translation understanding has the same DoF completeness.
The difference is, our proposed SO is grounded by languages, making it useful for open-world manipulation requiring complicated spatial reasoning~\cite{RobotsThatUseLanguage20,SayCan22,Open6DOR24}. 
In addition, our Semantic Orientation can be auto-labeled from Internet 3D data that achieves higher scalability, introduced in the next section.
\begin{figure}[t!]
\begin{center}
% \includegraphics[width=0.85\linewidth]{figs/src/spatialvqa_statistic.pdf}
\includegraphics[width=\linewidth]{figs/src/spatialvqa_statistic.pdf}
\vspace{-15pt}
\caption{\textbf{6-DoF SpatialBench statistics}. (a) Statistical analysis of the task type, question type, and object relation. (b) Word cloud visualization.}
\label{fig:spatialvqa_statistic}
\end{center}
\end{figure}



\subsection{Comparison to Concurrent Works}
\begin{figure*}[t!]
\includegraphics[width=\linewidth]{figs/src/simpler_visual.pdf}
\vspace{-15pt}
\caption{An example of \ours~how to finish ``move near'' task in SIMPLER~\cite{simplerenv24}.}
\label{fig:simpler_visual}
\end{figure*}

\subsubsection{Comparison with ReKep~\cite{ReKep24}}
Recently, ReKep has succeeded in executing complex robotic tasks, such as long-horizon manipulation, based on the relationships and constraints between spatial key points. 
Its structural design offers many insights that \sofar~can draw upon, yet it also presents several issues: 
(1) Overly customized prompt engineering. ReKep requires manually designed complex system prompts for each task during inference. 
While this approach may be described as ``no training'', it cannot be considered a true zero-shot transfer. In contrast, \sofar~achieves genuine zero-shot transfer by eliminating the need for any human involvement during inference; (2) Using constraints based solely on key points fails to capture the full 6-DoF pose integrity of objects. For example, in the ``pouring water'' task, merely bringing the spout of the kettle close to the cup may lead to incorrect solutions, such as the kettle overturning; (3) ReKep requires all key points to be present in the first frame, and each step of the process—from mask extraction to feature dimensionality reduction, clustering, and filtering—introduces additional hyperparameters.

\subsubsection{Comparison with Orient Anything~\cite{orient_anything24}}
Recently, Orient Anything also highlighted the importance of orientation in spatial perception and adopted a training data construction approach similar to Our PointSO. Our primary distinction lies in semantic orientation, which is language-conditioned orientation. In contrast, Orient Anything is limited to learning basic directions such as ``front'' and ``top''. By aligning with textual information, semantic orientation better enhances spatial perception, understanding, and robotic manipulation.

\subsection{Future Works}
Future work includes further expanding the OrienText300K with larger datasets like Objaverse-XL~\cite{ObjaverseXL23}, enhancing the performance of semantic orientation through self-supervised learning and pretraining methods~\cite{MAE,CLIP,ACT23,ReCon23}, and demonstrating its effectiveness in a broader range of robotic scenarios, such as navigation~\cite{GOAT24}, mobile manipulation~\cite{homerobot23}, lifelong learning~\cite{LIBERO23}, spatio-temporal reasoning~\cite{ReKep24,LeaFLF23,CrossVideoSC24,thinking24}, humanoid~\cite{OmniH2O24,SmoothHumanoidLCP24,Exbody24,humanup25}, and human-robot interaction~\cite{HOI4D22,InteractiveHO23}.

\begin{table*}[t!]
\centering
\caption{\textbf{Statistics of Open6DOR V2 Benchmark.} The entire benchmark comprises three independent tracks, each featuring diverse tasks with careful annotations. The tasks are divided into different levels based on instruction categories, with statistics demonstrated above.}
\label{tab:open6dorv2_statistic}
\resizebox{\textwidth}{!}{
\setlength{\tabcolsep}{3.5pt}
    \begin{tabular}{c|ccccc|cc|c|c|c|c|c}
    \toprule
        Track & \multicolumn{7}{c|}{Position-track} & \multicolumn{3}{c|}{Rotation-track} & 6-DoF-track & Totel\\
        \midrule
        Level  & \multicolumn{5}{c|}{Level 0} & \multicolumn{2}{c|}{Level 1}  & Level 0 & Level 1 & Level 2 & - & -\\
        \midrule
        Task Catog. &  Left & Right  & Top &  Behind &  Front & Between  & Center & Geometric & Directional & Semantic & - & - \\
        \midrule
        Task Stat. & 296 & 266 & 209 & 297 & 278 & 193 & 159 & 318 & 367 & 134 & 1810 & 4535\\
        
        \midrule
        Benchmark Stat. &\multicolumn{7}{c|}{1698} & \multicolumn{3}{c|}{1027} & 1810 & 4535\\
    \bottomrule 
    \end{tabular}
}
\end{table*}

\section{Additional Visualizations}\label{app:visualization}

\subsection{Robotic Manipulation}
As shown in \cref{fig:simpler_visual}, we present a visualization of executing a task named ``move near''.
According to the input image and task instruction - ``\textit{move blue plastic bottle near pepsi can}'', \ours~can predict the center coordinate of the target object (bottle) and relative target (pepsi can), and it would infer the place coordinate and produce a series of grasp pose.

\subsection{6-DoF SpatialBench}
To further evaluate 6-DoF spatial understanding, we construct a 6-DoF SpatialBench.
We present examples of question-answer pairs from the 6-DoF SpatialBench, with quantitative and qualitative questions shown in \cref{fig:spatialbench_show1,fig:spatialbench_show2}, respectively. 
The benchmark we constructed is both challenging and practical, potentially involving calculations based on the laws of motion, such as ``\textit{Assuming a moving speed of 0.5 m/s, how many seconds would it take to walk from here to the white flower?}'' Moreover, it covers a wide range of spatially relevant scenarios across both indoor and outdoor environments.


\subsection{System Prompts}
Prompt engineering significantly enhances ChatGPT's capabilities. The model's understanding and reasoning abilities can be greatly improved by leveraging techniques such as Chain-of-Thought~\cite{CoT22} and In-Context Learning~\cite{GPT3_20}. \cref{fig:filter_prompt,fig:instruction_prompt} illustrate the system prompt we used in constructing OrienText300K.
\cref{fig:open6dor_prompt}, \cref{fig:manip_prompt}, and \cref{fig:vqa_prompt} illustrate the system prompt we used when evaluating \sofar on Open6DOR (simulation), object manipulation (both simulation and real worlds), and VQA, respectively.
Note that different from previous methods~\cite{VoxPoser23,ReKep24}, \sofar does not require complicated in-context examples.


\begin{figure*}[h!]
\centering
\includegraphics[width=0.97\linewidth]{figs/src/show1.pdf}
\captionof{figure}{\textbf{Visualization example of 6-DoF SpatialBench data samples.}
% 6-DoF SpatialBench includes complex spatial reasoning of absolute numbers.
}
\label{fig:spatialbench_show1}
\end{figure*}

\begin{figure*}[h!]
\centering
\includegraphics[width=0.97\linewidth]{figs/src/show2.pdf}
\captionof{figure}{\textbf{Visualization example of 6-DoF SpatialBench data samples.}}
\label{fig:spatialbench_show2}
\end{figure*}

\begin{figure*}[h!]
\centering
\includegraphics[width=0.97\linewidth]{figs/src/filter_prompt.pdf}
\captionof{figure}{\textbf{The system prompt of ChatGPT-4o used for filtering Objaverse data.}}
\label{fig:filter_prompt}
\end{figure*}

\begin{figure*}[h!]
\centering
\includegraphics[width=0.97\linewidth]{figs/src/instruction_prompt.pdf}
\captionof{figure}{\textbf{The system prompt of ChatGPT-4o used for generating Semantic Direction-Index pairs.}}
\label{fig:instruction_prompt}
\end{figure*}

\begin{figure*}[h!]
\centering
\includegraphics[width=0.97\linewidth]{figs/src/open6dor_prompt.pdf}
\captionof{figure}{\textbf{The system prompt of Open6DOR tasks.}}
\label{fig:open6dor_prompt}
\end{figure*}

\begin{figure*}[h!]
\centering
\includegraphics[width=0.97\linewidth]{figs/src/manip_prompt.pdf}
\captionof{figure}{\textbf{The system prompt of general manipulation tasks.}}
\label{fig:manip_prompt}
\end{figure*}

\begin{figure*}[h!]
\centering
\includegraphics[width=0.97\linewidth]{figs/src/vqa_prompt.pdf}
\captionof{figure}{\textbf{The system prompt of visual-question-answering tasks.}}
\label{fig:vqa_prompt}
\end{figure*}


\end{document}
