% \section{code refinement}

% 在一次code review过程中,首先由developer提交了一份修改,用最初的代码C0修改到代码C1。而后,reviewer针对这次修改(C0->C1)提出了review comment(RC),无论是reviewer提交review,还是developer查看review,RC会呈现在某一行代码之后,我们称这一行代码是review line(RL)。通常review line是C0->C1的修改部分,review对上次的修改进行评价和建议,也有少部分情况review line是针对未修改代码提出新的修改建议。而后developer根据RC,对C1进行修改,得到新的代码版本C2。传统的code refinement任务我们称之为Basic Code Refinement,其input为:<C1, RC>,output为<C2>。然而,我们注意到一些数据只提供C1, RC是信息不足的。如图1所示,例子a需要提供review line才能确定review要删除哪一行。例子2中,需要指导C0才能确定如何回退代码。我们定义两种新型的任务,Position-Aware Code Refinement: 其输入为<C1,RC,RL>,输出是<C2>;	Comprehensive Code Refinement:输入为<C0,C1,RC,RL>,输出也是<C2>。本文主要研究的对象就是Comprehensive Code Refinement。
% 为了避免文字混淆,我们称C0版本的代码是initial code,C1版本的代码是original code,C2版本的代码是revised code。



% 目前coderefine方向有两个使用广泛的数据集,Tufano数据集和codereview数据集。如前文介绍的,我们需要数据集提供initial code, original code, review line, review comment, revised code等五个字段。Tufano数据集没有initial code字段,且没有提供原始数据的链接。而codereview数据集虽没有review line,initial code两个字段,但是提供了原始的数据连接,故而我们选择使用codereview数据,并补齐缺失字段。

% 首先介绍review line字段获取方法。我们观察到通过GitHub REST API获取code review信息时,可以获取到partial last code diff(在API返回的json中叫做diff hunk字段,给个角标https://api.github.com/repos/meganz/sdk/pulls/comments/326107667)。之所以我们称之为partial last code diff,是因为这段code diff只提供了review comment之前的修改信息。在例子中,原本的last code diff有三行代码删除,三行代码添加。review line在第一行代码添加之后。所以partial last code diff只有三行删除,一行添加。根据这个规律,我们可以得到review line就是partial last code diff的最后一行。

% 而后我们需要设计得到initial code的方法。我们观察到,一次code review可能是reviewer对前面多次commit的review。如果当前pull request有n个commit:(commit_1,commit_2,...,commit_n),reviewer可以选择commit_m到commit_n之间所有commit(n小于等于m),然后查看这些commit叠加后的文件变化,而后再给出review comment。而GitHub REST API只给出了最后一次commit的id,即commit_n,无法确定前面的commit_m。不过GitHub REST API中提供的partial alst code diff就是commit_m到commit_n的code diff。故而,我们倒序依次遍历前面的所有commit,并与commit_n做比较,得到review line附近的code diff。并与partial last code diff做对比,就可以找到与partial last code diff一致的,完整的last code diff。根据last code diff和original code就可以得到initial code。