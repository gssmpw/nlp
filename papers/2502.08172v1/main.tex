
\documentclass[10pt,conference,nocompress]{IEEEtran}
% \documentclass[10pt,conference,review]{IEEEtran}
\pdfoutput=1

\IEEEoverridecommandlockouts
% 投稿时添加
% \settopmatter{printacmref=false, printccs=false, printfolios=true}
% \renewcommand\footnotetextcopyrightpermission[1]{}
\usepackage[switch]{lineno}
\usepackage{hyperref}
\hypersetup{hidelinks}
\usepackage{graphicx} 
\usepackage{multirow}
\usepackage{tabularx}
\usepackage{url}
\usepackage{cite}
\usepackage{pifont}
\usepackage{xcolor}
\usepackage{enumerate}
\usepackage{hyperref}
\usepackage{enumitem}
\usepackage{caption}
% \usepackage[table,xcdraw]{xcolor} % 导入xcolor宏包
%%
%% end of the preamble, start of the body of the document source.
\begin{document}
\author{
    \IEEEauthorblockN{
    Qi Guo\textsuperscript{1}\IEEEauthorrefmark{1},
    Xiaofei Xie\textsuperscript{2},
    Shangqing Liu\textsuperscript{3}\IEEEauthorrefmark{2},
    Ming Hu\textsuperscript{2},
    Xiaohong Li\textsuperscript{1}\IEEEauthorrefmark{2},
    and Lei Bu\textsuperscript{3}
    \thanks{\IEEEauthorrefmark{1} This work was done during Qi Guo's visit to Singapore Management University.}
    \thanks{\IEEEauthorrefmark{2} Shangqing Liu and Xiaohong Li are corresponding authors.}}
    \IEEEauthorblockA{
    \textsuperscript{1} Tianjin University, P.R. China\\
    \textsuperscript{2} Singapore Management University, Singapore\\
    \textsuperscript{3} State Key Laboratory for Novel Software Technology, Nanjing University, P.R. China\\
    }
    \IEEEauthorblockA{bxguoqi@tju.edu.cn,
    xfxie@smu.edu.sg,
    shangqingliu@nju.edu.cn,\\
    ecnu\_hm@163.com,
    xiaohongli@tju.edu.cn,
    bulei@nju.edu.cn
    }
}
\makeatletter
\patchcmd{\@maketitle}
  {\addvspace{0.5\baselineskip}\egroup}
  {\addvspace{-1\baselineskip}\egroup}
  {}
  {}
\makeatother

%% The "title" command has an optional parameter,
%% allowing the author to define a "short title" to be used in page headers.
\title{Intention is All You Need: Refining Your Code from Your Intention}
% \author{Anonymous Author(s)*}
% \title{An Intention-Based Framework for Code Refinement}
\newcommand{\fei}[1]{{\textcolor{red}{Fei:#1}}}
\newcommand{\sql}[1]{\textcolor{red}{sql: #1}}
\newcommand{\guo}[1]{\textcolor{cyan}{#1}}
\newcommand{\major}[1]{\textcolor{black}{#1}}
%%
%% The abstract is a short summary of the work to be presented in the
%% article.
\maketitle



% % This is samplepaper.tex, a sample chapter demonstrating the
% LLNCS macro package for Springer Computer Science proceedings;
% Version 2.21 of 2022/01/12
%
\documentclass[runningheads]{llncs}
%
\bibliographystyle{splncs04}
\usepackage[T1]{fontenc}
% T1 fonts will be used to generate the final print and online PDFs,
% so please use T1 fonts in your manuscript whenever possible.
% Other font encondings may result in incorrect characters.
%
\usepackage{graphicx}
% Used for displaying a sample figure. If possible, figure files should
% be included in EPS format.
%
% If you use the hyperref package, please uncomment the following two lines
% to display URLs in blue roman font according to Springer's eBook style:
%\usepackage{color}
%\renewcommand\UrlFont{\color{blue}\rmfamily}
%\urlstyle{rm}
%
\usepackage{url}
\begin{document}
%
\title{Modeling the Training of Human and GPT-4 Social Engineering Attack Recognition}
%\title{Modeling and Evaluating Human and GPT-4 Social Engineering Attack Detection}
%Using Cognitive Models to Improve Training Against Human and GPT-4 Generated Social Engineering Attacks}
% Modeling the Training of Human and GPT-4 Social Engineering Attack Recognition
\titlerunning{Human and GPT-4 Social Engineering Attacks}
%
%\titlerunning{Abbreviated paper title}
% If the paper title is too long for the running head, you can set
% an abbreviated paper title here
%
\author{Tyler Malloy \and
Maria José Ferreira \and 
Fei Fang \and
Cleotilde Gonzalez}
%
\authorrunning{Malloy et al.}
\institute{Carnegie Mellon University, Pittsburgh PA 15222, USA}
%
\maketitle

\begin{abstract}
    Social engineering attacks remain a critical tool for cybercriminals seeking to exploit sensitive data. Although the threat of AI-generated content in such attacks is growing, current training methods predominantly rely on simplistic human-designed emails. This research introduces a novel experimental paradigm to investigate differences in the detection of human-generated versus AI-generated phishing emails. Our behavioral results reveal that emails co-created by humans and Generative-AI models pose a greater challenge to end users compared to those emails created by GPT-4 or Human only. We also propose a cognitive model that predicts user behavior during training, which offers the potential to be used in future training frameworks to improve training effectiveness. Our work contributes by (1) identifying critical weaknesses in current social engineering training and (2) proposing a cognitive model-driven solution to better equip users against evolving threats.
\end{abstract}

\section{Introduction}
Social engineering attacks are commonly used by cyber criminals to gain access to valuable and sensitive data. Recent Large Language Models (LLMs) such as GPT-4 have demonstrated the ability to produce convincing text that mimics human writing, and code that could be used to create fake emails and websites that appear to be legitimate. Research in cybersecurity has identified the risks of increased proliferation of social engineering attacks through the use of LLMs \cite{schmitt2024digital}. However, the efficacy of LLM-generated emails in training users against social engineering attacks has not been evaluated. Many training programs are based on simple human-designed emails in classroom-style instruction delivery \cite{wen2019hack}. In this work, we propose the use of GPT-4 to write convincing text that mimics real emails, as well as HTML and CSS code to stylize emails. To our knowledge, this is the first study designed to establish the efficacy of GPT-4-generated emails compared to those written by humans. We also evaluate the efficacy of emails that are co-created by humans and styled by GPT-4. 

Our research introduces an experimental paradigm to determine whether there is a difference in end user detection using human-written and GPT-4 generated emails. This was done in a two-by-two design that varied the original author of the email text (Human or GPT-4) as well as the style of the email (Plain-text or HTML/CSS). A pre-experiment quiz on the indicators of phishing emails served as a measure of the base phishing knowledge of participants, and a post-experiment questionnaire had participants indicate what proportion of the content they observed was generated by AI. Participants observed exclusively human-written or GPT-4 generated emails.

The results of the experiment show that emails written by humans and stylized using HTML/CSS code generated by GPT-4 are the most challenging for end users, with a significant interaction effect leading to the GPT-4 written and HTML/CSS stylized emails being the easiest for participants to categorize. Analysis of the performance of participants based on their perception of content as AI-written demonstrates a significant bias by which participants rate more emails as phishing if they believe a higher proportion of emails were generated by AI. This effect represents a novel \textit{AI-writing bias} that leads participants to assume that AI-written emails are phishing attempts. This bias is closely related to the well-studied phenomenon of algorithm aversion. Participants who had less initial knowledge of phishing emails performed worse on average under all experiment conditions compared to participants who performed better on the initial phishing quiz. These two groups, participants who have less initial knowledge about phishing and those who perceive all AI-written content as being more likely to be phishing, could improve their performance through a better method of selecting emails to show to participants.

Alongside this experiment, we propose an Instance-Based Learning (IBL) cognitive model that uses GPT-4 embeddings of emails as attributes to predict the user's behavior in the email categorization task. 
%This cognitive model can be used to predict the categorization of the emails of the participants and determine the optimal email to show to that participant. 
%This is done by iterating over all possible emails that could be shown to a participant, and selecting the email that has the highest probability of incorrect categorization by that participant at that time in the training. This is inspired by the intuition that more difficult emails will expand participant's ability to categorize a wider variety of emails. 

We demonstrate that the IBL model is capable of accurately predicting the user's classification. We also run a simulation study to demonstrate how the model could be used to predict the categorization of a user and, by this prediction, select an optimal email to show to that participant to optimize their training.

%This is done by iterating over all possible emails that could be shown to a participant, and selecting the email that has the highest probability of incorrect categorization by that participant at that time in the training. This is inspired by the intuition that more difficult emails will expand participant's ability to categorize a wider variety of emails. 
%results that predict the potential improvement in participant training outcomes through this email selection method.  

\section{Background}
Generative Artificial Intelligence (GAI) has the potential to improve education and training in a variety of settings through increased accessibility and reduced costs (for a review, see \cite{baldassarre2023social}. However, there are significant ethical concerns due to the potential negative societal impacts of these models being misused \cite{bommasani2021opportunities}, such as through the generation of social engineering attacks \cite{al2023chatgpt}. One commonly used and widely available class of GAI are pre-trained Large Language Models (LLMs) that can be prompted to produce highly convincing textual outputs that resemble human writing \cite{sejnowski2023large}. While these methods are trained to avoid producing potentially harmful content, they can be repeatedly prompted when changing the initial prompt or continuing with different prompts, in an effort to produce desired outputs \cite{white2023prompt}. The design of the prompts that are input into LLMs to produce text is call \textit{prompt engineering}, and can be used to improve the quality of the LLM output \cite{chen2023unleashing}. The repeated prompting of LLMs has been applied onto predicting how humans may speed up learning through the use of natural language instructions that can be used to inform the predicted value of actions without needing experience of performing those actions in a specific environment state \cite{mcdonald2023exploring}.

LLMs such as the Generative Pretrained Transformer 3 (GPT-3) \cite{brown2020language} have been evaluated in their social engineering ability and have shown lower performance in designing social engineering attacks compared to humans \cite{sharma2023well}. The ability of these models is constantly evolving, putting into question the ability of newer models to design social engineering attacks \cite{kumar2023certifying}. While more advanced models may be able to produce more human-like text, they also have more advanced methods to prevent misuse. This work seeks to evaluate the newer GPT-4 model \cite{achiam2023gpt} in its ability to design phishing emails, as well as to compare the effectiveness of social engineering attacks designed by humans and LLM alone and emails generated by different combinations of the output of the human and LLM model. 

This work introduces an experimental paradigm for evaluating the potential harm of LLM use in one specific area, social engineering attacks. This experimental paradigm is used to compare social engineering attacks in the form of phishing emails that are either fully written by human cybersecurity experts, fully written by GPT-4, or a combination of the two through prompt engineering. Alongside this experiment, we propose a method to mitigate the potential misuse of LLMs in cybersecurity contexts by improving training against social engineering attacks. This is done by using a cognitive model to trace and predict individual learning progress and determine the best educational examples to show to participants. 

Overall, the contributions of this work are, first, the outline of some limitations to current social engineering training methods and, second, the identification of a potential solution to these limitations through the use of a cognitive model to improve learning outcomes. A novel bias is presented, in which participants assumed that AI-written emails are more likely to be phishing, leading to worse categorization performance. We show through simulation that selecting educational example emails using an IBL cognitive model reduces the effect of the AI-writing bias we demonstrate. These results show the usefulness of cognitive models in predicting the learning progress of end users in training scenarios, and the difficulty of correctly identifying phishing emails that are written by humans and then stylized by GPT-4.

\subsection{Large Language Models and Social Engineering Attacks}
The use of LLMs in the production of social engineering attacks demonstrates a significant concern for cybersecurity \cite{gupta2023chatgpt}. The simplicity of Generative AI tools makes them easy to apply to tasks such as writing phishing emails from scratch or stylizing existing phishing emails to look more convincing, potentially increasing their effectiveness \cite{sharma2023well}. Modern LLMs are even capable of producing code \cite{khan2022automatic}, such as Javascript, HTML, and CSS, \cite{lajko2022towards} that can create highly convincing emails that resemble real emails sent from many companies \cite{park2024ai}. This adds an additional layer to the potential misuse of LLMs in social engineering attacks, as hand-writing code for realistic looking emails would normally take minutes or hours, and can be done in seconds with LLMs. These two areas, writing original phishing emails and stylizing emails with HTML and CSS code, are the main focus of our experiment to investigate how users may be susceptible to social engineering attacks from humans and LLMs. 

One method of reducing the potential harm of LLMs is through the use of specific training that can make LLMs less likely to produce harmful content \cite{cao2023defending}. This is typically done using feedback from humans, either machine learning engineers or crowd-sourced participants in user studies \cite{bai2022training}. This can train models to avoid producing content that is designed to trick or scam users, such as phishing emails. However, the effectiveness of these methods in preventing the generation of dangerous content forms is not perfect and can often be worked around with more complex prompt engineering \cite{fredrikson2015model}. More advanced prompting can also train a separate model to adjust the prompt until it is accepted by the LLM and the desired content is produced \cite{zou2023universal}. In this work, we focus on using relatively simple prompt engineering to faithfully replicate what we view as a realistic scenario of a cyber attacker applying an LLM to write a phishing email. 

\subsection{Social Engineering Training}
Training end users to identify social engineering attacks is an important part of cybersecurity \cite{back2021cyber}. Users without experience in security are vulnerable, making them the `weakest link' of cyber defense \cite{vishwanath2022weakest}. Phishing emails are an especially common method of social engineering due to the high volume of emails sent daily and the potential for compromising systems provided by redirecting users to unintended websites, among other methods \cite{gupta2016literature}. Typically, training users to identify phishing emails focuses on specific features of these emails that can indicate that they are phishing attempts, such as the use of urgent language; making requests of confidential information; making an offer; containing a link to a dangerous website; among other features \cite{kumaraguru2009school}. In the past, this has been done using plain text emails written by human cybersecurity experts \cite{weaver2021training}. These training paradigms are a large industry and are commonly required by individuals, universities, companies, and other groups that are interested in improving the ability of end users to identify phishing emails \cite{jampen2020don}. 

\begin{figure}[t!] 
\begin{centering}
  \includegraphics[width=\textwidth]{Figures/Trial.png} 
  \caption{An example of the email identification task shown to participants}\label{fig:Trial}
 \end{centering} 
\end{figure}

Given the ever-updated nature of phishing attempts and the ease of use of LLMs in creating social engineering attacks, it is important to understand how users make decisions and learn from examples of emails written or stylized by LLMs. The intelligence selection of training examples shown to students has been shown to improve their learning outcomes \cite{ferguson2006improving}. This can be done by applying cognitive modeling methods to predict participant learning and decision making \cite{feng2011student}. In this work, these cognitive models are adjusted to reflect human behavior and serve as a baseline that can test various methods to improve end-user training on the identification of phishing emails. 

\subsection{Cognitive Modeling}
Cognitive models have previously been applied to predict human learning in anti-phishing training \cite{singh2023cognitive}. Recently, Generative AI models have been integrated with cognitive models by forming \textit{representations}, of stimuli, such as textual information using LLM embeddings \cite{malloy2024applying}, \cite{malloy2024leveraging}. This approach has demonstrated human-like abilities to recognize new stimuli, even when they are informationally complex, based on past experiences \cite{malloy2024efficient}. We propose the use of LLM embeddings as attributes of a cognitive model to both predict student learning and evaluate them under different experimental conditions. These same models are also used to simulate possible improvements in phishing education that can be afforded by intelligently selecting email examples. 

An Instance-Based Learning (IBL) model is used to both predict human learning in each condition of our experiment, and simulate the potential improvement of human learning afforded by an intelligence selection of example emails. Using LLMs to form representations of emails allows us to use the same representation method in experimental conditions. Comparing the accuracy of the IBL model in predicting human behavior across conditions allows us to assess how effectively it can be used to predict general human behavior. Additionally, we perform a simulation of these IBL models that fit human learning and decision making that allows us to evaluate methods of improving user learning in the identification of phishing emails. These simulation results provide evidence for our proposed method of improving cognitive-based training to make participant learning outcomes as efficient and effective as possible. 

\subsection{Instance Based Learning}
IBL models work by storing instances $i$ in memory $\mathcal{M}$, composed of utility outcomes $u_i$ and options $k$ composed of features $j$ in the set of features $\mathcal{F}$ of environmental decision alternatives. In the case of predicting student learning from phishing emails, these options include labeling an email as being either dangerous (phishing) or benign (ham), the features correspond to the attributes of the email that are relevant for determining if it is a phishing email, in our model the LLM embeddings, and the outcome corresponds to the point feedback provided to students depending on whether they are correct (1 point) or incorrect (-1 points). These options are observed in an order represented by the time step $t$, and the time step in which an instance occurred is given $\mathcal{T}(i)$. When tracing human participant performance, the memory is composed of the options presented to participants, the options that they selected, and the utility reward that was presented to them. 

To model the retrieval of instances in memory when calculating the expected value of different option alternatives, IBL models calculate the activation of each instance in memory based on the current options available. In calculating this activation, the similarity between instances in memory and the current instance is represented by adding the value $S_{ij}$ over all attributes, which is the similarity of the attribute $j$ of instance $i$ to the current state. This gives the activation equation as: 
 
\begin{equation}
A_i(t) = \ln \Bigg( \sum_{t' \in \mathcal{T}_i(t)} (t - t')^{-d}\Bigg) + \mu \sum_{j \in \mathcal{F}} \omega_j (S_{ij} - 1) + \sigma \xi
\label{eq:activation}
\end{equation}
The parameters of the IBL model can either be fit to individual human performance, or set to their default values. These parameters are the decay parameter $d$; the mismatch penalty $\mu$; the attribute weight of each $j$ feature $\omega_j$; and the noise parameter $\sigma$. The default values for these parameters are $(d,\mu,\omega_j,\sigma) = (0.5, 1, 1, 0.25)$. The IBL models in this work use default values to predict individual student behaviors. The value $\xi$ is drawn from a normal distribution $\mathcal{N}(-1,1)$ and multiplied by the noise parameter $\sigma$ to add random noise to the activation. Varying these parameters impacts which instances are retrieved, and ultimately how the predicted utility of option alternatives is calculated.  

When predicting human learning and decision making based on textual information such as phishing emails, it is possible to use LLMs to form embeddings of these emails as attributes of the IBL model \cite{malloy2024applying}. To calculate the similarity metric $S_{ij}$ between two emails, we use the cosine similarity of their embeddings, as is done in \cite{malloy2024leveraging}. In this work, this has the benefit that the same method of forming attributes from emails can be used across experimental conditions. Thus, we can assess the effectiveness of an IBL+LLM cognitive model in predicting human learning and decision making during training. 

The blended value of an option $k$ is calculated at time step $t$ according to the utility outcomes $u_i$ weighted by the probability of retrieval of that instance $P_i$ and summing over all instances in memory $\mathcal{M}_k$ to give the equation:
\begin{equation}
V_k(t) = \sum_{i \in \mathcal{M}_k} P_i(t)u_i
\label{eq:blending}
\end{equation}

Where $P_i(t)$ is the probability of retrieval, calculated by an inverse-temperature weighted soft-max of all available instance activations. 

\begin{figure}[t!] 
\begin{centering}
  \includegraphics[width=0.7\textwidth]{Figures/Emails.png} 
  \caption{Top-Left: The original plain-text email written by human experts Bottom-Left: The GPT-4 stylized version of this original email. Bottom-Right: The fully GPT-4 rewritten and stylized version of the email. Top-Right: The stripped plain-text version of the fully GPT-4 rewritten email.}\label{fig:Emails}
 \end{centering} 
\end{figure}

\section{Experiment}
The recent proliferation of phishing emails written or styled by large language models (LLMs) brings into question our understanding of how users make judgments of phishing emails and how these judgments compare between human and LLM written content. These LLM written emails can either be fully authored by humans, by LLMs, or a combination of the two where a human creates one of either the text body or styling, and the LLM creates the other. To test these different options of generating emails, we use a 2x2 design varying author (Human or GPT-4) or style (plain-text or GPT-4 stylized). We designed an experiment to collect human judgments of phishing (dangerous) and ham (safe) emails and varied the author (Human or GPT-4) and style (Plain-text or Styled) in a between-subjects 2x2 design. 

An example of the experimental interface used to evaluate the identification training of phishing emails is shown in Figure \ref{fig:Trial}. In this example, the email being shown is a human-written and plain-text styled email. Importantly, for each experimental condition, the same set of 360 emails was used, all based on the original dataset of plain-text emails written by human cybersecurity experts that was used in a previous study \cite{singh2023cognitive}. These base emails were then either stylized by GPT-4, or rewritten entirely by prompting GPT-4 to write an email with the same attributes that the experts coded the original emails as having. The fully GPT-4 rewritten email is also stripped of HTML and CSS code and presented as the plain-text version of the GPT-4 written email. This resulted in 4 sets of 360 emails with the same general features and topics in each set. Figure \ref{fig:Emails} shows the same email that is stylized, fully rewritten, and the plain-text version of that email. 

\subsection{Methods}
This experiment compares human learning and decision making when categorizing emails as phishing (dangerous) or ham (safe) depending on the email author (Human or GPT-4) and style (plain-text or GPT-4 stylized). We are interested in determining which condition is the most difficult for humans to make accurate judgments in and whether there is a relationship between participant confidence, reaction time, and accuracy. This is an important potential relationship as it can aid in our overall goal of improving the quality of example emails shown to participants based on their performance.

\begin{figure}[t!] 
\begin{centering}
  \includegraphics[width=0.7\textwidth]{Figures/BarPerformance.png} 
  \caption{Pre and post-training categorization accuracy for ham and phishing emails by experimental condition.}\label{fig:BarPerformance}
 \end{centering} 
\end{figure}

This experiment included 10 pre-training trials without feedback, 40 training trials with feedback, and 10 post-training trials without feedback. During all trials, participants made judgments about emails as phishing or ham and indicated their confidence in their judgment. We recruited 268 participants online through the Amazon Mechanical Turk (AMT) platform. Of these participants, 44 did not complete all 60 trials and were excluded from further analysis. Of the remaining 224 participants, 18 were removed due to poor performance in the categorization task, as predefined in the study preregistration. This predefined criterion removed all participants who performed less than two standard deviations below the mean categorization improvement between pre-training and post-training trials. 

This exclusion resulted in a total of 207 participants used for the following analysis. Participants (69 Female, 137 Male, 1 Non-binary) had an average age of 40.02 with a standard deviation of 10.48 years. Of these participants, 25 had never received a phishing email, 101 had received phishing emails on a few occasions, and 79 had received phishing emails on many occasions.  Participants were compensated with a base payment of \$3 with the potential to earn up to a \$12 bonus payment depending on performance. This experiment was approved by the Carnegie Mellon University Institutional Review Board, and the study was pre-registered on OSF\footnote{\url{https://osf.io/wbg3r/}}. All participant data and analysis code is available on OSF. 

\subsection{Results}
The primary comparison between conditions is done in terms of the improvement in categorization accuracy percentage between the 10 pre-training trials and the 10 post-training trials. These results are shown in Figure \ref{fig:BarPerformance}, with the pre-training performance lightly shaded and the post-training performance a darker shade. The only decrease in performance between pre and post-training was in the Human written and GPT-4 styled ham email categorization. 

A mixed repeated measure analysis of variance of the effect of the author of the email and the style of the email on the improvement of categorization demonstrated no significant variation in author ($F=1.101,p=0.295,\eta_p^2=0.005$) but a significant variation of style ($F=12.261$, $p=0.001$, $\eta_p^2=0.057$) as well as a significant interaction between author and style ($F=14.344$, $p<0.001$, $\eta_p^2=0.066$). A post-hoc multi-comparison Tukey test showed that the improvement of the human subject in the human written and GPT-4-styled condition had a significantly lower improvement from the prior training to the post-training categorization accuracy ($p=0.033$) when compared to the GPT-4-written and GPT-4-styled condition. All other comparisons between conditions did not show a significant difference in the effect. This indicates that the smallest improvement in participant categorization accuracy was the Human written and GPT-4 styled condition ($\mu=0.015$) while the largest improvement was in the GPT-4 written and styled condition ($\mu=0.104$).

These results demonstrate the difficulty of training participants to identify emails that were written by human cybersecurity experts and stylized by GPT-4. Interestingly, the highest accuracy for the detection of phishing emails after training was observed with the written and styled by GPT-4. This is potentially due to the safety methods built into the GPT-4 model which could have hindered the model's ability to write convincing phishing emails. Alternative approaches to the GPT-4 model prompting, such as prompt attack, could produce more convincing phishing emails, though these complex methods may be outside of the skill set of most cybersecurity attackers.    

The results of this analysis indicate that GPT-4 stylized human-written phishing emails present the most challenging learning and decision-making paradigm. There was a strong interaction effect between the author of the email and the style, whereby the author was less relevant in plain-text emails, but became significant in stylized emails. This is crucial to our understanding of phishing email training, since many existing platforms still use plain-text emails in training examples.

\subsection{Participant AI Identification}
\begin{figure}[t!] 
\begin{centering}
  \includegraphics[width=\textwidth]{Figures/AIBias.png} 
  \caption{Linear regression comparing the percentage of emails categorized as being phishing emails and the proportion of emails identified as being AI written. Regressions are split between each of the four experimental conditions. Shaded regions represent 95\% confidence intervals of linear regression with $R^2$ and slope labeled.}\label{fig:PerceptionCondition}
 \end{centering} 
\end{figure}
To capture human participant identification of how emails were created, they were asked four questions at the end of the experiment to estimate the number of emails that they saw that were AI generated. The next comparison we performed was to assess the overall probability of categorizing an email as phishing based on how likely a participant was to categorize an email as being phishing based on their identification of emails as AI-generated or created by humans.  

These results are shown in Figure \ref{fig:PerceptionCondition} which shows a regression of the average percent of emails classified as phishing, since half of all emails shown to the participants were phishing, a correct categorization of all emails would result in 50\% emails being classified as phishing. In general, the participants tended to categorize more than half of the emails they were shown as phishing emails. Additionally, there was an overall trend across each condition that the higher the proportion of emails identified as AI written, the higher the probability of categorizing any email as being phishing.

It may seem surprising that the increased perception of emails as written by an AI model would lead to this bias in categorizing emails as being phishing. However, people generally demonstrate a poor ability to detect AI-written content \cite{kobis2021artificial}, which could interact with general aversion to algorithms \cite{burton2020systematic}) which has been shown to be higher in people who have experience with algorithms making incorrect judgments \cite{dietvorst2015algorithm}. 

We can see from this regression that participants who identified emails as being AI written in both of the GPT-4 styled conditions were more likely to categorize emails as being phishing if they had a higher identification of emails as being AI written. This represents an important bias in participant identification of emails that could potentially be exploited by cybersecurity attackers. This further motivates the improvement of training for detecting social engineering attacks that are designed by both humans and LLMs. 

A comparison of the slopes of these regressions in Figure \ref{fig:PerceptionCondition} demonstrates that this effect of phishing categorization bias is not equal across conditions. Notably, the likelihood of categorizing emails as being phishing has both a higher slope and a higher $R^2$ for emails that were styled by GPT-4. Looking back to the four example emails shown in Figure \ref{fig:Emails}, we can see that both of the GPT-4 styled conditions include banners, logos, bold text and other styled text that may draw the attention of participants. It is likely that participants were attending to these more salient features in the GPT-4 styled conditions, which if perceived as being AI generated could bias participants into believing that emails are phishing. 

These comparisons demonstrate that there is a difference between experimental conditions in how identifying emails as being AI written impacts the likelihood of categorizing emails as being phishing. This has important implications for both understanding how participants make judgments of emails in different contexts, as well as how best to design training when incorporating LLMs into the design of example emails. It is important that participants not over attend to irrelevant features like the perception of content as being AI written, and focus on relevant features like the presence of offers or incorrect sender addresses. 

\begin{figure}[!t] 
\begin{centering}
\includegraphics[width=\textwidth]{Figures/Simulations.png} 
  \caption{All improvement measures refer to the percentage point difference between pre-training and post-training accuracy. Left: Human participants (pastel colors) compared to IBL model training (bright colors) improvement under randomized email selection. Right: Simulated IBL student model improvement under IBL teacher model email selection (dark colors) compared to IBL model training under randomized email selection (bright colors). Color indicates condition, shade indicates training method, error bars indicate standard deviation.}\label{fig:training}
 \end{centering} 
\end{figure}

\subsection{Cognitive Modeling}
Before describing our proposed method for improving phishing training against human and LLM attackers, we must determine the appropriate method of modeling human behavior in this task. The emails used in each condition have the same base email, a plain-text and human-written email that had hand-crafted attributes associated with it. An alternative to using these hand-crafted features is to use LLM embedding representations of emails. However, the complex nature of these embedding representations means they may be difficult to use in a cognitive model that seeks to reflect the realities of human cognition. 

To assess these two different approaches in their ability to model and predict human-like learning in this task, we compared an IBL model that used hand-crafted attributes with one that used LLM model embeddings (IBL+LLM). This was done using a model-tracing approach for each individual participant, which works by setting the IBL model to select the same choice made by an individual participant, and observing the same utility outcome from that choice that the participant observed. This allows us to compare the IBL and the performance of human participants with the same experience.  

\subsection{Proposed Phishing Training supported by IBL}
Our proposed method to improve the learning outcomes of phishing training is based on the use of an IBL model to perform model training during the experiment and select emails to show to participants based on that model. Specifically, this model will be trained on all trials of an experiment based on the emails shown to a participant. During the pre-training and post-training trial blocks, the emails will be selected randomly from all possible emails. Then, during the training block where participants receive feedback, the model will search through all possible emails to find the email with the highest probability of being incorrectly categorized. 

The theory behind this approach is that emails should be selected to show participants when there is a high probability that the participant will misclassify them. This can ensure that participants observe a diverse and challenging set of emails, based on their individual performance on past trials. Since we only have data from human participants in trials in which emails are selected at random, we instead compare these two email selection training approaches using IBL+LLM models. The average percentage point improvement in participant categorization accuracy between the pre-training and post-training trials is shown in the lighter shaded bars on the left column of Figure \ref{fig:training}.

IBL+LLM simulated students have the same training as the experiment, with 10 pre-training trials without feedback, 40 training trials with feedback, and 10 post-training trials without feedback. These IBL models trained with a random sampling of emails are compared to the same IBL models trained with emails selected by a separate IBL teacher model. This teacher model is structured in the same way as the IBL tracing models described in previous results. These IBL teachers predict the behavior of simulated IBL students. After each trial of the main training portion of the experiment, the IBL teacher model iterates over all emails that have not yet been shown to the IBL student, and selects the email that has assigns the highest expected utility to the incorrect categorization for that email. 

Other than this training period with emails selected by the IBL teacher, the same standard pre and post-training periods are performed with randomized emails. Results from this training method are shown on the right column of Figure \ref{fig:training}, and demonstrate a clear and significant improvement between the training outcomes, as measured by pre-post-training improvement in terms of percentage point accuracy, between the random email sampling and the IBL+LLM teacher sampling. This suggests that selecting emails to show students using an IBL teaching model may improve the quality of educational outcomes.  

Overall, this comparison of different methods to train simulated IBL+LLM student models provides support for our planned study that will use a IBL teacher model to select the emails that real human participants will observe. This future study will confirm the benefit afforded by using an IBL teacher model to trace the performance of human students and select emails to show to them that will maximize their learning outcomes. The selection of emails to choose those that are most difficult for an individual student effectively broadens the range of emails they experience in the training block when they are receiving feedback. 

\section{Discussion}
In this work, we present a method for assessing different potential uses of GPT-4 by human cyberattackers interested in crafting phishing emails. Results from this experimentation highlights an issue of current methods of training end users to identify phishing emails and improve cybersecurity. Alongside this, we present a proposed solution to the issues that we highlight, to improve the quality of phishing email identification training through the use of a cognitive model. This is done by using an Instance Based Learning model to select the emails that are shown to participants and improve their learning outcomes.

Several interesting and surprising results from analyses of human behavior were revealed in our experimental result. Firstly, the most significant different between any two conditions of the experiment was in the human-written and GPT-4-styled condition and the GPT-4-written and GPT-4-styled condition. Comparing pre-training performance and improvement in the plain-text styled conditions showed little difference between different email authors. This interaction demonstrates that the GPT-4 model is unlikely to write convincing phishing emails from scratch without more advanced prompt engineering.

Another important result from experimental analysis was the observed bias between the perception of emails as being generated by an AI model. As participants were more likely to perceive emails as being written or stylized by AI, the worse their performance in categorizing ham emails. It is possible that the presence of this bias could be incorporated into improved feedback to students, to point out that AI generated writing does not necessarily indicate that an email is phishing. 

Improving education of AI-generated content is an important step to preventing the misuse of LLMs in the future, by improving the public awareness of the capabilities of LLMs, and how best to detect when they are potentially being used for nefarious purposes. A significant area of research in machine learning is seeking to further the capabilities of LLMs, aligning their outputs to human goals and use cases, and make misuse more difficult. However, it is unlikely that a perfect model will ever be trained, as it is possible to train separate models to learn how to best prompt LLMs to allow for unintended use cases. Thus, proper education and training is a crucial step to reducing the potential harm of LLMs in the future. 

\section*{Acknowledgments}
This research was sponsored by the Army Research Office and accomplished under Australia-US MURI Grant Number W911NF-20-S-000, and the AI Research Institutes Program funded by the National Science Foundation under AI Institute for Societal Decision Making (AI-SDM), Award No. 2229881. Compute resources and GPT model credits were provided by the Microsoft Accelerate Foundation Models Research Program grant ``Personalized Education with Foundation Models via Cognitive Modeling"

\bibliography{springer}

\end{document}


During the early stages of interface design, designers need to produce multiple sketches to explore a design space.  Design tools often fail to support this critical stage, because they insist on specifying more details than necessary. Although recent advances in generative AI have raised hopes of solving this issue, in practice they fail because expressing loose ideas in a prompt is impractical. In this paper, we propose a diffusion-based approach to the low-effort generation of interface sketches. It breaks new ground by allowing flexible control of the generation process via three types of inputs: A) prompts, B) wireframes, and C) visual flows. The designer can provide any combination of these as input at any level of detail, and will get a diverse gallery of low-fidelity solutions in response. The unique benefit is that large design spaces can be explored rapidly with very little effort in input-specification. We present qualitative results for various combinations of input specifications. Additionally, we demonstrate that our model aligns more accurately with these specifications than other models. 

% OLD ABSTRACT
%When sketching Graphical User Interfaces (GUIs), designers need to explore several aspects of visual design simultaneously, such as how to guide the user’s attention to the right aspects of the design while making the intended functionality visible. Although current Large Language Models (LLMs) can generate GUIs, they do not offer the finer level of control necessary for this kind of exploration. To address this, we propose a diffusion-based model with multi-modal conditional generation. In practice, our model optionally takes semantic segmentation, prompt guidance, and flow direction to generate multiple GUIs that are aligned with the input design specifications. It produces multiple examples. We demonstrate that our approach outperforms baseline methods in producing desirable GUIs and meets the desired visual flow.

% Designing visually engaging Graphical User Interfaces (GUIs) is a challenge in HCI research. Effective GUI design must balance visual properties, like color and positioning, with user behaviors to ensure GUIs easy to comprehend and guide attention to critical elements. Modern GUIs, with their complex combinations of text, images, and interactive components, make it difficult to maintain a coherent visual flow during design.
% Although current Large Language Models (LLMs) can generate GUIs, they often lack the fine control necessary for ensuring a coherent visual flow. To address this, we propose a diffusion-based model that effectively handles multi-modal conditional generation. Our model takes semantic segmentation, optional prompt guidance, and ordered viewing elements to generate high-fidelity GUIs that are aligned with the input design specifications.
% We demonstrate that our approach outperforms baseline methods in producing desirable GUIs and meets the desired visual flow. Moreover, a user study involving XX designers indicates that our model enhances the efficiency of the GUI design ideation process and provides designers with greater control compared to existing methods.    



% %%%%%%%%%%%%%%%%%%%%%%%%%%%%%%%%%%%%%%%%%%%%%%%%%%%%%%
% % Writing Clinic Comments:
% %%%%%%%%%%%%%%%%%%%%%%%%%%%%%%%%%%%%%%%%%%%%%%%%%%%%%%
% % Define: Effective UI design
% % Motivate GANs and write in full form.
% % LLMs vs ControlNet vs GANs
% % Say something about the Figma plugin?
% % Write the work is novel or what has been done before
% % What is desirable UI and how to evalutate that?
% % Visual Flow - main theme (center around it)
% % Re-Title: use word Flow!
% % Use ControlNet++ & SPADE for abstract.
% % Write about input/output. 
% % Why better than previous work?
% %%%%%%%%%%%%%%%%%%%%%%%%%%%%%%%%%%%%%%%%%%%%%%%%%%%%%

% % v2:
% % \noindent \textcolor{red}{\textbf{NEW Abstract!} (Post Writing Clinic 1 - 25-Jun)}

% % \noindent \textcolor{red}{----------------------------------------------------------------------}

% % \noindent Designing user interfaces (UIs) is a time-consuming process, particularly for novice designers. 
% % Creating UI designs that are effective in market funneling or any other designer defined goal requires a good understanding of the visual flow to guide users' attention to UI elements in the desired order. 
% % While current Large Language Models (LLMs) can generate UIs from just prompts, they often lack finer pixel-precise control and fail to consider visual flow. 
% % In this work, we present a UI synthesis method that incorporates visual flow alongside prompts and semantic layouts. 
% % Our efficient approach uses a carefully designed Generative Adversarial Network (GAN) optimized for scenarios with limited data, making it more suitable than diffusion-based and large vision-language models.
% % We demonstrate that our method produces more "desirable" UIs according to the well-known contrast, repetition, alignment, and proximity principles of design. 
% % We further validate our method through comprehensive automatic non-reference, human-preference aligned network scoring and subjective human evaluations.
% % Finally, an evaluation with xx non-expert designers using our contributed Figma plugin shows that <method-name> improves the time-efficiency as well as the overall quality of the UI design development cycle.

% % \noindent \textcolor{red}{----------------------------------------------------------------------}


% \noindent \textcolor{blue}{\textbf{NEW Abstract!} (Pre Writing Clinic 9-July)}

% \noindent \textcolor{blue}{----------------------------------------------------------------------}

% \noindent Exploring different graphical user interface (GUI) design ideas is time-consuming, particularly for novice designers. 
% Given the segmentation masks, design requirement as prompt, and/or preferred visual flow, we aim to facilitate creative exploration for GUI design and generate different UI designs for inspiration.
% While current Vision Language Models (VLMs) can generate GUIs from just prompts, they often lack control over visual concepts and flow that are difficult to convey through language during the generation process. 
% In this work, we present FlowGenUI, a semantic map-guided GUI synthesis method that optionally incorporates visual flow information based on the user's choice alongside language prompts. 
% We demonstrate that our model not only creates more realistic GUIs but also creates "predictable" (how users pay attention to and order of looking at GUI elements) GUIs.
% Our approach uses Stable Diffusion (SD), a large paired image-text pretrained diffusion model with a rich latent space that we steer toward realistic GUIs using a trainable copy of SD's encoder for every condition (segmentation masks, prompts, and visual flow). 
% We further provide a semantic typography feature to create custom text-fonts and styles while also alleviating SD's inherent limitations in drawing coherent, meaningful and correct aspect-ratio text. 
% Finally, a subjective evaluation study of XX non-expert and expert designers demonstrates the efficiency and fidelity of our method.


% This process encourages creativity and prevents designers from falling into habitual patterns.


% ------------------------------------------------------------------
% Joongi Why is it important to create realistic GUI?
% I do not see how the Visual Flow given on the left hand side is reflected in the results on the right hand side. 
% I’d avoid making unsubstantiated claims about designers (falling into habitual patterns).
% The UIs you generate do not “align with users’ attention patterns” but rather try to control it (that’s what visual flow means)
% ------------------------------------------------------------------
% Comments - Writing Clinic - 9th July:
% Improve title. More names: FlowGen
% Figure 1: Use an inference time hand-drawn mask
% Figure 1: Show both workflows. Add a designer --> Input.
% Figure 1: Make them more diverse
% ------------------------------------------------------------------
% Designing graphical user interfaces (GUIs) requires human creativity and time. Designers often fall into habitual patterns, which can limit the exploration of new ideas. 
% To address this, we introduce FlowGenUI, a method that facilitates creative exploration and generates diverse GUI designs for inspiration. By using segmentation masks, design requirements as prompts, and/or selected visual flows, our approach enhances control over the visual concepts and flows during the generation process, which current Vision Language Models (VLMs) often lack.
% FlowGenUI uses Stable Diffusion (SD), a largely pretrained text-to-image diffusion model, and guides it to create realistic GUIs. 
% We achieve this by using a trainable copy of SD's encoder for each condition (segmentation masks, prompts, and visual flow). 
% This method enables the creation of more realistic and predictable GUIs that align with users' attention patterns and their preferred order of viewing elements.
% We also offer a semantic typography feature that creates custom text fonts and styles while addressing SD's limitations in generating coherent, meaningful, and correctly aspect-ratio text.
% Our approach's efficiency and fidelity are evaluated through a subjective user study involving XX designers. 
% The results demonstrate the effectiveness of FlowGenUI in generating high-quality GUI designs that meet user requirements and visual expectations.

% ---------------------------------------


%A critical and general issue remains while using such deep generative priors: creating coherent, meaningful and correct aspect-ratio text. 
%We tackle this issue within our framework and additionally provide a semantic typography feature to create custom text-fonts and styles. 


% %Creating UI designs that are effective in market funneling or any other designer-defined goal requires a good understanding of the visual flow to guide users' attention to UI elements in the desired order. 
% %While current largely pre-trained Vision Language Models (VLMs) can generate GUIs from just prompts, they often lack finer or pixel-precise control which can be crucial for many easy-to-understand visual concepts but difficult to convey through language. 
% % However, obtaining such pixe-level labels is an extremely expensive so we
% % For example - overlaying text on images with certain aspect ratios and two equally separated buttons 
% Additionally, all prior GUI generation work fails to consider visual flow information during the generation process. 
% We demonstrate that visual flow-informed generation not only creates more realistic and human-friendly GUIs but also creates "predictable" (how users pay attention to and order of looking at GUI elements) UIs that could be beneficial for designers for tasks like creating effective market funnels.
% In this work, we present a semantic map-guided GUI synthesis method that optionally incorporates visual flow information based on the user's choice alongside language prompts. 
% Our approach uses Stable Diffusion, a large (billions) paired image-text pretrained diffusion model with a rich latent space that we steer toward realistic GUIs using an ensemble of ControlNets. 
% % TODO: Mention it in 1 sentence:
% A critical and general issue remains while using such deep generative priors: creating coherent, meaningful and correct aspect-ratio text. 
% We tackle this issue within our framework and additionally provide a semantic typography feature to create custom text-fonts and styles. 
% To evaluate our method, we demonstrate that our method produces more "desirable" UIs according to the well-known contrast, repetition, alignment, and proximity principles of design. 
% % We further validate our method through comprehensive automatic non-reference and human-preference aligned scores. (TODO: Maybe Unskip if we get UIClip from Jason!)
% % TODO: Re-word this and only keep ideation cycles and time-efficiency.
% Finally, a subjective evaluation study of XX non-expert and expert designers demonstrates the efficiency and fidelity of our method.
% % improves the time-efficiency by quick iterations of the UI design ideation process.
% %Finally, an evaluation with xx non-expert designers using our contributed <method-name> improves the time-efficiency by quick iterations of the UI design ideation cycle.

%\noindent \textcolor{blue}{----------------------------------------------------------------------}


%In an evaluation with xx designers, we found that GenerativeLayout: 1) enhances designers' exploration by expanding the coverage of the design space, 2) reduces the time required for exploration, and 3) maintains a perceived level of control similar to that of manual exploration.



% Present-day graphical user interfaces (GUIs) exhibit diverse arrangements of text, graphics, and interactive elements such as buttons and menus, but representations of GUIs have not kept up. They do not encapsulate both semantic and visuo-spatial relationships among elements. %\color{red} 
% To seize machine learning's potential for GUIs more efficiently, \papername~ exploits graph neural networks to capture individual elements' properties and their semantic—visuo-spatial constraints in a layout. The learned representation demonstrated its effectiveness in multiple tasks, especially generating designs in a challenging GUI autocompletion task, which involved predicting the positions of remaining unplaced elements in a partially completed GUI. The new model's suggestions showed alignment and visual appeal superior to the baseline method and received higher subjective ratings for preference. 
% Furthermore, we demonstrate the practical benefits and efficiency advantages designers perceive when utilizing our model as an autocompletion plug-in.


% Overall pipeline: Maybe drop semantic typography / visual flow?
\begin{IEEEkeywords}
code refinement, intention-based generation, large language model
\end{IEEEkeywords}
\section{Introduction}

Tutoring has long been recognized as one of the most effective methods for enhancing human learning outcomes and addressing educational disparities~\citep{hill2005effects}. 
By providing personalized guidance to students, intelligent tutoring systems (ITS) have proven to be nearly as effective as human tutors in fostering deep understanding and skill acquisition, with research showing comparable learning gains~\citep{vanlehn2011relative,rus2013recent}.
More recently, the advancement of large language models (LLMs) has offered unprecedented opportunities to replicate these benefits in tutoring agents~\citep{dan2023educhat,jin2024teach,chen2024empowering}, unlocking the enormous potential to solve knowledge-intensive tasks such as answering complex questions or clarifying concepts.


\begin{figure}[t!]
\centering
\includegraphics[width=1.0\linewidth]{Figs/Fig.intro.pdf}
\caption{An illustration of coding tutoring, where a tutor aims to proactively guide students toward completing a target coding task while adapting to students' varying levels of background knowledge. \vspace{-5pt}}
\label{fig:example}
\end{figure}

\begin{figure}[t!]
\centering
\includegraphics[width=1.0\linewidth]{Figs/Fig.scaling.pdf}
\caption{\textsc{Traver} with the trained verifier shows inference-time scaling for coding tutoring (detailed in \S\ref{sec:scaling_analysis}). \textbf{Left}: Performance vs. sampled candidate utterances per turn. \textbf{Right}: Performance vs. total tokens consumed per tutoring session. \vspace{-15pt}}
\label{fig:scale}
\end{figure}


Previous research has extensively explored tutoring in educational fields, including language learning~\cite{swartz2012intelligent,stasaski-etal-2020-cima}, math reasoning~\cite{demszky-hill-2023-ncte,macina-etal-2023-mathdial}, and scientific concept education~\cite{yuan-etal-2024-boosting,yang2024leveraging}. 
Most aim to enhance students' understanding of target knowledge by employing pedagogical strategies such as recommending exercises~\cite{deng2023towards} or selecting teaching examples~\cite{ross-andreas-2024-toward}. 
However, these approaches fall short in broader situations requiring both understanding and practical application of specific pieces of knowledge to solve real-world, goal-driven problems. 
Such scenarios demand tutors to proactively guide people toward completing targeted tasks (e.g., coding).
Furthermore, the tutoring outcomes are challenging to assess since targeted tasks can often be completed by open-ended solutions.



To bridge this gap, we introduce \textbf{coding tutoring}, a promising yet underexplored task for LLM agents.
As illustrated in Figure~\ref{fig:example}, the tutor is provided with a target coding task and task-specific knowledge (e.g., cross-file dependencies and reference solutions), while the student is given only the coding task. The tutor does not know the student's prior knowledge about the task.
Coding tutoring requires the tutor to proactively guide the student toward completing the target task through dialogue.
This is inherently a goal-oriented process where tutors guide students using task-specific knowledge to achieve predefined objectives. 
Effective tutoring requires personalization, as tutors must adapt their guidance and communication style to students with varying levels of prior knowledge. 


Developing effective tutoring agents is challenging because off-the-shelf LLMs lack grounding to task-specific knowledge and interaction context.
Specifically, tutoring requires \textit{epistemic grounding}~\citep{tsai2016concept}, where domain expertise and assessment can vary significantly, and \textit{communicative grounding}~\citep{chai2018language}, necessary for proactively adapting communications to students' current knowledge.
To address these challenges, we propose the \textbf{Tra}ce-and-\textbf{Ver}ify (\textbf{\model}) agent workflow for building effective LLM-powered coding tutors. 
Leveraging knowledge tracing (KT)~\citep{corbett1994knowledge,scarlatos2024exploring}, \model explicitly estimates a student's knowledge state at each turn, which drives the tutor agents to adapt their language to fill the gaps in task-specific knowledge during utterance generation. 
Drawing inspiration from value-guided search mechanisms~\citep{lightman2023let,wang2024math,zhang2024rest}, \model incorporates a turn-by-turn reward model as a verifier to rank candidate utterances. 
By sampling more candidate tutor utterances during inference (see Figure~\ref{fig:scale}), \model ensures the selection of optimal utterances that prioritize goal-driven guidance and advance the tutoring progression effectively. 
Furthermore, we present \textbf{Di}alogue for \textbf{C}oding \textbf{T}utoring (\textbf{\eval}), an automatic protocol designed to assess the performance of tutoring agents. 
\eval employs code generation tests and simulated students with varying levels of programming expertise for evaluation. While human evaluation remains the gold standard for assessing tutoring agents, its reliance on time-intensive and costly processes often hinders rapid iteration during development. 
By leveraging simulated students, \eval serves as an efficient and scalable proxy, enabling reproducible assessments and accelerated agent improvement prior to final human validation. 



Through extensive experiments, we show that agents developed by \model consistently demonstrate higher success rates in guiding students to complete target coding tasks compared to baseline methods. We present detailed ablation studies, human evaluations, and an inference time scaling analysis, highlighting the transferability and scalability of our tutoring agent workflow.

\section{Background on Causal Inference}
\label{sec:background-causal} 



 \newtextold{In this section, we 
 %formalize the notion of {\em Average Treatment Effect and understand the 
 review the basic concepts and key assumptions for inferring the effects of an intervention on the outcome on collected datasets without performing randomized controlled experiments. 
We use {\em Pearl's graphical causal model} for {\em observational causal analysis} \cite{pearl2009causal} to define these concepts.}


\par
\paratitle{Causal Inference and Causal DAGs} The primary goal of causal inference is to model causal dependencies between attributes and evaluate how changing one variable (referred to as intervention) would affect the other.
Pearl's Probabilistic Graphical Causal Model \cite{pearl2009causal} can be written as a tuple $(\exo, \edvar, Pr_{\exo}, \psi)$, where $\exo$ is a set of {\em exogenous} variables, $\Pr_{\exo}$ is the joint distribution of \exo, and $\edvar$ is a set of observed {\em endogenous variables}.
Here $\psi$ is a set of structural equations that encode dependencies among variables. The equation for $A \in \edvar$ takes the following form:
%that encode the dependencies among the variables.  These equations are of the form 
$$\psi_{A}: 
\dom(Pa_{\exo}(A)) {\times} \dom(Pa_{\edvar}(A)) \to \dom(A)$$
Here $Pa_{\exo}(A) {\subseteq} {\exo}$ and $Pa_{\edvar}(A) {\subseteq} \edvar \setminus \{A\}$ respectively denote the exogenous and endogenous parents of $A$. A causal relational model is associated with a directed acyclic graph ({\em causal DAG}) $G$, whose nodes are the endogenous variables $\edvar$ and there is a directed edge from $X$ to $O$ if  $X {\in} Pa_{\edvar}(O)$. The causal DAG obfuscates exogenous variables as they are unobserved. %Any given set of values for the exogenous variables completely determines the values of the endogenous variables by the structural equations (we do not need any known closed-form expressions of the structural equations in this work). 
The probability distribution $\Pr_{\exo}$ on exogenous variables $\exo$ induces a probability distribution  
on the endogenous variables $\edvar$ by the structural equations $\psi$.  A causal DAG can be constructed by a domain expert as in the above example, or using existing {\em causal discovery} algorithms~\cite{glymour2019review}. 



\begin{figure}
    \centering
    \small
    \begin{tikzpicture}[node distance=0.6cm and 1cm, every node/.style={minimum size=0.5cm}]
        \tikzset{vertex/.style = {draw, circle, align=center}}

        \node[vertex] (Ethnicity) {\bf\scriptsize{{Ethnicity}}};
        \node[vertex, right=0.3cm of Ethnicity] (Gender) {\bf{\scriptsize{Gender}}};
        \node[vertex, right=0.3cm of Gender] (Age) {\bf{\scriptsize{Age}}};
        \node[vertex, below=0.3cm of Gender] (Role) {\bf{\scriptsize{Role}}};
        \node[vertex, right=0.3cm of Role] (Education) {\bf{\small{\scriptsize{Education}}}};
        \node[vertex, below=0.3cm of Role] (Salary) {\bf{\scriptsize{Salary}}};

        \draw[->] (Ethnicity) -- (Salary);
        \draw[->] (Gender) -- (Role);
        \draw[->] (Age) -- (Role);
         \draw[->] (Education) -- (Role);
           \draw[->] (Education) -- (Salary);
             \draw[->] (Ethnicity) -- (Education);
                \draw[->] (Ethnicity) -- (Role);
             \draw[->] (Gender) -- (Education);
               \draw[->] (Age) -- (Education);
                 \draw[->] (Role) -- (Salary);
        \draw[->] (Gender) to[bend right] (Salary);
        \draw[->] (Age) -- (Salary);
    \end{tikzpicture}
    \caption{Partial causal DAG for the Stack Overflow dataset.}
    \label{fig:causal_DAG}
\end{figure}



 \begin{example}
Figure \ref{fig:causal_DAG} depicts a partial causal DAG for the SO dataset over the attributes in Table \ref{tab:data} as endogenous variables (we use a larger causal DAG with all 20 attributes in our experiments). 
  Given this causal DAG, we can observe that the role that a coder has in their company depends on their education, age gender and ethnicity.
\end{example}
\par


\par
\paratitle{Intervention} In Pearl's model, a treatment $T = t$ (on one or more variables) is considered as an {\em intervention} to a causal DAG by mechanically changing the DAG such that the values of node(s) of $T$ in $G$ are set to the value(s) in $t$, which is denoted by $\doop(T = t)$. Following this operation, the probability distribution of the nodes in the graph changes as the treatment nodes no longer depend on the values of their parents. Pearl's model gives an approach to estimate the new probability distribution by identifying the confounding factors $Z$ described earlier using conditions such as {\em d-separation} and {\em backdoor criteria} \cite{pearl2009causal}, which we do not discuss in this paper.


\par
\paratitle{Average Treatment Effect} The effects of an intervention are often measured by evaluating
% \par
% \paratitle{Causal inference, Treatment, ATE, and CATE}
% \newtextold{One of the primary goals  of {\em causal inference} is to estimate the effect of making a change in terms of a {\em treatment} $T$ (often referred to as an intervention)
% on the outcome $O$. 
% %A variable that is modified is often referred to as the treatment variable $T$ and the metric used to captures 
% The effect of treatment $T$ on outcome $O$ is measured by 
% %is known as 
{\em Conditional Average treatment effect (CATE)}, 
%a {\em treatment variable} $T$ on an outcome variable $O$ (e.g., what is the effect of higher \verb|Education| on \verb|Salary|). 
measuring the effect of an intervention on a subset of records~\cite{rubin1971use,holland1986statistics} by calculating the difference in average outcomes between the group that receives the treatment and the group that does not (called the {\em control} group), providing an estimate of how the intervention by $T$ influences an outcome $O$ for a given subpopulation. 
% Mathematically,
% \begin{equation}
%     %{\small ATE(T,O) = \mathbb{E}[O \mid \doop(T=1)] -      \mathbb{E}[O \mid \doop(T=0)]}
%     {\small ATE(T, O) = \mathbb{E}[O \mid \doop(T=1)] -  
%     \mathbb{E}[O \mid \doop(T=0)]}
% \label{eq:ate}
% \end{equation}
% In our work, where the treatment with maximum effect may vary among different subpopulations, we are interested in computing the \emph{Conditional Average Treatment Effect} (CATE), which measures the effect of a treatment on an outcome on \emph{a subset of input units}~\cite{rubin1971use,holland1986statistics}. 
Given a subset of the records defined by (a vector of) attributes $B$ and their values $b$, 
%g {\in} \Qagg(\db)$ defined by a predicate $G {=} g$ 
we can compute $CATE(T,O \mid B = b)$ as:
{
\begin{eqnarray}    
    %CATE(T,O \mid G=g) = \mathbb{E}[O \mid \doop(T=1)&, G=g] -  \mathbb{E}[O \mid \doop(T=0), G=g] 
   % CATE(T,O \mid B = b) = 
    \mathbb{E}[O \mid \doop(T=1), B = b] -  
    \mathbb{E}[O \mid \doop(T=0), B = b]\label{eq:cate}
\end{eqnarray}
}
Setting $B=\phi$ is equivalent to the ATE estimate.
The above definitions assumes that the treatment assigned to one unit does not affect the outcome of another unit (called the {Stable Unit Treatment Value Assumption (SUTVA)) \cite{rubin2005causal}}\footnote{This assumption does not hold for causal inference on multiple tables and even on a single table where tuples depend on each other.}. 


The ideal way of estimating the ATE and CATE is through {\em randomized controlled experiments}, 
where the population is randomly divided into two groups (treated and control, for binary treatments): 
%treated group that receives the treatment and control group that does not (denoted by 
%{the \em treated} group 
denoted by 
$\doop(T = 1)$ 
%for a binary treatment)  (the {\em control} group, 
and $\doop(T = 0)$ resp.)~\cite{pearl2009causal}.
%\sr{edited up to here, going to read the rest first, this section should not look like causumx}
%\par
%\par
However, randomized experiments cannot always be performed due to ethical or feasibility issues. In these scenarios, observational data is used to estimate the treatment effect, which requires the following additional assumptions. 
% {\em Observational Causal Analysis} still allows sound causal inference under additional assumptions. Randomization in controlled trials mitigates the effect of {\em confounding factors}, i.e., attributes that can affect the treatment assignment and outcome. Suppose we want to understand the causal effect of \verb|Education| on \verb|Salary| from the SO dataset.  %in Example~\ref{ex:running_example}. 
% We no longer apply Eq. (\ref{eq:ate}) since the values of \verb|Education| were not assigned at random in this data, and obtaining higher education largely depends on other attributes like \verb|Gender|, \verb|Age|, and \verb|Country|. 
% Pearl's model provides ways to account for these confounding attributes $Z$ to get an unbiased causal estimate from observational data under the following assumptions ($\independent$ denotes independence):
% \vspace{-2mm}
\newtextold{
The first assumption is called {\em unconfoundedness} or {\em strong ignorability}  \cite{rosenbaum1983central} says that the independence of outcome $O$ and treatment $T$ conditioning on a set of confounder variables  (covariates) $Z$, i.e.,
%\begin{eqnarray}
 $    O \independent T | Z {=} z$.
 %\label{eq:unconfoundedness}
%\end{eqnarray}
The second assumption called {\em overlap or positivity} says that there is a chance of observing individuals in both the treatment and control groups for every combination of covariate values, i.e., 
%\begin{eqnarray}
   $ 0 < Pr(T {=} 1 ~~|~~Z {=} z)< 1 $.
   %\label{eq:overlap}
%\end{eqnarray}
}
%\sg{Is this overlap or positivity? maybe both are the same?} \sr{yeah - same - from Google AI - The overlap assumption, also known as the positivity assumption, is a key assumption in causal inference that states that there is a chance of observing individuals in both the treatment and control groups for every combination of covariate values.}
% The above conditions are known as {\em Strong Ignorability} in Rubin's model \cite{rubin2005causal}.
The unconfoundedness assumption requires that the treatment $T$ and the outcome $O$ be independent when conditioned on a set of variables $Z$. In SO, assuming that only $Z$ =\{\verb|Gender|, \verb|Age|, \verb|Country|\} affects $T = $ \verb|Education|, if we condition on a fixed set of values of $Z$, i.e., consider people of a given gender, from a given country, and at a given age, then $T = $ \verb|Education| and $O = $ \verb|Salary| are independent. For such confounding factors $Z$,  Eq. (\ref{eq:cate}) reduces to the following form 
(positivity 
gives the feasibility of the expectation difference): 
 \vspace{-1mm}
{\small
\begin{flalign}    
% \begin{eqnarray}
   % % & ATE(T,O) = \mathbb{E}_Z \left[\mathbb{E}[O \mid T=1, Z = z] -  
   %  \mathbb{E}[O \mid T=0, Z = z] \right] \label{eq:conf-ate}\\
 & CATE(T,O {\mid} B {=} b) {=} \nonumber
    \mathbb{E}_Z \left[\mathbb{E}[O {\mid} T{=}1, B {=} b, Z {=} z] {-}  
    \mathbb{E}[O {\mid} T{=}0, B {=} b, Z {=} z]\right]\label{eq:conf-cate}
\end{flalign}
% \end{eqnarray}
}
% \vspace{-4mm}
This equation contains conditional probabilities and not $\doop(T = b)$, which can be estimated from an observed data. 
Pearl's model gives a systematic way to find such a $Z$ when a causal DAG is available. 




% \section{code refinement}

% 在一次code review过程中,首先由developer提交了一份修改,用最初的代码C0修改到代码C1。而后,reviewer针对这次修改(C0->C1)提出了review comment(RC),无论是reviewer提交review,还是developer查看review,RC会呈现在某一行代码之后,我们称这一行代码是review line(RL)。通常review line是C0->C1的修改部分,review对上次的修改进行评价和建议,也有少部分情况review line是针对未修改代码提出新的修改建议。而后developer根据RC,对C1进行修改,得到新的代码版本C2。传统的code refinement任务我们称之为Basic Code Refinement,其input为:<C1, RC>,output为<C2>。然而,我们注意到一些数据只提供C1, RC是信息不足的。如图1所示,例子a需要提供review line才能确定review要删除哪一行。例子2中,需要指导C0才能确定如何回退代码。我们定义两种新型的任务,Position-Aware Code Refinement: 其输入为<C1,RC,RL>,输出是<C2>;	Comprehensive Code Refinement:输入为<C0,C1,RC,RL>,输出也是<C2>。本文主要研究的对象就是Comprehensive Code Refinement。
% 为了避免文字混淆,我们称C0版本的代码是initial code,C1版本的代码是original code,C2版本的代码是revised code。



% 目前coderefine方向有两个使用广泛的数据集,Tufano数据集和codereview数据集。如前文介绍的,我们需要数据集提供initial code, original code, review line, review comment, revised code等五个字段。Tufano数据集没有initial code字段,且没有提供原始数据的链接。而codereview数据集虽没有review line,initial code两个字段,但是提供了原始的数据连接,故而我们选择使用codereview数据,并补齐缺失字段。

% 首先介绍review line字段获取方法。我们观察到通过GitHub REST API获取code review信息时,可以获取到partial last code diff(在API返回的json中叫做diff hunk字段,给个角标https://api.github.com/repos/meganz/sdk/pulls/comments/326107667)。之所以我们称之为partial last code diff,是因为这段code diff只提供了review comment之前的修改信息。在例子中,原本的last code diff有三行代码删除,三行代码添加。review line在第一行代码添加之后。所以partial last code diff只有三行删除,一行添加。根据这个规律,我们可以得到review line就是partial last code diff的最后一行。

% 而后我们需要设计得到initial code的方法。我们观察到,一次code review可能是reviewer对前面多次commit的review。如果当前pull request有n个commit:(commit_1,commit_2,...,commit_n),reviewer可以选择commit_m到commit_n之间所有commit(n小于等于m),然后查看这些commit叠加后的文件变化,而后再给出review comment。而GitHub REST API只给出了最后一次commit的id,即commit_n,无法确定前面的commit_m。不过GitHub REST API中提供的partial alst code diff就是commit_m到commit_n的code diff。故而,我们倒序依次遍历前面的所有commit,并与commit_n做比较,得到review line附近的code diff。并与partial last code diff做对比,就可以找到与partial last code diff一致的,完整的last code diff。根据last code diff和original code就可以得到initial code。



\begin{table*}[ht!]
\centering
\small % Reduce font size
\setlength{\tabcolsep}{4pt} % Reduce horizontal padding if needed
\begin{tabular}{l p{5.5cm} p{5.5cm}}
    \toprule
    \textbf{Dataset} & \textbf{Answerable} & \textbf{Unanswerable} \\
    \midrule
    SQUAD & 
    \textbf{Passage:} The first beer pump known in England is believed to [\dots]. %have been invented by John Lofting (b. Netherlands 1659--d. Great Marlow, Buckinghamshire 1742), an inventor, manufacturer, and merchant of London.
    \newline 
    \textbf{Question:} When was John Lofting born? 
    & 
    \textbf{Passage:} Starting in 2010/2011, Hauptschulen were merged  [\dots]. %with Realschulen and Gesamtschulen to form a new type of comprehensive school  in the German States of Berlin and Hamburg---called Stadtteilschule in Hamburg and Sekundarschule in Berlin (see: Education in Berlin, Education in Hamburg). 
    \newline 
    \textbf{Question:} In what school year were Hauptschulen last combined with Realschulen and Gesamtschulen? \\
    \midrule
    IDK & 
    \textbf{Passage:} Singapore has reported 16 deaths. \newline 
    \textbf{Question:} Where are the deaths? 
    & 
    \textbf{Passage:} Showed the arrest of the prime suspect. \newline 
    \textbf{Question:} Where was the arrest? \\
    \midrule
    BoolQ & 
    \textbf{Passage:} On April 20, 2018, ABC officially renewed \textit{Grey's Anatomy} for a network primetime drama record-tying fifteenth season. \newline 
    \textbf{Question:} Is season 14 the last of \textit{Grey's Anatomy}? 
    & 
    \textbf{Passage:} Discover is the fourth largest credit card brand in the U.S., behind Visa, MasterCard, and American Express, with nearly 44 million cardholders. \newline 
    \textbf{Question:} Are pasilla chiles and poblano chiles the same? \\
    \midrule
    Equation & 
    \textbf{Given equations:} \newline n = 53 \newline v = 90 \newline 
    \textbf{Final equation:} \newline n / v = 
    & 
    \textbf{Given equations:} \newline n = 17 \newline u = 38 \newline 
    \textbf{Final equation:} \newline n * t = \\
    \midrule
    Celebrity & 
    \textbf{Article:} Yesterday, I saw an article about Gerard Butler. They really are a great actor. \newline 
    \textbf{Question:} Do you know what their age is? 
    & 
    \textbf{Article:} Yesterday, I saw an article about Tania Scott. They really are a great actor. \newline 
    \textbf{Question:} Do you know what their age is? \\
    \bottomrule
\end{tabular}
\caption{Answerable and Unanswerable Examples from Different Datasets}
\label{tab:answerability}
\end{table*}




\section{Methodology}
We evaluate SAE probes for answerability detection with a specific focus on  generalization.

\myparagraph{SAE Probes}
We use the "Gemma Scope" SAEs pretrained by \citet{lieberum2024gemma} for the instruction-tuned model Gemma 2 \citep{team2024gemma}, and specifically the largest available ones with a width (number of dimensions) of 131k. \citet{lieberum2024gemma} provide SAEs trained on layers 20 and 31.
%
Note that answerability more generally (i.e., beyond specific types of answerability) is a rather high-level concept, which we assume to be represented in intermediate and later layers.
Unless otherwise specified, we search for features using 2k samples of SQUAD (balanced, leaving 1.8k for testing). We collect the feature activations on the last token position and then use 5-fold cross validation for finding SAE features that are predictive for answerability, thus obtaining 1-sparse SAE probes \citep{gurnee2023finding}.
%Note that the common naming as ``probes'' can be misleading, with these probes there are no parameters to be trained.
We then train final probes\footnote{We use SAE probes and SAE features synonymously.} (i.e., scale and bias) for best performing features, which are used for the out-of-distribution evaluation. See Appendix~\ref{app:prelims} for details. ``Top'' features are selected based on training set performance.




\myparagraph{Baselines: Linear Probes}
We train simple linear residual stream probes on the (in-domain) training dataset we also use for finding the SAE features. To ensure robustness, we employ bootstrap analysis across different training splits. %, revealing significant variability in out-of-distribution generalization [i don't recall if this true anymore]. % VT commented based on comment
Since we also focus on SAE features for the residual stream, this probing represents an upper bound for the SAE probing performance on in-domain data.
Observe that these probes achieve 85-90\% accuracy on the in-domain SQUAD data, and thus provide a strong benchmark for comparison. 

\myparagraph{Datasets}
We focus on context-based question answering in the English language.
%In the main experiments, we train and evaluate on SQUAD \cite{rajpurkar2018know} and additionally
We use established data as well as datasets specifically constructed  for out-of-distribution evaluation; for examples see Table~\ref{tab:answerability}.
%We use two established answerability datasets (BoolQ \cite{} and IDK \cite{}), and create two synthetic specialized datasets (math equations, celebrity names):
\begin{itemize}[leftmargin=*,topsep=0pt,noitemsep]
    \item \textbf{SQUAD} \citep{rajpurkar2018know}: Established dataset, passages plus questions relating to them. %Dataset consisting of a short context passage and a question relating to the context. We follow the training data split and prompting template provided by \citet{slobodkin2023curious}.
    \item \textbf{IDK} \citep{sulem2021we}: Dataset with questions in the style of SQUAD.
    % , containing both answerable and unanswerable examples. We specifically use the non-competitive and unanswerable subsets of the ACE-whQA dataset.
    \item \textbf{BoolQ\_3L} 
    \citep{sulem2022yes}: Context-based yes/no questions. % with answerable and unanswerable subsets.
    \item \textbf{Math Equations}: Synthetic dataset contrasting solvable equations with equations containing unknown variables.
    \item \textbf{Celebrity Recognition}: Context-based queries requiring background knowledge about celebrities; for construction details, see Appendix~\ref{app:datasets}.
    % For construction, we use a public dataset of actors and movies from IMDB\footnote{\url{https://www.kaggle.com/datasets/darinhawley/imdb-films-by-actor-for-10k-actors}}, and generate a list of the 1000 most popular actors after 1990, as measured by the total number of ratings their movies received. We construct an additional dataset of non-celebrity names by randomly generating first and last name combinations using the most common North American names from Wikipedia\footnote{\url{https://en.wikipedia.org/wiki/Lists_of_most_common_surnames_in_North_American_countries} and \url{https://en.wikipedia.org/wiki/List_of_most_popular_given_names?utm_source=chatgpt.com}}. 
\end{itemize}

% For the question-answering datasets (SQUAD, BoolQ, and IDK) we use the following consistent formatting:
%
% \begin{verbatim}
% Given the following passage and question, answer the question:
% Passage: {passage}
% Question: {question}
% \end{verbatim}
%
% For the equation dataset, we construct ...
%
% \begin{verbatim}
% "Given the following equations, determine the result of the final equation.
% Given equations:
% {eq_1}
% {eq_2}
% Final equation:
% {eq_3}"
% \end{verbatim}
%
% For the celebrity dataset, we use a public list of celebrity names from \cite{} and generate additional non-celebrity names using the most common north american first names and last names from wikipedia. 
%
% \begin{verbatim}
% "Yesterday, I saw an article about {name}. They really are a great actor. Do you know what their age is?"
% \end{verbatim}

% \subsection{Generalization Analysis}

% We evaluate generalization through multiple lenses:
% \begin{itemize}
%     \item Cross-dataset performance comparing SQUAD-trained features on out-of-distribution data
%     \item Prompt variation analysis testing robustness to input formatting
%     \item Analysis of using a combination of up to five SAE features 
%     \item Similarity analysis of top-performing SAE features and learned residual stream probes
% %    \item Feature consistency analysis across domains using attribution scores
% %    \item Ablation studies measuring the impact of individual features versus feature combinations
% \end{itemize}
%This approach reveals variance in feature transfer ability---while some SAE features show impressive generalization, others remain strongly domain-specific. Moreover, despite strong in-domain performance, residual stream probes exhibit inconsistent transfer when it comes to generalisation.

% \vt{brief intro sentence, why we choose the feature, compare (only) to regular probes}

% \myparagraph{Hypothesis} 
% % vs probe esp in terms of generalization hold their promise?

% \myparagraph{Datasets} 
% % mainly context-based QA
% % squad, generaliz on ambigqa, boolq, math celeb
% % figure with some prompt example, rest into appendix


% \myparagraph{Methods}  % if we don't have more about method we can rename it to models
% \vt{gemma scope since... , also experimented briefly with llama scope; @lovis anything what we could mention here specifically?}

% \paragraph{Linear probes}

% \lh{
% - Trained on 2000 samples of SQUAD (balanced, leaving 1800 for testing)
% - last token position
% - 5 fold cross validation
% - sweeping over regularization parameters with 26 logarithmically spaced steps between 0.0001 and 1
% - Fitting the final probe with the best regularization parameter on the whole training set
% - Trained probes are then evaluated on OOD datasets
% - Analysis is repeated 10 times with different random training set splits

% }



% \begin{table*}[h]
%     \centering
%     \begin{tabular}{|l|p{6cm}|p{6cm}|}
%         \hline
%         \textbf{Dataset} & \textbf{Answerable} & \textbf{Unanswerable} \\
%         \hline
%         SQUAD & Given the following passage and question, answer the question: \newline \textbf{Passage:} The first beer pump known in England is believed to have been invented by John Lofting (b. Netherlands 1659-d. Great Marlow Buckinghamshire 1742) an inventor, manufacturer and merchant of London.\newline \textbf{Question:} When was John Lofting born? & Given the following passage and question, answer the question:\newline \textbf{Passage:} Starting in 2010/2011, Hauptschulen were merged with Realschulen and Gesamtschulen to form a new type of comprehensive school in the German States of Berlin and Hamburg, called Stadtteilschule in Hamburg and Sekundarschule in Berlin (see: Education in Berlin, Education in Hamburg). \newline \textbf{Question:} In what school year were Hauptschulen last combined with Realschulen and Gesamtschulen?\\
%         \hline
%         IDK & Given the following passage and question, answer the question: \newline \textbf{Passage:} Singapore has reported 16 deaths.\newline \textbf{Question:} Where are the deaths? & Given the following passage and question, answer the question:\newline \textbf{Passage:} Showed the arrest of the prime suspect.\newline \textbf{Question:} Where was the arrest? \\
%         \hline
%         BoolQ & Given the following passage and question, answer the question: \newline \textbf{Passage:} On April 20, 2018, ABC officially renewed Grey's Anatomy for a network primetime drama record-tying fifteenth season.\newline \textbf{Question:} Is season 14 the last of grey's anatomy? & Given the following passage and question, answer the question:\newline \textbf{Passage:} Discover is the fourth largest credit card brand in the U.S., behind Visa, MasterCard and American Express, with nearly 44 million cardholders. \newline \textbf{Question:} Are pasilla chiles and poblano chiles the same?\\
%         \hline
%         Equation & Given the following equations, determine the result of the final equation.\newline \textbf{Given equations}:\newline n = 53\newline v = 90\newline \textbf{Final equation}:\newline n / v = & Given the following equations, determine the result of the final equation.\newline \textbf{Given equations}: \newline n = 17\newline u = 38\newline \textbf{Final equation}:\newline n * t = \\
%         \hline
%         Celebrity & Yesterday, I saw an article about Gerard Butler. They really are a great actor. Do you know what their age is? & Yesterday, I saw an article about Tania Scott. They really are a great actor. Do you know what their age is? \\
%         \hline
%     \end{tabular}
%     \caption{Answerable and Unanswerable Examples from Different Datasets}
%     \label{tab:answerability}
% \end{table*}







\section{Experimental Setup}
To evaluate the effectiveness of the proposed approach, we design the following research questions.
\begin{itemize}[leftmargin=*,topsep=2pt]
    \item \major{RQ1: How accurate are LLMs in extracting intentions?}
    \item RQ2: To what extent is the Intention-based framework effective in code refinement?
    \item RQ3: To what extent does the framework improve performance across different intention categories?
    \item RQ4: To what extent is intention-based dataset cleaning effective?
\end{itemize}





% Our objective is to systematically evaluate the effectiveness of the Intention-based Code Refinement Framework in addressing the Comprehensive Code Refinement task. The research questions include:

% 1. To What Extent is the Intention Extraction Accurate?

% 2. To What Extent is the Intention-based Framework Effective in Code Refinement?

% 3. To What Extent Does the Framework Improve Performance Across Different Intention Categories?

% 4. To What Extent is Intention-Based Dataset Cleaning Effective?
 
% 我们的目标是系统的评估Intention-based code refinement framework解决Comprehensive Code Refinement任务的有效性。研究问题包括:首先,需要先调研Extracting Intention这个步骤中取得Intention的正确率。其次,验证整体框架的有效性,即评估模型通过框架解决code Refinement任务的正确率,对比不使用本框架的方法是否有提升。再次,调研不同Intention类别的数据,使用框架可以提升多少效果。最后,研究使用Intention作为数据质量判别的方法,对清洗数据集的效果。

\subsection{RQ1: Intention Extraction Accuracy}
The accuracy of the intention understanding is important for code refinement. 
% The concept of intention is a novel proposition we introduced, which currently lacks an existing benchmark. 
To measure the accuracy, we constructed a test dataset from the existing code review dataset~\cite{li2022automating}. We randomly select 2,000 samples for manual annotation to assess the accuracy of different LLMs in intention extraction. 
% Correct intention extraction implies that the intention accurately summarizes the reviewer’s suggested solution, allowing the developer to refine the code without having to focus on the review comments but solely relying on the intention.


% As presented in Fig.~\ref{fig:framework}, three general categories are designed including explicit code suggestion, reversion suggestion and general suggestion. Followed by the rule-based intention analyzer, we use GPT4o to classify the categories. 
As shown in Fig.~\ref{fig:framework}, for reversion intention and general intention, we use the LLM to extract the intention. To check the accuracy of the extracted intention, we invited two PhD candidates to conduct manual annotations. In cases of differing results, the annotators reached a consensus through discussion to finalize the annotations. During the annotation process, we not only assessed the correctness of intention extraction but also filtered out invalid data. Specifically, cases where the revised code was unrelated to the review comments were excluded to ensure that only relevant and high-quality data were used in subsequent research questions.

% We categorized the selected samples based on the actual \textit{Intentions} into three categories: Explicit Code Suggestion, Reversion Suggestion, and General Suggestion. The following criteria define how to determine the correctness of the framework’s intention extraction:

% 1. Explicit Code Suggestion: If the framework correctly classifies the intention as an Explicit Code Suggestion, the extracted intention is deemed correct.

% 2. Reversion Suggestion: There are two scenarios where the framework's extraction is considered correct for this category:

%    - The framework directly categorizes the intention as a Reversion Suggestion.

%    - The framework classifies the intention as a General Suggestion, but following the suggested fixes achieves the same outcome as the Revised Code.

% 3. General Suggestion: For data categorized under General Suggestion, the framework must classify the intention as a General Suggestion, and the proposed solution must be consistent with the original review comments to be deemed correct.

% Given the extensive nature of manual annotations, we focused solely on evaluating the results of the GPT-4o model in the intention extraction process. Two Ph.D. candidates in software engineering conducted the annotations. 

% Intention的正确性代表了模型真正理解了任务的要求,是做好后续步骤的前提。Intention的概念是我们首次提出,目前还没有benchmark,我们使用codereview数据集的测试数据,随机选取2000个,而后用人工标记的方式来判断Intention的正确率。所谓Intention的正确就是指Intention概括总结了reviewer的建议方案。developer可以不再关注ReviewComment,只根据Intention就能正确修复代码。
% 我们按照真实的Intention把数据分成Explicit Code Suggestion,Reversion Suggestion和General Suggestion这三类。接下来定义如何判断framework提取的Intention是否正确。对于Explicit Code Suggestion类别的数据,只要framework正确将Intention分成Explicit Code Suggestion类别,就认为提取的Intention是正确的。对于Reversion Suggestion类别的数据,framework提取的Intention有两种情况是正确的。一种是framework直接将Intention分成Reversion Suggestion类别。另一种是framework将Intention分类成General Suggestion类别,并且按照修复建议,达到的效果和Revised Code是一致的。对于General Suggestion类别的数据,需要framework将Intention分类到General Suggestion,并且修复的方案和原始ReviewComment是一致的,才算是framework提取的Intention是正确的。
% 由于涉及大量人工标注,我们仅评估了GPT-4o模型提取Intention的结果。共有两个软件工程的PHD进行标注,出现不同结果时,由两人商议再最终确认。在标注过程中,我们不仅标注了Intention提取的正确性,也标注了原始数据的合理性。对于revised code与review comment无关的不合理的数据进行过滤,保证在后续的RQ中,只使用合理的数据。

\begin{figure}[!t]
\centering
\includegraphics[width=0.85\linewidth]{fig/prompt2.pdf}
\vspace{-2mm}
\caption{The used prompt format for different tasks.}
\label{fig:prompt2}
\vspace{-4mm}
\end{figure}


\subsection{RQ2: Intention-based Framework Effectiveness}
To evaluate the effectiveness of our proposed intention-based code refinement, we select two state-of-the-art baselines for comparison. 

% To What Extent is the Intention-Based Framework Effective in Code Refinement?
% In this study, we compare the results of our framework with other code refinement methods. Our selected baselines include two categories: pre-trained models and large model prompt techniques.

% For pre-trained models, we have chosen CodeReviewer~\cite{li2022automating} and T5CR~\cite{tufano2022using} as our baselines.

\noindent \textbf{CodeReviewer~\cite{li2022automating}:} 
It designed three pre-training objectives, i.e., Diff Tag Prediction, Denoising Objective, and Review Comment Generation, to pre-train the model based on CodeT5~\cite{codet5} for code review activities. Several downstream tasks, including code change quality estimation, code review generation and code refinement, are selected to evaluate the effectiveness of the proposed models. 


\noindent \textbf{T5CR~\cite{tufano2022using}:} It utilized two datasets including the official Stack Overflow dump (i.e., SOD) and CodeSearchNet (i.e., CSN) to pre-train the code review model based on T5 architecture. A tokenizer, i.e., SentencePiece~\cite{kudo2018sentencepiece}, is adopted to tokenize the source code, and the input sequence's maximum length is increased to 512 for training. 

Apart from these baselines, we also comprehensively evaluate the effectiveness of different LLMs and prompt strategies. In particular, we select three closed-source LLMs i.e., GPT-4o-2024-05-13 (GPT4o)~\cite{achiam2023gpt}, GPT-3.5-turbo-0125 (GPT3.5)~\cite{ouyang2022training}, DeepSeek-Coder-V2-0724 (DeepSeekV2)~\cite{zhu2024deepseek} and two open-source large models: CodeQwen1.5-7B-Chat (CodeQwen7B)~\cite{bai2023qwen} and Deepseek-coder-6.7b-instruct (DeepSeek7B)~\cite{guo2024deepseek} for evaluation.
\major{Note that, since different models have varying capabilities in extracting intentions, we present results using both the intentions extracted by the model itself and those extracted by a high-quality model (i.e., GPT-4o). The use of GPT-4o intentions allows us to evaluate whether providing the correct intention can enhance performance in code refinement tasks.}


% The T5~\cite{raffel2020exploring} model serves as the foundational model, pre-trained on a dataset comprising 1.5 million Java-English corresponding pairs through the Masked Language Model Task. In this experiment, we employed the same pre-trained model and also fine-tuned it using the $CodeReview_{train}$ datasets.

% Since our framework, like prompt techniques, is model-agnostic, we conducted a comprehensive and detailed comparison of five models and six common prompt strategies under three code refinement tasks as baselines.

% The models include three closed-source large models: gpt-4o-2024-05-13 (Gpt4o)~\cite{achiam2023gpt}, gpt-3.5-turbo-0125 (Gpt3.5)~\cite{ouyang2022training}, and DeepSeek-Coder-V2-0724 (DeepSeekV2)~\cite{zhu2024deepseek} and two open-source large models: CodeQwen1.5-7B-Chat (CodeQwen7B)~\cite{bai2023qwen} and Deepseek-coder-6.7b-instruct (DeepSeek7B)~\cite{guo2024deepseek}.

% The three tasks include (see Section 3): Basic Code Refinement, Position-aware Code Refinement, and Comprehensive Code Refinement. This allows for a thorough comparison of the effectiveness of various refinements under different input scenarios.
For LLMs, we selected different prompting strategies:

\noindent \textbf{Simple Prompt:} As shown in Fig.~\ref{fig:prompt2}, the Simple Prompt first describes the scenario, then introduces the input information in the task, and finally requests the generation of revised code based on the provided information. This is a concise and effective prompt design used in Guo et al.~\cite{guo2024exploring}.

\noindent \textbf{Simple COT Prompt:} This prompt builds upon the simple prompt by adding the phrase ``Let's think step by step." This technique is employed in model reasoning~\cite{wei2022chain,wang2022self}.

\noindent \textbf{Tufano COT Prompt:} It is introduced in Tufano et al.~\cite{tufano2024code}, which first requires determining which of the following six categories the modification belongs to before completing the modification and then utilizes LLM for the generation.

% The six modification categories are:

% - Changes have been required to refactor the code to improve its quality;

% - Changes have been required since tests for this code must be written;

% - Changes have been required to better align this code to good object-oriented design principles;

% - Changes have been required to fix one or more bugs;

% - Changes have been required to improve the logging of its execution;

% - Changes have been required for other reasons not listed above.

\noindent \textbf{Random Few-shot Prompt:} For each case, three data examples are randomly selected as examples for the prompt as few-shot (excluding self-examples). The data fields provided as examples depend on the task's input fields, such as \texttt{OriginalCode}, \texttt{ReviewComment}, \texttt{ReviewLine}, \texttt{RevisedCode} in Comprehensive Code Refinement, and only \texttt{OriginalCode}, \texttt{ReviewComment}, \texttt{RevisedCode} in Basic Code Refinement. After providing examples as prompt hints, they are concatenated with the simple prompt.

% \noindent \textbf{RAG Prompt:} Similar to Random Fewshot, but the example selection is based on the most similar data retrieved using the BM25~\cite{robertson2009probabilistic} method. We use the \texttt{ReviewComment} as the key for retrieval, with the complete data as the value stored in the retrieval set. For each case, its \texttt{ReviewComment} is used to retrieve three nearest neighbors (excluding itself). After retrieval, the prompt construction method is the same as Random Fewshot.

\noindent \major{\textbf{RAG Prompt:} 
To compare with the Random Few-Shot Prompt, we design a Retrieval-Augmented Generation (RAG) prompting method as an alternative approach to few-shot prompting, leveraging relevant example selection. The retrieval database is constructed from the dataset used in RQ1, specifically the 2,000 randomly selected samples from the CodeReview dataset. However, only 1,337 of these samples are included in the database, as the remaining 663 cases exhibit low-quality refinements that do not align well with the review comments.
We selected these samples for two reasons: 1) The CodeReview dataset uniquely provides the complete data required for this study, including \texttt{OriginalCode}, \texttt{ReviewComment}, \texttt{ReviewLine}, \texttt{RevisedCode}, and \texttt{LastCodeDiffHunk}. 2) The quality of the retrieval dataset is crucial, and these 2,000 samples have been manually analyzed in RQ1, ensuring their reliability.
During testing, for each test case, we retrieve three samples to construct a 3-shot prompt. If the test data is included in the retrieval results, it is excluded and replaced with another sample. To enhance contextual relevance, we use BM25 to select semantically similar examples, ensuring that the retrieved examples closely align with the test data's context..}

\noindent \textbf{Self-generated Prompt:} This prompt technique addresses mathematical and reasoning problems by utilizing the model's own understanding ability~\cite{yasunaga2023large}. It first generates several examples and then uses these examples to inspire the model to answer the original question.

% The format of these prompts is provided on our website~\cite{IntentionWebsite}. 

\noindent \textbf{Evaluation Metrics:} To fully automate the code refinement task, we prioritize the exact match (EM) between the model's output and the actual revised code. While metrics like BLEU and Code-BLEU can reflect proximity to the ground truth, they do not effectively gauge the degree to which the model's output aids programmers in code refinement. Particularly in simple tasks, if the model's output is not completely correct, it may be less beneficial for programmers to use the model's output for refinement than to directly use the review comment. Therefore, our evaluation metric only includes the EM value.

% Dataset: The data we used is the manually labeled valid data from RQ1.

% In this experiment, our framework utilizes the RAG prompt method. By comparing its performance with these baselines, we can assess whether our framework surpasses existing methods.


% To What Extent is the Intention-based Framework Effective in Code Refinement? 我们将framework的结果与其他Code Refinement方法进行比较。
% 我们选用的baseline包括预训练模型和大模型的prompt技术两种。
% 预训练模型,我们选择的baseline是CodeReviewer和T5CR。
% \textbf{CodeReviewer:} The model is initialized using the weight parameters of CodeT5 ~\cite{codet5}. Subsequently, the pre-training is carried out with three objectives: Diff Tag Prediction, Denoising Objective, and Review Comment Generation. In this experiment, we employed the same pre-trained CodeReviewer model and fine-tuned it using the $CodeReview_{train}$ datasets.
% T5CR:使用的T5作为基础模型,在一份1.5million个的,关于java和英语对应关系的数据集上,进行Masked Language Model Task任务预训练。In this experiment, we employed the same pre-trained model and fine-tuned it using the $CodeReview_{train}$ datasets.
% 因为我们的框架和prompt技术一样,都是模型无关的。所以我们全面而详细的对比5种模型,6种常见的prompt,在三种code Refinement任务下的效果作为baseline。
% 模型我们采用了3个闭源大模型,gpt-4o-2024-05-13,gpt-3.5-turbo-0125,DeepSeek-Coder-V2-0724,和2个开源大模型CodeQwen1.5-7B-Chat,deepseek-coder-6.7b-instruct。
% 三种任务包括(见第三节):Basic code Refinement,Position-aware code Refinement和Comprehensive code Refinement等三种任务。这样充分对比不同输入信息的情况下,哪种修复的效果最好。
% Prompt方法包括了Simple Prompt,Simple COT,Tufuno COT,Random Fewshot,RAG Prompt, Self-generated Prompt等6种常见的prompt策略。
% Simple Prompt:如图3所示,Simple Prompt首先描述场景,然后介绍任务中的输入信息,最后要求根据提供的信息,生成revised code。是Guo的文章中使用的,简洁有效的Prompt设计。
% Simple COT Prompt: 在simple prompt的基础上,加上“Let's think step by step.”。这是在多个推理任务中使用的prompt方法。
% Tufano COT Prompt:在Tufano论文中提到的一种COT技术,首先要求判断修改类型属于以下六个类别的哪一种,然后在完成修改。六种修改类别为:
% Changes have been required to refactor the code to improve its quality;
% Changes have been required since tests for this code must be written;
% Changes have been required to better align this code to good object-oriented design principles;
% Changes have been required to fix one or more bugs;
% Changes have been required to improve the logging of its execution;
% Changes have been required for other reasons not listed above.
% Random Fewshot Prompt:用有效数据为样例选取数据集。对每个case,随机选取3个数据作为examples提供给prompt作为fewshot(已排除掉自己给自己做example的情况)。例子提供的数据字段依照任务本身输入字段,如Comprehensive Code Refinement里包含Initial code,original code,review comment,review line,revised code等5个字段。而Basic Code Refinement只包含original code,review comment,revised code三个字段。提供例子作为prompt提示后,再拼接上simple prompt。
% RAG Prompt:与random fewshot类似,只不是选取example时,使用的是用BM25方法检索到的最相似的数据。我们使用review comment作为检索的key,完整的数据作为value,存储再检索集中。对每个case,使用其review comment去检索相近的3个邻居(排除掉自身)。检索到example之后,后面构造prompt的方法与random fewshot相同。
% Self-generated Prompt:这是解决数学和推理问题一种prompt方法,利用模型自身对问题的理解能力,先生成若干个例子,然后再根据这些例子来启发模型,回答最初的问题。
% 详细的prompt设计我们已上传到网站。
% 评价指标:为了完全自动化code Refinement任务,我们更关模型输出结果与真实RevisedCode是否完全匹配,即EM。BLEU与Code-BLEU等指标虽然可以反映与真实值的接近程度,但是对code Refinement任务不能很有效的反映模型输出结果对程序员的帮助程度。特别是对于简单的任务,如果模型输出结果不是完全正确,那么程序员利用模型输出的结果再去做Refinement,可能不如直接利用ReviewComment来做Refinement。所以,我们这次的评价指标只选择了EM值。
% 数据集:我们所用的数据是RQ1中人工标记valid的数据
% 在这个实验中,我们框架选用RAG的prompt方法。通过与这些baseline的效果做对比,可以看到我们的framework是否可以超过现有的方法。

\subsection{RQ3: Improvement Across Different Intention Categories}

% To What Extent Does the Framework Improve Performance Across Different Intention Categories?

In this experiment, we mainly evaluated the effects of different prompting strategies and three prompting strategies: Simple Prompt, RAG, and Self-generated Prompt. We aim to understand how much our framework improves performance for each of Intention types compared to methods that do not use the framework.

\major{Additionally, we conducted ablation experiments on the Intention Extraction component of our framework, specifically evaluating the impact of the three agents (Agent1, Agent2, and Agent3). For each agent, we tested the Exact Match (EM) values by removing the respective agent and comparing the results. This analysis allows us to quantify the individual contribution of each agent to the overall performance of the framework and to identify which components are most critical for improving the accuracy of intention-based predictions.}

% The model, data, and experimental settings are the same as those used in RQ2.

% To What Extent Does the Framework Improve Performance Across Different Intention Categories? 
% 在这个实验中,我们对比了Simple Prompt,RAG和Self-generated Prompt等三种方法,在使用和不使用框架的效果。特别的,对于每一种Intention类型都进行了详细的对比。我们希望了解,相较于不使用框架的方法,我们的框架对于3种不同的Intention类型的数据,效果分别提升了多少?
% 模型,数据等实验设置与RQ2相同

\subsection{RQ4: Intention-Based Dataset Cleaning}

Lastly, we aim to explore the capability of using intention to enhance data quality. A significant challenge in code refinement data quality is the inconsistency between the content expressed in the review comment and the modifications made in the revised code. Currently, no practical method exists for aligning the review comment and revised code at the semantic level. We will attempt to use intention to determine whether the modifications in the revised code meet the reviewer's requirements.

The Intention Method we designed is as follows: we provide the model with information about the Intention, Original Code, Review Line, and Revised Code and use GPT4o as a classification model to determine whether the modifications in the Revised Code meet the Intention requirements.

We also compared this with a common method that does not use the intention: we provide the model with information about the Review Comment, Original Code, Review Line, and Revised Code and use GPT4o as a classification model to determine whether the revised code meets the requirements of the review comment.

% By systematically evaluating these methods, we aim to demonstrate how intention can potentially improve the alignment between review comments and revised code, thereby enhancing the overall data quality in code refinement tasks.

% 最后,我们要探索,使用Intention来提升数据质量的能力。review comment所表达内容与revised code所修改的内容不一致,是影响code Refinement数据质量的一个难题,目前还没有很好的方法完成review comment和revised code语义级别的对齐。
% 我们将尝试使用Intention来判断revised code的修改是否符合要求。
% 我们设计的方法是:告诉模型Intention,OriginalCode,ReviewLine和RevisedCode等信息,使用Gpt4o作为分类模型,让模型判断RevisedCode的修改是否符合Intention要求。
% 我们对比了不使用Intention的方法:告诉模型ReviewComment,OriginalCode,ReviewLine和RevisedCode,使用Gpt4o作为分类模型,让模型判断RevisedCode是否符合ReviewComment的要求。

\section{experimental results}

\subsection{RQ1: Intention Extraction Accuracy}

\begin{table}[!t]
\caption{\major{Results of intention accuracy.}}
% \vspace{-4mm}
\footnotesize
\centering
\label{tb:rq1-1}
\resizebox{.4\textwidth}{!}{
\begin{tabular}{ccccc}
\hline
           & Explicit & Reversion        & General          & All clean data   \\ \hline
\#Samples  & 175      & 308              & 854              & 1337             \\
\major{GPT3.5}     & -        & \major{84.42\%}          & \major{49.18\%}          & \major{63.95\%}          \\
GPT4o      & -        & \textbf{99.03\%} & \textbf{66.86\%} & \textbf{78.61\%} \\
\major{DeepSeekV2} & -        & \major{95.45\%}          & \major{62.88\%}          & \major{75.24\%}          \\
\major{DeepSeek7B} & -        & \major{80.19\%}          & \major{46.72\%}          & \major{61.41\%}          \\
\major{CodeQwen7B} & -        & \major{82.14\%}          & \major{43.91\%}          & \major{60.06\%}          \\ \hline
\end{tabular}
}
\vspace{-4mm}
\end{table}




\major{
We manually annotated the data to select valid samples, with 1,337 out of a total of 2,000 identified as valid. To ensure annotation quality, we enlisted two senior PhD candidates specializing in code learning to conduct the manual labeling. Disagreements between the annotators were observed in 249 instances, and inter-rater reliability was assessed using Cohen’s Kappa coefficient, which yielded a value of 0.719, indicating acceptable agreement.
Subsequently, we used various models to generate intentions using our framework, and then manually assessed each generated intention for its correctness. Notably, since the Explicit category is classified using predefined rules, we just measure the accuracy of the reversion intention and general intention. 
% As shown in Table~\ref{tb:rq1-1}, 1,337 cases out of the total 2,000 samples were identified as valid. We used different models to predict the intentions and then manually checked the accuracy of these predictions.
% To ensure annotation quality, we invited two senior PhD candidates specializing in code learning to manually label the data. There were 249 instances of disagreement between the annotators, and the inter-rater reliability was evaluated using Cohen's Kappa coefficient, which yielded a value of 0.719, indicating acceptable agreement.
}

% As shown in Table~\ref{tb:rq1-1}, there are 1,337 clean cases, accounting for 66.9\% of the total 2,000 samples. 
\major{As shown in Table~\ref{tb:rq1-1}, we observe that GPT-4o and DeepSeekV2 are significantly more effective than the other three models. Among these, Reversion Intention is relatively easier to extract, with all models achieving at least 80\% accuracy, while GPT-4o achieves an impressive 99.03\%.}


It is noteworthy that 20\% of Reversion Suggestions overlap with General Suggestions, representing a dual classification challenge. In these instances, predicting the intention as either category is considered correct. 
For General Suggestions cases, the highest accuracy is 66.86\%, indicating that this category presents more significant challenges for the model to comprehend. Unlike the other categories, General Suggestions lack specific patterns or structured cues, making it more difficult for the model to pinpoint the exact intention behind the suggestions. This lower accuracy highlights the complexity and ambiguity inherent in general suggestions, which often require a deeper understanding of the context and the underlying reasoning behind the recommendations.



% Among these clean cases, the intention can be correctly predicted in 1,051 cases, which is 78.61\% of the clean samples. This result highlights the ability of the Intention-based Framework to effectively identify and predict intentions in a significant majority of the clean data. Overall, intention correctly predicts 1,371 cases, accounting for 68.6\% of the total samples, demonstrating its overall reliability across different types of suggestions.

% For Explicit Code Suggestions cases, the accuracy of intention prediction is 100\%. This high accuracy is achieved by employing regular expressions to determine whether a case belongs to this category. The rule-based prediction method ensures that each explicit suggestion is captured without errors, reflecting the structured and well-defined nature of these suggestions. Such a robust approach leaves no room for mistakes, making it a perfect classification for these cases.


% In the case of Reversion Suggestions, the accuracy is 99.03\%, reflecting a strong performance with a minimal margin for error. 





% The overall performance of the Intention-based Framework illustrates its potential in accurately predicting intentions across various suggestion types. The ability to achieve 100\% accuracy in Explicit Code Suggestions and near-perfect results in Reversion Suggestions reflects its robustness and precision in structured scenarios. However, the challenges faced in General Suggestions indicate areas for improvement, inviting further exploration into more sophisticated methods for handling complex and ambiguous cases. 

% As shown in Table~\ref{tb:rq1-1}, there are 1,337 clean cases, accounting for 66.9\% of the total 2,000 samples. Among these clean cases, the intention can be correctly predicted in 1,051 cases, which is 78.61\% of the clean samples. Overall, intention correctly predicts 1,371 cases, accounting for 68.6\% of the total samples.

% For Explicit Code Suggestions cases, the accuracy of intention prediction is 100\%, as we use regular expressions to determine whether a case belongs to this category. This rule-based prediction method is error-free.
% For Reversion Suggestions cases, the accuracy is 99.03\%. Notably, 20\% of Reversion Suggestions are also General Suggestions. For these dual classification tasks, predicting the Intention as either category is considered correct.
% For General Suggestions cases, the accuracy is only 66.86\%, indicating that this category is more challenging for the model to understand.

% Overall, for valid samples, the intention extraction accuracy is nearly 80\%, which is very effective. This suggests that using intention to replace review comments in the subsequent task is highly feasible.

% While observing the intention, we found that some cases, although correct, require significant effort to understand the reviewer's intent, even for humans. However, intention can help clarify the task clearly.

% **Case 1:** The review comment, “Any reason not to fold the `currentComponent` assignment into that?”, might be interpreted as asking the developer to comment on the `options.diffed = ()` part of the code, explaining why it is assigned this way. However, considering the previous modifications, the reviewer's real intent is to inquire about the reason for adding the code, implying a suggestion to possibly revert the last modification. Intention can directly discern this meaning and make the correct understanding.

% Additionally, we found during manual labeling that code refinement is a flexible task, often resulting in multiple correct answers for a single task.

% **Case 2:** The review comment, “set the scheduledExecutorService to null?”, can be interpreted in two ways: replacing `this.getScheduledExecutorService().shutdown();` with `this.scheduledExecutorService = null;`, or adding a line to set the service to null after the previous shutdown step. Although the intention and revised code might differ in their representation, both modifications are reasonable to an ordinary programmer.

\vspace{5pt}\noindent \fbox{
\parbox{0.95\linewidth}{\textbf{Answers to RQ1}: \major{GPT-4o achieves the highest accuracy in intention extraction, with DeepSeekV2 performing competitively (with 3\% less). General intentions, however, are more challenging to extract due to the inherent complexity and the often ambiguous nature in the review comments.}}
}


% 从表1可以看出,合理的case共1337个,占样本总数2000个的66.9%。而合理的case中,Intention可以预测正确的有1051个,占合理样本中的78.6%。对总体样本而言,Intention共预测对了1371个,占总体样本的68.6%。接下来具体分析每一个类别的正确率:Explicit code Suggestion类别的Intention预测准确率是100%,这是因为我们是通过正则表达式来判断是否属于第一类,这种rule-based的预测方法不会出错。Reversion Suggestion类别的Intention预测量是99.03%,值得注意的是,Reversion Suggestion有20%的数据同时也是General Suggestion的。也就是说,对于这些数据,如果理解成为要求退回上次修改是正确的Intention,如果按照Intention具体的执行change xxx code to xxx,也是可以正确修复的。最后,对于General Suggestion类别,Intention正确率只有66.86%,说明这一类理解比较困难。总体而言,对于合理样本,Intention的提取正确率接近80%具有非常好的效果,这说明使用Intention在接下来生成revised code步骤中代替review comment,具有较高可行性。

% 我们在观察Intention时,发现了一些case虽然是正确的,但是从即使人来做,也需要很大的努力才能理解其reviewer的意图。然而,Intention可以帮助人很明确的理解任务。
% 在case1中,如果通过review comment:“Any reason not to fold the `currentComponent` assignment into that?”理解,可能是需要developer对options.diffed = () 这部分代码进行注释,解释为什么要在这样赋值。然而综合考虑上一次的修改情况,review的真实意图是询问这次添加代码的原因,隐含了建议是可能要退回上次修改。而Intention可以直接洞察这种含义,做出正确的理解。

% 另外我们在人工标注时发现code Refinement是很灵活的任务,经常会出现一个任务存在多种正确回答的现象。
% 在case2中,review comment:“set the scheduledExecutorService to null?”,可以理解成把this.getScheduledExecutorService().shutdown();代码替换成this.scheduledExecutorService = null;。也可以理解成在前一步shutdown之后,再添加一行代码将service设为null。即使Intention和revised code表现结果不同,但是普通程序员看来,这两种修改都是合理的。

% 总之,Intention对合理数据的理解正确率高达78.61%,可以作为理解程序的一个可靠方式。


% \begin{table}[!t]
% \caption{Comparative Results with Pre-trained Models.}
% % \vspace{-4mm}
% \footnotesize
% \label{tb:rq2-2}
% \resizebox{.49\textwidth}{!}{
% \begin{tabular}{ccccccccc}
% \hline
%    & \multicolumn{5}{c}{Intention-based   Framework}           &  & \multicolumn{2}{c}{Pre-trained Model} \\ \cline{2-6} \cline{8-9} 
%    & GPT3.5 & GPT4o & DeepSeekV2     & DeepSeek7B & CodeQwen7B &  & CodeReviewer          & T5CR          \\ \hline
% EM & 53.40  & \textbf{64.77} & \major{64.25} & 49.29      & 48.47      &  & 52.13                 & 18.84         \\ \hline
% \end{tabular}
% }
% \vspace{-4mm}
% \end{table}



\begin{table*}[!t]
\caption{\major{Comparative results between our intention-based framework and LLM-based baselines.}}
% \vspace{-4mm}
\footnotesize
\label{tb:rq2-1}
% \vspace{-2mm}
\resizebox{.99\textwidth}{!}{
\begin{tabular}{ccccccccccccccccccccccccc|cc}
\hline
           & \multicolumn{3}{c}{Simple Prompt} &  & \multicolumn{3}{c}{Simple COT} &  & \multicolumn{3}{c}{Tufuno COT} &  & \multicolumn{3}{c}{Random fewshot} &  & \multicolumn{3}{c}{RAG} &  & \multicolumn{3}{c}{Self-generated} &  & \multicolumn{2}{c}{Intention   Framework with RAG} \\ \cline{2-4} \cline{6-8} \cline{10-12} \cline{14-16} \cline{18-20} \cline{22-24} \cline{26-27} 
           & Basic     & P-A       & Comp.     &  & Basic    & P-A      & Comp.    &  & Basic    & P-A      & Comp.    &  & Basic      & P-A       & Comp.     &  & Basic  & P-A    & Comp. &  & Basic      & P-A       & Comp.     &  & Same             & \major{GPT4o Inte.}            \\ \hline
GPT3.5     & 38.00     & 38.29     & 36.42     &  & 34.85    & 36.87    & 32.76    &  & 19.97    & 17.20    & 11.74    &  & 45.25      & 48.62     & 32.39     &  & 46.45  & 49.89  & 33.96 &  & 31.86      & 25.88     & 27.97     &  & 53.40                & \major{\textbf{58.86}}              \\
GPT4o      & 41.14     & 46.60     & 29.54     &  & 39.72    & 43.68    & 26.25    &  & 20.12    & 24.98    & 13.76    &  & 49.21      & 52.88     & 35.00     &  & 49.59  & 54.08  & 36.95 &  & 51.01      & 56.84     & 34.41     &  & 64.77                & \major{\textbf{64.77}}              \\
DeepSeekV2 & 34.63     & 38.07     & 27.60     &  & 41.29    & 46.37    & 31.49    &  & 24.31    & 27.60    & 17.43    &  & 46.45      & 52.66     & 42.41     &  & 48.09  & 55.35  & 45.62 &  & 44.95      & 53.03     & 38.37     &  & 64.25                & \major{\textbf{68.06}}              \\
DeepSeek7B & 28.50     & 34.48     & 24.38     &  & 29.92    & 36.28    & 24.01    &  & 5.76     & 6.24     & 3.60     &  & 33.28      & 39.49     & 25.58     &  & 36.05  & 41.59  & 25.88 &  & 7.26       & 8.68      & 5.01      &  & 49.29                & \major{\textbf{56.40}}              \\
CodeQwen7B & 26.10     & 33.13     & 19.22     &  & 25.06    & 31.04    & 16.98    &  & 2.85     & 3.29     & 1.97     &  & 35.75      & 40.99     & 17.80     &  & 35.83  & 44.05  & 20.12 &  & 13.69      & 19.67     & 7.93      &  & 48.47                & \major{\textbf{54.75}}              \\ \hline
\end{tabular}
}
\vspace{-4mm}
\end{table*}

\subsection{RQ2: Intention-based Framework Effectiveness}\label{sec:rq2}

\major{Table~\ref{tb:rq2-1} presents a comparison between our method and the LLM-based baselines. For the LLM baselines, we made extensive efforts to optimize their performance by setting multiple configurations, including six different prompts and three types of inputs (i.e., basic inputs, position-aware code refinement, and comprehensive code refinement; details are provided in Section~\ref{sec:background}).
For our method, we show the results of the RAG-based approach and the results with other prompts are shown in Table~\ref{tb:rq3-1}. Note that we present two types of results for our method: (1) using the same model for both intention extraction and code refinement (Column \textit{Same}) and (2) extracting accurate intentions with GPT-4o and generating revised code using other LLMs based on these intentions (Column \textit{GPT4o Inte.}).}



% We then compare the performance of other prompt methods based on large models.
From Table~\ref{tb:rq2-1}, we can observe that using the Intention-based method improved the performance of all models compared to all baselines. The model with the highest improvement was DeepSeekV2, which saw an increase from 55.35\% to 64.25\%, a 9 percentage points improvement. The model with the least improvement was GPT3.5, which increased from 49.89\% to 53.40\%, a mere 3 percentage points improvement.  We provide several representative examples to demonstrate the advantages of the Intention-based framework, which are available on our website~\cite{IntentionWebsite}.
% 我们展示了一些有代表性的例子以说明Intention-based framework的优势,相较于其他baseline方法,做到了理解任务再完成任务。限于文章篇幅,我们将例子放到网站中。

\noindent
\major{
\textbf{From the Perspective of Intention Quality}:
Comparing the results of our method using intentions generated by the model itself and those generated by GPT4o, it is evident that intention quality plays a critical role. While using the intentions generated by the same model (Column \textit{Same}) generally improves performance compared to LLM baselines, leveraging higher-quality intentions (Column \textit{GPT4o Inte.}) further enhances the results. This is due to the fact that some models, as shown in Table~\ref{tb:rq1-1}, are not as effective as GPT4o in extracting accurate intentions. Surprisingly, we also observed that DeepSeekV2 outperformed GPT4o (68.06 VS 64.77) when provided with the same intentions (from GPT4o). This demonstrates that, while DeepSeekV2 may not excel in intention extraction, it has superior code refinement capabilities when given accurate intentions. The results underscore the strength of our framework, which allows for separate optimization of the agents responsible for intention understanding and code refinement under a given intent. A similar trend is observed for other prompts, as shown in Table~\ref{tb:rq3-1}.
}

\noindent
\textbf{From the Perspective of Input Complexity:} By comparing the baseline results, we found that across all six prompt strategies, the Position-Aware Code Refinement task performed the best, followed by the Basic Code Refinement task, and lastly the Comprehensive Code Refinement task. Although the Comprehensive Code Refinement task provided the most information, including all necessary data, all models and prompt methods struggled to effectively utilize this information. This may be due to the complexity of understanding \texttt{LastCodeDiffHunk} and the potential for overly long prompts to reduce model focus, impairing comprehension of key information. Incomplete input information reduces model effectiveness, while excessive input information hampers task understanding and reduces effectiveness. The Position-Aware Code Refinement task, which includes \texttt{OriginalCode}, \texttt{ReviewComment}, and \texttt{ReviewLine}, balances the completeness of information with the model's processing capability (5 percentage points better than Basic Code Refinement and about 10 percentage points better than Comprehensive Code Refinement in general). 

% We observed that under prompt strategies like Simple Prompt, Simple COT, Random few-shot, and RAG, the Position-Aware Code Refinement task generally performed about 5\% better than Basic Code Refinement and about 10\% better than Comprehensive Code Refinement.

\noindent
\textbf{From the Perspective of Prompt Strategy:} 
% discrepancies between our implementation and the described methodology in the paper, as we did not have access to the original code. Additionally, the prompt design lacked detailed execution steps, failing to enhance model performance effectively.
Except for the DeepSeekV2 model, other models showed only slight improvements with the Simple COT Prompt method over the Simple Prompt, and even showed declines for GPT3.5 and GPT4o in the Basic task. This indicates that for most models like GPT3.5, GPT4o, DeepSeek7B, and CodeQwen7B, Simple COT Prompt without clear step-by-step guidance does not significantly enhance code refinement tasks. However, for the DeepSeek model, using COT significantly improved accuracy in Basic, Position-Aware, and Comprehensive tasks by up to 8 percentage points, demonstrating DeepSeekV2's strong reasoning capabilities. 
% The Tufano COT method performed significantly worse than others. This may be due to that the prompt design lacks detailed steps, failing to enhance model performance effectively.

Consistent with our intuition, the Random Few-Shot Prompt outperformed the Simple COT Prompt and Simple Prompt. Furthermore, RAG demonstrated superior performance compared to the Random Few-Shot Prompt across almost all models and tasks. \major{This indicates that retrieving more similar examples is highly beneficial, as it provides refinement guidance that aligns closely with both the format and the content of the task.
Therefore, we recommend that users intending to use the RAG prompt for code review tasks construct the retrieval database using a relevant historical code review dataset. Relevance can be assessed from various perspectives, including project similarity, task similarity, author alignment, dataset quality, and fine-grained intention matching. Specifically, retrieved data from the same or similar project, addressing similar tasks, or authored by the same individuals or the same group is likely to offer better guidance for new code refinement tasks, as these examples share meaningful similarities and contextual relevance.}


For Self-generation Prompt, compared to the RAG method, it showed a slight improvement for the GPT4o model, a slight decline for the DeepSeekV2 model, and a significant decline for the other three models. Notably, GPT4o and DeepSeekV2, the two largest models with the best overall performance in other prompt methods, benefited from Self-generation. This suggests that Self-generation Prompt is more effective for models with large parameter sizes and strong reasoning capabilities. 



Last, we compared our method with pre-trained models, specifically CodeReviewer~\cite{li2022automating} and T5CR~\cite{tufano2022using}, which achieved 52.14\% and 18.84\% accuracy, respectively. As expected, our intention-based method demonstrates superior effectiveness, achieving significantly higher performance due to the use of LLMs and the incorporation of intention extraction (e.g., 64.77\% with GPT4o).


% We also compared our method with that of the pre-trained models, i.e., CodeReviewer and T5CR, which achieve 52.14\% and 18..84\%, respectively. It is expected that our intention-based method is more effective due to the use of LLMs and the intention extraction. (e.g., 64.77\% with GPT4o).

% It is expected that 
% According to Table~\ref{tb:rq2-2}, we observe that all five models perform better with the Intention-based framework than with the pre-trained models. Among them, the GPT4o model achieves the best performance, reaching 64.77\%, which is 12.64 percentage points higher than the CodeReviewer model (52.13\%). 

\vspace{5pt}\noindent \fbox{
\parbox{0.95\linewidth}{\textbf{Answers to RQ2}: The results show that the Intention-based framework outperforms the existing baselines, including both pre-trained models and LLMs with diverse prompts. 
\major{High-quality intentions play a critical role in achieving effective code refinement.}
% Specifically, for the DeepSeekV2 model, the framework achieves an Exact Match (EM) accuracy of 68\%, which is 13\% higher than the baseline. 
% Additionally, we observed that among these baseline methods, using large models for the code refinement task yields the best performance for the Position-Aware Code Refinement task, and the most effective prompt strategy is RAG.
}
}


\begin{table}[!t]
\centering
\caption{\major{Results of our method with different prompts.}}
\footnotesize
\label{tb:rq3-1}
\resizebox{.49\textwidth}{!}{
\begin{tabular}{cccclccc}
\hline
           & \multicolumn{3}{c}{Our Method (Same)}                    &  & \multicolumn{3}{c}{\major{Our Method (GPT4o   Inte.)}} \\ \cline{2-4} \cline{6-8} 
           & Simple Prompt  & RAG            & Self-generated &  & \major{Simple Prompt}  & \major{RAG}             & \major{Self-generated} \\ \hline
GPT3.5     & 53.03          & \textbf{53.40} & 43.38          &  & \major{57.14}          & \major{\textbf{58.86}}  & \major{46.60}          \\
GPT4o      & 64.10          & 64.77          & \textbf{65.97} &  & \major{64.10}          & \major{64.77}           & \major{\textbf{65.97}} \\
DeepSeekV2 & 60.88          & \textbf{64.25} & 61.03          &  & \major{61.03}          & \major{\textbf{68.06}}  & \major{64.10}          \\
DeepSeek7B & \textbf{50.04} & 49.29          & 26.55          &  & \major{54.45}          & \major{\textbf{56.40}}  & \major{27.75}          \\
CodeQwen7B & 46.00          & \textbf{48.47} & 29.32          &  & \major{50.11}          & \major{\textbf{54.75}}  & \major{27.45}          \\ \hline
\end{tabular}
}
\vspace{-4mm}
\end{table}

% 从表2中,我们首先可以看到,所有模型,使用Intention-based方法对比所有baseline均有提升。提升最高的模型是deepseek,从55.35%提升到了68.06%提升了13%。提升最少的模型是GPT3.5,从49.89%提升到53.03%只提升了3%。

% 我们也从baseline的结果,得到更多的结论。

% 从任务角度观察:我们通过对比baseline的结果发现:在所有6种不同的策略下,Position-Aware code Refinement任务都是最优的,其次是Basic code Refinement,最后是comprehensive code Refinement。虽然comprehensive code Refinement任务提供了最多的信息,提供了全部的数据所需要的信息,但是所有模型,所有prompt方法都不能很好的理解这些信息。这一方面可能是由于code diff是一种难以理解的结构,另一方面可能是因为过长的prompt会降低模型的注意力,反而降低了对关键有效信息的理解。
% 输入信息不全会降低模型的效果;输入信息太多会影响模型对任务的理解,也会导致降低模型的效果。Position-Aware code Refinement任务输入是original code,review comment和review line信息,很好的平衡了信息完整程度和模型对信息的处理能力。我们观察到simple prompt,simple cot,Random fewshot,RAG等prompt策略之下,大部分情况Position-Aware code Refinement任务效果比Bacis code Refinement的效果高5%左右,比comprehensive code Refinement效果高10%左右。

% 从prompt策略角度观察:Tufuno COT的效果远低于其他,这一方面是因为我们是根据她论文描述而写的,没有找到她的代码实现,可能跟他的实现有出入。另一方面是因为prompt设计时也没有详细执行步骤,未能有效提高模型效果。

% 另外可以观察到,除了deepseek模型以外,其他模型在simple COT方法的效果只是略优于simple prompt,甚至对于Basic任务中,GPT3.5和GPT4o的效果还有下降。这说明对于GPT3.5,GPT4o,deepseekcode,code-qwen等模型,没有明确步骤指导的COT对code refinement任务提升不大。而对于deepseek模型,在Basic,P-Aware,Comprehensive等三个任务上,使用COT可以显著提升准确率,最多8%。这说明deepseek具有较强的推理能力。

% 与我们的直觉一致的,random fewshot的效果好于simple cot和simple prompt。而RAG的效果也好于random fewshot。这对几乎对所有模型和所有任务都成立。

% 最后,我们尝试了self-generation方法。相比于RAG方法,我们发现只有对GPT4o模型效果有小幅提升,deepseek模型小幅下降,其他三个模型效果都大幅下降。恰巧GPT4o和deepseek是参数量最大的两个模型,也是在其他prompt方法综合效果最好的两个模型。这说明只有对参数规模较大,自身推理能力较强的模型,self-generation方法才能充分发挥效果。

% 总之,我们在5个模型,3种类型的code Refinement任务上充分比较了现有的6种prompt方法的效果。发现对模型的输入信息不能过多也不能过少,要平衡信息完整性和模型理解能力,效果最好的任务是Position-Aware。还发现了对于一般的模型RAG是最优的prompt策略,而对于参数量大推理能力强的模型,self-generation也是很好的prompt策略。而我们提出的Intention框架,充分利用了所有信息,效果比目前所有的prompt策略都要高3%-13%。

% % 设置表格中所有文本颜色为蓝色
% \begingroup
% \color{blue}  % 将以下内容颜色设置为蓝色
\begin{table*}[!t]
\caption{\major{Ablation results on different intention-based components.}}
\footnotesize
\label{tb:rq3-3}
\resizebox{.99\textwidth}{!}{
\major{
\begin{tabular}{cccccccccccccccc}
\hline
           & \multicolumn{3}{c}{Our Method (with Intention)}         &  & \multicolumn{3}{c}{w/o Explicit  Intention} &  & \multicolumn{3}{c}{w/o Reversion   Intention} &  & \multicolumn{3}{c}{w/o General Intention} \\ \cline{2-4} \cline{6-8} \cline{10-12} \cline{14-16} 
           & Simple Prompt & RAG            & Self-generated &  & Simple Prompt    & RAG             & Self-generated   &  & Simple Prompt    & RAG             & Self-generated    &  & Simple Prompt         & RAG                 & Self-generated        \\ \hline
GPT3.5     & 57.14         & \textbf{58.86} & 46.60          &  & 56.25(-0.9)      & 56.39(-2.47)    & 42.63(-3.96)     &  & 49.59(-7.55)     & 51.6(-7.26)     & 35.53(-11.07)     &  & 51.53(-5.61)          & 58.41(-0.45)        & 44.05(-2.54)          \\
GPT4o      & 64.10         & 64.77          & \textbf{65.97} &  & 64.03(-0.07)     & 63.65(-1.12)    & 65.45(-0.52)     &  & 59.61(-4.49)     & 60.21(-4.56)    & 61.41(-4.56)      &  & 57.37(-6.73)          & 62.3(-2.47)         & 65.75(-0.22)          \\
DeepSeekV2 & 61.03         & \textbf{68.06} & 64.10          &  & 60.58(-0.45)     & 65.52(-2.54)    & 61.41(-2.69)     &  & 55.27(-5.76)     & 61.85(-6.21)    & 56.32(-7.78)      &  & 52.35(-8.68)          & 63.95(-4.11)        & 63.05(-1.05)          \\
DeepSeek7B & 54.45         & \textbf{56.40} & 27.75          &  & 53.18(-1.27)     & 53.48(-2.92)    & 22.59(-5.16)     &  & 45.85(-8.6)      & 44.2(-12.19)    & 8.97(-18.77)      &  & 48.62(-5.83)          & 52.51(-3.89)        & 29.39(+1.65)          \\
CodeQwen7B & 50.11         & \textbf{54.75} & 27.45          &  & 46.97(-3.14)     & 52.81(-1.94)    & 19.75(-7.7)      &  & 38.0(-12.12)     & 46.6(-8.15)     & 6.96(-20.49)      &  & 47.8(-2.32)           & 54.75(0.00)          & 39.72(+12.27)         \\ \hline
\end{tabular}
}
}
\end{table*}
% \endgroup


\subsection{RQ3: Ablation Study on Different Intention Categories}

\major{Table~\ref{tb:rq3-3} presents the ablation study results. The first column shows the performance with all intention-handling components included, while the subsequent columns illustrate the effects of removing intention-handling for each category (i.e., removing each agent in Fig.~\ref{fig:framework}). For example, removing the Explicit Intention component means that code reviews with explicit suggestions are instead handled by Agent 2 and Agent 3. Similarly, removing the Reversion Intention redirects code reviews with reversion suggestions to Agent 1 and Agent 3. For Agent 3, the general intention extraction is removed, and LLMs are used to handle these cases directly.
The results indicate that removing any of these intention components leads to performance drops across almost all scenarios, demonstrating the importance of each component. Notably, removing the Reversion Intention results in significant declines in EM scores across all models and prompt methods, with reductions ranging from 5 to 15 percentage points, highlighting its critical role in improving this category of code reviews.
}

\major{An exception is observed with the Self-generated prompt under the condition of ``w/o General Intention'', where performance increases slightly. This is likely because Self-generated prompts tend to perform worse with LLMs that have weaker reasoning capabilities, as discussed in RQ2. However, the Simple Prompt shows a significant drop, indicating that the intention-based method provides substantial improvements when using very basic prompts. In contrast, RAG-based and Self-generated prompts exhibit relatively smaller performance declines. These advanced prompts compensate for the models' limitations in implicitly understanding intentions, even when intentions are not explicitly provided.}

% We conducted ablation experiments on the three agents used in the framework, and the results are presented in Table 6. First, analyzing the impact of Agent1 on the Exact Match (EM) scores, we observe that removing Agent1 causes a decline in the EM values for all models and all prompt methods. Notably, weaker models such as GPT-3.5, CodeQwen7B, and DeepSeek7B exhibit a significant drop of approximately 6 percentage points, whereas stronger models like GPT-4o and DeepSeekV2 experience a smaller decrease of 1-3 percentage points. 
% Next, examining the impact of Agent2, we find that its removal leads to a substantial decline in the EM scores across all models and prompt methods, with reductions ranging from 5 to 15 percentage points. 
% Finally, considering the influence of Agent3, we observe that it provides a notable improvement for stronger models when using the Simple Prompt approach (e.g., GPT-4o and DeepSeekV2 improve by 6.8 and 8.5 percentage points, respectively) and a slight improvement when using the RAG method (e.g., GPT-4o and DeepSeekV2 improve by 2.4 and 0.3 percentage points, respectively). However, for weaker models, Agent3 only shows a slight benefit for Simple Prompt (e.g., GPT-3.5 and DeepSeek7B), while for the RAG method, its influence is negligible or even detrimental in most cases. This outcome can be attributed to Agent3's ability to extract intentions that help the models better understand the reviewers' goals under the Simple Prompt setting. In contrast, the RAG method compensates for the models' inherent deficiencies in interpreting intentions, reducing the added value of Agent3. 
% Further analysis reveals that weaker models occasionally produce incorrect intention extraction in Agent3. When replacing the extracted intentions with those generated by GPT-4o, all models' performance improves, especially for the RAG method, where all models see an improvement of more than 4 percentage points.
% These findings highlight that Agent1 and Agent2 significantly enhance the results, while Agent3 offers improvements primarily for stronger models. However, for weaker models, particularly under the RAG method, Agent3 provides limited or no benefit.}


\begin{table*}[!t]
\caption{\major{Improvement results with different intention-based components.}}
\footnotesize
\label{tb:rq3-2}
\resizebox{.99\textwidth}{!}{
\begin{tabular}{cccccccccccc}
\hline
           & \multicolumn{3}{c}{Simple Prompt}           &  & \multicolumn{3}{c}{RAG}                     &  & \multicolumn{3}{c}{Self-generated}          \\ \cline{2-4} \cline{6-8} \cline{10-12} 
           & Explicit & Reversion & General (\major{Same/GPT4o}) &  & Explicit & Reversion & General (\major{Same/GPT4o}) &  & Explicit & Reversion & General (\major{Same/GPT4o}) \\ \hline
GPT3.5     & 29.71    & 62.01     & 2.86/\major{12.78}           &  & 8.57     & 57.47     & -9.57/\major{2.64}           &  & 38.29    & 70.13     & -1.29/\major{5.73}           \\
GPT4o      & 6.86     & 74.35     & 13.22/\major{13.22}          &  & 2.29     & 68.83     & 4.85/\major{4.85}            &  & 4.00     & 71.10      & 0.44/\major{0.44}            \\
DeepSeekV2 & 21.14    & 76.95     & 16.74/\major{16.45}          &  & 7.43     & 67.53     & 0.59/\major{7.49}            &  & 13.14    & 67.53     & -3.96/\major{1.47}           \\
DeepSeek7B & 25.71    & 65.91     & 2.71/\major{11.75}           &  & 14.29    & 60.71     & -6.14/\major{8.52}           &  & 38.86    & 73.38     & -5.43/\major{-3.23}          \\
CodeQwen7B & 29.71    & 67.21     & -3.43/\major{6.75}           &  & 24.00    & 57.14     & -12.00/\major{0.00}          &  & 50.29    & 68.51     & -19.86/\major{-22.91}        \\ \hline
\end{tabular}
}
\vspace{-4mm}
% \vspace{-4mm}
\end{table*}

% We experimented with different models under the Intention framework using three distinct prompt strategies: Simple Prompt, RAG, and Self-generated Prompt. The results are presented in Table~\ref{tb:rq3-1}. We can observe that DeepSeek7B achieved the best results with the Simple Prompt strategy, GPT4o achieved the best results with the Self-generated Prompt strategy, and the other three models achieved high accuracy with the RAG Prompt strategy.

% Comparing the Self-generated Prompt with the RAG Prompt, the results align with our previous analysis in RQ2: the larger the model's parameters and the stronger its reasoning capabilities, the better it performs with the Self-generated Prompt. For example, GPT4o and DeepSeekV2 achieved 65.97\% and 64.10\% accuracy, respectively. Conversely, other models perform much worse with the Self-generated Prompt compared to other prompt strategies.

% Notably, even with the Simple Prompt method, all models still achieved very high accuracy. Compared to the Simple Prompt result in Table~\ref{tb:rq2-1}, the accuracy of the five models increased by 15 to 23 percentage points. This indicates that the Intention framework makes it easier for models to understand reviewer's requirements.

\major{To further understand the improvements that are caused by our three intention-based components, we calculate the EM improvements on the test samples that are handled by each component, i.e., the EM difference compared to the best performance of the LLM baselines (see Table~\ref{tb:rq2-1}). Note that, due to that the intention quality is important, similar to RQ2, we show both results using intentions extracted by the same model and GPT4o. 
% The results are shown in Table~\ref{tb:rq3-2}.
}

% Next, we compare the improvement of each category using the Intention-based framework with Simple Prompt, RAG, and Self-Generated Prompt strategies against the ordinary tasks. For better demonstration of the Intention-based framework's effect, we chose the Position-Aware task, which performed the best in RQ2, as the control group.

As shown in Table~\ref{tb:rq3-2}, the Intention-based framework demonstrates improvements in most cases. Notably, the most significant improvement is observed in the Reversion Suggestion category, with accuracy increases ranging from 57 to 76 percentage
points. This substantial gain is attributed to the rule-based refinement strategy employed for Reversion Suggestions, which ensures correct refinement when the given sample is accurately classified into this category during the intention analysis.

For the Explicit Code Suggestion category, our framework also shows improvement. For relatively weaker models like GPT3.5, DeepSeek7B, and CodeQwen7B, the improvement is particularly notable, often exceeding 20 percentage points, with a maximum increase of 50 percentage points. For stronger models like GPT4o and DeepSeekV2, it still provides slight improvements. 
\major{This is because explicit code suggestions are relatively easier for stronger models to handle, as the explicit intentions are more straightforward for them to extract. However, weaker models struggle to effectively extract these intentions and refine code, resulting in more significant improvements when using our method.}
% This is due to the repair rules used in the framework, which can correct errors generated by the model.
\major{For tasks in the General Suggestion category, we observed that our method improves results when high-quality intention extraction (e.g., using GPT4o) is employed. However, when using the same model for both intention extraction and code refinement, the results decrease in some cases. This is because weaker models may extract incorrect intentions, which not only fail to assist the refinement process but can also lead to the generation of incorrect code.}

\major{Additionally, with the Simple Prompt strategy, the overall improvement is notable since the original model performs poorly, leaving substantial room for improvement. On the other hand, for RAG-based and Self-generated prompts, the overall performance is already enhanced, so the use of low-quality intentions (extracted by the same model) can more easily degrade performance.}

\major{
Furthermore, we found that the improvement in the General Suggestion category (even with high-quality intentions) is lower compared to Explicit Suggestion and Reversion Suggestion categories. This is because intentions in General Suggestions remain relatively high-level and less concrete, making it more challenging for the model to execute precise refinements compared to the other two categories.}

% the framework provides slight improvements for the stronger models, GPT4o and DeepSeekV2. Specifically, for the RAG prompt method, the improvements are 4.85 and 7.49 percentage points, respectively. This could be attributed to the repair strategy within the framework or more effective indexing after intention extraction, leading to more relevant task retrieval. However, for the three weaker models, GPT3.5, DeepSeek7B, and CodeQwen7B, the framework shows almost no improvement for these tasks, and in some cases, the performance decreases. Particularly with the RAG method, the performance drops by 6 to 12 percentage points. This might be because weaker models extract less accurate intentions, and using these intentions for indexing can lead to retrieving irrelevant examples, thereby interfering with the results.





\vspace{5pt}\noindent \fbox{
\parbox{0.95\linewidth}{\textbf{Answers to RQ3}: \major{The results demonstrate that all three components contribute to performance improvement. However, their effectiveness varies. The Reversion Intention component achieves the largest improvement (over 50 percentage points) due to its rule-based code refinement, which ensures high accuracy. In contrast, the General Intention component shows relatively less improvement, as its intentions are less concrete compared to Explicit and Reversion Intentions, introducing larger uncertainty into the refinement process. Moreover, performance can decrease if the General Intention is not accurately extracted.}
}
}

% The results show that using the Intention-based framework improves performance for all models. 对Reversion Suggestion类型的任务提升50%以上。对Explicit Code Suggestion类型任务提升20%以上。如果使用比较强的模型,如GPT4o和DeepSeekV2,对General Suggestion tasks也有小幅提升4.85\% and 7.49\%。
% Strong models show improvements across Explicit Code Suggestion, Reversion Suggestion, and General Suggestion tasks. Weaker models only show improvements in the first two tasks, with performance decreasing in the third task. 


% 我们尝试了不同模型在Intention框架下,使用simple prompt,rag,self-generation三种不同的prompt策略。结果如表3所示,我们可以观察到,code-deepseek在simple prompt策略下取得最好结果。GPT4o在self-generation下取得最好结果,其余三个模型都在RAG的prompt策略下取得高的准确率。

% 对比self-generation方法与RAG方法,与之前在RQ2中分析的结果一致,模型参数量越大推理能力越强,在self-generation方法的效果越好,例如GPT4o和deepseek效果分别为65.97%和64.10%。反之,一般的模型在self-generation上的效果相较于其他prompt策略会落后很多。

% 另外值得注意的是,即便只是使用最简单的simple prompt方法,所有模型仍可取得非常高的准确率。相较于表2中的simple prompt方法,五个模型分别提升了15%-23%的正确率。这说明Intention的框架使模型更加容易理解任务的需求。

% 而后我们再比较一下相较于普通任务中的simple prompt,RAG,self-generation策略,使用Intention框架后对每一类别的提升。为了更好的展示Intention的效果,对照组我们选择了在RQ2中表现最好的Position-Aware任务。

% 如表4所示,几乎大部分情况,使用Intention框架都有提升。特别的,对于Reversion Suggestion任务提升最为明显,均提升57%-76%之间。这是因为Intention框架中对Reversion Suggestion的修复策略是基于规则的,如果能在extraction Intention过程中正确分类,识别出任务属于这一类Intention,那么后续修复就保证修复正确。
% 其次,Explicit code Suggestion类别,使用Intention框架也均有提升。对于相对弱一些的模型,如GPT3.5,deepseekcoder,code-qwen,提升尤为明显,大部分情况可以提升20%以上,最高提升有50%。而对于能力较强GPT4o和deepseek,Intention框架也可以小幅提升。这是因为Intention框架中使用了repair规则,可以修复模型生成中的错误。
% 最后,对于general的任务,我们发现对于能力较强的模型GPT4o和deepseek,Intention框架仍可以进行小幅提升。特别的,对于效果最好的RAG prompt方法,提升分别为4.85%和7.49%。这可能要归功于Intention框架中的repair策略,或者是Intention提取后使得索引更有效,可以索引到更相近的任务。然而,对于能力较弱的3个模型,GPT35,deepseekcoder,codeqwen,general任务中Intention方法就几乎没有提升,反而效果会下降,特别是RAG方法上,效果下降最明显,下降了6%-12%。这可能是因为弱模型本身生成Intention的正确率就不高,再继续用Intention做索引会索引到不相关的例子,反而干扰了结果。

% 总之,无论是能力强的模型还是能力弱的模型使用Intention框架效果都会有提升。能力强的模型在Explicit code Suggestion,Reversion Suggestion,general Suggestion三个任务上都有提升。而能力弱的模型只在前两个任务上有提升,第三个任务上会有下降。

\subsection{RQ4: Intention-Based Dataset Cleaning}



\begin{table}[!t]
\centering
\caption{Effectiveness on data cleansing.}
\scriptsize
\label{tb:rq4}
\begin{tabular}{ccccccc}
\hline
          & TP   & FP  & TN  & FN  & Accuracy & Precision \\ \hline
Intention-based & 1076 & 110 & 553 & 261 & 81.45\%  & 90.73\%   \\
Comment-based    & 1201 & 312 & 351 & 136 & 77.60\%  & 79.38\%   \\ \hline
\end{tabular}
\vspace{-4mm}
\end{table}

We further investigated the role of intention in data cleaning. As shown in Table~\ref{tb:rq4}, compared to directly using the review comment to verify whether the code modification meets the reviewer’s requirements, using Intention to verify the code modification proved to be more effective. The accuracy increased from 77.60\% to 81.45\%, and the precision increased from 79.38\% to 90.73\%.

% The improvement in True Positives (TP) and the reduction in False Positives (FP) also highlights the effectiveness of the Intention-based approach. Specifically, the True Positives for the Intention-based method are 1076, compared to 1201 for the Comment-based method, showing a more conservative approach that reduces unnecessary inclusions. The False Positives dropped significantly from 312 to 110, indicating a more precise identification of valid code modifications. Similarly, True Negatives (TN) rose from 351 to 553, and False Negatives (FN) increased slightly from 136 to 261, showing that while some valid changes were missed, the overall precision improved.

Such improvements in accuracy and precision underscore the significance of utilizing intention as a guiding principle for code verification. Since dataset construction places a higher emphasis on data quality, the 12\% improvement in precision is significant for enhancing data quality. This increase ensures that the cleansed data is not only more accurate but also more reliable for downstream applications and analysis.

Overall, the intention-based approach demonstrates a more balanced and effective methodology for ensuring that code modifications align closely with the original reviewer's intentions, resulting in cleaner and more precise datasets. This shift toward a more intention-driven process marks a substantial advancement in the field of data cleansing, providing developers and data scientists with a more robust tool for maintaining code integrity and quality.

\vspace{5pt}\noindent \fbox{
\parbox{0.95\linewidth}{\textbf{Answers to RQ4}: The results indicate that intention-based cleansing is more effective for code refinement data cleansing than comment-based methods, achieving an accuracy of 81\% and a precision of 91\%.
}
}

% 我们进一步研究Intention在筛选数据中的作用。结果如表5所示,相较于直接用review comment检验代码修改是否符合要求,使用Intention来检验代码修改是否符合要求的效果更好。准确率从78%提升到了81%,而精确率从79%提升到了91%。因为构造数据集更在意数据质量,精确率提升的12%对数据质量的提升意义很重要。
\section{DISCUSSION}
\subsection{Discussions on Intention Classification and Extraction}


\major{Our framework divides the code refinement task into two steps: \textbf{intention analysis} and \textbf{intention-guided refinement}. This separation allows for improvements in effectiveness by enhancing both steps individually. Ideally, we aim to extract intentions as concretely as possible, such as Explicit and Reversion Intentions, which simplify the following refinement process. However, when the intention is less concrete (e.g., for General Suggestions), the refinement process involves more understanding difficulties, making performance improvements less significant.}

\major{
Our results also indicate that inaccurate intention extraction can degrade performance compared to an end-to-end refinement approach (see RQ3). This highlights why we only extract high-level intentions for General Suggestions, as it is challenging to ensure fully accurate and concrete intention extraction in these cases. Extracting incorrect intentions can lead to misguided refinements, which we aim to avoid.
}

\major{
In the future, refining the categories of General Intentions could further enhance the refinement process. For example, if the categories are more concrete, a rule-based method can be easily designed, or weaker models may better understand and refine the code. An ideal scenario would be that we have a complete classification of intention categories, where each category is sufficiently concrete to allow the use of reliable rule-based methods or even very weak models for effective refinement.
Explicit Suggestions and Reversion Suggestions are prime examples of such concrete categories.}



\subsection{Threats to Validity}
\noindent \textit{Model Threats:} We only tested the 7B versions of open-source code models, DeepSeek7B and CodeQwen7B, due to the resource limit. However, based on the performance of general open-source models like GPT4o and DeepSeekV2, our Intention-based framework performs excellently in code refinement tasks compared to other prompt techniques.
\major{Another potential threat is the length of the prompt templates. While longer prompts can pose input challenges for some models, the longest prompt template used in our study (the RAG prompt) contains only 141 tokens. The fields for each case (e.g., \texttt{OriginalCode}, \texttt{ReviewLine}, \texttt{Intention}) typically remain below 200 tokens. Even when including three-shot examples, the total input length remains under 1,000 tokens. Therefore, the prompt length is unlikely to affect the validity of our results.}

 \noindent \textit{Data Threats:} We only selected CodeReviewer dataset because other datasets lack some data fields and do not provide the link to the original data, making it impossible to use the Intention framework. However, the CodeReviewer dataset has been used in many papers and is recognized as a relatively complete and objective dataset. In the future, we will also try to collect more comprehensive and higher-quality datasets.
\major{We also acknowledge the potential risk of data leakage, particularly when using LLMs. While it is challenging to entirely rule out the possibility of data leakage within LLMs, our experimental results demonstrate that utilizing the Intention framework consistently yields better outcomes compared to LLMs not using it. This indicates that even in scenarios where data leakage may occur, the framework's design and methodology provide a significant performance advantage.}

 \noindent \textit{Efficiency Threats:} Our framework involves multiple LLM calls for classification and code generation, potentially making it slightly less efficient than other prompt strategies. However, the model’s classification response speed is relatively fast, and for Explicit Code Suggestions and Reversion Suggestions, we only need one step LLM call. Therefore, the overall impact on efficiency is not significant.

% 模型的THREATS:我们只测试了7B版本的开源代码模型,deepseekcoder和code-qwen,对于更大参数量的代码模型是否会有其他性质,我们没有计算资源去测试。不过从开源通用模型的效果来看,如gpt4o和deepseek,我们的Intention框架相较于其他prompt技术可以很出色的完成code Refinement任务

% 数据的THREATS:我只选择了一个测试数据集,这是因为其他数据集没有提供原始数据难以完成数据补全,也就没法使用Intention框架的。不过这个测试集已经在很多论文中使用,是公认的比较完整客观的数据集。以后我们也会尝试收集更全面,质量更好的数据集。

% 效率的THREATS:因为我们框架调用了多次LLM进行分类和生成代码,整体效率可能会略低于其他prompt策略。不过用模型分类的响应速度比较快,而且对于Explicit code Suggestion,和Reversion Suggestion我们只需要通过一次LLM。所以整体效率影响不是很大。
\section{Related Work} \label{sec:related}

In this section, we discuss related work that has not been covered in previous sections.

\paragraph{Static resource analysis}
%
AARA is not the only approach for type-based resource analysis.
%
For example, there are approaches based on sized types~\cite{phd:Vasconcelos08,ICFP:AL17}, dependent types~\cite{LICS:LG11,POPL:LP13,POPL:RGG21}, refinement types~\cite{POPL:HVH20,POPL:CBG17,ESOP:CGA15,OOPSLA:WWC17},
recurrence extraction~\cite{POPL:KML20,ICFP:DLR15}, logical frameworks~\cite{POPL:NSG22,POPL:GNS24}, and annotated types~\cite{POPL:CW00,POPL:Danielsson08}.
%
None of the aforementioned type systems considered supporting heap-manipulating programs with Rust's borrow mechanisms.
%
Besides type systems, there are also static resource-analysis techniques based on
defunctionalization~\cite{ICFP:ALM15}, recurrence relations~\cite{JAR:AAG11,TACAS:AFR15,APLAS:FH14,PLDI:BCK20,POPL:KBC19,PLDI:KBB17},
term rewriting~\cite{RTA:AM13,TACAS:BEG14,IJCAR:FNH16,JAR:NEG13}, and
abstract interpretation~\cite{SAS:ZSG11,LPAR:BHH10,CAV:SZV14,kn:DHW07,misc:fbinfer20,SAS:AG12}.
%
Some of the aforementioned techniques work on imperative programs (e.g., C programs) with arrays,
such as C4B~\cite{PLDI:CHS15}, SPEED~\cite{POPL:GMC09}, COSTA~\cite{JAR:AAG11}, ICRA~\cite{PLDI:KBB17}, etc., but none of them considered exploiting Rust's borrow mechanisms in the design.
%
It would be interesting future research direction to investigate how different static resource-analysis techniques can benefit from Rust's safety guarantees.

% Automatic Amortized Resource Analysis(AARA) first presented by \cite{AARA-Linear}, enriched type system  with resource annotations for a first-order functional language, to derive linear upper bounds on the heap-memory usage of list via linear programming. Derivative works support non-linear worst-case upper bound, like polynomial \cite{AARA-Poly}, multivariate polynomial \cite{AARA-Poly-Multivar}, exponential \cite{AARA-Exp} and logarithmic \cite {AARA-Log}. AARA can also support features like higher-order functions \cite{AARA-HigherOrder}, algebraic data types \cite{AARA-ADT} and general recursive data types \cite{AARA-GeneralRecursive}. Original AARA method does not support imperative mutation, whereas our work features AARA mainly with it and Rust borrow mechanism. Compared with AARA derivative works, our formalization is currently limited to single variate linear upper bound, however in our view, our extension is orthogonal to other features. Therefore it can be easy, as future works, to support complex recursive data structures and various upper bounds in Rust resource analyzer.

\paragraph{Graded type system}
Graded types \cite{Granule} introduce a graded modality for associating types with elements from a resource algebra. 
%
A graded type system can account for program variables' exact usages, security levels, and potentials (conceptually). 
%
Graded types seem to provide a more general mechanism than AARA types to reason about more general resources. 
%
On the other hand, our work focuses on how Rust's borrow mechanisms can aid resource analysis and chooses AARA as the starting point because AARA admits efficient type inference via linear programming. 

\paragraph{Program verification for Rust}
%
RustBelt~\cite{RustBelt} pioneers a line of work to use semantic typing and separation logic to verify Rust programs with both safe and unsafe code.
%
% Stacked Borrow \cite{StackedBorrow} presents an operational semantics for memory accesses in Rust, and defines an aliasing discipline to cooperate raw pointers and borrows when verifying. 
%
RustHorn~\cite{RustHorn} uses prophecy variables to model the future values of mutable borrows and proposes an automatic verification algorithm based on constrained Horn clauses.
%
% Another approach is to leverage Rust type system, restricting on safe Rust with borrow mechanism. 
Aeneas~\cite{Aeneas}---which inspires our development of RABC and supports our prototype implementation---uses LLBC to translate Rust programs into equivalent pure functional programs via symbolic execution.
%
Such pure functional programs can be ported into theorem provers such as Coq and F* to enable verification of functional correctness.
%
Prusti~\cite{OOPSLA:AMP19} also leverages Rust's advanced type system to devise a modular and automated verification approach.
%
\citet{master:Engel21} proposed a method to verify user-provided asymptotic resource bounds in Prusti.
%
In contrast, our work focuses on the automatic inference of concrete resource bounds via a type system.
%
% Refinement types based method has also been applied to verification, like Flux \cite{Flux} proposing a liquid type system and RefinedRust \cite{RefinedRust} exploiting prophecy variables. Our work follows type system based approach and focuses on analysis instead of verification. We admits safe restriction of Rust from Stacked Borrow, resembling Aeneas working on borrow calculus, but unnecessarily translating to functional program, straightforward analyzing resource consumption. We also adopt prophecy variables to model mutable borrows.

% \todo{MAY DELETE THIS merging? might similar to static analysis, when coming across loop}
\section{Conclusion}

Subgroup analysis is an important, yet under-utilized tool in data science.
Our results suggest that combining algorithm-generated, rule-based insights with human intuition and experimentation in an interactive workflow can help practitioners develop a thorough understanding of complex datasets.
By implementing these interactions in a lightweight notebook-based tool, we hope to lower the barrier for data scientists to try subgroup discovery and to curate unexpected, interesting subpopulations in their data.
Divisi is available as an open-source package so that data scientists and HCI researchers can build on this work, helping to make exploratory subgroup analysis more feasible for a wider range of contexts.

\section*{Acknowledgment}
This work was partially supported by the National Natural Science Foundation of China (Key Program, Grant No. 62332005), the National Research Foundation, Singapore, and the Cyber Security Agency under its National Cybersecurity R\&D Programme (NCRP25-P04-TAICeN). Lei Bu is supported in part by the Leading-edge Technology Program of Jiangsu Natural Science Foundation (No. BK20202001), the National Natural Science Foundation of China (No. 62232008, 62172200). Any opinions, findings and conclusions or recommendations expressed in this material are those of the author(s) and do not reflect the views of National Research Foundation, Singapore and Cyber Security Agency of Singapore.


%%
%% This command processes the author and affiliation and title
%% information and builds the first part of the formatted document.

% 构建一个benchmark,参考https://arxiv.org/pdf/2404.00599.pdf
% 新内容包括:问题的行号信息,更全的codediff,完整的project信息
% 新数据集保证了:更全的program language,随时更新的数据集(防止data leak)
% 新metric:EM-trim,BLEU-trim,编辑距离(David Lo文章)
% LLM测试包括:GPT3.5, GPT4, DeepSeek Code, StarCoder2, CodeLLaMa, Gemma, Qwen1.5
% 2个benchmark,micro, tufuno的





% \bibliographystyle{ACM-Reference-Format}
% \bibliography{software}
\bibliographystyle{IEEEtran}
\bibliography{software}

\end{document}
\endinput
%%
%% End of file `sample-sigconf-authordraft.tex'.
