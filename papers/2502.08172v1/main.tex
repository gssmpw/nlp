
\documentclass[10pt,conference,nocompress]{IEEEtran}
% \documentclass[10pt,conference,review]{IEEEtran}
\pdfoutput=1

\IEEEoverridecommandlockouts
% 投稿时添加
% \settopmatter{printacmref=false, printccs=false, printfolios=true}
% \renewcommand\footnotetextcopyrightpermission[1]{}
\usepackage[switch]{lineno}
\usepackage{hyperref}
\hypersetup{hidelinks}
\usepackage{graphicx} 
\usepackage{multirow}
\usepackage{tabularx}
\usepackage{url}
\usepackage{cite}
\usepackage{pifont}
\usepackage{xcolor}
\usepackage{enumerate}
\usepackage{hyperref}
\usepackage{enumitem}
\usepackage{caption}
% \usepackage[table,xcdraw]{xcolor} % 导入xcolor宏包
%%
%% end of the preamble, start of the body of the document source.
\begin{document}
\author{
    \IEEEauthorblockN{
    Qi Guo\textsuperscript{1}\IEEEauthorrefmark{1},
    Xiaofei Xie\textsuperscript{2},
    Shangqing Liu\textsuperscript{3}\IEEEauthorrefmark{2},
    Ming Hu\textsuperscript{2},
    Xiaohong Li\textsuperscript{1}\IEEEauthorrefmark{2},
    and Lei Bu\textsuperscript{3}
    \thanks{\IEEEauthorrefmark{1} This work was done during Qi Guo's visit to Singapore Management University.}
    \thanks{\IEEEauthorrefmark{2} Shangqing Liu and Xiaohong Li are corresponding authors.}}
    \IEEEauthorblockA{
    \textsuperscript{1} Tianjin University, P.R. China\\
    \textsuperscript{2} Singapore Management University, Singapore\\
    \textsuperscript{3} State Key Laboratory for Novel Software Technology, Nanjing University, P.R. China\\
    }
    \IEEEauthorblockA{bxguoqi@tju.edu.cn,
    xfxie@smu.edu.sg,
    shangqingliu@nju.edu.cn,\\
    ecnu\_hm@163.com,
    xiaohongli@tju.edu.cn,
    bulei@nju.edu.cn
    }
}
\makeatletter
\patchcmd{\@maketitle}
  {\addvspace{0.5\baselineskip}\egroup}
  {\addvspace{-1\baselineskip}\egroup}
  {}
  {}
\makeatother

%% The "title" command has an optional parameter,
%% allowing the author to define a "short title" to be used in page headers.
\title{Intention is All You Need: Refining Your Code from Your Intention}
% \author{Anonymous Author(s)*}
% \title{An Intention-Based Framework for Code Refinement}
\newcommand{\fei}[1]{{\textcolor{red}{Fei:#1}}}
\newcommand{\sql}[1]{\textcolor{red}{sql: #1}}
\newcommand{\guo}[1]{\textcolor{cyan}{#1}}
\newcommand{\major}[1]{\textcolor{black}{#1}}
%%
%% The abstract is a short summary of the work to be presented in the
%% article.
\maketitle



% % ###############################################
% Start of file - draft.tex
% ###############################################

% ===============================================
% Preamble
% ===============================================
\documentclass[conference]{IEEEtran}
\IEEEoverridecommandlockouts

% ===============================================
% Bibliography set-up
% ===============================================
\usepackage[style=ieee, citestyle=numeric-comp, backend=biber]{biblatex}
\addbibresource{bibliography/references.bib}

% ###############################################
% Required extra packages
% ###############################################

% ===============================================
% Math mode
% ===============================================
\usepackage{amsmath, amssymb, amsfonts}

% ===============================================
% Hyperlinks
% ===============================================
%\usepackage{hyperref}

% ===============================================
% Figures and sub-figures
% ===============================================
\usepackage{graphicx}
\usepackage{subcaption}

% ===============================================
% Inline list and itemisation adjustment
% ===============================================
\usepackage[inline]{enumitem}
\setlist*[itemize]{labelindent=10pt, itemindent=0pt, leftmargin=*}

% ===============================================
% Table style
% ===============================================
\usepackage{booktabs}
\usepackage{multirow}
\usepackage{tabularx}

% ===============================================
% Box plot
% ===============================================
\usepackage{pgfplots}
\pgfplotsset{compat=1.18}
\usepgfplotslibrary{statistics}

% ===============================================
% Cross-referencing
% ===============================================
\usepackage[nameinlink]{cleveref}

% ===============================================
% Quotation marks
% ===============================================
\usepackage{csquotes}
\usepackage{textcomp}

% ===============================================
% Formattling the last page
% ===============================================
\usepackage{balance}

% ###############################################
% Document start
% ###############################################
\begin{document}

% ===============================================
% Title & acknowledgements
% ===============================================
\title{InfoPos: A ML-Assisted Solution Design Support Framework for Industrial Cyber-Physical Systems \\
\thanks{This publication is part of the project ZORRO with project number KICH1.ST02.21.003 of the research programme Key Enabling Technologies (KIC), which is (partly) financed by the Dutch Research Council (NWO).}
%\thanks{Redacted acknowledgements ...}
}

% ============================================================
% Authors compact
% ============================================================
\author{
\IEEEauthorblockN{Uraz {Odyurt}\IEEEauthorrefmark{1}, Richard {Loendersloot}\IEEEauthorrefmark{1}, Tiedo {Tinga}\IEEEauthorrefmark{1}}
\IEEEauthorblockA{\IEEEauthorrefmark{1}Faculty of Engineering Technology, University of Twente, Enschede, The Netherlands \\
Email: \{u.odyurt, r.loendersloot, t.tinga\}@utwente.nl}
}

% ============================================================
% Authors blind
% ============================================================
%\author{Redacted author segment ...}

% ===============================================
% Make title command
% ===============================================
\maketitle

% ###############################################
% Abstract and keywords
% ###############################################
\begin{abstract}
The variety of building blocks and algorithms incorporated in data-centric and ML-assisted solutions is high, contributing to two challenges: selection of most effective set and order of building blocks, as well as achieving such a selection with minimum cost. Considering that ML-assisted solution design is influenced by the extent of available data, as well as available knowledge of the target system, it is advantageous to be able to select matching building blocks. We introduce the first iteration of our InfoPos framework, allowing the placement of use-cases considering the available positions (levels), i.e., from poor to rich, of knowledge and data dimensions. With that input, designers and developers can reveal the most effective corresponding choice(s), streamlining the solution design process. The results from our demonstrator, an anomaly identification use-case for industrial Cyber-Physical Systems, reflects achieved effects upon the use of different building blocks throughout knowledge and data positions. The achieved ML model performance is considered as the indicator. Our data processing code and the composed data sets are publicly available.
\end{abstract}

\begin{IEEEkeywords}
Information position, Anomaly identification, Machine learning, Data-centric, Fine-tuning
\end{IEEEkeywords}

% ###############################################
% Text body
% ###############################################
% ==============================================================================
\section{Introduction}


\acrfull{chis} have become an important tool in modern healthcare and serve a variety of functions. 
%
They can provide consumers with a comprehensive understanding of a disease, specifically addressing general knowledge of health-related issues, effects, ways, and measures to maintain and possibly restore health. 
%
They can also enable early detection, diagnosis, treatment, palliation, rehabilitation, and follow-up care for diseases, along with associated medical decisions, care, and coping strategies for daily life with these diseases~\cite{RN1}. 
%
Typically, \chis\ are provided in a static and linear manner, meaning the same medical content is presented to everyone in the same structure. 
%
However, a linear reading or navigation may not be the best solution for everyone to extract relevant information because patients differ in terms of their prior knowledge, information needs, personal preferences and styles, and health situation, which may depend on factors such as gender, age, personality, and perception~\cite{RN3}. 
%
To address this problem, an \emph{adaptive and interactive visual \chis\ that supports document exploration with adaptive focus and detail views} is needed. 
%
By tailoring content to individual user needs and preferences, such as adjusting the level of detail or focusing on specific topics, this type of system would allow users to personalize their experience.


The primary research goal of this work is to develop novel concepts for advanced, interactive, adaptive, and visual \chis\ (called \apluschis). 
%
We selected \acrfull{ttwodm} as a pilot disease for our research because it is a complex disease with high prevalence and relevance to the public health system. 
%
Also, its progression over time requires changes and adaptations, including tailored knowledge, flexible treatment, new drug classes, improved patient education, sustainable follow-up practices, and screening for complications. Managing \acrshort{ttwodm} is still a difficult and time-consuming task, as it is common, serious, and under-treated. 
%
This is a major challenge for healthcare services, and patients and therapists must be prepared to deal with it effectively. 
%
Consequently, those affected by \acrshort{ttwodm} have an ongoing requirement for timely and pertinent information.~\cite{RN6}.



\begin{figure*}[ht!]
    \centering
    \includegraphics[width=\linewidth]{figures/apchis_library.png}
    \caption{
    The proposed \apluschis\ supports dynamic levels of detail. 
    A \DocumentLibrary\ (top) allows users to select one particular document they want to explore. 
    At document level, an interactive \TableOfContents\ (bottom left) preserves the global linear structure of the document while a chapter's substructure and content is visualized with dedicated visualization techniques for textual and pictorial data.
    }
    \label{fig:concept-dl-and-toc}
\end{figure*}

In this paper, we present a novel visual document exploration system with multi-dimensional adaptivity to help health information consumers better understand medical content by combining close and distant reading approaches.
%
It is targeted towards non-medical users of all adult age groups, which have either \acrshort{ttwodm}, are a relative to a \acrshort{ttwodm} patient or are interested in the disease for another reason. 
%
After a development and evaluation phase the \apluschis\ should  eventually be freely discoverable on the Internet. 
%
We work thus with an unpremeditated userbase in mind, which should be able to use the \acrshort{chis} intuitively, without the need for a supervised introduction. 

To this end, we propose an innovative document exploration system that provides multi-level navigation from high-level (topic overview) to mid-level (keyword occurrence and highlighting) to low-level (full text) views (\figref{fig:concept-dl-and-toc}). 
%
The basic idea is to allow users to efficiently navigate through documents, overview the content, find topics of interest, and finally switch to a close read on specific information contents. 
%
In our system, we make use of well-known document visualization approaches to enhance the learning process of medical content. To visualize high-level structures of a document, we propose dynamic table-of-content which represents sub-chapters by means of a \WordCloud\ containing keywords from a topic-modeling approach. High-level structures and mid-level document information are linked by using Tile Bars as visual navigator. 
%
The visual navigator shows topic occurrences within the underlying document and allows users to quickly explore the content by text snippets.
%
Although these text visualization techniques are not novel themselves, we tie them together in an integrated implementation prototype and adapt them for the specific requirements of the health domain. The system also serves as a platform to test and evaluate approaches for adaptive document and interaction provenance visualization (cf.\ also Section \ref{sec:interaction-analysis}).
%
Our system introduces the notions of  level of detail and adaptive visual presentations for document-based health information exploration for \acrshort{ttwodm}. 
%
Existing \acrshort{chis} are largely static in nature and do not adapt either of these dimensions to their users.


As a consequence, we conducted a user study to characterize the usage behavior of health information seekers adopting our approach. 
%
We show the usability of our system by comparing linear reading with our multi-level approach, and illustrate the results by two provenance visualizations.


This paper presents a comprehensive extension of one of our previous papers \cite{lin2023ivapp} delving further into our developed system and offering additional insights and analyses. 
%
Besides a more verbose and detailed outline of our work, this paper comes with the following additions with regards to its predecessor:
\begin{enumerate}
    \item An extension to our system design with an updated visual representation as well as newly-added components such as a \DocumentLibrary\ for the exploration at documents level and an alternative to the Word Cloud representation (\acrlong{topiccloud});
    \item A through analysis of the interaction data obtained through supervised evaluations with actual users;
    \item Different customized visual analytics tools for analysing and evaluating said interaction data.
\end{enumerate}
%
In the next section (\secref{sec:related-work}) we provide an overview on relevant previous works from the consumer health information domain and reveal the research gap regarding adaptive systems for said domain, before our proposed system design is outlined in \secref{sec:proposed-design}. 
%
\secref{sec:evaluation} describes a formative evaluation that incorporates quantitative and qualitative methods, performed by a number of participants on an implementation prototype. 
%
Following this formative evaluation it was observed that the rich interaction data collected in its course merits the effort of a more detailed analysis and the development of additional visual analytics tool (\secref{sec:interaction-analysis}).
%
On that note, we also investigated the suitability of user interactions for proposing adaptive visualizations. 
%
\secref{sec:discussion} and \ref{sec:conclusion} conclude the paper with a thorough discussion regarding next steps and open research challenges. 

% ==============================================================================
\section{Related Work} \label{sec:related-work}

In this section, we present an overview of important visualization techniques that have inspired our approach and review the need for adaptive and interactive consumer health information systems.

% ------------------------------------------------------------------------------
\subsection{Visualization Techniques for Text Documents and Health Care}
Visualizing large text corpora is a challenging task. 
%
Usually, the involved data sets are inherently complex, containing structural and content-related information. 
%
Most linguistic and text visualization approaches rely on text-mining techniques to reveal semantic information from raw text data. 
%
Therefore, simple statistical processing (\eg\ word frequency and bag-of-words concept) as well as natural language processing approaches (\eg\ named-entity recognition, relationship extraction and sentiment analysis) may be used~\cite{10.1002/widm.1071, 7156366}.


A widely-used visualization technique for text data is the Word Cloud representation (also known as Tag Cloud) which presents an overview of the most frequent or important words by using different type or font sizes~\cite{6758829}.
%
This technique is also known as distant-reading technique~\cite{moretti2005graphs} and allows users to approach literature in a new way.
%
Instead of reading texts in the traditional way, i.e., linear or close-reading, the focus of distant-reading approaches is to count, graph, and map textual data by a visual representation~\cite{janicke2015close}.
%
In recent years, much research has been conducted on distant-reading and Word Cloud visualizations. 
%
For instance, Kim et al.~\cite{5718617} proposed WordBridge, which utilizes graph-based visualization techniques to connect multiple Word Clouds with information-rich edges. 
%
Further extensions of Word Cloud exist that focus on semantic contour lines~\cite{2011.01923.x} and images~\cite{doi:10.1177}. 
%
In our work, we rely on traditional Word Clouds to foster distant-reading within single documents.


For the exploration of larger document collections, additional document features such as metadata information and co-authorships, could be considered to gain a better understanding of the contents of those documents~\cite{6392833, 7583708}. 
%
Another interesting approach by Strobelt et al.~\cite{5290723} called Document Cards, utilizes a mixture of images and important keywords to visualize key semantics of a document. 
%
To visualize distributional properties within a document, Tile Bars~\cite{hearst1995tilebars,keim2007literature} could be considered, which is a compact pixel-based visualization technique that reveals the relative length of a document and the relative frequency of one or more query terms. 
%
In our work, we utilize Tile Bars to represent the relative frequency and distribution of terms from a Word Cloud.



Data visualizations are becoming increasingly important for various fields of application, as well as in healthcare. 
%
Visual representations may help patients as well as physicians to gain a better understanding of health records, \eg\ information on medical diagnostics, treatments, and health histories~\cite{HCI-039}. 
%
For example, the LifeLine system was among the first exploration systems that supports visual patient treatment histories~\cite{10.1145/286498.286513}. 
%
An extensive survey about visualization techniques for electronic health records and population health records are given in~\cite{DBLP:journals/cgf/WangL22}.


Recently, many of the mentioned document visualization techniques are also applied in a medical context. 
%
For instance, Facetatlas by Cao et al.~\cite{5613456} used linked Word Clouds to visualize entity-relational text document of diseases such as Type 1 and Type 2 Diabetes
Mellitus. 
%
The linked Word Clouds are used to represent global relations by using a density map and local relations by using edge bundling techniques. 
%
Another interesting multifaceted text visualization is SolarMap~\cite{6137214} which combines a labeled contour-based cluster visualization with a radially-oriented word cloud.
%
Furthermore, SolarMap can visualize topic distribution of entities from one facet together with keyword distributions that convey the semantic definition along a secondary facet.


With the advent of novel visualization techniques in different domains, visualization literacy, i.e., user understanding and discovery of visual patterns, is becoming increasingly important. 
%
Developing visual literacy is essential to support cognition and evolve toward a more informed society~\cite{doi:10.1177/14738716221081831}. 
%
In our work, we intend to increase visual and health literacy by gathering user information during exploration and providing adapted health information based on that.



% ------------------------------------------------------------------------------
\subsection{Need for Adaptive \chis} \label{subsec:chis}

As part of this work, we examined current sources of \chis\ related to \acrshort{ttwodm} across multiple media platforms, including websites, digital documents (PDFs), print media, apps, and videos. Our goal was to identify elements and modes of presentation within a representative sample of these sources that users can customize to their needs and preferences.
%
Our results suggest that the potential for adaptation in existing CHIS is only realized to a limited extent.
%
We did not find any adaptive elements in print media or digital documents (PDF) while websites, apps, and videos offer some customization options related to presentation format, such as adjusting font size and color. 
%
Some \chis\ also included features such as text-to-speech or language-switching~\cite{RN7, RN8}.
%
However, in terms of personalized medical information, only a few \chis\ had mechanisms to pre-filter medical content based on a user's diabetes profile~\cite{RN9}.
%
Most \chis\ included a standard table of contents, with or without hyperlinks to the respective chapters. Some sources also contained links within the text or cross-references to other sections or chapters. 
%
However, none of the \acrshort{ttwodm} \chis\ we analyzed used a visual document exploration system with multi-dimensional adaptivity for health information consumers.



These results show that existing \chis\ on \acrshort{ttwodm} fall short of the potential of presenting health information in an \emph{interactive}, \emph{adaptive} and/or \emph{personalized} way, while there is evidently a need for it. 
%
The knowledge domain of \acrshort{ttwodm} is complex and comprehensive, with a wide range of information sources (brochures, websites, medical doctors, etc.) and high diversity of topics (such as symptoms, treatments, nutrition, etc.). 
%
This might be overwhelming for laypersons without medical expertise seeking knowledge in the field. Such complex information situations usually put a high \emph{intrinsic cognitive load}~\cite{sweller2005implications} on the working memory during information processing and often lead to information seekers applying heuristics and cognitive biases at every stage of information processing. 
%
Such cognitive biases, misconceptions, and even myths about \acrshort{ttwodm} may lead to unhealthy behavior with severe health-related consequences. 


An \emph{interactive} \chis\ has the potential to (i) track behavioral patterns and explicit feedback of consumers, (ii) interpret these indicators in terms of certain cognitive biases (\eg\ the confirmation bias), and (iii) intervene if necessary (\eg\ by suggesting other pieces or sources of information). 

An \emph{adaptive} \chis\ can match the information units to the users and their current information needs. 
%
It can thus balance the \emph{intrinsic cognitive load} to a medium level and ensure that the consumer is neither too bored nor too overwhelmed. 
%
This is in line with the transfer of Vygotsky’s concept of the \emph{zone of proximal development} to digital learning environments~\cite{luckin2010re}, the outer fringe as suggested by the competence-based knowledge space theory~\cite{heller2006competence} and constitutes a solid basis for an enjoyable flow for consumers~\cite{schiefele2011skills}. 
%
All these theories emphasize that a medium difficulty of information units lead to a successful processing outcome. The intermediate goal is to successively reach more advanced levels of learning outcomes as suggested by Krathwohl~\cite{krathwohl2002revision} which means not simply remembering information, but also applying and evaluating it.


A \emph{personalized} \chis\ can foster a consumers’ personal commitment to engage with the system and information, and thus help to close the `intention-behaviour gap’ or `attitude-action gap' \cite{schwarzer2008modeling}. 
%
This is the ultimate goal of any \chis. 
%
We strive to achieve this through different added values of our advanced \chis\ compared to more `traditional’ digital \chis\ (\eg, a brochure in PDF format or plain webpage): the guarantee of high quality and evidence-based medical information, the reduction of complexity to a medium level, and the recommendation of information units that fit a consumers’ information needs. 
%
In addition, tools and functionalities to get an overview of the knowledge domain, to efficiently answer certain questions, and to easily navigate through different sub-topics will be offered for good user experience. With our formative evaluation activities (\secref{sec:evaluation}) we monitor progress towards these goals.






% ==============================================================================
\section{Adaptive Document Exploration Design} \label{sec:proposed-design}
%
In the following \secref{sec:design-requirements} we discuss the design considerations, taking into account the intended userbase and use case scenarios. 
%
Informed by this, the resulting system design is outlined in detail in \secref{sec:implementation}.


% ------------------------------------------------------------------------------
\subsection{Design Requirements/Considerations}\label{sec:design-requirements}
In view of our intended users, who consists of all adult age groups, genders, and digital proficiency we decided to rely on document visualization techniques that are intuitive to understand and easy to use. 
%
Although the participants of the formative evaluation study which was conducted within the scope of this project (\secref{sec:evaluation}) received a brief introduction into visual components of the system, we want to support the scenario that an uninformed user, which discovers the \acrshort{chis} on the Internet is able to use it without any prior supervised introduction. 
%
That is, visualizations should base upon tried and tested concepts, which have been shown to be usable by the vast majority of people. 
%
Yet, at the same time, more advanced users should have to option to alternatively work with more complex visualizations, which allow for a more efficient exploration at the cost of a steeper learning curve. 


\begin{figure*}[ht!]
    \centering
    \includegraphics[width=\textwidth]{figures/system_new.png}
    \caption{
    The main components of \apluschis\ shown by an example of exploring a German diabetes health brochure \cite{aok}:
    \acrfull{toc}, \acrfull{wc}/\acrfull{hwc}, \acrfull{is}, \acrfull{tileb}, \acrfull{topicb}, \acrfull{snps}, and \acrfull{fulltext}.
    Different actions (illustrated as blue arrows) allow a user to navigate from one view to another.}    
    \label{fig:exploration-mockup}
\end{figure*}

% ------------------------------------------------------------------------------
\subsection{Implementation}\label{sec:implementation}
Informed by the considerations above we designed a system comprised of several components, which support the exploration of documents at different levels of visual granularity. 
%
Starting from a \acrfull{dl} showing an overview of the documents supported by the \apluschis, a user is able to explore individual documents on both high and low levels (\figref{fig:exploration-mockup}).
%
For the prior, the system provides an expandable and interactive \acrfull{toc} while the latter is enabled though a series of text abstraction methods such as \acrfull{wc}, fingerprinting in the form of a \acrfull{tileb}, and topic modeling by means of a \acrfull{topicb}.
%
Pictorial content is provided in an \acrfull{is} component.
%
On the lowest level, a user can also review sections of the original text sources in the form of \acrfull{snps} or even the untampered full text. 
%
These different concepts are implemented in different subsystems such that an unintermitted exploration process is possible while alternating between levels.
%
In the background, we track user interactions to determine which parts of the content have already been visited and consumed by the user. 
%
This information is also displayed to the user in order to indicate which information has not yet been scrutinized.



\paragraph*{\DocumentLibrary}
%
The \dl\ is the first view a user is greeted with upon logging into the \apluschis.
%
All available documents are presented in a grid arrangement with their respective covers and titles (\figref{fig:concept-dl-and-toc}).
%
Upon hovering over a certain document a preview appears, showing both related metadata and a histogram of the document's most frequent terms. 
%
This initial view should serve the users in determining which of the documents is the most appropriate for addressing their respective information need.
%
Clicking on a document transfers the user to its \toc.


\paragraph*{\TableOfContents}
The exploration process on document level is supported by an interactive \toc.
%
We base this view on a document's inherent linear structure with chapters, sub-chapters, and so on. 
%
While we aim to preserve the outermost structure (`chapters' in most cases) we abstract all lower levels of structure and content with dedicated text- and multimedia visualizations (\figref{fig:concept-dl-and-toc}). 
%
To this end, we employ \wc s -- and most recently also an alternative representation (\secref{sec:future-work}) -- for the textual content, \is s for the pictorial content, and further subsystems (\tib\ and \snps) for structure and lower levels of visual granularity.
%
These dedicated visualizations are interweaved into the linear chapter structure, based loosely on the \emph{document card} design concept by Strobelt et al.~\cite{5290723} -- i.e., different visualization techniques are used to display the textual and visual contents.
%
On a per-chapter basis, these visualizations can be expanded or collapsed. 

Additionally, already `consumed' content is tracked and indicated with a `history' version of the respective \wc s and \is s. 
%
Specifically, terms and images are added to these components after they have been reviewed (i.e., clicked on) by the user. 
%
This \acrfull{hwc} visualizes the context of the exploration to the user. 
%
Alternatively, it can   be used to display non-clicked terms as to suggest content to the user.



\paragraph*{\WordCloud\ with \Topicbar}
To generate the word clouds, natural language processing is used to extract `significant' terms from each chapter. 
%
This pre-processing comprises steps such as tokenization (separation/segmentation into individual parts) and stop-word removal (filtering of irrelevant/insignificant words).
%
Subsequently, the set of remaining words are subjected to a lemmatization (transformation into their canonical form or dictionary form) and grammatically tagged (part-of-speech tagging). 
%
However, we do not use the resulting normalized words directly to fill the \wc, but instead subject them to the Latent Dirichlet Allocation~\cite{blei2003latent} in order to obtain meaningful topic models on them. 
%
The appropriate number of topic models per-chapter (5) was determined empirically by visually evaluating the resulting topics for different values in the low integer range.
%
That is, each topic model is defined by a weighted vector containing a subset of the chapter's extracted nouns. 

The concated vectors of all topics serve as the input for a chapter's \wc, where the weights are used to determine a word's size within the cloud.
%
For the arrangement of terms, we rely on the \emph{Wordle word cloud} algorithm \cite{steele2010beautiful} while the mapping between words and their belonging topic is established through a qualitative color mapping (\figref{fig:exploration-mockup}, \wc). 
%
Note that this means certain terms could appear redundantly as topics can exhibit overlapping term compositions.



Since the concurrent display of all topics can be overwhelming, we provide a means to toggle the visibility of individual topics. 
%
This subsystem -- the \tob\ directly above the \wc\ -- consists of 5 colorized toggle buttons mapping to the respective topic models.
%
Hovering over a term toggles the \Tilebar\ component for the respective term above it, while clicking it initiates the \snps\ view. 
%
To this end, we track the interactions -- how often the user has clicked a certain term -- in the \wc.
%
These click counts are the basis for the so-called \acrfull{hwc} on the right-hand side of a chapter visualization (\figref{fig:exploration-mockup}, \hwc) where the count determines the size of a term in the cloud.
%
The same hover-and-click interactions as with the `regular' word cloud are possible with the \hwc.



\paragraph*{\Tilebar}

Hovering over a term in the \wc\ triggers the display of the \tib\ component above it, which allows the user to efficiently grasp the term's occurrences over the whole document. 
%
This visualization is inspired by the \emph{literature fingerprinting} concept by Keim and Oelke~\cite{keim2007literature} which shows various document properties in a drilled-down manner. 
%
To this end, we compute the respective term's frequency over equal-sized text chunks and visualize the resulting 'intensities' in a colorized fashion, following the linear document structure (i.e., from top to bottom and left to right, see \figref{fig:exploration-mockup}, \tib). 
%
That is, the rows of the grid symbolize the document's chapters and the columns the ordered text chunks within a chapter.
%
Cells which stand for text chunks in which the term does not appear at all are filled with uniform gray color.
%
The \tib\ allows a user to quickly answer such questions as \emph{``does another chapter also cover this topic?''} or \emph{``how frequently is it mentioned overall?''}.



\paragraph*{\Snippets /\Fulltext}
%
The above mentioned abstractions are vital to gain an overview of the information covered by a document and determine which sections are the most appropriate to answer a specific information need. 
%
Ultimately however, it is necessary to provide a user with text chunks to enable them to answer their specific information need. 
%
To address this issue, we added two additional levels of visual granularity.
%
Firstly, a \emph{\Snippets} view which pops up to the right hand-side of \toc\ if a term in the \wc\ or \hwc\ is clicked.
%
Within this view, all sentences containing the clicked term are displayed with a highlighting of the term (\figref{fig:exploration-mockup}, \snps). 
%
Handles at the beginning and the end of a sentence allow to reveal the preceding and succeeding sentence. 
%
Those can be clicked iteratively to  display larger parts of the document before and after the found position. Alternatively, the section headers, which are also shown in the snippets view, can be clicked to display a section's whole content immediately (\figref{fig:exploration-mockup}, \fullt).
%
The top of the \snps\ view also contains a \Searchbar, which allows to readily change the term in question.


\paragraph*{\ImageSlider}
%
Besides the abstractions for textual content (\wc/\hwc), we would like to indicate the presence of a document's pictorial content. 
%
To this end, we employed an off-the-shelf \is\ component, next to the \wc\ to display a chapter's images (\figref{fig:exploration-mockup}, \is). 
%
The thumbnails of the images can be viewed directly in this slider, yet if further details are needed, an image can be clicked which shows it together with its caption in a full screen modal dialog (henceforth referred to as \isS\ and \isL\ respectively). 
%
As this component shows just five images at a time, we aim to determine an image's relevance in order to sort the list of images, resulting in the most relevant images being shown initially.
%
To determine said relevance, we make two assumptions. 
%
Firstly, we assume that images without a caption (\eg\ scenic backgrounds at the beginning of a chapter) are rather unimportant. 
%
Secondly, we split the set of images with captions into two tiers with the first tier being comprised of images showing tables, diagrams, and flow charts, or convey any sort of structured information, while all the others belong to the second tier.
%
The information whether or not an image has a caption is obtained in the extraction process. 
%
In a similar fashion to the provenance version of the \wc\ -- the \hwc\ -- we provide an additional image slider in the bottom of the \hwc, showing exclusively the chapter's images which have already been clicked (and thus `consumed') by the user.




% ==============================================================================
\section{Formative Evaluation}\label{sec:evaluation}

The formative study aimed to 
(a) explore how the design of the system and its components would be perceived by users, 
(b) evaluate the system in comparison to a linear and static \acrshort{chis}, 
(c) highlight potential areas for improvement, and 
(d) identify prospective research questions.
%
To maintain methodological consistency and participant comparability, we focused on one particular document for the evaluation setup. 
%
That is, we used a \acrshort{ttwodm} information brochure of the German health insurance provider AOK~\cite{aok} as data basis for evaluation. 
%
The text document in PDF format comprises more than $130$ pages of comprehensive and detailed health information, including figures, tables, and info-graphics. 
%
Full texts with health information were extracted with Adobe PDFBox\footnote{\url{https://pdfbox.apache.org/}} library and the images were extracted manually. 
%
Subsequently, the sub-chapters and images were sorted by the main chapters. 


% ------------------------------------------------------------------------------
\subsection{Participants}
%
Overall, 12 participants (four females) took part in the study, representing the different potential users of \apluschis\ with ages between 26 and 62 years ($M = 40 \unit{yrs.}$, $SD = 14 \unit{yrs.}$). 
%
Additionally, the participants had different levels of knowledge and competence. On a 5-point rating scale (from $0 := $ very low to $4 := $ very high) they self-assessed their prior knowledge of \acrshort{ttwodm} ($M = 1.00$, $SD = 1.21$), computer and software skills ($M = 2.25$, $SD = .97$), as well as previous experiences with visualizations ($M = 2.58$, $SD = 1.24$).

% ------------------------------------------------------------------------------
\subsection{Procedure} \label{sec:procedure}
 The overall procedure for the participants can be divided into the following phases: 
 (i) \emph{Instruction}, 
 (ii) \emph{Cognitive Walkthrough}, 
 (iii) \emph{Forced Choice} and finally, 
 (iv) \emph{Semi-Structured Interviews.} 
 %
 The phases (ii) to (iv) are described in more detail in according subsections below. 
 %
 The sessions that lasted between 60 and 90 minutes per participant were carried out individually, lead by one investigator. 
 %
 Three of the authors took the role of an investigator, each for four participants. 
 %
 In phase (i), i.e. \emph{Instruction}, participants received a short explanation of basic \chis\ functions, such as search functions, to ensure that they all began the evaluation process from a common usage knowledge base. 
 %
 The participants were then set in a real-world usage scenario: they were asked to imagine that they themselves or someone in their family was diagnosed with \acrshort{ttwodm} during a health check-up. 
 %
 That is why they want to find out more about this disease.
 
 

The above-mentioned AOK brochure~\cite{aok} was presented in \apluschis, where participants started the first task with the \toc\ view of the brochure, as well as in a PDF viewer (Adobe Acrobat Reader). 
%
The comparison with the PDF viewer was chosen not only because this is the original, static format of the brochure, but also because PDF viewers are widely used and therefore people are usually accustomed to them. 
%
The PDF viewer is therefore a challenging benchmark that allowed us to evaluate the expected added value of \apluschis\ compared to conventional \chis. 
%
The investigators recorded all audio and on-screen activities and additionally made manual notes of their observations. 



\paragraph*{\acrlong{cwt}}
To encourage interaction with \apluschis, showing how intuitive its functions are and how quickly the content can be grasped, participants were given pre-defined tasks to explore \apluschis. 
%
This evaluation method is known as a \acrfull{cwt}~\cite{hollingsed2007usability}.  
%
The pre-defined tasks also enabled comparable conditions across all participants for the subsequent evaluation steps. 
%
We defined them in such a way that they represent realistic search and evaluation tasks in the course of an information search. 
%
For the purpose of the study, the tasks also had to be linked to a measurable goal achievement, for example, the participants had to find and report a specific piece of information. 
%
It had to be possible to achieve this goal in both PDF viewer and \apluschis. 
%
For each tool, approximately the same number of tasks were designed, including tasks from the two categories of `generating an overview' vs. `finding specific information'. 
%
Initially, we defined 11 tasks: four were aimed at using the \WordCloud\ including \Topicbar\ (\taskWcOne--4),
 four at using the \Tilebar\ (\taskTibOne--4), and three at using the \ImageSlider\ (\taskIsOne--3). 
%
Since we intended to compare the PDF viewer with \apluschis, two parallel versions were created for the 11 tasks that were comparable in terms of difficulty and functions used. 
%
For example, the task \taskTibFour\ `In which chapter would you most likely start if you wanted to find out more about blood pressure?' had the parallel version `In which chapter would you most likely start if you wanted to find out more about insulin?'. 
%
This resulted in a total of 22 tasks (\tableref{tab:CWT}) that each participant processed in a within-subject design via the PDF viewer and \apluschis. 
%
We balanced task order in terms of system and components to avoid sequential effects. 
%
In addition to the \cwt, participants were asked to express their thoughts during the tasks (i.e., \emph{think-aloud}). 
%
The investigators also noted behavioral observations during the \cwt\ tasks to complement on-screen and think-aloud activities.

\bgroup
\newcommand{\theader}[1]{\multicolumn{1}{c}{\textbf{#1}}}
\begin{table*}[ht!]
    \centering
    \caption{The \acrshort{cwt} Tasks.}\label{tab:CWT}
    \begin{tabularx}{\textwidth}{l X X}
        \hline
        \theader{Task} & \theader{Parallel Version 1} & \theader{Parallel Version 2} \\
        \hline\hline
        \taskWcOne & Which contents/subject areas do you think are included in chapter 5? & Which contents/subject areas do you think are included in chapter 3?  \\
        \taskWcTwo & What is the   waist size that creates a greatly increased risk for men? & What daily amounts of beer/alcohol are just acceptable for men? \\
        \taskWcThree & How does type I diabetes mellitus develop? & How does type II diabetes mellitus develop?  
        \\
        \taskWcFour & What contents/terms have you searched for so far? & What contents/terms have you searched for so far?  
        \\
        \taskTibOne & How often does the term `smoking' appear in chapter 1? & How often does the term `stress' appear in chapter 1?  
        \\
        \taskTibTwo & Is it worth reading beyond chapter 2 if you want to learn exclusively about `diastole'? & Is it worth reading beyond chapter 4 if you want to learn exclusively about `medication'?  
        \\
        \taskTibThree & Does chapter 2 give a better insight into the topic of `care' than chapter 6? & Does chapter 6 give a better insight into the topic of `(diabetic) foot' or `foot syndrome' than chapter 3?
        \\
        \taskTibFour & In which chapter would you most likely start if you wanted to find out more about `blood pressure'? & In which chapter would you most likely start if you wanted to find out more about `insulin'? 
        \\
        \taskIsOne & According to an illustration in chapter 5, is a sugar value of 100 mg/dl alarming or safe? & Which 3 stages of diabetes therapy are shown graphically in chapter 4? 
        \\
        \taskIsTwo & Which picture in chapter 6 do you think conveys the most relevant information about type II diabetes mellitus? & Which picture in chapter 6 do you think conveys the least relevant information about type II diabetes mellitus?  
        \\
        \taskIsThree & Search for a graphic on the topic of 'sugar metabolism'. & Search for a graphic on the topic of 'nutrition pyramid'.  
        \\
        \hline
    \end{tabularx}
\end{table*}
\egroup


\paragraph*{Forced Choice} 
Following the \cwt, participants were asked to choose between the PDF viewer and \apluschis\ regarding performance goals of system use. 
%
This evaluation method is known as \emph{forced choice}. 
%
The following nine performance goals were evaluated: `Would you rather use Adobe Acrobat Reader or the \apluschis\ to 
(a) get an overview of the domain, 
(b) develop a general understanding on \acrshort{ttwodm}, 
(c) search for specific keywords, 
(d) capture the main content, 
(e) search for specific images, 
(f) get an overview of the most informative images, 
(g) efficiently navigate through different topics of the content, 
(h) get answers to questions you might have in mind, and finally, 
(i) trace past searches'. 
%
We decided to choose a \emph{forced choice} rather than a questionnaire format with Likert scales for two main reasons: first, as a formative and explorative study we aimed for an explicit comparison between \apluschis\ and a challenging benchmark with regards to the nine above mentioned performance goals. 
%
Deciding between two alternatives might be easier in particular for the inexperienced participants of our sample. 
%
Second, considering the sample size which seems reasonable for a first formative evaluation but rather small for a summative one, means and standard deviations which could be derived by Likert scales would not allow for inferential statistical methods.




\paragraph*{Semi-Structured Interviews} 
Lastly, we conducted \emph{semi-structured interviews} with participants to ask open and generic as well as closed and specific questions about \apluschis\ as well as further inquiries on the comparison between the two systems. 
%
The open questions included for example: `Which system would you rather recommend to a person as a first source of information to get started?' or about the non-linear content exploration in \apluschis\ such as ‘How much does this interactive system encourage you to explore further content?’. 
%
More specific and closed (yes-no) questions were if the participants had already seen or used the different components of \apluschis\ (such as the \WordCloud\ or \ImageSlider, etc.) prior to the session, if they considered them as helpful and if they would like to use them again.



% ------------------------------------------------------------------------------
\subsection{Results and Discussion}
In the following section, we outline and discuss the results, starting from a more global evaluation and continuing with more specific results with respect to the components of \apluschis. 


\paragraph*{Global Evaluation} 
As outlined in the previous section, in the course of the semi-structured interviews, participants were asked closed questions if they had already seen or used a 
(i) Word Cloud, 
(ii) Topic Bar, 
(iii) Tile Bar, or 
(iv) Image Slider prior to the session. 
%
In addition, they were asked if the components were considered as helpful and if they would like to use them again. 
%
A `yes-answer' has been coded as `1', a `no-answer' as `0', and indifference as `0.5'.
%
\figref{fig:seen_used_hepful_useagain} shows the results of this global evaluation: while \WordCloud\ and \ImageSlider\ were well known by participants, as around 75\% had seen it before, no participant had come across a \Tilebar\ and only one has come across a \Topicbar\ prior to the session. 
%
The \ImageSlider\ had been used most often before (ca. 63\%). 
%
Interestingly, the \WordCloud\ had rarely been used despite being well known. 
%
Several participants noted that they were familiar with a \WordCloud\ as an overview graphic, but not as an interactive element. 
%
Regarding helpfulness and future use, more than half of the participants found the \WordCloud, the \Tilebar, and the \ImageSlider\ helpful and would consider using them again. 
%
However, the \Topicbar\ in its current form was not perceived as very helpful, and only 25\% would consider using it again.

\begin{figure}
    \centering
    \includegraphics[width=\mediumwidth\linewidth]{figures/seen_used_helpful_useagain_v0.2.png}
    \caption{
    The evaluation regarding the prior knowledge and usefulness of \apluschis\ components by the \cwt\ participants.
    }
    \label{fig:seen_used_hepful_useagain}
\end{figure}



\paragraph*{Evaluation of \apluschis\ Components} 
The \cwt\ tasks provided information on how the participants used the different components. 
%
We observed mixed results regarding the intuitiveness of searching with the \WordCloud. 
%
Some participants were not hesitant to use the \WordCloud\ and performed efficient searches for keywords and contents, as one participant explained: \emph{‘I simply looked at the largest terms’ (\PTwo)}. 
%
Others preferred the use of more familiar features, such as the \TableOfContents\ or the \Searchbar. 
%
However, the \WordCloud\ seemed to be a tool that was easy to learn. We observed that participants often started to use the \WordCloud\ when other features were not perceived as helpful due to the nature of the task or because the traditional search was simply inefficient. 


Similarly, we observed that the \Tilebar\ component was mostly intuitive and easy to learn. Several participants were completely unfamiliar with the \Tilebar\, but most were able to quickly understand its function and extract information from it, as this quote demonstrates: \emph{‘There is a box [tile] in chapter 2 and then no more boxes in the chapters after that, so it [the searched term] is not mentioned anymore’ (\PSix)}. 
%
Overall, the use of \Tilebar\ allowed the participants to efficiently search for the desired information. 
%
With regard to the design, the participants identified a need for optimization; for example, more contrasting colors for the tiles and a clearer labeling of the chapters within the \Tilebar\ would promote more intuitive use.



Finally, the \ImageSlider\ was perceived as rather intuitive and efficient for finding images because a number of participants already knew it from other applications and the operation was thus familiar. 
%
However, using the search bar was often preferred over browsing the \ImageSlider, particularly because the latter was time-consuming in chapters with many images, as this participant explained: \emph{‘I don't want to click through all of them; they are unnecessary and very small. 
%
They don't contain any information; they are just photos without text’ (\PSeven)}. 
%
This quote also highlights the desired reduction of unnecessary images as well as the wish for more context for those images that were deemed not self-explanatory, as expressed by several participants.



\paragraph*{Performance Goals}
%
Besides describing the sample as a whole, as part of our explorative research, we compared the results of several subgroups: 
(i) female vs. male participants, and - based on a Median-split - 
(ii) `younger' vs. `older' participants, `higher' vs. `lower' levels of self-assessed 
(iii) prior knowledge of \acrshort{ttwodm}, 
(iv) computer and software skills, and 
(v) experiences with visualizations. 
%
Only the comparison between the two age-groups shows some differences at face level. 
%
All other pairs of `sub-groups' are rather similar in their decisions with regards to the \emph{forced choice} items and thus, we do not report the according results.
%
\figref{fig:Forced Choice} illustrates the findings of the forced choices participants had to make between the PDF viewer and \apluschis\ regarding defined performance goals. 
%
If participants chose \apluschis\ as their information source of choice for a performance goal, it received a value of `+1'. 
%
Conversely, `-1' was the value for their choice of the PDF viewer and `0' for an indiscriminate choice. 
%
Thus, the ordinate ranges from `-12' (all participants chose the PDF viewer) to `12' (all participants chose \apluschis). 
%
In \figref{fig:Forced Choice}, the values for the whole sample are reported with green circles, while grey circles represent only the older participants ($n = 6$, $\geq 38 \unit{yrs.}$) and orange circles only the younger ones ($n = 6$, $< 38 \unit{yrs.}$).



The results show the potential of \apluschis\ with regard to the fulfillment of the performance goals. 
%
The values for all nine performance goals are either close to the horizontal `middle-line' (indicating that participants are equally inclined towards \apluschis\ and PDF viewer and/or are indifferent) or clearly above, such as for \emph{efficiently navigating} through different topics of the content, getting an \emph{overview} of the most \emph{important images}, or \emph{tracing searches}. 
%
We could also observe differences between the two age groups; however, this should not be over-interpreted due to the small sample size but should be given consideration in future research. 
%
Overall, \apluschis\ is evaluated as a good tool for information searches compared to the PDF viewer, despite participants' familiarity with the latter.

\begin{figure}[ht!]
    \centering
    \includegraphics[width=\mediumwidth\linewidth]{figures/forced-choice-results.pdf}
    \caption{
    The results of the forced choice evaluation between \apluschis\ and the PDF viewer w.r.t. the defined performance goals.
    The performance goals in information processing are sorted along a continuum from more abstract (left) to more specific goals (right). 
    }
    \label{fig:Forced Choice}
\end{figure}



\paragraph*{Non-Linear Content Exploration} 
One area of particular interest was to find out how non-linear content exploration was received by participants. 
%
Think aloud during the \cwt\ tasks and the subsequent interviews showed that the system arouses curiosity and motivates further exploration of system features and contents. 
%
The participants perceived the search as enjoyable and found the aesthetic design with colors appealing. 
%
This applies in particular to the \WordCloud, which manages to cultivate curiosity about further topics, as one participant explained: \emph{‘[...] I can imagine that if there is such a \WordCloud\, I would at least take a look at what topics there are, whether I missed something that would interest me. And I would rather do that than browse through the brochure [...]’ (\PTen)}. 
%
It is an added value compared to a non-interactive system, that the \WordCloud\ fosters engagement with the content further via making interesting and frequent terms more visible. 



However, some participants felt that they could not explore the content effectively because they were unsure about how to use the system due to its novelty. 
%
In particular, the lack of a familiar linear structure was found confusing and made it difficult for some participants to keep track of the content. 
%
As one participant noted: \emph{‘[...] I’d like to know what’s in it. 
%
I don’t know what to expect, what the tool offers me.
%
It doesn't give an impression of what it actually is now. 
%
At least not a quick one. 
%
I don’t have that much patience for it [...] (\PSeven)’}. 
%
It seems that the open structure overwhelmed some participants, leading to a sense of inadequacy to complete the information search. 
%
More support through assistance in the \apluschis\ system could help users overcome these initial uncertainties so that they can capitalize on the strengths mentioned above and utilize the advantages to the fullest.



% ==============================================================================
\section{Interaction Analysis} \label{sec:interaction-analysis}
Our data collection efforts, including the on-screen and audio recordings as well as the application of the think-aloud method (\secref{sec:procedure}), yielded valuable insights and enabled us to evaluate the interaction data of the 12 participants during the 11 \acrshort{cwt} tasks in more detail. 
%
Thus, the analysis of these interaction data is a secondary analysis of the data obtained during the \acrshortpl{cwt} as part of the formative evaluation described in the previous section.


These interactions, such as clicks, scrolls, and key-presses, which are performed in order to derive a new insight, are referred to as \emph{(insight) provenance} \cite{gotz2009characterizing} and are a vital cue for the analysis for cognitive processes (\secref{sec:background}).
%
After describing the steps involved in capturing our provenance data such as tools and components used by the participants and the processes involved (\secref{sec:tools-and-processes}), the obtained records were analysed on a quantitative basis (\secref{sec:cwt-results}). 
%
Lastly, we will show how an interactive visual analysis of such provenance data can be enabled through custom-made visual analytics systems (\secref{sec:provenance-visualization}).


% ------------------------------------------------------------------------------
\subsection{Background} \label{sec:background}
%
The underlying working hypothesis of the analysis of interaction data was inspired by behavioral mapping, by process models in the field of information seeking and retrieval (e.g., Joseph et al.~\cite{joseph2013models}), as well by the work of Pohl et al.~\cite{pohl2016using} who applied a lag-sequential analysis (e.g., Bakeman and Gottman~\cite{bakeman1997observing}) to investigate interactions, sequences of interactions and users' activities and processes when engaging with a visualization system. 
%
Behavioral mapping is a well-established research method where the paths, movements, and activities within a physical space of participants (when carrying out certain tasks) are recorded and transferred onto a map for further analysis. 
%
As an example, Shepley~\cite{shepley2002predesign} investigated the ways of staff members and their time spent on walking from activity to activity at a neonatal intensive care unit. 
%
One of the goals of behavioral mapping is to re-arrange the physical space and workstations to avoid unnecessary paths and to make the workflow more efficient. 
%
The underlying principles, questions, and metrics have been transferred to the virtual space of the \apluschis\ platform and the \acrshort{cwt} tasks, having in mind exploratory research questions such as: \emph{What are the processes and interactions of the participants? 
%
Are there efficient and inefficient participants with regards to specific tasks, what are their characteristics and how could we support this sub-group of users? 
%
Could inefficient series of loops and cycles be avoided by a re-arrangement of the tools and components within the platform, by providing more guidance support, or by highlighting central tools and components which are used by a majority of participants across several tasks?} 
%
To answer such questions as these, quantitatively and by means of metrics from behavioral mapping~\cite{ng2016behavioral} and graph theory (e.g. \emph{centrality}), a formal description of the interaction data of each \cwt\ task per participant, the tasks across all participants, as well as the participants across all tasks are required. 


The interaction data of one or more participants or for one or more tasks can be graphically represented as a directed, labeled multi-graph, with the tools as vertices and the processes that lead from one vertex $n$ to a consecutive vertex $n+1$ as edge-labels (or vice versa). 


% ------------------------------------------------------------------------------
\subsection{Tools and Processes} \label{sec:tools-and-processes}
Before the data processing of the on-screen and audio recordings, a comprehensive list of (cognitive) processes have been defined.
%
Overall, 28 processes have been pre-defined by the three investigators of the formative evaluation study, whereas two additional processes had to be included during the coding process to ensure that all processes are covered by the raw data. 
%
Some examples of these processes include \emph{commenting} (when evaluating pictures), \emph{reading} (if text passages have been reproduced verbatim), \emph{interpreting} (concerning pictures and text passages, if participants summarized or evaluated the content in their own words), \emph{pause} (if participants did not do or say anything), but also more basic processes when interacting with the \apluschis\ platform, such as \emph{scrolling}, \emph{sliding} (through the pictures of the \is), or \emph{hovering} and \emph{click on} (e.g. a certain term in the \wc). 
%
The tools and exploration subsystems (\secref{sec:proposed-design}), have been further distinguished by considering also the chapter of the brochure~\cite{aok} into account; i.e., instead of distinguishing ‘only’ between \ImageSlider, \WordCloud, etc., it has been further differentiated between the preview (small) and the enlarged \ImageSlider\ as well as the \WordCloud, \Tilebar, \Snippets, etc. for each of the seven chapters of the brochure, resulting in 86 ‘tools’.
%
The coding of the on-screen and audio recordings has been to a large extent carried out individually by the above-mentioned investigators.
%
However, in cases the investigators were not absolutely sure on how to interpret a participants activity and to code it into one of the pre-defined processes, he or she noted the time-stamp and the coding of such ambiguous activities has been done collaboratively by reaching a consensus.
%
The processes have been coded only if they exceeded a duration of around $1 \unit{s}$. For the computation of several metrics and further analysis, the individual interaction data for each of the tasks has been represented as sequence of 
$\langle \mathit{tool}_{\mathit{src}}, \mathit{process}, \mathit{tool}_{\mathit{tar}} \rangle$
triples. 



% ------------------------------------------------------------------------------
\subsection{Results and Discussion} \label{sec:cwt-results}
%
Overall, i.e., across all 12 participants and 11 \cwt\ tasks, 1,870 processes (=edges) have been observed, whereas 21 out of the 30 processes were applied at least once by the participants. The three most frequently applied processes are \emph{scrolling} ($n = 268$), \emph{sliding} ($n = 240$) and \emph{click on} ($n = 235$). 
%
On average, a process took $4.18 \unit{s}$ (\emph{Mdn} = $2 \unit{s}$, \emph{SD} = $5.03 \unit{s}$), ranging from $1$ to $64 \unit{s}$. 
%
The tasks (\tableref{tab:CWT}) where the most processes have been applied by all participants are \emph{IS2} ($n = 337$), \emph{WC2} ($n = 267$) and \emph{IS3} ($n = 242$); whereas the tasks with the least amount of processes are \emph{WC4} ($n = 65$), \emph{TiB1} ($n = 77$) and \emph{WC1} ($n = 94$). 

With regards to the `tools', 75 out of the 86 differentiated tools have been applied at least once by the participants, whereas the three most frequently applied tools are the \emph{\isL}\ (the \ImageSlider\ in enlarged form) \emph{for chapter 6} ($n = 308$), the \emph{\snps\ for chapter 1} ($n = 124$) and the \emph{\Searchbar} ($n = 103$). 
%
To put these numbers in context, the $75$ tools which have been applied at least once, were visited $2,002$ times across all participants and \acrshort{cwt} tasks. 
%
The three most applied tools, i.e., those with the highest \emph{centrality}, cover around a quarter of all incoming and outgoing edges.


The triple sequence $T$ allows to easily evaluate further metrics, such as the number of
(i) \emph{loops}, i.e., $|\{\langle s, ., t \rangle \in T : s = t \}|$,
(ii) \emph{multiple edges}, i.e., $|\{\langle s, ., t \rangle \in T : (\exists \langle s', ., t' \rangle \in T\setminus \{\langle s, ., t \rangle\})[s = s' \wedge t = t']\}|$,
and (iii) \emph{identical triples}, i.e., 
$|\{\mathbf{t} \in T : (\exists \mathbf{t}' \in T\setminus \{\mathbf{t}\})[\mathbf{t} = \mathbf{t}']\}|$.
% 
Overall, $1,870$ triples have been identified. 
%
$1,030$ of them are \emph{loops}, comprising $59$ (of the $75$) tools. 
%
The three tools with the most \emph{loops} are the \emph{\isL\ for chapter 6} ($n = 288$), the \emph{\snps\ for chapter 1} ($n = 91$) and the \emph{\snps\ for chapter 3}. 
%
These three cases are identical to the three most prominent \emph{multiple edges}. $1,479$ triples represent \emph{multiple edges} with $197$ unique tool combinations.
%
Finally, $1,177$ triples represent \emph{identical triples}, with $171$ distinct instances. 
%
The three most prominent ones are subsets of the above mentioned \emph{loops} and \emph{multiple edges} concerning the tool \emph{\isL\ for chapter 6}, with the processes \emph{sliding} ($n = 170$), \emph{commenting} ($n = 53$) and \emph{viewing} ($n = 41$). 


The fact that certain chapters of the brochure occur several times within the top-three of the applied tools, the \emph{loops}, \emph{multiple edges} and \emph{identical triples}, is of course caused by the concrete \cwt\ tasks (e.g., the tasks \taskTibThree\ and \taskIsTwo\ which specifically ask for a comparison between the content in chapter 6 and other chapters; see \tableref{tab:CWT}). 


The transitions between tools (agnostic with regards to chapters) are shown in the matrix in \figref{fig:adjacency-matrix}. 
%
The \emph{loops} correspond to the main diagonal, while all cells with a value $> 1$ contain \emph{multiple edges}. 
%
Please note that several transitions from one tool to another are technically not possible; in particular, several transitions from or to the \isL. 
%
When distinguishing only between the more broadly defined $10$ tools as in \figref{fig:adjacency-matrix}, out of the $1,870$ triples, $1,259$ are \emph{loops}, whereas all tools are affected, $1,802$ triples represent \emph{multiple edges}, with $58$ distinct tool combinations, and $1,616$ represent \emph{identical triples}, with $221$ unique instances. 
%
The three most observed \emph{identical triples} are 
$\langle \mathit{\isL}, \mathit{sliding}, \mathit{\isL} \rangle$ ($n = 221$), 
$\langle \mathit{\wc}, \mathit{scrolling}, \mathit{\wc} \rangle$ ($n = 77$), and 
$\langle \mathit{\wc}, \mathit{scanning}, \mathit{\wc} \rangle$ ($n = 74$). 

\mediumwidth
\begin{figure}[ht!]
    \centering
    \includegraphics[width=\mediumwidth\linewidth]{figures/adjacency_matrix/adjacency_matrix_v0.4_viridis.pdf}
    \caption{
    The adjacency matrix of transitions between the `high-level' tools. 
    The entries in the main diagonal corresponds to our notion of \emph{loops}, while \emph{multiple edges} populate the remaining cells.
    The cells reflecting transitions which are technically not possible are intentionally left blank.
    }
    \label{fig:adjacency-matrix}
\end{figure}



The often observed \emph{loops}, \emph{multiple edges}, and \emph{identical triples} with regards to the 
\emph{\isL}, reveal some potential for improvement to make the information search for users more efficient. 
%
One potential improvement approach has been suggested by a participant in the course of the formative evaluation study (\emph{\POne}: An \emph{Image Tile Display} that allows for more efficient scanning through the images of a certain chapter at once).
%



% ------------------------------------------------------------------------------
\subsection{Provenance Visualization} \label{sec:provenance-visualization}


While a `global' evaluation of the provenance -- e.g., obtaining quantitative measures for \emph{loops} or usage of individual components -- can be conducted based on the raw interaction transcripts, a more in-depth analysis requires dedicated visualizations.
%
Those would allow the analysis of interactions on a per-user and/or per-task basis and could answer additional questions, such as
`Are the different groups of users observable?' (e.g., depth-first search vs. breath-first search exploration process), `Are there any outliers?' (users whose exploration process differs significantly from all others), etc.
%
The data's underlying graph structure invites the application of different established visualizations. 
%
We implement two customized visual analytics tools, a graph as well as a matrix layout, which allow us to investigate different orthogonal aspects of the interaction data.



\paragraph*{Provenance Graph}
In a \emph{Provenance Graph}, we visualize a user's alteration and switching between different processes; i.e., in a directed weighted graph, we show how often a user switched between different processes (\figref{fig:provenance-vis}, bottom). 
%
With this type of visualization, it is possible to, for instance, spot if a user goes back and forth between two different processes or if they go through the same sequence of processes over and over (cycles). 
%
We indicate the time spent on specific processes through the size of the respective node and, conversely, the number of transitions between processes though the thickness of the respective links.


\paragraph*{Provenance Matrix}
%
Besides the graph representation, we also visualize the provenance information in a matrix layout where the sorted rows (\Startscreen/\dl\ $>$ \toc\ $>$ \wc/\hwc/\is\ $>$ \tob/\tib\ $>$ \snps/\fullt) represent the high-level tools at  different levels of visual granularity and abstraction (overview to closeup $\sim$ top to bottom). 
%
The columns reflect the different processes, sorted by their type (from basal/technical processes such as `scrolling' to cognitive/psychological processes such as `interpreting').
%
The matrix cells exhibit a color-coding, indicating the overall time a user spent on a tool-process pair.
%
Additionally, we show the sequence of the exploration with arrows spanning consecutively visited cells. 
%
As the display of `all' arrows would overload the visualization, we display only the most recent transitions, with the recentness modelled by an alpha-drop-off. 
%
Hovering over a cell triggers a tooltip which lists all the interaction triples responsible for said cell; i.e., as opposed to the provenance graph, the matrix is able to reveal correlations between processes and components, i.e., it shows 
(i) at which levels of visual abstraction a user predominantly operates, 
(ii) which processes they carry out at which level, 
(iii) which processes they carry out using which component, and 
(iv) whether they exhibit a rather vertical ($\sim$ depth-first search) or horizontal ($\sim$ breath-first search) exploration behavior.


\paragraph*{Use Case Examples}
Finally, we investigate the effectiveness of the proposed provenance visualizations for the visual analysis of tasks conducted by different users.
%
To this end, we take a look at one exemplar task (\taskWcThree, Version 1: `How does type I diabetes mellitus develop?') which should both, encourage an `open' exploration process, and result in comparable interactions amongst users. 
%
We compare the respective provenance visualizations of two participants (\POne\ and \PTwelve) with vastly different visual analytics 
proficiencies. 
%
It took \POne\ $188 \unit{s}$ to successfully complete the task, while \PTwelve\ required only $63 \unit{s}$.
%
\figref{fig:provenance-vis} shows the respective provenance graph and matrix, illustrating the interactions captured while working on said task. 
%
Even at a first glance it is obvious that \POne\ underwent a much more laborious exploration as \PTwelve\ which is in line with our impression during the supervised evaluation.

On a closer look, we can see \PTwelve\ used only 5 distinct processes without much back-and-forth between any of them. 
%
An inspection of the respective arrows in the provenance matrix also reveals that this participant generally moved from basal processes at a high level of abstraction (top left corner of the matrix) to rather cognitive processes at detail level (bottom right corner of the matrix). 
%
This is an expected exploration pattern of a competent information seeker.


\POne, on the other hand, required not just more time overall, but had a lot more back-and-forth between different technical and cognitive processes, i.e., `Scrolling' $\rightleftarrows$ `Scanning' and `Reading' $\rightleftarrows$ `Interpreting'. 
%
This interaction pattern is also confirmed by the provenance matrix which further reveals that \POne\ has several movements from low- to high abstraction levels such as from \fullt\ to \wc, or from \wc\ to \toc. 
%
We take these bottom-to-top patterns as a cue for an individual who ran into an exploratory wrong track, hence they had to backtrack to a higher level in order to find the correct path to their desired information.


\begin{figure}[ht!]
    \centering
    \begin{minipage}{\mediumwidth\linewidth}
        \includegraphics[width=0.49\linewidth]{figures/interaction-visualization/P01_graph_v1.0.png} 
        \includegraphics[width=0.49\linewidth]{figures/interaction-visualization/P12_graph.png} \\
        \includegraphics[width=\linewidth]{figures/interaction-visualization/P01_12_matrix_v1.0.pdf}
    \end{minipage}
    \caption{
    Provenance visualizations for two very different users, \POne\ (left) and \PTwelve\ (right), working on task \taskWcThree. 
    Their respective provenance graphs (top) reveal the alternations between processes, while the provenance matrices (bottom) clearly show the alternations between levels of visual granularity.
    }
    \label{fig:provenance-vis}
\end{figure}

The next research focus will be on the automatic analysis and the clustering of users based on their interaction patterns, as this information can be leveraged to propose adequate visualization to them \cite{gotz_behavior-driven_2009}. 




% ==============================================================================
\section{Overall Discussion} \label{sec:discussion} 

Our exploration system allows users to navigate through documents and adapt the visual representation and level of detail by using well-known visual analysis techniques such as word clouds, topic models, tile bars, and keyword search. 
%
The system provides the users with a two-fold document exploration: a traditional linear and non-linear document navigation by switching between content and detail. 
%
Furthermore, it allows users to follow both the edited content of a given document (supervised structure) as well as an automatically computed topic models (unsupervised structure). 
%
To the best of our knowledge, there are few empirical studies on the cognitive and motivational aspects of using document visualizations (\eg\ tile bars and word clouds, with those of linear document readers). 
%
Our evaluation is a first confirmation that our approach could foster interest and heighten curiosity by using a distant-reading approach for exploring the content of interest more efficiently.
%
Further key findings are that the participants in general enjoyed the non-linear content exploration, even if some participants would have needed more support functions at least in beginning of use. 
%
There were mixed results regarding intuitiveness of the \WordCloud\, whereas the \Tilebar\ and the \ImageSlider\ have been evaluated as being intuitive; which is also reflected in their agreement to the question if they would use these components again.



Our study showed that users did not make great use of the topic model structure. 
%
This may be partly due to the unfamiliar representation of topic models. 
%
Recently, some studies have investigated the impact of word clouds for topic understanding \cite{10.1162/tacl_a_00042} and keyword summaries \cite{8017641}. 
%
Word clouds, as it transpired, are particularly useful for quickly identifying the most common and frequent terms, while disadvantages may arise in decoding numeric values from font sizes for larger sets of keywords. 
%
An alternative visualization could be a simple word list with a frequency encoding (\eg\ font size, bars) that represent each topic. 
%
\figref{fig:wc-tc} shows such an alternative representation (2) to improve topic understanding where the keywords for each topic are displayed among each other. 
%
By using such a layout, overlapping term composition may be included to increase topic understanding and cause confusion as with the previous layout (1). 
%
Advanced topic model visualization will be considered in future work.

\begin{figure}
    \centering
    \includegraphics[width=0.6\linewidth]{figures/wc-tc.png}
    \caption{Adaptive word cloud layouts: for improving topic understanding the arrangement can be switched from Wordle layout (1) to an alternative list representation (2).}
    \label{fig:wc-tc}
\end{figure}

One key element of our system is its ``user tracking'' during document exploration.
%
In particular, we track which keywords have been explored, with which visual component, and for how long. 
%
This is considered \emph{important information provenance data} which, in our system, is used for provenance visualizations (\eg\ history word cloud, interaction matrix and graph). 
%
The latter is an important functionality for content recommendation, and forms the basis for the mitigation of cognitive biases and potentially harmful and wrong pre-conceptions.
%
The provenance visualizations may reveal emerging difficulties of users during information seeking tasks by showing outstanding patterns of user behavior, \eg\ horizontal, vertical or loop-like patterns. Our evaluation is a first step in this direction. 
%
To investigate user behavior during the CWT tasks, we integrated the CWT tasks in our provenance visualizations.



% ------------------------------------------------------------------------------
\subsection{Future Work} \label{sec:future-work}
The advantages of aggregated document representations warrant further examination in future research; in particular, the specific benefits that can be derived from using aggregated document representations. 
%
Furthermore, our future work will involve exploring potential misunderstandings that may arise from the highly aggregated nature of certain content presentations.


As evident from \secref{sec:interaction-analysis}, a vital cue which we plan to leverage for adaptive visualizations are \emph{user interactions}.
%
Research has demonstrated that user interactions can effectively be utilized for recommending particular types of visualizations \cite{gotz_behavior-driven_2009} and even help mitigate cognitive biases \cite{gotz_adaptive_2016}.
%
An investigation of the user interactions obtained from the \acrshort{cwt} (\secref{sec:cwt-results}) with customized visual analysis tools (\secref{sec:provenance-visualization}) revealed that meaningful interaction patterns, reflecting different types of users, can be observed.
%
We assume that users can be clustered into cohesive groups using said patterns, which, in turn, reflect a user group's need for specific types of visualizations.
%
Therefore, an open challenge is the robust and continuous tracking and classification of these interactions.
%
While this was done manually for the 12 participants of the \acrshort{cwt}, a purely automatic solution is necessary for the long-term.
%
Our prototype already comprises a tracking of the components a user is interacting with. % using.
%
To this end, we leverage the mouse pointer position together with the established assumption that a user's cursor movements are correlated with their gaze~\cite{reichle_ez_2006, buscher_eye_2008}.
%
The other aspect of our interaction logs -- the process a user is occupied with (\secref{sec:tools-and-processes}) -- is significantly harder to track automatically. 
%
Even though purely technical processes such as `scrolling' or `clicking' are trivial to track, the capturing of cognitive processes such as `reading' or `interpreting' pose a non-trivial challenge.
%
Yet, even such can be determined using low-level mouse interactions together with well-defined heuristics~\cite{10.1145/2207676.2208591, kirsh_virtual_2022}.



Additionally, we have planned further user studies. 
%
These will focus on the planned enhancements of the system, such as the representation of behavioral patterns, which can support users in reflecting on their information-seeking behavior and, thus, potentially contribute to the detection and prevention of biases, especially confirmation bias. 
%
This will include investigating how different ways of presenting the interactions can promote unbiased information-seeking by also taking into account differences between users, such as different visualization preferences, different states of knowledge or, as already mentioned above, age groups. 
%
Moreover, the need for support, as found in our current study for some users due to the novelty of the system, will be addressed.


In future work, we also intend to research automatic recommendation and develop adaptation methods based on the current system. 
%
Such a further developed system version will also be subjected to an extensive case study in order to evaluate the system in detail.


% ------------------------------------------------------------------------------
\subsection{A Model for Adaptation of Content and Presentation}

Our initial motivation was to provide an interactive and adaptive \chis. 
%
These systems should adapt the content and presentation to the users' information needs and preferences. 
%
Our document exploration design has a number of variables which could be controlled and adjusted by the system to adapt itself to the users' information need and preferences, and recommend views and content. 
%
To that end, we define a model based on the following dimensions along which automatic adaptation can be performed. 
%
In future work, we will focus on the prediction of dimensions to use and how to set them for specific users.



\paragraph*{Content Navigation} 
Here, \emph{what to show} when the user is doing a mouse-over on a term of the Word Cloud is to be decided. 
%
The principal options include (a) go to the best matching full text position (full detail), (b) show the text snippets of multiple matches (an intermediate detail level), or (c) show the tile bar (lowest level of detail) from which the user may pick a finding location. 
%
The system could determine the answer based on the previous selections made by the user, presuming the user has a stable preference. 
%
On the other hand, the system could track which background knowledge is already available, and show the higher levels of details for topics which are not well-known.


\paragraph*{Document Structure} 
Our \apluschis\ design shows the section Word Clouds together with the (a) section headings or (b) topic models computed for each section. 
%
These represent both supervised and unsupervised content structures. 
%
When the user requests a section, the system might adapt to show either one of these. 
%
To this end, the system might predict whether the user prefers the traditional, edited document structure given by the headings, or the computed content structure from a topic model. 
%
The latter provides an opportunity to compute the topic models such as to represent the users' interest and background information.


\paragraph*{Configuration of the History Cloud} 
The system may present to the user either a History Cloud of (a) what has already been explored, or (b) what has not been explored yet. 
%
The system could predict if the user would want to deepen their knowledge on a particular topic or broaden their horizon into new topics. Depending on the user's intent, the system can choose from which terms to create the history Word Cloud. 
%
This might also be a good starting point to mitigate biases once they are detected in the users.
% 
It is important to recognize that these are concepts, and thus require specific implementations to monitor, track, and characterize the users' background knowledge, preferences, and interests. 
%
In order to explore potential implementations and solutions to realize these concepts, both direct and indirect methods will be examined in our future research work.



% ==============================================================================
\section{Conclusion} \label{sec:conclusion}
%
Currently, a significant number of existing CHIS fall short when it comes to presenting health information in interactive, adaptive, and/or personalized ways. 
%
In this regard, interactive document visualization techniques are more conducive to enhancing user engagement and promoting a deeper understanding of complex information, thus providing an amplitude of opportunities for improved support of information-seeking tasks. 
%
We presented a novel and innovative document exploration system that allows users to visually explore health documents at various levels of detail and abstraction. 
%
We incorporated well-known document visualization techniques, such as \WordCloud s and \Tilebar s, into our design and applied it in the domain of \acrshort{ttwodm}. 
%
We evaluated our implemented system by performing a formative study based on a \cwt\ and compared our approach with linear and close reading. 
%
The evaluation results are promising and constructive, and demonstrate that our approach is easy to use and helps in content exploration, further facilitating more interactive content navigation as well as motivating users to engage with our implemented system at different levels of visual granularity and detail. 
%
We also presented concepts for possible visual adaptation by collecting user interaction data and visualizing this data in two provenance visualizations to reveal specific information needs as well as reading preferences.





% ===============================================
% Balance columns for the last page
% ===============================================
\balance

% ###############################################
% Bibliography
% ###############################################
\printbibliography

% ###############################################
% Document end
% ###############################################
\end{document}

% ###############################################
% End of file
% ###############################################

Humor is a social binding agent. It is an act of creativity that can provoke emotional reactions on a broad range of topics. Humor has long been thought to be “too human” for AI to generate. However, humans are complex, and humor requires our complex set of skills: cognitive reasoning, social understanding, a broad base of knowledge, creative thinking, and audience understanding. We explore whether giving AI such skills enables it to write humor. We target one audience: Gen Z humor fans. We ask people to rate meme caption humor from three sources: highly upvoted human captions, 2) basic LLMs, and 3) LLMs captions with humor skills. We find that users like LLMs captions with humor skills more than basic LLMs and almost on par with top-rated humor written by people. We discuss how giving AI human-like skills can help it generate communication that resonates with people. 

\begin{IEEEkeywords}
code refinement, intention-based generation, large language model
\end{IEEEkeywords}
%!TEX root=main.tex

\section{Introduction}
% Decision-makers, analysts, data scientists, and policymakers frequently rely on data to draw conclusions and extract insights. This data-driven approach helps them identify actionable recommendations aimed at influencing an outcome of interest, such as increasing product satisfaction or income levels or decreasing the likelihood of experiencing serious health conditions \cite{galhotra2022hyper,lakkaraju2016interpretable,agrawal1994fast}. 
\revc{Prescriptions, or actionable recommendations, are commonly generated across various fields to influence key outcomes such as improving product satisfaction, enhancing economic policies, or increasing business efficiency. 
%Decision- or policy-makers, analysts, data scientists, and 
Policymakers in government, decision-makers in businesses, and data scientists in various fields, often rely on data-driven approaches to identify 
%actionable recommendations 
potential actions to influence an outcome of interest, such as increasing income levels or loan approval rates}.
% , or decreasing the likelihood of experiencing serious health conditions. 
%
While association or prediction-based methods are extensively used in practice to draw useful insights from data, they typically identify correlations among variables and may fail to reveal the underlying causal factors, i.e., which actions may result in an improved outcome, needed for informed decision-making. 
%For recommendations to be truly impactful, there must be a clear  explanation that justifies why a particular decision is appropriate for a specific subpopulation~\cite{sun2021treatment,plecko2022causal}. 

\emph{Causal analysis} or {\em causal inference}, therefore, is considered one of the most important requirements to generate prescriptions that are {\em actionable} and aligned with human reasoning~\cite{imbens2024causal}. Causal inference, and in particular {\em observational studies} for causal inference on collected data (when controlled trials are impossible due to cost or ethical reasons), have been extensively studied in the statistics and artificial intelligence (AI) literature for several decades \cite{rubin2005causal, pearl2009causal}. Motivated by this foundational work on causal inference, the notion of causality has also influenced the field of database research. The causal models from AI have been extended to relational databases \cite{salimi2020causal},  and causality has been incorporated into various data management tasks such as finding responsibilities of inputs toward query answers ~\cite{meliou2010causality, meliou2009so, meliou2014causality}, explanations for query answers \cite{roy2014formal, DBLP:journals/pacmmod/YoungmannCGR24}, data discovery~\cite{galhotra2023metam,youngmann2023causal}, data cleaning~\cite{pirhadi2024otclean,salimi2019interventional}, hypothetical reasoning \cite{galhotra2022causal}, and large system diagnostics~\cite{markakis2024sawmill,causalsim,sage, gudmundsdottir2017demonstration}. 


\revc{If-then rules are generally considered interpretable by humans~\cite{lakkaraju2016interpretable,guidotti2018local,van2021evaluating,pradhan2022interpretable,chen2018optimization}.
We give a concrete example of the difference between association and causation in generating prescriptions or recommended actions in the form of if-then rules below}:
\begin{example}	%
\label{example:ex1} {\bf Importance of causal prescriptions:}
Consider the Stack Overflow (SO) annual developer survey
\cite{stackoverflowreport}, where respondents from around the world answer
questions about their jobs and demographics. A sample of the dataset \reva{with a subset of the
attributes (there are 20 attributes)} is presented in \cref{tab:data}.
%
Alice, a researcher in the United Nations (UN) finance department, is interested in discovering ways to increase the salaries of high-tech employees worldwide. She is looking for a set of actionable recommendations 
%(that we call a prescription rules) 
to raise the overall average salary.
%
Using association-based approaches~\cite{chen2018optimization,lakkaraju2016interpretable}, she may discover that individuals residing in the US who identify as straight or heterosexual tend to earn higher salaries (see \cref{exp:quality} for full details). However, this observation merely indicates a correlation: people living in the US, for example, generally earn more than those outside the country. Their comparatively higher salaries are primarily attributable to the country's economy and are unrelated to their sexual orientation. Thus, this observation cannot be used as a prescription rule to increase salary. 
Our causal analysis, on the other hand, reveals that individuals aged 25-34 with dependents would benefit from working as front-end developers.
This results in a \$44,009 annual salary increase on average. \reva{Another observation is that students should pursue an
undergraduate major in CS. %Computer Science (CS). 
This can boost their salary by \$22,174 per year} (see details in \cref{sec:casestudy}).
\end{example}

%It has been incorporated into various tasks including . 
%Causal interventions are often more relatable and easier to understand, as they offer insight into the underlying reasons behind the recommendations and allow unraveling complex cause-effect relationships that govern our world~\cite{pearl2009causality}. Furthermore, causal interventions often have long-lasting effects~\cite{imbens2024causal}.

%, making it essential that the prescribed actions are not only actionable but also 

%causally consistent. 

%Decision makings, in particular, high-stak

\cut{
In this work, {we study the problem of generating causal insights (referred to as \emph{prescription rules}), which serve as actionable recommendations} to improve an outcome of interest.
Recent works have introduced causality to the field of database research~\cite{meliou2010causality,  meliou2014causality,salimi2020causal,10.14778/3554821.3554902}. It has been incorporated into various tasks including data discovery~\cite{galhotra2023metam,youngmann2023causal}, data cleaning~\cite{pirhadi2024otclean,salimi2019interventional}, and large system diagnostics~\cite{markakis2024sawmill,causalsim,sage, gudmundsdottir2017demonstration}. 
We propose using causal inference to generate prescription rules that are both actionable and justifiable.
}

While generating prescriptions based on causal inference may help in robust decision-making, causal prescriptions that solely consider the betterment of an outcome (like salary) are not enough in practice. 
It is well-known that decision-making in many high-stake applications (like hiring policy, or policy for approving loans by banks) may lead to disparate societal or economic impact on different sub-populations. 
As a shocking example from a recent work called 
%For example, 
CauSumX~\cite{DBLP:journals/pacmmod/YoungmannCGR24} that generates a set of causal explanations for an aggregated view, the explanations generated %by CauSumX %recommendations which 
suggest that male individuals do a Bachelor's degree to increase their salary while %suggesting that 
being an unmarried woman 
%the recommendation for women includes getting married 
has the most adverse effect on salary
(borrowed directly 
from Fig.~19 in~\cite{youngmann2024summarizedcausalexplanationsaggregate}). 
%We demonstrate the advantage of using causal reasoning to generate actionable recommendations and the limitations of not considering fairness requirements in the following example. 
We explored this further in the context of generating prescriptions and observed that prescriptions that are not fairness-aware can generate unfair outcomes to some subpopulations which we refer to as the {\em protected group}. Examples include women, Black, Latino, or Native Americans, individuals with a disability, countries with a weaker economy, or other protected groups specific to an application. %Here is a concrete example:


% Understanding the causal factors behind these recommendations is crucial to ensuring that decisions lead to fair and equitable outcomes, particularly in sensitive applications where biased decisions can perpetuate or even exacerbate societal inequalities.
% While prior work has extensively explored techniques for association rule mining~\cite{kumbhare2014overview}, recent efforts have focused on deriving causal explanations for individual data points or entire datasets~\cite{salimi2018bias,youngmann2022explaining,ma2023xinsight}. Although some of these methods produce causally consistent insights, the absence of fairness considerations in the process can lead to unfair outcomes, further reinforcing existing biases. For example, CauSumX~\cite{DBLP:journals/pacmmod/YoungmannCGR24} generates causal recommendation which suggest male individuals to do a Bachelor's degree to increase salary while the recommendation for women include getting married (borrowed directly from Figure~19 in the paper~\cite{youngmann2024summarizedcausalexplanationsaggregate}). 





%\emph{Causal inference} has been thoroughly studied in AI and Statistics~\cite{pearl2009causal,rubin2005causal}. Causal analysis is a vital tool in determining the effect of a \emph{treatment} on an \emph{outcome}, and has been used in decision-making in medicine \cite{robins2000marginal}, economics \cite{banerjee2011poor}, biology \cite{shipley2016cause}, and in high-stakes areas such as identifying the root causes of failures in critical infrastructure systems to prevent catastrophic outcomes. Recent works have introduced causality to the field of database research~\cite{meliou2010causality,  meliou2014causality,salimi2020causal,10.14778/3554821.3554902}. It has been incorporated into various tasks including data discovery~\cite{galhotra2023metam,youngmann2023causal}, query result explanation~\cite{salimi2018bias,youngmann2022explaining,DBLP:journals/pacmmod/YoungmannCGR24}, and large system diagnostics~\cite{markakis2024sawmill,causalsim,sage, gudmundsdottir2017demonstration}. We propose leveraging causal inference to generate interpretable and justifiable insights (referred to as \emph{prescription rules}), which serve as actionable recommendations to improve an outcome of interest. Causal reasoning is considered one of the most important requirements,  to generate insights that are actionable and aligned with human reasoning.




\begin{table*}[]
\footnotesize
    \centering
    	\caption{\textnormal{A subset of the Stack Overflow dataset.}}
         \label{tab:data}
    	% \vspace{-4mm}
  			\begin{tabular}[b]{|l|l|l|c|l|l|c|l|c|}
  			
				%\multicolumn{9}{c}{\textbf{Users}}\\ 
				\hline

				\textbf{ID}
    
    % \textbf{Country}& \textbf{Continent} 
    
    &\textbf{Gender} &\textbf{Ethnicity}&
				\textbf{Age} &\textbf{Role} &
				\textbf{Education} &\textbf{Country}&\textbf{Undergrad Major}&\textbf{Salary}
				\\ \hline

				1 &Male&White&26&Data Scientist & PhD& US&Computer Science&180k\\
    		2 &Non-binary&White&32&QA developer & Bachelor's degree& US&Mechanical Eng.&83k\\

 3 &Male&South Asian&29&C-suite executive  & Bachelor's degree & India&Computer Science&24k\\

  % 4 &Female&South Asian&25&Back-end developer  & Master's degree & India&Mathematics&7.5k\\

  4 &Female&East Asian&21&Back-end developer & Bachelor's degree & China&Computer Science&19k\\
  

        % $\ldots$ &  $\ldots$&  $\ldots$&  $\ldots$&  $\ldots$&  $\ldots$&  $\ldots$&  $\ldots$&  $\ldots$&  $\ldots$&  $\ldots$\\
    \hline
			\end{tabular}
            \vspace{-5mm}
\end{table*}




\begin{example}	%
\label{example:ex2}
{\bf Importance of fair prescriptions:}
Continuing Example~\ref{example:ex1}, while those causal prescription rules are highly beneficial for the overall population, they are considerably less effective for individuals residing in countries with a low GDP (indicating a weaker economy). For this group, the average expected increase in salary is only approximately \$13,000 per year (in contrast to \$44,009 for the entire group). % \sr{add which rule 44k or 25k} 
Consequently, implementing these rules would exacerbate the disparity between those living in countries with strong economies and those in countries with weaker economies.
\end{example}




% Our objective is to generate a small set of prescription rules aimed at increasing (or decreasing) an outcome of interest. This is framed as an optimization problem where the goal is to select the fewest prescription rules that maximize utility (i.e., the expected increase or decrease in the outcome). However, 

The example above shows that focusing solely on maximizing utility (\revc{i.e., increasing income}) can result in a scenario where only some of the population receive significant improvement, while others experience no benefit (\revc{only a small benefit for individuals from countries with weaker economies in our example}). Additionally, even if a large portion of the population receives recommendations, a protected subpopulation might not share the benefits and, worse, their situation could deteriorate, exacerbating inequalities.

Examples~\ref{example:ex1} and \ref{example:ex2} show that it is crucial to provide recommendations that are (1) {\em causal} for the outcome (beyond associations),  and (2) also {\em fair or equitable} in terms of the outcome for both the protected and non-protected groups. While recent work in database research
has focused on deriving {\em causal explanations} for individual data points, aggregated view, or entire datasets~\cite{salimi2018bias,youngmann2022explaining,ma2023xinsight, DBLP:journals/pacmmod/YoungmannCGR24}, and in particular \cite{DBLP:journals/pacmmod/YoungmannCGR24} has considered generating a set of causal explanations for an aggregated view that resemble a ruleset, 
%Although some of these methods produce causally consistent insights, 
the absence of fairness considerations in generating these causal explanations can lead to unfair outcomes for the protected group.
%further reinforcing existing biases.


%\red{We, therefore, enable users to incorporate various \emph{coverage and fairness constraints} along with the overall objective of improving an outcome of interest. }

\medskip
\noindent
\textbf{Our contributions.~} 
Motivated by the dual goals of generating causal and fair prescriptions for the betterment of an outcome, we introduce a {\em fairness-aware framework leveraging causal reasoning for generating a set of actionable prescription rules (ruleset)} called \sysName\ (\underline{Fair} \underline{CA}usal \underline{P}rescription).
%
Following research on fairness in data management~\cite{stoyanovich2020responsible,galhotra2022causal}, we assume the existence of a \emph{protected subpopulation}, defined by an attribute such as gender or race for people, or GDP of a country. Motivated by the causal explanation rules for an aggregated view \cite{DBLP:journals/pacmmod/YoungmannCGR24}, each prescription rule in our ruleset applies to a sub-population defined by a {\em grouping attribute}, and prescribes a {\em treatment or intervention} to improve the {\em outcome} for this sub-population. Fairness constraints ensure that the expected utility of the protected population is {\em comparable} to the utility of the unprotected individuals. We borrow the notions of \emph{group and individual fairness} from the fairness literature but tailor them for prescription rules. In addition to the fairness constraints, our coverage constraints ensure that a substantial fraction of the population and protected subpopulation receives at least one recommendation. 
%We demonstrate how such constraints ensure that the generated rules apply to a large portion of the population and ensure fairness through the following example.

\begin{example}
\label{ex:intro_example_3}
Continuing Examples~\ref{example:ex1} and \ref{example:ex2}, Alice uses our proposed system, called \sysName, to impose fairness and coverage constraints to discover useful and equitable recommendations for increasing salaries worldwide. In particular,
Alice chooses to implement a coverage constraint to ensure that the selected rules apply to a significant portion of people worldwide, including a sufficiently large number of individuals from countries with low GDP (the protected group). She also imposes a fairness constraint to ensure that the expected gains for both protected and non-protected groups are comparable.
\reva{She discovers, for example, that for individuals with 6-8 years of coding experience (a subpopulation comprising 21\% of the entire dataset and 25\% of the protected group), pursuing a bachelor’s degree in computer science will increase the expected salary by $\$14.9k$ for protected and by $\$17.8k$ for non-protected}. (See \cref{sec:casestudy} for more details.) This prescription rule applies to a large portion of the population and ensures fairness by providing a similar expected gain for both protected and non-protected groups, and the allowed difference of outcomes between these two populations may be adjusted by choosing appropriate thresholds in the fairness definitions. 
\end{example}


\noindent
Our main contributions are as follows. \\
%\begin{itemize}[leftmargin=*,topsep=0pt]
{\bf (1)} We {\bf develop a framework that generates a set of prescription rules to enhance an outcome of interest (Section~\ref{sec:problem})}. A prescription rule consists of a \emph{grouping pattern} and an \emph{intervention pattern}, representing the target subpopulation and the actionable recommendation for that group, respectively. The strength of the {\em conditional causal effect} (Section~\ref{sec:background-causal}) of this intervention on the subgroup is used to measure the expected utility of a rule. Our objective is to identify the smallest set of rules that maximizes overall expected utility. We refer to this problem as the {\em \probName} problem.
We adopt several notions of fairness (individual vs. group, statistical parity vs. bounded group loss) from the literature to define the {\bf fairness constraints} for our problem. In addition, {\bf coverage constraints} (for individual rules or for a group) ensure that the solution for the \probName\ problem is applied to a sufficient number of individuals and to minimize inequalities. We show NP-hardness for different variants of the problems and properties (matroid) useful in our algorithms. 
%We establish several definitions for group and individual fairness constraints tailored for prescription rules.
\smallskip
    \par
    \noindent
{\bf (2)} We {\bf develop a general three-step algorithm named \sysName to solve the optimization problem of selecting a fair prescription ruleset (Section~\ref{sec:algo})}. The first step involves mining frequent grouping patterns using the Apriori algorithm~\cite{agrawal1994fast}. In the second step, we employ a lattice-based algorithm to find high utility and fair intervention patterns for grouping patterns identified in the previous step. Finally, the third step applies a greedy approach to determine a solution. \sysName\ can be easily adapted to accommodate all variants of the \probName\ problem.

\smallskip
\par
\noindent
{\bf (3) We provide a detailed  case study  (Section~\ref{sec:casestudy}) and experimental analysis (Section~\ref{sec:experiments}) to evaluate our framework and algorithms.}
The case study shows the qualitative difference of different variants of our problem for different choices of the fairness and coverage constraints. The experiments include two datasets, three baselines, and 18 variations of our problem with different constraints. Our evaluations suggest that fairness may come at the cost of expected
utility for everyone. However, without fairness constraints, we often observe a significant disparity between the protected and non-protected. We also observe that
achieving individual fairness is harder than group fairness,
as most high-utility or high-coverage rules are unfair. Lastly, we show that \sysName\ can generate  prescription rules over large datasets in a reasonable time. 

%\end{itemize}


%\paragraph*{Paper outline} 
We discuss related work in \cref{sec:related}, review background on causal inference in \Cref{sec:background-causal}, %and our problem formulation can be found in \cref{sec:problem}. Our algorithmic framework is presented in \cref{sec:algo}. A case study demonstrating the impact of different constraint configurations on the solution is given in \cref{exp:problem_variants}, and our experimental evaluation is detailed in \cref{sec:experiments}. Finally, we 
and discuss the limitations of our framework and future work in \cref{sec:conc}.

% \noindent
% \boxed{\parbox{\columnwidth}{$\bullet$ 
% For people with a professional degree, move to the United Kingdom
%  (coverage = 435 (20), coverage-protected = 20 (13), utility = 186855, utility-protected = 0.)\\
% $\bullet$ For graphic developers, move to the	United States
%  (coverage = 116 (29), coverage-protected = 8 (2), utility = 169431, utility-protected = 0).\\
% $\bullet$ For people who have no formal education, move to the United States
%  (coverage = 123 (34), coverage-protected = 7 (2), utility = 206742, utility-protected = 0).\\
% % \textcolor{red}{size = 38, length = 76, overlap = 64029181, utility = 1659307}\\
% \textcolor{blue}{overall coverage =674, expected utility = 187485
% coverage-protected = 35, expected utility-protected = 0}
% \sr{should mention protected group, and possibly not mention coverage in the intro or just intuitively like high coverage}
% }}


% Alice notes that although these rules result in a \$187,485 increase in the overall salary for those to whom they apply, they only affect a small fraction of the population, specifically 674 individuals. Additionally, although the expected salary increase is substantial, there is no expected increase in salary for non-males, a subpopulation of particular interest to Alice. In other words, applying these rules would result in no gain for non-males.
% \end{example}

% \begin{example}[Episode 2 - coverage and fairness constraints]
% Alice introduces coverage and fairness constraints to ensure that enough people will benefit from the rules and that they will be \emph{fair} with respect to non-males. Specifically, she demands that the benefit for a randomly chosen individual to whom one of the rules applies is nearly the same as the benefit for a randomly chosen individual who identifies as non-male and to whom one of the rules applies.

% After adding these constraints, \sysName\ recommends the following set of prescription rules:



% \noindent
% \boxed{\parbox{\columnwidth}{$\bullet$ 
% For people who have no formal education, move to the United States
%  (coverage = 123 (34), coverage-protected = 7 (2), utility = 206742, utility-protected = 0)\\
% $\bullet$ 
% For females, change role to	DevOps specialist (coverage = 2256 (47), coverage-protected = 2256 (47), utility = 90023, utility-protected = 90023).\\
% $\bullet$ For people with a Master's degree, move to the	United States
%  (coverage = 9097 (2222), coverage-protected = 642 (236), utility = 85390, utility-protected = 84201).\\
% % \textcolor{red}{size = 38, length = 76, overlap = 64029181, utility = 1659307}\\
% \textcolor{blue}{overall coverage =11476	
% , expected utility = 87601,
% coverage-protected = 2905, expected utility-protected = 88519}
% }} 







% \begin{figure}[t]
%         \centering
%         \begin{minipage}[b]{1.0\linewidth}
%             \small
%             \begin{tcolorbox}[colback=white]
%             \vspace{-2mm}
% $\bullet$ For backend developers, the treatment with the highest effect on salary is “Country = US” effect size = 78646
% \begin{itemize}
%     \item For non-male the effect is only: 59429
%     \item For male the effect is 80454
% \end{itemize}

% $\bullet$ For frontend developers, the treatment with the highest effect is :Formal Education = Bachelor's degree” effect size: 17340
% \begin{itemize}
%     \item For white the effect is 33464
%     \item For non-white the effect is 15320
% \end{itemize}


% $\bullet$ For people in Europe, the treatment with the highest effect on salary is “DevType = C-suite executive” effect size = 53254
% \begin{itemize}
%     \item For white the effect is 55112
%     \item For non-white 35249
% \end{itemize}



%             \vspace{-2mm}
%             \end{tcolorbox}
%         \end{minipage}%%
%          % \vspace{-4mm}
%         \caption{Set of prescription rules.}
%         \label{fig:so-explanation}
%     \end{figure}

\section{Background and Motivation}
\label{sec:background}

We introduce the background on serverless workload serving and motivate the use of runtime resource adaptation to address resource inefficiency in existing serverless platforms.

\subsection{Resource Inefficiency with Early Binding}
% In current serverless platforms, developers are required to specify immutable sizes for their deployed functions.
% Then, providers consider functions' runtime workloads  (e.g., concurrency)  and resource usage to scale out/in their instances.
% Moreover, due to high runtime variability, functions must size their functions for worst-case scenarios.
% This, however, incurs considerable resource inefficiency.
Current serverless workflow platforms (e.g., AWS Step Functions~\cite{aws-step-function} and Azure Durable Functions~\cite{azure-durable-function}) offer the opportunity for developers to build various applications with advanced logic like chaining, branching, and parallel execution.
These applications can be defined by JSON-based structured languages (e.g., Amazon States Language) or other programming languages.
Meanwhile, developers require to specify resource configurations, including memory size, CPU cores, and scaling options, for individual functions---an early-binding approach.
The serverless platform is responsible for monitoring the workload intensity and resource usage at runtime and scaling out/in function instances accordingly.
To account for potential runtime variability, developers must size the functions in their application workflow accounting for the worst case in order to provide SLO guarantees over the end-to-end delay of request processing, e.g., the 99th percentile (P99) of the end-to-end delay must be within a given target. 
After deployment, the function sizes become immutable. The worst case is not representative and over-shoots most of the time, leading to resource inefficiency. 


To verify this claim, we conduct a data-driven analysis with a dataset from Microsoft Azure Functions~\cite{azure-dataset} to explicitly demonstrate the resource inefficiency issue. % , deriving from the worst-case based early bind.
To quantify the inefficiency, we define a metric called \emph{slack}---the margin between the actual execution time and the SLO, which is calculated as $1-l/T$ with $l$ and $T$ representing end-to-end latency and SLO, respectively.
Under certain SLO defined with P99 latency as done by existing works (e.g., \cite{osdi22-orion,mac22-wisefuse}),  we can see from Figure \ref{fig:bg:slack} that more than 60\% function invocations have slacks over 60\%.
Particularly, we analyze slacks of the top 100 most popular functions, whose invocations account for 81.6\% of the total function invocations. % (depicted in Figure~\ref{fig:bg:popular_func}) of overall invocations.
The result shows that only 20\% of the invocations of the popular functions (blue line in Figure~\ref{fig:bg:slack}) have slacks less than 40\%.
This means the majority of requests are processed faster than necessary.
Notably, in DAG-based workloads (i.e., Azure Durable Functions), the resource inefficiency further deteriorates wherein the ratio between the 95th percentile and 50th percentile is by up to three times \cite{mac22-wisefuse}.

% \begin{figure}[t!]
% \centering
% \includegraphics[width=0.25\textwidth]{./figure/motivation/Average_P99_cdf_top=100.pdf}
% \vspace{-0.3cm}
% \caption{Sufficient function slacks in production traces.}
% \label{fig:bg:slack}
% \end{figure}

\subsection{Runtime Dynamics}
\label{sec:bg:worst-case}

The resource inefficiency caused by the large slack can be mainly attributed to the over-provisioning of resources by the developer. This is to ensure that the SLO is guaranteed even in the worst case (i.e., P99). However, normal cases deviate from the worst case significantly due to runtime dynamics. 
In particular, we observe that functions face two major dynamic factors at runtime: varying working sets and inevitable performance interference. These two factors contribute significantly to the variance of the function execution time. 
% Functions face two remarkably dynamic factors at runtime: working sets and performance interference, which lead to considerable variance of execution latency.

\begin{figure*}[!t]
	\centering
	\subfloat[]{
		\includegraphics[width=0.24\textwidth]{./figure/motivation/Average_P99_cdf_top=100.pdf}
		\label{fig:bg:slack}
	}
	\hspace{8mm}
	\subfloat[]{
		\includegraphics[width=0.25\textwidth]{./figure/motivation/function-latency-ml-analyze-varying-worksets.pdf}
		\label{fig:bg:ml-func-latency}
	}
	\hspace{8mm}
	\subfloat[]{
	\includegraphics[width=0.28\textwidth]{./figure/motivation/coresident-perf.pdf}   
	\label{fig:bg:perf-inteference}
	}
	%\vspace{-0.1cm}
	\caption{(a) slacks of function invocations in production traces, (b) function latency variance caused by varying input worksets for functions object detection (OD), question answering (QA), and and text-to-speech (TS), respectively,
 (c) performance interference attributed to co-location of homogeneous function with different dominant resource demands.}
 %\vspace{-0.4cm}
\end{figure*}

%'ml-analyze':{'text-to-speech': 'text-to-speech', 'question-answer': 'question answer',
%                      'object-detection': 'object detection'
\textbf{\textit{Varying working sets.}} 
The working set, i.e., input data like videos, audios, and texts, can have varying sizes.
Taking Microsoft Azure Function Blobs (storage service) as an example, their data size difference can be as high as nine orders of magnitude~\cite{azure-function-blob}.
Such a large difference results in substantial variance of the execution time even for the same function~\cite{socc21-faast,eurosys21-ofc}.
Specifically, we measure the execution time of three functions under different working sets (detailed in \S\ref{exp:setup}).
Figure~\ref{fig:bg:ml-func-latency} illustrates the results, where we can observe a variance of up to 3.8 times in function execution caused by varying working set sizes.

% \begin{figure}[t!]
% \centering
% \includegraphics[width=0.25\textwidth]{././figure/motivation/function-latency-ml-analyze-varying-worksets.pdf}
% \vspace{-0.3cm}
% \caption{Function latency variance caused by varying input worksets}
% \label{fig:bg:ml-func-latency}
% \end{figure}	

\textbf{\textit{Performance interference.}}
% On the other hand, function deployment, which decides when and where to deploy functions, is completely undertaken by providers.
For simplicity and security, commercial serverless platforms, such as Alibaba Function Compute, Microsoft Azure, and AWS Lambda, exclusively deploy function instances belonging to the same tenant, or even belonging to the same function, in the same virtual machine~\cite{socc22-owl,atc18-peek-bench}.
For example, the empirical study in~\cite{socc22-owl} shows that in Alibaba Function Compute 65\% of the virtual machines exclusively deploy instances of the same function.
This co-location of homogeneous function instances, however, can incur severe resource contention on the same resource dimensions, particularly for network bandwidth and memory bandwidth of virtual machines~\cite{sc21-gsight,micro19-faaSprofiler,socc22-owl,atc18-peek-bench}.
To verify this observation, we use a virtual machine to run a function increasing the number of co-located instances from one to six while measuring the execution time of four different functions with resource dominance on different dimensions namely computing, I/O, network, and memory, respectively (detailed in \S\ref{exp:setup}). 
As shown in Figure~\ref{fig:bg:perf-inteference}, the co-location of homogeneous functions leads to substantial resource contention and performance interference, prolonging the function execution time up to 8.1 times. The performance interference is often hard to model and predict.

% this co-residency results in substantial increase of execution latency by up to 8.1 times,leading to considerable variance in function execution time.
% when compared with that with concurrency as one.

%for CPU-, IO-, network- and memory-intensive functions as the concurrency rises from one to six.
%Figure shows that significant performance interference can be observed, . 
%compared with the inclusive deployment (concurrency as one), 
% this exclusive deployment (gray bar) results in substantial increase of execution latency by up to 8.1$\times$ for CPU-, IO-, network- and memory-intensive functions as the concurrency rises from one to six.

% this exclusive deployment (gray bar) results in substantial increase of execution latency by up to 8.1$\times$ for CPU-, IO-, network- and memory-intensive functions as the concurrency rises from one to six.
% As depicted in Figure~\ref{fig:bg:concurrent_latency}, with the concurrency rising  from one to six,  the exclusive deployment results in substantial increase of execution latency by up to 8.1$\times$.
% This significantly magnifies execution latency variance.

% \begin{figure}[t!]
% \centering
% \includegraphics[width=0.25\textwidth]{./figure/motivation/coresident-perf.pdf}
% \vspace{-0.3cm}
% \caption{Performance interference attributed to co-residency of homogeneous function.}
% \label{fig:bg:perf-inteference}
% \end{figure}




\subsection{Runtime Resource Adaptation}
\label{sec:bg:adaptive-allocation}
To tackle the aforementioned resource inefficiency issue, we can adopt a late-binding approach through \emph{runtime resource adaptation}, which resizes functions on the fly based on runtime information (e.g., function slacks), achieving higher resource efficiency without violating SLO. For example, given a workflow as a chain of functions, the resource allocation of the downstream functions can be adjusted when the first function finishes execution. This way, the slack from the first function can be exploited to optimize resource efficiency. 

The idea sounds straightforward and has been considered in some existing works \cite{infocom22-stepconf,middleware20-fifer,socc21-llama,socc21-kraken,middleware20-xanadu}.
However, most of these works make an unrealistic assumption that either the developer performs the adaptation decision with access to runtime information or the serverless platform provider performs the adaptation with domain knowledge of the application workflow. These assumptions render these solutions impractical to deploy in real-world serverless systems. The information barrier between the developer and the provider calls for a new solution. 

We identify the following challenges and opportunities for a full-fledged design for runtime resource adaptation. 

\textbf{\textit{Skewed function execution time distribution.}} 
Resource allocation for a serverless workflow is typically done by leveraging performance profiles of all the functions in the workflow. 
During the offline profiling, the execution time distribution for each function is first obtained by running the function with a variety of sample inputs under different resource conditions. Then, given a time budget, existing approaches typically use P99 of the function execution time as a target and calculate the corresponding resource demands. However, due to the high runtime variability, the distribution of the function execution time is highly skewed where the difference between P50 and P99 can be as high as 100 times~\cite{socc23-huawei-cloud}. This means that if only the function execution time at a single percentile (P50 or P99) is used for resource allocation, there will be significant resource under-provisioning and over-provisioning for most requests at runtime. To address this issue, our idea is to allow for the exploration of the function execution time at diverse percentiles during resource allocation. 


% It is a prerequisite to profile execution latency for adaptive resource allocation.  
% As aforementioned, owing to a variety of unexpected runtime dynamics,  execution latency demonstrates skewed distributions, by up to 100$\times$ between 99\% percentile and 50\% percentile on Huawei cloud serverless~\cite{socc23-huawei-cloud} .
% This makes the current a single statistic (e.g., mean) or 99\% percentile distribution based profiling suffer significant under- and over-estimation.
% To fix this issue, our insight is to \textit{introduce more diverse percentiles to profile execution latency}. 

\textbf{\textit{Dependencies of adaptation decisions.}}
As the function execution progresses, a sub-workflow will be generated by removing the finished function(s) from the workflow. Within each sub-workflow, the resource adaptation decisions for remaining functions are dependent on each other due to the constraint imposed by the end-to-end latency SLO. For example, under-provisioning a function will result in a reduction of the time budget for executing its downstream functions, thus calling for more resources for these downstream functions to avoid SLO violations. Meanwhile, the selection of the percentile for the execution time of each function dictates resource-latency tradeoff for that function. For example, a higher percentile means that more resources will be allocated to ensure that more requests processed by the function will finish within the given time budget. On the contrary, a lower percentile means that more requests will risk SLO violation, but at the benefits of reduced resource consumption. To address such complex dependencies, we propose the following ideas: (1) We introduce two metrics (i.e., the timeout metric and the resilience metric detailed in \S\ref{sec:profilier}) to balance the resource adaptation decisions of the head function of the current sub-workflow and those of the remaining downstream functions. These metrics help us connect the decision making across sub-workflows and avoids sub-optimal adaptation decisions in each sub-workflow. 
(2) We explore lower percentiles for the head function and a high percentile (i.e., P99) for other functions in each sub-workflow. Using lower percentiles maximizes the opportunity for resource optimization since any over-time execution of the head function can later be compensated by resource adaptation in the next round. The high percentile ensures that the resource adaptation is not too radical to cause SLO violations. 

% Each workflow generates multiple sub-workflows as the execution moves forwards. 
% Within sub-workflows, the provisioning is inter-corrected.
% For instance, under-provisioning upstream functions may directly shrink the time budget for downstream functions, which dictates more resources required by the latter against (sub-) SLO violation. 
% This makes sub-workflows generally adopted as the basic unit to make adaptation decisions~\cite{socc21-llama,rtas22-fa2}. 
%  Moreover,  due to the high variance of execution performance, runtime adaptation requires to carry out function by function, i.e.,  discrete adaptation.
%  This, however, can easily lead to a sub-optimal (analyzed in~\S~\ref{sec:synthesizer:generate}).
% Our insight is to \emph{introduce a metric (i.e., resilience detailed in \S~\ref{sec:profilier}) to quantify the inter-correlation as well as a heuristic design (i.e., heavier head explained in \S~\ref{sec:synthesizer:generate})  to calibrate the sub-optimal,  such that resource adaptation can explore higher resource efficiency without SLO guarantee}.

% In particular, latency percentiles (introduced by the profiling)  involves resource adaptation as a new knob.
% Specifically, higher percentile earns  stronger guarantees in SLOs but may be highly prone to resource over-allocation because of its latency over-estimation, impairing resource efficiency.
% In contrast, decreasing percentiles offers the opportunity to explore higher resource efficiency, but suffers the risk of timeout, i.e., execution latency beyond specified time budget, and  may thus incur  SLO violations.
% Here, our insight is to \emph{moderately explore percentiles (detailed in~\S~\ref{sec:synthesizer:generate}), where head functions of  (sub-)workflows can explore lower percentiles because this creates the opportunity to reap higher resource efficiency while possible timeout can be recovered by subsequent functions' re-adaptive allocation.
% On the other head, non head functions maintain percentiles as 99\%}.
% This can well keep the trade-off between opportunities of exploring higher resource efficiency and risks of SLO violations. 
% Additionally, it effectively shrinks the searching space, benefiting the adaptation with higher time-efficiency.


\textbf{\textit{Tight resource adaptation window.}}
Runtime resource adaptation requires to calculate a new resource allocation decision for the remaining sub-workflow immediately when a function finishes execution. Since serverless functions are typically short-lived (less than 1s on average)~\cite{atc18-peek-bench,socc22-owl,atc20-serverless-in-the-wild,socc23-huawei-cloud}, the window for resource adaptation is quite tight. Assuming the serverless platform will perform the runtime adaptation on behalf of the developer since the platform has access to full runtime information, the resource adaptation decision making should be fast without involving complex calculations and logic or exploring a large space. As discussed before, the serverless platform provider does not have domain knowledge of the serverless workflow. Hence, the developer must pass the necessary information to the serverless platform for runtime adaptation decision making. Our idea is to let the developer synthesize critical hints containing resource allocation rules and options, which the serverless platform provider utilizes to perform runtime resource adaptation. The hints should be highly condensed so the serverless platform can make adaptation decisions quickly enough. 


% Apart from highly varying execution performance, serverless functions are also short-living (less than 1s on average)~\cite{atc18-peek-bench,socc22-owl,atc20-serverless-in-the-wild,socc23-huawei-cloud}, so is the window for adaptive allocation. 
% This variance and volatility calls for a well-preparation of hints for all possible runtime situations while promising them compact and straightforward enough for providers to easily take action.

% Here, our insight is to \emph{holistically synthesize hints in an offline manner, and then utilize the discreteness of adaptive allocation in both decision-making and decision-executing (detailed in~\S~\ref{sec:synthesizer:condense}) to fully condense the hints.
% Finally, hints are warped into a simple and compact table.
% Base on that, providers can accomplish the runtime adaption promptly and properly}.

To demonstrate the potential of runtime resource adaptation incorporating all the above ideas, we take a real-world serverless workflow (explained in \S\ref{exp:setup}) as an example, and evaluate its end-to-end latency (denoted by E2E) and resource consumption (CPU cores).
As illustrated in Figure~\ref{fig:bg:size}, the late-binding (blue triangle) reduces the resource consumption by up to 42.2\% compared with existing early-binding solutions (orange circle), while ensuring SLO guarantees. This highlights the significant gains from runtime resource adaptation. 


\begin{figure}[t!]
\centering
\includegraphics[width=0.45\textwidth]{./figure/motivation/size_early_bind_vs_ours.pdf}
%\vspace{-0.1cm}
\caption{Performance comparison between early-binding (left)~\cite{eurosys19-grandslam} and late-binding (runtime resource adaptation), where the CPU consumption (right) is normalized by the optimal obtained with exhaustive search.} 
%\vspace{-0.3cm}
\label{fig:bg:size}
\end{figure}

   
	







% \section{code refinement}

% 在一次code review过程中,首先由developer提交了一份修改,用最初的代码C0修改到代码C1。而后,reviewer针对这次修改(C0->C1)提出了review comment(RC),无论是reviewer提交review,还是developer查看review,RC会呈现在某一行代码之后,我们称这一行代码是review line(RL)。通常review line是C0->C1的修改部分,review对上次的修改进行评价和建议,也有少部分情况review line是针对未修改代码提出新的修改建议。而后developer根据RC,对C1进行修改,得到新的代码版本C2。传统的code refinement任务我们称之为Basic Code Refinement,其input为:<C1, RC>,output为<C2>。然而,我们注意到一些数据只提供C1, RC是信息不足的。如图1所示,例子a需要提供review line才能确定review要删除哪一行。例子2中,需要指导C0才能确定如何回退代码。我们定义两种新型的任务,Position-Aware Code Refinement: 其输入为<C1,RC,RL>,输出是<C2>;	Comprehensive Code Refinement:输入为<C0,C1,RC,RL>,输出也是<C2>。本文主要研究的对象就是Comprehensive Code Refinement。
% 为了避免文字混淆,我们称C0版本的代码是initial code,C1版本的代码是original code,C2版本的代码是revised code。



% 目前coderefine方向有两个使用广泛的数据集,Tufano数据集和codereview数据集。如前文介绍的,我们需要数据集提供initial code, original code, review line, review comment, revised code等五个字段。Tufano数据集没有initial code字段,且没有提供原始数据的链接。而codereview数据集虽没有review line,initial code两个字段,但是提供了原始的数据连接,故而我们选择使用codereview数据,并补齐缺失字段。

% 首先介绍review line字段获取方法。我们观察到通过GitHub REST API获取code review信息时,可以获取到partial last code diff(在API返回的json中叫做diff hunk字段,给个角标https://api.github.com/repos/meganz/sdk/pulls/comments/326107667)。之所以我们称之为partial last code diff,是因为这段code diff只提供了review comment之前的修改信息。在例子中,原本的last code diff有三行代码删除,三行代码添加。review line在第一行代码添加之后。所以partial last code diff只有三行删除,一行添加。根据这个规律,我们可以得到review line就是partial last code diff的最后一行。

% 而后我们需要设计得到initial code的方法。我们观察到,一次code review可能是reviewer对前面多次commit的review。如果当前pull request有n个commit:(commit_1,commit_2,...,commit_n),reviewer可以选择commit_m到commit_n之间所有commit(n小于等于m),然后查看这些commit叠加后的文件变化,而后再给出review comment。而GitHub REST API只给出了最后一次commit的id,即commit_n,无法确定前面的commit_m。不过GitHub REST API中提供的partial alst code diff就是commit_m到commit_n的code diff。故而,我们倒序依次遍历前面的所有commit,并与commit_n做比较,得到review line附近的code diff。并与partial last code diff做对比,就可以找到与partial last code diff一致的,完整的last code diff。根据last code diff和original code就可以得到initial code。
\begin{figure*}[hbt]
  \centering
  \includegraphics[width=\textwidth]{figs/pipeline.pdf} 
  \caption{Overview of the framework.}
  \label{fig:overview}
\end{figure*}

\section{Method}
\subsection{Overview}
Our method, \textsc{Drangen3D}, takes an image as input and generate a 3D object represented by 3D Guassians with multi-view geometric consistency, allowing user interaction of editing the geometry during the process. As illustrated in Fig.~\ref{fig:overview}, we first train an Anchor-Gaussian (Anchor-GS) VAE that encodes complex 3D information into a latent space and decodes it into 3DGS, enabling subsequent generation in the latent space (Sec.~\ref{sec:anchor-vae}).  
%
Then, we propose Seed-Point-Driven Controllable Generation module for 3D generation from a single image. This module starts with the generation of the rough initial geometry represented by a set of sparse surface points, named seed points, where we can apply the editing by deforming the seed points. After that, a mapping module is designed to map the (edited) seed point information to the latent space, which can be decode to 3DGS subsequently (Sec.~\ref{sec:seed-point-driven}). 

% \todo{Note that we do not directly generate anchor points because: computational complex; easy to learn; generated anchor points contains noise that affect the final geoemtry}


% As illustrated in Fig.~\ref{fig:overview}, the framework of \textsc{Dragen3D} mainly consists of two parts. We first adopt an anchor-based approach to obtain 3D Gaussians and propose the Anchor-GS VAE that constructs 3D Gaussians by leveraging points sampled from 3D assets and rendered images.
% %
% Through the anchor-based representation, we can compress both geometric and texture information into a set of anchor latents, which is beneficial for latent generation, as described in Sec.~\ref{sec:anchor-vae}

% Then we generate a set of sparse seed points to control the anchor latents and thus deform the final output 3DGS, designing and utilizing the Seed-Anchor Mapping module, as described in Sec. \ref{sec:seed-point-driven}.

% Overall, combining anchor-based 3DGS VAE and seed-point-driven generation approach gives large flexibility in geometric control and deformation during the 3D generation process.

\subsection{Background}
\paragraph{Gaussian Splatting}
% 3D Gaussian Splatting represents 3D objects explicitly through radiance-based methods, leveraging high-degree shape and color features for multi-view synthesis. Its differentiable volume rendering properties make it suitable for image-to-3D object generation, allowing for the alignment of conditioned image textures. This generation process $G$ can be outlined as follows:
% \begin{equation}
%     G:\{C\}_{x',y'} \rightarrow {\{\mu,o,A,F\}}_p,
% \end{equation}
% where $\{C\}_{x',y'}$ is pixels of the input at $(w, h)$. 
% % \mc{need to use different symbols to describe pixal and gaussian positions}.
% $\mu$, $o$, $A$, and $F$ denote the position, opacity, covariance matrix, and spherical harmonic features of each Gaussian particle $p$, respectively. The discretized splatting rendering renders at $\{C\}_{x',y'}$:
% \begin{equation}
%     \displaystyle\sum_{i \in \mathcal{N}}  \alpha_i \boldsymbol{SH}(r|_{x',y'};F_i)                      \displaystyle\prod_{j=1}^{i-1} (1-\alpha_j )\rightarrow \{C\}_{x',y'},
% \end{equation}
% where $\alpha_i$ is the product of $o_i$ and the projected Gaussian density of where the kernel interacts with the ray in direction $r|_{x',y'}$ from the specific pixel at $(x',y')$, and $\boldsymbol{SH}$ denotes the color calculated with features $F_i$. 
% This is equivalent to computing a depth-and-opacity weighted average color of particles along the ray direction. Consequently, individual Gaussian points possess adequate geometric and chromatic information.
% \subsection{3DGS}
% gaussian splatting将静态场景表示为一组各向异性的3d gaussians,每个像素的颜色通过基于点的alpha混合渲染来获得,从而实现高保真度的实时新视角合成。
Gaussian splatting represents scenes as a collection of anisotropic 3D Gaussians. Each Gaussian primitive $\mathcal{G}_i$ is parameterized by a center $\mu \in \mathbb{R}^3$, opacity $\alpha \in \mathbb{R}$, color $c \in \mathbb{R}^{3(n+1)^2}$ which is represented by n-degree SH coefficients and 3D covariance matrix $\Sigma \in \mathbb{R}^{3 \times 3}$,which can be represented by scaling $s\in \mathbb{R}^3$ and rotation $r\in \mathbb{R}^4$.
% \begin{equation}
%   \mathcal{G}(x) = e^{-\frac{1}{2}(x-\mu)^T\Sigma^{-1}(x-\mu)}
%   \label{3dgs_define}
% \end{equation}

% 为了保持协方差矩阵的物理意义,它必须是半正定的。因此,将协方差矩阵分解为一个缩放矩阵S和一个旋转矩阵R,其中S和R分别使用一个缩放向量和一个四元数旋转向量来表示。
% To maintain the physical meaning of the covariance matrix, it must be positive semi-definite.Therefore, the covariance matrix $\Sigma$ can be decomposed into a scaling matrix $S$ and a rotation matrix $R$:
% \begin{equation}
%   \Sigma = RSS^T R^T
%   \label{3dgs_decomposed}
% \end{equation}

% 渲染时首先将3D gaussian投影到2D空间。给定视角变换W,可以计算得到2D协方差矩阵,其中J是the Jacobian of the affine approximation of the projective transformation.随后基于深度对覆盖一个像素的高斯进行排序,使用基于点的alpha混合渲染得到像素的颜色。
During rendering, the 3D Gaussian is first projected onto 2D space. Given a view transformation matrix $W$, the 2D covariance matrix $\Sigma'$ can be computed as :
$\Sigma' = JW\Sigma W^T J^T$, where $J$ is the Jacobian of the affine approximation of the projective transformation. Subsequently, the Gaussians covering a pixel are sorted based on depth. The color of the pixel is obtained using point-based alpha blending rendering:
\begin{equation}
  c = \sum_{i=1}^n c_i \alpha_i \prod_{j=1}^{i-1}(1-\alpha_i)
  \label{3dgs_render}
\end{equation}

\paragraph{Rectified Flow Model}
% Recitified flow modle有建立两个分布\pi_0, \pi_1之间mapping的能力,所以很适合我们的任务。给定x_0 ~ \pi_0 和 相对应的x_1 ~ \pi_1, 我们可以通过liner interpolation  得到 x(t) = (1-t) x_0 + t x_1 at timestamp t. And a vector filed v_sita parameterized by a neural network 被用来drive the flow from source distribution \pi_0 to target distribution \pi_1 by minimizing the conditional flow matching objective:
The Rectified Flow Model \cite{liu2022flow, lipman2022flow} has the capability to establish a mapping between two distributions, \( \pi_0 \) and \( \pi_1 \), making it well-suited for our task of mapping seed point latents to anchor latents. Given \( x_0 \sim \pi_0 \) and the corresponding \( x_1 \sim \pi_1 \), we can obtain \( x(t) = (1 - t) x_0 + t x_1 \) at timestamp \( t \in [0,1]\) through linear interpolation. A vector field \( v_{\theta} \) parameterized by a neural network is used to drive the flow from the source distribution \( \pi_0 \) to the target distribution \( \pi_1 \) by minimizing the conditional flow matching objective:
\begin{equation}
    L(\theta) = E_{t,x_0,x_1,y}||v_{\theta}(x_t, t,y) - (x_1 - x_0)||
    \label{eq:flow matching}
\end{equation}
Here, $v_{\theta}(x_t, t, y)$ is the predicted flow at time $t$ for a given point $x_t$, $y$ refers to the image  condition that guides the flow matching.
\section{Experimental Setup}
To evaluate the effectiveness of the proposed approach, we design the following research questions.
\begin{itemize}[leftmargin=*,topsep=2pt]
    \item \major{RQ1: How accurate are LLMs in extracting intentions?}
    \item RQ2: To what extent is the Intention-based framework effective in code refinement?
    \item RQ3: To what extent does the framework improve performance across different intention categories?
    \item RQ4: To what extent is intention-based dataset cleaning effective?
\end{itemize}





% Our objective is to systematically evaluate the effectiveness of the Intention-based Code Refinement Framework in addressing the Comprehensive Code Refinement task. The research questions include:

% 1. To What Extent is the Intention Extraction Accurate?

% 2. To What Extent is the Intention-based Framework Effective in Code Refinement?

% 3. To What Extent Does the Framework Improve Performance Across Different Intention Categories?

% 4. To What Extent is Intention-Based Dataset Cleaning Effective?
 
% 我们的目标是系统的评估Intention-based code refinement framework解决Comprehensive Code Refinement任务的有效性。研究问题包括:首先,需要先调研Extracting Intention这个步骤中取得Intention的正确率。其次,验证整体框架的有效性,即评估模型通过框架解决code Refinement任务的正确率,对比不使用本框架的方法是否有提升。再次,调研不同Intention类别的数据,使用框架可以提升多少效果。最后,研究使用Intention作为数据质量判别的方法,对清洗数据集的效果。

\subsection{RQ1: Intention Extraction Accuracy}
The accuracy of the intention understanding is important for code refinement. 
% The concept of intention is a novel proposition we introduced, which currently lacks an existing benchmark. 
To measure the accuracy, we constructed a test dataset from the existing code review dataset~\cite{li2022automating}. We randomly select 2,000 samples for manual annotation to assess the accuracy of different LLMs in intention extraction. 
% Correct intention extraction implies that the intention accurately summarizes the reviewer’s suggested solution, allowing the developer to refine the code without having to focus on the review comments but solely relying on the intention.


% As presented in Fig.~\ref{fig:framework}, three general categories are designed including explicit code suggestion, reversion suggestion and general suggestion. Followed by the rule-based intention analyzer, we use GPT4o to classify the categories. 
As shown in Fig.~\ref{fig:framework}, for reversion intention and general intention, we use the LLM to extract the intention. To check the accuracy of the extracted intention, we invited two PhD candidates to conduct manual annotations. In cases of differing results, the annotators reached a consensus through discussion to finalize the annotations. During the annotation process, we not only assessed the correctness of intention extraction but also filtered out invalid data. Specifically, cases where the revised code was unrelated to the review comments were excluded to ensure that only relevant and high-quality data were used in subsequent research questions.

% We categorized the selected samples based on the actual \textit{Intentions} into three categories: Explicit Code Suggestion, Reversion Suggestion, and General Suggestion. The following criteria define how to determine the correctness of the framework’s intention extraction:

% 1. Explicit Code Suggestion: If the framework correctly classifies the intention as an Explicit Code Suggestion, the extracted intention is deemed correct.

% 2. Reversion Suggestion: There are two scenarios where the framework's extraction is considered correct for this category:

%    - The framework directly categorizes the intention as a Reversion Suggestion.

%    - The framework classifies the intention as a General Suggestion, but following the suggested fixes achieves the same outcome as the Revised Code.

% 3. General Suggestion: For data categorized under General Suggestion, the framework must classify the intention as a General Suggestion, and the proposed solution must be consistent with the original review comments to be deemed correct.

% Given the extensive nature of manual annotations, we focused solely on evaluating the results of the GPT-4o model in the intention extraction process. Two Ph.D. candidates in software engineering conducted the annotations. 

% Intention的正确性代表了模型真正理解了任务的要求,是做好后续步骤的前提。Intention的概念是我们首次提出,目前还没有benchmark,我们使用codereview数据集的测试数据,随机选取2000个,而后用人工标记的方式来判断Intention的正确率。所谓Intention的正确就是指Intention概括总结了reviewer的建议方案。developer可以不再关注ReviewComment,只根据Intention就能正确修复代码。
% 我们按照真实的Intention把数据分成Explicit Code Suggestion,Reversion Suggestion和General Suggestion这三类。接下来定义如何判断framework提取的Intention是否正确。对于Explicit Code Suggestion类别的数据,只要framework正确将Intention分成Explicit Code Suggestion类别,就认为提取的Intention是正确的。对于Reversion Suggestion类别的数据,framework提取的Intention有两种情况是正确的。一种是framework直接将Intention分成Reversion Suggestion类别。另一种是framework将Intention分类成General Suggestion类别,并且按照修复建议,达到的效果和Revised Code是一致的。对于General Suggestion类别的数据,需要framework将Intention分类到General Suggestion,并且修复的方案和原始ReviewComment是一致的,才算是framework提取的Intention是正确的。
% 由于涉及大量人工标注,我们仅评估了GPT-4o模型提取Intention的结果。共有两个软件工程的PHD进行标注,出现不同结果时,由两人商议再最终确认。在标注过程中,我们不仅标注了Intention提取的正确性,也标注了原始数据的合理性。对于revised code与review comment无关的不合理的数据进行过滤,保证在后续的RQ中,只使用合理的数据。

\begin{figure}[!t]
\centering
\includegraphics[width=0.85\linewidth]{fig/prompt2.pdf}
\vspace{-2mm}
\caption{The used prompt format for different tasks.}
\label{fig:prompt2}
\vspace{-4mm}
\end{figure}


\subsection{RQ2: Intention-based Framework Effectiveness}
To evaluate the effectiveness of our proposed intention-based code refinement, we select two state-of-the-art baselines for comparison. 

% To What Extent is the Intention-Based Framework Effective in Code Refinement?
% In this study, we compare the results of our framework with other code refinement methods. Our selected baselines include two categories: pre-trained models and large model prompt techniques.

% For pre-trained models, we have chosen CodeReviewer~\cite{li2022automating} and T5CR~\cite{tufano2022using} as our baselines.

\noindent \textbf{CodeReviewer~\cite{li2022automating}:} 
It designed three pre-training objectives, i.e., Diff Tag Prediction, Denoising Objective, and Review Comment Generation, to pre-train the model based on CodeT5~\cite{codet5} for code review activities. Several downstream tasks, including code change quality estimation, code review generation and code refinement, are selected to evaluate the effectiveness of the proposed models. 


\noindent \textbf{T5CR~\cite{tufano2022using}:} It utilized two datasets including the official Stack Overflow dump (i.e., SOD) and CodeSearchNet (i.e., CSN) to pre-train the code review model based on T5 architecture. A tokenizer, i.e., SentencePiece~\cite{kudo2018sentencepiece}, is adopted to tokenize the source code, and the input sequence's maximum length is increased to 512 for training. 

Apart from these baselines, we also comprehensively evaluate the effectiveness of different LLMs and prompt strategies. In particular, we select three closed-source LLMs i.e., GPT-4o-2024-05-13 (GPT4o)~\cite{achiam2023gpt}, GPT-3.5-turbo-0125 (GPT3.5)~\cite{ouyang2022training}, DeepSeek-Coder-V2-0724 (DeepSeekV2)~\cite{zhu2024deepseek} and two open-source large models: CodeQwen1.5-7B-Chat (CodeQwen7B)~\cite{bai2023qwen} and Deepseek-coder-6.7b-instruct (DeepSeek7B)~\cite{guo2024deepseek} for evaluation.
\major{Note that, since different models have varying capabilities in extracting intentions, we present results using both the intentions extracted by the model itself and those extracted by a high-quality model (i.e., GPT-4o). The use of GPT-4o intentions allows us to evaluate whether providing the correct intention can enhance performance in code refinement tasks.}


% The T5~\cite{raffel2020exploring} model serves as the foundational model, pre-trained on a dataset comprising 1.5 million Java-English corresponding pairs through the Masked Language Model Task. In this experiment, we employed the same pre-trained model and also fine-tuned it using the $CodeReview_{train}$ datasets.

% Since our framework, like prompt techniques, is model-agnostic, we conducted a comprehensive and detailed comparison of five models and six common prompt strategies under three code refinement tasks as baselines.

% The models include three closed-source large models: gpt-4o-2024-05-13 (Gpt4o)~\cite{achiam2023gpt}, gpt-3.5-turbo-0125 (Gpt3.5)~\cite{ouyang2022training}, and DeepSeek-Coder-V2-0724 (DeepSeekV2)~\cite{zhu2024deepseek} and two open-source large models: CodeQwen1.5-7B-Chat (CodeQwen7B)~\cite{bai2023qwen} and Deepseek-coder-6.7b-instruct (DeepSeek7B)~\cite{guo2024deepseek}.

% The three tasks include (see Section 3): Basic Code Refinement, Position-aware Code Refinement, and Comprehensive Code Refinement. This allows for a thorough comparison of the effectiveness of various refinements under different input scenarios.
For LLMs, we selected different prompting strategies:

\noindent \textbf{Simple Prompt:} As shown in Fig.~\ref{fig:prompt2}, the Simple Prompt first describes the scenario, then introduces the input information in the task, and finally requests the generation of revised code based on the provided information. This is a concise and effective prompt design used in Guo et al.~\cite{guo2024exploring}.

\noindent \textbf{Simple COT Prompt:} This prompt builds upon the simple prompt by adding the phrase ``Let's think step by step." This technique is employed in model reasoning~\cite{wei2022chain,wang2022self}.

\noindent \textbf{Tufano COT Prompt:} It is introduced in Tufano et al.~\cite{tufano2024code}, which first requires determining which of the following six categories the modification belongs to before completing the modification and then utilizes LLM for the generation.

% The six modification categories are:

% - Changes have been required to refactor the code to improve its quality;

% - Changes have been required since tests for this code must be written;

% - Changes have been required to better align this code to good object-oriented design principles;

% - Changes have been required to fix one or more bugs;

% - Changes have been required to improve the logging of its execution;

% - Changes have been required for other reasons not listed above.

\noindent \textbf{Random Few-shot Prompt:} For each case, three data examples are randomly selected as examples for the prompt as few-shot (excluding self-examples). The data fields provided as examples depend on the task's input fields, such as \texttt{OriginalCode}, \texttt{ReviewComment}, \texttt{ReviewLine}, \texttt{RevisedCode} in Comprehensive Code Refinement, and only \texttt{OriginalCode}, \texttt{ReviewComment}, \texttt{RevisedCode} in Basic Code Refinement. After providing examples as prompt hints, they are concatenated with the simple prompt.

% \noindent \textbf{RAG Prompt:} Similar to Random Fewshot, but the example selection is based on the most similar data retrieved using the BM25~\cite{robertson2009probabilistic} method. We use the \texttt{ReviewComment} as the key for retrieval, with the complete data as the value stored in the retrieval set. For each case, its \texttt{ReviewComment} is used to retrieve three nearest neighbors (excluding itself). After retrieval, the prompt construction method is the same as Random Fewshot.

\noindent \major{\textbf{RAG Prompt:} 
To compare with the Random Few-Shot Prompt, we design a Retrieval-Augmented Generation (RAG) prompting method as an alternative approach to few-shot prompting, leveraging relevant example selection. The retrieval database is constructed from the dataset used in RQ1, specifically the 2,000 randomly selected samples from the CodeReview dataset. However, only 1,337 of these samples are included in the database, as the remaining 663 cases exhibit low-quality refinements that do not align well with the review comments.
We selected these samples for two reasons: 1) The CodeReview dataset uniquely provides the complete data required for this study, including \texttt{OriginalCode}, \texttt{ReviewComment}, \texttt{ReviewLine}, \texttt{RevisedCode}, and \texttt{LastCodeDiffHunk}. 2) The quality of the retrieval dataset is crucial, and these 2,000 samples have been manually analyzed in RQ1, ensuring their reliability.
During testing, for each test case, we retrieve three samples to construct a 3-shot prompt. If the test data is included in the retrieval results, it is excluded and replaced with another sample. To enhance contextual relevance, we use BM25 to select semantically similar examples, ensuring that the retrieved examples closely align with the test data's context..}

\noindent \textbf{Self-generated Prompt:} This prompt technique addresses mathematical and reasoning problems by utilizing the model's own understanding ability~\cite{yasunaga2023large}. It first generates several examples and then uses these examples to inspire the model to answer the original question.

% The format of these prompts is provided on our website~\cite{IntentionWebsite}. 

\noindent \textbf{Evaluation Metrics:} To fully automate the code refinement task, we prioritize the exact match (EM) between the model's output and the actual revised code. While metrics like BLEU and Code-BLEU can reflect proximity to the ground truth, they do not effectively gauge the degree to which the model's output aids programmers in code refinement. Particularly in simple tasks, if the model's output is not completely correct, it may be less beneficial for programmers to use the model's output for refinement than to directly use the review comment. Therefore, our evaluation metric only includes the EM value.

% Dataset: The data we used is the manually labeled valid data from RQ1.

% In this experiment, our framework utilizes the RAG prompt method. By comparing its performance with these baselines, we can assess whether our framework surpasses existing methods.


% To What Extent is the Intention-based Framework Effective in Code Refinement? 我们将framework的结果与其他Code Refinement方法进行比较。
% 我们选用的baseline包括预训练模型和大模型的prompt技术两种。
% 预训练模型,我们选择的baseline是CodeReviewer和T5CR。
% \textbf{CodeReviewer:} The model is initialized using the weight parameters of CodeT5 ~\cite{codet5}. Subsequently, the pre-training is carried out with three objectives: Diff Tag Prediction, Denoising Objective, and Review Comment Generation. In this experiment, we employed the same pre-trained CodeReviewer model and fine-tuned it using the $CodeReview_{train}$ datasets.
% T5CR:使用的T5作为基础模型,在一份1.5million个的,关于java和英语对应关系的数据集上,进行Masked Language Model Task任务预训练。In this experiment, we employed the same pre-trained model and fine-tuned it using the $CodeReview_{train}$ datasets.
% 因为我们的框架和prompt技术一样,都是模型无关的。所以我们全面而详细的对比5种模型,6种常见的prompt,在三种code Refinement任务下的效果作为baseline。
% 模型我们采用了3个闭源大模型,gpt-4o-2024-05-13,gpt-3.5-turbo-0125,DeepSeek-Coder-V2-0724,和2个开源大模型CodeQwen1.5-7B-Chat,deepseek-coder-6.7b-instruct。
% 三种任务包括(见第三节):Basic code Refinement,Position-aware code Refinement和Comprehensive code Refinement等三种任务。这样充分对比不同输入信息的情况下,哪种修复的效果最好。
% Prompt方法包括了Simple Prompt,Simple COT,Tufuno COT,Random Fewshot,RAG Prompt, Self-generated Prompt等6种常见的prompt策略。
% Simple Prompt:如图3所示,Simple Prompt首先描述场景,然后介绍任务中的输入信息,最后要求根据提供的信息,生成revised code。是Guo的文章中使用的,简洁有效的Prompt设计。
% Simple COT Prompt: 在simple prompt的基础上,加上“Let's think step by step.”。这是在多个推理任务中使用的prompt方法。
% Tufano COT Prompt:在Tufano论文中提到的一种COT技术,首先要求判断修改类型属于以下六个类别的哪一种,然后在完成修改。六种修改类别为:
% Changes have been required to refactor the code to improve its quality;
% Changes have been required since tests for this code must be written;
% Changes have been required to better align this code to good object-oriented design principles;
% Changes have been required to fix one or more bugs;
% Changes have been required to improve the logging of its execution;
% Changes have been required for other reasons not listed above.
% Random Fewshot Prompt:用有效数据为样例选取数据集。对每个case,随机选取3个数据作为examples提供给prompt作为fewshot(已排除掉自己给自己做example的情况)。例子提供的数据字段依照任务本身输入字段,如Comprehensive Code Refinement里包含Initial code,original code,review comment,review line,revised code等5个字段。而Basic Code Refinement只包含original code,review comment,revised code三个字段。提供例子作为prompt提示后,再拼接上simple prompt。
% RAG Prompt:与random fewshot类似,只不是选取example时,使用的是用BM25方法检索到的最相似的数据。我们使用review comment作为检索的key,完整的数据作为value,存储再检索集中。对每个case,使用其review comment去检索相近的3个邻居(排除掉自身)。检索到example之后,后面构造prompt的方法与random fewshot相同。
% Self-generated Prompt:这是解决数学和推理问题一种prompt方法,利用模型自身对问题的理解能力,先生成若干个例子,然后再根据这些例子来启发模型,回答最初的问题。
% 详细的prompt设计我们已上传到网站。
% 评价指标:为了完全自动化code Refinement任务,我们更关模型输出结果与真实RevisedCode是否完全匹配,即EM。BLEU与Code-BLEU等指标虽然可以反映与真实值的接近程度,但是对code Refinement任务不能很有效的反映模型输出结果对程序员的帮助程度。特别是对于简单的任务,如果模型输出结果不是完全正确,那么程序员利用模型输出的结果再去做Refinement,可能不如直接利用ReviewComment来做Refinement。所以,我们这次的评价指标只选择了EM值。
% 数据集:我们所用的数据是RQ1中人工标记valid的数据
% 在这个实验中,我们框架选用RAG的prompt方法。通过与这些baseline的效果做对比,可以看到我们的framework是否可以超过现有的方法。

\subsection{RQ3: Improvement Across Different Intention Categories}

% To What Extent Does the Framework Improve Performance Across Different Intention Categories?

In this experiment, we mainly evaluated the effects of different prompting strategies and three prompting strategies: Simple Prompt, RAG, and Self-generated Prompt. We aim to understand how much our framework improves performance for each of Intention types compared to methods that do not use the framework.

\major{Additionally, we conducted ablation experiments on the Intention Extraction component of our framework, specifically evaluating the impact of the three agents (Agent1, Agent2, and Agent3). For each agent, we tested the Exact Match (EM) values by removing the respective agent and comparing the results. This analysis allows us to quantify the individual contribution of each agent to the overall performance of the framework and to identify which components are most critical for improving the accuracy of intention-based predictions.}

% The model, data, and experimental settings are the same as those used in RQ2.

% To What Extent Does the Framework Improve Performance Across Different Intention Categories? 
% 在这个实验中,我们对比了Simple Prompt,RAG和Self-generated Prompt等三种方法,在使用和不使用框架的效果。特别的,对于每一种Intention类型都进行了详细的对比。我们希望了解,相较于不使用框架的方法,我们的框架对于3种不同的Intention类型的数据,效果分别提升了多少?
% 模型,数据等实验设置与RQ2相同

\subsection{RQ4: Intention-Based Dataset Cleaning}

Lastly, we aim to explore the capability of using intention to enhance data quality. A significant challenge in code refinement data quality is the inconsistency between the content expressed in the review comment and the modifications made in the revised code. Currently, no practical method exists for aligning the review comment and revised code at the semantic level. We will attempt to use intention to determine whether the modifications in the revised code meet the reviewer's requirements.

The Intention Method we designed is as follows: we provide the model with information about the Intention, Original Code, Review Line, and Revised Code and use GPT4o as a classification model to determine whether the modifications in the Revised Code meet the Intention requirements.

We also compared this with a common method that does not use the intention: we provide the model with information about the Review Comment, Original Code, Review Line, and Revised Code and use GPT4o as a classification model to determine whether the revised code meets the requirements of the review comment.

% By systematically evaluating these methods, we aim to demonstrate how intention can potentially improve the alignment between review comments and revised code, thereby enhancing the overall data quality in code refinement tasks.

% 最后,我们要探索,使用Intention来提升数据质量的能力。review comment所表达内容与revised code所修改的内容不一致,是影响code Refinement数据质量的一个难题,目前还没有很好的方法完成review comment和revised code语义级别的对齐。
% 我们将尝试使用Intention来判断revised code的修改是否符合要求。
% 我们设计的方法是:告诉模型Intention,OriginalCode,ReviewLine和RevisedCode等信息,使用Gpt4o作为分类模型,让模型判断RevisedCode的修改是否符合Intention要求。
% 我们对比了不使用Intention的方法:告诉模型ReviewComment,OriginalCode,ReviewLine和RevisedCode,使用Gpt4o作为分类模型,让模型判断RevisedCode是否符合ReviewComment的要求。

\section{experimental results}

\subsection{RQ1: Intention Extraction Accuracy}

\begin{table}[!t]
\caption{\major{Results of intention accuracy.}}
% \vspace{-4mm}
\footnotesize
\centering
\label{tb:rq1-1}
\resizebox{.4\textwidth}{!}{
\begin{tabular}{ccccc}
\hline
           & Explicit & Reversion        & General          & All clean data   \\ \hline
\#Samples  & 175      & 308              & 854              & 1337             \\
\major{GPT3.5}     & -        & \major{84.42\%}          & \major{49.18\%}          & \major{63.95\%}          \\
GPT4o      & -        & \textbf{99.03\%} & \textbf{66.86\%} & \textbf{78.61\%} \\
\major{DeepSeekV2} & -        & \major{95.45\%}          & \major{62.88\%}          & \major{75.24\%}          \\
\major{DeepSeek7B} & -        & \major{80.19\%}          & \major{46.72\%}          & \major{61.41\%}          \\
\major{CodeQwen7B} & -        & \major{82.14\%}          & \major{43.91\%}          & \major{60.06\%}          \\ \hline
\end{tabular}
}
\vspace{-4mm}
\end{table}




\major{
We manually annotated the data to select valid samples, with 1,337 out of a total of 2,000 identified as valid. To ensure annotation quality, we enlisted two senior PhD candidates specializing in code learning to conduct the manual labeling. Disagreements between the annotators were observed in 249 instances, and inter-rater reliability was assessed using Cohen’s Kappa coefficient, which yielded a value of 0.719, indicating acceptable agreement.
Subsequently, we used various models to generate intentions using our framework, and then manually assessed each generated intention for its correctness. Notably, since the Explicit category is classified using predefined rules, we just measure the accuracy of the reversion intention and general intention. 
% As shown in Table~\ref{tb:rq1-1}, 1,337 cases out of the total 2,000 samples were identified as valid. We used different models to predict the intentions and then manually checked the accuracy of these predictions.
% To ensure annotation quality, we invited two senior PhD candidates specializing in code learning to manually label the data. There were 249 instances of disagreement between the annotators, and the inter-rater reliability was evaluated using Cohen's Kappa coefficient, which yielded a value of 0.719, indicating acceptable agreement.
}

% As shown in Table~\ref{tb:rq1-1}, there are 1,337 clean cases, accounting for 66.9\% of the total 2,000 samples. 
\major{As shown in Table~\ref{tb:rq1-1}, we observe that GPT-4o and DeepSeekV2 are significantly more effective than the other three models. Among these, Reversion Intention is relatively easier to extract, with all models achieving at least 80\% accuracy, while GPT-4o achieves an impressive 99.03\%.}


It is noteworthy that 20\% of Reversion Suggestions overlap with General Suggestions, representing a dual classification challenge. In these instances, predicting the intention as either category is considered correct. 
For General Suggestions cases, the highest accuracy is 66.86\%, indicating that this category presents more significant challenges for the model to comprehend. Unlike the other categories, General Suggestions lack specific patterns or structured cues, making it more difficult for the model to pinpoint the exact intention behind the suggestions. This lower accuracy highlights the complexity and ambiguity inherent in general suggestions, which often require a deeper understanding of the context and the underlying reasoning behind the recommendations.



% Among these clean cases, the intention can be correctly predicted in 1,051 cases, which is 78.61\% of the clean samples. This result highlights the ability of the Intention-based Framework to effectively identify and predict intentions in a significant majority of the clean data. Overall, intention correctly predicts 1,371 cases, accounting for 68.6\% of the total samples, demonstrating its overall reliability across different types of suggestions.

% For Explicit Code Suggestions cases, the accuracy of intention prediction is 100\%. This high accuracy is achieved by employing regular expressions to determine whether a case belongs to this category. The rule-based prediction method ensures that each explicit suggestion is captured without errors, reflecting the structured and well-defined nature of these suggestions. Such a robust approach leaves no room for mistakes, making it a perfect classification for these cases.


% In the case of Reversion Suggestions, the accuracy is 99.03\%, reflecting a strong performance with a minimal margin for error. 





% The overall performance of the Intention-based Framework illustrates its potential in accurately predicting intentions across various suggestion types. The ability to achieve 100\% accuracy in Explicit Code Suggestions and near-perfect results in Reversion Suggestions reflects its robustness and precision in structured scenarios. However, the challenges faced in General Suggestions indicate areas for improvement, inviting further exploration into more sophisticated methods for handling complex and ambiguous cases. 

% As shown in Table~\ref{tb:rq1-1}, there are 1,337 clean cases, accounting for 66.9\% of the total 2,000 samples. Among these clean cases, the intention can be correctly predicted in 1,051 cases, which is 78.61\% of the clean samples. Overall, intention correctly predicts 1,371 cases, accounting for 68.6\% of the total samples.

% For Explicit Code Suggestions cases, the accuracy of intention prediction is 100\%, as we use regular expressions to determine whether a case belongs to this category. This rule-based prediction method is error-free.
% For Reversion Suggestions cases, the accuracy is 99.03\%. Notably, 20\% of Reversion Suggestions are also General Suggestions. For these dual classification tasks, predicting the Intention as either category is considered correct.
% For General Suggestions cases, the accuracy is only 66.86\%, indicating that this category is more challenging for the model to understand.

% Overall, for valid samples, the intention extraction accuracy is nearly 80\%, which is very effective. This suggests that using intention to replace review comments in the subsequent task is highly feasible.

% While observing the intention, we found that some cases, although correct, require significant effort to understand the reviewer's intent, even for humans. However, intention can help clarify the task clearly.

% **Case 1:** The review comment, “Any reason not to fold the `currentComponent` assignment into that?”, might be interpreted as asking the developer to comment on the `options.diffed = ()` part of the code, explaining why it is assigned this way. However, considering the previous modifications, the reviewer's real intent is to inquire about the reason for adding the code, implying a suggestion to possibly revert the last modification. Intention can directly discern this meaning and make the correct understanding.

% Additionally, we found during manual labeling that code refinement is a flexible task, often resulting in multiple correct answers for a single task.

% **Case 2:** The review comment, “set the scheduledExecutorService to null?”, can be interpreted in two ways: replacing `this.getScheduledExecutorService().shutdown();` with `this.scheduledExecutorService = null;`, or adding a line to set the service to null after the previous shutdown step. Although the intention and revised code might differ in their representation, both modifications are reasonable to an ordinary programmer.

\vspace{5pt}\noindent \fbox{
\parbox{0.95\linewidth}{\textbf{Answers to RQ1}: \major{GPT-4o achieves the highest accuracy in intention extraction, with DeepSeekV2 performing competitively (with 3\% less). General intentions, however, are more challenging to extract due to the inherent complexity and the often ambiguous nature in the review comments.}}
}


% 从表1可以看出,合理的case共1337个,占样本总数2000个的66.9%。而合理的case中,Intention可以预测正确的有1051个,占合理样本中的78.6%。对总体样本而言,Intention共预测对了1371个,占总体样本的68.6%。接下来具体分析每一个类别的正确率:Explicit code Suggestion类别的Intention预测准确率是100%,这是因为我们是通过正则表达式来判断是否属于第一类,这种rule-based的预测方法不会出错。Reversion Suggestion类别的Intention预测量是99.03%,值得注意的是,Reversion Suggestion有20%的数据同时也是General Suggestion的。也就是说,对于这些数据,如果理解成为要求退回上次修改是正确的Intention,如果按照Intention具体的执行change xxx code to xxx,也是可以正确修复的。最后,对于General Suggestion类别,Intention正确率只有66.86%,说明这一类理解比较困难。总体而言,对于合理样本,Intention的提取正确率接近80%具有非常好的效果,这说明使用Intention在接下来生成revised code步骤中代替review comment,具有较高可行性。

% 我们在观察Intention时,发现了一些case虽然是正确的,但是从即使人来做,也需要很大的努力才能理解其reviewer的意图。然而,Intention可以帮助人很明确的理解任务。
% 在case1中,如果通过review comment:“Any reason not to fold the `currentComponent` assignment into that?”理解,可能是需要developer对options.diffed = () 这部分代码进行注释,解释为什么要在这样赋值。然而综合考虑上一次的修改情况,review的真实意图是询问这次添加代码的原因,隐含了建议是可能要退回上次修改。而Intention可以直接洞察这种含义,做出正确的理解。

% 另外我们在人工标注时发现code Refinement是很灵活的任务,经常会出现一个任务存在多种正确回答的现象。
% 在case2中,review comment:“set the scheduledExecutorService to null?”,可以理解成把this.getScheduledExecutorService().shutdown();代码替换成this.scheduledExecutorService = null;。也可以理解成在前一步shutdown之后,再添加一行代码将service设为null。即使Intention和revised code表现结果不同,但是普通程序员看来,这两种修改都是合理的。

% 总之,Intention对合理数据的理解正确率高达78.61%,可以作为理解程序的一个可靠方式。


% \begin{table}[!t]
% \caption{Comparative Results with Pre-trained Models.}
% % \vspace{-4mm}
% \footnotesize
% \label{tb:rq2-2}
% \resizebox{.49\textwidth}{!}{
% \begin{tabular}{ccccccccc}
% \hline
%    & \multicolumn{5}{c}{Intention-based   Framework}           &  & \multicolumn{2}{c}{Pre-trained Model} \\ \cline{2-6} \cline{8-9} 
%    & GPT3.5 & GPT4o & DeepSeekV2     & DeepSeek7B & CodeQwen7B &  & CodeReviewer          & T5CR          \\ \hline
% EM & 53.40  & \textbf{64.77} & \major{64.25} & 49.29      & 48.47      &  & 52.13                 & 18.84         \\ \hline
% \end{tabular}
% }
% \vspace{-4mm}
% \end{table}



\begin{table*}[!t]
\caption{\major{Comparative results between our intention-based framework and LLM-based baselines.}}
% \vspace{-4mm}
\footnotesize
\label{tb:rq2-1}
% \vspace{-2mm}
\resizebox{.99\textwidth}{!}{
\begin{tabular}{ccccccccccccccccccccccccc|cc}
\hline
           & \multicolumn{3}{c}{Simple Prompt} &  & \multicolumn{3}{c}{Simple COT} &  & \multicolumn{3}{c}{Tufuno COT} &  & \multicolumn{3}{c}{Random fewshot} &  & \multicolumn{3}{c}{RAG} &  & \multicolumn{3}{c}{Self-generated} &  & \multicolumn{2}{c}{Intention   Framework with RAG} \\ \cline{2-4} \cline{6-8} \cline{10-12} \cline{14-16} \cline{18-20} \cline{22-24} \cline{26-27} 
           & Basic     & P-A       & Comp.     &  & Basic    & P-A      & Comp.    &  & Basic    & P-A      & Comp.    &  & Basic      & P-A       & Comp.     &  & Basic  & P-A    & Comp. &  & Basic      & P-A       & Comp.     &  & Same             & \major{GPT4o Inte.}            \\ \hline
GPT3.5     & 38.00     & 38.29     & 36.42     &  & 34.85    & 36.87    & 32.76    &  & 19.97    & 17.20    & 11.74    &  & 45.25      & 48.62     & 32.39     &  & 46.45  & 49.89  & 33.96 &  & 31.86      & 25.88     & 27.97     &  & 53.40                & \major{\textbf{58.86}}              \\
GPT4o      & 41.14     & 46.60     & 29.54     &  & 39.72    & 43.68    & 26.25    &  & 20.12    & 24.98    & 13.76    &  & 49.21      & 52.88     & 35.00     &  & 49.59  & 54.08  & 36.95 &  & 51.01      & 56.84     & 34.41     &  & 64.77                & \major{\textbf{64.77}}              \\
DeepSeekV2 & 34.63     & 38.07     & 27.60     &  & 41.29    & 46.37    & 31.49    &  & 24.31    & 27.60    & 17.43    &  & 46.45      & 52.66     & 42.41     &  & 48.09  & 55.35  & 45.62 &  & 44.95      & 53.03     & 38.37     &  & 64.25                & \major{\textbf{68.06}}              \\
DeepSeek7B & 28.50     & 34.48     & 24.38     &  & 29.92    & 36.28    & 24.01    &  & 5.76     & 6.24     & 3.60     &  & 33.28      & 39.49     & 25.58     &  & 36.05  & 41.59  & 25.88 &  & 7.26       & 8.68      & 5.01      &  & 49.29                & \major{\textbf{56.40}}              \\
CodeQwen7B & 26.10     & 33.13     & 19.22     &  & 25.06    & 31.04    & 16.98    &  & 2.85     & 3.29     & 1.97     &  & 35.75      & 40.99     & 17.80     &  & 35.83  & 44.05  & 20.12 &  & 13.69      & 19.67     & 7.93      &  & 48.47                & \major{\textbf{54.75}}              \\ \hline
\end{tabular}
}
\vspace{-4mm}
\end{table*}

\subsection{RQ2: Intention-based Framework Effectiveness}\label{sec:rq2}

\major{Table~\ref{tb:rq2-1} presents a comparison between our method and the LLM-based baselines. For the LLM baselines, we made extensive efforts to optimize their performance by setting multiple configurations, including six different prompts and three types of inputs (i.e., basic inputs, position-aware code refinement, and comprehensive code refinement; details are provided in Section~\ref{sec:background}).
For our method, we show the results of the RAG-based approach and the results with other prompts are shown in Table~\ref{tb:rq3-1}. Note that we present two types of results for our method: (1) using the same model for both intention extraction and code refinement (Column \textit{Same}) and (2) extracting accurate intentions with GPT-4o and generating revised code using other LLMs based on these intentions (Column \textit{GPT4o Inte.}).}



% We then compare the performance of other prompt methods based on large models.
From Table~\ref{tb:rq2-1}, we can observe that using the Intention-based method improved the performance of all models compared to all baselines. The model with the highest improvement was DeepSeekV2, which saw an increase from 55.35\% to 64.25\%, a 9 percentage points improvement. The model with the least improvement was GPT3.5, which increased from 49.89\% to 53.40\%, a mere 3 percentage points improvement.  We provide several representative examples to demonstrate the advantages of the Intention-based framework, which are available on our website~\cite{IntentionWebsite}.
% 我们展示了一些有代表性的例子以说明Intention-based framework的优势,相较于其他baseline方法,做到了理解任务再完成任务。限于文章篇幅,我们将例子放到网站中。

\noindent
\major{
\textbf{From the Perspective of Intention Quality}:
Comparing the results of our method using intentions generated by the model itself and those generated by GPT4o, it is evident that intention quality plays a critical role. While using the intentions generated by the same model (Column \textit{Same}) generally improves performance compared to LLM baselines, leveraging higher-quality intentions (Column \textit{GPT4o Inte.}) further enhances the results. This is due to the fact that some models, as shown in Table~\ref{tb:rq1-1}, are not as effective as GPT4o in extracting accurate intentions. Surprisingly, we also observed that DeepSeekV2 outperformed GPT4o (68.06 VS 64.77) when provided with the same intentions (from GPT4o). This demonstrates that, while DeepSeekV2 may not excel in intention extraction, it has superior code refinement capabilities when given accurate intentions. The results underscore the strength of our framework, which allows for separate optimization of the agents responsible for intention understanding and code refinement under a given intent. A similar trend is observed for other prompts, as shown in Table~\ref{tb:rq3-1}.
}

\noindent
\textbf{From the Perspective of Input Complexity:} By comparing the baseline results, we found that across all six prompt strategies, the Position-Aware Code Refinement task performed the best, followed by the Basic Code Refinement task, and lastly the Comprehensive Code Refinement task. Although the Comprehensive Code Refinement task provided the most information, including all necessary data, all models and prompt methods struggled to effectively utilize this information. This may be due to the complexity of understanding \texttt{LastCodeDiffHunk} and the potential for overly long prompts to reduce model focus, impairing comprehension of key information. Incomplete input information reduces model effectiveness, while excessive input information hampers task understanding and reduces effectiveness. The Position-Aware Code Refinement task, which includes \texttt{OriginalCode}, \texttt{ReviewComment}, and \texttt{ReviewLine}, balances the completeness of information with the model's processing capability (5 percentage points better than Basic Code Refinement and about 10 percentage points better than Comprehensive Code Refinement in general). 

% We observed that under prompt strategies like Simple Prompt, Simple COT, Random few-shot, and RAG, the Position-Aware Code Refinement task generally performed about 5\% better than Basic Code Refinement and about 10\% better than Comprehensive Code Refinement.

\noindent
\textbf{From the Perspective of Prompt Strategy:} 
% discrepancies between our implementation and the described methodology in the paper, as we did not have access to the original code. Additionally, the prompt design lacked detailed execution steps, failing to enhance model performance effectively.
Except for the DeepSeekV2 model, other models showed only slight improvements with the Simple COT Prompt method over the Simple Prompt, and even showed declines for GPT3.5 and GPT4o in the Basic task. This indicates that for most models like GPT3.5, GPT4o, DeepSeek7B, and CodeQwen7B, Simple COT Prompt without clear step-by-step guidance does not significantly enhance code refinement tasks. However, for the DeepSeek model, using COT significantly improved accuracy in Basic, Position-Aware, and Comprehensive tasks by up to 8 percentage points, demonstrating DeepSeekV2's strong reasoning capabilities. 
% The Tufano COT method performed significantly worse than others. This may be due to that the prompt design lacks detailed steps, failing to enhance model performance effectively.

Consistent with our intuition, the Random Few-Shot Prompt outperformed the Simple COT Prompt and Simple Prompt. Furthermore, RAG demonstrated superior performance compared to the Random Few-Shot Prompt across almost all models and tasks. \major{This indicates that retrieving more similar examples is highly beneficial, as it provides refinement guidance that aligns closely with both the format and the content of the task.
Therefore, we recommend that users intending to use the RAG prompt for code review tasks construct the retrieval database using a relevant historical code review dataset. Relevance can be assessed from various perspectives, including project similarity, task similarity, author alignment, dataset quality, and fine-grained intention matching. Specifically, retrieved data from the same or similar project, addressing similar tasks, or authored by the same individuals or the same group is likely to offer better guidance for new code refinement tasks, as these examples share meaningful similarities and contextual relevance.}


For Self-generation Prompt, compared to the RAG method, it showed a slight improvement for the GPT4o model, a slight decline for the DeepSeekV2 model, and a significant decline for the other three models. Notably, GPT4o and DeepSeekV2, the two largest models with the best overall performance in other prompt methods, benefited from Self-generation. This suggests that Self-generation Prompt is more effective for models with large parameter sizes and strong reasoning capabilities. 



Last, we compared our method with pre-trained models, specifically CodeReviewer~\cite{li2022automating} and T5CR~\cite{tufano2022using}, which achieved 52.14\% and 18.84\% accuracy, respectively. As expected, our intention-based method demonstrates superior effectiveness, achieving significantly higher performance due to the use of LLMs and the incorporation of intention extraction (e.g., 64.77\% with GPT4o).


% We also compared our method with that of the pre-trained models, i.e., CodeReviewer and T5CR, which achieve 52.14\% and 18..84\%, respectively. It is expected that our intention-based method is more effective due to the use of LLMs and the intention extraction. (e.g., 64.77\% with GPT4o).

% It is expected that 
% According to Table~\ref{tb:rq2-2}, we observe that all five models perform better with the Intention-based framework than with the pre-trained models. Among them, the GPT4o model achieves the best performance, reaching 64.77\%, which is 12.64 percentage points higher than the CodeReviewer model (52.13\%). 

\vspace{5pt}\noindent \fbox{
\parbox{0.95\linewidth}{\textbf{Answers to RQ2}: The results show that the Intention-based framework outperforms the existing baselines, including both pre-trained models and LLMs with diverse prompts. 
\major{High-quality intentions play a critical role in achieving effective code refinement.}
% Specifically, for the DeepSeekV2 model, the framework achieves an Exact Match (EM) accuracy of 68\%, which is 13\% higher than the baseline. 
% Additionally, we observed that among these baseline methods, using large models for the code refinement task yields the best performance for the Position-Aware Code Refinement task, and the most effective prompt strategy is RAG.
}
}


\begin{table}[!t]
\centering
\caption{\major{Results of our method with different prompts.}}
\footnotesize
\label{tb:rq3-1}
\resizebox{.49\textwidth}{!}{
\begin{tabular}{cccclccc}
\hline
           & \multicolumn{3}{c}{Our Method (Same)}                    &  & \multicolumn{3}{c}{\major{Our Method (GPT4o   Inte.)}} \\ \cline{2-4} \cline{6-8} 
           & Simple Prompt  & RAG            & Self-generated &  & \major{Simple Prompt}  & \major{RAG}             & \major{Self-generated} \\ \hline
GPT3.5     & 53.03          & \textbf{53.40} & 43.38          &  & \major{57.14}          & \major{\textbf{58.86}}  & \major{46.60}          \\
GPT4o      & 64.10          & 64.77          & \textbf{65.97} &  & \major{64.10}          & \major{64.77}           & \major{\textbf{65.97}} \\
DeepSeekV2 & 60.88          & \textbf{64.25} & 61.03          &  & \major{61.03}          & \major{\textbf{68.06}}  & \major{64.10}          \\
DeepSeek7B & \textbf{50.04} & 49.29          & 26.55          &  & \major{54.45}          & \major{\textbf{56.40}}  & \major{27.75}          \\
CodeQwen7B & 46.00          & \textbf{48.47} & 29.32          &  & \major{50.11}          & \major{\textbf{54.75}}  & \major{27.45}          \\ \hline
\end{tabular}
}
\vspace{-4mm}
\end{table}

% 从表2中,我们首先可以看到,所有模型,使用Intention-based方法对比所有baseline均有提升。提升最高的模型是deepseek,从55.35%提升到了68.06%提升了13%。提升最少的模型是GPT3.5,从49.89%提升到53.03%只提升了3%。

% 我们也从baseline的结果,得到更多的结论。

% 从任务角度观察:我们通过对比baseline的结果发现:在所有6种不同的策略下,Position-Aware code Refinement任务都是最优的,其次是Basic code Refinement,最后是comprehensive code Refinement。虽然comprehensive code Refinement任务提供了最多的信息,提供了全部的数据所需要的信息,但是所有模型,所有prompt方法都不能很好的理解这些信息。这一方面可能是由于code diff是一种难以理解的结构,另一方面可能是因为过长的prompt会降低模型的注意力,反而降低了对关键有效信息的理解。
% 输入信息不全会降低模型的效果;输入信息太多会影响模型对任务的理解,也会导致降低模型的效果。Position-Aware code Refinement任务输入是original code,review comment和review line信息,很好的平衡了信息完整程度和模型对信息的处理能力。我们观察到simple prompt,simple cot,Random fewshot,RAG等prompt策略之下,大部分情况Position-Aware code Refinement任务效果比Bacis code Refinement的效果高5%左右,比comprehensive code Refinement效果高10%左右。

% 从prompt策略角度观察:Tufuno COT的效果远低于其他,这一方面是因为我们是根据她论文描述而写的,没有找到她的代码实现,可能跟他的实现有出入。另一方面是因为prompt设计时也没有详细执行步骤,未能有效提高模型效果。

% 另外可以观察到,除了deepseek模型以外,其他模型在simple COT方法的效果只是略优于simple prompt,甚至对于Basic任务中,GPT3.5和GPT4o的效果还有下降。这说明对于GPT3.5,GPT4o,deepseekcode,code-qwen等模型,没有明确步骤指导的COT对code refinement任务提升不大。而对于deepseek模型,在Basic,P-Aware,Comprehensive等三个任务上,使用COT可以显著提升准确率,最多8%。这说明deepseek具有较强的推理能力。

% 与我们的直觉一致的,random fewshot的效果好于simple cot和simple prompt。而RAG的效果也好于random fewshot。这对几乎对所有模型和所有任务都成立。

% 最后,我们尝试了self-generation方法。相比于RAG方法,我们发现只有对GPT4o模型效果有小幅提升,deepseek模型小幅下降,其他三个模型效果都大幅下降。恰巧GPT4o和deepseek是参数量最大的两个模型,也是在其他prompt方法综合效果最好的两个模型。这说明只有对参数规模较大,自身推理能力较强的模型,self-generation方法才能充分发挥效果。

% 总之,我们在5个模型,3种类型的code Refinement任务上充分比较了现有的6种prompt方法的效果。发现对模型的输入信息不能过多也不能过少,要平衡信息完整性和模型理解能力,效果最好的任务是Position-Aware。还发现了对于一般的模型RAG是最优的prompt策略,而对于参数量大推理能力强的模型,self-generation也是很好的prompt策略。而我们提出的Intention框架,充分利用了所有信息,效果比目前所有的prompt策略都要高3%-13%。

% % 设置表格中所有文本颜色为蓝色
% \begingroup
% \color{blue}  % 将以下内容颜色设置为蓝色
\begin{table*}[!t]
\caption{\major{Ablation results on different intention-based components.}}
\footnotesize
\label{tb:rq3-3}
\resizebox{.99\textwidth}{!}{
\major{
\begin{tabular}{cccccccccccccccc}
\hline
           & \multicolumn{3}{c}{Our Method (with Intention)}         &  & \multicolumn{3}{c}{w/o Explicit  Intention} &  & \multicolumn{3}{c}{w/o Reversion   Intention} &  & \multicolumn{3}{c}{w/o General Intention} \\ \cline{2-4} \cline{6-8} \cline{10-12} \cline{14-16} 
           & Simple Prompt & RAG            & Self-generated &  & Simple Prompt    & RAG             & Self-generated   &  & Simple Prompt    & RAG             & Self-generated    &  & Simple Prompt         & RAG                 & Self-generated        \\ \hline
GPT3.5     & 57.14         & \textbf{58.86} & 46.60          &  & 56.25(-0.9)      & 56.39(-2.47)    & 42.63(-3.96)     &  & 49.59(-7.55)     & 51.6(-7.26)     & 35.53(-11.07)     &  & 51.53(-5.61)          & 58.41(-0.45)        & 44.05(-2.54)          \\
GPT4o      & 64.10         & 64.77          & \textbf{65.97} &  & 64.03(-0.07)     & 63.65(-1.12)    & 65.45(-0.52)     &  & 59.61(-4.49)     & 60.21(-4.56)    & 61.41(-4.56)      &  & 57.37(-6.73)          & 62.3(-2.47)         & 65.75(-0.22)          \\
DeepSeekV2 & 61.03         & \textbf{68.06} & 64.10          &  & 60.58(-0.45)     & 65.52(-2.54)    & 61.41(-2.69)     &  & 55.27(-5.76)     & 61.85(-6.21)    & 56.32(-7.78)      &  & 52.35(-8.68)          & 63.95(-4.11)        & 63.05(-1.05)          \\
DeepSeek7B & 54.45         & \textbf{56.40} & 27.75          &  & 53.18(-1.27)     & 53.48(-2.92)    & 22.59(-5.16)     &  & 45.85(-8.6)      & 44.2(-12.19)    & 8.97(-18.77)      &  & 48.62(-5.83)          & 52.51(-3.89)        & 29.39(+1.65)          \\
CodeQwen7B & 50.11         & \textbf{54.75} & 27.45          &  & 46.97(-3.14)     & 52.81(-1.94)    & 19.75(-7.7)      &  & 38.0(-12.12)     & 46.6(-8.15)     & 6.96(-20.49)      &  & 47.8(-2.32)           & 54.75(0.00)          & 39.72(+12.27)         \\ \hline
\end{tabular}
}
}
\end{table*}
% \endgroup


\subsection{RQ3: Ablation Study on Different Intention Categories}

\major{Table~\ref{tb:rq3-3} presents the ablation study results. The first column shows the performance with all intention-handling components included, while the subsequent columns illustrate the effects of removing intention-handling for each category (i.e., removing each agent in Fig.~\ref{fig:framework}). For example, removing the Explicit Intention component means that code reviews with explicit suggestions are instead handled by Agent 2 and Agent 3. Similarly, removing the Reversion Intention redirects code reviews with reversion suggestions to Agent 1 and Agent 3. For Agent 3, the general intention extraction is removed, and LLMs are used to handle these cases directly.
The results indicate that removing any of these intention components leads to performance drops across almost all scenarios, demonstrating the importance of each component. Notably, removing the Reversion Intention results in significant declines in EM scores across all models and prompt methods, with reductions ranging from 5 to 15 percentage points, highlighting its critical role in improving this category of code reviews.
}

\major{An exception is observed with the Self-generated prompt under the condition of ``w/o General Intention'', where performance increases slightly. This is likely because Self-generated prompts tend to perform worse with LLMs that have weaker reasoning capabilities, as discussed in RQ2. However, the Simple Prompt shows a significant drop, indicating that the intention-based method provides substantial improvements when using very basic prompts. In contrast, RAG-based and Self-generated prompts exhibit relatively smaller performance declines. These advanced prompts compensate for the models' limitations in implicitly understanding intentions, even when intentions are not explicitly provided.}

% We conducted ablation experiments on the three agents used in the framework, and the results are presented in Table 6. First, analyzing the impact of Agent1 on the Exact Match (EM) scores, we observe that removing Agent1 causes a decline in the EM values for all models and all prompt methods. Notably, weaker models such as GPT-3.5, CodeQwen7B, and DeepSeek7B exhibit a significant drop of approximately 6 percentage points, whereas stronger models like GPT-4o and DeepSeekV2 experience a smaller decrease of 1-3 percentage points. 
% Next, examining the impact of Agent2, we find that its removal leads to a substantial decline in the EM scores across all models and prompt methods, with reductions ranging from 5 to 15 percentage points. 
% Finally, considering the influence of Agent3, we observe that it provides a notable improvement for stronger models when using the Simple Prompt approach (e.g., GPT-4o and DeepSeekV2 improve by 6.8 and 8.5 percentage points, respectively) and a slight improvement when using the RAG method (e.g., GPT-4o and DeepSeekV2 improve by 2.4 and 0.3 percentage points, respectively). However, for weaker models, Agent3 only shows a slight benefit for Simple Prompt (e.g., GPT-3.5 and DeepSeek7B), while for the RAG method, its influence is negligible or even detrimental in most cases. This outcome can be attributed to Agent3's ability to extract intentions that help the models better understand the reviewers' goals under the Simple Prompt setting. In contrast, the RAG method compensates for the models' inherent deficiencies in interpreting intentions, reducing the added value of Agent3. 
% Further analysis reveals that weaker models occasionally produce incorrect intention extraction in Agent3. When replacing the extracted intentions with those generated by GPT-4o, all models' performance improves, especially for the RAG method, where all models see an improvement of more than 4 percentage points.
% These findings highlight that Agent1 and Agent2 significantly enhance the results, while Agent3 offers improvements primarily for stronger models. However, for weaker models, particularly under the RAG method, Agent3 provides limited or no benefit.}


\begin{table*}[!t]
\caption{\major{Improvement results with different intention-based components.}}
\footnotesize
\label{tb:rq3-2}
\resizebox{.99\textwidth}{!}{
\begin{tabular}{cccccccccccc}
\hline
           & \multicolumn{3}{c}{Simple Prompt}           &  & \multicolumn{3}{c}{RAG}                     &  & \multicolumn{3}{c}{Self-generated}          \\ \cline{2-4} \cline{6-8} \cline{10-12} 
           & Explicit & Reversion & General (\major{Same/GPT4o}) &  & Explicit & Reversion & General (\major{Same/GPT4o}) &  & Explicit & Reversion & General (\major{Same/GPT4o}) \\ \hline
GPT3.5     & 29.71    & 62.01     & 2.86/\major{12.78}           &  & 8.57     & 57.47     & -9.57/\major{2.64}           &  & 38.29    & 70.13     & -1.29/\major{5.73}           \\
GPT4o      & 6.86     & 74.35     & 13.22/\major{13.22}          &  & 2.29     & 68.83     & 4.85/\major{4.85}            &  & 4.00     & 71.10      & 0.44/\major{0.44}            \\
DeepSeekV2 & 21.14    & 76.95     & 16.74/\major{16.45}          &  & 7.43     & 67.53     & 0.59/\major{7.49}            &  & 13.14    & 67.53     & -3.96/\major{1.47}           \\
DeepSeek7B & 25.71    & 65.91     & 2.71/\major{11.75}           &  & 14.29    & 60.71     & -6.14/\major{8.52}           &  & 38.86    & 73.38     & -5.43/\major{-3.23}          \\
CodeQwen7B & 29.71    & 67.21     & -3.43/\major{6.75}           &  & 24.00    & 57.14     & -12.00/\major{0.00}          &  & 50.29    & 68.51     & -19.86/\major{-22.91}        \\ \hline
\end{tabular}
}
\vspace{-4mm}
% \vspace{-4mm}
\end{table*}

% We experimented with different models under the Intention framework using three distinct prompt strategies: Simple Prompt, RAG, and Self-generated Prompt. The results are presented in Table~\ref{tb:rq3-1}. We can observe that DeepSeek7B achieved the best results with the Simple Prompt strategy, GPT4o achieved the best results with the Self-generated Prompt strategy, and the other three models achieved high accuracy with the RAG Prompt strategy.

% Comparing the Self-generated Prompt with the RAG Prompt, the results align with our previous analysis in RQ2: the larger the model's parameters and the stronger its reasoning capabilities, the better it performs with the Self-generated Prompt. For example, GPT4o and DeepSeekV2 achieved 65.97\% and 64.10\% accuracy, respectively. Conversely, other models perform much worse with the Self-generated Prompt compared to other prompt strategies.

% Notably, even with the Simple Prompt method, all models still achieved very high accuracy. Compared to the Simple Prompt result in Table~\ref{tb:rq2-1}, the accuracy of the five models increased by 15 to 23 percentage points. This indicates that the Intention framework makes it easier for models to understand reviewer's requirements.

\major{To further understand the improvements that are caused by our three intention-based components, we calculate the EM improvements on the test samples that are handled by each component, i.e., the EM difference compared to the best performance of the LLM baselines (see Table~\ref{tb:rq2-1}). Note that, due to that the intention quality is important, similar to RQ2, we show both results using intentions extracted by the same model and GPT4o. 
% The results are shown in Table~\ref{tb:rq3-2}.
}

% Next, we compare the improvement of each category using the Intention-based framework with Simple Prompt, RAG, and Self-Generated Prompt strategies against the ordinary tasks. For better demonstration of the Intention-based framework's effect, we chose the Position-Aware task, which performed the best in RQ2, as the control group.

As shown in Table~\ref{tb:rq3-2}, the Intention-based framework demonstrates improvements in most cases. Notably, the most significant improvement is observed in the Reversion Suggestion category, with accuracy increases ranging from 57 to 76 percentage
points. This substantial gain is attributed to the rule-based refinement strategy employed for Reversion Suggestions, which ensures correct refinement when the given sample is accurately classified into this category during the intention analysis.

For the Explicit Code Suggestion category, our framework also shows improvement. For relatively weaker models like GPT3.5, DeepSeek7B, and CodeQwen7B, the improvement is particularly notable, often exceeding 20 percentage points, with a maximum increase of 50 percentage points. For stronger models like GPT4o and DeepSeekV2, it still provides slight improvements. 
\major{This is because explicit code suggestions are relatively easier for stronger models to handle, as the explicit intentions are more straightforward for them to extract. However, weaker models struggle to effectively extract these intentions and refine code, resulting in more significant improvements when using our method.}
% This is due to the repair rules used in the framework, which can correct errors generated by the model.
\major{For tasks in the General Suggestion category, we observed that our method improves results when high-quality intention extraction (e.g., using GPT4o) is employed. However, when using the same model for both intention extraction and code refinement, the results decrease in some cases. This is because weaker models may extract incorrect intentions, which not only fail to assist the refinement process but can also lead to the generation of incorrect code.}

\major{Additionally, with the Simple Prompt strategy, the overall improvement is notable since the original model performs poorly, leaving substantial room for improvement. On the other hand, for RAG-based and Self-generated prompts, the overall performance is already enhanced, so the use of low-quality intentions (extracted by the same model) can more easily degrade performance.}

\major{
Furthermore, we found that the improvement in the General Suggestion category (even with high-quality intentions) is lower compared to Explicit Suggestion and Reversion Suggestion categories. This is because intentions in General Suggestions remain relatively high-level and less concrete, making it more challenging for the model to execute precise refinements compared to the other two categories.}

% the framework provides slight improvements for the stronger models, GPT4o and DeepSeekV2. Specifically, for the RAG prompt method, the improvements are 4.85 and 7.49 percentage points, respectively. This could be attributed to the repair strategy within the framework or more effective indexing after intention extraction, leading to more relevant task retrieval. However, for the three weaker models, GPT3.5, DeepSeek7B, and CodeQwen7B, the framework shows almost no improvement for these tasks, and in some cases, the performance decreases. Particularly with the RAG method, the performance drops by 6 to 12 percentage points. This might be because weaker models extract less accurate intentions, and using these intentions for indexing can lead to retrieving irrelevant examples, thereby interfering with the results.





\vspace{5pt}\noindent \fbox{
\parbox{0.95\linewidth}{\textbf{Answers to RQ3}: \major{The results demonstrate that all three components contribute to performance improvement. However, their effectiveness varies. The Reversion Intention component achieves the largest improvement (over 50 percentage points) due to its rule-based code refinement, which ensures high accuracy. In contrast, the General Intention component shows relatively less improvement, as its intentions are less concrete compared to Explicit and Reversion Intentions, introducing larger uncertainty into the refinement process. Moreover, performance can decrease if the General Intention is not accurately extracted.}
}
}

% The results show that using the Intention-based framework improves performance for all models. 对Reversion Suggestion类型的任务提升50%以上。对Explicit Code Suggestion类型任务提升20%以上。如果使用比较强的模型,如GPT4o和DeepSeekV2,对General Suggestion tasks也有小幅提升4.85\% and 7.49\%。
% Strong models show improvements across Explicit Code Suggestion, Reversion Suggestion, and General Suggestion tasks. Weaker models only show improvements in the first two tasks, with performance decreasing in the third task. 


% 我们尝试了不同模型在Intention框架下,使用simple prompt,rag,self-generation三种不同的prompt策略。结果如表3所示,我们可以观察到,code-deepseek在simple prompt策略下取得最好结果。GPT4o在self-generation下取得最好结果,其余三个模型都在RAG的prompt策略下取得高的准确率。

% 对比self-generation方法与RAG方法,与之前在RQ2中分析的结果一致,模型参数量越大推理能力越强,在self-generation方法的效果越好,例如GPT4o和deepseek效果分别为65.97%和64.10%。反之,一般的模型在self-generation上的效果相较于其他prompt策略会落后很多。

% 另外值得注意的是,即便只是使用最简单的simple prompt方法,所有模型仍可取得非常高的准确率。相较于表2中的simple prompt方法,五个模型分别提升了15%-23%的正确率。这说明Intention的框架使模型更加容易理解任务的需求。

% 而后我们再比较一下相较于普通任务中的simple prompt,RAG,self-generation策略,使用Intention框架后对每一类别的提升。为了更好的展示Intention的效果,对照组我们选择了在RQ2中表现最好的Position-Aware任务。

% 如表4所示,几乎大部分情况,使用Intention框架都有提升。特别的,对于Reversion Suggestion任务提升最为明显,均提升57%-76%之间。这是因为Intention框架中对Reversion Suggestion的修复策略是基于规则的,如果能在extraction Intention过程中正确分类,识别出任务属于这一类Intention,那么后续修复就保证修复正确。
% 其次,Explicit code Suggestion类别,使用Intention框架也均有提升。对于相对弱一些的模型,如GPT3.5,deepseekcoder,code-qwen,提升尤为明显,大部分情况可以提升20%以上,最高提升有50%。而对于能力较强GPT4o和deepseek,Intention框架也可以小幅提升。这是因为Intention框架中使用了repair规则,可以修复模型生成中的错误。
% 最后,对于general的任务,我们发现对于能力较强的模型GPT4o和deepseek,Intention框架仍可以进行小幅提升。特别的,对于效果最好的RAG prompt方法,提升分别为4.85%和7.49%。这可能要归功于Intention框架中的repair策略,或者是Intention提取后使得索引更有效,可以索引到更相近的任务。然而,对于能力较弱的3个模型,GPT35,deepseekcoder,codeqwen,general任务中Intention方法就几乎没有提升,反而效果会下降,特别是RAG方法上,效果下降最明显,下降了6%-12%。这可能是因为弱模型本身生成Intention的正确率就不高,再继续用Intention做索引会索引到不相关的例子,反而干扰了结果。

% 总之,无论是能力强的模型还是能力弱的模型使用Intention框架效果都会有提升。能力强的模型在Explicit code Suggestion,Reversion Suggestion,general Suggestion三个任务上都有提升。而能力弱的模型只在前两个任务上有提升,第三个任务上会有下降。

\subsection{RQ4: Intention-Based Dataset Cleaning}



\begin{table}[!t]
\centering
\caption{Effectiveness on data cleansing.}
\scriptsize
\label{tb:rq4}
\begin{tabular}{ccccccc}
\hline
          & TP   & FP  & TN  & FN  & Accuracy & Precision \\ \hline
Intention-based & 1076 & 110 & 553 & 261 & 81.45\%  & 90.73\%   \\
Comment-based    & 1201 & 312 & 351 & 136 & 77.60\%  & 79.38\%   \\ \hline
\end{tabular}
\vspace{-4mm}
\end{table}

We further investigated the role of intention in data cleaning. As shown in Table~\ref{tb:rq4}, compared to directly using the review comment to verify whether the code modification meets the reviewer’s requirements, using Intention to verify the code modification proved to be more effective. The accuracy increased from 77.60\% to 81.45\%, and the precision increased from 79.38\% to 90.73\%.

% The improvement in True Positives (TP) and the reduction in False Positives (FP) also highlights the effectiveness of the Intention-based approach. Specifically, the True Positives for the Intention-based method are 1076, compared to 1201 for the Comment-based method, showing a more conservative approach that reduces unnecessary inclusions. The False Positives dropped significantly from 312 to 110, indicating a more precise identification of valid code modifications. Similarly, True Negatives (TN) rose from 351 to 553, and False Negatives (FN) increased slightly from 136 to 261, showing that while some valid changes were missed, the overall precision improved.

Such improvements in accuracy and precision underscore the significance of utilizing intention as a guiding principle for code verification. Since dataset construction places a higher emphasis on data quality, the 12\% improvement in precision is significant for enhancing data quality. This increase ensures that the cleansed data is not only more accurate but also more reliable for downstream applications and analysis.

Overall, the intention-based approach demonstrates a more balanced and effective methodology for ensuring that code modifications align closely with the original reviewer's intentions, resulting in cleaner and more precise datasets. This shift toward a more intention-driven process marks a substantial advancement in the field of data cleansing, providing developers and data scientists with a more robust tool for maintaining code integrity and quality.

\vspace{5pt}\noindent \fbox{
\parbox{0.95\linewidth}{\textbf{Answers to RQ4}: The results indicate that intention-based cleansing is more effective for code refinement data cleansing than comment-based methods, achieving an accuracy of 81\% and a precision of 91\%.
}
}

% 我们进一步研究Intention在筛选数据中的作用。结果如表5所示,相较于直接用review comment检验代码修改是否符合要求,使用Intention来检验代码修改是否符合要求的效果更好。准确率从78%提升到了81%,而精确率从79%提升到了91%。因为构造数据集更在意数据质量,精确率提升的12%对数据质量的提升意义很重要。
\section{DISCUSSION}
\subsection{Discussions on Intention Classification and Extraction}


\major{Our framework divides the code refinement task into two steps: \textbf{intention analysis} and \textbf{intention-guided refinement}. This separation allows for improvements in effectiveness by enhancing both steps individually. Ideally, we aim to extract intentions as concretely as possible, such as Explicit and Reversion Intentions, which simplify the following refinement process. However, when the intention is less concrete (e.g., for General Suggestions), the refinement process involves more understanding difficulties, making performance improvements less significant.}

\major{
Our results also indicate that inaccurate intention extraction can degrade performance compared to an end-to-end refinement approach (see RQ3). This highlights why we only extract high-level intentions for General Suggestions, as it is challenging to ensure fully accurate and concrete intention extraction in these cases. Extracting incorrect intentions can lead to misguided refinements, which we aim to avoid.
}

\major{
In the future, refining the categories of General Intentions could further enhance the refinement process. For example, if the categories are more concrete, a rule-based method can be easily designed, or weaker models may better understand and refine the code. An ideal scenario would be that we have a complete classification of intention categories, where each category is sufficiently concrete to allow the use of reliable rule-based methods or even very weak models for effective refinement.
Explicit Suggestions and Reversion Suggestions are prime examples of such concrete categories.}



\subsection{Threats to Validity}
\noindent \textit{Model Threats:} We only tested the 7B versions of open-source code models, DeepSeek7B and CodeQwen7B, due to the resource limit. However, based on the performance of general open-source models like GPT4o and DeepSeekV2, our Intention-based framework performs excellently in code refinement tasks compared to other prompt techniques.
\major{Another potential threat is the length of the prompt templates. While longer prompts can pose input challenges for some models, the longest prompt template used in our study (the RAG prompt) contains only 141 tokens. The fields for each case (e.g., \texttt{OriginalCode}, \texttt{ReviewLine}, \texttt{Intention}) typically remain below 200 tokens. Even when including three-shot examples, the total input length remains under 1,000 tokens. Therefore, the prompt length is unlikely to affect the validity of our results.}

 \noindent \textit{Data Threats:} We only selected CodeReviewer dataset because other datasets lack some data fields and do not provide the link to the original data, making it impossible to use the Intention framework. However, the CodeReviewer dataset has been used in many papers and is recognized as a relatively complete and objective dataset. In the future, we will also try to collect more comprehensive and higher-quality datasets.
\major{We also acknowledge the potential risk of data leakage, particularly when using LLMs. While it is challenging to entirely rule out the possibility of data leakage within LLMs, our experimental results demonstrate that utilizing the Intention framework consistently yields better outcomes compared to LLMs not using it. This indicates that even in scenarios where data leakage may occur, the framework's design and methodology provide a significant performance advantage.}

 \noindent \textit{Efficiency Threats:} Our framework involves multiple LLM calls for classification and code generation, potentially making it slightly less efficient than other prompt strategies. However, the model’s classification response speed is relatively fast, and for Explicit Code Suggestions and Reversion Suggestions, we only need one step LLM call. Therefore, the overall impact on efficiency is not significant.

% 模型的THREATS:我们只测试了7B版本的开源代码模型,deepseekcoder和code-qwen,对于更大参数量的代码模型是否会有其他性质,我们没有计算资源去测试。不过从开源通用模型的效果来看,如gpt4o和deepseek,我们的Intention框架相较于其他prompt技术可以很出色的完成code Refinement任务

% 数据的THREATS:我只选择了一个测试数据集,这是因为其他数据集没有提供原始数据难以完成数据补全,也就没法使用Intention框架的。不过这个测试集已经在很多论文中使用,是公认的比较完整客观的数据集。以后我们也会尝试收集更全面,质量更好的数据集。

% 效率的THREATS:因为我们框架调用了多次LLM进行分类和生成代码,整体效率可能会略低于其他prompt策略。不过用模型分类的响应速度比较快,而且对于Explicit code Suggestion,和Reversion Suggestion我们只需要通过一次LLM。所以整体效率影响不是很大。
\section{related work}
Code refinement is a critical core component in the code review process, and numerous scholars have conducted research on the automation of code refinement. Tufano M.~\cite{tufano2019learning} were the pioneers in proposing the use of Neural Machine Translation (NMT) to learn the automated modification of Java methods based on review comments. Subsequently, Tufano R.~\cite{tufano2021towards} and Thongtanunam~\cite{thongtanunam2022autotransform} employed transformer~\cite{vaswani2017attention} models to train and enhance the original task’s performance. 
Tufano R.~\cite{tufano2022using} and Li~\cite{li2022automating} further advanced this field by using the Text-To-Text Transfer Transformer (T5) model~\cite{raffel2020exploring} and CodeT5 model~\cite{wang2021codet5}, pre-training code review-related tasks to enable the model to comprehend the meaning of the code and review comments. These approaches yielded significant improvements in downstream code refinement tasks.

% Compared to pre-trained models, large models exhibit significant advantages in understanding instructions and generating code. 
With the rise of LLMs, many researchers have attempted to leverage them in software engineering \cite{ma2024specgen, ma2024speceval, kong2024contrastrepair, guo2024ft2ra, xia2023keep}. In particular, Guo~\cite{guo2024exploring} explored using ChatGPT for code refinement tasks, uncovering some prompt design techniques.
% as well as the strengths and weaknesses of Large Language Models (LLMs) in this context. 
Tufano R.~\cite{tufano2024code} manually analyzed over 2,000 code refinement examples, evaluating three code review models~\cite{hong2022commentfinder, li2022automating, tufano2022using} and comparing their performance to ChatGPT. This analysis revealed that ChatGPT is highly competitive compared to previous methods. 
Pornprasit~\cite{pornprasit2024fine} experimented with various prompt strategies for LLMs in code refinement. They also fine-tuned ChatGPT using an API~\cite{ChatGPTblog}, enhancing the effectiveness of LLMs in this task.
% Pornprasit~\cite{pornprasit2024fine} experimented with various prompt strategies for LLM in code refinement, including zero-shot, few-shot, and persona-based approaches. Additionally, Pornprasit fine-tuned ChatGPT using an API~\cite{ChatGPTblog}, which further improved the effectiveness of LLMs in this task.

Some studies have also involved classifying code refinement tasks.
Tufano~\cite{tufano2024code} categorizes code refinement based on the type of task and examines the performance of pre-trained models on different task types. 
Kononenko~\cite{kononenko2016code} studied the time and effort required by programmers for different types of code review tasks. Bacchelli~\cite{bacchelli2013expectations} investigated the categories of code review tasks, focusing on developer motivation and response speed. Pascarella~\cite{pascarella2018information} explored the information needed for different types of code refinement tasks, but their classification method is more oriented toward human understanding rather than guiding model modifications.

% 另外,将code Refinement分类,也有很多研究工作。Tufano是从任务的类型角度来分类,查看pre-trained模型在不同任务类型上的效果。Kononenko[kononenko2016code]研究了不同类别的code review任务所消耗程序员的时间和精力。Bacchelli[bacchelli2013expectations]调研了不同code review任务类别,开发者的motivation和响应速度。Pascarella[pascarella2018information]研究了不同类别的code Refinement任务的所需信息,不过他们的分类方式更偏重于让人类理解,而不是指导模型进行修改。


% Code Refinement自动化任务:code Refinement任务作为code review流程中的关键核心任务,有很多学者进行过code Refinement自动化相关研究。最早由Tufano M.等人[tufano2019learning]提出使用NMT去学习根据review comment自动化的修改java method。而后Tufano R.等人[tufano2021towards],Thongtanunam等人[thongtanunam2022autotransform]分别使用transformer模型训练并提升了原任务的效果。Tufano R.[tufano2022using]进一步使用Text-To-Text Transfer Transformer (T5)模型[raffel2020exploring],allowing the model to work with raw source code by keeping under control the vocabulary size, 解决了之前工作需要将变量名做化简代替的问题。而后,Li等人[li2022automating]首先使用了pre-trained模型,设计了 Diff Tag Prediction,Denoising Objective,Review Comment Generation等三个预训练任务,在CodeT5模型[wang2021codet5]的基础上预训练code review相关任务,让模型理解代码的含义,以及代码和review comment的对应关系。并在下游code Refinement任务上取得了很好的效果。

% 大模型for code Refinement任务:相较于预训练模型,大模型在理解指令,和生成代码方面具明显优势。随着大模型的兴起,很多研究者也尝试用大模型解决code Refinement任务。Guo[guo2024exploring]尝试用ChatGPT处理code Refinement任务,发现了一些prompt设计技巧,以及LLM在此任务上的优势和不足。Tufano R.[tufano2024code]手工分析了2000多个code Refinement例子,评估了三种code review模型[hong2022commentfinder, li2022automating, tufano2022using]并与ChatGPT的效果对比,发现ChatGPT相较于之前的方法具有很强的竞争力,并且通过引导模型先去思考问题类型,再做修改的COT方法,可以进一步提升修复效果。Pornprasit[pornprasit2024fine]尝试了大模型解决code Refinement问题时的多种prompt策略,包括zeroshot,fewshot和Persona,并且用API的方式finetune了ChatGPT,可以进一步提升大模型的效果。
\section{Conclusion}



This paper introduces \sysname, an AI-assisted system designed to enhance the process of visual blend ideation by leveraging metaphors. 
%Our system utilizes large language models and commonsense knowledge bases to explore objects and their associated attributes, forming metaphorical connections with abstract concepts.
Our system utilizes LLMs and commonsense knowledge bases to explore objects and their associated attributes, forming metaphorical connections with abstract concepts. 
It offers the capability to automatically generate blending proposals based on user selections, facilitating rapid creative realization for verification through the T2I model.
To evaluate the system, we conducted a user study involving 24 participants who had AI experience. The findings demonstrate that \sysname\ has the potential to enhance the creativity of the generated ideation results and enable the expression of abstract concepts more metaphorically.
Additionally, this research offers insights into user preferences regarding visual blend design and potential future approaches for supporting design with generative AI.



\section*{Acknowledgment}
This work was partially supported by the National Natural Science Foundation of China (Key Program, Grant No. 62332005), the National Research Foundation, Singapore, and the Cyber Security Agency under its National Cybersecurity R\&D Programme (NCRP25-P04-TAICeN). Lei Bu is supported in part by the Leading-edge Technology Program of Jiangsu Natural Science Foundation (No. BK20202001), the National Natural Science Foundation of China (No. 62232008, 62172200). Any opinions, findings and conclusions or recommendations expressed in this material are those of the author(s) and do not reflect the views of National Research Foundation, Singapore and Cyber Security Agency of Singapore.


%%
%% This command processes the author and affiliation and title
%% information and builds the first part of the formatted document.

% 构建一个benchmark,参考https://arxiv.org/pdf/2404.00599.pdf
% 新内容包括:问题的行号信息,更全的codediff,完整的project信息
% 新数据集保证了:更全的program language,随时更新的数据集(防止data leak)
% 新metric:EM-trim,BLEU-trim,编辑距离(David Lo文章)
% LLM测试包括:GPT3.5, GPT4, DeepSeek Code, StarCoder2, CodeLLaMa, Gemma, Qwen1.5
% 2个benchmark,micro, tufuno的





% \bibliographystyle{ACM-Reference-Format}
% \bibliography{software}
\bibliographystyle{IEEEtran}
\bibliography{software}

\end{document}
\endinput
%%
%% End of file `sample-sigconf-authordraft.tex'.
