\begin{abstract}
% Large language models are exposed to privacy risks since they are trained on large text corpus, which may include sensitive or private information.
Various studies have attempted to remove sensitive or private knowledge from a language model to prevent its unauthorized exposure.
However, prior studies have overlooked the complex and interconnected nature of knowledge, where related knowledge must be carefully examined.
Specifically, they have failed to evaluate whether an unlearning method faithfully erases interconnected knowledge that should be removed, retaining knowledge that appears relevant but exists in a completely different context.
To resolve this problem, we first define a new concept called \textit{\textbf{superficial unlearning,}} which refers to the phenomenon where an unlearning method either fails to erase the interconnected knowledge it should remove or unintentionally erases irrelevant knowledge.
Based on the definition, we introduce a new benchmark, \textbf{\ourdata}, to analyze and evaluate the faithfulness of unlearning in real-world knowledge QA settings.
Furthermore, we propose a novel unlearning method, \textbf{\ourmodel}, which updates only knowledge-related neurons to achieve faithful unlearning.
\ourmodel~identifies knowledge neurons using an explainability method and updates only those neurons using selected unforgotten samples. 
Experimental results demonstrate that widely-used unlearning methods fail to ensure faithful unlearning, while our method shows significant effectiveness in real-world QA unlearning.
\end{abstract}
