\section*{Method}

We propose a comprehensive verifier-based evaluation method that assesses the correctness of a solution \( \hat{Y} \) by evaluating both individual reasoning steps and the overall reasoning structure. This involves combining step-level scores with reasoning-level scores to account for planning errors and missing steps.

\subsection*{Step-Level Evaluation}

1. \textbf{Premise Identification}:  
   For each reasoning step \( S_t \), we identify its premises \( \{P_{t,1}, P_{t,2}, \ldots, P_{t,m}\} \) using a pre-trained language model (PLM). The PLM processes the entire reasoning chain up to \( S_t \) to determine which previous steps or external information serve as premises for \( S_t \):
   \[
   \{P_{t,1}, P_{t,2}, \ldots, P_{t,m}\} = \mathrm{PLM}(X, \{S_1, S_2, \ldots, S_{t-1}\}, S_t)
   \]

2. \textbf{Error Detection and Scoring}:  
   The verifier \( V \) evaluates each reasoning step \( S_t \) by supplying the PLM with \( S_t \) and its premises. It checks for the following types of errors:
   \begin{itemize}
       \item \textbf{Logical Inconsistency (\( \mathbb{L}_t \))}: Contradictions between \( S_t \) and its premises.
       \item \textbf{Mathematical Error (\( \mathbb{M}_t \))}: Incorrect calculations or quantitative reasoning in \( S_t \).
       \item \textbf{Accumulation Error (\( \mathbb{A}_t \))}: Errors in \( S_t \) due to incorrect premises.
       \item \textbf{Redundancy (\( \mathbb{R}_t \))}: Unnecessary repetition of information in \( S_t \).
   \end{itemize}

   Based on the detected errors, we assign penalties to the step score.

3. \textbf{Mathematical Formulation of Step Score}:
   \begin{itemize}
       \item \textbf{Premise Correctness}:  
       Each premise \( P_{t,i} \) is assigned a correctness score \( c_{t,i} \in [0, 1] \). The aggregated premise correctness for \( S_t \) is:
       \[
       C_t = \frac{1}{m} \sum_{i=1}^m c_{t,i}
       \]

       \item \textbf{Error Penalties}:  
       We define binary indicators for each error type:
       \[
       \mathbb{L}_t, \mathbb{M}_t, \mathbb{A}_t, \mathbb{R}_t \in \{0, 1\}
       \]
       with corresponding penalty weights \( p_L, p_M, p_A, p_R \in [0, 1] \). The cumulative error penalty for \( S_t \) is:
       \[
       P_t = 1 - \left( p_L \cdot \mathbb{L}_t + p_M \cdot \mathbb{M}_t + p_A \cdot \mathbb{A}_t + p_R \cdot \mathbb{R}_t \right)
       \]

       \item \textbf{Adjusted Premise Correctness}:  
       For accumulation errors, we adjust the premise correctness:
       \[
       C_t' = C_t \cdot (1 - \beta \cdot \mathbb{A}_t)
       \]
       where \( \beta \in [0, 1] \) is the weight for accumulation errors.

       \item \textbf{Step Score}:  
       The final score for step \( S_t \) is:
       \[
       V_t = C_t' \cdot P_t
       \]
   \end{itemize}

\subsection*{Reasoning-Level Evaluation}

In addition to step-level evaluation, we assess the overall reasoning chain for higher-level errors:

1. \textbf{Planning Error (\( \mathbb{P} \))}:  
   Occurs when the sequence of steps lacks logical coherence or fails to form a valid strategy to solve the problem. Detected by analyzing the reasoning chain using the PLM.

2. \textbf{Missing Step (\( \mathbb{M}_s \))}:  
   Occurs when essential steps necessary for reaching the correct solution are omitted. Identified by comparing the reasoning chain to an expected solution path.

3. \textbf{Reasoning-Level Scores}:
   \begin{itemize}
       \item \textbf{Planning Error Penalty}:
       \[
       R_P = 1 - p_P \cdot \mathbb{P}
       \]
       where \( p_P \in [0, 1] \) is the penalty weight for planning errors.

       \item \textbf{Missing Step Penalty}:
       \[
       R_M = 1 - p_{Ms} \cdot \mathbb{M}_s
       \]
       where \( p_{Ms} \in [0, 1] \) is the penalty weight for missing steps.

       \item \textbf{Combined Reasoning-Level Score}:
       \[
       R = R_P \cdot R_M
       \]
   \end{itemize}

\subsection*{Overall Solution Score}

The final score \( S \) for the solution \( \hat{Y} \) combines the step-level scores and the reasoning-level scores:
\[
S = \left( \prod_{t=1}^N V_t \right) \cdot R
\]
where:
\begin{itemize}
    \item \( V_t \) is the step score for \( S_t \).
    \item \( R \) is the combined reasoning-level score.
\end{itemize}

\subsection*{Lemma: Comprehensive Correctness Lemma}

If both step-level and reasoning-level evaluations accurately reflect the correctness of their respective components, then the overall score \( S \) provides a comprehensive measure of the solution \( \hat{Y} \)'s validity.

\textbf{Proof Sketch}:
\begin{itemize}
    \item \textbf{Step-Level Validity}: The product \( \prod_{t=1}^N V_t \) encapsulates the correctness of individual steps, factoring in both premise correctness and penalties for specific error types.
    \item \textbf{Reasoning-Level Validity}: The combined reasoning-level score \( R \) accounts for higher-level structural issues that are not captured at the step level, such as planning errors and missing steps.
    \item \textbf{Overall Validity}: By combining step-level and reasoning-level scores multiplicatively, the overall score \( S \) accurately reflects the cumulative effect of errors at all levels.
\end{itemize}
