\section{Appendix}

\subsection{Experimental Details}
\label{appx:model-choice}
\paragraph{Model} 
For the Llama model \cite{grattafiori2024llama3herdmodels}, we used vLLM \cite{kwon2023efficientmemorymanagementlarge} for model serving and AzureOpenAI for the GPT4o and GPT4-o1 \cite{openai2024o1} models. To ensure reproducibility, all generations were performed with a temperature=0. For all models, we used their instruct variant. 

\subsection{Data statistics}  
\label{appx:data-stats}
This results in a total of 607 reasoning chains with 203 positives, 214 negatives, and 190 synthetic negatives. In total, we have 2,134 steps annotated as \textit{Correct}, 321 steps as \textit{Mathematical Error}, 443 steps as \textit{Logical Inconsistency}, and 741 steps as \textit{Accumulation Error}, indicating that native and accumulation errors appear at almost equal rates. Additionally, we model each chain-of-thought as a directed acyclic graph (PARC) based on the step-level premise annotations. Across 1,370 PARCs, each chain contains on average 7.30 steps, representing the length of the chain-of-thought; 11.27 premises, corresponding to the total premise references across steps; and 10.42 edges linking premises to conclusions. The average depth of the PARC is 6.02, measuring the longest path of dependencies, while the maximum width is 1.90, quantifying how many steps can appear at the same layer in the PARC. Finally, the branching factor is 1.37, which is the ratio of edges to nodes, indicating that each step typically spawns fewer than two subsequent steps on average.

\begin{table*}[!t]
\centering
\setlength\tabcolsep{4pt}
\fontsize{8pt}{9pt}\selectfont
\begin{tabular}{lcccc|cccc|cccc|cccc}
\toprule
\multirow{2}[2]{*}{\textbf{Method}} & \multicolumn{4}{c|}{\textbf{GSM8K}} & \multicolumn{4}{c|}{\textbf{MATH}} & \multicolumn{4}{c|}{\textbf{Orca Math}} & \multicolumn{4}{c}{\textbf{MetaMathQA}} \\
\cmidrule(lr){2-5} \cmidrule(lr){6-9} \cmidrule(lr){10-13} \cmidrule(lr){14-17}
 & Neg & Syn & Pos & Avg & Neg & Syn & Pos & Avg & Neg & Syn & Pos & Avg & Neg & Syn & Pos & Avg \\
\midrule
\textit{\textbf{Full Context}} \\
Llama 3.1 8b & 48.87 & 57.68 & 91.9 & 58.6 & 46.91 & 60.29 & 90.61 & 60.2 & 43.26 & 51.2 & 96.7 & 55.06 & 50.47 & 55.15 & 95.17 & 59.45 \\
Llama 3.1 70b & 58.35 & 77.03 & 98.61 & 71.37 & 55.55 & 72.83 & 96.24 & 69.77 & 50.33 & 64.25 & 96.74 & 63.52 & 55.59 & 73.13 & 97.93 & 69.49 \\
Qwen 2.5 7b & 40.88 & 54.13 & 100 & 54.94 & 40.47 & 53.60 & 98.07 & 56.26 & 41.20 & 53.69 & 100 & 55.85 & 43.36 & 53.18 & 99.39 & 56.26 \\
Qwen 2.5 32b & 46.76 & 69.26 & 99.8 & 63.56 & 49.58 & 66.5 & 98 & 65.11 & 49.44 & 63.01 & 99.5 & 63.06 & 49.7 & 67.8 & 99.6 & 65.02 \\
Qwen 2.5 72b & 47.48 & 67.17 & 100 & 63.11 & 53.19 & 67.87 & 99.79 & 67.52 & 50.41 & 60.33 & 100 & 62.50 & 47.31 & 63.59 & 99.05 & 62.20 \\
GPT4o-mini & 52.68 & 73.14 & 95.02 & 66.68 & 50.41 & 62.39 & 93.4 & 63.03 & 55.22 & 61.18 & 98.6 & 64.59 & 51.52 & 65.42 & 97.35 & 64.47 \\
GPT-4o & 53.81 & 75.3 & 98.33 & 68.52 & 50.9 & 68.41 & 98.57 & 66.52 & 59.55 & 65.75 & 100 & 68.55 & 56.26 & 71.72 & 99.3 & 69.41 \\
\midrule
\textit{\textbf{Oracle Premises}} \\
Llama 3.1 8b & 57.02 & 82.64 & 74.61 & 69.26 & 54.58 & 60.94 & 51.65 & 56.47 & 57.43 & 60.61 & 59.48 & 59.09 & 52.65 & 68.48 & 60.51 & 60.37 \\
Llama 3.1 70b & 74.41 & 85.12 & 94.69 & 81.46 & 68.55 & 79.88 & 78.11 & 74.68 & 66.78 & 72.65 & 93.37 & 73.45 & 63.31 & 83.15 & 92.16 & 76.05 \\
Qwen 2.5 7b & 52.40 & 79.37 & 89.56 & 69.74 & 56.86 & 70.23 & 79.85 & 65.96 & 55.01 & 66.66 & 76.65 & 63.39 & 56.32 & 77.06 & 91.94 & 70.45 \\
Qwen 32b & 66.57 & 88.64 & 96.95 & 80.76 & 69.24 & 82.26 & 89.32 & 77.98 & 63.58 & 69.88 & 96.31 & 71.4 & 59.84 & 85.74 & 96.87 & 76.38 \\
Qwen 2.5 72b & 69.48 & 86.34 & 98.98 & 80.58 & 72.98 & 81.10 & 91.03 & 79.50 & 69.76 & 72.77 & 93.24 & 74.87 & 66.23 & 86.21 & 98.06 & 79.50 \\
GPT4o-mini & 63.83 & 85.77 & 91.29 & 76.16 & 65.86 & 74.99 & 78.41 & 71.71 & 64.03 & 72.66 & 83.56 & 70.74 & 58.77 & 85.11 & 91.44 & 74.73 \\
GPT-4o & 67.03 & 87.2 & 94.76 & 78.74 & 70.32 & 77.3 & 90.65 & 76.79 & 71.19 & 73.69 & 92.15 & 75.56 & 72.76 & 88.95 & 94.82 & 82.88 \\
\midrule
\textit{\textbf{Model Premises}} \\
Llama 3.1 8b & 54.61 & 76.74 & 65.44 & 64.53 & 52.16 & 59.83 & 50.96 & 54.88 & 55.57 & 59.82 & 67.36 & 59.22 & 54.52 & 69.13 & 62.63 & 61.78 \\
Llama 3.1 70b & 74.51 & 86.22 & 94.69 & 81.92 & 70.94 & 79.29 & 77.91 & 75.45 & 64.95 & 72.28 & 93.37 & 72.52 & 65.69 & 82.74 & 89.71 & 76.52 \\
Qwen 2.5 7b & 53.03 & 78.40 & 90.37 & 69.73 & 58.47 & 66.87 & 82.72 & 65.96 & 52.07 & 71.12 & 84.28 & 65.38 & 54.98 & 73.87 & 91.02 & 68.43 \\
Qwen 32b & 65.58 & 86.34 & 96.95 & 79.37 & 68.61 & 76.96 & 88.71 & 75.57 & 61.89 & 69.24 & 95.25 & 70.26 & 62.18 & 85.60 & 97.32 & 77.40 \\
Qwen 2.5 72b & 69.87 & 85.12 & 96.38 & 80.56 & 69.51 & 79.92 & 90.84 & 77.28 & 69.45 & 71.34 & 94.19 & 74.41 & 64.73 & 85.52 & 98.12 & 78.53 \\
GPT4o-mini & 58.25 & 80.81 & 86.72 & 72.33 & 65.84 & 72.4 & 71.88 & 69.45 & 61.18 & 72.48 & 85.94 & 69.93 & 57.04 & 81.63 & 91.15 & 72.51 \\
GPT-4o & 65.23 & 86.67 & 99.76 & 79.82 & 70.12 & 76.67 & 91.84 & 76.45 & 66.33 & 71.74 & 93.14 & 72.96 & 67.28 & 88.19 & 89.85 & 79.41 \\
\bottomrule
\end{tabular}

\caption{Detailed results for error identification for all dataset and models}
\label{tab:error_identification}
\end{table*}


\begin{table*}[!t]
    \centering
    \setlength\tabcolsep{4pt}
    \fontsize{8pt}{9pt}\selectfont
    \begin{tabular}{cccc|ccc|ccc}
    \toprule
    \multirow{2}[2]{*}{\textbf{Model Name}} & \multicolumn{3}{c|}{\textbf{Positives}} & \multicolumn{3}{c|}{\textbf{Negatives}} & \multicolumn{3}{c}{\textbf{Synthetic Negatives }} \\
    \cmidrule(lr){2-4} \cmidrule(lr){5-7} \cmidrule(lr){8-10}
     & Precision & Recall & F1 & Precision & Recall & F1 & Precision & Recall & F1 \\
    \midrule
    Llama 3.1 8b & 67.80 & 89.91 & 77.31 & 65.03 & 91.31 & 75.96 & 64.08 & 86.76 & 73.71 \\
    Llama 3.1 70b & 83.49 & 97.09 & 89.78 & 80.36 & 98.87 & 88.66 & 79.92 & 96.68 & 87.51 \\
    \midrule
    Qwen 7b & 69.94 & 62.85 & 66.21 & 66.14 & 67.60 & 66.86 & 66.18 & 60.15 & 63.02 \\ 
    Qwen 32b & 85.98 & 93.49 & 89.58 & 83.78 & 96.95 & 89.88 & 82.45 & 95.61 & 88.54 \\ 
    \midrule
    GPT4o-mini & 74.93 & 84.81 & 79.56 & 70.85 & 87.01 & 78.10 & 71.84 & 86.73 & 78.59 \\
    GPT-4o & 89.38 & 93.66 & 91.47 & 84.97 & 94.95 & 89.69 & 83.43 & 94.65 & 88.72 \\
    \bottomrule
    \end{tabular}
    
    \caption{Precision, Recall and F1 score for premise identification with GSM8K}
    \label{tab:gsm8k_premise_identification_per_candidate_score}
\end{table*}

\begin{table*}[!t]
    \centering
    \setlength\tabcolsep{4pt}
    \fontsize{8pt}{9pt}\selectfont
    \begin{tabular}{cccc|ccc|ccc}
    \toprule
    \multirow{2}[2]{*}{\textbf{Model Name}} & \multicolumn{3}{c|}{\textbf{Positives}} & \multicolumn{3}{c|}{\textbf{Negatives}} & \multicolumn{3}{c}{\textbf{Synthetic Negatives }} \\
    \cmidrule(lr){2-4} \cmidrule(lr){5-7} \cmidrule(lr){8-10}
     & Precision & Recall & F1 & Precision & Recall & F1 & Precision & Recall & F1 \\
    \midrule
    Llama 3.1 8b & 53.98 & 82.36 & 65.22 & 46.39 & 82.43 & 59.37 & 52.63 & 82.42 & 64.24 \\
    Llama 3.1 70b & 71.13 & 95.82 & 81.65 & 67.82 & 97.16 & 79.88 & 70.46 & 97.47 & 81.80 \\
    \midrule
    Qwen 7b & 59.40 & 60/30 & 59.85 & 48.82 & 56.02 & 52.17 & 59.40 & 62.87 & 61.09 \\
    Qwen 32b & 74.57 & 87.30 & 80.43 & 69.98 & 87.31 & 77.69 & 73.13 & 91.04 & 81.11 \\
    \midrule
    GPT4o-mini & 57.58 & 66.57 & 61.75 & 54.42 & 69.01 & 60.85 & 61.18 & 72.29 & 66.27 \\
    GPT-4o & 69.76 & 90.19 & 78.67 & 66.95 & 87.16 & 75.73 & 71.48 & 91.99 & 80.45 \\
    \bottomrule
    \end{tabular}
    
    \caption{Precision, Recall and F1 score for premise identification in MATH}
    \label{tab:math_premise_identification_per_candidate_score}
\end{table*}

\begin{table*}[!t]
    \centering
    \setlength\tabcolsep{4pt}
    \fontsize{8pt}{9pt}\selectfont
    \begin{tabular}{cccc|ccc|ccc}
    \toprule
    \multirow{2}[2]{*}{\textbf{Model Name}} & \multicolumn{3}{c|}{\textbf{Positives}} & \multicolumn{3}{c|}{\textbf{Negatives}} & \multicolumn{3}{c}{\textbf{Synthetic Negatives }} \\
    \cmidrule(lr){2-4} \cmidrule(lr){5-7} \cmidrule(lr){8-10}
     & Precision & Recall & F1 & Precision & Recall & F1 & Precision & Recall & F1 \\
    \midrule
    Llama 3.1 8b & 58.94 & 84.52 & 69.45 & 44.14 & 76.76 & 56.05 & 58.67 & 82.99 & 68.74 \\
    Llama 3.1 70b & 77.82 & 97.37 & 86.50 & 65.18 & 96.13 & 77.69 & 76.29 & 95.88 & 84.97 \\
    \midrule
    Qwen 7b & 58.05 & 59.09 & 58.57 & 48.72 & 66.48 & 56.23 & 59.86 & 59.41 & 59.63 \\
    Qwen 32b & 78.73 & 91.60 & 84.68 & 69.89 & 88.99 & 78.29 & 78.09 & 92.07 & 84.50 \\
    \midrule
    GPT4o-mini & 65.06 & 76.92 & 70.50 & 51.74 & 73.81 & 60.84 & 65.52 & 79.37 & 71.78 \\
    GPT-4o & 79.91 & 92.51 & 84.68 & 67.13 & 88.11 & 76.26 & 79.42 & 90.49 & 84.62 \\
    \bottomrule
    \end{tabular}
    
    \caption{Precision, Recall and F1 score for premise identification with MetaMathQA.}
    \label{tab:meta_math_premise_identification_per_candidate_score}
\end{table*}

\begin{table*}[!t]
    \centering
    \setlength\tabcolsep{4pt}
    \fontsize{8pt}{9pt}\selectfont
    \begin{tabular}{cccc|ccc|ccc}
    \toprule
    \multirow{2}[2]{*}{\textbf{Model Name}} & \multicolumn{3}{c|}{\textbf{Positives}} & \multicolumn{3}{c|}{\textbf{Negatives}} & \multicolumn{3}{c}{\textbf{Synthetic Negatives }} \\
    \cmidrule(lr){2-4} \cmidrule(lr){5-7} \cmidrule(lr){8-10}
     & Precision & Recall & F1 & Precision & Recall & F1 & Precision & Recall & F1 \\
    \midrule
    Llama 3.1 8b & 56.02 & 80.12 & 65.94 & 52.60 & 78.92 & 63.13 & 55.31 & 80.26 & 65.49 \\
    Llama 3.1 70b & 72.82 & 96.83 & 83.13 & 68.15 & 96.42 & 79.86 & 72.17 & 96.93 & 82.74 \\
    \midrule
    Qwen 7b & 62.33 & 61.81 & 62.07 & 68.15 & 96.42 & 79.86 & 58.75 & 59.69 & 59.22 \\
    Qwen 32b & 77.73 & 91.18 & 83.92 & 71.55 & 89.55 & 79.54 & 73.55 & 91.16 & 81.41 \\
    \midrule
    GPT4o-mini & 63.24 & 75.36 & 68.77 & 57.36 & 73.38 & 64.39 & 59.75 & 73.13 & 65.77 \\
    GPT-4o & 73.90 & 92.63 & 83.92 & 66.97 & 92.87 & 77.92 & 72.50 & 91.02 & 80.53 \\
    \bottomrule
    \end{tabular}
    
    \caption{Precision, Recall and F1 score for premise identification in OrcaMath}
    \label{tab:orca_math_premise_identification_per_candidate_score}
\end{table*}


\subsection{Prompt for Error Annotation with O1}
You are an expert mathematical reasoning analyzer. Your task is to analyze mathematical solutions and generate detailed error annotations in a specific JSON format. For each solution provided, you must carefully examine the reasoning chain and individual steps to identify any errors or issues.

Response Format

Your response must be a valid JSON object following exactly this structure:

\begin{verbatim}
{
  "error_annotations": {
    "chain_level": {
      "has_errors": boolean,
      "errors": [
        {
          "error_type": string,
          "error_description": string
        }
      ]
    },
    "step_level": [
      {
        "step_number": ,
        "has_error": boolean,
        "errors": [
          {
            "error_type": ,
            "error_description": 
          }
        ]
      }
    ]
  }
}
\end{verbatim}

Chain-Level Error Types

1. ``Missing\_Steps" \newline
\textbf{Definition:} Solution lacks crucial concluding steps or final answer derivation \newline
\textbf{Examples:} \newline 
  - Not showing the final calculated value \newline
  - Missing the ultimate conclusion \newline
  - Failing to complete the proof \newline
\textbf{Impact:} Makes the solution incomplete or inconclusive

2. ``Planning\_Error" \newline
\textbf{Definition:} The reasoning takes an invalid or fundamentally incorrect approach \newline
\textbf{Examples:} \newline
  - Using inapplicable theorems or methods \newline
  - Solving for incorrect variables \newline
  - Taking an approach that cannot possibly lead to a solution \newline
\textbf{Impact:} Makes the entire solution path invalid \newline
- Note: Valid but longer approaches (e.g., integration by parts instead of a substitution trick) should NOT be marked as errors

Step-Level Error Types

1. ``Logical\_Inconsistency" \newline
\textbf{Definition:} Steps that violate logical principles or make unjustified conclusions \newline
\textbf{Examples:} \newline
  - False equivalences \newline 
  - Invalid deductions \newline
  - Unsupported assumptions \newline
  - Note that incorrect use of previous information (example the step uses a wrong value of a variable) is a Logical\_Inconsistency \newline
\textbf{Impact:} Breaks the logical flow of the solution 

2. ``Mathematical\_Error" \newline
\textbf{Definition:} Incorrect calculations, misuse of formulas, or mathematical operations \newline
\textbf{Examples:} \newline
  - Arithmetic mistakes \newline
  - Incorrect algebraic manipulations \newline
  - Wrong formula application \newline
  - Note that Mathematical\_Error can only appear when there is an error in calculation \newline
\textbf{Impact:} Produces incorrect numerical or algebraic results 

3. ``Accumulation\_Error" \newline
\textbf{Definition:} Errors that propagate from previous incorrect steps \newline
\textbf{Examples:} \newline
  - Using wrong intermediate results \newline
  - Building upon previously miscalculated values \newline 
\textbf{Impact:} Compounds previous mistakes into larger errors \newline

4. ``Other" \newline
\textbf{Definition:} Any error that doesn't fit into the above categories
Examples: \newline
  - Notation mistakes \newline
  - Unclear explanations \newline
  - Formatting issues \newline
\textbf{Impact:} Varies depending on the specific error 

Analysis Requirements \newline
1. Examine each step against mathematical principles and theorems \newline
2. Verify all calculations and mathematical operations \newline
3. Check for proper use of definitions and formulas \newline
4. Ensure logical flow between steps \newline
5. Compare against the provided ground truth answer \newline
6. Consider the completeness of the solution 

Important Notes \newline
- Provide ONLY the JSON output, no additional text or explanations \newline
- Every step in the solution must have a corresponding entry in step\_level array \newline
- Keep error descriptions clear, specific, and mathematically precise \newline
- Use empty arrays for errors when no errors exist \newline
- Ensure your response is always valid JSON that matches the exact format specified \newline
- Each error must have both an error\_type and a corresponding detailed error\_description \newline
- Error descriptions should be specific to the mathematical context of the problem \newline
- Do NOT penalize valid but verbose approaches (e.g., breaking down algebra into multiple steps) \newline
- Do NOT mark alternative solution methods as errors unless they are genuinely invalid \newline
- Focus on correctness rather than elegance or brevity \newline

Workflow \newline
1. Read and understand the problem statement \newline
2. Analyze the reasoning chain step-by-step \newline
3. Check for chain-level errors \newline
4. Analyze each step for specific errors \newline
5. Verify all premises and justifications \newline
6. Ensure completeness of the solution \newline
\label{sec:appendix_prompt}


\subsection{Prompt for Premise Annotation with O1}
The system prompt and the instruction for the O1 model for identifying premises for a given step is shared in Table \ref{tab:premise_prompt}

\subsection{Prompt for Error identification}
The prompts used for the baseline approach are shared in Appendix \ref{tab:baseline_error_id_instruction} and \ref{tab:baseline_error_id_sys} 
\label{sec:baseline_error_prompts}. The evaluation for our error identification with premises are done with the prompts outlined in Tables \ref{tab:ours_math_error} and \ref{tab:ours_logical_error},




\begin{table*}[h!]
\centering
\begin{tcolorbox}[width=0.9\linewidth, colback=gray!10, colframe=black, sharp corners]
{\bf Instruction:} \newline
Given this math word problem and its solution steps, identify the key premises and their relationships. \newline
{\bf Problem:} \{problem['question']\} \newline
{\bf Solution Steps:} \newline
\{chr(10).join(problem['steps'])\} \newline
{\bf Return your analysis in this exact JSON format:} \newline
\{json\_template\} \newline
{\bf Critical Rules for Premises:} \newline
1. A step can NEVER use itself as a premise. For example, Step 3 cannot use any premise labeled as [3, "..."]. \newline
2. Premises can only come from: \newline
   - Step 0 (problem statement and fundamental mathematical principles) \newline
   - Previous steps (steps with lower index) \newline
3. Any intermediate calculations or logical steps within a single step should be part of that step's reasoning, not treated as separate premises. \newline
4. Mathematical principles (like properties of operations) should be treated as part of Step 0. \newline
\newline
{\bf Additional Requirements:} \newline
1. Start with Step 0 containing the problem statement. \newline
2. For each step after 0, copy the EXACT text from the student's solution into `original\_step`. \newline
3. Show clear reasoning for how premises lead to conclusions. \newline
4. Return ONLY valid JSON with no additional text. \newline
5. Do not use any special characters like \&, <, >, etc. \newline
6. Do not add any additional text for formatting it (e.g., "json"), just output the raw JSON. \newline
7. Maintain the exact same number of steps as in the original solution. \newline
\newline
{\bf Remember:} \newline
- Each step's premises must strictly come from either the problem statement (Step 0) or previous steps. Never from the current step. \newline
- Keep each step atomic—do not split steps into multiple substeps even if they contain multiple calculations. \newline
- The number of steps in your output (excluding Step 0) must match exactly with the number of steps in the student's solution. \newline
\newline
{\bf System Prompt:} \newline
You are an expert in mathematical reasoning. Your task is to analyze solution steps and output a JSON object containing: \newline
1. The premises used in each step. \newline
2. The conclusion reached. \newline
3. The reasoning that connects premises to conclusions. \newline
\newline
Output MUST be valid JSON with no additional text or explanation.
\end{tcolorbox}
\caption{\label{tab:premise_prompt} System prompt and instruction for the O1 model to identify premises for a given step}
\end{table*}

\clearpage


\begin{table*}[h!]
\centering
\begin{tcolorbox}[width=0.9\linewidth]
{\bf Instruction:}\\ 
Question: \{question\} \newline
Solution so far: \{solution\} \newline \newline


1. ``Logical\_Inconsistency" \newline
- \textbf{Definition}: Steps that violate logical principles or make unjustified conclusions \newline
- \textbf{Examples}: \newline
  - False equivalences \newline
  - Invalid deductions \newline
  - Unsupported assumptions \newline
  - Note that incorrect use of previous information (example the step uses a wrong value of a variable) is a Logical\_Inconsistency \newline
- \textbf{Impact}: Breaks the logical flow of the solution \newline

2. ``Mathematical\_Error" \newline
- \textbf{Definition}: Incorrect calculations, misuse of formulas, or mathematical operations \newline
- \textbf{Examples}: \newline
  - Arithmetic mistakes \newline
  - Incorrect algebraic manipulations \newline
  - Wrong formula application \newline
  - Note that Mathematical\_Error can only appear when there is an error in calculation \newline
- \textbf{Impact}: Produces incorrect numerical or algebraic results \newline

3. ``Accumulation\_Error" \newline
- \textbf{Definition}: Errors that propagate from previous incorrect steps \newline
- \textbf{Examples}: \newline
  - Using wrong intermediate results \newline
  - Building upon previously miscalculated values \newline
- \textbf{Impact}: Compounds previous mistakes into larger errors \newline

4. ``Other" \newline
- \textbf{Definition}: Any error that doesn't fit into the above categories \newline
- \textbf{Examples}: \newline
  - Notation mistakes \newline
  - Unclear explanations \newline
  - Formatting issues \newline
- \textbf{Impact}: Varies depending on the specific error \newline


Statement to analyze:
{step}

Format your response as:
Reasoning: [detailed analysis of the statement's validity]
Verdict: [CORRECT,  Mathematical\_Error, Logical\_Inconsistency, or Accumulation\_Error]
\end{tcolorbox}
\caption{\label{tab:baseline_error_id_instruction} Baseline error identification Instruction }
\end{table*}

\begin{table*}[h!]
\centering
\begin{tcolorbox}[width=0.9\linewidth]
{\bf System Prompt:}\\ 
You are a helpful AI assistant that analyzes mathematical solution steps. 
    Your task is to determine if each statement is COMPLETELY correct by carefully analyzing its validity.
    Focus ONLY on whether the current step is valid - do not consider whether it helps reach the final answer or whether better steps could have been taken.
    Mark a statement as CORRECT unless you find a specific error.
\end{tcolorbox}
\caption{\label{tab:baseline_error_id_sys} Baseline error identification System prompt }
\end{table*}

\begin{table*}[h!]
\centering
\begin{tcolorbox}[width=0.9\linewidth]
{\bf System Prompt:}\\ 
Your task is to determine whether a given sentence contains any mathematical errors. 
For mathematical error, check if the sentence contains mathematical calculations (arithmetic or algebraic), and whether they are incorrect. If there are such errors, mark the sentence as "mathematical\_error"
-   Mathematical errors can only come from incorrect results of mathematical operations
If no such errors exist, mark it as "correct".

Note: mathematical error can only come from incorrect numerical or algebraic calculations (i.e. wrong multiplication, wrong addition etc.)
if there are no numerical or algebraic calculations done, you can mark it as correct \newline
{\bf Instruction :}\\ 
Statement to analyze:\newline
{step}\newline
Format your response as:\newline
Reasoning: [detailed analysis of the statement's validity]\newline
Verdict: [correct or  mathematical\_error]
\end{tcolorbox}
\caption{\label{tab:ours_math_error} Math error identification instruction }
\end{table*}

\begin{table*}[h!]
\centering
\begin{tcolorbox}[width=0.9\linewidth]
{\bf System Prompt:}\\ 
You are provided with a math question, a statement which is a step in the solution to the question and the premises to this steps (the question is also a premise). Your task is to identify whether the step follow naturally from the premises or not. 
If the current step contradicts the premises, mark is as a logical\_inconsistency
If the step can be directly inferred from the premises, mark it as correct.
You should not check whether the premises are correct, assume they are correct. Only check the sentence given.\newline
{\bf Instruction :}\\ 
Given Premises: \newline
Question:  \{question\} \newline
Previous steps as premise: \{premises\} \newline
Statement to analyze: \{step\} \newline
Guidelines:\newline
1. for logical\_inconsistency check if the step was performed under misinterpretation of the premises, made invalid deductions or had unsupported assumptions \newline
2. Don't check for correctness of the premises, your only task is to check correctness of the given sentence


Format your response as:           
Reasoning: [detailed analysis of the statement's validity]
Verdict: [correct, logical\_inconsistency]
\end{tcolorbox}
\caption{\label{tab:ours_logical_error} Logical error identification instruction }
\end{table*}

\begin{table*}[h!]
\centering
\begin{tcolorbox}[width=0.9\linewidth]
{\bf Instruction:}\\ 
You are provided with a question, a partial solution, and the next step in the solution. \newline
Your task is to identify the steps that serve as premises for the given next step.\newline
A step qualifies as a premise if the next step directly relies on information from that step. Based on the identified premises, the correctness of the next step should be fully verifiable.\newline
Question (Step 0):\newline
\{question\}\newline
Solution so far:\newline
\{solution\}\newline
Next step to analyze:\newline
\{step\}\newline
For the step above, identify which previous steps (including Step 0 - the question) are premises and explain why each one is necessary. Remember:\newline
1. A step cannot be a premise to itself\newline
2. The question (Step 0) can be a premise if used directly\newline
Generate ONLY the premises and nothing else.\newline
Format your response with one premise per line as:\newline
Step X: [explanation of why this step is necessary for the current step]\newline
\end{tcolorbox}
\caption{\label{tab:premise_ideintification_prompt} Prompt for evaluating models in the premise identification task (zero shot) }
\end{table*}
\appendix
