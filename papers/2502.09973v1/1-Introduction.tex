\section{Introduction}

% \revision{1. An excessive amount of details about the two studies and the InteRecon are presented. I suggest leaving some of these details in the subsequent sections.}

% \revision{2. explain why we choose mixed methods. What are the benefits and limitations of it? }

% \revision{3. Also, what research communities are your target audiences? Who would benefit from this work’s research results and insights?}

% \revision{4. highlight Assigning physical attributes to digital entities, and its significance and potential applications}
% Digitization for mementos is used for a wide range of personal memory archives as mementos can be a trigger to playback to a certain phase of memory through invoking important places, people, times, etc\cite{}.
% Digitization for mementos is widely used for building personal memory archives, as mementos can serve a trigger to represent a certain phase of memory through invoking and symbolizing important places, times, things, people, and experiences.




% A memory artifact refers to any object or item that holds personal significance or memories for an individual. These artifacts can range from physical objects like photographs, letters, souvenirs, and personal belongings to digital items such as digital photos, videos, or audio recordings. They serve as tangible or digital reminders of past experiences, people, places, or moments in time, playing a crucial role in preserving personal history and evoking memories.

% A memory artifact is characterised as an intentionally made object or structure that is involved in memory practices and aids its user in remembering an experience, event, fact, or other unit of information

Memorable personal items function as catalysts, invoking and symbolizing significant locations, moments, objects, individuals, and experiences to represent memory phases \cite{10.1145/1806923.1806924,petrelli2010family, bowen2011remembering}. Humans seek lifelong preservation of personal memory, mainly through the digitization of physical content.
Current digitization methods for personal memories are largely based on videos or pictures, with an auxiliary focus on capturing textual descriptions and physical appearances of memorable items \cite{petrelli2010family,KALNIKAITE2011298}. 
However, digital copies created by these methods are often unable to fully express the richness of memories, as they do not inherit the interactive features (e.g., the physical, sensory, or functional use) from the physical items. 
Imagine that using your grandmother’s jam book or soup ladle occasionally ignites a story your mother told about your grandmother---a digital copy of videos or pictures could never be used in daily life \cite{10.1145/1806923.1806924}.
Moreover, prior work found the stories of past memories associated with memorable items may be `stumbled across' in people's interactions or functional use of physical objects in the real world, while digital photos or videos tend to be stored away and do not become integrated into people's real life \cite{petrelli2010family,KALNIKAITE2011298,west2007memento}.
\revision{Therefore, personal reconstructing the interactivity of memorable items in their digital counterparts is valuable in facilitating personal memory archives in people's daily lives.}


In this paper, we present \textit{Interactive Digital Item} (IDI), a novel concept for personal item digital reconstruction featuring the preservation of its physical interactivity. 
The concept of IDI extends beyond the conventional `interactive 3D models' typically created by developers or modelers. IDI includes an authoring process that is as accessible as photo capturing, allowing non-experts to contribute to enriching personal memory archives.
We started with a formative study to investigate users' expectations of the essential attributes that constitute the physical interactivity of memorable items.
Two prominent types of memorable items are proposed by participants as most required for reconstruction - obsolete electronic devices (e.g., music players, cassette players, game controllers) and physical artifacts (e.g., dairies, albums, toys) - because users found the interactive features in these items largely convey the traces of memories.
Three design goals were devised from the study that needed to be incorporated into IDI, in which IDI should: 1) be a reconstructed 3D model from the physical item with similar visual properties (e.g., size, shape, texture) and physical properties (e.g., gravity, collisions, motions), 2) demonstrate the original interactivity within the interfaces of electronic devices by activating the tangible widgets (e.g., pressing buttons, dragging sliders) on devices and 3) preserve the embedded content of digital files (e.g., photos, songs, or software) in electronic devices and the usage scenario contexts for physical artifacts.





% ------------>

% Inspired by the above findings, we present \textit{Interactive Digital Item} (IDI), a novel concept for item digitization featuring the preservation of physical interactivity for personal, memorable items, outputting three design goals including the reconstructing the geometry, the interface, and the embedded content. Specifically, we summarize the design goal of IDI as: 
% % We further envisioned and detailed the features on IDM according to the design goals and considering the distinct functional manifestations of these two prominent types of memorable items identified earlier.
% % For each design goal, we proposed IDM should: 
% 1) be a reconstructed 3D model with similar visual properties (e.g., size, shape, texture, and material, etc) and physical properties (e.g., gravity, collisions, etc.) for physical artifacts; 2) demonstrate the original interactivity within the interfaces of electronic devices by showcasing the reconstructed tangible widgets (e.g., pressing buttons, dragging sliders, etc.) and ; 3) preserve the stored or embedded content of the digital files (e.g., photos, songs, or software, etc.) in electronic devices and the usage scenarios for the physical artifacts.
% ---------------<
Based on IDI's design goals, we implemented an AR prototype, namely \textit{InteRecon}, to support the end-user creation of IDI from a physical item in the Mixed Reality (MR) environment. 
InteRecon provides four core functions - reconstructing 3D appearance, adding physical transforms, reconstructing interface, and adding embedded content - for physical interactivity reconstruction. 
We conducted a two-session user study with 16 participants to understand the feasibility of using InteRecon to create IDI and further explore the participants’ experiences of using InteRecon to reconstruct their own items, collecting feedback on the challenges and future opportunities of IDI.
Results show that InteRecon was effective and enjoyable for IDI creation, and the physical interactivity created by participants augmented the realistic experiences of participants. Moreover, participants also proposed that creative reconstructions go beyond real-world interactivity, along with in-situ and life-logging usage scenarios of IDIs, to enrich memory archives. 
We make the following contributions:
\begin{itemize}
    \item A novel concept of IDI derived from a formative study, a digital personalized reconstruction of memorable personal items while maintaining their physical interactivity features.
    % incorporating the design goals of reconstructing geometry, interface, and embedded content from a formative study.
    \item An AR prototype, InteRecon, enables end-users to create IDI incorporating design goals of reconstructing geometry, interface, and embedded content.
    \item Potential opportunities and applications of IDI discovered from a the user study in terms of realism creation, interactivity approaches, and customization potentials. Future avenues are also discussed to augment the creation and usage of IDI for enriching personal memory archives. 
\end{itemize}




% end-users to 1) reconstruct and import the 3D model of the physical memento by 3D scanning on mobile devices, 2) map tangible widgets of digital devices to the 3D model in AR, 3) map the mechanical component of physical artifacts to the 3D model in AR, 4) upload and modify digital files on mobile devices to AR environment, 5) try out the IDM by bare-hands within the AR environment.
% InteRecon is a multi-device system: an mobile application designed for 3D scanning for a physical object and upload digital files to AR environment, and an authoring interface built on a optical see-through AR glasses.
% By blending the authoring process into the AR environment, users could seamlessly transition between the physical and virtual realms, which eliminates the workload for switching from physical objects to its associated AR content.
% The choice to use mobile devices is driven by their accessibility for capturing the physical appearance (such as 3D scanning) and reading digital files from memorable items.
% We conducted a two-session user study to evaluate the user experience of utilizing InteRecon to create IDM within a mixed reality environment and further investigated the feedback and ideas on the challenges, future opportunities, and applications of InteRecon.
% % We conducted a user study to assess the feasibility of digitizing physical artifacts and adding interactability through MemoTool in terms of realism creation, interactivity approaches, and customization potentials. Moreover, we highlight future avenues to enrich the creation and usage of IDM. 
% The first session included 4 pre-designed tasks covering all the atomic interactions broke down from functionalities of InteRecon.
% The second is an exploratory session where we encouraged our participants to freely utilize InteRecon to reconstruct an IDM without any requirements on their physical counterparts and follow up with an in-depth interview to gather qualitative feedback.
% The results from our study show that InteRecon is an effective, expressive, and enjoyable tool for creating IDM.
% The results of qualitative feedback illustrated that InteRecon could reconstruct the realism with real-world features on memorable items, the ease of InteRecon's interaction approaches, and the customization potentials were rich to extend.
% Our participants also expressed the expectations of creating a virtual community that includes various pre-set resources (e.g., the 3D models of memorable items, software, operating systems, complex mechanical components, etc.) to expand the accessibility of IDM and InteRecon for more non-technician users.
% realism with real-world features on physcial mementos have been reconstructed through MemoTool, the ease of interaction approaches of MemoTool, the customization potentials and the design space of 


% digitizing physical artifacts 
% 1、我们通过访谈了解了用户对于digital memento的期待,并提出了IDM的概念
%  2、我们基于IDM的概念实现MemTool,有XXX的功能
%  3、通过用户实验,我们证明了memtool在xxx的优越性,并得到未来完善IDM实现的一些指导


% which included the insights from the 



% have contextual information---the status viewed from different perspectives—of the physical memento and its associated AR content. 

% based on the necessity to seamlessly transition between the physical and virtual realms

% These results highlight the need and feasibility for the automatic creation of optimal mappings, as users had a consistent preference for good mappings and could not design their favorite mappings on their own. 

% three characteristics are important for enabling functions that 

% the digitization of physical mementos. 
% We invited 10 participants to engage in semi-structured interviews aiming to uncover the important characteristics that constitute the functions of a physical memento. 

% where the interactability expressed where?
% Where is interactivity manifested
% the important elements that consititute the interactabilities of a physical memento?



% 1. two types of mementos
% 2. personal usage traces on physical appearances of mementos
% 3. interactions of mementos
% 4. embeded content of mementos



% This study recruited 16 participants 




% To make digitization of mementos more close to reality, facilitating to preserve memory, we believe the digitization process of mementos could keep the context of interactability.

% To achieving this goal, we started from a formative study to investigate the users' expectations on the interactability of mementos. 
% The study recruited xxx participants xxx.
% The results informed that the characteristics for a memento to express interactability is around xx, xx and xx.
% We proposed a concept of Interactive Digital Memento (IDM) to describe and embed the characteristics from the formative study. 
% We believe the concept of IDM could put forward the interactability preserving for mementos, aiming to facilitate personal archives.

% We propose MemoTool, an xxx authoring tool that supports the end-user to build IDM.
% Considering AR technologies could provide a xxxx environment with advantages of xxx, we further implemented MemoTool, 
% MemoTool allows end-users to create IDM within an immersive environment by leveraging the functions of xx, xx, and xx.

% We conducted an two-session user study to evaluate the xxx, xxx, and xxx. We first evaluated the usability and the user experience of xxx by xxx micro tasks.
% We further investigated the qualitative feedback from xxxx through xxx.
% The results showed that xxx.


% Sentimental artifacts can invoke and symbolize important places, times, things, people, and experiences. And it is through this process that they garner their value.

% For static mementos such as xxx, this approach works well on xxxx of memories, 
% However, in more scenarios, dynamic mementos that have the context of interactability (e.g., functionalities and manipulation) tend to be more prevalence and impressive for people \cite{}. 
% Moreover, interactability is confirmed to be the essential element in personal memory in terms of xxx, xx, xxx. 
% Thus, it still remains a problem that how to digitize the memento with preserving its interactability. 



% 现在个人的数字重建是一种对于增强memory的重要方式,但是现在的数字重建停留在模型阶段,只保留了物理外观。然而,前人的工作证明了和物理的artefacts交互是保留记忆的重要因素,同时也是保留记忆的重要环节。所以为了让数字化physical artefacts更加接近真实,更加发挥增强记忆的作用,我们希望数字化的过程可以保留用户和physical artefacts的交互或用法等等。

% 为了实现这一目标,我们做了一个formative study,调研用户对DA的交互细节的期待,实验结果表明,xx,xx,xx是典型的,可以概括用户期待的三个方面,根据这三个方面的design consideration, 我们进行了prototype的开发,实现了可以让用户在沉浸式环境中enable三个方面的workflow和有代表性的模型。

% we present IDAR, an AR-based interactive system to enable users to build an interactive digital assets, based on a physical artefacts with its interactive function preserving. To validate the usability of IDAR, we conducted an evaluation to validate its usability. Results found that the xxxx.


















% AR/VR could be a good tool for reconstructing interactability as its xxxx













% Digitization for mementos is an important method for xxx personal memeory archives. 
% As mementos can be a trigger to invoke memeory points such as important xxxx.








% % benefits of digitalization of reconstruction in immersive environments
% Physical artefacts (e.g., diaries, albums, paintings, etc.) have been confirmed their value as a representation of memory. For example, prior work attributes physical artefacts as a mediation to yield important insights into the goals and purposes of sharing memories or 'passing on' memories for the family \cite{10.1145/3173574.3173998,10.1145/1806923.1806924}. However, compared with digital artefacts, physical artefacts have its disadvantages inaccurate indexing and the need for the amount of physical space \cite{10.1145/3173574.3173998}, which is challenging to keep and make them 'alive' in an individual's memory. 

% Digitalization, achieved through the virtual reconstruction of physical artifacts using immersive technologies such as VR or AR, offers users an effective means to make their memories visible and reduce preservation costs \cite{}. Several studies have explored the significance of virtual representations in preserving memories and functioning as digital mementos, as they retain the benefits of both digital attributes (e.g., easy sharing) and physical attributes (e.g., more emotional expressive) simultaneously. 

% The HCI community has explored virtual reconstructions for environments or physical artefacts that are sentimental and valuable for individual's memory preservation such as homes, books, diaries, paintings, and photos \cite{}. 




% % --------------------------------------------------------


% We need technologies that respect the highly personal nature of people’s collections, and that can incorporate idiosyncratic physical objects \cite{10.1145/1518701.1518966}.




% Building on this notion of heirlooms again raises the potential for taking from the physical notions of frailty and degradation or patina, issues which were raised throughout the fieldwork as people marveled at objects they owned such as the old photographs where the very survival of the images through the ravages of time was seen as a mark of value. Turning the perceived value of the digital’s lasting permanence on its head by offering digital files that gracefully degrade through use might then offer richer ways of interacting with the digital in line with the values of observed decay in curated spaces spoken about by DeSilvey

% frailty and degradation or patina




% Consequently, it is worth asking if a system designed to help capture, manage, share and keep safe digital materials could be expanded upon and enriched to incorporate aspects of the vast array of physical things that we may also keep and care about. 

% —Affordances of the digital. It is worth initially considering what we see the benefits of digital storage to be. Evidently, there are a diverse range of natural digital behaviors which facilitate such activities as the accurate indexing, cataloging and retrieval of information, combined with the ability to edit content, append meta-data and otherwise manipulate and move it through various means of sharing, alongside a significant reduction in the amount of physical space that is necessarily required for the storage of sentimental artefacts. Consequently, it is worth asking if a system designed to help capture, manage, share, and keep safe digital materials could be expanded upon and enriched to incorporate aspects of the vast array of physical things that we may also keep and care about. In other words, when we look at our computers and think about existing tools for helping to manage and archive digital photographs and videos, we can begin to ask ourselves what a digital archiving system would look like if it took the diversity of other kinds of objects into account and tried to apply the strengths of digital archiving to a broader set of artefacts. How might this change the nature of physical artefacts such that they can be incorporated with digital artefacts or be connected to them in richer ways?



% These research were always focus on improve the fidelity of 3D reconstruction of the artefacts, 

% This body of prior work has addressed the ways that the “stuff” of family memory can be mediated through physical and digital technological artifacts and has yielded important insights into the goals and purposes of sharing memories for the family.



% !!! AR UI/INTERACTIVE DESIGN ALSO COUNTS

% There is also a focus on understanding how to create sensory affordance for the reconstructions comparable to their physical counterparts, such as xxx. However, in addition to  







% there is a research gap in exploring the 


% xxx ; xxx; xxx;





% Among these physical artefacts, digital devices have their uniqueness in reconstructing virtually 


% have challenges in 

% Among works on virtual reconstruction for enhance memory replications,  targets, such as spaces, movements, 

% xx, xx, xx



% There is also a focus on understanding how to create sensory affordance for the reconstructions comparable to their physical counterparts [70, 71] (e.g. tactile sensations for virtual surfaces [75]). This can be useful for activities requiring precise sensory input for skilled physical tasks (e.g., cutting through material for surgery).

% The increasing access to immersive technologies is also leading to a growing body of HCI work on system building and interaction design for reconstructing real physical objects and translating their attributes to virtual representations.  

% system building and interaction design for informal settings with remote friends and family.

% When compared with traditional 2D media, XR provides a more effective way for users to perform spatial referencing and demonstrate actions remotely [45]. This can especially be helpful during collaboration and expert guidance for physical tasks in professional settings (e.g. surgical training [20, 57, 95], field servicing [100]). The increasing access to XR devices is also leading to a growing body of HCI work on system building and interaction design for informal settings with remote friends and family. These works can involve, for example, games [2], learning [82], and other general-purpose telepresence applications [47, 48]. Virtual representation of the users, their surrounding physical spaces, and objects in XR is an important element of this interaction design for collaboration or instruction. Recent advances in creating high-fidelity 3D environments [102, 103] and avatar reconstructions [38, 101] for XR devices are allowing remote collaborators to explore and obtain better spatial and semantic information from a remote environment.
% % Virtual representations of surrounding physical spaces and objects provides a more efficient way for users to perform spatial referencing, especially a feeling of scaling properties \cite[]. These physical properties is especially important for users to conduct reminiscence activities. Many works used the virtual reconstructions to preserve the memory and feeling associated with individuals’ private spaces by reconstructing the environment in immersive technologies.
% % Digitalization by virtually reconstructing the environment or physical artifacts in immersive technologies (VR or AR) provides an efficient way for users to make their memory visible and reduce the keeping cost. Many works discussed the value of virtual representations for preserving shared memories and as mementoes in terms of their simoutenously maintaining the advantages of digital (e.g., can be shared easily) and physical (e.g., more emotionally expressive) properties. 

% investigated the importance and the connections of an physical artifacts and memory.  

% Spatial referencing and physical properties of an environment or an object is especially important for reminiscence activities \cite[]. 



% In this way, users could get a better understanding of spatial properties of an object 

% This can be helpful during tasks with strongly physical dependencies (teaching\cite{}, field servicing\cite{}, games\cite{}, learning\cite{}). 

% In this way, users could get a better understanding of 


% make immersive reminiscence experiences and have the potential to be used in their digital assets to keep. 



% perform spatial referencing, especially a feeling of scaling properties \cite[]. In this way, users could get a better understanding of 



% This can be helpful during tasks with strongly physical dependencies (teaching\cite{}, field servicing\cite{}, games\cite{}, learning\cite{}). 


% AR Digitalization for Outdated Devices
% Why Digitalization?
% make it alive for users, and reduce the keeping cost
% make broken devices alive in a new format
% can be used in museums or education scenarios
% What are the values of outdated devices?
% including memory on interactive workflow/ methods
% should conduct a formative study to explore more
% Why AR?

