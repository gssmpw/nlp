\newpage
\onecolumn
\begin{table*}[tbh!]
  % \vspace{-0.3cm}
  \centering
  \caption{\textbf{Descriptions of the pre-defined physical joint categories.}}
  ~\label{tab:joints}
    \vspace{-0.3cm}
    \resizebox{\linewidth}{!}{
    \begin{tabular}{c|c|c}
    \toprule
    \textbf{Index} & \textbf{Name} & \textbf{Description} \\
    \midrule
    1  & Pivot & It enables rotational movement around a single axis. \\

    2  & Ball-and-socket & It offers the widest range of motion, allowing for movement in all three planes, including rotation. \\

    3  & Hinge & It allows movement in one plane, similar to the way a door hinge works. \\

    4  & Condyloid & It allows movement in two planes: flexion and extension, as well as abduction and adduction. \\

    5  & Plane & It involves two flat surfaces that slide over each other, allowing for limited movement in multiple directions. \\

    6  & Saddle & It allows angular movements similar to Condyloid joints but with a greater range of motion. \\

    \bottomrule
  \end{tabular}
  }
\end{table*}



% 这个最好放到文章里
\begin{table}[tbh!]
  % \vspace{-0.3cm}
  \centering
  \caption{\textbf{The metrics and the questions in the questionnaire}}
  ~\label{tab:questionnaire}
    \vspace{-0.3cm}
    \resizebox{\linewidth}{!}{
    \begin{tabular}{c|c|c}
    \toprule
    \textbf{Index} & \textbf{Metric} & \textbf{Question} \\
    \midrule
    
    1  & Easiness to use & I think this function is easy to use. \\

    2  & Learnability & I think I learned this function quickly. \\

    3  & Helpfulness & I think this function is helpful for realizing the IDI concept.\\

    4  & Expressiveness & I think this function is able to express my thinking for reconstruction. \\

    5  & Non-frustration & I think I can complete this function without frustration. \\
    \bottomrule
  \end{tabular}
  }
\end{table}


% Ball and socket joint – the rounded head of one bone sits within the cup of another, such as the hip joint or shoulder joint. Movement in all directions is allowed.
% Saddle joint – this permits movement back and forth and from side to side, but does not allow rotation, such as the joint at the base of the thumb.
% Hinge joint – the two bones open and close in one direction only (along one plane) like a door, such as the knee and elbow joints.
% Condyloid joint – this permits movement without rotation, such as in the jaw or finger joints.
% Pivot joint – one bone swivels around the ring formed by another bone, such as the joint between the first and second vertebrae in the neck.
% Gliding joint – or plane joint. Smooth surfaces slip over one another, allowing limited movement, such as the wrist joints.


% \section{Appendix}


% \begin{table*}[tbh!]
%   % \vspace{-0.3cm}
%   \centering
%   \caption{\textbf{Descriptions of the pre-designed demonstrated physical joints.}}
%   ~\label{tab:joints}
%     \vspace{-0.3cm}
%     \resizebox{\linewidth}{!}{
%     \begin{tabular}{c|c|c}
%     \toprule
%     \textbf{Index} & \textbf{Name} & \textbf{Description} \\
%     \midrule
%     1  & Plane & Smooth surfaces slip over one another, allowing limited movement, such as the wrist joints. \\
    
%     2  & Hinge & The two bones open and close in one direction only (along one plane) like a door, such as the knee and elbow joints. \\

%     3  & Condyloid & This allows movement in two planes: flexion and extension, as well as abduction and adduction. \\
    
%     4  & Pivot & One bone swivels around the ring formed by another bone, such as the joint between the first and second vertebrae in the neck. \\
    
%     5  & Saddle & This permits movement back and forth and from side to side, but does not allow rotation, such as the joint at the base of the thumb. \\
     
%     6  & Ball-and-socket & This permits swing in two planes and rotate around the other object.\\
%     \bottomrule
%   \end{tabular}
%   }
% \end{table*}







% \subsection{}
