\section{Related Work}
% real world features ?
% real-world interactivity ?
\subsection{Real-world Interactive Features of Memorable Items}
% \revision{Why real-world interactivity of memory artifacts is important to trigger memory? }

Previous research has explored the characteristics of memorable personal items that activate recollections. 
Physical objects, along with the environment or context in which a remembered activity occurred, play an important role in how an activity is represented in one's memory ____. 
The tangible nature of physical objects, including elements you can visually perceive, touch, or interact with, are important in understanding the essence of memory activities ____. 
% This physicality may be even more significant than the abstract emotional feelings behind them ____.
For instance, the physical interactions involved in an activity, such as the tools utilized, the space it occurs in, and other physical aspects, can create enduring memories ____.
These memories can easily resurface in everyday life, particularly when one incidentally encounters them through the functional use or the random interactions of a physical object ____.  
% Memories are not only naturally triggered when encountering the memento but can also be prompted by activating certain features of it ____.
____ mentioned that tangible real-world interaction such as flipping pages or arranging blocks can improve concentration in collocated communication about memories and even stimulate collaboration in memory activities. 
____ found people favored using physical souvenirs from travel to access photo sets, appreciating the serendipitous sharing of physical souvenirs.
____ investigated the advantages of sharing family photographs facilitated the physical affordances of objects (e.g., the ability to store and display images) within the home environment, integrating them into daily routines.
Additionally, there is a growing focus on how physical interactions that can open up new ways to organize and collective memories, provoke social conversations ____, and support collective experiences of reminiscence and reflection ____.

Although existing work has incorporated different interactive features into personal items to improve usability and immersiveness, none of them has explicitly revealed the user expectations regarding the indispensable attributes of physical interactivity for activating personal memories.
% entire definition space of interactivity that is crucial for activating personal memories. 
In this paper, we bridge this gap by conducting a formative study to identify essential physical interactivity attributes of memorable personal items, with the goal of reconstructing memorable personal items with interactivity integrated into the digital realm to augment the preservation of personal memories. 
% We also showcase a prototype facilitating user-created interactivity reconstruction.

% with the goal of reconstructing memory artifacts with interactivity integrated in the digital realm to preserve or expand personal memories, 

% However, existing research has inadequately investigated user expectations regarding the crucial or indispensable aspects of interactivity necessary for activating personal memories.
% In this paper, with the goal of reconstructing memory artifacts with interactivity integrated in the digital realm to preserve or expand personal memories, we conducted a formative study to identify essential interactivity aspects of memory artifacts and then developed a prototype facilitating user-created interactivity reconstruction.

% We first conducted a formative study to identify essential interactivity aspects of memory artifacts and then developed a prototype facilitating user-created interactivity.
% with the goal of replicating these memory artifacts in the digital realm to preserve or expand personal memories, However, research has not sufficiently explored the users' expectation of  interactivity that are more important or indispensable for activiting personal memories. 
% Promptify adds to this research area by contributing a novel interface that positions and clusters images by CLIP em- bedding similarity, providing a more efcient browsing experience than the traditional folder view.

% In line with prior work, we aimed to reconstruct the interactivity of memory artifacts using digital technologies to 

% through new technologies and push the boundary


% Additionally, interactions around physical sensory attributes, such as tactile ____, olfactory ____, hearing ____, and visual ____ triggered by mementos could influence each other and leave a more lasting impact on memories.

% In reminiscence activities, tangible interactions have been investigated that can open up new ways to organize and collective memories mediated by digital photos ____, provoke social conversations ____, and support collective experiences of reminiscence and reflection 

% ____.




% ChatGPT
% Banks and colleagues investigated the advantages of sharing family photographs facilitated by the physical properties of objects (for example, the ability to store and display images) within the home environment, integrating them into daily routines.

% In this paper we describe a concept and working prototype
% called “Shoebox” which aims to explore notions of storage
% and display of images in the home through creating an amalgam of physical and digital affordances.

% Interacting with memory artifacts physically can also inspire individuals to thoughtfully preserve and engage in reminiscence as part of their daily lives ____. 

% In reminiscence activities, tangible interactions have been investigated that can open up new ways to organize and collective memories mediated by digital photos ____, provoke social conversations ____, and support collective experiences of reminiscence and reflection ____.

% ____


% to help users to share photos in reminscence scenarios. 

% Shoebox describe a concept and working prototype
% called “Shoebox” which aims to explore notions of storage
% and display of images in the home through creating an amalgam of physical and digital affordances.
% For example, the physical interactions that occur during an activity, including the tools utilized, the environment where it takes place, and other material elements, can lead to the formation of lasting memories. These memories can easily resurface in everyday life, particularly when one incidentally encounters them while using a physical object for its intended purpose.
% Moreover, the use of physical memory artifacts, particularly common items like tableware, toys, and electronic devices, can powerfully evoke past experiences and memories when people incorporate these objects into their everyday routines, effectively taking them 'back' to those moments in time ____.

% For instance, the tangible interactions involved in an activity, such as the tools utilized, the location where it unfolds, and other physical aspects, can create enduring memories and be effortlessly recalled in daily life, as these experiences can occur spontaneously without deliberate intention.


% Their work revealed that the attention to form, materials, and interaction triggered people to carefully protect the design artifact and led to increased perceived value in the overall digital photo archive itself. 
% Another major area of focus in the HCI community has been in how tangible interactions with digital photo archives can open up new ways to organize and share collective memories mediated by digital photos [32, 67], provoke social conversations [25, 30], and support collective experiences of reminiscence and reflection [25, 32, 44, 48, 60, 61, 65, 66, 71].




% Tangible interaction such as flipping pages (C2) or arranging blocks (C4) can limit distraction in collocated communication and even stimulate collaboration ____.

% The tangible nature of physical objects, including elements you can visually perceive, touch, or engage with, plays a vital role in understanding the essence of memory activities. 
% This physicality may be even more significant than the abstract emotional feelings behind them.
% The materiality of the physical objects such as these tangible interactive elements—things you can see, touch, or interact with—are crucial for grasping what the memory activity is really about, perhaps even more so than abstract concepts themselves

% We conjecture that the essence of an activity, even if social, is better captured through its materiality, i.e. objects and context. Our self-defined photo-based cues also differ from souvenir photos and their self presentation quality. In contrast, we found limited photo-based cues featuring selfies, suggesting that such cues are for exclusive private functional use
% Materiality, in this context, refers to the physical objects and the environment or setting in which the activity takes place. The idea is that these tangible elements—things you can see, touch, or interact with—are crucial for grasping what the activity is really about, perhaps even more so than abstract concepts or the interactions themselves. This perspective implies that the concrete aspects of an activity, such as the tools used, the space it occurs in, and other physical components, are key to understanding its essence or core.
% -> Physical objects and the environment or setting in which the activity takes place, are key to represent an activity in one's memory.
% -> concrete aspects of an activity, such as the tools used, the space it occurs in, and other physical components, are key to understanding its essence or core.
% -> tangible elements—things you can see, touch, or interact with—are crucial for grasping what the activity is really about
% -> Objects and context, which is the materiality of an memory  such as  is essencial for an memory activity, objects and context are the mater



\subsection{Digital and Physical Forms of Personal Memory Archives}
% Digital capturing or reconstruction for personal memory archives have been extensively explored ____.
% Photo or video-based recordings for life experiences are the main formats of digital reconstruction of personal memory archives ____.
% 补充________.
Two primitive forms of digital media used for personal memory archives are photographs and videos. 
To facilitate efficient content creation____, organization____, and access____ processes for personal memory archives, previous research has investigated novel digital designs by extending the original photographs and video forms____.
% The primary methods of digital capturing for personal memory archives currently include photographs and videos ____.
For example, \textit{Rewind}____ proposed to associate one's daily excursion videos with a sequence of street-level images from self-tracked location data, aiming to organize video contents in a geographical for better visualization and access.
\textit{Chronoscope} is domestic technology leveraging temporal metadata in digital photos as a resource to encourage diverse and open-ended experiences when revisiting one's personal digital photo archive ____. 
\emph{Shoebox} offers a digital solution for combining the storage and display of digital images within the home environment ____.
% Although photo and video-based digital captures offer immediate representational meaning and visual accuracy, they often fall short of the symbolic and emotional depth that physical objects associated with memory experiences provide ____. 

Although the above digital capturing methods may offer visual accuracy and immediate representational value, they often lack the symbolic and emotional meanings inherent to physical objects associated with memory experiences ____.
These meanings are essential for enhancing personal memory archives and have a lasting impact beyond the immediate representational value. ____. 
Research has extensively compared the impact of physical objects versus digital photos and videos on memory activation, consistently finding a preference for physical objects ____. 
% This preference underscores the unique value that physical objects bring to the recollection of memories, offering a depth of connection that digital formats struggle to replicate ____.
Physical objects, with their tangible presence, allow for direct physical interaction, capturing the essence of experiences more profoundly than digital alternatives ____.
For instance, Petrelli et al. ____ found physical mementos are highly valued heterogeneous and support different types of recollection compared to digital photos and videos.
____ discovered that it is more common for participants to actively share memories with physical items than digital artifacts such as photos and videos when they are outside the home. 
While physical items play a crucial role in triggering memories, they have disadvantages, such as fragility, degradation, and the need for additional storage space, which increases costs ____. 
To address these challenges, digital preservation becomes important to ensure the long-term safeguarding of these items. Despite their importance, the interactive attributes of physical items have rarely been considered in the digital reconstruction of memory archives.

Our work aligns with previous efforts to achieve digital longevity, ensure the persistence of memories, and facilitate the efficient evocation of memories and emotions ____. 
However, unlike earlier approaches, we enrich the concept of personal digital reconstruction by allowing end-users to integrate physical interactivity into digitally reconstructed items. Additionally, we explore the design space of this reconstruction process across three dimensions: geometry, interface, and embedded content.




% However, physical items have their disadvantages, such as frailty, degradation, and the need for additional physical space, which increases costs ____. This makes digital preservation important to ensure their long-term preservation. 

% Although these physical items and their interactive attributes are essential for memory triggers, they have rarely been considered in digital reconstruction forms of memory archives. 
% Our work shares a similar goal with prior efforts to achieve digital longevity, ensure memory persistence, and facilitate efficient evocations of memories and emotions ____. 
% Still, unlike previous works, we enrich the concept of personal digital reconstruction by enabling end-users to incorporate physical interactivity to digitally reconstructed items. 
% Additionally, we explore the design space of this reconstruction process in three dimensions - geometry, interface, and embedded content. 



% we extend the concept of personal digital reconstruction by allowing end-users to incorporate physical interactivity into digitally reconstructed items. Additionally, we explore the design space of this reconstruction process across three dimensions: geometry, interface, and embedded content.


% focused on achieving high-fidelity reconstruction for physical items, highlighting their physical interactivity. 




% ------creative process for buiding tools
% A surprising finding however is the prevalence of doodles as second preferred cue format. In contrast to photos, the doodles’ content needs to be created. This involves a creative component which most people enjoyed. Doodles are particularly interesting as they are mostly preferred for capturing event’s emotional meaning. They are also used for abstract qualities of the experience which are difficult to capture through readily available cues.
% -> memory artifacts that need to be created instead of directly captured make most people enjoyed. Additionally, they are used for abstract qualities of a certain experience which are difficult to capture through readily available cues. 

% With respect to objects, most of the memorable events consist of human activities. When defining the cues, participants tend to identify key aspects which could stand for the entire activity. Such aspects include materials and objects instrumental for the completion of the activity, or objects representing the result of the activity.
% The preference to use the doodles to express emotions suggests the value of creative and more agentic forms of participation in generating the cues.
% -> memory artifacts tend to represent a set of activity, which includes the interactions between the objects and human
% -> people tend to abstract their moods in the process of 'creating' memory artifacts
% -> creative forms of participation could generate more memory cues

% ____


% -----------------------
% Digital capturing or reconstruction for personal memory archives in people's daily lives have been extensively investigated ____.
% Prior research investigated ways of using videos or photos with textual information to enhance memory ____.
% For example, ____ introduced a tagging system designed to capture and share object-related stories, including their historical context, associations, locations, and the memories of their owners.
% ____ developed an email system that integrated data mining techniques with an interactive interface to stimulate users' memory by analyzing email archives and generating memory cues.







\subsection{Immersive Technologies to Facilitate Memory Reconstruction}
Memory reconstruction is a process where the user records memory elements, archives them in digital forms, and revisits them for recollection____. 
Immersive technologies, typically built upon AR and VR platforms, play an important role in assisting memory reconstruction in different stages ____.
Their immersive display and interaction capabilities stand out as key advantages for memory reconstruction, while the mobility of these devices further enhances their utility in common memory-sharing and communication scenarios ____.



% 3d display
With high-fidelity 3D display technologies, AR or VR could create more immersive and realistic representations of memories ____.
They thus provide historians an invaluable tool for the realistic reconstruction of ancient artifacts, past heritages, and memories, giving them more longevity within the digital realm ____.
For example, Valtolina et al. ____ and Yang et al. ____ were devoted to reconstructing ancient sites in the form of accurate 3D models in mixed reality, enhancing the visitors' understanding, accessibility, and engagement with historical narratives.
____ leveraged VR to create a virtual graveyard as an accurate simulation of the Salla graveyard as possible even with its atmosphere, providing a deeply immersive experience for visitors.

% Additionally, the interaction technologies of AR or VR could enable more user-friendly memory archives operations, such as bring better visualized overview within huge memory collections ____ and presenting methods with flexible and intuitive editing functions for individuals to review and archive their memory contents (e.g., photos, videos, mementos, souvenirs, etc.) ____. 

% 3d contents like scenarios, items, figures, are important for personal memories. 
% interactive techs provided by arvr can augment the presentation of 3D contents, creating a more immersive and intuitive interface for humans. 
% The line of work that involves digitizing memorable physical objects (e.g., 2D [2], and 3D [3] objects) for communication. 
The interaction methods offered by immersive technologies can provide more intuitive and flexible operations for 3D content (e.g., models, scenarios, etc) for non-expert users____. 
For example, \textit{GesturAR} is an end-to-end authoring tool that supports users to create in-situ freehand AR applications to interact with 3D contents ____.
3D content plays an important role that is augmented for memory reconstruction by immersive technologies in prior works. 
% The interactive technologies offered by AR and VR can also significantly enhance operations for memory archives, such as content navigation and management of folder hierarchies, thanks to their advanced graphical visualization capabilities  ____. 
% Also, the memory elements presenting operations are enhanced by the flexible and intuitive interaction of immersive technologies. 
For example, Li et al. demonstrated how AR can be used to support intergenerational memory storytelling by augmenting photos, videos, music, and 3D models that recount past experiences in AR ____.
Kang et al. created hybrid tangible AR souvenirs with different input modalities of AR (e.g., hand gestures and voice) that combine a physical firework launcher and AR models, improving the connection between physical souvenirs and their contexts ____.
Considering the advantages of immersive technologies in terms of high-fidelity 3D display and interaction methods aimed to augment 3D contexts, we pioneer the use of AR in reconstructing personal memorable items, by introducing specifically designed interactions to achieve personalized physical interactivity reconstruction for memory archiving.




% \subsection{Technologies to Simulate Physics-based Interaction}
\subsection{Enriching Physical Interactivity in Virtual Environment}
Physical interactivity in virtual environments allows users to manipulate virtual objects realistically, where the virtual environment responds to the user's behaviors according to physical laws, such as the effects of gravity, friction, inertia, collision dynamics, mechanical structures, and materials ____. This provides users with more realistic and intuitive experiences in virtual environments.
Ideally, achieving the most realistic physical interactivity requires the combined use of a powerful physics engine and fine-grained modeling of 3D assets.

Prior works have explored automatic approaches through ML models to infer 3D assets' materials to model their physical dynamics. 
These methods utilize visual data from the real world to identify materials and incorporate material-aware dynamics, enabling physically plausible interactions for 3D assets.
% In a system that incorporates physics-based interactions, objects behave in a manner that mimics how they would in the real world, allowing for more realistic and intuitive user experiences ____.
% Previous research introduced automatic approaches for reconstructing physical interactivity within objects. These methods detect the material and properties of objects using visual data from the real world and integrate appropriate physics dynamics to render the objects, enabling realistic physical interaction.
For example, \textit{PhysGaussian} seamlessly integrates physically grounded Newtonian dynamics within 3D Gaussians to achieve high-quality motion synthesis ____.
\textit{PhysDreamer} is a physics-based approach that endows static 3D objects with interactive dynamics by leveraging the object dynamics priors learned by video generation models ____.
\textit{VR-GS} enables real-time execution with realistic dynamic responses on virtual objects by developing a physical dynamics-aware interactive Gaussian Splatting in a VR setting ____.
These methods require precise specification of boundary conditions or material properties of the object to be simulated as a pre-requirement in addition to the visual input. 
These boundary conditions are essentially locations and magnitudes of forces, and fixed locations that specify how the interaction will proceed ____. 
Material properties, such as Young's Modulus and Poisson ratio, are also specified.
Combining this information allows the system to predict how the object will respond to forces, i.e., stretch, compress, or even fracture. 
These specifications are traditionally designed by mechanical engineers, based on the model's use case and requirements of the simulation ____. 
We cannot expect end-users to have this specialized knowledge, and thus this is outside the scope of our work.
If there are discrepancies between the simulated interactions and the physical world, it remains difficult for end-users to make effective edits.
Processing more complex objects (e.g., objects with mechanical structures, objects with non-uniform materials) often requires skilled modelers or designers to manually rig models using professional software (e.g., Blender, Maya, etc.), and to integrate physical properties using physical engines embedded within game engines (e.g., Unity3D and Unreal Engine) ____.

Considering that our work is aimed at end-users without specialized knowledge and addresses the need for personalized 3D physical interaction reconstruction, our prototype first introduces a combined manual and automated 3D assets segmentation method ____. Then we employ predefined physical constraints in the AR interface to simplify and abstract the physics engine, making it easier and accessible for users without domain expertise to create physical interactions based on their own understanding of the objects. \revision{As most related to our work in terms of AR functions, \textit{GesturAR} enables users to create interactive hand gestures for virtual objects ____, while \textit{Ubi Edge} allows end-users to customize edges on daily objects as TUI inputs to control varied digital functions ____. Different from their explorations on specific interactive functions, our prototype aims to achieve a more comprehensive pipeline to allow users to reconstruct and customize physical objects in AR in terms of geometry, interface, and embedded content. Our prototype, with its emphasis on realistic reconstruction within the virtual environment, also provides a comprehensive description of future user-defined interactive virtual objects, originating from the physical world.}


% Our prototype, with its emphasis on realistic reconstruction within the virtual environment, also delivers an extensive description of future interactive virtual objects that users can define, originating from the physical world.


% However, our system is more focused on the physical world's tangible ui while preserving its real-world properties and appearances
% The features of the AR system are quite similar to GesturAR[1] and Ubi Edge[2]. GesturAR allows virtual model creation through scanning, segmentation, and adding mechanical constraints via joints, while Ubi Edge supports UI placement and defining virtual functions. The authors should more clearly differentiate their system by analyzing the distinct emphasis between these works.




% \zisu{As most related to our work, gesturar and ubi-edge present xxxx, our system is inspired by gesturar and ubi edge}
% intro / related work (reco) / design
% empirical comparison, 可能交互上
% differences: 1.pipeline更完整,允许用户自己做,有分割,允许用户高度自定义,more integrated,2.我们的idi的概念更完整,不仅有物理分割,还有界面和interface,这个概念是对未来可交互对象的完整描述存在的,

% \zisu{The features of the AR system are quite similar to GesturAR[1] and Ubi Edge[2]. GesturAR allows virtual model creation through scanning, segmentation, and adding mechanical constraints via joints, while Ubi Edge supports UI placement and defining virtual functions. The authors should more clearly differentiate their system by analyzing the distinct emphasis between these works.}





% Therefore, there is still a gap to reconstruct physical interactivity without requirements for technical expertise. To take a further step, InteRecon is designed to enrich the reconstruction of physical-aligned interaction for end-users. It combines an automatic model segmentation methods for users to refer ____, and provide intuitive user-editable methods in AR environment. 
% Considering our research goal is to create a tool for end-users to enable 

% Therefore, there is still a gap to reconstruct physical interactivity without requirements for technical expertise. To take a further step, InteRecon is designed to enrich the reconstruction of physical-aligned interaction for end-users. It combines an automatic model segmentation methods for users to refer ____, and provide intuitive user-editable methods in AR environment. 

% Although these methods are efficient for certain widely studied materials (e.g., elastic, rigid, etc), a single visual channel cannot accurately predict the dynamics of objects without uniform kinematics dynamics, such as objects with articulated mechanical structures or with non-uniform materials ____. 

% the dynamics of objects with heterogeneous materials or objects with hinges, which are more common in reality ____.
% This makes it challenging to handle objects without uniform kinematics dynamics, such as objects with articulated mechanical structures or with non-uniform materials ____.

% Although these methods are efficient for several materials, 
% the lack of ground truth for the object's properties often results in discrepancies between the simulated physical interaction and how humans experience it in the real world ____.


% Although these methods are automated and do not require manual input, the lack of ground truth for the object's properties often results in discrepancies between the simulated physical interaction and how humans experience it in the real world ____.
% Furthermore, these methods are generally applicable to objects with homogeneous materials and are limited to a few predefined materials (e.g., elastic, rigid, etc). This makes it challenging to handle objects without uniform kinematics dynamics, such as objects with articulated mechanical structures or with non-uniform materials ____.


% In the industry, processing these objects with physical interactivity needs skilled modelers or designers to manually rig models using professional software (e.g., Blender, Maya, etc.), and to integrate physical properties using physics engines embedded within game engines (e.g., Unity3D and Unreal Engine) ____.

% Therefore, there is still a gap to reconstruct physical interactivity without requirements for technical expertise. To take a further step, InteRecon is designed to enrich the reconstruction of physical-aligned interaction for end-users. It combines an automatic model segmentation methods for users to refer ____, and provide intuitive user-editable methods in AR environment. 

% The primary issue lies in that vision-only information cannot 

% the lack of material ground-truth data ____, making it difficult for users to make further adjustments if the results are unrealistic or not as expected.
% Furthermore, these methods are limited to a few materials (e.g., elastic, rigid) recognized with visual information and objects with simple overall structures (e.g., consistent kinematics dynamics) ____. 
% However, for objects with complex mechanical structures (e.g., articulation objects), predicting their physical dynamics solely through visual channels is challenging and requires significant learning effort and cost. 

% For complex modeling in virtual environments, skilled modelers or designers are needed to manually rig models using professional software (e.g., Blender, Maya, etc.), and to integrate physical properties using physics engines embedded within game engines (e.g., Unity3D and Unreal Engine) ____.

% Contrasting with these engineering-intensive and expertise-intensive solutions, our system allows non-expert users to reconstruct personal physical objects, while preserving their physical interactivity. 
% As an initial and first attempt to propose a practical and accessible authoring workflow, we implemented the physics-based interaction by the automatic 3D model pre-processing methods ____ while enabling a level of editing freedom for users and also provided alternative personal intuitive authoring solutions in AR without requirements for domain knowledge.


% In contrast to these hardware-intensive solu- tions, our work focuses on developing a hardware-free approach to address object-sharing challenges when the user holds and orients objects towards the camera, with both dynamic live sharing and static non-live storage of physical objects in the virtual space.


% Therefore, 

% we want to minimal the use of professional softwares.


% In contrast, users can easily understand how these objects move because they are familiar with them. 
% Therefore, as an initial and first attempt to propose an authoring workflow for end-users, we used the automatic 3D model pre-processing methods as the auxiliary approach and also provided alternative user authoring methods in AR. 




% we  enables users to add and edit the physical interactivity of their items, with automatic 3D model processing methods serving as the auxiliary approach.
%  to develop an authoring tool for end-users to add physical interactivity for personal items in AR, we combined the automatic and manual methods to achieve user-friendly authoring and 






% Inspired by prior immersive authoring tools tailored for non-expert users ____, we also provided some pre-designed choices of joints to help users easily create mechanical structures without any domain knowledge. 






% 这些自动化的结果生成的是以object为基础的,这些target objects are limited to several materials with a relative 整体的结构,而对于具有复杂机械结构的物体,很难或者需要大量学习成本仅通过视觉通道来预测其物理动力学,而用户却可以轻而易举的知道物体是如何运动的,因为他们拥有这个物体,所以我们希望提供一种authoring workflow来让用户编辑属于自己的物体的physical interactivity.



% 所以这些方法在对于手物交互的physics interactivity上的表现不好
% 对于human interaction trigger, for example, hands, not perform well 

% Additionally, personal items have many types, including different materials, geometry, frictions, and densities, it is still hard to always generate realistic physics dynamics for them based on limited 


% 仅依靠视觉通道分析物理动力学对于personal items来说很难取得很好的效果,



% However, these automatic methods to create physical interactivity for personal items are hard to apply to users who do not have relative domain knowledge due to their limited customized space (e.g., hard to further adjusting the results, limited object sets trained).
% 对结果难以进一步调整
% 有限的



% the goal of our work is to create a tool for non-expert users to create physical interactivity of their personal items by themselves. 
% In this case, it is hard for non-expert users to use 

% adjusting the results of these automatically physics dynamics generation, 


% Physics-based interactions are based on the real world physics dynamics, always describing an object can be interacted and cause physical effects, such as motion and deformation. 


% interactions that cause physical changes in the real world, such as motion and deformation.
% Immersive technologies have been explored to simulate physical-based interactions to improve the realism in virtual environments. 


% physics dynamics including 

% a tedious multi-stage process: constructing the geometry, making it simulation-ready (of- ten through techniques like tetrahedralization), simulating it with physics, and finally rendering the scene







% COULD ADD WHT END USER???

% Inspired by the works above, we are the first to leverage AR to facilitate memory artifacts reconstruction with designed interactions in the immersive environment to enable real-world interactivity of them.


% digitize physical memory artifacts with interactions for reconstructing real-world interactivity on them.

% 3DDocumentation:everythingfromsitesurveytoepigraphy
% • 3DRepresentation:fromhistoricreconstructiontovisualization
% • 3DDissemination:fromimmersivenetworkedworldsto‘in-situ’augmented
% reality.

% Through nostalgic elements triggered by the AR/VR three-dimensional model and video/audio interaction, the feasibility of our integrated system for reminiscence therapy is thus verified. 

% To conquer these problems and to make personalized souvenirs a part of the visiting experience, we create a hybrid tangible Augmented Reality(AR) souvenir that combines a physical firework launcher and AR models. An application called AR Firework is designed for customizing the hybrid souvenir as well as interactive learning in an exhibition in the wild. 


% It is a big challenges for designers to find 


% Presenting or reconstructing personal memory contents is the main problem in 




% xxx We consider one of the big challenges for designers to find solutions to bring better overview in the digital space, without losing mobility of the devices.















% --------------------------
% StoryPlaces authoring tool supports constraints-based approach to creat narrative stories ____.

% ____This paper introduces a user-centric authoring tool that enables common users to transform a static photo into a temporal presentation, or story, which can be shared with close friends and relatives. 


% Firstly, it ofers a detailed refective account that unpacks how we made decisions and intuitive adjustments to attend to the materiality and temporality of a shape-changing artifact that supports everyday interactions ____.


% ____: digital photos extensions

% Pensieve, a system that supports everyday reminiscence by emailing memory triggers to people that contain either social media content they previously created on third-party websites or text prompts about common life experiences ____.


% Augmented-, Virtual- or Mixed Reality technology might be an interesting next step (e.g. Microsoft's HoloLens (microsoft.com/microsoft- hololens/)), as long as designers at the same time innovate existing archiving structures and folder hierarchies, because otherwise, the digital files are as hard to navigate as books in an attic ____. 


% ____To promote in-home photo sharing, we designed Souvenirs, a system that lets people link digital photo sets to physical memorabilia.

% ____memory compass

% For example, ____ introduced a tagging system designed to capture and share memory stories, including their historical context, associations, locations, and the memories of their owners.


% ____museum physical obejcts, ar

% ____ar souvenirs uist

% ____According with this, the paper presents Memodules Framework which enables personal objects as TUIs for memory recollection and sharing





% --------
% \subsection{Immersive Authoring Tools}
% Prior work has investigated immersive authoring tools to facilitate users to create 3D contents ____. 
% For example, ProGesAR____ allows users to quickly prototype proxemic and gestural interactions with actual IoT in augmented reality. 
% By tying static graphics to moving ones, RealitySketch____ allows users to make their static graphics responsive. 
% Based on the intuitive manipulations enabled by immersive technologies, these tools can be quickly learned and used by end-users without any professional modeling, programming, or sketching skills. 
% Following this paradigm, Window Shaping ____ and SceneCtrl ____ empowered the creation of static 3D models and virtual scenes by leveraging the spatial perception of AR devices. 
% ARAnimator ____ and PuppetPhone ____ allow users to author an animation sequence of a virtual character using a smartphone as the motion controller.
% Informed by this line of work, we aimed to develop an end-user authoring tool in AR to reconstruct motions, 3D models, and virtual scenes as the real-world features for physical mementos.

% As defining or reconstructing physical or virtual interactions are logically complex____, end-users might encounter difficulties in directly modifying interactions within the user interface. 
% Moreover, reconstructing or simulating physical objects with an emphasis on their real-world attributes in AR/VR remains relatively unexplored.
% Thus, in the designing of an authoring tool, we sought to design features to enable end-users to reconstruct physical mementos in AR with functions of mapping physical joints, mapping tangible widgets, enabling bare-hand interactions.