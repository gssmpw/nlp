\section{Conclusion}
In this work, we presented the concept of Interactive Digital Item (IDI), emphasizing the reconstruction of memorable personal items while preserving their interactivity, which represented the design goals of reconstructing geometry, interface, and embedded content.
% informed by a formative study aiming to understand the participants' expectations on the important factors that constitute the real-world features of physical mementos. 
IDI is built on the knowledge of a formative study we conducted to understand the participants' expectations on the interactivity features of memorable personal items that wanted to be reconstructed for digital replications.
We developed an AR prototype, InteRecon, which allows users to create and present IDI within MR environments from physical items, including four main categories of functions: 1) Reconstructing 3D appearance, 2) Adding physical transforms, 3) Reconstructing interface, 4) Adding embedded content.
We conducted a two-session user study to assess the feasibility of using InteRecon to create IDI and collecting users' feedback on the challenges, future opportunities,
and applications of IDI. 
We collected quantitative and qualitative feedback and the results revealed that InteRecon is effective, expressive, and enjoyable for creating IDIs. 
Furthermore, the qualitative feedback illustrated that 1) IDIs brought realistic experiences with physical interactivity; 2) The immersive interactions enabled by InteRecon empowered the IDI creation; and 3) IDIs have the potential to enrich personal memory archives.





% evaluate the user experience of utilizing MemoTool to create IDM and gather in-depth ideas of challenges, future opportunities, and applications of IDM and MemoTool.  

% includes four main categories
% of functions to enable end-users to create IDI



% We derived the design goals of reconstructing these three real-world features and present MemoTool, an immersive authoring tool to enable end-users to create their IDMs.

% MemoTool allows users to create and present IDM from the physical memento by: 1) reconstruct the physical appearance, 2) map tangible widgets, 3) customize embeded content, 4) reconstruct the mechanical component, and 5) try out the IDM by free-hands.
% We conducted a two-session user study to evaluate the user experience of utilizing MemoTool to create IDM and gather in-depth ideas of challenges, future opportunities, and applications of IDM and MemoTool.  
% We collected quantitative and qualitative feedback from 16 participants and the results indicated that MemoTool are effective, expressive, and enjoyable for creating IDM.
% Additionally, the qualitative feedback illustrated that MemoTool could reconstruct the realism with real-world features on physical mementos, the ease of MemoTool's interaction approaches, and the customization potentials were rich to extend.



% The results of qualitative feedback illustrated that MemoTool could reconstruct the realism with real-world features on physical mementos, the ease of MemoTool's interaction approaches, and the customization potentials were rich to extend.

% finding design  regarding the essential for reconstructing appearance, interactivity, and embedded content. 
% e present MemoTool, an immersive authoring tool to enable end-users to create their IDMs.









% We also discussed the extensions on the implementation and interactions of MemoTool the envisioned ecosystem of IDM concept.

% We have come up with 28 designs around this interaction technique and conducted two main user studies. The first user study investigated Hand Interfaces in object retrieval tasks, and the second study in- vestigated Hand Interfaces in interactive control tasks. Each study included two baseline techniques, which we drew from prior work and existing applications. We collected quantitative and qualitative feedback from 17 participants and the results indicated that Hand Interfaces are effective, expressive, and fun to use. We demonstrated example applications centering around three domains — entertain- ment, education, and ubiquitous computing. All these efforts have been open-sourced to facilitate future research.




% In this work, we present MechARspace, an authoring tool that en- ables novice designers and DIY-ers to create AR applications based on their physical toys. MechARspace allows users to program cus- tomized toy-AR interactions in-situ by demonstration while using relevant contextual elements as references. We start by compiling the bidirectional interaction model that maps various types of toys’ actions to responsive behaviors of virtual contents. Following this interaction model, we design our immersive visual programming interface so that users can author corresponding interaction modal- ities through simple trigger-action programming. Furthermore, we develop a collection of IoT modules to help designers efortlessly integrate their toys into the AR scene without lengthy electronic prototyping processes. To explore the capability of MechARspace, we demonstrate four groups of applications scenarios. Through a two-session user study, we frst proved our system’s usability and then its utility as a design tool to help impromptu AR-enhanced toy design. Thus, MechARspace provides the HCI community with a unifed framework and an open landscape into future designs of AR authoring tools for toys.