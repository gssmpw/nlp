\section{Discussion}
\subsection{Value and Challenges of Reconstructing Interactivity in Virtual Environments}
Physical items and their interactivity play an important role in human memory, serving as memory triggers and markers \cite{10.1145/1806923.1806924,petrelli2010family,10.1145/3024969.3024996}, which cannot be effectively captured by photos or videos. 
Our user study reveals that IDI can fill this gap, recording physical interactivity for personal items as a novel representation to enrich personal memory.
% Our study reveals that with interactivity reconstructed, IDI can be more alive and realistic, enriching personal memory archives.
While IDI can't fully be an exact equivalent of physical items due to the lack of tactile aspects, it still offers a high level of realism by preserving interactivity. 
% Because of the realism created, users could feel 
% ccccThis bridges the gap between physical and digital experiences, allowing users to engage with the item in ways that feel personal and meaningful.
With this realism enabled, IDI is no longer a "cold and lifeless" virtual artifact.
Instead, it becomes a dynamic, cyber personal asset owned by users, fostering a stronger emotional connection between users and virtual objects compared to traditional recording formats like photos or videos.

Our findings also indicate that the reconstruction quality of IDI in terms of realism, influences participants' perceptions of ownership over IDI.
This aligns with the existing research, which has established a connection between the quality of 3D object reconstruction and human perceptions \cite{scarfe2015using, crete2012reconstructing, 10.1145/1518701.1518966}.
For example, participants tended to form a strong perception of ownership and connection to IDIs (Sec. \ref{fi:Realistic}) when they had been reconstructed with high fidelity, and viewed the IDI as an extension of their personal belongings.
This perception can be enhanced by incorporating more fine-grained reconstruction functions into InteRecon (Sec. \ref{fi:Realistic}).

% \hl{In the current representation of computational objects, 3D models within volumetric videos are the richest forms of expression. As the metaverse evolves, future object representations will require new approaches. 
% IDI can offer end-users a non-parametric object representation, which, through graph structures, relationship networks, and connection mechanisms, can scale to various scenarios.
% For example, a video can be linked to an object, centering on the real object, with its attributes being expandable. Interactivity is difficult to express through meshes (mesh), as it functions more like a data structure, similar to a schema or knowledge graph, working in conjunction with multiple factors to enable flexible expansion.}

Interactivity in the real world is multidimensional, encompassing interactions between people, objects, and environments, forming a non-parametric and complex structure that is difficult to describe with existing methods \cite{gibson1977theory,10.1145/3446370}. While IDI took the first step to identifying some interactivity features of memorable personal items, defining, creating, and editing interactivity in broader virtual environments like the metaverse remains an open question. This challenge not only requires technical support but also involves context, design, and psychological considerations. Interactivity in virtual worlds must dynamically adapt to user behavior and the environment, potentially combining user input and machine learning to create a more flexible, natural, and user-aligned interactive experience.

%在我们的工作中,个人记忆中的交互性物品是由用户定义的,虽然我们从formative study中获取了memorable items的交互性的重要特征,但是在更大的context下,对于现实世界的交互性,它很难被简单概括,它更多的是一种non-parametric 的结构,不仅以物体为中心,还会和环境、交互对象息息相关,所以当重建整个现实世界时,例如在元宇宙的context下,交互性如何被定义、创建、编辑,也是一个open question。



% idi的交互性是以对象为中心的,由终端用户创建的,它更多的是一种物体的affordance,然而,现实世界中的物体的交互性可能会有不和物体相关的参数,更广泛的意义上,它可能是由多个人或物组成的,具有知识图谱等数据结构的信息

% 在今天的计算机对象的表示形式上,现在的表示形式无非是3d模型,体积视频,是最rich的对于世界的表示,这里面未来在原宇宙中,大家需要怎么去表示,idi可以是一种作为对终端用户的非,especially for endusers, 原宇宙的non-parematric化的对象表示,通过图的结构、关系网络、连接结构,可以scalable到不同的场景
% 比如把一段视频link到一个object上,以真实对象为中心的,它的属性是可扩展的,interactivity是很难用mesh去表示的,一个对象的inter如何去定义,inetractivity是类似schema,knowledgegraph的结构



% Such enhancement can evoke the same emotional attachments associated with the physical item.
% Conversely, if the reconstruction fails to capture the item's interactivity, it can lead to an emotional detachment from the digital replication.
% IDI is a digital replication for reconstructing the interactivity of memorable personal items. 
% Our findings indicate that the reconstruction quality of IDI, especially its realism, impacts the participants' perceptions of the ownership of these digital replicas.



% and its impact on individuals' perception of the resultant digital assets is a nuanced aspect of digital replication.

% While IDI is initially designed to aid in personal memory archiving, its potential extends to numerous other fields. Beyond personal items, users can reconstruct a wide range of objects by adhering to the IDI workflow. The core of IDI is its focus on the interactivity of objects, a vital concept that significantly impacts daily life. This interactivity can encompass the characteristics, uses, and affordances of objects, as well as richer contexts like the people, environment, and emotions associated with the object. 
% Beyond simply recording personal memories, reconstructing interactivity fulfills various professional needs, including instant sharing, effective teaching, cost-efficient simulation, or even entertainment.
\subsection{IDI Ecosystem and Application Scenarios}
% 1.ecosystem
% 2.richer content reconstruction for more physical things that cannot be portable
% 1. interactive 

% Consequently, future research could concentrate on establishing such a community and on designing and implementing systems for user-created virtual interactive 3D asset sharing and communication.
% Therefore, how such a community can be formed and how can it contribute to future VR/AR and metaverse applications is an important question for further research.
% With such wide-ranging applications, IDIs present opportunities to form a decentralized community where users, rather than professional modelers, can create and accumulate more personalized and interactive 3D assets in the future.
% 未来的工作可以集中在如何提高虚拟复制品在特定场景里的交流效率,
% Echoing past research on using virtual replicas for remote collaboration and instruction [8, 17, 36, 63], par- ticipants in our study found virtual replicas enhance interpretability. In contrast, representing objects as primitives adds an interpretation layer and hinders efciency. Currently VirtualNexus takes 1–3 mins to create a unique virtual replica. Usability research by Nielsen [35] suggests that a response time over 10 seconds risks losing users’ attention, but can be alleviated by providing a progress bar. As we managed the virtual replica creation in a separate thread behind the scenes, we expect the users can work on other sub-tasks in parallel. For example, we observed some AR participants proceed to sketch on the whiteboard with the remote VR user. Future work can reduce the processing time for object reconstruction by replacing Colmap [45, 46] with refned HoloLens camera poses.

% Virtual Nexus\cite{10.1145/3654777.3676377}
% Thing2reality\cite{10.1145/3672539.3686740}
% doga\cite{10.1145/3654777.3676379}
% fulfills various professional needs, including instant sharing, effective teaching, cost-efficient simulation, or even entertainment.
% 1. things with physical interactivity
% 2. things with specific usages in specific scens 
% 3. not just record for memory, but with the aim of instant sharing, clarified teaching, low-cost simulating, or just for fun. 
% 4. reconstructing interactivity by end-users is a meaningful concept 
% 5. personal items (e.g., legal documents, heirlooms)




% \subsubsection{Richer Personal Content Reconstructed by Individuals}
\subsubsection{IDI as Storage and Dissemination Tool for Personal Memory Archives}
In our user study, participants showed great interest in IDI for its potential and benefits in reconstructing personalized content. As mentioned by some participants, the use of IDI can be extended to represent, store, and disseminate personal memories and experiences associated with personal items.
Key elements like people, scenes, objects, and their interactions are valuable information to be recorded for high-fidelity memory preservation. 
Although previous work has explored using MR to rebuild these elements, such as reconstructing human figures \cite{10.1145/3139131.3139156}, reconstructing human-object interactions in office settings \cite{10.1145/3491102.3501836}, and capturing key contexts from travel \cite{10.1145/3613904.3642320}, a unified representation form is needed to organize all these multi-source data, where IDI could be a good fit.
Based on IDI representations, research questions about how to reconstruct human figures in personal memories without triggering the uncanny valley effect, how to provide more user-friendly and editable 3D personal memory reconstruction interfaces, and how to enable multi-user collaboration and cross-user dissemination to enhance reconstruction are all valuable for further research.

\subsubsection{\revision{Richer IDI Conceptual Extensions}}
\revision{While IDI is initially designed to facilitate personal memory archiving, it can be extended to become a generalized digital interactive asset across various user groups and scenarios. 
Beyond personal items, users can reconstruct a wide range of objects by adhering to the IDI workflow.
The core of IDI is its focus on the interactivity of objects, a vital concept that significantly impacts daily life.
This interactivity can encompass the characteristics, uses, and affordance of objects, as well as richer contexts like the people, environment, and emotions associated with the object.
For example, meaningful documents or heirlooms' interactivity may lie in the stories behind them about the time, place, or people connected to them.
By transforming them into IDIs, individuals can preserve these stories anchored to the object and extend their presence in the digital realm.
Echoing past research on sharing physical environments in broader contexts \cite{10.1145/3654777.3676377,10.1145/3672539.3686740,10.1145/3654777.3676379}, beyond the aim of recording personal memories, our participants also found reconstructing interactivity meets a wider range of user needs, such as instant and interpretable sharing, effective instruction, cost-efficient simulation, or even entertainment (Sec. \ref{broader}).
}
\revision{With such wide-ranging applications, }IDIs offer opportunities to form a decentralized community where users, rather than professional modelers, can create and accumulate more personalized and interactive 3D assets in the future. 
While there are some websites (e.g., sketchfab\footnote{sketchfab: https://sketchfab.com/feed}) for experts to upload or share their 3D models, there is no existing prevalent online 3D asset platform tailored for non-expert users. 
\revision{Therefore, future work can focus on establishing such a community and on designing and implementing systems for user-created virtual interactive 3D asset sharing and communication. }

% \zisu{different user types and informal interview}
% \zisu{Overall, I would like to see more emphasis on how memorable items relate to the system in the revision. Additionally, please clarify the relationship between memorable items and IDI. Are memorable items just one part of what IDI encompasses, or is IDI solely focused on memorable items?}
% \zisu{1AC: One main issue is the limited scope and unclear motivation (R1, 2AC). While the focus on personal memory archiving is understandable, the paper could benefit from exploring broader applications and discussing the real-world utility of this technology beyond niche use cases (R1). Additionally, the significance of the work could be increased by emphasizing its societal or broader implications (2AC).
% }
% \subsubsection{\revision{application scenarios perceived by different user groups}}
% \zisu{Overall, I would like to see more emphasis on how memorable items relate to the system in the revision. Additionally, please clarify the relationship between memorable items and IDI. Are memorable items just one part of what IDI encompasses, or is IDI solely focused on memorable items?}

% 我们的用户整体上认为interecon提供的功能是轻松易使用的,并且interecon的框架和工作流对于重建物理交互性来说是完整的,尽管会有在功能xx里模版不足的缺失。
% 我们认为,开发一种面向终端用户的重建工具需要提供直觉性和人性化的操作。specifically,intereon realize the intuitive operations by providing preview and templates for users to map, and giving them a level of freedom to edit.  
% 我们认为interrecon以后需要enable更多的直觉性操作,来降低工具的使用门槛,使得3d交互性重建就和拍照一样简单易用,这样才能真正让interecon成为一个普遍的个人记忆的工具。



\revision{\subsection{Humane Considerations with Sentimental Value for Digital Item Creation}}
% \zisu{for future research on sentimental value in digital item creation.}
We believe that developing a 3D reconstruction tool for end-users requires providing both intuitive and humane operations. 
Participants find the functions provided by InteRecon to be easy and effortless to use, and the framework and workflow are complete for reconstructing physical interactivity.
Specifically for intuitive operations, InteRecon realized by offering previews and templates for users to map, while giving them editing freedom.
We envision that InteRecon should enable more intuitive operations in the future. 
A good example is demonstration-based programming \cite{10.1145/3472749.3474769,lu2013gesture}, enabling users to define an interaction by just demonstrating this interaction.
Intuitive interactions are to lower the barrier to using InteRecon, making 3D interactive reconstruction as simple and easy as taking a photo. 
% In this way, InteRecon can truly become a universal tool for personal memory reconstruction.

\revision{
However, intuitive operations in 3D environment may cause sentimental hurt due to its realistic effects. 
Since the reconstructed object holds significant personal memories and value for users, they may experience emotional disturbances, such as feelings of discomfort or cruelty, while reconstructing it in a 3D environment (Sec. \ref{fi:immersive}). These feelings are particularly common during operations like segmentation, splicing, deformation, and color alteration. Additionally, hyper-realistic or imperfectly replicated objects that appear temporarily during the process can evoke eerie emotions.
Therefore, we believe that humane designs regarding sentimental value should be integrated throughout the entire interactive process of creating personal digital items. 

An alternative approach of using the 2D interface for editing is mentioned in our user study, which aligns with previous research \cite{seymour2021have,mcmahan2016interaction} and may help reduce emotional disturbances. From this observation, we suggest that the pre-editing process for a 3D model, which might involve disruptive changes for 3d models, can be done in a less immersive setting: In a 2D interface with grid lines and toolboxes, the immersion is reduced, allowing users to focus more on editing without the realistic sensation of harming a cherished item.
However, manipulating a 3D model in a 2D interface may not be as intuitive as in a 3D environment. Therefore, this still needs empirical studies to explore how to seamlessly integrate 2D intermediary steps in the future. The goal is to provide users with a more intuitive and also emotionally acceptable experience when their cherished items are converted into a virtual format.
}

% Humane operations are also expected in the context of personal memory reconstruction. 
% In our study, participants were unwilling to do operations that might cause "damage", even if such damage occurs only in the virtual realm, such as segmenting the model. 
% Because they might lead to distress or unease (Sec. \ref{fi:immersive}), contradicting the preservation intent associated with the reconstruction ~\cite{wang2018effects,lambooij2007visual}. To reduce discomfort when developing memory reconstruction tools, it's crucial to optimize processes that may damage personal items or memories (whether virtual or real), such as providing previews and automatically filling cut surfaces after segmentation. This humane approach should be integrated throughout the entire tool's usage.




% Meanwhile, to address the limitations in defining physical motions, which some users found to be unnatural and cumbersome, we intend to develop more intuitive control mechanisms as an end-user tool. 
% These will likely involve sophisticated hand-object interaction technologies \cite{christen2022d, zhang2023artigrasp}, allowing users to define the body and moving parts of digital items more naturally and accurately. 
% Moreover, for broader types of physical effects, future research incorporating advanced physics engines and algorithms could be instrumental in enriching InteRecon's interaction space by simulating more refined and realistic physical effects, such as flexible/elastic materials \cite{baraff2023large}, hairs \cite{daviet2023interactive}, and fluid \cite{rioux2022monte, li2022fluidic}.
% Moreover, the IDI creation study suggests that in fully user-defined reconstructions for personal items, we should minimize operations that might cause "damage", even if such damage occurs only in the virtual realm, such as segmenting the model. 
% Because they might lead to distress or unease (Sec. \ref{fi:immersive}), as they contradict the preservation intent associated with the reconstruction ~\cite{wang2018effects, lambooij2007visual}.
% Therefore, it is important to optimize the operation processes that potentially destroy the model, such as offering a preview (e.g., presenting the divided model in 2D dimension) and automatically filling the cutting surfaces after a segmenting operation to mitigate any feelings of distress or unease.




\subsection{Incorporating Broader Technical Development}
InteRecon successfully captures basic physical motions and mechanical structures associated with a variety of physical objects. 
Yet, it is imperative to broaden this scope to encompass more complex and varied physical effects. 
\revision{For the reconstruction fidelity, material generation models (e.g., ControlMat \cite{10.1145/3688830}, MatFuse \cite{Vecchio_2024}, TileGen \cite{zhou2022tilegentileablecontrollablematerial}) can be integrated to process complex material reconstruction (e.g., metal, leather, clay). }
As for physical transforms, despite the capability of defining a close set of common joints, future work could generalize to reconstruct physical deformations and motions at different dimensions (e.g., more complex mechanical structures such as pulley blocks and difference types of force such as torque and friction) and scales (e.g., segments with different sizes and resolutions) by incorporating advanced machine learning models \cite{ ao2023gesturediffuclip, lesser2022loki}. 
\revision{For physical-based human-object interaction authoring, Multi-modal Large Language Model (MLLM) combined with physics-based simulation can be integrated to automatically predict 3D objects' interactive dynamics under forces (e.g., SimAnything \cite{zhao2024automated3dphysicalsimulation}, DreamPhysics \cite{huang2024dreamphysicslearningphysicalproperties}, DreamGaussian4D \cite{ren2024dreamgaussian4dgenerative4dgaussian}). Thus, future work in this area can explore interaction capabilities such as dynamic controlling of flexibility, exploring thermal properties of 3D models, dynamic linkages with existing objects in an environment~\cite{stemasov2020mix,10.1145/3654777.3676379}, etc.}

InteRecon's functions of reconstructing electronic device interfaces has been initially set to basic interactions within widgets such as buttons, knobs, screens, and more mechanical triggers. 
To extend the utility and applicability of InteRecon, we propose enhancing the system to support a wider array of electrical functions and components. 
This expansion would involve integrating more complex and diverse electronic mechanisms, allowing users to simulate and interact with a broader range of device interfaces. 
For instance, incorporating touch-sensitive screens, sensor-based interactions, and advanced control systems \cite{tatzgern2022airres, liao2022realitytalk, kim2022spinocchietto} could significantly enrich user experiences. 
By enabling more sophisticated electrical interactions, users can not only recreate but also prototype new device functionalities within the InteRecon environment. 
This would not only enhance the simulation fidelity but also provide a robust platform for education, design, and prototyping activities.




% ----------
% \revision{\subsection{Design Implications for InteRecon}
% In our initial exploration with InteRecon, we have harnessed the power of reconstructing physical artifacts’ motions and electronic device interfaces. This reconstruction has set a foundation for a more immersive and interactive experience.
% However, there is substantial room for enhancement and expansion in these domains.

% % \paragraph{Physical Artifacts: Motion Generalization}
% InteRecon successfully captures basic physical motions and mechanical structures associated with a variety of physical objects. 
% Yet, it is imperative to broaden this scope to encompass more complex and varied physical effects. 
% As for joints and physical transforms, despite the capability of defining a close set of common joints (e.g., Table\ref{tab:joints}), future work could generalize to reconstruct physical deformations and motions at different dimensions (e.g., more complex mechanical structures such as pulley blocks and difference types of force such as torque and friction) and scales (e.g., segments with different sizes and resolutions) by incorporating advanced machine learning models \cite{ ao2023gesturediffuclip, lesser2022loki}. 
% Meanwhile, to address the limitations in defining physical motions, which some users found to be unnatural and cumbersome, we intend to develop more intuitive control mechanisms as an end-user tool. 
% These will likely involve sophisticated hand-object interaction technologies \cite{christen2022d, zhang2023artigrasp}, allowing users to define the body and moving parts of digital items more naturally and accurately. 
% Moreover, for broader types of physical effects, future research incorporating advanced physics engines and algorithms could be instrumental in enriching InteRecon's interaction space by simulating more refined and realistic physical effects, such as flexible/elastic materials \cite{baraff2023large}, hairs \cite{daviet2023interactive}, and fluid \cite{rioux2022monte, li2022fluidic}.
% Moreover, the IDI creation study suggests that in fully user-defined reconstructions for personal items, we should minimize operations that might cause "damage", even if such damage occurs only in the virtual realm, such as segmenting the model. 
% Because they might lead to distress or unease (Sec. \ref{fi:immersive}), as they contradict the preservation intent associated with the reconstruction ~\cite{wang2018effects, lambooij2007visual}.
% Therefore, it is important to optimize the operation processes that potentially destroy the model, such as offering a preview (e.g., presenting the divided model in 2D dimension) and automatically filling the cutting surfaces after a segmenting operation to mitigate any feelings of distress or unease.

% The InteRecon system's capability with electronic device interfaces has been initially set to basic interactions within widgets such as buttons, knobs, screens, and more mechanical triggers. 
% To extend the utility and applicability of InteRecon, we propose enhancing the system to support a wider array of electrical functions and components. 
% This expansion would involve integrating more complex and diverse electronic mechanisms, allowing users to simulate and interact with a broader range of device interfaces. 
% For instance, incorporating touch-sensitive screens, sensor-based interactions, and advanced control systems \cite{tatzgern2022airres, liao2022realitytalk, kim2022spinocchietto} could significantly enrich the user experience. 
% By enabling more sophisticated electrical interactions, users can not only recreate but also prototype new device functionalities within the InteRecon environment. 
% This would not only enhance the simulation fidelity but also provide a robust platform for educational, design, and prototyping activities.}
% ------------

% Moreover, the IDI creation study suggests that in fully user-defined reconstructions for personal items, we should minimize operations that might cause "damage", even if such damage occurs only in the virtual realm, such as segmenting the model. 
% Because they might lead to distress or unease (Sec. \ref{fi:immersive}), as they contradict the preservation intent associated with the reconstruction ~\cite{wang2018effects, lambooij2007visual}.
% Therefore, it is important to optimize the operation processes that potentially destroy the model, such as offering a preview (e.g., presenting the divided model in 2D dimension) and automatically filling the cutting surfaces after a segmenting operation to mitigate any feelings of distress or unease.
% By doing so, the reconstruction can maintain the integrity and emotional value of memorable personal items on their digital counterparts, making them a cherished extension of the owner's personal memory collection.




% sharing
% beyond physical objects: scenarios, humans, site-seeings, 




% This ecosystem might feature various interconnected roles that collectively facilitate ecosystem building.
% We envision the IDI creation process as one that incorporates modular immersive interactions. Initially, designers and developers could create a range of reusable templates for editing 3D content.
% Then, the elements within the categories of reconstruction aspects-geometry, interface, and embedded content-should be durable, modular, and immediately operational.
% For example, a database of virtual widget sets could be enriched by developers to facilitate the reconstruction of a wider array of electronic devices. 
% Furthermore, individual users could use InteRecon with various templates to reconstruct memorable items and moments of their own. 
% After posting to the platform, further feedback from end-users can foster the growth of an organic virtual community, thereby transforming the platform into a hub for life-logging and content sharing.

















% -----------------------------
% \subsection{Richer Collaborative Interaction Support for IDI Sharing}

% Our study reveals a strong desire among participants to share their IDI with its associated memories (Sec. \ref{fi:potentials}).
% In the future, IDI sharing might potentially be a common scenario in which multiple people will engage.
% % Therefore, a collaborative user need for IDI sharing naturally emerges.
% Therefore, the need for collaboration interactions in these scenarios during IDI sharing naturally emerges.
% In this scenario, the challenge remains in designing support for IDI sharing that incorporates collaborative updating or editing interactions.
% Prior work has touched upon collaborative behaviors of incorporating physical mementos in reminiscent activities.
% For example, \citet{west2007memento} discussed activities of collaborative editing for the physical scrapbook with digital elements that could evoke memory sharing among older adults.
% \citet{li2022supporting} explored the roles of memento in a cooperation process in which both the storytellers and listeners are actively involved.
% Considering that IDIs represent 3D digital replicas with physical interactivity reconstructed from physical items, the novel form presents new opportunities for collaborative interactions in reminiscence activities.
% For example, how individuals could collaboratively edit or update IDI within Mixed Reality environments. 
% Another question is how users might present the interactive features of IDIs in MR to others. Would there be comparisons between IDIs and their physical counterparts?

% Additionally, our user study reveals that IDI has the potential of being in-situ reconstructed for objects with memorable moments, whether during travels or in settings outside the home in Sec. \ref{fi:potentials}.
% Previous research has demonstrated that MR applications can offer in-situ support for replaying significant events \cite{10.1145/3610888,10.1145/3491102.3501836}.
% With this new paradigm of life recording, will people be more active in sharing their memories using IDIs?
% Future investigations should explore the ways in which individuals use IDIs to chronicle their experiences anytime and anywhere.
% Additionally, it would be worthwhile to explore different roles involved in sharing IDIs: could listeners propose more creative interactions for IDI within MR environments to enhance communication between the sharer and the listener?

% --------------

% \subsection{Ecosystem for Extended Digital Content}
% % extend persoanl memory to personal life-logging
% Personal digital content, commonly photos and videos, is popular with individual users to create and share via social media platforms as a life-logging method.
% As a further step from 2D media, IDIs signify a shift from static 3D content created by professional modelers to a form of more informative, customized, and interactive 3D content. 
% While there are some websites (e.g., sketchfab\footnote{https://sketchfab.com/feed}) for experts to upload or share their 3D models, there is no 3D content social platform tailored for non-expert users.  
% Therefore, a trend emerged for sharing IDI via social media platforms (Sec. \ref{fi:potentials}).
% To address this gap, a more accessible 3D content ecosystem is needed to extend the presentation formats of 3D content, making it a more important role in enriching diverse personal content.
% This ecosystem might feature various interconnected roles that collectively facilitate ecosystem building.


% We envision the IDI creation process as one that incorporates modular immersive interactions. Initially, designers and developers could create a range of reusable templates for editing 3D content.
% Then, the elements within the categories of reconstruction aspects-geometry, interface, and embedded content-should be durable, modular, and immediately operational.
% For example, a database of virtual widget sets could be enriched by developers to facilitate the reconstruction of a wider array of electronic devices. 
% Furthermore, individual users could use InteRecon with various templates to reconstruct memorable items and moments of their own. 
% After posting to the platform, further feedback from end-users can foster the growth of an organic virtual community, thereby transforming the platform into a hub for life-logging and content sharing.


% 1、内容形式上:照片/视频非常普及 -- 3D有少量的专业平台 -- IDI我们首先提出
% 所以未来的趋势应该是,一个是让3D平台更accessible、更普及;一个是继续拓展3D的表达形式,使得3D内容真正能在交互/XX之类的场景发挥作用

% As the realism created by InteRecon, our participants have a sense of ownership of IDI and recognize IDI as a personal digital content or asset (Sec. 5.4.1). 
% Furthermore, IDI was believed more informative than photos or videos that can capture and replicate diverse interactions, enhancing the impression of the personal item in memories (Sec. 5.4.2).  
% Due to the realism and the more informative characteristics of IDI, there emerged a natural trend that the IDI could be a new internet content that can be created, shared or even purchased via a social media platform.
% While there are some websites (e.g., sketchfab\footnote{https://sketchfab.com/feed}) for experts to upload or share their 3D models, there is no 3D content social platform tailored for non-expert users. 
% Therefore, we propose to build an ecosystem for richer 3D content to empower life-logging and enrich diverse personal content. 
% This ecosystem might feature various interconnected roles that collectively facilitate ecosystem building.





% 1. 模块化
% 1. for end-users' ecosystem
% 2. for among users from diverse backgrounds (e.g., modeling, animation, programming, designing, etc.) ecosystem (generality)

% Based on the results of the qualitative feedback, we learned that users have high expectations of building an interactive virtual community of IDM.


% ----------------
% \subsection{Concept Development and the Ecosystem Creation of IDM}
% % 1. for end-users' ecosystem
% % 2. for among users from diverse backgrounds (e.g., modeling, animation, programming, designing, etc.) ecosystem (generality)

% % Based on the results of the qualitative feedback, we learned that users have high expectations of building an interactive virtual community of IDM.
% Digital assets have been prevalence with the development of Metaverse with VR/AR technologies due to their abilities in modeling, content creation and presentation \cite{cannavo2020blockchain,truong2023blockchain}.
% Informed by the results of our user study (Sec.~\ref{sec: Qualitative Feedback --- Expectations on the virtual community creation of IDM}), we could found our participants have a sense of ownership of IDM and hope the IDM could be indexed in their memory archives.
% The properties of the ownership and indexing are similar of digital assets \cite{banta2016property,austerberry2012digital}, thus we envisioned IDM could also be a new type of digital assets owned by individuals, or even shared or purchased by people. 

% To facilitate the dissemination and sharing of IDM, we propose creating an ecosystem around IDM and MemoTool in the future.
% This ecosystem should include a complex environment of interrelated software applications (e.g., the application works for converting 2D content to 3D, the application works for specially developing extensions of programming interactions for IDM, etc.), a virtual user community to share resources, an extension marketplace, and so on.
% These components could enroll developers, designers, and other users with various backgrounds to create, share, develop, and purchase IDM.
% Specifically, technicians, such as designers and developers, can generate and distribute functional applications or resources within the IDM framework with a lower learning curve within this ecosystem. 
% This, in turn, contributes positively to the development of the IDM ecosystem.
% Users could provide feedback, report issues, and create a network effect by promoting the ecosystem's adoption.
% Additionally, it is essential to ensure the privacy and security of IDM in order to alleviate user concerns \cite{guseva2020conceptual, zhang2010security}.




% contains the virtual community of IDM that allows end-users, developers, and designers to communicate and share objects and functionalities is worth building for future work. 
% The ecosystem provides end users with a more intuitive and interactive experience than existing digital ecosystems (e.g. video content~\cite{kallinikos2011video}, cultural heritage~\cite{dochev2019towards}, etc.), as end-users can share the mesh they scanned. 
% For non-technician users (e.g. developers, and designers), they can also easily create and share functional applications under the IDM framework without much learning effort, which could also enhance productivity and efficiency. 


% For future work, as we have the functionalities of creating and sharing interactive digital assets of memory, our ecosystem should also focus on privacy and security to eliminate the concerns of users~\cite{guseva2020conceptual, zhang2010security}.


% Marketplace or Repository: Some ecosystems have a centralized marketplace or repository where users can find and install plugins, extensions, or apps created by third-party developers. This makes it easy to customize and extend the core software.

% Software ecosystems often have a user base or community that actively uses, supports, and sometimes even contributes to the ecosystem. Users provide feedback, report issues, and create a network effect by promoting the ecosystem's adoption.



% Digital assets have been prevalence with the development of AR/VR due to the xxx xxx and xxx. 
% Our research illustrated the 
% We envision the concept of IDM could be included and potentially be owned through the format of digital assets.  
% From our research, we 


% We believe that IDM can serve as digital assets owned by individual or community in several aspects. First is emotional value, These mementos often hold significant emotional value, and their interactivity can help to evoke and enhance those emotions, making them even more valuable to individuals. Second, sharing and connection. In some cases, interactive digital mementos can have monetary value, especially if they are rare or highly sought after. They can be sold or traded as collectibles. We envisioned IDM could serve as the tangible digital assets which can be shared or even purchased by people.  


% \subsection{IDM and digital assets}

% 加上digital assets 和 idm的关系: 我们希望idm在未来可以被认可为一种个人的数字资产,因为其具有以下价值,xxx,we will follow this direction to explore the concrete advantages of IDM. 





% \subsection{Digitization for more than personal archives completed under the workflow of MemoTool(application)}
% sample scenarios: cultual heritage, intangible cultural heritage (handicraft), 
% 更多的tools more than memory  % 1. combine the visiual and interaction authoring tools
% \subsection{More than IDM for Personal Archives}

% While the archives of human memory is important, we hope the more application scenarios could be incorporated into the future use of MemoTool with the concept of IDM being extended. 

% For example, in terms of cultural heritage \cite{}, digitization is also an effective method to preserve them, more importantly, some intangible cultural heritages contains skills, more broadly, interactions, simply 3D reconstruction could not satisfy the preservation of the detailed interaction and the causality in completion. 

% We proposed a new way to 'model' physical mementos with all the context. We build a bridge between virtual and physical world and enable end-users to 'model' their cherished mementos without the need of professional modeling devices. 









% While some literature has delved into collaborative reminiscence activities, with importing the new paradigm of memory triggers-IDI, it should be investigated that how people 


% these activities would introduces unique complexities due to the varied roles users may adopt

% Our current work explores ways to allow users to create IDI, but the further functions to support better reminiscenet of these related 

% Through our study, we explored the dynamics of creating Interactive Digital Items (IDI) in an immersive environment, particularly focusing on collaborative efforts. 

% We discovered a strong desire among individuals to share their IDIs.
% This IDI sharing scenario will naturally foster new interactions (among people, and between people and IDIs), stories, and memories. 
% This sharing context opens up intriguing possibilities, such as whether the interactions between people and IDIs could also be recorded or reconstructed. 
% In scenarios where multiple people collaborate on reminiscence, the challenge of how to facilitate the updating or editing of IDIs in a collaborative setting arises. While some literature has delved into collaborative 3D content editing within immersive environments, the specific case of IDIs introduces unique complexities due to the varied roles users may adopt. For instance, differences in interaction and editing rights could emerge between the IDI's owner and those with whom the IDI is shared. This distinction highlights the need for nuanced approaches to support collaborative editing and interaction within the context of IDIs, suggesting a fertile area for further research and development.



% Our current work explores ways to allow users to reconstruct the digital copy of a physical memento (artifacts) and map real-world features to it. 
% Mementos can remind users of related people or events in the past. 
% Our current work has not investigated further functions to support better reminiscent of these related people or events. 
% The HCI community has explored ways to help people reminisce with physical artifacts. 
% For example,~\citet{10.1145/3027063.3052756} has explored the role of physical artifacts in social dynamics by proposing Memory Dialogue, and~\citet{10.1145/2470654.2466453} has investigated the object's ability to provide access to an intellectual work. 
% These works have important implications for how to improve memory for physical artifacts in terms of social connection, and can also serve as a future extension of the concept of IDM.
% \subsection{Future Technical Development of InteRecon}
% In our initial exploration with InteRecon, we have harnessed the power of integrating physical artifacts' motions and electronic device interfaces.
% This integration has set a foundation for a more immersive and interactive experience. 
% However, there is substantial room for enhancement and expansion in these domains.

% % \paragraph{Physical Artifacts: Motion Generalization}
% InteRecon successfully captures basic physical motions and mechanical structures associated with a variety of physical objects. Yet, it is imperative to broaden this scope to encompass more complex and varied physical effects. 
% As for joints and physical transforms, despite the capability of defining a close set of common joints (e.g., Table\ref{tab:joints}), future work could generalize to reconstruct physical deformations and motions at different dimensions (e.g., more complex mechanical structures such as pully blocks and difference types of force such as torque and friction) and scales (e.g., segments with different sizes and resolutions) by incorporating advanced machine learning models \cite{ ao2023gesturediffuclip, lesser2022loki}. 
% Meanwhile, to address the limitations in defining physical motions and mechanical structures, which some users found to be unnatural and cumbersome, we intend to develop more intuitive control mechanisms. 
% These will likely involve sophisticated hand-object interaction technologies \cite{christen2022d, zhang2023artigrasp}, allowing users to define the body and moving parts of digital items more naturally and accurately. 
% Moreover, for broader types of physical effects, future research incorporating advanced physics engines and algorithms could be instrumental in enriching InteRecon's interaction space by simulating more refined and realistic physical effects, such as flexible/elastic materials \cite{baraff2023large}, hairs \cite{daviet2023interactive}, and fluid \cite{rioux2022monte, li2022fluidic}.

% to understand better and replicate the nuanced physics of real-world interactions 
% By doing so, users can expect a more refined and lifelike simulation of physical motions, enabling them to recreate and interact with digital replicas that behave more consistently with their real-world counterparts. 

% \paragraph{Electronic Devices: Interface Expansion} 

\subsection{Limitations}

Our work took a further step in investigating interactivity-aware reconstruction of personal items in the MR environment, but it still has limitations that motivate potential future research directions.

Firstly, we believe integrating a more powerful physical simulation engine would improve the realism of InteRecon's hand-to-object interactions, including fine-grained simulation for furry textures or soft bodies and detailed modeling of human hand joints. 
Thus, the collision, touching, or other random interactions between hands and virtual objects will be more realistic. 
Furthermore, it is noteworthy that participants encountered initial challenges when learning to navigate the AR interface in Hololens2 by using hand gestures for selecting models or buttons, partly due to lower pinching gesture accuracy. Therefore, a more robust hand-tracking algorithm should be implemented to improve InteRecon in the future.

The creation of virtual assets (e.g., embedded content, widgets) also potentially be augmented by AI-based generation technology by text prompt \cite{song2023consistency, zhang2023amphion, liu2024sora} or image prompt \cite{podell2023sdxl, zhu2023minigpt, wang2023dreamvideo}. For example, users can leverage text or pictures to generate the corresponding 3D models if they do not have the physical item at hand. 
However, the trade-off between AI generation and the freedom of user customization needs to be further explored and validated through experiments, as personal items can contain many unique and individualized features.

% Our user study discovered that users often prefer to segment areas on models where functionalities are present (such as widgets on electronic devices or joints on humanoid or animal-shaped toys). 
% This process could be enhanced by incorporating state-of-the-art AI-based 3D object semantic segmentation models \cite{qian2022pointnext, chen2024pointgpt, fang2024explore}, eliminating the learning curve for users when performing segmentation. Additionally, further experiments are needed to observe whether users tend to segment certain non-semantic areas when reconstructing their own items.


% scalability

% Secondly, the model segmentations can be augmented by AI algorithms. For example, the 3D segmentation model based on functionalities or affordances, such as xx, xx, can be embedded into InteRecon to relieve the user's workload.









