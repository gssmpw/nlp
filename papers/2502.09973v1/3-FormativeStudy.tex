\section{Identifying Key Interactive Attributes for Memorable Items}
We first conducted a formative study to identify the key physical interactivity attributes of memorable items with the aim of reconstructing interactive digital replications.
% The results of this study informed the design choices we made for our further concept proposal and prototype design. 

% Furthermore, it is not clear whether participants were referring to the actual objects they value as memories and their actual behavior, or the hypothetical concept. 
% A lot of the discussion revolves around cassette players, record players, TVs with physical knobs, but the study participants average age was 26 years (SD = 4.8), i.e., the likelihood that many of them grew up with those devices is minimal. Future versions of the paper should make clear which comments were based on discussion about future usage, or participants’ lived experiences.

% - “All participants highly recognized the value of memorable personal items, showing their willingness to share these items and digitize them.” When filtering participants, did the authors remove people who do not recognize the value? Furthermore, I think this system may benefit or target a specific group of people that often digitize things with current technology or traditional ways, such as those who take diaries, or record their moments, rather than people who merely recognize the value with willingness to share.

\subsection{Participants and Procedure}
% Ten participants with a variety of ChatGPT experiences were re- cruited, including 2 first-time users, 6 casual users who are famil- iar with ChatGPT, and 2 experienced users who use it daily with advanced prompting techniques and have developed applications using OpenAI’s API. The study sessions were conducted over Zoom for an hour each, and participants received 15 USD as compensation.

% We recruited ten participants from a local university who possessed personal memory artifacts. These individuals were eager to present their items to our researcher and expressed interest in having them digitized.




Ten participants were recruited from a local campus (5 male, 5 female; age: avg = 25.80, std = 4.83). 
% All participants highly recognized the value of memorable personal items, showing their willingness to share these items and digitize them.
%of eager to present their items to our experimenter and were interested in digitizing them (5 male, 5 female; age: avg = 25.80, sd = 4.83).
We conducted a semi-structured interview to investigate participants' expectations on the key interactivity attributes of memorable personal items they wanted to preserve in reconstruction.
% in the context of creating their digital counterparts.
% The study was conducted in person. 
Each interview session took about one hour and each participant received financial compensation based on local standards. 

After completing the consent form and demographic questionnaire, participants were asked questions on 1) the metadata (names, creation or ownership dates, stories behind, types, etc.) of the physical memorable personal items brought by them, 2) key factors, especially related to the real-world interactivity, that make these memorable personal items meaningful to participants' memories, 3) the detailed expectations when converting the memorable personal items to digital replications, and 4) the potential usage scenarios of the digital replications.
The participants were also asked to use pencils and papers provided by experimenters to sketch their ideas, especially for the detailed expectations when converting the memory artifact to a digital replication.
The experimenter took field notes and video-recorded the whole interview.
% and each interview session's video was recorded. 

\subsection{Results}
We collected 1) video recordings of the interviews (10 hours in total), 2) field notes taken by the experimenter, and 3) sketches created in the interviews from all participants.
% the data including approximately ten hours of video recording from ten participants in total, the field notes from the experimenter and the sketches created by participants from the interview. 
The video recordings were transcribed to paragraphs using a commercial ASR system (iFLYTEK6\footnote{iFLYTEK6: https://www.iflyrec.com/zhuanwenzi.html}) and checked by the research team for correctness. 
The data were analyzed using the reflexive thematic analysis method \cite{braun2006using}. Below we summarized the interview results.

\subsubsection{R1: Two prominent types of memorable personal items}
Two types of memorable personal items were frequently mentioned by participants (N=10), which are physical artifacts (non-electronic devices) (e.g., diaries, albums, toys, boxes to keep physical photos, cards, ornaments, artworks, etc.) and electronic devices (e.g., music players, cassette/record players, game controllers, game consoles, instant cameras, film cameras, etc.). 
Both types of memorable personal items were mentioned to require a significant need to reconstruct their physical interactivity features.
For example, P6 mentioned, \engquote{I cherish my old MP3 player, which was my companion for several years. Unfortunately, it's now broken. I would like to recreate a digital copy that can also play the music from my damaged MP3.}
Our participants emphasized that these items hold value not only because they evoke memories when seen but also because they were used or interacted with regularly in the past.
When these items break or become obsolete, users may lose the majority of their traces in memories.  

% Both types of mementos require a significant need for reconstructing their real-world functionality. For instance, participant P6 mentioned, "I cherish my old MP3 player, which was my companion for several years. Unfortunately, it's now broken. I would like to recreate a digital copy that can also play the music from my damaged MP3."

% Our study participants emphasized that these mementos hold value not only because they evoke memories when seen but also because they were used or interacted with regularly in the past. When these mementos break or become obsolete, they risk losing a significant part of their memory association.

% Physical artifacts take the form of souvenirs, crafted by the individuals themselves or given as meaningful presents by friends and family. 
% Digital devices become the constant companions during certain phases of one's life journey. 


% We also found two concrete prominent types of mementos that most require the reconstruction of real-world features: obsolete digital devices (e.g., xx) and physical artifacts (e.g., xx), otherwise, they may lose the majority of their traces in memories. 


\subsubsection{R2: Physical traces and transformations for memorable personal items}
% unique usage marks
The usage traces found on memorable personal items revealed the past interactions between the artifact and the owner. 
The personal traces also represent the ownership of the memory artifact and distinguish the objects with personal memories from brand-new ones. 
% The personal usage traces left on the physical appearance of mementos are also a crucial factor in comprising the real-world features---usage traces and significance of memorial represent the ownership of the memento and distinguish objects with personal memories from brand-new ones. 
Most participants (N=9) mentioned that scratches and traces of wear and tear on mementos can swiftly trigger associated memories. 
For instance, a scratch might symbolize \engquote{a past misuse} (P1), and the traces of wear and tear could indicate a frequently used part of the object. 
% When being asked to digitize the memento, participants highlighted the preservation of usage marks in the digital world because \engquote{In this era, industrial products can be easily purchased brand new, yet only memory artifacts imprinted on my personal usage marks hold value in terms of preserving memories.} (P8).

For physical artifacts, transformations triggered by motions also contribute to making them more distinct and memorable, such as \engquote{adjusting a Transformer's arms to change its pose} (P8), \engquote{loading a toy gun} (P2), or \engquote{touching the wind chime to make it sway} (P9), and so on.
When discussing digital replicas, participants wanted these copies to maintain the ability to undergo transformations through gestures or motions, emphasizing that \engquote{digital versions should preserve physical interaction like their real counterparts instead of being static displays in museums.} (P9). 

% When discussing digital replicas of mementos, participants wanted these copies to maintain the ability to undergo changes through gestures or motions, highlighting that "digital versions should preserve the physical interaction characteristic of their real counterparts, rather than being static displays."




% \textit{``In this era, industrial products can be easily purchased brand new, yet mementos imprinted my personal usage marks hold value in terms of preserving memories.''(P8)}. 

\subsubsection{R3: Distinct Interfaces of memorable personal items}
Special and vintage interfaces of electronic devices (e.g., music players, cassette/ record players, game controllers, game consoles, instant cameras, film cameras, etc.) were also mentioned to enhance the concreteness and tangibility of personal memories.
Participants tend to reminisce about the distinct ways of engagement and the interface design of vintage electronic devices, together with their interactions with the tangible widgets like \engquote{gaming console directional pads} (P10), \engquote{vintage record player sliders and knobs} (P1), and \engquote{distinctive music player buttons} (P3). 
However, with the advent of more advanced technologies like smartphones, these physical widgets and their interface designs may become obsolete. 
P3 observed that \engquote{encountering older TVs, which allow channel switching through a rotatable knob, has become increasingly rare.}
Furthermore, these widgets are often triggered by their internal mechanical structures prone to damage, \engquote{making them easily breakable} (P2, P3, P8). 
For these reasons, the participants stressed the crucial importance of preserving the unique interfaces of electronic devices. 

\subsubsection{R4: Embedded content of memorable personal items}
% 1)for digital devices
% 2)for physical objects with mechanical components
The embedded content of memorable personal items was viewed as cherished assets by participants and needed digital preservation. 
% The embedded storage of memory artifacts was viewed as cherished assets by participants
For artifacts of electronic devices, songs in music players, video games and software in-game consoles, and old photos in a film camera, were mentioned by almost all participants to be the important part of their memory. 
However, the continuous iteration and updates of devices have made it challenging to \engquote{access these older contents on newer devices.} (P1). 
If the old devices were accidentally damaged, \engquote{these contents might become permanently inaccessible. } (P6). 
For physical artifacts, the embedded content included contextual details that evoke and symbolize significant places, times, things, people, and experiences, such as \engquote{the game mechanics of card games} (P6), \engquote{the rules of using a Kendo sword \footnote{Kendo: https://en.wikipedia.org/wiki/Kendo}} (P2) and \engquote{the usage scenario of a wooden table} (P3). 
For example, P3 mentioned, \engquote{Every time I see the table, I'm reminded of the experiences from my first year at university.}. 
Seven participants expressed the desire to incorporate the embedded content into the mementos' digital copies.
% This inclination arises from the fact that such content is prone to be lost and can be quite implicit (e.g., the contextual information for an old sword), making it challenging to preserve and access for a long time.
This preference arises from the vulnerability of such content to loss and its often implicit nature (e.g., the contextual details of an ancient sword), which poses challenges to its long-term preservation and accessibility.

\subsection{Summary}
% We conducted a formative study to identify the key physical interactivity attributes of memory artifacts with the aim of reconstructing interactive digital replications.
The formative study uncovers participants' expectations regarding the key physical interactivity attributes of memorable personal items when creating their digital counterparts.
\textit{Physical artifacts} and \textit{electronic devices} emerge as two significant types of memorable personal items that are instrumental in evoking personal memories.
Participants noted that the \textit{physical traces and transformations} symbolized the memorable personal items' uniqueness, linking them closely to personal memories. 
Preserving the unique and vintage \textit{interfaces of older electronic devices} is essential for reinforcing the memories' concreteness and tangibility. 
Furthermore, the \textit{embedded content} is highly valued by participants and crucial for digital reconstruction.
Overall, the findings suggest the need to explore user-oriented and memory-evoked digital replications of memorable personal items, emphasizing their physical interactivity.

