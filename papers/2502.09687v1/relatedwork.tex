\section{Related work}
\subsection{Persuasion by LLMs}

The extensive utilization of LLMs in a multitude of tasks has rised significant concerns regarding their potential to generate detrimental outputs, including discrimination, exclusion, toxicity, information hazards, misinformation, malicious uses, and harm to human-computer interaction \cite{weidinger2021ethical}. 

Recent studies have identified manipulative content produced by LLMs as a function of detected personality of a person interacting with the model \cite{mieleszczenko2024dark}. Other studies \cite{goldstein2024persuasive,karinshak2023working} showed that LLMs can be as persuasive as humans, both in writing extensive articles or short texts. 

The capabilities of LLMs are linked to both their knowledge-driven stylistic approach and their integration of moral-emotional language \cite{carrasco2024large,breum2024persuasive,wilczynski2024resistance}.

\subsection{{Persuasion prompting}}

In previous studies some authors prompted the models with suggestions which persuasion tactic should be included in the output\cite{carrasco2024large,zeng2024johnny,pauli2024measuring}.  These approaches aimed at enhancing the effectiveness of generated content by incorporating rhetorical strategies such as emotional appeals, logical arguments, or credibility cues \cite{wilczynski2024resistance}.
Other notable works encompassed categories of persuasion as rational or manipulative \cite{pauli2024measuring} or focused on the characteristic of communication, e.g., static persuasion, interacting with LLMs, or interacting with humans \cite{jones2024lies}.  
These studies mainly focused on complex persuasion techniques, neglecting basic tactics, such as emotional and rational persuasion, which have been well documented in the literature \cite{rosselli1995processing,miceli2006emotional}. 

\subsection{Principles of social influence}

Widely recognized conception of social influence was proposed by Robert Cialdini, who argued that all influence attempts fall into one of six principles, i.e., commitment and consistency, reciprocity, scarcity, liking and sympathy, authority, and social proof \cite{cialdini2021influence}.  
The principle of \textit{commitment and consistency} means that people can use their natural tendency to remain consistent in their decisions and behaviors.  If an individual succeeds in forcing an initial commitment on someone, it will be much easier to persuade that person to meet subsequent demands. The principle of \textit{reciprocity} states that one should always reciprocate for what one receives from someone. The ability to induce a sense of obligation for the future is a critical element in conducting successful transactions, socially beneficial exchanges, and establishing lasting relationships. The principle of \textit{scarcity }states that when something is limited, we experience discomfort because of the reduced opportunities. This emotional response often leads people to make quick decisions. The principle of \textit{liking and sympathy} suggests that people are more likely to help those they like or consider friends, even if the request is uncomfortable. In addition, we tend to be more sympathetic to those who share our beliefs or resemble us. The principle of \textit{authority} explains that individuals are more likely to trust and accept the ideas of experts rather than forming independent opinions. This is why people often respect professionals such as doctors, lawyers, and military officers. The principle of \textit{social proof }can be used to get someone to comply by showing that many other people (including widely admired and well-known individuals) have already agreed to the demand.
Based on previous studies, it can be argued that each of the aforementioned principles of social influencing others contains the possibility of having a manipulative effect on those involved in the communication process.