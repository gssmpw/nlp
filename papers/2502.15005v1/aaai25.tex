%File: anonymous-submission-latex-2025.tex
\documentclass[letterpaper]{article} % DO NOT CHANGE THIS
\usepackage{aaai25}  % DO NOT CHANGE THIS
\usepackage{times}  % DO NOT CHANGE THIS
\usepackage{helvet}  % DO NOT CHANGE THIS
\usepackage{courier}  % DO NOT CHANGE THIS
\usepackage[hyphens]{url}  % DO NOT CHANGE THIS
\usepackage{graphicx} % DO NOT CHANGE THIS
\urlstyle{rm} % DO NOT CHANGE THIS
\def\UrlFont{\rm}  % DO NOT CHANGE THIS
\usepackage{natbib}  % DO NOT CHANGE THIS AND DO NOT ADD ANY OPTIONS TO IT
\usepackage{caption} % DO NOT CHANGE THIS AND DO NOT ADD ANY OPTIONS TO IT
\usepackage{xcolor}
\usepackage{enumitem}
\frenchspacing  % DO NOT CHANGE THIS
\setlength{\pdfpagewidth}{8.5in} % DO NOT CHANGE THIS
\setlength{\pdfpageheight}{11in} % DO NOT CHANGE THIS
%
% These are recommended to typeset algorithms but not required. See the subsubsection on algorithms. Remove them if you don't have algorithms in your paper.
\usepackage{algorithm}
\usepackage{algorithmic}

%
% These are are recommended to typeset listings but not required. See the subsubsection on listing. Remove this block if you don't have listings in your paper.
\usepackage{newfloat}
\usepackage{listings}
\DeclareCaptionStyle{ruled}{labelfont=normalfont,labelsep=colon,strut=off} % DO NOT CHANGE THIS
\lstset{%
	basicstyle={\footnotesize\ttfamily},% footnotesize acceptable for monospace
	numbers=left,numberstyle=\footnotesize,xleftmargin=2em,% show line numbers, remove this entire line if you don't want the numbers.
	aboveskip=0pt,belowskip=0pt,%
	showstringspaces=false,tabsize=2,breaklines=true}
\floatstyle{ruled}
\newfloat{listing}{tb}{lst}{}
\floatname{listing}{Listing}
%
% Keep the \pdfinfo as shown here. There's no need
% for you to add the /Title and /Author tags.
\pdfinfo{
/TemplateVersion (2025.1)
}

% DISALLOWED PACKAGES
% \usepackage{authblk} -- This package is specifically forbidden
% \usepackage{balance} -- This package is specifically forbidden
% \usepackage{color (if used in text)
% \usepackage{CJK} -- This package is specifically forbidden
% \usepackage{float} -- This package is specifically forbidden
% \usepackage{flushend} -- This package is specifically forbidden
% \usepackage{fontenc} -- This package is specifically forbidden
% \usepackage{fullpage} -- This package is specifically forbidden
% \usepackage{geometry} -- This package is specifically forbidden
% \usepackage{grffile} -- This package is specifically forbidden
% \usepackage{hyperref} -- This package is specifically forbidden
% \usepackage{navigator} -- This package is specifically forbidden
% (or any other package that embeds links such as navigator or hyperref)
% \indentfirst} -- This package is specifically forbidden
% \layout} -- This package is specifically forbidden
% \multicol} -- This package is specifically forbidden
% \nameref} -- This package is specifically forbidden
% \usepackage{savetrees} -- This package is specifically forbidden
% \usepackage{setspace} -- This package is specifically forbidden
% \usepackage{stfloats} -- This package is specifically forbidden
% \usepackage{tabu} -- This package is specifically forbidden
% \usepackage{titlesec} -- This package is specifically forbidden
% \usepackage{tocbibind} -- This package is specifically forbidden
% \usepackage{ulem} -- This package is specifically forbidden
% \usepackage{wrapfig} -- This package is specifically forbidden
% DISALLOWED COMMANDS
% \nocopyright -- Your paper will not be published if you use this command
% \addtolength -- This command may not be used
% \balance -- This command may not be used
% \baselinestretch -- Your paper will not be published if you use this command
% \clearpage -- No page breaks of any kind may be used for the final version of your paper
% \columnsep -- This command may not be used
% \newpage -- No page breaks of any kind may be used for the final version of your paper
% \pagebreak -- No page breaks of any kind may be used for the final version of your paperr
% \pagestyle -- This command may not be used
% \tiny -- This is not an acceptable font size.
% \vspace{- -- No negative value may be used in proximity of a caption, figure, table, section, subsection, subsubsection, or reference
% \vskip{- -- No negative value may be used to alter spacing above or below a caption, figure, table, section, subsection, subsubsection, or reference

\setcounter{secnumdepth}{0} %May be changed to 1 or 2 if section numbers are desired.

% The file aaai25.sty is the style file for AAAI Press
% proceedings, working notes, and technical reports.
%

% Title
%\iffalse


% Your title must be in mixed case, not sentence case.
% That means all verbs (including short verbs like be, is, using,and go),
% nouns, adverbs, adjectives should be capitalized, including both words in hyphenated terms, while
% articles, conjunctions, and prepositions are lower case unless they
% directly follow a colon or long dash
%\title{Knowledge Organization Systems and RAG Combine for an Explainable AI Agent Focused on Research Topics, to Increase Visibility of Emerging Researchers}

%\author{
    %Authors
    % All authors must be in the same font size and format.
%    Written by AAAI Press Staff\textsuperscript{\rm 1}\thanks{With help from the AAAI Publications Committee.}\\
 %   AAAI Style Contributions by Pater Patel Schneider,
 %   Sunil Issar,\\
 %   J. Scott Penberthy,
  %  George Ferguson,
 %   Hans Guesgen,
  %  Francisco Cruz\equalcontrib,
 %   Marc Pujol-Gonzalez\equalcontrib
%}
%\affiliations{
    %Afiliations
%    \textsuperscript{\rm 1}Association for the Advancement of Artificial Intelligence\\
    % If you have multiple authors and multiple affiliations
    % use superscripts in text and roman font to identify them.
    % For example,

    % Sunil Issar\textsuperscript{\rm 2},
    % J. Scott Penberthy\textsuperscript{\rm 3},
    % George Ferguson\textsuperscript{\rm 4},
    % Hans Guesgen\textsuperscript{\rm 5}
    % Note that the comma should be placed after the superscript

%    1101 Pennsylvania Ave, NW Suite 300\\
 %   Washington, DC 20004 USA\\
    % email address must be in roman text type, not monospace or sans serif
 %   proceedings-questions@aaai.org
%
% See more examples next
%}
%\fi



%\iffalse
%Example, Multiple Authors, ->> remove \iffalse,\fi and place them surrounding AAAI title to use it
%\title{Grounding natural language queries on research topics to knowledged organization systems}
%\title{An Explainable AI Agent Bridging Knowledge Organization Systems and human understandable research topics with RAG}
\title{A Socratic RAG Approach to Connect Natural Language Queries on Research Topics with Knowledge Organization Systems}
\author{
    %Authors
   Lew Lefton\textsuperscript{\rm 1},\
   Kexin Rong\textsuperscript{\rm 1},\
   Chinar Dankhara\textsuperscript{\rm 1},\
   Lila Ghemri\textsuperscript{\rm 2},\
   Firdous Kausar\textsuperscript{\rm 3},\
   A. Hannibal Hamdallahi\textsuperscript{\rm 3}
}
\affiliations{
    %Affiliations
\textsuperscript{\rm 1}Georgia Institute of Technology\\
\textsuperscript{\rm 2}Texas Southern University\\
\textsuperscript{\rm 3}Fisk University\\
lew.lefton@gatech.edu, krong@gatech.edu, chinardankhara@gatech.edu, lila.ghemri@tsu.edu, fkausar@fisk.edu, aleach@fisk.edu
}



% REMOVE THIS: bibentry
% This is only needed to show inline citations in the guidelines document. You should not need it and can safely delete it.
%\usepackage{bibentry}
% END REMOVE bibentry



\begin{document}

\maketitle{}

\begin{abstract}
In this paper, we propose a Retrieval Augmented Generation (RAG) agent that maps natural language queries about research topics to precise, machine-interpretable semantic entities. Our approach combines RAG with Socratic dialogue to align users' intuitive understanding of research topics with established Knowledge Organization Systems (KOSs). The proposed approach will effectively bridge "little semantics" (domain-specific KOS structures) with "big semantics" (broad bibliometric repositories), making complex academic taxonomies more accessible. Such agents have the potential for broad use. We illustrate with a sample application called CollabNext, which is a person-centric knowledge graph connecting people, organizations, and research topics. We further describe how the application design has an intentional focus on HBCUs and emerging researchers to raise visibility of people historically rendered invisible in the current science system.

\end{abstract}

% Uncomment the following to link to your code, datasets, an extended version or similar.
%
% \begin{links}
%     \link{Code}{https://aaai.org/example/code}
%     \link{Datasets}{https://aaai.org/example/datasets}
%     \link{Extended version}{https://aaai.org/example/extended-version}
% \end{links}

\section{Introduction}
What is a research topic? Research is carried out on almost everything, so the concept space is wide and deep. It is unsurprising that there is no universal resource that covers all research concepts. Instead, many domain-specific Knowledge Organization Systems (KOS) have emerged providing ontologies, taxonomies, lists, and thesauri of their areas \cite{salatino:2024}. These KOSs are good sources of expert-curated ground truth. Unfortunately, in a KOS, topic data typically consists only of strings and not semantic entities, which makes research topics a weak link in knowledge axiomatization and research classification.

A knee-jerk reaction to this challenge is to leverage a Large Language Model (LLM) to generate research topics, based on a corpus of research papers, datasets, etc. In fact, there are many tools in the area of topic classification that do precisely that. Unfortunately, LLM generated topics do not align well with how human beings talk and think about research. 

Alignment of natural language research topics with structured KOSs is needed for CollabNext \cite{CollabNext}, an application currently under development as part of the National Science Foundation's Prototype Open Knowledge Network \cite{ProtoOKN}.  CollabNext allows users to explore who is working in what research area and where they are. While this seems simple and straightforward to answer, it turns out to be quite nuanced and challenging. Name collisions make it difficult to identify the "who", especially for common names or Asian names \cite{namedisambig}. Organizations, the "where", can also have multiple names, campuses, etc.  Fortunately, for both individuals and organizations, there are established and open semantic structures, for example in schema.org \cite{schemaorg} or Wikidata.org \cite{wikidata}, and available persistent identifiers, e.g. ORCID \cite{orcid} and ROR \cite{ror}.
%
Research topics, the "what", turn out to be more complicated. There is no universal list of research concepts, and certainly nothing that compares to a persistent identifier. 

This paper begins with a brief overview of related work on the challenge of organizing research topics. We then provide a short description of CollabNext, including its intentional design to increase the visibility of emerging researchers. We share an example that highlights the challenge and importance of aligning natural language research topic queries with robust, open, and structured data in KOSs. Next we proceed to lay out a plan for a RAG agent designed to meet this challenge using a novel Socratic approach. The final section has a conclusion and considers the next steps.

\section{Related Work}
In this section, we review approaches for automatically generating topics in scientific publications. We focus on three main paradigms: supervised topic classification, unsupervised topic modeling, and knowledge organization systems.

\subsection{Topic Classification} Topic classification takes a supervised approach, using often manually curated document topics to train models for categorizing new publications. However, obtaining high-quality ground truth labels at scale remains challenging, forcing many methods to build topic taxonomies from scratch~\cite{liu2014hierarchical}.

The Microsoft Academic Graph (MAG)~\cite{sinha2015overview} represents a significant effort in this direction. 
Their methodology began with manual definition of a small number of top-level concepts, followed by extraction of 28,000 academic concepts from Wikipedia articles~\cite{shen2018web}. MAG implemented topic assignment as a multilabel classification problem, enabling papers to be tagged with multiple concepts.
After the discontinuation of MAG in 2021, OpenAlex attempted to reproduce the topic classification model using historical MAG data as training labels~\cite{priem2022openalex}. During this process, OpenAlex identified several limitations of MAG concepts, including term polysemy, ambiguity, and the static nature of concepts that failed to evolve with research trends.
OpenAlex subsequently adopted a labeled dataset from CWTS~\cite{van2024open} and integrated it with concepts from the All Science Journal Classification (ASJC) codes. 
However, obtaining high-quality training labels at scale remains a challenge, often forcing methods to construct topic taxonomies from scratch~\cite{liu2014hierarchical}.


\subsection{Topic Modeling} 
Unsupervised topic modeling techniques discover common themes within document collections by clustering similar documents and deriving topic labels from shared characteristics within clusters.
For scientific papers, these methods typically leverage multiple signals including semantic similarity from titles and abstracts, citation network structures, and shared references.
Domain-specific pretrained models like SciBERT~\cite{hosokawa2024} have been shown to enhance clustering quality of abstract embeddings by using purpose-built tokenizers and specialized pretraining that capture field-specific terminology.

Similarly, BERTopic~\cite{grootendorst2022bertopic} uses a pre-trained language model to generate document embedding, performs dimensionality reduction, and creates semantically similar clusters (using HDBSCAN) that each represent a distinct topic. BERTopic uses a variant of TF-IDF to extract topic representations from each topic. 
CWTS~\cite{van2024open} leverages large language models (LLMs) to generate research topic labels. The LLM receives titles of the 250 most cited publications in each cluster and generates a label and a brief summary of the research area. 

However, clustering-based approaches face limitations: extracted topics often fail to align with established ontologies, and poor clustering quality can result in topics lacking clear interpretability.

\subsection{Knowledge Organization Systems} 

A Knowledge Organization System (KOS) \cite{salatino:2024} is a structured framework that enables the arrangement, management, retrieval, and dissemination of information. 
These systems are crucial to library science, information retrieval, and knowledge management, offering standardized vocabularies and defined relationships between concepts to improve information accessibility and interoperability~\cite{Hodge:2000}. 

KOS encompasses several key methodologies:
\begin{itemize}[leftmargin=*]
    \item \emph{Classification schemes}: Systems such as the Dewey Decimal Classification (DDC)~\cite{Dewey:2011} and Library of Congress Classification (LCC) systematically group related concepts based on shared characteristics, providing logical organization for library resources.
    \item \emph{Thesauri}: Controlled vocabularies that map relationships between terms, including synonyms, antonyms, and hierarchical relationships. A prominent example is the Medical Subject Headings (MeSH), which is the controlled vocabulary used to index the huge amount of biomedical literature \cite{NLM:2022}.
    \item \emph{Taxonomies}: Hierarchical systems organizing concepts into parent-child relationships, commonly used in web navigation and content management systems.
    \item \emph{Ontologies}: Formal and machine-interpretable specifications that define domain concepts and their relationships~\cite{Guarino:2009}. These are crucial for semantic web technologies and AI applications, enabling data interoperability and automated reasoning.
    \item \emph{Controlled Vocabularies}: Curated term lists ensure consistent language across collections, enhancing retrieval precision and accuracy.
\end{itemize}

KOSs help to uncover knowledge from various sources, especially in the context of big data and the semantic web. Integrating and discovering knowledge from heterogeneous sources becomes a lot easier with KOS, which is also very usefu; for advanced research and development tasks. The Simple Knowledge Organization System (SKOS) presents a model for sharing and linking knowledge organization systems on the web \cite{Miles:2009}. This W3C-endorsed model based on Resource Description Framework (RDF) promotes interoperability, as well as the seamless exchange of information between heterogeneous systems. 
%%\textcolor{blue}{KR: Is SKOS important to cover?}
%% Yes, it was mentioned several times at the ISWC so I believe it is still an active w3c standard https://www.w3.org/TR/skos-reference/

Despite their utility, KOSs encounter some issues, such as how to remain pertinent in rapidly changing areas of knowledge and how to ensure that they work well with other systems. \cite{lauruhn:2016} comment on the need for KOSs to have adaptive management strategies that allow them to change along with our knowledge structures.

Recent research shows promises of integrating KOS with large language models (LLMs) to enhance AI explainability. \cite{ahmed:2023} demonstrated that the combination of knowledge graphs with LLM produces more interpretable outputs than LLMs alone, while \cite{krause:2024} showed that LLMs coupled with external knowledge sources improve commonsense reasoning capabilities in AI systems.


\section{CollabNext}

There are many potentially impactful applications which would benefit from having a RAG agent that provides an explainable topic identifier based on established ground truth of human-generated KOSs. To serve as an illustrative example, we describe CollabNext \cite{CollabNext} which is currently under development as part of the NSF's Prototype Open Knowledge Network \cite{ProtoOKN}. This application depends on open, robust, and structured data on research topics.

\subsection{Application Overview}
CollabNext implements a knowledge graph with entities consisting of people, organizations, and research topics. The primary data source is currently the bibliometric dataset from OpenAlex \cite{priem2022openalex}, but the intention is to include other relevant data sources as development continues. Relationships between people and organizations are available directly in OpenAlex via author and institution tables. Relationships between people and topics are inferred, since people are connected to OpenAlex works as authors, and works are connected to topics. See Figure 1 for a conceptual schema of the CollabNext knowledge graph. Note, that the schema in Figure 1 includes additional data like Grants and Patents which are not yet in OpenAlex, but these objects could still be assigned topics via methods discussed above.  


\begin{figure}[t]
\centering
\includegraphics[width=0.9\columnwidth]{CollabNext_Schema_Idea.png} % Reduce the figure size so that it is slightly narrower than the column. Don't use precise values for figure width.This setup will avoid overfull boxes.
\caption{Conceptual schema of CollabNext knowledge graph}
\label{fig1}
\end{figure}

The CollabNext application is being developed using an intentional design approach, initially prioritizing Historically Black Colleges and Universities (HBCUs) and emerging researchers. This is a deliberate effort to counterbalance the accumulated advantage of well-resourced research organizations, which is an example of the Matthew Effect \cite{Merton-Matthew-1968}. This effect is a common phenomenon in social systems, and can be summarized as "the rich get richer and the poor get poorer." CollabNext is being designed to counterbalance this bias, specifically in the social system of science research \cite{Bol2018, Rossiter1993, Petersen2011}.

One of the ways to do this is to, by default, focus on HBCUs and other organization where there are a high percentage of underrepresented researchers. Another design decision which can help raise visibility of invisible researchers is to not order researchers by citation count, which exacerbates the Matthew effect, but rather default to use geospatial data like location and distance as a primary filter.  This is effectively adding an implicit "near me" search filter and showing people who work on a specific research topic who are also geographically close, e.g. at your same institution, in your same state, or your same timezone. The rationale for this approach is based on work in Team Science, which shows that geographical distance is a major barrier for scientific collaboration \cite{HOEKMAN2024104927}. The work of \cite{Katz1994-rv} shows that the probability of collaboration decreases as distance grows, based on co-authorship relations.


\subsection{Challenge: Topic Search Needs an NLP interface}
During the development of CollabNext, it became apparent that there was a disconnect between how people see research topics and how research topics are classified by AIs.  We will illustrate with an example.

Suppose a user wants to find researchers who are working on the topic of \textit{plastic recycling} at a nearby University.  There is no OpenAlex topic that directly matches this.  Matching just for \textit{plastic} yields 14 topic ids in the subfield  \textit{Polymers and Plastics}. This topic could also be considered \textit{Waste Management and Disposal} which is in the subfield \textit{Environmental Science}. Or it may be the topic \textit{Biodegradable Polymers as Biomaterials and Packaging} which is in the subfield \textit{Biomaterials}.  It may even be the case that the user is not interested in the Chemistry of plastic recycling at all, but rather the economics or logistics of it. Similar collisions and ambiguities arise when the keyword \textit{recycling} is used. More clarification is needed from the user. This subtle but important challenge can be improved with the RAG agent described below. 

%\section{Methodology}
\label{sec:approach}

\begin{figure}[!t]
\centering
\includegraphics[width=0.5\textwidth]{Pipeline.png}
\caption{Workflow. For each synthesis or sketching task, we create an input query for the LLM such that the query contains the target property in natural language or Alloy (depending on the kind of task), run the query, get the LLM's output, and use the Alloy analyzer to validate it with respect to a reference (ground truth) formula.}
\label{fig:workflow}
\end{figure}

We consider the following three methods for employing large language models (LLMs) to create Alloy formulas to investigate the capabilities and limitations of LLMs in writing Alloy:

\begin{enumerate}
\item
{\bf English to Alloy}. We employ LLMs to write complete Alloy formulas in multiple different ways from given natural language descriptions (in English);
\item
{\bf Alloy to Alloy}. We employ LLMs to create multiple alternative but equivalent formulas in Alloy with respect to given formulas in Alloy; and
\item
{\bf Sketch to Alloy}. We employ LLMs to complete sketches~\cite{SolarLazemaPhD2008,WangETALABZ2018ASketch} of Alloy
formulas and populate the holes in the sketches by synthesizing Alloy
expressions and operators so that the completed formulas accurately
represent the desired properties (that are given in natural language).  \end{enumerate}

\begin{table}[!t]
\begin{tabular}{r@{\hskip 0.2cm}|l|p{4cm}|p{5cm}}
& \multicolumn{1}{c|}{\Intro{Property}} & \multicolumn{1}{c|}{\Intro{Natural language desc.}} & \multicolumn{1}{c}{\Intro{Reference Alloy formula}}\\
\hline
1 & DAG & Directed acyclic graph &
\begin{lstlisting}[style=AlloyTable]
all n: Node | n !in n.^link
\end{lstlisting} \\
\hline
2 & Cycle & Graph with directed cycle &
\begin{lstlisting}[style=AlloyTable]
some n: Node | n in n.^link
\end{lstlisting} \\
\hline
3 & Circular & The number of nodes is equal to the number of edges and the graph has a directed cycle that visits all nodes &
\begin{lstlisting}[style=AlloyTable]
#Node = #link
all n: Node | one n.link
all m, n: Node | m in n.^link
\end{lstlisting} \\
\hline
4 & Connex & For every pair of elements in S, either the first is related to the second or vice versa &
\begin{lstlisting}[style=AlloyTable]
all s, t: S |
  s->t in r or t->s in r
\end{lstlisting} \\
\hline
5 & Reflexive & Every element in S is related to itself &
\begin{lstlisting}[style=AlloyTable]
all s: S | s->s in r
\end{lstlisting} \\
\hline
6 & Symmetric & If element x in S is related to y, then y is also related to x &
\begin{lstlisting}[style=AlloyTable]
all s, t: S |
  s->t in r implies t->s in r
\end{lstlisting} \\
\hline
7 & Transitive & If element x in S is related to y and y is related to z, then x is also related to z &
\begin{lstlisting}[style=AlloyTable]
all s, t, u: S |
  s->t in r and t->u in r
    implies s->u in r
\end{lstlisting} \\
\hline
8 & Antisymmetric & If element x in S is related to y and y is related to x, then x and y are the same element &
\begin{lstlisting}[style=AlloyTable]
all s, t: S |
  s->t in r and t->s in r
    implies s = t
\end{lstlisting} \\
\hline
9 & Irreflexive & No element in S is related to itself &
\begin{lstlisting}[style=AlloyTable]
all s, t: S |
  s->t in r implies s != t
\end{lstlisting} \\
\hline
10 & Functional & Every element in S is related to at most one element (making r a partial function) &
\begin{lstlisting}[style=AlloyTable]
all s: S | lone s.r
\end{lstlisting} \\
\hline
11 & Function & Every element in S is related to exactly one element (making r a total function) &
\begin{lstlisting}[style=AlloyTable]
all s: S | one s.r
\end{lstlisting} \\
\hline
\end{tabular}
\vspace*{2ex}
\caption{Subject properties. The table lists for each property, its
  natural language description that defines the corresponding natural
  language to Alloy task, and its reference formulation in Alloy that
  defines the corresponding Alloy to Alloy
  task.}\label{tab:subjects-synthesis}
\vspace*{-4ex}
\end{table}


\begin{table}[!h]
\centering
\begin{tabular}{p{12cm}}
\hline
\begin{lstlisting}[style=AlloyTable]
pred DAG {
  // Directed acyclic graph
  all n: Node | \E,e\ \CO,co\ \E,e\
}
co := {| =|in|!=|!in |}
e := {| Node|n|((Node|n).(*|^)link) |}
\end{lstlisting} \\ \hline

\begin{lstlisting}[style=AlloyTable]
pred Cycle {
  // Graph with directed cycle
  some n: Node | \E,e\ \CO,co\ \E,e\
}
co := {| =|in|!=|!in |}
e := {| Node|n|((Node|n).(*|^)link) |}
\end{lstlisting} \\ \hline

\begin{lstlisting}[style=AlloyTable]
pred Circular {
  // The number of nodes is equal to the number of edges and the graph has a directed cycle that visits all nodes
#Node = #link
  all n: Node | one n.link
  all m, n: Node | \E,e\ \CO,co\ \E,e\
}
co := {| =|in|!=|!in |}
e := {| (Node|m|n).(*|^)link |}
\end{lstlisting} \\ \hline

\end{tabular}
\vspace*{2ex}
\caption{Sketches for Alloy specifications for Properties 1--3.}
\vspace*{-8ex}
\label{tab:sketches-1-3}
\end{table}

Figure~\ref{fig:workflow} graphically illustrates our approach.
For each synthesis or sketching task, we create an input query for the LLM such that the query contains the target property in natural language or Alloy (depending on the kind of task), run the query, get the LLM's output, and run the Alloy analyzer to validate it with respect to a ground truth formula, which we provide to the analyzer. There are three possible outcomes of running the Alloy analyzer: (1) the LLM's answer is correct (when the analyzer does not find a counterexample to the equivalence of the LLM's answer and ground truth); (2) the LLM's answer has a syntax error (when the analyzer fails to compile the LLM's answer); and (3) the LLM's answer is wrong (when the analyzer finds a counterexample to the equivalence of the LLM's answer and ground truth). Note for "Alloy to Alloy" synthesis tasks, the ground truth formula is the reference formula given as input to the LLM. Note also that for any "English to Alloy" synthesis task and for any "Sketch to Alloy" sketching task, the input to the LLM does not include the ground truth formula.

We employ the LLMs directly as available for public use.  Specifically, we do not fine-tune them.  Moreover, the queries we write are minimalistic in their description of the problem domain and do not provide instructions to the LLM on how to approach solving any given task.

\subsection{Subject tasks}

We use \NumSubjects~well-known properties of graphs and binary relations to create \NumTotalTasks~tasks for the LLMs to answer.  Three of the properties (DAG, Cycle, and Circular) are regarding edge-labeled graphs, and the remaining eight properties (Connex, Reflexive, Symmetric, Transitive, Antisymmetric, Irreflexive, Functional, and Function) are regarding binary relations.  In Alloy, in general, we can use one signature $S$ and one binary relation $r: S\times S$ to represent either an edge-labeled graph or a binary relation. However, in view of the specific domain of graphs, we name the signature `\CodeIn{Node}' and the binary relation `\CodeIn{link}' when creating the tasks relating graph properties. For the tasks relating properties of binary relations, we name the signature `\CodeIn{S}' and the relation `\CodeIn{r}'.

For each property, we create 2~kinds of synthesis tasks: (1) create 20~unique Alloy formulas that represent the given natural language description of the property; and (2) create 20~unique Alloy formulas that are equivalent to the given Alloy formula that captures the property, which is also included as a natural language comment in the prompt.  In addition, for each property, we create one sketching task: complete the given sketch of the property with respect to its natural language description that is included as a comment in the prompt.  Thus, for each property, we have a total of 3~tasks for the LLM to answer.

Table~\ref{tab:subjects-synthesis} lists each property, its natural language description, and a reference (ground truth) formula that characterizes it in Alloy. Moreover, Tables~\ref{tab:sketches-1-3}, \ref{tab:sketches-4-8} (Appendix), and \ref{tab:sketches-9-11} (Appendix) list each property, its sketch that defines the corresponding sketching problem. Together these four tables summarize the key elements of our tasks for the LLMs. To illustrate, consider the DAG property.  Figure~\ref{fig:three-tasks-for-DAG} describes the actual prompts we run against each LLM for this property.

\begin{figure}[!p]
\centering
\begin{tcolorbox}[mytextbox]
Give me 20 unique solutions to the problem of synthesizing the body of the following Alloy predicate (without markdown or comments) with respect to the property described in the comments:
\begin{lstlisting}
sig Node {
  link: set Node
}
pred DAG{
  // Directed acyclic graph
  // your code go here
}
\end{lstlisting}
\end{tcolorbox}
(a) "English to Alloy" task\\
\begin{tcolorbox}[mytextbox]
Give me 20 unique solutions to the problem of synthesizing the body of the following Alloy predicate (without markdown or comments) with respect to the property described in the comments:
\begin{lstlisting}
sig Node {
  link: set Node
}
pred DAG{
  // Directed acyclic graph
  all n: Node | n !in n.^link
}
\end{lstlisting}
\end{tcolorbox}
(b) "Alloy to Alloy" task\\
\begin{tcolorbox}[mytextbox]
Complete the following sketch of the Alloy predicate (without markdown or comments) by selecting values for the holes with respect to the given constraints such that the predicate is correct with respect to the property described in the comments:

\begin{lstlisting}
sig Node {
  link: set Node
}
pred DAG {
  // Directed acyclic graph
  all n: Node | \E,e\ \CO,co\ \E,e\
}

co := {| =|in|!=|!in |}
e := {| Node|n|((Node|n).(*|^)link) |}
\end{lstlisting}
\end{tcolorbox}
(c) "Sketch to Alloy" task
\caption{Three tasks for the LLMs with respect to the DAG property.}
\label{fig:three-tasks-for-DAG}
\end{figure}

In a predicate sketch, certain components of the predicate are placeholder holes~\cite{WangETALABZ2018ASketch}. These holes can be of different forms, e.g., comparison operator holes, expression holes, and quantifier holes.  For all our sketching tasks, we only use two kinds of holes: comparison operator holes and expression holes. A predicate sketch includes a definition of the sets of possible values that each hole can be completed with.  These sets are typically defined using regular expressions~\cite{SolarLazemaPhD2008}.  For our DAG sketching task, the comparison operator hole may be completed with one of four possible values from the set \{ `\CodeIn{=}', `\CodeIn{in}', `\CodeIn{!=}', `\CodeIn{!in}'\}, and each expression hole may be completed with one of six possible values from the set \{ `\CodeIn{Node}', `\CodeIn{n}', `\CodeIn{Node.*link}', `\CodeIn{Node.\^{}link}', `\CodeIn{n.*link}', `\CodeIn{n.\^{}link}' \}.



\section{Proposed Approach}
%\textcolor{blue}{KR: Wrote a version based on my understanding. Feel free to edit. }
% Looks great! Thank you.

We propose an LLM agent leveraging multi-round Retrieval-Augmented Generation (RAG) to bridge the semantic gap between users' natural language queries on research topics and structured KOSs. While LLMs have demonstrated remarkable capabilities, they often struggle with domain-specific knowledge and are prone to hallucinations~\cite{zhao2023survey}, particularly in specialized academic contexts. Retrieval-augmented approaches have emerged as a promising solution by grounding LLM outputs in external knowledge bases~\cite{lewis2020retrieval}.

Traditional RAG systems typically employ single-round retrieval, where knowledge is retrieved solely based on the initial query~\cite{izacard2022few}.  However, this approach proves insufficient for complex questions that require iterative refinement and disambiguation, such as mapping colloquial research descriptions to formal taxonomic structures. 
Recent work has demonstrated the effectiveness of multi-round retrieval and reasoning~\cite{borgeaud2022improving,shao2023enhancing,zhuang2024efficientrag}, where each round of interaction refines the query and retrieval process. 

Building on these advances, we propose a Socratic dialogue approach where the agent guides users through structured exploration of research topics, by asking questions of users, rather than simply offering answers. We anticipate that this dialog will be more than simple clarification questions. The agent should engage in meaningful conversation to examine and understand the user's research topic of interest at multiple levels of granularity. The agent will be guided by reasoning across the multiple KOS's on which it was trained. By grounding these conversations in human-curated KOSs as the knowledge foundation, the system provides transparent data provenance and explainable recommendations.

If successful, the proposed agent would be able to effectively bridge "little semantics" (domain-specific KOS structures) with "big semantics" (broad bibliometric repositories), making complex academic taxonomies more accessible to users who naturally express research interests in colloquial language. This approach enables precise mapping between informal research descriptions and formal knowledge structures while maintaining transparency and interpretability. Figure~\ref{fig2} illustrates the high-level workflow of our proposed system, and we overview the detailed design below.

\subsubsection{Topic Retrieval}
To effectively identify relevant research topics from KOS taxonomies, we propose a novel hierarchical retrieval mechanism. We will use domain-specific, pretrained models such as SciBERT~\cite{beltagy2019scibert} to generate embeddings for each topic node. The retrieval process occurs in two stages:
\begin{enumerate}
    \item Initial Semantic Search: Topic embeddings are represented as dense embeddings of topic title and description strings. These are retrieved based on embedding similarity with the user query, using cosine similarity normalized by a temperature parameter $\tau$ to control retrieval diversity - higher values generate more diverse candidates while lower values prioritize closer semantic matches.
    \item Hierarchy-Aware Reranking: Given the initial topic list, we rerank them by considering the ontological structure of topics. The reranking score combines three components: (1) direct semantic similarity between query and topic; (2) weighted ancestor path similarity computed as $$score(t)= sim(q,t)+\alpha\sum_{a \in ancestors(t)} \beta^d \times sim(q, a)$$ where $\alpha$ controls the overall influence of ancestral relationships and $\beta$ determines the decay rate with ancestral distance $d$; and (3) sibling coherence that boosts topics whose siblings also show high query similarity. This scoring function ensures that retrieved topics are both semantically relevant and structurally coherent within their respective taxonomies.
\end{enumerate}





\subsubsection{Multi-round RAG}
The academic knowledge landscape is characterized by fragmented KOS resources: broad multi-field systems that cover multiple disciplines but lack depth, and specialized single-field KOS that provide granular topic categorization within specific domains. Our proposed multi-round RAG process is specifically designed to navigate this fragmented structure through a two-phase approach.

In the first phase, the system queries broad multi-field KOS systems such as All Science Journal Classification Codes, OpenAIRE’s Field of Science Taxonomy, or Dewey Decimal Classification to identify relevant high-level research areas. For each retrieved topic, the LLM generates contextual explanations by synthesizing definitions and descriptions from linked knowledge bases. The agent engages in iterative dialogue with users, presenting candidate topics along with their explanations and soliciting feedback to refine the search. This phase continues until users confirm their broad research direction.

The second phase starts when users confirm topics that have corresponding single-field KOS coverage. For instance, in Computer Science, the system can leverage multiple specialized taxonomies including the Computer Science Ontology (CSO), ACM Computing Classification System (CCS), and Wikipedia's Computer Science Subject Headings. The agent uses context accumulated from the first phase to guide exploration within these detailed taxonomies. For fields lacking comprehensive KOS coverage~\cite{salatino:2024}, the system employs more stringent filtering within multi-field KOS to identify the most precise available classifications.





\begin{figure}[t]
\centering
\includegraphics[width=0.95\columnwidth]{overview} % Reduce the figure size so that it is slightly narrower than the column. Don't use precise values for figure width.This setup will avoid overfull boxes.
\caption{Proposed multi-round RAG design that bridges natural language queries on research topics with structured KOS data. }
\label{fig2}
\end{figure}



\section{Conclusion and Next Steps}

We have outlined a RAG agent that takes natural language queries about research topics and provides focused machine-representable semantic entity identifiers based on the ground truth of human generated KOSs. We have also described the CollabNext application. One of the goals of CollabNext is to construct a knowledge graph that makes visible the relationships between people, research organizations, and research topics. CollabNext is being intentionally designed to make emerging researchers more visible and thereby help reduce the Matthew Effect in scientific research. We  
further illustrate how the RAG agent could be used, and provided an approach to building the agent.


The agent described here has potential beyond the use case of CollabNext. Indeed, \textit{any application that links its data to research topics via an existing KOS could also leverage the agent as a research topic "Rosetta Stone."} Moreover, disparate research artifacts could be semantically linked using research topics as an index, including articles, datasets, websites, AI models, GitHub repositories, theses, posters, etc. An existing example of this is the OpenAlex "aboutness" endpoint. This could lead to broader integration of research knowledge and data across diverse domains using a well-defined topic architecture. New research topics at disciplinary boundaries could also emerge and be naturally included as domain specific KOSs incorporate new structure and data. 

We note that there are multiple other data-driven products which are relevant for this discussion.   Examples include Google Scholar, LinkedIn (owned by Microsoft), Clarivate, Elsevier, Semantic Scholar, Metaphacts, System.com, etc. Some of them are commercial and are only available to well resourced institutions. Others leverage proprietary data and are not well suited to open systems. Also, these tools are more focused on research artifacts, and less on the researchers themselves.

In some sense, the RAG agent serves the role of a topic disambiguation tool.  This is clearly needed for tools that work at the interface of humans and AIs.  Nuance and subtlety of how different researchers think about the “same” thing may provide insight and new results, and allow researchers to see integrative work and connect with adjacent researchers whom they did now know about.

This paper is admittedly only a high level description of the agent. We are taking to heart the workshop submission guidelines \cite{tika25} which welcome short papers \textit{focused on case studies, work-in-progress, or visionary ideas}. There are many unanswered questions including
\begin{enumerate}
    \item What happens if there is a topic that cannot be resolved by the KOSs that the agent knows about. This knowledge gap could be a signal of an emerging research area or perhaps a need for KOS improvement. Some KOSs are updated more frequently than others, so this may be an opportunity to provide a feedback loop.
    \item Would it be possible for users to explore topics both up and down a refinement, i.e. broadening and narrowing vertically, as well as exploring adjacent siblings (horizontally). 
    \item How would user interactions be incorporated into the agent's knowledge over time? Could histories and usage patterns help with noisy/overlapping topics, or identify frequently asked questions that could indicate gaps between user needs and KOS data structures.
\end{enumerate}

The CollabNext team has plans to build a simple proof-of-concept prototype of the proposed RAG agent. Such a prototype would likely be focused on only a handful of available KOSs, but would likely include both OpenAlex and Wikidata as a starting point. 

\section{Acknowledgments}
This work was supported by the National Science Foundation TIP Directorate under Award Number 2333737. We would also like to acknowledge helpful guidance from the reviewers of this submission, along with the following members of the CollabNext Team who have helped develop the ideas here and the alpha version of the CollabNext application: Didier Contis, Kinnis Gosha, John Porter, Christopher Thomas, Craig Abbey, Leslie Collins, Sufyan Baksh, Sajid Hussain, Robert Briggs, Samarth Chandna, Sambridhi Deo, Lazarus Egwurube, Diamond GC, Ilia Khalighi, Vidushi Maheshwari, Ikechukwu Mgbemele, Christian Moore, Rejin Nepal, Mingxuan Nie, Netra Nyaupane, Nirmal Patel, Maruti Ram Ponnaganti, Ram Sharma, Keller Smith, Siddharth Singh Solanki, and Jaelyn Sykes.

\bibliography{aaai25}

\end{document}
