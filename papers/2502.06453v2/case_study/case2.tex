\begin{figure*}[t]
\begin{small}
    \centering
    \begin{minipage}[c]{0.42\textwidth}
        \begin{tcolorbox}[colback=white, colframe=gray!30, title=\textcolor{black}{\textsf{Original}}]
        \textbf{Problem}: Ten people are sitting around a round table.  Three of them are chosen at random to give a presentation.  What is the probability that the three chosen people were sitting in consecutive seats?

        \dashline

        \textbf{Solution}:

        Let's count as our outcomes the ways to select 3 people without regard to order.  There are $\binom{10}{3} = 120$ ways to select any 3 people.  The number of successful outcomes is the number of ways to select 3 consecutive people.  There are only 10 ways to do this -- think of first selecting the middle person, then we take his or her two neighbors.  Therefore, the probability is $\frac{10}{120} = \boxed{\frac{1}{12}}$.
        \vspace{-1mm}
        \end{tcolorbox}

    \end{minipage}
    \hfill
    \begin{minipage}[c]{0.53\textwidth}
        \begin{tcolorbox}[colback=white, colframe=gray!30, title=\HARD]
            \textbf{Problem}: Ten people are sitting around a round table.  Three of them are chosen at random to give a presentation in a specific order.  What is the probability that \textcolor{brown}{the first and second presenter were sitting in consecutive seats and at the same time the second and third presenter were sitting in consecutive seats?}

            \dashline

            \textbf{Claude-3.5-Sonnet}: \\
            1) First, let's understand what we're looking for: \\
           * We need 3 people in a specific order \\
           * The 1st and 2nd people must be sitting next to each other \\
           * The 2nd and 3rd people must be sitting next to each other \\
           * \textcolor{red}{This means all three people must be sitting consecutively around the table} \\
    \textcolor{gray}{\texttt{<--Omitting several lines-->}}





Therefore, the probability is $\boxed{\frac{1}{12}}$. %
        \end{tcolorbox}
        
    \end{minipage}
\end{small}
    \caption{An example of memorization \textit{coupled with} incorrect reasoning: The model incorrectly reduces the modified condition to the original condition, and then follows the original reasoning pattern. The correct answer is $\boxed{\nicefrac{1}{36}}$. {We manually performed 20 repeated trials and found that Claude-3.5-Sonnet has a pass rate of 50\%. Among the mistakes, 30\% are due to the memorization issue above.}}
    \label{fig:case:2}
    \vspace{-3mm}
\end{figure*}
