\begin{figure*}[ht]
\begin{small}
    \centering
    \begin{minipage}[c]{0.54\textwidth}
        \begin{tcolorbox}[colback=white, colframe=gray!30, title=\textcolor{black}{\textsf{Original}}]
        \textbf{Problem}: If $\left\lfloor n^2/4 \right\rfloor - \lfloor n/2 \rfloor^2 = 2$, then find all integer values of $n$.
        
        \dashline
        
        \textbf{Solution:}
        
        If $n$ is even, then we can write $n = 2m$ for some integer $m$. Substituting, $$\left \lfloor (2m)^2/4 \right\rfloor - \left\lfloor (2m)/2 \right\rfloor^2 = m^2 - m^2 = 0.$$Hence, $n$ must be odd; we can write $n = 2m+1$ for some integer $m$. Substituting,   
        \begin{align*}
        &\left \lfloor (2m+1)^2/4 \right. \rfloor - \left\lfloor (2m+1)/2 \right\rfloor^2\\
        & \text{\textcolor{gray}{\texttt{<--Omitting several lines-->}}}\\
        & = m.
        \end{align*}Thus, we find $m = 2$ and $n = \boxed{5}$ as the unique integer solution.
        \vspace{1mm}
        \end{tcolorbox}
    \end{minipage}
    \hfill
    \begin{minipage}[c]{0.44\textwidth}
        \begin{tcolorbox}[colback=white, colframe=gray!30, title=\HARD]
            \textbf{Problem}: If \textcolor{brown}{$\left\lfloor n^2/9 \right\rfloor - \lfloor n/3 \rfloor^2 = 2$}, then find \textcolor{brown}{the smallest} integer value of $n$.

        \dashline

        \textbf{o1-mini:} 
        
        Let's solve the equation step-by-step:
        
        \textcolor{gray}{\texttt{<--Omitting several lines-->}}

5. **Analyze for each value of \(r\):**

    \textcolor{gray}{\texttt{<--Omitting several lines-->}}
    
   - **For \(r = 1\):**
     \[
     2 \leq \frac{2k}{3} + \frac{1}{9} < 3
     \]
     
     Solving gives \(k = 3\) and \(k = 4\). Therefore, \(n = 10\) and \(n = 13\).
   

\textcolor{gray}{\texttt{<--Omitting several lines-->}}

**Final Answer:**

\[\boxed{\textcolor{red}{10 \text{ and } 13}}\] 
        \end{tcolorbox}
    \end{minipage}
\end{small}
    \caption{An example of memorizing the desired outcome. The model outputs all integer values instead of the smallest integer value. The correct answer is $\boxed{10}$. {We manually performed 20 repeated trials and found that o1-mini has a pass rate of 75\%. All the 25\% errors are due to this specific memorization issue above.}}
    \label{fig:case:1}
    \vspace{-3mm}
\end{figure*}
