%\newcommand{\CLASSINPUTrighttextmargin}{0.63in}
%\newcommand{\CLASSINPUTbottomtextmargin}{0.99in}
\documentclass[10pt, conference]{IEEEtran} % use the `journal` option for ITherm conference style
\usepackage{caption}

\usepackage[top=0.65in, bottom=1.02in, left=0.64in, right=0.64in]{geometry}
\setlength{\columnsep}{0.2in}
\IEEEoverridecommandlockouts

\usepackage{subcaption}
\usepackage{amsmath,mathtools}
\usepackage{amsfonts}
\usepackage{amssymb}
\usepackage{amsthm}
\usepackage{algorithm}
\usepackage{algpseudocode}
\usepackage{booktabs} % For better table lines
%\usepackage{romannum}
%\usepackage{float}
%\usepackage{algorithm}
\usepackage{fancyhdr}
%\usepackage{graphicx}
%\usepackage{textcomp}
\usepackage{xcolor}
\usepackage{multicol}
\usepackage{graphicx}
\usepackage{caption}
\usepackage{subcaption}  % Add these packages in the preamble
%\usepackage[utf8]{inputenc}
%\usepackage[normalem]{ulem}
\hyphenation{op-tical net-works semi-conduc-tor}
\usepackage{soul}
\usepackage{pifont}
\usepackage[breaklinks=true]{hyperref}
\usepackage{boxedminipage}
\usepackage{tikz}
%\usepackage{pgfplots}
\usepackage{multirow}
\usepackage{array}
\usepackage{enumitem,kantlipsum}
\usepackage{comment}
\usepackage{tikz}
\usetikzlibrary{patterns}
\usepackage{tikzscale}
%\usepackage{bbm}
%\usepackage{hyperref}
\newtheorem{theorem}{Motivation}
\newtheorem{observation}{Observation}
\usepackage{url}

%\usepackage{cite}
%\usepackage[natbib=true, style=numeric, sorting=none]{biblatex}

%\addbibresource{ref.bib}
%\usepackage[usenames]{color}
%\def\BibTeX{{\rm B\kern-.05em{\sc i\kern-.025em b}\kern-.08em
 %   T\kern-.1667em\lower.7ex\hbox{E}\kern-.125emX}}
 \newcommand{\todo}[1]{\noindent\textcolor{red}{\bf }[#1]}
 
\begin{document}

\title{\LARGE{ACCORD: \underline{A}pplication \underline{C}ontext-aware \underline{C}ross-layer \underline{O}ptimization and \underline{R}esource \underline{D}esign for 5G/NextG Machine-centric Applications}}
\author{\IEEEauthorblockN{Azuka Chiejina\IEEEauthorrefmark{1}, Subhramoy Mohanti\IEEEauthorrefmark{2}, and Vijay K. Shah\IEEEauthorrefmark{1}}
\IEEEauthorblockA{\IEEEauthorrefmark{1}NC State University, USA, \IEEEauthorrefmark{2}InterDigital Communications, Inc., USA}
\IEEEauthorblockA{Email: ajchieji@ncsu.edu, subhramoy.mohanti@interdigital.com, vijay.shah@ncsu.edu}}
\maketitle
%%%%%%%%%%%%%%%%%%%%%%%%%%%%%%%%%%%%
\begin{abstract}
Retrieval-Augmented Generation (RAG) is often used with Large Language Models (LLMs) to infuse domain knowledge or user-specific information. In RAG, given a user query, a retriever extracts chunks of relevant text from a knowledge base. These chunks are sent to an LLM as part of the input prompt. Typically, any given chunk is repeatedly retrieved across user questions. However, currently, for every question, attention-layers in LLMs fully compute the key values (KVs) repeatedly for the input chunks, as state-of-the-art methods cannot reuse KV-caches when chunks appear at arbitrary locations with arbitrary contexts. Naive reuse leads to output quality degradation.  This leads to potentially redundant computations on expensive GPUs and increases latency. In this work, we propose \sys, a system for managing and reusing precomputed KVs corresponding to the text chunks (we call \textit{chunk-caches}) in RAG-based systems. We present how to identify \hl{\textit{chunk-caches} that are reusable}, how to efficiently perform a small fraction of recomputation to \textit{fix} the cache to maintain output quality, and how to efficiently store and evict \textit{chunk-caches} in the hardware for maximizing reuse while masking any overheads. With real production workloads as well as synthetic datasets, we show that \sys reduces redundant computation by \textbf{51\%} over SOTA prefix-caching and \textbf{75\%} over full recomputation.
\hl{Additionally, with continuous batching on a real production workload, we get a \textbf{1.6$\times$} speedup in throughput and a \textbf{2$\times$} reduction in end-to-end response latency over prefix-caching while maintaining quality, for both the \llama-3-8B and \llama-3-70B models. 
}
\end{abstract}





%%%%%%%%%%%%%%%%%%%%%%%%%%%%%%%%%%%%
\documentclass[../main.tex]{subfiles}
\graphicspath{{../images/}}
\makeatletter
\def\input@path{{../images/}}
\makeatother
\begin{document}
\section{Introduction}
\begin{figure}
\centering
\begin{tikzpicture}
\node[inner sep=0pt] (ws) at (0, 0) {
\includegraphics[height=.4\textwidth, trim={10cm 0 10cm 0},clip]{world_space.png}};
\node[inner sep=0pt] (cs) at (6,0) {\includegraphics[height=.4\textwidth, trim={10cm 1cm 10cm 4cm},clip]{conf_space.png}};
\end{tikzpicture}
\vspace{-5pt}
\label{fig:pbrm_intro}
\caption{\textbf{Left}: Shows world space obstacles as grey spheres. Robots start and goal configuration is colored red and green, respectively. Configurations along the computed path are colored transparent blue. \textbf{Right:} Mapped world space scenario to configuration space. Obstacle region is the grey mesh. Red spheres are collision-free regions computed by the neural SCDF. The optimized shortest path in the convex corridor is the blue curve.}
\vspace{-25pt}
\end{figure}
Motion planning is the problem of finding a collision-free trajectory that connects a given start and goal configuration. The planning takes place in the configuration space of the robot. For single body robots, like mobile robots or drones, the configuration space and the world space are usually the same. This simplifies the planning, since explicit obstacle representations are available which enables geometrical tools like separating hyperplanes, smallest distance to obstacles etc., to be used when designing motion planning algorithms. For multi-body robots like manipulators, the situation is completely different. The world space obstacles are usually mapped to non-convex regions, and to make the problem even harder, the mapping is usually not known. Forming explicit representations of the obstacle region in the configuration space is usually too expensive or intractable. Despite all of this, sampling based planners are used with great success, which mainly is due to their use of implicit representations of the obstacle region. The basic idea is to construct a graph in the configuration space that covers and connects the collision-free region. From this graph, a path can be extracted that connects a given start and goal configuration. The approach is computationally expensive, since the graph is constructed with the smallest geometrical building block available, points, which represents a collision-check. Furthermore, the extracted paths from the graph are non-smooth and jagged due to the stochastic nature of the approach. This adds an additional post-processing step to the process, where the paths are shortcutted and smoothened, before the path can be used for tracking. Clearly a lot of time is invested to form this graph and produce smooth paths. Thus, if the obstacles start to move, then all of this work is done in no use, since all points that make up this graph need to be re-verified, which is simply too time consuming to be done in real time.
\\\\
In this work, we want to address the existing drawbacks of the sampling based planners. Our main contribution is an improved motion planner where each vertex in the graph covers a collision-free region in the form of a sphere instead of a point and where the edges are formed with neighboring intersecting spheres. This representation has the advantage of instead of returning piecewise linear paths, returning a sequence of overlapping spheres, i.e. a convex corridor, that connects a given start and goal configuration, illustrated in Figure \ref{fig:pbrm_intro}. This convex corridor allows us to use convex optimization to produce smooth trajectories, instead of computationally expensive post-processing methods. The representation further allows us to estimate the coverage of the collision-free space, which gives us awareness and feedback in the offline roadmap construction phase. Finally, our representation is simple to adapt to moving obstacles, simply requery for the new radii and recheck for intersections. 
\\\\
The spherical collision-free regions are formed using a signed distance function (SDF), which is a function that returns the smallest distance from an arbitrary point to the boundary of an obstacle. As the name implies, the distance is signed, thus if the point is inside the obstacle it is negative otherwise positive. If the distance is positive, a sphere with radius equal to the distance is guaranteed to cover a collision-free region. Using an SDF in motion planning is not new, but what is novel about our approach is that we express the distance in the configuration space instead of the world space and by doing so allows us to form these convex collision-free regions. We refer to the resulting SDF as a signed configuration distance function (SCDF). Computing an SCDF analytically is non-trivial, our approach is therefore to parameterize the SCDF with a deep neural network and learn the mapping by supervised learning. Our resulting neural SCDF can compute distances for different parameter values of obstacle shapes and we also show how multiple distances can be combined, thus making our approach flexible.
\section{Related work}
Motion planning algorithms can roughly be divided into three families, grid-based, sampling based and optimization based methods. Grid-based methods (GBM) discretize the planning space from which a graph is then compiled. A standard search method is A$^\star$ \citep{a_star}, which is classified as an \textit{informed} search method, since it employs a heuristic function to speed up the search. A$^\star$ guarantees to return an optimal path at the level of discretization used. GBMs usually discretize the planning space by a regular lattice and this limits the GBMs to problems with low dimensionality due to the curse of dimensionality. Thus, GBMs are usually limited to single-body robots where the degrees of freedom (DOF) are low. To overcome the inherent scaling problem with the GBMs, stochastic methods are usually used for multi-body robots. These methods are termed as sampling-based methods (SBM) and core members within this family are the rapidly-exploring random trees (RRT) \citep{rrt} and the probabilistic roadmap (PRM) \citep{prm}. RRT grows a tree from the start configuration and explores the collision-free region in a rapid way until it is able to connect to the goal region. RRT is usually improved by bi-directional planning \citep{rrt_connect}, i.e. an additional tree is grown from the goal configuration and the trees are tested for connection after any tree has been expanded. RRT is a single-query method, thus it searches for a path from scratch each time it is queried. Contrary to this, PRM is a multi-query method, which solves for multiple queries without starting from scratch. PRM does this by creating a roadmap (graph) that covers the collision-free space as an offline step. The graph is then used to solve for multiple queries. PRMs are used in cases where the environment does not change since the extra offline step is too computationally costly and needs to be re-done if the environment is changed. In our work, we address this inherent issue by using a different roadmap representation. Our vertices in the graph cover a collision-free region in the form of spheres and we form the edges by checking for intersecting spheres. If something in the environment changes, we recompute the spheres radii and recheck the intersections, without relying on collision detection. We use a trained neural network to compute the sphere radius, therefore querying for the radius can be done fast, hence our representation enables the PRM for dynamic environments.
\\\\
In the recent decades, optimization based methods (OBM) \citep{chomp, schulman, itomp, stomp} have been introduced as an alternative to SBM for multi-body robots. Like the SBM, the OBMs scale well to higher dimensional problems and produce smoother motion. It is common to use a SDF in the optimization since it is a smooth function, thus enabling gradient-based methods. However, the standard way of expressing the SDF is in world space. The distance therefore needs to be mapped to the configuration space by the forward kinematics. This mapping makes the optimization problem a non-linear program (NLP), which is computationally expensive to solve. Recently, a different approach has been proposed. In \cite{mp_gcs} motion planning is formulated as a convex optimization problem by using the graph of convex sets framework \citep{gcs}. The underlying idea is to decompose the collision-free space into intersecting convex sets from which a convex optimization problem is formulated. In cases where an explicit representation of the obstacles in the configuration space exists, like for single-body robots, creating collision-free convex regions can be done fast \citep{iris}. For multi-body robots, this is non-trivial. Existing work does this successfully \citep{iris_nlp, iris_c} by an optimization based approach, but the methods are still too time consuming to be used in the presence of moving obstacles. Our approach is instead to use deep learning to learn an SDF expressed in the configuration space. With this, we can query for shortest distances to the collision boundary, which allows us to expand spherical regions which are collision-free. Our approach is fast and therefore enables our suggested roadmap planner to be used in dynamic environments.
\\\\
Recent research has focused on learning collision detection \citep{fk_kernel_distance, diffco, graphdistnet} by predicting the signed distance between the robot links and the surrounding obstacles in the world space. The learned SDF is used in trajectory optimization but since the distance is expressed in the world space, the problem becomes an NLP and therefore takes a long time to solve. We take a novel approach and suggest to instead express the signed distance in the configuration space. This allows us to improve the PRM at the same time as it enables convex optimization for trajectory optimization, which runs faster and is more reliable than NLP solvers. In \cite{cspf} a learned signed distance function in the configuration space is proposed similar to our approach. However, their approach is restricted to point cloud representations, while we propose to represent the obstacles as parameterized geometric shapes, e.g. spheres. Furthermore, we also show how to use our learned SCDF to improve an existing roadmap planner.
\section{Problem formulation}
A robot is located in the world space, $\W \subset \R^3 $. The unique location of the robot is given by its configuration $\q \in \C$, where $\C$ is the configuration space. The set of points covered by the robots bodies at a certain configuration is expressed as $\B(\q) \subset \W$. The robot is surrounded by $\NrObst$ obstacles $\O = \bigcup_{i=1}^{\NrObst} \O_i$, where  $\O_i \subset \W$. The representation of the obstacle in the configuration space is the set $\C\O_i = \{\q \in \C \: |\: \B(\q) \cap \O_i \neq \emptyset \}$. The obstacle space is formed as $\Co = \bigcup_{i=1}^{\NrObst} \C \O_i$. The complement is referred to as the free space, $\Cf = \C \setminus \Co$. The path planning problem is a tuple, ($\Cf$, $\qStart$, $\qGoal$), where we want to connect a query pair, consisting of a start, $\qStart$, and goal configuration, $\qGoal$, with a geometric path, $\q(s): [0, 1] \mapsto \Cf$, such that $\q(0)=\qStart$ and $\q(1)=\qGoal$, or report correctly when such a path does not exist.
\end{document}

%%%%%%%%%%%%%%%%%%%%%%%%%%%%%%%%%%%%
\section{Related Work}
% \subsection{Vision Language Model}
% 시각장애인에서 상황을 설명할 DB가 없으니 만들었다. 그리고 이를 VLM에 튜닝했다.
\subsection{Technical approaches for assisting the visually-impaired}


\subsection{Datasets for visual instruction tuning}

%%%%%%%%%%%%%%%%%%%%%%%%%%%%%%%%%%%%
\section{Preliminary Experiments}
\label{sec:preliminary_experiment}
\subsection{Characterizing MCA requirement:}
%\vspace{-0.05in}
MCAs often exhibit dynamic and unpredictable behavior, influenced by external factors such as user actions and environmental conditions.
%We then create the MCA context by combining the requirement with information that can help the network to properly allocate resources to meet the MCA demand. 
Consider the example shown in Fig.\,\ref{fig:app_charac}, where a vehicle is approaching a mobile monitoring device equipped with a camera. This device captures video frames and transmits them to a remote server for license plate identification. The time required for the vehicle to be detected and identified, and for the feedback to be relayed back to the device (e.g., ``track vehicle") determines the necessary round-trip latency of the application. This latency is affected by various factors, including the vehicle speed, the camera field of view (FoV), and the processing time required for identification. In this particular scenario, the round-trip latency is $\sim$1\,second, (including uplink latency), as the vehicle remains within the camera field of view for 33 frames at a rate of 30 frames per second (FPS). If the vehicle's speed were to change or the camera field of view were different, the required latency would also change. 
%Contextual factors that enable the network to properly allocate resources, such as channel quality, MCA traffic information, user mobility, and network buffer are used to make the network {\em aware} of the MCA context.
This example highlights the dynamic nature of MCAs and the need for network optimization strategies that can adapt to their unique and changing requirements.
%Here we showcase through a real-world example how unpredictable external factors such as user behavior, environment parameters, etc., determine machine-centric application requirements such as round-trip latency.\newline
%Fig.\,\ref{fig:app_charac}, shows some of the extracted frames from the video of a car navigating a stretch of a rural highway and approaching a mobile monitoring device with a camera. The camera captures video frames in 1080-pixel resolution at 30 frames per second (FPS). As the car is detected in Frame \#1 through local object detection, the frame is compressed and transmitted via the network to the remote server application for the compute-intensive inferencing task of license plate identification. This frame may not be suitable for license plate identification because of factors that include but are not limited to the distance of the vehicle from the device, the frame compression technique, environmental conditions, etc. 
%The edge server application informs the edge device to continue transmitting frames where the object is detected for edge-based tracking and future position estimation of the object in a future frame. The edge server subsequently informs the edge device through the network to capture a future frame at a specific compression technique where the object is projected to be at a suitable location in the frame with suitable features for carrying out the machine task of object classification and license plate identification. 
%Over the next few frames, as the approaching car size becomes bigger, the license plate becomes easier to identify.
%At frame \#345, the license plate could be positively identified by the remote server application. The car moves away from the camera field of view (FoV) after frame \#378. If the server decides to generate feedback for the client device to take follow-on actions such as track/follow the car, this message needs to be delivered to the client device before the car moves away from the camera FoV. In this scenario, the duration of time the car stays within the camera FoV is determined by external factors such as the vehicle speed, camera field of view span, etc. This in turn determines the application round trip time from license plate identification at frame \#345, to feedback to the device (e.g., follow car) by frame \#378. If the inference decision reaches the mobile device after this time frame, then the feedback will be stale since the vehicle has gone past the camera field of view and the device cannot perform the task required in the feedback, i.e., to follow the car. The time duration from frame \#345 to frame \#378 (total of 33 frames) can be calculated based on the camera frame rate (frames per second or FPS). Here at 30 FPS, the round-trip latency thus needs to be $<$1 second.\newline
%If the car speed was different or changed while the video was being taken or if the camera FoV was narrower or wider then in each scenario the round-trip latency would be different.
\vspace{-0.05in}
\subsection{Network Characterization:}
%\vspace{-0.05in}
We conduct simulation experiments with an MCA (similar to Sec.\,\ref{sec:preliminary_experiment}.A) over a 5G network to understand the uplink performance as the MCA transmit video to a 5G network-connected server.
%To understand how legacy 5G networks perform while serving MCAs, we conduct simulation experiments with an MCA, as explained in Sec.\,\ref{sec:preliminary_experiment}.A, generating and uploading video to a 5G network-connected server. 
We track the application data flow through the network stack and quantify the limitations of using standard-compliant fixed configurations across the PHY, MAC, RLC, and APP layers\,\cite{fezeu:techreport2023}\,\cite{3gppmac}\,\cite{3gpprlc}. 
\begin{table}[t!]
\vspace{-1mm}
 \caption{Configuration Profiles}
 \vspace{-1mm}
 \begin{center}
 %\renewcommand{\arraystretch}{0.65}
 \resizebox{0.482\textwidth}{!}{
  %\begin{tabu}{ | X[0.001cm l] | X[0.6cm l]|| X[0.001cm l] | X[0.6cm l] |}
  %\vspace{-0.6mm}
  \begin{tabular}{|c|c|c|c|}
 \hline
\textbf{Profile}  & \textbf{Frame Rate (FPS)} & \textbf{MAC scheduling}  & \textbf{RLC buffer (KB)}\\
\hline
 1 & 30 & Round Robin & 6 \\
 \hline
 2 & 60 & Round Robin & 6 \\
 \hline
 3 & 60 & Round Robin & 10 \\
 \hline
 4 & 60 & Proportional Fair & 6 \\
 \hline
 \end{tabular}}
 \end{center}
 \label{tab:notations}
 \vspace{-6mm}
\end{table}

\noindent $\bullet$ \textbf {Experiment Setup:} We use a 3GPP-compliant MATLAB 5G toolbox featuring a base station (gNB) having an 8x8 MIMO configuration with a bandwidth (BW) of 10\,MHz with 30\,KHz subcarrier spacing (SCS), and 4 UEs each having 2x2 MIMO configuration at different locations from the gNB (300\,m, 1200\,m, 1500\,m, 3000\,m) and experiencing channel conditions based on the 3GPP standard clustered delay line (CDL) channel model. 
The MCA on each UE generated uplink video frames of size 7.5\,KB, at rates varying between 30\,FPS and 60\,FPS. The MCA requirement was an uplink latency of 18\,ms based on the vehicle speed and camera FoV, calculated based on a scenario similar to the one described in Sec.\,\ref{sec:preliminary_experiment}.A. 
%The video streaming application on each UE generated uplink video frames, with each frame having a size of 7.5\,KB, and frame rates varying between 30\,FPS and 60\,FPS. 
%CQI of 15,13,13 and 8 respectively. These CQI levels reflect how the channel quality of the users is affected by their positions. 
%Each UE is connected to the gNB using the 3GPP clustered delay line (CDL) channel. 
We consider three different PHY slot configurations that are downlink-heavy with 7\,downlink (DL)-3\,uplink (UL) slots \texttt{(Configuration A)}, equalized with 5\,DL-5\,UL slots \texttt{(Configuration B)}, and an uplink-heavy configuration with 3\,DL-7\,UL slots \texttt{(Configuration C)}, each with a periodicity of 5\,ms. Table\,\ref{tab:notations} highlights the considered baseline 3GPP parameters across the other layers. 
%We study the baseline 3GPP Round Robin (RR) and Proportional Fair (PF) algorithms at the MAC layer. The evaluated UL RLC buffer size was 6\,KB and 10\,KB, and the application generated video frames of size 7.5\,KB with a frame rate of 30\,FPS and 60\,FPS.
% \begin{figure}[t!]
% \centering
% \includegraphics[trim=10 1 10 10,clip,width=0.55\linewidth]{Figures/prelim_experiment.png}
% \caption{Preliminary experimental setup for network characterization} %\cite{tilt}}
% \vspace{-5mm}
% \label{fig:prelim-exper}
% \end{figure}

\begin{comment}
\begin{table}[t!]
    \centering
    \vspace{-0.2in}
    \caption{Experiment Parameters}
    \vspace{-0.1in}
    \footnotesize
    
\begin{tabular}{|>{\centering\arraybackslash}p{0.2in}|>{\centering\arraybackslash}p{0.2in}|>{\centering\arraybackslash}p{0.7in}|>{\centering\arraybackslash}p{0.8in}|>{\centering\arraybackslash}p{0.7in}|}
    \hline
    \textbf{Layer} & \textbf{Index} & \multicolumn{3}{c|}{\textbf{Configurations}} \\ \hline
    \multirow{4}{*}{PHY}  &  & \textbf{DL Slots} & \textbf{UL Slots} & \textbf{Periodicity} \\ \cline{2-5}
    & 1 & 7 & 3 & 5 \\ \cline{2-5}
    & 2 & 5 & 5 & 5 \\ \cline{2-5}
    & 3 & 3 & 7 & 5 \\ \hline

    \multirow{4}{*}{MAC} &  & \textbf{Scheduler} & \textbf{Resource Allocation Type} & \textbf{-} \\ \cline{2-5}
    & 1 & Round Robin (RR) & 0 & - \\ \cline{2-5}
    % & Round Robin (RR) & 1 & - \\ \cline{2-5}
    & 2 & Proportional Fair (PF) & 1 & - \\ \hline

    \multirow{2}{*}{RLC} &  & \textbf{TX Buffer Size} & \textbf{SN Field Length} & \textbf{Entity} \\ \cline{2-5}
    & 1 & 2 & 6 & UMUL \\ \cline{2-5}
    & 2 & 4 & 6 & UMUL \\ \cline{2-5}
    & 3 & 6 & 6 & UMUL \\ \cline{2-5}
    & 4 & 8 & 6 & UMUL \\ \cline{2-5}
    & 5 & 10 & 6 & UMUL \\ \hline

    \multirow{3}{*}{APP} &  & \textbf{Frame Size (kb)} & \textbf{Frame Interval (ms)} & \textbf{Frame Rate (FPS)} \\ \cline{2-5}
    & 1 & 7.5 & 33 & 30 \\ \cline{2-5}
    & 2 & 7.5 & 16 & 60 \\ \hline
\end{tabular}
\label{tab:prelim-exper}
\end{table}
\end{comment}
%Table \ref{tab:prelim-exper} highlights the different parameters across the different layers considered, specifically, we consider parameters across the layers that affect the latency of UL data transmission. The following should also be noted: For the MAC layer, there was no custom scheduler implemented, rather, we consider the scheduling algorithms already implemented in the 5G toolbox. For the RLC layer, other parameters like prioritized bitrate, bucket size duration, and priority have significant effect on latency when considering multiple applications running on the same UE. This is a result of the logical channel prioritization in 5G (cite 3GPP spec), which is designed in such a way as to ensure that the UE satisfies the QoS requirements of each configured radio bearer for each application. But for our experiment, we are considering a single type of video streaming MCA application running on each UE. 
\noindent $\bullet$ \textbf {Observation:} The experimental results presented in Fig.\,\ref{fig:prelim_images} illustrate the impact of various 5G layer configurations on achieving the \textcolor{black}{target latency of 18\,ms} for 4\,UEs situated at different distances from the gNB. In Profile\,1, (Fig.\,\ref{fig:prelim_images_a}), modifying only the PHY parameters, specifically the DL and UL slot allocations, resulted in latency improvements for UEs\,2,\,3,\,and\,4, while UE\,1 experienced no significant change. This suggests that PHY layer adjustments alone may not guarantee latency requirements for UEs experiencing different channel conditions.
%particularly those closer to the gNB. 
Profiles\,2,\,3,\,and\,4 (Fig.\,\ref{fig:prelim_images_b},\,\ref{fig:prelim_images_c},\,\,\ref{fig:prelim_images_d}) demonstrate the impact of adjusting parameters across PHY, MAC, RLC, and APP layers together in meeting the required latency under different channel conditions. These configurations yielded more consistent latency improvements across all the UEs for specific configuration profiles.
%highlighting the importance of a holistic approach to network optimization.
%Using the discussed experiment setup and configuration parameters across the different layers, we show in Fig.\,\ref{fig:prelim_images}, the results of the achieved latency of each UE connected to the gNB in meeting a desired latency requirement desired by the application running. In Fig.\,\ref{fig:prelim_images}a, using the first combination of configuration parameters from the table, we can observe how changing parameters in the PHY layer while keeping other parameters fixed improves the required latency for the last three UEs while the first UE does not have any performance gain. For Fig. \ref{fig:prelim_images}b-c, we can observe how in addition to adjusting parameters in the PHY layer, how adjusting parameters in the MAC, RLC and APP layers improve the overall latency accross all UEs.

%This gives us an intuition that just adjusting the PHY layer parameters does not ensure that the application requirements are met for all UEs, therefore, other higher blayers have to be considered for each UE to meet the desired latency threshold requirements. In addition to these observations, we can observe how this approach of optimizing the network to meet the demands of MTC applications in real-time would be ineffective as traditional networks do not optimize the networks this way to 
{\em Our observations highlight the limitations of relying solely on fixed configurations or isolated layer-specific optimizations in 5G to meet diverse MCA requirements, especially in varying channel conditions.
%As evidenced by the varying latency responses across different UEs, a more nuanced and adaptive approach is needed. 
This motivates the need for exploring optimization strategies that consider the interplay between network and application layers, enabling dynamic adjustments tailored to individual MCA needs. }
%Such an approach is crucial for ensuring consistent and reliable QoS across emerging 5G applications and deployment scenarios.
%Adjusting certain NW layer parameters by following the fixed configurations across does not ensure that the application requirements are met for all UEs, therefore, informed optimization across the NW and application layers needs to be considered for each UE to meet the desired latency threshold requirements.
% The results from the experiments are shown in Fig.\,\ref{fig:prelim-thpt}.
% Here, we observe UL throughput of $\sim$8\,Mbps for the 7\,DL 3\,UL  configuration, $\sim$10\,Mbps for the 6\,DL 4\,UL configuration, $\sim$4.5\,Mbps for the 8\,DL 2\,UL configuration and $\sim$14\,Mbps for 5\,DL 5\,UL configuration. DL throughput ranged between 25\,-\,35\,Mbps in these frame configurations.
% These results underscore a clear inverse relationship between DL-heavy PHY frame configurations and UL throughput. Note that the UL to DL throughput ratio does not exactly follow the slot allocation ratio for two reasons: First, for UL transmissions, the UE utilizes one whole F/mixed slot to send a scheduling request to notify the NW regarding incoming UL data transmission in the upcoming frame cycle. Second, in 3GPP for a given MCS, the maximum data transport block size for DL is higher compared to UL\,\cite{snr-cqi-mcs}, leading to such asymmetry in throughput.
% \textcolor{blue}{to add a sentence to justify thpt difference between UL and DL}
\begin{comment}
\begin{figure}[t!]
	\centering
	\includegraphics[trim=0 30 0 0,clip,width=0.6\linewidth]{Figures/prelim_thpt.pdf}
	\caption{Iperf uplink throughput in OAI with different PHY frame configurations (7-2-1, 6-3-1, 8-1-1) } %\cite{tilt}}
 \vspace{-4mm}
	\label{fig:prelim-thpt}
\end{figure}
\end{comment}
%The 8\,DL 2\,UL slot configuration exhibited the lowest UL throughput, while the 5\,DL 5\,UL configuration demonstrated the highest among the considered PHY frame configurations, which is due to the reason of 5\,DL 5\,UL configuration having more UL slots compared to the others. 
%\vspace{-1mm}
\begin{comment}
\begin{theorem}
Reactive time domain resource allocation based on fixed PHY frame configurations may excel in either DL or UL, but in the presence of heterogeneous traffic demands, this procedure may struggle to achieve a balance between UL and DL transmissions. Dynamic PHY frame reconfiguration, informed by predicted channel conditions and application demand, holds promise in mitigating this imbalance, fostering coexistence between DL and UL-heavy applications.
\end{theorem}
\end{comment}
\begin{comment}
\noindent $\bullet$ \textbf{Latency performance:}  In this study, we conducted a series of experiments, varying packet sizes and evaluating the latency across multiple DL-heavy slot configurations within the context of the 5G PHY frame. The results are shown in Fig.\,\ref{fig:prelim-lat}. The observed latency values revealed a compelling pattern. For a 100\,KB packet size, we measured latency of 0.09 seconds with the 6-3-1 frame configuration, 0.16 seconds with the 7-2-1 configuration, and 0.22 seconds with the 8-1-1 configuration. As the packet size increased to 459\,KB, latency grew to 0.3\,seconds for the 6-3-1 configuration, 0.48\,seconds for the 7-2-1 configuration, and 0.8\,seconds for the 8-1-1 configuration. Similarly, with a 1\,MB packet size, we observed latency of 0.7\,seconds for the 6-3-1 configuration, 1.1\,seconds for the 7-2-1 configuration, and 1.9\,seconds for the 8-1-1 configuration. These results distinctly highlight the adverse effect of DL-heavy slot configurations on UL latency, with the 8-1-1 configuration demonstrating the highest latency and the 6-3-1 configuration achieving the lowest latency among the considered PHY frame configurations.

The KPI results from the preliminary experiments offer valuable insights into optimizing 5G network performance, emphasizing the importance of carefully balancing DL and UL resource allocation for enhanced efficiency and maintaining diverse application QoS thresholds.
\end{comment}
\begin{comment}
\begin{figure}[t!]
	\centering
	\includegraphics[trim=0 0 0 0,clip,width=0.7\linewidth]{Figures/prelim_latency.pdf}
	\caption{Latency in transmitting data of varying sizes through OAI uplink with different PHY frame configurations (7-2-1, 6-3-1, 8-1-1) } %\cite{tilt}}
 \vspace{-4mm}
	\label{fig:prelim-lat}
\end{figure}
\end{comment}
\begin{comment}
\begin{figure}[t!]
  \centering
  \begin{minipage}[t]{\linewidth}  
  \subfloat[Uplink throughput]{\includegraphics[trim=0 0 10 0,clip,width=0.5\textwidth]{Figures/prelim_thpt.pdf}\label{fig:prelim-thpt}}
  \hfill
  \subfloat[Uplink latency]{\includegraphics[trim=0 0 0 0,clip,width=0.5\textwidth]{Figures/prelim_latency.pdf}\label{fig:prelim-lat}}
  \label{}
  \caption{KPIs for legacy 5G OAI network, with (a) Iperf uplink throughput in OAI with different PHY frame configurations (7-2-1, 6-3-1, 8-1-1, and (b) Latency in transmitting data of varying sizes through OAI uplink with different PHY frame configurations (7-2-1, 6-3-1, 8-1-1) [to be updated]}
  \vspace{-3mm}
  \end{minipage}
  \vspace{-3mm}
\end{figure}
\end{comment}
\begin{comment}
\begin{figure}[t!]
    \centering
    \includegraphics[trim=0 0 0 0,width=0.6\linewidth]{Figures/OAI_legacy_thpt.png}
    \caption{KPIs for legacy 5G network, with Iperf uplink throughput in OAI with different PHY frame configurations.}
    \vspace{-6mm}
    \label{fig:prelim-thpt}
\end{figure}
\end{comment}
%\vspace{-1mm}
%%%%%%%%%%%%%%%%%%%%%%%%%%%%%%%%%%%%
%\vspace{-0.1in}
\section{Accord framework description}
\label{sec:solution}
\vspace{0.05in}
To enable the cross-network and application layer optimization, we develop and implement the ACCORD framework over a 5G network. We first provide a detailed workflow of the framework, referring to Fig.\,\ref{fig:arch}, and then explain the details of the DRL solution that resides at the core of the framework.
%\vspace{-0.05in}
\subsection{ACCORD workflow}
%\vspace{-0.05in}
As the MCA client-server communication instantiates over 5G, ACCORD starts to optimize the network and application through the following steps:

\noindent $\bullet$ \textbf{Step 1-2: } An MCA client, running on the UE, generates video frames (user data) by observing events in the environment (e.g., vehicle detection). After characterizing the requirement (explained in Sec.\,\ref{sec:preliminary_experiment}.A), the MCA client transmits the user data to the server using legacy network configuration parameters. The MCA client also transmits the calculated requirement along with information about UE buffer space and channel quality indicator (CQI) through uplink control information (UCI) messages to the gNB.

\noindent $\bullet$ \textbf{Step 3 -  Building MCA context: } After receiving the user and control data from the UE, ACCORD combines this information with network control data (UE mobility/position and MCS), to build the context for MCA requirement.

\noindent $\bullet$ \textbf{Step 4-5:} The context information is used as input to the DRL agent to perform cross-layer optimization by selecting the optimal set of configurations across the network and application layers, that meet the MCA requirement.

\noindent $\bullet$ \textbf{Step 6-7:} The lower network layer configurations (PHY, MAC) from ACCORD are implemented at the network side (e.g., gNB). The RLC and APP configurations are transmitted to the UE through the downlink control information (DCI).

In \texttt{Step\,3.1}, the MCA server uses the uplink data from the client (that was transmitted in Step 2) to make inferences and generates that feedback to the UE through the network as seen in \texttt{Steps\,3.2} and \texttt{3.3}.\newline
%This information is passed through the downlink channel to the UE then the UE performs necessary reconfiguration.
% \begin{figure}[b!]
%     \centering
%     \includegraphics[width=1\linewidth]{Figures/drl_figures/1uestatic_rewards.pdf}
%     \caption{Performance in total rewards during training on the y-axis and training episodes on the x-axis. The solid lines
% represent smoothed (window size of 2 episodes) averages, while shaded areas
% show the standard deviations. Epsilon greedy is the acting policy while the greedy policy is the updating policy}
%     % \vspace{-5mm}
%     \label{fig:1ue_static_rewards}
% \end{figure}

% \begin{figure}[b!]
%     \centering
%     \includegraphics[width=1\linewidth]{Figures/drl_figures/1uemobile_rewards.pdf}
%     \caption{Total rewards gotten per episode for a single mobile user for different latency requirements.}
%     % \vspace{-5mm}
%     \label{fig:1ue_mobile_rewards}
% \end{figure}

% \begin{figure}[b!]
%     \centering
%     \includegraphics[width=1\linewidth]{Figures/drl_figures/2uestaticandmobile_rewards.pdf}
%     \caption{Total rewards achieved per episode for two static and mobile users for latency requirements of 24ms and 30ms respectively.}
%     % \vspace{-5mm}
%     \label{fig:2ue_static_rewards}
% \end{figure}



% \begin{figure}[b!]
%     \centering
%     \includegraphics[width=1\linewidth]{Figures/drl_figures/1uestaticandmobile_losses.pdf}
%     \caption{Loss 1UE.}
%     % \vspace{-5mm}
%     \label{fig:2ue_static_rewards}
% \end{figure}


% \begin{figure}[b!]
%     \centering
%     \includegraphics[width=1\linewidth]{Figures/drl_figures/2uestaticandmobile_losses.pdf}
%     \caption{Loss 1UE.}
%     % \vspace{-5mm}
%     \label{fig:2ue_static_rewards}
% \end{figure}


% \begin{figure}[t!]
%     \centering
%     \includegraphics[trim=30 0 30 0,width=0.75\linewidth]{Figures/dynamicTDD_flow_1.pdf}
%     \caption{Dynamic PHY frame reconfiguration at NW aided by UE forecast of local channel metrics.}
%     \vspace{-5mm}
%     \label{fig:phy_reconfig}
% \end{figure}
% \begin{figure}[t!]
% \centering
% \includegraphics[trim=25 10 20 0,width=0.85\linewidth]{Figures/framework_overview.png}
% \caption{Sliding window mechanism at UE in using historical channel metrics (in orange) to predict the channel metrics of future time slots (in green).}
% \vspace{-5mm}
% \label{fig:sliding_window}
% \end{figure}
%\vspace{-0.26in}
\vspace{-4mm}
\subsection{Proposed DRL approach for context awareness}
%\vspace{-0.05in}
To develop the context-aware cross-network and application layer optimization in ACCORD, we train a DRL agent by making it interact with the 5G environment simulated in the 3GPP-compliant MATLAB 5G toolbox. The DRL objective is to learn the optimal configuration parameters across the PHY, MAC, RLC, and APP layers based on the MCA context. 
%This context is constructed from the APP requirements, channel quality indicator (CQI), MCS, UE buffer size, UE position, and APP throughput. 
Through this interaction, the agent obtains a reward related to the objective function of meeting the latency requirements of the UE while using minimal network resources to save spectrum. This approach was used as opposed to the supervised learning procedure which necessitates a comprehensive dataset capturing all possible network conditions, UE behaviors, and requirements, which is impractical considering the unpredictable MCA requirement and complexity of real-world situations. The optimization problem can be expressed as a Markov decision process (MDP) of the DRL, which we describe as follows:
% The overall framework for solving the problem was designed using DQN, with the 3GPP compliant MATLAB 5G toolbox using the MATLAB engine API\,\cite{matlab_engine_python}. This framework is shown in Fig.\,\ref{fig:contextaware_architecture}. DQN is a value-based method\,\cite{mnih2013playing} that uses neural networks for training agents to find an optimal policy through a parameterized optimal action-value function $Q(s, a; \theta)$, where $\theta$ are the parameters of the neural network that inputs the current system state $s$ at time $t$ to produce all possible state-action values.
% The goal of the agent is to interact with the environment by selecting actions in a way that maximizes future rewards, with an assumption that future rewards are discounted by a factor of $\gamma$ per time-step, and define the future discounted return at time $t$ as $R_t = \sum_{t}^{T} \gamma^{t} r_{t}$, where $\gamma\in [0, 1]$ weights the importance of future rewards in the cumulative sum of rewards and  $T$ is the time step termination occurs, in our case this is when the episode ends.
% \begin{figure}[t!]
%     \centering
%     \vspace{-2mm}
% \includegraphics[width=0.85\linewidth]{Figures/drl_figures/drl_agent_architecture.pdf}
%     \vspace{-2mm}
%     \caption{Context Aware Cross Layer DRL Framework.}
%     \vspace{-5mm} \label{fig:contextaware_architecture}
% \end{figure}
% To achieve this goal, we need to find an optimal policy that minimizes the overall network latency based on user application requirements and network conditions. This optimal policy which is also a greedy policy would be derived from the optimal action-value function as $\pi_\theta(s) = \underset{a}{\operatorname{argmax}} \, Q(s, a;\theta)$
% \begin{comment}
% \begin{equation}
% \pi_\theta(s) = \underset{a}{\operatorname{argmax}} \, Q(s, a;\theta)
% \label{eq:optimal_policy}
% \end{equation}
% \end{comment}
% To derive this optimal policy, we go into details on how we designed the framework for training the DQN by explaining our state, action, and reward spaces along with the neural network and environment used to generate the states.

\noindent $\bullet$ \textbf{State: } The state space $s_{t}$ can be modeled as the combination of the application required latency ($\mathcal{L}_t$), transmitted bytes ($c_t$)), UE location ($g_t$), MAC buffer status report (BSR) ($k_t$), RLC buffer size ($p_t$), CQI ($v_t$) and MCS ($u_t$) at current time window $t$ with a window size of $w$ for all the UEs in set $U$. Overall, $s_{t}$ is formulated as $s_{t} = \{ \mathcal{L}_t, c_t, g_t, k_t, p_t, v_t, u_t\}$. The $g_t$ is a tuple of dimension 2 corresponding to the 2D coordinates of the target location.
%It should be noted that these are all possible features that can be gotten from the environment and all would necessarily not be used for training the DQN as we would see in the performance evaluation.

\noindent $\bullet$ \textbf{Action: } The action space is designed to allow the DQN agent to generate the PHY, MAC, RLC and APP layer configuration parameters at time $t$ for time window $t+w$ (the next time window). We define the action space $\mathcal{A}_{t}$ for all the UEs in $U$ ($|U| = \mathcal{K}$), as the combination of $C_{t}^{PHY}$ (PHY frame configuration), $C_{t}^{MAC}$ (MAC configuration), $C_{t}^{RLC}$ (RLC configuration) and $C_{t}^{APP}$ (APP configuration), at time $t$ for time window $t+w$, hence, $\mathcal{A}_{t} =\mathcal{K} \times \{C_{t}^{PHY}, C_{t}^{MAC}, C_{t}^{RLC}, C_{t}^{RLC}\}$. The action $\mathcal{A}_{t}$ is generated by the designed policy $\pi_{\theta}({s_t})$ from the state $s_{t}$. Hence, the action of time $t$ is formulated as ${A}_{t} = \pi_{\theta}(s_{t})$.

\noindent $\bullet$ \textbf {Reward: } The reward space is a function of the action $\mathcal{A}_{t}$ generated at time $t$ and transition to the next state $s_{t+w}$ at time $t+w$ for each UE as: $f^{\mathcal{R}}_{\mathcal{A}}(\mathcal{A}_{t},s_{t+w}) = r_{t}$, where $r_{t}$ is the achieved reward for a UE at time $t$. Our reward function is defined as $r_{t}=
\left( \frac{1}{1 + e^{-\beta (l_t - \mathcal{L}_t)}} \right) \times 100$ for $l_t \leq \mathcal{L}_t$ and $r_{t}= 0$ otherwise.
% We carefully design our reward function to account for instances where the achieved latency $l_t$ at time $t$ is either greater or less than the application required latency $\mathcal{L}_t$. This reward is defined as:
\begin{comment}
\begin{equation}
r_{t}=
\begin{cases} 
\left( \frac{1}{1 + e^{-\beta (l_t - \mathcal{L}_t)}} \right) \times 100 & \text{for } l_t \leq \mathcal{L}_t \\[10pt]
0 & \text{else }
\end{cases}
\label{eq:reward}
\end{equation}
\end{comment}
$\beta$ is chosen in such a way as to control the steepness of the sigmoid function utilized in the equation. Smaller values of $\beta$ ensures that the function is smoother, causing $r_t$ to change more gradually as $l_t$ moves around $\mathcal{L}_t$.\newline
% \noindent $\bullet$ \textbf {Environment: } Finally the environment is parameterized as ${s_{t+w}, r_{t}} = \psi(\mathcal{A}_{t})$, where the environment $\psi$ takes the inputs of the generated action by the DRL network at time window $t$ and generates the next state for time window $t+w$ and reward for taking action $A_t$ at time $t$.

{\em Our goal is to find an optimal policy that minimizes the overall network latency based on real-time MCA requirements and network conditions while conserving network resources. This is achieved by mapping the state space to an action space that maximizes the accumulated reward. We employ the Deep Q-Network (DQN) algorithm\,\cite{mnih2013playing} explained in Algorithm\,\ref{Algo:dqn} and adapt it to solve our MDP.}
%This algorithm is explained in Algorithm \ref{Algo:dqn}.
\begin{algorithm}[t!] 
\caption{Context-Aware DQN in ACCORD}
\scriptsize
\begin{algorithmic}
\State \textbf{Initialization:} Experience replay memory $\mathcal{D}$ to capacity $N$, batch size, main network parameters $\theta$, target network parameters $\theta^- = \theta$, update rate of target network $\tau$, number of episodes $N_{\text{eps}}$, number of time steps in each episode $N_{\text{step}}$, learning rate and epsilon $\epsilon$ for the exploitation and exploration.
\For{\textit{episode} = 1 : $N_{\text{eps}}$}
    \State Initialize processed state from environment $\phi_t$ =  $\phi(s_t)$;
    \For{\textit{step} = 1 : $N_{\text{step}}$}
        \State Using epsilon greedy as in \cite{mnih2013playing};
        \State With probability $\epsilon$, select a random action $a_t$
        \State Otherwise, select $a_t = \max_a Q^*(\phi_t, a; \theta)$
        \State Execute action $a_t$ and observe reward $r_t$ and $s_{t+w}$
        \State Process $s_{t+w}$ and store $(\phi_t, a_t,\phi_{t+w}, r_t)$ in $\mathcal{D}$
        \State \textbf{Optimize model:}
         \If{memory has sufficient transitions}
            \State Sample a random mini-batch of transitions $(\phi_k, a_k, \phi_{k+w},r_k)$ from $\mathcal{D}$
            \State Compute $Q(\phi_k, a_k; \theta)$ for each state-action pair in the batch
             \State Compute target Q values based on if $\phi_{k+w}$ is terminal or not as in \cite{mnih2013playing}
            \State Compute loss between $Q(\phi_k, a_k; \theta)$ and the target Q values
            \State Perform gradient descent to update $\theta$;
        \EndIf
        %\State \textbf{Update target network:}
            \State Soft update target network: $\theta^{-} \gets \tau \theta + (1 - \tau)\theta^{-}$
    \EndFor
\EndFor
\end{algorithmic}
\label{Algo:dqn}
%\vspace{-0.5mm}
\end{algorithm}
%\vspace{-4mm}
%Based on this framework, we now give an overview of how the DQN is initialized and trained using experienced replay. After the environment is initialized, the behavior distribution of the action selection follows a tradeoff between exploration and exploitation through the standard $\epsilon-$greedy policy. A standard approach for handling this exploration and exploitation is that with probability $\epsilon$, a random action is selected, while with probability 1-$\epsilon$, a greedy action is specified earlier. Based on this policy, at step $t$, the agent performs an action, transitions to the next state, and receives a reward. This whole process forms an experience of the tuple $(s_t,a_t,r_t,s_{t+w})$ and is stored in an initialized experience replay buffer.

% \begin{figure*}[t!]
% \centering % <-- added
% \begin{subfigure}{0.2\linewidth}
%         \includegraphics[width=\linewidth]{Figures/drl_figures/1uestatic_rewards.pdf}
%         \caption{Single Static UE}
%         \label{fig:single_static_ue}
% \end{subfigure}\hfil % <-- added
% \begin{subfigure}{0.2\linewidth}
%         \includegraphics[width=\linewidth]{Figures/drl_figures/1uemobile_rewards.pdf}
%         \caption{Single Mobile UE}
%         \label{fig:single_mobile_ue}
% \end{subfigure}\hfil % <-- added
% \begin{subfigure}{0.2\linewidth}
%         \includegraphics[width=\linewidth]{Figures/drl_figures/2uestaticandmobile_rewards.pdf}
%         \caption{Multiple UEs}
%         \label{fig:multiple_ues}
% \end{subfigure}\hfil % <-- added
% \caption{Rewards achieved by the DRL agent for optimizing the network for the different network setup scenarios considered \textcolor{red}{remove fig.b and place the remaining two plots side by side in one column like fig.7}}
% \vspace{-3mm}
% \label{fig:rewards}
% \end{figure*}


% \begin{figure}[t!]
%     \centering
%     %\vspace{-3mm}
% \includegraphics[width=\linewidth]{Figures/UE-NW-signaling.pdf}
%     \vspace{-3mm}
%     \caption{ACCORD implementation in 5G}
%     \vspace{-5mm}
% \label{fig:drl_agent_architecture}
% \end{figure}


%In training the DQN, there are two neural networks used, one for the policy network and the other for the target network. The neural network architecture used for both networks can be seen in Fig. \ref{fig:drl_agent_architecture}. We randomly sample a batch $B$ of experiences from the replay memory and calculate the target Q-value over the batch of samples as $y = r$ if $s_{t+w}$ is terminal and $y= [r + \gamma \max_{a_{t+w}} Q_{\text{target}}(s_{t+w}, a_{t+w})]$ otherwise.
\begin{comment}
\begin{equation}
y = 
\begin{cases} 
r & \text{if } s_{t+w} \text{ is terminal} \\
r + \gamma \max_{a_{t+w}} Q_{\text{target}}(s_{t+w}, a_{t+w}) & \text{otherwise}
\end{cases}
\end{equation}
\end{comment}
%Here, $Q_{target}$ is a copy of the Q-network but is updated less frequently. Finally, we optimize the Q-network by minimizing the loss between the predicted Q-value and the target Q-value and then use gradient descent to update the Q-network weights $\theta$, where $L(\theta) = [\frac{1}{|B|} \sum (y - Q(s, a; \theta))^2]$
\begin{comment}
\begin{equation}
L(\theta) = \frac{1}{|B|} \sum (y - Q(s, a; \theta))^2
\end{equation}
\end{comment}
%Periodically, the weights from the Q-network are copied to the target network. This process is repeated while the experiences are being collected over multiple episodes until the agent learns an optimal policy.
%\vspace{-1mm}
%%%%%%%%%%%%%%%%%%%%%%%%%%%%%%%%%%%%
%\section{Data Collection}
We conducted our experiment in both English and Chinese settings. 
We collected 800 English texts from arXiv, Quora, Wiki, and CNN news in equal proportions to form a original English text dataset.
For Chinese, We randomly selected 800 Chinese translations from WMT2019.
Both datasets were recursively paraphrased 15 times by the current widely used LLMs respectively. 
Additionally, we tested four different prompts to further explore the universality of our findings.
Table 1 shows the details of our experiment setting.
\begin{table*}[h!]
\centering
\begin{tabular}{lccccc}
\toprule
\textbf{Model} & \textbf{arXiv} & \textbf{Quora} & \textbf{Wiki} & \textbf{CNN} & \textbf{WMT2019} \\
\midrule
Original & 200 & 200 & 200 & 200 & 800 \\
chatgpt & 3000 & 3000 & 3000 & 3000 &- \\
llama3-8B-chat & 3000 & 3000 & 3000 & 3000 &- \\
Mistral & 3000 & 3000 & 3000 & 3000 &- \\
GLM4-chat & 3000 & 3000 & 3000 & 3000 & 12000 \\
Baichuan3-Turbo & 3000 & 3000 & 3000 & 3000 & 12000 \\
\bottomrule
\end{tabular}
\caption{Model and Text Number Information.The paraphrase text number is determined by multiplying the number of stages by the number of texts in each stage. }
\label{table:model_text_number_info}
\end{table*}

%%%%%%%%%%%%%%%%%%%%%%%%%%%%%%%%%%%%
%\section{Implementation Environment}
\label{sec:implementation_environment}

Here we introduce the detailed implementation details and environment for reproducibility purpose. For our model, we choose hyperparameters based on the performance on validation set (Document classification task in the main paper explains how we split validation set). The results in the main paper are obtain by 5 independent runs. The standard deviations reported in the main paper are 1-sigma error bars and are obtained by calling its corresponding function in Excel library. All the experiments were done on Linux server with an NVIDIA A40 GPU with 46,068 MiB. Its operating system is CentOS Linux 7 (Core). We implemented our proposed model GTFormer using Python 3.10 as programming language and PyTorch 2.0.0 as deep learning library. Other frameworks include NumPy 1.23.1, sklearn 0.23.2, and scipy 1.5.2. We emphasize that the main focus of our model is effectiveness, instead of running efficiency. But for completeness, we still make a short comment on execution time. Our model is efficient, on the largest dataset Web, the training takes less than 40 hours to converge. We will release code and datasets upon publication.
%%%%%%%%%%%%%%%%%%%%%%%%%%%%%%%%%%%%
% \vspace{-0.01in}
\section{Performance Evaluation}
%\vspace{-0.05in}
We evaluate the performance of ACCORD in 5G when serving the example MCA in two distinct scenarios, each representing a practical use case with unique characteristics and challenges.
%\vspace{-0.05in}
\subsection{Experiments}
%\vspace{-1mm}
The considered scenarios are simulated with gNB, UEs and MCA configurations mirroring those in the preliminary experiments (Sec.\,\ref{sec:preliminary_experiment}) but with different latency requirements in a reduced BW of 5\,MHz to show spectrum efficiency.\newline
\noindent $\bullet$ \textbf{Scenario A -  Stationary UEs: }
%This scenario simulates applications with stationary UEs, such as fixed traffic cameras for edge-assisted navigation. 
We consider two cases: (1) a single UE positioned 5000\,m from the gNB with a CQI of 10, and (2) two UEs at distances of 600\,m and 3600\,m, with CQIs of 15 and 8, respectively.\newline
\noindent $\bullet$ \textbf{Scenario B - Mobile UEs: }
%This scenario emulates applications involving mobile UEs, such as drone-based tracking. 
Here also we consider two cases: (1) a single UE moving away from the gNB at 60\,mph with decreasing CQI over time, and (2) two UEs, one with degrading CQI and the other with improving CQI.\newline
In both scenarios, the gNB and UEs are initialized with random configurations across all layers.\newline
\noindent $\bullet$ \textbf{DRL Architecture \& Training: }The DRL agent in ACCORD employs a DQN algorithm described in Sec.\,\ref{sec:solution}.B and Algorithm\,\ref{Algo:dqn} with the following hyper-parameters: batch size\,128, discount factor (gamma)\,0.2, replay memory capacity\,10,000, and Adam optimizer with a learning rate\,0.0001. The neural network architecture, depicted in Fig.\,\ref{fig:drl_agent_architecture}, utilizes a multilayer perceptron with ReLU activation functions, fully connected layers (FC) with 256 neurons each, and batch normalization (BN) after specific layers. The input to the network consists of key performance metrics collected every 10\,ms that form the State space described in Sec.\,\ref{sec:solution}.B.
%uplink transmitted bytes, MCS, CQI, and application-required latency. 
These metrics are aggregated over a 100\,ms window, preprocessed, and fed into the network as a vector of shape (40,1) for a single UE and (80,1) for two UEs. The network output is mapped to specific configurations across the PHY, MAC, RLC, and APP layers using a predefined codebook. Based on the selected action, the reward is generated using the reward function described in Sec.\,\ref{sec:solution}.B.

\begin{figure}[t!]
    \centering
    %\vspace{-3mm}
\includegraphics[trim=20 10 40 20,clip,width=\linewidth]{Figures/drl_figures/drl_network_1.pdf}
    \vspace{-5mm}
    \caption{MLP Model used for training the DQN in ACCORD.}
    \vspace{-2mm}
\label{fig:drl_agent_architecture}
\end{figure}

\begin{figure}[h!] % Force placement at the top of the column
    \centering
    %\vspace{-1mm} % Adjust as needed to control vertical spacing
    \begin{subfigure}{0.48\columnwidth} % Adjust width to fit within a single column
        \includegraphics[width=\linewidth]{Figures/drl_figures/1uestatic_rewards.pdf}
        \caption{Single static UE.}
        \label{fig:static_ue_a}
    \end{subfigure}
    \hspace{1mm} % Small horizontal space between subfigures
    \begin{subfigure}{0.48\columnwidth} % Adjust width to fit within a single column
        \includegraphics[width=\linewidth]{Figures/drl_figures/2uestaticandmobile_rewards_with_1UE.pdf}
        \caption{Multiple UEs, static\,\&\,mobile.}
        \label{fig:static_ue_b}
    \end{subfigure}
    \vspace{-5mm}
    \caption{Rewards achieved by the DRL agent for optimizing the network for different network setup scenarios considered.}
    \vspace{-3mm} % Reduce space after the figure
    \label{fig:rewards}
\end{figure}
\begin{comment}
\begin{table}[h]
\centering
\caption{Common DRL Hyperparameters and Values}
\begin{tabular}{ll}
\toprule
\textbf{Hyperparameters} & \textbf{Value} \\
\midrule
\multicolumn{2}{l}{\textbf{DQN Agent}} \\
Batch size & 128 \\
Gamma & 0.2 \\
Replay Memory capacity & 10000 \\
% Number of iterations & 400 \\
% Training steps & 100000 \\

\midrule
\multicolumn{2}{l}{\textbf{Optimizer}} \\
Optimizer & AdamOptimizer \\
Learning rate & $0.0001$ \\

\midrule
\multicolumn{2}{l}{\textbf{Neural Network (Fig. \ref{fig:drl_agent_architecture})}} \\
% Conv1D Layer & filters=32 \\
% & kernel size=$B=8$ \\
% & strides=$B=8$ \\
% & activation=ReLU \\
% Flatten Layer & 225 neurons \\
% Dense Layer 1 & 128 neurons \\
% Dense Layer 2 & 32 neurons \\
% Dense Layer 3 & 1400 neurons \\
\bottomrule
\end{tabular}
\label{tab:dqnparams}
\end{table}
\end{comment}
% \vspace{-2mm}
%\textbf{Evaluated DRL Architecture:}
%For training the DRL agent using the DQN algorithm described in section \ref{sec:solution}, the hyperparameters used were \texttt{Batch Size 128}, \texttt{Gamma 0.2}, \texttt{Replay memory} capacity \texttt{10000}, and \texttt{Adam optimizer} with learning rate \texttt{0.0001}. These hyperparameters are common to both scenarios considered and the multilayer perceptron architecture used is shown in Fig. \ref{fig:drl_agent_architecture}. For the input, we narrow down our metrics considered to the uplink transmitted bytes, MCS, CQI and the application required latency. These metrics are analyzed for every 10ms frame which contain 20 slots based on the periodicity we have chosen. These metrics are then accumulated for 100ms, are then preprocessed and then form a shape of (40,1) in the case of a single UE and (80,1), in the case of two UEs. The input is fed into the neural network to produce an action that corresponds to the optimal configuration across the layers of interest to minimize the latency. We have created a codebook that maps output of the neural network to these configurations.  Based on this selected action, the reward is generated using the reward function described in section \ref{sec:solution}.
% \vspace{-0.1in}

\begin{figure}[t!] % Adjust to place the figure at the top of the column
    \centering
    %\vspace{-2mm} % Adjust as necessary to control vertical spacing
    \begin{subfigure}{0.48\columnwidth} % Width adjusted for side-by-side placement in a column
        \includegraphics[width=\linewidth]{Figures/netwrk_perf/1uestatic_req_latencies.pdf}
        \caption{Single Static UE at a fixed location of 5000\,m from the gNB with a constant CQI of 10.}
        \label{fig:static_ues_a}
    \end{subfigure}
    \hspace{1mm} % Small horizontal space between the subfigures
    \begin{subfigure}{0.48\columnwidth} % Width adjusted for side-by-side placement in a column
        \includegraphics[width=\linewidth]{Figures/netwrk_perf/2uestatic_req_latencies.pdf}
        \caption{Two static UEs located 600m and 3600m from the gNB with CQI 15 and 8, respectively.}
        \label{fig:static_ues_b}
    \end{subfigure}
    \caption{Performance of various RAN configurations in meeting different latency requirements for static UEs.}
    \vspace{-3mm} % Reduce vertical space after the figure
    \label{fig:static_ues}
\end{figure}

\vspace{-0.2in}
\begin{figure*}[h!]
\centering
\begin{subfigure}{0.24\textwidth} % Adjusted to 0.23\textwidth for consistent layout
    \includegraphics[width=\textwidth]{Figures/netwrk_perf/1uemobile_30ms.pdf}
    \caption{Latency \& CQI trends}
    \label{fig:1ue_time_cdf_plots_a}
\end{subfigure}\hfil % <-- added
\begin{subfigure}{0.22\textwidth} % Adjusted to 0.23\textwidth for consistent layout
    \includegraphics[width=\textwidth]{Figures/netwrk_perf/1uemobile_30ms_cdf.pdf}
    \caption{30\,ms latency}
    \label{fig:1ue_time_cdf_plots_b}
\end{subfigure}\hfil % <-- added
\begin{subfigure}{0.24\textwidth} % Adjusted to 0.23\textwidth for consistent layout
    \includegraphics[width=\textwidth]{Figures/netwrk_perf/1uemobile_18ms.pdf}
    \caption{Latency \& CQI trends}
    \label{fig:1ue_time_cdf_plots_c}
\end{subfigure}\hfil % <-- added
\begin{subfigure}{0.22\textwidth} % Adjusted to 0.23\textwidth for consistent layout
    \includegraphics[width=\textwidth]{Figures/netwrk_perf/1uemobile_18ms_cdf.pdf}
    \caption{18\,ms latency}
    \label{fig:1ue_time_cdf_plots_d}
\end{subfigure}
%\vspace{-0.05in}
\caption{Performance of various RAN configurations in meeting various latency requirements (30\,ms \& 18\,ms) for a mobile single UE with varying CQI over time (Emulating a UE moving away from gNB).}
\vspace{-2mm}
\label{fig:1ue_time_cdf_plots}
\end{figure*}
\begin{figure*}[h!]
\centering
\begin{subfigure}{0.24\textwidth} % Adjusted width to fit four images in two columns
    \includegraphics[width=\textwidth]{Figures/netwrk_perf/2uemobileue1_30ms.pdf}
    \caption{UE\,1 latency \& CQI trends.}
    \label{fig:2ue_time_cdf_plots_a}
\end{subfigure}\hfil
\begin{subfigure}{0.22\textwidth} % Adjusted width
    \includegraphics[width=\textwidth]{Figures/netwrk_perf/2uemobileue1_30ms_cdf.pdf}
    \caption{30\,ms latency (UE1).}
    \label{fig:2ue_time_cdf_plots_b}
\end{subfigure}\hfil
\begin{subfigure}{0.24\textwidth} % Adjusted width
    \includegraphics[width=\textwidth]{Figures/netwrk_perf/2uemobileue2_30ms.pdf}
    \caption{UE\,2 latency \& CQI trends.}
    \label{fig:2ue_time_cdf_plots_c}
\end{subfigure}\hfil
\begin{subfigure}{0.22\textwidth} % Adjusted width
    \includegraphics[width=\textwidth]{Figures/netwrk_perf/2uemobileue2_30ms_cdf.pdf}
    \caption{30\,ms latency (UE2).}
    \label{fig:2ue_time_cdf_plots_d}
\end{subfigure}
%\vspace{-0.05in}
\caption{Performance of various RAN configurations for two mobile UEs in meeting a latency requirement of 30\,ms. Both UEs have varying CQI over time (UE\,1 emulates moving away from gNB while UE\,2 emulates moving towards gNB).}
\vspace{-4mm}
\label{fig:2ue_time_cdf_plots}
\end{figure*}
\vspace{0.2in}
\noindent $\bullet$ \textbf{Learning Performance: }Fig.\,\ref{fig:rewards} illustrates the learning progress of the DRL agent in each scenario, indicated by the total reward accumulated per episode during training. For the single stationary UE scenario (Fig.\,\ref{fig:static_ue_a}), the agent exhibits rapid convergence to the maximum reward, 
%achieving it around episode 20 
for all the considered uplink latency thresholds (24\,ms, 30\,ms, and 40\,ms). With 150 iterations per episode and a maximum reward of 50 per iteration, the achievable maximum reward per episode is 7500. Similarly, for the multiple UEs scenario (Fig.\,\ref{fig:static_ue_b}), the agent demonstrates faster convergence in the static case compared to the mobile case. This observation validates the effectiveness of the DRL agent in learning optimal policies for both single and multiple UEs.
%while also highlighting the increased complexity associated with mobile environments. 
To facilitate learning in the more challenging mobile scenario, we employed longer training episodes, allowing the agent more time to explore the state-action space and adapt to the dynamic channel conditions.\newline

{\em These results underscore the capability of the DRL agent to effectively learn and adapt to diverse network conditions, optimizing configuration parameters across multiple layers to meet the different latency requirements of MCAs across multiple UEs. The observed convergence behavior further suggests the robustness and stability of the DRL in ACCORD.}
%of the DRL agent for optimizing our proposed ACCORD framework.
%shows the reward accumulated by the agent in each considered network scenario we described in the network setup during the training of the DRL agent. In Fig.\,\ref{fig:single_static_ue}, we can observe that in the static case, that the agent converges quickly to the maximum rewards at around episode 20 for all considered required application latency. In this case, the number of iterations for each episode was 150 and the maximum reward that can be gotten in an iteration is 50. Therefore, the maximum reward in an episode is 7500. 
%In the case of a mobile UE as shown in Fig. \ref{fig:rewards} (b), we can observe that it takes longer (Up to 50th episode) to converge to the maximum rewards and this is because as the UE moves away from the gNB, the signal quality degrades and changes over time, so the agent needs more time to converge. Also, we can see that the agent is able to achieve maximum rewards for latency requirements of 24 and 30ms but fails to reach the maximum rewards for the 18ms latency requirement. Here, the number of iterations was increased to 200 for each episode, leading to a maximum reward for an episode to be 10000. We would also observe in further analysis how this situation now either requires the increase in bandwidth or for the optimization of the application layer in addition to the PHY, MAC and RLC layers already being optimized. 
%Similarly, in Fig.\,\ref{fig:multiple_ues}, in the case of multiple UEs, for the static case, the agent converges faster compared to the mobile case, therefore validating the training for single and multiple UEs. In the case of multiple UEs, we train for longer iterations per episode for better learning.
\vspace{-0.01in}
\subsection{ACCORD Performance Results}
%\vspace{-0.05in}
%We now comment on the the performance of the agent in meeting different latency requirements for each network scenario we considered in comparison with the fixed cross-network configuration approach.
This section analyzes the performance of ACCORD in comparison to the fixed configuration approach of legacy 5G networks.\newline
%focusing on ACCORD's ability to meet diverse latency requirements across different scenarios.\newline
\noindent $\bullet$ \textbf{Observation (Scenario A - Stationary UEs): }Fig.\,\ref{fig:static_ues_a} illustrates the achieved latency for a single static UE under varying latency requirements (24\,ms, 30\,ms, and 40\,ms). ACCORD consistently outperforms the fixed configurations of legacy 5G networks, demonstrating its ability to use information from the context to dynamically adjust network parameters and closely match the desired latency. This adaptability is crucial in resource-constrained environments, as the framework avoids over-provisioning resources and efficiently utilizes available bandwidth. For instance, when the required latency is 24\,ms, ACCORD converges to a configuration of 5\,DL-5\,UL slots, Round Robin scheduling, and RLC buffer size of 6\,KB, to meet the target latency. Conversely, for a 40\,ms latency requirement, the agent selects a less resource-intensive configuration of 6\,DL-4\,UL slots, Round Robin scheduling, and a buffer size of 2\,KB.  This adaptive behavior contrasts with the fixed configurations, which often overshoot or undershoot the latency target, leading to either over-allocation of network resources or performance degradation. Similar trends are observed in Fig.\,\ref{fig:static_ues_b} for the multiple static UEs scenario.  Despite the varying channel conditions and individual latency requirements of the two UEs, ACCORD successfully meets the target latency of 24\,ms for both, demonstrating its ability to handle diverse user demands and optimize network resources accordingly.\newline

\vspace{-0.1in}
\noindent $\bullet$ \textbf{Observation (Scenario B - Single Mobile UE): }Fig. \ref{fig:1ue_time_cdf_plots_a} depicts the performance of ACCORD and two fixed 5G configurations in meeting the latency requirement of a single mobile UE.
%over a 2500\,ms period. 
As the UE moves away from the gNB, its channel quality given by the CQI degrades.
%from 15 to 10.  
While the fixed configurations can initially meet the latency requirement, their performance deteriorates as the CQI drops below 13. In contrast, ACCORD proactively adapts to the changing channel conditions and MCA context, dynamically adjusting network parameters to consistently maintain the desired latency. This adaptability is further evident in the statistical analysis presented in Fig.\,\ref{fig:1ue_time_cdf_plots_b}. The cumulative distribution function (CDF) of achieved latency demonstrates that ACCORD not only consistently meets the latency requirement but also operates close to the target without excessive over-provisioning of resources. For instance, ACCORD initially employs a configuration of 7\,DL-3\,UL slots, Proportional Fair scheduling, and a buffer size of 6\,KB at a CQI of 15. As the CQI degrades to 12 and the latency approaches the threshold, ACCORD incrementally increases the buffer size to 8\,KB and then 10\,KB.  With further CQI degradation, ACCORD switches to a new configuration with 6\,DL-4\,UL slots, Round Robin scheduling, and a buffer size of 2\,KB. This dynamic adaptation highlights ACCORD's ability to effectively utilize network resources and respond to changing channel conditions. Fig.\,\ref{fig:1ue_time_cdf_plots_c} examines a more stringent latency requirement, where the fixed configurations struggle to meet the target as the UE moves away from the gNB.  This scenario underscores the limitations of fixed configurations in dynamic environments, particularly with limited bandwidth (5\,MHz). To address this, the agent optimizes APP layer parameters, enabling it to consistently meet the latency requirement even with significant CQI degradation.

\noindent $\bullet$ \textbf{Observation (Scenario B - Multiple Mobile UEs):} Similar to the single mobile UE case, ACCORD outperforms the fixed 5G configurations in the multiple mobile UE scenario, effectively handling the complexities introduced by varying channel conditions and individual UE mobility patterns.\newline
%we compare two different fixed RAN configurations with the DRL agent in achieving the UE required latency for 2500 ms. As observed in the plot, we can see how the CQI of the UEs drops from 15 to 10 as the UE moves away from the gNB. The fixed configuration approach can meet the latency requirement up until the CQI drops to 13, whereas the agent was able to determine that that configuration was not optimal and generated the right configuration to meet the UE latency requirement and channel conditions. We can also analyze this result statistically as seen in Fig.\,\ref{fig:1ue_time_cdf_plots_b}, we can see that based on the required latency, the agent always achieves a latency that is not just less than the required latency but also as close as possible to the required latency thereby indicating that the agent also ensures that the network does not over-allocate resources to meet latency requirements. For instance, the agent initially converges to a configuration of 7 DL, 3 UL, PF, and a buffer size of 6 at CQI of 15, and at CQI of 12, as soon as the latency achieved tries to overshoot the required latency, the agent converges to a new configuration by first just increasing the buffer size to 8 and then 10. Then as the CQI keeps on degrading, increasing the buffer size does not help anymore so the agent then gets a new configuration that includes changing the slot configuration to 6 DL and 4 UL, then using RR and buffer size of 2.
%In Fig.\,\ref{fig:1ue_time_cdf_plots_c} (c), we analyze the performance when we use a lower latency threshold. As seen in the figure, the fixed configuration approach is hardly able to meet the desired latency requirement as the UE moves from the base station. 
%For the DRL agent, on the other hand, recall from Fig.\,\ref{fig:rewards_b}, the total rewards achieved for the 18ms latency requirement were very low as a result of just optimizing the PHY, MAC, and RLC layers. 
%In worst-case scenarios, after the CQI has degraded up to a certain point, there is no possible RAN configuration across these layers that can meet the desired latency given the limited BW of 5MHz that we considered. We either have to increase bandwidth or we then optimize the application layer by signaling the UE to change its application layer parameters. This is reflected in Fig.\,\ref{fig:1ue_time_cdf_plots_c}, we can see that in optimizing the application layer parameters, we can ensure that the UE latency requirement is met at all CQI levels. Similar to the statistical analysis for a single mobile UE, we can see that in the case of the two mobile UEs that the agents outperforms the fixed RAN configurations in all scenarios.
%\vspace{-0.1in}
Fig.\,\ref{fig:2ue_time_cdf_plots} illustrates the performance analysis for a scenario with two mobile UEs experiencing dynamic channel conditions due to their movement relative to the gNB. ACCORD successfully meets the latency requirements of both UEs by selecting unique configurations at the RLC layer while maintaining common configurations across the PHY and MAC layers.  This differentiation stems from the varying CQI levels experienced by each UE, with one moving away from the gNB (degrading CQI) and the other moving towards it (improving CQI).  Applying a uniform configuration across both UEs would fail to address their individual needs and maintain consistent latency performance.  ACCORD intelligently adjusts the RLC layer buffer size to compensate for the varying channel conditions, ensuring that both UEs meet their respective latency requirements despite the dynamic environment.  This result underscores the importance of learning and responding to individual UE channel conditions in a multi-user scenario.\newline
These results collectively demonstrate the efficacy of ACCORD in optimizing 5G network performance for mobile applications. The ability to dynamically adapt to changing channel conditions, efficiently utilize network resources and optimize application layer parameters positions it as a powerful solution for ensuring QoS with efficient spectrum utilization in dynamic and demanding 5G environments.

{\em Overall, ACCORD's ability to differentiate configurations at the RLC layer while maintaining common parameters at the PHY and MAC layers highlights its nuanced and efficient resource allocation capacity, enabling the framework to effectively manage diverse user demands and optimize network performance even in complex and dynamic environments.}
%Fig.\,\ref{fig:2ue_time_cdf_plots} presents the performance analysis for the scenario with two mobile UEs, each experiencing dynamic channel conditions due to their movement relative to the gNB. Remarkably, ACCORD successfully meets the latency requirements of both UEs by selecting configurations that are unique at the RLC layer but common across the PHY and MAC layers. This differentiated approach stems from the varying CQI levels experienced by each UE, with one moving away from the gNB (experiencing degrading CQI) and the other moving towards it (experiencing improving CQI). Applying a uniform configuration across both UEs would fail to address their individual needs and maintain consistent latency performance. ACCORD recognizes this and intelligently adjusts the RLC layer buffer size, to compensate for the varying channel conditions. This custom optimization ensures that both UEs meet their respective latency requirements despite the dynamic environment.\newline
%This result underscores the importance of learning and responding to individual UE channel conditions in a multi-user scenario. ACCORD's ability to differentiate configurations at the RLC layer while maintaining common parameters at the PHY and MAC layers highlights its nuanced and efficient resource allocation capacity. This adaptive behavior enables the framework to effectively manage diverse user demands and optimize network performance even in complex and dynamic environments.
%\vspace{-2mm}

%%%%%%%%%%%%%%%%%%%%%%%%%%%%%%%%%%%%
\section*{Conclusion}
This paper aims to enhance our understanding of the computational complexity of computing various Shapley value variants. We found that for various ML models --- including decision trees, regression tree ensembles, weighted automata, and linear regression --- both local and global interventional and baseline SHAP can be computed in polynomial time under HMM modeled distributions. This extends popular algorithms, such as TreeSHAP, beyond their empirical distributional scope. We also establish strict complexity gaps between the various SHAP variants (baseline, interventional, and conditional) and prove the intractability of computing SHAP for tree ensembles and neural networks in simplified scenarios. Overall, we present SHAP as a versatile framework whose complexity depends on four key factors: \begin{inparaenum}[(i)] \item model type, \item SHAP variant, \item distribution modeling approach, \item and local vs. global explanations\end{inparaenum}. We believe this perspective provides deeper insight into the computational complexity of SHAP, paving the way for future work.




%We believe that our framework provides a more intricate understanding of SHAP computation complexity across different models, distributions, and variants, paving the way for further research.

Our work opens promising directions for future research. First, expanding our computational analysis to other SHAP-related metrics, such as asymmetric SHAP~\citep{frye20} and SAGE~\citep{covert2020understanding}, would be valuable. Additionally, we aim to explore more expressive distribution classes and relaxed assumptions beyond those in Section \ref{sec:tractable} while maintaining tractable SHAP computation. Finally, when exact computation is intractable (Section \ref{sec:intractable}), investigating the approximability of SHAP metrics through approximation and parameterized complexity theory~\citep{downey2012parameterized} is an important direction.

%Our work opens several promising avenues for future research on the computational properties of explainable AI methods, with a particular focus on SHAP. First, it would be interesting to broaden the computational analysis conducted in this work to include other popular SHAP-related metrics in the literature, such as asymmetric SHAP \cite{frye20} and SAGE \cite{covert2020understanding}. Also, in the future, we aim to explore more expressive distribution classes and relaxed distributional assumptions—extending beyond those examined in Section \ref{sec:tractable} —that still yield tractable SHAP computation. Finally, when exact computation proves intractable (Section \ref{sec:intractable}), it is worthwhile to theoretically investigate the question of the approximability of computing the SHAP metrics across various configurations, through the lens of approximation and parametrized complexity theory \cite{arora2009computational}.

%This paper aims to deepen our understanding of the computational complexity involved in obtaining different Shapley value variants. We found that for a variety of ML models, including decision trees, tree ensembles for regression, weighted automata, and linear regression models — computing both local and global interventional and baseline SHAP can be done in polynomial time when distributions are modeled by HMMs. This extends the distributional scope of popular algorithms like TreeSHAP, which is limited to empirical distributions. Additionally, we demonstrate a strict complexity gap between SHAP variants, showing that interventional and baseline SHAP can be strictly easier to compute than conditional SHAP. Despite these positive results, we uncovered intractability for various SHAP variants in neural networks and tree ensembles. Finally, we provided generalized complexity relations across SHAP variants. We believe that our framework offers a deeper understanding of the complexity involved in computing SHAP across various variants, models, distributions, as well as in both local and global computations, laying the groundwork for future research.
%%%%%%%%%%%%%%%%%%%%%%%%%%%%%%%%%%%%

%\section{Future Direction/Further Research}


%\section*{Acknowledgment}




%section*{References}

% Please number citations consecutively within brackets \cite{b1}. The
% sentence punctuation follows the bracket \cite{b2}. Refer simply to the reference
% number, as in \cite{b3}---do not use ``Ref. \cite{b3}'' or ``reference \cite{b3}'' except at
% the beginning of a sentence: ``Reference \cite{b3} was the first $\ldots$''

% Number footnotes separately in superscripts. Place the actual footnote at
% the bottom of the column in which it was cited. Do not put footnotes in the
% abstract or reference list. Use letters for table footnotes.

% Unless there are six authors or more give all authors' names; do not use
% ``et al.''. Papers that have not been published, even if they have been
% submitted for publication, should be cited as ``unpublished'' \cite{b4}. Papers
% that have been accepted for publication should be cited as ``in press'' \cite{b5}.
% Capitalize only the first word in a paper title, except for proper nouns and
% element symbols.

% For papers published in translation journals, please give the English
% citation first, followed by the original foreign-language citation \cite{b6}.

% % \begin{thebibliography}{00}
% %     \bibitem{b1} J. Shen, Z. Du, Z.Zhang, N. Yang, H. Tang ``5G NR and Enhancements From R15 to R16", 2022
% %     \bibitem{b2} 3GPP TS 38.213 V17.4.0, 2022
% %     \bibitem{b3} Erik Dahlman, Stefan Parkvall ``5G NR The Next Generation Wireless Access Technology"
% %     \bibitem{b4} M. Chiang, P. Hande, T. Lan, C. W. Tan ``Power Control in Wireless Cellular Networks", 2008
% %     \bibitem{b5} Ram Kumar Sharma (2022, Nov), Ericsson website, Available: https://www.ericsson.com/en/blog/2022/11/uplink-power-control-reinforcement-learning
    
% %     % \bibitem{b6} Y. Yorozu, M. Hirano, K. Oka, and Y. Tagawa, ``Electron spectroscopy studies on magneto-optical media and plastic substrate interface,'' IEEE Transl. J. Magn. Japan, vol. 2, pp. 740--741, August 1987 [Digests 9th Annual Conf. Magnetics Japan, p. 301, 1982].
% %     % \bibitem{b7} M. Young, The Technical Writer's Handbook. Mill Valley, CA: University Science, 1989.
% %     \bibitem{b6} R. Falkenberg, B. Sliwa, N. Piatkowski, C. Wietfeld ``Machine Learning Based Uplink Transmission Power Prediction for LTE and Upcoming 5G Networks using Passive Downlink Indicators", IEEE 88th Vehicular Technology Conference, 2018
% %     \bibitem{b7} N. Salhab, R. Rahim, R. Langar, R. Boutaba ``Deep Neural Networks approach for Power Head-Room Predicitons in 5G Networks and Beyond", 2020
% %     \bibitem{8} S. Dzulkifly, L. Giupponi, F. Said, M. Dohler ``Decentralized Q-Learning for Uplink Power Control", IEEE 20th International Workshop on Computer Aided Modelling and Design of Communication Links and Networks (CAMAD), 2015
% %     \bibitem{9} F. Hugo Costa Neto, D. Costa Araújo, M. Pontes Mota, T. F. Maciel, A. L. F. de Almeida ``Uplink Power Control Framework Based on Reinforcement Learning for 5G Networks", IEEE Transactions on Vehicular Technology, Vol.70, No.6, June 2021
% %     \bibitem{10} Z. Kaleem, A. Ahmad, M. Husain Rehmani ``Neighbors' Interference Situation-Aware Power Control Scheme for Dense 5G Mobile Communication System", March 2018
% %     \bibitem{11} 3GPP TS 38.331 V17.3.0, 2022
% %     \bibitem{12} ShareTechnote Website 
% %     \\Available: https://www.sharetechnote.com/html/Handbook\_LTE\_PHR.html
% %     \bibitem{13} Gurucharan M. K. (2020, Jul), Towards Data Science Website, Available: https://towardsdatascience.com/machine-learning-basics-decision-tree-regression-1d73ea003fda
% %     \bibitem{14} Jason Brownlee (2021, March), Machine Learning Mastery Website, Available: https://machinelearningmastery.com/xgboost-for-regression/
% %     \bibitem{15} https://scdn.rohde-schwarz.com/ur/pws/dl_downloads/dl_common_library/dl_brochures_and_datasheets/pdf_1/QualiPoc_Android_bro_en_3607-1607-12_v0400.pdf
     
    
% \end{thebibliography}
%\printbibliography
%\vspace{-1mm}
\bibliographystyle{IEEEtran}

\bibliography{ref}
% \vspace{12pt}

\end{document}
