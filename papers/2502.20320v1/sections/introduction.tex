\section{Introduction}
%\vspace{-0.05in}
%\vspace{-1mm}
%The continuous development of 5G networks has ushered in a new era of connectivity, enabling a surge of innovative machine-type applications (MTAs) with transformative potential. 
The continuous evolution of cellular networks has enabled unprecedented connectivity, characterized by faster speeds and lower latency. While initially designed for human-centric applications like video streaming, the horizon of network usage is rapidly expanding. Recent advancements in artificial intelligence (AI) have fueled the growth of machine-centric applications (MCAs), such as object detection and tracking, where machines rely on inferences derived from sensor data (e.g., video) to perform machine tasks like surveillance and navigation. These MCAs are transforming emerging use-cases such as connected vehicles\,\cite{V2X-DoT}, smart surveillance\,\cite{akyildiz2022wireless}, and automated factories\,\cite{bai2020industry}. Existing research has focused on enhancing the communication architecture of MCAs, enabling resource-constrained end-devices to offload their sensor data to powerful remote servers for complex inference processing and generating feedback for executing the machine task. This strategy, known as {\em remote inferencing}\,\cite{matsubara2022split}, effectively overcomes the limitations of individual end-devices, and has led to the rise of machine-type communications (MTC). However, the unique Quality of Service (QoS) requirements of MTC present significant challenges for 5G networks.\newline
\begin{figure}[t!]
\centering
\includegraphics[trim=20 15 20 12,clip,width=\linewidth]{Figures/sys_arch_v2.pdf}
%\vspace{-3mm}
\caption{ACCORD framework in 5G. The application data (yellow) and network control data (blue) provide the context that is used as input for the deep reinforcement learning (DRL) to generate the configurations for the network (blue) and application (magenta) for machine-type communication.} 
%\cite{tilt}}
%\vspace{-6mm}
\label{fig:arch}
\end{figure}
\begin{figure*}[t!]
\centering
\includegraphics[trim=0 0 0 0,clip,width=\linewidth]{Figures/app_characterization_v2.pdf}
%\vspace{-5mm}
\caption{Example scenario for estimating the threshold of round-trip time of video frame generation to inference feedback reception from remote server for a machine-centric task such as smart surveillance.} %\cite{tilt}}
\vspace{-3mm}
\label{fig:app_charac}
\end{figure*}
\noindent $\bullet$ \textbf {Challenges for 5G in serving MCAs:} Human-centric applications like video streaming or online gaming have predictable needs for throughput and latency. They are mostly downlink-heavy\,\cite{ghoshal2022depth}\,\cite{fezeu2023depth}, and have predictable traffic patterns that allow statistical modeling\,\cite{rao2011network}. The QoS in 5G is managed by grouping applications into {\em classes}\,\cite{5QI}, with fixed upper and lower bounds of the Key Performance Indicators (KPI) like latency, based on the similarity of the corresponding statistical traffic models. 
%5G networks rely on the fixed 5QI table\,\cite{5QI} that determines the QoS for each application class. 
%For example, live video streaming apps are all grouped and assigned the same priority with pre-defined upper and lower bounds of the key performance indicators (KPIs). This works well for video meant for human consumption, but not for MCAs that use video differently. 
In contrast, MCAs are mostly uplink-heavy and are often triggered by {\em events} in the real world, leading to sudden {\em bursts} of {\em time-sensitive} data that leads to {\em unpredictable} QoS demands\,\cite{liu2019edge}\,\cite{feng2021ultra}\,\cite{kong2017object}.
%that can overwhelm legacy 5G networks. 
%This {\em bursty} traffic from MCAs can overwhelm 5G networks, which are designed for the predictable traffic patterns of human-centric applications that do not deviate from pre-defined statistical models\,\cite{rao2011network}. 
An MCA using video for smart surveillance through remote inferencing in a sparse environment (e.g., rural highway) might have very different QoS needs than the same MCA in a dense environment with fast-moving objects (e.g., crowded urban intersection). For 5G to provide {\em guaranteed} service to MCAs with dynamic upper and lower bounds in the KPIs, the network has to over-allocate resources for each MCA to meet the unpredictable demands. This can overwhelm the limited spectrum and is not scalable.
%even though 5G can classify them as belonging to the same class and having the same priority. 
%Legacy 5G networks need to consider these characteristics of remote inferencing MCAs, the This rigid approach cannot scale to meet the dynamic and diverse needs of MCAs, which can change dramatically based on the machine's task, environment, and user behavior. 
Manually configuring network parameters for each MCA scenario is also impractical due to the sheer number of parameters and the complexity of real-world situations. \newline
%MCA communication needs can fluctuate based on real-time user behavior, environmental triggers, or task-specific demands. Such sporadic and unpredictable traffic patterns can strain the resource allocation and scheduling mechanisms of 5G networks, which are optimized for steady-state human-centric traffic. In legacy 5G, to ensure adherence to quality of service (QoS) thresholds for specific application classes (e.g., video), the UE configures the application and upper network layer parameters (e.g., data rate, data size, TCP/UDP, RLC buffer, etc.), whereas for the lower network layers, 3GPP provides the flexibility of configuring network resources like PHY frame configuration, MAC scheduling\,\cite{3gpp-control},\,\cite{3gpprlc}. This process is managed by the Network (NW), encompassing both the base station (gNB) and the core network (CN). The NW relies on UE reports of estimated wireless channel conditions and pending traffic for specific application classes, to determine the allocation of network resources to the UEs\,\cite{snr-cqi-mcs}. An MCA that uses video (e.g., object detection) over 5G, can receive the same static QoS bounds (latency, throughput, etc.), as a traditional video application since it belongs to the same application class, even though the MCA may have different QoS requirements at different time instances. Moreover, 5G maps an application to one of the 27 application classes in the 5QI table\,\cite{5QI}, based on which the static QoS bounds of the application are determined. 
%Since event-based MCA traffic can have unique QoS requirements based on the specific machine task, environment, and user behavior, the existing 5QI table-based QoS bounds determination for MCAs can have a negative impact on the MCA performance over 5G. Also, given the large number of available parameters in the NW and application layers that need to be customized for diverse and custom QoS requirements of MCA, hand-engineering specific configurations of such parameters for each real-world scenario is not a scalable and efficient approach. \newline
\noindent \textbf {Proposed Work:} 
To overcome the limitations of legacy 5G networks in handling the unique demands of MCAs, we propose an \underline{A}pplication \underline{C}ontext-aware \underline{C}ross-layer \underline{O}ptimization and \underline{R}esource \underline{D}esign (ACCORD) framework in 5G, that is depicted in Fig.\,\ref{fig:arch}. ACCORD understands the dynamic demands of MCAs by {\em learning} the application {\em context}.
%and can re-configure the network and application resources to meet this demand in real-time. 
%Our goal is to enable future cellular networks to understand the {\em context} of each application. 
This context information includes the application latency requirement, observed latency, device configuration, user mobility, environmental dynamics, and observed network conditions. This information is used as input state representation to a Deep Reinforcement Learning (DRL) model that learns how to meet the application latency requirement in a spectrum-efficient manner by observing the change in the context after applying different configurations of the network layers (PHY, MAC, RLC) and the application layer. 
%is trained to meet the application requirement by configuring the parameters of the network layer (PHY, MAC, RLC) and the application layer in a spectrum-efficient manner. 
%At the core of ACCORD, we utilize Deep Reinforcement Learning (DRL), where the application context provide the state representation of the UE environment and we optimize configurations across multiple layers—specifically PHY, MAC, RLC, and application layers. 
%This allows us to ensure that MCAs receive exactly the resources they need to meet their quality of service (QoS) demands, while also conserving spectrum. 
%Our framework, depicted in Fig.,\ref{fig:arch}, is designed to anticipate the evolving needs of MCAs and adapt in real-time, delivering personalized services and maintaining optimal performance even in highly dynamic and unpredictable environments. 
%This adaptability leads to more efficient resource management and enhanced spectrum utilization. 
For practical implementation of this framework, the RAN must be capable of exposing real-time data, providing analytics, and supporting closed-loop control. While traditional 5G networks lack these functionalities, the advent of Open RAN paradigm has enabled the design of NextG networks in a modular and disaggregated fashion\,\cite{polese2023understanding}, with open and standardized interfaces that enable access to the necessary data and analytics for implementing frameworks like ACCORD. \newline
%This shift paves the way for implementing such adaptive and intelligent frameworks, driving performance improvements in future networks.\newline
%We aim to address these drawbacks by enabling the next-generation cellular networks to understand the \emph{context} of the UE machine-type task performance through a deep reinforcement learning (DRL) method. We define this context as a function of the application requirements and characteristics, UE characteristics, wireless channel conditions and the NW configuration and requirements. The knowledge of the UE application context can enable the DRL module in the NW to \emph{learn} the optimal configuration across the application and the network layers (PHY, MAC, RLC) and meet the dynamic, event-based, custom QoS requirement of MCA traffic. With this approach, the NW can be aware of the event-based application demands and quickly converge to optimal NW and application configurations for the custom and dynamic QoS requirements of emerging MCAs. 
%Our proposed solution, visualized in Fig.\,\ref{fig:arch} enhances the existing 5G standard to better support MCAs. Here, a UE with low computing capability (e.g., a surveillance drone with a camera) performs smart surveillance with assistance from a remote server that can execute complex machine tasks. When the UE detects mobile objects, like cars, it transmits to the network the relevant video frames, along with the {\em context}, i.e., information about the UE location and mobility, network conditions, and the urgency of the task, such as round trip time latency (RTT). The network uses the trained DRL model to analyze this information and optimize the network and application configurations that can meet the application-requested RTT for the machine-centric communication between the UE and the remote server. The server then performs complex machine tasks, like identifying the car's license plate, and the results are then promptly sent back to the UE, allowing it to take further action, such as tracking the identified car. 
%This entire process is designed to occur within the application-requested round trip time latency, ensuring real-time responsiveness for critical tasks.
%Our solution is designed to operate with the existing 3GPP standard with added enhancements shown in Fig.\,\ref{fig:arch}: The UE, represented by a machine-type device such as an unmanned aerial vehicle equipped with sensors (e.g., camera), that is executing a machine-type task such as smart surveillance by capturing the environment through the camera sensors. The sensor data is encoded and offloaded to an edge/remote server when some triggering event occurs such as the detection of cars through local object detection models running on the UE. In this case, the number of cars and the duration of time for which the cars remain in the field of view of the camera determines the application data size and rate. The UE transmits in the uplink the NW application data along with this metadata, which also contains the UE NW layer information, estimated round-trip latency, estimated channel conditions, and UE location. The NW uses the UE metadata along with its locally derived NW layer parameters as input to a trained deep reinforcement learning (DRL) model for generating the optimal NW and application layer configuration (PHY, MAC, RLC, APP) that can meet the application requested latency bound for the event. The NW informs the UE of the required configuration through downlink control information (DCI). The received user data is forwarded to the ML module to run inference on heavy-weight and privacy-preserving AI/ML tasks such as license plate identification. The inference feedback is then transmitted to the UE in the downlink using the DRL-provided configuration. The inference feedback, which reaches the UE within the latency threshold, may contain information such as positive identification of a car and follow-on actions such as tracking/following the positively identified car.
%This \textit{reactive} approach may exhibit sluggish convergence to an optimal configuration across the NW layers in dynamic environments characterized by high variance in channel quality and the event-driven nature of the application traffic. This potentially increases the likelihood of failing to meet the QoS thresholds for event-driven MTAs. Moreover, in legacy 5G, 
%the application and NW layers do not {\em talk} with each other to perform joint optimization to reach a more fine-grained QoS, as required by the MTAs \textcolor{red}{[cite]}.
%Most cellular network deployments are configured and optimized to meet demands from HTC which are usually characterized by applications requiring high demand in the downlink traffic such as video and VoIP. This current approach of optimization has proven to be inefficient and unscalable \cite{park2020extreme}. According to  \cite{3gpp-5gs}, an approach to meet the requirements of applications in terms of QoS is by setting up different bearers which are associated with a QoS and standardized 5QI. It is the gNB responsibility to ensure that the necessary QoS for each bearer over the radio interface is met with each bearer using the standardized 5QI to QoS mapping using characteristics like priority, packet delay budget and packet error rate. The goal here is to enable frequently used services to benefit from optimized signalling by using standardized QoS characteristics that maps applications or services directly to fixed packet delay budget or latency requirement. For instance, with characteristics from the 5QI table like priority and packet delay budget, the gNB can determine the RLC configuration mode to use and how the scheduler in the MAC may handle packets in terms of scheduling policy. This fixed approach is based on specific use cases and lacks context/state awareness of the network or dynamic application needs. In essence, for MTC applications, the tasks and requirements may vary at different times based on context and this may not be directly mapped to one of the 5QI rows and may lead to problems such as inefficient optimization of the gNB in various layers leading to inefficient resource allocation/waste of available resources and reduced task precision for MTC applications.
%On the other hand, MTC applications are characterized by high uplink throughput and dynamic latency/reliability requirements \cite{3gpp5gservicerequirement},\,\cite{feng2021ultra} which are pivotal for ensuring seamless communication, particularly in scenarios characterized by mobility and high user density \cite{ghoshal:imc2023}. To ensure adherence to application QoS thresholds, considering the time division duplexing (TDD) approach, a possible optimization solution according to 3GPP is the flexible selection of the PHY frame configuration, comprising of a specific number of downlink (DL) and uplink (UL) slots in the time domain, from a range of available options\,\cite{3gpp-control} in order to aid time-domain resource allocation. This process is managed by the Network (NW), encompassing both the base station (gNB) and the core network (CN). This proactive/flexible method of selecting the optimal slot allocation in both uplink and downlink based on QoS requirements has been demonstrated in "citation", showing promising results. 
%The approach of optimizing the network at the PHY layer to meet different application thresholds can only be beneficial up to a certain point and depends on parameters such as available bandwidth, subcarrier spacing, DLUL-periodicity and allowable slot configuration/allocation in the network . For instance, without taking the downlink QoS into consideration, if the available bandwidth or resources are large enough (which is usually not the case as spectrum is a scarce resource), MTC applications requiring stringent latency requirements can benefit from slot configuration that allocates more slots in the UL, a higher subcarrier spacing and a lower DLUL-periodicity. On the other hand, if the downlink QoS is taken into consideration, just giving all resources to the uplink in the PHY layer would be detrimental to the downlnk applications. In essence, just optimizing the PHY layer alone has its benefits but also has its drawbacks especially when considering limited spectrum or downlink applications coexisting. 
%\textbf{Contributions.} We therefore take into considerations these drawbacks, specifically looking at the perspective of limited spectrum availability. In this case, we investigate not just the PHY layer, but the higher layers above including the medium access control (MAC) layer, radio link control (RLC) layer and the Application layer. Using these layers, we assess how the 5G protocol stack can be optimized in an adaptive manner to meet the dynamic mobility and network requirements of MTC applications. This approach uses metrics from the network layers such as total transmitted bytes, MCS, CQI, application frame size/rate and application required QoS, while leveraging a deep-reinforcement learning (DRL) agent to dynamically select optimal parameters in the PHY layer, as well as the MAC, RLC and application layers to meet the input QoS requirement thereby forming a cross-layer optimized system.
%outloOur proposed solution is designed to operate with the existing 3GPP standard with added enhancements and features shown in Fig. \ref{fig:arch} \textcolor{red}{(Would work on this after working on fig.1)}
The main contributions of this paper are as follows:
%\vspace{-0.1in}
\begin{itemize}
    % \item We introduce ACCORD, a 5G enabled framework that is designed to adapt to the application and network  and QoS requirements of MCAs.
    \item We perform experimental evaluations of an example generalizable MCA and characterize its requirements in a real-world scenario. We showcase how MCA requirements are driven by the triggering event characteristics, the environment, the behavior of the user (e.g mobility) and the objects in the environment, thereby making these requirements unpredictable.
    \item We leverage 3GPP-compliant Matlab 5G toolbox\,\cite{matlab}, to run data communication experiments with an example generalizable MCA. We characterize the performance limitations of legacy 5G network optimizations in the PHY, MAC, RLC, and application (APP) layers when serving MCAs with dynamic requirements.  
    \item We explain the building blocks of ACCORD, and through extensive experiments we showcase its performance in a 5G environment, serving MCAs in different scenarios. We show how our proposed approach ensures the MCA QoS through better resource management and spectrum utilization, compared to legacy 5G network solutions.
    %introduce a 3GPP-compliant context-aware optimization framework enabling dynamic allotment of DL and UL slots in the PHY layer, as well as the preferred configuration of parameters across the MAC, RLC, and APP layers that meets the MCA QoS requirements. This approach uses metrics from the network such as CQI, and transmitted bytes, and also requires information from the UE such as the application-requested RTT and frame size/rate. 
    %This proposed approach enhances network QoS, leading to better resource management and spectrum utilization.
    %\item \textcolor{red}{[how this is 3gpp-compliant without oai systems implementation]}. We specify how our proposed method can be integrated seamlessly into a 3GPP-compliant system by walking through a possible implementation approach as shown in our proposed system framework in \ref{fig:arch}. Using this framework, we propose how the control message information exchange between the network and UE can be achieved to get all necessary data for the agent. We also explore the possible deployment options of the DRL agent.
    \end{itemize}
   \begin{comment} 
    \item \textcolor{red}{remove this part}We deploy our proposed method on the 3GPP-compliant OAI 5G testbed\,\cite{phy-test} to showcase its practicality and adherence to 3GPP standards. By comparing our proposed solution with the baseline legacy 5G network configurations across all layers through simulations and testbed experiments, we demonstrate enhanced UL QoS performance
    These findings underscore the critical role of a context aware cross-layer optimization approach in accomodating the diverse demands of emerging applications within dynamic channel environments.
%\end{itemize}
\end{comment}
% \vspace{-1.5mm}



% To ensure adherence to application QoS thresholds, 3GPP provides the flexibility of selecting a PHY frame configuration, comprising of a specific number of downlink (DL) and uplink (UL) slots in the time domain, from a range of available options\,\cite{3gpp-control},\,\cite{3gpprlc}, to aid time-domain resource allocation. This process is managed by the Network (NW), encompassing both the base station (gNB) and the core network (CN). 

% \noindent $\bullet$ \textbf {Bridging the gap in 5G:} Allocation of the DL/UL slots in a PHY frame among available UEs is achieved through time division duplexing (TDD). Here, the NW relies on UE reports of estimated wireless channel conditions, pending application traffic for the UEs, and their QoS requirements to determine the allocation of UEs to specific DL or UL slots in a PHY frame, along with UE transmission configurations like MCS\,\cite{snr-cqi-mcs}.
%This allocation information is conveyed to UEs via downlink control information (DCI) commands transmitted through the physical downlink control channel (PDCCH). In instances of degraded wireless channel conditions reported by UEs, the NW can discern such deterioration from channel estimation reports. Subsequently, the NW adjusts resource allocation for UEs in subsequent slots or frames based on this information. 
% This \textit{reactive} approach may exhibit sluggish convergence to an optimal PHY frame configuration, along with optimal UE allocation to DL/UL slots, and MCS selection in dynamic environments characterized by high variance in channel metrics such as SINR, RSRP, RSSI. This potentially increases the likelihood of failing to meet the QoS thresholds for emerging applications.
%Recent research has also unveiled a conspicuous disparity in the performance of 5G in DL and UL, resembling a pattern observed in its predecessor technologies\,\cite{fezeu2023depth}. For instance, commercial 5G-mmWave-enabled smartphones are capable of achieving remarkable DL speeds, often surpassing 3.5\,Gbps. In contrast, the UL throughput consistently falls significantly behind, typically constrained to an order of magnitude lower than its DL counterpart\,\cite{dinh2022demystifying}. Although numerous factors contribute to this asymmetric behavior, a significant portion of it can be attributed to inefficient DL and UL resource scheduling executed by the NW through PHY time slot configurations that are not updated based on user demands or channel conditions.
% Commercial 5G networks get around this problem by having a \textit{fixed} PHY frame configuration with more DL slots than UL\,\cite{fezeu2023depth}, giving more priority to existing DL-heavy applications like video streaming, and resulting in a noticeable performance gap between the DL and UL in terms of throughput\,\cite{ghoshal:imc2023}. While various factors contribute to this asymmetry, the reactive scheduling by the NW, particularly the lack of updates to DL and UL slot configurations in the PHY frame based on user demands and channel conditions, plays a significant role.\newline
\begin{comment}
\begin{figure}[t!]
	\centering
	\includegraphics[trim=0 0 0 30,clip,width=0.45\linewidth]{Figures/dynamic_scheduling.pdf}
	\vspace{-1mm}
	\caption{Schematic of legacy dynamic scheduling in 5G.}
 \vspace{-4mm}
	\label{fig:ds}
\end{figure}
\end{comment}
% \noindent $\bullet$ \textbf {Proposed framework description:} We explore the possibility of addressing the aforementioned gaps in 5G by investigating
% %time domain resource allocation for future PHY frames by enabling 
% \textit{proactive} PHY frame reconfiguration. Here, the number of DL and UL slots in a PHY frame is decided based on the real-time application traffic demand and {\em predictions} of the UE channel metrics. We propose an adaptive and autonomous UE-based machine learning (ML)-enabled solution, that allows the UEs to predict their local channel metrics for future time instances, and report them to the NW. This {\em lookahead} from all the available UEs can enable the NW to proactively configure the DL and UL slots for future PHY frames and pre-allocate UEs to these slots to meet the diverse application requirements in dynamic channel conditions. 
% While the NW can store channel metric data from all served UEs, executing ML algorithms on each UE's reported channel data for real-time forecasting is not scalable from the NW side. A more practical approach is for individual UEs to forecast their channel metrics and transmit them to the NW, facilitating proactive resource optimization.

% Our solution is designed to operate with the existing 3GPP standard with added enhancements shown in Fig.\,\ref{fig:arch}: (i) Deployed UEs with trained machine learning (ML) modules for local channel metric prediction, that are reported to the NW along with the application traffic demand, 
% %(2) the NW, that forwards the traffic information and predicted channel metrics from the UE to the network optimization suite, 
% and (ii) the optimization suite in the NW, which receives this forwarded information from the UEs and proactively reconfigures the PHY frame with an optimal number of DL and UL slots. This helps to allocate UEs to the best possible slots in the PHY frame and maintain the QoS requirement.

\begin{comment}
We are trying to solve:
\begin{itemize}
    \item item 1
    \item item 2
    \item item 3
\end{itemize}
\begin{figure}[H]
    \centering
    \includegraphics[trim=20 20 20 20,width=0.46\textwidth]{Figures/system_arch.png}
    \caption{System architecture diagram }
    \label{fig:sys_arch}
\end{figure} 
\end{comment}
\begin{comment}
\begin{figure}[t!]
	\centering
	\includegraphics[trim=0 0 0 10,clip,width=0.9\linewidth]{Figures/5g_frame_structure.pdf}
	\vspace{-1mm}
	\caption{Legacy 5G frame structure with numerology 1}
 \vspace{-4mm}
	\label{fig:5g_frame}
\end{figure}
\end{comment}
%\vspace{-2mm}

% The main contributions of the paper are as follows:
% \begin{itemize}
%     \item We leverage 3GPP-compliant Matlab 5G toolbox\,\cite{matlab}, to characterize how legacy 5G allocates UEs in the PHY frame in the time domain based on UE reported channel metrics, and motivate how this approach can be a bottleneck for emerging UL-heavy applications. 
%     \item We introduce a proactive PHY frame configuration framework enabling dynamic allotment of DL and UL slots in a PHY frame through {\em look-ahead forecasts} of UE channel metrics and application demand, enhancing network QoS compared to reactive resource allocation. Crucially, our framework is straightforward and aligns with the 3GPP standard.
%     \item We deploy our proposed method on the 3GPP-compliant OAI 5G testbed\,\cite{phy-test} to showcase its practicality and adherence to 3GPP standards. By comparing our solution with the baseline reactive legacy 5G TDD, through both testbed experiments and simulations, we demonstrate an enhanced equilibrium between DL and UL performance. These findings underscore the critical role of proactive PHY frame reconfiguration in accommodating the diverse demands of emerging applications within dynamic channel environments.
% \end{itemize}
% \vspace{-1.5mm}


%The rise of machine-to-machine (M2M) communications marked a significant shift in the wireless industry, where machines are no longer just passive tools but active participants in an interconnected digital ecosystem\,\cite{wang2017survey}. Powered by machine learning, machine-type applications (MTA) are the end users participating in M2M communication, enabling autonomous communication and interaction between devices without human intervention. From smart cities and industrial automation to autonomous vehicles and remote healthcare, MTAs are poised to revolutionize various sectors. However, the resource constraints of low-cost end devices, coupled with the increasing complexity of machine learning algorithms, present challenges for the widespread deployment of these applications\,\cite{V2X-DoT},\,\cite{akyildiz2022wireless},\,\cite{bai2020industry}. In recent times, there has been a paradigm shift in MTA architecture where low-cost devices, often equipped with limited computational capabilities, can seamlessly offload their sensor data to edge or remote servers. This strategic offloading allows for the execution of compute-intensive inference processes in a resource-rich environment. The results of these complex computations are then efficiently relayed back to the MTA client residing on the end device, providing actionable insights and facilitating real-time decision-making\,\cite{liu2019edge}.
%Next-generation cellular systems including 5G/6G promise to support a wide range of emerging machine-type application (MTA) use cases in the near future, like those found in connected vehicles, smart factories, smart surveillance, etc.\,\cite{V2X-DoT},\,\cite{akyildiz2022wireless},\,\cite{bai2020industry}. MTA-enabled devices can communicate with each other and perform distributed operations without human intervention. They can also offload machine tasks such as object detection/recognition/tracking to edge/cloud servers\,\cite{liu2019edge}, due to compute/energy limitations or to conserve these resources. 
%thereby presenting a distinctive array of challenges in contrast to legacy human-type communication (HTC) applications like video streaming, real time gaming, conversational video and conversational voice. 
%In contrast to HTC, the array of challenges faced by these machine-type applications are not limited to end to end (E2E) latency which could be affected by the application server performance and not under the network control. Another challenge is the fact that 
%One key differentiator between the HTA and MTA is that the MTA requirements may vary over time based on different application-specific tasks and situations or based on the behavior of the objects in the vicinity of the UE. Taking the edge-assisted navigation system as an example, where a smart surveillance system (e.g., UAV) is monitoring objects in an environment, the application requirement (e.g., latency) in the low-density environment is moderate compared to high density environments where low-latency is critical. Similarly, the latency requirement for a surveillance system trying to monitor a slow moving vehicle on a high way is different from monitoring a high-speed moving vehicle.
\begin{figure*}[t!]
    \centering
    %\vspace{-2mm} % Adjust vertical spacing if necessary
    \begin{subfigure}[b]{0.24\textwidth} % Set each subfigure to 24% of text width
        \centering
        \includegraphics[width=\linewidth]{Figures/netwrk_perf/prelim1.pdf}
        \caption{Profile\,1}
        \label{fig:prelim_images_a}
    \end{subfigure}
    \hfill % Use \hfill to evenly space out the subfigures
    \begin{subfigure}[b]{0.24\textwidth}
        \centering
        \includegraphics[width=\linewidth]{Figures/netwrk_perf/prelim2.pdf}
        \caption{Profile\,2}
        \label{fig:prelim_images_b}
    \end{subfigure}
    \hfill
    \begin{subfigure}[b]{0.24\textwidth}
        \centering
        \includegraphics[width=\linewidth]{Figures/netwrk_perf/prelim3.pdf}
        \caption{Profile\,3}
        \label{fig:prelim_images_c}
    \end{subfigure}
    \hfill
    \begin{subfigure}[b]{0.24\textwidth}
        \centering
        \includegraphics[width=\linewidth]{Figures/netwrk_perf/prelim4.pdf}
        \caption{Profile\,4}
        \label{fig:prelim_images_d}
    \end{subfigure}
    %\vspace{-3mm}
    \caption{Impact of legacy 5G configurations on achieving a target latency of 18\,ms for 4 UEs at varying distances from the gNB.}
    \label{fig:prelim_images}
    \vspace{-2mm} % Adjust vertical space below the figure
\end{figure*}

