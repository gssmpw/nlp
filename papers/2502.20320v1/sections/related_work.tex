\section{Related Work}
%\vspace{-0.07in}
Optimizing Radio Access Network (RAN) performance through network-layer enhancements, particularly in Time Division Duplex (TDD) slot configurations, has been extensively studied. Static TDD configurations are inherently inefficient as they cannot adapt to real-time network traffic variations. To address this, various dynamic TDD techniques leveraging deep reinforcement learning (DRL) have been proposed to optimize throughput and latency. However, these approaches face challenges related to scalability and the need for real-time decision-making.

Several works\,\cite{bagaa2021using, boutiba2023enabling, hassan2024wixor, boutiba2023multi, park2024deep, sun2015d2d, dao2021deep, ghoshal2024enabling, tang2020deep} explore DRL-driven dynamic TDD solutions across different scenarios, including high-density environments, device-to-device (D2D) communications, massive IoT, and public/private 5G networks. In particular,\,\cite{bagaa2021using, boutiba2023enabling} introduce a TDD xApp that leverages buffer status reports (BSR) and downlink buffer size to prevent buffer overflow. Similarly,\,\cite{hassan2024wixor} enhances user experience (QoE) by dynamically configuring TDD slots and symbols, utilizing both BSR and channel quality indicators (CQI) for adaptive modulation and coding schemes (MCS). This method operates at the radio protocol layer, making it application-transparent. Meanwhile,\,\cite{ghoshal2024enabling} employs UE-side wireless channel metrics for proactive PHY frame configuration.

While most research primarily focuses on PHY-layer optimization, some studies explore higher-layer enhancements. MAC scheduling optimization has been examined in\,\cite{yan2019intelligent, chen2024elase}, while \cite{kumar2018dynamic} introduces a dynamic RLC buffer sizing algorithm that minimizes latency by adjusting buffer size based on real-time queuing delay measurements. Additionally,\,\cite{fezeu2024roaming} provides a comprehensive cross-layer measurement study, linking PHY, MAC, and RLC configurations to their impact on overall network performance.

A similar adaptive approach has long been applied in internet and video streaming domains, where adaptive bitrate streaming techniques assess network conditions and adjust video quality accordingly. These techniques involve selecting the appropriate resolution, modifying frame rates, and optimizing encoding strategies\,\cite{nam2014youslow, zuo2022adaptive, tran2020bitrate}. Such methodologies have been widely adopted in modern streaming platforms like Netflix and YouTube.

Building on these foundations, our work characterizes MCA requirements in real-world scenarios and develops a multi-layer optimization framework spanning PHY, MAC, and RLC layers. Our approach ensures adaptive resource allocation under diverse mobility and channel conditions, optimizing performance with minimal resource overhead.

