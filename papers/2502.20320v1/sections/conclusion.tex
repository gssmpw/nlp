% \vspace{-0.05in}
\section{Conclusion}
%\vspace{-0.08in}
This research demonstrates the effectiveness of ACCORD, a context-aware DRL framework for optimizing 5G resource allocation in machine-centric applications, compared to legacy 5G solutions. ACCORD uses application requirements supplemented by contextual information as input to a DQN agent to meet event-driven latency requirements. This agent learns to dynamically adjust network parameters across the PHY, MAC, RLC, and application layers, adjusting to the needs of individual devices and channel conditions. The framework's ability to increase spectrum efficiency and deliver consistent latency performance was validated by evaluations conducted in various scenarios. For future work, ACCORD will be generalized for heterogeneous applications, real-world 5G deployments, and optimization across extra network layers.
%This research demonstrated the efficacy of our proposed ACCORD framework by using context information along with application requirements as input to a deep reinforcement learning network for optimizing 5G resource allocation to meet the dynamic latency requirements of machine-centric applications. By leveraging a DQN agent, we achieved adaptive control over network parameters across the PHY, MAC, and RLC layers, enabling fine-grained optimization tailored to individual UE needs and channel conditions. Our evaluations, spanning both static and mobile scenarios with single and multiple UEs, consistently showed the superiority of the ACCORD framework over legacy 5G. The DRL agent in the ACCORD framework has the ability to dynamically adjust to varying channel conditions and efficiently utilize network resources resulting in consistent latency performance and avoided unnecessary over-provisioning leading to spectrum efficiency. Future work will focus on extending this solution to real-world 5G deployments, incorporating optimization across additional network layers, and addressing the challenges of generalized optimization for diverse applications with varying requirements. We believe that this research provides a significant step towards realizing intelligent and adaptive 5G networks capable of supporting the evolving demands of human-centric and machine-centric applications.