\section{Related Works}
\label{sec:background_motivation}
%\vspace{-1mm}
%3GPP employs dynamic scheduling to allocate resources to UEs in the time domain, representing the prevalent method in commercial cellular networks\,\cite{fezeu2023depth}. Our proposed solution builds upon this approach, aiming to enhance its efficacy.
%\vspace{-1mm}
%\subsection{Related Works \textcolor{red}{(to-do)}}
\textcolor{red}{to-do}Existing studies investigate 5G performance improvements through layer specific optimization solutions, such as application layer solutions [cite], RLC layer solutions [cite], MAC layer solutions [cite], and PHY layer solutions [cite]. Studies such as [cite] investigate the effect of 5G on emerging human-centric applications such as XR and propose application-aware network layer solutions such as PDU sets, but these solutions are for a specific class of applications and are confined to a specific layer of the network stack. Overall, these studies do not consider emerging machine-centric application requirements and also do not explore the effect of joint optimization across the different layers of the network stack through a learnable approach.
%\subsection{Legacy 5G network optimization}
\vspace{-1mm}
%A frame in 5G is 10\,ms long, which is broken down into 10 subframes. Depending on the numerology, the number of slots in a subframe varies. Here, we consider a subframe of 30\,KHz subcarrier spacing (SCS) spanning two slots. In TDD, the DL-UL-periodicity determines the time for which there can be a consecutive set of DL and UL slots. Each slot is further broken down into 14 symbols assuming a normal cyclic prefix (CP). 
\begin{comment}
\noindent $\bullet$ \textbf{Time domain resource allocation: }For allocating resources in the time domain in 5G, the NW informs the UE about which slots/symbols can be used for data transmission/reception through signaling of time-domain resources using DCI in PDCCH. In 5G, DCI formats [{0\_0}] and [{0\_1}] allocate time-domain resources for UL data in the physical uplink shared channel (PUSCH). DCI formats {0\_0} and {0\_1} carry a 4-bit field named ‘time domain resource assignment’ which points to one of the 16 rows of a look-up table\,\cite{snr-cqi-mcs}. Each row in the look-up table provides the following parameters:
\begin{figure}[t!]
	\centering
	\includegraphics[trim=0 0 0 0,clip,width=\linewidth]{Figures/time_domain_allocation.pdf}
	\vspace{-2mm}
	\caption{Example time domain resource assignment from NW to UE in 5G.}
 \vspace{-4mm}
	\label{fig:time_domain_alloc}
\end{figure}
\begin{itemize}
    \item \textit{Slot offset K2.} This parameter is used to derive the slot in which PUSCH transmission occurs.
    \item Jointly coded \textit{Start and Length Indicator Values (SLIV)}, or individual values for the start symbol $S$ and the allocation length $L$.
    \item \textit{PUSCH mapping type} to be applied on the PUSCH transmission.
\end{itemize}
\end{comment}
%\noindent $\bullet$ \textbf{Dynamic scheduling procedure: }
%The time domain resource allocation for UL transmission in 5G are shown in Fig.\,\ref{fig:time_domain_alloc}. 
%In TDD, the NW divides the time domain into either dedicated DL or UL or mixed/flexible (F) slots. Here, the UE first sends a scheduling request (SR) as control information in either a UL or an F slot, informing the NW of pending data to transmit. The UE informs the NW of the most recent channel estimates and the available data volume through the buffer status report (BSR) as control information\,\cite{3gppmac}. The NW after receiving these control data, informs the UE of the assigned resources via DCI, which indicates the specific slot, starting OFDM symbol, and symbol length in the following PHY frame for uplink data transmission. The UE then prepares the transport block sizes (TBS) based on the MCS and the assigned OFDM symbols and transmits the application data in the scheduled UL slots. 3GPP allows flexible scheduling in which slots are dedicated for DL vs. UL transmissions\,\cite{3gpprlc}.
%Existing 5G optimization techniques primarily focus on individual network layers. Application data flows through the network stack until it reaches the RLC buffer where it can accumulate due to wireless channel degradation, network congestion, etc. The RLC layer has FIFO queues (RLC channels) per application class [cite]. Packets wait at the RLC buffer until the MAC scheduler allocates Resource Blocks (RBs), in the network signaled PHY frame slots, to the data in the RLC buffer. This allocation process considers various factors such as channel conditions, application class KPI bounds, buffer status, etc. While commercial networks employ proprietary non-public MAC scheduling algorithms tailored to their specific needs, 3GPP provides standardized algorithms like \textit{Round Robin}, \textit{Proportional Fair}, and \textit{bestCQI} as benchmarks for evaluating new scheduling methods. Commercial 5G networks such as AT\&T, T-Mobile, and Verizon all configure the 5G PHY frame of 10\,ms duration, with the same {\em fixed} configuration that has more downlink (DL) than uplink (UL) slots\,\cite{fezeu:techreport2023}, leading to imbalanced traffic distribution. The resource allocation in the PHY layer is \textit{reactive}, where the UE triggers the dynamic scheduling process by transmitting a Scheduling Request (SR) to the NW in a UL or F slot, indicating pending UL data, and providing channel estimates via the physical uplink control channel (PUCCH)\,\cite{3gppmac}. Subsequently, the NW informs the UEs of resource allocation using downlink control information (DCI) in the downlink control channel (PDCCH). Here, among other things, the NW specifies slot assignments for UEs in the uplink service channel (PUSCH) and downlink service channel (PDSCH) for UL transmission and DL reception, respectively. 

%\noindent $\bullet$ \textbf {PHY layer:} The 5G PHY layer performance is based on the resource allocation in the PHY frames which are divided into DL and UL slots. Commercial 5G networks such as AT\&T, T-Mobile, and Verizon all configure the 5G PHY frame of 10\,ms duration, with the same {\em fixed} configuration that has more DL than UL slots\,\cite{fezeu:techreport2023}, leading to imbalanced traffic distribution.
%In TDD, the NW configures the PHY frame of 10\,ms duration into dedicated DL, UL, and mixed/flexible (F) slots in the time domain (20 slots per frame in this example), as shown in Fig.\,\ref{fig:7_2_1}. 
%The resource allocation in the PHY layer is \textit{reactive}, where the UE triggers the dynamic scheduling process by transmitting a Scheduling Request (SR) to the NW in a UL or F slot, indicating pending UL data, and providing channel estimates via the physical uplink control channel (PUCCH)\,\cite{3gppmac}. Subsequently, the NW informs the UEs of resource allocation using downlink control information (DCI) in the downlink control channel (PDCCH). Here, among other things, the NW specifies slot assignments for UEs in the uplink service channel (PUSCH) and downlink service channel (PDSCH) for UL transmission and DL reception, respectively. 

%\noindent $\bullet$ \textbf {MAC layer:} 
%The MAC scheduler at each base station (gNB) decides the UE-wise PRB allocation for every slot. In FDD, PDSCH and PUSCH allocations are output per slot, while in TDD, the appropriate (PDSCH or PUSCH) allocation is output every slot (DL or UL). The scheduler uses the information related to the SINR, CQI, MCS, number of MIMO layers, buffer status, and HARQ  as inputs for each gNB and attached UE. The scheduler also uses the number of PRBs available in the gNB. Re-transmissions are prioritized over first transmissions. The scheduler's output is UE-wise PRB allocation in the UL and DL, at every slot. MAC scheduling algorithms in commercial networks are customized for specific business needs and are unavailable in the public domain. 3GPP provides some baseline MAC scheduling algorithms (Round Robin, Proportional Fair, and bestCQI) for evaluating the performance of proposed methods in the upstream and downstream functions. The \textit{RoundRobin} method divides the available PRBs among the logical channels that have a non-empty RLC queue. The MCS for each user is calculated according to the received CQIs. The \textit{Proportional Fair} method works by scheduling a (active) user when its instantaneous channel quality is high relative to its own average channel condition over time. The PF scheme is based on the current data rate for each user and the exponentially weighted moving average (EWMA) data rate over an immediately prior predetermined interval for each user. In comparison with the round-robin (RR) scheduler in which UEs are cyclically scheduled irrespective of the channel condition, the PF scheduler maximizes the system throughput while maintaining long-term fairness in the allocation of resources between users. The \textit{bestCQI} method allocates PRBs to the active flow(s) to maximize the achievable rate. It selects the user that sees the highest CQI.
%In the 5G MAC layer, the scheduler at each base station (gNB) allocates Physical Resource Blocks (PRBs) to User Equipments (UEs) for every time slot. This allocation process considers various factors such as Signal-to-Interference-plus-Noise Ratio (SINR), Channel Quality Indicator (CQI), Modulation and Coding Scheme (MCS), number of Multiple-Input Multiple-Output (MIMO) layers, buffer status, and Hybrid Automatic Repeat reQuest (HARQ) information. 
%The scheduler prioritizes re-transmissions over first transmissions to ensure data reliability. 
%While commercial networks employ proprietary non-public MAC scheduling algorithms tailored to their specific needs, 3GPP provides standardized algorithms like \textit{Round Robin}, \textit{Proportional Fair}, and \textit{bestCQI} as benchmarks for evaluating new scheduling methods. 
%The Round Robin algorithm distributes PRBs equally among active logical channels, while the Proportional Fair algorithm dynamically allocates resources to users with favorable instantaneous channel conditions relative to their average channel quality, thereby maximizing system throughput and ensuring long-term fairness. The \textit{bestCQI} algorithm prioritizes the user with the highest CQI to maximize the achievable data rate.

%\noindent $\bullet$ \textbf {Application layer:} 
%Flexible  scheduling, sanctioned by 3GPP, assigns slots for DL and UL transmissions as needed.

%Commercial 5G networks such as AT\&T, T-Mobile, and Verizon all use the same {\em fixed} PHY frame configuration that has more DL than UL slots -- 7\,DL-2\,UL-1\,F (7\,DL 3\,UL)\,\cite{fezeu:techreport2023}, with a periodicity of 5\,ms (Fig.\,\ref{fig:7_2_1}), leading to imbalanced traffic distribution.
%Existing drive-test studies on these commercial networks using 5G smartphones reveal impressive DL speeds, surpassing 3.5\,Gbps, while UL throughput remains consistently lower\,\cite{ghoshal:imc2023}. The fixed DL-heavy PHY frame configuration may strain 5G networks, particularly with the rise of uplink-heavy applications. The use of this fixed PHY frame configuration persists due to the prevalence of existing DL-heavy applications and the 3GPP-supported {\em reactive} scheduling method based on UE-reported channel metrics. The lack of channel forecast knowledge provides little incentive for the NW to modify PHY frame configuration relying solely on UE-reported channel metrics as this process may take time to converge and can result in sub-optimal NW performance. 
\vspace{-1mm}
