\section{Performance Analysis}
In this section, we present the results and findings of our study on the feasibility of dynamic and proactive PHY frame configuration to serve emerging low-latency applications in next generation cellular networks. We explore the performance and scalability of our proposed solution with different traffic models and different number of users in different dynamic channel conditions.
\begin{figure*}[t!]
	\centering
 \begin{minipage}[t]{\linewidth}  
	\includegraphics[trim=0 0 0 0,clip,height=1.7in,width=0.9\linewidth]{Figures/latency_cdf_25.pdf}
	\caption{Latency with proactive PHY frame configuration compared to static frame configurations in OAI} %\cite{tilt}}
	\label{fig:lat_new}
 \end{minipage}
\end{figure*}
\begin{figure*}[t!]
	\centering
  \begin{minipage}[t]{\linewidth}  
	\includegraphics[trim=0 0 0 0,clip,height=1.8in,width=0.9\linewidth]{Figures/tput_cdf_25.pdf}
	\caption{Throughput with proactive PHY frame configuration compared to static frame configurations in OAI} %\cite{tilt}}
	\label{fig:tput_new}
 \end{minipage}
\end{figure*}
\subsection{Experimental Methodology}
% We introduced modifications to the OAI code, enhancing both UE and NW components. When receiving the UE's prediction of channel and traffic metrics in the uplink, the NW is now equipped to proactively allocate the UEs in the best possible slots based on their channel conditions and traffic requirements, ensuring the UE can complete their transmission within the required latency bounds. With these OAI code adaptations on the testbed, we conducted experiments in an indoor setting with the same network configurations as depicted during our preliminary experiments in Section\,\ref{sec:background_motivation}.D. We systematically extracted control and data channel logs from various layers within both the UE and the NW to evaluate the performance enhancements achieved by our proposed approach. For the latency and throughput performance, we connect 3 UEs with the NW. The first UE (UE 0) was made to transmit UL data of application packet size 90\,KB, the second UE (UE 1) transmitted UL data with application packet size of 200\,KB and the third UE (UE 2) was made to receive DL data of application packet size 400\,KB. 

\subsection{PHY latency performance}
The latency deadline for each UE was kept at 200\,ms and the latency performance is shown in Fig.\,\ref{fig:lat_new}. When employing a static DL heavy configuration of 8-1-1 (8\,DL-2\,UL in Fig.\,\ref{fig:lat_new}), both UL UEs (UE 0 and UE 1) had the worst performance where the probability of achieving a threshold was 250\,ms and 450\,ms respectively, whereas the DL UE (UE 2) was able to meet the latency deadline. Using the 7-2-1 (7\,DL-3\,UL in Fig.\,\ref{fig:lat_new}) and 6-3-1 (6\,DL-4\,UL in Fig.\,\ref{fig:lat_new}) PHY frame configurations, made UE\,0 meet the latency deadline but still resulted in UE\,1 overshooting the latency bound because of large data packets. The 6-3-1 configuration also resulted in the DL UE\,2 not meeting the latency deadline because of less number of DL slots. With the proactive PHY frame configuration (Dynamic configuration in Fig.\,\ref{fig:lat_new}) all the UEs were able to meet the latency deadline of 200\,ms. The average latency with the proposed approach was $\sim$150\,ms. This was possible because the NW had accurate forecast of UE traffic and channel conditions which helped the NW to optimize the slots and allocate the UEs accordingly in each frame such that all the UEs can meet their respective latency deadlines. In static frame configurations, even with the forecast of channel and traffic metrics, the NW could not allocate UL UEs to DL slots, resulting in reduced performance.
\subsection{PHY throughput performance}
For throughput performance, the same 3 UEs with same traffic characteristics were used, and the results are shown in Fig.\,\ref{fig:tput_new}. With static frame configurations, the UL UEs (UE\,0 and UE\,1) suffered the most compared to DL UE (UE\,2). This is because the static frame configurations were all DL heavy. With the proposed approach (dynamic configuration), all the UEs were able to experience significantly better throughput than the static PHY frame configuration. With the proposed approach, the NW could optimally allocate the UEs to the best possible slots based on the UE reported local channel and traffic metrics, achieving better throughput results compared to static PHY configurations, where even with the UE predictions of channel and traffic metrics, the NW could not allocate UL UEs to DL slots. 