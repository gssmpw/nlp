\section{Preliminary Experiments}
\label{sec:preliminary_experiment}
\subsection{Characterizing MCA requirement:}
%\vspace{-0.05in}
MCAs often exhibit dynamic and unpredictable behavior, influenced by external factors such as user actions and environmental conditions.
%We then create the MCA context by combining the requirement with information that can help the network to properly allocate resources to meet the MCA demand. 
Consider the example shown in Fig.\,\ref{fig:app_charac}, where a vehicle is approaching a mobile monitoring device equipped with a camera. This device captures video frames and transmits them to a remote server for license plate identification. The time required for the vehicle to be detected and identified, and for the feedback to be relayed back to the device (e.g., ``track vehicle") determines the necessary round-trip latency of the application. This latency is affected by various factors, including the vehicle speed, the camera field of view (FoV), and the processing time required for identification. In this particular scenario, the round-trip latency is $\sim$1\,second, (including uplink latency), as the vehicle remains within the camera field of view for 33 frames at a rate of 30 frames per second (FPS). If the vehicle's speed were to change or the camera field of view were different, the required latency would also change. 
%Contextual factors that enable the network to properly allocate resources, such as channel quality, MCA traffic information, user mobility, and network buffer are used to make the network {\em aware} of the MCA context.
This example highlights the dynamic nature of MCAs and the need for network optimization strategies that can adapt to their unique and changing requirements.
%Here we showcase through a real-world example how unpredictable external factors such as user behavior, environment parameters, etc., determine machine-centric application requirements such as round-trip latency.\newline
%Fig.\,\ref{fig:app_charac}, shows some of the extracted frames from the video of a car navigating a stretch of a rural highway and approaching a mobile monitoring device with a camera. The camera captures video frames in 1080-pixel resolution at 30 frames per second (FPS). As the car is detected in Frame \#1 through local object detection, the frame is compressed and transmitted via the network to the remote server application for the compute-intensive inferencing task of license plate identification. This frame may not be suitable for license plate identification because of factors that include but are not limited to the distance of the vehicle from the device, the frame compression technique, environmental conditions, etc. 
%The edge server application informs the edge device to continue transmitting frames where the object is detected for edge-based tracking and future position estimation of the object in a future frame. The edge server subsequently informs the edge device through the network to capture a future frame at a specific compression technique where the object is projected to be at a suitable location in the frame with suitable features for carrying out the machine task of object classification and license plate identification. 
%Over the next few frames, as the approaching car size becomes bigger, the license plate becomes easier to identify.
%At frame \#345, the license plate could be positively identified by the remote server application. The car moves away from the camera field of view (FoV) after frame \#378. If the server decides to generate feedback for the client device to take follow-on actions such as track/follow the car, this message needs to be delivered to the client device before the car moves away from the camera FoV. In this scenario, the duration of time the car stays within the camera FoV is determined by external factors such as the vehicle speed, camera field of view span, etc. This in turn determines the application round trip time from license plate identification at frame \#345, to feedback to the device (e.g., follow car) by frame \#378. If the inference decision reaches the mobile device after this time frame, then the feedback will be stale since the vehicle has gone past the camera field of view and the device cannot perform the task required in the feedback, i.e., to follow the car. The time duration from frame \#345 to frame \#378 (total of 33 frames) can be calculated based on the camera frame rate (frames per second or FPS). Here at 30 FPS, the round-trip latency thus needs to be $<$1 second.\newline
%If the car speed was different or changed while the video was being taken or if the camera FoV was narrower or wider then in each scenario the round-trip latency would be different.
\vspace{-0.05in}
\subsection{Network Characterization:}
%\vspace{-0.05in}
We conduct simulation experiments with an MCA (similar to Sec.\,\ref{sec:preliminary_experiment}.A) over a 5G network to understand the uplink performance as the MCA transmit video to a 5G network-connected server.
%To understand how legacy 5G networks perform while serving MCAs, we conduct simulation experiments with an MCA, as explained in Sec.\,\ref{sec:preliminary_experiment}.A, generating and uploading video to a 5G network-connected server. 
We track the application data flow through the network stack and quantify the limitations of using standard-compliant fixed configurations across the PHY, MAC, RLC, and APP layers\,\cite{fezeu:techreport2023}\,\cite{3gppmac}\,\cite{3gpprlc}. 
\begin{table}[t!]
\vspace{-1mm}
 \caption{Configuration Profiles}
 \vspace{-1mm}
 \begin{center}
 %\renewcommand{\arraystretch}{0.65}
 \resizebox{0.482\textwidth}{!}{
  %\begin{tabu}{ | X[0.001cm l] | X[0.6cm l]|| X[0.001cm l] | X[0.6cm l] |}
  %\vspace{-0.6mm}
  \begin{tabular}{|c|c|c|c|}
 \hline
\textbf{Profile}  & \textbf{Frame Rate (FPS)} & \textbf{MAC scheduling}  & \textbf{RLC buffer (KB)}\\
\hline
 1 & 30 & Round Robin & 6 \\
 \hline
 2 & 60 & Round Robin & 6 \\
 \hline
 3 & 60 & Round Robin & 10 \\
 \hline
 4 & 60 & Proportional Fair & 6 \\
 \hline
 \end{tabular}}
 \end{center}
 \label{tab:notations}
 \vspace{-6mm}
\end{table}

\noindent $\bullet$ \textbf {Experiment Setup:} We use a 3GPP-compliant MATLAB 5G toolbox featuring a base station (gNB) having an 8x8 MIMO configuration with a bandwidth (BW) of 10\,MHz with 30\,KHz subcarrier spacing (SCS), and 4 UEs each having 2x2 MIMO configuration at different locations from the gNB (300\,m, 1200\,m, 1500\,m, 3000\,m) and experiencing channel conditions based on the 3GPP standard clustered delay line (CDL) channel model. 
The MCA on each UE generated uplink video frames of size 7.5\,KB, at rates varying between 30\,FPS and 60\,FPS. The MCA requirement was an uplink latency of 18\,ms based on the vehicle speed and camera FoV, calculated based on a scenario similar to the one described in Sec.\,\ref{sec:preliminary_experiment}.A. 
%The video streaming application on each UE generated uplink video frames, with each frame having a size of 7.5\,KB, and frame rates varying between 30\,FPS and 60\,FPS. 
%CQI of 15,13,13 and 8 respectively. These CQI levels reflect how the channel quality of the users is affected by their positions. 
%Each UE is connected to the gNB using the 3GPP clustered delay line (CDL) channel. 
We consider three different PHY slot configurations that are downlink-heavy with 7\,downlink (DL)-3\,uplink (UL) slots \texttt{(Configuration A)}, equalized with 5\,DL-5\,UL slots \texttt{(Configuration B)}, and an uplink-heavy configuration with 3\,DL-7\,UL slots \texttt{(Configuration C)}, each with a periodicity of 5\,ms. Table\,\ref{tab:notations} highlights the considered baseline 3GPP parameters across the other layers. 
%We study the baseline 3GPP Round Robin (RR) and Proportional Fair (PF) algorithms at the MAC layer. The evaluated UL RLC buffer size was 6\,KB and 10\,KB, and the application generated video frames of size 7.5\,KB with a frame rate of 30\,FPS and 60\,FPS.
% \begin{figure}[t!]
% \centering
% \includegraphics[trim=10 1 10 10,clip,width=0.55\linewidth]{Figures/prelim_experiment.png}
% \caption{Preliminary experimental setup for network characterization} %\cite{tilt}}
% \vspace{-5mm}
% \label{fig:prelim-exper}
% \end{figure}

\begin{comment}
\begin{table}[t!]
    \centering
    \vspace{-0.2in}
    \caption{Experiment Parameters}
    \vspace{-0.1in}
    \footnotesize
    
\begin{tabular}{|>{\centering\arraybackslash}p{0.2in}|>{\centering\arraybackslash}p{0.2in}|>{\centering\arraybackslash}p{0.7in}|>{\centering\arraybackslash}p{0.8in}|>{\centering\arraybackslash}p{0.7in}|}
    \hline
    \textbf{Layer} & \textbf{Index} & \multicolumn{3}{c|}{\textbf{Configurations}} \\ \hline
    \multirow{4}{*}{PHY}  &  & \textbf{DL Slots} & \textbf{UL Slots} & \textbf{Periodicity} \\ \cline{2-5}
    & 1 & 7 & 3 & 5 \\ \cline{2-5}
    & 2 & 5 & 5 & 5 \\ \cline{2-5}
    & 3 & 3 & 7 & 5 \\ \hline

    \multirow{4}{*}{MAC} &  & \textbf{Scheduler} & \textbf{Resource Allocation Type} & \textbf{-} \\ \cline{2-5}
    & 1 & Round Robin (RR) & 0 & - \\ \cline{2-5}
    % & Round Robin (RR) & 1 & - \\ \cline{2-5}
    & 2 & Proportional Fair (PF) & 1 & - \\ \hline

    \multirow{2}{*}{RLC} &  & \textbf{TX Buffer Size} & \textbf{SN Field Length} & \textbf{Entity} \\ \cline{2-5}
    & 1 & 2 & 6 & UMUL \\ \cline{2-5}
    & 2 & 4 & 6 & UMUL \\ \cline{2-5}
    & 3 & 6 & 6 & UMUL \\ \cline{2-5}
    & 4 & 8 & 6 & UMUL \\ \cline{2-5}
    & 5 & 10 & 6 & UMUL \\ \hline

    \multirow{3}{*}{APP} &  & \textbf{Frame Size (kb)} & \textbf{Frame Interval (ms)} & \textbf{Frame Rate (FPS)} \\ \cline{2-5}
    & 1 & 7.5 & 33 & 30 \\ \cline{2-5}
    & 2 & 7.5 & 16 & 60 \\ \hline
\end{tabular}
\label{tab:prelim-exper}
\end{table}
\end{comment}
%Table \ref{tab:prelim-exper} highlights the different parameters across the different layers considered, specifically, we consider parameters across the layers that affect the latency of UL data transmission. The following should also be noted: For the MAC layer, there was no custom scheduler implemented, rather, we consider the scheduling algorithms already implemented in the 5G toolbox. For the RLC layer, other parameters like prioritized bitrate, bucket size duration, and priority have significant effect on latency when considering multiple applications running on the same UE. This is a result of the logical channel prioritization in 5G (cite 3GPP spec), which is designed in such a way as to ensure that the UE satisfies the QoS requirements of each configured radio bearer for each application. But for our experiment, we are considering a single type of video streaming MCA application running on each UE. 
\noindent $\bullet$ \textbf {Observation:} The experimental results presented in Fig.\,\ref{fig:prelim_images} illustrate the impact of various 5G layer configurations on achieving the \textcolor{black}{target latency of 18\,ms} for 4\,UEs situated at different distances from the gNB. In Profile\,1, (Fig.\,\ref{fig:prelim_images_a}), modifying only the PHY parameters, specifically the DL and UL slot allocations, resulted in latency improvements for UEs\,2,\,3,\,and\,4, while UE\,1 experienced no significant change. This suggests that PHY layer adjustments alone may not guarantee latency requirements for UEs experiencing different channel conditions.
%particularly those closer to the gNB. 
Profiles\,2,\,3,\,and\,4 (Fig.\,\ref{fig:prelim_images_b},\,\ref{fig:prelim_images_c},\,\,\ref{fig:prelim_images_d}) demonstrate the impact of adjusting parameters across PHY, MAC, RLC, and APP layers together in meeting the required latency under different channel conditions. These configurations yielded more consistent latency improvements across all the UEs for specific configuration profiles.
%highlighting the importance of a holistic approach to network optimization.
%Using the discussed experiment setup and configuration parameters across the different layers, we show in Fig.\,\ref{fig:prelim_images}, the results of the achieved latency of each UE connected to the gNB in meeting a desired latency requirement desired by the application running. In Fig.\,\ref{fig:prelim_images}a, using the first combination of configuration parameters from the table, we can observe how changing parameters in the PHY layer while keeping other parameters fixed improves the required latency for the last three UEs while the first UE does not have any performance gain. For Fig. \ref{fig:prelim_images}b-c, we can observe how in addition to adjusting parameters in the PHY layer, how adjusting parameters in the MAC, RLC and APP layers improve the overall latency accross all UEs.

%This gives us an intuition that just adjusting the PHY layer parameters does not ensure that the application requirements are met for all UEs, therefore, other higher blayers have to be considered for each UE to meet the desired latency threshold requirements. In addition to these observations, we can observe how this approach of optimizing the network to meet the demands of MTC applications in real-time would be ineffective as traditional networks do not optimize the networks this way to 
{\em Our observations highlight the limitations of relying solely on fixed configurations or isolated layer-specific optimizations in 5G to meet diverse MCA requirements, especially in varying channel conditions.
%As evidenced by the varying latency responses across different UEs, a more nuanced and adaptive approach is needed. 
This motivates the need for exploring optimization strategies that consider the interplay between network and application layers, enabling dynamic adjustments tailored to individual MCA needs. }
%Such an approach is crucial for ensuring consistent and reliable QoS across emerging 5G applications and deployment scenarios.
%Adjusting certain NW layer parameters by following the fixed configurations across does not ensure that the application requirements are met for all UEs, therefore, informed optimization across the NW and application layers needs to be considered for each UE to meet the desired latency threshold requirements.
% The results from the experiments are shown in Fig.\,\ref{fig:prelim-thpt}.
% Here, we observe UL throughput of $\sim$8\,Mbps for the 7\,DL 3\,UL  configuration, $\sim$10\,Mbps for the 6\,DL 4\,UL configuration, $\sim$4.5\,Mbps for the 8\,DL 2\,UL configuration and $\sim$14\,Mbps for 5\,DL 5\,UL configuration. DL throughput ranged between 25\,-\,35\,Mbps in these frame configurations.
% These results underscore a clear inverse relationship between DL-heavy PHY frame configurations and UL throughput. Note that the UL to DL throughput ratio does not exactly follow the slot allocation ratio for two reasons: First, for UL transmissions, the UE utilizes one whole F/mixed slot to send a scheduling request to notify the NW regarding incoming UL data transmission in the upcoming frame cycle. Second, in 3GPP for a given MCS, the maximum data transport block size for DL is higher compared to UL\,\cite{snr-cqi-mcs}, leading to such asymmetry in throughput.
% \textcolor{blue}{to add a sentence to justify thpt difference between UL and DL}
\begin{comment}
\begin{figure}[t!]
	\centering
	\includegraphics[trim=0 30 0 0,clip,width=0.6\linewidth]{Figures/prelim_thpt.pdf}
	\caption{Iperf uplink throughput in OAI with different PHY frame configurations (7-2-1, 6-3-1, 8-1-1) } %\cite{tilt}}
 \vspace{-4mm}
	\label{fig:prelim-thpt}
\end{figure}
\end{comment}
%The 8\,DL 2\,UL slot configuration exhibited the lowest UL throughput, while the 5\,DL 5\,UL configuration demonstrated the highest among the considered PHY frame configurations, which is due to the reason of 5\,DL 5\,UL configuration having more UL slots compared to the others. 
%\vspace{-1mm}
\begin{comment}
\begin{theorem}
Reactive time domain resource allocation based on fixed PHY frame configurations may excel in either DL or UL, but in the presence of heterogeneous traffic demands, this procedure may struggle to achieve a balance between UL and DL transmissions. Dynamic PHY frame reconfiguration, informed by predicted channel conditions and application demand, holds promise in mitigating this imbalance, fostering coexistence between DL and UL-heavy applications.
\end{theorem}
\end{comment}
\begin{comment}
\noindent $\bullet$ \textbf{Latency performance:}  In this study, we conducted a series of experiments, varying packet sizes and evaluating the latency across multiple DL-heavy slot configurations within the context of the 5G PHY frame. The results are shown in Fig.\,\ref{fig:prelim-lat}. The observed latency values revealed a compelling pattern. For a 100\,KB packet size, we measured latency of 0.09 seconds with the 6-3-1 frame configuration, 0.16 seconds with the 7-2-1 configuration, and 0.22 seconds with the 8-1-1 configuration. As the packet size increased to 459\,KB, latency grew to 0.3\,seconds for the 6-3-1 configuration, 0.48\,seconds for the 7-2-1 configuration, and 0.8\,seconds for the 8-1-1 configuration. Similarly, with a 1\,MB packet size, we observed latency of 0.7\,seconds for the 6-3-1 configuration, 1.1\,seconds for the 7-2-1 configuration, and 1.9\,seconds for the 8-1-1 configuration. These results distinctly highlight the adverse effect of DL-heavy slot configurations on UL latency, with the 8-1-1 configuration demonstrating the highest latency and the 6-3-1 configuration achieving the lowest latency among the considered PHY frame configurations.

The KPI results from the preliminary experiments offer valuable insights into optimizing 5G network performance, emphasizing the importance of carefully balancing DL and UL resource allocation for enhanced efficiency and maintaining diverse application QoS thresholds.
\end{comment}
\begin{comment}
\begin{figure}[t!]
	\centering
	\includegraphics[trim=0 0 0 0,clip,width=0.7\linewidth]{Figures/prelim_latency.pdf}
	\caption{Latency in transmitting data of varying sizes through OAI uplink with different PHY frame configurations (7-2-1, 6-3-1, 8-1-1) } %\cite{tilt}}
 \vspace{-4mm}
	\label{fig:prelim-lat}
\end{figure}
\end{comment}
\begin{comment}
\begin{figure}[t!]
  \centering
  \begin{minipage}[t]{\linewidth}  
  \subfloat[Uplink throughput]{\includegraphics[trim=0 0 10 0,clip,width=0.5\textwidth]{Figures/prelim_thpt.pdf}\label{fig:prelim-thpt}}
  \hfill
  \subfloat[Uplink latency]{\includegraphics[trim=0 0 0 0,clip,width=0.5\textwidth]{Figures/prelim_latency.pdf}\label{fig:prelim-lat}}
  \label{}
  \caption{KPIs for legacy 5G OAI network, with (a) Iperf uplink throughput in OAI with different PHY frame configurations (7-2-1, 6-3-1, 8-1-1, and (b) Latency in transmitting data of varying sizes through OAI uplink with different PHY frame configurations (7-2-1, 6-3-1, 8-1-1) [to be updated]}
  \vspace{-3mm}
  \end{minipage}
  \vspace{-3mm}
\end{figure}
\end{comment}
\begin{comment}
\begin{figure}[t!]
    \centering
    \includegraphics[trim=0 0 0 0,width=0.6\linewidth]{Figures/OAI_legacy_thpt.png}
    \caption{KPIs for legacy 5G network, with Iperf uplink throughput in OAI with different PHY frame configurations.}
    \vspace{-6mm}
    \label{fig:prelim-thpt}
\end{figure}
\end{comment}
%\vspace{-1mm}