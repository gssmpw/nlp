\begin{abstract}
%The rise of the Internet of Things (IoT) and advancements in artificial intelligence (AI) have led to the proliferation of machine-type applications (MTAs). Unlike legacy human-type applications (HTAs) designed for human consumption (e.g., live video streaming), which have predictable Quality of Service (QoS) requirements, MTAs such as connected vehicles, industrial automation systems, etc., primarily interact with machines and sensors, which can be through the wireless medium for flexibility, and they present distinct characteristics and requirements.
%For MTAs involving low-cost edge devices that do not have enough computing capability, the AI-driven MTAs reside remotely in edge/cloud servers and interact with the devices via the wireless medium. 
%MTAs demand low latency, high throughput, scalability, and the ability to handle massive volumes of heterogeneous data. 5G networks, with their high bandwidth, low latency, and massive device connectivity, can be an important enabler for network-assisted MTAs. These applications can be primarily event-driven (e.g., object detection-based distributed machine tasks), resulting in heterogeneous data rates and sizes for the same application, that can be hard to predict. Coupled with the dynamic nature of the wireless channel, these requirements may exceed the capabilities of legacy network solutions designed for HTAs, highlighting a potential bottleneck in realizing the full potential of network-assisted MTAs. 
%This paper explores the evolution of MTAs, delves into their unique requirements, and investigates how legacy network architectures may prove insufficient for supporting their demands. 
%By understanding these challenges, researchers and network engineers can pave the way for the development of novel networking solutions that enable seamless and efficient operation of MTAs in the increasingly interconnected world. 
%Recent advancements in artificial intelligence (AI) have spurred the growth of emerging machine-centric applications (MCA) such as object detection and tracking in use cases like connected vehicles, smart surveillance, etc. These applications engage primarily with machines and sensors, often offloading the inference tasks to remote servers due to local computing and/or energy limitations.
%The proliferation of the Internet of Things (IoT) and advancements in artificial intelligence (AI) have spurred the growth of machine-type applications (MTAs)
%In contrast to legacy human-type applications (HTAs), which exhibit predictable Quality of Service (QoS) requirements, 
%such as those found in connected vehicles, industrial automation systems, etc., and they engage primarily with machines and sensors, often leveraging wireless communication for enhanced flexibility. 
%This distinction introduces unique characteristics and requirements. 
%MTAs necessitate low latency, high throughput, scalability, and the capacity to manage substantial volumes of heterogeneous data. 
%The inherent event-driven triggering of these MCAs can lead to unpredictable and diverse requirements, which when compounded by the dynamic nature of wireless channels, may strain legacy network solutions originally designed for human-centric applications (e.g., video streaming).
%To assess the performance of legacy 5G networks for supporting these applications, we focus on the network layers (PHY, MAC, RLC) of the 5G stack, together with the application characteristics.
%This potential mismatch underscores a critical challenge in fully realizing the potential of network-assisted MTAs. 
%5G networks, with their inherent capabilities of high bandwidth, low latency, and massive device connectivity, are poised to play a pivotal role in addressing these challenges and enabling the widespread adoption of MTAs.
%Our preliminary experiments with 3GPP compliant Matlab 5G toolbox reveal limitations of the {\em fixed} configuration of the network layers (PHY, MAC, RLC), that are typically used by commercial 5G networks, hindering the adaptability to heterogeneous MCA demands, affecting MCA quality of service (QoS) and spectrum efficiency. To overcome this drawback, we propose an {\em application-aware} deep reinforcement learning-enabled framework in the network that {\em learns} the optimal configuration across the network and application layers, based on the {\em context} of the UE MCA characteristics and proactively re-configures the parameters of these layers to maintain the MCA QoS. Integration and validation of our approach with the 3GPP-compliant Matlab 5G toolbox demonstrate the practicality of our solution. Overall, our application-centric context-aware optimization solution consistently meets heterogeneous traffic demands from the same type of MCA and meets the application-requested latency with minimal network resources
%by achieving a mean latency that is close to the requested application latency 
%compared to any fixed configuration of the network and application layers across various scenarios. 

% better \textcolor{red}{(quantify this)} than any fixed configuration of the network and application layers across various scenarios. 
Recent advancements in artificial intelligence (AI) and edge computing have accelerated the development of machine-centric applications (MCAs), such as smart surveillance systems. In these applications, video cameras and sensors offload inference tasks like license plate recognition and vehicle tracking to remote servers due to local computing and energy constraints. However, legacy network solutions, designed primarily for human-centric applications, struggle to reliably support these MCAs, which demand heterogeneous and fluctuating quality of service (QoS) (due to diverse application inference tasks), further challenged by dynamic wireless network conditions and limited spectrum resources. To tackle these challenges, we propose an Application Context-aware Cross-layer Optimization and Resource Design (ACCORD) framework. This innovative framework anticipates the evolving demands of MCAs in real time, quickly adapting to provide customized QoS and optimal performance, even for the most dynamic and unpredictable MCAs. This also leads to improved network resource management and spectrum utilization. ACCORD operates as a closed feedback-loop system between the application client and network and consists of two key components: (1) Building Application Context: It focuses on understanding the specific context of MCA requirements. Contextual factors include device capabilities, user behavior (e.g., mobility speed), and network channel conditions. (2) Cross-layer Network Parameter Configuration: Utilizing a deep reinforcement learning (DRL) approach, this component leverages the contextual information to optimize network configuration parameters across various layers, including physical (PHY), medium access control (MAC), and radio link control (RLC), as well as the application layer, to meet the desired QoS requirement in real-time. Extensive evaluation with the 3GPP-compliant MATLAB 5G toolbox demonstrates the practicality and effectiveness of our proposed ACCORD framework.
\end{abstract}
%\vspace{-2mm}