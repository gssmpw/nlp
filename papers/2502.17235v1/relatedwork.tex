\section{RELATED WORK}
Tidying up is an object rearrangement problem occurring in situations where the goal is not explicitly provided. 
Instead of a specific goal, several research approaches involve expressing goals in natural language \cite{StructFormer, StructDiffusion, LGMCTS}, 
or finding functional arrangements based on user preferences \cite{myhouse, RobotOrganize}.
Additionally, there are studies that directly learn the degree of tidiness as a score function and plan trajectories to achieve a tidied scene \cite{SceneScore, TarGF}.

Recent studies such as StructFormer \cite{StructFormer} and StructDiffusion \cite{StructDiffusion} find appropriate positions for objects guided by natural language instructions. 
Both methods take language tokens and object point clouds as inputs to find arrangements that satisfy language conditions. 
Studies such as \cite{myhouse} and \cite{RobotOrganize} learn user preferences to find organized arrangements without explicit goals.
For instance, \cite{myhouse} uses scene graphs to encode scenes and learns user preference vectors, 
and \cite{RobotOrganize} addresses tasks involving the organization of various items into containers or shelves, learning pairwise preferences of objects. % through collaborative filtering.
These studies rely more on semantic information rather than visual information of the objects.
There exist diffusion based methods to directly generate final arrangement images \cite{DalleBot}, \cite{LVDiffuser}.
These studies rely on the commonsense knowledge inherent in large language models (LLMs) and vision language models (VLMs) to find arrangements that are similar to human intentions. 

Similar to the current work, studies such as \cite{SceneScore} and \cite{TarGF} learn to quantify the degree of tidiness with a score function. 
\cite{SceneScore} uses an energy-based model to learn and predict the cost, which is most relevant to our work. 
While \cite{SceneScore} focused on finding positions for just one missing object, our study plans to find the optimal state by moving all movable objects on a table. 
\cite{TarGF} learns a score function to calculate the likelihood with the target distribution for each task, using this score to learn a policy for rearranging objects. 
Each task requires a separate target distribution, and the score function is trained separately for each task, whereas our study uses a single score function to tidy up across various environments.

Language-guided Monte-Carlo tree search (LGMCTS) \cite{LGMCTS} uses the MCTS algorithm to find trajectories to obtain arrangements that satisfy language conditions. 
LGMCTS assumes that explicit spatial conditions can be derived based on language conditions. 
They first establish these spatial conditions and then find a trajectory that arranges the objects to satisfy all these conditions. 
In this paper, we propose an algorithm that learns a score function to find various tidied arrangements without the guidance of language.