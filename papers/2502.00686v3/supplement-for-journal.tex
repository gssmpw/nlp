\documentclass{article}
\usepackage[margin=1in]{geometry}
% \setlength{\textwidth}{12.2cm}
% \setlength{\textheight}{19.3cm}

%
% RECOMMENDED %%%%%%%%%%%%%%%%%%%%%%%%%%%%%%%%%%%%%%%%%%%%%%%%%%%
%
%%%% Standard Packages
%%<additional latex packages if required can be included here>
\usepackage{graphicx}%
\usepackage{multirow}%
\usepackage{amsmath,amssymb,amsfonts}%
\usepackage{mathrsfs}%
\usepackage[title]{appendix}%
\usepackage{xcolor}%
\usepackage{textcomp}%
\usepackage{manyfoot}%
\usepackage{booktabs}%
\usepackage{algorithm}%
\usepackage{algorithmicx}%
\usepackage{algpseudocode}%
\usepackage{listings}%
\usepackage{subcaption}%

\definecolor{codegreen}{rgb}{0,0.6,0}
\definecolor{codegray}{rgb}{0.5,0.5,0.5}
\definecolor{codepurple}{rgb}{0.58,0,0.82}
\definecolor{backcolour}{rgb}{0.95,0.95,0.92}

\lstdefinestyle{mystyle}{
    backgroundcolor=\color{backcolour},   
    commentstyle=\color{codegreen},
    keywordstyle=\color{magenta},
    numberstyle=\tiny\color{codegray},
    stringstyle=\color{codepurple},
    basicstyle=\ttfamily\footnotesize,
    breakatwhitespace=false,         
    breaklines=true,                 
    captionpos=b,                    
    keepspaces=true,                 
    numbers=left,                    
    numbersep=5pt,                  
    showspaces=false,                
    showstringspaces=false,
    showtabs=false,                  
    tabsize=2
}

\lstset{style=mystyle}

% \renewcommand{\thesection}{S\arabic{section}}  
% \renewcommand{\thetable}{S\arabic{table}}  
% \renewcommand{\thefigure}{S\arabic{figure}}
\renewcommand\thesection{\Alph{section}}
\renewcommand{\thetable}{\Alph{table}}  
\renewcommand{\thefigure}{\Alph{figure}}


\usepackage{longtable}% added by min for long table
\usepackage{multirow}% added by min for description length multicolumn table
% to typeset URLs, URIs, and DOIs
\usepackage{url}
\def\UrlFont{\rmfamily}
\def\orcidID#1{\unskip$^{[#1]}$} % added MR 201803-10

% Use the PLoS provided BiBTeX style
% \bibliographystyle{plos2015}
\bibliographystyle{plain}

% Remove brackets from numbering in List of References
\makeatletter
\renewcommand{\@biblabel}[1]{\quad#1.}
\makeatother


\begin{document}


%\mainmatter              % start of a contribution
%
\title{Supplementary Materials for Improved Community Detection using Stochastic Block Models}
%
%\titlerunning{Running title}  % abbreviated title (for running head)
%                                     also used for the TOC unless
%                                     \toctitle is used
%

\author{Minhyuk Park\textsuperscript{1} \and 
Daniel Wang Feng\textsuperscript{1} \and
Siya Digra\textsuperscript{1} \and 
The-Anh Vu-Le\textsuperscript{1} \and 
Lahari Anne \textsuperscript{1} \and
George Chacko \textsuperscript{1} \and 
Tandy Warnow\textsuperscript{1}}


\date{}
% \authorrunning{Park et al.} % abbreviated author list (for running head)



% \institute{Department of Computer Science, University of Illinois Urbana-Champaign, Urbana IL 61801\\
% \email{\{minhyuk2,warnow,chackoge\}@illinois.edu}\\ 
% }

\maketitle % typeset the title of the contribution

\textsuperscript{1}\emph{Siebel School of Computing and Data Science, University of Illinois Urbana-Champaign, Urbana, IL 61801. \{minhyuk2, chackoge, warnow\}@illinois.edu}\}

% \clearpage
\tableofcontents
\listoffigures
\listoftables

\clearpage
\section{Additional Figures}
\begin{figure}[!ht]
\centering
% \includegraphics[]{./figs/medium_large_cluster_size_boxplot_fonts_embedded.eps}
\includegraphics[]{figs/fig_s1.eps}
\caption[Experiment 2: Impact of treatment on the non-singleton cluster size distributions of SBM clusterings on medium and large real-world networks (non-bipartite)]{\textbf{Experiment 2: Impact of treatment on the non-singleton cluster size distributions of SBM clusterings on medium and large real-world networks (non-bipartite) }
We show that SBM+WCC is competitive compared to Leiden-CPM(0.001) or Leiden-mod with the exception of very few cases. For 84 real-world networks grouped by the number of nodes, labeled medium ($1000 < n < 1,000,000$) and large ($n \geq{} 1,000,000$), we show the distribution of non-singleton cluster sizes resulting from clustering these networks with the lowest description model of SBM as well as its CC, WCC, and CM treatment clusterings where SBM has the largest cluster sizes followed by CM and then CC and WCC.. The boxplots are shown with a log scale y-axis with the whiskers indicating the smallest and largest cluster sizes for each clustering method in a dataset group. On one of the lage-sized datasets, WCC encountered an out-of-memory error on a machine with 256GB of RAM and is not included in any of the statistics for SBM, SBM+CC, or SBM+CM. The median and maximum cluster sizes for each box plot are listed here. Medium group median/max: SBM: 91/132109, SBM+CC: 3/38539, SBM+WCC: 3/2966, SBM+CM: 6/2169. Large group median/max: SBM: 933/262816, SBM+CC: 4/10187, SBM+WCC: 3/4387, SBM+CM: 9/3258% }\textbf{\textcolor{red}{Re-make table without bipartite graphs today (1/17) - update: it's been updated for both the actual figure and the caption listing the numbers. The boxes shifted down.}


% The distribution of non-singleton cluster sizes  is shown as a boxplot for the selected SBM and its treatments. The y-axis is plotted on a log scale with the whiskers indicating the minimum and maximum cluster sizes in all of the networks in the group.
% Both groups and treatments have minimum cluster size of 2 for SBM clusterings whether treated or not, but differ in the medians and maxes, as follows.
}\label{fig:s1}
\end{figure}
\begin{figure}[!h]
\centering
% \includegraphics[]{./figs/lfr_reccs_nmi_accuracies_individual.eps}
\includegraphics[]{figs/fig_s2.eps}
\caption[Experiment 3b: NMI accuracies of SBM+WCC against various methods and their treatments on LFR and RECCS synthetic networks]{\textbf{Experiment 3b: NMI accuracies of SBM+WCC against various methods and their treatments on LFR and RECCS synthetic networks.} We show that SBM+WCC is competitive compared to Leiden-CPM(0.001) or Leiden-mod with the exception of very few cases. Each subplot shows results for one synthetic network (either LFR or RECCS), defined by the real-world network (vertical) axes and clustering (horizontal axes).
The ``N/A'' cells are those networks that are not available; see text for explanation.
There are  three out-of-memory (oom) entries (top row) and one ``time-out" entry (bottom row).}
\label{fig:s2}
\end{figure}

 
\begin{figure}[!h]
\centering
% \includegraphics[]{./figs/lfr_reccs_rnmi_accuracies_individual.eps}
\includegraphics[]{figs/fig_s3.eps}
\caption[Experiment 3b: RMI accuracies of SBM+WCC against various methods and their treatments on LFR and RECCS synthetic networks]{\textbf{Experiment 3b: RMI accuracies of SBM+WCC against various methods and their treatments on LFR and RECCS synthetic networks.}
Each subplot shows results for one synthetic network (either LFR or RECCS), defined by the real-world network (vertical) axes and clustering (horizontal axes).
The ``N/A'' cells are those networks that are not available; see text for explanation.
There are  three out-of-memory (oom) entries (top row) and one ``time-out" entry (bottom row).}
\label{fig:s3}
\end{figure}
\clearpage
\section{Extra Tables}
\begin{table}[!ht]
\centering
\caption[Experiment 2: Impact of treatment on node coverage on non-bipartite real-world networks]{\textbf{Experiment 2: impact of treatment on node coverage on non-bipartite real-world networks. } For 84 real-world networks (non-bipartite) grouped by the number of nodes, labeled small ($n \leq{} 1000$), medium ($1000 < n < 1,000,000$), and large ($n \geq{} 1,000,000$), we show the percentage of nodes in non-singleton clusters for the selected SBM clusterings (i.e., model achieving the lowest description length), with and without CC, WCC, and CM treatments. The node coverages are averaged across networks. On one of the lage-sized datasets, WCC encountered an out-of-memory error on a machine with 256GB of RAM and is not included in any of the statistics for SBM, SBM+CC, or SBM+CM.}
\label{tab:s1}
\begin{tabular}{lccc}
\hline
clustering & \multicolumn{3}{c}{node coverage} \\
& small  & medium  & large \\
\hline
Selected SBM & 74\% & 100\% & 100\% \\
Selected SBM + CC& 74\% & 64\% & 54\% \\ % used to be 59
Selected SBM + WCC& 67\% & 48\% & 38\% \\
Selected SBM + CM& 67\% & 35\% & 32\% \\ % used to be 32
\hline
\end{tabular}
\begin{flushleft}

\end{flushleft}
\end{table}

\begin{table}[!ht]
\centering
\caption[Experiment 2: Impact of treatment on node coverage on bipartite real-world networks]{\textbf{Experiment 2: impact of treatment on node coverage on bipartite real-world networks. } For 35 real-world networks (bipartite) grouped by the number of nodes, labeled small ($n \leq{} 1000$), medium ($1000 < n < 1,000,000$), and large ($n \geq{} 1,000,000$), we show the percentage of nodes in non-singleton clusters for the selected SBM clusterings (i.e., model achieving the lowest description length), with and without CC, WCC, and CM treatments. The node coverages are averaged across networks.}
\label{tab:s2}
\begin{tabular}{lccc}
\hline
clustering & \multicolumn{3}{c}{node coverage} \\
& small  & medium  & large \\
\hline
Selected SBM & 100\% & 100\% & 100\% \\
Selected SBM + CC& 100\% & 11\% & 27\% \\
Selected SBM + WCC& 82\% & 9\% & 12\% \\
Selected SBM + CM& 82\% & 0\% & 5\%\\
\hline
\end{tabular}
\begin{flushleft}

\end{flushleft}
\end{table}


\begin{table}[!ht]
% \begin{adjustwidth}{-2.25in}{0in} % Comment out/remove adjustwidth environment if table fits in text column.
\centering
\caption[Experiment 2: Average number of non-singleton clusters for the selected SBM, both treated and untreated, per non-bipartite real-world network group]{\textbf{Experiment 2: Average number of non-singleton clusters for the selected SBM, both treated and untreated, per non-bipartite real-world network group} For 84 real-world networks grouped by the number of nodes, labeled small ($n \leq{} 1000$),  medium ($1000 < n < 1,000,000$), and large ($n \geq{} 1,000,000$), we show the average number of non-singleton clusters for each method. The model of SBM for each dataset is determined by whichever model, chosen from degree corrected, non degree corrected, and planted partition, produced the clustering with the lowest description length. The number of clusters are averaged across networks. The follow-up CC, WCC, and CM treatments were done on the lowest description length SBM clustering. On one of the lage-sized datasets, WCC encountered an out-of-memory error on a machine with 256GB of RAM and is not included in any of the statistics for SBM, SBM+CC, or SBM+CM. }
\begin{tabular}{@{}lccc@{}}
\hline
 & small & medium & large \\
 \hline
Selected SBM & 1.43& 398.96& 3024.50\\ % used to be 2539.80
Selected SBM + CC & 1.43& 3214.05& 25955.00\\ % used to be 66431.20
Selected SBM + WCC & 1.43& 4171.45& 45969.25\\ 
Selected SBM + CM & 1.43& 2426.70 & 48331.75 \\ % used to be 39814.00
\hline
\end{tabular}
\begin{flushleft}

\end{flushleft}
\label{tab:s3}
% \end{adjustwidth}
\end{table}
% \section{Extra Figures}
% \begin{figure}[htpb!]
%     \centering
%     \includegraphics[width=1\linewidth]{./figs/lfr_reccs_node_coverage_individual.eps}
%     \caption[\textbf{Node Coverage of Various Methods and Their Treatments on LFR and RECCS synthetic networks}]{\textbf{Node Coverage of Various Methods and Their Treatments on LFR and RECCS synthetic networks} Chosen SBM-WCC on RECCS wiki\_topcats as well as Leiden-mod-WCC on RECCS cit\_patents, RECCS wiki\_topcats, and RECCS cen ran into a memory error with 256GB of RAM. \textcolor{red}{chosen SBM on RECCS cen has not been started yet.}}
%     \label{fig:enter-label}
% \end{figure}
% \begin{figure}[htpb!]
%     \centering
%     \includegraphics[width=1\linewidth]{./figs/lfr_reccs_nmi_accuracies_individual.eps}
%     \caption[\textbf{NMI Accuracies of Various Methods and Their Treatments on LFR and RECCS synthetic networks}]{\textbf{NMI Accuracies of Various Methods and Their Treatments on LFR and RECCS synthetic networks} Chosen SBM-WCC on RECCS wiki\_topcats as well as Leiden-mod-WCC on RECCS cit\_patents, RECCS wiki\_topcats, and RECCS cen ran into a memory error with 256GB of RAM. \textcolor{red}{chosen SBM on RECCS cen has not been started yet.}}
%     \label{fig:enter-label}
% \end{figure}
% \begin{figure}[htpb!]
%     \centering
%     \includegraphics[width=1\linewidth]{./figs/lfr_reccs_ari_accuracies_individual.eps}
%     \caption[\textbf{ARI Accuracies of Various Methods and Their Treatments on LFR and RECCS synthetic networks}]{\textbf{ARI Accuracies of Various Methods and Their Treatments on LFR and RECCS synthetic networks} Chosen SBM-WCC on RECCS wiki\_topcats as well as Leiden-mod-WCC on RECCS cit\_patents, RECCS wiki\_topcats, and RECCS cen ran into a memory error with 256GB of RAM. \textcolor{red}{chosen SBM on RECCS cen has not been started yet.}}
%     \label{fig:enter-label}
% \end{figure}
% \begin{figure}[htpb!]
%     \centering
%     \includegraphics[width=1\linewidth]{./figs/lfr_reccs_ami_accuracies_individual.eps}
%     \caption[\textbf{AMI Accuracies of Various Methods and Their Treatments on LFR and RECCS synthetic networks}]{\textbf{AMI Accuracies of Various Methods and Their Treatments on LFR and RECCS synthetic networks} Chosen SBM-WCC on RECCS wiki\_topcats as well as Leiden-mod-WCC on RECCS cit\_patents, RECCS wiki\_topcats, and RECCS cen ran into a memory error with 256GB of RAM. \textcolor{red}{1. Chosen SBM on RECCS cen has not been started yet. 2. The clusterings for Leiden-CPM(0.001)-WCC, Leiden-CPM(0.001)-WCC, and Leiden-mod-CM on RECCS cen are done but their AMI, AGRI, and RMI accuracies have not been computed yet. Before the campuscluster went down, they had about 30 hours to produce the results but all of them timed out.}}
%     \label{fig:enter-label}
% \end{figure}
% \begin{figure}[htpb!]
%     \centering
%     \includegraphics[width=1\linewidth]{./figs/lfr_reccs_agri_accuracies_individual.eps}
%     \caption[\textbf{AGRI Accuracies of Various Methods and Their Treatments on LFR and RECCS synthetic networks}]{\textbf{AGRI Accuracies of Various Methods and Their Treatments on LFR and RECCS synthetic networks} Chosen SBM-WCC on RECCS wiki\_topcats as well as Leiden-mod-WCC on RECCS cit\_patents, RECCS wiki\_topcats, and RECCS cen ran into a memory error with 256GB of RAM. \textcolor{red}{1. Chosen SBM on RECCS cen has not been started yet. 2. The clusterings for Leiden-CPM(0.001)-WCC, Leiden-CPM(0.001)-WCC, and Leiden-mod-CM on RECCS cen are done but their AMI, AGRI, and RMI accuracies have not been computed yet. Before the campuscluster went down, they had about 30 hours to produce the results but all of them timed out.}}
%     \label{fig:enter-label}
% \end{figure}
% \begin{figure}[htpb!]
%     \centering
%     \includegraphics[width=1\linewidth]{./figs/lfr_reccs_rnmi_accuracies_individual.eps}
%     \caption[\textbf{RMI Accuracies of Various Methods and Their Treatments on LFR and RECCS synthetic networks}]{\textbf{RMI Accuracies of Various Methods and Their Treatments on LFR and RECCS synthetic networks} Chosen SBM-WCC on RECCS wiki\_topcats as well as Leiden-mod-WCC on RECCS cit\_patents, RECCS wiki\_topcats, and RECCS cen ran into a memory error with 256GB of RAM. \textcolor{red}{1. Chosen SBM on RECCS cen has not been started yet. 2. The clusterings for Leiden-CPM(0.001)-WCC, Leiden-CPM(0.001)-WCC, and Leiden-mod-CM on RECCS cen are done but their AMI, AGRI, and RMI accuracies have not been computed yet. Before the campuscluster went down, they had about 30 hours to produce the results but all of them timed out.}}
%     \label{fig:enter-label}
% \end{figure}


% \begin{figure}[htpb!]
%     \centering
%     \includegraphics[width=1\linewidth]{./figs/large_empirical_sbm_treatment_runtime.eps}
%     \caption[\textbf{Runtime of SBM and SBM + Treatments on Large Empirical Networks (no bipartite graphs)}]{\textbf{Runtime of SBM and SBM + Treatments on Large Empirical Networks (no bipartite graphs)} Here we show the runtime of each method in hours:minutes:seconds format. For Chosen SBM-CC and Chosen SBM-WCC, only the time it took to run the treatment is shown. Number of nodes: orkut - 3,072,441; livejournal - 4,847,571; bitcoin - 6,336,770; cen - 75,025,194}
%     \label{fig:enter-label}
% \end{figure}

% \begin{figure}
%     \centering
%     \includegraphics[width=1\linewidth]{./figs/best_sbm_connectivity_no_bipartite.eps}
%     \caption[\textbf{Cluster Connectivity of SBM on Real-World Networks (no bipartite graphs)}]{\textbf{Cluster Connectivity of SBM (no bipartite graphs)} Neworks: 7 small, 73 medium, 3 large, cen, and orkut.}
%     \label{fig:enter-label}
% \end{figure}
% \begin{figure}
%     \centering
%     \includegraphics[width=1\linewidth]{./figs/best_sbm_connectivity_only_bipartite.eps}
%     \caption[\textbf{Cluster Connectivity of SBM on Real-World Networks (only bipartite graphs)}]{\textbf{Cluster Connectivity of SBM (only bipartite graphs)} Neworks: 1 small, 30 medium, and 4 large.}
%     \label{fig:enter-label}
% \end{figure}

% \begin{figure}
% \begin{subfigure}{0.5\textwidth}
%     \includegraphics[width=\textwidth]{./figs/cc_node_coverage_no_bipartite.eps}
%     \caption{No Bipartite}
% \end{subfigure}
% \begin{subfigure}{0.5\textwidth}
%     \includegraphics[width=\textwidth]{./figs/cc_node_coverage_only_bipartite.eps}
%     \caption{Only Bipartite}
% \end{subfigure}
% \caption[\textbf{Node coverage of SBM-CC on Real-World Networks }]{\textbf{Node coverage of SBM-CC}}
% \end{figure}
% \begin{figure}
% \begin{subfigure}{0.5\textwidth}
%     \includegraphics[width=\textwidth]{./figs/wcc_node_coverage_no_bipartite.eps}
%     \caption{No Bipartite}
% \end{subfigure}
% \begin{subfigure}{0.5\textwidth}
%     \includegraphics[width=\textwidth]{./figs/wcc_node_coverage_only_bipartite.eps}
%     \caption{Only Bipartite}
% \end{subfigure}
% \caption[\textbf{Node coverage of SBM-WCC on Real-World Networks }]{\textbf{Node coverage of SBM-WCC}}
% \end{figure}
\clearpage
\section{Software} \label{sec:software}
The main implementation for WCC, and CC, are written in C++ and is hosted at this repository \url{https://github.com/MinhyukPark/constrained-clustering}.

%\section*{Commands and Versions} \label{sec:commands-and-versions}
 CC and WCC treatments on Leiden-\{mod, cpm\} on LFR networks were done with the Python code base. CC and WCC treatments on all models of SBM were done with the C++ code base. CC of both code bases are deterministic while WCC of each code bases may result in different outcomes each time depending on which mincut is selected when there is a tie. CM was only done through the Python code base.

\paragraph{SBM} graph-tool can be found at \url{https://graph-tool.skewed.de/static/doc/} \cite{graph-tool}. We used version 2.59 on Python 3.9.18.
\begin{lstlisting}[language=Python]
import graph_tool.all as gt
g = gt.load_graph_from_csv(inputGraphName, csv_options = {'delimiter': '\t'})
```
clustering = gt.minimize_blockmodel_dl(g, state=gt.BlockState, state_args=dict(deg_corr=True)) # for DC sbm
clustering = gt.minimize_blockmodel_dl(g, state=gt.BlockState, state_args=dict(deg_corr=False)) # for Non DC sbm
clustering = gt.minimize_blockmodel_dl(g, state=gt.PPBlockState) # for PP sbm
```
description_length = clustering.entropy()
membershipFile = open(outputMemberName, "w+")
blockMembership = clustering.get_blocks()
for v in g.vertices():
    nodeId = g.vp.name[v]
    cur_block = blockMembership[v]
    if cur_block < 0 or cur_block > num_nodes_total - 1:
        continue
    membershipFile.write((str)(nodeId) + "\t" + (str)(cur_block) + "\n")
membershipFile.close()
\end{lstlisting}

\paragraph{CC} CC code can be found at \url{https://github.com/MinhyukPark/constrained-clustering}. The tag is v1.1.0.
\begin{lstlisting}
./constrained_clustering MincutOnly --edgelist <tab separated edgelist network> --existing-clustering <input clustering> --num-processors <maximum allowed parallelism> --output-file <output file path> --log-file <log file path> --log-level 1 --connectedness-criterion 0
\end{lstlisting}

\paragraph{WCC} WCC code can be found at \url{https://github.com/MinhyukPark/constrained-clustering} The tag is v1.1.0.
\begin{lstlisting}
./constrained_clustering MincutOnly --edgelist <tab separated edgelist network> --existing-clustering <input clustering> --num-processors <maximum allowed parallelism> --output-file <output file path> --log-file <log file path> --log-level 1 --connectedness-criterion 1
\end{lstlisting}

\paragraph{CM} CM code can be found at \url{https://github.com/illinois-or-research-analytics/cm_pipeline} The commit was a1c1d29.
\begin{lstlisting}
python3 -m hm01.cm -i <tab separated edgelist network> -e <input clustering> -o <output file path> -c external -cfile <clusterer file path e.g., path to hm01/clusterers/external_clusterers/sbm_wrapper.py> --threshold <threhsold e.g., 1log10> -cargs <clusterer args path (json detail found in repository) > --nprocs <number of processors>
\end{lstlisting}
\clearpage
\section{Real-world networks}
The lowest description length SBM clustering on ceo\_club and elite networks from the small networks returned all singleton clusters. These datasets were excluded from our study.\\


\noindent{}\textbf{Small networks (10) from \cite{peixoto2020netzschleuder} sorted by increasing node count} \\

\noindent{}\textit{bipartite}: sa\_companies; ceo\_club; elite; \\

\noindent{}\textit{non-bipartite}: november17; dutch\_criticism; macaque\_neural; sp\_kenyan\_households; contiguous\_usa; cs\_department; dolphins \\
% The selected SBM clustering of the ceo\_club and elite networks returned only singleton clusters.

\noindent{}\textbf{Medium-size networks (103) from \cite{peixoto2020netzschleuder} sorted by increasing node count} \\

\noindent{}\textit{bipartite}: plant\_pol\_robertson; escorts; movielens\_100k; nematode\_mammal; paris\_transportation; jester; dbpedia\_writer; digg\_votes; dbtropes\_feature; dbpedia\_starring; github; dbpedia\_recordlabel; dbpedia\_producer; dbpedia\_location; dbpedia\_occupation; dbpedia\_genre; discogs\_label; wiki\_article\_words; corporate\_directors; lkml\_thread; bookcrossing; flickr\_groups; visualizeus; dbpedia\_country; stackoverflow; eu\_procurements; epinions; citeulike; dbpedia\_team; bibsonomy \\

\noindent{}\textit{non-bipartite}: dnc; uni\_email; polblogs; faa\_routes; netscience; new\_zealand\_collab; collins\_yeast; interactome\_stelzl; bible\_nouns; at\_migrations; interactome\_figeys; us\_air\_traffic; drosophila\_flybi; fly\_larva; interactome\_vidal; openflights; bitcoin\_alpha; fediverse; power; advogato; bitcoin\_trust; jung; reactome; jdk; elec; chess; sp\_infectious; wiki\_rfa; dblp\_cite; anybeat; chicago\_road; foldoc; inploid; google; marvel\_universe; fly\_hemibrain; internet\_as; word\_assoc; cora; lkml\_reply; linux; topology; email\_enron; pgp\_strong; facebook\_wall; slashdot\_threads; python\_dependency; marker\_cafe; epinions\_trust; slashdot\_zoo; twitter\_15m; prosper; wiki\_link\_dyn; livemocha; wikiconflict; lastfm\_aminer; wiki\_users; wordnet; douban; academia\_edu; google\_plus; libimseti; email\_eu; stanford\_web; dblp\_coauthor\_snap; notre\_dame\_web; citeseer; twitter; petster; yahoo\_ads; berkstan\_web; myspace\_aminer; google\_web \\

\noindent{}\textbf{Large networks (7) from \cite{peixoto2020netzschleuder} sorted by increasing node count} \\

\noindent{}\textit{bipartite}: reuters; discogs\_affiliation; amazon\_ratings; dblp\_author\_paper \\

\noindent{}\textit{non-bipartite}: hyves; livejournal; bitcoin \\

% WCC on the bitcoin network ran into memory errors with 256GB.
\noindent{}\textbf{Other datasets (2) from \cite{park2024}} \\

\noindent{}\textit{non-bipartite}: orkut; CEN (Curated Exosome Network) \\

\clearpage
\section{RECCS synthetic network generation protocol}
RECCS synthetic network generator takes two inputs: a network and clustering on that network. Using these two inputs, it can then output a synthetic network using the input clustering as ground-truth. For our study, we used three real-world networks (cit\_hepph, cit\_patents, and wiki\_topcats) and clustered each of them using Leiden-CPM(0.01). These real-world network and clustering pairs were given to RECCS to generate the final synthetic networks. The details for how to run RECCS and the software can be found at \url{https://github.com/illinois-or-research-analytics/lanne2_networks/tree/main/generate_synthetic_networks}.


% The first step in the RECCS pipeline is to separate the isolated nodes which may be included in $C$. Given the set of nodes $V_{clustered}$ and $V_{isolated}$, we can create an induced subnetwork of $N_{real-world}$ on $V_{clustered}$. The command for this is shown below. Running the command yields $N_{clustered}$ for the induced subnetwork. For convenience, we will denote $C$ induced on $V_{clustered}$ as $C_{clustered}$.

% \begin{lstlisting}[mathescape]
% python3 clean_outliers.py --input-network <$N_{real-world}$> --input-clustering <$C$> --output-folder <output directory>
% \end{lstlisting}

% Using the induced subnetwork $N_{clustered}$, which is $N_{real-world}$ on the set of nodes that are in non-singleton clusters of $C$, we generated the SBM network using the following command. This command will generate two different networks whose filenames include ``v1'' and ``v2''. For our study, we used the v1 output, and we will call this v1 output network $SBM_{clustered}$. 

% \begin{lstlisting}[mathescape]
% python gen_SBM.py -f <$N_{clustered}$> -c <$C_{clustered}$> -o <output directory>
% \end{lstlisting}

% Once the SBM network on $V_{clustered}$ is generated, we run the next command to ensure the minimum connectivity of each cluster.

% \begin{lstlisting}[mathescape]
% python reccs.py -f <$SBM_{clustered}$> -c <$C_{clustered}$> -o <output directory> -ef <$N_{clustered}$>
% \end{lstlisting}

% Finally, we add the isolated nodes to $SBM_{clustered}$ to generate the final synthetic network output $SBM_{clustered+isolated}$ using the command below.

% \begin{lstlisting}[mathescape]
% python outliers_strategy1.py -f <$N_{clustered}$> -c <$C_{clustered}$> -o <output directory> -s <$SBM_{clustered}$>
% \end{lstlisting}

\bibliography{clustering}
\end{document}

\end{document}