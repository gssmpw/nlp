\section{Introduction}

Parametric Computer-Aided Design (CAD) systems revolutionized manufacturing-oriented design by introducing a paradigm where geometry is created through a sequence of constructive operations. This approach enables both accuracy and precision in modeling and offers flexibility in design editing. Essentially, CAD systems use domain-specific languages (DSLs) to express geometry as a program, with CAD GUIs as end-user programming interfaces.

Recent advances in generative AI have significantly enhanced the creation of 2D and 3D geometry, yet achieving the precision, detail, and editability provided by CAD models remains a challenge. To bridge this gap, one promising strategy is to harness the powerful code generation capabilities of pre-trained large language models (LLMs) and the geometry-as-a-program paradigm from CAD. 
Rather than generating the geometries directly, we generate CAD programs that produce the geometric structures. However, this raises a crucial question: \textbf{How can we reimagine the traditional CAD DSL principles, which have been designed for a constant visual feedback loop, to craft innovative languages for design in an age where code is generated with support from AIs?}

In this work, we address this question and propose a new DSL for CAD modeling with LLMs, which we call AIDL: AI Design Language. Through experiments with different existing models and prior work that analyzes their observed behavior, we identify four key \textbf{\textit{design goals}} for our DSL. Namely, we propose a \textit{solver-aided approach} that enables LLMs to concentrate on high-level reasoning that they excel at while offloading finer computational tasks that demand precision to external solvers. For CAD, this means that the DSL should enable implicitly referencing previously constructed geometry (\dgone{}) and specifying relationships between parts that can then be solved by the solver (\dgtwo{}). Further, we aim to create \textit{semantically meaningful abstractions} that leverage the LLM's proficiency in understanding and manipulating natural language (\dgthree{}). Finally, we advocate for a \textit{hierarchical design approach}, which allows for encapsulating reasoning within different model parts and enhancing editability (\dgfour{}). 

Our analysis of existing CAD DSLs reveals that none achieve all four design goals, and supporting all goals simultaneously presents challenges due to conflicting requirements. For example, the ability to unambiguously reference all intermediate parts of the geometry (\dgone{}) is a known challenge in CAD. While recent work proposes a language that supports unambiguous referencing, it requires semantic complexity (\dgthree{}). Additionally, while constraints are widely used in specific aspects of CAD design, such as assembly modeling (\dgtwo{}), supporting them in a complex model with hierarchically defined constraints (\dgfour{}) is computationally challenging. 
%AIDL addresses these challenges through the use of novel domain-specific constructs and a back-end recursive solver, demonstrating support for all design goals.
Our key insight is that we can address these challenges by both limiting and expanding different language constructs from prior CAD DSLs.
While we limit the use of references to \emph{constructed} geometry, without losing geometric expressivity, we expand the use of constraints to hierarchical groups of geometry, so called \emph{structures}. We support these novel language constructs with a recursive constraint solver that leverages the hierarchical structure to tractably solve global constraint systems.

%\jz{To evaluate AIDL's capability in supporting LLM-driven CAD design, we built a font-end generation pipeline which interacts with the back-end solver to provide feedback, including syntactic errors and solver issues, to the LLM.} \adriana{this is no longer true is it?}
We present a series of text-to-CAD results in 2D generated with our language, and we evaluate the importance of different aspects of AIDL by comparing it to OpenSCAD, a popular CAD language, and subsets of the AIDL language that has hierarchy or constraints disabled. For these methods, we report CLIP scores of the generated results and conducted a perceptual study on the generated CAD renderings. Our experiments show that AIDL programs are visually on-par with or better than their OpenSCAD counterparts despite the LLM not seeing AIDL code in its training data, while having superior editability, and our ablations demonstrate that introducing hierarchy contributes to local editability, while constraints allow complex multi-part objects to be composed precisely. With AIDL we show that language design alone can improve LLM performance in CAD generation.


\begin{figure}[!htbp]
\centering
\includegraphics[width=\linewidth]{images/telephone_teaser_linefix.png}
% \begin{minipage}{.5\textwidth}
%   \centering
%   \includegraphics[scale=0.12]{images/carbon (2).png}
%   \label{fig:test1}
% \end{minipage}%
% \begin{minipage}{.6\textwidth}
%   \centering
%   \includegraphics[scale=0.75]{images/phone_full.pdf}
%   \label{fig:test2}
% \end{minipage}
  \captionof{figure}{\textbf{A 2D CAD program in AIDL, generated using the prompt ``old-school telephone".} The LLM generates AIDL code in a hierarchical fashion, adding constraints using naturally named operators. AIDL's backend solver produces the final CAD shape rendered on the right.\maaz{rewrote the caption, please revise if necessary. syntax highlighting would help code readability}}
  \label{fig:teaser}
\end{figure}

\iffalse

% Outline

%CAD design is used to make everything manufactured
%CAD formats have advantage of precision and accuracy for manufacturability, and editability for design iteration
%These properties have their origins in the way CAD objects are constructed; procedurally.
%Generative AI is revolutionizing design by automating the production of details from high-level ideas as well as helping to generate concepts.
%Gen AI can make images and even 3D models, but the structures produced lack precision, accuracy, and editability.
%If we can generate procedures for generating CAD shapes, then the produced models will have these properties.

%While gen AI can produce syntactically correct CAD programs, and often structurally correct programs, they lack spatial understanding and reasoning necessary to construct complex objects.

%We propose to solve this by designing a domain-specific-language for CAD that is made to overcome the limitations of LLMs in spatial reasoning.
%It is a solver-aided, hierarchical language.
%The constraint solver allows the language model to specify geometric concepts, like spatial relationships, in natural language, and rely on the constraint-solving runtime to enforce them. The hierarchical structure allows for the natural decomposition of larger assemblies into managable peices, as well as for precise addressing of subparts of the program based on semantic, hierarchical naming.

% Generated from outline:

% Computer-aided design (CAD) stands as a cornerstone in the manufacturing world, celebrated for its precise and accurate output that is essential for manufacturing processes. At the heart of CAD's effectiveness lies its procedural approach to constructing objects, a method that ensures a high degree of precision, accuracy, and notably, editability. Editability in CAD allows for iterative design modifications, a critical aspect in engineering and manufacturing that fosters continual improvement and adaptation of designs.
% \adriana{is this what you actually what to discuss here? is editability the key? }

% Paragraph1: CAD is program CAD systems give you a DSL and the UI is an end-user programming interface 
Parametric Computer-Aided Design (CAD) systems revolutionized manufacturing-oriented design by introducing a paradigm where geometry is created through a sequence of constructive operations. This approach enhances accuracy and precision in modeling and offers flexibility in design editing. In its essence, CAD treats geometry as a \textit{program}, with CAD systems providing a Domain-Specific Language (DSL) for design and CAD GUIs providing end-user programming interfaces. 


As generative AI advances in program generation, we can imagine using these systems to create CAD objects, but this dream fall far short in reality because CAD programming requires geometric acumen that generative models do not have. In this work we argue that we need novel DSLs and end-user programming interfaces that are tailored for use by generative models.  %How can we reimagine the traditional CAD DSL principles, which are centered on end-user programming, to craft an innovative languages for design in an age where code is generated by AIs?


% Paragraph 3: we did an analysis and concluded X
To achieve this, we first analyzed the success and failures of various generative language models in CAD generation tasks. From this study we distilled areas of strength and weakness which we used to make these design principles; TODO add principles here

% Paragraph 4 - from analysis, here are principles

% Paragraph 5 - built an end-user programming tool for AI (feedback loop)

% Paragraph 6 - we validated it


% We want to do it with AI - what is the DSL, and what is the programming interface
% Now there are 2 end-users; the AI, and the person trying to direct the AI (for discussion / future work)
%
% - CAD is programming w/ DSL and end-user env.
% - AI is a new kind of end-user, than fails in particular ways due to geometric understanding, we need novel DSLs and interfaces to 
% 
%
% This kind of programming is uniquely hard for AIs due to its geometric
%
Advances in AI suggest we can now automatically generate designs with high complexity. 

The advent of generative artificial intelligence (AI) is reshaping the landscape of design, automating the creation of detailed elements from overarching ideas and aiding in the generation of innovative concepts. Despite these advancements, the outputs of generative AI, including images and 3D models, often fall short in embodying the quintessential CAD characteristics of precision, accuracy, and editability~\vk{Would be good to refer to an example here. We can have a small Figure on 2nd page, showing limitations of various baselines discussed in the intro}\maaz{I like this suggestion. A well chosen example can save us a lot of text}. Recognizing this gap, the focus shifts towards the potential of using generative AI to create procedural instructions for CAD models~\vk{our focus or community's? if latter, share some citations...}. Such a shift promises to imbue AI-generated designs with the inherent advantages of traditional CAD models.

However, a critical challenge emerges when employing generative AI in this context. While it can produce CAD programs that are syntactically correct, and often structurally sound, they lack the spatial understanding and reasoning needed to construct complex objects. \vk{Again, perhaps, show a few limitations of naive baselines? is it true that generating something with correct syntax is easy?}
\vk{I would expand / re-structure above a bit more: itemize main limitations of existing / naive solutions, try to refer to specific examples in results section or add a few most prominent failures here. I expect existing systems to struggle with: preserving relational constraints, understanding global objectives, producing interpretable code, with meaningful relations, etc...}\maaz{I would argue LLMs make syntactic mistakes quite often, altho that is tangential to the point of this paper and therefore we want to drive the conversation to the aspects of the problem that we are trying to address. Furthermore, if the point is "LLMs may be okay at getting syntax correct but they are especially bad at spatial reasoning" i think it would be valuable to provide some intuition as to WHY that is the case.}
To overcome this, the paper proposes the design of a specialized CAD language tailored for integration with large language models (LLMs). This language is distinctive in two key aspects: it is solver-aided and hierarchical.~\vk{Explain how these aspects are derived directly from the limitations listed above, it's not immediately... }\maaz{I would even split this into two short paragraphs, each leading the reader from one problem to its eventual solution}

The solver-aided aspect of the language allows for the expression of geometric concepts and spatial relationships in a natural language format. The constraint-solving runtime of this setup plays a pivotal role in ensuring these concepts are accurately enforced, addressing the spatial reasoning limitations of LLMs. The hierarchical structure of the language, on the other hand, enables efficient breakdown of larger assemblies into smaller, manageable segments. This structure also facilitates precise identification and modification of specific parts within a CAD program, leveraging a semantic, hierarchical naming system.

By developing this domain-specific language, the aim is to harmonically blend the generative power of AI with the stringent requirements of CAD modeling. Such an integration has the potential to not only enhance the efficiency and creativity of design processes but also to pave the way for more sophisticated and intricate manufactured products.\maaz{This feels weak and vague. Rather than say "harmonically blend", say something more explicit along the lines of...we leverage the strength of each component: LLMs "understanding" of how things relate to each other and the solver's ability to generate a model that enforces those constraints. }


Other motivation is enabling collaborative design and interactive workflows. Generating design alternatives that are then editable either by the person or by the AI. Gaining the ability to edit or update part of a model while having gaurantees that other parts of the design will remain fixed (not guaranteed by more general autoregressive techniques or latent-space based exploration, but maybe skexgen?).

%% Adriana's below

In this work we propose a novel method for automatically generating CAD models from high level textual specification. We focus on 2D models, and also show examples of compositions in 3D/laser cutting. Generating CAD models from high level specifications is hard


To address this challenge, we propose a solution inspired by representing CAD model as programs. The perspective of designs as programs, originating from the early days of computer graphics and procedural modeling, has been gaining new traction with recent advancements in the realm of program synthesis. In particular, we leverage this perspective to draw from two advances in the field: code generation through Large Language Models (LLMs), which has recently emerged as a powerful tool, and  \emph{solver-aided programming paradigm}, which has significantly advanced accessible programming over the past decade~\cite{torlak2013growing}. Our fundamental insight is to combine both of these approaches to develop innovative solutions in design automation. 

Solver-aided programming aspires to simplify and democratize programming by harnessing the capabilities of domain-specific languages (DSLs) and productivity tools based on constraint solvers, such as verification or synthesis. Solver-Aided DSLs (SDSLs) can be thought of as high-level languages wherein programmers articulate merely a \emph{partial} specification, with solvers seamlessly filling in the gaps. Such abstractions not only make code writing faster but also reduce errors since the solvers are used to ensure that contains are met instead of relying on the programmer. Inspired by this, we propose to use a SDSL to interface with AI-driven code generation. In the context of design, this abstraction enables the generative model to focus on high-level design decisions, while formal methods ensure that the design meets feasibility constraints. 

However, scalability is a problem. We circumvent these challenges by crafting hierarchical SDSL.

\fi