% \theoremstyle{definition}
% \newtheorem{definition}{Definition}[section]
% \newtheorem{problem}{Problem Statement}[section]

\newif\ifshowauthorcomments

\newcommand{\langname}[1]{AIDL}
\newcommand{\dgone}[1]{\textbf{\textit{dependencies}}}
\newcommand{\dgtwo}[1]{\textbf{\textit{constraints}}}
\newcommand{\dgthree}[1]{\textbf{\textit{semantics}}}
\newcommand{\dgfour}[1]{\textbf{\textit{hierarchy}}}

%\showauthorcommentstrue
\showauthorcommentsfalse
\newcommand{\authorcomment}[3]{\ifshowauthorcomments{\bfseries \scriptsize \color{#3} #1: #2}\fi}
\newcommand{\adriana}[1]{\authorcomment{AS}{#1}{red}}
\newcommand{\ben}[1]{\authorcomment{BJ}{#1}{magenta}}
\newcommand{\maaz}[1]{\authorcomment{MA}{#1}{olive}}
\newcommand{\vk}[1]{\authorcomment{VK}{#1}{cyan}}
\newcommand{\felix}[1]{\authorcomment{FH}{#1}{purple}}
\newcommand{\jz}[1]{{\color[rgb]{0.20,0.60,0.10}#1}}

%Multi-line comment to hide content
\newcommand{\mlcomment}[1]{}


\newif\ifseechanges
%\seechangestrue % uncomment to see changes

\ifseechanges
\newcommand{\revision}[2]{\textcolor{red}{\stkout{#1}}\textcolor{blue}{#2}}
\newcommand{\revisionj}[2]{\textcolor{red}{\stkout{#1}}\textcolor{blue}{#2}}
\else
\newcommand{\revision}[2]{#2}
\newcommand{\revisionj}[2]{#2}
\fi

% Claims and Evidence List
\newcommand{\claim}{\item \color{blue}}
\newcommand{\evidence}{\item \color{olive}}
\newcommand{\experiment}{\item \color{magenta}}
\newcommand{\exptodo}{\item \color{lime}}
\newcommand{\expskip}{\item \color{lightgray}}


% From: https://tex.stackexchange.com/questions/546259/create-chat-interface-in-latex-with-tikz

\usepackage{tikz}
\usepackage{xparse}

\newcounter{chatlinenum}

%% Adjust text width to suit
\tikzset{chatstyle/.style={text width=0.8*\linewidth,rounded corners=2pt}}

%% Adjust width of minipage to suit, but greater than TikZ text width
\NewDocumentEnvironment{chat}{}{%
   \setcounter{chatlinenum}{0}
   \begin{minipage}{\linewidth}
       \everypar={\chatline}
}{%
   \end{minipage}
}

\definecolor{mygreen}{HTML}{88EABB}

\definecolor{cbgreen}{HTML}{1b9e77}
\definecolor{cborange}{HTML}{d95f02}
\definecolor{cbpurple}{HTML}{7570b3}
\definecolor{cbpink}{HTML}{e7298a}
\definecolor{cblightgreen}{HTML}{66a61e}
\definecolor{cbyello}{HTML}{e6ab02}
\definecolor{cbbrown}{HTML}{a6761d}
\definecolor{cbgrey}{HTML}{666666}







%% Alter colors to suit
\def\chatline#1\par{%
   \stepcounter{chatlinenum}%
   \noindent
   \ifodd\thechatlinenum
       \tikz[]{\node[fill=lightgray,chatstyle]{\strut#1\strut};}%
   \else
       \hfill
       \tikz[]{\node[fill=mygreen,chatstyle,align=right]{\strut#1\unskip\strut};}%
   \fi
   \par
   \smallskip
}

%% |=====8><-----| %% New solution:

%% Alter colors to suit
\begingroup
    \lccode`~=`\^^M
    \lowercase{%
\endgroup
    \def\newchatline#1~{%
        \stepcounter{chatlinenum}%
        \ifodd\thechatlinenum
            \tikz[]{\node[fill=lightgray,chatstyle]{\strut#1\strut};}%
        \else
            \hfill
            \tikz[]{\node[fill=mygreen,chatstyle,align=right]{\strut#1\strut};}%
        \fi
        ~
        \smallskip
    }%
}

\NewDocumentEnvironment{newchat}{}{%
    \setcounter{chatlinenum}{0}
    \begin{minipage}{2.0in}
        \obeylines
        \everypar={\newchatline}
}{%
    \end{minipage}
}



% https://tex.stackexchange.com/questions/247681/how-to-create-checkbox-todo-list
\newlist{todolist}{itemize}{2}
\setlist[todolist]{label=$\square$}
\usepackage{pifont}
\newcommand{\cmark}{\ding{51}}%
\newcommand{\xmark}{\ding{55}}%
\newcommand{\done}{\rlap{$\square$}{\raisebox{2pt}{\large\hspace{1pt}\cmark}}%
\hspace{-2.5pt}}
\newcommand{\wontfix}{\rlap{$\square$}{\large\hspace{1pt}\xmark}}


% Load computed values from a file
%\DTLsetseparator{ = }
%\DTLloaddb[noheader, keys={key,value}]{metrics}{data/metrics.dat}
%\newcommand{\data}[1]{\DTLfetch{metrics}{key}{#1}{value}}

%% Command to place a TikZ anchor at the current position
%\newcommand{\mytikzmark}[2]{%
%  \tikz[overlay,remember picture,baseline] \coordinate (#2) at (0,0) {};}
%
%\newcommand{\highlight}[3]{%
%  \draw[#1,line width=14pt,opacity=0.5]%
%    ([yshift=4pt]#2) -- ([yshift=4pt]#3);%
%}
%\hfsetbordercolor{yellow}
%\hfsetfillcolor{yellow}


\lstnewenvironment{code}[1][]%
{\lstset{basicstyle=\ttfamily, breaklines=true, breakatwhitespace=false, #1}}{}

\newcommand{\iverb}[1]{\lstinline[basicstyle=\ttfamily, breaklines=true, breakatwhitespace=true]{#1}}


%\lstnewenvironment{code}[1][firstnumber=\themain,name=main]
%$  {\lstset{language=haskell,
%$           basicstyle=\small\ttfamily,
%$           numbers=left,
%$           numberstyle=\tiny\color{gray},
%$           backgroundcolor=\color{lightgray},
%$           #1
%$          }
%$}
%${\setcounter{main}{\value{lstnumber}}}