
\documentclass{article} % For LaTeX2e
\usepackage{iclr2025_conference,times}

% Optional math commands from https://github.com/goodfeli/dlbook_notation.
%%%%% NEW MATH DEFINITIONS %%%%%

\usepackage{amsmath,amsfonts,bm}
\usepackage{derivative}
% Mark sections of captions for referring to divisions of figures
\newcommand{\figleft}{{\em (Left)}}
\newcommand{\figcenter}{{\em (Center)}}
\newcommand{\figright}{{\em (Right)}}
\newcommand{\figtop}{{\em (Top)}}
\newcommand{\figbottom}{{\em (Bottom)}}
\newcommand{\captiona}{{\em (a)}}
\newcommand{\captionb}{{\em (b)}}
\newcommand{\captionc}{{\em (c)}}
\newcommand{\captiond}{{\em (d)}}

% Highlight a newly defined term
\newcommand{\newterm}[1]{{\bf #1}}

% Derivative d 
\newcommand{\deriv}{{\mathrm{d}}}

% Figure reference, lower-case.
\def\figref#1{figure~\ref{#1}}
% Figure reference, capital. For start of sentence
\def\Figref#1{Figure~\ref{#1}}
\def\twofigref#1#2{figures \ref{#1} and \ref{#2}}
\def\quadfigref#1#2#3#4{figures \ref{#1}, \ref{#2}, \ref{#3} and \ref{#4}}
% Section reference, lower-case.
\def\secref#1{section~\ref{#1}}
% Section reference, capital.
\def\Secref#1{Section~\ref{#1}}
% Reference to two sections.
\def\twosecrefs#1#2{sections \ref{#1} and \ref{#2}}
% Reference to three sections.
\def\secrefs#1#2#3{sections \ref{#1}, \ref{#2} and \ref{#3}}
% Reference to an equation, lower-case.
\def\eqref#1{equation~\ref{#1}}
% Reference to an equation, upper case
\def\Eqref#1{Equation~\ref{#1}}
% A raw reference to an equation---avoid using if possible
\def\plaineqref#1{\ref{#1}}
% Reference to a chapter, lower-case.
\def\chapref#1{chapter~\ref{#1}}
% Reference to an equation, upper case.
\def\Chapref#1{Chapter~\ref{#1}}
% Reference to a range of chapters
\def\rangechapref#1#2{chapters\ref{#1}--\ref{#2}}
% Reference to an algorithm, lower-case.
\def\algref#1{algorithm~\ref{#1}}
% Reference to an algorithm, upper case.
\def\Algref#1{Algorithm~\ref{#1}}
\def\twoalgref#1#2{algorithms \ref{#1} and \ref{#2}}
\def\Twoalgref#1#2{Algorithms \ref{#1} and \ref{#2}}
% Reference to a part, lower case
\def\partref#1{part~\ref{#1}}
% Reference to a part, upper case
\def\Partref#1{Part~\ref{#1}}
\def\twopartref#1#2{parts \ref{#1} and \ref{#2}}

\def\ceil#1{\lceil #1 \rceil}
\def\floor#1{\lfloor #1 \rfloor}
\def\1{\bm{1}}
\newcommand{\train}{\mathcal{D}}
\newcommand{\valid}{\mathcal{D_{\mathrm{valid}}}}
\newcommand{\test}{\mathcal{D_{\mathrm{test}}}}

\def\eps{{\epsilon}}


% Random variables
\def\reta{{\textnormal{$\eta$}}}
\def\ra{{\textnormal{a}}}
\def\rb{{\textnormal{b}}}
\def\rc{{\textnormal{c}}}
\def\rd{{\textnormal{d}}}
\def\re{{\textnormal{e}}}
\def\rf{{\textnormal{f}}}
\def\rg{{\textnormal{g}}}
\def\rh{{\textnormal{h}}}
\def\ri{{\textnormal{i}}}
\def\rj{{\textnormal{j}}}
\def\rk{{\textnormal{k}}}
\def\rl{{\textnormal{l}}}
% rm is already a command, just don't name any random variables m
\def\rn{{\textnormal{n}}}
\def\ro{{\textnormal{o}}}
\def\rp{{\textnormal{p}}}
\def\rq{{\textnormal{q}}}
\def\rr{{\textnormal{r}}}
\def\rs{{\textnormal{s}}}
\def\rt{{\textnormal{t}}}
\def\ru{{\textnormal{u}}}
\def\rv{{\textnormal{v}}}
\def\rw{{\textnormal{w}}}
\def\rx{{\textnormal{x}}}
\def\ry{{\textnormal{y}}}
\def\rz{{\textnormal{z}}}

% Random vectors
\def\rvepsilon{{\mathbf{\epsilon}}}
\def\rvphi{{\mathbf{\phi}}}
\def\rvtheta{{\mathbf{\theta}}}
\def\rva{{\mathbf{a}}}
\def\rvb{{\mathbf{b}}}
\def\rvc{{\mathbf{c}}}
\def\rvd{{\mathbf{d}}}
\def\rve{{\mathbf{e}}}
\def\rvf{{\mathbf{f}}}
\def\rvg{{\mathbf{g}}}
\def\rvh{{\mathbf{h}}}
\def\rvu{{\mathbf{i}}}
\def\rvj{{\mathbf{j}}}
\def\rvk{{\mathbf{k}}}
\def\rvl{{\mathbf{l}}}
\def\rvm{{\mathbf{m}}}
\def\rvn{{\mathbf{n}}}
\def\rvo{{\mathbf{o}}}
\def\rvp{{\mathbf{p}}}
\def\rvq{{\mathbf{q}}}
\def\rvr{{\mathbf{r}}}
\def\rvs{{\mathbf{s}}}
\def\rvt{{\mathbf{t}}}
\def\rvu{{\mathbf{u}}}
\def\rvv{{\mathbf{v}}}
\def\rvw{{\mathbf{w}}}
\def\rvx{{\mathbf{x}}}
\def\rvy{{\mathbf{y}}}
\def\rvz{{\mathbf{z}}}

% Elements of random vectors
\def\erva{{\textnormal{a}}}
\def\ervb{{\textnormal{b}}}
\def\ervc{{\textnormal{c}}}
\def\ervd{{\textnormal{d}}}
\def\erve{{\textnormal{e}}}
\def\ervf{{\textnormal{f}}}
\def\ervg{{\textnormal{g}}}
\def\ervh{{\textnormal{h}}}
\def\ervi{{\textnormal{i}}}
\def\ervj{{\textnormal{j}}}
\def\ervk{{\textnormal{k}}}
\def\ervl{{\textnormal{l}}}
\def\ervm{{\textnormal{m}}}
\def\ervn{{\textnormal{n}}}
\def\ervo{{\textnormal{o}}}
\def\ervp{{\textnormal{p}}}
\def\ervq{{\textnormal{q}}}
\def\ervr{{\textnormal{r}}}
\def\ervs{{\textnormal{s}}}
\def\ervt{{\textnormal{t}}}
\def\ervu{{\textnormal{u}}}
\def\ervv{{\textnormal{v}}}
\def\ervw{{\textnormal{w}}}
\def\ervx{{\textnormal{x}}}
\def\ervy{{\textnormal{y}}}
\def\ervz{{\textnormal{z}}}

% Random matrices
\def\rmA{{\mathbf{A}}}
\def\rmB{{\mathbf{B}}}
\def\rmC{{\mathbf{C}}}
\def\rmD{{\mathbf{D}}}
\def\rmE{{\mathbf{E}}}
\def\rmF{{\mathbf{F}}}
\def\rmG{{\mathbf{G}}}
\def\rmH{{\mathbf{H}}}
\def\rmI{{\mathbf{I}}}
\def\rmJ{{\mathbf{J}}}
\def\rmK{{\mathbf{K}}}
\def\rmL{{\mathbf{L}}}
\def\rmM{{\mathbf{M}}}
\def\rmN{{\mathbf{N}}}
\def\rmO{{\mathbf{O}}}
\def\rmP{{\mathbf{P}}}
\def\rmQ{{\mathbf{Q}}}
\def\rmR{{\mathbf{R}}}
\def\rmS{{\mathbf{S}}}
\def\rmT{{\mathbf{T}}}
\def\rmU{{\mathbf{U}}}
\def\rmV{{\mathbf{V}}}
\def\rmW{{\mathbf{W}}}
\def\rmX{{\mathbf{X}}}
\def\rmY{{\mathbf{Y}}}
\def\rmZ{{\mathbf{Z}}}

% Elements of random matrices
\def\ermA{{\textnormal{A}}}
\def\ermB{{\textnormal{B}}}
\def\ermC{{\textnormal{C}}}
\def\ermD{{\textnormal{D}}}
\def\ermE{{\textnormal{E}}}
\def\ermF{{\textnormal{F}}}
\def\ermG{{\textnormal{G}}}
\def\ermH{{\textnormal{H}}}
\def\ermI{{\textnormal{I}}}
\def\ermJ{{\textnormal{J}}}
\def\ermK{{\textnormal{K}}}
\def\ermL{{\textnormal{L}}}
\def\ermM{{\textnormal{M}}}
\def\ermN{{\textnormal{N}}}
\def\ermO{{\textnormal{O}}}
\def\ermP{{\textnormal{P}}}
\def\ermQ{{\textnormal{Q}}}
\def\ermR{{\textnormal{R}}}
\def\ermS{{\textnormal{S}}}
\def\ermT{{\textnormal{T}}}
\def\ermU{{\textnormal{U}}}
\def\ermV{{\textnormal{V}}}
\def\ermW{{\textnormal{W}}}
\def\ermX{{\textnormal{X}}}
\def\ermY{{\textnormal{Y}}}
\def\ermZ{{\textnormal{Z}}}

% Vectors
\def\vzero{{\bm{0}}}
\def\vone{{\bm{1}}}
\def\vmu{{\bm{\mu}}}
\def\vtheta{{\bm{\theta}}}
\def\vphi{{\bm{\phi}}}
\def\va{{\bm{a}}}
\def\vb{{\bm{b}}}
\def\vc{{\bm{c}}}
\def\vd{{\bm{d}}}
\def\ve{{\bm{e}}}
\def\vf{{\bm{f}}}
\def\vg{{\bm{g}}}
\def\vh{{\bm{h}}}
\def\vi{{\bm{i}}}
\def\vj{{\bm{j}}}
\def\vk{{\bm{k}}}
\def\vl{{\bm{l}}}
\def\vm{{\bm{m}}}
\def\vn{{\bm{n}}}
\def\vo{{\bm{o}}}
\def\vp{{\bm{p}}}
\def\vq{{\bm{q}}}
\def\vr{{\bm{r}}}
\def\vs{{\bm{s}}}
\def\vt{{\bm{t}}}
\def\vu{{\bm{u}}}
\def\vv{{\bm{v}}}
\def\vw{{\bm{w}}}
\def\vx{{\bm{x}}}
\def\vy{{\bm{y}}}
\def\vz{{\bm{z}}}

% Elements of vectors
\def\evalpha{{\alpha}}
\def\evbeta{{\beta}}
\def\evepsilon{{\epsilon}}
\def\evlambda{{\lambda}}
\def\evomega{{\omega}}
\def\evmu{{\mu}}
\def\evpsi{{\psi}}
\def\evsigma{{\sigma}}
\def\evtheta{{\theta}}
\def\eva{{a}}
\def\evb{{b}}
\def\evc{{c}}
\def\evd{{d}}
\def\eve{{e}}
\def\evf{{f}}
\def\evg{{g}}
\def\evh{{h}}
\def\evi{{i}}
\def\evj{{j}}
\def\evk{{k}}
\def\evl{{l}}
\def\evm{{m}}
\def\evn{{n}}
\def\evo{{o}}
\def\evp{{p}}
\def\evq{{q}}
\def\evr{{r}}
\def\evs{{s}}
\def\evt{{t}}
\def\evu{{u}}
\def\evv{{v}}
\def\evw{{w}}
\def\evx{{x}}
\def\evy{{y}}
\def\evz{{z}}

% Matrix
\def\mA{{\bm{A}}}
\def\mB{{\bm{B}}}
\def\mC{{\bm{C}}}
\def\mD{{\bm{D}}}
\def\mE{{\bm{E}}}
\def\mF{{\bm{F}}}
\def\mG{{\bm{G}}}
\def\mH{{\bm{H}}}
\def\mI{{\bm{I}}}
\def\mJ{{\bm{J}}}
\def\mK{{\bm{K}}}
\def\mL{{\bm{L}}}
\def\mM{{\bm{M}}}
\def\mN{{\bm{N}}}
\def\mO{{\bm{O}}}
\def\mP{{\bm{P}}}
\def\mQ{{\bm{Q}}}
\def\mR{{\bm{R}}}
\def\mS{{\bm{S}}}
\def\mT{{\bm{T}}}
\def\mU{{\bm{U}}}
\def\mV{{\bm{V}}}
\def\mW{{\bm{W}}}
\def\mX{{\bm{X}}}
\def\mY{{\bm{Y}}}
\def\mZ{{\bm{Z}}}
\def\mBeta{{\bm{\beta}}}
\def\mPhi{{\bm{\Phi}}}
\def\mLambda{{\bm{\Lambda}}}
\def\mSigma{{\bm{\Sigma}}}

% Tensor
\DeclareMathAlphabet{\mathsfit}{\encodingdefault}{\sfdefault}{m}{sl}
\SetMathAlphabet{\mathsfit}{bold}{\encodingdefault}{\sfdefault}{bx}{n}
\newcommand{\tens}[1]{\bm{\mathsfit{#1}}}
\def\tA{{\tens{A}}}
\def\tB{{\tens{B}}}
\def\tC{{\tens{C}}}
\def\tD{{\tens{D}}}
\def\tE{{\tens{E}}}
\def\tF{{\tens{F}}}
\def\tG{{\tens{G}}}
\def\tH{{\tens{H}}}
\def\tI{{\tens{I}}}
\def\tJ{{\tens{J}}}
\def\tK{{\tens{K}}}
\def\tL{{\tens{L}}}
\def\tM{{\tens{M}}}
\def\tN{{\tens{N}}}
\def\tO{{\tens{O}}}
\def\tP{{\tens{P}}}
\def\tQ{{\tens{Q}}}
\def\tR{{\tens{R}}}
\def\tS{{\tens{S}}}
\def\tT{{\tens{T}}}
\def\tU{{\tens{U}}}
\def\tV{{\tens{V}}}
\def\tW{{\tens{W}}}
\def\tX{{\tens{X}}}
\def\tY{{\tens{Y}}}
\def\tZ{{\tens{Z}}}


% Graph
\def\gA{{\mathcal{A}}}
\def\gB{{\mathcal{B}}}
\def\gC{{\mathcal{C}}}
\def\gD{{\mathcal{D}}}
\def\gE{{\mathcal{E}}}
\def\gF{{\mathcal{F}}}
\def\gG{{\mathcal{G}}}
\def\gH{{\mathcal{H}}}
\def\gI{{\mathcal{I}}}
\def\gJ{{\mathcal{J}}}
\def\gK{{\mathcal{K}}}
\def\gL{{\mathcal{L}}}
\def\gM{{\mathcal{M}}}
\def\gN{{\mathcal{N}}}
\def\gO{{\mathcal{O}}}
\def\gP{{\mathcal{P}}}
\def\gQ{{\mathcal{Q}}}
\def\gR{{\mathcal{R}}}
\def\gS{{\mathcal{S}}}
\def\gT{{\mathcal{T}}}
\def\gU{{\mathcal{U}}}
\def\gV{{\mathcal{V}}}
\def\gW{{\mathcal{W}}}
\def\gX{{\mathcal{X}}}
\def\gY{{\mathcal{Y}}}
\def\gZ{{\mathcal{Z}}}

% Sets
\def\sA{{\mathbb{A}}}
\def\sB{{\mathbb{B}}}
\def\sC{{\mathbb{C}}}
\def\sD{{\mathbb{D}}}
% Don't use a set called E, because this would be the same as our symbol
% for expectation.
\def\sF{{\mathbb{F}}}
\def\sG{{\mathbb{G}}}
\def\sH{{\mathbb{H}}}
\def\sI{{\mathbb{I}}}
\def\sJ{{\mathbb{J}}}
\def\sK{{\mathbb{K}}}
\def\sL{{\mathbb{L}}}
\def\sM{{\mathbb{M}}}
\def\sN{{\mathbb{N}}}
\def\sO{{\mathbb{O}}}
\def\sP{{\mathbb{P}}}
\def\sQ{{\mathbb{Q}}}
\def\sR{{\mathbb{R}}}
\def\sS{{\mathbb{S}}}
\def\sT{{\mathbb{T}}}
\def\sU{{\mathbb{U}}}
\def\sV{{\mathbb{V}}}
\def\sW{{\mathbb{W}}}
\def\sX{{\mathbb{X}}}
\def\sY{{\mathbb{Y}}}
\def\sZ{{\mathbb{Z}}}

% Entries of a matrix
\def\emLambda{{\Lambda}}
\def\emA{{A}}
\def\emB{{B}}
\def\emC{{C}}
\def\emD{{D}}
\def\emE{{E}}
\def\emF{{F}}
\def\emG{{G}}
\def\emH{{H}}
\def\emI{{I}}
\def\emJ{{J}}
\def\emK{{K}}
\def\emL{{L}}
\def\emM{{M}}
\def\emN{{N}}
\def\emO{{O}}
\def\emP{{P}}
\def\emQ{{Q}}
\def\emR{{R}}
\def\emS{{S}}
\def\emT{{T}}
\def\emU{{U}}
\def\emV{{V}}
\def\emW{{W}}
\def\emX{{X}}
\def\emY{{Y}}
\def\emZ{{Z}}
\def\emSigma{{\Sigma}}

% entries of a tensor
% Same font as tensor, without \bm wrapper
\newcommand{\etens}[1]{\mathsfit{#1}}
\def\etLambda{{\etens{\Lambda}}}
\def\etA{{\etens{A}}}
\def\etB{{\etens{B}}}
\def\etC{{\etens{C}}}
\def\etD{{\etens{D}}}
\def\etE{{\etens{E}}}
\def\etF{{\etens{F}}}
\def\etG{{\etens{G}}}
\def\etH{{\etens{H}}}
\def\etI{{\etens{I}}}
\def\etJ{{\etens{J}}}
\def\etK{{\etens{K}}}
\def\etL{{\etens{L}}}
\def\etM{{\etens{M}}}
\def\etN{{\etens{N}}}
\def\etO{{\etens{O}}}
\def\etP{{\etens{P}}}
\def\etQ{{\etens{Q}}}
\def\etR{{\etens{R}}}
\def\etS{{\etens{S}}}
\def\etT{{\etens{T}}}
\def\etU{{\etens{U}}}
\def\etV{{\etens{V}}}
\def\etW{{\etens{W}}}
\def\etX{{\etens{X}}}
\def\etY{{\etens{Y}}}
\def\etZ{{\etens{Z}}}

% The true underlying data generating distribution
\newcommand{\pdata}{p_{\rm{data}}}
\newcommand{\ptarget}{p_{\rm{target}}}
\newcommand{\pprior}{p_{\rm{prior}}}
\newcommand{\pbase}{p_{\rm{base}}}
\newcommand{\pref}{p_{\rm{ref}}}

% The empirical distribution defined by the training set
\newcommand{\ptrain}{\hat{p}_{\rm{data}}}
\newcommand{\Ptrain}{\hat{P}_{\rm{data}}}
% The model distribution
\newcommand{\pmodel}{p_{\rm{model}}}
\newcommand{\Pmodel}{P_{\rm{model}}}
\newcommand{\ptildemodel}{\tilde{p}_{\rm{model}}}
% Stochastic autoencoder distributions
\newcommand{\pencode}{p_{\rm{encoder}}}
\newcommand{\pdecode}{p_{\rm{decoder}}}
\newcommand{\precons}{p_{\rm{reconstruct}}}

\newcommand{\laplace}{\mathrm{Laplace}} % Laplace distribution

\newcommand{\E}{\mathbb{E}}
\newcommand{\Ls}{\mathcal{L}}
\newcommand{\R}{\mathbb{R}}
\newcommand{\emp}{\tilde{p}}
\newcommand{\lr}{\alpha}
\newcommand{\reg}{\lambda}
\newcommand{\rect}{\mathrm{rectifier}}
\newcommand{\softmax}{\mathrm{softmax}}
\newcommand{\sigmoid}{\sigma}
\newcommand{\softplus}{\zeta}
\newcommand{\KL}{D_{\mathrm{KL}}}
\newcommand{\Var}{\mathrm{Var}}
\newcommand{\standarderror}{\mathrm{SE}}
\newcommand{\Cov}{\mathrm{Cov}}
% Wolfram Mathworld says $L^2$ is for function spaces and $\ell^2$ is for vectors
% But then they seem to use $L^2$ for vectors throughout the site, and so does
% wikipedia.
\newcommand{\normlzero}{L^0}
\newcommand{\normlone}{L^1}
\newcommand{\normltwo}{L^2}
\newcommand{\normlp}{L^p}
\newcommand{\normmax}{L^\infty}

\newcommand{\parents}{Pa} % See usage in notation.tex. Chosen to match Daphne's book.

\DeclareMathOperator*{\argmax}{arg\,max}
\DeclareMathOperator*{\argmin}{arg\,min}

\DeclareMathOperator{\sign}{sign}
\DeclareMathOperator{\Tr}{Tr}
\let\ab\allowbreak

\usepackage{hyperref}
\usepackage{url}

% Custom imports
\usepackage{fullpage}
\usepackage{appendix}
\usepackage[sort, authoryear, round]{natbib}
\usepackage{fancyhdr}
\usepackage[format=plain,
            labelfont={it,it},
            textfont=it]{caption}
\usepackage[stable, bottom, flushmargin]{footmisc}
\usepackage{graphicx}
\usepackage{xcolor}    
\usepackage{breakcites}
\usepackage{amsmath,amssymb,amsthm}
\usepackage{nicefrac}

\definecolor{lightblue}{RGB}{0,102,204}
\usepackage[pagebackref, colorlinks=true, linkcolor=lightblue, citecolor=lightblue, urlcolor=lightblue]{hyperref}

\usepackage{pgfplots}
\pgfplotsset{compat = newest}
\usepackage{wrapfig}

\usepackage{enumerate}
\usepackage{xparse, expl3}
\usepackage{indentfirst}
\usepackage{float}
\usepackage{tikz}
\usepackage{tikzscale}
\usetikzlibrary{arrows.meta,arrows}
\usetikzlibrary{decorations.pathreplacing}
\usetikzlibrary{positioning,chains,fit,shapes,calc,shapes.misc}

\usepackage[capitalize]{cleveref}

\newtheorem{theorem}{Theorem}[section]
\newtheorem{proposition}[theorem]{Proposition}
\newtheorem{lemma}[theorem]{Lemma}
\newtheorem{corollary}[theorem]{Corollary}
\newtheorem{definition}[theorem]{Definition}
\newtheorem{example}[theorem]{Example}
\newtheorem{remark}[theorem]{Remark}
\newtheorem{algorithm}[theorem]{Algorithm}
\newtheorem{conjecture}[theorem]{Conjecture}
\newtheorem{problem}{Open Problem}

\crefname{theorem}{Theorem}{Theorems}
\crefname{proposition}{Proposition}{Propositions}
\crefname{lemma}{Lemma}{Lemmas}
\crefname{corollary}{Corollary}{Corollaries}
\crefname{definition}{Definition}{Definitions}
\crefname{example}{Example}{Examples}
\crefname{remark}{Remark}{Remarks}
\crefname{algorithm}{Algorithm}{Algorithms}
\crefname{equation}{Equation}{Equations}
\crefname{section}{Section}{Sections}
\crefname{subsection}{Section}{Sections}
\crefname{conjecture}{Conjecture}{Conjectures}

\usepackage{microtype}
\usepackage{hyperref}
\usepackage{graphicx}

\usepackage{bbm,bm}
\usepackage{xfrac}
\usepackage{algorithm}
\usepackage{algpseudocode}
\usepackage{float}


%
\setlength\unitlength{1mm}
\newcommand{\twodots}{\mathinner {\ldotp \ldotp}}
% bb font symbols
\newcommand{\Rho}{\mathrm{P}}
\newcommand{\Tau}{\mathrm{T}}

\newfont{\bbb}{msbm10 scaled 700}
\newcommand{\CCC}{\mbox{\bbb C}}

\newfont{\bb}{msbm10 scaled 1100}
\newcommand{\CC}{\mbox{\bb C}}
\newcommand{\PP}{\mbox{\bb P}}
\newcommand{\RR}{\mbox{\bb R}}
\newcommand{\QQ}{\mbox{\bb Q}}
\newcommand{\ZZ}{\mbox{\bb Z}}
\newcommand{\FF}{\mbox{\bb F}}
\newcommand{\GG}{\mbox{\bb G}}
\newcommand{\EE}{\mbox{\bb E}}
\newcommand{\NN}{\mbox{\bb N}}
\newcommand{\KK}{\mbox{\bb K}}
\newcommand{\HH}{\mbox{\bb H}}
\newcommand{\SSS}{\mbox{\bb S}}
\newcommand{\UU}{\mbox{\bb U}}
\newcommand{\VV}{\mbox{\bb V}}


\newcommand{\yy}{\mathbbm{y}}
\newcommand{\xx}{\mathbbm{x}}
\newcommand{\zz}{\mathbbm{z}}
\newcommand{\sss}{\mathbbm{s}}
\newcommand{\rr}{\mathbbm{r}}
\newcommand{\pp}{\mathbbm{p}}
\newcommand{\qq}{\mathbbm{q}}
\newcommand{\ww}{\mathbbm{w}}
\newcommand{\hh}{\mathbbm{h}}
\newcommand{\vvv}{\mathbbm{v}}

% Vectors

\newcommand{\av}{{\bf a}}
\newcommand{\bv}{{\bf b}}
\newcommand{\cv}{{\bf c}}
\newcommand{\dv}{{\bf d}}
\newcommand{\ev}{{\bf e}}
\newcommand{\fv}{{\bf f}}
\newcommand{\gv}{{\bf g}}
\newcommand{\hv}{{\bf h}}
\newcommand{\iv}{{\bf i}}
\newcommand{\jv}{{\bf j}}
\newcommand{\kv}{{\bf k}}
\newcommand{\lv}{{\bf l}}
\newcommand{\mv}{{\bf m}}
\newcommand{\nv}{{\bf n}}
\newcommand{\ov}{{\bf o}}
\newcommand{\pv}{{\bf p}}
\newcommand{\qv}{{\bf q}}
\newcommand{\rv}{{\bf r}}
\newcommand{\sv}{{\bf s}}
\newcommand{\tv}{{\bf t}}
\newcommand{\uv}{{\bf u}}
\newcommand{\wv}{{\bf w}}
\newcommand{\vv}{{\bf v}}
\newcommand{\xv}{{\bf x}}
\newcommand{\yv}{{\bf y}}
\newcommand{\zv}{{\bf z}}
\newcommand{\zerov}{{\bf 0}}
\newcommand{\onev}{{\bf 1}}

% Matrices

\newcommand{\Am}{{\bf A}}
\newcommand{\Bm}{{\bf B}}
\newcommand{\Cm}{{\bf C}}
\newcommand{\Dm}{{\bf D}}
\newcommand{\Em}{{\bf E}}
\newcommand{\Fm}{{\bf F}}
\newcommand{\Gm}{{\bf G}}
\newcommand{\Hm}{{\bf H}}
\newcommand{\Id}{{\bf I}}
\newcommand{\Jm}{{\bf J}}
\newcommand{\Km}{{\bf K}}
\newcommand{\Lm}{{\bf L}}
\newcommand{\Mm}{{\bf M}}
\newcommand{\Nm}{{\bf N}}
\newcommand{\Om}{{\bf O}}
\newcommand{\Pm}{{\bf P}}
\newcommand{\Qm}{{\bf Q}}
\newcommand{\Rm}{{\bf R}}
\newcommand{\Sm}{{\bf S}}
\newcommand{\Tm}{{\bf T}}
\newcommand{\Um}{{\bf U}}
\newcommand{\Wm}{{\bf W}}
\newcommand{\Vm}{{\bf V}}
\newcommand{\Xm}{{\bf X}}
\newcommand{\Ym}{{\bf Y}}
\newcommand{\Zm}{{\bf Z}}

% Calligraphic

\newcommand{\Ac}{{\cal A}}
\newcommand{\Bc}{{\cal B}}
\newcommand{\Cc}{{\cal C}}
\newcommand{\Dc}{{\cal D}}
\newcommand{\Ec}{{\cal E}}
\newcommand{\Fc}{{\cal F}}
\newcommand{\Gc}{{\cal G}}
\newcommand{\Hc}{{\cal H}}
\newcommand{\Ic}{{\cal I}}
\newcommand{\Jc}{{\cal J}}
\newcommand{\Kc}{{\cal K}}
\newcommand{\Lc}{{\cal L}}
\newcommand{\Mc}{{\cal M}}
\newcommand{\Nc}{{\cal N}}
\newcommand{\nc}{{\cal n}}
\newcommand{\Oc}{{\cal O}}
\newcommand{\Pc}{{\cal P}}
\newcommand{\Qc}{{\cal Q}}
\newcommand{\Rc}{{\cal R}}
\newcommand{\Sc}{{\cal S}}
\newcommand{\Tc}{{\cal T}}
\newcommand{\Uc}{{\cal U}}
\newcommand{\Wc}{{\cal W}}
\newcommand{\Vc}{{\cal V}}
\newcommand{\Xc}{{\cal X}}
\newcommand{\Yc}{{\cal Y}}
\newcommand{\Zc}{{\cal Z}}

% Bold greek letters

\newcommand{\alphav}{\hbox{\boldmath$\alpha$}}
\newcommand{\betav}{\hbox{\boldmath$\beta$}}
\newcommand{\gammav}{\hbox{\boldmath$\gamma$}}
\newcommand{\deltav}{\hbox{\boldmath$\delta$}}
\newcommand{\etav}{\hbox{\boldmath$\eta$}}
\newcommand{\lambdav}{\hbox{\boldmath$\lambda$}}
\newcommand{\epsilonv}{\hbox{\boldmath$\epsilon$}}
\newcommand{\nuv}{\hbox{\boldmath$\nu$}}
\newcommand{\muv}{\hbox{\boldmath$\mu$}}
\newcommand{\zetav}{\hbox{\boldmath$\zeta$}}
\newcommand{\phiv}{\hbox{\boldmath$\phi$}}
\newcommand{\psiv}{\hbox{\boldmath$\psi$}}
\newcommand{\thetav}{\hbox{\boldmath$\theta$}}
\newcommand{\tauv}{\hbox{\boldmath$\tau$}}
\newcommand{\omegav}{\hbox{\boldmath$\omega$}}
\newcommand{\xiv}{\hbox{\boldmath$\xi$}}
\newcommand{\sigmav}{\hbox{\boldmath$\sigma$}}
\newcommand{\piv}{\hbox{\boldmath$\pi$}}
\newcommand{\rhov}{\hbox{\boldmath$\rho$}}
\newcommand{\upsilonv}{\hbox{\boldmath$\upsilon$}}

\newcommand{\Gammam}{\hbox{\boldmath$\Gamma$}}
\newcommand{\Lambdam}{\hbox{\boldmath$\Lambda$}}
\newcommand{\Deltam}{\hbox{\boldmath$\Delta$}}
\newcommand{\Sigmam}{\hbox{\boldmath$\Sigma$}}
\newcommand{\Phim}{\hbox{\boldmath$\Phi$}}
\newcommand{\Pim}{\hbox{\boldmath$\Pi$}}
\newcommand{\Psim}{\hbox{\boldmath$\Psi$}}
\newcommand{\Thetam}{\hbox{\boldmath$\Theta$}}
\newcommand{\Omegam}{\hbox{\boldmath$\Omega$}}
\newcommand{\Xim}{\hbox{\boldmath$\Xi$}}


% Sans Serif small case

\newcommand{\Gsf}{{\sf G}}

\newcommand{\asf}{{\sf a}}
\newcommand{\bsf}{{\sf b}}
\newcommand{\csf}{{\sf c}}
\newcommand{\dsf}{{\sf d}}
\newcommand{\esf}{{\sf e}}
\newcommand{\fsf}{{\sf f}}
\newcommand{\gsf}{{\sf g}}
\newcommand{\hsf}{{\sf h}}
\newcommand{\isf}{{\sf i}}
\newcommand{\jsf}{{\sf j}}
\newcommand{\ksf}{{\sf k}}
\newcommand{\lsf}{{\sf l}}
\newcommand{\msf}{{\sf m}}
\newcommand{\nsf}{{\sf n}}
\newcommand{\osf}{{\sf o}}
\newcommand{\psf}{{\sf p}}
\newcommand{\qsf}{{\sf q}}
\newcommand{\rsf}{{\sf r}}
\newcommand{\ssf}{{\sf s}}
\newcommand{\tsf}{{\sf t}}
\newcommand{\usf}{{\sf u}}
\newcommand{\wsf}{{\sf w}}
\newcommand{\vsf}{{\sf v}}
\newcommand{\xsf}{{\sf x}}
\newcommand{\ysf}{{\sf y}}
\newcommand{\zsf}{{\sf z}}


% mixed symbols

\newcommand{\sinc}{{\hbox{sinc}}}
\newcommand{\diag}{{\hbox{diag}}}
\renewcommand{\det}{{\hbox{det}}}
\newcommand{\trace}{{\hbox{tr}}}
\newcommand{\sign}{{\hbox{sign}}}
\renewcommand{\arg}{{\hbox{arg}}}
\newcommand{\var}{{\hbox{var}}}
\newcommand{\cov}{{\hbox{cov}}}
\newcommand{\Ei}{{\rm E}_{\rm i}}
\renewcommand{\Re}{{\rm Re}}
\renewcommand{\Im}{{\rm Im}}
\newcommand{\eqdef}{\stackrel{\Delta}{=}}
\newcommand{\defines}{{\,\,\stackrel{\scriptscriptstyle \bigtriangleup}{=}\,\,}}
\newcommand{\<}{\left\langle}
\renewcommand{\>}{\right\rangle}
\newcommand{\herm}{{\sf H}}
\newcommand{\trasp}{{\sf T}}
\newcommand{\transp}{{\sf T}}
\renewcommand{\vec}{{\rm vec}}
\newcommand{\Psf}{{\sf P}}
\newcommand{\SINR}{{\sf SINR}}
\newcommand{\SNR}{{\sf SNR}}
\newcommand{\MMSE}{{\sf MMSE}}
\newcommand{\REF}{{\RED [REF]}}

% Markov chain
\usepackage{stmaryrd} % for \mkv 
\newcommand{\mkv}{-\!\!\!\!\minuso\!\!\!\!-}

% Colors

\newcommand{\RED}{\color[rgb]{1.00,0.10,0.10}}
\newcommand{\BLUE}{\color[rgb]{0,0,0.90}}
\newcommand{\GREEN}{\color[rgb]{0,0.80,0.20}}

%%%%%%%%%%%%%%%%%%%%%%%%%%%%%%%%%%%%%%%%%%
\usepackage{hyperref}
\hypersetup{
    bookmarks=true,         % show bookmarks bar?
    unicode=false,          % non-Latin characters in AcrobatÕs bookmarks
    pdftoolbar=true,        % show AcrobatÕs toolbar?
    pdfmenubar=true,        % show AcrobatÕs menu?
    pdffitwindow=false,     % window fit to page when opened
    pdfstartview={FitH},    % fits the width of the page to the window
%    pdftitle={My title},    % title
%    pdfauthor={Author},     % author
%    pdfsubject={Subject},   % subject of the document
%    pdfcreator={Creator},   % creator of the document
%    pdfproducer={Producer}, % producer of the document
%    pdfkeywords={keyword1} {key2} {key3}, % list of keywords
    pdfnewwindow=true,      % links in new window
    colorlinks=true,       % false: boxed links; true: colored links
    linkcolor=red,          % color of internal links (change box color with linkbordercolor)
    citecolor=green,        % color of links to bibliography
    filecolor=blue,      % color of file links
    urlcolor=blue           % color of external links
}
%%%%%%%%%%%%%%%%%%%%%%%%%%%%%%%%%%%%%%%%%%%


%% Requires \usepackage{xcolor}
%% Requires \usepackage{lstlistings}
\definecolor{codegreen}{rgb}{0,0.6,0}
\definecolor{codeblue}{rgb}{.11,.56,1}
\definecolor{codegray}{rgb}{0.5,0.5,0.5}
\definecolor{codepurple}{rgb}{0.58,0,0.82}

\definecolor{codeKeyword}{RGB}{211	54	130}
% \definecolor{codeKeyword}{RGB}{133, 153, 0}
\definecolor{codeComment}{RGB}{42	161	152}
\definecolor{codeOmitted}{RGB}{108	113	196}
\definecolor{codeNumbers}{rgb}{0.5,0.5,0.5}
\definecolor{codeString}{RGB}{128, 161, 16}

\definecolor{textusercolor}{RGB}{40 20 10}
\definecolor{textgptcolor}{RGB}{62, 65, 115}

\definecolor{codebackcolour}{RGB}{	253	246	227}
\definecolor{backuserprompt}{RGB}{
253, 250, 250}
%\definecolor{backgptresponse}{rgb}{0.93,0.93,0.98}

\definecolor{backgptresponse}{RGB}{226 228 255}

\newcommand{\gpticon}{images/chatgpt-logo.png}
\newcommand{\usericon}{images/person-raising-hand.png}



%% Defining custom languages -- try not to put styling here, only keywords/comment signifiers/etc. this way all languages will look the same. 
\lstdefinelanguage{JavaScript}{
  keywords={typeof, new, true, false, catch, function, return, null, catch, switch, var, if, in, while, do, else, case, break, const},
  ndkeywords={class, export, boolean, throw, implements, import, this, require},
  sensitive=false,
  comment=[l]{//},
  morecomment=[s]{/*}{*/},
  morestring=[b]',
  morestring=[b]"
}

%% consistent styling used for all languages
\lstdefinestyle{codestyle}{
    commentstyle=\color{codeComment},
    keywordstyle=\color{codeKeyword},
    numberstyle=\tiny\color{codeNumbers},
    stringstyle=\color{codeString},
    basicstyle=\linespread{0.85}\footnotesize,
    columns=flexible,
    breakatwhitespace=false,         
    breaklines=true,                 
    captionpos=b,                    
    % numbers=left,                    
    % numbersep=5pt,     
    showspaces=false,
    showstringspaces=false,
    showtabs=false,
    tabsize=2,
    escapeinside={\$}{\$},
}

\surroundwithmdframed[
  hidealllines=true,
  backgroundcolor=codebackcolour,
  innerleftmargin=0pt,
  innertopmargin=0pt,
  innerbottommargin=0pt]{gptcodeblock}


\newcommand\colboxcolor{codeComment} % temporary
%% colored box -- takes background color as parameter
\newsavebox{\savedcolorbox}
\newenvironment{colbox}[2]
  {\renewcommand\colboxcolor{#1}%
   \begin{lrbox}{\savedcolorbox}%
    \begin{minipage}{\dimexpr\columnwidth-2\fboxsep\relax}

   \footnotesize
   \bgroup\color{#2}
   }
  {\egroup\end{minipage}\end{lrbox}%
   \begin{center}
   \colorbox{\colboxcolor}{\usebox{\savedcolorbox}}
   \end{center}
}



%% text responses
\newsavebox{\savedfigurebox}
\newenvironment{blurbwithfig}[5]
{
    \newcommand{\figurewidth}{#1}
    \newcommand{\iconwidth}{0.025\linewidth}
    \newcommand{\blurbwidth}{0.89\linewidth}
    \newcommand{\imagetoshow}{#2}
    \newcommand{\backgroundcolor}{#3}
    \newcommand{\boxtextcolor}{#4}
    \newcommand{\icontoshow}{#5}

    % collect the to-be-right-aligned figure block
    \begin{lrbox}{\savedfigurebox}%
    \begin{minipage}[t]{\figurewidth}
        \vspace{3pt}
        \ifthenelse{\equal{\imagetoshow}{}}{}{\includegraphics[width=\linewidth]{\imagetoshow}}
        % \captionof{figure}{note}
        % \label{fig:figure2}
    \end{minipage}\end{lrbox}%

    % start the user icon
    \noindent
    \begin{minipage}[t]{\iconwidth}
    \vspace{2pt}
    \centering
    \includegraphics[width=\linewidth]{\icontoshow}
    \end{minipage}
    %
    % start the left-aligned chat block
    \noindent
    \begin{minipage}[t]{\blurbwidth}
    \vspace{0pt}
    \begin{colbox}{\backgroundcolor}{\boxtextcolor}
}
{
    \end{colbox}
    \end{minipage}
    % end the primary prompt area
    \hfill
    % begin the image portion
    \usebox{\savedfigurebox}
}


\newenvironment{userprompt}[2]
{
    \begin{blurbwithfig}{#1}{#2}{backuserprompt}{textusercolor}{\usericon}
}
{
    \end{blurbwithfig}
}


\newenvironment{gptresponse}[2]
{
    \begin{blurbwithfig}{#1}{#2}{backgptresponse}{textgptcolor}{\gpticon}
}
{

    \end{blurbwithfig}
}


\lstnewenvironment{gptcodeblock}[1]
{
    \lstset{style=codestyle} %% change this to be a local setting, so it doesn't affect others
    \lstset{language=#1}
}
{}



\newenvironment{chatenv}[1]
{
    \newcommand{\preventbreaks}{#1}
    \begin{center}
    \mdfsetup{nobreak=\preventbreaks}
    \begin{mdframed}[
        linecolor=black,
        innerleftmargin=0.04cm,
        innerrightmargin=0cm
        innertopmargin=0cm
        innerbottommargin=0cm
    ]{}

}
{ 
    \end{mdframed}
    \end{center}
}


%% ============================
%% Macros to be used inside the chat environment, to help shorten/clarify the included chats
%% ============================

\newcommand{\authorremark}[1]{\footnotesize\textit{(Author remark: #1)}}

% parameters: #1 is the summary of the omitted content (will show up in doc if provided; can be left blank), #2 can be placed around the content to omit, so you can leave it in the tex, it just doesn't get rendered anywhere
\newcommand{\omitted}[2]{
    % \textcolor{codeblue}{
    \ifthenelse{\equal{#1}{}}{\textit{(... content omitted by authors ...)}} 
                            {\textit{(... omitted by authors: #1 ...)}}
}

% parameters: #1 is the summary of the omitted code (will show up in doc if provided; can be left blank), #2 can be placed around the content to omit, so you can leave it in the tex, it just doesn't get rendered anywhere
%% IMPORTANT: to use this inside a \gptcodeblock, you must wrap it in $$ to escape into latex mode. For example: 

% for x in range(0,3):
% $\omittedCode{update and output i}{
%.     i = x
%      print("the value of i is " + str(i))
% }$

\newcommand{\omittedCode}[2]{
    \textcolor{codeOmitted}{
    \ifthenelse{\equal{#1}{}}{\textit{(... code omitted by authors ...)}} 
                            {\textit{(... omitted by authors: #1 ...)}}
    }
}



\title{A Solver-Aided Hierarchical Language for LLM-Driven CAD Design}

\newcommand{\PH}{\text{[PH]}} % Placeholder

% Authors must not appear in the submitted version. They should be hidden
% as long as the \iclrfinalcopy macro remains commented out below.
% Non-anonymous submissions will be rejected without review.

\author{Benjamin T. Jones, Felix Hähnlein, Zihan Zhang, Maaz Ahmad, Vladimir Kim \& Adriana Schulz \\
MIT CSAIL \\
\texttt{bt\_jones@mit.edu} \\
Department of Computer Science\\
University of Washington\\
Seattle, WA 98195, USA \\
\texttt{\{fhahnlei, jzhang18, adriana\}@cs.washington.edu} \\
Adobe \\
Seattle, USA \\
\texttt{\{vokim, mahmad\}@adobe.com}
}

% The \author macro works with any number of authors. There are two commands
% used to separate the names and addresses of multiple authors: \And and \AND.
%
% Using \And between authors leaves it to \LaTeX{} to determine where to break
% the lines. Using \AND forces a linebreak at that point. So, if \LaTeX{}
% puts 3 of 4 authors names on the first line, and the last on the second
% line, try using \AND instead of \And before the third author name.

\newcommand{\fix}{\marginpar{FIX}}
\newcommand{\new}{\marginpar{NEW}}

\iclrfinalcopy % Uncomment for camera-ready version, but NOT for submission.

\begin{document}


\maketitle

\begin{abstract}
Large language models (LLMs) have been enormously successful in solving a wide variety
of structured and unstructured generative tasks,
but they struggle to generate procedural geometry in Computer Aided Design (CAD). These difficulties
arise from an inability to do spatial reasoning and the necessity to guide a model through complex,
long range planning to generate complex geometry. We enable generative CAD Design with LLMs
through the introduction of a solver-aided, hierarchical domain specific language (DSL) called AIDL, which offloads the spatial reasoning requirements to a geometric constraint solver. Additionally, we show that in the few-shot regime, AIDL outperforms even a language with in-training data (OpenSCAD), both in terms of generating visual results closer to the prompt and creating objects that are easier to post-process and reason about.
\end{abstract}

% \begin{figure}[h]
% \begin{center}
% %\framebox[4.0in]{$\;$}
%  \includegraphics[width=\textwidth]{images/teaser_template.pdf}
% \end{center}
% \caption{We generate CAD models from a text prompt in a new DSL that expresses CAD models as hierarchical constraint systems. Our system interactively prompts the LLM to build up a CAD program in our DSL, using feedback from executing and solving the intermediate programs to guide the generation process. The constraint system is solved by a novel hierarchical constraint solver, and we fabricate the resulting designs using a laser cutter.}
%   \label{fig:teaser}
% \end{figure}



\section{Introduction}

In sensor networks, maintaining data freshness is crucial to support diverse applications such as environmental monitoring, industrial automation, and smart cities \cite{kandris2020applications}. A critical metric for quantifying data freshness is the Age of Information (AoI), which measures the time elapsed since the last received update was generated \cite{yates2012}. Minimizing AoI is essential in dynamic environments, where obsolete information can result in inaccurate decisions or missed opportunities. Efficient AoI management involves balancing update frequency, data relevance, and network resource constraints to ensure decision-makers have timely and accurate information when required \cite{yates2021age}. The significance of AoI has led to extensive research on its optimization across various domains, including single-server systems with one or multiple sources \cite{modiano2015,mm1,sun2016,najm2018,soysal2019,9137714,yates2019,zou2023costly}, scheduling strategies \cite{modiano-sch-1,9007478,sch-igor-1,9241401,sch-li,sch-sun}, and analysis of resource-constrained systems \cite{const-ulukus,const-biyikoglu,const-arafa,const-farazi,const-parisa}. 

%\ali{A good transition here would be: one particular area that has been garnering focus by the AoI researchers and that is correalted systems. In fact, sensor networks often handle...}

Among the strategies for AoI minimization, packet preemption is regarded as a cornerstone approach for ensuring the timeliness of information in communication networks, especially when resources such as service rates are limited \cite{yates2021age}. By prioritizing critical updates, preemption ensures that the most valuable data reaches its destination promptly, as demonstrated in the context of single-sensor, memoryless systems \cite{kaul2012status,inoue2019general}. Beyond this specific scenario, numerous studies have extensively investigated its role in optimizing AoI across diverse settings. For example, \cite{maatouk2019age} analyzes systems with prioritized information streams sharing a common server, where lower-priority packets may be buffered or discarded. Similarly, \cite{wang2019preempt} and \cite{kavitha2021controlling} examine preemption strategies for rate-limited links and lossy systems, identifying in the process the optimal policies for minimizing the AoI.

On the other hand, one particular area that has been garnering focus among AoI researchers is correlated systems. In fact, sensor networks often handle correlated data streams, where relationships between data collected by different sensors can be leveraged to enhance decision-making, reduce redundancy, and improve overall system performance \cite{mahmood2015reliability,yetgin2017survey}. This correlation often arises when multiple sensors monitor overlapping areas or related phenomena, allowing them to collaboratively exchange information and optimize resource usage. The role of correlation in sensor networks has further been explored in studies focusing on its potential to optimize system efficiency and effectiveness \cite{he2018,tong2022,popovski2019,modiano2022,ramakanth2023monitoring,erbayat2024}.











% The importance of AoI and correlation in sensor networks has motivated extensive research into optimizing AoI within correlated sensor systems. For example, \cite{he2018} studied sensor networks with overlapping fields of view, proposing a joint optimization framework for fog node assignment and transmission scheduling to reduce the AoI of multi-view image data. Similarly, \cite{tong2022} focused on overlapping camera networks, introducing scheduling algorithms for multi-channel systems designed to minimize AoI. Other works, such as \cite{popovski2019, modiano2022}, leveraged probabilistic correlation models to formulate sensor scheduling strategies aimed at lowering AoI. Additionally, \cite{ramakanth2023monitoring} treated the correlation of status updates as a discrete-time Wiener process, developing a scheduling policy that balances AoI reduction with monitoring accuracy. Furthermore, \cite{erbayat2024} analyzed the impact of optimal correlation probabilities under varying environmental conditions, addressing the interplay between error minimization and AoI.

%\ali{On the other hand, Preemption in AoI systems has been widely studied...Also, Id say reduce the size of this paragraph} 



%\ali{I don't like this transition here. Talk about correlated systems in the previous paragraph and how AoI is of interest. Then, switch here to preemption is still open question. Do not focus on your paper as you did here}
As part of ongoing efforts in this area, the potential of leveraging interdependencies between sensors to reduce the AoI in correlated systems has been studied, but the benefits and challenges of employing preemption in multi-sensor systems with correlated data streams remain an open question. While preemption is a potential strategy to minimize AoI in a network, it is not always the optimal strategy \cite{yates2019}. This approach must account for the specific features of the packets being transmitted since preempting leads to prioritization. For example, a sensor with a lower arrival rate may track a unique process that no other sensor monitors, making its packets particularly valuable and critical to retain. On the other hand, preempting a packet from a sensor with a high arrival rate may not significantly reduce AoI, as the frequent updates from such sensors diminish the impact of losing a single packet.


%\ali{Here you make the connection between preemption and multi-sensor correlated systems}

%\ali{Its good to emphasize that we have correlation here so it is different than typical AoI system}.

To address this gap, this paper introduces adaptable and probabilistic preemption mechanisms that dynamically balance priorities across sensors, considering their unique correlation characteristics and resource demands. To that end, the main contributions of this paper are summarized as follows:

%To address these challenges, we propose a system where the ability of a packet to preempt an ongoing transmission depends on its source, allowing for a more adaptable approach to managing updates. We also introduce the concept of probabilistic preemption, where preemption decisions are guided by source-specific probabilities rather than fixed or deterministic rules. This probabilistic method improves efficiency by giving higher-priority updates a better chance to preempt, keeping the information more up-to-date. By incorporating stochastic hybrid system modeling, we derive a closed-form expression for the AoI, providing a theoretical foundation to analyze the impact of probabilistic preemption on network performance. Building on this system, we explore how varying preemption probabilities can influence the total AoI in multi-sensor systems, considering the interplay between diverse sensors and their shared resources. Furthermore, we establish that the problem of deciding optimal preemption strategies can be framed as a sum of linear ratios problem. We derive an upper bound on the number of iterations required using a branch-and-bound algorithm, ensuring computational efficiency in identifying optimal solutions. Through this analysis, we identify optimal preemption strategies that minimize the total AoI, balancing the timeliness and relevance of updates across all monitored processes to achieve an efficient and well-coordinated system.

%Interestingly, the results show how the system adjusts priorities between sensors to keep the AoI as low as possible. For example, if one sensor spreads its updates more evenly across multiple processes, the system tends to rely on it more, even if another sensor is sending updates less often. As arrival rates or service rates change, the system shifts its strategy to stay efficient.\footnote{Due to size limitations, we present the proof details in \url{https://github.com/erbayat/xxxx}}.


\begin{itemize}
    \item As a first step, we propose a system where the ability of a packet to preempt an ongoing transmission probabilistically depends on its source rather than being fixed or following deterministic rules. Subsequently, using stochastic hybrid system modeling, we derive a closed-form expression for AoI to analyze the impact of probabilistic preemption on network performance.
    
    %enabling a more adaptable approach to manage updates by giving higher-priority updates a better chance to preempt, ensuring information remains up-to-date.

    \item Following that, we investigate optimizing the total AoI in multi-sensor systems, considering the interplay between diverse sensors and shared resources. Building on this, we frame the problem of deciding optimal preemption strategies as a sum of linear ratios problem, which is generally an NP-Hard problem\cite{freund2001solving}. However, by analyzing its unique characteristics, we derive an upper bound on the number of iterations required to identify optimal preemption strategies using a branch-and-bound algorithm, thus ensuring computational efficiency in finding the optimal solution.
    %\ali{You are using a lot the , ensuring... it sounds very chatgpt liky, try to minimize those when possible. Also, talk about the bounds and the impact of these results on getting an efficient solution}
    \item Lastly, we validate our findings with numerical results and evaluate optimal preemption strategies to minimize AoI. Our findings demonstrate how correlation influences preemption strategies. Notably, when a source provides a lower aggregate number of updates while distributing them more evenly, the system prioritizes it for preemption, even if another sensor updates less frequently.\ifthenelse{\boolean{withappendix}}
{}
{\footnote{Due to space limitations, we present the proof details in \cite{technicalNote}.}}
 %\ali{Dont forget to put the right link}
\end{itemize}


%These results not only support the theory but also offer practical ideas for real-world use, such as in IoT networks, factories, or autonomous systems, where staying up-to-date is very important.

%The remainder of this paper is structured as follows. Section \ref{system-model} introduces the system model and key assumptions. In Section \ref{aoi-S}, we derive the closed-form expression for the AoI within the proposed system. Section \ref{aoi-opt} outlines the optimization problem and details the process of determining the optimal preemption probabilities. The numerical results are presented in Section \ref{numerical}, and the paper concludes with a summary and discussion in Section \ref{conc}.



\section{Related Work}

\subsection{CAD Generation}
The compilation of large CAD datasets in recent years~\citep{koch_abc_2019,willis_fusion_2021,jones_automate_2021,willis_joinable_2022} has inspired a wealth of research on synthesizing CAD models. These efforts fall into two broad categories; those which generate CAD geometry directly~\citep{willis_engineering_2021,guo_complexgen_2022,jayaraman_solidgen_2023,nash_polygen_2020,xu2024brepgen,liu2024point2cad}, and those which generate a \emph{procedure} that generates CAD geometry~\citep{wu_deepcad_2021,ellis_learning_2017, ellis_learning_2018, ganin_computer-aided_2021,ren_extrudenet_2022,li_secad-net_2023,xu_skexgen_2022,lambourne_reconstructing_2022,para_sketchgen_2021, seff_vitruvion_2022,willis_fusion_2021,ma2024draw,li2024sfmcad,khan2024cad}. A fundamental challenge with these tools is the ability to control the generation. While many methods can be conditioned on an input allowing for reverse engineering applications~\citep{lambourne_reconstructing_2022,guo_complexgen_2022}, the few methods that directly focus on generation give limited control over their output \citep{jayaraman_solidgen_2023, wu2021deepcad,xu2024brepgen,seff_vitruvion_2022}. The highest degree of control is afforded by those that take sketches as input, such as Free2CAD~\citep{li_free2cad_2022} but these are effectively reverse reverse engineering an existing geometric design rather than enabling high level guidance. The goal of AIDL is to enable control without direct geometric supervision, and to incorporate semantic understanding beyond that of existing CAD programs. We have thus chosen to design our system around \textit{general purpose} language models rather than CAD specific models, and focus on DSL design rather than the design or training of a generative model. Importantly, all prior works use CAD DSLs that have limitations when it comes to LLM needs, as we discuss in Section~\ref{sec:analysis_llm}.


%There have been several attempts in recent years to generate CAD models. Some attempt to directly generate symbolic geometry\cite{willis_engineering_2021,guo_complexgen_2022,jayaraman_solidgen_2023,nash_polygen_2020,xu2024brepgen,liu2024point2cad}, while others use procedural representations~\cite{wu_deepcad_2021,ellis_learning_2017, ellis_learning_2018, ganin_computer-aided_2021,ren_extrudenet_2022,li_secad-net_2023,xu_skexgen_2022,lambourne_reconstructing_2022,para_sketchgen_2021, seff_vitruvion_2022,willis_fusion_2021,ma2024draw,li2024sfmcad,khan2024cad}. Some procedural approaches employ symbolic techniques either alone or in concert with learned heuristics \cite{nandi_functional_2018,xu_inferring_2021,du_inversecsg_2018}. A fundamental challenge with these tools is the ability to control the generation. While many methods can be conditioned on an input allowing for reverse engineering applications~\cite{lambourne_reconstructing_2022,guo_complexgen_2022}, the few methods that directly focus on generation give limited control over their output \cite{jayaraman_solidgen_2023, wu2021deepcad,xu2024brepgen,seff_vitruvion_2022}. The highest degree of control is afforded by those that take sketches as input, such as Free2CAD~\cite{li_free2cad_2022} but these are effectively reverse reverse engineering an existing geometric design rather than enabling high level guidance. 

%While many generative and reconstructive models treat the procedural form of CAD models as linear output and rely on large-data statistics to take care of syntactic and semantic concerns, some works employ symbolic techniques directly, either alone or in concert with learned heuristics. ReIncarnate~\cite{nandi_functional_2018}, Zone Graphs~\cite{xu_inferring_2021}, and InverseCSG~\cite{du_inversecsg_2018} treat reconstruction as a synthesis problem given inferred initial geometry, but use different synthesis techniques; oracle guided heuristics, neurally guided search, and constraint guided synthesis. Program synthesis can also be used to improve the robustness of existing CAD programs.  The goal of AIDL is to enable control without direct geometric supervision, which is why we propose to use an LLM to create CAD models through code, and therefore focus on DSL design rather than the design or training of a generative model. Importantly, all prior work uses CAD DSLs that are have limitations when it comes to LLM needs as will be discussed and surveyed in details in Section~\ref{sec:analysis_llm}. With AIDL, we want to incorporate semantic understanding beyond that of existing CAD programs, and therefore design our system around \textit{general purpose} language models rather than CAD specific models.

%\cite{mathur_constraint_2021} synthesizes symbolic constraints on input parameters that exclude configurations which break the model. Some of these constraints are discoverable through static program analysis, but their work highlights the necessity of access to a CAD kernel to understand the semantics of a CAD program, necessitating \emph{dynamic} analysis as well. \cite{mathur_interactive_2020} improves robustness by replacing specific references with programs synthesized with a small program size heuristic, using the hypothesis that smaller programs will generalize better, thus capturing user intent.



\subsection{Code Generation with LLMs}

Software engineering has been one of the marquee applications of LLMs, so a detailed enumeration of works in the field is beyond the scope of this paper. We instead refer the reader to a survey \citet{zhang2024unifying}, and reserve this section to position AIDL within the space. The majority of research on using LLMs for coding focus on how to make LLMs work more effectively with existing programming languages. A popular approach is to specifically train or fine-tune a model on code repositories and coding specific tasks \citep{li_starcoder_2023,lozhkov_starcoder_2024,grattafiori2023code}, or more recently to use LLMs to generate higher complexity training examples \citep{xu_wizardlm_2023,luo_wizardcoder_2023}. Other approaches tackle prompt complexity through system design, exploring prompt engineering and multi-agent strategies for pre-planning or coordinating a divide-and-conquer strategy \citep{dong_self-collaboration_2023,bairi_codeplan_2023,silver_generalized_2023}. AIDL approaches LLM code generation from an entirely different perspective, by asking which \emph{language features} will best enable an LLM to work with a programming system. Most similar is BOSQUE, a proposed general purpose programming language \citep{marron_towards_2023}. In particular, BOSQUE's embrace of pre and post conditions mirrors AIDL's use of constraints and strong validation, but does not go so far as to employ a solver to enforce constraints.

% Most work on coding LLMs focuses on how to get an LLM to work with existing code and coding systems
% This is done by training models specifically for coding
% fine-tuning existing models for coding tasks
% or creating prompting / querying strategies to improve the performance of LLMs with code
% AIDL asks a different question; what can we do to a coding language to make it work better with an LLM? Most similar to this is the proposed BOSQUE language...

% Code Gen w/ LLMs
% - train models specifically on / for code
% - fine-tune existing models for code
% - prompting / generation strategies for code
% - design languages for use with LLM

%Code generation has been one of the headline applications of the recent LLM explosion. There are several coding-specific LLMs that have been specifically trained or fine-tuned on code repositories and coding specific tasks \cite{li_starcoder_2023,lozhkov_starcoder_2024,grattafiori2023code}. Generating solutions to complex prompts is difficult. Some works explore prompt engineering and multi-agent strategies for pre-planning or coordinating a divide-and-conquer strategy for complexity \cite{dong_self-collaboration_2023,bairi_codeplan_2023,silver_generalized_2023}, while more recently models have been fine-tuned on synthetic, complex examples generated by LLMs \cite{xu_wizardlm_2023,luo_wizardcoder_2023}. LLMs can perform worse generating code in DSL than general purpose languages since they will have fewer or no examples of these in their training data. For context-free languages, grammar prompting \cite{wang_grammar_2023} can constrain output to valid DSL expressions. This approach is not feasible for more complex languages like AIDL, however there are prompting strategies that can increase the likelihood of valid program structures \cite{jain_generating_2023}. Another recent work proposes an LLM specific general purpose language, BOSQUE \cite{marron_towards_2023}, that shares a similar the philosophy of AIDL to design the language around an LLM's strengths and weaknesses. In particular, BOSQUE's embrace of pre and post conditions mirrors AIDL's use of constraints and strong validation, but does not go so far as to employ a solver to enforce constraints.

%\paragraph{Code Generation}
%There is extensive body of work focusing on \em{program synthesis}, the task of generating programs that satisfy user-provided specifications. A number of approaches have been developed to solve the synthesis problem, including enumerative, constraint-based and stochastic search algorithms. However, traditional synthesis algorithms require \em{formal} specifications while our goal is to generate programs from natural language specifications.

\subsection{CAD DSLs}
\label{sec:background}
% We review existing CAD DSLs and give a high-level explanation of how they work.
While there are many CAD DSLs, they can be grouped intro three broad categories: %While these categories are not black and white, they emphasize the main features of their respective languages.
%More specifically, industry standard CAD systems tackle real-world problems with different DSLs for different stages of the design process.
%Whereas the modeling stage often uses a CSG or query-based DSL, the assembly stage uses a constraint-based DSL.

% \felix{Maybe put a figure with the same geometry, designed in all three different languages?} \maaz{This would be nice if we can afford the space but I don't think its necessary.}

\paragraph{Constructive Solid Geometry (CSG)}
% The first step of each CAD design is to create geometric primitives.
In CSG, users can specify 2D and 3D parametric primitives, such as rectangles or spheres, directly in global coordinates.
Using boolean operations, such as union or intersection, users then combine these primitives in a hierarchical tree structure to achieve complex designs.
%Examples of languages for this category are OpenSCAD, and Szalinski \cite{nandi2020synthesizing}.
While some CSG languages, such as OpenSCAD, allow the use of variables or expressions for primitive parameters, they do not support specifying relationships or dependencies between different parts of the geometry.
This absence of dependencies simplifies the abstraction, making CSG widely used in inverse design and reconstruction tasks \citep{du_inversecsg_2018, nandi2020synthesizing, yu2022capri, michel2021dag}. 
However, this limitation also makes modeling more challenging, which is why CSG is not commonly used in most commercial CAD tools.

% CSG operations reference the entire geometry directly and unambiguously, which means that primitives can be safely edited even after the fact.
% Geometric relationships between geometric elements are not part of the language.% itself and are often handled by external solvers in a design tool, e.g. Illustrator's alignment feature for SVG objects. \adriana{I'm not sure I agree that SVG is a programming language and is CSG-like.}
% Due to the binary nature of boolean operations, designs in CSG languages result in a hierarchical tree structure, which is analogous to an abstract syntax tree (AST).

%  % \adriana{would be it useful to talk about how this simplicity allows for applications like reverse engineering cite chandra's work, inverseCSG, capri-net. but is not usually what is most used for modeling because of the needs to specify global coordinates with all primitives.}\felix{It's worth talking about this point.}\maaz{I think briefly qualifying the strengths and weaknesses of CSG DSLs can be nice for less familiar readers but I would not spend more than a few lines.}

\paragraph{Query-based CAD}
Most commercial CAD tools use query-based languages, such as FeatureScript \citep{featurescript}, which employ a sequence of operators to create and modify models (e.g., extrude, fillet, chamfer). 
These operators reference intermediate geometry---e.g., a chamfer operator takes a reference to an edge. 
This referencing creates implicit dependencies, simplifying modeling and enabling easy editing as operations propagate when intermediate geometry is updated. 
However, a challenge arises when edits lead to topological changes, making reference resolution ambiguous. 
For example, if an edge gets split or disappears, where should the chamfer be applied? To address this, these languages do not reference geometry explicitly. 
Instead, geometric references are specified \emph{implicitly} via a language construct called \emph{queries}. 
These queries are resolved during runtime by a solver~\citep{cadquery, featurescript}, which typically uses heuristics to resolve ambiguities. 
This makes automating design challenging, and generative tools that use CAD operators restrict themselves to sequences where references are not needed, such as sketch and extrude~\citep{wu2021deepcad, willis_fusion_2021, lambourne_reconstructing_2022}. 
While recent work allows for the unambiguous direct specification of references~\citep{cascaval2023lineage}, mastering this language is complex and demands significant expertise.

 

\paragraph{Constraint-based CAD}
As the name implies, constraint-based CAD DSLs natively enable users to create geometric constraints between geometric primitives. This frees designers from specifying parameters consistently, allowing for freeform design while ensuring that relationships between parts are preserved. This approach is used in content creation languages like Shape-Assembly \citep{jones2020shapeassembly}, GeoCode \citep{pearl2022geocode}, and SketchGen \citep{para2021sketchgen}.
 In typical commercial CAD tools, constraint-based abstractions are used in sketches---2D drawings that get extruded to form 3D geometry---and during assembly modeling, but not during solid modeling which uses queries. 
These languages do not provide operations to modify primitives or to create intermediate geometry and therefore they reference geometry directly.
Designs specified in these languages are non-hierarchical, all constraints are being solved simultaneously.



\section{AIDL - A Language for AI Design}
\label{sec:dsl}




In this section, we present \langname{}, a CAD DSL for LLM-generated designs.



\subsection{LLM Analysis and Design Goals}
\label{sec:analysis_llm}
We review the strengths and weaknesses of LLMs and formulate design goals that our DSL should support.


\paragraph{Direct vs. indirect computation}
% In their seminal paper, \cite{bubeck2023sparks} observe that GPT-4 is unable to solve mathematical equations directly, but that it can make correctly use of external math libraries. 
% Similarly, \cite{makatura2023can} show that with a simple CSG DSL, geometric primitives often intersect each other and parts can be disconnected, but using a query-based CAD DSL, GPT-4 more successfully constructs connected objects.
% These results suggest that LLMs perform better in tandem with external solvers.
% For CAD, we want to give the LLM the means to express design intent not through direct computation, but by specifying geometric relationshiops.
% \cite{bubeck2023sparks} note that GPT-4 struggles with solving equations directly but performs well when using external math libraries. Similarly, \cite{makatura2023can} show that a simple CSG DSL often leads to intersecting and disconnected parts, whereas a query-based CAD DSL allows GPT-4 to create more connected objects. 
Findings by \cite{bubeck2023sparks} and \cite{makatura2023can} suggest LLMs perform better with external solvers. For CAD, we aim to enable LLMs to express design intent by specifying geometric relationships instead of performing direct computation. In modern CAD tools, geometric relationships can be defined using implicit dependencies or explicit constraints, each with trade-offs. Geometric dependencies create implicit constraints that are easy to enforce, but long chains of references are challenging to reason over \citep{makatura2023can}. 
Users typically avoid this issue by generating references automatically through CAD state interaction rather than writing CAD code directly. Explicit constraints, like those in CAD sketches or assemblies are easier to reason about, but harder to solve. It is also challenging to add just the right number of constraints so that the system is neither 
often under-or over-constrained. 
To achieve the best of both worlds, we aim to support both \emph{implicit constraints through geometric dependencies (\dgone{})} and \emph{specification of geometric relationships via constraints (\dgtwo{})}. 


\paragraph{Named variables and semantic cues}
LLMs are designed to manipulate words, i.e., terms with semantic meaning.
In their experiments, \cite{makatura2023can} reparametrize CSG programs with and without informing the LLM about the modeled object.
Their results suggest that better reparametrizations are obtained by providing additional semantic knowledge.
% In computational design DSLs, users can employ semantic variables or named constraints, but they are not required to.
% For example, programmers using OpenSCAD can directly specify dimensions and automatically generated FeatureScript programs in Onshape are tedious to parse. 
% In particular, we note that queries can quickly increase in \emph{semantic complexity} making them harder to use for LLMs.
Our CAD DSL should use \emph{intuitively named terms (\dgthree{})} for design operations, references and constraints.
Our language should also expose geometric entities easily, without many semantic indirections.

\paragraph{Design complexity and modularity}
\cite{bubeck2023sparks} observe that GPT-4 can generate ``syntactically invalid or semantically incorrect code, especially for longer or more complex programs." Similarly, \cite{makatura2023can} note that complex designs may miss components or have them incorrectly placed. To address this, our CAD DSL should treat \emph{hierarchical design that supports modularity (\dgfour{})} as a first-class construct, enabling the breakdown of complex problems into manageable units. This hierarchy should facilitate planning and iteration in code generation.


%Real-world designs can be arbitrarily complex and the industry has developed several layers of abstraction for designs, such as sketches, parts, multi-parts and assemblies.
%Since in our case designs are primarily represented as programs, LLMs capabilities to scale to complex designs should correlate with their capability to scale to complex code.

%\cite{makatura2023can} experiment with complex designs, such as a cabinet with multiple shelves and a tricycle design with multiple components.
%In particular, we want to enable the evaluation of only partially complete designs. |
% One of the proposed strategies for dealing with increased complexity is to introduce "subassemblies".
% However, a complex design can still be challenging to generate if the entire task is tackled all at once.
% In both programming \cite{bubeck_sparks_2023}, and in design \cite{makatura2023can}, up-front planning and iteration lead to better results.
%\cite{makatura2023can} often obtain more complex results after rounds of iteration during which the LLM tries to correct its previous proposal.
%Similarly, \cite{bubeck2023sparks} experiment with arithmetic problems which GPT-4 does not manage to solve when asked directly for a solution.
%However, when asked to plan the solution approach and to explain it, it arrives at the correct result.

\jz{
\begin{table}[!ht]
\centering
\caption{We review how well the three major CAD DSL groups align with our design goals.
Neither of the existing paradigms complies with all of the desiderata.\maaz{I like captions and figures that are self sufficient. Consider: "None of the three major CAD DSL (csg, constraing based and query based dsls) groups achieve our formulated design goals. A brief summary of why}}
\label{tab:cad_dsls}
\renewcommand{\arraystretch}{1.2}
\footnotesize % Reduce font size
\resizebox{0.8\columnwidth}{!}{%
\begin{tabular}{|c|c|c|c|c|}
    \hline
    \textbf{Language} & \dgone{} & \dgtwo{} & \dgthree{} & \dgfour{} \\
    \hline
    CSG & - & - & \cmark & \cmark \\
    Constraint-based & - & \cmark & \cmark & - \\
    Query-based & \cmark & - & - & - \\
    \langname{} (Ours) & \cmark & \cmark & \cmark & \cmark \\
    \hline
\end{tabular}
}
\end{table}

}

None of the existing CAD DSLs fully support all of these design goals, as shown in Table \ref{tab:cad_dsls}. CSG DSLs are inherently hierarchical and can have intuitively named operations, but they do not support constraints, either implicitly through references or explicitly. Query-based DSLs allow implicit constraints via dependencies, but since references must be solved for though queries, they cannot be named directly, reducing semantic clarity. This also impacts modularity, as queries create chains of dependencies between distant parts of the program. Constraint-based CAD DSLs use intuitively named constraints, such as ``coincident" or ``symmetric," but they do not generate dependencies and lack hierarchy, as constraint solving is performed globally over a flat design.


% Notably, non of the existing CAD DSLs support all of these designs goals, as  shown in Table \ref{tab:cad_dsls}. CSG is hierarchical by nature and operations can be intuitively named.
% However, CSG DSLs do not enable users to specify constraints, neither implicitly via references nor explicitly via constraints. Query-based DSLs allow implicit constraints through dependencies, but because references need to be solved for, they cannot be named directly, compromising on semantics. This also compromises modularity,  since queries create a chain of dependencies between distant parts of the program. Constraint-based CAD DSLs feature constraints between geometric entities, which are often intuitively named---e.g., "coincident" or "symmetric", but they do not create dependencies (\dgone{}). 
% They further lack hierarchy, since solving constraints is performed in a global pass over a flat design. 



\subsection{Key Challenges and DSL Design Decisions}

% The goal of \langname{} is to fulfill the four design goals from Sec.~\ref{sec:analysis}. 
Combining all of the goals above in a single CAD DSL requires addressing two key challenges.

% The first is creating dependencies on previously constructed geometry (\dgone{}) without increasing the semantic complexity of operators (\dgthree{}). 
% As explained in Sec.~\ref{sec:background}, 
% previously constructed  geometry cannot be persistently named because parametric valiaations often lead to topological changes, so DSLs that reference   e previously constructed  geometry use queries, which are essentially algorithms in their own write that retrieve the geometry at a given state -- e.g. an edge where we will apply a fillet.  This means that goemtris cannot be semantically named increatind semantic complexity. 

The first challenge is creating dependencies on previously constructed geometry (\dgone{}) without increasing the semantic complexity of operators (\dgthree{}). 
As explained in Sec.~\ref{sec:background}, previously constructed geometry cannot be persistently named because parametric variations often lead to topological changes. DSLs that reference previously constructed geometry use queries—algorithms that retrieve the geometry at a given state. However, this solution prevents assigning persistent semantic names to geometric entities, increasing semantic complexity and, our analysis shows that LLMs struggle to reason about queries with long chains, motivating our choice to disable them by design.



Our solution to enable dependencies without queries arises from the observation that all geometric primitives in CAD are created either through constructive operations that instantiate primitives or through boolean operations (e.g., when two edges intersect, a new vertex is generated). While this is evident for CSG DSLs we note that query-based CAD DSLs are not more expressive than CSG DSLs since all CAD operators (e.g. chamfering) can be expressed as a combination of a constructive and a boolean operation \cite{cascaval2023lineage}.
Reference challenges emerge from boolean operations, as changes in parameters can lead to a varying number of generated primitives.

While we still want the geometric expressivity enabled by boolean operations, we want to reference geometry without queries. % since our analysis, see Sec.~\ref{sec:analysis_llm}, shows that LLMs are struggling to reason about queries with long chains, motivating our choice to disable them by design.
To overcome this problem, we decide to restrict our DSL to only use references for geometry created before boolean operations. 
In our DSL, boolean operations are applied to \emph{structures}, which is an intermediate type to create tree-structured hierarchies, see Fig.~\ref{fig:language-grammar}.
The result of these booleans cannot be referenced, just as with CSG DSLs, however, we can reference \emph{constructed} geometry and structures themselves. 
Although this introduces a language limitation, it does not affect 1) geometric expressivity, since in the worst case, you can have one geometry per structure, achieving the same expressiveness as CSG, and 2) dependency expressivity, as AIDL allows for arbitrary parametric expressions, meaning that in the worst case, dependencies can still be expressed manually, albeit with more effort. %\adriana{and because we also allow for parametric expressions to be explicitly written down, it also does not impact variability expressivess - say that in an understandable way}
%Further, our analysis, see Sec.~\ref{sec:analysis_llm}, shows that LLMs are struggling to reason about queries with long chains, motivating our choice to disable them by design. 
%Importantly, this gives us the same geometric expressivity as typical CAD DSLs, while being able to reference most geometry and without resorting to queries.
%Note that query-based CAD DSLs are not more expressive than CSG DSLs since all CAD operators can be expressed as a combination of a constructive and a boolean operation \cite{cascaval2023lineage}.

%In other words, we opt for a trade-off which allows us to reference unaltered geometry and to consider boolean operations as post-construction step.
%This allows us to benefit from the geometric expressive power of boolean operations without resorting to queries. 

%In practice, we do not even feature explicit boolean operator in our language, but we define types of geometry as either additive or subtractive, thus disabling by design the referencing of newly created geometry.

Second, using constraints (\dgtwo{}) to specify the relationship between elements within hierarchical designs (\dgfour{}) is computationally challenging.
Hierarchical designs encourage growing complexity and an increasing number of constraints, driving down solver performance. 
Query-based languages deal with this complexity by solving constraints in intermediate, \emph{flat} designs, e.g constraints between sketch elements in a CAD sketch are first solved before the user can extrude the sketch.
Solving constraints from all CAD operations simultaneously is computationally too expensive for these systems. 
To tackle this challenge, we introduce (1) two types of constraints, one between geometry and one between \emph{structures}, and (2) a custom recursive solver to hierarchically solve constraints in a design.
This strategy allows us to explicitly define the hierarchy of constraints and to practically solve it, without providing intermediate feedback to the LLM.

% In summary, \langname{} is a hybrid DSL which allows users to reference \emph{constructed} geometry \textbf{(DG1)}, to benefit from boolean operations, to apply constraints between geometry and hierarchical structures \textbf{(DG2)}, to explicitly define hierarchical designs \textbf{(DG4)} and to use constraints and references with minimal semantic effort \textbf{(DG3)}.

%Central to the design of AIDL are three key considerations. First, we propose a \textit{solver-aided approach} that aligns with best practices in using Large Language Models (LLMs). This approach enables LLMs to concentrate on high-level reasoning while offloading computational tasks to external solvers. Second, we advocate for a \textit{hierarchical design approach}, which allows for encapsulating reasoning within different model parts. This facilitates developing and validating each section independently, providing recursive feedback to the LLM. Finally, we aim to create s\textit{emantically meaningful constructs} that leverage the LLM's proficiency in understanding and manipulating language.


% Next, we showcase \langname{} by example and show how the different language constructs fulfill our design goals.
% We start with a simple design to show how our DSL invokes external solvers.
% Then, a more complex design showcases how to construct hierarchical models in AIDL.
% The full language grammar for AIDL is shown in Fig.~\ref{fig:language-grammar}.

%Design Decisions
%Syntax Description
%Interpretter / Compiler (use keywords like type-checking)
% - show example with clapper with delayed type-checking before vs. after
%Hierarchical Solver - we'd rather have rel positions of parts than everything be free
%
%Syntax; introduce constructs in an order that does not depend on each other
% \begin{figure*}[!ht]
% %\begin{multicols}{2}  % This will split the inside of the figure into two columns
% \begin{tabular}{m{0.5\linewidth}|m{0.5\linewidth}}
% \centering
% %\lstset{moredelim=[is][\color{red}]{[*}{*]}, 
% %basicstyle=\ttfamily, numbers=left,numberstyle=\small, stepnumber=1, firstnumber=1, numberfirstline=true,  breaklines=true, postbreak=\mbox{\textcolor{red}{$\hookrightarrow$}\space}, language=Python, escapechar=$,
% \lstset{
% moredelim=[is][\bfseries\color{cbbrown}]{[b}{b]}, 
% moredelim=[is][\bfseries\color{cbpurple}]{[pu}{pu]}, 
% moredelim=[is][\bfseries\color{cbpurple}]{[g}{g]}, 
% moredelim=[is][\bfseries\color{cbpink}]{[pi}{pi]}, 
% %basicstyle=\ttfamily, 
% numbers=left,
% numberstyle=\footnotesize, 
% stepnumber=1, 
% firstnumber=1, 
% numberfirstline=true,  
% breaklines=true, 
% postbreak=\mbox{\textcolor{red}{$\hookrightarrow$}\space}, 
% %language=Python, 
% escapechar=$,
% basicstyle=\scriptsize}
% \begin{lstlisting}[escapechar=$]
% # 1) Create hierarchy
% phone = Solid() 
% phone.handset = Solid()
% phone.base = Solid()
% base_solid = Solid()
% base_hole = Hole()
% phone.dial = Solid()
% # 2) Specify geometry of structures
% phone.handset = ... # program from Fig.$\ref{fig:phone_handset}$
% base_solid = ... #
% base_hole = ... #
% phone.dial = ... #
% # 3) Compositional constraints between structures
% phone.AddConstraint(Centered(
%     phone.base, phone.dial))
% phone.AddConstraint(Above(phone.handset, phone.base, 3*mm))
% # more constraints ...
% # 4) Boolean composition
% phone.base.base_solid = base_solid
% phone.base.base_hole = base_hole
% # 5) Solve system
% solved = phone.Solve()
% \end{lstlisting} &

% \centering
% \includegraphics[height=7cm]{images/phone_full.pdf}
% %\begin{tikzpicture}[remember picture, overlay]
% %$    \highlight{cbbrown}{hl1Start}{hl1End}
% %\end{tikzpicture}   
% %\end{multicols}
% \end{tabular}
% \caption{Design of a phone assembly. (Left) \langname{} code for hierarchical design. (Right) Hierarchical tree structure of design. Solids are marked in \textbf{\color{cbpurple}{Purple}} and the Hole is marked in \textbf{\color{cborange}{Orange}}.}
% \label{fig:phone_full}
% \end{figure*}
\subsection{\langname{} by example}
\label{sec:references_constraints}

\begin{figure*}[!ht]
\centering
\includegraphics[width=\linewidth]{images/handset_lineno.png}
%\begin{multicols}{1}  % This will split the inside of the figure into two columns
% \begin{tabular}{m{1\linewidth}}
% \begin{tabular}{c|c}
% %\centering
% \includegraphics[width=0.4\linewidth]{images/phone_before_constraints.pdf} &
% %\centering
% \includegraphics[width=0.4\linewidth]{images/phone_after_constraints.pdf} 
% \end{tabular}

% \hrulefill

% \lstset{
% moredelim=[is][\bfseries\color{cbbrown}]{[b}{b]}, 
% moredelim=[is][\bfseries\color{cbpurple}]{[pu}{pu]}, 
% moredelim=[is][\bfseries\color{cbpurple}]{[g}{g]}, 
% moredelim=[is][\bfseries\color{cbpink}]{[pi}{pi]}, 
% % basicstyle=\tiny\ttfamily, 
% numbers=left,
% numberstyle=\scriptsize, 
% stepnumber=1, 
% firstnumber=1, 
% numberfirstline=true,  
% breaklines=true, 
% postbreak=\mbox{\textcolor{red}{$\hookrightarrow$}\space}, 
% %language=Python, 
% % escapechar=$,
% basicstyle=\scriptsize}
% \begin{lstlisting}
% # 1) Create structure and geometric primitives
% handset = Solid() 
% handset.base = Rectangle( Point(0, 0), 5, 1)
% handset.left_round = Arc(Point(-3, 0), handset.base.top_left, Point(-4, -1.5))
% handset.left_line = Line(handset.left_round.end, Point(-3, -1.5))
% handset.right_round = Arc(Point(3, 0), Point(4, -1.5), Point(2.5, 1))
% handset.right_line = Line(Point(2, -1), Point(4, -1))
% ... # more primitives
% # 2) Add constraints
% handset.AddConstraint(Coincident(handset.right_round.end, handset.base.top_right))
% handset.AddConstraint(Horizontal(handset.left_line))
% handset.AddConstraint(Equation(handset.left_line.length == handset.right_line.length))
% handset.AddConstraint(HorizontallySymmetric(handset.left_round.center, handset.right_round.center))
% handset.AddConstraint(Equal(handset.left_fillet, handset.right_fillet))
% handset.AddConstraint(Diameter(handset.left_fillet, 1.5))
% ... # more constraints
% # 3) Solve system
% solved = handset.Solve()
% \end{lstlisting} 
% \end{tabular} 

%\begin{tikzpicture}[remember picture, overlay]
%$    \highlight{cbbrown}{hl1Start}{hl1End}
%\end{tikzpicture}   
%\end{multicols}
\caption{AIDL allows LLMs to express constraints using semantically meaningful operators. This figure demonstrates how adding constraints (highlighted in red) in an AIDL program for a phone handset eliminates geometrical flaws in the generated 2D sketch. 
(\textbf{Left}) \langname{} code for handset design.
(\textbf{Top right}) Design before constraints applied.
(\textbf{Bottom right}) Design after constraints applied.}
\label{fig:phone_handset}
\end{figure*}

Next, we showcase \langname{} by example and show how the different language constructs fulfill our design goals. First, we will illustrate the basic constructs of \langname{} with the phone handset example in Fig.~\ref{fig:phone_handset}.
An \langname{} program starts by defining the high-level logic of a design.
These high-level building blocks are called structures and they are of different types, such as $\verb|Solid|$ and $\verb|Hole|$, and they can be empty, a list of primitives, a list of substructures or any combination of these, see Fig.~\ref{fig:language-grammar}.

In the handset example, we first define an empty structure, L.2, which we populate with primitives, such as rectangles, lines and arcs, L.3-L.8.
Next, we add unary and binary geometric constraints, e.g. Horizontal and Coincident, between these primitives, L.10-L.16.
Finally, we solve the constraint system to optimize for the final parameters of each geometric primitive, L.18.

\paragraph{References}
%The first major language construct in \langname{} are references.
In \langname{}, references are pointers to geometry, parameters or structures. They have various usages.

First, instead of specifying coordinates directly such as in L.3, we can use references to reuse already defined geometry.
For example, in L.4, we define an Arc, which in the \langname{} API is defined via $\verb|Arc(center, start, end)|$.
The $\verb|left_round|$ arc starts at the upper left corner of the $\verb|base|$ rectangle via the reference $\verb|handset.base.top_left|$.
This strategy lowers the risk of erroneously recomputing coordinates of the upper left point.
%Using this reference, we do not risk making an error in computing the coordinates for the upper left point of the $\verb|base|$ rectangle.
Second, this reference ensures that $\verb|base|$ and $\verb|left_round|$ stay attached during the constraint solving process.
Indeed, by sharing a common point, we \emph{implicitly} define a coincidence constraint between them.

Geometric primitives can also be referenced within constraint calls.
In L.10, we \emph{explicitly} define a coincidence constraint between the upper right corner of $\verb|base|$ and the end point of the arc $\verb|right_round|$.
The arc $\verb|right_round|$ has been defined with explicit coordinates in L.6, which, without further constraints, is not necessarily connected to the rest of the shape, see Fig.~\ref{fig:phone_handset} (top right).

Lastly, as can be seen in Fig.~\ref{fig:language-grammar}, references can also point to parameters of geometric primitives.
This allows for more control and more expressivity when defining geometry and constraints.
Consider L.12, where we used equation constraints to express a symmetric design intent on the two lines $\verb|left_line|$ and $\verb|right_line|$.
L.12 declares that both lines should have the same $\verb|length|$, which is a parameter of the $\verb|Line|$ primitive.
%L.19 states that the $\verb|v|$ parameters, i.e., the vertical parameter of both lines' endpoints should be the same.
Parameters are referenceable on the same level as geometry and structures, making them first-class constructs in our language.

\paragraph{Constraints}
Constraints express design intent, i.e., the way that geometry should behave under change.
As we have already seen, in \langname{}, constraints can be implied by sharing a reference, see L.4, or by explicitly adding them to the design via $\verb|AddConstraint|$ calls.
Constraint operations have a certain constraint type and they take as input references.
Depending on the constraint type, either equality or inequality constraints will be enforced on the geometric parameters specified by the input references.
For example, in L.14, the $\verb|Equal|$ constraint type enforces the $\verb|diameter|$ of the two arcs $\verb|left_fillet|$ and $\verb|right_fillet|$ to be the same.

Using references and constraints, we can explicitly state the design intent, which will be realized by an external solver, L.18, (\dgone{}), (\dgtwo{}).

\mlcomment{
-- necessary, basic features. Solver-aided
How to construct something minimal in our language?
Get familiar with the syntax.
Here, we want to talk about references and constraints.

Simple example: ..
We create primitives.
We set constraints between these geometric primitives.
1. Explicit constraints because LLMs bad at global positioning
2. Explicit constraints can be written as arbitrary equations.
6. Solver comports with LLM best practice of using external tools
What we can see here are explicit constraints between geometric entities.
Geometric entities are being \emph{referenced} by constraints.

However, constraints can also be implicitly created by references. Dualism between references and constraints.
As you can see in the example, a point has a reference to two lines. This creates an implicit constraint.
4. Implicit constraints can be specified by shared references
(\dgone{}),
\textbf{(DG2)}

So far, we have talked about constraints and references between geometric entities.
But as you might have noticed on line N, something else has been referenced: a parameter.
11. In our DSL parameters and constants are first class citizens. The leaf nodes are parameters and constants, which allows for arbitrary relationships and referenced expressions (e.g. length, width, etc.). Allows for the specification of non-local and inequality relations
}

\paragraph{Synonymous operators}
%--- Redundancy and Reachability/Referenceability
References and constraints in a DSL are useful if they are easy to use.
For human users, learning a new DSL can be challenging if its API is long and redundant.
Concise APIs are usually preferred.
However, designing a DSL for LLMs introduces a different criteria, which is that the LLM might write a function call which is not part of the API, but which is semantically equivalent.
For example, consider the two constraint calls: (1) \iverb{AddConstraint(Perpendicular( line_1, line_2))} and (2) \iverb{AddConstraint(Orthogonal( line_1, line_2))}. 

Intuitively, both $\verb|Perpendicular|$ and \iverb{Orthogonal} should enforce the same angle between the two lines, i.e., they are synonyms.
However, to reduce redundancy, most APIs will choose only one of them.
In \langname{}, we expose both constraint types, to account for syntactical weaknesses of LLMs and to take advantage of their semantic versatility (\dgthree{}).
More generally, we opt for a robust API vocabulary, allowing for different ways of constructing primitives, e.g. \iverb{Triangle(center, base, height)} vs. \iverb{Triangle(pt_a, pt_b, pt_c)}.% and synonymous combinations of compositional constraints, e.g. $\verb|Underneath|$ is equivalent to $\verb|Below|$ plus $\verb|HorizontallyInside|$. \ben{Compositional constraints are no longer named}

%References and constraints are great, but they can be complex to use.
%We can make them more accessible to LLMs by providing redundancy.
%Talk about the trade-off between uniqueness and synonyms, a DSL for people vs. for LLMs.
%New example here.
%3. Explicit constraint can be specified from a library of semantically meaningful relationships
%8. Robust vocabulary. Verbose semantically named constructs give the LLM a lot of natural ways to refer geometry and describe relationships. (many named constraints, many paths to references, many named expressions). Many synonyms. Natural languages tend to prefer conciseness given by a larger vocabulary, so create many variations and pre-named objects and constraints. (coding languages vs natural languages)
%\textbf{(DG3)}
%References can also be constructed synonymously, e.g. the lower left point of a rectangle can be referenced via \iverb{rect.bottomLeft} or \iverb{rect.bottom.start}.

Note that even though we have synonymous references in \langname{}, they are all being compiled to unique identifiers.
During the interpretation of the program, we include only referenced entities in the model.
%As such, referenceability, reachability by references, is the actual definition of the design's constituent entities. 

%\adriana{worth adding at some point that we can also have multiple names for references for example bottom left of rectagle or rectangle.bottom endge, left point (I don't know the exact syntax)}
%\felix{It would be \iverb{rect.bottomLeft} or \iverb{rect.bottom.start}. But there is no notion of "left" for a line. So this doesn't really work. I don't know if there's a good example for synonymous references.}

%\felix{What about persistent naming?}

%These redundant references are being solved/compiled to unique identifiers.
%In our language, being able to reference entities is not only a gimmick, but referencability, i.e., reachability by references, defines what is actually included in the model.
%Redundancy helps us to ensure that the LLM can effortlessly define a model.
%7. Reachability by named references defines inclusion into model -> everything has a meaningful name, and potentially more than one for different contextual meanings (corner vs end, etc.) We can name everything that exists at solve time (e.g. user or LLM can express). ``All refs can be reached by persistent, semantic name''. No need for queries, all references are valid forever (persistent naming).



\paragraph{Hierarchical designs}
Next, we illustrate the use of hierarchical designs with a complete phone design, see Fig.~\ref{fig:teaser}.
%The handset example in Fig.~\ref{fig:phone_handset} is a flat design, where all geometric primitives are equally important and all constraints are solved at the same time.
%To achieve more complex outcomes, \langname{} allows for hierarchical designs.
%In this section, we refer to the phone design shown in Fig.~\ref{fig:phone_full}.
%Hierarchical designs start by specifying the high-level structure.
The phone is an assembly made out of three different structures, the $\verb|base|$, $\verb|receiver|$ and $\verb|dial_plate|$, which are all $\verb|Solid|$ structures.
These structures are directly attached to the $\verb|telephone|$ structure on lines 5, 9 and 13.
As for the handset design in Fig.~\ref{fig:phone_handset}, each structure defines its own geometry and and constraints, e.g. the constraints for the receiver, L.20-21.
Constraints can also be enforced between structures, which will be solved iteratively in tandem with structure-internal constraints, see Sec.~\ref{sec:solver}.
%Next, we specify the geometry and the internal constraints for each structure, L.9-L.12.
%An example for this is the phone handset from Fig.~\ref{fig:phone_handset}.
%Then, we specify constraints between structures, L.14-L.17.
%For example, the constraint operation \iverb{CenterInside( phone.base, phone.dial)} places the dial in the middle of the base.
%Constraints between structures act on the respective bounding boxes, for more details, see Sec.\ref{sec:solver}.
%Then, we specify boolean operations between structures.
%Here, the phone base is a result of a solid shape from which we cut out a hole to place for the dial, L.19-L.20.
%Finally, we solve the entire, hierarchical design in L.21.

%The structure of this example program closely matches our code generation strategy described in Sec.~\ref{sec:code_generation}.
%Here, we follow with a few remarks about the language constructs which enable this strategy.
%
%In our top-down approach, we encourage planning ahead by first building up a tree structure, see Fig.~\ref{fig:phone_full} (right) with empty nodes.
%In \langname{}, at any point in time we have a valid design since structures can be empty, see Fig.\ref{fig:language-grammar}.
%In other words, we allow for partial evaluation, which we realize by instantiating empty structures with virtual bounding boxes.

Finally, in \langname{}, the result of a boolean operation cannot be referenced, since the parameter-dependent topological outcome requires queries, see Sec.~\ref{sec:analysis_llm}.
To implement this, boolean operations are implied by using different structure types and then applied after constraint solving in a boolean post-process.
%In L.5-L.6, we define two structures as \iverb{Solid} and as \iverb{Hole} and in L.19-L.20, we assign them to the same parent, which realizes the desired subtraction operation seen in Fig.~\ref{fig:phone_full} (right).

%\felix{TODO: Talk about this final paragraph}
%Finally, note that even though there are multiple structures with internal geometry and internal constraints, L.9-L.12, there is only a single solve at the end, L.21.
%Whereas in traditional CAD systems, each substructure would first solve its constraint, fix the geometry, and then define subsequent substructures, in \langname{}, we first specify the entirety of the design and then perform one global, hierarchical constraint solve.
%Our approach means that we do not have to provide the LLM with feedback from intermediate solves at the cost of a more difficult, hierarchical solve, see Sec~\ref{sec:solver}. \adriana{this doesn't seem right to me. I think we do solver hierarchically and give feedback.}

\mlcomment{
-- medium features. Hierarchical language
Here, we are introducing more complex designs.
9. Tree structures allows for guided hierarchical reasoning
5. Having one global solve step removes the need to reason about intermediate solve results, disadvantages: harder solve (use references + hierarchy to help + custom solver), do not get intermediate feedback (but we want hierarchical planning instead of sequential, so make that tradeoff)
10. Allow evaluation of partial trees with empty nodes to give intermediate feedback. We are linearizing the process of planning and fulfilling plan. Explicitly describe the top-down reasoning that is usually outside of CAD tools. (LLMs are known bad at this, so we help them).
\textbf{(DG4)},
\textbf{(DG5)}

-- advanced features
Let's make our language CAD-complete by adding booleans.
12. Booleans at the end; supporting booleans allows for richer specification, by putting them at the end we maintain reference validity and completeness up to the solve. Primitives + booleans = most things you can do with CAD. (inspired by CSG). We disallow the operations with implicit booleans. Don't need to handle issues of persistent referencing.

}


\mlcomment{
several reasons to decide to not have constructive operations, we have it break down hierarchical instead. Sequential is useful when you have a human going with sequential feedback. We've made the black box of hierarchical planning visible, but to do this we needed to get rid of sequential feedback for references / queries. (terms we could use for sequential execution; coarse-to-fine modeling, embodied creativity with material in front of you) - we keep track of the plan so the LLM doesn't have to.


Uses Constraints throughout:
tradcad doesnt need global solve after everything
doesn't need to reason about intermediate solve results
LLMs bad at global positioning
External tool aligns with LLM usage
(no free lunch, makes the solve hard, so we need the tree structure + feedback)

Programmer constructed entities are all stored by reference. (Persistent naming)
- this gives them semantically meaningful and understandable names
- removes need to reason about querying
- removes need to reason about reference validity
- indirect specification of relationships through shared references (also simplifies the constraint solving problem a bit) -- don't need to reason about implicit constraints that would be found by a GUI
- all operations pre-solve create deterministic referenceable entities, so no need for queries
- existence in the model is determined by reachability (makes everything have names)

(dependencies in geometry - refering to geometry to specify constraint or construction - relationships can be specified through constraints or edits. In trad cad you)

Names belong to references -- persistent naming --> any topology changing operations happen after solving (so we know ahead of time exactly which references come out), this leads to certain geometry of the final design being unreferencable, but also means that no references can be invalidated after they are created. Tie to the LLM; doesn't need to reason about queries, validity of references, etc., plus uses names that are easy for it to understand. References also allow for the indirect specification of relationships through reference sharing. (We don't have selectability completeness)

Tree Structured Models (with DAG leaves for geometry and expressions)
- breaks down global solve
- gives multiple meaning names

Name tree relationships and DAG relationships -> multiple semantically meaningful references that are context based



% Questions: Organize goals -> decisions or decisions <- goals?
% 
%

In this section we translate the design goals into concrete design decisions, including language limitations and tradeoffs.

Design Decisions / Key Constructs

Hierarchical Model Structure:
Booleans (e.g. constructive operations) after solve. This allows for references to be deterministic while the program is running, so we do not need queries, which would be difficult for an LLM to reason about. The compromise here is that some entities in the final design, e.g. topology that comes into being as a result of final, solved operations, are unreferenceable.

Indirect Specification of geometry and geometric relationships:
This leads to a few choices, namely the combination of constraints, parameters, and references for both geometry and parameters. Since parameters are stored and passed by reference, geometry can be implicitly related by sharing references (e.g. a rectangle stays together by its edges sharing references to the corner points rather than an explicit constraint equation saying their endpoints are coincident). Constraints allow indirect specification of relationships in the sense that the LLM does not need to construct objects to directly have particular relationships.

Geometric Constraints:

Compositional Constraints:

Direct Access to Parameters and directly expressible constraint equations:
In traditional GUI-based CAD, every reference and constraint needs to be visually expressible in order to be able to be operated upon by the user. Since this is a programming language based approach, we can and do support arbitrary constraint expressions.

Verbose Language with Semantically Meaningful Names:
This comes in two parts. One is in the language's standard library, which contains many natural sounding synonyms for describing constraints (above, on top of, below, underneath, etc.). The other is in the requirement that the relationships in the hierarchy are all named. This creates context dependent, named references for the each structure, geometry, and parameter (and multiple ways of referencing the same geometry since each unique reference path generates a different semantic name, e.g. start point of the bottom edge of a rectangle versus bottom left point of a rectangle).

To support planning and iteration, we allow evaluation and feedback on partial designs that contain no geometry, by allowing compositional constraints to still have meaning even when there is no geometry.    


Validate Design Decisions -- Could align to section 4 instead (key ``constructs'' / design decisions)
\begin{itemize}
    \item Indirect specification of geometry - covered by language comparison
    \item Indirect specification of geometric relationships - covered by language comparison
    \item Intuitively named operators - no sugaring, just write constraint equations
    \item Hierarchical Design - no hierarchical solve
    \item Partial Evaluation - no feedback loop
\end{itemize}


\begin{itemize}
    \item Intuitively named operators
    \item Hierarchical Design
    
    \item Partial Evaluation
\end{itemize}
}





\subsection{Compilation and Constraint Solving}
\label{sec:solver}

%==================== NEW TEXT ==================================

The hierarchical organization of AIDL models allows for recursive constraint solving. We employ an iterative deepening, recursive solver strategy that allows AIDL to solve a minimal constraint problem at each stage, and also keeps substructures fixed as much as is possible to avoid unintuitive changes to substructures due to higher-level constraints. (translations of substructures are preferred over modification of internal geometry to satisfy constraints). To facilitate this recursive solving, AIDL models are first \emph{validated} to ensure that each substructure is independently solvable, then \emph{compiled} into a hierarchy of geometric constraint problems that we solve with an iterated Newton's method solver. The solved model is then \emph{post-processed} to perform boolean operations and generate the final geometry.

When an AIDL program is run as a Python program, it generates a Structure tree data structure. An AIDL model is valid if Geometry only references other Geometry belonging to the same Structure, and Constraints only reference Geometry, Parameters and Structures within the same subtree. Definition of constraint equations in AIDL is \emph{deferred} until after the tree structure is finalized because bounding boxes and some geometric constraints are not well defined until the model topology and initial parameters are fixed. Two non-inversion constraints are added to each bounding box, $height >= 0$ and $width >= 0$, using a slack variable formulation borrowed from linear programming (e.g. $height + s == 0 \land s - |s| == 0$).

The constraint system of an AIDL model is solved hierarchically as described in \Cref{app:solver} using an iterated Newton's method solver (based on SolveSpace~\cite{westhues_solvespace_2022}). Iteration is used to support bounding boxes; at each iteration we fix the expression of each bounding box limit relative to the initial positions of its geometry, then re-check and re-solve if a different piece of geometry now defines the limit. Solved AIDL models are post-processed to apply boolean operations defined by Solid and Hole Structures. Curve geometry is recursively aggregated to discover closed faces which are boolean unioned or subtracted from each other depending on the type of Structure they belong to. We use  the OpenCascade Modeling Kernel~\cite{occt3d_opencascade_2021} to perform boolean operations and generate output in the CAD standard STEP format.

\iffalse

%==================== OLD TEXT ==================================
%Points to make:
%Validation and compilation is designed to make the models hierarchically solvable. Reasons for hierarchical solvability:
%- want to minimize the size of the constraint problems being solved for tractability
%- want to isolate substructure solves as-much-as-possible for consistency; e.g. if a structure can be left alone that is best, if it can be rigidly translated, and only if that is not possible do we allow

%- We solve using a modified Newton solver (forked from SolveSpace [cite Westhues 2022])




%Primary design goals:

%Hierarchical






After a model is created in the AIDL DSL, its validity is checked and it is compiled into an AIDL model, which is a tree structure in which every node is also a valid AIDL model. 
The AIDL model defines a system of geometry, parameters, and constraints in a hierarchy. 
The constraint system is then solved recursively bottom-up using an iterated Newton's method solver to find parameter values that satisfy all constraints while minimizing the size of constraint subproblems. 
Finally, the geometry is aggregated up the tree from leaves-to-root, combined via boolean operations to form the final geometry.

\paragraph{Validation and Compilation}
To compile an AIDL model, we flatten Geometry nodes such that just primitive geometry (Points, Lines, Circles, and Arcs), Parameters, and Constraints remain, and are attached directly to a Structure. 
We ensure every subtree can be solved independently by validating that Geometry only references other Geometry on the same Structure, and Constraints only reference Geometry and Parameters within their own subtree. 
%Next, we concretize the constraint equations that are deferred. 

Many numeric Expressions in AIDL can only be defined once the model structure is finalized; these are called \emph{deferred} expressions, and exist as Expression-valued functions of the final model hierarchy that get evaluated at the end of compilation. Two kinds of expressions get deferred in AIDL; ambiguous geometric constraints, and bounding boxes. Some geometric constraints, like fixed ``Angle'' have multiple common interpretations (clockwise or counter-clockwise); AIDL resolves this to whichever is closest to being satisfied by the initialization values. Bounding boxes are deferred because both local geometry and structural hierarchy are not fixed until compilation time. 
%In order to support validating constraints between empty Structures before geometry has been generated, we generate width and height Parameters for empty Structures at compile time if their bounding boxes are involved in a deferred constraint expression.
To support validating deferred constraints involving yet empty Structures, we create virtual bounding boxes.

\paragraph{Constraint Solving}
Constraints in AIDL are solved using an iterated Newton's method solver. AIDL Expressions support ``min'' and ``max'', which are used to express bounding boxes. The piece-wise nature of these functions can hinder the converge of Newton's method. 
To promote convergence, we remove these discontinuities by re-writing Expressions involving ``min'' and ``max'' on the active branch, using the model parameter initialization before the Newton solve. 
We then check if the original constraint equations are satisfied, and iterate this process until convergence or failure.
%they are, or one of the re-written problems fails to converge.

In addition to ``min'' and ``max,'' AIDL models can also contain inequality constraints. 
These occur when bounding boxes are constructed for empty structures to ensure that the boxes do not invert, e.g. $height >= 0$. To support these constraints, we borrow the idea of slack variables from linear programming, and rewrite constraints of the form $A <= b$ as $A - b + s == 0 \land s - |s| == 0$. The second equality is equivalent to $s >= 0$, ensuring that the slack variable $s$ is positive. Using inequality constraints in CAD is usually unhelpful because they negate the accuracy benefits of CAD models, and tend to lead to very unpredictable solutions. For this reason, we do not expose the inequality capabilities of AIDL to the programmer. In our system, inequality constraints only appear at intermediate stages of programming when they are used to \textit{validate} the feasibility of a constraint system before concrete geometry is specified.

\paragraph{Model Solving}
Because \langname{} can express compositional constraints between structures, the constraint system can involve the entire tree and grow in complexity, making it difficult to solve and its solution unintuitive. By structuring AIDL models into trees of small, independent constraint problems, which is being reflected by our validation criteria, we ameliorate both issues. With the exception of constraints with references into substructures, we can recursively solve the AIDL model node-by-node.


If constraints refer to substructures, it may not be possible to solve the constraints locally, only modifying their defining structure. 
We iteratively deepen the parameters and constraints considered, one tree-level at a time, whenever a subproblem cannot be solved. This is a greedy approach to finding a minimal solveable subproblem at each stage of the recursive solve. To further reduce changes to solved substructures, we attempt the iterative deepening in several stages; first only adding 2D structure translational degrees of freedom on deeper attempts, then 3D, and finally allowing the local parameters and geometry on subnodes to be modified on the final pass. This preferentially preserves the local geometry of already solved nodes, helping to preserve solutions near the initial parameter values.

\paragraph{Boolean Post-Processing}
The final stage of evaluating an AIDL model is to recursively collect and combine geometry. AIDL supports boolean union and subtraction through its Solid and Hole Structures, operations that are performed on surfaces and volumes, but its highest dimensional geometry primitives are curves. To allow booleans, curves endpoints of Solid and Hole Structures are matched to find closed loops that bound planar regions. This includes geometry from attached child nodes, post boolean. At each node, a boolean union is first carried out over faces discovered from local and Solid descendant edges, and then boolean subtraction performed with faces from Hole children. This recursion continues until an Assembly Structure or the root is reached, at which point face geometry is propagated without further boolean operations. Drawing geometry is interpreted purely as edges and so propagated directly to the root Structure.

\paragraph{Implementation}
We forked our constraint solver from SolveSpace~\cite{westhues_solvespace_2022}, adding support for arbitrary constraint expressions, inequality constraints, extra arithmetic operators, dynamic activate of parameters and constraints, and iterative Newton solving. Boolean operations are implemented with the OpenCascade Modeling Kernel~\cite{occt3d_opencascade_2021}.

\fi

\section{Experiments}

% In this section, we first describe our front-end generative pipeline used to construct 2D CAD shapes in AIDL (\Cref{sec:code_generation}), then we show the results of LLM-based generation of 2D CAD shapes (\Cref{sec:main_results}). In addition, we compare AIDL with the OpenSCAD language on 2D CAD generation and ablate on the language design choices we have made (\Cref{sec:comparisons}).
\paragraph{Implementation}
For our experiments, we perform LLM-driven 2D CAD generations with AIDL. AIDL enables LLM-driven text-to-CAD through a front-end generation pipeline. The pipeline follows a common \textbf{validate-until-correct} pattern. 
We first prompt the LLM with a detailed language description of AIDL, which includes AIDL's syntax, primitive geometry types, and available constraints. Then the LLM is prompted with six manually designed example programs in AIDL for these objects: bottle opener, ruler, hanger, key, toy sword, and wrench. Please refer to the supplemental material for the full list of prompts. Finally, it is prompted to generated the full AIDL program of the desired model. The front-end then executes the generated program, returning tracebacks directly to the LLM in case of failure and prompting the LLM to fix the error. This generation loop is repeated until either a syntactically correct program is found or after $N=5$ failed attempts, taking advantage of incomplete executability to give feedback on partial generations. 
%We manually generated thirty-six test prompts, see the supplemental for the full list. 
For all our experiments, we use the OpenAI's gpt-4o model without finetuning, and we run each prompt ten times with different seeds and collect the runs that generated a valid program.

%For comparison, we perform 2D text-to-CAD with the OpenSCAD language, the most common option for directly coding geometries in CAD, unlike other languages that are typically used with UIs for end-user programming. Additionally, we ablate our language design by comparing AIDL against two variants: $\text{AIDL}{\text{no hierarchy}}$ and $\text{AIDL}{\text{no constraints}}$, which disable hierarchy or constraints, respectively. 
%We report CLIP scores for the renderings of the generated programs across all four approaches in \Cref{tab:clip}. 



%In addition, we conduct a perceptual study where, for each method, we show all the renderings of valid CAD programs for a particular prompt and ask users to select the best one. Due to the high number of prompts, the study is split into four blocks (one for each method) and we show the user one out of four blocks randomly. We collected $32$ responses in total ($8$ on average per method) and we show the aggregated results in\cref{app:user study results}.


\paragraph{Results}
We report both the rendering and program of all runs of on 36 manually generated prompts in the supplemental material. In \Cref{fig:Main result}, we show renderings for a diverse subset of the generated AIDL programs. Despite the LLM not being finetuned with our AIDL language, it successfully generates accurate CAD geometry based on its prior knowledge of these objects. Furthermore, the geometries are grouped hierarchically by semantically meaningful structures and constraints, making them easy to edit. See \cref{app:editability} for an illustration of how an AIDL model can be modified. 
%The CLIP scores in \cref{tab:clip} shows that objects generated with AIDL is visually closer to the user prompt compared to other languages. \adriana{I'm not sure we can say this}


\begin{figure}[htbp!]
  \centering
  \includegraphics[width=\linewidth]{images/main_gallery_linefix.png}
  \caption{\textbf{A sample of LLM-guided 2D CAD generations using AIDL.} An untuned general purpose LLM is able to generate a diverse range of objects with accuracy after being prompted by the AIDL language syntax and a few example programs.}
  \label{fig:Main result}
\end{figure}

\paragraph{Comparisons} For comparison, we perform 2D text-to-CAD with the OpenSCAD language, the most common language for directly coding geometries in CAD, unlike other languages that are typically used with GUIs for end-user programming. We directly prompt the LLM to generated CAD geometry in the OpenSCAD language since the gpt-4o model has prior knowledge about its syntax. We used the same 36 prompts and report all results in the supplemental material. Despite the LLM’s familiarity with OpenSCAD, we observe that AIDL results are often closer to the prompt and achieve higher CLIP scores (see \Cref{tab:clip}). In addition to better prompt alignment, AIDL results exhibit more semantic structure.  
In particular, the OpenSCAD language does not support specifying relationships or dependencies between components, thus the LLM would often opt to generate polygons of the desired shape by specifying explicitly the vertex coordinates (see \Cref{fig:Comparison}), making the resulting program highly difficult to edit. 

We also attempted using FeatureScript and the DSL from the recent work~\cite{cascaval2023lineage} for LLM-drive 2D CAD generations. However, the LLM failed to generate syntactically correct programs in almost all cases. This issue was not rectified even when prompting the LLM with example programs and code documentations in those languages. These two languages are not syntactically based on common programming languages usually found in LLM training sets. This indicates the importance of designing a semantically rich language that is easy for the LLM to use and manipulate. 

\paragraph{Ablations}We ablate our language design choices by comparing AIDL  against two variants: $\text{AIDL}_{\text{no hierarchy}}$ and $\text{AIDL}_{\text{no constraints}}$, which disable hierarchy and constraints respectively. In $\text{AIDL}_{\text{no hierarchy}}$, all the geometries of a program will live on the same level under a single Structure instance, and all constraints will also be attributed to this single Structure. On the other hand, $\text{AIDL}_{\text{no constraints}}$ is a subset of AIDL where we have simply removed the ability to specify any constraints. For these language subsets, we modify our initial prompts to the LLM to reflect the altered language features. We report all runs on the same 36 prompts in the supplemental material.

%We observe that programs generated with   $\text{AIDL}_{\text{no constraints}}$ will more often exhibit disconnected components that should have been attached, as illustrated in~\Cref{fig:Comparison}. This makes editing the program difficult since simple scaling will require different changes be made to all the components, where having constraints allow a single edit to affect all constrained geometries. More results generated with $\text{AIDL}_{\text{no constraints}}$ in the supplemental show that, while the programs may occasionally have the correct placement, they more often exhibit detached components due to lack of constraints (e.g. results with prompt "barn on a farm" and "lighthouse") \adriana{say that this is particularly the case for models where multiple parts need to be aligned for example in fountain pen.}.  \jz{However, this does not seem to adversely affect CLIP scores for renderings of shapes generated with $\text{AIDL}_\text{no constraints}$ (\Cref{tab:clip}), due to the fact that CLIP scores placing more emphasis on local semantics than having precisely connected geometries.} 
While $\text{AIDL}_{\text{no constraints}}$ occasionally places components correctly, editing such programs is difficult because scaling requires individual adjustments for each component, whereas constraints allow a single edit to affect all geometries. Additionally, it often produces detached components due to the lack of constraints (see \Cref{fig:Comparison} and the ``fountain pen" example in the supplemental material).


%No hierarchy
Programs generated with $\text{AIDL}_{\text{no hierarchy}}$, while being visually similar to the ones generated with $\text{AIDL}$, are harder to refine, since the user cannot choose a particular part of the CAD shape to make local edits, as shown in \Cref{fig:Comparison}.

We observe that neither variation of AIDL significantly impacts CLIP scores for the renderings (\Cref{tab:clip}), because that CLIP scores do not take into account editability and they place more emphasis on local semantics than having precisely connected geometries.

%Programs generated with $\text{AIDL}_{\text{no hierarchy}}$, while being visually simlar to the ones generated with $\text{AIDL}$, are harder to refine, since the user cannot choose a particular part of the CAD shape to make local edits. \adriana{I would move the clip score discussion to the beginning of the section..}

\paragraph{Results Across Multiple Runs} All methods produced at least one valid output per prompt, with success rates as follows: ours: 64\%, $\text{AIDL}_{\text{no constraints}}$: 94\%, $\text{AIDL}_{\text{no hierarchy}}$: 77\%, and OpenSCAD: 79\%. Notably, our method's success rate is only slightly lower than OpenSCAD, which is included in the training data. To showcase the highest-quality output for each method side by side, considering that LLMs produce varying outputs across runs, we conducted a perceptual study to rank the valid CAD programs generated from the 10 runs per method and prompt. The study details are discussed in \cref{app:study}, and the results are provided in the supplemental material.
%\adriana{or are they going back to the appendix?}

\paragraph{Limitations}
Our experiments revealed limitations of our system, particularly around model complexity and underused language features. AIDL supports rectangle rotation, yet all rectangles used in generated examples are axis-aligned. Looking at the generated code and conversation history (see supplemental) shows that the LLM did frequently specify that rectangles were rotatable (a flag in the Rectangle constructor), but failed to rotate them. One shortcoming of the AIDL library is that rectangles can only be rotated by the constraint solver, so an appropriate constraint (usually \verb|Angle|) must be imposed to cause a rectangle to rotate. In cases where the LLM attempted to do this, it hallucinated a non-existent constraint like \verb|Rotate| instead. When errors are reported to the LLM, the most common response is to try removing constraints or structures until the error goes away. Since we apply a validate-until-correct pattern, this means that the removed design intent (e.g. rectangle rotation) is never returned to the model. These limitations stem from our choice to focus on DSL design rather than the complementary approaches of model training or tuning, or prompt engineering. Fine-tuning a model on AIDL code could reduce the incidence of language feature hallucination, and crafting a more interactive prompting and feedback system could allow an LLM to recover lost complexity and design intent in the face of errors.

%The primary limitations of our system lie in the interaction with the LLM. Several features of AIDL were rarely or never utilized in our experiments, despite being present in the in-context examples. Rectangles in AIDL are axis-aligned by default because we noticed, anecdotally, that rectangles in most designs are axis aligned, and found rotating a rectangle to satisfy a constraint to be an unintuitive solution. To address this AIDL provides an optional boolean flag on Rectangle construction that removes the axis alignment constraint, however this flag was only used in one in-context example and does not appear in any of the generated results. Boolean operations in AIDL are defined implicitly by the type of each structure: Solid, Hole, Drawing, or Assembly. While most of the in-context examples used a variety of structure types, Solids were over-represented, with Drawings being scarce and no Assembly examples given. The LLM underutilized non-Solid structures, such as in the ``old-school telephone'' example in \Cref{fig:teaser}, where the base \verb|telephone| structure would be best represented as an Assembly, and \verb|dial_plate.holes| would be better expressed as a Hole-type substructure. In addition to un-and-underused language features, our prompting strategy's termination condition encouraged the LLM to produce less complex models. When confronted with an error such as an unsolvable constraint, the LLM most commonly reduced the complexity of the model until no error was found. Because we stop generation as soon as a valid model is generated, the LLM was not given an opportunity to re-introduce complexity into the model. In the future we would address these issues by curating a more robust set of in-context examples and by crafting a more interactive feedback loop with finer-grained error messages. \adriana{we could also fine tune the LLM to work with aidl}

%As a remark, the decision space of language design is enormous, so we had to make some decisions about what to explore in the language design of AIDL. In particular, we did not build a new constraint system from scratch and instead developed ours based on an open-source constraint solver. This limited the types of primitives we allow, e.g. ellipses are not currently supported. Additionally, rectangles in AIDL are constrained to be axis-aligned by default because we found that in most use cases, a rectangle being rotated by the solver was unintuitive, and we included a parameter in the language allows rectangles to be marked as rotatable. While this feature was included in the prompts to the LLM, it was never used by the model. Furthermore, in testing our front-end, we observed that repeated instances of feedback tends to reduce the complexity of models as the LLM would frequently address the errors by removing the offending entity. This leads to unnecessarily removed details. More extensive prompt-engineering could be employed in future work to encourage the LLM to more frequently modify, rather than remove, to fix these errors. \adriana{no idea what this paragraph is trying to say}

% What this is trying to say:
% - smaller set of primitives than full-fat CAD systems have
% - ***LLM did not pick up on some language features
%   - Doesn't always use booleans when it should
%   - Doesn't use the axis-alignment flag
% - *** LLM will react to solver failures by reducing model complexity, and does not try to add it back in.
% Limitations on the solver

% \subsection{Front End}
% \label{sec:code_generation}

% % \begin{figure}
% %     \centering
% %     \includegraphics[scale=0.3]{images/pipeline.png}
% %     \caption{\textbf{LLM Generation Pipeline.} An initial text prompt is given to the LLM along with instructions to generate a hierarchical AIDL code. Guided by Solver feedback, the LLM then iterates to add compositional constraints between these structures until a feasible code is found. 
% %     % From these we generate a template AIDL program, and direct the solver to add to it, in a bottom-up fashion, to add first geometry and local constraints, followed by global constraints. Solver feedback is used to iterate until correctness at each of these stages as well.
% %     }
% %     \label{fig:frontend}
% % \end{figure}
% AIDL enables LLM-driven text-to-CAD through a front-end generation pipeline. The pipeline follows a common \textbf{validate-until-correct} pattern. 
% We first prompt the LLM with a detailed language description of AIDL, which includes AIDL's syntax, primitive geometry types, and available constraints. Then the LLM is prompted with a few example AIDL programs. Finally, it is prompted to generated the full AIDL program of the desired prompt. The front-end then executes the generated program, returning tracebacks directly to the LLM in case of failure and prompting the LLM to fix the error. This generation loop is repeated until either a syntactically correct program is found or after $N=5$ failed attempts, taking advantage of incomplete executability to give feedback on partial generations. For all our experiments, we use the OpenAI's gpt-4o model without finetuning.


% \subsection{Text-to-CAD Results}
% \label{sec:main_results}
% We show results of LLM-guided 2D CAD generation with AIDL. 
% To generate 2D CAD models with AIDL, \jz{the LLM is prompted with a detailed language description and six manually designed example programs in AIDL for these objects: bottle opener, ruler, hanger, key, toy sword, and wrench. }\adriana{how many?}
% In \Cref{fig:Main result}, we show renderings for a diverse subset of the generated AIDL programs. \adriana{point to supplemental where you have all the rest and say you run 10 times and they are all there (the ones that worked}. Despite the LLM not being finetuned with our AIDL language, it is still able to generate accurate CAD geometry with the LLM's prior knowledge of these objects. 


% \adriana{report how many didn't work - how many runs don't actually get you a result - either max out or don't render. }


% \subsection{Comparison and Ablation}
% \label{sec:comparisons}
% \jz{We compare our language on LLM-driven CAD generation with OpenSCAD, a CSG CAD language. To generate CAD in OpenSCAD, we directly prompt the LLM to generated CAD geometry in the OpenSCAD language since the gpt-4o model has prior knowledge about its syntax. We also ablate our language design choices by comparing AIDL to $\text{AIDL}_{\text{no hierarchy}}$ and $\text{AIDL}_{\text{no constraints}}$. Spefically, $\text{AIDL}_{\text{no hierarchy}}$ is a subset of the AIDL language where hierarchy is removed from the design. Namely, all the geometries of a program will live on the same level under a single Structure instance, and all constraints will also be attributed to this single Structure. On the other hand, $\text{AIDL}_{\text{no constraints}}$ is a subset of AIDL where we have simply removed the ability to specify any constraints. For these language subsets, we modify our initial prompts to the LLM to reflect the altered language features.

% We report CLIP scores for the renderings of the generated programs across all four methods in \Cref{tab:clip}. In addition, we conduct a perceptual study where, for each method,  we show all the renderings of valid CAD programs for a particular prompt and ask users to select the best one. Due to the high number of prompts, the study is split into four blocks (one for each method) and we show the user one out of four blocks randomly. We collected 32 responses in total (8 on average per method) and we show the aggregated results in \cref{app:user study results}.
% } 

% \adriana{I would add a paragraph here describing the study and pointing to the results. you can introduce what the comparison and the 2 ablations are, say that you run them all 10 times and you ask users to get the best and show the result in the figure.  }

% To generate CAD in OpenSCAD, we directly prompt the LLM to generated CAD geometry in the OpenSCAD language since the gpt-4o model has prior knowledge about its syntax. 
% 
% \Cref{fig:Comparison} shows a comparison between geometries generated with AIDL and with OpenSCAD.
% In particular, the OpenSCAD language doesn't support specifying relationships or dependencies between components, thus the LLM would often opt to generate polygons of the desired shape by specifying explicitly the vertex coordinates, making the resulting program highly difficult to edit. \adriana{Would it be good to also point to the results of the study in the appendix here?  and say that even though gpt doesn not know our language in training the generated results often look better? and then mention  that beyond visual appearance the programs we generate are more editable. and say what you had before }

% \label{sec:ablations}
% To highlight the benefits of our language design choices, we ablate AIDL by restricting it to certain subsets of functionalities. With the same LLM-based CAD generation task, we experiment with $\text{AIDL}_{\text{no hierarchy}}$ and $\text{AIDL}_{\text{no constraints}}$. 

% Spefically, $\text{AIDL}_{\text{no hierarchy}}$ is a subset of the AIDL language where hierarchy is removed from the design. Namely, all the geometries of a program will live on the same level under a single Structure instance, and all constraints will also be attributed to this single Structure. On the other hand, $\text{AIDL}_{\text{no constraints}}$ is a subset of AIDL where we have simply removed the ability to specify any constraints. For these language subsets, we modify our initial prompts to the LLM to reflect the altered language features. 
% We show the generated results in \Cref{fig:Comparison}. In particular, programs generated with   $\text{AIDL}_{\text{no constraints}}$ will more often exhibit disconnected components that should have been attached. This makes editing the program difficult since simple scaling will require different changes be made to all the components, where having constraints allow a single edit to affect all constrained geometries. Programs generated with $\text{AIDL}_{\text{no hierarchy}}$, while being visually simlar to the ones generated with $\text{AIDL}$, are harder to refine, since the user cannot choose a particular part of the CAD shape to make local edits. \adriana{I would move the clip score discussion to the beginning of the section..}

% Finally, we conducted a perceptual study where the users are shown all generated examples for a particular prompt and method and are instructed to pick the one that most resembles the prompt. We collected $32$ responses and we show the aggregated the results in \cref{app:user study results}. \adriana{32 responded feels misleading here. I would move this up to the beginning of the section(see comment above).}



\begin{figure}[htbp!]
  \centering
  \includegraphics[width=\linewidth]{images/all_comparison_3.png}
  \caption{\textbf{Comparison and Ablation.} For the task of text-to-CAD, we compare our language to OpenSCAD and ablate on our language design choices. (\textbf{Top Left}) In particular, generated OpenSCAD programs exihibit manually drawn polygons with explicit vertex positions which are difficult to edit. (\textbf{Bottom Left}) Programs generated with $\textbf{AIDL}_{\text{no constraints}}$ has detached parts due to not being able to constrain the relative positions of part geometries. (\textbf{Right}) When an AIDL model is created with a structure hierarchy it is easier to locally edit because of modular substructures (left), while a similar edit on a non-hierarchical model (right) results in the model breaking (the dial moves without the dial holes). Performing the same edit in a non-hierarchical model requires multiple, non-concurrent edits.}
  \label{fig:Comparison}
\end{figure}

% \begin{figure}[htbp]
%   \centering
%   \fbox{\parbox[c][0.3\linewidth][c]{0.8\linewidth}{\centering Ablation result PH}}
%   \caption{Ablation result ph}
%   \label{fig:Ablation result ph}
% \end{figure}


% ours avg: 27.861057512524784
% nc avg: 27.661548950525162
% nh avg: 27.75429911960118
% openscad avg: 26.510543021170914
% ours std: 2.157481509541088
% nc std: 1.9855580138660853
% nh std: 2.006689119383239
% openscad std: 1.752424687473368


\begin{table}[ht]

\centering
\caption{\textbf{Average CLIP scores for all prompts.} We perform text-to-CAD generation with $\textbf{AIDL}$, $\textbf{AIDL}_{\text{no hierarchy}}$, $\textbf{AIDL}_{\text{no constraints}}$, and \textbf{OpenSCAD} on our list of prompts for ten iterations each and show the average CLIP scores over the ones that produced valid programs.}
\label{tab:clip}
\begin{tabular}{lcccc}
\toprule
 & \textbf{AIDL} &$\textbf{AIDL}_{\text{no hierarchy}}$ & $\textbf{AIDL}_{\text{no constraints}}$ & \textbf{OpenSCAD} \\ 
\midrule
\textbf{$\uparrow$ CLIP Score Avg.}              & \textbf{28.90} & 28.64 & 28.89 &27.32 \\ 
\textbf{$\ \ $ CLIP Score Var.}              & 2.24& 1.98 & 2.05 & 1.87 \\ 
\bottomrule
\end{tabular}
\end{table}

% What we need to show:

% Basic Language Usage - hand-crafted examples showcase language features (could be in language section / implementation)
% \begin{itemize}
%     \item clapboard - structure, geometry constraints, expression constraints
%     \item ruler - compositional constraints, functional requirements
%     \item comb - spacing, linear pattern, union
%     \item bottle opener - compositional constraint, alignment, boolean difference
%     \item sundial - equational reasoning, basic 3D, functional requirements - (maybe bottom of results)
%     \item chair - good illustration of compositional constraints defining class - (maybe bottom of results)
% \end{itemize}

% The Language Can Support complex designs - show 1-3 complex designs (hand-crafted, LLM crafted if possible)

% The LLM frontend works:
% \begin{itemize}
%     \item Selected examples with discussion
%     \item Quantitative metrics about large-scale experiment
%     \begin{itemize}
%         \item How many valid programs?
%         \item How many ``look correct''
%         \item How complex are the generated examples
%     \end{itemize}
% \end{itemize}

% Insights about frontend:
% \begin{itemize}
%     \item Stats on frequency of types of feedback
%     \item Example failure cases (could be in limitations section)
% \end{itemize}

% Validate Design Decisions -- Could align to section 4 instead (key ``constructs'' / design decisions)
% \begin{itemize}
%     \item Indirect specification of geometry - covered by language comparison
%     \item Indirect specification of geometric relationships - covered by language comparison
%     \item Intuitively named operators - no sugaring, just write constraint equations
%     \item Hierarchical Design - no hierarchical solve
%     \item Partial Evaluation - no feedback loop
% \end{itemize}

% Front End Ablations - no feedback loop

% Comparisons to Existing Languages - line up to basic kinds of languages:
% \begin{itemize}
%     \item Constraint Based - SolveSpace (bare)
%     \item CSG - Open SCAD
%     \item Reference Based - CAD Query or Dan's language
% \end{itemize}
% Faster Alternative: Base on Table 1, ablate our language on these instead




\iffalse
\subsection{Examples}

We created \data{numExamples} examples. Testing loading \data{foo}. What about math mode?
\[
x + \data{bar}
\]

\begin{figure}
    \centering
    \includegraphics[width=\linewidth]{images/ours.pdf}
    \caption{Success rate at each pipeline stage. Each prompt with tried with 10 random seeds.}
    \label{fig:ours-success}
\end{figure}

\begin{figure}
    \centering
    \includegraphics[width=\linewidth]{images/complexity.pdf}
    \caption{Complexity of solutions under various ablations, shows that adding our stuff allows for more complex models to be generated}
    \label{fig:ablation-complexity}
\end{figure}

\begin{figure*}
    \centering
    \includegraphics[width=\textwidth]{images/results-mock.png}
    \caption{Fabricated results generated with GPT.}
    \label{fig:fabricated}
\end{figure*}

\begin{table}[h]
    \centering
    \begin{tabular}{c|c}
        A & B \\
        C & D
    \end{tabular}
    \caption{Ablation results. TODO - import from csv}
    \label{tab:ablations}
\end{table}

% Subjective Failure Examples
\begin{wrapfigure}{r}{0.25\textwidth}
    \centering
    \includegraphics[width=0.25\textwidth]{example-image-c}
\end{wrapfigure}

\begin{figure}
    \centering
    \includegraphics[width=\linewidth]{example-image-a}
    \caption{Removing hierarchy and compositional constraints causes these failures in spatial reasoning.}
    \label{fig:ablation-failures}
\end{figure}


\paragraph{Fabricated Examples}
\langname{} currently supports compositions sketch-and-extrude solids and draw sketches, which we realized using laser cutting. Using GPT-4-Turbo as our LM, we produced 5 physical models:

\begin{itemize}
    \item ruler: shows code-reuse (two scales), necessity of using a DSL over just structure (looping over markings), mixing solids and drawings
    \item comb: shows-off substructure for solids (teeth), and the use of layout constraints to align them
    \item sundial: shows-off 3D from 2D, cross-level constraints, mixing solids and drawings, very-much needs the DSL support to compute the correct marking positions and shows that the language models know about and can provide these functional requirements
    \item glasses: repeated substructures (earpieces), everything is symmetric
    \item table: simple model really only shows off the inside and on-top-of layout constraints, is here to demonstrate utility beyond just laser-cutting since we could realize by cutting thicker stock by another means
    \item box: Solver handling 3D reasoning (in front, behind, on top, perpendicular) so that language doesn't have to
\end{itemize}

\subsection{Ablations}

The design decisions come in many categories:
\begin{itemize}
    \item Arbitrary decisions because I had to choose something. These will be mentioned in the future work section.
    \item Design decisions made for implementation convenience / technical limitations. Specific benefits / limitations of these will be left for the future work section.
    \item Principled decisions made based on conjecture or experience, but without experimental backing
    \item Principled decision based on anecdotal observations and ad-hoc experiments. We could attempt to make rigorous ablations for each of these, or to construct single illustrative examples, but there is likely diminishing returns to that. These reasons and justifications will only exist in the methods / system design section
    \item Principled decisions based on evidence from related work
    \item Principled decisions that we ablate on. These will be mentioned when the design choice is specified in methods / system design, along with a reference to the ablation section where it is re-emphasized and demonstrated. This takes effort to do and paper space, so the most important decisions should be prioritized.
\end{itemize}

\paragraph{Language Design Ablations}

Design decisions to justify:

hierarchical structure:
Claimed Benefits:
 - enables LLM to find more complex designs
   - experiment: Run same prompts with flat versus hierarchical version of language, show visually the differences, measure how frequently it sucseeds at fixing from different seeds
 - assists in targeting feedback from solver
   - Break a flat vs. structured program in the same way
 - allows for precise scoping while editing
 - allows for 1:1 named references - e.g. click on a part and can see exactly what it's called in the program. E.g. we can do this like point to a substructer and say "make this taller" or "expose a parameter to adjust this height"
    - Experiment: handmake two programs, give to GPT-4, and ask for the same modification and for the same exposed parameter. Qualitatively compare and run N times to see how often it succeeds / fails

Experiments:
- Run with just the basic geometry and geometric helpers (this allows layout constraints to be used)
- To be fair, we should still include the natural-language reasoning prompt to illicit reasoning about substructure in the model context before the program is written
- Expectation: it naturally create some sub-structure in the code, but will fail in relative substructure positioning (e.g. frequently putting the scale outside of the ruler body, etc.). It may also be difficult to get it to correct individual parts of the program without modifying other pieces

Explicitly naming structures:
- assists in targetting feedback from the solver
  -  break programs with and without semantic names identically then try and repair, measure how frequently it can from different seeds

solver-aided design:
- exp 1: Just primitives, no constraints. Still create the subtrees, but just place objects in space within each subtree. The only exception is to allow ``perpendicular'' as a layout constraint to disentangle objects since we don't explicitly set workplanes and rely on constraints for relative 3D positioning
   - Is able to get small-scale details correct, but cannot get large scale details working
- exp 2: just low-level constraints, no high level. Can do some connectivity, but large-scale planning isn't working
- exp 3: only high-level constraints (no geometric CAD constraints). Connectivity between things (like comb-tooth corners) does not work, relative scales are wonky because when we need to resolve high-level, the low level shapes get stretched and skewed

DSL vs data-structure generation:
The claimed benefits of using a python-based DSL are

Modular Code reuse: comb teeth, marker scales, glasses earpieces
Calculated dimensions: sundial hour marks
Take advantage of lots of python pre-training: does DSL generation have a higher validation success rate or take fewer feedback iterations to get correct (valid)?


Prompt with the IR representation as json. Optionally use json-mode output from GPT4-turbo to guarantee well-formatted JSON output. Expect that repeated details are missed or have less detail (the ruler marks), and that it won't be able to get the sundial's marks correct at all because those require solving trigonometric functions.

\paragraph{Code Generation Ablations}

Pre-code natural-language reasoning: expect that we'll see less complex structures without it.

Multi-prompt vs all-at-once generation: after pre-code reasoning prompt, just ask it to generate the whole codebase at once

Few-Shot Learning:
 - providing API code
 - providing example programs
 - providing example interaction


\fi
% \section{Discussion and Future Work}

% %Onscad is insample data cause these LLMS have seen openscad

% The decision space of language design is enormous, so we had to make some decisions about what to explore in the language design of AIDL. In particular, we did not build a new constraint system from scratch and instead developed ours based on an open-source constraint solver. This limited the types of primitives we allow, e.g. ellipses are not currently supported. \jz{Additionally, rectangles in AIDL are constrained to be axis-aligned by default because we found that in most use cases, a rectangle being rotated by the solver was unintuitive, and we included a parameter in the language allows rectangles to be marked as rotatable. While this feature was included in the prompts to the LLM, it was never used by the model. We hope to explore prompt-engineering techniques to rectify this issue in the future. Similarly, we hope to reduce the frequency of solver errors by providing better prompts for explaining the available constraints.} \adriana{Add two other limitations to this paragraph: that we typically noticed that things are axis alignment, say why we use this as default and in the future could try to get the gpt to not use default more often. Mention that we still have Solver failures that could be addressed by better engineering in future. }

% In testing our front-end, we observed that repeated instances of feedback tends to reduce the complexity of models as the LLM would frequently address the errors by removing the offending entity. This leads to unnecessarily removed details. More extensive prompt-engineering could be employed in future work to encourage the LLM to more frequently modify, rather than remove, to fix these errors. \adriana{no idea what this paragraph is trying to say}


% \adriana{This seems  like a future work paragraph so maybe start by saying that in the future you could do other front end or fine tune a model with aidl, we just tested the few shot.  } \jz{In the future, we hope to improve our front-end generation pipeline by finetuning a pretrained LLM on example AIDL programs.} In addition, multi-modal vision-langauge model development has exploded in recent months. Visual modalities are an obvious fit for CAD modeling -- in fact, most procedural CAD models are produced in visual editors -- but we decided not to explore visual inputs yet based on reports ([PH] cite OPENAIs own GPT4V paper) that current vision-language models suffter from the same spatial reasoning issues as purely textual models do (identifying relative positions like above, left of, etc.). This also informed our decision to omit spline curves which are difficult to describe in natural language. This deficit is being addressed by the development of new spatial reasoning datasets ([PH] cite visual math reasoning paper), so allowing visual user input as well as visual feedback in future work with the next generation of models seems promising.

% The decision space of language design is enormous, so it was impossible to explore it all here. We had to make some decisions about what to explore, guided by experience, conjecture, technical limitations, and anecdotal experience. Since we primarily explore the interaction between language design and language models in order to overcome the shortcomings in the latter, we did not wish to focus effort on building new constraint systems. This led us to use an open-source constraint solver to build our solver off of. This limited the types of primitives we allow; in particular, most commercial geometric solvers also support ellipses.

% In testing our generation frontend, we observed that repeated instances of feedback tended to reduce the complexity of models as the LLM would frequently address the errors by removing the offending entity. This is a fine strategy for over-constrained systems, but can unnecessarily remove detail when done in response to a syntax or validation error. More extensive prompt-engineering could be employed to encourage the LLM to more frequently modify, rather than remove, to fix these errors


% In recent months, multi-modal vision-language model development has exploded. Visual modalities are an obvious fit for CAD modeling -- in fact, most procedural CAD models are produced in visual editors -- but we decided not to explore visual input yet based on reports (cite OpenAIs own GPT4V paper) that current vision-language models suffer from the same spatial reasoning issues as purely textual models do (identifying relative positions like above, left of, etc.). This also informed our decision to omit spline curves; they are not easily described in natural language. This deficit is being addressed by the development of new spatial reasoning datasets (cite visual math reasoning paper), so allowing visual user input as well as visual feedback in future work with the next generation of models seems promising.



\section{Conclusion}

AIDL is an experiment in a new way of building graphics systems for language models; what if, instead of tuning a model for a graphics system, we build a graphics system tailored for language models? By taking this approach, we are able to draw on the rich literature of programming languages, crafting a language that supports language-based dependency reasoning through semantically meaningful references, separation of concerns with a modular, hierarchical structure, and that compliments the shortcomings of LLMs with a solver assistance. Taking this neurosymbolic, procedural approach allows our system to tap into the general knowledge of LLMs as well as being more applicable to CAD by promoting precision, accuracy, and editability. Framing AI CAD generation as a language design problem is a complementary approach to model training and prompt engineering, and we are excited to see how advance in these fields will synergize with AIDL and its successors, especially as the capabilities of multi-modal vision-language models improve. AI-driven, procedural design coming to CAD, and AIDL provides a template for that future.

% Using procedural generation instead of direct geometric generation enables greater editability, accuracy, and precision
% Using a general language model allows for generalizability beyond existing CAD datasets and control via common language.
% Approaches code gen in LLMs through language design rather than training the model or constructing complexing querying algorithms. This could be a complimentary approach
% Embedding as a DSL in a popular language allows us to leverage the LLMs syntactic knowledge while exploiting our domain knowledge in the language design
% LLM-CAD languages should hierarchical, semantic, support constraints and dependencies




%In this paper, we proposed AIDL, a language designed specifically for LLM-driven CAD design. The AIDL language simultaneously supports 1) references to constructed geometry (\dgone{}), 2) geometric constraints between components (\dgtwo{}), 3) naturally named operators (\dgthree{}), and 4) first-class hierarchical design (\dgfour{}), while none of the existing languages supports all the above. These novel designs in AIDL allow users to tap into LLMs' knowledge about objects and their compositionalities and generate complex geometry in a hierarchical and constrained fashion. Specifically, the solver for AIDL supports iterative editing by the LLM by providing intermediate feedback, and remedies the LLM's weakness of providing explicit positions for geometries.

%\adriana{This seems  like a future work paragraph so maybe start by saying that in the future you could do other front end or fine tune a model with aidl, we just tested the few shot.  }
%\paragraph{Future work} In recent months, multi-modal vision-language model development has exploded. Visual modalities are an obvious fit for CAD modeling -- in fact, most procedural CAD models are produced in visual editors -- but we decided not to explore visual input yet based on reports (cite OpenAIs own GPT4V paper) that current vision-language models suffer from the same spatial reasoning issues as purely textual models do (identifying relative positions like above, left of, etc.). This also informed our decision to omit spline curves; they are not easily described in natural language. This deficit is being addressed by the development of new spatial reasoning datasets (cite visual math reasoning paper), so allowing visual user input as well as visual feedback in future work with the next generation of models seems promising. 


% \subsubsection*{Acknowledgments}
% Use unnumbered third level headings for the acknowledgments. All
% acknowledgments, including those to funding agencies, go at the end of the paper.



\bibliography{bibliography, generals}
\bibliographystyle{iclr2025_conference}

\newpage
\appendix
\subsection{Lloyd-Max Algorithm}
\label{subsec:Lloyd-Max}
For a given quantization bitwidth $B$ and an operand $\bm{X}$, the Lloyd-Max algorithm finds $2^B$ quantization levels $\{\hat{x}_i\}_{i=1}^{2^B}$ such that quantizing $\bm{X}$ by rounding each scalar in $\bm{X}$ to the nearest quantization level minimizes the quantization MSE. 

The algorithm starts with an initial guess of quantization levels and then iteratively computes quantization thresholds $\{\tau_i\}_{i=1}^{2^B-1}$ and updates quantization levels $\{\hat{x}_i\}_{i=1}^{2^B}$. Specifically, at iteration $n$, thresholds are set to the midpoints of the previous iteration's levels:
\begin{align*}
    \tau_i^{(n)}=\frac{\hat{x}_i^{(n-1)}+\hat{x}_{i+1}^{(n-1)}}2 \text{ for } i=1\ldots 2^B-1
\end{align*}
Subsequently, the quantization levels are re-computed as conditional means of the data regions defined by the new thresholds:
\begin{align*}
    \hat{x}_i^{(n)}=\mathbb{E}\left[ \bm{X} \big| \bm{X}\in [\tau_{i-1}^{(n)},\tau_i^{(n)}] \right] \text{ for } i=1\ldots 2^B
\end{align*}
where to satisfy boundary conditions we have $\tau_0=-\infty$ and $\tau_{2^B}=\infty$. The algorithm iterates the above steps until convergence.

Figure \ref{fig:lm_quant} compares the quantization levels of a $7$-bit floating point (E3M3) quantizer (left) to a $7$-bit Lloyd-Max quantizer (right) when quantizing a layer of weights from the GPT3-126M model at a per-tensor granularity. As shown, the Lloyd-Max quantizer achieves substantially lower quantization MSE. Further, Table \ref{tab:FP7_vs_LM7} shows the superior perplexity achieved by Lloyd-Max quantizers for bitwidths of $7$, $6$ and $5$. The difference between the quantizers is clear at 5 bits, where per-tensor FP quantization incurs a drastic and unacceptable increase in perplexity, while Lloyd-Max quantization incurs a much smaller increase. Nevertheless, we note that even the optimal Lloyd-Max quantizer incurs a notable ($\sim 1.5$) increase in perplexity due to the coarse granularity of quantization. 

\begin{figure}[h]
  \centering
  \includegraphics[width=0.7\linewidth]{sections/figures/LM7_FP7.pdf}
  \caption{\small Quantization levels and the corresponding quantization MSE of Floating Point (left) vs Lloyd-Max (right) Quantizers for a layer of weights in the GPT3-126M model.}
  \label{fig:lm_quant}
\end{figure}

\begin{table}[h]\scriptsize
\begin{center}
\caption{\label{tab:FP7_vs_LM7} \small Comparing perplexity (lower is better) achieved by floating point quantizers and Lloyd-Max quantizers on a GPT3-126M model for the Wikitext-103 dataset.}
\begin{tabular}{c|cc|c}
\hline
 \multirow{2}{*}{\textbf{Bitwidth}} & \multicolumn{2}{|c|}{\textbf{Floating-Point Quantizer}} & \textbf{Lloyd-Max Quantizer} \\
 & Best Format & Wikitext-103 Perplexity & Wikitext-103 Perplexity \\
\hline
7 & E3M3 & 18.32 & 18.27 \\
6 & E3M2 & 19.07 & 18.51 \\
5 & E4M0 & 43.89 & 19.71 \\
\hline
\end{tabular}
\end{center}
\end{table}

\subsection{Proof of Local Optimality of LO-BCQ}
\label{subsec:lobcq_opt_proof}
For a given block $\bm{b}_j$, the quantization MSE during LO-BCQ can be empirically evaluated as $\frac{1}{L_b}\lVert \bm{b}_j- \bm{\hat{b}}_j\rVert^2_2$ where $\bm{\hat{b}}_j$ is computed from equation (\ref{eq:clustered_quantization_definition}) as $C_{f(\bm{b}_j)}(\bm{b}_j)$. Further, for a given block cluster $\mathcal{B}_i$, we compute the quantization MSE as $\frac{1}{|\mathcal{B}_{i}|}\sum_{\bm{b} \in \mathcal{B}_{i}} \frac{1}{L_b}\lVert \bm{b}- C_i^{(n)}(\bm{b})\rVert^2_2$. Therefore, at the end of iteration $n$, we evaluate the overall quantization MSE $J^{(n)}$ for a given operand $\bm{X}$ composed of $N_c$ block clusters as:
\begin{align*}
    \label{eq:mse_iter_n}
    J^{(n)} = \frac{1}{N_c} \sum_{i=1}^{N_c} \frac{1}{|\mathcal{B}_{i}^{(n)}|}\sum_{\bm{v} \in \mathcal{B}_{i}^{(n)}} \frac{1}{L_b}\lVert \bm{b}- B_i^{(n)}(\bm{b})\rVert^2_2
\end{align*}

At the end of iteration $n$, the codebooks are updated from $\mathcal{C}^{(n-1)}$ to $\mathcal{C}^{(n)}$. However, the mapping of a given vector $\bm{b}_j$ to quantizers $\mathcal{C}^{(n)}$ remains as  $f^{(n)}(\bm{b}_j)$. At the next iteration, during the vector clustering step, $f^{(n+1)}(\bm{b}_j)$ finds new mapping of $\bm{b}_j$ to updated codebooks $\mathcal{C}^{(n)}$ such that the quantization MSE over the candidate codebooks is minimized. Therefore, we obtain the following result for $\bm{b}_j$:
\begin{align*}
\frac{1}{L_b}\lVert \bm{b}_j - C_{f^{(n+1)}(\bm{b}_j)}^{(n)}(\bm{b}_j)\rVert^2_2 \le \frac{1}{L_b}\lVert \bm{b}_j - C_{f^{(n)}(\bm{b}_j)}^{(n)}(\bm{b}_j)\rVert^2_2
\end{align*}

That is, quantizing $\bm{b}_j$ at the end of the block clustering step of iteration $n+1$ results in lower quantization MSE compared to quantizing at the end of iteration $n$. Since this is true for all $\bm{b} \in \bm{X}$, we assert the following:
\begin{equation}
\begin{split}
\label{eq:mse_ineq_1}
    \tilde{J}^{(n+1)} &= \frac{1}{N_c} \sum_{i=1}^{N_c} \frac{1}{|\mathcal{B}_{i}^{(n+1)}|}\sum_{\bm{b} \in \mathcal{B}_{i}^{(n+1)}} \frac{1}{L_b}\lVert \bm{b} - C_i^{(n)}(b)\rVert^2_2 \le J^{(n)}
\end{split}
\end{equation}
where $\tilde{J}^{(n+1)}$ is the the quantization MSE after the vector clustering step at iteration $n+1$.

Next, during the codebook update step (\ref{eq:quantizers_update}) at iteration $n+1$, the per-cluster codebooks $\mathcal{C}^{(n)}$ are updated to $\mathcal{C}^{(n+1)}$ by invoking the Lloyd-Max algorithm \citep{Lloyd}. We know that for any given value distribution, the Lloyd-Max algorithm minimizes the quantization MSE. Therefore, for a given vector cluster $\mathcal{B}_i$ we obtain the following result:

\begin{equation}
    \frac{1}{|\mathcal{B}_{i}^{(n+1)}|}\sum_{\bm{b} \in \mathcal{B}_{i}^{(n+1)}} \frac{1}{L_b}\lVert \bm{b}- C_i^{(n+1)}(\bm{b})\rVert^2_2 \le \frac{1}{|\mathcal{B}_{i}^{(n+1)}|}\sum_{\bm{b} \in \mathcal{B}_{i}^{(n+1)}} \frac{1}{L_b}\lVert \bm{b}- C_i^{(n)}(\bm{b})\rVert^2_2
\end{equation}

The above equation states that quantizing the given block cluster $\mathcal{B}_i$ after updating the associated codebook from $C_i^{(n)}$ to $C_i^{(n+1)}$ results in lower quantization MSE. Since this is true for all the block clusters, we derive the following result: 
\begin{equation}
\begin{split}
\label{eq:mse_ineq_2}
     J^{(n+1)} &= \frac{1}{N_c} \sum_{i=1}^{N_c} \frac{1}{|\mathcal{B}_{i}^{(n+1)}|}\sum_{\bm{b} \in \mathcal{B}_{i}^{(n+1)}} \frac{1}{L_b}\lVert \bm{b}- C_i^{(n+1)}(\bm{b})\rVert^2_2  \le \tilde{J}^{(n+1)}   
\end{split}
\end{equation}

Following (\ref{eq:mse_ineq_1}) and (\ref{eq:mse_ineq_2}), we find that the quantization MSE is non-increasing for each iteration, that is, $J^{(1)} \ge J^{(2)} \ge J^{(3)} \ge \ldots \ge J^{(M)}$ where $M$ is the maximum number of iterations. 
%Therefore, we can say that if the algorithm converges, then it must be that it has converged to a local minimum. 
\hfill $\blacksquare$


\begin{figure}
    \begin{center}
    \includegraphics[width=0.5\textwidth]{sections//figures/mse_vs_iter.pdf}
    \end{center}
    \caption{\small NMSE vs iterations during LO-BCQ compared to other block quantization proposals}
    \label{fig:nmse_vs_iter}
\end{figure}

Figure \ref{fig:nmse_vs_iter} shows the empirical convergence of LO-BCQ across several block lengths and number of codebooks. Also, the MSE achieved by LO-BCQ is compared to baselines such as MXFP and VSQ. As shown, LO-BCQ converges to a lower MSE than the baselines. Further, we achieve better convergence for larger number of codebooks ($N_c$) and for a smaller block length ($L_b$), both of which increase the bitwidth of BCQ (see Eq \ref{eq:bitwidth_bcq}).


\subsection{Additional Accuracy Results}
%Table \ref{tab:lobcq_config} lists the various LOBCQ configurations and their corresponding bitwidths.
\begin{table}
\setlength{\tabcolsep}{4.75pt}
\begin{center}
\caption{\label{tab:lobcq_config} Various LO-BCQ configurations and their bitwidths.}
\begin{tabular}{|c||c|c|c|c||c|c||c|} 
\hline
 & \multicolumn{4}{|c||}{$L_b=8$} & \multicolumn{2}{|c||}{$L_b=4$} & $L_b=2$ \\
 \hline
 \backslashbox{$L_A$\kern-1em}{\kern-1em$N_c$} & 2 & 4 & 8 & 16 & 2 & 4 & 2 \\
 \hline
 64 & 4.25 & 4.375 & 4.5 & 4.625 & 4.375 & 4.625 & 4.625\\
 \hline
 32 & 4.375 & 4.5 & 4.625& 4.75 & 4.5 & 4.75 & 4.75 \\
 \hline
 16 & 4.625 & 4.75& 4.875 & 5 & 4.75 & 5 & 5 \\
 \hline
\end{tabular}
\end{center}
\end{table}

%\subsection{Perplexity achieved by various LO-BCQ configurations on Wikitext-103 dataset}

\begin{table} \centering
\begin{tabular}{|c||c|c|c|c||c|c||c|} 
\hline
 $L_b \rightarrow$& \multicolumn{4}{c||}{8} & \multicolumn{2}{c||}{4} & 2\\
 \hline
 \backslashbox{$L_A$\kern-1em}{\kern-1em$N_c$} & 2 & 4 & 8 & 16 & 2 & 4 & 2  \\
 %$N_c \rightarrow$ & 2 & 4 & 8 & 16 & 2 & 4 & 2 \\
 \hline
 \hline
 \multicolumn{8}{c}{GPT3-1.3B (FP32 PPL = 9.98)} \\ 
 \hline
 \hline
 64 & 10.40 & 10.23 & 10.17 & 10.15 &  10.28 & 10.18 & 10.19 \\
 \hline
 32 & 10.25 & 10.20 & 10.15 & 10.12 &  10.23 & 10.17 & 10.17 \\
 \hline
 16 & 10.22 & 10.16 & 10.10 & 10.09 &  10.21 & 10.14 & 10.16 \\
 \hline
  \hline
 \multicolumn{8}{c}{GPT3-8B (FP32 PPL = 7.38)} \\ 
 \hline
 \hline
 64 & 7.61 & 7.52 & 7.48 &  7.47 &  7.55 &  7.49 & 7.50 \\
 \hline
 32 & 7.52 & 7.50 & 7.46 &  7.45 &  7.52 &  7.48 & 7.48  \\
 \hline
 16 & 7.51 & 7.48 & 7.44 &  7.44 &  7.51 &  7.49 & 7.47  \\
 \hline
\end{tabular}
\caption{\label{tab:ppl_gpt3_abalation} Wikitext-103 perplexity across GPT3-1.3B and 8B models.}
\end{table}

\begin{table} \centering
\begin{tabular}{|c||c|c|c|c||} 
\hline
 $L_b \rightarrow$& \multicolumn{4}{c||}{8}\\
 \hline
 \backslashbox{$L_A$\kern-1em}{\kern-1em$N_c$} & 2 & 4 & 8 & 16 \\
 %$N_c \rightarrow$ & 2 & 4 & 8 & 16 & 2 & 4 & 2 \\
 \hline
 \hline
 \multicolumn{5}{|c|}{Llama2-7B (FP32 PPL = 5.06)} \\ 
 \hline
 \hline
 64 & 5.31 & 5.26 & 5.19 & 5.18  \\
 \hline
 32 & 5.23 & 5.25 & 5.18 & 5.15  \\
 \hline
 16 & 5.23 & 5.19 & 5.16 & 5.14  \\
 \hline
 \multicolumn{5}{|c|}{Nemotron4-15B (FP32 PPL = 5.87)} \\ 
 \hline
 \hline
 64  & 6.3 & 6.20 & 6.13 & 6.08  \\
 \hline
 32  & 6.24 & 6.12 & 6.07 & 6.03  \\
 \hline
 16  & 6.12 & 6.14 & 6.04 & 6.02  \\
 \hline
 \multicolumn{5}{|c|}{Nemotron4-340B (FP32 PPL = 3.48)} \\ 
 \hline
 \hline
 64 & 3.67 & 3.62 & 3.60 & 3.59 \\
 \hline
 32 & 3.63 & 3.61 & 3.59 & 3.56 \\
 \hline
 16 & 3.61 & 3.58 & 3.57 & 3.55 \\
 \hline
\end{tabular}
\caption{\label{tab:ppl_llama7B_nemo15B} Wikitext-103 perplexity compared to FP32 baseline in Llama2-7B and Nemotron4-15B, 340B models}
\end{table}

%\subsection{Perplexity achieved by various LO-BCQ configurations on MMLU dataset}


\begin{table} \centering
\begin{tabular}{|c||c|c|c|c||c|c|c|c|} 
\hline
 $L_b \rightarrow$& \multicolumn{4}{c||}{8} & \multicolumn{4}{c||}{8}\\
 \hline
 \backslashbox{$L_A$\kern-1em}{\kern-1em$N_c$} & 2 & 4 & 8 & 16 & 2 & 4 & 8 & 16  \\
 %$N_c \rightarrow$ & 2 & 4 & 8 & 16 & 2 & 4 & 2 \\
 \hline
 \hline
 \multicolumn{5}{|c|}{Llama2-7B (FP32 Accuracy = 45.8\%)} & \multicolumn{4}{|c|}{Llama2-70B (FP32 Accuracy = 69.12\%)} \\ 
 \hline
 \hline
 64 & 43.9 & 43.4 & 43.9 & 44.9 & 68.07 & 68.27 & 68.17 & 68.75 \\
 \hline
 32 & 44.5 & 43.8 & 44.9 & 44.5 & 68.37 & 68.51 & 68.35 & 68.27  \\
 \hline
 16 & 43.9 & 42.7 & 44.9 & 45 & 68.12 & 68.77 & 68.31 & 68.59  \\
 \hline
 \hline
 \multicolumn{5}{|c|}{GPT3-22B (FP32 Accuracy = 38.75\%)} & \multicolumn{4}{|c|}{Nemotron4-15B (FP32 Accuracy = 64.3\%)} \\ 
 \hline
 \hline
 64 & 36.71 & 38.85 & 38.13 & 38.92 & 63.17 & 62.36 & 63.72 & 64.09 \\
 \hline
 32 & 37.95 & 38.69 & 39.45 & 38.34 & 64.05 & 62.30 & 63.8 & 64.33  \\
 \hline
 16 & 38.88 & 38.80 & 38.31 & 38.92 & 63.22 & 63.51 & 63.93 & 64.43  \\
 \hline
\end{tabular}
\caption{\label{tab:mmlu_abalation} Accuracy on MMLU dataset across GPT3-22B, Llama2-7B, 70B and Nemotron4-15B models.}
\end{table}


%\subsection{Perplexity achieved by various LO-BCQ configurations on LM evaluation harness}

\begin{table} \centering
\begin{tabular}{|c||c|c|c|c||c|c|c|c|} 
\hline
 $L_b \rightarrow$& \multicolumn{4}{c||}{8} & \multicolumn{4}{c||}{8}\\
 \hline
 \backslashbox{$L_A$\kern-1em}{\kern-1em$N_c$} & 2 & 4 & 8 & 16 & 2 & 4 & 8 & 16  \\
 %$N_c \rightarrow$ & 2 & 4 & 8 & 16 & 2 & 4 & 2 \\
 \hline
 \hline
 \multicolumn{5}{|c|}{Race (FP32 Accuracy = 37.51\%)} & \multicolumn{4}{|c|}{Boolq (FP32 Accuracy = 64.62\%)} \\ 
 \hline
 \hline
 64 & 36.94 & 37.13 & 36.27 & 37.13 & 63.73 & 62.26 & 63.49 & 63.36 \\
 \hline
 32 & 37.03 & 36.36 & 36.08 & 37.03 & 62.54 & 63.51 & 63.49 & 63.55  \\
 \hline
 16 & 37.03 & 37.03 & 36.46 & 37.03 & 61.1 & 63.79 & 63.58 & 63.33  \\
 \hline
 \hline
 \multicolumn{5}{|c|}{Winogrande (FP32 Accuracy = 58.01\%)} & \multicolumn{4}{|c|}{Piqa (FP32 Accuracy = 74.21\%)} \\ 
 \hline
 \hline
 64 & 58.17 & 57.22 & 57.85 & 58.33 & 73.01 & 73.07 & 73.07 & 72.80 \\
 \hline
 32 & 59.12 & 58.09 & 57.85 & 58.41 & 73.01 & 73.94 & 72.74 & 73.18  \\
 \hline
 16 & 57.93 & 58.88 & 57.93 & 58.56 & 73.94 & 72.80 & 73.01 & 73.94  \\
 \hline
\end{tabular}
\caption{\label{tab:mmlu_abalation} Accuracy on LM evaluation harness tasks on GPT3-1.3B model.}
\end{table}

\begin{table} \centering
\begin{tabular}{|c||c|c|c|c||c|c|c|c|} 
\hline
 $L_b \rightarrow$& \multicolumn{4}{c||}{8} & \multicolumn{4}{c||}{8}\\
 \hline
 \backslashbox{$L_A$\kern-1em}{\kern-1em$N_c$} & 2 & 4 & 8 & 16 & 2 & 4 & 8 & 16  \\
 %$N_c \rightarrow$ & 2 & 4 & 8 & 16 & 2 & 4 & 2 \\
 \hline
 \hline
 \multicolumn{5}{|c|}{Race (FP32 Accuracy = 41.34\%)} & \multicolumn{4}{|c|}{Boolq (FP32 Accuracy = 68.32\%)} \\ 
 \hline
 \hline
 64 & 40.48 & 40.10 & 39.43 & 39.90 & 69.20 & 68.41 & 69.45 & 68.56 \\
 \hline
 32 & 39.52 & 39.52 & 40.77 & 39.62 & 68.32 & 67.43 & 68.17 & 69.30  \\
 \hline
 16 & 39.81 & 39.71 & 39.90 & 40.38 & 68.10 & 66.33 & 69.51 & 69.42  \\
 \hline
 \hline
 \multicolumn{5}{|c|}{Winogrande (FP32 Accuracy = 67.88\%)} & \multicolumn{4}{|c|}{Piqa (FP32 Accuracy = 78.78\%)} \\ 
 \hline
 \hline
 64 & 66.85 & 66.61 & 67.72 & 67.88 & 77.31 & 77.42 & 77.75 & 77.64 \\
 \hline
 32 & 67.25 & 67.72 & 67.72 & 67.00 & 77.31 & 77.04 & 77.80 & 77.37  \\
 \hline
 16 & 68.11 & 68.90 & 67.88 & 67.48 & 77.37 & 78.13 & 78.13 & 77.69  \\
 \hline
\end{tabular}
\caption{\label{tab:mmlu_abalation} Accuracy on LM evaluation harness tasks on GPT3-8B model.}
\end{table}

\begin{table} \centering
\begin{tabular}{|c||c|c|c|c||c|c|c|c|} 
\hline
 $L_b \rightarrow$& \multicolumn{4}{c||}{8} & \multicolumn{4}{c||}{8}\\
 \hline
 \backslashbox{$L_A$\kern-1em}{\kern-1em$N_c$} & 2 & 4 & 8 & 16 & 2 & 4 & 8 & 16  \\
 %$N_c \rightarrow$ & 2 & 4 & 8 & 16 & 2 & 4 & 2 \\
 \hline
 \hline
 \multicolumn{5}{|c|}{Race (FP32 Accuracy = 40.67\%)} & \multicolumn{4}{|c|}{Boolq (FP32 Accuracy = 76.54\%)} \\ 
 \hline
 \hline
 64 & 40.48 & 40.10 & 39.43 & 39.90 & 75.41 & 75.11 & 77.09 & 75.66 \\
 \hline
 32 & 39.52 & 39.52 & 40.77 & 39.62 & 76.02 & 76.02 & 75.96 & 75.35  \\
 \hline
 16 & 39.81 & 39.71 & 39.90 & 40.38 & 75.05 & 73.82 & 75.72 & 76.09  \\
 \hline
 \hline
 \multicolumn{5}{|c|}{Winogrande (FP32 Accuracy = 70.64\%)} & \multicolumn{4}{|c|}{Piqa (FP32 Accuracy = 79.16\%)} \\ 
 \hline
 \hline
 64 & 69.14 & 70.17 & 70.17 & 70.56 & 78.24 & 79.00 & 78.62 & 78.73 \\
 \hline
 32 & 70.96 & 69.69 & 71.27 & 69.30 & 78.56 & 79.49 & 79.16 & 78.89  \\
 \hline
 16 & 71.03 & 69.53 & 69.69 & 70.40 & 78.13 & 79.16 & 79.00 & 79.00  \\
 \hline
\end{tabular}
\caption{\label{tab:mmlu_abalation} Accuracy on LM evaluation harness tasks on GPT3-22B model.}
\end{table}

\begin{table} \centering
\begin{tabular}{|c||c|c|c|c||c|c|c|c|} 
\hline
 $L_b \rightarrow$& \multicolumn{4}{c||}{8} & \multicolumn{4}{c||}{8}\\
 \hline
 \backslashbox{$L_A$\kern-1em}{\kern-1em$N_c$} & 2 & 4 & 8 & 16 & 2 & 4 & 8 & 16  \\
 %$N_c \rightarrow$ & 2 & 4 & 8 & 16 & 2 & 4 & 2 \\
 \hline
 \hline
 \multicolumn{5}{|c|}{Race (FP32 Accuracy = 44.4\%)} & \multicolumn{4}{|c|}{Boolq (FP32 Accuracy = 79.29\%)} \\ 
 \hline
 \hline
 64 & 42.49 & 42.51 & 42.58 & 43.45 & 77.58 & 77.37 & 77.43 & 78.1 \\
 \hline
 32 & 43.35 & 42.49 & 43.64 & 43.73 & 77.86 & 75.32 & 77.28 & 77.86  \\
 \hline
 16 & 44.21 & 44.21 & 43.64 & 42.97 & 78.65 & 77 & 76.94 & 77.98  \\
 \hline
 \hline
 \multicolumn{5}{|c|}{Winogrande (FP32 Accuracy = 69.38\%)} & \multicolumn{4}{|c|}{Piqa (FP32 Accuracy = 78.07\%)} \\ 
 \hline
 \hline
 64 & 68.9 & 68.43 & 69.77 & 68.19 & 77.09 & 76.82 & 77.09 & 77.86 \\
 \hline
 32 & 69.38 & 68.51 & 68.82 & 68.90 & 78.07 & 76.71 & 78.07 & 77.86  \\
 \hline
 16 & 69.53 & 67.09 & 69.38 & 68.90 & 77.37 & 77.8 & 77.91 & 77.69  \\
 \hline
\end{tabular}
\caption{\label{tab:mmlu_abalation} Accuracy on LM evaluation harness tasks on Llama2-7B model.}
\end{table}

\begin{table} \centering
\begin{tabular}{|c||c|c|c|c||c|c|c|c|} 
\hline
 $L_b \rightarrow$& \multicolumn{4}{c||}{8} & \multicolumn{4}{c||}{8}\\
 \hline
 \backslashbox{$L_A$\kern-1em}{\kern-1em$N_c$} & 2 & 4 & 8 & 16 & 2 & 4 & 8 & 16  \\
 %$N_c \rightarrow$ & 2 & 4 & 8 & 16 & 2 & 4 & 2 \\
 \hline
 \hline
 \multicolumn{5}{|c|}{Race (FP32 Accuracy = 48.8\%)} & \multicolumn{4}{|c|}{Boolq (FP32 Accuracy = 85.23\%)} \\ 
 \hline
 \hline
 64 & 49.00 & 49.00 & 49.28 & 48.71 & 82.82 & 84.28 & 84.03 & 84.25 \\
 \hline
 32 & 49.57 & 48.52 & 48.33 & 49.28 & 83.85 & 84.46 & 84.31 & 84.93  \\
 \hline
 16 & 49.85 & 49.09 & 49.28 & 48.99 & 85.11 & 84.46 & 84.61 & 83.94  \\
 \hline
 \hline
 \multicolumn{5}{|c|}{Winogrande (FP32 Accuracy = 79.95\%)} & \multicolumn{4}{|c|}{Piqa (FP32 Accuracy = 81.56\%)} \\ 
 \hline
 \hline
 64 & 78.77 & 78.45 & 78.37 & 79.16 & 81.45 & 80.69 & 81.45 & 81.5 \\
 \hline
 32 & 78.45 & 79.01 & 78.69 & 80.66 & 81.56 & 80.58 & 81.18 & 81.34  \\
 \hline
 16 & 79.95 & 79.56 & 79.79 & 79.72 & 81.28 & 81.66 & 81.28 & 80.96  \\
 \hline
\end{tabular}
\caption{\label{tab:mmlu_abalation} Accuracy on LM evaluation harness tasks on Llama2-70B model.}
\end{table}

%\section{MSE Studies}
%\textcolor{red}{TODO}


\subsection{Number Formats and Quantization Method}
\label{subsec:numFormats_quantMethod}
\subsubsection{Integer Format}
An $n$-bit signed integer (INT) is typically represented with a 2s-complement format \citep{yao2022zeroquant,xiao2023smoothquant,dai2021vsq}, where the most significant bit denotes the sign.

\subsubsection{Floating Point Format}
An $n$-bit signed floating point (FP) number $x$ comprises of a 1-bit sign ($x_{\mathrm{sign}}$), $B_m$-bit mantissa ($x_{\mathrm{mant}}$) and $B_e$-bit exponent ($x_{\mathrm{exp}}$) such that $B_m+B_e=n-1$. The associated constant exponent bias ($E_{\mathrm{bias}}$) is computed as $(2^{{B_e}-1}-1)$. We denote this format as $E_{B_e}M_{B_m}$.  

\subsubsection{Quantization Scheme}
\label{subsec:quant_method}
A quantization scheme dictates how a given unquantized tensor is converted to its quantized representation. We consider FP formats for the purpose of illustration. Given an unquantized tensor $\bm{X}$ and an FP format $E_{B_e}M_{B_m}$, we first, we compute the quantization scale factor $s_X$ that maps the maximum absolute value of $\bm{X}$ to the maximum quantization level of the $E_{B_e}M_{B_m}$ format as follows:
\begin{align}
\label{eq:sf}
    s_X = \frac{\mathrm{max}(|\bm{X}|)}{\mathrm{max}(E_{B_e}M_{B_m})}
\end{align}
In the above equation, $|\cdot|$ denotes the absolute value function.

Next, we scale $\bm{X}$ by $s_X$ and quantize it to $\hat{\bm{X}}$ by rounding it to the nearest quantization level of $E_{B_e}M_{B_m}$ as:

\begin{align}
\label{eq:tensor_quant}
    \hat{\bm{X}} = \text{round-to-nearest}\left(\frac{\bm{X}}{s_X}, E_{B_e}M_{B_m}\right)
\end{align}

We perform dynamic max-scaled quantization \citep{wu2020integer}, where the scale factor $s$ for activations is dynamically computed during runtime.

\subsection{Vector Scaled Quantization}
\begin{wrapfigure}{r}{0.35\linewidth}
  \centering
  \includegraphics[width=\linewidth]{sections/figures/vsquant.jpg}
  \caption{\small Vectorwise decomposition for per-vector scaled quantization (VSQ \citep{dai2021vsq}).}
  \label{fig:vsquant}
\end{wrapfigure}
During VSQ \citep{dai2021vsq}, the operand tensors are decomposed into 1D vectors in a hardware friendly manner as shown in Figure \ref{fig:vsquant}. Since the decomposed tensors are used as operands in matrix multiplications during inference, it is beneficial to perform this decomposition along the reduction dimension of the multiplication. The vectorwise quantization is performed similar to tensorwise quantization described in Equations \ref{eq:sf} and \ref{eq:tensor_quant}, where a scale factor $s_v$ is required for each vector $\bm{v}$ that maps the maximum absolute value of that vector to the maximum quantization level. While smaller vector lengths can lead to larger accuracy gains, the associated memory and computational overheads due to the per-vector scale factors increases. To alleviate these overheads, VSQ \citep{dai2021vsq} proposed a second level quantization of the per-vector scale factors to unsigned integers, while MX \citep{rouhani2023shared} quantizes them to integer powers of 2 (denoted as $2^{INT}$).

\subsubsection{MX Format}
The MX format proposed in \citep{rouhani2023microscaling} introduces the concept of sub-block shifting. For every two scalar elements of $b$-bits each, there is a shared exponent bit. The value of this exponent bit is determined through an empirical analysis that targets minimizing quantization MSE. We note that the FP format $E_{1}M_{b}$ is strictly better than MX from an accuracy perspective since it allocates a dedicated exponent bit to each scalar as opposed to sharing it across two scalars. Therefore, we conservatively bound the accuracy of a $b+2$-bit signed MX format with that of a $E_{1}M_{b}$ format in our comparisons. For instance, we use E1M2 format as a proxy for MX4.

\begin{figure}
    \centering
    \includegraphics[width=1\linewidth]{sections//figures/BlockFormats.pdf}
    \caption{\small Comparing LO-BCQ to MX format.}
    \label{fig:block_formats}
\end{figure}

Figure \ref{fig:block_formats} compares our $4$-bit LO-BCQ block format to MX \citep{rouhani2023microscaling}. As shown, both LO-BCQ and MX decompose a given operand tensor into block arrays and each block array into blocks. Similar to MX, we find that per-block quantization ($L_b < L_A$) leads to better accuracy due to increased flexibility. While MX achieves this through per-block $1$-bit micro-scales, we associate a dedicated codebook to each block through a per-block codebook selector. Further, MX quantizes the per-block array scale-factor to E8M0 format without per-tensor scaling. In contrast during LO-BCQ, we find that per-tensor scaling combined with quantization of per-block array scale-factor to E4M3 format results in superior inference accuracy across models. 

% \section{Appendix}
% You may include other additional sections here.


\end{document}
