During this study, we accomplished our aim of using FFF 3D printing to fabricate an optical sensor that can be both distributed and highly integrated and we demonstrated how this fabrication technique could be used to achieve proprioception in a multi-material 3D printed finger. Our compliant tendon-actuated finger is able to perform pose estimation in the presence of static forces and, with training data, it is able to detect contact with the environment.

There are still aspects that warrant further improvement and investigation. In particular, further development of the embedding process is needed to increase its reproducibility and improve overall output repeatability among different sensors. In addition, future work needs to build on our contribution by focusing on how the embodiment of the sensor affects its response characteristics. Our experience and the results from the study suggest that better sensor output might be achieved by optimizing the flexure morphology. Thanks to the accessibility and design flexibility of FFF 3D printing and the low-cost power efficient nature of our proposed sensing methodology, we envision it being adapted to existing soft robotics design and used in larger distributed sensing networks to finally achieve accurate proprioception in the soft robots of the future.