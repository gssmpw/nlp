\begin{figure}
    \centering
    \includegraphics[width=0.9\columnwidth]{ExperimentalSetup_w_scalebar.pdf}
    \caption{Setup used for the characterization of the fingers. The photo is taken according to the POV of the camera used for the tracking of the markers.}
    \label{fig:setup_figure}
    \vspace{-1em}
\end{figure}
\begin{figure*}[t]
\centering
\includegraphics[width=0.9\textwidth]
{FingersComparison_compressed_revised.png}
\caption{Characterizations of the sensors. (a),(b),(c) The plots show measured optical power loss against the MCP, PIP, and DIP joint rotations for each finger during the quasi-static trials with their respective best-fit lines. (d) The sensitivity achieved for each sensor is shown grouped by finger on the bar chart. (e) RMS error in degrees for each joint is shown. (f) Hysteresis as percentage of full scale range for each joint is shown.}
\label{fig:fingers_results}
\vspace{-1em}
\end{figure*}
In this section, we describe a series of tests in which we analyze characteristics of the sensors that are important for engineering and control applications. Specifically, we characterize the reproducibility, repeatability, sensitivity, linearity, and hysteresis of the sensors, while also investigating the sensors' ability to retain their response characteristics over their operational life. Three sensorized flexures were 3D printed to evaluate sensor fabrication reproducibility, each of which were inserted into the same multi-material finger body.
\subsection{Protocol}\label{subsec:protocol}
\subsubsection{Quasi-Static Test}
To evaluate the sensor performance as-actuated, we used a 2-step pulley mounted on a Dynamixel MX-106R servo motor (Robotis, Seoul, South Korea) to actuate both the flexor and extensor tendons of the multi-material finger. After applying a \qty{15}{rpm} \qty{90}{\degree} transient rotation to the pulley, we observed a settling time under \qty{125}{\ms}. To characterize the sensors in quasi-static conditions, the pulley was rotated from \qty{0}{\degree} to \qty{263.6}{\degree} in increments of \qty{0.44}{\degree}. After each rotation, a pause of an eight of a second was performed before taking a measurement from the sensors. The entire procedure was repeated five times for each of three different fingers, resulting in $N=12010$ readings from each joint of each finger over the full test. The test was conducted with the apparatus placed under an LED lamp to control for ambient light. 

A webcam was placed square to the finger to enable joint angle measurements via image processing by detecting the pose of ArUco markers~\cite{ArUco} affixed to each phalanx of the finger (Fig. \ref{fig:setup_figure}) and taking the Euler angle of the markers in the plane of the webcam image. The rotations $\theta_\textrm{MCP}$, $\theta_\textrm{PIP}$, and $\theta_\textrm{DIP}$ of the metacarpophalangeal, proximal interphalangeal, and distal interphalangeal joints were then calculated by the difference between the marker rotations of their respective phalanges. The servo motor, microcontroller, and webcam were all controlled by the same laptop running a data collection script in MATLAB. 

\subsubsection{Stress Test}
To evaluate the sensor's ability to retain its response characteristics over time, we selected one finger and subjected it to cyclic motion over the full \qty{263.6}{\degree} motion range of the setup. We chose an angular speed for the tendon pulley of \qty{60}{rpm} corresponding with an actuation frequency of \qty{1.46}{Hz}, in line with existing actuation mechanisms in soft robots \cite{Li2023}. After performing all other experiments and applying 50 cycles to the finger to break it in, we applied 450 cycles in constant lighting conditions. Sensor data was acquired at \qty{8}{Hz}.

\subsection{Sensor Model}\label{subsec:sensor_model}
To obtain a calibration line for each sensor, we applied a least-squares linear regression to the data collected from all five quasi-static trials for each sensor according to a linear model of the form 
\begin{equation}\label{eq:model}
   \hat{y}=\begin{bmatrix}1 & \theta\end{bmatrix}\begin{bmatrix}\beta_0 \\ \beta_1\end{bmatrix}
\end{equation}
where $\hat{y}$ is the best fit of the sensor output $y$, $\begin{bmatrix}\beta_0 & \beta_1\end{bmatrix}^\top$ is the vector of regression coefficients. $y$ is expressed as optical power loss in decibels and is calculated as $y=-10\log_{10}(I/I_0)$, where $I$ is the raw digital value in nW/cm$^{2}$ provided by the sensor and $I_0$ is the raw value measured at the beginning of each trial.

\subsection{Results}
To evaluate the utility of the sensors for closed-loop control of soft robots, we examine the quality of the models in predicting the joint angle by calculating metrics for reproducibility, repeatability, sensitivity, linearity, and hysteresis. We first examined reproducibility by conducting an ANCOVA test using $\theta$ as the covariate and individual sensor responses as categorical factors. We observed a highly statistically significant difference between the responses [F(8,108072)=9822, p\textless 0.01], indicating that unique calibration lines should be used for each sensor. 

The sensitivity of each sensor was taken to be the slope of the calibration line, $\beta_1$. Percentage linearity was calculated as a percentage of full scale range (FSR) according to the formula in \cite{Fleming2013}: \begin{equation}
    \textrm{Linearity(\% FSR)}=\left(1-\frac{\max|y-\hat{y}|}{\max |y|}\right)\times 100\%
\end{equation}
To give a metric for repeatability, we calculated the RMS error measured in degrees $\sigma_\textrm{RMS}$ of each sensor according to the formula:
\begin{equation}
    \sigma_\textrm{RMS}=\frac{1}{\beta_1}\sqrt{\sum_{i=1}^N\frac{\left(y_i-\hat{y}_i\right)^2}{N-1}}
\end{equation}
The linearity values we achieved typically ranged between 59\% (PIP joint of finger 1) and 88\% (DIP joint of finger 2) with a mean value of 70\% and a single outlier due to the PIP joint of finger 3 with a value of 37\%. We note from these calibrations that the sensitivities are higher for fingers 2 and 3 than for finger 1. When taken with the results of the ANCOVA test, this suggests that the manual steps in the fabrication process, although limited, still affect the resulting sensor array.

To quantify hysteresis for each sensor, we took the difference between the mean sensor reading from the loading portions of the quasi-static trials and the mean sensor reading from the unloading portions of the quasi-static trials. We express this difference as a percentage of the FSR for each sensor in Fig. \ref{fig:fingers_results}f. We note that sensors with greater hysteresis also show less repeatability, as expected. 

We quantified the results of the stress test by measuring the normalized amplitude $(I_\textrm{max}-I_\textrm{min})/I_\textrm{max}$ during each cycle. Finger 3 completed the test with no detectable change in cycle amplitude on any of the three sensor readings (Fig. \ref{fig:stress}). This result indicates that the sensor is able to retain its response characteristics over a typical research service life. 
\begin{figure}
\centering
\includegraphics[width=0.9\columnwidth]{stress_plot.pdf}
\caption{Results of a stress test on one of the fingers. All sensors survived undergoing 450 actuation cycles, as demonstrated by their normalized cycle amplitude that do not decay over time.}
\label{fig:stress}
\vspace{-1em}
\end{figure}